\documentclass[12pt]{article}
\usepackage[margin=1.2in]{geometry}
\usepackage{graphicx}
\usepackage{amsmath}
\usepackage{physics}
\usepackage{tabto}
\usepackage{float}
\usepackage{amssymb}
\usepackage{pgfplots}
\usepackage{verbatim}
\usepackage{tcolorbox}
\usepackage{listings}
\usepackage{xcolor}
\usepackage{siunitx}
\pgfplotsset{compat=1.16}
\definecolor{stringcolor}{HTML}{C792EA}
\definecolor{codeblue}{HTML}{2162DB}
\definecolor{commentcolor}{HTML}{4A6E46}
\lstset{
    basicstyle=\ttfamily\footnotesize,
    commentstyle=\color{commentcolor},
    keywordstyle=\color{codeblue},
    stringstyle=\color{stringcolor},
    showstringspaces=false,
    numbers=left,
    upquote=true,
    captionpos=t,
    abovecaptionskip=12pt,
    belowcaptionskip=12pt,
    language=Python,
    breaklines=true,
    frame=single
    }
\renewcommand{\lstlistingname}{Appendix}

\title{More Advanced Model Fitting and Plotting}
\author{\textbf{PHY2004W \hspace{8cm} KDSMIL001}}
\date{\textbf{24 Feb 2020}}

\begin{document}

    \begin{titlepage}
        \maketitle
        \tableofcontents
    \end{titlepage}

    \section{Answers}
    The first section of the activity was plotting the best-fit curve for a set of non-linear 
    points. The code for everything in this section can be found in Appendix 1.
    Firstly, we were asked to plot the data supplied to us in DampedData.txt, which was a text 
    file containing time and position data of a damped oscillator. To do this we used the errorbar 
    function in \texttt{matplotlib.pyplot} [line 31]. Next, we defined a function that takes in 
    a set of parameters and returns a value for $y(t)$ where

    \begin{equation}
        y(t) = A+Be^{-\gamma t}cos(\omega t-\alpha)
    \end{equation}
    
    \noindent
    This is the equation for the position of a damped oscillator with respect to time. $A, B, 
    \gamma, \omega,$ and $\alpha$ are parameters that change the shape of the curve plotted 
    by this function in various ways. The function defined on line 36 takes these parameters, 
    as well as $t$, and returns a value for the position. \newline
    In order to begin fitting a curve to this data, we first need a set of initial parameters. 
    These are defined on line 20 and were obtained by guessing a few and then adjusting them 
    until we reached a curve that very roughly fit the data points. They are defined in an array 
    in order to be passed to the function that we'll be using later to properly fit the curve 
    to the data. Below, in Figure \ref{fig:Initial Guess}, you can see the curve of our initial 
    guess along with the values of each parameter in Table \ref{table:Initial Params}.
    
    \begin{figure}[H]
        \begin{center}
           \scalebox{.8}{%% Creator: Matplotlib, PGF backend
%%
%% To include the figure in your LaTeX document, write
%%   \input{<filename>.pgf}
%%
%% Make sure the required packages are loaded in your preamble
%%   \usepackage{pgf}
%%
%% Figures using additional raster images can only be included by \input if
%% they are in the same directory as the main LaTeX file. For loading figures
%% from other directories you can use the `import` package
%%   \usepackage{import}
%% and then include the figures with
%%   \import{<path to file>}{<filename>.pgf}
%%
%% Matplotlib used the following preamble
%%
\begingroup%
\makeatletter%
\begin{pgfpicture}%
\pgfpathrectangle{\pgfpointorigin}{\pgfqpoint{6.400000in}{4.800000in}}%
\pgfusepath{use as bounding box, clip}%
\begin{pgfscope}%
\pgfsetbuttcap%
\pgfsetmiterjoin%
\definecolor{currentfill}{rgb}{1.000000,1.000000,1.000000}%
\pgfsetfillcolor{currentfill}%
\pgfsetlinewidth{0.000000pt}%
\definecolor{currentstroke}{rgb}{1.000000,1.000000,1.000000}%
\pgfsetstrokecolor{currentstroke}%
\pgfsetdash{}{0pt}%
\pgfpathmoveto{\pgfqpoint{0.000000in}{0.000000in}}%
\pgfpathlineto{\pgfqpoint{6.400000in}{0.000000in}}%
\pgfpathlineto{\pgfqpoint{6.400000in}{4.800000in}}%
\pgfpathlineto{\pgfqpoint{0.000000in}{4.800000in}}%
\pgfpathclose%
\pgfusepath{fill}%
\end{pgfscope}%
\begin{pgfscope}%
\pgfsetbuttcap%
\pgfsetmiterjoin%
\definecolor{currentfill}{rgb}{1.000000,1.000000,1.000000}%
\pgfsetfillcolor{currentfill}%
\pgfsetlinewidth{0.000000pt}%
\definecolor{currentstroke}{rgb}{0.000000,0.000000,0.000000}%
\pgfsetstrokecolor{currentstroke}%
\pgfsetstrokeopacity{0.000000}%
\pgfsetdash{}{0pt}%
\pgfpathmoveto{\pgfqpoint{0.800000in}{0.528000in}}%
\pgfpathlineto{\pgfqpoint{5.760000in}{0.528000in}}%
\pgfpathlineto{\pgfqpoint{5.760000in}{4.224000in}}%
\pgfpathlineto{\pgfqpoint{0.800000in}{4.224000in}}%
\pgfpathclose%
\pgfusepath{fill}%
\end{pgfscope}%
\begin{pgfscope}%
\pgfsetbuttcap%
\pgfsetroundjoin%
\definecolor{currentfill}{rgb}{0.000000,0.000000,0.000000}%
\pgfsetfillcolor{currentfill}%
\pgfsetlinewidth{0.803000pt}%
\definecolor{currentstroke}{rgb}{0.000000,0.000000,0.000000}%
\pgfsetstrokecolor{currentstroke}%
\pgfsetdash{}{0pt}%
\pgfsys@defobject{currentmarker}{\pgfqpoint{0.000000in}{-0.048611in}}{\pgfqpoint{0.000000in}{0.000000in}}{%
\pgfpathmoveto{\pgfqpoint{0.000000in}{0.000000in}}%
\pgfpathlineto{\pgfqpoint{0.000000in}{-0.048611in}}%
\pgfusepath{stroke,fill}%
}%
\begin{pgfscope}%
\pgfsys@transformshift{1.025455in}{0.528000in}%
\pgfsys@useobject{currentmarker}{}%
\end{pgfscope}%
\end{pgfscope}%
\begin{pgfscope}%
\definecolor{textcolor}{rgb}{0.000000,0.000000,0.000000}%
\pgfsetstrokecolor{textcolor}%
\pgfsetfillcolor{textcolor}%
\pgftext[x=1.025455in,y=0.430778in,,top]{\color{textcolor}\rmfamily\fontsize{10.000000}{12.000000}\selectfont \(\displaystyle 0\)}%
\end{pgfscope}%
\begin{pgfscope}%
\pgfsetbuttcap%
\pgfsetroundjoin%
\definecolor{currentfill}{rgb}{0.000000,0.000000,0.000000}%
\pgfsetfillcolor{currentfill}%
\pgfsetlinewidth{0.803000pt}%
\definecolor{currentstroke}{rgb}{0.000000,0.000000,0.000000}%
\pgfsetstrokecolor{currentstroke}%
\pgfsetdash{}{0pt}%
\pgfsys@defobject{currentmarker}{\pgfqpoint{0.000000in}{-0.048611in}}{\pgfqpoint{0.000000in}{0.000000in}}{%
\pgfpathmoveto{\pgfqpoint{0.000000in}{0.000000in}}%
\pgfpathlineto{\pgfqpoint{0.000000in}{-0.048611in}}%
\pgfusepath{stroke,fill}%
}%
\begin{pgfscope}%
\pgfsys@transformshift{1.927273in}{0.528000in}%
\pgfsys@useobject{currentmarker}{}%
\end{pgfscope}%
\end{pgfscope}%
\begin{pgfscope}%
\definecolor{textcolor}{rgb}{0.000000,0.000000,0.000000}%
\pgfsetstrokecolor{textcolor}%
\pgfsetfillcolor{textcolor}%
\pgftext[x=1.927273in,y=0.430778in,,top]{\color{textcolor}\rmfamily\fontsize{10.000000}{12.000000}\selectfont \(\displaystyle 1\)}%
\end{pgfscope}%
\begin{pgfscope}%
\pgfsetbuttcap%
\pgfsetroundjoin%
\definecolor{currentfill}{rgb}{0.000000,0.000000,0.000000}%
\pgfsetfillcolor{currentfill}%
\pgfsetlinewidth{0.803000pt}%
\definecolor{currentstroke}{rgb}{0.000000,0.000000,0.000000}%
\pgfsetstrokecolor{currentstroke}%
\pgfsetdash{}{0pt}%
\pgfsys@defobject{currentmarker}{\pgfqpoint{0.000000in}{-0.048611in}}{\pgfqpoint{0.000000in}{0.000000in}}{%
\pgfpathmoveto{\pgfqpoint{0.000000in}{0.000000in}}%
\pgfpathlineto{\pgfqpoint{0.000000in}{-0.048611in}}%
\pgfusepath{stroke,fill}%
}%
\begin{pgfscope}%
\pgfsys@transformshift{2.829091in}{0.528000in}%
\pgfsys@useobject{currentmarker}{}%
\end{pgfscope}%
\end{pgfscope}%
\begin{pgfscope}%
\definecolor{textcolor}{rgb}{0.000000,0.000000,0.000000}%
\pgfsetstrokecolor{textcolor}%
\pgfsetfillcolor{textcolor}%
\pgftext[x=2.829091in,y=0.430778in,,top]{\color{textcolor}\rmfamily\fontsize{10.000000}{12.000000}\selectfont \(\displaystyle 2\)}%
\end{pgfscope}%
\begin{pgfscope}%
\pgfsetbuttcap%
\pgfsetroundjoin%
\definecolor{currentfill}{rgb}{0.000000,0.000000,0.000000}%
\pgfsetfillcolor{currentfill}%
\pgfsetlinewidth{0.803000pt}%
\definecolor{currentstroke}{rgb}{0.000000,0.000000,0.000000}%
\pgfsetstrokecolor{currentstroke}%
\pgfsetdash{}{0pt}%
\pgfsys@defobject{currentmarker}{\pgfqpoint{0.000000in}{-0.048611in}}{\pgfqpoint{0.000000in}{0.000000in}}{%
\pgfpathmoveto{\pgfqpoint{0.000000in}{0.000000in}}%
\pgfpathlineto{\pgfqpoint{0.000000in}{-0.048611in}}%
\pgfusepath{stroke,fill}%
}%
\begin{pgfscope}%
\pgfsys@transformshift{3.730909in}{0.528000in}%
\pgfsys@useobject{currentmarker}{}%
\end{pgfscope}%
\end{pgfscope}%
\begin{pgfscope}%
\definecolor{textcolor}{rgb}{0.000000,0.000000,0.000000}%
\pgfsetstrokecolor{textcolor}%
\pgfsetfillcolor{textcolor}%
\pgftext[x=3.730909in,y=0.430778in,,top]{\color{textcolor}\rmfamily\fontsize{10.000000}{12.000000}\selectfont \(\displaystyle 3\)}%
\end{pgfscope}%
\begin{pgfscope}%
\pgfsetbuttcap%
\pgfsetroundjoin%
\definecolor{currentfill}{rgb}{0.000000,0.000000,0.000000}%
\pgfsetfillcolor{currentfill}%
\pgfsetlinewidth{0.803000pt}%
\definecolor{currentstroke}{rgb}{0.000000,0.000000,0.000000}%
\pgfsetstrokecolor{currentstroke}%
\pgfsetdash{}{0pt}%
\pgfsys@defobject{currentmarker}{\pgfqpoint{0.000000in}{-0.048611in}}{\pgfqpoint{0.000000in}{0.000000in}}{%
\pgfpathmoveto{\pgfqpoint{0.000000in}{0.000000in}}%
\pgfpathlineto{\pgfqpoint{0.000000in}{-0.048611in}}%
\pgfusepath{stroke,fill}%
}%
\begin{pgfscope}%
\pgfsys@transformshift{4.632727in}{0.528000in}%
\pgfsys@useobject{currentmarker}{}%
\end{pgfscope}%
\end{pgfscope}%
\begin{pgfscope}%
\definecolor{textcolor}{rgb}{0.000000,0.000000,0.000000}%
\pgfsetstrokecolor{textcolor}%
\pgfsetfillcolor{textcolor}%
\pgftext[x=4.632727in,y=0.430778in,,top]{\color{textcolor}\rmfamily\fontsize{10.000000}{12.000000}\selectfont \(\displaystyle 4\)}%
\end{pgfscope}%
\begin{pgfscope}%
\pgfsetbuttcap%
\pgfsetroundjoin%
\definecolor{currentfill}{rgb}{0.000000,0.000000,0.000000}%
\pgfsetfillcolor{currentfill}%
\pgfsetlinewidth{0.803000pt}%
\definecolor{currentstroke}{rgb}{0.000000,0.000000,0.000000}%
\pgfsetstrokecolor{currentstroke}%
\pgfsetdash{}{0pt}%
\pgfsys@defobject{currentmarker}{\pgfqpoint{0.000000in}{-0.048611in}}{\pgfqpoint{0.000000in}{0.000000in}}{%
\pgfpathmoveto{\pgfqpoint{0.000000in}{0.000000in}}%
\pgfpathlineto{\pgfqpoint{0.000000in}{-0.048611in}}%
\pgfusepath{stroke,fill}%
}%
\begin{pgfscope}%
\pgfsys@transformshift{5.534545in}{0.528000in}%
\pgfsys@useobject{currentmarker}{}%
\end{pgfscope}%
\end{pgfscope}%
\begin{pgfscope}%
\definecolor{textcolor}{rgb}{0.000000,0.000000,0.000000}%
\pgfsetstrokecolor{textcolor}%
\pgfsetfillcolor{textcolor}%
\pgftext[x=5.534545in,y=0.430778in,,top]{\color{textcolor}\rmfamily\fontsize{10.000000}{12.000000}\selectfont \(\displaystyle 5\)}%
\end{pgfscope}%
\begin{pgfscope}%
\definecolor{textcolor}{rgb}{0.000000,0.000000,0.000000}%
\pgfsetstrokecolor{textcolor}%
\pgfsetfillcolor{textcolor}%
\pgftext[x=3.280000in,y=0.251766in,,top]{\color{textcolor}\rmfamily\fontsize{10.000000}{12.000000}\selectfont t(s)}%
\end{pgfscope}%
\begin{pgfscope}%
\pgfsetbuttcap%
\pgfsetroundjoin%
\definecolor{currentfill}{rgb}{0.000000,0.000000,0.000000}%
\pgfsetfillcolor{currentfill}%
\pgfsetlinewidth{0.803000pt}%
\definecolor{currentstroke}{rgb}{0.000000,0.000000,0.000000}%
\pgfsetstrokecolor{currentstroke}%
\pgfsetdash{}{0pt}%
\pgfsys@defobject{currentmarker}{\pgfqpoint{-0.048611in}{0.000000in}}{\pgfqpoint{0.000000in}{0.000000in}}{%
\pgfpathmoveto{\pgfqpoint{0.000000in}{0.000000in}}%
\pgfpathlineto{\pgfqpoint{-0.048611in}{0.000000in}}%
\pgfusepath{stroke,fill}%
}%
\begin{pgfscope}%
\pgfsys@transformshift{0.800000in}{0.625268in}%
\pgfsys@useobject{currentmarker}{}%
\end{pgfscope}%
\end{pgfscope}%
\begin{pgfscope}%
\definecolor{textcolor}{rgb}{0.000000,0.000000,0.000000}%
\pgfsetstrokecolor{textcolor}%
\pgfsetfillcolor{textcolor}%
\pgftext[x=0.455863in,y=0.577042in,left,base]{\color{textcolor}\rmfamily\fontsize{10.000000}{12.000000}\selectfont \(\displaystyle 0.24\)}%
\end{pgfscope}%
\begin{pgfscope}%
\pgfsetbuttcap%
\pgfsetroundjoin%
\definecolor{currentfill}{rgb}{0.000000,0.000000,0.000000}%
\pgfsetfillcolor{currentfill}%
\pgfsetlinewidth{0.803000pt}%
\definecolor{currentstroke}{rgb}{0.000000,0.000000,0.000000}%
\pgfsetstrokecolor{currentstroke}%
\pgfsetdash{}{0pt}%
\pgfsys@defobject{currentmarker}{\pgfqpoint{-0.048611in}{0.000000in}}{\pgfqpoint{0.000000in}{0.000000in}}{%
\pgfpathmoveto{\pgfqpoint{0.000000in}{0.000000in}}%
\pgfpathlineto{\pgfqpoint{-0.048611in}{0.000000in}}%
\pgfusepath{stroke,fill}%
}%
\begin{pgfscope}%
\pgfsys@transformshift{0.800000in}{1.054109in}%
\pgfsys@useobject{currentmarker}{}%
\end{pgfscope}%
\end{pgfscope}%
\begin{pgfscope}%
\definecolor{textcolor}{rgb}{0.000000,0.000000,0.000000}%
\pgfsetstrokecolor{textcolor}%
\pgfsetfillcolor{textcolor}%
\pgftext[x=0.455863in,y=1.005884in,left,base]{\color{textcolor}\rmfamily\fontsize{10.000000}{12.000000}\selectfont \(\displaystyle 0.25\)}%
\end{pgfscope}%
\begin{pgfscope}%
\pgfsetbuttcap%
\pgfsetroundjoin%
\definecolor{currentfill}{rgb}{0.000000,0.000000,0.000000}%
\pgfsetfillcolor{currentfill}%
\pgfsetlinewidth{0.803000pt}%
\definecolor{currentstroke}{rgb}{0.000000,0.000000,0.000000}%
\pgfsetstrokecolor{currentstroke}%
\pgfsetdash{}{0pt}%
\pgfsys@defobject{currentmarker}{\pgfqpoint{-0.048611in}{0.000000in}}{\pgfqpoint{0.000000in}{0.000000in}}{%
\pgfpathmoveto{\pgfqpoint{0.000000in}{0.000000in}}%
\pgfpathlineto{\pgfqpoint{-0.048611in}{0.000000in}}%
\pgfusepath{stroke,fill}%
}%
\begin{pgfscope}%
\pgfsys@transformshift{0.800000in}{1.482951in}%
\pgfsys@useobject{currentmarker}{}%
\end{pgfscope}%
\end{pgfscope}%
\begin{pgfscope}%
\definecolor{textcolor}{rgb}{0.000000,0.000000,0.000000}%
\pgfsetstrokecolor{textcolor}%
\pgfsetfillcolor{textcolor}%
\pgftext[x=0.455863in,y=1.434725in,left,base]{\color{textcolor}\rmfamily\fontsize{10.000000}{12.000000}\selectfont \(\displaystyle 0.26\)}%
\end{pgfscope}%
\begin{pgfscope}%
\pgfsetbuttcap%
\pgfsetroundjoin%
\definecolor{currentfill}{rgb}{0.000000,0.000000,0.000000}%
\pgfsetfillcolor{currentfill}%
\pgfsetlinewidth{0.803000pt}%
\definecolor{currentstroke}{rgb}{0.000000,0.000000,0.000000}%
\pgfsetstrokecolor{currentstroke}%
\pgfsetdash{}{0pt}%
\pgfsys@defobject{currentmarker}{\pgfqpoint{-0.048611in}{0.000000in}}{\pgfqpoint{0.000000in}{0.000000in}}{%
\pgfpathmoveto{\pgfqpoint{0.000000in}{0.000000in}}%
\pgfpathlineto{\pgfqpoint{-0.048611in}{0.000000in}}%
\pgfusepath{stroke,fill}%
}%
\begin{pgfscope}%
\pgfsys@transformshift{0.800000in}{1.911792in}%
\pgfsys@useobject{currentmarker}{}%
\end{pgfscope}%
\end{pgfscope}%
\begin{pgfscope}%
\definecolor{textcolor}{rgb}{0.000000,0.000000,0.000000}%
\pgfsetstrokecolor{textcolor}%
\pgfsetfillcolor{textcolor}%
\pgftext[x=0.455863in,y=1.863567in,left,base]{\color{textcolor}\rmfamily\fontsize{10.000000}{12.000000}\selectfont \(\displaystyle 0.27\)}%
\end{pgfscope}%
\begin{pgfscope}%
\pgfsetbuttcap%
\pgfsetroundjoin%
\definecolor{currentfill}{rgb}{0.000000,0.000000,0.000000}%
\pgfsetfillcolor{currentfill}%
\pgfsetlinewidth{0.803000pt}%
\definecolor{currentstroke}{rgb}{0.000000,0.000000,0.000000}%
\pgfsetstrokecolor{currentstroke}%
\pgfsetdash{}{0pt}%
\pgfsys@defobject{currentmarker}{\pgfqpoint{-0.048611in}{0.000000in}}{\pgfqpoint{0.000000in}{0.000000in}}{%
\pgfpathmoveto{\pgfqpoint{0.000000in}{0.000000in}}%
\pgfpathlineto{\pgfqpoint{-0.048611in}{0.000000in}}%
\pgfusepath{stroke,fill}%
}%
\begin{pgfscope}%
\pgfsys@transformshift{0.800000in}{2.340634in}%
\pgfsys@useobject{currentmarker}{}%
\end{pgfscope}%
\end{pgfscope}%
\begin{pgfscope}%
\definecolor{textcolor}{rgb}{0.000000,0.000000,0.000000}%
\pgfsetstrokecolor{textcolor}%
\pgfsetfillcolor{textcolor}%
\pgftext[x=0.455863in,y=2.292409in,left,base]{\color{textcolor}\rmfamily\fontsize{10.000000}{12.000000}\selectfont \(\displaystyle 0.28\)}%
\end{pgfscope}%
\begin{pgfscope}%
\pgfsetbuttcap%
\pgfsetroundjoin%
\definecolor{currentfill}{rgb}{0.000000,0.000000,0.000000}%
\pgfsetfillcolor{currentfill}%
\pgfsetlinewidth{0.803000pt}%
\definecolor{currentstroke}{rgb}{0.000000,0.000000,0.000000}%
\pgfsetstrokecolor{currentstroke}%
\pgfsetdash{}{0pt}%
\pgfsys@defobject{currentmarker}{\pgfqpoint{-0.048611in}{0.000000in}}{\pgfqpoint{0.000000in}{0.000000in}}{%
\pgfpathmoveto{\pgfqpoint{0.000000in}{0.000000in}}%
\pgfpathlineto{\pgfqpoint{-0.048611in}{0.000000in}}%
\pgfusepath{stroke,fill}%
}%
\begin{pgfscope}%
\pgfsys@transformshift{0.800000in}{2.769475in}%
\pgfsys@useobject{currentmarker}{}%
\end{pgfscope}%
\end{pgfscope}%
\begin{pgfscope}%
\definecolor{textcolor}{rgb}{0.000000,0.000000,0.000000}%
\pgfsetstrokecolor{textcolor}%
\pgfsetfillcolor{textcolor}%
\pgftext[x=0.455863in,y=2.721250in,left,base]{\color{textcolor}\rmfamily\fontsize{10.000000}{12.000000}\selectfont \(\displaystyle 0.29\)}%
\end{pgfscope}%
\begin{pgfscope}%
\pgfsetbuttcap%
\pgfsetroundjoin%
\definecolor{currentfill}{rgb}{0.000000,0.000000,0.000000}%
\pgfsetfillcolor{currentfill}%
\pgfsetlinewidth{0.803000pt}%
\definecolor{currentstroke}{rgb}{0.000000,0.000000,0.000000}%
\pgfsetstrokecolor{currentstroke}%
\pgfsetdash{}{0pt}%
\pgfsys@defobject{currentmarker}{\pgfqpoint{-0.048611in}{0.000000in}}{\pgfqpoint{0.000000in}{0.000000in}}{%
\pgfpathmoveto{\pgfqpoint{0.000000in}{0.000000in}}%
\pgfpathlineto{\pgfqpoint{-0.048611in}{0.000000in}}%
\pgfusepath{stroke,fill}%
}%
\begin{pgfscope}%
\pgfsys@transformshift{0.800000in}{3.198317in}%
\pgfsys@useobject{currentmarker}{}%
\end{pgfscope}%
\end{pgfscope}%
\begin{pgfscope}%
\definecolor{textcolor}{rgb}{0.000000,0.000000,0.000000}%
\pgfsetstrokecolor{textcolor}%
\pgfsetfillcolor{textcolor}%
\pgftext[x=0.455863in,y=3.150092in,left,base]{\color{textcolor}\rmfamily\fontsize{10.000000}{12.000000}\selectfont \(\displaystyle 0.30\)}%
\end{pgfscope}%
\begin{pgfscope}%
\pgfsetbuttcap%
\pgfsetroundjoin%
\definecolor{currentfill}{rgb}{0.000000,0.000000,0.000000}%
\pgfsetfillcolor{currentfill}%
\pgfsetlinewidth{0.803000pt}%
\definecolor{currentstroke}{rgb}{0.000000,0.000000,0.000000}%
\pgfsetstrokecolor{currentstroke}%
\pgfsetdash{}{0pt}%
\pgfsys@defobject{currentmarker}{\pgfqpoint{-0.048611in}{0.000000in}}{\pgfqpoint{0.000000in}{0.000000in}}{%
\pgfpathmoveto{\pgfqpoint{0.000000in}{0.000000in}}%
\pgfpathlineto{\pgfqpoint{-0.048611in}{0.000000in}}%
\pgfusepath{stroke,fill}%
}%
\begin{pgfscope}%
\pgfsys@transformshift{0.800000in}{3.627158in}%
\pgfsys@useobject{currentmarker}{}%
\end{pgfscope}%
\end{pgfscope}%
\begin{pgfscope}%
\definecolor{textcolor}{rgb}{0.000000,0.000000,0.000000}%
\pgfsetstrokecolor{textcolor}%
\pgfsetfillcolor{textcolor}%
\pgftext[x=0.455863in,y=3.578933in,left,base]{\color{textcolor}\rmfamily\fontsize{10.000000}{12.000000}\selectfont \(\displaystyle 0.31\)}%
\end{pgfscope}%
\begin{pgfscope}%
\pgfsetbuttcap%
\pgfsetroundjoin%
\definecolor{currentfill}{rgb}{0.000000,0.000000,0.000000}%
\pgfsetfillcolor{currentfill}%
\pgfsetlinewidth{0.803000pt}%
\definecolor{currentstroke}{rgb}{0.000000,0.000000,0.000000}%
\pgfsetstrokecolor{currentstroke}%
\pgfsetdash{}{0pt}%
\pgfsys@defobject{currentmarker}{\pgfqpoint{-0.048611in}{0.000000in}}{\pgfqpoint{0.000000in}{0.000000in}}{%
\pgfpathmoveto{\pgfqpoint{0.000000in}{0.000000in}}%
\pgfpathlineto{\pgfqpoint{-0.048611in}{0.000000in}}%
\pgfusepath{stroke,fill}%
}%
\begin{pgfscope}%
\pgfsys@transformshift{0.800000in}{4.056000in}%
\pgfsys@useobject{currentmarker}{}%
\end{pgfscope}%
\end{pgfscope}%
\begin{pgfscope}%
\definecolor{textcolor}{rgb}{0.000000,0.000000,0.000000}%
\pgfsetstrokecolor{textcolor}%
\pgfsetfillcolor{textcolor}%
\pgftext[x=0.455863in,y=4.007775in,left,base]{\color{textcolor}\rmfamily\fontsize{10.000000}{12.000000}\selectfont \(\displaystyle 0.32\)}%
\end{pgfscope}%
\begin{pgfscope}%
\definecolor{textcolor}{rgb}{0.000000,0.000000,0.000000}%
\pgfsetstrokecolor{textcolor}%
\pgfsetfillcolor{textcolor}%
\pgftext[x=0.400308in,y=2.376000in,,bottom,rotate=90.000000]{\color{textcolor}\rmfamily\fontsize{10.000000}{12.000000}\selectfont y(m)}%
\end{pgfscope}%
\begin{pgfscope}%
\pgfpathrectangle{\pgfqpoint{0.800000in}{0.528000in}}{\pgfqpoint{4.960000in}{3.696000in}}%
\pgfusepath{clip}%
\pgfsetbuttcap%
\pgfsetroundjoin%
\pgfsetlinewidth{0.501875pt}%
\definecolor{currentstroke}{rgb}{0.000000,0.000000,1.000000}%
\pgfsetstrokecolor{currentstroke}%
\pgfsetdash{}{0pt}%
\pgfpathmoveto{\pgfqpoint{1.043491in}{3.541390in}}%
\pgfpathlineto{\pgfqpoint{1.043491in}{3.627158in}}%
\pgfusepath{stroke}%
\end{pgfscope}%
\begin{pgfscope}%
\pgfpathrectangle{\pgfqpoint{0.800000in}{0.528000in}}{\pgfqpoint{4.960000in}{3.696000in}}%
\pgfusepath{clip}%
\pgfsetbuttcap%
\pgfsetroundjoin%
\pgfsetlinewidth{0.501875pt}%
\definecolor{currentstroke}{rgb}{0.000000,0.000000,1.000000}%
\pgfsetstrokecolor{currentstroke}%
\pgfsetdash{}{0pt}%
\pgfpathmoveto{\pgfqpoint{1.061527in}{3.412738in}}%
\pgfpathlineto{\pgfqpoint{1.061527in}{3.498506in}}%
\pgfusepath{stroke}%
\end{pgfscope}%
\begin{pgfscope}%
\pgfpathrectangle{\pgfqpoint{0.800000in}{0.528000in}}{\pgfqpoint{4.960000in}{3.696000in}}%
\pgfusepath{clip}%
\pgfsetbuttcap%
\pgfsetroundjoin%
\pgfsetlinewidth{0.501875pt}%
\definecolor{currentstroke}{rgb}{0.000000,0.000000,1.000000}%
\pgfsetstrokecolor{currentstroke}%
\pgfsetdash{}{0pt}%
\pgfpathmoveto{\pgfqpoint{1.079564in}{3.069664in}}%
\pgfpathlineto{\pgfqpoint{1.079564in}{3.155433in}}%
\pgfusepath{stroke}%
\end{pgfscope}%
\begin{pgfscope}%
\pgfpathrectangle{\pgfqpoint{0.800000in}{0.528000in}}{\pgfqpoint{4.960000in}{3.696000in}}%
\pgfusepath{clip}%
\pgfsetbuttcap%
\pgfsetroundjoin%
\pgfsetlinewidth{0.501875pt}%
\definecolor{currentstroke}{rgb}{0.000000,0.000000,1.000000}%
\pgfsetstrokecolor{currentstroke}%
\pgfsetdash{}{0pt}%
\pgfpathmoveto{\pgfqpoint{1.097600in}{2.640823in}}%
\pgfpathlineto{\pgfqpoint{1.097600in}{2.726591in}}%
\pgfusepath{stroke}%
\end{pgfscope}%
\begin{pgfscope}%
\pgfpathrectangle{\pgfqpoint{0.800000in}{0.528000in}}{\pgfqpoint{4.960000in}{3.696000in}}%
\pgfusepath{clip}%
\pgfsetbuttcap%
\pgfsetroundjoin%
\pgfsetlinewidth{0.501875pt}%
\definecolor{currentstroke}{rgb}{0.000000,0.000000,1.000000}%
\pgfsetstrokecolor{currentstroke}%
\pgfsetdash{}{0pt}%
\pgfpathmoveto{\pgfqpoint{1.115636in}{2.169097in}}%
\pgfpathlineto{\pgfqpoint{1.115636in}{2.254865in}}%
\pgfusepath{stroke}%
\end{pgfscope}%
\begin{pgfscope}%
\pgfpathrectangle{\pgfqpoint{0.800000in}{0.528000in}}{\pgfqpoint{4.960000in}{3.696000in}}%
\pgfusepath{clip}%
\pgfsetbuttcap%
\pgfsetroundjoin%
\pgfsetlinewidth{0.501875pt}%
\definecolor{currentstroke}{rgb}{0.000000,0.000000,1.000000}%
\pgfsetstrokecolor{currentstroke}%
\pgfsetdash{}{0pt}%
\pgfpathmoveto{\pgfqpoint{1.133673in}{1.611603in}}%
\pgfpathlineto{\pgfqpoint{1.133673in}{1.697371in}}%
\pgfusepath{stroke}%
\end{pgfscope}%
\begin{pgfscope}%
\pgfpathrectangle{\pgfqpoint{0.800000in}{0.528000in}}{\pgfqpoint{4.960000in}{3.696000in}}%
\pgfusepath{clip}%
\pgfsetbuttcap%
\pgfsetroundjoin%
\pgfsetlinewidth{0.501875pt}%
\definecolor{currentstroke}{rgb}{0.000000,0.000000,1.000000}%
\pgfsetstrokecolor{currentstroke}%
\pgfsetdash{}{0pt}%
\pgfpathmoveto{\pgfqpoint{1.151709in}{1.311414in}}%
\pgfpathlineto{\pgfqpoint{1.151709in}{1.397182in}}%
\pgfusepath{stroke}%
\end{pgfscope}%
\begin{pgfscope}%
\pgfpathrectangle{\pgfqpoint{0.800000in}{0.528000in}}{\pgfqpoint{4.960000in}{3.696000in}}%
\pgfusepath{clip}%
\pgfsetbuttcap%
\pgfsetroundjoin%
\pgfsetlinewidth{0.501875pt}%
\definecolor{currentstroke}{rgb}{0.000000,0.000000,1.000000}%
\pgfsetstrokecolor{currentstroke}%
\pgfsetdash{}{0pt}%
\pgfpathmoveto{\pgfqpoint{1.169745in}{1.225646in}}%
\pgfpathlineto{\pgfqpoint{1.169745in}{1.311414in}}%
\pgfusepath{stroke}%
\end{pgfscope}%
\begin{pgfscope}%
\pgfpathrectangle{\pgfqpoint{0.800000in}{0.528000in}}{\pgfqpoint{4.960000in}{3.696000in}}%
\pgfusepath{clip}%
\pgfsetbuttcap%
\pgfsetroundjoin%
\pgfsetlinewidth{0.501875pt}%
\definecolor{currentstroke}{rgb}{0.000000,0.000000,1.000000}%
\pgfsetstrokecolor{currentstroke}%
\pgfsetdash{}{0pt}%
\pgfpathmoveto{\pgfqpoint{1.187782in}{1.311414in}}%
\pgfpathlineto{\pgfqpoint{1.187782in}{1.397182in}}%
\pgfusepath{stroke}%
\end{pgfscope}%
\begin{pgfscope}%
\pgfpathrectangle{\pgfqpoint{0.800000in}{0.528000in}}{\pgfqpoint{4.960000in}{3.696000in}}%
\pgfusepath{clip}%
\pgfsetbuttcap%
\pgfsetroundjoin%
\pgfsetlinewidth{0.501875pt}%
\definecolor{currentstroke}{rgb}{0.000000,0.000000,1.000000}%
\pgfsetstrokecolor{currentstroke}%
\pgfsetdash{}{0pt}%
\pgfpathmoveto{\pgfqpoint{1.205818in}{1.611603in}}%
\pgfpathlineto{\pgfqpoint{1.205818in}{1.697371in}}%
\pgfusepath{stroke}%
\end{pgfscope}%
\begin{pgfscope}%
\pgfpathrectangle{\pgfqpoint{0.800000in}{0.528000in}}{\pgfqpoint{4.960000in}{3.696000in}}%
\pgfusepath{clip}%
\pgfsetbuttcap%
\pgfsetroundjoin%
\pgfsetlinewidth{0.501875pt}%
\definecolor{currentstroke}{rgb}{0.000000,0.000000,1.000000}%
\pgfsetstrokecolor{currentstroke}%
\pgfsetdash{}{0pt}%
\pgfpathmoveto{\pgfqpoint{1.223855in}{2.169097in}}%
\pgfpathlineto{\pgfqpoint{1.223855in}{2.254865in}}%
\pgfusepath{stroke}%
\end{pgfscope}%
\begin{pgfscope}%
\pgfpathrectangle{\pgfqpoint{0.800000in}{0.528000in}}{\pgfqpoint{4.960000in}{3.696000in}}%
\pgfusepath{clip}%
\pgfsetbuttcap%
\pgfsetroundjoin%
\pgfsetlinewidth{0.501875pt}%
\definecolor{currentstroke}{rgb}{0.000000,0.000000,1.000000}%
\pgfsetstrokecolor{currentstroke}%
\pgfsetdash{}{0pt}%
\pgfpathmoveto{\pgfqpoint{1.241891in}{2.597939in}}%
\pgfpathlineto{\pgfqpoint{1.241891in}{2.683707in}}%
\pgfusepath{stroke}%
\end{pgfscope}%
\begin{pgfscope}%
\pgfpathrectangle{\pgfqpoint{0.800000in}{0.528000in}}{\pgfqpoint{4.960000in}{3.696000in}}%
\pgfusepath{clip}%
\pgfsetbuttcap%
\pgfsetroundjoin%
\pgfsetlinewidth{0.501875pt}%
\definecolor{currentstroke}{rgb}{0.000000,0.000000,1.000000}%
\pgfsetstrokecolor{currentstroke}%
\pgfsetdash{}{0pt}%
\pgfpathmoveto{\pgfqpoint{1.259927in}{3.026780in}}%
\pgfpathlineto{\pgfqpoint{1.259927in}{3.112549in}}%
\pgfusepath{stroke}%
\end{pgfscope}%
\begin{pgfscope}%
\pgfpathrectangle{\pgfqpoint{0.800000in}{0.528000in}}{\pgfqpoint{4.960000in}{3.696000in}}%
\pgfusepath{clip}%
\pgfsetbuttcap%
\pgfsetroundjoin%
\pgfsetlinewidth{0.501875pt}%
\definecolor{currentstroke}{rgb}{0.000000,0.000000,1.000000}%
\pgfsetstrokecolor{currentstroke}%
\pgfsetdash{}{0pt}%
\pgfpathmoveto{\pgfqpoint{1.277964in}{3.326969in}}%
\pgfpathlineto{\pgfqpoint{1.277964in}{3.412738in}}%
\pgfusepath{stroke}%
\end{pgfscope}%
\begin{pgfscope}%
\pgfpathrectangle{\pgfqpoint{0.800000in}{0.528000in}}{\pgfqpoint{4.960000in}{3.696000in}}%
\pgfusepath{clip}%
\pgfsetbuttcap%
\pgfsetroundjoin%
\pgfsetlinewidth{0.501875pt}%
\definecolor{currentstroke}{rgb}{0.000000,0.000000,1.000000}%
\pgfsetstrokecolor{currentstroke}%
\pgfsetdash{}{0pt}%
\pgfpathmoveto{\pgfqpoint{1.296000in}{3.455622in}}%
\pgfpathlineto{\pgfqpoint{1.296000in}{3.541390in}}%
\pgfusepath{stroke}%
\end{pgfscope}%
\begin{pgfscope}%
\pgfpathrectangle{\pgfqpoint{0.800000in}{0.528000in}}{\pgfqpoint{4.960000in}{3.696000in}}%
\pgfusepath{clip}%
\pgfsetbuttcap%
\pgfsetroundjoin%
\pgfsetlinewidth{0.501875pt}%
\definecolor{currentstroke}{rgb}{0.000000,0.000000,1.000000}%
\pgfsetstrokecolor{currentstroke}%
\pgfsetdash{}{0pt}%
\pgfpathmoveto{\pgfqpoint{1.314036in}{3.455622in}}%
\pgfpathlineto{\pgfqpoint{1.314036in}{3.541390in}}%
\pgfusepath{stroke}%
\end{pgfscope}%
\begin{pgfscope}%
\pgfpathrectangle{\pgfqpoint{0.800000in}{0.528000in}}{\pgfqpoint{4.960000in}{3.696000in}}%
\pgfusepath{clip}%
\pgfsetbuttcap%
\pgfsetroundjoin%
\pgfsetlinewidth{0.501875pt}%
\definecolor{currentstroke}{rgb}{0.000000,0.000000,1.000000}%
\pgfsetstrokecolor{currentstroke}%
\pgfsetdash{}{0pt}%
\pgfpathmoveto{\pgfqpoint{1.332073in}{3.241201in}}%
\pgfpathlineto{\pgfqpoint{1.332073in}{3.326969in}}%
\pgfusepath{stroke}%
\end{pgfscope}%
\begin{pgfscope}%
\pgfpathrectangle{\pgfqpoint{0.800000in}{0.528000in}}{\pgfqpoint{4.960000in}{3.696000in}}%
\pgfusepath{clip}%
\pgfsetbuttcap%
\pgfsetroundjoin%
\pgfsetlinewidth{0.501875pt}%
\definecolor{currentstroke}{rgb}{0.000000,0.000000,1.000000}%
\pgfsetstrokecolor{currentstroke}%
\pgfsetdash{}{0pt}%
\pgfpathmoveto{\pgfqpoint{1.350109in}{2.898128in}}%
\pgfpathlineto{\pgfqpoint{1.350109in}{2.983896in}}%
\pgfusepath{stroke}%
\end{pgfscope}%
\begin{pgfscope}%
\pgfpathrectangle{\pgfqpoint{0.800000in}{0.528000in}}{\pgfqpoint{4.960000in}{3.696000in}}%
\pgfusepath{clip}%
\pgfsetbuttcap%
\pgfsetroundjoin%
\pgfsetlinewidth{0.501875pt}%
\definecolor{currentstroke}{rgb}{0.000000,0.000000,1.000000}%
\pgfsetstrokecolor{currentstroke}%
\pgfsetdash{}{0pt}%
\pgfpathmoveto{\pgfqpoint{1.368145in}{2.469286in}}%
\pgfpathlineto{\pgfqpoint{1.368145in}{2.555055in}}%
\pgfusepath{stroke}%
\end{pgfscope}%
\begin{pgfscope}%
\pgfpathrectangle{\pgfqpoint{0.800000in}{0.528000in}}{\pgfqpoint{4.960000in}{3.696000in}}%
\pgfusepath{clip}%
\pgfsetbuttcap%
\pgfsetroundjoin%
\pgfsetlinewidth{0.501875pt}%
\definecolor{currentstroke}{rgb}{0.000000,0.000000,1.000000}%
\pgfsetstrokecolor{currentstroke}%
\pgfsetdash{}{0pt}%
\pgfpathmoveto{\pgfqpoint{1.386182in}{2.083329in}}%
\pgfpathlineto{\pgfqpoint{1.386182in}{2.169097in}}%
\pgfusepath{stroke}%
\end{pgfscope}%
\begin{pgfscope}%
\pgfpathrectangle{\pgfqpoint{0.800000in}{0.528000in}}{\pgfqpoint{4.960000in}{3.696000in}}%
\pgfusepath{clip}%
\pgfsetbuttcap%
\pgfsetroundjoin%
\pgfsetlinewidth{0.501875pt}%
\definecolor{currentstroke}{rgb}{0.000000,0.000000,1.000000}%
\pgfsetstrokecolor{currentstroke}%
\pgfsetdash{}{0pt}%
\pgfpathmoveto{\pgfqpoint{1.404218in}{1.568719in}}%
\pgfpathlineto{\pgfqpoint{1.404218in}{1.654487in}}%
\pgfusepath{stroke}%
\end{pgfscope}%
\begin{pgfscope}%
\pgfpathrectangle{\pgfqpoint{0.800000in}{0.528000in}}{\pgfqpoint{4.960000in}{3.696000in}}%
\pgfusepath{clip}%
\pgfsetbuttcap%
\pgfsetroundjoin%
\pgfsetlinewidth{0.501875pt}%
\definecolor{currentstroke}{rgb}{0.000000,0.000000,1.000000}%
\pgfsetstrokecolor{currentstroke}%
\pgfsetdash{}{0pt}%
\pgfpathmoveto{\pgfqpoint{1.422255in}{1.354298in}}%
\pgfpathlineto{\pgfqpoint{1.422255in}{1.440067in}}%
\pgfusepath{stroke}%
\end{pgfscope}%
\begin{pgfscope}%
\pgfpathrectangle{\pgfqpoint{0.800000in}{0.528000in}}{\pgfqpoint{4.960000in}{3.696000in}}%
\pgfusepath{clip}%
\pgfsetbuttcap%
\pgfsetroundjoin%
\pgfsetlinewidth{0.501875pt}%
\definecolor{currentstroke}{rgb}{0.000000,0.000000,1.000000}%
\pgfsetstrokecolor{currentstroke}%
\pgfsetdash{}{0pt}%
\pgfpathmoveto{\pgfqpoint{1.440291in}{1.354298in}}%
\pgfpathlineto{\pgfqpoint{1.440291in}{1.440067in}}%
\pgfusepath{stroke}%
\end{pgfscope}%
\begin{pgfscope}%
\pgfpathrectangle{\pgfqpoint{0.800000in}{0.528000in}}{\pgfqpoint{4.960000in}{3.696000in}}%
\pgfusepath{clip}%
\pgfsetbuttcap%
\pgfsetroundjoin%
\pgfsetlinewidth{0.501875pt}%
\definecolor{currentstroke}{rgb}{0.000000,0.000000,1.000000}%
\pgfsetstrokecolor{currentstroke}%
\pgfsetdash{}{0pt}%
\pgfpathmoveto{\pgfqpoint{1.458327in}{1.482951in}}%
\pgfpathlineto{\pgfqpoint{1.458327in}{1.568719in}}%
\pgfusepath{stroke}%
\end{pgfscope}%
\begin{pgfscope}%
\pgfpathrectangle{\pgfqpoint{0.800000in}{0.528000in}}{\pgfqpoint{4.960000in}{3.696000in}}%
\pgfusepath{clip}%
\pgfsetbuttcap%
\pgfsetroundjoin%
\pgfsetlinewidth{0.501875pt}%
\definecolor{currentstroke}{rgb}{0.000000,0.000000,1.000000}%
\pgfsetstrokecolor{currentstroke}%
\pgfsetdash{}{0pt}%
\pgfpathmoveto{\pgfqpoint{1.476364in}{1.783140in}}%
\pgfpathlineto{\pgfqpoint{1.476364in}{1.868908in}}%
\pgfusepath{stroke}%
\end{pgfscope}%
\begin{pgfscope}%
\pgfpathrectangle{\pgfqpoint{0.800000in}{0.528000in}}{\pgfqpoint{4.960000in}{3.696000in}}%
\pgfusepath{clip}%
\pgfsetbuttcap%
\pgfsetroundjoin%
\pgfsetlinewidth{0.501875pt}%
\definecolor{currentstroke}{rgb}{0.000000,0.000000,1.000000}%
\pgfsetstrokecolor{currentstroke}%
\pgfsetdash{}{0pt}%
\pgfpathmoveto{\pgfqpoint{1.494400in}{2.297750in}}%
\pgfpathlineto{\pgfqpoint{1.494400in}{2.383518in}}%
\pgfusepath{stroke}%
\end{pgfscope}%
\begin{pgfscope}%
\pgfpathrectangle{\pgfqpoint{0.800000in}{0.528000in}}{\pgfqpoint{4.960000in}{3.696000in}}%
\pgfusepath{clip}%
\pgfsetbuttcap%
\pgfsetroundjoin%
\pgfsetlinewidth{0.501875pt}%
\definecolor{currentstroke}{rgb}{0.000000,0.000000,1.000000}%
\pgfsetstrokecolor{currentstroke}%
\pgfsetdash{}{0pt}%
\pgfpathmoveto{\pgfqpoint{1.512436in}{2.726591in}}%
\pgfpathlineto{\pgfqpoint{1.512436in}{2.812359in}}%
\pgfusepath{stroke}%
\end{pgfscope}%
\begin{pgfscope}%
\pgfpathrectangle{\pgfqpoint{0.800000in}{0.528000in}}{\pgfqpoint{4.960000in}{3.696000in}}%
\pgfusepath{clip}%
\pgfsetbuttcap%
\pgfsetroundjoin%
\pgfsetlinewidth{0.501875pt}%
\definecolor{currentstroke}{rgb}{0.000000,0.000000,1.000000}%
\pgfsetstrokecolor{currentstroke}%
\pgfsetdash{}{0pt}%
\pgfpathmoveto{\pgfqpoint{1.530473in}{3.069664in}}%
\pgfpathlineto{\pgfqpoint{1.530473in}{3.155433in}}%
\pgfusepath{stroke}%
\end{pgfscope}%
\begin{pgfscope}%
\pgfpathrectangle{\pgfqpoint{0.800000in}{0.528000in}}{\pgfqpoint{4.960000in}{3.696000in}}%
\pgfusepath{clip}%
\pgfsetbuttcap%
\pgfsetroundjoin%
\pgfsetlinewidth{0.501875pt}%
\definecolor{currentstroke}{rgb}{0.000000,0.000000,1.000000}%
\pgfsetstrokecolor{currentstroke}%
\pgfsetdash{}{0pt}%
\pgfpathmoveto{\pgfqpoint{1.548509in}{3.284085in}}%
\pgfpathlineto{\pgfqpoint{1.548509in}{3.369854in}}%
\pgfusepath{stroke}%
\end{pgfscope}%
\begin{pgfscope}%
\pgfpathrectangle{\pgfqpoint{0.800000in}{0.528000in}}{\pgfqpoint{4.960000in}{3.696000in}}%
\pgfusepath{clip}%
\pgfsetbuttcap%
\pgfsetroundjoin%
\pgfsetlinewidth{0.501875pt}%
\definecolor{currentstroke}{rgb}{0.000000,0.000000,1.000000}%
\pgfsetstrokecolor{currentstroke}%
\pgfsetdash{}{0pt}%
\pgfpathmoveto{\pgfqpoint{1.566545in}{3.369854in}}%
\pgfpathlineto{\pgfqpoint{1.566545in}{3.455622in}}%
\pgfusepath{stroke}%
\end{pgfscope}%
\begin{pgfscope}%
\pgfpathrectangle{\pgfqpoint{0.800000in}{0.528000in}}{\pgfqpoint{4.960000in}{3.696000in}}%
\pgfusepath{clip}%
\pgfsetbuttcap%
\pgfsetroundjoin%
\pgfsetlinewidth{0.501875pt}%
\definecolor{currentstroke}{rgb}{0.000000,0.000000,1.000000}%
\pgfsetstrokecolor{currentstroke}%
\pgfsetdash{}{0pt}%
\pgfpathmoveto{\pgfqpoint{1.584582in}{3.326969in}}%
\pgfpathlineto{\pgfqpoint{1.584582in}{3.412738in}}%
\pgfusepath{stroke}%
\end{pgfscope}%
\begin{pgfscope}%
\pgfpathrectangle{\pgfqpoint{0.800000in}{0.528000in}}{\pgfqpoint{4.960000in}{3.696000in}}%
\pgfusepath{clip}%
\pgfsetbuttcap%
\pgfsetroundjoin%
\pgfsetlinewidth{0.501875pt}%
\definecolor{currentstroke}{rgb}{0.000000,0.000000,1.000000}%
\pgfsetstrokecolor{currentstroke}%
\pgfsetdash{}{0pt}%
\pgfpathmoveto{\pgfqpoint{1.602618in}{3.069664in}}%
\pgfpathlineto{\pgfqpoint{1.602618in}{3.155433in}}%
\pgfusepath{stroke}%
\end{pgfscope}%
\begin{pgfscope}%
\pgfpathrectangle{\pgfqpoint{0.800000in}{0.528000in}}{\pgfqpoint{4.960000in}{3.696000in}}%
\pgfusepath{clip}%
\pgfsetbuttcap%
\pgfsetroundjoin%
\pgfsetlinewidth{0.501875pt}%
\definecolor{currentstroke}{rgb}{0.000000,0.000000,1.000000}%
\pgfsetstrokecolor{currentstroke}%
\pgfsetdash{}{0pt}%
\pgfpathmoveto{\pgfqpoint{1.620655in}{2.726591in}}%
\pgfpathlineto{\pgfqpoint{1.620655in}{2.812359in}}%
\pgfusepath{stroke}%
\end{pgfscope}%
\begin{pgfscope}%
\pgfpathrectangle{\pgfqpoint{0.800000in}{0.528000in}}{\pgfqpoint{4.960000in}{3.696000in}}%
\pgfusepath{clip}%
\pgfsetbuttcap%
\pgfsetroundjoin%
\pgfsetlinewidth{0.501875pt}%
\definecolor{currentstroke}{rgb}{0.000000,0.000000,1.000000}%
\pgfsetstrokecolor{currentstroke}%
\pgfsetdash{}{0pt}%
\pgfpathmoveto{\pgfqpoint{1.638691in}{2.340634in}}%
\pgfpathlineto{\pgfqpoint{1.638691in}{2.426402in}}%
\pgfusepath{stroke}%
\end{pgfscope}%
\begin{pgfscope}%
\pgfpathrectangle{\pgfqpoint{0.800000in}{0.528000in}}{\pgfqpoint{4.960000in}{3.696000in}}%
\pgfusepath{clip}%
\pgfsetbuttcap%
\pgfsetroundjoin%
\pgfsetlinewidth{0.501875pt}%
\definecolor{currentstroke}{rgb}{0.000000,0.000000,1.000000}%
\pgfsetstrokecolor{currentstroke}%
\pgfsetdash{}{0pt}%
\pgfpathmoveto{\pgfqpoint{1.656727in}{1.868908in}}%
\pgfpathlineto{\pgfqpoint{1.656727in}{1.954676in}}%
\pgfusepath{stroke}%
\end{pgfscope}%
\begin{pgfscope}%
\pgfpathrectangle{\pgfqpoint{0.800000in}{0.528000in}}{\pgfqpoint{4.960000in}{3.696000in}}%
\pgfusepath{clip}%
\pgfsetbuttcap%
\pgfsetroundjoin%
\pgfsetlinewidth{0.501875pt}%
\definecolor{currentstroke}{rgb}{0.000000,0.000000,1.000000}%
\pgfsetstrokecolor{currentstroke}%
\pgfsetdash{}{0pt}%
\pgfpathmoveto{\pgfqpoint{1.674764in}{1.568719in}}%
\pgfpathlineto{\pgfqpoint{1.674764in}{1.654487in}}%
\pgfusepath{stroke}%
\end{pgfscope}%
\begin{pgfscope}%
\pgfpathrectangle{\pgfqpoint{0.800000in}{0.528000in}}{\pgfqpoint{4.960000in}{3.696000in}}%
\pgfusepath{clip}%
\pgfsetbuttcap%
\pgfsetroundjoin%
\pgfsetlinewidth{0.501875pt}%
\definecolor{currentstroke}{rgb}{0.000000,0.000000,1.000000}%
\pgfsetstrokecolor{currentstroke}%
\pgfsetdash{}{0pt}%
\pgfpathmoveto{\pgfqpoint{1.692800in}{1.440067in}}%
\pgfpathlineto{\pgfqpoint{1.692800in}{1.525835in}}%
\pgfusepath{stroke}%
\end{pgfscope}%
\begin{pgfscope}%
\pgfpathrectangle{\pgfqpoint{0.800000in}{0.528000in}}{\pgfqpoint{4.960000in}{3.696000in}}%
\pgfusepath{clip}%
\pgfsetbuttcap%
\pgfsetroundjoin%
\pgfsetlinewidth{0.501875pt}%
\definecolor{currentstroke}{rgb}{0.000000,0.000000,1.000000}%
\pgfsetstrokecolor{currentstroke}%
\pgfsetdash{}{0pt}%
\pgfpathmoveto{\pgfqpoint{1.710836in}{1.440067in}}%
\pgfpathlineto{\pgfqpoint{1.710836in}{1.525835in}}%
\pgfusepath{stroke}%
\end{pgfscope}%
\begin{pgfscope}%
\pgfpathrectangle{\pgfqpoint{0.800000in}{0.528000in}}{\pgfqpoint{4.960000in}{3.696000in}}%
\pgfusepath{clip}%
\pgfsetbuttcap%
\pgfsetroundjoin%
\pgfsetlinewidth{0.501875pt}%
\definecolor{currentstroke}{rgb}{0.000000,0.000000,1.000000}%
\pgfsetstrokecolor{currentstroke}%
\pgfsetdash{}{0pt}%
\pgfpathmoveto{\pgfqpoint{1.728873in}{1.654487in}}%
\pgfpathlineto{\pgfqpoint{1.728873in}{1.740256in}}%
\pgfusepath{stroke}%
\end{pgfscope}%
\begin{pgfscope}%
\pgfpathrectangle{\pgfqpoint{0.800000in}{0.528000in}}{\pgfqpoint{4.960000in}{3.696000in}}%
\pgfusepath{clip}%
\pgfsetbuttcap%
\pgfsetroundjoin%
\pgfsetlinewidth{0.501875pt}%
\definecolor{currentstroke}{rgb}{0.000000,0.000000,1.000000}%
\pgfsetstrokecolor{currentstroke}%
\pgfsetdash{}{0pt}%
\pgfpathmoveto{\pgfqpoint{1.746909in}{2.083329in}}%
\pgfpathlineto{\pgfqpoint{1.746909in}{2.169097in}}%
\pgfusepath{stroke}%
\end{pgfscope}%
\begin{pgfscope}%
\pgfpathrectangle{\pgfqpoint{0.800000in}{0.528000in}}{\pgfqpoint{4.960000in}{3.696000in}}%
\pgfusepath{clip}%
\pgfsetbuttcap%
\pgfsetroundjoin%
\pgfsetlinewidth{0.501875pt}%
\definecolor{currentstroke}{rgb}{0.000000,0.000000,1.000000}%
\pgfsetstrokecolor{currentstroke}%
\pgfsetdash{}{0pt}%
\pgfpathmoveto{\pgfqpoint{1.764945in}{2.469286in}}%
\pgfpathlineto{\pgfqpoint{1.764945in}{2.555055in}}%
\pgfusepath{stroke}%
\end{pgfscope}%
\begin{pgfscope}%
\pgfpathrectangle{\pgfqpoint{0.800000in}{0.528000in}}{\pgfqpoint{4.960000in}{3.696000in}}%
\pgfusepath{clip}%
\pgfsetbuttcap%
\pgfsetroundjoin%
\pgfsetlinewidth{0.501875pt}%
\definecolor{currentstroke}{rgb}{0.000000,0.000000,1.000000}%
\pgfsetstrokecolor{currentstroke}%
\pgfsetdash{}{0pt}%
\pgfpathmoveto{\pgfqpoint{1.782982in}{2.812359in}}%
\pgfpathlineto{\pgfqpoint{1.782982in}{2.898128in}}%
\pgfusepath{stroke}%
\end{pgfscope}%
\begin{pgfscope}%
\pgfpathrectangle{\pgfqpoint{0.800000in}{0.528000in}}{\pgfqpoint{4.960000in}{3.696000in}}%
\pgfusepath{clip}%
\pgfsetbuttcap%
\pgfsetroundjoin%
\pgfsetlinewidth{0.501875pt}%
\definecolor{currentstroke}{rgb}{0.000000,0.000000,1.000000}%
\pgfsetstrokecolor{currentstroke}%
\pgfsetdash{}{0pt}%
\pgfpathmoveto{\pgfqpoint{1.801018in}{3.112549in}}%
\pgfpathlineto{\pgfqpoint{1.801018in}{3.198317in}}%
\pgfusepath{stroke}%
\end{pgfscope}%
\begin{pgfscope}%
\pgfpathrectangle{\pgfqpoint{0.800000in}{0.528000in}}{\pgfqpoint{4.960000in}{3.696000in}}%
\pgfusepath{clip}%
\pgfsetbuttcap%
\pgfsetroundjoin%
\pgfsetlinewidth{0.501875pt}%
\definecolor{currentstroke}{rgb}{0.000000,0.000000,1.000000}%
\pgfsetstrokecolor{currentstroke}%
\pgfsetdash{}{0pt}%
\pgfpathmoveto{\pgfqpoint{1.819055in}{3.284085in}}%
\pgfpathlineto{\pgfqpoint{1.819055in}{3.369854in}}%
\pgfusepath{stroke}%
\end{pgfscope}%
\begin{pgfscope}%
\pgfpathrectangle{\pgfqpoint{0.800000in}{0.528000in}}{\pgfqpoint{4.960000in}{3.696000in}}%
\pgfusepath{clip}%
\pgfsetbuttcap%
\pgfsetroundjoin%
\pgfsetlinewidth{0.501875pt}%
\definecolor{currentstroke}{rgb}{0.000000,0.000000,1.000000}%
\pgfsetstrokecolor{currentstroke}%
\pgfsetdash{}{0pt}%
\pgfpathmoveto{\pgfqpoint{1.837091in}{3.284085in}}%
\pgfpathlineto{\pgfqpoint{1.837091in}{3.369854in}}%
\pgfusepath{stroke}%
\end{pgfscope}%
\begin{pgfscope}%
\pgfpathrectangle{\pgfqpoint{0.800000in}{0.528000in}}{\pgfqpoint{4.960000in}{3.696000in}}%
\pgfusepath{clip}%
\pgfsetbuttcap%
\pgfsetroundjoin%
\pgfsetlinewidth{0.501875pt}%
\definecolor{currentstroke}{rgb}{0.000000,0.000000,1.000000}%
\pgfsetstrokecolor{currentstroke}%
\pgfsetdash{}{0pt}%
\pgfpathmoveto{\pgfqpoint{1.855127in}{3.198317in}}%
\pgfpathlineto{\pgfqpoint{1.855127in}{3.284085in}}%
\pgfusepath{stroke}%
\end{pgfscope}%
\begin{pgfscope}%
\pgfpathrectangle{\pgfqpoint{0.800000in}{0.528000in}}{\pgfqpoint{4.960000in}{3.696000in}}%
\pgfusepath{clip}%
\pgfsetbuttcap%
\pgfsetroundjoin%
\pgfsetlinewidth{0.501875pt}%
\definecolor{currentstroke}{rgb}{0.000000,0.000000,1.000000}%
\pgfsetstrokecolor{currentstroke}%
\pgfsetdash{}{0pt}%
\pgfpathmoveto{\pgfqpoint{1.873164in}{2.941012in}}%
\pgfpathlineto{\pgfqpoint{1.873164in}{3.026780in}}%
\pgfusepath{stroke}%
\end{pgfscope}%
\begin{pgfscope}%
\pgfpathrectangle{\pgfqpoint{0.800000in}{0.528000in}}{\pgfqpoint{4.960000in}{3.696000in}}%
\pgfusepath{clip}%
\pgfsetbuttcap%
\pgfsetroundjoin%
\pgfsetlinewidth{0.501875pt}%
\definecolor{currentstroke}{rgb}{0.000000,0.000000,1.000000}%
\pgfsetstrokecolor{currentstroke}%
\pgfsetdash{}{0pt}%
\pgfpathmoveto{\pgfqpoint{1.891200in}{2.597939in}}%
\pgfpathlineto{\pgfqpoint{1.891200in}{2.683707in}}%
\pgfusepath{stroke}%
\end{pgfscope}%
\begin{pgfscope}%
\pgfpathrectangle{\pgfqpoint{0.800000in}{0.528000in}}{\pgfqpoint{4.960000in}{3.696000in}}%
\pgfusepath{clip}%
\pgfsetbuttcap%
\pgfsetroundjoin%
\pgfsetlinewidth{0.501875pt}%
\definecolor{currentstroke}{rgb}{0.000000,0.000000,1.000000}%
\pgfsetstrokecolor{currentstroke}%
\pgfsetdash{}{0pt}%
\pgfpathmoveto{\pgfqpoint{1.909236in}{2.254865in}}%
\pgfpathlineto{\pgfqpoint{1.909236in}{2.340634in}}%
\pgfusepath{stroke}%
\end{pgfscope}%
\begin{pgfscope}%
\pgfpathrectangle{\pgfqpoint{0.800000in}{0.528000in}}{\pgfqpoint{4.960000in}{3.696000in}}%
\pgfusepath{clip}%
\pgfsetbuttcap%
\pgfsetroundjoin%
\pgfsetlinewidth{0.501875pt}%
\definecolor{currentstroke}{rgb}{0.000000,0.000000,1.000000}%
\pgfsetstrokecolor{currentstroke}%
\pgfsetdash{}{0pt}%
\pgfpathmoveto{\pgfqpoint{1.927273in}{1.783140in}}%
\pgfpathlineto{\pgfqpoint{1.927273in}{1.868908in}}%
\pgfusepath{stroke}%
\end{pgfscope}%
\begin{pgfscope}%
\pgfpathrectangle{\pgfqpoint{0.800000in}{0.528000in}}{\pgfqpoint{4.960000in}{3.696000in}}%
\pgfusepath{clip}%
\pgfsetbuttcap%
\pgfsetroundjoin%
\pgfsetlinewidth{0.501875pt}%
\definecolor{currentstroke}{rgb}{0.000000,0.000000,1.000000}%
\pgfsetstrokecolor{currentstroke}%
\pgfsetdash{}{0pt}%
\pgfpathmoveto{\pgfqpoint{1.945309in}{1.568719in}}%
\pgfpathlineto{\pgfqpoint{1.945309in}{1.654487in}}%
\pgfusepath{stroke}%
\end{pgfscope}%
\begin{pgfscope}%
\pgfpathrectangle{\pgfqpoint{0.800000in}{0.528000in}}{\pgfqpoint{4.960000in}{3.696000in}}%
\pgfusepath{clip}%
\pgfsetbuttcap%
\pgfsetroundjoin%
\pgfsetlinewidth{0.501875pt}%
\definecolor{currentstroke}{rgb}{0.000000,0.000000,1.000000}%
\pgfsetstrokecolor{currentstroke}%
\pgfsetdash{}{0pt}%
\pgfpathmoveto{\pgfqpoint{1.963345in}{1.482951in}}%
\pgfpathlineto{\pgfqpoint{1.963345in}{1.568719in}}%
\pgfusepath{stroke}%
\end{pgfscope}%
\begin{pgfscope}%
\pgfpathrectangle{\pgfqpoint{0.800000in}{0.528000in}}{\pgfqpoint{4.960000in}{3.696000in}}%
\pgfusepath{clip}%
\pgfsetbuttcap%
\pgfsetroundjoin%
\pgfsetlinewidth{0.501875pt}%
\definecolor{currentstroke}{rgb}{0.000000,0.000000,1.000000}%
\pgfsetstrokecolor{currentstroke}%
\pgfsetdash{}{0pt}%
\pgfpathmoveto{\pgfqpoint{1.981382in}{1.568719in}}%
\pgfpathlineto{\pgfqpoint{1.981382in}{1.654487in}}%
\pgfusepath{stroke}%
\end{pgfscope}%
\begin{pgfscope}%
\pgfpathrectangle{\pgfqpoint{0.800000in}{0.528000in}}{\pgfqpoint{4.960000in}{3.696000in}}%
\pgfusepath{clip}%
\pgfsetbuttcap%
\pgfsetroundjoin%
\pgfsetlinewidth{0.501875pt}%
\definecolor{currentstroke}{rgb}{0.000000,0.000000,1.000000}%
\pgfsetstrokecolor{currentstroke}%
\pgfsetdash{}{0pt}%
\pgfpathmoveto{\pgfqpoint{1.999418in}{1.783140in}}%
\pgfpathlineto{\pgfqpoint{1.999418in}{1.868908in}}%
\pgfusepath{stroke}%
\end{pgfscope}%
\begin{pgfscope}%
\pgfpathrectangle{\pgfqpoint{0.800000in}{0.528000in}}{\pgfqpoint{4.960000in}{3.696000in}}%
\pgfusepath{clip}%
\pgfsetbuttcap%
\pgfsetroundjoin%
\pgfsetlinewidth{0.501875pt}%
\definecolor{currentstroke}{rgb}{0.000000,0.000000,1.000000}%
\pgfsetstrokecolor{currentstroke}%
\pgfsetdash{}{0pt}%
\pgfpathmoveto{\pgfqpoint{2.017455in}{2.211981in}}%
\pgfpathlineto{\pgfqpoint{2.017455in}{2.297750in}}%
\pgfusepath{stroke}%
\end{pgfscope}%
\begin{pgfscope}%
\pgfpathrectangle{\pgfqpoint{0.800000in}{0.528000in}}{\pgfqpoint{4.960000in}{3.696000in}}%
\pgfusepath{clip}%
\pgfsetbuttcap%
\pgfsetroundjoin%
\pgfsetlinewidth{0.501875pt}%
\definecolor{currentstroke}{rgb}{0.000000,0.000000,1.000000}%
\pgfsetstrokecolor{currentstroke}%
\pgfsetdash{}{0pt}%
\pgfpathmoveto{\pgfqpoint{2.035491in}{2.597939in}}%
\pgfpathlineto{\pgfqpoint{2.035491in}{2.683707in}}%
\pgfusepath{stroke}%
\end{pgfscope}%
\begin{pgfscope}%
\pgfpathrectangle{\pgfqpoint{0.800000in}{0.528000in}}{\pgfqpoint{4.960000in}{3.696000in}}%
\pgfusepath{clip}%
\pgfsetbuttcap%
\pgfsetroundjoin%
\pgfsetlinewidth{0.501875pt}%
\definecolor{currentstroke}{rgb}{0.000000,0.000000,1.000000}%
\pgfsetstrokecolor{currentstroke}%
\pgfsetdash{}{0pt}%
\pgfpathmoveto{\pgfqpoint{2.053527in}{2.898128in}}%
\pgfpathlineto{\pgfqpoint{2.053527in}{2.983896in}}%
\pgfusepath{stroke}%
\end{pgfscope}%
\begin{pgfscope}%
\pgfpathrectangle{\pgfqpoint{0.800000in}{0.528000in}}{\pgfqpoint{4.960000in}{3.696000in}}%
\pgfusepath{clip}%
\pgfsetbuttcap%
\pgfsetroundjoin%
\pgfsetlinewidth{0.501875pt}%
\definecolor{currentstroke}{rgb}{0.000000,0.000000,1.000000}%
\pgfsetstrokecolor{currentstroke}%
\pgfsetdash{}{0pt}%
\pgfpathmoveto{\pgfqpoint{2.071564in}{3.112549in}}%
\pgfpathlineto{\pgfqpoint{2.071564in}{3.198317in}}%
\pgfusepath{stroke}%
\end{pgfscope}%
\begin{pgfscope}%
\pgfpathrectangle{\pgfqpoint{0.800000in}{0.528000in}}{\pgfqpoint{4.960000in}{3.696000in}}%
\pgfusepath{clip}%
\pgfsetbuttcap%
\pgfsetroundjoin%
\pgfsetlinewidth{0.501875pt}%
\definecolor{currentstroke}{rgb}{0.000000,0.000000,1.000000}%
\pgfsetstrokecolor{currentstroke}%
\pgfsetdash{}{0pt}%
\pgfpathmoveto{\pgfqpoint{2.089600in}{3.241201in}}%
\pgfpathlineto{\pgfqpoint{2.089600in}{3.326969in}}%
\pgfusepath{stroke}%
\end{pgfscope}%
\begin{pgfscope}%
\pgfpathrectangle{\pgfqpoint{0.800000in}{0.528000in}}{\pgfqpoint{4.960000in}{3.696000in}}%
\pgfusepath{clip}%
\pgfsetbuttcap%
\pgfsetroundjoin%
\pgfsetlinewidth{0.501875pt}%
\definecolor{currentstroke}{rgb}{0.000000,0.000000,1.000000}%
\pgfsetstrokecolor{currentstroke}%
\pgfsetdash{}{0pt}%
\pgfpathmoveto{\pgfqpoint{2.107636in}{3.241201in}}%
\pgfpathlineto{\pgfqpoint{2.107636in}{3.326969in}}%
\pgfusepath{stroke}%
\end{pgfscope}%
\begin{pgfscope}%
\pgfpathrectangle{\pgfqpoint{0.800000in}{0.528000in}}{\pgfqpoint{4.960000in}{3.696000in}}%
\pgfusepath{clip}%
\pgfsetbuttcap%
\pgfsetroundjoin%
\pgfsetlinewidth{0.501875pt}%
\definecolor{currentstroke}{rgb}{0.000000,0.000000,1.000000}%
\pgfsetstrokecolor{currentstroke}%
\pgfsetdash{}{0pt}%
\pgfpathmoveto{\pgfqpoint{2.125673in}{3.069664in}}%
\pgfpathlineto{\pgfqpoint{2.125673in}{3.155433in}}%
\pgfusepath{stroke}%
\end{pgfscope}%
\begin{pgfscope}%
\pgfpathrectangle{\pgfqpoint{0.800000in}{0.528000in}}{\pgfqpoint{4.960000in}{3.696000in}}%
\pgfusepath{clip}%
\pgfsetbuttcap%
\pgfsetroundjoin%
\pgfsetlinewidth{0.501875pt}%
\definecolor{currentstroke}{rgb}{0.000000,0.000000,1.000000}%
\pgfsetstrokecolor{currentstroke}%
\pgfsetdash{}{0pt}%
\pgfpathmoveto{\pgfqpoint{2.143709in}{2.812359in}}%
\pgfpathlineto{\pgfqpoint{2.143709in}{2.898128in}}%
\pgfusepath{stroke}%
\end{pgfscope}%
\begin{pgfscope}%
\pgfpathrectangle{\pgfqpoint{0.800000in}{0.528000in}}{\pgfqpoint{4.960000in}{3.696000in}}%
\pgfusepath{clip}%
\pgfsetbuttcap%
\pgfsetroundjoin%
\pgfsetlinewidth{0.501875pt}%
\definecolor{currentstroke}{rgb}{0.000000,0.000000,1.000000}%
\pgfsetstrokecolor{currentstroke}%
\pgfsetdash{}{0pt}%
\pgfpathmoveto{\pgfqpoint{2.161745in}{2.469286in}}%
\pgfpathlineto{\pgfqpoint{2.161745in}{2.555055in}}%
\pgfusepath{stroke}%
\end{pgfscope}%
\begin{pgfscope}%
\pgfpathrectangle{\pgfqpoint{0.800000in}{0.528000in}}{\pgfqpoint{4.960000in}{3.696000in}}%
\pgfusepath{clip}%
\pgfsetbuttcap%
\pgfsetroundjoin%
\pgfsetlinewidth{0.501875pt}%
\definecolor{currentstroke}{rgb}{0.000000,0.000000,1.000000}%
\pgfsetstrokecolor{currentstroke}%
\pgfsetdash{}{0pt}%
\pgfpathmoveto{\pgfqpoint{2.179782in}{2.169097in}}%
\pgfpathlineto{\pgfqpoint{2.179782in}{2.254865in}}%
\pgfusepath{stroke}%
\end{pgfscope}%
\begin{pgfscope}%
\pgfpathrectangle{\pgfqpoint{0.800000in}{0.528000in}}{\pgfqpoint{4.960000in}{3.696000in}}%
\pgfusepath{clip}%
\pgfsetbuttcap%
\pgfsetroundjoin%
\pgfsetlinewidth{0.501875pt}%
\definecolor{currentstroke}{rgb}{0.000000,0.000000,1.000000}%
\pgfsetstrokecolor{currentstroke}%
\pgfsetdash{}{0pt}%
\pgfpathmoveto{\pgfqpoint{2.197818in}{1.740256in}}%
\pgfpathlineto{\pgfqpoint{2.197818in}{1.826024in}}%
\pgfusepath{stroke}%
\end{pgfscope}%
\begin{pgfscope}%
\pgfpathrectangle{\pgfqpoint{0.800000in}{0.528000in}}{\pgfqpoint{4.960000in}{3.696000in}}%
\pgfusepath{clip}%
\pgfsetbuttcap%
\pgfsetroundjoin%
\pgfsetlinewidth{0.501875pt}%
\definecolor{currentstroke}{rgb}{0.000000,0.000000,1.000000}%
\pgfsetstrokecolor{currentstroke}%
\pgfsetdash{}{0pt}%
\pgfpathmoveto{\pgfqpoint{2.215855in}{1.611603in}}%
\pgfpathlineto{\pgfqpoint{2.215855in}{1.697371in}}%
\pgfusepath{stroke}%
\end{pgfscope}%
\begin{pgfscope}%
\pgfpathrectangle{\pgfqpoint{0.800000in}{0.528000in}}{\pgfqpoint{4.960000in}{3.696000in}}%
\pgfusepath{clip}%
\pgfsetbuttcap%
\pgfsetroundjoin%
\pgfsetlinewidth{0.501875pt}%
\definecolor{currentstroke}{rgb}{0.000000,0.000000,1.000000}%
\pgfsetstrokecolor{currentstroke}%
\pgfsetdash{}{0pt}%
\pgfpathmoveto{\pgfqpoint{2.233891in}{1.568719in}}%
\pgfpathlineto{\pgfqpoint{2.233891in}{1.654487in}}%
\pgfusepath{stroke}%
\end{pgfscope}%
\begin{pgfscope}%
\pgfpathrectangle{\pgfqpoint{0.800000in}{0.528000in}}{\pgfqpoint{4.960000in}{3.696000in}}%
\pgfusepath{clip}%
\pgfsetbuttcap%
\pgfsetroundjoin%
\pgfsetlinewidth{0.501875pt}%
\definecolor{currentstroke}{rgb}{0.000000,0.000000,1.000000}%
\pgfsetstrokecolor{currentstroke}%
\pgfsetdash{}{0pt}%
\pgfpathmoveto{\pgfqpoint{2.251927in}{1.697371in}}%
\pgfpathlineto{\pgfqpoint{2.251927in}{1.783140in}}%
\pgfusepath{stroke}%
\end{pgfscope}%
\begin{pgfscope}%
\pgfpathrectangle{\pgfqpoint{0.800000in}{0.528000in}}{\pgfqpoint{4.960000in}{3.696000in}}%
\pgfusepath{clip}%
\pgfsetbuttcap%
\pgfsetroundjoin%
\pgfsetlinewidth{0.501875pt}%
\definecolor{currentstroke}{rgb}{0.000000,0.000000,1.000000}%
\pgfsetstrokecolor{currentstroke}%
\pgfsetdash{}{0pt}%
\pgfpathmoveto{\pgfqpoint{2.269964in}{1.954676in}}%
\pgfpathlineto{\pgfqpoint{2.269964in}{2.040445in}}%
\pgfusepath{stroke}%
\end{pgfscope}%
\begin{pgfscope}%
\pgfpathrectangle{\pgfqpoint{0.800000in}{0.528000in}}{\pgfqpoint{4.960000in}{3.696000in}}%
\pgfusepath{clip}%
\pgfsetbuttcap%
\pgfsetroundjoin%
\pgfsetlinewidth{0.501875pt}%
\definecolor{currentstroke}{rgb}{0.000000,0.000000,1.000000}%
\pgfsetstrokecolor{currentstroke}%
\pgfsetdash{}{0pt}%
\pgfpathmoveto{\pgfqpoint{2.288000in}{2.383518in}}%
\pgfpathlineto{\pgfqpoint{2.288000in}{2.469286in}}%
\pgfusepath{stroke}%
\end{pgfscope}%
\begin{pgfscope}%
\pgfpathrectangle{\pgfqpoint{0.800000in}{0.528000in}}{\pgfqpoint{4.960000in}{3.696000in}}%
\pgfusepath{clip}%
\pgfsetbuttcap%
\pgfsetroundjoin%
\pgfsetlinewidth{0.501875pt}%
\definecolor{currentstroke}{rgb}{0.000000,0.000000,1.000000}%
\pgfsetstrokecolor{currentstroke}%
\pgfsetdash{}{0pt}%
\pgfpathmoveto{\pgfqpoint{2.306036in}{2.683707in}}%
\pgfpathlineto{\pgfqpoint{2.306036in}{2.769475in}}%
\pgfusepath{stroke}%
\end{pgfscope}%
\begin{pgfscope}%
\pgfpathrectangle{\pgfqpoint{0.800000in}{0.528000in}}{\pgfqpoint{4.960000in}{3.696000in}}%
\pgfusepath{clip}%
\pgfsetbuttcap%
\pgfsetroundjoin%
\pgfsetlinewidth{0.501875pt}%
\definecolor{currentstroke}{rgb}{0.000000,0.000000,1.000000}%
\pgfsetstrokecolor{currentstroke}%
\pgfsetdash{}{0pt}%
\pgfpathmoveto{\pgfqpoint{2.324073in}{2.941012in}}%
\pgfpathlineto{\pgfqpoint{2.324073in}{3.026780in}}%
\pgfusepath{stroke}%
\end{pgfscope}%
\begin{pgfscope}%
\pgfpathrectangle{\pgfqpoint{0.800000in}{0.528000in}}{\pgfqpoint{4.960000in}{3.696000in}}%
\pgfusepath{clip}%
\pgfsetbuttcap%
\pgfsetroundjoin%
\pgfsetlinewidth{0.501875pt}%
\definecolor{currentstroke}{rgb}{0.000000,0.000000,1.000000}%
\pgfsetstrokecolor{currentstroke}%
\pgfsetdash{}{0pt}%
\pgfpathmoveto{\pgfqpoint{2.342109in}{3.112549in}}%
\pgfpathlineto{\pgfqpoint{2.342109in}{3.198317in}}%
\pgfusepath{stroke}%
\end{pgfscope}%
\begin{pgfscope}%
\pgfpathrectangle{\pgfqpoint{0.800000in}{0.528000in}}{\pgfqpoint{4.960000in}{3.696000in}}%
\pgfusepath{clip}%
\pgfsetbuttcap%
\pgfsetroundjoin%
\pgfsetlinewidth{0.501875pt}%
\definecolor{currentstroke}{rgb}{0.000000,0.000000,1.000000}%
\pgfsetstrokecolor{currentstroke}%
\pgfsetdash{}{0pt}%
\pgfpathmoveto{\pgfqpoint{2.360145in}{3.198317in}}%
\pgfpathlineto{\pgfqpoint{2.360145in}{3.284085in}}%
\pgfusepath{stroke}%
\end{pgfscope}%
\begin{pgfscope}%
\pgfpathrectangle{\pgfqpoint{0.800000in}{0.528000in}}{\pgfqpoint{4.960000in}{3.696000in}}%
\pgfusepath{clip}%
\pgfsetbuttcap%
\pgfsetroundjoin%
\pgfsetlinewidth{0.501875pt}%
\definecolor{currentstroke}{rgb}{0.000000,0.000000,1.000000}%
\pgfsetstrokecolor{currentstroke}%
\pgfsetdash{}{0pt}%
\pgfpathmoveto{\pgfqpoint{2.378182in}{3.112549in}}%
\pgfpathlineto{\pgfqpoint{2.378182in}{3.198317in}}%
\pgfusepath{stroke}%
\end{pgfscope}%
\begin{pgfscope}%
\pgfpathrectangle{\pgfqpoint{0.800000in}{0.528000in}}{\pgfqpoint{4.960000in}{3.696000in}}%
\pgfusepath{clip}%
\pgfsetbuttcap%
\pgfsetroundjoin%
\pgfsetlinewidth{0.501875pt}%
\definecolor{currentstroke}{rgb}{0.000000,0.000000,1.000000}%
\pgfsetstrokecolor{currentstroke}%
\pgfsetdash{}{0pt}%
\pgfpathmoveto{\pgfqpoint{2.396218in}{2.941012in}}%
\pgfpathlineto{\pgfqpoint{2.396218in}{3.026780in}}%
\pgfusepath{stroke}%
\end{pgfscope}%
\begin{pgfscope}%
\pgfpathrectangle{\pgfqpoint{0.800000in}{0.528000in}}{\pgfqpoint{4.960000in}{3.696000in}}%
\pgfusepath{clip}%
\pgfsetbuttcap%
\pgfsetroundjoin%
\pgfsetlinewidth{0.501875pt}%
\definecolor{currentstroke}{rgb}{0.000000,0.000000,1.000000}%
\pgfsetstrokecolor{currentstroke}%
\pgfsetdash{}{0pt}%
\pgfpathmoveto{\pgfqpoint{2.414255in}{2.683707in}}%
\pgfpathlineto{\pgfqpoint{2.414255in}{2.769475in}}%
\pgfusepath{stroke}%
\end{pgfscope}%
\begin{pgfscope}%
\pgfpathrectangle{\pgfqpoint{0.800000in}{0.528000in}}{\pgfqpoint{4.960000in}{3.696000in}}%
\pgfusepath{clip}%
\pgfsetbuttcap%
\pgfsetroundjoin%
\pgfsetlinewidth{0.501875pt}%
\definecolor{currentstroke}{rgb}{0.000000,0.000000,1.000000}%
\pgfsetstrokecolor{currentstroke}%
\pgfsetdash{}{0pt}%
\pgfpathmoveto{\pgfqpoint{2.432291in}{2.383518in}}%
\pgfpathlineto{\pgfqpoint{2.432291in}{2.469286in}}%
\pgfusepath{stroke}%
\end{pgfscope}%
\begin{pgfscope}%
\pgfpathrectangle{\pgfqpoint{0.800000in}{0.528000in}}{\pgfqpoint{4.960000in}{3.696000in}}%
\pgfusepath{clip}%
\pgfsetbuttcap%
\pgfsetroundjoin%
\pgfsetlinewidth{0.501875pt}%
\definecolor{currentstroke}{rgb}{0.000000,0.000000,1.000000}%
\pgfsetstrokecolor{currentstroke}%
\pgfsetdash{}{0pt}%
\pgfpathmoveto{\pgfqpoint{2.450327in}{1.954676in}}%
\pgfpathlineto{\pgfqpoint{2.450327in}{2.040445in}}%
\pgfusepath{stroke}%
\end{pgfscope}%
\begin{pgfscope}%
\pgfpathrectangle{\pgfqpoint{0.800000in}{0.528000in}}{\pgfqpoint{4.960000in}{3.696000in}}%
\pgfusepath{clip}%
\pgfsetbuttcap%
\pgfsetroundjoin%
\pgfsetlinewidth{0.501875pt}%
\definecolor{currentstroke}{rgb}{0.000000,0.000000,1.000000}%
\pgfsetstrokecolor{currentstroke}%
\pgfsetdash{}{0pt}%
\pgfpathmoveto{\pgfqpoint{2.468364in}{1.740256in}}%
\pgfpathlineto{\pgfqpoint{2.468364in}{1.826024in}}%
\pgfusepath{stroke}%
\end{pgfscope}%
\begin{pgfscope}%
\pgfpathrectangle{\pgfqpoint{0.800000in}{0.528000in}}{\pgfqpoint{4.960000in}{3.696000in}}%
\pgfusepath{clip}%
\pgfsetbuttcap%
\pgfsetroundjoin%
\pgfsetlinewidth{0.501875pt}%
\definecolor{currentstroke}{rgb}{0.000000,0.000000,1.000000}%
\pgfsetstrokecolor{currentstroke}%
\pgfsetdash{}{0pt}%
\pgfpathmoveto{\pgfqpoint{2.486400in}{1.611603in}}%
\pgfpathlineto{\pgfqpoint{2.486400in}{1.697371in}}%
\pgfusepath{stroke}%
\end{pgfscope}%
\begin{pgfscope}%
\pgfpathrectangle{\pgfqpoint{0.800000in}{0.528000in}}{\pgfqpoint{4.960000in}{3.696000in}}%
\pgfusepath{clip}%
\pgfsetbuttcap%
\pgfsetroundjoin%
\pgfsetlinewidth{0.501875pt}%
\definecolor{currentstroke}{rgb}{0.000000,0.000000,1.000000}%
\pgfsetstrokecolor{currentstroke}%
\pgfsetdash{}{0pt}%
\pgfpathmoveto{\pgfqpoint{2.504436in}{1.654487in}}%
\pgfpathlineto{\pgfqpoint{2.504436in}{1.740256in}}%
\pgfusepath{stroke}%
\end{pgfscope}%
\begin{pgfscope}%
\pgfpathrectangle{\pgfqpoint{0.800000in}{0.528000in}}{\pgfqpoint{4.960000in}{3.696000in}}%
\pgfusepath{clip}%
\pgfsetbuttcap%
\pgfsetroundjoin%
\pgfsetlinewidth{0.501875pt}%
\definecolor{currentstroke}{rgb}{0.000000,0.000000,1.000000}%
\pgfsetstrokecolor{currentstroke}%
\pgfsetdash{}{0pt}%
\pgfpathmoveto{\pgfqpoint{2.522473in}{1.826024in}}%
\pgfpathlineto{\pgfqpoint{2.522473in}{1.911792in}}%
\pgfusepath{stroke}%
\end{pgfscope}%
\begin{pgfscope}%
\pgfpathrectangle{\pgfqpoint{0.800000in}{0.528000in}}{\pgfqpoint{4.960000in}{3.696000in}}%
\pgfusepath{clip}%
\pgfsetbuttcap%
\pgfsetroundjoin%
\pgfsetlinewidth{0.501875pt}%
\definecolor{currentstroke}{rgb}{0.000000,0.000000,1.000000}%
\pgfsetstrokecolor{currentstroke}%
\pgfsetdash{}{0pt}%
\pgfpathmoveto{\pgfqpoint{2.540509in}{2.169097in}}%
\pgfpathlineto{\pgfqpoint{2.540509in}{2.254865in}}%
\pgfusepath{stroke}%
\end{pgfscope}%
\begin{pgfscope}%
\pgfpathrectangle{\pgfqpoint{0.800000in}{0.528000in}}{\pgfqpoint{4.960000in}{3.696000in}}%
\pgfusepath{clip}%
\pgfsetbuttcap%
\pgfsetroundjoin%
\pgfsetlinewidth{0.501875pt}%
\definecolor{currentstroke}{rgb}{0.000000,0.000000,1.000000}%
\pgfsetstrokecolor{currentstroke}%
\pgfsetdash{}{0pt}%
\pgfpathmoveto{\pgfqpoint{2.558545in}{2.469286in}}%
\pgfpathlineto{\pgfqpoint{2.558545in}{2.555055in}}%
\pgfusepath{stroke}%
\end{pgfscope}%
\begin{pgfscope}%
\pgfpathrectangle{\pgfqpoint{0.800000in}{0.528000in}}{\pgfqpoint{4.960000in}{3.696000in}}%
\pgfusepath{clip}%
\pgfsetbuttcap%
\pgfsetroundjoin%
\pgfsetlinewidth{0.501875pt}%
\definecolor{currentstroke}{rgb}{0.000000,0.000000,1.000000}%
\pgfsetstrokecolor{currentstroke}%
\pgfsetdash{}{0pt}%
\pgfpathmoveto{\pgfqpoint{2.576582in}{2.769475in}}%
\pgfpathlineto{\pgfqpoint{2.576582in}{2.855244in}}%
\pgfusepath{stroke}%
\end{pgfscope}%
\begin{pgfscope}%
\pgfpathrectangle{\pgfqpoint{0.800000in}{0.528000in}}{\pgfqpoint{4.960000in}{3.696000in}}%
\pgfusepath{clip}%
\pgfsetbuttcap%
\pgfsetroundjoin%
\pgfsetlinewidth{0.501875pt}%
\definecolor{currentstroke}{rgb}{0.000000,0.000000,1.000000}%
\pgfsetstrokecolor{currentstroke}%
\pgfsetdash{}{0pt}%
\pgfpathmoveto{\pgfqpoint{2.594618in}{2.983896in}}%
\pgfpathlineto{\pgfqpoint{2.594618in}{3.069664in}}%
\pgfusepath{stroke}%
\end{pgfscope}%
\begin{pgfscope}%
\pgfpathrectangle{\pgfqpoint{0.800000in}{0.528000in}}{\pgfqpoint{4.960000in}{3.696000in}}%
\pgfusepath{clip}%
\pgfsetbuttcap%
\pgfsetroundjoin%
\pgfsetlinewidth{0.501875pt}%
\definecolor{currentstroke}{rgb}{0.000000,0.000000,1.000000}%
\pgfsetstrokecolor{currentstroke}%
\pgfsetdash{}{0pt}%
\pgfpathmoveto{\pgfqpoint{2.612655in}{3.112549in}}%
\pgfpathlineto{\pgfqpoint{2.612655in}{3.198317in}}%
\pgfusepath{stroke}%
\end{pgfscope}%
\begin{pgfscope}%
\pgfpathrectangle{\pgfqpoint{0.800000in}{0.528000in}}{\pgfqpoint{4.960000in}{3.696000in}}%
\pgfusepath{clip}%
\pgfsetbuttcap%
\pgfsetroundjoin%
\pgfsetlinewidth{0.501875pt}%
\definecolor{currentstroke}{rgb}{0.000000,0.000000,1.000000}%
\pgfsetstrokecolor{currentstroke}%
\pgfsetdash{}{0pt}%
\pgfpathmoveto{\pgfqpoint{2.630691in}{3.112549in}}%
\pgfpathlineto{\pgfqpoint{2.630691in}{3.198317in}}%
\pgfusepath{stroke}%
\end{pgfscope}%
\begin{pgfscope}%
\pgfpathrectangle{\pgfqpoint{0.800000in}{0.528000in}}{\pgfqpoint{4.960000in}{3.696000in}}%
\pgfusepath{clip}%
\pgfsetbuttcap%
\pgfsetroundjoin%
\pgfsetlinewidth{0.501875pt}%
\definecolor{currentstroke}{rgb}{0.000000,0.000000,1.000000}%
\pgfsetstrokecolor{currentstroke}%
\pgfsetdash{}{0pt}%
\pgfpathmoveto{\pgfqpoint{2.648727in}{3.026780in}}%
\pgfpathlineto{\pgfqpoint{2.648727in}{3.112549in}}%
\pgfusepath{stroke}%
\end{pgfscope}%
\begin{pgfscope}%
\pgfpathrectangle{\pgfqpoint{0.800000in}{0.528000in}}{\pgfqpoint{4.960000in}{3.696000in}}%
\pgfusepath{clip}%
\pgfsetbuttcap%
\pgfsetroundjoin%
\pgfsetlinewidth{0.501875pt}%
\definecolor{currentstroke}{rgb}{0.000000,0.000000,1.000000}%
\pgfsetstrokecolor{currentstroke}%
\pgfsetdash{}{0pt}%
\pgfpathmoveto{\pgfqpoint{2.666764in}{2.812359in}}%
\pgfpathlineto{\pgfqpoint{2.666764in}{2.898128in}}%
\pgfusepath{stroke}%
\end{pgfscope}%
\begin{pgfscope}%
\pgfpathrectangle{\pgfqpoint{0.800000in}{0.528000in}}{\pgfqpoint{4.960000in}{3.696000in}}%
\pgfusepath{clip}%
\pgfsetbuttcap%
\pgfsetroundjoin%
\pgfsetlinewidth{0.501875pt}%
\definecolor{currentstroke}{rgb}{0.000000,0.000000,1.000000}%
\pgfsetstrokecolor{currentstroke}%
\pgfsetdash{}{0pt}%
\pgfpathmoveto{\pgfqpoint{2.684800in}{2.555055in}}%
\pgfpathlineto{\pgfqpoint{2.684800in}{2.640823in}}%
\pgfusepath{stroke}%
\end{pgfscope}%
\begin{pgfscope}%
\pgfpathrectangle{\pgfqpoint{0.800000in}{0.528000in}}{\pgfqpoint{4.960000in}{3.696000in}}%
\pgfusepath{clip}%
\pgfsetbuttcap%
\pgfsetroundjoin%
\pgfsetlinewidth{0.501875pt}%
\definecolor{currentstroke}{rgb}{0.000000,0.000000,1.000000}%
\pgfsetstrokecolor{currentstroke}%
\pgfsetdash{}{0pt}%
\pgfpathmoveto{\pgfqpoint{2.702836in}{2.297750in}}%
\pgfpathlineto{\pgfqpoint{2.702836in}{2.383518in}}%
\pgfusepath{stroke}%
\end{pgfscope}%
\begin{pgfscope}%
\pgfpathrectangle{\pgfqpoint{0.800000in}{0.528000in}}{\pgfqpoint{4.960000in}{3.696000in}}%
\pgfusepath{clip}%
\pgfsetbuttcap%
\pgfsetroundjoin%
\pgfsetlinewidth{0.501875pt}%
\definecolor{currentstroke}{rgb}{0.000000,0.000000,1.000000}%
\pgfsetstrokecolor{currentstroke}%
\pgfsetdash{}{0pt}%
\pgfpathmoveto{\pgfqpoint{2.720873in}{1.997561in}}%
\pgfpathlineto{\pgfqpoint{2.720873in}{2.083329in}}%
\pgfusepath{stroke}%
\end{pgfscope}%
\begin{pgfscope}%
\pgfpathrectangle{\pgfqpoint{0.800000in}{0.528000in}}{\pgfqpoint{4.960000in}{3.696000in}}%
\pgfusepath{clip}%
\pgfsetbuttcap%
\pgfsetroundjoin%
\pgfsetlinewidth{0.501875pt}%
\definecolor{currentstroke}{rgb}{0.000000,0.000000,1.000000}%
\pgfsetstrokecolor{currentstroke}%
\pgfsetdash{}{0pt}%
\pgfpathmoveto{\pgfqpoint{2.738909in}{1.740256in}}%
\pgfpathlineto{\pgfqpoint{2.738909in}{1.826024in}}%
\pgfusepath{stroke}%
\end{pgfscope}%
\begin{pgfscope}%
\pgfpathrectangle{\pgfqpoint{0.800000in}{0.528000in}}{\pgfqpoint{4.960000in}{3.696000in}}%
\pgfusepath{clip}%
\pgfsetbuttcap%
\pgfsetroundjoin%
\pgfsetlinewidth{0.501875pt}%
\definecolor{currentstroke}{rgb}{0.000000,0.000000,1.000000}%
\pgfsetstrokecolor{currentstroke}%
\pgfsetdash{}{0pt}%
\pgfpathmoveto{\pgfqpoint{2.756945in}{1.697371in}}%
\pgfpathlineto{\pgfqpoint{2.756945in}{1.783140in}}%
\pgfusepath{stroke}%
\end{pgfscope}%
\begin{pgfscope}%
\pgfpathrectangle{\pgfqpoint{0.800000in}{0.528000in}}{\pgfqpoint{4.960000in}{3.696000in}}%
\pgfusepath{clip}%
\pgfsetbuttcap%
\pgfsetroundjoin%
\pgfsetlinewidth{0.501875pt}%
\definecolor{currentstroke}{rgb}{0.000000,0.000000,1.000000}%
\pgfsetstrokecolor{currentstroke}%
\pgfsetdash{}{0pt}%
\pgfpathmoveto{\pgfqpoint{2.774982in}{1.740256in}}%
\pgfpathlineto{\pgfqpoint{2.774982in}{1.826024in}}%
\pgfusepath{stroke}%
\end{pgfscope}%
\begin{pgfscope}%
\pgfpathrectangle{\pgfqpoint{0.800000in}{0.528000in}}{\pgfqpoint{4.960000in}{3.696000in}}%
\pgfusepath{clip}%
\pgfsetbuttcap%
\pgfsetroundjoin%
\pgfsetlinewidth{0.501875pt}%
\definecolor{currentstroke}{rgb}{0.000000,0.000000,1.000000}%
\pgfsetstrokecolor{currentstroke}%
\pgfsetdash{}{0pt}%
\pgfpathmoveto{\pgfqpoint{2.793018in}{2.040445in}}%
\pgfpathlineto{\pgfqpoint{2.793018in}{2.126213in}}%
\pgfusepath{stroke}%
\end{pgfscope}%
\begin{pgfscope}%
\pgfpathrectangle{\pgfqpoint{0.800000in}{0.528000in}}{\pgfqpoint{4.960000in}{3.696000in}}%
\pgfusepath{clip}%
\pgfsetbuttcap%
\pgfsetroundjoin%
\pgfsetlinewidth{0.501875pt}%
\definecolor{currentstroke}{rgb}{0.000000,0.000000,1.000000}%
\pgfsetstrokecolor{currentstroke}%
\pgfsetdash{}{0pt}%
\pgfpathmoveto{\pgfqpoint{2.811055in}{2.297750in}}%
\pgfpathlineto{\pgfqpoint{2.811055in}{2.383518in}}%
\pgfusepath{stroke}%
\end{pgfscope}%
\begin{pgfscope}%
\pgfpathrectangle{\pgfqpoint{0.800000in}{0.528000in}}{\pgfqpoint{4.960000in}{3.696000in}}%
\pgfusepath{clip}%
\pgfsetbuttcap%
\pgfsetroundjoin%
\pgfsetlinewidth{0.501875pt}%
\definecolor{currentstroke}{rgb}{0.000000,0.000000,1.000000}%
\pgfsetstrokecolor{currentstroke}%
\pgfsetdash{}{0pt}%
\pgfpathmoveto{\pgfqpoint{2.829091in}{2.597939in}}%
\pgfpathlineto{\pgfqpoint{2.829091in}{2.683707in}}%
\pgfusepath{stroke}%
\end{pgfscope}%
\begin{pgfscope}%
\pgfpathrectangle{\pgfqpoint{0.800000in}{0.528000in}}{\pgfqpoint{4.960000in}{3.696000in}}%
\pgfusepath{clip}%
\pgfsetbuttcap%
\pgfsetroundjoin%
\pgfsetlinewidth{0.501875pt}%
\definecolor{currentstroke}{rgb}{0.000000,0.000000,1.000000}%
\pgfsetstrokecolor{currentstroke}%
\pgfsetdash{}{0pt}%
\pgfpathmoveto{\pgfqpoint{2.847127in}{2.812359in}}%
\pgfpathlineto{\pgfqpoint{2.847127in}{2.898128in}}%
\pgfusepath{stroke}%
\end{pgfscope}%
\begin{pgfscope}%
\pgfpathrectangle{\pgfqpoint{0.800000in}{0.528000in}}{\pgfqpoint{4.960000in}{3.696000in}}%
\pgfusepath{clip}%
\pgfsetbuttcap%
\pgfsetroundjoin%
\pgfsetlinewidth{0.501875pt}%
\definecolor{currentstroke}{rgb}{0.000000,0.000000,1.000000}%
\pgfsetstrokecolor{currentstroke}%
\pgfsetdash{}{0pt}%
\pgfpathmoveto{\pgfqpoint{2.865164in}{3.026780in}}%
\pgfpathlineto{\pgfqpoint{2.865164in}{3.112549in}}%
\pgfusepath{stroke}%
\end{pgfscope}%
\begin{pgfscope}%
\pgfpathrectangle{\pgfqpoint{0.800000in}{0.528000in}}{\pgfqpoint{4.960000in}{3.696000in}}%
\pgfusepath{clip}%
\pgfsetbuttcap%
\pgfsetroundjoin%
\pgfsetlinewidth{0.501875pt}%
\definecolor{currentstroke}{rgb}{0.000000,0.000000,1.000000}%
\pgfsetstrokecolor{currentstroke}%
\pgfsetdash{}{0pt}%
\pgfpathmoveto{\pgfqpoint{2.883200in}{3.069664in}}%
\pgfpathlineto{\pgfqpoint{2.883200in}{3.155433in}}%
\pgfusepath{stroke}%
\end{pgfscope}%
\begin{pgfscope}%
\pgfpathrectangle{\pgfqpoint{0.800000in}{0.528000in}}{\pgfqpoint{4.960000in}{3.696000in}}%
\pgfusepath{clip}%
\pgfsetbuttcap%
\pgfsetroundjoin%
\pgfsetlinewidth{0.501875pt}%
\definecolor{currentstroke}{rgb}{0.000000,0.000000,1.000000}%
\pgfsetstrokecolor{currentstroke}%
\pgfsetdash{}{0pt}%
\pgfpathmoveto{\pgfqpoint{2.901236in}{3.069664in}}%
\pgfpathlineto{\pgfqpoint{2.901236in}{3.155433in}}%
\pgfusepath{stroke}%
\end{pgfscope}%
\begin{pgfscope}%
\pgfpathrectangle{\pgfqpoint{0.800000in}{0.528000in}}{\pgfqpoint{4.960000in}{3.696000in}}%
\pgfusepath{clip}%
\pgfsetbuttcap%
\pgfsetroundjoin%
\pgfsetlinewidth{0.501875pt}%
\definecolor{currentstroke}{rgb}{0.000000,0.000000,1.000000}%
\pgfsetstrokecolor{currentstroke}%
\pgfsetdash{}{0pt}%
\pgfpathmoveto{\pgfqpoint{2.919273in}{2.941012in}}%
\pgfpathlineto{\pgfqpoint{2.919273in}{3.026780in}}%
\pgfusepath{stroke}%
\end{pgfscope}%
\begin{pgfscope}%
\pgfpathrectangle{\pgfqpoint{0.800000in}{0.528000in}}{\pgfqpoint{4.960000in}{3.696000in}}%
\pgfusepath{clip}%
\pgfsetbuttcap%
\pgfsetroundjoin%
\pgfsetlinewidth{0.501875pt}%
\definecolor{currentstroke}{rgb}{0.000000,0.000000,1.000000}%
\pgfsetstrokecolor{currentstroke}%
\pgfsetdash{}{0pt}%
\pgfpathmoveto{\pgfqpoint{2.937309in}{2.683707in}}%
\pgfpathlineto{\pgfqpoint{2.937309in}{2.769475in}}%
\pgfusepath{stroke}%
\end{pgfscope}%
\begin{pgfscope}%
\pgfpathrectangle{\pgfqpoint{0.800000in}{0.528000in}}{\pgfqpoint{4.960000in}{3.696000in}}%
\pgfusepath{clip}%
\pgfsetbuttcap%
\pgfsetroundjoin%
\pgfsetlinewidth{0.501875pt}%
\definecolor{currentstroke}{rgb}{0.000000,0.000000,1.000000}%
\pgfsetstrokecolor{currentstroke}%
\pgfsetdash{}{0pt}%
\pgfpathmoveto{\pgfqpoint{2.955345in}{2.469286in}}%
\pgfpathlineto{\pgfqpoint{2.955345in}{2.555055in}}%
\pgfusepath{stroke}%
\end{pgfscope}%
\begin{pgfscope}%
\pgfpathrectangle{\pgfqpoint{0.800000in}{0.528000in}}{\pgfqpoint{4.960000in}{3.696000in}}%
\pgfusepath{clip}%
\pgfsetbuttcap%
\pgfsetroundjoin%
\pgfsetlinewidth{0.501875pt}%
\definecolor{currentstroke}{rgb}{0.000000,0.000000,1.000000}%
\pgfsetstrokecolor{currentstroke}%
\pgfsetdash{}{0pt}%
\pgfpathmoveto{\pgfqpoint{2.973382in}{2.211981in}}%
\pgfpathlineto{\pgfqpoint{2.973382in}{2.297750in}}%
\pgfusepath{stroke}%
\end{pgfscope}%
\begin{pgfscope}%
\pgfpathrectangle{\pgfqpoint{0.800000in}{0.528000in}}{\pgfqpoint{4.960000in}{3.696000in}}%
\pgfusepath{clip}%
\pgfsetbuttcap%
\pgfsetroundjoin%
\pgfsetlinewidth{0.501875pt}%
\definecolor{currentstroke}{rgb}{0.000000,0.000000,1.000000}%
\pgfsetstrokecolor{currentstroke}%
\pgfsetdash{}{0pt}%
\pgfpathmoveto{\pgfqpoint{2.991418in}{1.997561in}}%
\pgfpathlineto{\pgfqpoint{2.991418in}{2.083329in}}%
\pgfusepath{stroke}%
\end{pgfscope}%
\begin{pgfscope}%
\pgfpathrectangle{\pgfqpoint{0.800000in}{0.528000in}}{\pgfqpoint{4.960000in}{3.696000in}}%
\pgfusepath{clip}%
\pgfsetbuttcap%
\pgfsetroundjoin%
\pgfsetlinewidth{0.501875pt}%
\definecolor{currentstroke}{rgb}{0.000000,0.000000,1.000000}%
\pgfsetstrokecolor{currentstroke}%
\pgfsetdash{}{0pt}%
\pgfpathmoveto{\pgfqpoint{3.009455in}{1.740256in}}%
\pgfpathlineto{\pgfqpoint{3.009455in}{1.826024in}}%
\pgfusepath{stroke}%
\end{pgfscope}%
\begin{pgfscope}%
\pgfpathrectangle{\pgfqpoint{0.800000in}{0.528000in}}{\pgfqpoint{4.960000in}{3.696000in}}%
\pgfusepath{clip}%
\pgfsetbuttcap%
\pgfsetroundjoin%
\pgfsetlinewidth{0.501875pt}%
\definecolor{currentstroke}{rgb}{0.000000,0.000000,1.000000}%
\pgfsetstrokecolor{currentstroke}%
\pgfsetdash{}{0pt}%
\pgfpathmoveto{\pgfqpoint{3.027491in}{1.740256in}}%
\pgfpathlineto{\pgfqpoint{3.027491in}{1.826024in}}%
\pgfusepath{stroke}%
\end{pgfscope}%
\begin{pgfscope}%
\pgfpathrectangle{\pgfqpoint{0.800000in}{0.528000in}}{\pgfqpoint{4.960000in}{3.696000in}}%
\pgfusepath{clip}%
\pgfsetbuttcap%
\pgfsetroundjoin%
\pgfsetlinewidth{0.501875pt}%
\definecolor{currentstroke}{rgb}{0.000000,0.000000,1.000000}%
\pgfsetstrokecolor{currentstroke}%
\pgfsetdash{}{0pt}%
\pgfpathmoveto{\pgfqpoint{3.045527in}{1.868908in}}%
\pgfpathlineto{\pgfqpoint{3.045527in}{1.954676in}}%
\pgfusepath{stroke}%
\end{pgfscope}%
\begin{pgfscope}%
\pgfpathrectangle{\pgfqpoint{0.800000in}{0.528000in}}{\pgfqpoint{4.960000in}{3.696000in}}%
\pgfusepath{clip}%
\pgfsetbuttcap%
\pgfsetroundjoin%
\pgfsetlinewidth{0.501875pt}%
\definecolor{currentstroke}{rgb}{0.000000,0.000000,1.000000}%
\pgfsetstrokecolor{currentstroke}%
\pgfsetdash{}{0pt}%
\pgfpathmoveto{\pgfqpoint{3.063564in}{2.169097in}}%
\pgfpathlineto{\pgfqpoint{3.063564in}{2.254865in}}%
\pgfusepath{stroke}%
\end{pgfscope}%
\begin{pgfscope}%
\pgfpathrectangle{\pgfqpoint{0.800000in}{0.528000in}}{\pgfqpoint{4.960000in}{3.696000in}}%
\pgfusepath{clip}%
\pgfsetbuttcap%
\pgfsetroundjoin%
\pgfsetlinewidth{0.501875pt}%
\definecolor{currentstroke}{rgb}{0.000000,0.000000,1.000000}%
\pgfsetstrokecolor{currentstroke}%
\pgfsetdash{}{0pt}%
\pgfpathmoveto{\pgfqpoint{3.081600in}{2.426402in}}%
\pgfpathlineto{\pgfqpoint{3.081600in}{2.512170in}}%
\pgfusepath{stroke}%
\end{pgfscope}%
\begin{pgfscope}%
\pgfpathrectangle{\pgfqpoint{0.800000in}{0.528000in}}{\pgfqpoint{4.960000in}{3.696000in}}%
\pgfusepath{clip}%
\pgfsetbuttcap%
\pgfsetroundjoin%
\pgfsetlinewidth{0.501875pt}%
\definecolor{currentstroke}{rgb}{0.000000,0.000000,1.000000}%
\pgfsetstrokecolor{currentstroke}%
\pgfsetdash{}{0pt}%
\pgfpathmoveto{\pgfqpoint{3.099636in}{2.640823in}}%
\pgfpathlineto{\pgfqpoint{3.099636in}{2.726591in}}%
\pgfusepath{stroke}%
\end{pgfscope}%
\begin{pgfscope}%
\pgfpathrectangle{\pgfqpoint{0.800000in}{0.528000in}}{\pgfqpoint{4.960000in}{3.696000in}}%
\pgfusepath{clip}%
\pgfsetbuttcap%
\pgfsetroundjoin%
\pgfsetlinewidth{0.501875pt}%
\definecolor{currentstroke}{rgb}{0.000000,0.000000,1.000000}%
\pgfsetstrokecolor{currentstroke}%
\pgfsetdash{}{0pt}%
\pgfpathmoveto{\pgfqpoint{3.117673in}{2.855244in}}%
\pgfpathlineto{\pgfqpoint{3.117673in}{2.941012in}}%
\pgfusepath{stroke}%
\end{pgfscope}%
\begin{pgfscope}%
\pgfpathrectangle{\pgfqpoint{0.800000in}{0.528000in}}{\pgfqpoint{4.960000in}{3.696000in}}%
\pgfusepath{clip}%
\pgfsetbuttcap%
\pgfsetroundjoin%
\pgfsetlinewidth{0.501875pt}%
\definecolor{currentstroke}{rgb}{0.000000,0.000000,1.000000}%
\pgfsetstrokecolor{currentstroke}%
\pgfsetdash{}{0pt}%
\pgfpathmoveto{\pgfqpoint{3.135709in}{2.983896in}}%
\pgfpathlineto{\pgfqpoint{3.135709in}{3.069664in}}%
\pgfusepath{stroke}%
\end{pgfscope}%
\begin{pgfscope}%
\pgfpathrectangle{\pgfqpoint{0.800000in}{0.528000in}}{\pgfqpoint{4.960000in}{3.696000in}}%
\pgfusepath{clip}%
\pgfsetbuttcap%
\pgfsetroundjoin%
\pgfsetlinewidth{0.501875pt}%
\definecolor{currentstroke}{rgb}{0.000000,0.000000,1.000000}%
\pgfsetstrokecolor{currentstroke}%
\pgfsetdash{}{0pt}%
\pgfpathmoveto{\pgfqpoint{3.153745in}{3.026780in}}%
\pgfpathlineto{\pgfqpoint{3.153745in}{3.112549in}}%
\pgfusepath{stroke}%
\end{pgfscope}%
\begin{pgfscope}%
\pgfpathrectangle{\pgfqpoint{0.800000in}{0.528000in}}{\pgfqpoint{4.960000in}{3.696000in}}%
\pgfusepath{clip}%
\pgfsetbuttcap%
\pgfsetroundjoin%
\pgfsetlinewidth{0.501875pt}%
\definecolor{currentstroke}{rgb}{0.000000,0.000000,1.000000}%
\pgfsetstrokecolor{currentstroke}%
\pgfsetdash{}{0pt}%
\pgfpathmoveto{\pgfqpoint{3.171782in}{2.983896in}}%
\pgfpathlineto{\pgfqpoint{3.171782in}{3.069664in}}%
\pgfusepath{stroke}%
\end{pgfscope}%
\begin{pgfscope}%
\pgfpathrectangle{\pgfqpoint{0.800000in}{0.528000in}}{\pgfqpoint{4.960000in}{3.696000in}}%
\pgfusepath{clip}%
\pgfsetbuttcap%
\pgfsetroundjoin%
\pgfsetlinewidth{0.501875pt}%
\definecolor{currentstroke}{rgb}{0.000000,0.000000,1.000000}%
\pgfsetstrokecolor{currentstroke}%
\pgfsetdash{}{0pt}%
\pgfpathmoveto{\pgfqpoint{3.189818in}{2.812359in}}%
\pgfpathlineto{\pgfqpoint{3.189818in}{2.898128in}}%
\pgfusepath{stroke}%
\end{pgfscope}%
\begin{pgfscope}%
\pgfpathrectangle{\pgfqpoint{0.800000in}{0.528000in}}{\pgfqpoint{4.960000in}{3.696000in}}%
\pgfusepath{clip}%
\pgfsetbuttcap%
\pgfsetroundjoin%
\pgfsetlinewidth{0.501875pt}%
\definecolor{currentstroke}{rgb}{0.000000,0.000000,1.000000}%
\pgfsetstrokecolor{currentstroke}%
\pgfsetdash{}{0pt}%
\pgfpathmoveto{\pgfqpoint{3.207855in}{2.597939in}}%
\pgfpathlineto{\pgfqpoint{3.207855in}{2.683707in}}%
\pgfusepath{stroke}%
\end{pgfscope}%
\begin{pgfscope}%
\pgfpathrectangle{\pgfqpoint{0.800000in}{0.528000in}}{\pgfqpoint{4.960000in}{3.696000in}}%
\pgfusepath{clip}%
\pgfsetbuttcap%
\pgfsetroundjoin%
\pgfsetlinewidth{0.501875pt}%
\definecolor{currentstroke}{rgb}{0.000000,0.000000,1.000000}%
\pgfsetstrokecolor{currentstroke}%
\pgfsetdash{}{0pt}%
\pgfpathmoveto{\pgfqpoint{3.225891in}{2.383518in}}%
\pgfpathlineto{\pgfqpoint{3.225891in}{2.469286in}}%
\pgfusepath{stroke}%
\end{pgfscope}%
\begin{pgfscope}%
\pgfpathrectangle{\pgfqpoint{0.800000in}{0.528000in}}{\pgfqpoint{4.960000in}{3.696000in}}%
\pgfusepath{clip}%
\pgfsetbuttcap%
\pgfsetroundjoin%
\pgfsetlinewidth{0.501875pt}%
\definecolor{currentstroke}{rgb}{0.000000,0.000000,1.000000}%
\pgfsetstrokecolor{currentstroke}%
\pgfsetdash{}{0pt}%
\pgfpathmoveto{\pgfqpoint{3.243927in}{2.126213in}}%
\pgfpathlineto{\pgfqpoint{3.243927in}{2.211981in}}%
\pgfusepath{stroke}%
\end{pgfscope}%
\begin{pgfscope}%
\pgfpathrectangle{\pgfqpoint{0.800000in}{0.528000in}}{\pgfqpoint{4.960000in}{3.696000in}}%
\pgfusepath{clip}%
\pgfsetbuttcap%
\pgfsetroundjoin%
\pgfsetlinewidth{0.501875pt}%
\definecolor{currentstroke}{rgb}{0.000000,0.000000,1.000000}%
\pgfsetstrokecolor{currentstroke}%
\pgfsetdash{}{0pt}%
\pgfpathmoveto{\pgfqpoint{3.261964in}{1.868908in}}%
\pgfpathlineto{\pgfqpoint{3.261964in}{1.954676in}}%
\pgfusepath{stroke}%
\end{pgfscope}%
\begin{pgfscope}%
\pgfpathrectangle{\pgfqpoint{0.800000in}{0.528000in}}{\pgfqpoint{4.960000in}{3.696000in}}%
\pgfusepath{clip}%
\pgfsetbuttcap%
\pgfsetroundjoin%
\pgfsetlinewidth{0.501875pt}%
\definecolor{currentstroke}{rgb}{0.000000,0.000000,1.000000}%
\pgfsetstrokecolor{currentstroke}%
\pgfsetdash{}{0pt}%
\pgfpathmoveto{\pgfqpoint{3.280000in}{1.783140in}}%
\pgfpathlineto{\pgfqpoint{3.280000in}{1.868908in}}%
\pgfusepath{stroke}%
\end{pgfscope}%
\begin{pgfscope}%
\pgfpathrectangle{\pgfqpoint{0.800000in}{0.528000in}}{\pgfqpoint{4.960000in}{3.696000in}}%
\pgfusepath{clip}%
\pgfsetbuttcap%
\pgfsetroundjoin%
\pgfsetlinewidth{0.501875pt}%
\definecolor{currentstroke}{rgb}{0.000000,0.000000,1.000000}%
\pgfsetstrokecolor{currentstroke}%
\pgfsetdash{}{0pt}%
\pgfpathmoveto{\pgfqpoint{3.298036in}{1.826024in}}%
\pgfpathlineto{\pgfqpoint{3.298036in}{1.911792in}}%
\pgfusepath{stroke}%
\end{pgfscope}%
\begin{pgfscope}%
\pgfpathrectangle{\pgfqpoint{0.800000in}{0.528000in}}{\pgfqpoint{4.960000in}{3.696000in}}%
\pgfusepath{clip}%
\pgfsetbuttcap%
\pgfsetroundjoin%
\pgfsetlinewidth{0.501875pt}%
\definecolor{currentstroke}{rgb}{0.000000,0.000000,1.000000}%
\pgfsetstrokecolor{currentstroke}%
\pgfsetdash{}{0pt}%
\pgfpathmoveto{\pgfqpoint{3.316073in}{1.954676in}}%
\pgfpathlineto{\pgfqpoint{3.316073in}{2.040445in}}%
\pgfusepath{stroke}%
\end{pgfscope}%
\begin{pgfscope}%
\pgfpathrectangle{\pgfqpoint{0.800000in}{0.528000in}}{\pgfqpoint{4.960000in}{3.696000in}}%
\pgfusepath{clip}%
\pgfsetbuttcap%
\pgfsetroundjoin%
\pgfsetlinewidth{0.501875pt}%
\definecolor{currentstroke}{rgb}{0.000000,0.000000,1.000000}%
\pgfsetstrokecolor{currentstroke}%
\pgfsetdash{}{0pt}%
\pgfpathmoveto{\pgfqpoint{3.334109in}{2.254865in}}%
\pgfpathlineto{\pgfqpoint{3.334109in}{2.340634in}}%
\pgfusepath{stroke}%
\end{pgfscope}%
\begin{pgfscope}%
\pgfpathrectangle{\pgfqpoint{0.800000in}{0.528000in}}{\pgfqpoint{4.960000in}{3.696000in}}%
\pgfusepath{clip}%
\pgfsetbuttcap%
\pgfsetroundjoin%
\pgfsetlinewidth{0.501875pt}%
\definecolor{currentstroke}{rgb}{0.000000,0.000000,1.000000}%
\pgfsetstrokecolor{currentstroke}%
\pgfsetdash{}{0pt}%
\pgfpathmoveto{\pgfqpoint{3.352145in}{2.512170in}}%
\pgfpathlineto{\pgfqpoint{3.352145in}{2.597939in}}%
\pgfusepath{stroke}%
\end{pgfscope}%
\begin{pgfscope}%
\pgfpathrectangle{\pgfqpoint{0.800000in}{0.528000in}}{\pgfqpoint{4.960000in}{3.696000in}}%
\pgfusepath{clip}%
\pgfsetbuttcap%
\pgfsetroundjoin%
\pgfsetlinewidth{0.501875pt}%
\definecolor{currentstroke}{rgb}{0.000000,0.000000,1.000000}%
\pgfsetstrokecolor{currentstroke}%
\pgfsetdash{}{0pt}%
\pgfpathmoveto{\pgfqpoint{3.370182in}{2.726591in}}%
\pgfpathlineto{\pgfqpoint{3.370182in}{2.812359in}}%
\pgfusepath{stroke}%
\end{pgfscope}%
\begin{pgfscope}%
\pgfpathrectangle{\pgfqpoint{0.800000in}{0.528000in}}{\pgfqpoint{4.960000in}{3.696000in}}%
\pgfusepath{clip}%
\pgfsetbuttcap%
\pgfsetroundjoin%
\pgfsetlinewidth{0.501875pt}%
\definecolor{currentstroke}{rgb}{0.000000,0.000000,1.000000}%
\pgfsetstrokecolor{currentstroke}%
\pgfsetdash{}{0pt}%
\pgfpathmoveto{\pgfqpoint{3.388218in}{2.898128in}}%
\pgfpathlineto{\pgfqpoint{3.388218in}{2.983896in}}%
\pgfusepath{stroke}%
\end{pgfscope}%
\begin{pgfscope}%
\pgfpathrectangle{\pgfqpoint{0.800000in}{0.528000in}}{\pgfqpoint{4.960000in}{3.696000in}}%
\pgfusepath{clip}%
\pgfsetbuttcap%
\pgfsetroundjoin%
\pgfsetlinewidth{0.501875pt}%
\definecolor{currentstroke}{rgb}{0.000000,0.000000,1.000000}%
\pgfsetstrokecolor{currentstroke}%
\pgfsetdash{}{0pt}%
\pgfpathmoveto{\pgfqpoint{3.406255in}{2.983896in}}%
\pgfpathlineto{\pgfqpoint{3.406255in}{3.069664in}}%
\pgfusepath{stroke}%
\end{pgfscope}%
\begin{pgfscope}%
\pgfpathrectangle{\pgfqpoint{0.800000in}{0.528000in}}{\pgfqpoint{4.960000in}{3.696000in}}%
\pgfusepath{clip}%
\pgfsetbuttcap%
\pgfsetroundjoin%
\pgfsetlinewidth{0.501875pt}%
\definecolor{currentstroke}{rgb}{0.000000,0.000000,1.000000}%
\pgfsetstrokecolor{currentstroke}%
\pgfsetdash{}{0pt}%
\pgfpathmoveto{\pgfqpoint{3.424291in}{2.983896in}}%
\pgfpathlineto{\pgfqpoint{3.424291in}{3.069664in}}%
\pgfusepath{stroke}%
\end{pgfscope}%
\begin{pgfscope}%
\pgfpathrectangle{\pgfqpoint{0.800000in}{0.528000in}}{\pgfqpoint{4.960000in}{3.696000in}}%
\pgfusepath{clip}%
\pgfsetbuttcap%
\pgfsetroundjoin%
\pgfsetlinewidth{0.501875pt}%
\definecolor{currentstroke}{rgb}{0.000000,0.000000,1.000000}%
\pgfsetstrokecolor{currentstroke}%
\pgfsetdash{}{0pt}%
\pgfpathmoveto{\pgfqpoint{3.442327in}{2.898128in}}%
\pgfpathlineto{\pgfqpoint{3.442327in}{2.983896in}}%
\pgfusepath{stroke}%
\end{pgfscope}%
\begin{pgfscope}%
\pgfpathrectangle{\pgfqpoint{0.800000in}{0.528000in}}{\pgfqpoint{4.960000in}{3.696000in}}%
\pgfusepath{clip}%
\pgfsetbuttcap%
\pgfsetroundjoin%
\pgfsetlinewidth{0.501875pt}%
\definecolor{currentstroke}{rgb}{0.000000,0.000000,1.000000}%
\pgfsetstrokecolor{currentstroke}%
\pgfsetdash{}{0pt}%
\pgfpathmoveto{\pgfqpoint{3.460364in}{2.726591in}}%
\pgfpathlineto{\pgfqpoint{3.460364in}{2.812359in}}%
\pgfusepath{stroke}%
\end{pgfscope}%
\begin{pgfscope}%
\pgfpathrectangle{\pgfqpoint{0.800000in}{0.528000in}}{\pgfqpoint{4.960000in}{3.696000in}}%
\pgfusepath{clip}%
\pgfsetbuttcap%
\pgfsetroundjoin%
\pgfsetlinewidth{0.501875pt}%
\definecolor{currentstroke}{rgb}{0.000000,0.000000,1.000000}%
\pgfsetstrokecolor{currentstroke}%
\pgfsetdash{}{0pt}%
\pgfpathmoveto{\pgfqpoint{3.478400in}{2.512170in}}%
\pgfpathlineto{\pgfqpoint{3.478400in}{2.597939in}}%
\pgfusepath{stroke}%
\end{pgfscope}%
\begin{pgfscope}%
\pgfpathrectangle{\pgfqpoint{0.800000in}{0.528000in}}{\pgfqpoint{4.960000in}{3.696000in}}%
\pgfusepath{clip}%
\pgfsetbuttcap%
\pgfsetroundjoin%
\pgfsetlinewidth{0.501875pt}%
\definecolor{currentstroke}{rgb}{0.000000,0.000000,1.000000}%
\pgfsetstrokecolor{currentstroke}%
\pgfsetdash{}{0pt}%
\pgfpathmoveto{\pgfqpoint{3.496436in}{2.297750in}}%
\pgfpathlineto{\pgfqpoint{3.496436in}{2.383518in}}%
\pgfusepath{stroke}%
\end{pgfscope}%
\begin{pgfscope}%
\pgfpathrectangle{\pgfqpoint{0.800000in}{0.528000in}}{\pgfqpoint{4.960000in}{3.696000in}}%
\pgfusepath{clip}%
\pgfsetbuttcap%
\pgfsetroundjoin%
\pgfsetlinewidth{0.501875pt}%
\definecolor{currentstroke}{rgb}{0.000000,0.000000,1.000000}%
\pgfsetstrokecolor{currentstroke}%
\pgfsetdash{}{0pt}%
\pgfpathmoveto{\pgfqpoint{3.514473in}{2.083329in}}%
\pgfpathlineto{\pgfqpoint{3.514473in}{2.169097in}}%
\pgfusepath{stroke}%
\end{pgfscope}%
\begin{pgfscope}%
\pgfpathrectangle{\pgfqpoint{0.800000in}{0.528000in}}{\pgfqpoint{4.960000in}{3.696000in}}%
\pgfusepath{clip}%
\pgfsetbuttcap%
\pgfsetroundjoin%
\pgfsetlinewidth{0.501875pt}%
\definecolor{currentstroke}{rgb}{0.000000,0.000000,1.000000}%
\pgfsetstrokecolor{currentstroke}%
\pgfsetdash{}{0pt}%
\pgfpathmoveto{\pgfqpoint{3.532509in}{1.868908in}}%
\pgfpathlineto{\pgfqpoint{3.532509in}{1.954676in}}%
\pgfusepath{stroke}%
\end{pgfscope}%
\begin{pgfscope}%
\pgfpathrectangle{\pgfqpoint{0.800000in}{0.528000in}}{\pgfqpoint{4.960000in}{3.696000in}}%
\pgfusepath{clip}%
\pgfsetbuttcap%
\pgfsetroundjoin%
\pgfsetlinewidth{0.501875pt}%
\definecolor{currentstroke}{rgb}{0.000000,0.000000,1.000000}%
\pgfsetstrokecolor{currentstroke}%
\pgfsetdash{}{0pt}%
\pgfpathmoveto{\pgfqpoint{3.550545in}{1.826024in}}%
\pgfpathlineto{\pgfqpoint{3.550545in}{1.911792in}}%
\pgfusepath{stroke}%
\end{pgfscope}%
\begin{pgfscope}%
\pgfpathrectangle{\pgfqpoint{0.800000in}{0.528000in}}{\pgfqpoint{4.960000in}{3.696000in}}%
\pgfusepath{clip}%
\pgfsetbuttcap%
\pgfsetroundjoin%
\pgfsetlinewidth{0.501875pt}%
\definecolor{currentstroke}{rgb}{0.000000,0.000000,1.000000}%
\pgfsetstrokecolor{currentstroke}%
\pgfsetdash{}{0pt}%
\pgfpathmoveto{\pgfqpoint{3.568582in}{1.868908in}}%
\pgfpathlineto{\pgfqpoint{3.568582in}{1.954676in}}%
\pgfusepath{stroke}%
\end{pgfscope}%
\begin{pgfscope}%
\pgfpathrectangle{\pgfqpoint{0.800000in}{0.528000in}}{\pgfqpoint{4.960000in}{3.696000in}}%
\pgfusepath{clip}%
\pgfsetbuttcap%
\pgfsetroundjoin%
\pgfsetlinewidth{0.501875pt}%
\definecolor{currentstroke}{rgb}{0.000000,0.000000,1.000000}%
\pgfsetstrokecolor{currentstroke}%
\pgfsetdash{}{0pt}%
\pgfpathmoveto{\pgfqpoint{3.586618in}{2.126213in}}%
\pgfpathlineto{\pgfqpoint{3.586618in}{2.211981in}}%
\pgfusepath{stroke}%
\end{pgfscope}%
\begin{pgfscope}%
\pgfpathrectangle{\pgfqpoint{0.800000in}{0.528000in}}{\pgfqpoint{4.960000in}{3.696000in}}%
\pgfusepath{clip}%
\pgfsetbuttcap%
\pgfsetroundjoin%
\pgfsetlinewidth{0.501875pt}%
\definecolor{currentstroke}{rgb}{0.000000,0.000000,1.000000}%
\pgfsetstrokecolor{currentstroke}%
\pgfsetdash{}{0pt}%
\pgfpathmoveto{\pgfqpoint{3.604655in}{2.340634in}}%
\pgfpathlineto{\pgfqpoint{3.604655in}{2.426402in}}%
\pgfusepath{stroke}%
\end{pgfscope}%
\begin{pgfscope}%
\pgfpathrectangle{\pgfqpoint{0.800000in}{0.528000in}}{\pgfqpoint{4.960000in}{3.696000in}}%
\pgfusepath{clip}%
\pgfsetbuttcap%
\pgfsetroundjoin%
\pgfsetlinewidth{0.501875pt}%
\definecolor{currentstroke}{rgb}{0.000000,0.000000,1.000000}%
\pgfsetstrokecolor{currentstroke}%
\pgfsetdash{}{0pt}%
\pgfpathmoveto{\pgfqpoint{3.622691in}{2.597939in}}%
\pgfpathlineto{\pgfqpoint{3.622691in}{2.683707in}}%
\pgfusepath{stroke}%
\end{pgfscope}%
\begin{pgfscope}%
\pgfpathrectangle{\pgfqpoint{0.800000in}{0.528000in}}{\pgfqpoint{4.960000in}{3.696000in}}%
\pgfusepath{clip}%
\pgfsetbuttcap%
\pgfsetroundjoin%
\pgfsetlinewidth{0.501875pt}%
\definecolor{currentstroke}{rgb}{0.000000,0.000000,1.000000}%
\pgfsetstrokecolor{currentstroke}%
\pgfsetdash{}{0pt}%
\pgfpathmoveto{\pgfqpoint{3.640727in}{2.769475in}}%
\pgfpathlineto{\pgfqpoint{3.640727in}{2.855244in}}%
\pgfusepath{stroke}%
\end{pgfscope}%
\begin{pgfscope}%
\pgfpathrectangle{\pgfqpoint{0.800000in}{0.528000in}}{\pgfqpoint{4.960000in}{3.696000in}}%
\pgfusepath{clip}%
\pgfsetbuttcap%
\pgfsetroundjoin%
\pgfsetlinewidth{0.501875pt}%
\definecolor{currentstroke}{rgb}{0.000000,0.000000,1.000000}%
\pgfsetstrokecolor{currentstroke}%
\pgfsetdash{}{0pt}%
\pgfpathmoveto{\pgfqpoint{3.658764in}{2.898128in}}%
\pgfpathlineto{\pgfqpoint{3.658764in}{2.983896in}}%
\pgfusepath{stroke}%
\end{pgfscope}%
\begin{pgfscope}%
\pgfpathrectangle{\pgfqpoint{0.800000in}{0.528000in}}{\pgfqpoint{4.960000in}{3.696000in}}%
\pgfusepath{clip}%
\pgfsetbuttcap%
\pgfsetroundjoin%
\pgfsetlinewidth{0.501875pt}%
\definecolor{currentstroke}{rgb}{0.000000,0.000000,1.000000}%
\pgfsetstrokecolor{currentstroke}%
\pgfsetdash{}{0pt}%
\pgfpathmoveto{\pgfqpoint{3.676800in}{2.941012in}}%
\pgfpathlineto{\pgfqpoint{3.676800in}{3.026780in}}%
\pgfusepath{stroke}%
\end{pgfscope}%
\begin{pgfscope}%
\pgfpathrectangle{\pgfqpoint{0.800000in}{0.528000in}}{\pgfqpoint{4.960000in}{3.696000in}}%
\pgfusepath{clip}%
\pgfsetbuttcap%
\pgfsetroundjoin%
\pgfsetlinewidth{0.501875pt}%
\definecolor{currentstroke}{rgb}{0.000000,0.000000,1.000000}%
\pgfsetstrokecolor{currentstroke}%
\pgfsetdash{}{0pt}%
\pgfpathmoveto{\pgfqpoint{3.694836in}{2.941012in}}%
\pgfpathlineto{\pgfqpoint{3.694836in}{3.026780in}}%
\pgfusepath{stroke}%
\end{pgfscope}%
\begin{pgfscope}%
\pgfpathrectangle{\pgfqpoint{0.800000in}{0.528000in}}{\pgfqpoint{4.960000in}{3.696000in}}%
\pgfusepath{clip}%
\pgfsetbuttcap%
\pgfsetroundjoin%
\pgfsetlinewidth{0.501875pt}%
\definecolor{currentstroke}{rgb}{0.000000,0.000000,1.000000}%
\pgfsetstrokecolor{currentstroke}%
\pgfsetdash{}{0pt}%
\pgfpathmoveto{\pgfqpoint{3.712873in}{2.812359in}}%
\pgfpathlineto{\pgfqpoint{3.712873in}{2.898128in}}%
\pgfusepath{stroke}%
\end{pgfscope}%
\begin{pgfscope}%
\pgfpathrectangle{\pgfqpoint{0.800000in}{0.528000in}}{\pgfqpoint{4.960000in}{3.696000in}}%
\pgfusepath{clip}%
\pgfsetbuttcap%
\pgfsetroundjoin%
\pgfsetlinewidth{0.501875pt}%
\definecolor{currentstroke}{rgb}{0.000000,0.000000,1.000000}%
\pgfsetstrokecolor{currentstroke}%
\pgfsetdash{}{0pt}%
\pgfpathmoveto{\pgfqpoint{3.730909in}{2.640823in}}%
\pgfpathlineto{\pgfqpoint{3.730909in}{2.726591in}}%
\pgfusepath{stroke}%
\end{pgfscope}%
\begin{pgfscope}%
\pgfpathrectangle{\pgfqpoint{0.800000in}{0.528000in}}{\pgfqpoint{4.960000in}{3.696000in}}%
\pgfusepath{clip}%
\pgfsetbuttcap%
\pgfsetroundjoin%
\pgfsetlinewidth{0.501875pt}%
\definecolor{currentstroke}{rgb}{0.000000,0.000000,1.000000}%
\pgfsetstrokecolor{currentstroke}%
\pgfsetdash{}{0pt}%
\pgfpathmoveto{\pgfqpoint{3.748945in}{2.426402in}}%
\pgfpathlineto{\pgfqpoint{3.748945in}{2.512170in}}%
\pgfusepath{stroke}%
\end{pgfscope}%
\begin{pgfscope}%
\pgfpathrectangle{\pgfqpoint{0.800000in}{0.528000in}}{\pgfqpoint{4.960000in}{3.696000in}}%
\pgfusepath{clip}%
\pgfsetbuttcap%
\pgfsetroundjoin%
\pgfsetlinewidth{0.501875pt}%
\definecolor{currentstroke}{rgb}{0.000000,0.000000,1.000000}%
\pgfsetstrokecolor{currentstroke}%
\pgfsetdash{}{0pt}%
\pgfpathmoveto{\pgfqpoint{3.766982in}{2.254865in}}%
\pgfpathlineto{\pgfqpoint{3.766982in}{2.340634in}}%
\pgfusepath{stroke}%
\end{pgfscope}%
\begin{pgfscope}%
\pgfpathrectangle{\pgfqpoint{0.800000in}{0.528000in}}{\pgfqpoint{4.960000in}{3.696000in}}%
\pgfusepath{clip}%
\pgfsetbuttcap%
\pgfsetroundjoin%
\pgfsetlinewidth{0.501875pt}%
\definecolor{currentstroke}{rgb}{0.000000,0.000000,1.000000}%
\pgfsetstrokecolor{currentstroke}%
\pgfsetdash{}{0pt}%
\pgfpathmoveto{\pgfqpoint{3.785018in}{2.083329in}}%
\pgfpathlineto{\pgfqpoint{3.785018in}{2.169097in}}%
\pgfusepath{stroke}%
\end{pgfscope}%
\begin{pgfscope}%
\pgfpathrectangle{\pgfqpoint{0.800000in}{0.528000in}}{\pgfqpoint{4.960000in}{3.696000in}}%
\pgfusepath{clip}%
\pgfsetbuttcap%
\pgfsetroundjoin%
\pgfsetlinewidth{0.501875pt}%
\definecolor{currentstroke}{rgb}{0.000000,0.000000,1.000000}%
\pgfsetstrokecolor{currentstroke}%
\pgfsetdash{}{0pt}%
\pgfpathmoveto{\pgfqpoint{3.803055in}{1.997561in}}%
\pgfpathlineto{\pgfqpoint{3.803055in}{2.083329in}}%
\pgfusepath{stroke}%
\end{pgfscope}%
\begin{pgfscope}%
\pgfpathrectangle{\pgfqpoint{0.800000in}{0.528000in}}{\pgfqpoint{4.960000in}{3.696000in}}%
\pgfusepath{clip}%
\pgfsetbuttcap%
\pgfsetroundjoin%
\pgfsetlinewidth{0.501875pt}%
\definecolor{currentstroke}{rgb}{0.000000,0.000000,1.000000}%
\pgfsetstrokecolor{currentstroke}%
\pgfsetdash{}{0pt}%
\pgfpathmoveto{\pgfqpoint{3.821091in}{1.997561in}}%
\pgfpathlineto{\pgfqpoint{3.821091in}{2.083329in}}%
\pgfusepath{stroke}%
\end{pgfscope}%
\begin{pgfscope}%
\pgfpathrectangle{\pgfqpoint{0.800000in}{0.528000in}}{\pgfqpoint{4.960000in}{3.696000in}}%
\pgfusepath{clip}%
\pgfsetbuttcap%
\pgfsetroundjoin%
\pgfsetlinewidth{0.501875pt}%
\definecolor{currentstroke}{rgb}{0.000000,0.000000,1.000000}%
\pgfsetstrokecolor{currentstroke}%
\pgfsetdash{}{0pt}%
\pgfpathmoveto{\pgfqpoint{3.839127in}{2.083329in}}%
\pgfpathlineto{\pgfqpoint{3.839127in}{2.169097in}}%
\pgfusepath{stroke}%
\end{pgfscope}%
\begin{pgfscope}%
\pgfpathrectangle{\pgfqpoint{0.800000in}{0.528000in}}{\pgfqpoint{4.960000in}{3.696000in}}%
\pgfusepath{clip}%
\pgfsetbuttcap%
\pgfsetroundjoin%
\pgfsetlinewidth{0.501875pt}%
\definecolor{currentstroke}{rgb}{0.000000,0.000000,1.000000}%
\pgfsetstrokecolor{currentstroke}%
\pgfsetdash{}{0pt}%
\pgfpathmoveto{\pgfqpoint{3.857164in}{2.254865in}}%
\pgfpathlineto{\pgfqpoint{3.857164in}{2.340634in}}%
\pgfusepath{stroke}%
\end{pgfscope}%
\begin{pgfscope}%
\pgfpathrectangle{\pgfqpoint{0.800000in}{0.528000in}}{\pgfqpoint{4.960000in}{3.696000in}}%
\pgfusepath{clip}%
\pgfsetbuttcap%
\pgfsetroundjoin%
\pgfsetlinewidth{0.501875pt}%
\definecolor{currentstroke}{rgb}{0.000000,0.000000,1.000000}%
\pgfsetstrokecolor{currentstroke}%
\pgfsetdash{}{0pt}%
\pgfpathmoveto{\pgfqpoint{3.875200in}{2.469286in}}%
\pgfpathlineto{\pgfqpoint{3.875200in}{2.555055in}}%
\pgfusepath{stroke}%
\end{pgfscope}%
\begin{pgfscope}%
\pgfpathrectangle{\pgfqpoint{0.800000in}{0.528000in}}{\pgfqpoint{4.960000in}{3.696000in}}%
\pgfusepath{clip}%
\pgfsetbuttcap%
\pgfsetroundjoin%
\pgfsetlinewidth{0.501875pt}%
\definecolor{currentstroke}{rgb}{0.000000,0.000000,1.000000}%
\pgfsetstrokecolor{currentstroke}%
\pgfsetdash{}{0pt}%
\pgfpathmoveto{\pgfqpoint{3.893236in}{2.640823in}}%
\pgfpathlineto{\pgfqpoint{3.893236in}{2.726591in}}%
\pgfusepath{stroke}%
\end{pgfscope}%
\begin{pgfscope}%
\pgfpathrectangle{\pgfqpoint{0.800000in}{0.528000in}}{\pgfqpoint{4.960000in}{3.696000in}}%
\pgfusepath{clip}%
\pgfsetbuttcap%
\pgfsetroundjoin%
\pgfsetlinewidth{0.501875pt}%
\definecolor{currentstroke}{rgb}{0.000000,0.000000,1.000000}%
\pgfsetstrokecolor{currentstroke}%
\pgfsetdash{}{0pt}%
\pgfpathmoveto{\pgfqpoint{3.911273in}{2.812359in}}%
\pgfpathlineto{\pgfqpoint{3.911273in}{2.898128in}}%
\pgfusepath{stroke}%
\end{pgfscope}%
\begin{pgfscope}%
\pgfpathrectangle{\pgfqpoint{0.800000in}{0.528000in}}{\pgfqpoint{4.960000in}{3.696000in}}%
\pgfusepath{clip}%
\pgfsetbuttcap%
\pgfsetroundjoin%
\pgfsetlinewidth{0.501875pt}%
\definecolor{currentstroke}{rgb}{0.000000,0.000000,1.000000}%
\pgfsetstrokecolor{currentstroke}%
\pgfsetdash{}{0pt}%
\pgfpathmoveto{\pgfqpoint{3.929309in}{2.898128in}}%
\pgfpathlineto{\pgfqpoint{3.929309in}{2.983896in}}%
\pgfusepath{stroke}%
\end{pgfscope}%
\begin{pgfscope}%
\pgfpathrectangle{\pgfqpoint{0.800000in}{0.528000in}}{\pgfqpoint{4.960000in}{3.696000in}}%
\pgfusepath{clip}%
\pgfsetbuttcap%
\pgfsetroundjoin%
\pgfsetlinewidth{0.501875pt}%
\definecolor{currentstroke}{rgb}{0.000000,0.000000,1.000000}%
\pgfsetstrokecolor{currentstroke}%
\pgfsetdash{}{0pt}%
\pgfpathmoveto{\pgfqpoint{3.947345in}{2.941012in}}%
\pgfpathlineto{\pgfqpoint{3.947345in}{3.026780in}}%
\pgfusepath{stroke}%
\end{pgfscope}%
\begin{pgfscope}%
\pgfpathrectangle{\pgfqpoint{0.800000in}{0.528000in}}{\pgfqpoint{4.960000in}{3.696000in}}%
\pgfusepath{clip}%
\pgfsetbuttcap%
\pgfsetroundjoin%
\pgfsetlinewidth{0.501875pt}%
\definecolor{currentstroke}{rgb}{0.000000,0.000000,1.000000}%
\pgfsetstrokecolor{currentstroke}%
\pgfsetdash{}{0pt}%
\pgfpathmoveto{\pgfqpoint{3.965382in}{2.855244in}}%
\pgfpathlineto{\pgfqpoint{3.965382in}{2.941012in}}%
\pgfusepath{stroke}%
\end{pgfscope}%
\begin{pgfscope}%
\pgfpathrectangle{\pgfqpoint{0.800000in}{0.528000in}}{\pgfqpoint{4.960000in}{3.696000in}}%
\pgfusepath{clip}%
\pgfsetbuttcap%
\pgfsetroundjoin%
\pgfsetlinewidth{0.501875pt}%
\definecolor{currentstroke}{rgb}{0.000000,0.000000,1.000000}%
\pgfsetstrokecolor{currentstroke}%
\pgfsetdash{}{0pt}%
\pgfpathmoveto{\pgfqpoint{3.983418in}{2.726591in}}%
\pgfpathlineto{\pgfqpoint{3.983418in}{2.812359in}}%
\pgfusepath{stroke}%
\end{pgfscope}%
\begin{pgfscope}%
\pgfpathrectangle{\pgfqpoint{0.800000in}{0.528000in}}{\pgfqpoint{4.960000in}{3.696000in}}%
\pgfusepath{clip}%
\pgfsetbuttcap%
\pgfsetroundjoin%
\pgfsetlinewidth{0.501875pt}%
\definecolor{currentstroke}{rgb}{0.000000,0.000000,1.000000}%
\pgfsetstrokecolor{currentstroke}%
\pgfsetdash{}{0pt}%
\pgfpathmoveto{\pgfqpoint{4.001455in}{2.555055in}}%
\pgfpathlineto{\pgfqpoint{4.001455in}{2.640823in}}%
\pgfusepath{stroke}%
\end{pgfscope}%
\begin{pgfscope}%
\pgfpathrectangle{\pgfqpoint{0.800000in}{0.528000in}}{\pgfqpoint{4.960000in}{3.696000in}}%
\pgfusepath{clip}%
\pgfsetbuttcap%
\pgfsetroundjoin%
\pgfsetlinewidth{0.501875pt}%
\definecolor{currentstroke}{rgb}{0.000000,0.000000,1.000000}%
\pgfsetstrokecolor{currentstroke}%
\pgfsetdash{}{0pt}%
\pgfpathmoveto{\pgfqpoint{4.019491in}{2.383518in}}%
\pgfpathlineto{\pgfqpoint{4.019491in}{2.469286in}}%
\pgfusepath{stroke}%
\end{pgfscope}%
\begin{pgfscope}%
\pgfpathrectangle{\pgfqpoint{0.800000in}{0.528000in}}{\pgfqpoint{4.960000in}{3.696000in}}%
\pgfusepath{clip}%
\pgfsetbuttcap%
\pgfsetroundjoin%
\pgfsetlinewidth{0.501875pt}%
\definecolor{currentstroke}{rgb}{0.000000,0.000000,1.000000}%
\pgfsetstrokecolor{currentstroke}%
\pgfsetdash{}{0pt}%
\pgfpathmoveto{\pgfqpoint{4.037527in}{2.169097in}}%
\pgfpathlineto{\pgfqpoint{4.037527in}{2.254865in}}%
\pgfusepath{stroke}%
\end{pgfscope}%
\begin{pgfscope}%
\pgfpathrectangle{\pgfqpoint{0.800000in}{0.528000in}}{\pgfqpoint{4.960000in}{3.696000in}}%
\pgfusepath{clip}%
\pgfsetbuttcap%
\pgfsetroundjoin%
\pgfsetlinewidth{0.501875pt}%
\definecolor{currentstroke}{rgb}{0.000000,0.000000,1.000000}%
\pgfsetstrokecolor{currentstroke}%
\pgfsetdash{}{0pt}%
\pgfpathmoveto{\pgfqpoint{4.055564in}{2.040445in}}%
\pgfpathlineto{\pgfqpoint{4.055564in}{2.126213in}}%
\pgfusepath{stroke}%
\end{pgfscope}%
\begin{pgfscope}%
\pgfpathrectangle{\pgfqpoint{0.800000in}{0.528000in}}{\pgfqpoint{4.960000in}{3.696000in}}%
\pgfusepath{clip}%
\pgfsetbuttcap%
\pgfsetroundjoin%
\pgfsetlinewidth{0.501875pt}%
\definecolor{currentstroke}{rgb}{0.000000,0.000000,1.000000}%
\pgfsetstrokecolor{currentstroke}%
\pgfsetdash{}{0pt}%
\pgfpathmoveto{\pgfqpoint{4.073600in}{1.997561in}}%
\pgfpathlineto{\pgfqpoint{4.073600in}{2.083329in}}%
\pgfusepath{stroke}%
\end{pgfscope}%
\begin{pgfscope}%
\pgfpathrectangle{\pgfqpoint{0.800000in}{0.528000in}}{\pgfqpoint{4.960000in}{3.696000in}}%
\pgfusepath{clip}%
\pgfsetbuttcap%
\pgfsetroundjoin%
\pgfsetlinewidth{0.501875pt}%
\definecolor{currentstroke}{rgb}{0.000000,0.000000,1.000000}%
\pgfsetstrokecolor{currentstroke}%
\pgfsetdash{}{0pt}%
\pgfpathmoveto{\pgfqpoint{4.091636in}{2.040445in}}%
\pgfpathlineto{\pgfqpoint{4.091636in}{2.126213in}}%
\pgfusepath{stroke}%
\end{pgfscope}%
\begin{pgfscope}%
\pgfpathrectangle{\pgfqpoint{0.800000in}{0.528000in}}{\pgfqpoint{4.960000in}{3.696000in}}%
\pgfusepath{clip}%
\pgfsetbuttcap%
\pgfsetroundjoin%
\pgfsetlinewidth{0.501875pt}%
\definecolor{currentstroke}{rgb}{0.000000,0.000000,1.000000}%
\pgfsetstrokecolor{currentstroke}%
\pgfsetdash{}{0pt}%
\pgfpathmoveto{\pgfqpoint{4.109673in}{2.169097in}}%
\pgfpathlineto{\pgfqpoint{4.109673in}{2.254865in}}%
\pgfusepath{stroke}%
\end{pgfscope}%
\begin{pgfscope}%
\pgfpathrectangle{\pgfqpoint{0.800000in}{0.528000in}}{\pgfqpoint{4.960000in}{3.696000in}}%
\pgfusepath{clip}%
\pgfsetbuttcap%
\pgfsetroundjoin%
\pgfsetlinewidth{0.501875pt}%
\definecolor{currentstroke}{rgb}{0.000000,0.000000,1.000000}%
\pgfsetstrokecolor{currentstroke}%
\pgfsetdash{}{0pt}%
\pgfpathmoveto{\pgfqpoint{4.127709in}{2.340634in}}%
\pgfpathlineto{\pgfqpoint{4.127709in}{2.426402in}}%
\pgfusepath{stroke}%
\end{pgfscope}%
\begin{pgfscope}%
\pgfpathrectangle{\pgfqpoint{0.800000in}{0.528000in}}{\pgfqpoint{4.960000in}{3.696000in}}%
\pgfusepath{clip}%
\pgfsetbuttcap%
\pgfsetroundjoin%
\pgfsetlinewidth{0.501875pt}%
\definecolor{currentstroke}{rgb}{0.000000,0.000000,1.000000}%
\pgfsetstrokecolor{currentstroke}%
\pgfsetdash{}{0pt}%
\pgfpathmoveto{\pgfqpoint{4.145745in}{2.512170in}}%
\pgfpathlineto{\pgfqpoint{4.145745in}{2.597939in}}%
\pgfusepath{stroke}%
\end{pgfscope}%
\begin{pgfscope}%
\pgfpathrectangle{\pgfqpoint{0.800000in}{0.528000in}}{\pgfqpoint{4.960000in}{3.696000in}}%
\pgfusepath{clip}%
\pgfsetbuttcap%
\pgfsetroundjoin%
\pgfsetlinewidth{0.501875pt}%
\definecolor{currentstroke}{rgb}{0.000000,0.000000,1.000000}%
\pgfsetstrokecolor{currentstroke}%
\pgfsetdash{}{0pt}%
\pgfpathmoveto{\pgfqpoint{4.163782in}{2.683707in}}%
\pgfpathlineto{\pgfqpoint{4.163782in}{2.769475in}}%
\pgfusepath{stroke}%
\end{pgfscope}%
\begin{pgfscope}%
\pgfpathrectangle{\pgfqpoint{0.800000in}{0.528000in}}{\pgfqpoint{4.960000in}{3.696000in}}%
\pgfusepath{clip}%
\pgfsetbuttcap%
\pgfsetroundjoin%
\pgfsetlinewidth{0.501875pt}%
\definecolor{currentstroke}{rgb}{0.000000,0.000000,1.000000}%
\pgfsetstrokecolor{currentstroke}%
\pgfsetdash{}{0pt}%
\pgfpathmoveto{\pgfqpoint{4.181818in}{2.812359in}}%
\pgfpathlineto{\pgfqpoint{4.181818in}{2.898128in}}%
\pgfusepath{stroke}%
\end{pgfscope}%
\begin{pgfscope}%
\pgfpathrectangle{\pgfqpoint{0.800000in}{0.528000in}}{\pgfqpoint{4.960000in}{3.696000in}}%
\pgfusepath{clip}%
\pgfsetbuttcap%
\pgfsetroundjoin%
\pgfsetlinewidth{0.501875pt}%
\definecolor{currentstroke}{rgb}{0.000000,0.000000,1.000000}%
\pgfsetstrokecolor{currentstroke}%
\pgfsetdash{}{0pt}%
\pgfpathmoveto{\pgfqpoint{4.199855in}{2.898128in}}%
\pgfpathlineto{\pgfqpoint{4.199855in}{2.983896in}}%
\pgfusepath{stroke}%
\end{pgfscope}%
\begin{pgfscope}%
\pgfpathrectangle{\pgfqpoint{0.800000in}{0.528000in}}{\pgfqpoint{4.960000in}{3.696000in}}%
\pgfusepath{clip}%
\pgfsetbuttcap%
\pgfsetroundjoin%
\pgfsetlinewidth{0.501875pt}%
\definecolor{currentstroke}{rgb}{0.000000,0.000000,1.000000}%
\pgfsetstrokecolor{currentstroke}%
\pgfsetdash{}{0pt}%
\pgfpathmoveto{\pgfqpoint{4.217891in}{2.898128in}}%
\pgfpathlineto{\pgfqpoint{4.217891in}{2.983896in}}%
\pgfusepath{stroke}%
\end{pgfscope}%
\begin{pgfscope}%
\pgfpathrectangle{\pgfqpoint{0.800000in}{0.528000in}}{\pgfqpoint{4.960000in}{3.696000in}}%
\pgfusepath{clip}%
\pgfsetbuttcap%
\pgfsetroundjoin%
\pgfsetlinewidth{0.501875pt}%
\definecolor{currentstroke}{rgb}{0.000000,0.000000,1.000000}%
\pgfsetstrokecolor{currentstroke}%
\pgfsetdash{}{0pt}%
\pgfpathmoveto{\pgfqpoint{4.235927in}{2.812359in}}%
\pgfpathlineto{\pgfqpoint{4.235927in}{2.898128in}}%
\pgfusepath{stroke}%
\end{pgfscope}%
\begin{pgfscope}%
\pgfpathrectangle{\pgfqpoint{0.800000in}{0.528000in}}{\pgfqpoint{4.960000in}{3.696000in}}%
\pgfusepath{clip}%
\pgfsetbuttcap%
\pgfsetroundjoin%
\pgfsetlinewidth{0.501875pt}%
\definecolor{currentstroke}{rgb}{0.000000,0.000000,1.000000}%
\pgfsetstrokecolor{currentstroke}%
\pgfsetdash{}{0pt}%
\pgfpathmoveto{\pgfqpoint{4.253964in}{2.640823in}}%
\pgfpathlineto{\pgfqpoint{4.253964in}{2.726591in}}%
\pgfusepath{stroke}%
\end{pgfscope}%
\begin{pgfscope}%
\pgfpathrectangle{\pgfqpoint{0.800000in}{0.528000in}}{\pgfqpoint{4.960000in}{3.696000in}}%
\pgfusepath{clip}%
\pgfsetbuttcap%
\pgfsetroundjoin%
\pgfsetlinewidth{0.501875pt}%
\definecolor{currentstroke}{rgb}{0.000000,0.000000,1.000000}%
\pgfsetstrokecolor{currentstroke}%
\pgfsetdash{}{0pt}%
\pgfpathmoveto{\pgfqpoint{4.272000in}{2.469286in}}%
\pgfpathlineto{\pgfqpoint{4.272000in}{2.555055in}}%
\pgfusepath{stroke}%
\end{pgfscope}%
\begin{pgfscope}%
\pgfpathrectangle{\pgfqpoint{0.800000in}{0.528000in}}{\pgfqpoint{4.960000in}{3.696000in}}%
\pgfusepath{clip}%
\pgfsetbuttcap%
\pgfsetroundjoin%
\pgfsetlinewidth{0.501875pt}%
\definecolor{currentstroke}{rgb}{0.000000,0.000000,1.000000}%
\pgfsetstrokecolor{currentstroke}%
\pgfsetdash{}{0pt}%
\pgfpathmoveto{\pgfqpoint{4.290036in}{2.297750in}}%
\pgfpathlineto{\pgfqpoint{4.290036in}{2.383518in}}%
\pgfusepath{stroke}%
\end{pgfscope}%
\begin{pgfscope}%
\pgfpathrectangle{\pgfqpoint{0.800000in}{0.528000in}}{\pgfqpoint{4.960000in}{3.696000in}}%
\pgfusepath{clip}%
\pgfsetbuttcap%
\pgfsetroundjoin%
\pgfsetlinewidth{0.501875pt}%
\definecolor{currentstroke}{rgb}{0.000000,0.000000,1.000000}%
\pgfsetstrokecolor{currentstroke}%
\pgfsetdash{}{0pt}%
\pgfpathmoveto{\pgfqpoint{4.308073in}{2.169097in}}%
\pgfpathlineto{\pgfqpoint{4.308073in}{2.254865in}}%
\pgfusepath{stroke}%
\end{pgfscope}%
\begin{pgfscope}%
\pgfpathrectangle{\pgfqpoint{0.800000in}{0.528000in}}{\pgfqpoint{4.960000in}{3.696000in}}%
\pgfusepath{clip}%
\pgfsetbuttcap%
\pgfsetroundjoin%
\pgfsetlinewidth{0.501875pt}%
\definecolor{currentstroke}{rgb}{0.000000,0.000000,1.000000}%
\pgfsetstrokecolor{currentstroke}%
\pgfsetdash{}{0pt}%
\pgfpathmoveto{\pgfqpoint{4.326109in}{2.040445in}}%
\pgfpathlineto{\pgfqpoint{4.326109in}{2.126213in}}%
\pgfusepath{stroke}%
\end{pgfscope}%
\begin{pgfscope}%
\pgfpathrectangle{\pgfqpoint{0.800000in}{0.528000in}}{\pgfqpoint{4.960000in}{3.696000in}}%
\pgfusepath{clip}%
\pgfsetbuttcap%
\pgfsetroundjoin%
\pgfsetlinewidth{0.501875pt}%
\definecolor{currentstroke}{rgb}{0.000000,0.000000,1.000000}%
\pgfsetstrokecolor{currentstroke}%
\pgfsetdash{}{0pt}%
\pgfpathmoveto{\pgfqpoint{4.344145in}{2.040445in}}%
\pgfpathlineto{\pgfqpoint{4.344145in}{2.126213in}}%
\pgfusepath{stroke}%
\end{pgfscope}%
\begin{pgfscope}%
\pgfpathrectangle{\pgfqpoint{0.800000in}{0.528000in}}{\pgfqpoint{4.960000in}{3.696000in}}%
\pgfusepath{clip}%
\pgfsetbuttcap%
\pgfsetroundjoin%
\pgfsetlinewidth{0.501875pt}%
\definecolor{currentstroke}{rgb}{0.000000,0.000000,1.000000}%
\pgfsetstrokecolor{currentstroke}%
\pgfsetdash{}{0pt}%
\pgfpathmoveto{\pgfqpoint{4.362182in}{2.126213in}}%
\pgfpathlineto{\pgfqpoint{4.362182in}{2.211981in}}%
\pgfusepath{stroke}%
\end{pgfscope}%
\begin{pgfscope}%
\pgfpathrectangle{\pgfqpoint{0.800000in}{0.528000in}}{\pgfqpoint{4.960000in}{3.696000in}}%
\pgfusepath{clip}%
\pgfsetbuttcap%
\pgfsetroundjoin%
\pgfsetlinewidth{0.501875pt}%
\definecolor{currentstroke}{rgb}{0.000000,0.000000,1.000000}%
\pgfsetstrokecolor{currentstroke}%
\pgfsetdash{}{0pt}%
\pgfpathmoveto{\pgfqpoint{4.380218in}{2.254865in}}%
\pgfpathlineto{\pgfqpoint{4.380218in}{2.340634in}}%
\pgfusepath{stroke}%
\end{pgfscope}%
\begin{pgfscope}%
\pgfpathrectangle{\pgfqpoint{0.800000in}{0.528000in}}{\pgfqpoint{4.960000in}{3.696000in}}%
\pgfusepath{clip}%
\pgfsetbuttcap%
\pgfsetroundjoin%
\pgfsetlinewidth{0.501875pt}%
\definecolor{currentstroke}{rgb}{0.000000,0.000000,1.000000}%
\pgfsetstrokecolor{currentstroke}%
\pgfsetdash{}{0pt}%
\pgfpathmoveto{\pgfqpoint{4.398255in}{2.426402in}}%
\pgfpathlineto{\pgfqpoint{4.398255in}{2.512170in}}%
\pgfusepath{stroke}%
\end{pgfscope}%
\begin{pgfscope}%
\pgfpathrectangle{\pgfqpoint{0.800000in}{0.528000in}}{\pgfqpoint{4.960000in}{3.696000in}}%
\pgfusepath{clip}%
\pgfsetbuttcap%
\pgfsetroundjoin%
\pgfsetlinewidth{0.501875pt}%
\definecolor{currentstroke}{rgb}{0.000000,0.000000,1.000000}%
\pgfsetstrokecolor{currentstroke}%
\pgfsetdash{}{0pt}%
\pgfpathmoveto{\pgfqpoint{4.416291in}{2.597939in}}%
\pgfpathlineto{\pgfqpoint{4.416291in}{2.683707in}}%
\pgfusepath{stroke}%
\end{pgfscope}%
\begin{pgfscope}%
\pgfpathrectangle{\pgfqpoint{0.800000in}{0.528000in}}{\pgfqpoint{4.960000in}{3.696000in}}%
\pgfusepath{clip}%
\pgfsetbuttcap%
\pgfsetroundjoin%
\pgfsetlinewidth{0.501875pt}%
\definecolor{currentstroke}{rgb}{0.000000,0.000000,1.000000}%
\pgfsetstrokecolor{currentstroke}%
\pgfsetdash{}{0pt}%
\pgfpathmoveto{\pgfqpoint{4.434327in}{2.726591in}}%
\pgfpathlineto{\pgfqpoint{4.434327in}{2.812359in}}%
\pgfusepath{stroke}%
\end{pgfscope}%
\begin{pgfscope}%
\pgfpathrectangle{\pgfqpoint{0.800000in}{0.528000in}}{\pgfqpoint{4.960000in}{3.696000in}}%
\pgfusepath{clip}%
\pgfsetbuttcap%
\pgfsetroundjoin%
\pgfsetlinewidth{0.501875pt}%
\definecolor{currentstroke}{rgb}{0.000000,0.000000,1.000000}%
\pgfsetstrokecolor{currentstroke}%
\pgfsetdash{}{0pt}%
\pgfpathmoveto{\pgfqpoint{4.452364in}{2.812359in}}%
\pgfpathlineto{\pgfqpoint{4.452364in}{2.898128in}}%
\pgfusepath{stroke}%
\end{pgfscope}%
\begin{pgfscope}%
\pgfpathrectangle{\pgfqpoint{0.800000in}{0.528000in}}{\pgfqpoint{4.960000in}{3.696000in}}%
\pgfusepath{clip}%
\pgfsetbuttcap%
\pgfsetroundjoin%
\pgfsetlinewidth{0.501875pt}%
\definecolor{currentstroke}{rgb}{0.000000,0.000000,1.000000}%
\pgfsetstrokecolor{currentstroke}%
\pgfsetdash{}{0pt}%
\pgfpathmoveto{\pgfqpoint{4.470400in}{2.855244in}}%
\pgfpathlineto{\pgfqpoint{4.470400in}{2.941012in}}%
\pgfusepath{stroke}%
\end{pgfscope}%
\begin{pgfscope}%
\pgfpathrectangle{\pgfqpoint{0.800000in}{0.528000in}}{\pgfqpoint{4.960000in}{3.696000in}}%
\pgfusepath{clip}%
\pgfsetbuttcap%
\pgfsetroundjoin%
\pgfsetlinewidth{0.501875pt}%
\definecolor{currentstroke}{rgb}{0.000000,0.000000,1.000000}%
\pgfsetstrokecolor{currentstroke}%
\pgfsetdash{}{0pt}%
\pgfpathmoveto{\pgfqpoint{4.488436in}{2.812359in}}%
\pgfpathlineto{\pgfqpoint{4.488436in}{2.898128in}}%
\pgfusepath{stroke}%
\end{pgfscope}%
\begin{pgfscope}%
\pgfpathrectangle{\pgfqpoint{0.800000in}{0.528000in}}{\pgfqpoint{4.960000in}{3.696000in}}%
\pgfusepath{clip}%
\pgfsetbuttcap%
\pgfsetroundjoin%
\pgfsetlinewidth{0.501875pt}%
\definecolor{currentstroke}{rgb}{0.000000,0.000000,1.000000}%
\pgfsetstrokecolor{currentstroke}%
\pgfsetdash{}{0pt}%
\pgfpathmoveto{\pgfqpoint{4.506473in}{2.726591in}}%
\pgfpathlineto{\pgfqpoint{4.506473in}{2.812359in}}%
\pgfusepath{stroke}%
\end{pgfscope}%
\begin{pgfscope}%
\pgfpathrectangle{\pgfqpoint{0.800000in}{0.528000in}}{\pgfqpoint{4.960000in}{3.696000in}}%
\pgfusepath{clip}%
\pgfsetbuttcap%
\pgfsetroundjoin%
\pgfsetlinewidth{0.501875pt}%
\definecolor{currentstroke}{rgb}{0.000000,0.000000,1.000000}%
\pgfsetstrokecolor{currentstroke}%
\pgfsetdash{}{0pt}%
\pgfpathmoveto{\pgfqpoint{4.524509in}{2.597939in}}%
\pgfpathlineto{\pgfqpoint{4.524509in}{2.683707in}}%
\pgfusepath{stroke}%
\end{pgfscope}%
\begin{pgfscope}%
\pgfpathrectangle{\pgfqpoint{0.800000in}{0.528000in}}{\pgfqpoint{4.960000in}{3.696000in}}%
\pgfusepath{clip}%
\pgfsetbuttcap%
\pgfsetroundjoin%
\pgfsetlinewidth{0.501875pt}%
\definecolor{currentstroke}{rgb}{0.000000,0.000000,1.000000}%
\pgfsetstrokecolor{currentstroke}%
\pgfsetdash{}{0pt}%
\pgfpathmoveto{\pgfqpoint{4.542545in}{2.426402in}}%
\pgfpathlineto{\pgfqpoint{4.542545in}{2.512170in}}%
\pgfusepath{stroke}%
\end{pgfscope}%
\begin{pgfscope}%
\pgfpathrectangle{\pgfqpoint{0.800000in}{0.528000in}}{\pgfqpoint{4.960000in}{3.696000in}}%
\pgfusepath{clip}%
\pgfsetbuttcap%
\pgfsetroundjoin%
\pgfsetlinewidth{0.501875pt}%
\definecolor{currentstroke}{rgb}{0.000000,0.000000,1.000000}%
\pgfsetstrokecolor{currentstroke}%
\pgfsetdash{}{0pt}%
\pgfpathmoveto{\pgfqpoint{4.560582in}{2.254865in}}%
\pgfpathlineto{\pgfqpoint{4.560582in}{2.340634in}}%
\pgfusepath{stroke}%
\end{pgfscope}%
\begin{pgfscope}%
\pgfpathrectangle{\pgfqpoint{0.800000in}{0.528000in}}{\pgfqpoint{4.960000in}{3.696000in}}%
\pgfusepath{clip}%
\pgfsetbuttcap%
\pgfsetroundjoin%
\pgfsetlinewidth{0.501875pt}%
\definecolor{currentstroke}{rgb}{0.000000,0.000000,1.000000}%
\pgfsetstrokecolor{currentstroke}%
\pgfsetdash{}{0pt}%
\pgfpathmoveto{\pgfqpoint{4.578618in}{2.126213in}}%
\pgfpathlineto{\pgfqpoint{4.578618in}{2.211981in}}%
\pgfusepath{stroke}%
\end{pgfscope}%
\begin{pgfscope}%
\pgfpathrectangle{\pgfqpoint{0.800000in}{0.528000in}}{\pgfqpoint{4.960000in}{3.696000in}}%
\pgfusepath{clip}%
\pgfsetbuttcap%
\pgfsetroundjoin%
\pgfsetlinewidth{0.501875pt}%
\definecolor{currentstroke}{rgb}{0.000000,0.000000,1.000000}%
\pgfsetstrokecolor{currentstroke}%
\pgfsetdash{}{0pt}%
\pgfpathmoveto{\pgfqpoint{4.596655in}{2.083329in}}%
\pgfpathlineto{\pgfqpoint{4.596655in}{2.169097in}}%
\pgfusepath{stroke}%
\end{pgfscope}%
\begin{pgfscope}%
\pgfpathrectangle{\pgfqpoint{0.800000in}{0.528000in}}{\pgfqpoint{4.960000in}{3.696000in}}%
\pgfusepath{clip}%
\pgfsetbuttcap%
\pgfsetroundjoin%
\pgfsetlinewidth{0.501875pt}%
\definecolor{currentstroke}{rgb}{0.000000,0.000000,1.000000}%
\pgfsetstrokecolor{currentstroke}%
\pgfsetdash{}{0pt}%
\pgfpathmoveto{\pgfqpoint{4.614691in}{2.083329in}}%
\pgfpathlineto{\pgfqpoint{4.614691in}{2.169097in}}%
\pgfusepath{stroke}%
\end{pgfscope}%
\begin{pgfscope}%
\pgfpathrectangle{\pgfqpoint{0.800000in}{0.528000in}}{\pgfqpoint{4.960000in}{3.696000in}}%
\pgfusepath{clip}%
\pgfsetbuttcap%
\pgfsetroundjoin%
\pgfsetlinewidth{0.501875pt}%
\definecolor{currentstroke}{rgb}{0.000000,0.000000,1.000000}%
\pgfsetstrokecolor{currentstroke}%
\pgfsetdash{}{0pt}%
\pgfpathmoveto{\pgfqpoint{4.632727in}{2.169097in}}%
\pgfpathlineto{\pgfqpoint{4.632727in}{2.254865in}}%
\pgfusepath{stroke}%
\end{pgfscope}%
\begin{pgfscope}%
\pgfpathrectangle{\pgfqpoint{0.800000in}{0.528000in}}{\pgfqpoint{4.960000in}{3.696000in}}%
\pgfusepath{clip}%
\pgfsetbuttcap%
\pgfsetroundjoin%
\pgfsetlinewidth{0.501875pt}%
\definecolor{currentstroke}{rgb}{0.000000,0.000000,1.000000}%
\pgfsetstrokecolor{currentstroke}%
\pgfsetdash{}{0pt}%
\pgfpathmoveto{\pgfqpoint{4.650764in}{2.297750in}}%
\pgfpathlineto{\pgfqpoint{4.650764in}{2.383518in}}%
\pgfusepath{stroke}%
\end{pgfscope}%
\begin{pgfscope}%
\pgfpathrectangle{\pgfqpoint{0.800000in}{0.528000in}}{\pgfqpoint{4.960000in}{3.696000in}}%
\pgfusepath{clip}%
\pgfsetbuttcap%
\pgfsetroundjoin%
\pgfsetlinewidth{0.501875pt}%
\definecolor{currentstroke}{rgb}{0.000000,0.000000,1.000000}%
\pgfsetstrokecolor{currentstroke}%
\pgfsetdash{}{0pt}%
\pgfpathmoveto{\pgfqpoint{4.668800in}{2.469286in}}%
\pgfpathlineto{\pgfqpoint{4.668800in}{2.555055in}}%
\pgfusepath{stroke}%
\end{pgfscope}%
\begin{pgfscope}%
\pgfpathrectangle{\pgfqpoint{0.800000in}{0.528000in}}{\pgfqpoint{4.960000in}{3.696000in}}%
\pgfusepath{clip}%
\pgfsetbuttcap%
\pgfsetroundjoin%
\pgfsetlinewidth{0.501875pt}%
\definecolor{currentstroke}{rgb}{0.000000,0.000000,1.000000}%
\pgfsetstrokecolor{currentstroke}%
\pgfsetdash{}{0pt}%
\pgfpathmoveto{\pgfqpoint{4.686836in}{2.640823in}}%
\pgfpathlineto{\pgfqpoint{4.686836in}{2.726591in}}%
\pgfusepath{stroke}%
\end{pgfscope}%
\begin{pgfscope}%
\pgfpathrectangle{\pgfqpoint{0.800000in}{0.528000in}}{\pgfqpoint{4.960000in}{3.696000in}}%
\pgfusepath{clip}%
\pgfsetbuttcap%
\pgfsetroundjoin%
\pgfsetlinewidth{0.501875pt}%
\definecolor{currentstroke}{rgb}{0.000000,0.000000,1.000000}%
\pgfsetstrokecolor{currentstroke}%
\pgfsetdash{}{0pt}%
\pgfpathmoveto{\pgfqpoint{4.704873in}{2.769475in}}%
\pgfpathlineto{\pgfqpoint{4.704873in}{2.855244in}}%
\pgfusepath{stroke}%
\end{pgfscope}%
\begin{pgfscope}%
\pgfpathrectangle{\pgfqpoint{0.800000in}{0.528000in}}{\pgfqpoint{4.960000in}{3.696000in}}%
\pgfusepath{clip}%
\pgfsetbuttcap%
\pgfsetroundjoin%
\pgfsetlinewidth{0.501875pt}%
\definecolor{currentstroke}{rgb}{0.000000,0.000000,1.000000}%
\pgfsetstrokecolor{currentstroke}%
\pgfsetdash{}{0pt}%
\pgfpathmoveto{\pgfqpoint{4.722909in}{2.812359in}}%
\pgfpathlineto{\pgfqpoint{4.722909in}{2.898128in}}%
\pgfusepath{stroke}%
\end{pgfscope}%
\begin{pgfscope}%
\pgfpathrectangle{\pgfqpoint{0.800000in}{0.528000in}}{\pgfqpoint{4.960000in}{3.696000in}}%
\pgfusepath{clip}%
\pgfsetbuttcap%
\pgfsetroundjoin%
\pgfsetlinewidth{0.501875pt}%
\definecolor{currentstroke}{rgb}{0.000000,0.000000,1.000000}%
\pgfsetstrokecolor{currentstroke}%
\pgfsetdash{}{0pt}%
\pgfpathmoveto{\pgfqpoint{4.740945in}{2.855244in}}%
\pgfpathlineto{\pgfqpoint{4.740945in}{2.941012in}}%
\pgfusepath{stroke}%
\end{pgfscope}%
\begin{pgfscope}%
\pgfpathrectangle{\pgfqpoint{0.800000in}{0.528000in}}{\pgfqpoint{4.960000in}{3.696000in}}%
\pgfusepath{clip}%
\pgfsetbuttcap%
\pgfsetroundjoin%
\pgfsetlinewidth{0.501875pt}%
\definecolor{currentstroke}{rgb}{0.000000,0.000000,1.000000}%
\pgfsetstrokecolor{currentstroke}%
\pgfsetdash{}{0pt}%
\pgfpathmoveto{\pgfqpoint{4.758982in}{2.769475in}}%
\pgfpathlineto{\pgfqpoint{4.758982in}{2.855244in}}%
\pgfusepath{stroke}%
\end{pgfscope}%
\begin{pgfscope}%
\pgfpathrectangle{\pgfqpoint{0.800000in}{0.528000in}}{\pgfqpoint{4.960000in}{3.696000in}}%
\pgfusepath{clip}%
\pgfsetbuttcap%
\pgfsetroundjoin%
\pgfsetlinewidth{0.501875pt}%
\definecolor{currentstroke}{rgb}{0.000000,0.000000,1.000000}%
\pgfsetstrokecolor{currentstroke}%
\pgfsetdash{}{0pt}%
\pgfpathmoveto{\pgfqpoint{4.777018in}{2.640823in}}%
\pgfpathlineto{\pgfqpoint{4.777018in}{2.726591in}}%
\pgfusepath{stroke}%
\end{pgfscope}%
\begin{pgfscope}%
\pgfpathrectangle{\pgfqpoint{0.800000in}{0.528000in}}{\pgfqpoint{4.960000in}{3.696000in}}%
\pgfusepath{clip}%
\pgfsetbuttcap%
\pgfsetroundjoin%
\pgfsetlinewidth{0.501875pt}%
\definecolor{currentstroke}{rgb}{0.000000,0.000000,1.000000}%
\pgfsetstrokecolor{currentstroke}%
\pgfsetdash{}{0pt}%
\pgfpathmoveto{\pgfqpoint{4.795055in}{2.512170in}}%
\pgfpathlineto{\pgfqpoint{4.795055in}{2.597939in}}%
\pgfusepath{stroke}%
\end{pgfscope}%
\begin{pgfscope}%
\pgfpathrectangle{\pgfqpoint{0.800000in}{0.528000in}}{\pgfqpoint{4.960000in}{3.696000in}}%
\pgfusepath{clip}%
\pgfsetbuttcap%
\pgfsetroundjoin%
\pgfsetlinewidth{0.501875pt}%
\definecolor{currentstroke}{rgb}{0.000000,0.000000,1.000000}%
\pgfsetstrokecolor{currentstroke}%
\pgfsetdash{}{0pt}%
\pgfpathmoveto{\pgfqpoint{4.813091in}{2.340634in}}%
\pgfpathlineto{\pgfqpoint{4.813091in}{2.426402in}}%
\pgfusepath{stroke}%
\end{pgfscope}%
\begin{pgfscope}%
\pgfpathrectangle{\pgfqpoint{0.800000in}{0.528000in}}{\pgfqpoint{4.960000in}{3.696000in}}%
\pgfusepath{clip}%
\pgfsetbuttcap%
\pgfsetroundjoin%
\pgfsetlinewidth{0.501875pt}%
\definecolor{currentstroke}{rgb}{0.000000,0.000000,1.000000}%
\pgfsetstrokecolor{currentstroke}%
\pgfsetdash{}{0pt}%
\pgfpathmoveto{\pgfqpoint{4.831127in}{2.211981in}}%
\pgfpathlineto{\pgfqpoint{4.831127in}{2.297750in}}%
\pgfusepath{stroke}%
\end{pgfscope}%
\begin{pgfscope}%
\pgfpathrectangle{\pgfqpoint{0.800000in}{0.528000in}}{\pgfqpoint{4.960000in}{3.696000in}}%
\pgfusepath{clip}%
\pgfsetbuttcap%
\pgfsetroundjoin%
\pgfsetlinewidth{0.501875pt}%
\definecolor{currentstroke}{rgb}{0.000000,0.000000,1.000000}%
\pgfsetstrokecolor{currentstroke}%
\pgfsetdash{}{0pt}%
\pgfpathmoveto{\pgfqpoint{4.849164in}{2.126213in}}%
\pgfpathlineto{\pgfqpoint{4.849164in}{2.211981in}}%
\pgfusepath{stroke}%
\end{pgfscope}%
\begin{pgfscope}%
\pgfpathrectangle{\pgfqpoint{0.800000in}{0.528000in}}{\pgfqpoint{4.960000in}{3.696000in}}%
\pgfusepath{clip}%
\pgfsetbuttcap%
\pgfsetroundjoin%
\pgfsetlinewidth{0.501875pt}%
\definecolor{currentstroke}{rgb}{0.000000,0.000000,1.000000}%
\pgfsetstrokecolor{currentstroke}%
\pgfsetdash{}{0pt}%
\pgfpathmoveto{\pgfqpoint{4.867200in}{2.126213in}}%
\pgfpathlineto{\pgfqpoint{4.867200in}{2.211981in}}%
\pgfusepath{stroke}%
\end{pgfscope}%
\begin{pgfscope}%
\pgfpathrectangle{\pgfqpoint{0.800000in}{0.528000in}}{\pgfqpoint{4.960000in}{3.696000in}}%
\pgfusepath{clip}%
\pgfsetbuttcap%
\pgfsetroundjoin%
\pgfsetlinewidth{0.501875pt}%
\definecolor{currentstroke}{rgb}{0.000000,0.000000,1.000000}%
\pgfsetstrokecolor{currentstroke}%
\pgfsetdash{}{0pt}%
\pgfpathmoveto{\pgfqpoint{4.885236in}{2.126213in}}%
\pgfpathlineto{\pgfqpoint{4.885236in}{2.211981in}}%
\pgfusepath{stroke}%
\end{pgfscope}%
\begin{pgfscope}%
\pgfpathrectangle{\pgfqpoint{0.800000in}{0.528000in}}{\pgfqpoint{4.960000in}{3.696000in}}%
\pgfusepath{clip}%
\pgfsetbuttcap%
\pgfsetroundjoin%
\pgfsetlinewidth{0.501875pt}%
\definecolor{currentstroke}{rgb}{0.000000,0.000000,1.000000}%
\pgfsetstrokecolor{currentstroke}%
\pgfsetdash{}{0pt}%
\pgfpathmoveto{\pgfqpoint{4.903273in}{2.254865in}}%
\pgfpathlineto{\pgfqpoint{4.903273in}{2.340634in}}%
\pgfusepath{stroke}%
\end{pgfscope}%
\begin{pgfscope}%
\pgfpathrectangle{\pgfqpoint{0.800000in}{0.528000in}}{\pgfqpoint{4.960000in}{3.696000in}}%
\pgfusepath{clip}%
\pgfsetbuttcap%
\pgfsetroundjoin%
\pgfsetlinewidth{0.501875pt}%
\definecolor{currentstroke}{rgb}{0.000000,0.000000,1.000000}%
\pgfsetstrokecolor{currentstroke}%
\pgfsetdash{}{0pt}%
\pgfpathmoveto{\pgfqpoint{4.921309in}{2.383518in}}%
\pgfpathlineto{\pgfqpoint{4.921309in}{2.469286in}}%
\pgfusepath{stroke}%
\end{pgfscope}%
\begin{pgfscope}%
\pgfpathrectangle{\pgfqpoint{0.800000in}{0.528000in}}{\pgfqpoint{4.960000in}{3.696000in}}%
\pgfusepath{clip}%
\pgfsetbuttcap%
\pgfsetroundjoin%
\pgfsetlinewidth{0.501875pt}%
\definecolor{currentstroke}{rgb}{0.000000,0.000000,1.000000}%
\pgfsetstrokecolor{currentstroke}%
\pgfsetdash{}{0pt}%
\pgfpathmoveto{\pgfqpoint{4.939345in}{2.555055in}}%
\pgfpathlineto{\pgfqpoint{4.939345in}{2.640823in}}%
\pgfusepath{stroke}%
\end{pgfscope}%
\begin{pgfscope}%
\pgfpathrectangle{\pgfqpoint{0.800000in}{0.528000in}}{\pgfqpoint{4.960000in}{3.696000in}}%
\pgfusepath{clip}%
\pgfsetbuttcap%
\pgfsetroundjoin%
\pgfsetlinewidth{0.501875pt}%
\definecolor{currentstroke}{rgb}{0.000000,0.000000,1.000000}%
\pgfsetstrokecolor{currentstroke}%
\pgfsetdash{}{0pt}%
\pgfpathmoveto{\pgfqpoint{4.957382in}{2.683707in}}%
\pgfpathlineto{\pgfqpoint{4.957382in}{2.769475in}}%
\pgfusepath{stroke}%
\end{pgfscope}%
\begin{pgfscope}%
\pgfpathrectangle{\pgfqpoint{0.800000in}{0.528000in}}{\pgfqpoint{4.960000in}{3.696000in}}%
\pgfusepath{clip}%
\pgfsetbuttcap%
\pgfsetroundjoin%
\pgfsetlinewidth{0.501875pt}%
\definecolor{currentstroke}{rgb}{0.000000,0.000000,1.000000}%
\pgfsetstrokecolor{currentstroke}%
\pgfsetdash{}{0pt}%
\pgfpathmoveto{\pgfqpoint{4.975418in}{2.769475in}}%
\pgfpathlineto{\pgfqpoint{4.975418in}{2.855244in}}%
\pgfusepath{stroke}%
\end{pgfscope}%
\begin{pgfscope}%
\pgfpathrectangle{\pgfqpoint{0.800000in}{0.528000in}}{\pgfqpoint{4.960000in}{3.696000in}}%
\pgfusepath{clip}%
\pgfsetbuttcap%
\pgfsetroundjoin%
\pgfsetlinewidth{0.501875pt}%
\definecolor{currentstroke}{rgb}{0.000000,0.000000,1.000000}%
\pgfsetstrokecolor{currentstroke}%
\pgfsetdash{}{0pt}%
\pgfpathmoveto{\pgfqpoint{4.993455in}{2.812359in}}%
\pgfpathlineto{\pgfqpoint{4.993455in}{2.898128in}}%
\pgfusepath{stroke}%
\end{pgfscope}%
\begin{pgfscope}%
\pgfpathrectangle{\pgfqpoint{0.800000in}{0.528000in}}{\pgfqpoint{4.960000in}{3.696000in}}%
\pgfusepath{clip}%
\pgfsetbuttcap%
\pgfsetroundjoin%
\pgfsetlinewidth{0.501875pt}%
\definecolor{currentstroke}{rgb}{0.000000,0.000000,1.000000}%
\pgfsetstrokecolor{currentstroke}%
\pgfsetdash{}{0pt}%
\pgfpathmoveto{\pgfqpoint{5.011491in}{2.812359in}}%
\pgfpathlineto{\pgfqpoint{5.011491in}{2.898128in}}%
\pgfusepath{stroke}%
\end{pgfscope}%
\begin{pgfscope}%
\pgfpathrectangle{\pgfqpoint{0.800000in}{0.528000in}}{\pgfqpoint{4.960000in}{3.696000in}}%
\pgfusepath{clip}%
\pgfsetbuttcap%
\pgfsetroundjoin%
\pgfsetlinewidth{0.501875pt}%
\definecolor{currentstroke}{rgb}{0.000000,0.000000,1.000000}%
\pgfsetstrokecolor{currentstroke}%
\pgfsetdash{}{0pt}%
\pgfpathmoveto{\pgfqpoint{5.029527in}{2.726591in}}%
\pgfpathlineto{\pgfqpoint{5.029527in}{2.812359in}}%
\pgfusepath{stroke}%
\end{pgfscope}%
\begin{pgfscope}%
\pgfpathrectangle{\pgfqpoint{0.800000in}{0.528000in}}{\pgfqpoint{4.960000in}{3.696000in}}%
\pgfusepath{clip}%
\pgfsetbuttcap%
\pgfsetroundjoin%
\pgfsetlinewidth{0.501875pt}%
\definecolor{currentstroke}{rgb}{0.000000,0.000000,1.000000}%
\pgfsetstrokecolor{currentstroke}%
\pgfsetdash{}{0pt}%
\pgfpathmoveto{\pgfqpoint{5.047564in}{2.597939in}}%
\pgfpathlineto{\pgfqpoint{5.047564in}{2.683707in}}%
\pgfusepath{stroke}%
\end{pgfscope}%
\begin{pgfscope}%
\pgfpathrectangle{\pgfqpoint{0.800000in}{0.528000in}}{\pgfqpoint{4.960000in}{3.696000in}}%
\pgfusepath{clip}%
\pgfsetbuttcap%
\pgfsetroundjoin%
\pgfsetlinewidth{0.501875pt}%
\definecolor{currentstroke}{rgb}{0.000000,0.000000,1.000000}%
\pgfsetstrokecolor{currentstroke}%
\pgfsetdash{}{0pt}%
\pgfpathmoveto{\pgfqpoint{5.065600in}{2.469286in}}%
\pgfpathlineto{\pgfqpoint{5.065600in}{2.555055in}}%
\pgfusepath{stroke}%
\end{pgfscope}%
\begin{pgfscope}%
\pgfpathrectangle{\pgfqpoint{0.800000in}{0.528000in}}{\pgfqpoint{4.960000in}{3.696000in}}%
\pgfusepath{clip}%
\pgfsetbuttcap%
\pgfsetroundjoin%
\pgfsetlinewidth{0.501875pt}%
\definecolor{currentstroke}{rgb}{0.000000,0.000000,1.000000}%
\pgfsetstrokecolor{currentstroke}%
\pgfsetdash{}{0pt}%
\pgfpathmoveto{\pgfqpoint{5.083636in}{2.297750in}}%
\pgfpathlineto{\pgfqpoint{5.083636in}{2.383518in}}%
\pgfusepath{stroke}%
\end{pgfscope}%
\begin{pgfscope}%
\pgfpathrectangle{\pgfqpoint{0.800000in}{0.528000in}}{\pgfqpoint{4.960000in}{3.696000in}}%
\pgfusepath{clip}%
\pgfsetbuttcap%
\pgfsetroundjoin%
\pgfsetlinewidth{0.501875pt}%
\definecolor{currentstroke}{rgb}{0.000000,0.000000,1.000000}%
\pgfsetstrokecolor{currentstroke}%
\pgfsetdash{}{0pt}%
\pgfpathmoveto{\pgfqpoint{5.101673in}{2.211981in}}%
\pgfpathlineto{\pgfqpoint{5.101673in}{2.297750in}}%
\pgfusepath{stroke}%
\end{pgfscope}%
\begin{pgfscope}%
\pgfpathrectangle{\pgfqpoint{0.800000in}{0.528000in}}{\pgfqpoint{4.960000in}{3.696000in}}%
\pgfusepath{clip}%
\pgfsetbuttcap%
\pgfsetroundjoin%
\pgfsetlinewidth{0.501875pt}%
\definecolor{currentstroke}{rgb}{0.000000,0.000000,1.000000}%
\pgfsetstrokecolor{currentstroke}%
\pgfsetdash{}{0pt}%
\pgfpathmoveto{\pgfqpoint{5.119709in}{2.126213in}}%
\pgfpathlineto{\pgfqpoint{5.119709in}{2.211981in}}%
\pgfusepath{stroke}%
\end{pgfscope}%
\begin{pgfscope}%
\pgfpathrectangle{\pgfqpoint{0.800000in}{0.528000in}}{\pgfqpoint{4.960000in}{3.696000in}}%
\pgfusepath{clip}%
\pgfsetbuttcap%
\pgfsetroundjoin%
\pgfsetlinewidth{0.501875pt}%
\definecolor{currentstroke}{rgb}{0.000000,0.000000,1.000000}%
\pgfsetstrokecolor{currentstroke}%
\pgfsetdash{}{0pt}%
\pgfpathmoveto{\pgfqpoint{5.137745in}{2.126213in}}%
\pgfpathlineto{\pgfqpoint{5.137745in}{2.211981in}}%
\pgfusepath{stroke}%
\end{pgfscope}%
\begin{pgfscope}%
\pgfpathrectangle{\pgfqpoint{0.800000in}{0.528000in}}{\pgfqpoint{4.960000in}{3.696000in}}%
\pgfusepath{clip}%
\pgfsetbuttcap%
\pgfsetroundjoin%
\pgfsetlinewidth{0.501875pt}%
\definecolor{currentstroke}{rgb}{0.000000,0.000000,1.000000}%
\pgfsetstrokecolor{currentstroke}%
\pgfsetdash{}{0pt}%
\pgfpathmoveto{\pgfqpoint{5.155782in}{2.211981in}}%
\pgfpathlineto{\pgfqpoint{5.155782in}{2.297750in}}%
\pgfusepath{stroke}%
\end{pgfscope}%
\begin{pgfscope}%
\pgfpathrectangle{\pgfqpoint{0.800000in}{0.528000in}}{\pgfqpoint{4.960000in}{3.696000in}}%
\pgfusepath{clip}%
\pgfsetbuttcap%
\pgfsetroundjoin%
\pgfsetlinewidth{0.501875pt}%
\definecolor{currentstroke}{rgb}{0.000000,0.000000,1.000000}%
\pgfsetstrokecolor{currentstroke}%
\pgfsetdash{}{0pt}%
\pgfpathmoveto{\pgfqpoint{5.173818in}{2.297750in}}%
\pgfpathlineto{\pgfqpoint{5.173818in}{2.383518in}}%
\pgfusepath{stroke}%
\end{pgfscope}%
\begin{pgfscope}%
\pgfpathrectangle{\pgfqpoint{0.800000in}{0.528000in}}{\pgfqpoint{4.960000in}{3.696000in}}%
\pgfusepath{clip}%
\pgfsetbuttcap%
\pgfsetroundjoin%
\pgfsetlinewidth{0.501875pt}%
\definecolor{currentstroke}{rgb}{0.000000,0.000000,1.000000}%
\pgfsetstrokecolor{currentstroke}%
\pgfsetdash{}{0pt}%
\pgfpathmoveto{\pgfqpoint{5.191855in}{2.426402in}}%
\pgfpathlineto{\pgfqpoint{5.191855in}{2.512170in}}%
\pgfusepath{stroke}%
\end{pgfscope}%
\begin{pgfscope}%
\pgfpathrectangle{\pgfqpoint{0.800000in}{0.528000in}}{\pgfqpoint{4.960000in}{3.696000in}}%
\pgfusepath{clip}%
\pgfsetbuttcap%
\pgfsetroundjoin%
\pgfsetlinewidth{0.501875pt}%
\definecolor{currentstroke}{rgb}{0.000000,0.000000,1.000000}%
\pgfsetstrokecolor{currentstroke}%
\pgfsetdash{}{0pt}%
\pgfpathmoveto{\pgfqpoint{5.209891in}{2.597939in}}%
\pgfpathlineto{\pgfqpoint{5.209891in}{2.683707in}}%
\pgfusepath{stroke}%
\end{pgfscope}%
\begin{pgfscope}%
\pgfpathrectangle{\pgfqpoint{0.800000in}{0.528000in}}{\pgfqpoint{4.960000in}{3.696000in}}%
\pgfusepath{clip}%
\pgfsetbuttcap%
\pgfsetroundjoin%
\pgfsetlinewidth{0.501875pt}%
\definecolor{currentstroke}{rgb}{0.000000,0.000000,1.000000}%
\pgfsetstrokecolor{currentstroke}%
\pgfsetdash{}{0pt}%
\pgfpathmoveto{\pgfqpoint{5.227927in}{2.683707in}}%
\pgfpathlineto{\pgfqpoint{5.227927in}{2.769475in}}%
\pgfusepath{stroke}%
\end{pgfscope}%
\begin{pgfscope}%
\pgfpathrectangle{\pgfqpoint{0.800000in}{0.528000in}}{\pgfqpoint{4.960000in}{3.696000in}}%
\pgfusepath{clip}%
\pgfsetbuttcap%
\pgfsetroundjoin%
\pgfsetlinewidth{0.501875pt}%
\definecolor{currentstroke}{rgb}{0.000000,0.000000,1.000000}%
\pgfsetstrokecolor{currentstroke}%
\pgfsetdash{}{0pt}%
\pgfpathmoveto{\pgfqpoint{5.245964in}{2.769475in}}%
\pgfpathlineto{\pgfqpoint{5.245964in}{2.855244in}}%
\pgfusepath{stroke}%
\end{pgfscope}%
\begin{pgfscope}%
\pgfpathrectangle{\pgfqpoint{0.800000in}{0.528000in}}{\pgfqpoint{4.960000in}{3.696000in}}%
\pgfusepath{clip}%
\pgfsetbuttcap%
\pgfsetroundjoin%
\pgfsetlinewidth{0.501875pt}%
\definecolor{currentstroke}{rgb}{0.000000,0.000000,1.000000}%
\pgfsetstrokecolor{currentstroke}%
\pgfsetdash{}{0pt}%
\pgfpathmoveto{\pgfqpoint{5.264000in}{2.812359in}}%
\pgfpathlineto{\pgfqpoint{5.264000in}{2.898128in}}%
\pgfusepath{stroke}%
\end{pgfscope}%
\begin{pgfscope}%
\pgfpathrectangle{\pgfqpoint{0.800000in}{0.528000in}}{\pgfqpoint{4.960000in}{3.696000in}}%
\pgfusepath{clip}%
\pgfsetbuttcap%
\pgfsetroundjoin%
\pgfsetlinewidth{0.501875pt}%
\definecolor{currentstroke}{rgb}{0.000000,0.000000,1.000000}%
\pgfsetstrokecolor{currentstroke}%
\pgfsetdash{}{0pt}%
\pgfpathmoveto{\pgfqpoint{5.282036in}{2.769475in}}%
\pgfpathlineto{\pgfqpoint{5.282036in}{2.855244in}}%
\pgfusepath{stroke}%
\end{pgfscope}%
\begin{pgfscope}%
\pgfpathrectangle{\pgfqpoint{0.800000in}{0.528000in}}{\pgfqpoint{4.960000in}{3.696000in}}%
\pgfusepath{clip}%
\pgfsetbuttcap%
\pgfsetroundjoin%
\pgfsetlinewidth{0.501875pt}%
\definecolor{currentstroke}{rgb}{0.000000,0.000000,1.000000}%
\pgfsetstrokecolor{currentstroke}%
\pgfsetdash{}{0pt}%
\pgfpathmoveto{\pgfqpoint{5.300073in}{2.683707in}}%
\pgfpathlineto{\pgfqpoint{5.300073in}{2.769475in}}%
\pgfusepath{stroke}%
\end{pgfscope}%
\begin{pgfscope}%
\pgfpathrectangle{\pgfqpoint{0.800000in}{0.528000in}}{\pgfqpoint{4.960000in}{3.696000in}}%
\pgfusepath{clip}%
\pgfsetbuttcap%
\pgfsetroundjoin%
\pgfsetlinewidth{0.501875pt}%
\definecolor{currentstroke}{rgb}{0.000000,0.000000,1.000000}%
\pgfsetstrokecolor{currentstroke}%
\pgfsetdash{}{0pt}%
\pgfpathmoveto{\pgfqpoint{5.318109in}{2.555055in}}%
\pgfpathlineto{\pgfqpoint{5.318109in}{2.640823in}}%
\pgfusepath{stroke}%
\end{pgfscope}%
\begin{pgfscope}%
\pgfpathrectangle{\pgfqpoint{0.800000in}{0.528000in}}{\pgfqpoint{4.960000in}{3.696000in}}%
\pgfusepath{clip}%
\pgfsetbuttcap%
\pgfsetroundjoin%
\pgfsetlinewidth{0.501875pt}%
\definecolor{currentstroke}{rgb}{0.000000,0.000000,1.000000}%
\pgfsetstrokecolor{currentstroke}%
\pgfsetdash{}{0pt}%
\pgfpathmoveto{\pgfqpoint{5.336145in}{2.383518in}}%
\pgfpathlineto{\pgfqpoint{5.336145in}{2.469286in}}%
\pgfusepath{stroke}%
\end{pgfscope}%
\begin{pgfscope}%
\pgfpathrectangle{\pgfqpoint{0.800000in}{0.528000in}}{\pgfqpoint{4.960000in}{3.696000in}}%
\pgfusepath{clip}%
\pgfsetbuttcap%
\pgfsetroundjoin%
\pgfsetlinewidth{0.501875pt}%
\definecolor{currentstroke}{rgb}{0.000000,0.000000,1.000000}%
\pgfsetstrokecolor{currentstroke}%
\pgfsetdash{}{0pt}%
\pgfpathmoveto{\pgfqpoint{5.354182in}{2.297750in}}%
\pgfpathlineto{\pgfqpoint{5.354182in}{2.383518in}}%
\pgfusepath{stroke}%
\end{pgfscope}%
\begin{pgfscope}%
\pgfpathrectangle{\pgfqpoint{0.800000in}{0.528000in}}{\pgfqpoint{4.960000in}{3.696000in}}%
\pgfusepath{clip}%
\pgfsetbuttcap%
\pgfsetroundjoin%
\pgfsetlinewidth{0.501875pt}%
\definecolor{currentstroke}{rgb}{0.000000,0.000000,1.000000}%
\pgfsetstrokecolor{currentstroke}%
\pgfsetdash{}{0pt}%
\pgfpathmoveto{\pgfqpoint{5.372218in}{2.169097in}}%
\pgfpathlineto{\pgfqpoint{5.372218in}{2.254865in}}%
\pgfusepath{stroke}%
\end{pgfscope}%
\begin{pgfscope}%
\pgfpathrectangle{\pgfqpoint{0.800000in}{0.528000in}}{\pgfqpoint{4.960000in}{3.696000in}}%
\pgfusepath{clip}%
\pgfsetbuttcap%
\pgfsetroundjoin%
\pgfsetlinewidth{0.501875pt}%
\definecolor{currentstroke}{rgb}{0.000000,0.000000,1.000000}%
\pgfsetstrokecolor{currentstroke}%
\pgfsetdash{}{0pt}%
\pgfpathmoveto{\pgfqpoint{5.390255in}{2.169097in}}%
\pgfpathlineto{\pgfqpoint{5.390255in}{2.254865in}}%
\pgfusepath{stroke}%
\end{pgfscope}%
\begin{pgfscope}%
\pgfpathrectangle{\pgfqpoint{0.800000in}{0.528000in}}{\pgfqpoint{4.960000in}{3.696000in}}%
\pgfusepath{clip}%
\pgfsetbuttcap%
\pgfsetroundjoin%
\pgfsetlinewidth{0.501875pt}%
\definecolor{currentstroke}{rgb}{0.000000,0.000000,1.000000}%
\pgfsetstrokecolor{currentstroke}%
\pgfsetdash{}{0pt}%
\pgfpathmoveto{\pgfqpoint{5.408291in}{2.169097in}}%
\pgfpathlineto{\pgfqpoint{5.408291in}{2.254865in}}%
\pgfusepath{stroke}%
\end{pgfscope}%
\begin{pgfscope}%
\pgfpathrectangle{\pgfqpoint{0.800000in}{0.528000in}}{\pgfqpoint{4.960000in}{3.696000in}}%
\pgfusepath{clip}%
\pgfsetbuttcap%
\pgfsetroundjoin%
\pgfsetlinewidth{0.501875pt}%
\definecolor{currentstroke}{rgb}{0.000000,0.000000,1.000000}%
\pgfsetstrokecolor{currentstroke}%
\pgfsetdash{}{0pt}%
\pgfpathmoveto{\pgfqpoint{5.426327in}{2.254865in}}%
\pgfpathlineto{\pgfqpoint{5.426327in}{2.340634in}}%
\pgfusepath{stroke}%
\end{pgfscope}%
\begin{pgfscope}%
\pgfpathrectangle{\pgfqpoint{0.800000in}{0.528000in}}{\pgfqpoint{4.960000in}{3.696000in}}%
\pgfusepath{clip}%
\pgfsetbuttcap%
\pgfsetroundjoin%
\pgfsetlinewidth{0.501875pt}%
\definecolor{currentstroke}{rgb}{0.000000,0.000000,1.000000}%
\pgfsetstrokecolor{currentstroke}%
\pgfsetdash{}{0pt}%
\pgfpathmoveto{\pgfqpoint{5.444364in}{2.383518in}}%
\pgfpathlineto{\pgfqpoint{5.444364in}{2.469286in}}%
\pgfusepath{stroke}%
\end{pgfscope}%
\begin{pgfscope}%
\pgfpathrectangle{\pgfqpoint{0.800000in}{0.528000in}}{\pgfqpoint{4.960000in}{3.696000in}}%
\pgfusepath{clip}%
\pgfsetbuttcap%
\pgfsetroundjoin%
\pgfsetlinewidth{0.501875pt}%
\definecolor{currentstroke}{rgb}{0.000000,0.000000,1.000000}%
\pgfsetstrokecolor{currentstroke}%
\pgfsetdash{}{0pt}%
\pgfpathmoveto{\pgfqpoint{5.462400in}{2.512170in}}%
\pgfpathlineto{\pgfqpoint{5.462400in}{2.597939in}}%
\pgfusepath{stroke}%
\end{pgfscope}%
\begin{pgfscope}%
\pgfpathrectangle{\pgfqpoint{0.800000in}{0.528000in}}{\pgfqpoint{4.960000in}{3.696000in}}%
\pgfusepath{clip}%
\pgfsetbuttcap%
\pgfsetroundjoin%
\pgfsetlinewidth{0.501875pt}%
\definecolor{currentstroke}{rgb}{0.000000,0.000000,1.000000}%
\pgfsetstrokecolor{currentstroke}%
\pgfsetdash{}{0pt}%
\pgfpathmoveto{\pgfqpoint{5.480436in}{2.640823in}}%
\pgfpathlineto{\pgfqpoint{5.480436in}{2.726591in}}%
\pgfusepath{stroke}%
\end{pgfscope}%
\begin{pgfscope}%
\pgfpathrectangle{\pgfqpoint{0.800000in}{0.528000in}}{\pgfqpoint{4.960000in}{3.696000in}}%
\pgfusepath{clip}%
\pgfsetbuttcap%
\pgfsetroundjoin%
\pgfsetlinewidth{0.501875pt}%
\definecolor{currentstroke}{rgb}{0.000000,0.000000,1.000000}%
\pgfsetstrokecolor{currentstroke}%
\pgfsetdash{}{0pt}%
\pgfpathmoveto{\pgfqpoint{5.498473in}{2.726591in}}%
\pgfpathlineto{\pgfqpoint{5.498473in}{2.812359in}}%
\pgfusepath{stroke}%
\end{pgfscope}%
\begin{pgfscope}%
\pgfpathrectangle{\pgfqpoint{0.800000in}{0.528000in}}{\pgfqpoint{4.960000in}{3.696000in}}%
\pgfusepath{clip}%
\pgfsetbuttcap%
\pgfsetroundjoin%
\pgfsetlinewidth{0.501875pt}%
\definecolor{currentstroke}{rgb}{0.000000,0.000000,1.000000}%
\pgfsetstrokecolor{currentstroke}%
\pgfsetdash{}{0pt}%
\pgfpathmoveto{\pgfqpoint{5.516509in}{2.769475in}}%
\pgfpathlineto{\pgfqpoint{5.516509in}{2.855244in}}%
\pgfusepath{stroke}%
\end{pgfscope}%
\begin{pgfscope}%
\pgfpathrectangle{\pgfqpoint{0.800000in}{0.528000in}}{\pgfqpoint{4.960000in}{3.696000in}}%
\pgfusepath{clip}%
\pgfsetbuttcap%
\pgfsetroundjoin%
\pgfsetlinewidth{0.501875pt}%
\definecolor{currentstroke}{rgb}{0.000000,0.000000,1.000000}%
\pgfsetstrokecolor{currentstroke}%
\pgfsetdash{}{0pt}%
\pgfpathmoveto{\pgfqpoint{5.534545in}{2.769475in}}%
\pgfpathlineto{\pgfqpoint{5.534545in}{2.855244in}}%
\pgfusepath{stroke}%
\end{pgfscope}%
\begin{pgfscope}%
\pgfpathrectangle{\pgfqpoint{0.800000in}{0.528000in}}{\pgfqpoint{4.960000in}{3.696000in}}%
\pgfusepath{clip}%
\pgfsetbuttcap%
\pgfsetroundjoin%
\definecolor{currentfill}{rgb}{0.000000,0.000000,1.000000}%
\pgfsetfillcolor{currentfill}%
\pgfsetlinewidth{0.501875pt}%
\definecolor{currentstroke}{rgb}{0.000000,0.000000,1.000000}%
\pgfsetstrokecolor{currentstroke}%
\pgfsetdash{}{0pt}%
\pgfsys@defobject{currentmarker}{\pgfqpoint{-0.027778in}{-0.000000in}}{\pgfqpoint{0.027778in}{0.000000in}}{%
\pgfpathmoveto{\pgfqpoint{0.027778in}{-0.000000in}}%
\pgfpathlineto{\pgfqpoint{-0.027778in}{0.000000in}}%
\pgfusepath{stroke,fill}%
}%
\begin{pgfscope}%
\pgfsys@transformshift{1.043491in}{3.541390in}%
\pgfsys@useobject{currentmarker}{}%
\end{pgfscope}%
\begin{pgfscope}%
\pgfsys@transformshift{1.061527in}{3.412738in}%
\pgfsys@useobject{currentmarker}{}%
\end{pgfscope}%
\begin{pgfscope}%
\pgfsys@transformshift{1.079564in}{3.069664in}%
\pgfsys@useobject{currentmarker}{}%
\end{pgfscope}%
\begin{pgfscope}%
\pgfsys@transformshift{1.097600in}{2.640823in}%
\pgfsys@useobject{currentmarker}{}%
\end{pgfscope}%
\begin{pgfscope}%
\pgfsys@transformshift{1.115636in}{2.169097in}%
\pgfsys@useobject{currentmarker}{}%
\end{pgfscope}%
\begin{pgfscope}%
\pgfsys@transformshift{1.133673in}{1.611603in}%
\pgfsys@useobject{currentmarker}{}%
\end{pgfscope}%
\begin{pgfscope}%
\pgfsys@transformshift{1.151709in}{1.311414in}%
\pgfsys@useobject{currentmarker}{}%
\end{pgfscope}%
\begin{pgfscope}%
\pgfsys@transformshift{1.169745in}{1.225646in}%
\pgfsys@useobject{currentmarker}{}%
\end{pgfscope}%
\begin{pgfscope}%
\pgfsys@transformshift{1.187782in}{1.311414in}%
\pgfsys@useobject{currentmarker}{}%
\end{pgfscope}%
\begin{pgfscope}%
\pgfsys@transformshift{1.205818in}{1.611603in}%
\pgfsys@useobject{currentmarker}{}%
\end{pgfscope}%
\begin{pgfscope}%
\pgfsys@transformshift{1.223855in}{2.169097in}%
\pgfsys@useobject{currentmarker}{}%
\end{pgfscope}%
\begin{pgfscope}%
\pgfsys@transformshift{1.241891in}{2.597939in}%
\pgfsys@useobject{currentmarker}{}%
\end{pgfscope}%
\begin{pgfscope}%
\pgfsys@transformshift{1.259927in}{3.026780in}%
\pgfsys@useobject{currentmarker}{}%
\end{pgfscope}%
\begin{pgfscope}%
\pgfsys@transformshift{1.277964in}{3.326969in}%
\pgfsys@useobject{currentmarker}{}%
\end{pgfscope}%
\begin{pgfscope}%
\pgfsys@transformshift{1.296000in}{3.455622in}%
\pgfsys@useobject{currentmarker}{}%
\end{pgfscope}%
\begin{pgfscope}%
\pgfsys@transformshift{1.314036in}{3.455622in}%
\pgfsys@useobject{currentmarker}{}%
\end{pgfscope}%
\begin{pgfscope}%
\pgfsys@transformshift{1.332073in}{3.241201in}%
\pgfsys@useobject{currentmarker}{}%
\end{pgfscope}%
\begin{pgfscope}%
\pgfsys@transformshift{1.350109in}{2.898128in}%
\pgfsys@useobject{currentmarker}{}%
\end{pgfscope}%
\begin{pgfscope}%
\pgfsys@transformshift{1.368145in}{2.469286in}%
\pgfsys@useobject{currentmarker}{}%
\end{pgfscope}%
\begin{pgfscope}%
\pgfsys@transformshift{1.386182in}{2.083329in}%
\pgfsys@useobject{currentmarker}{}%
\end{pgfscope}%
\begin{pgfscope}%
\pgfsys@transformshift{1.404218in}{1.568719in}%
\pgfsys@useobject{currentmarker}{}%
\end{pgfscope}%
\begin{pgfscope}%
\pgfsys@transformshift{1.422255in}{1.354298in}%
\pgfsys@useobject{currentmarker}{}%
\end{pgfscope}%
\begin{pgfscope}%
\pgfsys@transformshift{1.440291in}{1.354298in}%
\pgfsys@useobject{currentmarker}{}%
\end{pgfscope}%
\begin{pgfscope}%
\pgfsys@transformshift{1.458327in}{1.482951in}%
\pgfsys@useobject{currentmarker}{}%
\end{pgfscope}%
\begin{pgfscope}%
\pgfsys@transformshift{1.476364in}{1.783140in}%
\pgfsys@useobject{currentmarker}{}%
\end{pgfscope}%
\begin{pgfscope}%
\pgfsys@transformshift{1.494400in}{2.297750in}%
\pgfsys@useobject{currentmarker}{}%
\end{pgfscope}%
\begin{pgfscope}%
\pgfsys@transformshift{1.512436in}{2.726591in}%
\pgfsys@useobject{currentmarker}{}%
\end{pgfscope}%
\begin{pgfscope}%
\pgfsys@transformshift{1.530473in}{3.069664in}%
\pgfsys@useobject{currentmarker}{}%
\end{pgfscope}%
\begin{pgfscope}%
\pgfsys@transformshift{1.548509in}{3.284085in}%
\pgfsys@useobject{currentmarker}{}%
\end{pgfscope}%
\begin{pgfscope}%
\pgfsys@transformshift{1.566545in}{3.369854in}%
\pgfsys@useobject{currentmarker}{}%
\end{pgfscope}%
\begin{pgfscope}%
\pgfsys@transformshift{1.584582in}{3.326969in}%
\pgfsys@useobject{currentmarker}{}%
\end{pgfscope}%
\begin{pgfscope}%
\pgfsys@transformshift{1.602618in}{3.069664in}%
\pgfsys@useobject{currentmarker}{}%
\end{pgfscope}%
\begin{pgfscope}%
\pgfsys@transformshift{1.620655in}{2.726591in}%
\pgfsys@useobject{currentmarker}{}%
\end{pgfscope}%
\begin{pgfscope}%
\pgfsys@transformshift{1.638691in}{2.340634in}%
\pgfsys@useobject{currentmarker}{}%
\end{pgfscope}%
\begin{pgfscope}%
\pgfsys@transformshift{1.656727in}{1.868908in}%
\pgfsys@useobject{currentmarker}{}%
\end{pgfscope}%
\begin{pgfscope}%
\pgfsys@transformshift{1.674764in}{1.568719in}%
\pgfsys@useobject{currentmarker}{}%
\end{pgfscope}%
\begin{pgfscope}%
\pgfsys@transformshift{1.692800in}{1.440067in}%
\pgfsys@useobject{currentmarker}{}%
\end{pgfscope}%
\begin{pgfscope}%
\pgfsys@transformshift{1.710836in}{1.440067in}%
\pgfsys@useobject{currentmarker}{}%
\end{pgfscope}%
\begin{pgfscope}%
\pgfsys@transformshift{1.728873in}{1.654487in}%
\pgfsys@useobject{currentmarker}{}%
\end{pgfscope}%
\begin{pgfscope}%
\pgfsys@transformshift{1.746909in}{2.083329in}%
\pgfsys@useobject{currentmarker}{}%
\end{pgfscope}%
\begin{pgfscope}%
\pgfsys@transformshift{1.764945in}{2.469286in}%
\pgfsys@useobject{currentmarker}{}%
\end{pgfscope}%
\begin{pgfscope}%
\pgfsys@transformshift{1.782982in}{2.812359in}%
\pgfsys@useobject{currentmarker}{}%
\end{pgfscope}%
\begin{pgfscope}%
\pgfsys@transformshift{1.801018in}{3.112549in}%
\pgfsys@useobject{currentmarker}{}%
\end{pgfscope}%
\begin{pgfscope}%
\pgfsys@transformshift{1.819055in}{3.284085in}%
\pgfsys@useobject{currentmarker}{}%
\end{pgfscope}%
\begin{pgfscope}%
\pgfsys@transformshift{1.837091in}{3.284085in}%
\pgfsys@useobject{currentmarker}{}%
\end{pgfscope}%
\begin{pgfscope}%
\pgfsys@transformshift{1.855127in}{3.198317in}%
\pgfsys@useobject{currentmarker}{}%
\end{pgfscope}%
\begin{pgfscope}%
\pgfsys@transformshift{1.873164in}{2.941012in}%
\pgfsys@useobject{currentmarker}{}%
\end{pgfscope}%
\begin{pgfscope}%
\pgfsys@transformshift{1.891200in}{2.597939in}%
\pgfsys@useobject{currentmarker}{}%
\end{pgfscope}%
\begin{pgfscope}%
\pgfsys@transformshift{1.909236in}{2.254865in}%
\pgfsys@useobject{currentmarker}{}%
\end{pgfscope}%
\begin{pgfscope}%
\pgfsys@transformshift{1.927273in}{1.783140in}%
\pgfsys@useobject{currentmarker}{}%
\end{pgfscope}%
\begin{pgfscope}%
\pgfsys@transformshift{1.945309in}{1.568719in}%
\pgfsys@useobject{currentmarker}{}%
\end{pgfscope}%
\begin{pgfscope}%
\pgfsys@transformshift{1.963345in}{1.482951in}%
\pgfsys@useobject{currentmarker}{}%
\end{pgfscope}%
\begin{pgfscope}%
\pgfsys@transformshift{1.981382in}{1.568719in}%
\pgfsys@useobject{currentmarker}{}%
\end{pgfscope}%
\begin{pgfscope}%
\pgfsys@transformshift{1.999418in}{1.783140in}%
\pgfsys@useobject{currentmarker}{}%
\end{pgfscope}%
\begin{pgfscope}%
\pgfsys@transformshift{2.017455in}{2.211981in}%
\pgfsys@useobject{currentmarker}{}%
\end{pgfscope}%
\begin{pgfscope}%
\pgfsys@transformshift{2.035491in}{2.597939in}%
\pgfsys@useobject{currentmarker}{}%
\end{pgfscope}%
\begin{pgfscope}%
\pgfsys@transformshift{2.053527in}{2.898128in}%
\pgfsys@useobject{currentmarker}{}%
\end{pgfscope}%
\begin{pgfscope}%
\pgfsys@transformshift{2.071564in}{3.112549in}%
\pgfsys@useobject{currentmarker}{}%
\end{pgfscope}%
\begin{pgfscope}%
\pgfsys@transformshift{2.089600in}{3.241201in}%
\pgfsys@useobject{currentmarker}{}%
\end{pgfscope}%
\begin{pgfscope}%
\pgfsys@transformshift{2.107636in}{3.241201in}%
\pgfsys@useobject{currentmarker}{}%
\end{pgfscope}%
\begin{pgfscope}%
\pgfsys@transformshift{2.125673in}{3.069664in}%
\pgfsys@useobject{currentmarker}{}%
\end{pgfscope}%
\begin{pgfscope}%
\pgfsys@transformshift{2.143709in}{2.812359in}%
\pgfsys@useobject{currentmarker}{}%
\end{pgfscope}%
\begin{pgfscope}%
\pgfsys@transformshift{2.161745in}{2.469286in}%
\pgfsys@useobject{currentmarker}{}%
\end{pgfscope}%
\begin{pgfscope}%
\pgfsys@transformshift{2.179782in}{2.169097in}%
\pgfsys@useobject{currentmarker}{}%
\end{pgfscope}%
\begin{pgfscope}%
\pgfsys@transformshift{2.197818in}{1.740256in}%
\pgfsys@useobject{currentmarker}{}%
\end{pgfscope}%
\begin{pgfscope}%
\pgfsys@transformshift{2.215855in}{1.611603in}%
\pgfsys@useobject{currentmarker}{}%
\end{pgfscope}%
\begin{pgfscope}%
\pgfsys@transformshift{2.233891in}{1.568719in}%
\pgfsys@useobject{currentmarker}{}%
\end{pgfscope}%
\begin{pgfscope}%
\pgfsys@transformshift{2.251927in}{1.697371in}%
\pgfsys@useobject{currentmarker}{}%
\end{pgfscope}%
\begin{pgfscope}%
\pgfsys@transformshift{2.269964in}{1.954676in}%
\pgfsys@useobject{currentmarker}{}%
\end{pgfscope}%
\begin{pgfscope}%
\pgfsys@transformshift{2.288000in}{2.383518in}%
\pgfsys@useobject{currentmarker}{}%
\end{pgfscope}%
\begin{pgfscope}%
\pgfsys@transformshift{2.306036in}{2.683707in}%
\pgfsys@useobject{currentmarker}{}%
\end{pgfscope}%
\begin{pgfscope}%
\pgfsys@transformshift{2.324073in}{2.941012in}%
\pgfsys@useobject{currentmarker}{}%
\end{pgfscope}%
\begin{pgfscope}%
\pgfsys@transformshift{2.342109in}{3.112549in}%
\pgfsys@useobject{currentmarker}{}%
\end{pgfscope}%
\begin{pgfscope}%
\pgfsys@transformshift{2.360145in}{3.198317in}%
\pgfsys@useobject{currentmarker}{}%
\end{pgfscope}%
\begin{pgfscope}%
\pgfsys@transformshift{2.378182in}{3.112549in}%
\pgfsys@useobject{currentmarker}{}%
\end{pgfscope}%
\begin{pgfscope}%
\pgfsys@transformshift{2.396218in}{2.941012in}%
\pgfsys@useobject{currentmarker}{}%
\end{pgfscope}%
\begin{pgfscope}%
\pgfsys@transformshift{2.414255in}{2.683707in}%
\pgfsys@useobject{currentmarker}{}%
\end{pgfscope}%
\begin{pgfscope}%
\pgfsys@transformshift{2.432291in}{2.383518in}%
\pgfsys@useobject{currentmarker}{}%
\end{pgfscope}%
\begin{pgfscope}%
\pgfsys@transformshift{2.450327in}{1.954676in}%
\pgfsys@useobject{currentmarker}{}%
\end{pgfscope}%
\begin{pgfscope}%
\pgfsys@transformshift{2.468364in}{1.740256in}%
\pgfsys@useobject{currentmarker}{}%
\end{pgfscope}%
\begin{pgfscope}%
\pgfsys@transformshift{2.486400in}{1.611603in}%
\pgfsys@useobject{currentmarker}{}%
\end{pgfscope}%
\begin{pgfscope}%
\pgfsys@transformshift{2.504436in}{1.654487in}%
\pgfsys@useobject{currentmarker}{}%
\end{pgfscope}%
\begin{pgfscope}%
\pgfsys@transformshift{2.522473in}{1.826024in}%
\pgfsys@useobject{currentmarker}{}%
\end{pgfscope}%
\begin{pgfscope}%
\pgfsys@transformshift{2.540509in}{2.169097in}%
\pgfsys@useobject{currentmarker}{}%
\end{pgfscope}%
\begin{pgfscope}%
\pgfsys@transformshift{2.558545in}{2.469286in}%
\pgfsys@useobject{currentmarker}{}%
\end{pgfscope}%
\begin{pgfscope}%
\pgfsys@transformshift{2.576582in}{2.769475in}%
\pgfsys@useobject{currentmarker}{}%
\end{pgfscope}%
\begin{pgfscope}%
\pgfsys@transformshift{2.594618in}{2.983896in}%
\pgfsys@useobject{currentmarker}{}%
\end{pgfscope}%
\begin{pgfscope}%
\pgfsys@transformshift{2.612655in}{3.112549in}%
\pgfsys@useobject{currentmarker}{}%
\end{pgfscope}%
\begin{pgfscope}%
\pgfsys@transformshift{2.630691in}{3.112549in}%
\pgfsys@useobject{currentmarker}{}%
\end{pgfscope}%
\begin{pgfscope}%
\pgfsys@transformshift{2.648727in}{3.026780in}%
\pgfsys@useobject{currentmarker}{}%
\end{pgfscope}%
\begin{pgfscope}%
\pgfsys@transformshift{2.666764in}{2.812359in}%
\pgfsys@useobject{currentmarker}{}%
\end{pgfscope}%
\begin{pgfscope}%
\pgfsys@transformshift{2.684800in}{2.555055in}%
\pgfsys@useobject{currentmarker}{}%
\end{pgfscope}%
\begin{pgfscope}%
\pgfsys@transformshift{2.702836in}{2.297750in}%
\pgfsys@useobject{currentmarker}{}%
\end{pgfscope}%
\begin{pgfscope}%
\pgfsys@transformshift{2.720873in}{1.997561in}%
\pgfsys@useobject{currentmarker}{}%
\end{pgfscope}%
\begin{pgfscope}%
\pgfsys@transformshift{2.738909in}{1.740256in}%
\pgfsys@useobject{currentmarker}{}%
\end{pgfscope}%
\begin{pgfscope}%
\pgfsys@transformshift{2.756945in}{1.697371in}%
\pgfsys@useobject{currentmarker}{}%
\end{pgfscope}%
\begin{pgfscope}%
\pgfsys@transformshift{2.774982in}{1.740256in}%
\pgfsys@useobject{currentmarker}{}%
\end{pgfscope}%
\begin{pgfscope}%
\pgfsys@transformshift{2.793018in}{2.040445in}%
\pgfsys@useobject{currentmarker}{}%
\end{pgfscope}%
\begin{pgfscope}%
\pgfsys@transformshift{2.811055in}{2.297750in}%
\pgfsys@useobject{currentmarker}{}%
\end{pgfscope}%
\begin{pgfscope}%
\pgfsys@transformshift{2.829091in}{2.597939in}%
\pgfsys@useobject{currentmarker}{}%
\end{pgfscope}%
\begin{pgfscope}%
\pgfsys@transformshift{2.847127in}{2.812359in}%
\pgfsys@useobject{currentmarker}{}%
\end{pgfscope}%
\begin{pgfscope}%
\pgfsys@transformshift{2.865164in}{3.026780in}%
\pgfsys@useobject{currentmarker}{}%
\end{pgfscope}%
\begin{pgfscope}%
\pgfsys@transformshift{2.883200in}{3.069664in}%
\pgfsys@useobject{currentmarker}{}%
\end{pgfscope}%
\begin{pgfscope}%
\pgfsys@transformshift{2.901236in}{3.069664in}%
\pgfsys@useobject{currentmarker}{}%
\end{pgfscope}%
\begin{pgfscope}%
\pgfsys@transformshift{2.919273in}{2.941012in}%
\pgfsys@useobject{currentmarker}{}%
\end{pgfscope}%
\begin{pgfscope}%
\pgfsys@transformshift{2.937309in}{2.683707in}%
\pgfsys@useobject{currentmarker}{}%
\end{pgfscope}%
\begin{pgfscope}%
\pgfsys@transformshift{2.955345in}{2.469286in}%
\pgfsys@useobject{currentmarker}{}%
\end{pgfscope}%
\begin{pgfscope}%
\pgfsys@transformshift{2.973382in}{2.211981in}%
\pgfsys@useobject{currentmarker}{}%
\end{pgfscope}%
\begin{pgfscope}%
\pgfsys@transformshift{2.991418in}{1.997561in}%
\pgfsys@useobject{currentmarker}{}%
\end{pgfscope}%
\begin{pgfscope}%
\pgfsys@transformshift{3.009455in}{1.740256in}%
\pgfsys@useobject{currentmarker}{}%
\end{pgfscope}%
\begin{pgfscope}%
\pgfsys@transformshift{3.027491in}{1.740256in}%
\pgfsys@useobject{currentmarker}{}%
\end{pgfscope}%
\begin{pgfscope}%
\pgfsys@transformshift{3.045527in}{1.868908in}%
\pgfsys@useobject{currentmarker}{}%
\end{pgfscope}%
\begin{pgfscope}%
\pgfsys@transformshift{3.063564in}{2.169097in}%
\pgfsys@useobject{currentmarker}{}%
\end{pgfscope}%
\begin{pgfscope}%
\pgfsys@transformshift{3.081600in}{2.426402in}%
\pgfsys@useobject{currentmarker}{}%
\end{pgfscope}%
\begin{pgfscope}%
\pgfsys@transformshift{3.099636in}{2.640823in}%
\pgfsys@useobject{currentmarker}{}%
\end{pgfscope}%
\begin{pgfscope}%
\pgfsys@transformshift{3.117673in}{2.855244in}%
\pgfsys@useobject{currentmarker}{}%
\end{pgfscope}%
\begin{pgfscope}%
\pgfsys@transformshift{3.135709in}{2.983896in}%
\pgfsys@useobject{currentmarker}{}%
\end{pgfscope}%
\begin{pgfscope}%
\pgfsys@transformshift{3.153745in}{3.026780in}%
\pgfsys@useobject{currentmarker}{}%
\end{pgfscope}%
\begin{pgfscope}%
\pgfsys@transformshift{3.171782in}{2.983896in}%
\pgfsys@useobject{currentmarker}{}%
\end{pgfscope}%
\begin{pgfscope}%
\pgfsys@transformshift{3.189818in}{2.812359in}%
\pgfsys@useobject{currentmarker}{}%
\end{pgfscope}%
\begin{pgfscope}%
\pgfsys@transformshift{3.207855in}{2.597939in}%
\pgfsys@useobject{currentmarker}{}%
\end{pgfscope}%
\begin{pgfscope}%
\pgfsys@transformshift{3.225891in}{2.383518in}%
\pgfsys@useobject{currentmarker}{}%
\end{pgfscope}%
\begin{pgfscope}%
\pgfsys@transformshift{3.243927in}{2.126213in}%
\pgfsys@useobject{currentmarker}{}%
\end{pgfscope}%
\begin{pgfscope}%
\pgfsys@transformshift{3.261964in}{1.868908in}%
\pgfsys@useobject{currentmarker}{}%
\end{pgfscope}%
\begin{pgfscope}%
\pgfsys@transformshift{3.280000in}{1.783140in}%
\pgfsys@useobject{currentmarker}{}%
\end{pgfscope}%
\begin{pgfscope}%
\pgfsys@transformshift{3.298036in}{1.826024in}%
\pgfsys@useobject{currentmarker}{}%
\end{pgfscope}%
\begin{pgfscope}%
\pgfsys@transformshift{3.316073in}{1.954676in}%
\pgfsys@useobject{currentmarker}{}%
\end{pgfscope}%
\begin{pgfscope}%
\pgfsys@transformshift{3.334109in}{2.254865in}%
\pgfsys@useobject{currentmarker}{}%
\end{pgfscope}%
\begin{pgfscope}%
\pgfsys@transformshift{3.352145in}{2.512170in}%
\pgfsys@useobject{currentmarker}{}%
\end{pgfscope}%
\begin{pgfscope}%
\pgfsys@transformshift{3.370182in}{2.726591in}%
\pgfsys@useobject{currentmarker}{}%
\end{pgfscope}%
\begin{pgfscope}%
\pgfsys@transformshift{3.388218in}{2.898128in}%
\pgfsys@useobject{currentmarker}{}%
\end{pgfscope}%
\begin{pgfscope}%
\pgfsys@transformshift{3.406255in}{2.983896in}%
\pgfsys@useobject{currentmarker}{}%
\end{pgfscope}%
\begin{pgfscope}%
\pgfsys@transformshift{3.424291in}{2.983896in}%
\pgfsys@useobject{currentmarker}{}%
\end{pgfscope}%
\begin{pgfscope}%
\pgfsys@transformshift{3.442327in}{2.898128in}%
\pgfsys@useobject{currentmarker}{}%
\end{pgfscope}%
\begin{pgfscope}%
\pgfsys@transformshift{3.460364in}{2.726591in}%
\pgfsys@useobject{currentmarker}{}%
\end{pgfscope}%
\begin{pgfscope}%
\pgfsys@transformshift{3.478400in}{2.512170in}%
\pgfsys@useobject{currentmarker}{}%
\end{pgfscope}%
\begin{pgfscope}%
\pgfsys@transformshift{3.496436in}{2.297750in}%
\pgfsys@useobject{currentmarker}{}%
\end{pgfscope}%
\begin{pgfscope}%
\pgfsys@transformshift{3.514473in}{2.083329in}%
\pgfsys@useobject{currentmarker}{}%
\end{pgfscope}%
\begin{pgfscope}%
\pgfsys@transformshift{3.532509in}{1.868908in}%
\pgfsys@useobject{currentmarker}{}%
\end{pgfscope}%
\begin{pgfscope}%
\pgfsys@transformshift{3.550545in}{1.826024in}%
\pgfsys@useobject{currentmarker}{}%
\end{pgfscope}%
\begin{pgfscope}%
\pgfsys@transformshift{3.568582in}{1.868908in}%
\pgfsys@useobject{currentmarker}{}%
\end{pgfscope}%
\begin{pgfscope}%
\pgfsys@transformshift{3.586618in}{2.126213in}%
\pgfsys@useobject{currentmarker}{}%
\end{pgfscope}%
\begin{pgfscope}%
\pgfsys@transformshift{3.604655in}{2.340634in}%
\pgfsys@useobject{currentmarker}{}%
\end{pgfscope}%
\begin{pgfscope}%
\pgfsys@transformshift{3.622691in}{2.597939in}%
\pgfsys@useobject{currentmarker}{}%
\end{pgfscope}%
\begin{pgfscope}%
\pgfsys@transformshift{3.640727in}{2.769475in}%
\pgfsys@useobject{currentmarker}{}%
\end{pgfscope}%
\begin{pgfscope}%
\pgfsys@transformshift{3.658764in}{2.898128in}%
\pgfsys@useobject{currentmarker}{}%
\end{pgfscope}%
\begin{pgfscope}%
\pgfsys@transformshift{3.676800in}{2.941012in}%
\pgfsys@useobject{currentmarker}{}%
\end{pgfscope}%
\begin{pgfscope}%
\pgfsys@transformshift{3.694836in}{2.941012in}%
\pgfsys@useobject{currentmarker}{}%
\end{pgfscope}%
\begin{pgfscope}%
\pgfsys@transformshift{3.712873in}{2.812359in}%
\pgfsys@useobject{currentmarker}{}%
\end{pgfscope}%
\begin{pgfscope}%
\pgfsys@transformshift{3.730909in}{2.640823in}%
\pgfsys@useobject{currentmarker}{}%
\end{pgfscope}%
\begin{pgfscope}%
\pgfsys@transformshift{3.748945in}{2.426402in}%
\pgfsys@useobject{currentmarker}{}%
\end{pgfscope}%
\begin{pgfscope}%
\pgfsys@transformshift{3.766982in}{2.254865in}%
\pgfsys@useobject{currentmarker}{}%
\end{pgfscope}%
\begin{pgfscope}%
\pgfsys@transformshift{3.785018in}{2.083329in}%
\pgfsys@useobject{currentmarker}{}%
\end{pgfscope}%
\begin{pgfscope}%
\pgfsys@transformshift{3.803055in}{1.997561in}%
\pgfsys@useobject{currentmarker}{}%
\end{pgfscope}%
\begin{pgfscope}%
\pgfsys@transformshift{3.821091in}{1.997561in}%
\pgfsys@useobject{currentmarker}{}%
\end{pgfscope}%
\begin{pgfscope}%
\pgfsys@transformshift{3.839127in}{2.083329in}%
\pgfsys@useobject{currentmarker}{}%
\end{pgfscope}%
\begin{pgfscope}%
\pgfsys@transformshift{3.857164in}{2.254865in}%
\pgfsys@useobject{currentmarker}{}%
\end{pgfscope}%
\begin{pgfscope}%
\pgfsys@transformshift{3.875200in}{2.469286in}%
\pgfsys@useobject{currentmarker}{}%
\end{pgfscope}%
\begin{pgfscope}%
\pgfsys@transformshift{3.893236in}{2.640823in}%
\pgfsys@useobject{currentmarker}{}%
\end{pgfscope}%
\begin{pgfscope}%
\pgfsys@transformshift{3.911273in}{2.812359in}%
\pgfsys@useobject{currentmarker}{}%
\end{pgfscope}%
\begin{pgfscope}%
\pgfsys@transformshift{3.929309in}{2.898128in}%
\pgfsys@useobject{currentmarker}{}%
\end{pgfscope}%
\begin{pgfscope}%
\pgfsys@transformshift{3.947345in}{2.941012in}%
\pgfsys@useobject{currentmarker}{}%
\end{pgfscope}%
\begin{pgfscope}%
\pgfsys@transformshift{3.965382in}{2.855244in}%
\pgfsys@useobject{currentmarker}{}%
\end{pgfscope}%
\begin{pgfscope}%
\pgfsys@transformshift{3.983418in}{2.726591in}%
\pgfsys@useobject{currentmarker}{}%
\end{pgfscope}%
\begin{pgfscope}%
\pgfsys@transformshift{4.001455in}{2.555055in}%
\pgfsys@useobject{currentmarker}{}%
\end{pgfscope}%
\begin{pgfscope}%
\pgfsys@transformshift{4.019491in}{2.383518in}%
\pgfsys@useobject{currentmarker}{}%
\end{pgfscope}%
\begin{pgfscope}%
\pgfsys@transformshift{4.037527in}{2.169097in}%
\pgfsys@useobject{currentmarker}{}%
\end{pgfscope}%
\begin{pgfscope}%
\pgfsys@transformshift{4.055564in}{2.040445in}%
\pgfsys@useobject{currentmarker}{}%
\end{pgfscope}%
\begin{pgfscope}%
\pgfsys@transformshift{4.073600in}{1.997561in}%
\pgfsys@useobject{currentmarker}{}%
\end{pgfscope}%
\begin{pgfscope}%
\pgfsys@transformshift{4.091636in}{2.040445in}%
\pgfsys@useobject{currentmarker}{}%
\end{pgfscope}%
\begin{pgfscope}%
\pgfsys@transformshift{4.109673in}{2.169097in}%
\pgfsys@useobject{currentmarker}{}%
\end{pgfscope}%
\begin{pgfscope}%
\pgfsys@transformshift{4.127709in}{2.340634in}%
\pgfsys@useobject{currentmarker}{}%
\end{pgfscope}%
\begin{pgfscope}%
\pgfsys@transformshift{4.145745in}{2.512170in}%
\pgfsys@useobject{currentmarker}{}%
\end{pgfscope}%
\begin{pgfscope}%
\pgfsys@transformshift{4.163782in}{2.683707in}%
\pgfsys@useobject{currentmarker}{}%
\end{pgfscope}%
\begin{pgfscope}%
\pgfsys@transformshift{4.181818in}{2.812359in}%
\pgfsys@useobject{currentmarker}{}%
\end{pgfscope}%
\begin{pgfscope}%
\pgfsys@transformshift{4.199855in}{2.898128in}%
\pgfsys@useobject{currentmarker}{}%
\end{pgfscope}%
\begin{pgfscope}%
\pgfsys@transformshift{4.217891in}{2.898128in}%
\pgfsys@useobject{currentmarker}{}%
\end{pgfscope}%
\begin{pgfscope}%
\pgfsys@transformshift{4.235927in}{2.812359in}%
\pgfsys@useobject{currentmarker}{}%
\end{pgfscope}%
\begin{pgfscope}%
\pgfsys@transformshift{4.253964in}{2.640823in}%
\pgfsys@useobject{currentmarker}{}%
\end{pgfscope}%
\begin{pgfscope}%
\pgfsys@transformshift{4.272000in}{2.469286in}%
\pgfsys@useobject{currentmarker}{}%
\end{pgfscope}%
\begin{pgfscope}%
\pgfsys@transformshift{4.290036in}{2.297750in}%
\pgfsys@useobject{currentmarker}{}%
\end{pgfscope}%
\begin{pgfscope}%
\pgfsys@transformshift{4.308073in}{2.169097in}%
\pgfsys@useobject{currentmarker}{}%
\end{pgfscope}%
\begin{pgfscope}%
\pgfsys@transformshift{4.326109in}{2.040445in}%
\pgfsys@useobject{currentmarker}{}%
\end{pgfscope}%
\begin{pgfscope}%
\pgfsys@transformshift{4.344145in}{2.040445in}%
\pgfsys@useobject{currentmarker}{}%
\end{pgfscope}%
\begin{pgfscope}%
\pgfsys@transformshift{4.362182in}{2.126213in}%
\pgfsys@useobject{currentmarker}{}%
\end{pgfscope}%
\begin{pgfscope}%
\pgfsys@transformshift{4.380218in}{2.254865in}%
\pgfsys@useobject{currentmarker}{}%
\end{pgfscope}%
\begin{pgfscope}%
\pgfsys@transformshift{4.398255in}{2.426402in}%
\pgfsys@useobject{currentmarker}{}%
\end{pgfscope}%
\begin{pgfscope}%
\pgfsys@transformshift{4.416291in}{2.597939in}%
\pgfsys@useobject{currentmarker}{}%
\end{pgfscope}%
\begin{pgfscope}%
\pgfsys@transformshift{4.434327in}{2.726591in}%
\pgfsys@useobject{currentmarker}{}%
\end{pgfscope}%
\begin{pgfscope}%
\pgfsys@transformshift{4.452364in}{2.812359in}%
\pgfsys@useobject{currentmarker}{}%
\end{pgfscope}%
\begin{pgfscope}%
\pgfsys@transformshift{4.470400in}{2.855244in}%
\pgfsys@useobject{currentmarker}{}%
\end{pgfscope}%
\begin{pgfscope}%
\pgfsys@transformshift{4.488436in}{2.812359in}%
\pgfsys@useobject{currentmarker}{}%
\end{pgfscope}%
\begin{pgfscope}%
\pgfsys@transformshift{4.506473in}{2.726591in}%
\pgfsys@useobject{currentmarker}{}%
\end{pgfscope}%
\begin{pgfscope}%
\pgfsys@transformshift{4.524509in}{2.597939in}%
\pgfsys@useobject{currentmarker}{}%
\end{pgfscope}%
\begin{pgfscope}%
\pgfsys@transformshift{4.542545in}{2.426402in}%
\pgfsys@useobject{currentmarker}{}%
\end{pgfscope}%
\begin{pgfscope}%
\pgfsys@transformshift{4.560582in}{2.254865in}%
\pgfsys@useobject{currentmarker}{}%
\end{pgfscope}%
\begin{pgfscope}%
\pgfsys@transformshift{4.578618in}{2.126213in}%
\pgfsys@useobject{currentmarker}{}%
\end{pgfscope}%
\begin{pgfscope}%
\pgfsys@transformshift{4.596655in}{2.083329in}%
\pgfsys@useobject{currentmarker}{}%
\end{pgfscope}%
\begin{pgfscope}%
\pgfsys@transformshift{4.614691in}{2.083329in}%
\pgfsys@useobject{currentmarker}{}%
\end{pgfscope}%
\begin{pgfscope}%
\pgfsys@transformshift{4.632727in}{2.169097in}%
\pgfsys@useobject{currentmarker}{}%
\end{pgfscope}%
\begin{pgfscope}%
\pgfsys@transformshift{4.650764in}{2.297750in}%
\pgfsys@useobject{currentmarker}{}%
\end{pgfscope}%
\begin{pgfscope}%
\pgfsys@transformshift{4.668800in}{2.469286in}%
\pgfsys@useobject{currentmarker}{}%
\end{pgfscope}%
\begin{pgfscope}%
\pgfsys@transformshift{4.686836in}{2.640823in}%
\pgfsys@useobject{currentmarker}{}%
\end{pgfscope}%
\begin{pgfscope}%
\pgfsys@transformshift{4.704873in}{2.769475in}%
\pgfsys@useobject{currentmarker}{}%
\end{pgfscope}%
\begin{pgfscope}%
\pgfsys@transformshift{4.722909in}{2.812359in}%
\pgfsys@useobject{currentmarker}{}%
\end{pgfscope}%
\begin{pgfscope}%
\pgfsys@transformshift{4.740945in}{2.855244in}%
\pgfsys@useobject{currentmarker}{}%
\end{pgfscope}%
\begin{pgfscope}%
\pgfsys@transformshift{4.758982in}{2.769475in}%
\pgfsys@useobject{currentmarker}{}%
\end{pgfscope}%
\begin{pgfscope}%
\pgfsys@transformshift{4.777018in}{2.640823in}%
\pgfsys@useobject{currentmarker}{}%
\end{pgfscope}%
\begin{pgfscope}%
\pgfsys@transformshift{4.795055in}{2.512170in}%
\pgfsys@useobject{currentmarker}{}%
\end{pgfscope}%
\begin{pgfscope}%
\pgfsys@transformshift{4.813091in}{2.340634in}%
\pgfsys@useobject{currentmarker}{}%
\end{pgfscope}%
\begin{pgfscope}%
\pgfsys@transformshift{4.831127in}{2.211981in}%
\pgfsys@useobject{currentmarker}{}%
\end{pgfscope}%
\begin{pgfscope}%
\pgfsys@transformshift{4.849164in}{2.126213in}%
\pgfsys@useobject{currentmarker}{}%
\end{pgfscope}%
\begin{pgfscope}%
\pgfsys@transformshift{4.867200in}{2.126213in}%
\pgfsys@useobject{currentmarker}{}%
\end{pgfscope}%
\begin{pgfscope}%
\pgfsys@transformshift{4.885236in}{2.126213in}%
\pgfsys@useobject{currentmarker}{}%
\end{pgfscope}%
\begin{pgfscope}%
\pgfsys@transformshift{4.903273in}{2.254865in}%
\pgfsys@useobject{currentmarker}{}%
\end{pgfscope}%
\begin{pgfscope}%
\pgfsys@transformshift{4.921309in}{2.383518in}%
\pgfsys@useobject{currentmarker}{}%
\end{pgfscope}%
\begin{pgfscope}%
\pgfsys@transformshift{4.939345in}{2.555055in}%
\pgfsys@useobject{currentmarker}{}%
\end{pgfscope}%
\begin{pgfscope}%
\pgfsys@transformshift{4.957382in}{2.683707in}%
\pgfsys@useobject{currentmarker}{}%
\end{pgfscope}%
\begin{pgfscope}%
\pgfsys@transformshift{4.975418in}{2.769475in}%
\pgfsys@useobject{currentmarker}{}%
\end{pgfscope}%
\begin{pgfscope}%
\pgfsys@transformshift{4.993455in}{2.812359in}%
\pgfsys@useobject{currentmarker}{}%
\end{pgfscope}%
\begin{pgfscope}%
\pgfsys@transformshift{5.011491in}{2.812359in}%
\pgfsys@useobject{currentmarker}{}%
\end{pgfscope}%
\begin{pgfscope}%
\pgfsys@transformshift{5.029527in}{2.726591in}%
\pgfsys@useobject{currentmarker}{}%
\end{pgfscope}%
\begin{pgfscope}%
\pgfsys@transformshift{5.047564in}{2.597939in}%
\pgfsys@useobject{currentmarker}{}%
\end{pgfscope}%
\begin{pgfscope}%
\pgfsys@transformshift{5.065600in}{2.469286in}%
\pgfsys@useobject{currentmarker}{}%
\end{pgfscope}%
\begin{pgfscope}%
\pgfsys@transformshift{5.083636in}{2.297750in}%
\pgfsys@useobject{currentmarker}{}%
\end{pgfscope}%
\begin{pgfscope}%
\pgfsys@transformshift{5.101673in}{2.211981in}%
\pgfsys@useobject{currentmarker}{}%
\end{pgfscope}%
\begin{pgfscope}%
\pgfsys@transformshift{5.119709in}{2.126213in}%
\pgfsys@useobject{currentmarker}{}%
\end{pgfscope}%
\begin{pgfscope}%
\pgfsys@transformshift{5.137745in}{2.126213in}%
\pgfsys@useobject{currentmarker}{}%
\end{pgfscope}%
\begin{pgfscope}%
\pgfsys@transformshift{5.155782in}{2.211981in}%
\pgfsys@useobject{currentmarker}{}%
\end{pgfscope}%
\begin{pgfscope}%
\pgfsys@transformshift{5.173818in}{2.297750in}%
\pgfsys@useobject{currentmarker}{}%
\end{pgfscope}%
\begin{pgfscope}%
\pgfsys@transformshift{5.191855in}{2.426402in}%
\pgfsys@useobject{currentmarker}{}%
\end{pgfscope}%
\begin{pgfscope}%
\pgfsys@transformshift{5.209891in}{2.597939in}%
\pgfsys@useobject{currentmarker}{}%
\end{pgfscope}%
\begin{pgfscope}%
\pgfsys@transformshift{5.227927in}{2.683707in}%
\pgfsys@useobject{currentmarker}{}%
\end{pgfscope}%
\begin{pgfscope}%
\pgfsys@transformshift{5.245964in}{2.769475in}%
\pgfsys@useobject{currentmarker}{}%
\end{pgfscope}%
\begin{pgfscope}%
\pgfsys@transformshift{5.264000in}{2.812359in}%
\pgfsys@useobject{currentmarker}{}%
\end{pgfscope}%
\begin{pgfscope}%
\pgfsys@transformshift{5.282036in}{2.769475in}%
\pgfsys@useobject{currentmarker}{}%
\end{pgfscope}%
\begin{pgfscope}%
\pgfsys@transformshift{5.300073in}{2.683707in}%
\pgfsys@useobject{currentmarker}{}%
\end{pgfscope}%
\begin{pgfscope}%
\pgfsys@transformshift{5.318109in}{2.555055in}%
\pgfsys@useobject{currentmarker}{}%
\end{pgfscope}%
\begin{pgfscope}%
\pgfsys@transformshift{5.336145in}{2.383518in}%
\pgfsys@useobject{currentmarker}{}%
\end{pgfscope}%
\begin{pgfscope}%
\pgfsys@transformshift{5.354182in}{2.297750in}%
\pgfsys@useobject{currentmarker}{}%
\end{pgfscope}%
\begin{pgfscope}%
\pgfsys@transformshift{5.372218in}{2.169097in}%
\pgfsys@useobject{currentmarker}{}%
\end{pgfscope}%
\begin{pgfscope}%
\pgfsys@transformshift{5.390255in}{2.169097in}%
\pgfsys@useobject{currentmarker}{}%
\end{pgfscope}%
\begin{pgfscope}%
\pgfsys@transformshift{5.408291in}{2.169097in}%
\pgfsys@useobject{currentmarker}{}%
\end{pgfscope}%
\begin{pgfscope}%
\pgfsys@transformshift{5.426327in}{2.254865in}%
\pgfsys@useobject{currentmarker}{}%
\end{pgfscope}%
\begin{pgfscope}%
\pgfsys@transformshift{5.444364in}{2.383518in}%
\pgfsys@useobject{currentmarker}{}%
\end{pgfscope}%
\begin{pgfscope}%
\pgfsys@transformshift{5.462400in}{2.512170in}%
\pgfsys@useobject{currentmarker}{}%
\end{pgfscope}%
\begin{pgfscope}%
\pgfsys@transformshift{5.480436in}{2.640823in}%
\pgfsys@useobject{currentmarker}{}%
\end{pgfscope}%
\begin{pgfscope}%
\pgfsys@transformshift{5.498473in}{2.726591in}%
\pgfsys@useobject{currentmarker}{}%
\end{pgfscope}%
\begin{pgfscope}%
\pgfsys@transformshift{5.516509in}{2.769475in}%
\pgfsys@useobject{currentmarker}{}%
\end{pgfscope}%
\begin{pgfscope}%
\pgfsys@transformshift{5.534545in}{2.769475in}%
\pgfsys@useobject{currentmarker}{}%
\end{pgfscope}%
\end{pgfscope}%
\begin{pgfscope}%
\pgfpathrectangle{\pgfqpoint{0.800000in}{0.528000in}}{\pgfqpoint{4.960000in}{3.696000in}}%
\pgfusepath{clip}%
\pgfsetbuttcap%
\pgfsetroundjoin%
\definecolor{currentfill}{rgb}{0.000000,0.000000,1.000000}%
\pgfsetfillcolor{currentfill}%
\pgfsetlinewidth{0.501875pt}%
\definecolor{currentstroke}{rgb}{0.000000,0.000000,1.000000}%
\pgfsetstrokecolor{currentstroke}%
\pgfsetdash{}{0pt}%
\pgfsys@defobject{currentmarker}{\pgfqpoint{-0.027778in}{-0.000000in}}{\pgfqpoint{0.027778in}{0.000000in}}{%
\pgfpathmoveto{\pgfqpoint{0.027778in}{-0.000000in}}%
\pgfpathlineto{\pgfqpoint{-0.027778in}{0.000000in}}%
\pgfusepath{stroke,fill}%
}%
\begin{pgfscope}%
\pgfsys@transformshift{1.043491in}{3.627158in}%
\pgfsys@useobject{currentmarker}{}%
\end{pgfscope}%
\begin{pgfscope}%
\pgfsys@transformshift{1.061527in}{3.498506in}%
\pgfsys@useobject{currentmarker}{}%
\end{pgfscope}%
\begin{pgfscope}%
\pgfsys@transformshift{1.079564in}{3.155433in}%
\pgfsys@useobject{currentmarker}{}%
\end{pgfscope}%
\begin{pgfscope}%
\pgfsys@transformshift{1.097600in}{2.726591in}%
\pgfsys@useobject{currentmarker}{}%
\end{pgfscope}%
\begin{pgfscope}%
\pgfsys@transformshift{1.115636in}{2.254865in}%
\pgfsys@useobject{currentmarker}{}%
\end{pgfscope}%
\begin{pgfscope}%
\pgfsys@transformshift{1.133673in}{1.697371in}%
\pgfsys@useobject{currentmarker}{}%
\end{pgfscope}%
\begin{pgfscope}%
\pgfsys@transformshift{1.151709in}{1.397182in}%
\pgfsys@useobject{currentmarker}{}%
\end{pgfscope}%
\begin{pgfscope}%
\pgfsys@transformshift{1.169745in}{1.311414in}%
\pgfsys@useobject{currentmarker}{}%
\end{pgfscope}%
\begin{pgfscope}%
\pgfsys@transformshift{1.187782in}{1.397182in}%
\pgfsys@useobject{currentmarker}{}%
\end{pgfscope}%
\begin{pgfscope}%
\pgfsys@transformshift{1.205818in}{1.697371in}%
\pgfsys@useobject{currentmarker}{}%
\end{pgfscope}%
\begin{pgfscope}%
\pgfsys@transformshift{1.223855in}{2.254865in}%
\pgfsys@useobject{currentmarker}{}%
\end{pgfscope}%
\begin{pgfscope}%
\pgfsys@transformshift{1.241891in}{2.683707in}%
\pgfsys@useobject{currentmarker}{}%
\end{pgfscope}%
\begin{pgfscope}%
\pgfsys@transformshift{1.259927in}{3.112549in}%
\pgfsys@useobject{currentmarker}{}%
\end{pgfscope}%
\begin{pgfscope}%
\pgfsys@transformshift{1.277964in}{3.412738in}%
\pgfsys@useobject{currentmarker}{}%
\end{pgfscope}%
\begin{pgfscope}%
\pgfsys@transformshift{1.296000in}{3.541390in}%
\pgfsys@useobject{currentmarker}{}%
\end{pgfscope}%
\begin{pgfscope}%
\pgfsys@transformshift{1.314036in}{3.541390in}%
\pgfsys@useobject{currentmarker}{}%
\end{pgfscope}%
\begin{pgfscope}%
\pgfsys@transformshift{1.332073in}{3.326969in}%
\pgfsys@useobject{currentmarker}{}%
\end{pgfscope}%
\begin{pgfscope}%
\pgfsys@transformshift{1.350109in}{2.983896in}%
\pgfsys@useobject{currentmarker}{}%
\end{pgfscope}%
\begin{pgfscope}%
\pgfsys@transformshift{1.368145in}{2.555055in}%
\pgfsys@useobject{currentmarker}{}%
\end{pgfscope}%
\begin{pgfscope}%
\pgfsys@transformshift{1.386182in}{2.169097in}%
\pgfsys@useobject{currentmarker}{}%
\end{pgfscope}%
\begin{pgfscope}%
\pgfsys@transformshift{1.404218in}{1.654487in}%
\pgfsys@useobject{currentmarker}{}%
\end{pgfscope}%
\begin{pgfscope}%
\pgfsys@transformshift{1.422255in}{1.440067in}%
\pgfsys@useobject{currentmarker}{}%
\end{pgfscope}%
\begin{pgfscope}%
\pgfsys@transformshift{1.440291in}{1.440067in}%
\pgfsys@useobject{currentmarker}{}%
\end{pgfscope}%
\begin{pgfscope}%
\pgfsys@transformshift{1.458327in}{1.568719in}%
\pgfsys@useobject{currentmarker}{}%
\end{pgfscope}%
\begin{pgfscope}%
\pgfsys@transformshift{1.476364in}{1.868908in}%
\pgfsys@useobject{currentmarker}{}%
\end{pgfscope}%
\begin{pgfscope}%
\pgfsys@transformshift{1.494400in}{2.383518in}%
\pgfsys@useobject{currentmarker}{}%
\end{pgfscope}%
\begin{pgfscope}%
\pgfsys@transformshift{1.512436in}{2.812359in}%
\pgfsys@useobject{currentmarker}{}%
\end{pgfscope}%
\begin{pgfscope}%
\pgfsys@transformshift{1.530473in}{3.155433in}%
\pgfsys@useobject{currentmarker}{}%
\end{pgfscope}%
\begin{pgfscope}%
\pgfsys@transformshift{1.548509in}{3.369854in}%
\pgfsys@useobject{currentmarker}{}%
\end{pgfscope}%
\begin{pgfscope}%
\pgfsys@transformshift{1.566545in}{3.455622in}%
\pgfsys@useobject{currentmarker}{}%
\end{pgfscope}%
\begin{pgfscope}%
\pgfsys@transformshift{1.584582in}{3.412738in}%
\pgfsys@useobject{currentmarker}{}%
\end{pgfscope}%
\begin{pgfscope}%
\pgfsys@transformshift{1.602618in}{3.155433in}%
\pgfsys@useobject{currentmarker}{}%
\end{pgfscope}%
\begin{pgfscope}%
\pgfsys@transformshift{1.620655in}{2.812359in}%
\pgfsys@useobject{currentmarker}{}%
\end{pgfscope}%
\begin{pgfscope}%
\pgfsys@transformshift{1.638691in}{2.426402in}%
\pgfsys@useobject{currentmarker}{}%
\end{pgfscope}%
\begin{pgfscope}%
\pgfsys@transformshift{1.656727in}{1.954676in}%
\pgfsys@useobject{currentmarker}{}%
\end{pgfscope}%
\begin{pgfscope}%
\pgfsys@transformshift{1.674764in}{1.654487in}%
\pgfsys@useobject{currentmarker}{}%
\end{pgfscope}%
\begin{pgfscope}%
\pgfsys@transformshift{1.692800in}{1.525835in}%
\pgfsys@useobject{currentmarker}{}%
\end{pgfscope}%
\begin{pgfscope}%
\pgfsys@transformshift{1.710836in}{1.525835in}%
\pgfsys@useobject{currentmarker}{}%
\end{pgfscope}%
\begin{pgfscope}%
\pgfsys@transformshift{1.728873in}{1.740256in}%
\pgfsys@useobject{currentmarker}{}%
\end{pgfscope}%
\begin{pgfscope}%
\pgfsys@transformshift{1.746909in}{2.169097in}%
\pgfsys@useobject{currentmarker}{}%
\end{pgfscope}%
\begin{pgfscope}%
\pgfsys@transformshift{1.764945in}{2.555055in}%
\pgfsys@useobject{currentmarker}{}%
\end{pgfscope}%
\begin{pgfscope}%
\pgfsys@transformshift{1.782982in}{2.898128in}%
\pgfsys@useobject{currentmarker}{}%
\end{pgfscope}%
\begin{pgfscope}%
\pgfsys@transformshift{1.801018in}{3.198317in}%
\pgfsys@useobject{currentmarker}{}%
\end{pgfscope}%
\begin{pgfscope}%
\pgfsys@transformshift{1.819055in}{3.369854in}%
\pgfsys@useobject{currentmarker}{}%
\end{pgfscope}%
\begin{pgfscope}%
\pgfsys@transformshift{1.837091in}{3.369854in}%
\pgfsys@useobject{currentmarker}{}%
\end{pgfscope}%
\begin{pgfscope}%
\pgfsys@transformshift{1.855127in}{3.284085in}%
\pgfsys@useobject{currentmarker}{}%
\end{pgfscope}%
\begin{pgfscope}%
\pgfsys@transformshift{1.873164in}{3.026780in}%
\pgfsys@useobject{currentmarker}{}%
\end{pgfscope}%
\begin{pgfscope}%
\pgfsys@transformshift{1.891200in}{2.683707in}%
\pgfsys@useobject{currentmarker}{}%
\end{pgfscope}%
\begin{pgfscope}%
\pgfsys@transformshift{1.909236in}{2.340634in}%
\pgfsys@useobject{currentmarker}{}%
\end{pgfscope}%
\begin{pgfscope}%
\pgfsys@transformshift{1.927273in}{1.868908in}%
\pgfsys@useobject{currentmarker}{}%
\end{pgfscope}%
\begin{pgfscope}%
\pgfsys@transformshift{1.945309in}{1.654487in}%
\pgfsys@useobject{currentmarker}{}%
\end{pgfscope}%
\begin{pgfscope}%
\pgfsys@transformshift{1.963345in}{1.568719in}%
\pgfsys@useobject{currentmarker}{}%
\end{pgfscope}%
\begin{pgfscope}%
\pgfsys@transformshift{1.981382in}{1.654487in}%
\pgfsys@useobject{currentmarker}{}%
\end{pgfscope}%
\begin{pgfscope}%
\pgfsys@transformshift{1.999418in}{1.868908in}%
\pgfsys@useobject{currentmarker}{}%
\end{pgfscope}%
\begin{pgfscope}%
\pgfsys@transformshift{2.017455in}{2.297750in}%
\pgfsys@useobject{currentmarker}{}%
\end{pgfscope}%
\begin{pgfscope}%
\pgfsys@transformshift{2.035491in}{2.683707in}%
\pgfsys@useobject{currentmarker}{}%
\end{pgfscope}%
\begin{pgfscope}%
\pgfsys@transformshift{2.053527in}{2.983896in}%
\pgfsys@useobject{currentmarker}{}%
\end{pgfscope}%
\begin{pgfscope}%
\pgfsys@transformshift{2.071564in}{3.198317in}%
\pgfsys@useobject{currentmarker}{}%
\end{pgfscope}%
\begin{pgfscope}%
\pgfsys@transformshift{2.089600in}{3.326969in}%
\pgfsys@useobject{currentmarker}{}%
\end{pgfscope}%
\begin{pgfscope}%
\pgfsys@transformshift{2.107636in}{3.326969in}%
\pgfsys@useobject{currentmarker}{}%
\end{pgfscope}%
\begin{pgfscope}%
\pgfsys@transformshift{2.125673in}{3.155433in}%
\pgfsys@useobject{currentmarker}{}%
\end{pgfscope}%
\begin{pgfscope}%
\pgfsys@transformshift{2.143709in}{2.898128in}%
\pgfsys@useobject{currentmarker}{}%
\end{pgfscope}%
\begin{pgfscope}%
\pgfsys@transformshift{2.161745in}{2.555055in}%
\pgfsys@useobject{currentmarker}{}%
\end{pgfscope}%
\begin{pgfscope}%
\pgfsys@transformshift{2.179782in}{2.254865in}%
\pgfsys@useobject{currentmarker}{}%
\end{pgfscope}%
\begin{pgfscope}%
\pgfsys@transformshift{2.197818in}{1.826024in}%
\pgfsys@useobject{currentmarker}{}%
\end{pgfscope}%
\begin{pgfscope}%
\pgfsys@transformshift{2.215855in}{1.697371in}%
\pgfsys@useobject{currentmarker}{}%
\end{pgfscope}%
\begin{pgfscope}%
\pgfsys@transformshift{2.233891in}{1.654487in}%
\pgfsys@useobject{currentmarker}{}%
\end{pgfscope}%
\begin{pgfscope}%
\pgfsys@transformshift{2.251927in}{1.783140in}%
\pgfsys@useobject{currentmarker}{}%
\end{pgfscope}%
\begin{pgfscope}%
\pgfsys@transformshift{2.269964in}{2.040445in}%
\pgfsys@useobject{currentmarker}{}%
\end{pgfscope}%
\begin{pgfscope}%
\pgfsys@transformshift{2.288000in}{2.469286in}%
\pgfsys@useobject{currentmarker}{}%
\end{pgfscope}%
\begin{pgfscope}%
\pgfsys@transformshift{2.306036in}{2.769475in}%
\pgfsys@useobject{currentmarker}{}%
\end{pgfscope}%
\begin{pgfscope}%
\pgfsys@transformshift{2.324073in}{3.026780in}%
\pgfsys@useobject{currentmarker}{}%
\end{pgfscope}%
\begin{pgfscope}%
\pgfsys@transformshift{2.342109in}{3.198317in}%
\pgfsys@useobject{currentmarker}{}%
\end{pgfscope}%
\begin{pgfscope}%
\pgfsys@transformshift{2.360145in}{3.284085in}%
\pgfsys@useobject{currentmarker}{}%
\end{pgfscope}%
\begin{pgfscope}%
\pgfsys@transformshift{2.378182in}{3.198317in}%
\pgfsys@useobject{currentmarker}{}%
\end{pgfscope}%
\begin{pgfscope}%
\pgfsys@transformshift{2.396218in}{3.026780in}%
\pgfsys@useobject{currentmarker}{}%
\end{pgfscope}%
\begin{pgfscope}%
\pgfsys@transformshift{2.414255in}{2.769475in}%
\pgfsys@useobject{currentmarker}{}%
\end{pgfscope}%
\begin{pgfscope}%
\pgfsys@transformshift{2.432291in}{2.469286in}%
\pgfsys@useobject{currentmarker}{}%
\end{pgfscope}%
\begin{pgfscope}%
\pgfsys@transformshift{2.450327in}{2.040445in}%
\pgfsys@useobject{currentmarker}{}%
\end{pgfscope}%
\begin{pgfscope}%
\pgfsys@transformshift{2.468364in}{1.826024in}%
\pgfsys@useobject{currentmarker}{}%
\end{pgfscope}%
\begin{pgfscope}%
\pgfsys@transformshift{2.486400in}{1.697371in}%
\pgfsys@useobject{currentmarker}{}%
\end{pgfscope}%
\begin{pgfscope}%
\pgfsys@transformshift{2.504436in}{1.740256in}%
\pgfsys@useobject{currentmarker}{}%
\end{pgfscope}%
\begin{pgfscope}%
\pgfsys@transformshift{2.522473in}{1.911792in}%
\pgfsys@useobject{currentmarker}{}%
\end{pgfscope}%
\begin{pgfscope}%
\pgfsys@transformshift{2.540509in}{2.254865in}%
\pgfsys@useobject{currentmarker}{}%
\end{pgfscope}%
\begin{pgfscope}%
\pgfsys@transformshift{2.558545in}{2.555055in}%
\pgfsys@useobject{currentmarker}{}%
\end{pgfscope}%
\begin{pgfscope}%
\pgfsys@transformshift{2.576582in}{2.855244in}%
\pgfsys@useobject{currentmarker}{}%
\end{pgfscope}%
\begin{pgfscope}%
\pgfsys@transformshift{2.594618in}{3.069664in}%
\pgfsys@useobject{currentmarker}{}%
\end{pgfscope}%
\begin{pgfscope}%
\pgfsys@transformshift{2.612655in}{3.198317in}%
\pgfsys@useobject{currentmarker}{}%
\end{pgfscope}%
\begin{pgfscope}%
\pgfsys@transformshift{2.630691in}{3.198317in}%
\pgfsys@useobject{currentmarker}{}%
\end{pgfscope}%
\begin{pgfscope}%
\pgfsys@transformshift{2.648727in}{3.112549in}%
\pgfsys@useobject{currentmarker}{}%
\end{pgfscope}%
\begin{pgfscope}%
\pgfsys@transformshift{2.666764in}{2.898128in}%
\pgfsys@useobject{currentmarker}{}%
\end{pgfscope}%
\begin{pgfscope}%
\pgfsys@transformshift{2.684800in}{2.640823in}%
\pgfsys@useobject{currentmarker}{}%
\end{pgfscope}%
\begin{pgfscope}%
\pgfsys@transformshift{2.702836in}{2.383518in}%
\pgfsys@useobject{currentmarker}{}%
\end{pgfscope}%
\begin{pgfscope}%
\pgfsys@transformshift{2.720873in}{2.083329in}%
\pgfsys@useobject{currentmarker}{}%
\end{pgfscope}%
\begin{pgfscope}%
\pgfsys@transformshift{2.738909in}{1.826024in}%
\pgfsys@useobject{currentmarker}{}%
\end{pgfscope}%
\begin{pgfscope}%
\pgfsys@transformshift{2.756945in}{1.783140in}%
\pgfsys@useobject{currentmarker}{}%
\end{pgfscope}%
\begin{pgfscope}%
\pgfsys@transformshift{2.774982in}{1.826024in}%
\pgfsys@useobject{currentmarker}{}%
\end{pgfscope}%
\begin{pgfscope}%
\pgfsys@transformshift{2.793018in}{2.126213in}%
\pgfsys@useobject{currentmarker}{}%
\end{pgfscope}%
\begin{pgfscope}%
\pgfsys@transformshift{2.811055in}{2.383518in}%
\pgfsys@useobject{currentmarker}{}%
\end{pgfscope}%
\begin{pgfscope}%
\pgfsys@transformshift{2.829091in}{2.683707in}%
\pgfsys@useobject{currentmarker}{}%
\end{pgfscope}%
\begin{pgfscope}%
\pgfsys@transformshift{2.847127in}{2.898128in}%
\pgfsys@useobject{currentmarker}{}%
\end{pgfscope}%
\begin{pgfscope}%
\pgfsys@transformshift{2.865164in}{3.112549in}%
\pgfsys@useobject{currentmarker}{}%
\end{pgfscope}%
\begin{pgfscope}%
\pgfsys@transformshift{2.883200in}{3.155433in}%
\pgfsys@useobject{currentmarker}{}%
\end{pgfscope}%
\begin{pgfscope}%
\pgfsys@transformshift{2.901236in}{3.155433in}%
\pgfsys@useobject{currentmarker}{}%
\end{pgfscope}%
\begin{pgfscope}%
\pgfsys@transformshift{2.919273in}{3.026780in}%
\pgfsys@useobject{currentmarker}{}%
\end{pgfscope}%
\begin{pgfscope}%
\pgfsys@transformshift{2.937309in}{2.769475in}%
\pgfsys@useobject{currentmarker}{}%
\end{pgfscope}%
\begin{pgfscope}%
\pgfsys@transformshift{2.955345in}{2.555055in}%
\pgfsys@useobject{currentmarker}{}%
\end{pgfscope}%
\begin{pgfscope}%
\pgfsys@transformshift{2.973382in}{2.297750in}%
\pgfsys@useobject{currentmarker}{}%
\end{pgfscope}%
\begin{pgfscope}%
\pgfsys@transformshift{2.991418in}{2.083329in}%
\pgfsys@useobject{currentmarker}{}%
\end{pgfscope}%
\begin{pgfscope}%
\pgfsys@transformshift{3.009455in}{1.826024in}%
\pgfsys@useobject{currentmarker}{}%
\end{pgfscope}%
\begin{pgfscope}%
\pgfsys@transformshift{3.027491in}{1.826024in}%
\pgfsys@useobject{currentmarker}{}%
\end{pgfscope}%
\begin{pgfscope}%
\pgfsys@transformshift{3.045527in}{1.954676in}%
\pgfsys@useobject{currentmarker}{}%
\end{pgfscope}%
\begin{pgfscope}%
\pgfsys@transformshift{3.063564in}{2.254865in}%
\pgfsys@useobject{currentmarker}{}%
\end{pgfscope}%
\begin{pgfscope}%
\pgfsys@transformshift{3.081600in}{2.512170in}%
\pgfsys@useobject{currentmarker}{}%
\end{pgfscope}%
\begin{pgfscope}%
\pgfsys@transformshift{3.099636in}{2.726591in}%
\pgfsys@useobject{currentmarker}{}%
\end{pgfscope}%
\begin{pgfscope}%
\pgfsys@transformshift{3.117673in}{2.941012in}%
\pgfsys@useobject{currentmarker}{}%
\end{pgfscope}%
\begin{pgfscope}%
\pgfsys@transformshift{3.135709in}{3.069664in}%
\pgfsys@useobject{currentmarker}{}%
\end{pgfscope}%
\begin{pgfscope}%
\pgfsys@transformshift{3.153745in}{3.112549in}%
\pgfsys@useobject{currentmarker}{}%
\end{pgfscope}%
\begin{pgfscope}%
\pgfsys@transformshift{3.171782in}{3.069664in}%
\pgfsys@useobject{currentmarker}{}%
\end{pgfscope}%
\begin{pgfscope}%
\pgfsys@transformshift{3.189818in}{2.898128in}%
\pgfsys@useobject{currentmarker}{}%
\end{pgfscope}%
\begin{pgfscope}%
\pgfsys@transformshift{3.207855in}{2.683707in}%
\pgfsys@useobject{currentmarker}{}%
\end{pgfscope}%
\begin{pgfscope}%
\pgfsys@transformshift{3.225891in}{2.469286in}%
\pgfsys@useobject{currentmarker}{}%
\end{pgfscope}%
\begin{pgfscope}%
\pgfsys@transformshift{3.243927in}{2.211981in}%
\pgfsys@useobject{currentmarker}{}%
\end{pgfscope}%
\begin{pgfscope}%
\pgfsys@transformshift{3.261964in}{1.954676in}%
\pgfsys@useobject{currentmarker}{}%
\end{pgfscope}%
\begin{pgfscope}%
\pgfsys@transformshift{3.280000in}{1.868908in}%
\pgfsys@useobject{currentmarker}{}%
\end{pgfscope}%
\begin{pgfscope}%
\pgfsys@transformshift{3.298036in}{1.911792in}%
\pgfsys@useobject{currentmarker}{}%
\end{pgfscope}%
\begin{pgfscope}%
\pgfsys@transformshift{3.316073in}{2.040445in}%
\pgfsys@useobject{currentmarker}{}%
\end{pgfscope}%
\begin{pgfscope}%
\pgfsys@transformshift{3.334109in}{2.340634in}%
\pgfsys@useobject{currentmarker}{}%
\end{pgfscope}%
\begin{pgfscope}%
\pgfsys@transformshift{3.352145in}{2.597939in}%
\pgfsys@useobject{currentmarker}{}%
\end{pgfscope}%
\begin{pgfscope}%
\pgfsys@transformshift{3.370182in}{2.812359in}%
\pgfsys@useobject{currentmarker}{}%
\end{pgfscope}%
\begin{pgfscope}%
\pgfsys@transformshift{3.388218in}{2.983896in}%
\pgfsys@useobject{currentmarker}{}%
\end{pgfscope}%
\begin{pgfscope}%
\pgfsys@transformshift{3.406255in}{3.069664in}%
\pgfsys@useobject{currentmarker}{}%
\end{pgfscope}%
\begin{pgfscope}%
\pgfsys@transformshift{3.424291in}{3.069664in}%
\pgfsys@useobject{currentmarker}{}%
\end{pgfscope}%
\begin{pgfscope}%
\pgfsys@transformshift{3.442327in}{2.983896in}%
\pgfsys@useobject{currentmarker}{}%
\end{pgfscope}%
\begin{pgfscope}%
\pgfsys@transformshift{3.460364in}{2.812359in}%
\pgfsys@useobject{currentmarker}{}%
\end{pgfscope}%
\begin{pgfscope}%
\pgfsys@transformshift{3.478400in}{2.597939in}%
\pgfsys@useobject{currentmarker}{}%
\end{pgfscope}%
\begin{pgfscope}%
\pgfsys@transformshift{3.496436in}{2.383518in}%
\pgfsys@useobject{currentmarker}{}%
\end{pgfscope}%
\begin{pgfscope}%
\pgfsys@transformshift{3.514473in}{2.169097in}%
\pgfsys@useobject{currentmarker}{}%
\end{pgfscope}%
\begin{pgfscope}%
\pgfsys@transformshift{3.532509in}{1.954676in}%
\pgfsys@useobject{currentmarker}{}%
\end{pgfscope}%
\begin{pgfscope}%
\pgfsys@transformshift{3.550545in}{1.911792in}%
\pgfsys@useobject{currentmarker}{}%
\end{pgfscope}%
\begin{pgfscope}%
\pgfsys@transformshift{3.568582in}{1.954676in}%
\pgfsys@useobject{currentmarker}{}%
\end{pgfscope}%
\begin{pgfscope}%
\pgfsys@transformshift{3.586618in}{2.211981in}%
\pgfsys@useobject{currentmarker}{}%
\end{pgfscope}%
\begin{pgfscope}%
\pgfsys@transformshift{3.604655in}{2.426402in}%
\pgfsys@useobject{currentmarker}{}%
\end{pgfscope}%
\begin{pgfscope}%
\pgfsys@transformshift{3.622691in}{2.683707in}%
\pgfsys@useobject{currentmarker}{}%
\end{pgfscope}%
\begin{pgfscope}%
\pgfsys@transformshift{3.640727in}{2.855244in}%
\pgfsys@useobject{currentmarker}{}%
\end{pgfscope}%
\begin{pgfscope}%
\pgfsys@transformshift{3.658764in}{2.983896in}%
\pgfsys@useobject{currentmarker}{}%
\end{pgfscope}%
\begin{pgfscope}%
\pgfsys@transformshift{3.676800in}{3.026780in}%
\pgfsys@useobject{currentmarker}{}%
\end{pgfscope}%
\begin{pgfscope}%
\pgfsys@transformshift{3.694836in}{3.026780in}%
\pgfsys@useobject{currentmarker}{}%
\end{pgfscope}%
\begin{pgfscope}%
\pgfsys@transformshift{3.712873in}{2.898128in}%
\pgfsys@useobject{currentmarker}{}%
\end{pgfscope}%
\begin{pgfscope}%
\pgfsys@transformshift{3.730909in}{2.726591in}%
\pgfsys@useobject{currentmarker}{}%
\end{pgfscope}%
\begin{pgfscope}%
\pgfsys@transformshift{3.748945in}{2.512170in}%
\pgfsys@useobject{currentmarker}{}%
\end{pgfscope}%
\begin{pgfscope}%
\pgfsys@transformshift{3.766982in}{2.340634in}%
\pgfsys@useobject{currentmarker}{}%
\end{pgfscope}%
\begin{pgfscope}%
\pgfsys@transformshift{3.785018in}{2.169097in}%
\pgfsys@useobject{currentmarker}{}%
\end{pgfscope}%
\begin{pgfscope}%
\pgfsys@transformshift{3.803055in}{2.083329in}%
\pgfsys@useobject{currentmarker}{}%
\end{pgfscope}%
\begin{pgfscope}%
\pgfsys@transformshift{3.821091in}{2.083329in}%
\pgfsys@useobject{currentmarker}{}%
\end{pgfscope}%
\begin{pgfscope}%
\pgfsys@transformshift{3.839127in}{2.169097in}%
\pgfsys@useobject{currentmarker}{}%
\end{pgfscope}%
\begin{pgfscope}%
\pgfsys@transformshift{3.857164in}{2.340634in}%
\pgfsys@useobject{currentmarker}{}%
\end{pgfscope}%
\begin{pgfscope}%
\pgfsys@transformshift{3.875200in}{2.555055in}%
\pgfsys@useobject{currentmarker}{}%
\end{pgfscope}%
\begin{pgfscope}%
\pgfsys@transformshift{3.893236in}{2.726591in}%
\pgfsys@useobject{currentmarker}{}%
\end{pgfscope}%
\begin{pgfscope}%
\pgfsys@transformshift{3.911273in}{2.898128in}%
\pgfsys@useobject{currentmarker}{}%
\end{pgfscope}%
\begin{pgfscope}%
\pgfsys@transformshift{3.929309in}{2.983896in}%
\pgfsys@useobject{currentmarker}{}%
\end{pgfscope}%
\begin{pgfscope}%
\pgfsys@transformshift{3.947345in}{3.026780in}%
\pgfsys@useobject{currentmarker}{}%
\end{pgfscope}%
\begin{pgfscope}%
\pgfsys@transformshift{3.965382in}{2.941012in}%
\pgfsys@useobject{currentmarker}{}%
\end{pgfscope}%
\begin{pgfscope}%
\pgfsys@transformshift{3.983418in}{2.812359in}%
\pgfsys@useobject{currentmarker}{}%
\end{pgfscope}%
\begin{pgfscope}%
\pgfsys@transformshift{4.001455in}{2.640823in}%
\pgfsys@useobject{currentmarker}{}%
\end{pgfscope}%
\begin{pgfscope}%
\pgfsys@transformshift{4.019491in}{2.469286in}%
\pgfsys@useobject{currentmarker}{}%
\end{pgfscope}%
\begin{pgfscope}%
\pgfsys@transformshift{4.037527in}{2.254865in}%
\pgfsys@useobject{currentmarker}{}%
\end{pgfscope}%
\begin{pgfscope}%
\pgfsys@transformshift{4.055564in}{2.126213in}%
\pgfsys@useobject{currentmarker}{}%
\end{pgfscope}%
\begin{pgfscope}%
\pgfsys@transformshift{4.073600in}{2.083329in}%
\pgfsys@useobject{currentmarker}{}%
\end{pgfscope}%
\begin{pgfscope}%
\pgfsys@transformshift{4.091636in}{2.126213in}%
\pgfsys@useobject{currentmarker}{}%
\end{pgfscope}%
\begin{pgfscope}%
\pgfsys@transformshift{4.109673in}{2.254865in}%
\pgfsys@useobject{currentmarker}{}%
\end{pgfscope}%
\begin{pgfscope}%
\pgfsys@transformshift{4.127709in}{2.426402in}%
\pgfsys@useobject{currentmarker}{}%
\end{pgfscope}%
\begin{pgfscope}%
\pgfsys@transformshift{4.145745in}{2.597939in}%
\pgfsys@useobject{currentmarker}{}%
\end{pgfscope}%
\begin{pgfscope}%
\pgfsys@transformshift{4.163782in}{2.769475in}%
\pgfsys@useobject{currentmarker}{}%
\end{pgfscope}%
\begin{pgfscope}%
\pgfsys@transformshift{4.181818in}{2.898128in}%
\pgfsys@useobject{currentmarker}{}%
\end{pgfscope}%
\begin{pgfscope}%
\pgfsys@transformshift{4.199855in}{2.983896in}%
\pgfsys@useobject{currentmarker}{}%
\end{pgfscope}%
\begin{pgfscope}%
\pgfsys@transformshift{4.217891in}{2.983896in}%
\pgfsys@useobject{currentmarker}{}%
\end{pgfscope}%
\begin{pgfscope}%
\pgfsys@transformshift{4.235927in}{2.898128in}%
\pgfsys@useobject{currentmarker}{}%
\end{pgfscope}%
\begin{pgfscope}%
\pgfsys@transformshift{4.253964in}{2.726591in}%
\pgfsys@useobject{currentmarker}{}%
\end{pgfscope}%
\begin{pgfscope}%
\pgfsys@transformshift{4.272000in}{2.555055in}%
\pgfsys@useobject{currentmarker}{}%
\end{pgfscope}%
\begin{pgfscope}%
\pgfsys@transformshift{4.290036in}{2.383518in}%
\pgfsys@useobject{currentmarker}{}%
\end{pgfscope}%
\begin{pgfscope}%
\pgfsys@transformshift{4.308073in}{2.254865in}%
\pgfsys@useobject{currentmarker}{}%
\end{pgfscope}%
\begin{pgfscope}%
\pgfsys@transformshift{4.326109in}{2.126213in}%
\pgfsys@useobject{currentmarker}{}%
\end{pgfscope}%
\begin{pgfscope}%
\pgfsys@transformshift{4.344145in}{2.126213in}%
\pgfsys@useobject{currentmarker}{}%
\end{pgfscope}%
\begin{pgfscope}%
\pgfsys@transformshift{4.362182in}{2.211981in}%
\pgfsys@useobject{currentmarker}{}%
\end{pgfscope}%
\begin{pgfscope}%
\pgfsys@transformshift{4.380218in}{2.340634in}%
\pgfsys@useobject{currentmarker}{}%
\end{pgfscope}%
\begin{pgfscope}%
\pgfsys@transformshift{4.398255in}{2.512170in}%
\pgfsys@useobject{currentmarker}{}%
\end{pgfscope}%
\begin{pgfscope}%
\pgfsys@transformshift{4.416291in}{2.683707in}%
\pgfsys@useobject{currentmarker}{}%
\end{pgfscope}%
\begin{pgfscope}%
\pgfsys@transformshift{4.434327in}{2.812359in}%
\pgfsys@useobject{currentmarker}{}%
\end{pgfscope}%
\begin{pgfscope}%
\pgfsys@transformshift{4.452364in}{2.898128in}%
\pgfsys@useobject{currentmarker}{}%
\end{pgfscope}%
\begin{pgfscope}%
\pgfsys@transformshift{4.470400in}{2.941012in}%
\pgfsys@useobject{currentmarker}{}%
\end{pgfscope}%
\begin{pgfscope}%
\pgfsys@transformshift{4.488436in}{2.898128in}%
\pgfsys@useobject{currentmarker}{}%
\end{pgfscope}%
\begin{pgfscope}%
\pgfsys@transformshift{4.506473in}{2.812359in}%
\pgfsys@useobject{currentmarker}{}%
\end{pgfscope}%
\begin{pgfscope}%
\pgfsys@transformshift{4.524509in}{2.683707in}%
\pgfsys@useobject{currentmarker}{}%
\end{pgfscope}%
\begin{pgfscope}%
\pgfsys@transformshift{4.542545in}{2.512170in}%
\pgfsys@useobject{currentmarker}{}%
\end{pgfscope}%
\begin{pgfscope}%
\pgfsys@transformshift{4.560582in}{2.340634in}%
\pgfsys@useobject{currentmarker}{}%
\end{pgfscope}%
\begin{pgfscope}%
\pgfsys@transformshift{4.578618in}{2.211981in}%
\pgfsys@useobject{currentmarker}{}%
\end{pgfscope}%
\begin{pgfscope}%
\pgfsys@transformshift{4.596655in}{2.169097in}%
\pgfsys@useobject{currentmarker}{}%
\end{pgfscope}%
\begin{pgfscope}%
\pgfsys@transformshift{4.614691in}{2.169097in}%
\pgfsys@useobject{currentmarker}{}%
\end{pgfscope}%
\begin{pgfscope}%
\pgfsys@transformshift{4.632727in}{2.254865in}%
\pgfsys@useobject{currentmarker}{}%
\end{pgfscope}%
\begin{pgfscope}%
\pgfsys@transformshift{4.650764in}{2.383518in}%
\pgfsys@useobject{currentmarker}{}%
\end{pgfscope}%
\begin{pgfscope}%
\pgfsys@transformshift{4.668800in}{2.555055in}%
\pgfsys@useobject{currentmarker}{}%
\end{pgfscope}%
\begin{pgfscope}%
\pgfsys@transformshift{4.686836in}{2.726591in}%
\pgfsys@useobject{currentmarker}{}%
\end{pgfscope}%
\begin{pgfscope}%
\pgfsys@transformshift{4.704873in}{2.855244in}%
\pgfsys@useobject{currentmarker}{}%
\end{pgfscope}%
\begin{pgfscope}%
\pgfsys@transformshift{4.722909in}{2.898128in}%
\pgfsys@useobject{currentmarker}{}%
\end{pgfscope}%
\begin{pgfscope}%
\pgfsys@transformshift{4.740945in}{2.941012in}%
\pgfsys@useobject{currentmarker}{}%
\end{pgfscope}%
\begin{pgfscope}%
\pgfsys@transformshift{4.758982in}{2.855244in}%
\pgfsys@useobject{currentmarker}{}%
\end{pgfscope}%
\begin{pgfscope}%
\pgfsys@transformshift{4.777018in}{2.726591in}%
\pgfsys@useobject{currentmarker}{}%
\end{pgfscope}%
\begin{pgfscope}%
\pgfsys@transformshift{4.795055in}{2.597939in}%
\pgfsys@useobject{currentmarker}{}%
\end{pgfscope}%
\begin{pgfscope}%
\pgfsys@transformshift{4.813091in}{2.426402in}%
\pgfsys@useobject{currentmarker}{}%
\end{pgfscope}%
\begin{pgfscope}%
\pgfsys@transformshift{4.831127in}{2.297750in}%
\pgfsys@useobject{currentmarker}{}%
\end{pgfscope}%
\begin{pgfscope}%
\pgfsys@transformshift{4.849164in}{2.211981in}%
\pgfsys@useobject{currentmarker}{}%
\end{pgfscope}%
\begin{pgfscope}%
\pgfsys@transformshift{4.867200in}{2.211981in}%
\pgfsys@useobject{currentmarker}{}%
\end{pgfscope}%
\begin{pgfscope}%
\pgfsys@transformshift{4.885236in}{2.211981in}%
\pgfsys@useobject{currentmarker}{}%
\end{pgfscope}%
\begin{pgfscope}%
\pgfsys@transformshift{4.903273in}{2.340634in}%
\pgfsys@useobject{currentmarker}{}%
\end{pgfscope}%
\begin{pgfscope}%
\pgfsys@transformshift{4.921309in}{2.469286in}%
\pgfsys@useobject{currentmarker}{}%
\end{pgfscope}%
\begin{pgfscope}%
\pgfsys@transformshift{4.939345in}{2.640823in}%
\pgfsys@useobject{currentmarker}{}%
\end{pgfscope}%
\begin{pgfscope}%
\pgfsys@transformshift{4.957382in}{2.769475in}%
\pgfsys@useobject{currentmarker}{}%
\end{pgfscope}%
\begin{pgfscope}%
\pgfsys@transformshift{4.975418in}{2.855244in}%
\pgfsys@useobject{currentmarker}{}%
\end{pgfscope}%
\begin{pgfscope}%
\pgfsys@transformshift{4.993455in}{2.898128in}%
\pgfsys@useobject{currentmarker}{}%
\end{pgfscope}%
\begin{pgfscope}%
\pgfsys@transformshift{5.011491in}{2.898128in}%
\pgfsys@useobject{currentmarker}{}%
\end{pgfscope}%
\begin{pgfscope}%
\pgfsys@transformshift{5.029527in}{2.812359in}%
\pgfsys@useobject{currentmarker}{}%
\end{pgfscope}%
\begin{pgfscope}%
\pgfsys@transformshift{5.047564in}{2.683707in}%
\pgfsys@useobject{currentmarker}{}%
\end{pgfscope}%
\begin{pgfscope}%
\pgfsys@transformshift{5.065600in}{2.555055in}%
\pgfsys@useobject{currentmarker}{}%
\end{pgfscope}%
\begin{pgfscope}%
\pgfsys@transformshift{5.083636in}{2.383518in}%
\pgfsys@useobject{currentmarker}{}%
\end{pgfscope}%
\begin{pgfscope}%
\pgfsys@transformshift{5.101673in}{2.297750in}%
\pgfsys@useobject{currentmarker}{}%
\end{pgfscope}%
\begin{pgfscope}%
\pgfsys@transformshift{5.119709in}{2.211981in}%
\pgfsys@useobject{currentmarker}{}%
\end{pgfscope}%
\begin{pgfscope}%
\pgfsys@transformshift{5.137745in}{2.211981in}%
\pgfsys@useobject{currentmarker}{}%
\end{pgfscope}%
\begin{pgfscope}%
\pgfsys@transformshift{5.155782in}{2.297750in}%
\pgfsys@useobject{currentmarker}{}%
\end{pgfscope}%
\begin{pgfscope}%
\pgfsys@transformshift{5.173818in}{2.383518in}%
\pgfsys@useobject{currentmarker}{}%
\end{pgfscope}%
\begin{pgfscope}%
\pgfsys@transformshift{5.191855in}{2.512170in}%
\pgfsys@useobject{currentmarker}{}%
\end{pgfscope}%
\begin{pgfscope}%
\pgfsys@transformshift{5.209891in}{2.683707in}%
\pgfsys@useobject{currentmarker}{}%
\end{pgfscope}%
\begin{pgfscope}%
\pgfsys@transformshift{5.227927in}{2.769475in}%
\pgfsys@useobject{currentmarker}{}%
\end{pgfscope}%
\begin{pgfscope}%
\pgfsys@transformshift{5.245964in}{2.855244in}%
\pgfsys@useobject{currentmarker}{}%
\end{pgfscope}%
\begin{pgfscope}%
\pgfsys@transformshift{5.264000in}{2.898128in}%
\pgfsys@useobject{currentmarker}{}%
\end{pgfscope}%
\begin{pgfscope}%
\pgfsys@transformshift{5.282036in}{2.855244in}%
\pgfsys@useobject{currentmarker}{}%
\end{pgfscope}%
\begin{pgfscope}%
\pgfsys@transformshift{5.300073in}{2.769475in}%
\pgfsys@useobject{currentmarker}{}%
\end{pgfscope}%
\begin{pgfscope}%
\pgfsys@transformshift{5.318109in}{2.640823in}%
\pgfsys@useobject{currentmarker}{}%
\end{pgfscope}%
\begin{pgfscope}%
\pgfsys@transformshift{5.336145in}{2.469286in}%
\pgfsys@useobject{currentmarker}{}%
\end{pgfscope}%
\begin{pgfscope}%
\pgfsys@transformshift{5.354182in}{2.383518in}%
\pgfsys@useobject{currentmarker}{}%
\end{pgfscope}%
\begin{pgfscope}%
\pgfsys@transformshift{5.372218in}{2.254865in}%
\pgfsys@useobject{currentmarker}{}%
\end{pgfscope}%
\begin{pgfscope}%
\pgfsys@transformshift{5.390255in}{2.254865in}%
\pgfsys@useobject{currentmarker}{}%
\end{pgfscope}%
\begin{pgfscope}%
\pgfsys@transformshift{5.408291in}{2.254865in}%
\pgfsys@useobject{currentmarker}{}%
\end{pgfscope}%
\begin{pgfscope}%
\pgfsys@transformshift{5.426327in}{2.340634in}%
\pgfsys@useobject{currentmarker}{}%
\end{pgfscope}%
\begin{pgfscope}%
\pgfsys@transformshift{5.444364in}{2.469286in}%
\pgfsys@useobject{currentmarker}{}%
\end{pgfscope}%
\begin{pgfscope}%
\pgfsys@transformshift{5.462400in}{2.597939in}%
\pgfsys@useobject{currentmarker}{}%
\end{pgfscope}%
\begin{pgfscope}%
\pgfsys@transformshift{5.480436in}{2.726591in}%
\pgfsys@useobject{currentmarker}{}%
\end{pgfscope}%
\begin{pgfscope}%
\pgfsys@transformshift{5.498473in}{2.812359in}%
\pgfsys@useobject{currentmarker}{}%
\end{pgfscope}%
\begin{pgfscope}%
\pgfsys@transformshift{5.516509in}{2.855244in}%
\pgfsys@useobject{currentmarker}{}%
\end{pgfscope}%
\begin{pgfscope}%
\pgfsys@transformshift{5.534545in}{2.855244in}%
\pgfsys@useobject{currentmarker}{}%
\end{pgfscope}%
\end{pgfscope}%
\begin{pgfscope}%
\pgfpathrectangle{\pgfqpoint{0.800000in}{0.528000in}}{\pgfqpoint{4.960000in}{3.696000in}}%
\pgfusepath{clip}%
\pgfsetrectcap%
\pgfsetroundjoin%
\pgfsetlinewidth{0.501875pt}%
\definecolor{currentstroke}{rgb}{0.000000,0.500000,0.000000}%
\pgfsetstrokecolor{currentstroke}%
\pgfsetdash{}{0pt}%
\pgfpathmoveto{\pgfqpoint{1.025455in}{4.056000in}}%
\pgfpathlineto{\pgfqpoint{1.029968in}{4.033308in}}%
\pgfpathlineto{\pgfqpoint{1.034482in}{3.972686in}}%
\pgfpathlineto{\pgfqpoint{1.038995in}{3.875647in}}%
\pgfpathlineto{\pgfqpoint{1.048023in}{3.582378in}}%
\pgfpathlineto{\pgfqpoint{1.057050in}{3.180730in}}%
\pgfpathlineto{\pgfqpoint{1.075104in}{2.205106in}}%
\pgfpathlineto{\pgfqpoint{1.088645in}{1.495548in}}%
\pgfpathlineto{\pgfqpoint{1.097672in}{1.112916in}}%
\pgfpathlineto{\pgfqpoint{1.106699in}{0.842803in}}%
\pgfpathlineto{\pgfqpoint{1.111213in}{0.757651in}}%
\pgfpathlineto{\pgfqpoint{1.115727in}{0.708396in}}%
\pgfpathlineto{\pgfqpoint{1.120240in}{0.696000in}}%
\pgfpathlineto{\pgfqpoint{1.124754in}{0.720595in}}%
\pgfpathlineto{\pgfqpoint{1.129267in}{0.781481in}}%
\pgfpathlineto{\pgfqpoint{1.133781in}{0.877145in}}%
\pgfpathlineto{\pgfqpoint{1.142808in}{1.162934in}}%
\pgfpathlineto{\pgfqpoint{1.151835in}{1.551452in}}%
\pgfpathlineto{\pgfqpoint{1.169890in}{2.489112in}}%
\pgfpathlineto{\pgfqpoint{1.183431in}{3.166812in}}%
\pgfpathlineto{\pgfqpoint{1.192458in}{3.530132in}}%
\pgfpathlineto{\pgfqpoint{1.201485in}{3.784495in}}%
\pgfpathlineto{\pgfqpoint{1.205999in}{3.863555in}}%
\pgfpathlineto{\pgfqpoint{1.210512in}{3.908091in}}%
\pgfpathlineto{\pgfqpoint{1.215026in}{3.917241in}}%
\pgfpathlineto{\pgfqpoint{1.219540in}{3.890941in}}%
\pgfpathlineto{\pgfqpoint{1.224053in}{3.829924in}}%
\pgfpathlineto{\pgfqpoint{1.228567in}{3.735699in}}%
\pgfpathlineto{\pgfqpoint{1.237594in}{3.457306in}}%
\pgfpathlineto{\pgfqpoint{1.246621in}{3.081569in}}%
\pgfpathlineto{\pgfqpoint{1.264676in}{2.180521in}}%
\pgfpathlineto{\pgfqpoint{1.278216in}{1.533344in}}%
\pgfpathlineto{\pgfqpoint{1.287244in}{1.188441in}}%
\pgfpathlineto{\pgfqpoint{1.296271in}{0.949021in}}%
\pgfpathlineto{\pgfqpoint{1.300784in}{0.875707in}}%
\pgfpathlineto{\pgfqpoint{1.305298in}{0.835591in}}%
\pgfpathlineto{\pgfqpoint{1.309812in}{0.829441in}}%
\pgfpathlineto{\pgfqpoint{1.314325in}{0.857260in}}%
\pgfpathlineto{\pgfqpoint{1.318839in}{0.918287in}}%
\pgfpathlineto{\pgfqpoint{1.323352in}{1.011018in}}%
\pgfpathlineto{\pgfqpoint{1.332380in}{1.282102in}}%
\pgfpathlineto{\pgfqpoint{1.341407in}{1.645406in}}%
\pgfpathlineto{\pgfqpoint{1.359461in}{2.511152in}}%
\pgfpathlineto{\pgfqpoint{1.373002in}{3.129086in}}%
\pgfpathlineto{\pgfqpoint{1.382029in}{3.456429in}}%
\pgfpathlineto{\pgfqpoint{1.391057in}{3.681674in}}%
\pgfpathlineto{\pgfqpoint{1.395570in}{3.749572in}}%
\pgfpathlineto{\pgfqpoint{1.400084in}{3.785551in}}%
\pgfpathlineto{\pgfqpoint{1.404597in}{3.788932in}}%
\pgfpathlineto{\pgfqpoint{1.409111in}{3.759767in}}%
\pgfpathlineto{\pgfqpoint{1.413625in}{3.698843in}}%
\pgfpathlineto{\pgfqpoint{1.418138in}{3.607654in}}%
\pgfpathlineto{\pgfqpoint{1.427165in}{3.343783in}}%
\pgfpathlineto{\pgfqpoint{1.436193in}{2.992574in}}%
\pgfpathlineto{\pgfqpoint{1.472301in}{1.404794in}}%
\pgfpathlineto{\pgfqpoint{1.481329in}{1.140718in}}%
\pgfpathlineto{\pgfqpoint{1.485842in}{1.048532in}}%
\pgfpathlineto{\pgfqpoint{1.490356in}{0.985737in}}%
\pgfpathlineto{\pgfqpoint{1.494869in}{0.953627in}}%
\pgfpathlineto{\pgfqpoint{1.499383in}{0.952799in}}%
\pgfpathlineto{\pgfqpoint{1.503897in}{0.983148in}}%
\pgfpathlineto{\pgfqpoint{1.508410in}{1.043867in}}%
\pgfpathlineto{\pgfqpoint{1.512924in}{1.133471in}}%
\pgfpathlineto{\pgfqpoint{1.521951in}{1.390229in}}%
\pgfpathlineto{\pgfqpoint{1.530978in}{1.729680in}}%
\pgfpathlineto{\pgfqpoint{1.567087in}{3.249530in}}%
\pgfpathlineto{\pgfqpoint{1.576114in}{3.499094in}}%
\pgfpathlineto{\pgfqpoint{1.580628in}{3.585388in}}%
\pgfpathlineto{\pgfqpoint{1.585142in}{3.643377in}}%
\pgfpathlineto{\pgfqpoint{1.589655in}{3.671872in}}%
\pgfpathlineto{\pgfqpoint{1.594169in}{3.670352in}}%
\pgfpathlineto{\pgfqpoint{1.598682in}{3.638970in}}%
\pgfpathlineto{\pgfqpoint{1.603196in}{3.578550in}}%
\pgfpathlineto{\pgfqpoint{1.607710in}{3.490566in}}%
\pgfpathlineto{\pgfqpoint{1.616737in}{3.240815in}}%
\pgfpathlineto{\pgfqpoint{1.625764in}{2.912793in}}%
\pgfpathlineto{\pgfqpoint{1.661873in}{1.458166in}}%
\pgfpathlineto{\pgfqpoint{1.670900in}{1.222397in}}%
\pgfpathlineto{\pgfqpoint{1.675414in}{1.141679in}}%
\pgfpathlineto{\pgfqpoint{1.679927in}{1.088214in}}%
\pgfpathlineto{\pgfqpoint{1.684441in}{1.063094in}}%
\pgfpathlineto{\pgfqpoint{1.688954in}{1.066770in}}%
\pgfpathlineto{\pgfqpoint{1.693468in}{1.099045in}}%
\pgfpathlineto{\pgfqpoint{1.697982in}{1.159079in}}%
\pgfpathlineto{\pgfqpoint{1.702495in}{1.245413in}}%
\pgfpathlineto{\pgfqpoint{1.711522in}{1.488267in}}%
\pgfpathlineto{\pgfqpoint{1.720550in}{1.805183in}}%
\pgfpathlineto{\pgfqpoint{1.756658in}{3.197196in}}%
\pgfpathlineto{\pgfqpoint{1.765686in}{3.419852in}}%
\pgfpathlineto{\pgfqpoint{1.770199in}{3.495298in}}%
\pgfpathlineto{\pgfqpoint{1.774713in}{3.544506in}}%
\pgfpathlineto{\pgfqpoint{1.779226in}{3.566477in}}%
\pgfpathlineto{\pgfqpoint{1.783740in}{3.560825in}}%
\pgfpathlineto{\pgfqpoint{1.788254in}{3.527788in}}%
\pgfpathlineto{\pgfqpoint{1.792767in}{3.468218in}}%
\pgfpathlineto{\pgfqpoint{1.797281in}{3.383558in}}%
\pgfpathlineto{\pgfqpoint{1.806308in}{3.147488in}}%
\pgfpathlineto{\pgfqpoint{1.815335in}{2.841360in}}%
\pgfpathlineto{\pgfqpoint{1.851444in}{1.509450in}}%
\pgfpathlineto{\pgfqpoint{1.860471in}{1.299255in}}%
\pgfpathlineto{\pgfqpoint{1.864985in}{1.228794in}}%
\pgfpathlineto{\pgfqpoint{1.869499in}{1.183589in}}%
\pgfpathlineto{\pgfqpoint{1.874012in}{1.164554in}}%
\pgfpathlineto{\pgfqpoint{1.878526in}{1.172011in}}%
\pgfpathlineto{\pgfqpoint{1.883039in}{1.205688in}}%
\pgfpathlineto{\pgfqpoint{1.887553in}{1.264723in}}%
\pgfpathlineto{\pgfqpoint{1.892067in}{1.347690in}}%
\pgfpathlineto{\pgfqpoint{1.901094in}{1.577092in}}%
\pgfpathlineto{\pgfqpoint{1.910121in}{1.872745in}}%
\pgfpathlineto{\pgfqpoint{1.941716in}{3.018662in}}%
\pgfpathlineto{\pgfqpoint{1.950743in}{3.256656in}}%
\pgfpathlineto{\pgfqpoint{1.955257in}{3.345326in}}%
\pgfpathlineto{\pgfqpoint{1.959771in}{3.411076in}}%
\pgfpathlineto{\pgfqpoint{1.964284in}{3.452518in}}%
\pgfpathlineto{\pgfqpoint{1.968798in}{3.468820in}}%
\pgfpathlineto{\pgfqpoint{1.973311in}{3.459716in}}%
\pgfpathlineto{\pgfqpoint{1.977825in}{3.425512in}}%
\pgfpathlineto{\pgfqpoint{1.982339in}{3.367076in}}%
\pgfpathlineto{\pgfqpoint{1.986852in}{3.285818in}}%
\pgfpathlineto{\pgfqpoint{1.995880in}{3.062963in}}%
\pgfpathlineto{\pgfqpoint{2.004907in}{2.777481in}}%
\pgfpathlineto{\pgfqpoint{2.036502in}{1.680371in}}%
\pgfpathlineto{\pgfqpoint{2.045529in}{1.454900in}}%
\pgfpathlineto{\pgfqpoint{2.050043in}{1.371501in}}%
\pgfpathlineto{\pgfqpoint{2.054556in}{1.310203in}}%
\pgfpathlineto{\pgfqpoint{2.059070in}{1.272295in}}%
\pgfpathlineto{\pgfqpoint{2.063584in}{1.258536in}}%
\pgfpathlineto{\pgfqpoint{2.068097in}{1.269137in}}%
\pgfpathlineto{\pgfqpoint{2.072611in}{1.303763in}}%
\pgfpathlineto{\pgfqpoint{2.077124in}{1.361541in}}%
\pgfpathlineto{\pgfqpoint{2.081638in}{1.441081in}}%
\pgfpathlineto{\pgfqpoint{2.090665in}{1.657510in}}%
\pgfpathlineto{\pgfqpoint{2.104206in}{2.085434in}}%
\pgfpathlineto{\pgfqpoint{2.126774in}{2.852882in}}%
\pgfpathlineto{\pgfqpoint{2.135801in}{3.098880in}}%
\pgfpathlineto{\pgfqpoint{2.144828in}{3.275308in}}%
\pgfpathlineto{\pgfqpoint{2.149342in}{3.332402in}}%
\pgfpathlineto{\pgfqpoint{2.153856in}{3.366991in}}%
\pgfpathlineto{\pgfqpoint{2.158369in}{3.378388in}}%
\pgfpathlineto{\pgfqpoint{2.162883in}{3.366429in}}%
\pgfpathlineto{\pgfqpoint{2.167396in}{3.331478in}}%
\pgfpathlineto{\pgfqpoint{2.171910in}{3.274410in}}%
\pgfpathlineto{\pgfqpoint{2.180937in}{3.099866in}}%
\pgfpathlineto{\pgfqpoint{2.189965in}{2.859018in}}%
\pgfpathlineto{\pgfqpoint{2.203505in}{2.422656in}}%
\pgfpathlineto{\pgfqpoint{2.221560in}{1.839317in}}%
\pgfpathlineto{\pgfqpoint{2.230587in}{1.605631in}}%
\pgfpathlineto{\pgfqpoint{2.239614in}{1.439343in}}%
\pgfpathlineto{\pgfqpoint{2.244128in}{1.386218in}}%
\pgfpathlineto{\pgfqpoint{2.248641in}{1.354744in}}%
\pgfpathlineto{\pgfqpoint{2.253155in}{1.345540in}}%
\pgfpathlineto{\pgfqpoint{2.257669in}{1.358724in}}%
\pgfpathlineto{\pgfqpoint{2.262182in}{1.393911in}}%
\pgfpathlineto{\pgfqpoint{2.266696in}{1.450222in}}%
\pgfpathlineto{\pgfqpoint{2.275723in}{1.620382in}}%
\pgfpathlineto{\pgfqpoint{2.284750in}{1.853405in}}%
\pgfpathlineto{\pgfqpoint{2.302805in}{2.419386in}}%
\pgfpathlineto{\pgfqpoint{2.316345in}{2.830987in}}%
\pgfpathlineto{\pgfqpoint{2.325373in}{3.052928in}}%
\pgfpathlineto{\pgfqpoint{2.334400in}{3.209590in}}%
\pgfpathlineto{\pgfqpoint{2.338913in}{3.258969in}}%
\pgfpathlineto{\pgfqpoint{2.343427in}{3.287523in}}%
\pgfpathlineto{\pgfqpoint{2.347941in}{3.294694in}}%
\pgfpathlineto{\pgfqpoint{2.352454in}{3.280407in}}%
\pgfpathlineto{\pgfqpoint{2.356968in}{3.245068in}}%
\pgfpathlineto{\pgfqpoint{2.361481in}{3.189555in}}%
\pgfpathlineto{\pgfqpoint{2.370509in}{3.023735in}}%
\pgfpathlineto{\pgfqpoint{2.379536in}{2.798331in}}%
\pgfpathlineto{\pgfqpoint{2.397590in}{2.254374in}}%
\pgfpathlineto{\pgfqpoint{2.411131in}{1.861255in}}%
\pgfpathlineto{\pgfqpoint{2.420158in}{1.650515in}}%
\pgfpathlineto{\pgfqpoint{2.429186in}{1.502989in}}%
\pgfpathlineto{\pgfqpoint{2.433699in}{1.457144in}}%
\pgfpathlineto{\pgfqpoint{2.438213in}{1.431327in}}%
\pgfpathlineto{\pgfqpoint{2.442726in}{1.426037in}}%
\pgfpathlineto{\pgfqpoint{2.447240in}{1.441313in}}%
\pgfpathlineto{\pgfqpoint{2.451754in}{1.476727in}}%
\pgfpathlineto{\pgfqpoint{2.456267in}{1.531405in}}%
\pgfpathlineto{\pgfqpoint{2.465294in}{1.692932in}}%
\pgfpathlineto{\pgfqpoint{2.474322in}{1.910921in}}%
\pgfpathlineto{\pgfqpoint{2.492376in}{2.433638in}}%
\pgfpathlineto{\pgfqpoint{2.505917in}{2.809050in}}%
\pgfpathlineto{\pgfqpoint{2.514944in}{3.009107in}}%
\pgfpathlineto{\pgfqpoint{2.523971in}{3.147965in}}%
\pgfpathlineto{\pgfqpoint{2.528485in}{3.190478in}}%
\pgfpathlineto{\pgfqpoint{2.532998in}{3.213732in}}%
\pgfpathlineto{\pgfqpoint{2.537512in}{3.217281in}}%
\pgfpathlineto{\pgfqpoint{2.542026in}{3.201126in}}%
\pgfpathlineto{\pgfqpoint{2.546539in}{3.165706in}}%
\pgfpathlineto{\pgfqpoint{2.551053in}{3.111896in}}%
\pgfpathlineto{\pgfqpoint{2.560080in}{2.954611in}}%
\pgfpathlineto{\pgfqpoint{2.569107in}{2.743835in}}%
\pgfpathlineto{\pgfqpoint{2.587162in}{2.241599in}}%
\pgfpathlineto{\pgfqpoint{2.600703in}{1.883150in}}%
\pgfpathlineto{\pgfqpoint{2.609730in}{1.693279in}}%
\pgfpathlineto{\pgfqpoint{2.618757in}{1.562643in}}%
\pgfpathlineto{\pgfqpoint{2.623271in}{1.523271in}}%
\pgfpathlineto{\pgfqpoint{2.627784in}{1.502417in}}%
\pgfpathlineto{\pgfqpoint{2.632298in}{1.500473in}}%
\pgfpathlineto{\pgfqpoint{2.636811in}{1.517409in}}%
\pgfpathlineto{\pgfqpoint{2.641325in}{1.552768in}}%
\pgfpathlineto{\pgfqpoint{2.645839in}{1.605682in}}%
\pgfpathlineto{\pgfqpoint{2.654866in}{1.758782in}}%
\pgfpathlineto{\pgfqpoint{2.663893in}{1.962540in}}%
\pgfpathlineto{\pgfqpoint{2.700002in}{2.883603in}}%
\pgfpathlineto{\pgfqpoint{2.709029in}{3.036769in}}%
\pgfpathlineto{\pgfqpoint{2.713543in}{3.090232in}}%
\pgfpathlineto{\pgfqpoint{2.718056in}{3.126644in}}%
\pgfpathlineto{\pgfqpoint{2.722570in}{3.145255in}}%
\pgfpathlineto{\pgfqpoint{2.727083in}{3.145719in}}%
\pgfpathlineto{\pgfqpoint{2.731597in}{3.128097in}}%
\pgfpathlineto{\pgfqpoint{2.736111in}{3.092858in}}%
\pgfpathlineto{\pgfqpoint{2.740624in}{3.040863in}}%
\pgfpathlineto{\pgfqpoint{2.749651in}{2.891891in}}%
\pgfpathlineto{\pgfqpoint{2.758679in}{2.694955in}}%
\pgfpathlineto{\pgfqpoint{2.794788in}{1.813299in}}%
\pgfpathlineto{\pgfqpoint{2.803815in}{1.668550in}}%
\pgfpathlineto{\pgfqpoint{2.808328in}{1.618505in}}%
\pgfpathlineto{\pgfqpoint{2.812842in}{1.584881in}}%
\pgfpathlineto{\pgfqpoint{2.817356in}{1.568366in}}%
\pgfpathlineto{\pgfqpoint{2.821869in}{1.569264in}}%
\pgfpathlineto{\pgfqpoint{2.826383in}{1.587484in}}%
\pgfpathlineto{\pgfqpoint{2.830896in}{1.622549in}}%
\pgfpathlineto{\pgfqpoint{2.835410in}{1.673603in}}%
\pgfpathlineto{\pgfqpoint{2.844437in}{1.818509in}}%
\pgfpathlineto{\pgfqpoint{2.853464in}{2.008814in}}%
\pgfpathlineto{\pgfqpoint{2.889573in}{2.852633in}}%
\pgfpathlineto{\pgfqpoint{2.898600in}{2.989380in}}%
\pgfpathlineto{\pgfqpoint{2.903114in}{3.036191in}}%
\pgfpathlineto{\pgfqpoint{2.907628in}{3.067192in}}%
\pgfpathlineto{\pgfqpoint{2.912141in}{3.081749in}}%
\pgfpathlineto{\pgfqpoint{2.916655in}{3.079601in}}%
\pgfpathlineto{\pgfqpoint{2.921168in}{3.060864in}}%
\pgfpathlineto{\pgfqpoint{2.925682in}{3.026023in}}%
\pgfpathlineto{\pgfqpoint{2.930196in}{2.975927in}}%
\pgfpathlineto{\pgfqpoint{2.939223in}{2.835023in}}%
\pgfpathlineto{\pgfqpoint{2.948250in}{2.651162in}}%
\pgfpathlineto{\pgfqpoint{2.984359in}{1.843666in}}%
\pgfpathlineto{\pgfqpoint{2.993386in}{1.714525in}}%
\pgfpathlineto{\pgfqpoint{2.997900in}{1.670772in}}%
\pgfpathlineto{\pgfqpoint{3.002413in}{1.642240in}}%
\pgfpathlineto{\pgfqpoint{3.006927in}{1.629509in}}%
\pgfpathlineto{\pgfqpoint{3.011441in}{1.632803in}}%
\pgfpathlineto{\pgfqpoint{3.015954in}{1.651982in}}%
\pgfpathlineto{\pgfqpoint{3.020468in}{1.686553in}}%
\pgfpathlineto{\pgfqpoint{3.024981in}{1.735677in}}%
\pgfpathlineto{\pgfqpoint{3.034009in}{1.872644in}}%
\pgfpathlineto{\pgfqpoint{3.043036in}{2.050247in}}%
\pgfpathlineto{\pgfqpoint{3.079145in}{2.822876in}}%
\pgfpathlineto{\pgfqpoint{3.088172in}{2.944788in}}%
\pgfpathlineto{\pgfqpoint{3.092685in}{2.985650in}}%
\pgfpathlineto{\pgfqpoint{3.097199in}{3.011861in}}%
\pgfpathlineto{\pgfqpoint{3.101713in}{3.022889in}}%
\pgfpathlineto{\pgfqpoint{3.106226in}{3.018549in}}%
\pgfpathlineto{\pgfqpoint{3.110740in}{2.998999in}}%
\pgfpathlineto{\pgfqpoint{3.115253in}{2.964739in}}%
\pgfpathlineto{\pgfqpoint{3.119767in}{2.916598in}}%
\pgfpathlineto{\pgfqpoint{3.128794in}{2.783499in}}%
\pgfpathlineto{\pgfqpoint{3.137821in}{2.611975in}}%
\pgfpathlineto{\pgfqpoint{3.169417in}{1.947233in}}%
\pgfpathlineto{\pgfqpoint{3.178444in}{1.809191in}}%
\pgfpathlineto{\pgfqpoint{3.182958in}{1.757764in}}%
\pgfpathlineto{\pgfqpoint{3.187471in}{1.719635in}}%
\pgfpathlineto{\pgfqpoint{3.191985in}{1.695607in}}%
\pgfpathlineto{\pgfqpoint{3.196498in}{1.686163in}}%
\pgfpathlineto{\pgfqpoint{3.201012in}{1.691458in}}%
\pgfpathlineto{\pgfqpoint{3.205526in}{1.711313in}}%
\pgfpathlineto{\pgfqpoint{3.210039in}{1.745224in}}%
\pgfpathlineto{\pgfqpoint{3.214553in}{1.792374in}}%
\pgfpathlineto{\pgfqpoint{3.223580in}{1.921673in}}%
\pgfpathlineto{\pgfqpoint{3.232607in}{2.087297in}}%
\pgfpathlineto{\pgfqpoint{3.264202in}{2.723725in}}%
\pgfpathlineto{\pgfqpoint{3.273230in}{2.854503in}}%
\pgfpathlineto{\pgfqpoint{3.277743in}{2.902873in}}%
\pgfpathlineto{\pgfqpoint{3.282257in}{2.938420in}}%
\pgfpathlineto{\pgfqpoint{3.286770in}{2.960398in}}%
\pgfpathlineto{\pgfqpoint{3.291284in}{2.968367in}}%
\pgfpathlineto{\pgfqpoint{3.295798in}{2.962205in}}%
\pgfpathlineto{\pgfqpoint{3.300311in}{2.942105in}}%
\pgfpathlineto{\pgfqpoint{3.304825in}{2.908576in}}%
\pgfpathlineto{\pgfqpoint{3.309338in}{2.862423in}}%
\pgfpathlineto{\pgfqpoint{3.318366in}{2.736853in}}%
\pgfpathlineto{\pgfqpoint{3.331906in}{2.488595in}}%
\pgfpathlineto{\pgfqpoint{3.354474in}{2.043403in}}%
\pgfpathlineto{\pgfqpoint{3.363502in}{1.900714in}}%
\pgfpathlineto{\pgfqpoint{3.372529in}{1.798387in}}%
\pgfpathlineto{\pgfqpoint{3.377042in}{1.765278in}}%
\pgfpathlineto{\pgfqpoint{3.381556in}{1.745225in}}%
\pgfpathlineto{\pgfqpoint{3.386070in}{1.738626in}}%
\pgfpathlineto{\pgfqpoint{3.390583in}{1.745575in}}%
\pgfpathlineto{\pgfqpoint{3.395097in}{1.765863in}}%
\pgfpathlineto{\pgfqpoint{3.399611in}{1.798980in}}%
\pgfpathlineto{\pgfqpoint{3.408638in}{1.900254in}}%
\pgfpathlineto{\pgfqpoint{3.417665in}{2.039986in}}%
\pgfpathlineto{\pgfqpoint{3.431206in}{2.293132in}}%
\pgfpathlineto{\pgfqpoint{3.449260in}{2.631520in}}%
\pgfpathlineto{\pgfqpoint{3.458287in}{2.767068in}}%
\pgfpathlineto{\pgfqpoint{3.467315in}{2.863512in}}%
\pgfpathlineto{\pgfqpoint{3.471828in}{2.894319in}}%
\pgfpathlineto{\pgfqpoint{3.476342in}{2.912566in}}%
\pgfpathlineto{\pgfqpoint{3.480855in}{2.917894in}}%
\pgfpathlineto{\pgfqpoint{3.485369in}{2.910234in}}%
\pgfpathlineto{\pgfqpoint{3.489883in}{2.889810in}}%
\pgfpathlineto{\pgfqpoint{3.494396in}{2.857133in}}%
\pgfpathlineto{\pgfqpoint{3.503423in}{2.758403in}}%
\pgfpathlineto{\pgfqpoint{3.512451in}{2.623211in}}%
\pgfpathlineto{\pgfqpoint{3.530505in}{2.294872in}}%
\pgfpathlineto{\pgfqpoint{3.544046in}{2.056111in}}%
\pgfpathlineto{\pgfqpoint{3.553073in}{1.927376in}}%
\pgfpathlineto{\pgfqpoint{3.562100in}{1.836515in}}%
\pgfpathlineto{\pgfqpoint{3.566614in}{1.807881in}}%
\pgfpathlineto{\pgfqpoint{3.571127in}{1.791327in}}%
\pgfpathlineto{\pgfqpoint{3.575641in}{1.787179in}}%
\pgfpathlineto{\pgfqpoint{3.580155in}{1.795478in}}%
\pgfpathlineto{\pgfqpoint{3.584668in}{1.815989in}}%
\pgfpathlineto{\pgfqpoint{3.589182in}{1.848203in}}%
\pgfpathlineto{\pgfqpoint{3.598209in}{1.944414in}}%
\pgfpathlineto{\pgfqpoint{3.607236in}{2.075186in}}%
\pgfpathlineto{\pgfqpoint{3.625291in}{2.390747in}}%
\pgfpathlineto{\pgfqpoint{3.638832in}{2.618787in}}%
\pgfpathlineto{\pgfqpoint{3.647859in}{2.741024in}}%
\pgfpathlineto{\pgfqpoint{3.656886in}{2.826587in}}%
\pgfpathlineto{\pgfqpoint{3.661400in}{2.853171in}}%
\pgfpathlineto{\pgfqpoint{3.665913in}{2.868137in}}%
\pgfpathlineto{\pgfqpoint{3.670427in}{2.871195in}}%
\pgfpathlineto{\pgfqpoint{3.674940in}{2.862323in}}%
\pgfpathlineto{\pgfqpoint{3.679454in}{2.841768in}}%
\pgfpathlineto{\pgfqpoint{3.683968in}{2.810039in}}%
\pgfpathlineto{\pgfqpoint{3.692995in}{2.716320in}}%
\pgfpathlineto{\pgfqpoint{3.702022in}{2.589850in}}%
\pgfpathlineto{\pgfqpoint{3.720076in}{2.286611in}}%
\pgfpathlineto{\pgfqpoint{3.733617in}{2.068843in}}%
\pgfpathlineto{\pgfqpoint{3.742644in}{1.952802in}}%
\pgfpathlineto{\pgfqpoint{3.751672in}{1.872268in}}%
\pgfpathlineto{\pgfqpoint{3.756185in}{1.847616in}}%
\pgfpathlineto{\pgfqpoint{3.760699in}{1.834137in}}%
\pgfpathlineto{\pgfqpoint{3.765212in}{1.832088in}}%
\pgfpathlineto{\pgfqpoint{3.769726in}{1.841470in}}%
\pgfpathlineto{\pgfqpoint{3.774240in}{1.862027in}}%
\pgfpathlineto{\pgfqpoint{3.778753in}{1.893252in}}%
\pgfpathlineto{\pgfqpoint{3.787781in}{1.984511in}}%
\pgfpathlineto{\pgfqpoint{3.796808in}{2.106795in}}%
\pgfpathlineto{\pgfqpoint{3.814862in}{2.398152in}}%
\pgfpathlineto{\pgfqpoint{3.828403in}{2.606080in}}%
\pgfpathlineto{\pgfqpoint{3.837430in}{2.716212in}}%
\pgfpathlineto{\pgfqpoint{3.846457in}{2.791977in}}%
\pgfpathlineto{\pgfqpoint{3.850971in}{2.814807in}}%
\pgfpathlineto{\pgfqpoint{3.855485in}{2.826895in}}%
\pgfpathlineto{\pgfqpoint{3.859998in}{2.828013in}}%
\pgfpathlineto{\pgfqpoint{3.864512in}{2.818179in}}%
\pgfpathlineto{\pgfqpoint{3.869025in}{2.797657in}}%
\pgfpathlineto{\pgfqpoint{3.873539in}{2.766952in}}%
\pgfpathlineto{\pgfqpoint{3.882566in}{2.678122in}}%
\pgfpathlineto{\pgfqpoint{3.891593in}{2.559910in}}%
\pgfpathlineto{\pgfqpoint{3.927702in}{2.025606in}}%
\pgfpathlineto{\pgfqpoint{3.936729in}{1.936768in}}%
\pgfpathlineto{\pgfqpoint{3.941243in}{1.905763in}}%
\pgfpathlineto{\pgfqpoint{3.945757in}{1.884649in}}%
\pgfpathlineto{\pgfqpoint{3.950270in}{1.873862in}}%
\pgfpathlineto{\pgfqpoint{3.954784in}{1.873603in}}%
\pgfpathlineto{\pgfqpoint{3.959297in}{1.883835in}}%
\pgfpathlineto{\pgfqpoint{3.963811in}{1.904286in}}%
\pgfpathlineto{\pgfqpoint{3.968325in}{1.934457in}}%
\pgfpathlineto{\pgfqpoint{3.977352in}{2.020892in}}%
\pgfpathlineto{\pgfqpoint{3.986379in}{2.135146in}}%
\pgfpathlineto{\pgfqpoint{4.022488in}{2.646590in}}%
\pgfpathlineto{\pgfqpoint{4.031515in}{2.730545in}}%
\pgfpathlineto{\pgfqpoint{4.036029in}{2.759568in}}%
\pgfpathlineto{\pgfqpoint{4.040542in}{2.779065in}}%
\pgfpathlineto{\pgfqpoint{4.045056in}{2.788636in}}%
\pgfpathlineto{\pgfqpoint{4.049570in}{2.788106in}}%
\pgfpathlineto{\pgfqpoint{4.054083in}{2.777527in}}%
\pgfpathlineto{\pgfqpoint{4.058597in}{2.757177in}}%
\pgfpathlineto{\pgfqpoint{4.063110in}{2.727552in}}%
\pgfpathlineto{\pgfqpoint{4.072138in}{2.643477in}}%
\pgfpathlineto{\pgfqpoint{4.081165in}{2.533070in}}%
\pgfpathlineto{\pgfqpoint{4.117274in}{2.043576in}}%
\pgfpathlineto{\pgfqpoint{4.126301in}{1.964263in}}%
\pgfpathlineto{\pgfqpoint{4.130814in}{1.937115in}}%
\pgfpathlineto{\pgfqpoint{4.135328in}{1.919140in}}%
\pgfpathlineto{\pgfqpoint{4.139842in}{1.910704in}}%
\pgfpathlineto{\pgfqpoint{4.144355in}{1.911959in}}%
\pgfpathlineto{\pgfqpoint{4.148869in}{1.922838in}}%
\pgfpathlineto{\pgfqpoint{4.153382in}{1.943057in}}%
\pgfpathlineto{\pgfqpoint{4.157896in}{1.972126in}}%
\pgfpathlineto{\pgfqpoint{4.166923in}{2.053879in}}%
\pgfpathlineto{\pgfqpoint{4.175950in}{2.160547in}}%
\pgfpathlineto{\pgfqpoint{4.212059in}{2.628969in}}%
\pgfpathlineto{\pgfqpoint{4.221087in}{2.703871in}}%
\pgfpathlineto{\pgfqpoint{4.225600in}{2.729244in}}%
\pgfpathlineto{\pgfqpoint{4.230114in}{2.745788in}}%
\pgfpathlineto{\pgfqpoint{4.234627in}{2.753164in}}%
\pgfpathlineto{\pgfqpoint{4.239141in}{2.751246in}}%
\pgfpathlineto{\pgfqpoint{4.243655in}{2.740111in}}%
\pgfpathlineto{\pgfqpoint{4.248168in}{2.720049in}}%
\pgfpathlineto{\pgfqpoint{4.252682in}{2.691544in}}%
\pgfpathlineto{\pgfqpoint{4.261709in}{2.612076in}}%
\pgfpathlineto{\pgfqpoint{4.270736in}{2.509039in}}%
\pgfpathlineto{\pgfqpoint{4.306845in}{2.060843in}}%
\pgfpathlineto{\pgfqpoint{4.315872in}{1.990134in}}%
\pgfpathlineto{\pgfqpoint{4.320386in}{1.966438in}}%
\pgfpathlineto{\pgfqpoint{4.324899in}{1.951241in}}%
\pgfpathlineto{\pgfqpoint{4.329413in}{1.944851in}}%
\pgfpathlineto{\pgfqpoint{4.333927in}{1.947377in}}%
\pgfpathlineto{\pgfqpoint{4.338440in}{1.958726in}}%
\pgfpathlineto{\pgfqpoint{4.342954in}{1.978608in}}%
\pgfpathlineto{\pgfqpoint{4.347467in}{2.006542in}}%
\pgfpathlineto{\pgfqpoint{4.356495in}{2.083765in}}%
\pgfpathlineto{\pgfqpoint{4.365522in}{2.183276in}}%
\pgfpathlineto{\pgfqpoint{4.397117in}{2.568890in}}%
\pgfpathlineto{\pgfqpoint{4.406144in}{2.648958in}}%
\pgfpathlineto{\pgfqpoint{4.410658in}{2.678784in}}%
\pgfpathlineto{\pgfqpoint{4.415172in}{2.700896in}}%
\pgfpathlineto{\pgfqpoint{4.419685in}{2.714827in}}%
\pgfpathlineto{\pgfqpoint{4.424199in}{2.720298in}}%
\pgfpathlineto{\pgfqpoint{4.428712in}{2.717219in}}%
\pgfpathlineto{\pgfqpoint{4.433226in}{2.705693in}}%
\pgfpathlineto{\pgfqpoint{4.437740in}{2.686013in}}%
\pgfpathlineto{\pgfqpoint{4.442253in}{2.658655in}}%
\pgfpathlineto{\pgfqpoint{4.451280in}{2.583636in}}%
\pgfpathlineto{\pgfqpoint{4.460308in}{2.487549in}}%
\pgfpathlineto{\pgfqpoint{4.491903in}{2.118360in}}%
\pgfpathlineto{\pgfqpoint{4.500930in}{2.042506in}}%
\pgfpathlineto{\pgfqpoint{4.505444in}{2.014453in}}%
\pgfpathlineto{\pgfqpoint{4.509957in}{1.993839in}}%
\pgfpathlineto{\pgfqpoint{4.514471in}{1.981096in}}%
\pgfpathlineto{\pgfqpoint{4.518984in}{1.976481in}}%
\pgfpathlineto{\pgfqpoint{4.523498in}{1.980063in}}%
\pgfpathlineto{\pgfqpoint{4.528012in}{1.991730in}}%
\pgfpathlineto{\pgfqpoint{4.532525in}{2.011188in}}%
\pgfpathlineto{\pgfqpoint{4.537039in}{2.037967in}}%
\pgfpathlineto{\pgfqpoint{4.546066in}{2.110822in}}%
\pgfpathlineto{\pgfqpoint{4.559607in}{2.254848in}}%
\pgfpathlineto{\pgfqpoint{4.582175in}{2.513101in}}%
\pgfpathlineto{\pgfqpoint{4.591202in}{2.595867in}}%
\pgfpathlineto{\pgfqpoint{4.600229in}{2.655216in}}%
\pgfpathlineto{\pgfqpoint{4.604743in}{2.674416in}}%
\pgfpathlineto{\pgfqpoint{4.609257in}{2.686042in}}%
\pgfpathlineto{\pgfqpoint{4.613770in}{2.689863in}}%
\pgfpathlineto{\pgfqpoint{4.618284in}{2.685824in}}%
\pgfpathlineto{\pgfqpoint{4.622797in}{2.674048in}}%
\pgfpathlineto{\pgfqpoint{4.627311in}{2.654830in}}%
\pgfpathlineto{\pgfqpoint{4.636338in}{2.596069in}}%
\pgfpathlineto{\pgfqpoint{4.645365in}{2.515001in}}%
\pgfpathlineto{\pgfqpoint{4.658906in}{2.368143in}}%
\pgfpathlineto{\pgfqpoint{4.676961in}{2.171849in}}%
\pgfpathlineto{\pgfqpoint{4.685988in}{2.093226in}}%
\pgfpathlineto{\pgfqpoint{4.695015in}{2.037289in}}%
\pgfpathlineto{\pgfqpoint{4.699529in}{2.019424in}}%
\pgfpathlineto{\pgfqpoint{4.704042in}{2.008846in}}%
\pgfpathlineto{\pgfqpoint{4.708556in}{2.005762in}}%
\pgfpathlineto{\pgfqpoint{4.713069in}{2.010212in}}%
\pgfpathlineto{\pgfqpoint{4.717583in}{2.022067in}}%
\pgfpathlineto{\pgfqpoint{4.722097in}{2.041029in}}%
\pgfpathlineto{\pgfqpoint{4.731124in}{2.098314in}}%
\pgfpathlineto{\pgfqpoint{4.740151in}{2.176748in}}%
\pgfpathlineto{\pgfqpoint{4.758205in}{2.367225in}}%
\pgfpathlineto{\pgfqpoint{4.771746in}{2.505725in}}%
\pgfpathlineto{\pgfqpoint{4.780774in}{2.580397in}}%
\pgfpathlineto{\pgfqpoint{4.789801in}{2.633095in}}%
\pgfpathlineto{\pgfqpoint{4.794314in}{2.649699in}}%
\pgfpathlineto{\pgfqpoint{4.798828in}{2.659296in}}%
\pgfpathlineto{\pgfqpoint{4.803342in}{2.661696in}}%
\pgfpathlineto{\pgfqpoint{4.807855in}{2.656875in}}%
\pgfpathlineto{\pgfqpoint{4.812369in}{2.644970in}}%
\pgfpathlineto{\pgfqpoint{4.816882in}{2.626276in}}%
\pgfpathlineto{\pgfqpoint{4.825910in}{2.570453in}}%
\pgfpathlineto{\pgfqpoint{4.834937in}{2.494584in}}%
\pgfpathlineto{\pgfqpoint{4.852991in}{2.311520in}}%
\pgfpathlineto{\pgfqpoint{4.866532in}{2.179239in}}%
\pgfpathlineto{\pgfqpoint{4.875559in}{2.108337in}}%
\pgfpathlineto{\pgfqpoint{4.884586in}{2.058712in}}%
\pgfpathlineto{\pgfqpoint{4.889100in}{2.043296in}}%
\pgfpathlineto{\pgfqpoint{4.893614in}{2.034621in}}%
\pgfpathlineto{\pgfqpoint{4.898127in}{2.032853in}}%
\pgfpathlineto{\pgfqpoint{4.902641in}{2.038006in}}%
\pgfpathlineto{\pgfqpoint{4.907154in}{2.049936in}}%
\pgfpathlineto{\pgfqpoint{4.911668in}{2.068348in}}%
\pgfpathlineto{\pgfqpoint{4.920695in}{2.122725in}}%
\pgfpathlineto{\pgfqpoint{4.929722in}{2.196098in}}%
\pgfpathlineto{\pgfqpoint{4.947777in}{2.372013in}}%
\pgfpathlineto{\pgfqpoint{4.961318in}{2.498336in}}%
\pgfpathlineto{\pgfqpoint{4.970345in}{2.565644in}}%
\pgfpathlineto{\pgfqpoint{4.979372in}{2.612352in}}%
\pgfpathlineto{\pgfqpoint{4.983886in}{2.626647in}}%
\pgfpathlineto{\pgfqpoint{4.988399in}{2.634460in}}%
\pgfpathlineto{\pgfqpoint{4.992913in}{2.635643in}}%
\pgfpathlineto{\pgfqpoint{4.997427in}{2.630194in}}%
\pgfpathlineto{\pgfqpoint{5.001940in}{2.618263in}}%
\pgfpathlineto{\pgfqpoint{5.006454in}{2.600144in}}%
\pgfpathlineto{\pgfqpoint{5.015481in}{2.547195in}}%
\pgfpathlineto{\pgfqpoint{5.024508in}{2.476250in}}%
\pgfpathlineto{\pgfqpoint{5.042563in}{2.307228in}}%
\pgfpathlineto{\pgfqpoint{5.056103in}{2.186614in}}%
\pgfpathlineto{\pgfqpoint{5.065131in}{2.122734in}}%
\pgfpathlineto{\pgfqpoint{5.074158in}{2.078792in}}%
\pgfpathlineto{\pgfqpoint{5.078671in}{2.065553in}}%
\pgfpathlineto{\pgfqpoint{5.083185in}{2.058547in}}%
\pgfpathlineto{\pgfqpoint{5.087699in}{2.057904in}}%
\pgfpathlineto{\pgfqpoint{5.092212in}{2.063615in}}%
\pgfpathlineto{\pgfqpoint{5.096726in}{2.075525in}}%
\pgfpathlineto{\pgfqpoint{5.101239in}{2.093342in}}%
\pgfpathlineto{\pgfqpoint{5.110267in}{2.144882in}}%
\pgfpathlineto{\pgfqpoint{5.119294in}{2.213464in}}%
\pgfpathlineto{\pgfqpoint{5.155403in}{2.523411in}}%
\pgfpathlineto{\pgfqpoint{5.164430in}{2.574938in}}%
\pgfpathlineto{\pgfqpoint{5.168943in}{2.592919in}}%
\pgfpathlineto{\pgfqpoint{5.173457in}{2.605162in}}%
\pgfpathlineto{\pgfqpoint{5.177971in}{2.611414in}}%
\pgfpathlineto{\pgfqpoint{5.182484in}{2.611559in}}%
\pgfpathlineto{\pgfqpoint{5.186998in}{2.605618in}}%
\pgfpathlineto{\pgfqpoint{5.191512in}{2.593749in}}%
\pgfpathlineto{\pgfqpoint{5.196025in}{2.576241in}}%
\pgfpathlineto{\pgfqpoint{5.205052in}{2.526091in}}%
\pgfpathlineto{\pgfqpoint{5.214080in}{2.459806in}}%
\pgfpathlineto{\pgfqpoint{5.250188in}{2.163120in}}%
\pgfpathlineto{\pgfqpoint{5.259216in}{2.114426in}}%
\pgfpathlineto{\pgfqpoint{5.263729in}{2.097594in}}%
\pgfpathlineto{\pgfqpoint{5.268243in}{2.086289in}}%
\pgfpathlineto{\pgfqpoint{5.272756in}{2.080742in}}%
\pgfpathlineto{\pgfqpoint{5.277270in}{2.081055in}}%
\pgfpathlineto{\pgfqpoint{5.281784in}{2.087197in}}%
\pgfpathlineto{\pgfqpoint{5.286297in}{2.099007in}}%
\pgfpathlineto{\pgfqpoint{5.290811in}{2.116198in}}%
\pgfpathlineto{\pgfqpoint{5.299838in}{2.164978in}}%
\pgfpathlineto{\pgfqpoint{5.308865in}{2.229032in}}%
\pgfpathlineto{\pgfqpoint{5.344974in}{2.512984in}}%
\pgfpathlineto{\pgfqpoint{5.354001in}{2.558986in}}%
\pgfpathlineto{\pgfqpoint{5.358515in}{2.574730in}}%
\pgfpathlineto{\pgfqpoint{5.363028in}{2.585152in}}%
\pgfpathlineto{\pgfqpoint{5.367542in}{2.590041in}}%
\pgfpathlineto{\pgfqpoint{5.372056in}{2.589308in}}%
\pgfpathlineto{\pgfqpoint{5.376569in}{2.582992in}}%
\pgfpathlineto{\pgfqpoint{5.381083in}{2.571258in}}%
\pgfpathlineto{\pgfqpoint{5.385597in}{2.554390in}}%
\pgfpathlineto{\pgfqpoint{5.394624in}{2.506957in}}%
\pgfpathlineto{\pgfqpoint{5.403651in}{2.445073in}}%
\pgfpathlineto{\pgfqpoint{5.439760in}{2.173345in}}%
\pgfpathlineto{\pgfqpoint{5.448787in}{2.129902in}}%
\pgfpathlineto{\pgfqpoint{5.453301in}{2.115187in}}%
\pgfpathlineto{\pgfqpoint{5.457814in}{2.105595in}}%
\pgfpathlineto{\pgfqpoint{5.462328in}{2.101320in}}%
\pgfpathlineto{\pgfqpoint{5.466841in}{2.102438in}}%
\pgfpathlineto{\pgfqpoint{5.471355in}{2.108902in}}%
\pgfpathlineto{\pgfqpoint{5.475869in}{2.120545in}}%
\pgfpathlineto{\pgfqpoint{5.480382in}{2.137085in}}%
\pgfpathlineto{\pgfqpoint{5.489409in}{2.183193in}}%
\pgfpathlineto{\pgfqpoint{5.498437in}{2.242970in}}%
\pgfpathlineto{\pgfqpoint{5.534545in}{2.502965in}}%
\pgfpathlineto{\pgfqpoint{5.534545in}{2.502965in}}%
\pgfusepath{stroke}%
\end{pgfscope}%
\begin{pgfscope}%
\pgfpathrectangle{\pgfqpoint{0.800000in}{0.528000in}}{\pgfqpoint{4.960000in}{3.696000in}}%
\pgfusepath{clip}%
\pgfsetbuttcap%
\pgfsetroundjoin%
\definecolor{currentfill}{rgb}{0.000000,0.000000,1.000000}%
\pgfsetfillcolor{currentfill}%
\pgfsetlinewidth{0.501875pt}%
\definecolor{currentstroke}{rgb}{0.000000,0.000000,1.000000}%
\pgfsetstrokecolor{currentstroke}%
\pgfsetdash{}{0pt}%
\pgfsys@defobject{currentmarker}{\pgfqpoint{-0.027778in}{-0.000000in}}{\pgfqpoint{0.027778in}{0.000000in}}{%
\pgfpathmoveto{\pgfqpoint{0.027778in}{-0.000000in}}%
\pgfpathlineto{\pgfqpoint{-0.027778in}{0.000000in}}%
\pgfusepath{stroke,fill}%
}%
\begin{pgfscope}%
\pgfsys@transformshift{1.043491in}{3.584274in}%
\pgfsys@useobject{currentmarker}{}%
\end{pgfscope}%
\begin{pgfscope}%
\pgfsys@transformshift{1.061527in}{3.455622in}%
\pgfsys@useobject{currentmarker}{}%
\end{pgfscope}%
\begin{pgfscope}%
\pgfsys@transformshift{1.079564in}{3.112549in}%
\pgfsys@useobject{currentmarker}{}%
\end{pgfscope}%
\begin{pgfscope}%
\pgfsys@transformshift{1.097600in}{2.683707in}%
\pgfsys@useobject{currentmarker}{}%
\end{pgfscope}%
\begin{pgfscope}%
\pgfsys@transformshift{1.115636in}{2.211981in}%
\pgfsys@useobject{currentmarker}{}%
\end{pgfscope}%
\begin{pgfscope}%
\pgfsys@transformshift{1.133673in}{1.654487in}%
\pgfsys@useobject{currentmarker}{}%
\end{pgfscope}%
\begin{pgfscope}%
\pgfsys@transformshift{1.151709in}{1.354298in}%
\pgfsys@useobject{currentmarker}{}%
\end{pgfscope}%
\begin{pgfscope}%
\pgfsys@transformshift{1.169745in}{1.268530in}%
\pgfsys@useobject{currentmarker}{}%
\end{pgfscope}%
\begin{pgfscope}%
\pgfsys@transformshift{1.187782in}{1.354298in}%
\pgfsys@useobject{currentmarker}{}%
\end{pgfscope}%
\begin{pgfscope}%
\pgfsys@transformshift{1.205818in}{1.654487in}%
\pgfsys@useobject{currentmarker}{}%
\end{pgfscope}%
\begin{pgfscope}%
\pgfsys@transformshift{1.223855in}{2.211981in}%
\pgfsys@useobject{currentmarker}{}%
\end{pgfscope}%
\begin{pgfscope}%
\pgfsys@transformshift{1.241891in}{2.640823in}%
\pgfsys@useobject{currentmarker}{}%
\end{pgfscope}%
\begin{pgfscope}%
\pgfsys@transformshift{1.259927in}{3.069664in}%
\pgfsys@useobject{currentmarker}{}%
\end{pgfscope}%
\begin{pgfscope}%
\pgfsys@transformshift{1.277964in}{3.369854in}%
\pgfsys@useobject{currentmarker}{}%
\end{pgfscope}%
\begin{pgfscope}%
\pgfsys@transformshift{1.296000in}{3.498506in}%
\pgfsys@useobject{currentmarker}{}%
\end{pgfscope}%
\begin{pgfscope}%
\pgfsys@transformshift{1.314036in}{3.498506in}%
\pgfsys@useobject{currentmarker}{}%
\end{pgfscope}%
\begin{pgfscope}%
\pgfsys@transformshift{1.332073in}{3.284085in}%
\pgfsys@useobject{currentmarker}{}%
\end{pgfscope}%
\begin{pgfscope}%
\pgfsys@transformshift{1.350109in}{2.941012in}%
\pgfsys@useobject{currentmarker}{}%
\end{pgfscope}%
\begin{pgfscope}%
\pgfsys@transformshift{1.368145in}{2.512170in}%
\pgfsys@useobject{currentmarker}{}%
\end{pgfscope}%
\begin{pgfscope}%
\pgfsys@transformshift{1.386182in}{2.126213in}%
\pgfsys@useobject{currentmarker}{}%
\end{pgfscope}%
\begin{pgfscope}%
\pgfsys@transformshift{1.404218in}{1.611603in}%
\pgfsys@useobject{currentmarker}{}%
\end{pgfscope}%
\begin{pgfscope}%
\pgfsys@transformshift{1.422255in}{1.397182in}%
\pgfsys@useobject{currentmarker}{}%
\end{pgfscope}%
\begin{pgfscope}%
\pgfsys@transformshift{1.440291in}{1.397182in}%
\pgfsys@useobject{currentmarker}{}%
\end{pgfscope}%
\begin{pgfscope}%
\pgfsys@transformshift{1.458327in}{1.525835in}%
\pgfsys@useobject{currentmarker}{}%
\end{pgfscope}%
\begin{pgfscope}%
\pgfsys@transformshift{1.476364in}{1.826024in}%
\pgfsys@useobject{currentmarker}{}%
\end{pgfscope}%
\begin{pgfscope}%
\pgfsys@transformshift{1.494400in}{2.340634in}%
\pgfsys@useobject{currentmarker}{}%
\end{pgfscope}%
\begin{pgfscope}%
\pgfsys@transformshift{1.512436in}{2.769475in}%
\pgfsys@useobject{currentmarker}{}%
\end{pgfscope}%
\begin{pgfscope}%
\pgfsys@transformshift{1.530473in}{3.112549in}%
\pgfsys@useobject{currentmarker}{}%
\end{pgfscope}%
\begin{pgfscope}%
\pgfsys@transformshift{1.548509in}{3.326969in}%
\pgfsys@useobject{currentmarker}{}%
\end{pgfscope}%
\begin{pgfscope}%
\pgfsys@transformshift{1.566545in}{3.412738in}%
\pgfsys@useobject{currentmarker}{}%
\end{pgfscope}%
\begin{pgfscope}%
\pgfsys@transformshift{1.584582in}{3.369854in}%
\pgfsys@useobject{currentmarker}{}%
\end{pgfscope}%
\begin{pgfscope}%
\pgfsys@transformshift{1.602618in}{3.112549in}%
\pgfsys@useobject{currentmarker}{}%
\end{pgfscope}%
\begin{pgfscope}%
\pgfsys@transformshift{1.620655in}{2.769475in}%
\pgfsys@useobject{currentmarker}{}%
\end{pgfscope}%
\begin{pgfscope}%
\pgfsys@transformshift{1.638691in}{2.383518in}%
\pgfsys@useobject{currentmarker}{}%
\end{pgfscope}%
\begin{pgfscope}%
\pgfsys@transformshift{1.656727in}{1.911792in}%
\pgfsys@useobject{currentmarker}{}%
\end{pgfscope}%
\begin{pgfscope}%
\pgfsys@transformshift{1.674764in}{1.611603in}%
\pgfsys@useobject{currentmarker}{}%
\end{pgfscope}%
\begin{pgfscope}%
\pgfsys@transformshift{1.692800in}{1.482951in}%
\pgfsys@useobject{currentmarker}{}%
\end{pgfscope}%
\begin{pgfscope}%
\pgfsys@transformshift{1.710836in}{1.482951in}%
\pgfsys@useobject{currentmarker}{}%
\end{pgfscope}%
\begin{pgfscope}%
\pgfsys@transformshift{1.728873in}{1.697371in}%
\pgfsys@useobject{currentmarker}{}%
\end{pgfscope}%
\begin{pgfscope}%
\pgfsys@transformshift{1.746909in}{2.126213in}%
\pgfsys@useobject{currentmarker}{}%
\end{pgfscope}%
\begin{pgfscope}%
\pgfsys@transformshift{1.764945in}{2.512170in}%
\pgfsys@useobject{currentmarker}{}%
\end{pgfscope}%
\begin{pgfscope}%
\pgfsys@transformshift{1.782982in}{2.855244in}%
\pgfsys@useobject{currentmarker}{}%
\end{pgfscope}%
\begin{pgfscope}%
\pgfsys@transformshift{1.801018in}{3.155433in}%
\pgfsys@useobject{currentmarker}{}%
\end{pgfscope}%
\begin{pgfscope}%
\pgfsys@transformshift{1.819055in}{3.326969in}%
\pgfsys@useobject{currentmarker}{}%
\end{pgfscope}%
\begin{pgfscope}%
\pgfsys@transformshift{1.837091in}{3.326969in}%
\pgfsys@useobject{currentmarker}{}%
\end{pgfscope}%
\begin{pgfscope}%
\pgfsys@transformshift{1.855127in}{3.241201in}%
\pgfsys@useobject{currentmarker}{}%
\end{pgfscope}%
\begin{pgfscope}%
\pgfsys@transformshift{1.873164in}{2.983896in}%
\pgfsys@useobject{currentmarker}{}%
\end{pgfscope}%
\begin{pgfscope}%
\pgfsys@transformshift{1.891200in}{2.640823in}%
\pgfsys@useobject{currentmarker}{}%
\end{pgfscope}%
\begin{pgfscope}%
\pgfsys@transformshift{1.909236in}{2.297750in}%
\pgfsys@useobject{currentmarker}{}%
\end{pgfscope}%
\begin{pgfscope}%
\pgfsys@transformshift{1.927273in}{1.826024in}%
\pgfsys@useobject{currentmarker}{}%
\end{pgfscope}%
\begin{pgfscope}%
\pgfsys@transformshift{1.945309in}{1.611603in}%
\pgfsys@useobject{currentmarker}{}%
\end{pgfscope}%
\begin{pgfscope}%
\pgfsys@transformshift{1.963345in}{1.525835in}%
\pgfsys@useobject{currentmarker}{}%
\end{pgfscope}%
\begin{pgfscope}%
\pgfsys@transformshift{1.981382in}{1.611603in}%
\pgfsys@useobject{currentmarker}{}%
\end{pgfscope}%
\begin{pgfscope}%
\pgfsys@transformshift{1.999418in}{1.826024in}%
\pgfsys@useobject{currentmarker}{}%
\end{pgfscope}%
\begin{pgfscope}%
\pgfsys@transformshift{2.017455in}{2.254865in}%
\pgfsys@useobject{currentmarker}{}%
\end{pgfscope}%
\begin{pgfscope}%
\pgfsys@transformshift{2.035491in}{2.640823in}%
\pgfsys@useobject{currentmarker}{}%
\end{pgfscope}%
\begin{pgfscope}%
\pgfsys@transformshift{2.053527in}{2.941012in}%
\pgfsys@useobject{currentmarker}{}%
\end{pgfscope}%
\begin{pgfscope}%
\pgfsys@transformshift{2.071564in}{3.155433in}%
\pgfsys@useobject{currentmarker}{}%
\end{pgfscope}%
\begin{pgfscope}%
\pgfsys@transformshift{2.089600in}{3.284085in}%
\pgfsys@useobject{currentmarker}{}%
\end{pgfscope}%
\begin{pgfscope}%
\pgfsys@transformshift{2.107636in}{3.284085in}%
\pgfsys@useobject{currentmarker}{}%
\end{pgfscope}%
\begin{pgfscope}%
\pgfsys@transformshift{2.125673in}{3.112549in}%
\pgfsys@useobject{currentmarker}{}%
\end{pgfscope}%
\begin{pgfscope}%
\pgfsys@transformshift{2.143709in}{2.855244in}%
\pgfsys@useobject{currentmarker}{}%
\end{pgfscope}%
\begin{pgfscope}%
\pgfsys@transformshift{2.161745in}{2.512170in}%
\pgfsys@useobject{currentmarker}{}%
\end{pgfscope}%
\begin{pgfscope}%
\pgfsys@transformshift{2.179782in}{2.211981in}%
\pgfsys@useobject{currentmarker}{}%
\end{pgfscope}%
\begin{pgfscope}%
\pgfsys@transformshift{2.197818in}{1.783140in}%
\pgfsys@useobject{currentmarker}{}%
\end{pgfscope}%
\begin{pgfscope}%
\pgfsys@transformshift{2.215855in}{1.654487in}%
\pgfsys@useobject{currentmarker}{}%
\end{pgfscope}%
\begin{pgfscope}%
\pgfsys@transformshift{2.233891in}{1.611603in}%
\pgfsys@useobject{currentmarker}{}%
\end{pgfscope}%
\begin{pgfscope}%
\pgfsys@transformshift{2.251927in}{1.740256in}%
\pgfsys@useobject{currentmarker}{}%
\end{pgfscope}%
\begin{pgfscope}%
\pgfsys@transformshift{2.269964in}{1.997561in}%
\pgfsys@useobject{currentmarker}{}%
\end{pgfscope}%
\begin{pgfscope}%
\pgfsys@transformshift{2.288000in}{2.426402in}%
\pgfsys@useobject{currentmarker}{}%
\end{pgfscope}%
\begin{pgfscope}%
\pgfsys@transformshift{2.306036in}{2.726591in}%
\pgfsys@useobject{currentmarker}{}%
\end{pgfscope}%
\begin{pgfscope}%
\pgfsys@transformshift{2.324073in}{2.983896in}%
\pgfsys@useobject{currentmarker}{}%
\end{pgfscope}%
\begin{pgfscope}%
\pgfsys@transformshift{2.342109in}{3.155433in}%
\pgfsys@useobject{currentmarker}{}%
\end{pgfscope}%
\begin{pgfscope}%
\pgfsys@transformshift{2.360145in}{3.241201in}%
\pgfsys@useobject{currentmarker}{}%
\end{pgfscope}%
\begin{pgfscope}%
\pgfsys@transformshift{2.378182in}{3.155433in}%
\pgfsys@useobject{currentmarker}{}%
\end{pgfscope}%
\begin{pgfscope}%
\pgfsys@transformshift{2.396218in}{2.983896in}%
\pgfsys@useobject{currentmarker}{}%
\end{pgfscope}%
\begin{pgfscope}%
\pgfsys@transformshift{2.414255in}{2.726591in}%
\pgfsys@useobject{currentmarker}{}%
\end{pgfscope}%
\begin{pgfscope}%
\pgfsys@transformshift{2.432291in}{2.426402in}%
\pgfsys@useobject{currentmarker}{}%
\end{pgfscope}%
\begin{pgfscope}%
\pgfsys@transformshift{2.450327in}{1.997561in}%
\pgfsys@useobject{currentmarker}{}%
\end{pgfscope}%
\begin{pgfscope}%
\pgfsys@transformshift{2.468364in}{1.783140in}%
\pgfsys@useobject{currentmarker}{}%
\end{pgfscope}%
\begin{pgfscope}%
\pgfsys@transformshift{2.486400in}{1.654487in}%
\pgfsys@useobject{currentmarker}{}%
\end{pgfscope}%
\begin{pgfscope}%
\pgfsys@transformshift{2.504436in}{1.697371in}%
\pgfsys@useobject{currentmarker}{}%
\end{pgfscope}%
\begin{pgfscope}%
\pgfsys@transformshift{2.522473in}{1.868908in}%
\pgfsys@useobject{currentmarker}{}%
\end{pgfscope}%
\begin{pgfscope}%
\pgfsys@transformshift{2.540509in}{2.211981in}%
\pgfsys@useobject{currentmarker}{}%
\end{pgfscope}%
\begin{pgfscope}%
\pgfsys@transformshift{2.558545in}{2.512170in}%
\pgfsys@useobject{currentmarker}{}%
\end{pgfscope}%
\begin{pgfscope}%
\pgfsys@transformshift{2.576582in}{2.812359in}%
\pgfsys@useobject{currentmarker}{}%
\end{pgfscope}%
\begin{pgfscope}%
\pgfsys@transformshift{2.594618in}{3.026780in}%
\pgfsys@useobject{currentmarker}{}%
\end{pgfscope}%
\begin{pgfscope}%
\pgfsys@transformshift{2.612655in}{3.155433in}%
\pgfsys@useobject{currentmarker}{}%
\end{pgfscope}%
\begin{pgfscope}%
\pgfsys@transformshift{2.630691in}{3.155433in}%
\pgfsys@useobject{currentmarker}{}%
\end{pgfscope}%
\begin{pgfscope}%
\pgfsys@transformshift{2.648727in}{3.069664in}%
\pgfsys@useobject{currentmarker}{}%
\end{pgfscope}%
\begin{pgfscope}%
\pgfsys@transformshift{2.666764in}{2.855244in}%
\pgfsys@useobject{currentmarker}{}%
\end{pgfscope}%
\begin{pgfscope}%
\pgfsys@transformshift{2.684800in}{2.597939in}%
\pgfsys@useobject{currentmarker}{}%
\end{pgfscope}%
\begin{pgfscope}%
\pgfsys@transformshift{2.702836in}{2.340634in}%
\pgfsys@useobject{currentmarker}{}%
\end{pgfscope}%
\begin{pgfscope}%
\pgfsys@transformshift{2.720873in}{2.040445in}%
\pgfsys@useobject{currentmarker}{}%
\end{pgfscope}%
\begin{pgfscope}%
\pgfsys@transformshift{2.738909in}{1.783140in}%
\pgfsys@useobject{currentmarker}{}%
\end{pgfscope}%
\begin{pgfscope}%
\pgfsys@transformshift{2.756945in}{1.740256in}%
\pgfsys@useobject{currentmarker}{}%
\end{pgfscope}%
\begin{pgfscope}%
\pgfsys@transformshift{2.774982in}{1.783140in}%
\pgfsys@useobject{currentmarker}{}%
\end{pgfscope}%
\begin{pgfscope}%
\pgfsys@transformshift{2.793018in}{2.083329in}%
\pgfsys@useobject{currentmarker}{}%
\end{pgfscope}%
\begin{pgfscope}%
\pgfsys@transformshift{2.811055in}{2.340634in}%
\pgfsys@useobject{currentmarker}{}%
\end{pgfscope}%
\begin{pgfscope}%
\pgfsys@transformshift{2.829091in}{2.640823in}%
\pgfsys@useobject{currentmarker}{}%
\end{pgfscope}%
\begin{pgfscope}%
\pgfsys@transformshift{2.847127in}{2.855244in}%
\pgfsys@useobject{currentmarker}{}%
\end{pgfscope}%
\begin{pgfscope}%
\pgfsys@transformshift{2.865164in}{3.069664in}%
\pgfsys@useobject{currentmarker}{}%
\end{pgfscope}%
\begin{pgfscope}%
\pgfsys@transformshift{2.883200in}{3.112549in}%
\pgfsys@useobject{currentmarker}{}%
\end{pgfscope}%
\begin{pgfscope}%
\pgfsys@transformshift{2.901236in}{3.112549in}%
\pgfsys@useobject{currentmarker}{}%
\end{pgfscope}%
\begin{pgfscope}%
\pgfsys@transformshift{2.919273in}{2.983896in}%
\pgfsys@useobject{currentmarker}{}%
\end{pgfscope}%
\begin{pgfscope}%
\pgfsys@transformshift{2.937309in}{2.726591in}%
\pgfsys@useobject{currentmarker}{}%
\end{pgfscope}%
\begin{pgfscope}%
\pgfsys@transformshift{2.955345in}{2.512170in}%
\pgfsys@useobject{currentmarker}{}%
\end{pgfscope}%
\begin{pgfscope}%
\pgfsys@transformshift{2.973382in}{2.254865in}%
\pgfsys@useobject{currentmarker}{}%
\end{pgfscope}%
\begin{pgfscope}%
\pgfsys@transformshift{2.991418in}{2.040445in}%
\pgfsys@useobject{currentmarker}{}%
\end{pgfscope}%
\begin{pgfscope}%
\pgfsys@transformshift{3.009455in}{1.783140in}%
\pgfsys@useobject{currentmarker}{}%
\end{pgfscope}%
\begin{pgfscope}%
\pgfsys@transformshift{3.027491in}{1.783140in}%
\pgfsys@useobject{currentmarker}{}%
\end{pgfscope}%
\begin{pgfscope}%
\pgfsys@transformshift{3.045527in}{1.911792in}%
\pgfsys@useobject{currentmarker}{}%
\end{pgfscope}%
\begin{pgfscope}%
\pgfsys@transformshift{3.063564in}{2.211981in}%
\pgfsys@useobject{currentmarker}{}%
\end{pgfscope}%
\begin{pgfscope}%
\pgfsys@transformshift{3.081600in}{2.469286in}%
\pgfsys@useobject{currentmarker}{}%
\end{pgfscope}%
\begin{pgfscope}%
\pgfsys@transformshift{3.099636in}{2.683707in}%
\pgfsys@useobject{currentmarker}{}%
\end{pgfscope}%
\begin{pgfscope}%
\pgfsys@transformshift{3.117673in}{2.898128in}%
\pgfsys@useobject{currentmarker}{}%
\end{pgfscope}%
\begin{pgfscope}%
\pgfsys@transformshift{3.135709in}{3.026780in}%
\pgfsys@useobject{currentmarker}{}%
\end{pgfscope}%
\begin{pgfscope}%
\pgfsys@transformshift{3.153745in}{3.069664in}%
\pgfsys@useobject{currentmarker}{}%
\end{pgfscope}%
\begin{pgfscope}%
\pgfsys@transformshift{3.171782in}{3.026780in}%
\pgfsys@useobject{currentmarker}{}%
\end{pgfscope}%
\begin{pgfscope}%
\pgfsys@transformshift{3.189818in}{2.855244in}%
\pgfsys@useobject{currentmarker}{}%
\end{pgfscope}%
\begin{pgfscope}%
\pgfsys@transformshift{3.207855in}{2.640823in}%
\pgfsys@useobject{currentmarker}{}%
\end{pgfscope}%
\begin{pgfscope}%
\pgfsys@transformshift{3.225891in}{2.426402in}%
\pgfsys@useobject{currentmarker}{}%
\end{pgfscope}%
\begin{pgfscope}%
\pgfsys@transformshift{3.243927in}{2.169097in}%
\pgfsys@useobject{currentmarker}{}%
\end{pgfscope}%
\begin{pgfscope}%
\pgfsys@transformshift{3.261964in}{1.911792in}%
\pgfsys@useobject{currentmarker}{}%
\end{pgfscope}%
\begin{pgfscope}%
\pgfsys@transformshift{3.280000in}{1.826024in}%
\pgfsys@useobject{currentmarker}{}%
\end{pgfscope}%
\begin{pgfscope}%
\pgfsys@transformshift{3.298036in}{1.868908in}%
\pgfsys@useobject{currentmarker}{}%
\end{pgfscope}%
\begin{pgfscope}%
\pgfsys@transformshift{3.316073in}{1.997561in}%
\pgfsys@useobject{currentmarker}{}%
\end{pgfscope}%
\begin{pgfscope}%
\pgfsys@transformshift{3.334109in}{2.297750in}%
\pgfsys@useobject{currentmarker}{}%
\end{pgfscope}%
\begin{pgfscope}%
\pgfsys@transformshift{3.352145in}{2.555055in}%
\pgfsys@useobject{currentmarker}{}%
\end{pgfscope}%
\begin{pgfscope}%
\pgfsys@transformshift{3.370182in}{2.769475in}%
\pgfsys@useobject{currentmarker}{}%
\end{pgfscope}%
\begin{pgfscope}%
\pgfsys@transformshift{3.388218in}{2.941012in}%
\pgfsys@useobject{currentmarker}{}%
\end{pgfscope}%
\begin{pgfscope}%
\pgfsys@transformshift{3.406255in}{3.026780in}%
\pgfsys@useobject{currentmarker}{}%
\end{pgfscope}%
\begin{pgfscope}%
\pgfsys@transformshift{3.424291in}{3.026780in}%
\pgfsys@useobject{currentmarker}{}%
\end{pgfscope}%
\begin{pgfscope}%
\pgfsys@transformshift{3.442327in}{2.941012in}%
\pgfsys@useobject{currentmarker}{}%
\end{pgfscope}%
\begin{pgfscope}%
\pgfsys@transformshift{3.460364in}{2.769475in}%
\pgfsys@useobject{currentmarker}{}%
\end{pgfscope}%
\begin{pgfscope}%
\pgfsys@transformshift{3.478400in}{2.555055in}%
\pgfsys@useobject{currentmarker}{}%
\end{pgfscope}%
\begin{pgfscope}%
\pgfsys@transformshift{3.496436in}{2.340634in}%
\pgfsys@useobject{currentmarker}{}%
\end{pgfscope}%
\begin{pgfscope}%
\pgfsys@transformshift{3.514473in}{2.126213in}%
\pgfsys@useobject{currentmarker}{}%
\end{pgfscope}%
\begin{pgfscope}%
\pgfsys@transformshift{3.532509in}{1.911792in}%
\pgfsys@useobject{currentmarker}{}%
\end{pgfscope}%
\begin{pgfscope}%
\pgfsys@transformshift{3.550545in}{1.868908in}%
\pgfsys@useobject{currentmarker}{}%
\end{pgfscope}%
\begin{pgfscope}%
\pgfsys@transformshift{3.568582in}{1.911792in}%
\pgfsys@useobject{currentmarker}{}%
\end{pgfscope}%
\begin{pgfscope}%
\pgfsys@transformshift{3.586618in}{2.169097in}%
\pgfsys@useobject{currentmarker}{}%
\end{pgfscope}%
\begin{pgfscope}%
\pgfsys@transformshift{3.604655in}{2.383518in}%
\pgfsys@useobject{currentmarker}{}%
\end{pgfscope}%
\begin{pgfscope}%
\pgfsys@transformshift{3.622691in}{2.640823in}%
\pgfsys@useobject{currentmarker}{}%
\end{pgfscope}%
\begin{pgfscope}%
\pgfsys@transformshift{3.640727in}{2.812359in}%
\pgfsys@useobject{currentmarker}{}%
\end{pgfscope}%
\begin{pgfscope}%
\pgfsys@transformshift{3.658764in}{2.941012in}%
\pgfsys@useobject{currentmarker}{}%
\end{pgfscope}%
\begin{pgfscope}%
\pgfsys@transformshift{3.676800in}{2.983896in}%
\pgfsys@useobject{currentmarker}{}%
\end{pgfscope}%
\begin{pgfscope}%
\pgfsys@transformshift{3.694836in}{2.983896in}%
\pgfsys@useobject{currentmarker}{}%
\end{pgfscope}%
\begin{pgfscope}%
\pgfsys@transformshift{3.712873in}{2.855244in}%
\pgfsys@useobject{currentmarker}{}%
\end{pgfscope}%
\begin{pgfscope}%
\pgfsys@transformshift{3.730909in}{2.683707in}%
\pgfsys@useobject{currentmarker}{}%
\end{pgfscope}%
\begin{pgfscope}%
\pgfsys@transformshift{3.748945in}{2.469286in}%
\pgfsys@useobject{currentmarker}{}%
\end{pgfscope}%
\begin{pgfscope}%
\pgfsys@transformshift{3.766982in}{2.297750in}%
\pgfsys@useobject{currentmarker}{}%
\end{pgfscope}%
\begin{pgfscope}%
\pgfsys@transformshift{3.785018in}{2.126213in}%
\pgfsys@useobject{currentmarker}{}%
\end{pgfscope}%
\begin{pgfscope}%
\pgfsys@transformshift{3.803055in}{2.040445in}%
\pgfsys@useobject{currentmarker}{}%
\end{pgfscope}%
\begin{pgfscope}%
\pgfsys@transformshift{3.821091in}{2.040445in}%
\pgfsys@useobject{currentmarker}{}%
\end{pgfscope}%
\begin{pgfscope}%
\pgfsys@transformshift{3.839127in}{2.126213in}%
\pgfsys@useobject{currentmarker}{}%
\end{pgfscope}%
\begin{pgfscope}%
\pgfsys@transformshift{3.857164in}{2.297750in}%
\pgfsys@useobject{currentmarker}{}%
\end{pgfscope}%
\begin{pgfscope}%
\pgfsys@transformshift{3.875200in}{2.512170in}%
\pgfsys@useobject{currentmarker}{}%
\end{pgfscope}%
\begin{pgfscope}%
\pgfsys@transformshift{3.893236in}{2.683707in}%
\pgfsys@useobject{currentmarker}{}%
\end{pgfscope}%
\begin{pgfscope}%
\pgfsys@transformshift{3.911273in}{2.855244in}%
\pgfsys@useobject{currentmarker}{}%
\end{pgfscope}%
\begin{pgfscope}%
\pgfsys@transformshift{3.929309in}{2.941012in}%
\pgfsys@useobject{currentmarker}{}%
\end{pgfscope}%
\begin{pgfscope}%
\pgfsys@transformshift{3.947345in}{2.983896in}%
\pgfsys@useobject{currentmarker}{}%
\end{pgfscope}%
\begin{pgfscope}%
\pgfsys@transformshift{3.965382in}{2.898128in}%
\pgfsys@useobject{currentmarker}{}%
\end{pgfscope}%
\begin{pgfscope}%
\pgfsys@transformshift{3.983418in}{2.769475in}%
\pgfsys@useobject{currentmarker}{}%
\end{pgfscope}%
\begin{pgfscope}%
\pgfsys@transformshift{4.001455in}{2.597939in}%
\pgfsys@useobject{currentmarker}{}%
\end{pgfscope}%
\begin{pgfscope}%
\pgfsys@transformshift{4.019491in}{2.426402in}%
\pgfsys@useobject{currentmarker}{}%
\end{pgfscope}%
\begin{pgfscope}%
\pgfsys@transformshift{4.037527in}{2.211981in}%
\pgfsys@useobject{currentmarker}{}%
\end{pgfscope}%
\begin{pgfscope}%
\pgfsys@transformshift{4.055564in}{2.083329in}%
\pgfsys@useobject{currentmarker}{}%
\end{pgfscope}%
\begin{pgfscope}%
\pgfsys@transformshift{4.073600in}{2.040445in}%
\pgfsys@useobject{currentmarker}{}%
\end{pgfscope}%
\begin{pgfscope}%
\pgfsys@transformshift{4.091636in}{2.083329in}%
\pgfsys@useobject{currentmarker}{}%
\end{pgfscope}%
\begin{pgfscope}%
\pgfsys@transformshift{4.109673in}{2.211981in}%
\pgfsys@useobject{currentmarker}{}%
\end{pgfscope}%
\begin{pgfscope}%
\pgfsys@transformshift{4.127709in}{2.383518in}%
\pgfsys@useobject{currentmarker}{}%
\end{pgfscope}%
\begin{pgfscope}%
\pgfsys@transformshift{4.145745in}{2.555055in}%
\pgfsys@useobject{currentmarker}{}%
\end{pgfscope}%
\begin{pgfscope}%
\pgfsys@transformshift{4.163782in}{2.726591in}%
\pgfsys@useobject{currentmarker}{}%
\end{pgfscope}%
\begin{pgfscope}%
\pgfsys@transformshift{4.181818in}{2.855244in}%
\pgfsys@useobject{currentmarker}{}%
\end{pgfscope}%
\begin{pgfscope}%
\pgfsys@transformshift{4.199855in}{2.941012in}%
\pgfsys@useobject{currentmarker}{}%
\end{pgfscope}%
\begin{pgfscope}%
\pgfsys@transformshift{4.217891in}{2.941012in}%
\pgfsys@useobject{currentmarker}{}%
\end{pgfscope}%
\begin{pgfscope}%
\pgfsys@transformshift{4.235927in}{2.855244in}%
\pgfsys@useobject{currentmarker}{}%
\end{pgfscope}%
\begin{pgfscope}%
\pgfsys@transformshift{4.253964in}{2.683707in}%
\pgfsys@useobject{currentmarker}{}%
\end{pgfscope}%
\begin{pgfscope}%
\pgfsys@transformshift{4.272000in}{2.512170in}%
\pgfsys@useobject{currentmarker}{}%
\end{pgfscope}%
\begin{pgfscope}%
\pgfsys@transformshift{4.290036in}{2.340634in}%
\pgfsys@useobject{currentmarker}{}%
\end{pgfscope}%
\begin{pgfscope}%
\pgfsys@transformshift{4.308073in}{2.211981in}%
\pgfsys@useobject{currentmarker}{}%
\end{pgfscope}%
\begin{pgfscope}%
\pgfsys@transformshift{4.326109in}{2.083329in}%
\pgfsys@useobject{currentmarker}{}%
\end{pgfscope}%
\begin{pgfscope}%
\pgfsys@transformshift{4.344145in}{2.083329in}%
\pgfsys@useobject{currentmarker}{}%
\end{pgfscope}%
\begin{pgfscope}%
\pgfsys@transformshift{4.362182in}{2.169097in}%
\pgfsys@useobject{currentmarker}{}%
\end{pgfscope}%
\begin{pgfscope}%
\pgfsys@transformshift{4.380218in}{2.297750in}%
\pgfsys@useobject{currentmarker}{}%
\end{pgfscope}%
\begin{pgfscope}%
\pgfsys@transformshift{4.398255in}{2.469286in}%
\pgfsys@useobject{currentmarker}{}%
\end{pgfscope}%
\begin{pgfscope}%
\pgfsys@transformshift{4.416291in}{2.640823in}%
\pgfsys@useobject{currentmarker}{}%
\end{pgfscope}%
\begin{pgfscope}%
\pgfsys@transformshift{4.434327in}{2.769475in}%
\pgfsys@useobject{currentmarker}{}%
\end{pgfscope}%
\begin{pgfscope}%
\pgfsys@transformshift{4.452364in}{2.855244in}%
\pgfsys@useobject{currentmarker}{}%
\end{pgfscope}%
\begin{pgfscope}%
\pgfsys@transformshift{4.470400in}{2.898128in}%
\pgfsys@useobject{currentmarker}{}%
\end{pgfscope}%
\begin{pgfscope}%
\pgfsys@transformshift{4.488436in}{2.855244in}%
\pgfsys@useobject{currentmarker}{}%
\end{pgfscope}%
\begin{pgfscope}%
\pgfsys@transformshift{4.506473in}{2.769475in}%
\pgfsys@useobject{currentmarker}{}%
\end{pgfscope}%
\begin{pgfscope}%
\pgfsys@transformshift{4.524509in}{2.640823in}%
\pgfsys@useobject{currentmarker}{}%
\end{pgfscope}%
\begin{pgfscope}%
\pgfsys@transformshift{4.542545in}{2.469286in}%
\pgfsys@useobject{currentmarker}{}%
\end{pgfscope}%
\begin{pgfscope}%
\pgfsys@transformshift{4.560582in}{2.297750in}%
\pgfsys@useobject{currentmarker}{}%
\end{pgfscope}%
\begin{pgfscope}%
\pgfsys@transformshift{4.578618in}{2.169097in}%
\pgfsys@useobject{currentmarker}{}%
\end{pgfscope}%
\begin{pgfscope}%
\pgfsys@transformshift{4.596655in}{2.126213in}%
\pgfsys@useobject{currentmarker}{}%
\end{pgfscope}%
\begin{pgfscope}%
\pgfsys@transformshift{4.614691in}{2.126213in}%
\pgfsys@useobject{currentmarker}{}%
\end{pgfscope}%
\begin{pgfscope}%
\pgfsys@transformshift{4.632727in}{2.211981in}%
\pgfsys@useobject{currentmarker}{}%
\end{pgfscope}%
\begin{pgfscope}%
\pgfsys@transformshift{4.650764in}{2.340634in}%
\pgfsys@useobject{currentmarker}{}%
\end{pgfscope}%
\begin{pgfscope}%
\pgfsys@transformshift{4.668800in}{2.512170in}%
\pgfsys@useobject{currentmarker}{}%
\end{pgfscope}%
\begin{pgfscope}%
\pgfsys@transformshift{4.686836in}{2.683707in}%
\pgfsys@useobject{currentmarker}{}%
\end{pgfscope}%
\begin{pgfscope}%
\pgfsys@transformshift{4.704873in}{2.812359in}%
\pgfsys@useobject{currentmarker}{}%
\end{pgfscope}%
\begin{pgfscope}%
\pgfsys@transformshift{4.722909in}{2.855244in}%
\pgfsys@useobject{currentmarker}{}%
\end{pgfscope}%
\begin{pgfscope}%
\pgfsys@transformshift{4.740945in}{2.898128in}%
\pgfsys@useobject{currentmarker}{}%
\end{pgfscope}%
\begin{pgfscope}%
\pgfsys@transformshift{4.758982in}{2.812359in}%
\pgfsys@useobject{currentmarker}{}%
\end{pgfscope}%
\begin{pgfscope}%
\pgfsys@transformshift{4.777018in}{2.683707in}%
\pgfsys@useobject{currentmarker}{}%
\end{pgfscope}%
\begin{pgfscope}%
\pgfsys@transformshift{4.795055in}{2.555055in}%
\pgfsys@useobject{currentmarker}{}%
\end{pgfscope}%
\begin{pgfscope}%
\pgfsys@transformshift{4.813091in}{2.383518in}%
\pgfsys@useobject{currentmarker}{}%
\end{pgfscope}%
\begin{pgfscope}%
\pgfsys@transformshift{4.831127in}{2.254865in}%
\pgfsys@useobject{currentmarker}{}%
\end{pgfscope}%
\begin{pgfscope}%
\pgfsys@transformshift{4.849164in}{2.169097in}%
\pgfsys@useobject{currentmarker}{}%
\end{pgfscope}%
\begin{pgfscope}%
\pgfsys@transformshift{4.867200in}{2.169097in}%
\pgfsys@useobject{currentmarker}{}%
\end{pgfscope}%
\begin{pgfscope}%
\pgfsys@transformshift{4.885236in}{2.169097in}%
\pgfsys@useobject{currentmarker}{}%
\end{pgfscope}%
\begin{pgfscope}%
\pgfsys@transformshift{4.903273in}{2.297750in}%
\pgfsys@useobject{currentmarker}{}%
\end{pgfscope}%
\begin{pgfscope}%
\pgfsys@transformshift{4.921309in}{2.426402in}%
\pgfsys@useobject{currentmarker}{}%
\end{pgfscope}%
\begin{pgfscope}%
\pgfsys@transformshift{4.939345in}{2.597939in}%
\pgfsys@useobject{currentmarker}{}%
\end{pgfscope}%
\begin{pgfscope}%
\pgfsys@transformshift{4.957382in}{2.726591in}%
\pgfsys@useobject{currentmarker}{}%
\end{pgfscope}%
\begin{pgfscope}%
\pgfsys@transformshift{4.975418in}{2.812359in}%
\pgfsys@useobject{currentmarker}{}%
\end{pgfscope}%
\begin{pgfscope}%
\pgfsys@transformshift{4.993455in}{2.855244in}%
\pgfsys@useobject{currentmarker}{}%
\end{pgfscope}%
\begin{pgfscope}%
\pgfsys@transformshift{5.011491in}{2.855244in}%
\pgfsys@useobject{currentmarker}{}%
\end{pgfscope}%
\begin{pgfscope}%
\pgfsys@transformshift{5.029527in}{2.769475in}%
\pgfsys@useobject{currentmarker}{}%
\end{pgfscope}%
\begin{pgfscope}%
\pgfsys@transformshift{5.047564in}{2.640823in}%
\pgfsys@useobject{currentmarker}{}%
\end{pgfscope}%
\begin{pgfscope}%
\pgfsys@transformshift{5.065600in}{2.512170in}%
\pgfsys@useobject{currentmarker}{}%
\end{pgfscope}%
\begin{pgfscope}%
\pgfsys@transformshift{5.083636in}{2.340634in}%
\pgfsys@useobject{currentmarker}{}%
\end{pgfscope}%
\begin{pgfscope}%
\pgfsys@transformshift{5.101673in}{2.254865in}%
\pgfsys@useobject{currentmarker}{}%
\end{pgfscope}%
\begin{pgfscope}%
\pgfsys@transformshift{5.119709in}{2.169097in}%
\pgfsys@useobject{currentmarker}{}%
\end{pgfscope}%
\begin{pgfscope}%
\pgfsys@transformshift{5.137745in}{2.169097in}%
\pgfsys@useobject{currentmarker}{}%
\end{pgfscope}%
\begin{pgfscope}%
\pgfsys@transformshift{5.155782in}{2.254865in}%
\pgfsys@useobject{currentmarker}{}%
\end{pgfscope}%
\begin{pgfscope}%
\pgfsys@transformshift{5.173818in}{2.340634in}%
\pgfsys@useobject{currentmarker}{}%
\end{pgfscope}%
\begin{pgfscope}%
\pgfsys@transformshift{5.191855in}{2.469286in}%
\pgfsys@useobject{currentmarker}{}%
\end{pgfscope}%
\begin{pgfscope}%
\pgfsys@transformshift{5.209891in}{2.640823in}%
\pgfsys@useobject{currentmarker}{}%
\end{pgfscope}%
\begin{pgfscope}%
\pgfsys@transformshift{5.227927in}{2.726591in}%
\pgfsys@useobject{currentmarker}{}%
\end{pgfscope}%
\begin{pgfscope}%
\pgfsys@transformshift{5.245964in}{2.812359in}%
\pgfsys@useobject{currentmarker}{}%
\end{pgfscope}%
\begin{pgfscope}%
\pgfsys@transformshift{5.264000in}{2.855244in}%
\pgfsys@useobject{currentmarker}{}%
\end{pgfscope}%
\begin{pgfscope}%
\pgfsys@transformshift{5.282036in}{2.812359in}%
\pgfsys@useobject{currentmarker}{}%
\end{pgfscope}%
\begin{pgfscope}%
\pgfsys@transformshift{5.300073in}{2.726591in}%
\pgfsys@useobject{currentmarker}{}%
\end{pgfscope}%
\begin{pgfscope}%
\pgfsys@transformshift{5.318109in}{2.597939in}%
\pgfsys@useobject{currentmarker}{}%
\end{pgfscope}%
\begin{pgfscope}%
\pgfsys@transformshift{5.336145in}{2.426402in}%
\pgfsys@useobject{currentmarker}{}%
\end{pgfscope}%
\begin{pgfscope}%
\pgfsys@transformshift{5.354182in}{2.340634in}%
\pgfsys@useobject{currentmarker}{}%
\end{pgfscope}%
\begin{pgfscope}%
\pgfsys@transformshift{5.372218in}{2.211981in}%
\pgfsys@useobject{currentmarker}{}%
\end{pgfscope}%
\begin{pgfscope}%
\pgfsys@transformshift{5.390255in}{2.211981in}%
\pgfsys@useobject{currentmarker}{}%
\end{pgfscope}%
\begin{pgfscope}%
\pgfsys@transformshift{5.408291in}{2.211981in}%
\pgfsys@useobject{currentmarker}{}%
\end{pgfscope}%
\begin{pgfscope}%
\pgfsys@transformshift{5.426327in}{2.297750in}%
\pgfsys@useobject{currentmarker}{}%
\end{pgfscope}%
\begin{pgfscope}%
\pgfsys@transformshift{5.444364in}{2.426402in}%
\pgfsys@useobject{currentmarker}{}%
\end{pgfscope}%
\begin{pgfscope}%
\pgfsys@transformshift{5.462400in}{2.555055in}%
\pgfsys@useobject{currentmarker}{}%
\end{pgfscope}%
\begin{pgfscope}%
\pgfsys@transformshift{5.480436in}{2.683707in}%
\pgfsys@useobject{currentmarker}{}%
\end{pgfscope}%
\begin{pgfscope}%
\pgfsys@transformshift{5.498473in}{2.769475in}%
\pgfsys@useobject{currentmarker}{}%
\end{pgfscope}%
\begin{pgfscope}%
\pgfsys@transformshift{5.516509in}{2.812359in}%
\pgfsys@useobject{currentmarker}{}%
\end{pgfscope}%
\begin{pgfscope}%
\pgfsys@transformshift{5.534545in}{2.812359in}%
\pgfsys@useobject{currentmarker}{}%
\end{pgfscope}%
\end{pgfscope}%
\begin{pgfscope}%
\pgfsetrectcap%
\pgfsetmiterjoin%
\pgfsetlinewidth{0.803000pt}%
\definecolor{currentstroke}{rgb}{0.000000,0.000000,0.000000}%
\pgfsetstrokecolor{currentstroke}%
\pgfsetdash{}{0pt}%
\pgfpathmoveto{\pgfqpoint{0.800000in}{0.528000in}}%
\pgfpathlineto{\pgfqpoint{0.800000in}{4.224000in}}%
\pgfusepath{stroke}%
\end{pgfscope}%
\begin{pgfscope}%
\pgfsetrectcap%
\pgfsetmiterjoin%
\pgfsetlinewidth{0.803000pt}%
\definecolor{currentstroke}{rgb}{0.000000,0.000000,0.000000}%
\pgfsetstrokecolor{currentstroke}%
\pgfsetdash{}{0pt}%
\pgfpathmoveto{\pgfqpoint{5.760000in}{0.528000in}}%
\pgfpathlineto{\pgfqpoint{5.760000in}{4.224000in}}%
\pgfusepath{stroke}%
\end{pgfscope}%
\begin{pgfscope}%
\pgfsetrectcap%
\pgfsetmiterjoin%
\pgfsetlinewidth{0.803000pt}%
\definecolor{currentstroke}{rgb}{0.000000,0.000000,0.000000}%
\pgfsetstrokecolor{currentstroke}%
\pgfsetdash{}{0pt}%
\pgfpathmoveto{\pgfqpoint{0.800000in}{0.528000in}}%
\pgfpathlineto{\pgfqpoint{5.760000in}{0.528000in}}%
\pgfusepath{stroke}%
\end{pgfscope}%
\begin{pgfscope}%
\pgfsetrectcap%
\pgfsetmiterjoin%
\pgfsetlinewidth{0.803000pt}%
\definecolor{currentstroke}{rgb}{0.000000,0.000000,0.000000}%
\pgfsetstrokecolor{currentstroke}%
\pgfsetdash{}{0pt}%
\pgfpathmoveto{\pgfqpoint{0.800000in}{4.224000in}}%
\pgfpathlineto{\pgfqpoint{5.760000in}{4.224000in}}%
\pgfusepath{stroke}%
\end{pgfscope}%
\begin{pgfscope}%
\definecolor{textcolor}{rgb}{0.000000,0.000000,0.000000}%
\pgfsetstrokecolor{textcolor}%
\pgfsetfillcolor{textcolor}%
\pgftext[x=3.280000in,y=4.307333in,,base]{\color{textcolor}\rmfamily\fontsize{12.000000}{14.400000}\selectfont Damped Oscillator}%
\end{pgfscope}%
\begin{pgfscope}%
\pgfsetbuttcap%
\pgfsetmiterjoin%
\definecolor{currentfill}{rgb}{1.000000,1.000000,1.000000}%
\pgfsetfillcolor{currentfill}%
\pgfsetfillopacity{0.800000}%
\pgfsetlinewidth{1.003750pt}%
\definecolor{currentstroke}{rgb}{0.800000,0.800000,0.800000}%
\pgfsetstrokecolor{currentstroke}%
\pgfsetstrokeopacity{0.800000}%
\pgfsetdash{}{0pt}%
\pgfpathmoveto{\pgfqpoint{4.448078in}{3.725543in}}%
\pgfpathlineto{\pgfqpoint{5.662778in}{3.725543in}}%
\pgfpathquadraticcurveto{\pgfqpoint{5.690556in}{3.725543in}}{\pgfqpoint{5.690556in}{3.753321in}}%
\pgfpathlineto{\pgfqpoint{5.690556in}{4.126778in}}%
\pgfpathquadraticcurveto{\pgfqpoint{5.690556in}{4.154556in}}{\pgfqpoint{5.662778in}{4.154556in}}%
\pgfpathlineto{\pgfqpoint{4.448078in}{4.154556in}}%
\pgfpathquadraticcurveto{\pgfqpoint{4.420300in}{4.154556in}}{\pgfqpoint{4.420300in}{4.126778in}}%
\pgfpathlineto{\pgfqpoint{4.420300in}{3.753321in}}%
\pgfpathquadraticcurveto{\pgfqpoint{4.420300in}{3.725543in}}{\pgfqpoint{4.448078in}{3.725543in}}%
\pgfpathclose%
\pgfusepath{stroke,fill}%
\end{pgfscope}%
\begin{pgfscope}%
\pgfsetrectcap%
\pgfsetroundjoin%
\pgfsetlinewidth{0.501875pt}%
\definecolor{currentstroke}{rgb}{0.000000,0.500000,0.000000}%
\pgfsetstrokecolor{currentstroke}%
\pgfsetdash{}{0pt}%
\pgfpathmoveto{\pgfqpoint{4.475855in}{4.050389in}}%
\pgfpathlineto{\pgfqpoint{4.753633in}{4.050389in}}%
\pgfusepath{stroke}%
\end{pgfscope}%
\begin{pgfscope}%
\definecolor{textcolor}{rgb}{0.000000,0.000000,0.000000}%
\pgfsetstrokecolor{textcolor}%
\pgfsetfillcolor{textcolor}%
\pgftext[x=4.864744in,y=4.001778in,left,base]{\color{textcolor}\rmfamily\fontsize{10.000000}{12.000000}\selectfont Initial Guess}%
\end{pgfscope}%
\begin{pgfscope}%
\pgfsetbuttcap%
\pgfsetroundjoin%
\pgfsetlinewidth{0.501875pt}%
\definecolor{currentstroke}{rgb}{0.000000,0.000000,1.000000}%
\pgfsetstrokecolor{currentstroke}%
\pgfsetdash{}{0pt}%
\pgfpathmoveto{\pgfqpoint{4.614744in}{3.787272in}}%
\pgfpathlineto{\pgfqpoint{4.614744in}{3.926161in}}%
\pgfusepath{stroke}%
\end{pgfscope}%
\begin{pgfscope}%
\pgfsetbuttcap%
\pgfsetroundjoin%
\definecolor{currentfill}{rgb}{0.000000,0.000000,1.000000}%
\pgfsetfillcolor{currentfill}%
\pgfsetlinewidth{0.501875pt}%
\definecolor{currentstroke}{rgb}{0.000000,0.000000,1.000000}%
\pgfsetstrokecolor{currentstroke}%
\pgfsetdash{}{0pt}%
\pgfsys@defobject{currentmarker}{\pgfqpoint{-0.027778in}{-0.000000in}}{\pgfqpoint{0.027778in}{0.000000in}}{%
\pgfpathmoveto{\pgfqpoint{0.027778in}{-0.000000in}}%
\pgfpathlineto{\pgfqpoint{-0.027778in}{0.000000in}}%
\pgfusepath{stroke,fill}%
}%
\begin{pgfscope}%
\pgfsys@transformshift{4.614744in}{3.787272in}%
\pgfsys@useobject{currentmarker}{}%
\end{pgfscope}%
\end{pgfscope}%
\begin{pgfscope}%
\pgfsetbuttcap%
\pgfsetroundjoin%
\definecolor{currentfill}{rgb}{0.000000,0.000000,1.000000}%
\pgfsetfillcolor{currentfill}%
\pgfsetlinewidth{0.501875pt}%
\definecolor{currentstroke}{rgb}{0.000000,0.000000,1.000000}%
\pgfsetstrokecolor{currentstroke}%
\pgfsetdash{}{0pt}%
\pgfsys@defobject{currentmarker}{\pgfqpoint{-0.027778in}{-0.000000in}}{\pgfqpoint{0.027778in}{0.000000in}}{%
\pgfpathmoveto{\pgfqpoint{0.027778in}{-0.000000in}}%
\pgfpathlineto{\pgfqpoint{-0.027778in}{0.000000in}}%
\pgfusepath{stroke,fill}%
}%
\begin{pgfscope}%
\pgfsys@transformshift{4.614744in}{3.926161in}%
\pgfsys@useobject{currentmarker}{}%
\end{pgfscope}%
\end{pgfscope}%
\begin{pgfscope}%
\pgfsetbuttcap%
\pgfsetroundjoin%
\definecolor{currentfill}{rgb}{0.000000,0.000000,1.000000}%
\pgfsetfillcolor{currentfill}%
\pgfsetlinewidth{0.501875pt}%
\definecolor{currentstroke}{rgb}{0.000000,0.000000,1.000000}%
\pgfsetstrokecolor{currentstroke}%
\pgfsetdash{}{0pt}%
\pgfsys@defobject{currentmarker}{\pgfqpoint{-0.027778in}{-0.000000in}}{\pgfqpoint{0.027778in}{0.000000in}}{%
\pgfpathmoveto{\pgfqpoint{0.027778in}{-0.000000in}}%
\pgfpathlineto{\pgfqpoint{-0.027778in}{0.000000in}}%
\pgfusepath{stroke,fill}%
}%
\begin{pgfscope}%
\pgfsys@transformshift{4.614744in}{3.856716in}%
\pgfsys@useobject{currentmarker}{}%
\end{pgfscope}%
\end{pgfscope}%
\begin{pgfscope}%
\definecolor{textcolor}{rgb}{0.000000,0.000000,0.000000}%
\pgfsetstrokecolor{textcolor}%
\pgfsetfillcolor{textcolor}%
\pgftext[x=4.864744in,y=3.808105in,left,base]{\color{textcolor}\rmfamily\fontsize{10.000000}{12.000000}\selectfont Data}%
\end{pgfscope}%
\end{pgfpicture}%
\makeatother%
\endgroup%
}
           \caption{Initial Guess Curve}
           \label{fig:Initial Guess}
        \end{center}
    \end{figure}

    \begin{table}[H]
        \centering
        \begin{tabular}{c c c c c}
            \hline
            A & B & $\gamma$ & $\omega$ & $\alpha$ \\
            \hline
            0.28 & 0.04 & 0.4 & 30 & 0 \\
            \hline
        \end{tabular}
        \caption{Initial Guess Parameters}
        \label{table:Initial Params}
    \end{table}
    
    \noindent
    As you can see, the curve fits reasonably well to the data points. It has roughly the same 
    frequency, initial amplitude, and decay rate, which means it's a good initial guess for our 
    algorithm to start with. 
    \newline
    \newline
    The algorithm we used to fit the function to the set of data points was the Levenberg-Marquardt 
    Algorithm, implemented in the \texttt{scipy.optimize} module, specifically the function 
    \texttt{curve\_fit}. The implementation of this function [lines 47-49] is slightly complex 
    but, after providing it with the function we defined earlier, the data points we are 
    considering, and out initial guesses for the parameters, it returns an array of parameters 
    that it determines to be the optimal parameters to use in order to approximately fit the 
    curve to the data. We then feed these parameters back to our original function and it gives 
    us values for $y(t)$ that are very close to correct. The "goodness" of these values is 
    discussed later on. For now we can have a look at the plot of this curve [Figure \ref{fig:Best Fit}] 
    and see that it seems to be reasonably accurate. 

    \begin{figure}[H]
        \begin{center}
           \scalebox{.8}{%% Creator: Matplotlib, PGF backend
%%
%% To include the figure in your LaTeX document, write
%%   \input{<filename>.pgf}
%%
%% Make sure the required packages are loaded in your preamble
%%   \usepackage{pgf}
%%
%% Figures using additional raster images can only be included by \input if
%% they are in the same directory as the main LaTeX file. For loading figures
%% from other directories you can use the `import` package
%%   \usepackage{import}
%% and then include the figures with
%%   \import{<path to file>}{<filename>.pgf}
%%
%% Matplotlib used the following preamble
%%
\begingroup%
\makeatletter%
\begin{pgfpicture}%
\pgfpathrectangle{\pgfpointorigin}{\pgfqpoint{6.400000in}{4.800000in}}%
\pgfusepath{use as bounding box, clip}%
\begin{pgfscope}%
\pgfsetbuttcap%
\pgfsetmiterjoin%
\definecolor{currentfill}{rgb}{1.000000,1.000000,1.000000}%
\pgfsetfillcolor{currentfill}%
\pgfsetlinewidth{0.000000pt}%
\definecolor{currentstroke}{rgb}{1.000000,1.000000,1.000000}%
\pgfsetstrokecolor{currentstroke}%
\pgfsetdash{}{0pt}%
\pgfpathmoveto{\pgfqpoint{0.000000in}{0.000000in}}%
\pgfpathlineto{\pgfqpoint{6.400000in}{0.000000in}}%
\pgfpathlineto{\pgfqpoint{6.400000in}{4.800000in}}%
\pgfpathlineto{\pgfqpoint{0.000000in}{4.800000in}}%
\pgfpathclose%
\pgfusepath{fill}%
\end{pgfscope}%
\begin{pgfscope}%
\pgfsetbuttcap%
\pgfsetmiterjoin%
\definecolor{currentfill}{rgb}{1.000000,1.000000,1.000000}%
\pgfsetfillcolor{currentfill}%
\pgfsetlinewidth{0.000000pt}%
\definecolor{currentstroke}{rgb}{0.000000,0.000000,0.000000}%
\pgfsetstrokecolor{currentstroke}%
\pgfsetstrokeopacity{0.000000}%
\pgfsetdash{}{0pt}%
\pgfpathmoveto{\pgfqpoint{0.800000in}{0.528000in}}%
\pgfpathlineto{\pgfqpoint{5.760000in}{0.528000in}}%
\pgfpathlineto{\pgfqpoint{5.760000in}{4.224000in}}%
\pgfpathlineto{\pgfqpoint{0.800000in}{4.224000in}}%
\pgfpathclose%
\pgfusepath{fill}%
\end{pgfscope}%
\begin{pgfscope}%
\pgfsetbuttcap%
\pgfsetroundjoin%
\definecolor{currentfill}{rgb}{0.000000,0.000000,0.000000}%
\pgfsetfillcolor{currentfill}%
\pgfsetlinewidth{0.803000pt}%
\definecolor{currentstroke}{rgb}{0.000000,0.000000,0.000000}%
\pgfsetstrokecolor{currentstroke}%
\pgfsetdash{}{0pt}%
\pgfsys@defobject{currentmarker}{\pgfqpoint{0.000000in}{-0.048611in}}{\pgfqpoint{0.000000in}{0.000000in}}{%
\pgfpathmoveto{\pgfqpoint{0.000000in}{0.000000in}}%
\pgfpathlineto{\pgfqpoint{0.000000in}{-0.048611in}}%
\pgfusepath{stroke,fill}%
}%
\begin{pgfscope}%
\pgfsys@transformshift{1.025455in}{0.528000in}%
\pgfsys@useobject{currentmarker}{}%
\end{pgfscope}%
\end{pgfscope}%
\begin{pgfscope}%
\definecolor{textcolor}{rgb}{0.000000,0.000000,0.000000}%
\pgfsetstrokecolor{textcolor}%
\pgfsetfillcolor{textcolor}%
\pgftext[x=1.025455in,y=0.430778in,,top]{\color{textcolor}\rmfamily\fontsize{10.000000}{12.000000}\selectfont \(\displaystyle 0\)}%
\end{pgfscope}%
\begin{pgfscope}%
\pgfsetbuttcap%
\pgfsetroundjoin%
\definecolor{currentfill}{rgb}{0.000000,0.000000,0.000000}%
\pgfsetfillcolor{currentfill}%
\pgfsetlinewidth{0.803000pt}%
\definecolor{currentstroke}{rgb}{0.000000,0.000000,0.000000}%
\pgfsetstrokecolor{currentstroke}%
\pgfsetdash{}{0pt}%
\pgfsys@defobject{currentmarker}{\pgfqpoint{0.000000in}{-0.048611in}}{\pgfqpoint{0.000000in}{0.000000in}}{%
\pgfpathmoveto{\pgfqpoint{0.000000in}{0.000000in}}%
\pgfpathlineto{\pgfqpoint{0.000000in}{-0.048611in}}%
\pgfusepath{stroke,fill}%
}%
\begin{pgfscope}%
\pgfsys@transformshift{1.927273in}{0.528000in}%
\pgfsys@useobject{currentmarker}{}%
\end{pgfscope}%
\end{pgfscope}%
\begin{pgfscope}%
\definecolor{textcolor}{rgb}{0.000000,0.000000,0.000000}%
\pgfsetstrokecolor{textcolor}%
\pgfsetfillcolor{textcolor}%
\pgftext[x=1.927273in,y=0.430778in,,top]{\color{textcolor}\rmfamily\fontsize{10.000000}{12.000000}\selectfont \(\displaystyle 1\)}%
\end{pgfscope}%
\begin{pgfscope}%
\pgfsetbuttcap%
\pgfsetroundjoin%
\definecolor{currentfill}{rgb}{0.000000,0.000000,0.000000}%
\pgfsetfillcolor{currentfill}%
\pgfsetlinewidth{0.803000pt}%
\definecolor{currentstroke}{rgb}{0.000000,0.000000,0.000000}%
\pgfsetstrokecolor{currentstroke}%
\pgfsetdash{}{0pt}%
\pgfsys@defobject{currentmarker}{\pgfqpoint{0.000000in}{-0.048611in}}{\pgfqpoint{0.000000in}{0.000000in}}{%
\pgfpathmoveto{\pgfqpoint{0.000000in}{0.000000in}}%
\pgfpathlineto{\pgfqpoint{0.000000in}{-0.048611in}}%
\pgfusepath{stroke,fill}%
}%
\begin{pgfscope}%
\pgfsys@transformshift{2.829091in}{0.528000in}%
\pgfsys@useobject{currentmarker}{}%
\end{pgfscope}%
\end{pgfscope}%
\begin{pgfscope}%
\definecolor{textcolor}{rgb}{0.000000,0.000000,0.000000}%
\pgfsetstrokecolor{textcolor}%
\pgfsetfillcolor{textcolor}%
\pgftext[x=2.829091in,y=0.430778in,,top]{\color{textcolor}\rmfamily\fontsize{10.000000}{12.000000}\selectfont \(\displaystyle 2\)}%
\end{pgfscope}%
\begin{pgfscope}%
\pgfsetbuttcap%
\pgfsetroundjoin%
\definecolor{currentfill}{rgb}{0.000000,0.000000,0.000000}%
\pgfsetfillcolor{currentfill}%
\pgfsetlinewidth{0.803000pt}%
\definecolor{currentstroke}{rgb}{0.000000,0.000000,0.000000}%
\pgfsetstrokecolor{currentstroke}%
\pgfsetdash{}{0pt}%
\pgfsys@defobject{currentmarker}{\pgfqpoint{0.000000in}{-0.048611in}}{\pgfqpoint{0.000000in}{0.000000in}}{%
\pgfpathmoveto{\pgfqpoint{0.000000in}{0.000000in}}%
\pgfpathlineto{\pgfqpoint{0.000000in}{-0.048611in}}%
\pgfusepath{stroke,fill}%
}%
\begin{pgfscope}%
\pgfsys@transformshift{3.730909in}{0.528000in}%
\pgfsys@useobject{currentmarker}{}%
\end{pgfscope}%
\end{pgfscope}%
\begin{pgfscope}%
\definecolor{textcolor}{rgb}{0.000000,0.000000,0.000000}%
\pgfsetstrokecolor{textcolor}%
\pgfsetfillcolor{textcolor}%
\pgftext[x=3.730909in,y=0.430778in,,top]{\color{textcolor}\rmfamily\fontsize{10.000000}{12.000000}\selectfont \(\displaystyle 3\)}%
\end{pgfscope}%
\begin{pgfscope}%
\pgfsetbuttcap%
\pgfsetroundjoin%
\definecolor{currentfill}{rgb}{0.000000,0.000000,0.000000}%
\pgfsetfillcolor{currentfill}%
\pgfsetlinewidth{0.803000pt}%
\definecolor{currentstroke}{rgb}{0.000000,0.000000,0.000000}%
\pgfsetstrokecolor{currentstroke}%
\pgfsetdash{}{0pt}%
\pgfsys@defobject{currentmarker}{\pgfqpoint{0.000000in}{-0.048611in}}{\pgfqpoint{0.000000in}{0.000000in}}{%
\pgfpathmoveto{\pgfqpoint{0.000000in}{0.000000in}}%
\pgfpathlineto{\pgfqpoint{0.000000in}{-0.048611in}}%
\pgfusepath{stroke,fill}%
}%
\begin{pgfscope}%
\pgfsys@transformshift{4.632727in}{0.528000in}%
\pgfsys@useobject{currentmarker}{}%
\end{pgfscope}%
\end{pgfscope}%
\begin{pgfscope}%
\definecolor{textcolor}{rgb}{0.000000,0.000000,0.000000}%
\pgfsetstrokecolor{textcolor}%
\pgfsetfillcolor{textcolor}%
\pgftext[x=4.632727in,y=0.430778in,,top]{\color{textcolor}\rmfamily\fontsize{10.000000}{12.000000}\selectfont \(\displaystyle 4\)}%
\end{pgfscope}%
\begin{pgfscope}%
\pgfsetbuttcap%
\pgfsetroundjoin%
\definecolor{currentfill}{rgb}{0.000000,0.000000,0.000000}%
\pgfsetfillcolor{currentfill}%
\pgfsetlinewidth{0.803000pt}%
\definecolor{currentstroke}{rgb}{0.000000,0.000000,0.000000}%
\pgfsetstrokecolor{currentstroke}%
\pgfsetdash{}{0pt}%
\pgfsys@defobject{currentmarker}{\pgfqpoint{0.000000in}{-0.048611in}}{\pgfqpoint{0.000000in}{0.000000in}}{%
\pgfpathmoveto{\pgfqpoint{0.000000in}{0.000000in}}%
\pgfpathlineto{\pgfqpoint{0.000000in}{-0.048611in}}%
\pgfusepath{stroke,fill}%
}%
\begin{pgfscope}%
\pgfsys@transformshift{5.534545in}{0.528000in}%
\pgfsys@useobject{currentmarker}{}%
\end{pgfscope}%
\end{pgfscope}%
\begin{pgfscope}%
\definecolor{textcolor}{rgb}{0.000000,0.000000,0.000000}%
\pgfsetstrokecolor{textcolor}%
\pgfsetfillcolor{textcolor}%
\pgftext[x=5.534545in,y=0.430778in,,top]{\color{textcolor}\rmfamily\fontsize{10.000000}{12.000000}\selectfont \(\displaystyle 5\)}%
\end{pgfscope}%
\begin{pgfscope}%
\pgfsetbuttcap%
\pgfsetroundjoin%
\definecolor{currentfill}{rgb}{0.000000,0.000000,0.000000}%
\pgfsetfillcolor{currentfill}%
\pgfsetlinewidth{0.803000pt}%
\definecolor{currentstroke}{rgb}{0.000000,0.000000,0.000000}%
\pgfsetstrokecolor{currentstroke}%
\pgfsetdash{}{0pt}%
\pgfsys@defobject{currentmarker}{\pgfqpoint{-0.048611in}{0.000000in}}{\pgfqpoint{0.000000in}{0.000000in}}{%
\pgfpathmoveto{\pgfqpoint{0.000000in}{0.000000in}}%
\pgfpathlineto{\pgfqpoint{-0.048611in}{0.000000in}}%
\pgfusepath{stroke,fill}%
}%
\begin{pgfscope}%
\pgfsys@transformshift{0.800000in}{1.047166in}%
\pgfsys@useobject{currentmarker}{}%
\end{pgfscope}%
\end{pgfscope}%
\begin{pgfscope}%
\definecolor{textcolor}{rgb}{0.000000,0.000000,0.000000}%
\pgfsetstrokecolor{textcolor}%
\pgfsetfillcolor{textcolor}%
\pgftext[x=0.455863in,y=0.998941in,left,base]{\color{textcolor}\rmfamily\fontsize{10.000000}{12.000000}\selectfont \(\displaystyle 0.26\)}%
\end{pgfscope}%
\begin{pgfscope}%
\pgfsetbuttcap%
\pgfsetroundjoin%
\definecolor{currentfill}{rgb}{0.000000,0.000000,0.000000}%
\pgfsetfillcolor{currentfill}%
\pgfsetlinewidth{0.803000pt}%
\definecolor{currentstroke}{rgb}{0.000000,0.000000,0.000000}%
\pgfsetstrokecolor{currentstroke}%
\pgfsetdash{}{0pt}%
\pgfsys@defobject{currentmarker}{\pgfqpoint{-0.048611in}{0.000000in}}{\pgfqpoint{0.000000in}{0.000000in}}{%
\pgfpathmoveto{\pgfqpoint{0.000000in}{0.000000in}}%
\pgfpathlineto{\pgfqpoint{-0.048611in}{0.000000in}}%
\pgfusepath{stroke,fill}%
}%
\begin{pgfscope}%
\pgfsys@transformshift{0.800000in}{1.632443in}%
\pgfsys@useobject{currentmarker}{}%
\end{pgfscope}%
\end{pgfscope}%
\begin{pgfscope}%
\definecolor{textcolor}{rgb}{0.000000,0.000000,0.000000}%
\pgfsetstrokecolor{textcolor}%
\pgfsetfillcolor{textcolor}%
\pgftext[x=0.455863in,y=1.584218in,left,base]{\color{textcolor}\rmfamily\fontsize{10.000000}{12.000000}\selectfont \(\displaystyle 0.27\)}%
\end{pgfscope}%
\begin{pgfscope}%
\pgfsetbuttcap%
\pgfsetroundjoin%
\definecolor{currentfill}{rgb}{0.000000,0.000000,0.000000}%
\pgfsetfillcolor{currentfill}%
\pgfsetlinewidth{0.803000pt}%
\definecolor{currentstroke}{rgb}{0.000000,0.000000,0.000000}%
\pgfsetstrokecolor{currentstroke}%
\pgfsetdash{}{0pt}%
\pgfsys@defobject{currentmarker}{\pgfqpoint{-0.048611in}{0.000000in}}{\pgfqpoint{0.000000in}{0.000000in}}{%
\pgfpathmoveto{\pgfqpoint{0.000000in}{0.000000in}}%
\pgfpathlineto{\pgfqpoint{-0.048611in}{0.000000in}}%
\pgfusepath{stroke,fill}%
}%
\begin{pgfscope}%
\pgfsys@transformshift{0.800000in}{2.217720in}%
\pgfsys@useobject{currentmarker}{}%
\end{pgfscope}%
\end{pgfscope}%
\begin{pgfscope}%
\definecolor{textcolor}{rgb}{0.000000,0.000000,0.000000}%
\pgfsetstrokecolor{textcolor}%
\pgfsetfillcolor{textcolor}%
\pgftext[x=0.455863in,y=2.169495in,left,base]{\color{textcolor}\rmfamily\fontsize{10.000000}{12.000000}\selectfont \(\displaystyle 0.28\)}%
\end{pgfscope}%
\begin{pgfscope}%
\pgfsetbuttcap%
\pgfsetroundjoin%
\definecolor{currentfill}{rgb}{0.000000,0.000000,0.000000}%
\pgfsetfillcolor{currentfill}%
\pgfsetlinewidth{0.803000pt}%
\definecolor{currentstroke}{rgb}{0.000000,0.000000,0.000000}%
\pgfsetstrokecolor{currentstroke}%
\pgfsetdash{}{0pt}%
\pgfsys@defobject{currentmarker}{\pgfqpoint{-0.048611in}{0.000000in}}{\pgfqpoint{0.000000in}{0.000000in}}{%
\pgfpathmoveto{\pgfqpoint{0.000000in}{0.000000in}}%
\pgfpathlineto{\pgfqpoint{-0.048611in}{0.000000in}}%
\pgfusepath{stroke,fill}%
}%
\begin{pgfscope}%
\pgfsys@transformshift{0.800000in}{2.802998in}%
\pgfsys@useobject{currentmarker}{}%
\end{pgfscope}%
\end{pgfscope}%
\begin{pgfscope}%
\definecolor{textcolor}{rgb}{0.000000,0.000000,0.000000}%
\pgfsetstrokecolor{textcolor}%
\pgfsetfillcolor{textcolor}%
\pgftext[x=0.455863in,y=2.754772in,left,base]{\color{textcolor}\rmfamily\fontsize{10.000000}{12.000000}\selectfont \(\displaystyle 0.29\)}%
\end{pgfscope}%
\begin{pgfscope}%
\pgfsetbuttcap%
\pgfsetroundjoin%
\definecolor{currentfill}{rgb}{0.000000,0.000000,0.000000}%
\pgfsetfillcolor{currentfill}%
\pgfsetlinewidth{0.803000pt}%
\definecolor{currentstroke}{rgb}{0.000000,0.000000,0.000000}%
\pgfsetstrokecolor{currentstroke}%
\pgfsetdash{}{0pt}%
\pgfsys@defobject{currentmarker}{\pgfqpoint{-0.048611in}{0.000000in}}{\pgfqpoint{0.000000in}{0.000000in}}{%
\pgfpathmoveto{\pgfqpoint{0.000000in}{0.000000in}}%
\pgfpathlineto{\pgfqpoint{-0.048611in}{0.000000in}}%
\pgfusepath{stroke,fill}%
}%
\begin{pgfscope}%
\pgfsys@transformshift{0.800000in}{3.388275in}%
\pgfsys@useobject{currentmarker}{}%
\end{pgfscope}%
\end{pgfscope}%
\begin{pgfscope}%
\definecolor{textcolor}{rgb}{0.000000,0.000000,0.000000}%
\pgfsetstrokecolor{textcolor}%
\pgfsetfillcolor{textcolor}%
\pgftext[x=0.455863in,y=3.340049in,left,base]{\color{textcolor}\rmfamily\fontsize{10.000000}{12.000000}\selectfont \(\displaystyle 0.30\)}%
\end{pgfscope}%
\begin{pgfscope}%
\pgfsetbuttcap%
\pgfsetroundjoin%
\definecolor{currentfill}{rgb}{0.000000,0.000000,0.000000}%
\pgfsetfillcolor{currentfill}%
\pgfsetlinewidth{0.803000pt}%
\definecolor{currentstroke}{rgb}{0.000000,0.000000,0.000000}%
\pgfsetstrokecolor{currentstroke}%
\pgfsetdash{}{0pt}%
\pgfsys@defobject{currentmarker}{\pgfqpoint{-0.048611in}{0.000000in}}{\pgfqpoint{0.000000in}{0.000000in}}{%
\pgfpathmoveto{\pgfqpoint{0.000000in}{0.000000in}}%
\pgfpathlineto{\pgfqpoint{-0.048611in}{0.000000in}}%
\pgfusepath{stroke,fill}%
}%
\begin{pgfscope}%
\pgfsys@transformshift{0.800000in}{3.973552in}%
\pgfsys@useobject{currentmarker}{}%
\end{pgfscope}%
\end{pgfscope}%
\begin{pgfscope}%
\definecolor{textcolor}{rgb}{0.000000,0.000000,0.000000}%
\pgfsetstrokecolor{textcolor}%
\pgfsetfillcolor{textcolor}%
\pgftext[x=0.455863in,y=3.925326in,left,base]{\color{textcolor}\rmfamily\fontsize{10.000000}{12.000000}\selectfont \(\displaystyle 0.31\)}%
\end{pgfscope}%
\begin{pgfscope}%
\pgfpathrectangle{\pgfqpoint{0.800000in}{0.528000in}}{\pgfqpoint{4.960000in}{3.696000in}}%
\pgfusepath{clip}%
\pgfsetbuttcap%
\pgfsetroundjoin%
\pgfsetlinewidth{0.501875pt}%
\definecolor{currentstroke}{rgb}{0.000000,0.000000,1.000000}%
\pgfsetstrokecolor{currentstroke}%
\pgfsetdash{}{0pt}%
\pgfpathmoveto{\pgfqpoint{1.043491in}{3.856496in}}%
\pgfpathlineto{\pgfqpoint{1.043491in}{3.973552in}}%
\pgfusepath{stroke}%
\end{pgfscope}%
\begin{pgfscope}%
\pgfpathrectangle{\pgfqpoint{0.800000in}{0.528000in}}{\pgfqpoint{4.960000in}{3.696000in}}%
\pgfusepath{clip}%
\pgfsetbuttcap%
\pgfsetroundjoin%
\pgfsetlinewidth{0.501875pt}%
\definecolor{currentstroke}{rgb}{0.000000,0.000000,1.000000}%
\pgfsetstrokecolor{currentstroke}%
\pgfsetdash{}{0pt}%
\pgfpathmoveto{\pgfqpoint{1.061527in}{3.680913in}}%
\pgfpathlineto{\pgfqpoint{1.061527in}{3.797969in}}%
\pgfusepath{stroke}%
\end{pgfscope}%
\begin{pgfscope}%
\pgfpathrectangle{\pgfqpoint{0.800000in}{0.528000in}}{\pgfqpoint{4.960000in}{3.696000in}}%
\pgfusepath{clip}%
\pgfsetbuttcap%
\pgfsetroundjoin%
\pgfsetlinewidth{0.501875pt}%
\definecolor{currentstroke}{rgb}{0.000000,0.000000,1.000000}%
\pgfsetstrokecolor{currentstroke}%
\pgfsetdash{}{0pt}%
\pgfpathmoveto{\pgfqpoint{1.079564in}{3.212692in}}%
\pgfpathlineto{\pgfqpoint{1.079564in}{3.329747in}}%
\pgfusepath{stroke}%
\end{pgfscope}%
\begin{pgfscope}%
\pgfpathrectangle{\pgfqpoint{0.800000in}{0.528000in}}{\pgfqpoint{4.960000in}{3.696000in}}%
\pgfusepath{clip}%
\pgfsetbuttcap%
\pgfsetroundjoin%
\pgfsetlinewidth{0.501875pt}%
\definecolor{currentstroke}{rgb}{0.000000,0.000000,1.000000}%
\pgfsetstrokecolor{currentstroke}%
\pgfsetdash{}{0pt}%
\pgfpathmoveto{\pgfqpoint{1.097600in}{2.627414in}}%
\pgfpathlineto{\pgfqpoint{1.097600in}{2.744470in}}%
\pgfusepath{stroke}%
\end{pgfscope}%
\begin{pgfscope}%
\pgfpathrectangle{\pgfqpoint{0.800000in}{0.528000in}}{\pgfqpoint{4.960000in}{3.696000in}}%
\pgfusepath{clip}%
\pgfsetbuttcap%
\pgfsetroundjoin%
\pgfsetlinewidth{0.501875pt}%
\definecolor{currentstroke}{rgb}{0.000000,0.000000,1.000000}%
\pgfsetstrokecolor{currentstroke}%
\pgfsetdash{}{0pt}%
\pgfpathmoveto{\pgfqpoint{1.115636in}{1.983610in}}%
\pgfpathlineto{\pgfqpoint{1.115636in}{2.100665in}}%
\pgfusepath{stroke}%
\end{pgfscope}%
\begin{pgfscope}%
\pgfpathrectangle{\pgfqpoint{0.800000in}{0.528000in}}{\pgfqpoint{4.960000in}{3.696000in}}%
\pgfusepath{clip}%
\pgfsetbuttcap%
\pgfsetroundjoin%
\pgfsetlinewidth{0.501875pt}%
\definecolor{currentstroke}{rgb}{0.000000,0.000000,1.000000}%
\pgfsetstrokecolor{currentstroke}%
\pgfsetdash{}{0pt}%
\pgfpathmoveto{\pgfqpoint{1.133673in}{1.222749in}}%
\pgfpathlineto{\pgfqpoint{1.133673in}{1.339805in}}%
\pgfusepath{stroke}%
\end{pgfscope}%
\begin{pgfscope}%
\pgfpathrectangle{\pgfqpoint{0.800000in}{0.528000in}}{\pgfqpoint{4.960000in}{3.696000in}}%
\pgfusepath{clip}%
\pgfsetbuttcap%
\pgfsetroundjoin%
\pgfsetlinewidth{0.501875pt}%
\definecolor{currentstroke}{rgb}{0.000000,0.000000,1.000000}%
\pgfsetstrokecolor{currentstroke}%
\pgfsetdash{}{0pt}%
\pgfpathmoveto{\pgfqpoint{1.151709in}{0.813055in}}%
\pgfpathlineto{\pgfqpoint{1.151709in}{0.930111in}}%
\pgfusepath{stroke}%
\end{pgfscope}%
\begin{pgfscope}%
\pgfpathrectangle{\pgfqpoint{0.800000in}{0.528000in}}{\pgfqpoint{4.960000in}{3.696000in}}%
\pgfusepath{clip}%
\pgfsetbuttcap%
\pgfsetroundjoin%
\pgfsetlinewidth{0.501875pt}%
\definecolor{currentstroke}{rgb}{0.000000,0.000000,1.000000}%
\pgfsetstrokecolor{currentstroke}%
\pgfsetdash{}{0pt}%
\pgfpathmoveto{\pgfqpoint{1.169745in}{0.696000in}}%
\pgfpathlineto{\pgfqpoint{1.169745in}{0.813055in}}%
\pgfusepath{stroke}%
\end{pgfscope}%
\begin{pgfscope}%
\pgfpathrectangle{\pgfqpoint{0.800000in}{0.528000in}}{\pgfqpoint{4.960000in}{3.696000in}}%
\pgfusepath{clip}%
\pgfsetbuttcap%
\pgfsetroundjoin%
\pgfsetlinewidth{0.501875pt}%
\definecolor{currentstroke}{rgb}{0.000000,0.000000,1.000000}%
\pgfsetstrokecolor{currentstroke}%
\pgfsetdash{}{0pt}%
\pgfpathmoveto{\pgfqpoint{1.187782in}{0.813055in}}%
\pgfpathlineto{\pgfqpoint{1.187782in}{0.930111in}}%
\pgfusepath{stroke}%
\end{pgfscope}%
\begin{pgfscope}%
\pgfpathrectangle{\pgfqpoint{0.800000in}{0.528000in}}{\pgfqpoint{4.960000in}{3.696000in}}%
\pgfusepath{clip}%
\pgfsetbuttcap%
\pgfsetroundjoin%
\pgfsetlinewidth{0.501875pt}%
\definecolor{currentstroke}{rgb}{0.000000,0.000000,1.000000}%
\pgfsetstrokecolor{currentstroke}%
\pgfsetdash{}{0pt}%
\pgfpathmoveto{\pgfqpoint{1.205818in}{1.222749in}}%
\pgfpathlineto{\pgfqpoint{1.205818in}{1.339805in}}%
\pgfusepath{stroke}%
\end{pgfscope}%
\begin{pgfscope}%
\pgfpathrectangle{\pgfqpoint{0.800000in}{0.528000in}}{\pgfqpoint{4.960000in}{3.696000in}}%
\pgfusepath{clip}%
\pgfsetbuttcap%
\pgfsetroundjoin%
\pgfsetlinewidth{0.501875pt}%
\definecolor{currentstroke}{rgb}{0.000000,0.000000,1.000000}%
\pgfsetstrokecolor{currentstroke}%
\pgfsetdash{}{0pt}%
\pgfpathmoveto{\pgfqpoint{1.223855in}{1.983610in}}%
\pgfpathlineto{\pgfqpoint{1.223855in}{2.100665in}}%
\pgfusepath{stroke}%
\end{pgfscope}%
\begin{pgfscope}%
\pgfpathrectangle{\pgfqpoint{0.800000in}{0.528000in}}{\pgfqpoint{4.960000in}{3.696000in}}%
\pgfusepath{clip}%
\pgfsetbuttcap%
\pgfsetroundjoin%
\pgfsetlinewidth{0.501875pt}%
\definecolor{currentstroke}{rgb}{0.000000,0.000000,1.000000}%
\pgfsetstrokecolor{currentstroke}%
\pgfsetdash{}{0pt}%
\pgfpathmoveto{\pgfqpoint{1.241891in}{2.568887in}}%
\pgfpathlineto{\pgfqpoint{1.241891in}{2.685942in}}%
\pgfusepath{stroke}%
\end{pgfscope}%
\begin{pgfscope}%
\pgfpathrectangle{\pgfqpoint{0.800000in}{0.528000in}}{\pgfqpoint{4.960000in}{3.696000in}}%
\pgfusepath{clip}%
\pgfsetbuttcap%
\pgfsetroundjoin%
\pgfsetlinewidth{0.501875pt}%
\definecolor{currentstroke}{rgb}{0.000000,0.000000,1.000000}%
\pgfsetstrokecolor{currentstroke}%
\pgfsetdash{}{0pt}%
\pgfpathmoveto{\pgfqpoint{1.259927in}{3.154164in}}%
\pgfpathlineto{\pgfqpoint{1.259927in}{3.271219in}}%
\pgfusepath{stroke}%
\end{pgfscope}%
\begin{pgfscope}%
\pgfpathrectangle{\pgfqpoint{0.800000in}{0.528000in}}{\pgfqpoint{4.960000in}{3.696000in}}%
\pgfusepath{clip}%
\pgfsetbuttcap%
\pgfsetroundjoin%
\pgfsetlinewidth{0.501875pt}%
\definecolor{currentstroke}{rgb}{0.000000,0.000000,1.000000}%
\pgfsetstrokecolor{currentstroke}%
\pgfsetdash{}{0pt}%
\pgfpathmoveto{\pgfqpoint{1.277964in}{3.563858in}}%
\pgfpathlineto{\pgfqpoint{1.277964in}{3.680913in}}%
\pgfusepath{stroke}%
\end{pgfscope}%
\begin{pgfscope}%
\pgfpathrectangle{\pgfqpoint{0.800000in}{0.528000in}}{\pgfqpoint{4.960000in}{3.696000in}}%
\pgfusepath{clip}%
\pgfsetbuttcap%
\pgfsetroundjoin%
\pgfsetlinewidth{0.501875pt}%
\definecolor{currentstroke}{rgb}{0.000000,0.000000,1.000000}%
\pgfsetstrokecolor{currentstroke}%
\pgfsetdash{}{0pt}%
\pgfpathmoveto{\pgfqpoint{1.296000in}{3.739441in}}%
\pgfpathlineto{\pgfqpoint{1.296000in}{3.856496in}}%
\pgfusepath{stroke}%
\end{pgfscope}%
\begin{pgfscope}%
\pgfpathrectangle{\pgfqpoint{0.800000in}{0.528000in}}{\pgfqpoint{4.960000in}{3.696000in}}%
\pgfusepath{clip}%
\pgfsetbuttcap%
\pgfsetroundjoin%
\pgfsetlinewidth{0.501875pt}%
\definecolor{currentstroke}{rgb}{0.000000,0.000000,1.000000}%
\pgfsetstrokecolor{currentstroke}%
\pgfsetdash{}{0pt}%
\pgfpathmoveto{\pgfqpoint{1.314036in}{3.739441in}}%
\pgfpathlineto{\pgfqpoint{1.314036in}{3.856496in}}%
\pgfusepath{stroke}%
\end{pgfscope}%
\begin{pgfscope}%
\pgfpathrectangle{\pgfqpoint{0.800000in}{0.528000in}}{\pgfqpoint{4.960000in}{3.696000in}}%
\pgfusepath{clip}%
\pgfsetbuttcap%
\pgfsetroundjoin%
\pgfsetlinewidth{0.501875pt}%
\definecolor{currentstroke}{rgb}{0.000000,0.000000,1.000000}%
\pgfsetstrokecolor{currentstroke}%
\pgfsetdash{}{0pt}%
\pgfpathmoveto{\pgfqpoint{1.332073in}{3.446802in}}%
\pgfpathlineto{\pgfqpoint{1.332073in}{3.563858in}}%
\pgfusepath{stroke}%
\end{pgfscope}%
\begin{pgfscope}%
\pgfpathrectangle{\pgfqpoint{0.800000in}{0.528000in}}{\pgfqpoint{4.960000in}{3.696000in}}%
\pgfusepath{clip}%
\pgfsetbuttcap%
\pgfsetroundjoin%
\pgfsetlinewidth{0.501875pt}%
\definecolor{currentstroke}{rgb}{0.000000,0.000000,1.000000}%
\pgfsetstrokecolor{currentstroke}%
\pgfsetdash{}{0pt}%
\pgfpathmoveto{\pgfqpoint{1.350109in}{2.978581in}}%
\pgfpathlineto{\pgfqpoint{1.350109in}{3.095636in}}%
\pgfusepath{stroke}%
\end{pgfscope}%
\begin{pgfscope}%
\pgfpathrectangle{\pgfqpoint{0.800000in}{0.528000in}}{\pgfqpoint{4.960000in}{3.696000in}}%
\pgfusepath{clip}%
\pgfsetbuttcap%
\pgfsetroundjoin%
\pgfsetlinewidth{0.501875pt}%
\definecolor{currentstroke}{rgb}{0.000000,0.000000,1.000000}%
\pgfsetstrokecolor{currentstroke}%
\pgfsetdash{}{0pt}%
\pgfpathmoveto{\pgfqpoint{1.368145in}{2.393304in}}%
\pgfpathlineto{\pgfqpoint{1.368145in}{2.510359in}}%
\pgfusepath{stroke}%
\end{pgfscope}%
\begin{pgfscope}%
\pgfpathrectangle{\pgfqpoint{0.800000in}{0.528000in}}{\pgfqpoint{4.960000in}{3.696000in}}%
\pgfusepath{clip}%
\pgfsetbuttcap%
\pgfsetroundjoin%
\pgfsetlinewidth{0.501875pt}%
\definecolor{currentstroke}{rgb}{0.000000,0.000000,1.000000}%
\pgfsetstrokecolor{currentstroke}%
\pgfsetdash{}{0pt}%
\pgfpathmoveto{\pgfqpoint{1.386182in}{1.866554in}}%
\pgfpathlineto{\pgfqpoint{1.386182in}{1.983610in}}%
\pgfusepath{stroke}%
\end{pgfscope}%
\begin{pgfscope}%
\pgfpathrectangle{\pgfqpoint{0.800000in}{0.528000in}}{\pgfqpoint{4.960000in}{3.696000in}}%
\pgfusepath{clip}%
\pgfsetbuttcap%
\pgfsetroundjoin%
\pgfsetlinewidth{0.501875pt}%
\definecolor{currentstroke}{rgb}{0.000000,0.000000,1.000000}%
\pgfsetstrokecolor{currentstroke}%
\pgfsetdash{}{0pt}%
\pgfpathmoveto{\pgfqpoint{1.404218in}{1.164222in}}%
\pgfpathlineto{\pgfqpoint{1.404218in}{1.281277in}}%
\pgfusepath{stroke}%
\end{pgfscope}%
\begin{pgfscope}%
\pgfpathrectangle{\pgfqpoint{0.800000in}{0.528000in}}{\pgfqpoint{4.960000in}{3.696000in}}%
\pgfusepath{clip}%
\pgfsetbuttcap%
\pgfsetroundjoin%
\pgfsetlinewidth{0.501875pt}%
\definecolor{currentstroke}{rgb}{0.000000,0.000000,1.000000}%
\pgfsetstrokecolor{currentstroke}%
\pgfsetdash{}{0pt}%
\pgfpathmoveto{\pgfqpoint{1.422255in}{0.871583in}}%
\pgfpathlineto{\pgfqpoint{1.422255in}{0.988639in}}%
\pgfusepath{stroke}%
\end{pgfscope}%
\begin{pgfscope}%
\pgfpathrectangle{\pgfqpoint{0.800000in}{0.528000in}}{\pgfqpoint{4.960000in}{3.696000in}}%
\pgfusepath{clip}%
\pgfsetbuttcap%
\pgfsetroundjoin%
\pgfsetlinewidth{0.501875pt}%
\definecolor{currentstroke}{rgb}{0.000000,0.000000,1.000000}%
\pgfsetstrokecolor{currentstroke}%
\pgfsetdash{}{0pt}%
\pgfpathmoveto{\pgfqpoint{1.440291in}{0.871583in}}%
\pgfpathlineto{\pgfqpoint{1.440291in}{0.988639in}}%
\pgfusepath{stroke}%
\end{pgfscope}%
\begin{pgfscope}%
\pgfpathrectangle{\pgfqpoint{0.800000in}{0.528000in}}{\pgfqpoint{4.960000in}{3.696000in}}%
\pgfusepath{clip}%
\pgfsetbuttcap%
\pgfsetroundjoin%
\pgfsetlinewidth{0.501875pt}%
\definecolor{currentstroke}{rgb}{0.000000,0.000000,1.000000}%
\pgfsetstrokecolor{currentstroke}%
\pgfsetdash{}{0pt}%
\pgfpathmoveto{\pgfqpoint{1.458327in}{1.047166in}}%
\pgfpathlineto{\pgfqpoint{1.458327in}{1.164222in}}%
\pgfusepath{stroke}%
\end{pgfscope}%
\begin{pgfscope}%
\pgfpathrectangle{\pgfqpoint{0.800000in}{0.528000in}}{\pgfqpoint{4.960000in}{3.696000in}}%
\pgfusepath{clip}%
\pgfsetbuttcap%
\pgfsetroundjoin%
\pgfsetlinewidth{0.501875pt}%
\definecolor{currentstroke}{rgb}{0.000000,0.000000,1.000000}%
\pgfsetstrokecolor{currentstroke}%
\pgfsetdash{}{0pt}%
\pgfpathmoveto{\pgfqpoint{1.476364in}{1.456860in}}%
\pgfpathlineto{\pgfqpoint{1.476364in}{1.573916in}}%
\pgfusepath{stroke}%
\end{pgfscope}%
\begin{pgfscope}%
\pgfpathrectangle{\pgfqpoint{0.800000in}{0.528000in}}{\pgfqpoint{4.960000in}{3.696000in}}%
\pgfusepath{clip}%
\pgfsetbuttcap%
\pgfsetroundjoin%
\pgfsetlinewidth{0.501875pt}%
\definecolor{currentstroke}{rgb}{0.000000,0.000000,1.000000}%
\pgfsetstrokecolor{currentstroke}%
\pgfsetdash{}{0pt}%
\pgfpathmoveto{\pgfqpoint{1.494400in}{2.159193in}}%
\pgfpathlineto{\pgfqpoint{1.494400in}{2.276248in}}%
\pgfusepath{stroke}%
\end{pgfscope}%
\begin{pgfscope}%
\pgfpathrectangle{\pgfqpoint{0.800000in}{0.528000in}}{\pgfqpoint{4.960000in}{3.696000in}}%
\pgfusepath{clip}%
\pgfsetbuttcap%
\pgfsetroundjoin%
\pgfsetlinewidth{0.501875pt}%
\definecolor{currentstroke}{rgb}{0.000000,0.000000,1.000000}%
\pgfsetstrokecolor{currentstroke}%
\pgfsetdash{}{0pt}%
\pgfpathmoveto{\pgfqpoint{1.512436in}{2.744470in}}%
\pgfpathlineto{\pgfqpoint{1.512436in}{2.861525in}}%
\pgfusepath{stroke}%
\end{pgfscope}%
\begin{pgfscope}%
\pgfpathrectangle{\pgfqpoint{0.800000in}{0.528000in}}{\pgfqpoint{4.960000in}{3.696000in}}%
\pgfusepath{clip}%
\pgfsetbuttcap%
\pgfsetroundjoin%
\pgfsetlinewidth{0.501875pt}%
\definecolor{currentstroke}{rgb}{0.000000,0.000000,1.000000}%
\pgfsetstrokecolor{currentstroke}%
\pgfsetdash{}{0pt}%
\pgfpathmoveto{\pgfqpoint{1.530473in}{3.212692in}}%
\pgfpathlineto{\pgfqpoint{1.530473in}{3.329747in}}%
\pgfusepath{stroke}%
\end{pgfscope}%
\begin{pgfscope}%
\pgfpathrectangle{\pgfqpoint{0.800000in}{0.528000in}}{\pgfqpoint{4.960000in}{3.696000in}}%
\pgfusepath{clip}%
\pgfsetbuttcap%
\pgfsetroundjoin%
\pgfsetlinewidth{0.501875pt}%
\definecolor{currentstroke}{rgb}{0.000000,0.000000,1.000000}%
\pgfsetstrokecolor{currentstroke}%
\pgfsetdash{}{0pt}%
\pgfpathmoveto{\pgfqpoint{1.548509in}{3.505330in}}%
\pgfpathlineto{\pgfqpoint{1.548509in}{3.622386in}}%
\pgfusepath{stroke}%
\end{pgfscope}%
\begin{pgfscope}%
\pgfpathrectangle{\pgfqpoint{0.800000in}{0.528000in}}{\pgfqpoint{4.960000in}{3.696000in}}%
\pgfusepath{clip}%
\pgfsetbuttcap%
\pgfsetroundjoin%
\pgfsetlinewidth{0.501875pt}%
\definecolor{currentstroke}{rgb}{0.000000,0.000000,1.000000}%
\pgfsetstrokecolor{currentstroke}%
\pgfsetdash{}{0pt}%
\pgfpathmoveto{\pgfqpoint{1.566545in}{3.622386in}}%
\pgfpathlineto{\pgfqpoint{1.566545in}{3.739441in}}%
\pgfusepath{stroke}%
\end{pgfscope}%
\begin{pgfscope}%
\pgfpathrectangle{\pgfqpoint{0.800000in}{0.528000in}}{\pgfqpoint{4.960000in}{3.696000in}}%
\pgfusepath{clip}%
\pgfsetbuttcap%
\pgfsetroundjoin%
\pgfsetlinewidth{0.501875pt}%
\definecolor{currentstroke}{rgb}{0.000000,0.000000,1.000000}%
\pgfsetstrokecolor{currentstroke}%
\pgfsetdash{}{0pt}%
\pgfpathmoveto{\pgfqpoint{1.584582in}{3.563858in}}%
\pgfpathlineto{\pgfqpoint{1.584582in}{3.680913in}}%
\pgfusepath{stroke}%
\end{pgfscope}%
\begin{pgfscope}%
\pgfpathrectangle{\pgfqpoint{0.800000in}{0.528000in}}{\pgfqpoint{4.960000in}{3.696000in}}%
\pgfusepath{clip}%
\pgfsetbuttcap%
\pgfsetroundjoin%
\pgfsetlinewidth{0.501875pt}%
\definecolor{currentstroke}{rgb}{0.000000,0.000000,1.000000}%
\pgfsetstrokecolor{currentstroke}%
\pgfsetdash{}{0pt}%
\pgfpathmoveto{\pgfqpoint{1.602618in}{3.212692in}}%
\pgfpathlineto{\pgfqpoint{1.602618in}{3.329747in}}%
\pgfusepath{stroke}%
\end{pgfscope}%
\begin{pgfscope}%
\pgfpathrectangle{\pgfqpoint{0.800000in}{0.528000in}}{\pgfqpoint{4.960000in}{3.696000in}}%
\pgfusepath{clip}%
\pgfsetbuttcap%
\pgfsetroundjoin%
\pgfsetlinewidth{0.501875pt}%
\definecolor{currentstroke}{rgb}{0.000000,0.000000,1.000000}%
\pgfsetstrokecolor{currentstroke}%
\pgfsetdash{}{0pt}%
\pgfpathmoveto{\pgfqpoint{1.620655in}{2.744470in}}%
\pgfpathlineto{\pgfqpoint{1.620655in}{2.861525in}}%
\pgfusepath{stroke}%
\end{pgfscope}%
\begin{pgfscope}%
\pgfpathrectangle{\pgfqpoint{0.800000in}{0.528000in}}{\pgfqpoint{4.960000in}{3.696000in}}%
\pgfusepath{clip}%
\pgfsetbuttcap%
\pgfsetroundjoin%
\pgfsetlinewidth{0.501875pt}%
\definecolor{currentstroke}{rgb}{0.000000,0.000000,1.000000}%
\pgfsetstrokecolor{currentstroke}%
\pgfsetdash{}{0pt}%
\pgfpathmoveto{\pgfqpoint{1.638691in}{2.217720in}}%
\pgfpathlineto{\pgfqpoint{1.638691in}{2.334776in}}%
\pgfusepath{stroke}%
\end{pgfscope}%
\begin{pgfscope}%
\pgfpathrectangle{\pgfqpoint{0.800000in}{0.528000in}}{\pgfqpoint{4.960000in}{3.696000in}}%
\pgfusepath{clip}%
\pgfsetbuttcap%
\pgfsetroundjoin%
\pgfsetlinewidth{0.501875pt}%
\definecolor{currentstroke}{rgb}{0.000000,0.000000,1.000000}%
\pgfsetstrokecolor{currentstroke}%
\pgfsetdash{}{0pt}%
\pgfpathmoveto{\pgfqpoint{1.656727in}{1.573916in}}%
\pgfpathlineto{\pgfqpoint{1.656727in}{1.690971in}}%
\pgfusepath{stroke}%
\end{pgfscope}%
\begin{pgfscope}%
\pgfpathrectangle{\pgfqpoint{0.800000in}{0.528000in}}{\pgfqpoint{4.960000in}{3.696000in}}%
\pgfusepath{clip}%
\pgfsetbuttcap%
\pgfsetroundjoin%
\pgfsetlinewidth{0.501875pt}%
\definecolor{currentstroke}{rgb}{0.000000,0.000000,1.000000}%
\pgfsetstrokecolor{currentstroke}%
\pgfsetdash{}{0pt}%
\pgfpathmoveto{\pgfqpoint{1.674764in}{1.164222in}}%
\pgfpathlineto{\pgfqpoint{1.674764in}{1.281277in}}%
\pgfusepath{stroke}%
\end{pgfscope}%
\begin{pgfscope}%
\pgfpathrectangle{\pgfqpoint{0.800000in}{0.528000in}}{\pgfqpoint{4.960000in}{3.696000in}}%
\pgfusepath{clip}%
\pgfsetbuttcap%
\pgfsetroundjoin%
\pgfsetlinewidth{0.501875pt}%
\definecolor{currentstroke}{rgb}{0.000000,0.000000,1.000000}%
\pgfsetstrokecolor{currentstroke}%
\pgfsetdash{}{0pt}%
\pgfpathmoveto{\pgfqpoint{1.692800in}{0.988639in}}%
\pgfpathlineto{\pgfqpoint{1.692800in}{1.105694in}}%
\pgfusepath{stroke}%
\end{pgfscope}%
\begin{pgfscope}%
\pgfpathrectangle{\pgfqpoint{0.800000in}{0.528000in}}{\pgfqpoint{4.960000in}{3.696000in}}%
\pgfusepath{clip}%
\pgfsetbuttcap%
\pgfsetroundjoin%
\pgfsetlinewidth{0.501875pt}%
\definecolor{currentstroke}{rgb}{0.000000,0.000000,1.000000}%
\pgfsetstrokecolor{currentstroke}%
\pgfsetdash{}{0pt}%
\pgfpathmoveto{\pgfqpoint{1.710836in}{0.988639in}}%
\pgfpathlineto{\pgfqpoint{1.710836in}{1.105694in}}%
\pgfusepath{stroke}%
\end{pgfscope}%
\begin{pgfscope}%
\pgfpathrectangle{\pgfqpoint{0.800000in}{0.528000in}}{\pgfqpoint{4.960000in}{3.696000in}}%
\pgfusepath{clip}%
\pgfsetbuttcap%
\pgfsetroundjoin%
\pgfsetlinewidth{0.501875pt}%
\definecolor{currentstroke}{rgb}{0.000000,0.000000,1.000000}%
\pgfsetstrokecolor{currentstroke}%
\pgfsetdash{}{0pt}%
\pgfpathmoveto{\pgfqpoint{1.728873in}{1.281277in}}%
\pgfpathlineto{\pgfqpoint{1.728873in}{1.398333in}}%
\pgfusepath{stroke}%
\end{pgfscope}%
\begin{pgfscope}%
\pgfpathrectangle{\pgfqpoint{0.800000in}{0.528000in}}{\pgfqpoint{4.960000in}{3.696000in}}%
\pgfusepath{clip}%
\pgfsetbuttcap%
\pgfsetroundjoin%
\pgfsetlinewidth{0.501875pt}%
\definecolor{currentstroke}{rgb}{0.000000,0.000000,1.000000}%
\pgfsetstrokecolor{currentstroke}%
\pgfsetdash{}{0pt}%
\pgfpathmoveto{\pgfqpoint{1.746909in}{1.866554in}}%
\pgfpathlineto{\pgfqpoint{1.746909in}{1.983610in}}%
\pgfusepath{stroke}%
\end{pgfscope}%
\begin{pgfscope}%
\pgfpathrectangle{\pgfqpoint{0.800000in}{0.528000in}}{\pgfqpoint{4.960000in}{3.696000in}}%
\pgfusepath{clip}%
\pgfsetbuttcap%
\pgfsetroundjoin%
\pgfsetlinewidth{0.501875pt}%
\definecolor{currentstroke}{rgb}{0.000000,0.000000,1.000000}%
\pgfsetstrokecolor{currentstroke}%
\pgfsetdash{}{0pt}%
\pgfpathmoveto{\pgfqpoint{1.764945in}{2.393304in}}%
\pgfpathlineto{\pgfqpoint{1.764945in}{2.510359in}}%
\pgfusepath{stroke}%
\end{pgfscope}%
\begin{pgfscope}%
\pgfpathrectangle{\pgfqpoint{0.800000in}{0.528000in}}{\pgfqpoint{4.960000in}{3.696000in}}%
\pgfusepath{clip}%
\pgfsetbuttcap%
\pgfsetroundjoin%
\pgfsetlinewidth{0.501875pt}%
\definecolor{currentstroke}{rgb}{0.000000,0.000000,1.000000}%
\pgfsetstrokecolor{currentstroke}%
\pgfsetdash{}{0pt}%
\pgfpathmoveto{\pgfqpoint{1.782982in}{2.861525in}}%
\pgfpathlineto{\pgfqpoint{1.782982in}{2.978581in}}%
\pgfusepath{stroke}%
\end{pgfscope}%
\begin{pgfscope}%
\pgfpathrectangle{\pgfqpoint{0.800000in}{0.528000in}}{\pgfqpoint{4.960000in}{3.696000in}}%
\pgfusepath{clip}%
\pgfsetbuttcap%
\pgfsetroundjoin%
\pgfsetlinewidth{0.501875pt}%
\definecolor{currentstroke}{rgb}{0.000000,0.000000,1.000000}%
\pgfsetstrokecolor{currentstroke}%
\pgfsetdash{}{0pt}%
\pgfpathmoveto{\pgfqpoint{1.801018in}{3.271219in}}%
\pgfpathlineto{\pgfqpoint{1.801018in}{3.388275in}}%
\pgfusepath{stroke}%
\end{pgfscope}%
\begin{pgfscope}%
\pgfpathrectangle{\pgfqpoint{0.800000in}{0.528000in}}{\pgfqpoint{4.960000in}{3.696000in}}%
\pgfusepath{clip}%
\pgfsetbuttcap%
\pgfsetroundjoin%
\pgfsetlinewidth{0.501875pt}%
\definecolor{currentstroke}{rgb}{0.000000,0.000000,1.000000}%
\pgfsetstrokecolor{currentstroke}%
\pgfsetdash{}{0pt}%
\pgfpathmoveto{\pgfqpoint{1.819055in}{3.505330in}}%
\pgfpathlineto{\pgfqpoint{1.819055in}{3.622386in}}%
\pgfusepath{stroke}%
\end{pgfscope}%
\begin{pgfscope}%
\pgfpathrectangle{\pgfqpoint{0.800000in}{0.528000in}}{\pgfqpoint{4.960000in}{3.696000in}}%
\pgfusepath{clip}%
\pgfsetbuttcap%
\pgfsetroundjoin%
\pgfsetlinewidth{0.501875pt}%
\definecolor{currentstroke}{rgb}{0.000000,0.000000,1.000000}%
\pgfsetstrokecolor{currentstroke}%
\pgfsetdash{}{0pt}%
\pgfpathmoveto{\pgfqpoint{1.837091in}{3.505330in}}%
\pgfpathlineto{\pgfqpoint{1.837091in}{3.622386in}}%
\pgfusepath{stroke}%
\end{pgfscope}%
\begin{pgfscope}%
\pgfpathrectangle{\pgfqpoint{0.800000in}{0.528000in}}{\pgfqpoint{4.960000in}{3.696000in}}%
\pgfusepath{clip}%
\pgfsetbuttcap%
\pgfsetroundjoin%
\pgfsetlinewidth{0.501875pt}%
\definecolor{currentstroke}{rgb}{0.000000,0.000000,1.000000}%
\pgfsetstrokecolor{currentstroke}%
\pgfsetdash{}{0pt}%
\pgfpathmoveto{\pgfqpoint{1.855127in}{3.388275in}}%
\pgfpathlineto{\pgfqpoint{1.855127in}{3.505330in}}%
\pgfusepath{stroke}%
\end{pgfscope}%
\begin{pgfscope}%
\pgfpathrectangle{\pgfqpoint{0.800000in}{0.528000in}}{\pgfqpoint{4.960000in}{3.696000in}}%
\pgfusepath{clip}%
\pgfsetbuttcap%
\pgfsetroundjoin%
\pgfsetlinewidth{0.501875pt}%
\definecolor{currentstroke}{rgb}{0.000000,0.000000,1.000000}%
\pgfsetstrokecolor{currentstroke}%
\pgfsetdash{}{0pt}%
\pgfpathmoveto{\pgfqpoint{1.873164in}{3.037108in}}%
\pgfpathlineto{\pgfqpoint{1.873164in}{3.154164in}}%
\pgfusepath{stroke}%
\end{pgfscope}%
\begin{pgfscope}%
\pgfpathrectangle{\pgfqpoint{0.800000in}{0.528000in}}{\pgfqpoint{4.960000in}{3.696000in}}%
\pgfusepath{clip}%
\pgfsetbuttcap%
\pgfsetroundjoin%
\pgfsetlinewidth{0.501875pt}%
\definecolor{currentstroke}{rgb}{0.000000,0.000000,1.000000}%
\pgfsetstrokecolor{currentstroke}%
\pgfsetdash{}{0pt}%
\pgfpathmoveto{\pgfqpoint{1.891200in}{2.568887in}}%
\pgfpathlineto{\pgfqpoint{1.891200in}{2.685942in}}%
\pgfusepath{stroke}%
\end{pgfscope}%
\begin{pgfscope}%
\pgfpathrectangle{\pgfqpoint{0.800000in}{0.528000in}}{\pgfqpoint{4.960000in}{3.696000in}}%
\pgfusepath{clip}%
\pgfsetbuttcap%
\pgfsetroundjoin%
\pgfsetlinewidth{0.501875pt}%
\definecolor{currentstroke}{rgb}{0.000000,0.000000,1.000000}%
\pgfsetstrokecolor{currentstroke}%
\pgfsetdash{}{0pt}%
\pgfpathmoveto{\pgfqpoint{1.909236in}{2.100665in}}%
\pgfpathlineto{\pgfqpoint{1.909236in}{2.217720in}}%
\pgfusepath{stroke}%
\end{pgfscope}%
\begin{pgfscope}%
\pgfpathrectangle{\pgfqpoint{0.800000in}{0.528000in}}{\pgfqpoint{4.960000in}{3.696000in}}%
\pgfusepath{clip}%
\pgfsetbuttcap%
\pgfsetroundjoin%
\pgfsetlinewidth{0.501875pt}%
\definecolor{currentstroke}{rgb}{0.000000,0.000000,1.000000}%
\pgfsetstrokecolor{currentstroke}%
\pgfsetdash{}{0pt}%
\pgfpathmoveto{\pgfqpoint{1.927273in}{1.456860in}}%
\pgfpathlineto{\pgfqpoint{1.927273in}{1.573916in}}%
\pgfusepath{stroke}%
\end{pgfscope}%
\begin{pgfscope}%
\pgfpathrectangle{\pgfqpoint{0.800000in}{0.528000in}}{\pgfqpoint{4.960000in}{3.696000in}}%
\pgfusepath{clip}%
\pgfsetbuttcap%
\pgfsetroundjoin%
\pgfsetlinewidth{0.501875pt}%
\definecolor{currentstroke}{rgb}{0.000000,0.000000,1.000000}%
\pgfsetstrokecolor{currentstroke}%
\pgfsetdash{}{0pt}%
\pgfpathmoveto{\pgfqpoint{1.945309in}{1.164222in}}%
\pgfpathlineto{\pgfqpoint{1.945309in}{1.281277in}}%
\pgfusepath{stroke}%
\end{pgfscope}%
\begin{pgfscope}%
\pgfpathrectangle{\pgfqpoint{0.800000in}{0.528000in}}{\pgfqpoint{4.960000in}{3.696000in}}%
\pgfusepath{clip}%
\pgfsetbuttcap%
\pgfsetroundjoin%
\pgfsetlinewidth{0.501875pt}%
\definecolor{currentstroke}{rgb}{0.000000,0.000000,1.000000}%
\pgfsetstrokecolor{currentstroke}%
\pgfsetdash{}{0pt}%
\pgfpathmoveto{\pgfqpoint{1.963345in}{1.047166in}}%
\pgfpathlineto{\pgfqpoint{1.963345in}{1.164222in}}%
\pgfusepath{stroke}%
\end{pgfscope}%
\begin{pgfscope}%
\pgfpathrectangle{\pgfqpoint{0.800000in}{0.528000in}}{\pgfqpoint{4.960000in}{3.696000in}}%
\pgfusepath{clip}%
\pgfsetbuttcap%
\pgfsetroundjoin%
\pgfsetlinewidth{0.501875pt}%
\definecolor{currentstroke}{rgb}{0.000000,0.000000,1.000000}%
\pgfsetstrokecolor{currentstroke}%
\pgfsetdash{}{0pt}%
\pgfpathmoveto{\pgfqpoint{1.981382in}{1.164222in}}%
\pgfpathlineto{\pgfqpoint{1.981382in}{1.281277in}}%
\pgfusepath{stroke}%
\end{pgfscope}%
\begin{pgfscope}%
\pgfpathrectangle{\pgfqpoint{0.800000in}{0.528000in}}{\pgfqpoint{4.960000in}{3.696000in}}%
\pgfusepath{clip}%
\pgfsetbuttcap%
\pgfsetroundjoin%
\pgfsetlinewidth{0.501875pt}%
\definecolor{currentstroke}{rgb}{0.000000,0.000000,1.000000}%
\pgfsetstrokecolor{currentstroke}%
\pgfsetdash{}{0pt}%
\pgfpathmoveto{\pgfqpoint{1.999418in}{1.456860in}}%
\pgfpathlineto{\pgfqpoint{1.999418in}{1.573916in}}%
\pgfusepath{stroke}%
\end{pgfscope}%
\begin{pgfscope}%
\pgfpathrectangle{\pgfqpoint{0.800000in}{0.528000in}}{\pgfqpoint{4.960000in}{3.696000in}}%
\pgfusepath{clip}%
\pgfsetbuttcap%
\pgfsetroundjoin%
\pgfsetlinewidth{0.501875pt}%
\definecolor{currentstroke}{rgb}{0.000000,0.000000,1.000000}%
\pgfsetstrokecolor{currentstroke}%
\pgfsetdash{}{0pt}%
\pgfpathmoveto{\pgfqpoint{2.017455in}{2.042137in}}%
\pgfpathlineto{\pgfqpoint{2.017455in}{2.159193in}}%
\pgfusepath{stroke}%
\end{pgfscope}%
\begin{pgfscope}%
\pgfpathrectangle{\pgfqpoint{0.800000in}{0.528000in}}{\pgfqpoint{4.960000in}{3.696000in}}%
\pgfusepath{clip}%
\pgfsetbuttcap%
\pgfsetroundjoin%
\pgfsetlinewidth{0.501875pt}%
\definecolor{currentstroke}{rgb}{0.000000,0.000000,1.000000}%
\pgfsetstrokecolor{currentstroke}%
\pgfsetdash{}{0pt}%
\pgfpathmoveto{\pgfqpoint{2.035491in}{2.568887in}}%
\pgfpathlineto{\pgfqpoint{2.035491in}{2.685942in}}%
\pgfusepath{stroke}%
\end{pgfscope}%
\begin{pgfscope}%
\pgfpathrectangle{\pgfqpoint{0.800000in}{0.528000in}}{\pgfqpoint{4.960000in}{3.696000in}}%
\pgfusepath{clip}%
\pgfsetbuttcap%
\pgfsetroundjoin%
\pgfsetlinewidth{0.501875pt}%
\definecolor{currentstroke}{rgb}{0.000000,0.000000,1.000000}%
\pgfsetstrokecolor{currentstroke}%
\pgfsetdash{}{0pt}%
\pgfpathmoveto{\pgfqpoint{2.053527in}{2.978581in}}%
\pgfpathlineto{\pgfqpoint{2.053527in}{3.095636in}}%
\pgfusepath{stroke}%
\end{pgfscope}%
\begin{pgfscope}%
\pgfpathrectangle{\pgfqpoint{0.800000in}{0.528000in}}{\pgfqpoint{4.960000in}{3.696000in}}%
\pgfusepath{clip}%
\pgfsetbuttcap%
\pgfsetroundjoin%
\pgfsetlinewidth{0.501875pt}%
\definecolor{currentstroke}{rgb}{0.000000,0.000000,1.000000}%
\pgfsetstrokecolor{currentstroke}%
\pgfsetdash{}{0pt}%
\pgfpathmoveto{\pgfqpoint{2.071564in}{3.271219in}}%
\pgfpathlineto{\pgfqpoint{2.071564in}{3.388275in}}%
\pgfusepath{stroke}%
\end{pgfscope}%
\begin{pgfscope}%
\pgfpathrectangle{\pgfqpoint{0.800000in}{0.528000in}}{\pgfqpoint{4.960000in}{3.696000in}}%
\pgfusepath{clip}%
\pgfsetbuttcap%
\pgfsetroundjoin%
\pgfsetlinewidth{0.501875pt}%
\definecolor{currentstroke}{rgb}{0.000000,0.000000,1.000000}%
\pgfsetstrokecolor{currentstroke}%
\pgfsetdash{}{0pt}%
\pgfpathmoveto{\pgfqpoint{2.089600in}{3.446802in}}%
\pgfpathlineto{\pgfqpoint{2.089600in}{3.563858in}}%
\pgfusepath{stroke}%
\end{pgfscope}%
\begin{pgfscope}%
\pgfpathrectangle{\pgfqpoint{0.800000in}{0.528000in}}{\pgfqpoint{4.960000in}{3.696000in}}%
\pgfusepath{clip}%
\pgfsetbuttcap%
\pgfsetroundjoin%
\pgfsetlinewidth{0.501875pt}%
\definecolor{currentstroke}{rgb}{0.000000,0.000000,1.000000}%
\pgfsetstrokecolor{currentstroke}%
\pgfsetdash{}{0pt}%
\pgfpathmoveto{\pgfqpoint{2.107636in}{3.446802in}}%
\pgfpathlineto{\pgfqpoint{2.107636in}{3.563858in}}%
\pgfusepath{stroke}%
\end{pgfscope}%
\begin{pgfscope}%
\pgfpathrectangle{\pgfqpoint{0.800000in}{0.528000in}}{\pgfqpoint{4.960000in}{3.696000in}}%
\pgfusepath{clip}%
\pgfsetbuttcap%
\pgfsetroundjoin%
\pgfsetlinewidth{0.501875pt}%
\definecolor{currentstroke}{rgb}{0.000000,0.000000,1.000000}%
\pgfsetstrokecolor{currentstroke}%
\pgfsetdash{}{0pt}%
\pgfpathmoveto{\pgfqpoint{2.125673in}{3.212692in}}%
\pgfpathlineto{\pgfqpoint{2.125673in}{3.329747in}}%
\pgfusepath{stroke}%
\end{pgfscope}%
\begin{pgfscope}%
\pgfpathrectangle{\pgfqpoint{0.800000in}{0.528000in}}{\pgfqpoint{4.960000in}{3.696000in}}%
\pgfusepath{clip}%
\pgfsetbuttcap%
\pgfsetroundjoin%
\pgfsetlinewidth{0.501875pt}%
\definecolor{currentstroke}{rgb}{0.000000,0.000000,1.000000}%
\pgfsetstrokecolor{currentstroke}%
\pgfsetdash{}{0pt}%
\pgfpathmoveto{\pgfqpoint{2.143709in}{2.861525in}}%
\pgfpathlineto{\pgfqpoint{2.143709in}{2.978581in}}%
\pgfusepath{stroke}%
\end{pgfscope}%
\begin{pgfscope}%
\pgfpathrectangle{\pgfqpoint{0.800000in}{0.528000in}}{\pgfqpoint{4.960000in}{3.696000in}}%
\pgfusepath{clip}%
\pgfsetbuttcap%
\pgfsetroundjoin%
\pgfsetlinewidth{0.501875pt}%
\definecolor{currentstroke}{rgb}{0.000000,0.000000,1.000000}%
\pgfsetstrokecolor{currentstroke}%
\pgfsetdash{}{0pt}%
\pgfpathmoveto{\pgfqpoint{2.161745in}{2.393304in}}%
\pgfpathlineto{\pgfqpoint{2.161745in}{2.510359in}}%
\pgfusepath{stroke}%
\end{pgfscope}%
\begin{pgfscope}%
\pgfpathrectangle{\pgfqpoint{0.800000in}{0.528000in}}{\pgfqpoint{4.960000in}{3.696000in}}%
\pgfusepath{clip}%
\pgfsetbuttcap%
\pgfsetroundjoin%
\pgfsetlinewidth{0.501875pt}%
\definecolor{currentstroke}{rgb}{0.000000,0.000000,1.000000}%
\pgfsetstrokecolor{currentstroke}%
\pgfsetdash{}{0pt}%
\pgfpathmoveto{\pgfqpoint{2.179782in}{1.983610in}}%
\pgfpathlineto{\pgfqpoint{2.179782in}{2.100665in}}%
\pgfusepath{stroke}%
\end{pgfscope}%
\begin{pgfscope}%
\pgfpathrectangle{\pgfqpoint{0.800000in}{0.528000in}}{\pgfqpoint{4.960000in}{3.696000in}}%
\pgfusepath{clip}%
\pgfsetbuttcap%
\pgfsetroundjoin%
\pgfsetlinewidth{0.501875pt}%
\definecolor{currentstroke}{rgb}{0.000000,0.000000,1.000000}%
\pgfsetstrokecolor{currentstroke}%
\pgfsetdash{}{0pt}%
\pgfpathmoveto{\pgfqpoint{2.197818in}{1.398333in}}%
\pgfpathlineto{\pgfqpoint{2.197818in}{1.515388in}}%
\pgfusepath{stroke}%
\end{pgfscope}%
\begin{pgfscope}%
\pgfpathrectangle{\pgfqpoint{0.800000in}{0.528000in}}{\pgfqpoint{4.960000in}{3.696000in}}%
\pgfusepath{clip}%
\pgfsetbuttcap%
\pgfsetroundjoin%
\pgfsetlinewidth{0.501875pt}%
\definecolor{currentstroke}{rgb}{0.000000,0.000000,1.000000}%
\pgfsetstrokecolor{currentstroke}%
\pgfsetdash{}{0pt}%
\pgfpathmoveto{\pgfqpoint{2.215855in}{1.222749in}}%
\pgfpathlineto{\pgfqpoint{2.215855in}{1.339805in}}%
\pgfusepath{stroke}%
\end{pgfscope}%
\begin{pgfscope}%
\pgfpathrectangle{\pgfqpoint{0.800000in}{0.528000in}}{\pgfqpoint{4.960000in}{3.696000in}}%
\pgfusepath{clip}%
\pgfsetbuttcap%
\pgfsetroundjoin%
\pgfsetlinewidth{0.501875pt}%
\definecolor{currentstroke}{rgb}{0.000000,0.000000,1.000000}%
\pgfsetstrokecolor{currentstroke}%
\pgfsetdash{}{0pt}%
\pgfpathmoveto{\pgfqpoint{2.233891in}{1.164222in}}%
\pgfpathlineto{\pgfqpoint{2.233891in}{1.281277in}}%
\pgfusepath{stroke}%
\end{pgfscope}%
\begin{pgfscope}%
\pgfpathrectangle{\pgfqpoint{0.800000in}{0.528000in}}{\pgfqpoint{4.960000in}{3.696000in}}%
\pgfusepath{clip}%
\pgfsetbuttcap%
\pgfsetroundjoin%
\pgfsetlinewidth{0.501875pt}%
\definecolor{currentstroke}{rgb}{0.000000,0.000000,1.000000}%
\pgfsetstrokecolor{currentstroke}%
\pgfsetdash{}{0pt}%
\pgfpathmoveto{\pgfqpoint{2.251927in}{1.339805in}}%
\pgfpathlineto{\pgfqpoint{2.251927in}{1.456860in}}%
\pgfusepath{stroke}%
\end{pgfscope}%
\begin{pgfscope}%
\pgfpathrectangle{\pgfqpoint{0.800000in}{0.528000in}}{\pgfqpoint{4.960000in}{3.696000in}}%
\pgfusepath{clip}%
\pgfsetbuttcap%
\pgfsetroundjoin%
\pgfsetlinewidth{0.501875pt}%
\definecolor{currentstroke}{rgb}{0.000000,0.000000,1.000000}%
\pgfsetstrokecolor{currentstroke}%
\pgfsetdash{}{0pt}%
\pgfpathmoveto{\pgfqpoint{2.269964in}{1.690971in}}%
\pgfpathlineto{\pgfqpoint{2.269964in}{1.808026in}}%
\pgfusepath{stroke}%
\end{pgfscope}%
\begin{pgfscope}%
\pgfpathrectangle{\pgfqpoint{0.800000in}{0.528000in}}{\pgfqpoint{4.960000in}{3.696000in}}%
\pgfusepath{clip}%
\pgfsetbuttcap%
\pgfsetroundjoin%
\pgfsetlinewidth{0.501875pt}%
\definecolor{currentstroke}{rgb}{0.000000,0.000000,1.000000}%
\pgfsetstrokecolor{currentstroke}%
\pgfsetdash{}{0pt}%
\pgfpathmoveto{\pgfqpoint{2.288000in}{2.276248in}}%
\pgfpathlineto{\pgfqpoint{2.288000in}{2.393304in}}%
\pgfusepath{stroke}%
\end{pgfscope}%
\begin{pgfscope}%
\pgfpathrectangle{\pgfqpoint{0.800000in}{0.528000in}}{\pgfqpoint{4.960000in}{3.696000in}}%
\pgfusepath{clip}%
\pgfsetbuttcap%
\pgfsetroundjoin%
\pgfsetlinewidth{0.501875pt}%
\definecolor{currentstroke}{rgb}{0.000000,0.000000,1.000000}%
\pgfsetstrokecolor{currentstroke}%
\pgfsetdash{}{0pt}%
\pgfpathmoveto{\pgfqpoint{2.306036in}{2.685942in}}%
\pgfpathlineto{\pgfqpoint{2.306036in}{2.802998in}}%
\pgfusepath{stroke}%
\end{pgfscope}%
\begin{pgfscope}%
\pgfpathrectangle{\pgfqpoint{0.800000in}{0.528000in}}{\pgfqpoint{4.960000in}{3.696000in}}%
\pgfusepath{clip}%
\pgfsetbuttcap%
\pgfsetroundjoin%
\pgfsetlinewidth{0.501875pt}%
\definecolor{currentstroke}{rgb}{0.000000,0.000000,1.000000}%
\pgfsetstrokecolor{currentstroke}%
\pgfsetdash{}{0pt}%
\pgfpathmoveto{\pgfqpoint{2.324073in}{3.037108in}}%
\pgfpathlineto{\pgfqpoint{2.324073in}{3.154164in}}%
\pgfusepath{stroke}%
\end{pgfscope}%
\begin{pgfscope}%
\pgfpathrectangle{\pgfqpoint{0.800000in}{0.528000in}}{\pgfqpoint{4.960000in}{3.696000in}}%
\pgfusepath{clip}%
\pgfsetbuttcap%
\pgfsetroundjoin%
\pgfsetlinewidth{0.501875pt}%
\definecolor{currentstroke}{rgb}{0.000000,0.000000,1.000000}%
\pgfsetstrokecolor{currentstroke}%
\pgfsetdash{}{0pt}%
\pgfpathmoveto{\pgfqpoint{2.342109in}{3.271219in}}%
\pgfpathlineto{\pgfqpoint{2.342109in}{3.388275in}}%
\pgfusepath{stroke}%
\end{pgfscope}%
\begin{pgfscope}%
\pgfpathrectangle{\pgfqpoint{0.800000in}{0.528000in}}{\pgfqpoint{4.960000in}{3.696000in}}%
\pgfusepath{clip}%
\pgfsetbuttcap%
\pgfsetroundjoin%
\pgfsetlinewidth{0.501875pt}%
\definecolor{currentstroke}{rgb}{0.000000,0.000000,1.000000}%
\pgfsetstrokecolor{currentstroke}%
\pgfsetdash{}{0pt}%
\pgfpathmoveto{\pgfqpoint{2.360145in}{3.388275in}}%
\pgfpathlineto{\pgfqpoint{2.360145in}{3.505330in}}%
\pgfusepath{stroke}%
\end{pgfscope}%
\begin{pgfscope}%
\pgfpathrectangle{\pgfqpoint{0.800000in}{0.528000in}}{\pgfqpoint{4.960000in}{3.696000in}}%
\pgfusepath{clip}%
\pgfsetbuttcap%
\pgfsetroundjoin%
\pgfsetlinewidth{0.501875pt}%
\definecolor{currentstroke}{rgb}{0.000000,0.000000,1.000000}%
\pgfsetstrokecolor{currentstroke}%
\pgfsetdash{}{0pt}%
\pgfpathmoveto{\pgfqpoint{2.378182in}{3.271219in}}%
\pgfpathlineto{\pgfqpoint{2.378182in}{3.388275in}}%
\pgfusepath{stroke}%
\end{pgfscope}%
\begin{pgfscope}%
\pgfpathrectangle{\pgfqpoint{0.800000in}{0.528000in}}{\pgfqpoint{4.960000in}{3.696000in}}%
\pgfusepath{clip}%
\pgfsetbuttcap%
\pgfsetroundjoin%
\pgfsetlinewidth{0.501875pt}%
\definecolor{currentstroke}{rgb}{0.000000,0.000000,1.000000}%
\pgfsetstrokecolor{currentstroke}%
\pgfsetdash{}{0pt}%
\pgfpathmoveto{\pgfqpoint{2.396218in}{3.037108in}}%
\pgfpathlineto{\pgfqpoint{2.396218in}{3.154164in}}%
\pgfusepath{stroke}%
\end{pgfscope}%
\begin{pgfscope}%
\pgfpathrectangle{\pgfqpoint{0.800000in}{0.528000in}}{\pgfqpoint{4.960000in}{3.696000in}}%
\pgfusepath{clip}%
\pgfsetbuttcap%
\pgfsetroundjoin%
\pgfsetlinewidth{0.501875pt}%
\definecolor{currentstroke}{rgb}{0.000000,0.000000,1.000000}%
\pgfsetstrokecolor{currentstroke}%
\pgfsetdash{}{0pt}%
\pgfpathmoveto{\pgfqpoint{2.414255in}{2.685942in}}%
\pgfpathlineto{\pgfqpoint{2.414255in}{2.802998in}}%
\pgfusepath{stroke}%
\end{pgfscope}%
\begin{pgfscope}%
\pgfpathrectangle{\pgfqpoint{0.800000in}{0.528000in}}{\pgfqpoint{4.960000in}{3.696000in}}%
\pgfusepath{clip}%
\pgfsetbuttcap%
\pgfsetroundjoin%
\pgfsetlinewidth{0.501875pt}%
\definecolor{currentstroke}{rgb}{0.000000,0.000000,1.000000}%
\pgfsetstrokecolor{currentstroke}%
\pgfsetdash{}{0pt}%
\pgfpathmoveto{\pgfqpoint{2.432291in}{2.276248in}}%
\pgfpathlineto{\pgfqpoint{2.432291in}{2.393304in}}%
\pgfusepath{stroke}%
\end{pgfscope}%
\begin{pgfscope}%
\pgfpathrectangle{\pgfqpoint{0.800000in}{0.528000in}}{\pgfqpoint{4.960000in}{3.696000in}}%
\pgfusepath{clip}%
\pgfsetbuttcap%
\pgfsetroundjoin%
\pgfsetlinewidth{0.501875pt}%
\definecolor{currentstroke}{rgb}{0.000000,0.000000,1.000000}%
\pgfsetstrokecolor{currentstroke}%
\pgfsetdash{}{0pt}%
\pgfpathmoveto{\pgfqpoint{2.450327in}{1.690971in}}%
\pgfpathlineto{\pgfqpoint{2.450327in}{1.808026in}}%
\pgfusepath{stroke}%
\end{pgfscope}%
\begin{pgfscope}%
\pgfpathrectangle{\pgfqpoint{0.800000in}{0.528000in}}{\pgfqpoint{4.960000in}{3.696000in}}%
\pgfusepath{clip}%
\pgfsetbuttcap%
\pgfsetroundjoin%
\pgfsetlinewidth{0.501875pt}%
\definecolor{currentstroke}{rgb}{0.000000,0.000000,1.000000}%
\pgfsetstrokecolor{currentstroke}%
\pgfsetdash{}{0pt}%
\pgfpathmoveto{\pgfqpoint{2.468364in}{1.398333in}}%
\pgfpathlineto{\pgfqpoint{2.468364in}{1.515388in}}%
\pgfusepath{stroke}%
\end{pgfscope}%
\begin{pgfscope}%
\pgfpathrectangle{\pgfqpoint{0.800000in}{0.528000in}}{\pgfqpoint{4.960000in}{3.696000in}}%
\pgfusepath{clip}%
\pgfsetbuttcap%
\pgfsetroundjoin%
\pgfsetlinewidth{0.501875pt}%
\definecolor{currentstroke}{rgb}{0.000000,0.000000,1.000000}%
\pgfsetstrokecolor{currentstroke}%
\pgfsetdash{}{0pt}%
\pgfpathmoveto{\pgfqpoint{2.486400in}{1.222749in}}%
\pgfpathlineto{\pgfqpoint{2.486400in}{1.339805in}}%
\pgfusepath{stroke}%
\end{pgfscope}%
\begin{pgfscope}%
\pgfpathrectangle{\pgfqpoint{0.800000in}{0.528000in}}{\pgfqpoint{4.960000in}{3.696000in}}%
\pgfusepath{clip}%
\pgfsetbuttcap%
\pgfsetroundjoin%
\pgfsetlinewidth{0.501875pt}%
\definecolor{currentstroke}{rgb}{0.000000,0.000000,1.000000}%
\pgfsetstrokecolor{currentstroke}%
\pgfsetdash{}{0pt}%
\pgfpathmoveto{\pgfqpoint{2.504436in}{1.281277in}}%
\pgfpathlineto{\pgfqpoint{2.504436in}{1.398333in}}%
\pgfusepath{stroke}%
\end{pgfscope}%
\begin{pgfscope}%
\pgfpathrectangle{\pgfqpoint{0.800000in}{0.528000in}}{\pgfqpoint{4.960000in}{3.696000in}}%
\pgfusepath{clip}%
\pgfsetbuttcap%
\pgfsetroundjoin%
\pgfsetlinewidth{0.501875pt}%
\definecolor{currentstroke}{rgb}{0.000000,0.000000,1.000000}%
\pgfsetstrokecolor{currentstroke}%
\pgfsetdash{}{0pt}%
\pgfpathmoveto{\pgfqpoint{2.522473in}{1.515388in}}%
\pgfpathlineto{\pgfqpoint{2.522473in}{1.632443in}}%
\pgfusepath{stroke}%
\end{pgfscope}%
\begin{pgfscope}%
\pgfpathrectangle{\pgfqpoint{0.800000in}{0.528000in}}{\pgfqpoint{4.960000in}{3.696000in}}%
\pgfusepath{clip}%
\pgfsetbuttcap%
\pgfsetroundjoin%
\pgfsetlinewidth{0.501875pt}%
\definecolor{currentstroke}{rgb}{0.000000,0.000000,1.000000}%
\pgfsetstrokecolor{currentstroke}%
\pgfsetdash{}{0pt}%
\pgfpathmoveto{\pgfqpoint{2.540509in}{1.983610in}}%
\pgfpathlineto{\pgfqpoint{2.540509in}{2.100665in}}%
\pgfusepath{stroke}%
\end{pgfscope}%
\begin{pgfscope}%
\pgfpathrectangle{\pgfqpoint{0.800000in}{0.528000in}}{\pgfqpoint{4.960000in}{3.696000in}}%
\pgfusepath{clip}%
\pgfsetbuttcap%
\pgfsetroundjoin%
\pgfsetlinewidth{0.501875pt}%
\definecolor{currentstroke}{rgb}{0.000000,0.000000,1.000000}%
\pgfsetstrokecolor{currentstroke}%
\pgfsetdash{}{0pt}%
\pgfpathmoveto{\pgfqpoint{2.558545in}{2.393304in}}%
\pgfpathlineto{\pgfqpoint{2.558545in}{2.510359in}}%
\pgfusepath{stroke}%
\end{pgfscope}%
\begin{pgfscope}%
\pgfpathrectangle{\pgfqpoint{0.800000in}{0.528000in}}{\pgfqpoint{4.960000in}{3.696000in}}%
\pgfusepath{clip}%
\pgfsetbuttcap%
\pgfsetroundjoin%
\pgfsetlinewidth{0.501875pt}%
\definecolor{currentstroke}{rgb}{0.000000,0.000000,1.000000}%
\pgfsetstrokecolor{currentstroke}%
\pgfsetdash{}{0pt}%
\pgfpathmoveto{\pgfqpoint{2.576582in}{2.802998in}}%
\pgfpathlineto{\pgfqpoint{2.576582in}{2.920053in}}%
\pgfusepath{stroke}%
\end{pgfscope}%
\begin{pgfscope}%
\pgfpathrectangle{\pgfqpoint{0.800000in}{0.528000in}}{\pgfqpoint{4.960000in}{3.696000in}}%
\pgfusepath{clip}%
\pgfsetbuttcap%
\pgfsetroundjoin%
\pgfsetlinewidth{0.501875pt}%
\definecolor{currentstroke}{rgb}{0.000000,0.000000,1.000000}%
\pgfsetstrokecolor{currentstroke}%
\pgfsetdash{}{0pt}%
\pgfpathmoveto{\pgfqpoint{2.594618in}{3.095636in}}%
\pgfpathlineto{\pgfqpoint{2.594618in}{3.212692in}}%
\pgfusepath{stroke}%
\end{pgfscope}%
\begin{pgfscope}%
\pgfpathrectangle{\pgfqpoint{0.800000in}{0.528000in}}{\pgfqpoint{4.960000in}{3.696000in}}%
\pgfusepath{clip}%
\pgfsetbuttcap%
\pgfsetroundjoin%
\pgfsetlinewidth{0.501875pt}%
\definecolor{currentstroke}{rgb}{0.000000,0.000000,1.000000}%
\pgfsetstrokecolor{currentstroke}%
\pgfsetdash{}{0pt}%
\pgfpathmoveto{\pgfqpoint{2.612655in}{3.271219in}}%
\pgfpathlineto{\pgfqpoint{2.612655in}{3.388275in}}%
\pgfusepath{stroke}%
\end{pgfscope}%
\begin{pgfscope}%
\pgfpathrectangle{\pgfqpoint{0.800000in}{0.528000in}}{\pgfqpoint{4.960000in}{3.696000in}}%
\pgfusepath{clip}%
\pgfsetbuttcap%
\pgfsetroundjoin%
\pgfsetlinewidth{0.501875pt}%
\definecolor{currentstroke}{rgb}{0.000000,0.000000,1.000000}%
\pgfsetstrokecolor{currentstroke}%
\pgfsetdash{}{0pt}%
\pgfpathmoveto{\pgfqpoint{2.630691in}{3.271219in}}%
\pgfpathlineto{\pgfqpoint{2.630691in}{3.388275in}}%
\pgfusepath{stroke}%
\end{pgfscope}%
\begin{pgfscope}%
\pgfpathrectangle{\pgfqpoint{0.800000in}{0.528000in}}{\pgfqpoint{4.960000in}{3.696000in}}%
\pgfusepath{clip}%
\pgfsetbuttcap%
\pgfsetroundjoin%
\pgfsetlinewidth{0.501875pt}%
\definecolor{currentstroke}{rgb}{0.000000,0.000000,1.000000}%
\pgfsetstrokecolor{currentstroke}%
\pgfsetdash{}{0pt}%
\pgfpathmoveto{\pgfqpoint{2.648727in}{3.154164in}}%
\pgfpathlineto{\pgfqpoint{2.648727in}{3.271219in}}%
\pgfusepath{stroke}%
\end{pgfscope}%
\begin{pgfscope}%
\pgfpathrectangle{\pgfqpoint{0.800000in}{0.528000in}}{\pgfqpoint{4.960000in}{3.696000in}}%
\pgfusepath{clip}%
\pgfsetbuttcap%
\pgfsetroundjoin%
\pgfsetlinewidth{0.501875pt}%
\definecolor{currentstroke}{rgb}{0.000000,0.000000,1.000000}%
\pgfsetstrokecolor{currentstroke}%
\pgfsetdash{}{0pt}%
\pgfpathmoveto{\pgfqpoint{2.666764in}{2.861525in}}%
\pgfpathlineto{\pgfqpoint{2.666764in}{2.978581in}}%
\pgfusepath{stroke}%
\end{pgfscope}%
\begin{pgfscope}%
\pgfpathrectangle{\pgfqpoint{0.800000in}{0.528000in}}{\pgfqpoint{4.960000in}{3.696000in}}%
\pgfusepath{clip}%
\pgfsetbuttcap%
\pgfsetroundjoin%
\pgfsetlinewidth{0.501875pt}%
\definecolor{currentstroke}{rgb}{0.000000,0.000000,1.000000}%
\pgfsetstrokecolor{currentstroke}%
\pgfsetdash{}{0pt}%
\pgfpathmoveto{\pgfqpoint{2.684800in}{2.510359in}}%
\pgfpathlineto{\pgfqpoint{2.684800in}{2.627414in}}%
\pgfusepath{stroke}%
\end{pgfscope}%
\begin{pgfscope}%
\pgfpathrectangle{\pgfqpoint{0.800000in}{0.528000in}}{\pgfqpoint{4.960000in}{3.696000in}}%
\pgfusepath{clip}%
\pgfsetbuttcap%
\pgfsetroundjoin%
\pgfsetlinewidth{0.501875pt}%
\definecolor{currentstroke}{rgb}{0.000000,0.000000,1.000000}%
\pgfsetstrokecolor{currentstroke}%
\pgfsetdash{}{0pt}%
\pgfpathmoveto{\pgfqpoint{2.702836in}{2.159193in}}%
\pgfpathlineto{\pgfqpoint{2.702836in}{2.276248in}}%
\pgfusepath{stroke}%
\end{pgfscope}%
\begin{pgfscope}%
\pgfpathrectangle{\pgfqpoint{0.800000in}{0.528000in}}{\pgfqpoint{4.960000in}{3.696000in}}%
\pgfusepath{clip}%
\pgfsetbuttcap%
\pgfsetroundjoin%
\pgfsetlinewidth{0.501875pt}%
\definecolor{currentstroke}{rgb}{0.000000,0.000000,1.000000}%
\pgfsetstrokecolor{currentstroke}%
\pgfsetdash{}{0pt}%
\pgfpathmoveto{\pgfqpoint{2.720873in}{1.749499in}}%
\pgfpathlineto{\pgfqpoint{2.720873in}{1.866554in}}%
\pgfusepath{stroke}%
\end{pgfscope}%
\begin{pgfscope}%
\pgfpathrectangle{\pgfqpoint{0.800000in}{0.528000in}}{\pgfqpoint{4.960000in}{3.696000in}}%
\pgfusepath{clip}%
\pgfsetbuttcap%
\pgfsetroundjoin%
\pgfsetlinewidth{0.501875pt}%
\definecolor{currentstroke}{rgb}{0.000000,0.000000,1.000000}%
\pgfsetstrokecolor{currentstroke}%
\pgfsetdash{}{0pt}%
\pgfpathmoveto{\pgfqpoint{2.738909in}{1.398333in}}%
\pgfpathlineto{\pgfqpoint{2.738909in}{1.515388in}}%
\pgfusepath{stroke}%
\end{pgfscope}%
\begin{pgfscope}%
\pgfpathrectangle{\pgfqpoint{0.800000in}{0.528000in}}{\pgfqpoint{4.960000in}{3.696000in}}%
\pgfusepath{clip}%
\pgfsetbuttcap%
\pgfsetroundjoin%
\pgfsetlinewidth{0.501875pt}%
\definecolor{currentstroke}{rgb}{0.000000,0.000000,1.000000}%
\pgfsetstrokecolor{currentstroke}%
\pgfsetdash{}{0pt}%
\pgfpathmoveto{\pgfqpoint{2.756945in}{1.339805in}}%
\pgfpathlineto{\pgfqpoint{2.756945in}{1.456860in}}%
\pgfusepath{stroke}%
\end{pgfscope}%
\begin{pgfscope}%
\pgfpathrectangle{\pgfqpoint{0.800000in}{0.528000in}}{\pgfqpoint{4.960000in}{3.696000in}}%
\pgfusepath{clip}%
\pgfsetbuttcap%
\pgfsetroundjoin%
\pgfsetlinewidth{0.501875pt}%
\definecolor{currentstroke}{rgb}{0.000000,0.000000,1.000000}%
\pgfsetstrokecolor{currentstroke}%
\pgfsetdash{}{0pt}%
\pgfpathmoveto{\pgfqpoint{2.774982in}{1.398333in}}%
\pgfpathlineto{\pgfqpoint{2.774982in}{1.515388in}}%
\pgfusepath{stroke}%
\end{pgfscope}%
\begin{pgfscope}%
\pgfpathrectangle{\pgfqpoint{0.800000in}{0.528000in}}{\pgfqpoint{4.960000in}{3.696000in}}%
\pgfusepath{clip}%
\pgfsetbuttcap%
\pgfsetroundjoin%
\pgfsetlinewidth{0.501875pt}%
\definecolor{currentstroke}{rgb}{0.000000,0.000000,1.000000}%
\pgfsetstrokecolor{currentstroke}%
\pgfsetdash{}{0pt}%
\pgfpathmoveto{\pgfqpoint{2.793018in}{1.808026in}}%
\pgfpathlineto{\pgfqpoint{2.793018in}{1.925082in}}%
\pgfusepath{stroke}%
\end{pgfscope}%
\begin{pgfscope}%
\pgfpathrectangle{\pgfqpoint{0.800000in}{0.528000in}}{\pgfqpoint{4.960000in}{3.696000in}}%
\pgfusepath{clip}%
\pgfsetbuttcap%
\pgfsetroundjoin%
\pgfsetlinewidth{0.501875pt}%
\definecolor{currentstroke}{rgb}{0.000000,0.000000,1.000000}%
\pgfsetstrokecolor{currentstroke}%
\pgfsetdash{}{0pt}%
\pgfpathmoveto{\pgfqpoint{2.811055in}{2.159193in}}%
\pgfpathlineto{\pgfqpoint{2.811055in}{2.276248in}}%
\pgfusepath{stroke}%
\end{pgfscope}%
\begin{pgfscope}%
\pgfpathrectangle{\pgfqpoint{0.800000in}{0.528000in}}{\pgfqpoint{4.960000in}{3.696000in}}%
\pgfusepath{clip}%
\pgfsetbuttcap%
\pgfsetroundjoin%
\pgfsetlinewidth{0.501875pt}%
\definecolor{currentstroke}{rgb}{0.000000,0.000000,1.000000}%
\pgfsetstrokecolor{currentstroke}%
\pgfsetdash{}{0pt}%
\pgfpathmoveto{\pgfqpoint{2.829091in}{2.568887in}}%
\pgfpathlineto{\pgfqpoint{2.829091in}{2.685942in}}%
\pgfusepath{stroke}%
\end{pgfscope}%
\begin{pgfscope}%
\pgfpathrectangle{\pgfqpoint{0.800000in}{0.528000in}}{\pgfqpoint{4.960000in}{3.696000in}}%
\pgfusepath{clip}%
\pgfsetbuttcap%
\pgfsetroundjoin%
\pgfsetlinewidth{0.501875pt}%
\definecolor{currentstroke}{rgb}{0.000000,0.000000,1.000000}%
\pgfsetstrokecolor{currentstroke}%
\pgfsetdash{}{0pt}%
\pgfpathmoveto{\pgfqpoint{2.847127in}{2.861525in}}%
\pgfpathlineto{\pgfqpoint{2.847127in}{2.978581in}}%
\pgfusepath{stroke}%
\end{pgfscope}%
\begin{pgfscope}%
\pgfpathrectangle{\pgfqpoint{0.800000in}{0.528000in}}{\pgfqpoint{4.960000in}{3.696000in}}%
\pgfusepath{clip}%
\pgfsetbuttcap%
\pgfsetroundjoin%
\pgfsetlinewidth{0.501875pt}%
\definecolor{currentstroke}{rgb}{0.000000,0.000000,1.000000}%
\pgfsetstrokecolor{currentstroke}%
\pgfsetdash{}{0pt}%
\pgfpathmoveto{\pgfqpoint{2.865164in}{3.154164in}}%
\pgfpathlineto{\pgfqpoint{2.865164in}{3.271219in}}%
\pgfusepath{stroke}%
\end{pgfscope}%
\begin{pgfscope}%
\pgfpathrectangle{\pgfqpoint{0.800000in}{0.528000in}}{\pgfqpoint{4.960000in}{3.696000in}}%
\pgfusepath{clip}%
\pgfsetbuttcap%
\pgfsetroundjoin%
\pgfsetlinewidth{0.501875pt}%
\definecolor{currentstroke}{rgb}{0.000000,0.000000,1.000000}%
\pgfsetstrokecolor{currentstroke}%
\pgfsetdash{}{0pt}%
\pgfpathmoveto{\pgfqpoint{2.883200in}{3.212692in}}%
\pgfpathlineto{\pgfqpoint{2.883200in}{3.329747in}}%
\pgfusepath{stroke}%
\end{pgfscope}%
\begin{pgfscope}%
\pgfpathrectangle{\pgfqpoint{0.800000in}{0.528000in}}{\pgfqpoint{4.960000in}{3.696000in}}%
\pgfusepath{clip}%
\pgfsetbuttcap%
\pgfsetroundjoin%
\pgfsetlinewidth{0.501875pt}%
\definecolor{currentstroke}{rgb}{0.000000,0.000000,1.000000}%
\pgfsetstrokecolor{currentstroke}%
\pgfsetdash{}{0pt}%
\pgfpathmoveto{\pgfqpoint{2.901236in}{3.212692in}}%
\pgfpathlineto{\pgfqpoint{2.901236in}{3.329747in}}%
\pgfusepath{stroke}%
\end{pgfscope}%
\begin{pgfscope}%
\pgfpathrectangle{\pgfqpoint{0.800000in}{0.528000in}}{\pgfqpoint{4.960000in}{3.696000in}}%
\pgfusepath{clip}%
\pgfsetbuttcap%
\pgfsetroundjoin%
\pgfsetlinewidth{0.501875pt}%
\definecolor{currentstroke}{rgb}{0.000000,0.000000,1.000000}%
\pgfsetstrokecolor{currentstroke}%
\pgfsetdash{}{0pt}%
\pgfpathmoveto{\pgfqpoint{2.919273in}{3.037108in}}%
\pgfpathlineto{\pgfqpoint{2.919273in}{3.154164in}}%
\pgfusepath{stroke}%
\end{pgfscope}%
\begin{pgfscope}%
\pgfpathrectangle{\pgfqpoint{0.800000in}{0.528000in}}{\pgfqpoint{4.960000in}{3.696000in}}%
\pgfusepath{clip}%
\pgfsetbuttcap%
\pgfsetroundjoin%
\pgfsetlinewidth{0.501875pt}%
\definecolor{currentstroke}{rgb}{0.000000,0.000000,1.000000}%
\pgfsetstrokecolor{currentstroke}%
\pgfsetdash{}{0pt}%
\pgfpathmoveto{\pgfqpoint{2.937309in}{2.685942in}}%
\pgfpathlineto{\pgfqpoint{2.937309in}{2.802998in}}%
\pgfusepath{stroke}%
\end{pgfscope}%
\begin{pgfscope}%
\pgfpathrectangle{\pgfqpoint{0.800000in}{0.528000in}}{\pgfqpoint{4.960000in}{3.696000in}}%
\pgfusepath{clip}%
\pgfsetbuttcap%
\pgfsetroundjoin%
\pgfsetlinewidth{0.501875pt}%
\definecolor{currentstroke}{rgb}{0.000000,0.000000,1.000000}%
\pgfsetstrokecolor{currentstroke}%
\pgfsetdash{}{0pt}%
\pgfpathmoveto{\pgfqpoint{2.955345in}{2.393304in}}%
\pgfpathlineto{\pgfqpoint{2.955345in}{2.510359in}}%
\pgfusepath{stroke}%
\end{pgfscope}%
\begin{pgfscope}%
\pgfpathrectangle{\pgfqpoint{0.800000in}{0.528000in}}{\pgfqpoint{4.960000in}{3.696000in}}%
\pgfusepath{clip}%
\pgfsetbuttcap%
\pgfsetroundjoin%
\pgfsetlinewidth{0.501875pt}%
\definecolor{currentstroke}{rgb}{0.000000,0.000000,1.000000}%
\pgfsetstrokecolor{currentstroke}%
\pgfsetdash{}{0pt}%
\pgfpathmoveto{\pgfqpoint{2.973382in}{2.042137in}}%
\pgfpathlineto{\pgfqpoint{2.973382in}{2.159193in}}%
\pgfusepath{stroke}%
\end{pgfscope}%
\begin{pgfscope}%
\pgfpathrectangle{\pgfqpoint{0.800000in}{0.528000in}}{\pgfqpoint{4.960000in}{3.696000in}}%
\pgfusepath{clip}%
\pgfsetbuttcap%
\pgfsetroundjoin%
\pgfsetlinewidth{0.501875pt}%
\definecolor{currentstroke}{rgb}{0.000000,0.000000,1.000000}%
\pgfsetstrokecolor{currentstroke}%
\pgfsetdash{}{0pt}%
\pgfpathmoveto{\pgfqpoint{2.991418in}{1.749499in}}%
\pgfpathlineto{\pgfqpoint{2.991418in}{1.866554in}}%
\pgfusepath{stroke}%
\end{pgfscope}%
\begin{pgfscope}%
\pgfpathrectangle{\pgfqpoint{0.800000in}{0.528000in}}{\pgfqpoint{4.960000in}{3.696000in}}%
\pgfusepath{clip}%
\pgfsetbuttcap%
\pgfsetroundjoin%
\pgfsetlinewidth{0.501875pt}%
\definecolor{currentstroke}{rgb}{0.000000,0.000000,1.000000}%
\pgfsetstrokecolor{currentstroke}%
\pgfsetdash{}{0pt}%
\pgfpathmoveto{\pgfqpoint{3.009455in}{1.398333in}}%
\pgfpathlineto{\pgfqpoint{3.009455in}{1.515388in}}%
\pgfusepath{stroke}%
\end{pgfscope}%
\begin{pgfscope}%
\pgfpathrectangle{\pgfqpoint{0.800000in}{0.528000in}}{\pgfqpoint{4.960000in}{3.696000in}}%
\pgfusepath{clip}%
\pgfsetbuttcap%
\pgfsetroundjoin%
\pgfsetlinewidth{0.501875pt}%
\definecolor{currentstroke}{rgb}{0.000000,0.000000,1.000000}%
\pgfsetstrokecolor{currentstroke}%
\pgfsetdash{}{0pt}%
\pgfpathmoveto{\pgfqpoint{3.027491in}{1.398333in}}%
\pgfpathlineto{\pgfqpoint{3.027491in}{1.515388in}}%
\pgfusepath{stroke}%
\end{pgfscope}%
\begin{pgfscope}%
\pgfpathrectangle{\pgfqpoint{0.800000in}{0.528000in}}{\pgfqpoint{4.960000in}{3.696000in}}%
\pgfusepath{clip}%
\pgfsetbuttcap%
\pgfsetroundjoin%
\pgfsetlinewidth{0.501875pt}%
\definecolor{currentstroke}{rgb}{0.000000,0.000000,1.000000}%
\pgfsetstrokecolor{currentstroke}%
\pgfsetdash{}{0pt}%
\pgfpathmoveto{\pgfqpoint{3.045527in}{1.573916in}}%
\pgfpathlineto{\pgfqpoint{3.045527in}{1.690971in}}%
\pgfusepath{stroke}%
\end{pgfscope}%
\begin{pgfscope}%
\pgfpathrectangle{\pgfqpoint{0.800000in}{0.528000in}}{\pgfqpoint{4.960000in}{3.696000in}}%
\pgfusepath{clip}%
\pgfsetbuttcap%
\pgfsetroundjoin%
\pgfsetlinewidth{0.501875pt}%
\definecolor{currentstroke}{rgb}{0.000000,0.000000,1.000000}%
\pgfsetstrokecolor{currentstroke}%
\pgfsetdash{}{0pt}%
\pgfpathmoveto{\pgfqpoint{3.063564in}{1.983610in}}%
\pgfpathlineto{\pgfqpoint{3.063564in}{2.100665in}}%
\pgfusepath{stroke}%
\end{pgfscope}%
\begin{pgfscope}%
\pgfpathrectangle{\pgfqpoint{0.800000in}{0.528000in}}{\pgfqpoint{4.960000in}{3.696000in}}%
\pgfusepath{clip}%
\pgfsetbuttcap%
\pgfsetroundjoin%
\pgfsetlinewidth{0.501875pt}%
\definecolor{currentstroke}{rgb}{0.000000,0.000000,1.000000}%
\pgfsetstrokecolor{currentstroke}%
\pgfsetdash{}{0pt}%
\pgfpathmoveto{\pgfqpoint{3.081600in}{2.334776in}}%
\pgfpathlineto{\pgfqpoint{3.081600in}{2.451831in}}%
\pgfusepath{stroke}%
\end{pgfscope}%
\begin{pgfscope}%
\pgfpathrectangle{\pgfqpoint{0.800000in}{0.528000in}}{\pgfqpoint{4.960000in}{3.696000in}}%
\pgfusepath{clip}%
\pgfsetbuttcap%
\pgfsetroundjoin%
\pgfsetlinewidth{0.501875pt}%
\definecolor{currentstroke}{rgb}{0.000000,0.000000,1.000000}%
\pgfsetstrokecolor{currentstroke}%
\pgfsetdash{}{0pt}%
\pgfpathmoveto{\pgfqpoint{3.099636in}{2.627414in}}%
\pgfpathlineto{\pgfqpoint{3.099636in}{2.744470in}}%
\pgfusepath{stroke}%
\end{pgfscope}%
\begin{pgfscope}%
\pgfpathrectangle{\pgfqpoint{0.800000in}{0.528000in}}{\pgfqpoint{4.960000in}{3.696000in}}%
\pgfusepath{clip}%
\pgfsetbuttcap%
\pgfsetroundjoin%
\pgfsetlinewidth{0.501875pt}%
\definecolor{currentstroke}{rgb}{0.000000,0.000000,1.000000}%
\pgfsetstrokecolor{currentstroke}%
\pgfsetdash{}{0pt}%
\pgfpathmoveto{\pgfqpoint{3.117673in}{2.920053in}}%
\pgfpathlineto{\pgfqpoint{3.117673in}{3.037108in}}%
\pgfusepath{stroke}%
\end{pgfscope}%
\begin{pgfscope}%
\pgfpathrectangle{\pgfqpoint{0.800000in}{0.528000in}}{\pgfqpoint{4.960000in}{3.696000in}}%
\pgfusepath{clip}%
\pgfsetbuttcap%
\pgfsetroundjoin%
\pgfsetlinewidth{0.501875pt}%
\definecolor{currentstroke}{rgb}{0.000000,0.000000,1.000000}%
\pgfsetstrokecolor{currentstroke}%
\pgfsetdash{}{0pt}%
\pgfpathmoveto{\pgfqpoint{3.135709in}{3.095636in}}%
\pgfpathlineto{\pgfqpoint{3.135709in}{3.212692in}}%
\pgfusepath{stroke}%
\end{pgfscope}%
\begin{pgfscope}%
\pgfpathrectangle{\pgfqpoint{0.800000in}{0.528000in}}{\pgfqpoint{4.960000in}{3.696000in}}%
\pgfusepath{clip}%
\pgfsetbuttcap%
\pgfsetroundjoin%
\pgfsetlinewidth{0.501875pt}%
\definecolor{currentstroke}{rgb}{0.000000,0.000000,1.000000}%
\pgfsetstrokecolor{currentstroke}%
\pgfsetdash{}{0pt}%
\pgfpathmoveto{\pgfqpoint{3.153745in}{3.154164in}}%
\pgfpathlineto{\pgfqpoint{3.153745in}{3.271219in}}%
\pgfusepath{stroke}%
\end{pgfscope}%
\begin{pgfscope}%
\pgfpathrectangle{\pgfqpoint{0.800000in}{0.528000in}}{\pgfqpoint{4.960000in}{3.696000in}}%
\pgfusepath{clip}%
\pgfsetbuttcap%
\pgfsetroundjoin%
\pgfsetlinewidth{0.501875pt}%
\definecolor{currentstroke}{rgb}{0.000000,0.000000,1.000000}%
\pgfsetstrokecolor{currentstroke}%
\pgfsetdash{}{0pt}%
\pgfpathmoveto{\pgfqpoint{3.171782in}{3.095636in}}%
\pgfpathlineto{\pgfqpoint{3.171782in}{3.212692in}}%
\pgfusepath{stroke}%
\end{pgfscope}%
\begin{pgfscope}%
\pgfpathrectangle{\pgfqpoint{0.800000in}{0.528000in}}{\pgfqpoint{4.960000in}{3.696000in}}%
\pgfusepath{clip}%
\pgfsetbuttcap%
\pgfsetroundjoin%
\pgfsetlinewidth{0.501875pt}%
\definecolor{currentstroke}{rgb}{0.000000,0.000000,1.000000}%
\pgfsetstrokecolor{currentstroke}%
\pgfsetdash{}{0pt}%
\pgfpathmoveto{\pgfqpoint{3.189818in}{2.861525in}}%
\pgfpathlineto{\pgfqpoint{3.189818in}{2.978581in}}%
\pgfusepath{stroke}%
\end{pgfscope}%
\begin{pgfscope}%
\pgfpathrectangle{\pgfqpoint{0.800000in}{0.528000in}}{\pgfqpoint{4.960000in}{3.696000in}}%
\pgfusepath{clip}%
\pgfsetbuttcap%
\pgfsetroundjoin%
\pgfsetlinewidth{0.501875pt}%
\definecolor{currentstroke}{rgb}{0.000000,0.000000,1.000000}%
\pgfsetstrokecolor{currentstroke}%
\pgfsetdash{}{0pt}%
\pgfpathmoveto{\pgfqpoint{3.207855in}{2.568887in}}%
\pgfpathlineto{\pgfqpoint{3.207855in}{2.685942in}}%
\pgfusepath{stroke}%
\end{pgfscope}%
\begin{pgfscope}%
\pgfpathrectangle{\pgfqpoint{0.800000in}{0.528000in}}{\pgfqpoint{4.960000in}{3.696000in}}%
\pgfusepath{clip}%
\pgfsetbuttcap%
\pgfsetroundjoin%
\pgfsetlinewidth{0.501875pt}%
\definecolor{currentstroke}{rgb}{0.000000,0.000000,1.000000}%
\pgfsetstrokecolor{currentstroke}%
\pgfsetdash{}{0pt}%
\pgfpathmoveto{\pgfqpoint{3.225891in}{2.276248in}}%
\pgfpathlineto{\pgfqpoint{3.225891in}{2.393304in}}%
\pgfusepath{stroke}%
\end{pgfscope}%
\begin{pgfscope}%
\pgfpathrectangle{\pgfqpoint{0.800000in}{0.528000in}}{\pgfqpoint{4.960000in}{3.696000in}}%
\pgfusepath{clip}%
\pgfsetbuttcap%
\pgfsetroundjoin%
\pgfsetlinewidth{0.501875pt}%
\definecolor{currentstroke}{rgb}{0.000000,0.000000,1.000000}%
\pgfsetstrokecolor{currentstroke}%
\pgfsetdash{}{0pt}%
\pgfpathmoveto{\pgfqpoint{3.243927in}{1.925082in}}%
\pgfpathlineto{\pgfqpoint{3.243927in}{2.042137in}}%
\pgfusepath{stroke}%
\end{pgfscope}%
\begin{pgfscope}%
\pgfpathrectangle{\pgfqpoint{0.800000in}{0.528000in}}{\pgfqpoint{4.960000in}{3.696000in}}%
\pgfusepath{clip}%
\pgfsetbuttcap%
\pgfsetroundjoin%
\pgfsetlinewidth{0.501875pt}%
\definecolor{currentstroke}{rgb}{0.000000,0.000000,1.000000}%
\pgfsetstrokecolor{currentstroke}%
\pgfsetdash{}{0pt}%
\pgfpathmoveto{\pgfqpoint{3.261964in}{1.573916in}}%
\pgfpathlineto{\pgfqpoint{3.261964in}{1.690971in}}%
\pgfusepath{stroke}%
\end{pgfscope}%
\begin{pgfscope}%
\pgfpathrectangle{\pgfqpoint{0.800000in}{0.528000in}}{\pgfqpoint{4.960000in}{3.696000in}}%
\pgfusepath{clip}%
\pgfsetbuttcap%
\pgfsetroundjoin%
\pgfsetlinewidth{0.501875pt}%
\definecolor{currentstroke}{rgb}{0.000000,0.000000,1.000000}%
\pgfsetstrokecolor{currentstroke}%
\pgfsetdash{}{0pt}%
\pgfpathmoveto{\pgfqpoint{3.280000in}{1.456860in}}%
\pgfpathlineto{\pgfqpoint{3.280000in}{1.573916in}}%
\pgfusepath{stroke}%
\end{pgfscope}%
\begin{pgfscope}%
\pgfpathrectangle{\pgfqpoint{0.800000in}{0.528000in}}{\pgfqpoint{4.960000in}{3.696000in}}%
\pgfusepath{clip}%
\pgfsetbuttcap%
\pgfsetroundjoin%
\pgfsetlinewidth{0.501875pt}%
\definecolor{currentstroke}{rgb}{0.000000,0.000000,1.000000}%
\pgfsetstrokecolor{currentstroke}%
\pgfsetdash{}{0pt}%
\pgfpathmoveto{\pgfqpoint{3.298036in}{1.515388in}}%
\pgfpathlineto{\pgfqpoint{3.298036in}{1.632443in}}%
\pgfusepath{stroke}%
\end{pgfscope}%
\begin{pgfscope}%
\pgfpathrectangle{\pgfqpoint{0.800000in}{0.528000in}}{\pgfqpoint{4.960000in}{3.696000in}}%
\pgfusepath{clip}%
\pgfsetbuttcap%
\pgfsetroundjoin%
\pgfsetlinewidth{0.501875pt}%
\definecolor{currentstroke}{rgb}{0.000000,0.000000,1.000000}%
\pgfsetstrokecolor{currentstroke}%
\pgfsetdash{}{0pt}%
\pgfpathmoveto{\pgfqpoint{3.316073in}{1.690971in}}%
\pgfpathlineto{\pgfqpoint{3.316073in}{1.808026in}}%
\pgfusepath{stroke}%
\end{pgfscope}%
\begin{pgfscope}%
\pgfpathrectangle{\pgfqpoint{0.800000in}{0.528000in}}{\pgfqpoint{4.960000in}{3.696000in}}%
\pgfusepath{clip}%
\pgfsetbuttcap%
\pgfsetroundjoin%
\pgfsetlinewidth{0.501875pt}%
\definecolor{currentstroke}{rgb}{0.000000,0.000000,1.000000}%
\pgfsetstrokecolor{currentstroke}%
\pgfsetdash{}{0pt}%
\pgfpathmoveto{\pgfqpoint{3.334109in}{2.100665in}}%
\pgfpathlineto{\pgfqpoint{3.334109in}{2.217720in}}%
\pgfusepath{stroke}%
\end{pgfscope}%
\begin{pgfscope}%
\pgfpathrectangle{\pgfqpoint{0.800000in}{0.528000in}}{\pgfqpoint{4.960000in}{3.696000in}}%
\pgfusepath{clip}%
\pgfsetbuttcap%
\pgfsetroundjoin%
\pgfsetlinewidth{0.501875pt}%
\definecolor{currentstroke}{rgb}{0.000000,0.000000,1.000000}%
\pgfsetstrokecolor{currentstroke}%
\pgfsetdash{}{0pt}%
\pgfpathmoveto{\pgfqpoint{3.352145in}{2.451831in}}%
\pgfpathlineto{\pgfqpoint{3.352145in}{2.568887in}}%
\pgfusepath{stroke}%
\end{pgfscope}%
\begin{pgfscope}%
\pgfpathrectangle{\pgfqpoint{0.800000in}{0.528000in}}{\pgfqpoint{4.960000in}{3.696000in}}%
\pgfusepath{clip}%
\pgfsetbuttcap%
\pgfsetroundjoin%
\pgfsetlinewidth{0.501875pt}%
\definecolor{currentstroke}{rgb}{0.000000,0.000000,1.000000}%
\pgfsetstrokecolor{currentstroke}%
\pgfsetdash{}{0pt}%
\pgfpathmoveto{\pgfqpoint{3.370182in}{2.744470in}}%
\pgfpathlineto{\pgfqpoint{3.370182in}{2.861525in}}%
\pgfusepath{stroke}%
\end{pgfscope}%
\begin{pgfscope}%
\pgfpathrectangle{\pgfqpoint{0.800000in}{0.528000in}}{\pgfqpoint{4.960000in}{3.696000in}}%
\pgfusepath{clip}%
\pgfsetbuttcap%
\pgfsetroundjoin%
\pgfsetlinewidth{0.501875pt}%
\definecolor{currentstroke}{rgb}{0.000000,0.000000,1.000000}%
\pgfsetstrokecolor{currentstroke}%
\pgfsetdash{}{0pt}%
\pgfpathmoveto{\pgfqpoint{3.388218in}{2.978581in}}%
\pgfpathlineto{\pgfqpoint{3.388218in}{3.095636in}}%
\pgfusepath{stroke}%
\end{pgfscope}%
\begin{pgfscope}%
\pgfpathrectangle{\pgfqpoint{0.800000in}{0.528000in}}{\pgfqpoint{4.960000in}{3.696000in}}%
\pgfusepath{clip}%
\pgfsetbuttcap%
\pgfsetroundjoin%
\pgfsetlinewidth{0.501875pt}%
\definecolor{currentstroke}{rgb}{0.000000,0.000000,1.000000}%
\pgfsetstrokecolor{currentstroke}%
\pgfsetdash{}{0pt}%
\pgfpathmoveto{\pgfqpoint{3.406255in}{3.095636in}}%
\pgfpathlineto{\pgfqpoint{3.406255in}{3.212692in}}%
\pgfusepath{stroke}%
\end{pgfscope}%
\begin{pgfscope}%
\pgfpathrectangle{\pgfqpoint{0.800000in}{0.528000in}}{\pgfqpoint{4.960000in}{3.696000in}}%
\pgfusepath{clip}%
\pgfsetbuttcap%
\pgfsetroundjoin%
\pgfsetlinewidth{0.501875pt}%
\definecolor{currentstroke}{rgb}{0.000000,0.000000,1.000000}%
\pgfsetstrokecolor{currentstroke}%
\pgfsetdash{}{0pt}%
\pgfpathmoveto{\pgfqpoint{3.424291in}{3.095636in}}%
\pgfpathlineto{\pgfqpoint{3.424291in}{3.212692in}}%
\pgfusepath{stroke}%
\end{pgfscope}%
\begin{pgfscope}%
\pgfpathrectangle{\pgfqpoint{0.800000in}{0.528000in}}{\pgfqpoint{4.960000in}{3.696000in}}%
\pgfusepath{clip}%
\pgfsetbuttcap%
\pgfsetroundjoin%
\pgfsetlinewidth{0.501875pt}%
\definecolor{currentstroke}{rgb}{0.000000,0.000000,1.000000}%
\pgfsetstrokecolor{currentstroke}%
\pgfsetdash{}{0pt}%
\pgfpathmoveto{\pgfqpoint{3.442327in}{2.978581in}}%
\pgfpathlineto{\pgfqpoint{3.442327in}{3.095636in}}%
\pgfusepath{stroke}%
\end{pgfscope}%
\begin{pgfscope}%
\pgfpathrectangle{\pgfqpoint{0.800000in}{0.528000in}}{\pgfqpoint{4.960000in}{3.696000in}}%
\pgfusepath{clip}%
\pgfsetbuttcap%
\pgfsetroundjoin%
\pgfsetlinewidth{0.501875pt}%
\definecolor{currentstroke}{rgb}{0.000000,0.000000,1.000000}%
\pgfsetstrokecolor{currentstroke}%
\pgfsetdash{}{0pt}%
\pgfpathmoveto{\pgfqpoint{3.460364in}{2.744470in}}%
\pgfpathlineto{\pgfqpoint{3.460364in}{2.861525in}}%
\pgfusepath{stroke}%
\end{pgfscope}%
\begin{pgfscope}%
\pgfpathrectangle{\pgfqpoint{0.800000in}{0.528000in}}{\pgfqpoint{4.960000in}{3.696000in}}%
\pgfusepath{clip}%
\pgfsetbuttcap%
\pgfsetroundjoin%
\pgfsetlinewidth{0.501875pt}%
\definecolor{currentstroke}{rgb}{0.000000,0.000000,1.000000}%
\pgfsetstrokecolor{currentstroke}%
\pgfsetdash{}{0pt}%
\pgfpathmoveto{\pgfqpoint{3.478400in}{2.451831in}}%
\pgfpathlineto{\pgfqpoint{3.478400in}{2.568887in}}%
\pgfusepath{stroke}%
\end{pgfscope}%
\begin{pgfscope}%
\pgfpathrectangle{\pgfqpoint{0.800000in}{0.528000in}}{\pgfqpoint{4.960000in}{3.696000in}}%
\pgfusepath{clip}%
\pgfsetbuttcap%
\pgfsetroundjoin%
\pgfsetlinewidth{0.501875pt}%
\definecolor{currentstroke}{rgb}{0.000000,0.000000,1.000000}%
\pgfsetstrokecolor{currentstroke}%
\pgfsetdash{}{0pt}%
\pgfpathmoveto{\pgfqpoint{3.496436in}{2.159193in}}%
\pgfpathlineto{\pgfqpoint{3.496436in}{2.276248in}}%
\pgfusepath{stroke}%
\end{pgfscope}%
\begin{pgfscope}%
\pgfpathrectangle{\pgfqpoint{0.800000in}{0.528000in}}{\pgfqpoint{4.960000in}{3.696000in}}%
\pgfusepath{clip}%
\pgfsetbuttcap%
\pgfsetroundjoin%
\pgfsetlinewidth{0.501875pt}%
\definecolor{currentstroke}{rgb}{0.000000,0.000000,1.000000}%
\pgfsetstrokecolor{currentstroke}%
\pgfsetdash{}{0pt}%
\pgfpathmoveto{\pgfqpoint{3.514473in}{1.866554in}}%
\pgfpathlineto{\pgfqpoint{3.514473in}{1.983610in}}%
\pgfusepath{stroke}%
\end{pgfscope}%
\begin{pgfscope}%
\pgfpathrectangle{\pgfqpoint{0.800000in}{0.528000in}}{\pgfqpoint{4.960000in}{3.696000in}}%
\pgfusepath{clip}%
\pgfsetbuttcap%
\pgfsetroundjoin%
\pgfsetlinewidth{0.501875pt}%
\definecolor{currentstroke}{rgb}{0.000000,0.000000,1.000000}%
\pgfsetstrokecolor{currentstroke}%
\pgfsetdash{}{0pt}%
\pgfpathmoveto{\pgfqpoint{3.532509in}{1.573916in}}%
\pgfpathlineto{\pgfqpoint{3.532509in}{1.690971in}}%
\pgfusepath{stroke}%
\end{pgfscope}%
\begin{pgfscope}%
\pgfpathrectangle{\pgfqpoint{0.800000in}{0.528000in}}{\pgfqpoint{4.960000in}{3.696000in}}%
\pgfusepath{clip}%
\pgfsetbuttcap%
\pgfsetroundjoin%
\pgfsetlinewidth{0.501875pt}%
\definecolor{currentstroke}{rgb}{0.000000,0.000000,1.000000}%
\pgfsetstrokecolor{currentstroke}%
\pgfsetdash{}{0pt}%
\pgfpathmoveto{\pgfqpoint{3.550545in}{1.515388in}}%
\pgfpathlineto{\pgfqpoint{3.550545in}{1.632443in}}%
\pgfusepath{stroke}%
\end{pgfscope}%
\begin{pgfscope}%
\pgfpathrectangle{\pgfqpoint{0.800000in}{0.528000in}}{\pgfqpoint{4.960000in}{3.696000in}}%
\pgfusepath{clip}%
\pgfsetbuttcap%
\pgfsetroundjoin%
\pgfsetlinewidth{0.501875pt}%
\definecolor{currentstroke}{rgb}{0.000000,0.000000,1.000000}%
\pgfsetstrokecolor{currentstroke}%
\pgfsetdash{}{0pt}%
\pgfpathmoveto{\pgfqpoint{3.568582in}{1.573916in}}%
\pgfpathlineto{\pgfqpoint{3.568582in}{1.690971in}}%
\pgfusepath{stroke}%
\end{pgfscope}%
\begin{pgfscope}%
\pgfpathrectangle{\pgfqpoint{0.800000in}{0.528000in}}{\pgfqpoint{4.960000in}{3.696000in}}%
\pgfusepath{clip}%
\pgfsetbuttcap%
\pgfsetroundjoin%
\pgfsetlinewidth{0.501875pt}%
\definecolor{currentstroke}{rgb}{0.000000,0.000000,1.000000}%
\pgfsetstrokecolor{currentstroke}%
\pgfsetdash{}{0pt}%
\pgfpathmoveto{\pgfqpoint{3.586618in}{1.925082in}}%
\pgfpathlineto{\pgfqpoint{3.586618in}{2.042137in}}%
\pgfusepath{stroke}%
\end{pgfscope}%
\begin{pgfscope}%
\pgfpathrectangle{\pgfqpoint{0.800000in}{0.528000in}}{\pgfqpoint{4.960000in}{3.696000in}}%
\pgfusepath{clip}%
\pgfsetbuttcap%
\pgfsetroundjoin%
\pgfsetlinewidth{0.501875pt}%
\definecolor{currentstroke}{rgb}{0.000000,0.000000,1.000000}%
\pgfsetstrokecolor{currentstroke}%
\pgfsetdash{}{0pt}%
\pgfpathmoveto{\pgfqpoint{3.604655in}{2.217720in}}%
\pgfpathlineto{\pgfqpoint{3.604655in}{2.334776in}}%
\pgfusepath{stroke}%
\end{pgfscope}%
\begin{pgfscope}%
\pgfpathrectangle{\pgfqpoint{0.800000in}{0.528000in}}{\pgfqpoint{4.960000in}{3.696000in}}%
\pgfusepath{clip}%
\pgfsetbuttcap%
\pgfsetroundjoin%
\pgfsetlinewidth{0.501875pt}%
\definecolor{currentstroke}{rgb}{0.000000,0.000000,1.000000}%
\pgfsetstrokecolor{currentstroke}%
\pgfsetdash{}{0pt}%
\pgfpathmoveto{\pgfqpoint{3.622691in}{2.568887in}}%
\pgfpathlineto{\pgfqpoint{3.622691in}{2.685942in}}%
\pgfusepath{stroke}%
\end{pgfscope}%
\begin{pgfscope}%
\pgfpathrectangle{\pgfqpoint{0.800000in}{0.528000in}}{\pgfqpoint{4.960000in}{3.696000in}}%
\pgfusepath{clip}%
\pgfsetbuttcap%
\pgfsetroundjoin%
\pgfsetlinewidth{0.501875pt}%
\definecolor{currentstroke}{rgb}{0.000000,0.000000,1.000000}%
\pgfsetstrokecolor{currentstroke}%
\pgfsetdash{}{0pt}%
\pgfpathmoveto{\pgfqpoint{3.640727in}{2.802998in}}%
\pgfpathlineto{\pgfqpoint{3.640727in}{2.920053in}}%
\pgfusepath{stroke}%
\end{pgfscope}%
\begin{pgfscope}%
\pgfpathrectangle{\pgfqpoint{0.800000in}{0.528000in}}{\pgfqpoint{4.960000in}{3.696000in}}%
\pgfusepath{clip}%
\pgfsetbuttcap%
\pgfsetroundjoin%
\pgfsetlinewidth{0.501875pt}%
\definecolor{currentstroke}{rgb}{0.000000,0.000000,1.000000}%
\pgfsetstrokecolor{currentstroke}%
\pgfsetdash{}{0pt}%
\pgfpathmoveto{\pgfqpoint{3.658764in}{2.978581in}}%
\pgfpathlineto{\pgfqpoint{3.658764in}{3.095636in}}%
\pgfusepath{stroke}%
\end{pgfscope}%
\begin{pgfscope}%
\pgfpathrectangle{\pgfqpoint{0.800000in}{0.528000in}}{\pgfqpoint{4.960000in}{3.696000in}}%
\pgfusepath{clip}%
\pgfsetbuttcap%
\pgfsetroundjoin%
\pgfsetlinewidth{0.501875pt}%
\definecolor{currentstroke}{rgb}{0.000000,0.000000,1.000000}%
\pgfsetstrokecolor{currentstroke}%
\pgfsetdash{}{0pt}%
\pgfpathmoveto{\pgfqpoint{3.676800in}{3.037108in}}%
\pgfpathlineto{\pgfqpoint{3.676800in}{3.154164in}}%
\pgfusepath{stroke}%
\end{pgfscope}%
\begin{pgfscope}%
\pgfpathrectangle{\pgfqpoint{0.800000in}{0.528000in}}{\pgfqpoint{4.960000in}{3.696000in}}%
\pgfusepath{clip}%
\pgfsetbuttcap%
\pgfsetroundjoin%
\pgfsetlinewidth{0.501875pt}%
\definecolor{currentstroke}{rgb}{0.000000,0.000000,1.000000}%
\pgfsetstrokecolor{currentstroke}%
\pgfsetdash{}{0pt}%
\pgfpathmoveto{\pgfqpoint{3.694836in}{3.037108in}}%
\pgfpathlineto{\pgfqpoint{3.694836in}{3.154164in}}%
\pgfusepath{stroke}%
\end{pgfscope}%
\begin{pgfscope}%
\pgfpathrectangle{\pgfqpoint{0.800000in}{0.528000in}}{\pgfqpoint{4.960000in}{3.696000in}}%
\pgfusepath{clip}%
\pgfsetbuttcap%
\pgfsetroundjoin%
\pgfsetlinewidth{0.501875pt}%
\definecolor{currentstroke}{rgb}{0.000000,0.000000,1.000000}%
\pgfsetstrokecolor{currentstroke}%
\pgfsetdash{}{0pt}%
\pgfpathmoveto{\pgfqpoint{3.712873in}{2.861525in}}%
\pgfpathlineto{\pgfqpoint{3.712873in}{2.978581in}}%
\pgfusepath{stroke}%
\end{pgfscope}%
\begin{pgfscope}%
\pgfpathrectangle{\pgfqpoint{0.800000in}{0.528000in}}{\pgfqpoint{4.960000in}{3.696000in}}%
\pgfusepath{clip}%
\pgfsetbuttcap%
\pgfsetroundjoin%
\pgfsetlinewidth{0.501875pt}%
\definecolor{currentstroke}{rgb}{0.000000,0.000000,1.000000}%
\pgfsetstrokecolor{currentstroke}%
\pgfsetdash{}{0pt}%
\pgfpathmoveto{\pgfqpoint{3.730909in}{2.627414in}}%
\pgfpathlineto{\pgfqpoint{3.730909in}{2.744470in}}%
\pgfusepath{stroke}%
\end{pgfscope}%
\begin{pgfscope}%
\pgfpathrectangle{\pgfqpoint{0.800000in}{0.528000in}}{\pgfqpoint{4.960000in}{3.696000in}}%
\pgfusepath{clip}%
\pgfsetbuttcap%
\pgfsetroundjoin%
\pgfsetlinewidth{0.501875pt}%
\definecolor{currentstroke}{rgb}{0.000000,0.000000,1.000000}%
\pgfsetstrokecolor{currentstroke}%
\pgfsetdash{}{0pt}%
\pgfpathmoveto{\pgfqpoint{3.748945in}{2.334776in}}%
\pgfpathlineto{\pgfqpoint{3.748945in}{2.451831in}}%
\pgfusepath{stroke}%
\end{pgfscope}%
\begin{pgfscope}%
\pgfpathrectangle{\pgfqpoint{0.800000in}{0.528000in}}{\pgfqpoint{4.960000in}{3.696000in}}%
\pgfusepath{clip}%
\pgfsetbuttcap%
\pgfsetroundjoin%
\pgfsetlinewidth{0.501875pt}%
\definecolor{currentstroke}{rgb}{0.000000,0.000000,1.000000}%
\pgfsetstrokecolor{currentstroke}%
\pgfsetdash{}{0pt}%
\pgfpathmoveto{\pgfqpoint{3.766982in}{2.100665in}}%
\pgfpathlineto{\pgfqpoint{3.766982in}{2.217720in}}%
\pgfusepath{stroke}%
\end{pgfscope}%
\begin{pgfscope}%
\pgfpathrectangle{\pgfqpoint{0.800000in}{0.528000in}}{\pgfqpoint{4.960000in}{3.696000in}}%
\pgfusepath{clip}%
\pgfsetbuttcap%
\pgfsetroundjoin%
\pgfsetlinewidth{0.501875pt}%
\definecolor{currentstroke}{rgb}{0.000000,0.000000,1.000000}%
\pgfsetstrokecolor{currentstroke}%
\pgfsetdash{}{0pt}%
\pgfpathmoveto{\pgfqpoint{3.785018in}{1.866554in}}%
\pgfpathlineto{\pgfqpoint{3.785018in}{1.983610in}}%
\pgfusepath{stroke}%
\end{pgfscope}%
\begin{pgfscope}%
\pgfpathrectangle{\pgfqpoint{0.800000in}{0.528000in}}{\pgfqpoint{4.960000in}{3.696000in}}%
\pgfusepath{clip}%
\pgfsetbuttcap%
\pgfsetroundjoin%
\pgfsetlinewidth{0.501875pt}%
\definecolor{currentstroke}{rgb}{0.000000,0.000000,1.000000}%
\pgfsetstrokecolor{currentstroke}%
\pgfsetdash{}{0pt}%
\pgfpathmoveto{\pgfqpoint{3.803055in}{1.749499in}}%
\pgfpathlineto{\pgfqpoint{3.803055in}{1.866554in}}%
\pgfusepath{stroke}%
\end{pgfscope}%
\begin{pgfscope}%
\pgfpathrectangle{\pgfqpoint{0.800000in}{0.528000in}}{\pgfqpoint{4.960000in}{3.696000in}}%
\pgfusepath{clip}%
\pgfsetbuttcap%
\pgfsetroundjoin%
\pgfsetlinewidth{0.501875pt}%
\definecolor{currentstroke}{rgb}{0.000000,0.000000,1.000000}%
\pgfsetstrokecolor{currentstroke}%
\pgfsetdash{}{0pt}%
\pgfpathmoveto{\pgfqpoint{3.821091in}{1.749499in}}%
\pgfpathlineto{\pgfqpoint{3.821091in}{1.866554in}}%
\pgfusepath{stroke}%
\end{pgfscope}%
\begin{pgfscope}%
\pgfpathrectangle{\pgfqpoint{0.800000in}{0.528000in}}{\pgfqpoint{4.960000in}{3.696000in}}%
\pgfusepath{clip}%
\pgfsetbuttcap%
\pgfsetroundjoin%
\pgfsetlinewidth{0.501875pt}%
\definecolor{currentstroke}{rgb}{0.000000,0.000000,1.000000}%
\pgfsetstrokecolor{currentstroke}%
\pgfsetdash{}{0pt}%
\pgfpathmoveto{\pgfqpoint{3.839127in}{1.866554in}}%
\pgfpathlineto{\pgfqpoint{3.839127in}{1.983610in}}%
\pgfusepath{stroke}%
\end{pgfscope}%
\begin{pgfscope}%
\pgfpathrectangle{\pgfqpoint{0.800000in}{0.528000in}}{\pgfqpoint{4.960000in}{3.696000in}}%
\pgfusepath{clip}%
\pgfsetbuttcap%
\pgfsetroundjoin%
\pgfsetlinewidth{0.501875pt}%
\definecolor{currentstroke}{rgb}{0.000000,0.000000,1.000000}%
\pgfsetstrokecolor{currentstroke}%
\pgfsetdash{}{0pt}%
\pgfpathmoveto{\pgfqpoint{3.857164in}{2.100665in}}%
\pgfpathlineto{\pgfqpoint{3.857164in}{2.217720in}}%
\pgfusepath{stroke}%
\end{pgfscope}%
\begin{pgfscope}%
\pgfpathrectangle{\pgfqpoint{0.800000in}{0.528000in}}{\pgfqpoint{4.960000in}{3.696000in}}%
\pgfusepath{clip}%
\pgfsetbuttcap%
\pgfsetroundjoin%
\pgfsetlinewidth{0.501875pt}%
\definecolor{currentstroke}{rgb}{0.000000,0.000000,1.000000}%
\pgfsetstrokecolor{currentstroke}%
\pgfsetdash{}{0pt}%
\pgfpathmoveto{\pgfqpoint{3.875200in}{2.393304in}}%
\pgfpathlineto{\pgfqpoint{3.875200in}{2.510359in}}%
\pgfusepath{stroke}%
\end{pgfscope}%
\begin{pgfscope}%
\pgfpathrectangle{\pgfqpoint{0.800000in}{0.528000in}}{\pgfqpoint{4.960000in}{3.696000in}}%
\pgfusepath{clip}%
\pgfsetbuttcap%
\pgfsetroundjoin%
\pgfsetlinewidth{0.501875pt}%
\definecolor{currentstroke}{rgb}{0.000000,0.000000,1.000000}%
\pgfsetstrokecolor{currentstroke}%
\pgfsetdash{}{0pt}%
\pgfpathmoveto{\pgfqpoint{3.893236in}{2.627414in}}%
\pgfpathlineto{\pgfqpoint{3.893236in}{2.744470in}}%
\pgfusepath{stroke}%
\end{pgfscope}%
\begin{pgfscope}%
\pgfpathrectangle{\pgfqpoint{0.800000in}{0.528000in}}{\pgfqpoint{4.960000in}{3.696000in}}%
\pgfusepath{clip}%
\pgfsetbuttcap%
\pgfsetroundjoin%
\pgfsetlinewidth{0.501875pt}%
\definecolor{currentstroke}{rgb}{0.000000,0.000000,1.000000}%
\pgfsetstrokecolor{currentstroke}%
\pgfsetdash{}{0pt}%
\pgfpathmoveto{\pgfqpoint{3.911273in}{2.861525in}}%
\pgfpathlineto{\pgfqpoint{3.911273in}{2.978581in}}%
\pgfusepath{stroke}%
\end{pgfscope}%
\begin{pgfscope}%
\pgfpathrectangle{\pgfqpoint{0.800000in}{0.528000in}}{\pgfqpoint{4.960000in}{3.696000in}}%
\pgfusepath{clip}%
\pgfsetbuttcap%
\pgfsetroundjoin%
\pgfsetlinewidth{0.501875pt}%
\definecolor{currentstroke}{rgb}{0.000000,0.000000,1.000000}%
\pgfsetstrokecolor{currentstroke}%
\pgfsetdash{}{0pt}%
\pgfpathmoveto{\pgfqpoint{3.929309in}{2.978581in}}%
\pgfpathlineto{\pgfqpoint{3.929309in}{3.095636in}}%
\pgfusepath{stroke}%
\end{pgfscope}%
\begin{pgfscope}%
\pgfpathrectangle{\pgfqpoint{0.800000in}{0.528000in}}{\pgfqpoint{4.960000in}{3.696000in}}%
\pgfusepath{clip}%
\pgfsetbuttcap%
\pgfsetroundjoin%
\pgfsetlinewidth{0.501875pt}%
\definecolor{currentstroke}{rgb}{0.000000,0.000000,1.000000}%
\pgfsetstrokecolor{currentstroke}%
\pgfsetdash{}{0pt}%
\pgfpathmoveto{\pgfqpoint{3.947345in}{3.037108in}}%
\pgfpathlineto{\pgfqpoint{3.947345in}{3.154164in}}%
\pgfusepath{stroke}%
\end{pgfscope}%
\begin{pgfscope}%
\pgfpathrectangle{\pgfqpoint{0.800000in}{0.528000in}}{\pgfqpoint{4.960000in}{3.696000in}}%
\pgfusepath{clip}%
\pgfsetbuttcap%
\pgfsetroundjoin%
\pgfsetlinewidth{0.501875pt}%
\definecolor{currentstroke}{rgb}{0.000000,0.000000,1.000000}%
\pgfsetstrokecolor{currentstroke}%
\pgfsetdash{}{0pt}%
\pgfpathmoveto{\pgfqpoint{3.965382in}{2.920053in}}%
\pgfpathlineto{\pgfqpoint{3.965382in}{3.037108in}}%
\pgfusepath{stroke}%
\end{pgfscope}%
\begin{pgfscope}%
\pgfpathrectangle{\pgfqpoint{0.800000in}{0.528000in}}{\pgfqpoint{4.960000in}{3.696000in}}%
\pgfusepath{clip}%
\pgfsetbuttcap%
\pgfsetroundjoin%
\pgfsetlinewidth{0.501875pt}%
\definecolor{currentstroke}{rgb}{0.000000,0.000000,1.000000}%
\pgfsetstrokecolor{currentstroke}%
\pgfsetdash{}{0pt}%
\pgfpathmoveto{\pgfqpoint{3.983418in}{2.744470in}}%
\pgfpathlineto{\pgfqpoint{3.983418in}{2.861525in}}%
\pgfusepath{stroke}%
\end{pgfscope}%
\begin{pgfscope}%
\pgfpathrectangle{\pgfqpoint{0.800000in}{0.528000in}}{\pgfqpoint{4.960000in}{3.696000in}}%
\pgfusepath{clip}%
\pgfsetbuttcap%
\pgfsetroundjoin%
\pgfsetlinewidth{0.501875pt}%
\definecolor{currentstroke}{rgb}{0.000000,0.000000,1.000000}%
\pgfsetstrokecolor{currentstroke}%
\pgfsetdash{}{0pt}%
\pgfpathmoveto{\pgfqpoint{4.001455in}{2.510359in}}%
\pgfpathlineto{\pgfqpoint{4.001455in}{2.627414in}}%
\pgfusepath{stroke}%
\end{pgfscope}%
\begin{pgfscope}%
\pgfpathrectangle{\pgfqpoint{0.800000in}{0.528000in}}{\pgfqpoint{4.960000in}{3.696000in}}%
\pgfusepath{clip}%
\pgfsetbuttcap%
\pgfsetroundjoin%
\pgfsetlinewidth{0.501875pt}%
\definecolor{currentstroke}{rgb}{0.000000,0.000000,1.000000}%
\pgfsetstrokecolor{currentstroke}%
\pgfsetdash{}{0pt}%
\pgfpathmoveto{\pgfqpoint{4.019491in}{2.276248in}}%
\pgfpathlineto{\pgfqpoint{4.019491in}{2.393304in}}%
\pgfusepath{stroke}%
\end{pgfscope}%
\begin{pgfscope}%
\pgfpathrectangle{\pgfqpoint{0.800000in}{0.528000in}}{\pgfqpoint{4.960000in}{3.696000in}}%
\pgfusepath{clip}%
\pgfsetbuttcap%
\pgfsetroundjoin%
\pgfsetlinewidth{0.501875pt}%
\definecolor{currentstroke}{rgb}{0.000000,0.000000,1.000000}%
\pgfsetstrokecolor{currentstroke}%
\pgfsetdash{}{0pt}%
\pgfpathmoveto{\pgfqpoint{4.037527in}{1.983610in}}%
\pgfpathlineto{\pgfqpoint{4.037527in}{2.100665in}}%
\pgfusepath{stroke}%
\end{pgfscope}%
\begin{pgfscope}%
\pgfpathrectangle{\pgfqpoint{0.800000in}{0.528000in}}{\pgfqpoint{4.960000in}{3.696000in}}%
\pgfusepath{clip}%
\pgfsetbuttcap%
\pgfsetroundjoin%
\pgfsetlinewidth{0.501875pt}%
\definecolor{currentstroke}{rgb}{0.000000,0.000000,1.000000}%
\pgfsetstrokecolor{currentstroke}%
\pgfsetdash{}{0pt}%
\pgfpathmoveto{\pgfqpoint{4.055564in}{1.808026in}}%
\pgfpathlineto{\pgfqpoint{4.055564in}{1.925082in}}%
\pgfusepath{stroke}%
\end{pgfscope}%
\begin{pgfscope}%
\pgfpathrectangle{\pgfqpoint{0.800000in}{0.528000in}}{\pgfqpoint{4.960000in}{3.696000in}}%
\pgfusepath{clip}%
\pgfsetbuttcap%
\pgfsetroundjoin%
\pgfsetlinewidth{0.501875pt}%
\definecolor{currentstroke}{rgb}{0.000000,0.000000,1.000000}%
\pgfsetstrokecolor{currentstroke}%
\pgfsetdash{}{0pt}%
\pgfpathmoveto{\pgfqpoint{4.073600in}{1.749499in}}%
\pgfpathlineto{\pgfqpoint{4.073600in}{1.866554in}}%
\pgfusepath{stroke}%
\end{pgfscope}%
\begin{pgfscope}%
\pgfpathrectangle{\pgfqpoint{0.800000in}{0.528000in}}{\pgfqpoint{4.960000in}{3.696000in}}%
\pgfusepath{clip}%
\pgfsetbuttcap%
\pgfsetroundjoin%
\pgfsetlinewidth{0.501875pt}%
\definecolor{currentstroke}{rgb}{0.000000,0.000000,1.000000}%
\pgfsetstrokecolor{currentstroke}%
\pgfsetdash{}{0pt}%
\pgfpathmoveto{\pgfqpoint{4.091636in}{1.808026in}}%
\pgfpathlineto{\pgfqpoint{4.091636in}{1.925082in}}%
\pgfusepath{stroke}%
\end{pgfscope}%
\begin{pgfscope}%
\pgfpathrectangle{\pgfqpoint{0.800000in}{0.528000in}}{\pgfqpoint{4.960000in}{3.696000in}}%
\pgfusepath{clip}%
\pgfsetbuttcap%
\pgfsetroundjoin%
\pgfsetlinewidth{0.501875pt}%
\definecolor{currentstroke}{rgb}{0.000000,0.000000,1.000000}%
\pgfsetstrokecolor{currentstroke}%
\pgfsetdash{}{0pt}%
\pgfpathmoveto{\pgfqpoint{4.109673in}{1.983610in}}%
\pgfpathlineto{\pgfqpoint{4.109673in}{2.100665in}}%
\pgfusepath{stroke}%
\end{pgfscope}%
\begin{pgfscope}%
\pgfpathrectangle{\pgfqpoint{0.800000in}{0.528000in}}{\pgfqpoint{4.960000in}{3.696000in}}%
\pgfusepath{clip}%
\pgfsetbuttcap%
\pgfsetroundjoin%
\pgfsetlinewidth{0.501875pt}%
\definecolor{currentstroke}{rgb}{0.000000,0.000000,1.000000}%
\pgfsetstrokecolor{currentstroke}%
\pgfsetdash{}{0pt}%
\pgfpathmoveto{\pgfqpoint{4.127709in}{2.217720in}}%
\pgfpathlineto{\pgfqpoint{4.127709in}{2.334776in}}%
\pgfusepath{stroke}%
\end{pgfscope}%
\begin{pgfscope}%
\pgfpathrectangle{\pgfqpoint{0.800000in}{0.528000in}}{\pgfqpoint{4.960000in}{3.696000in}}%
\pgfusepath{clip}%
\pgfsetbuttcap%
\pgfsetroundjoin%
\pgfsetlinewidth{0.501875pt}%
\definecolor{currentstroke}{rgb}{0.000000,0.000000,1.000000}%
\pgfsetstrokecolor{currentstroke}%
\pgfsetdash{}{0pt}%
\pgfpathmoveto{\pgfqpoint{4.145745in}{2.451831in}}%
\pgfpathlineto{\pgfqpoint{4.145745in}{2.568887in}}%
\pgfusepath{stroke}%
\end{pgfscope}%
\begin{pgfscope}%
\pgfpathrectangle{\pgfqpoint{0.800000in}{0.528000in}}{\pgfqpoint{4.960000in}{3.696000in}}%
\pgfusepath{clip}%
\pgfsetbuttcap%
\pgfsetroundjoin%
\pgfsetlinewidth{0.501875pt}%
\definecolor{currentstroke}{rgb}{0.000000,0.000000,1.000000}%
\pgfsetstrokecolor{currentstroke}%
\pgfsetdash{}{0pt}%
\pgfpathmoveto{\pgfqpoint{4.163782in}{2.685942in}}%
\pgfpathlineto{\pgfqpoint{4.163782in}{2.802998in}}%
\pgfusepath{stroke}%
\end{pgfscope}%
\begin{pgfscope}%
\pgfpathrectangle{\pgfqpoint{0.800000in}{0.528000in}}{\pgfqpoint{4.960000in}{3.696000in}}%
\pgfusepath{clip}%
\pgfsetbuttcap%
\pgfsetroundjoin%
\pgfsetlinewidth{0.501875pt}%
\definecolor{currentstroke}{rgb}{0.000000,0.000000,1.000000}%
\pgfsetstrokecolor{currentstroke}%
\pgfsetdash{}{0pt}%
\pgfpathmoveto{\pgfqpoint{4.181818in}{2.861525in}}%
\pgfpathlineto{\pgfqpoint{4.181818in}{2.978581in}}%
\pgfusepath{stroke}%
\end{pgfscope}%
\begin{pgfscope}%
\pgfpathrectangle{\pgfqpoint{0.800000in}{0.528000in}}{\pgfqpoint{4.960000in}{3.696000in}}%
\pgfusepath{clip}%
\pgfsetbuttcap%
\pgfsetroundjoin%
\pgfsetlinewidth{0.501875pt}%
\definecolor{currentstroke}{rgb}{0.000000,0.000000,1.000000}%
\pgfsetstrokecolor{currentstroke}%
\pgfsetdash{}{0pt}%
\pgfpathmoveto{\pgfqpoint{4.199855in}{2.978581in}}%
\pgfpathlineto{\pgfqpoint{4.199855in}{3.095636in}}%
\pgfusepath{stroke}%
\end{pgfscope}%
\begin{pgfscope}%
\pgfpathrectangle{\pgfqpoint{0.800000in}{0.528000in}}{\pgfqpoint{4.960000in}{3.696000in}}%
\pgfusepath{clip}%
\pgfsetbuttcap%
\pgfsetroundjoin%
\pgfsetlinewidth{0.501875pt}%
\definecolor{currentstroke}{rgb}{0.000000,0.000000,1.000000}%
\pgfsetstrokecolor{currentstroke}%
\pgfsetdash{}{0pt}%
\pgfpathmoveto{\pgfqpoint{4.217891in}{2.978581in}}%
\pgfpathlineto{\pgfqpoint{4.217891in}{3.095636in}}%
\pgfusepath{stroke}%
\end{pgfscope}%
\begin{pgfscope}%
\pgfpathrectangle{\pgfqpoint{0.800000in}{0.528000in}}{\pgfqpoint{4.960000in}{3.696000in}}%
\pgfusepath{clip}%
\pgfsetbuttcap%
\pgfsetroundjoin%
\pgfsetlinewidth{0.501875pt}%
\definecolor{currentstroke}{rgb}{0.000000,0.000000,1.000000}%
\pgfsetstrokecolor{currentstroke}%
\pgfsetdash{}{0pt}%
\pgfpathmoveto{\pgfqpoint{4.235927in}{2.861525in}}%
\pgfpathlineto{\pgfqpoint{4.235927in}{2.978581in}}%
\pgfusepath{stroke}%
\end{pgfscope}%
\begin{pgfscope}%
\pgfpathrectangle{\pgfqpoint{0.800000in}{0.528000in}}{\pgfqpoint{4.960000in}{3.696000in}}%
\pgfusepath{clip}%
\pgfsetbuttcap%
\pgfsetroundjoin%
\pgfsetlinewidth{0.501875pt}%
\definecolor{currentstroke}{rgb}{0.000000,0.000000,1.000000}%
\pgfsetstrokecolor{currentstroke}%
\pgfsetdash{}{0pt}%
\pgfpathmoveto{\pgfqpoint{4.253964in}{2.627414in}}%
\pgfpathlineto{\pgfqpoint{4.253964in}{2.744470in}}%
\pgfusepath{stroke}%
\end{pgfscope}%
\begin{pgfscope}%
\pgfpathrectangle{\pgfqpoint{0.800000in}{0.528000in}}{\pgfqpoint{4.960000in}{3.696000in}}%
\pgfusepath{clip}%
\pgfsetbuttcap%
\pgfsetroundjoin%
\pgfsetlinewidth{0.501875pt}%
\definecolor{currentstroke}{rgb}{0.000000,0.000000,1.000000}%
\pgfsetstrokecolor{currentstroke}%
\pgfsetdash{}{0pt}%
\pgfpathmoveto{\pgfqpoint{4.272000in}{2.393304in}}%
\pgfpathlineto{\pgfqpoint{4.272000in}{2.510359in}}%
\pgfusepath{stroke}%
\end{pgfscope}%
\begin{pgfscope}%
\pgfpathrectangle{\pgfqpoint{0.800000in}{0.528000in}}{\pgfqpoint{4.960000in}{3.696000in}}%
\pgfusepath{clip}%
\pgfsetbuttcap%
\pgfsetroundjoin%
\pgfsetlinewidth{0.501875pt}%
\definecolor{currentstroke}{rgb}{0.000000,0.000000,1.000000}%
\pgfsetstrokecolor{currentstroke}%
\pgfsetdash{}{0pt}%
\pgfpathmoveto{\pgfqpoint{4.290036in}{2.159193in}}%
\pgfpathlineto{\pgfqpoint{4.290036in}{2.276248in}}%
\pgfusepath{stroke}%
\end{pgfscope}%
\begin{pgfscope}%
\pgfpathrectangle{\pgfqpoint{0.800000in}{0.528000in}}{\pgfqpoint{4.960000in}{3.696000in}}%
\pgfusepath{clip}%
\pgfsetbuttcap%
\pgfsetroundjoin%
\pgfsetlinewidth{0.501875pt}%
\definecolor{currentstroke}{rgb}{0.000000,0.000000,1.000000}%
\pgfsetstrokecolor{currentstroke}%
\pgfsetdash{}{0pt}%
\pgfpathmoveto{\pgfqpoint{4.308073in}{1.983610in}}%
\pgfpathlineto{\pgfqpoint{4.308073in}{2.100665in}}%
\pgfusepath{stroke}%
\end{pgfscope}%
\begin{pgfscope}%
\pgfpathrectangle{\pgfqpoint{0.800000in}{0.528000in}}{\pgfqpoint{4.960000in}{3.696000in}}%
\pgfusepath{clip}%
\pgfsetbuttcap%
\pgfsetroundjoin%
\pgfsetlinewidth{0.501875pt}%
\definecolor{currentstroke}{rgb}{0.000000,0.000000,1.000000}%
\pgfsetstrokecolor{currentstroke}%
\pgfsetdash{}{0pt}%
\pgfpathmoveto{\pgfqpoint{4.326109in}{1.808026in}}%
\pgfpathlineto{\pgfqpoint{4.326109in}{1.925082in}}%
\pgfusepath{stroke}%
\end{pgfscope}%
\begin{pgfscope}%
\pgfpathrectangle{\pgfqpoint{0.800000in}{0.528000in}}{\pgfqpoint{4.960000in}{3.696000in}}%
\pgfusepath{clip}%
\pgfsetbuttcap%
\pgfsetroundjoin%
\pgfsetlinewidth{0.501875pt}%
\definecolor{currentstroke}{rgb}{0.000000,0.000000,1.000000}%
\pgfsetstrokecolor{currentstroke}%
\pgfsetdash{}{0pt}%
\pgfpathmoveto{\pgfqpoint{4.344145in}{1.808026in}}%
\pgfpathlineto{\pgfqpoint{4.344145in}{1.925082in}}%
\pgfusepath{stroke}%
\end{pgfscope}%
\begin{pgfscope}%
\pgfpathrectangle{\pgfqpoint{0.800000in}{0.528000in}}{\pgfqpoint{4.960000in}{3.696000in}}%
\pgfusepath{clip}%
\pgfsetbuttcap%
\pgfsetroundjoin%
\pgfsetlinewidth{0.501875pt}%
\definecolor{currentstroke}{rgb}{0.000000,0.000000,1.000000}%
\pgfsetstrokecolor{currentstroke}%
\pgfsetdash{}{0pt}%
\pgfpathmoveto{\pgfqpoint{4.362182in}{1.925082in}}%
\pgfpathlineto{\pgfqpoint{4.362182in}{2.042137in}}%
\pgfusepath{stroke}%
\end{pgfscope}%
\begin{pgfscope}%
\pgfpathrectangle{\pgfqpoint{0.800000in}{0.528000in}}{\pgfqpoint{4.960000in}{3.696000in}}%
\pgfusepath{clip}%
\pgfsetbuttcap%
\pgfsetroundjoin%
\pgfsetlinewidth{0.501875pt}%
\definecolor{currentstroke}{rgb}{0.000000,0.000000,1.000000}%
\pgfsetstrokecolor{currentstroke}%
\pgfsetdash{}{0pt}%
\pgfpathmoveto{\pgfqpoint{4.380218in}{2.100665in}}%
\pgfpathlineto{\pgfqpoint{4.380218in}{2.217720in}}%
\pgfusepath{stroke}%
\end{pgfscope}%
\begin{pgfscope}%
\pgfpathrectangle{\pgfqpoint{0.800000in}{0.528000in}}{\pgfqpoint{4.960000in}{3.696000in}}%
\pgfusepath{clip}%
\pgfsetbuttcap%
\pgfsetroundjoin%
\pgfsetlinewidth{0.501875pt}%
\definecolor{currentstroke}{rgb}{0.000000,0.000000,1.000000}%
\pgfsetstrokecolor{currentstroke}%
\pgfsetdash{}{0pt}%
\pgfpathmoveto{\pgfqpoint{4.398255in}{2.334776in}}%
\pgfpathlineto{\pgfqpoint{4.398255in}{2.451831in}}%
\pgfusepath{stroke}%
\end{pgfscope}%
\begin{pgfscope}%
\pgfpathrectangle{\pgfqpoint{0.800000in}{0.528000in}}{\pgfqpoint{4.960000in}{3.696000in}}%
\pgfusepath{clip}%
\pgfsetbuttcap%
\pgfsetroundjoin%
\pgfsetlinewidth{0.501875pt}%
\definecolor{currentstroke}{rgb}{0.000000,0.000000,1.000000}%
\pgfsetstrokecolor{currentstroke}%
\pgfsetdash{}{0pt}%
\pgfpathmoveto{\pgfqpoint{4.416291in}{2.568887in}}%
\pgfpathlineto{\pgfqpoint{4.416291in}{2.685942in}}%
\pgfusepath{stroke}%
\end{pgfscope}%
\begin{pgfscope}%
\pgfpathrectangle{\pgfqpoint{0.800000in}{0.528000in}}{\pgfqpoint{4.960000in}{3.696000in}}%
\pgfusepath{clip}%
\pgfsetbuttcap%
\pgfsetroundjoin%
\pgfsetlinewidth{0.501875pt}%
\definecolor{currentstroke}{rgb}{0.000000,0.000000,1.000000}%
\pgfsetstrokecolor{currentstroke}%
\pgfsetdash{}{0pt}%
\pgfpathmoveto{\pgfqpoint{4.434327in}{2.744470in}}%
\pgfpathlineto{\pgfqpoint{4.434327in}{2.861525in}}%
\pgfusepath{stroke}%
\end{pgfscope}%
\begin{pgfscope}%
\pgfpathrectangle{\pgfqpoint{0.800000in}{0.528000in}}{\pgfqpoint{4.960000in}{3.696000in}}%
\pgfusepath{clip}%
\pgfsetbuttcap%
\pgfsetroundjoin%
\pgfsetlinewidth{0.501875pt}%
\definecolor{currentstroke}{rgb}{0.000000,0.000000,1.000000}%
\pgfsetstrokecolor{currentstroke}%
\pgfsetdash{}{0pt}%
\pgfpathmoveto{\pgfqpoint{4.452364in}{2.861525in}}%
\pgfpathlineto{\pgfqpoint{4.452364in}{2.978581in}}%
\pgfusepath{stroke}%
\end{pgfscope}%
\begin{pgfscope}%
\pgfpathrectangle{\pgfqpoint{0.800000in}{0.528000in}}{\pgfqpoint{4.960000in}{3.696000in}}%
\pgfusepath{clip}%
\pgfsetbuttcap%
\pgfsetroundjoin%
\pgfsetlinewidth{0.501875pt}%
\definecolor{currentstroke}{rgb}{0.000000,0.000000,1.000000}%
\pgfsetstrokecolor{currentstroke}%
\pgfsetdash{}{0pt}%
\pgfpathmoveto{\pgfqpoint{4.470400in}{2.920053in}}%
\pgfpathlineto{\pgfqpoint{4.470400in}{3.037108in}}%
\pgfusepath{stroke}%
\end{pgfscope}%
\begin{pgfscope}%
\pgfpathrectangle{\pgfqpoint{0.800000in}{0.528000in}}{\pgfqpoint{4.960000in}{3.696000in}}%
\pgfusepath{clip}%
\pgfsetbuttcap%
\pgfsetroundjoin%
\pgfsetlinewidth{0.501875pt}%
\definecolor{currentstroke}{rgb}{0.000000,0.000000,1.000000}%
\pgfsetstrokecolor{currentstroke}%
\pgfsetdash{}{0pt}%
\pgfpathmoveto{\pgfqpoint{4.488436in}{2.861525in}}%
\pgfpathlineto{\pgfqpoint{4.488436in}{2.978581in}}%
\pgfusepath{stroke}%
\end{pgfscope}%
\begin{pgfscope}%
\pgfpathrectangle{\pgfqpoint{0.800000in}{0.528000in}}{\pgfqpoint{4.960000in}{3.696000in}}%
\pgfusepath{clip}%
\pgfsetbuttcap%
\pgfsetroundjoin%
\pgfsetlinewidth{0.501875pt}%
\definecolor{currentstroke}{rgb}{0.000000,0.000000,1.000000}%
\pgfsetstrokecolor{currentstroke}%
\pgfsetdash{}{0pt}%
\pgfpathmoveto{\pgfqpoint{4.506473in}{2.744470in}}%
\pgfpathlineto{\pgfqpoint{4.506473in}{2.861525in}}%
\pgfusepath{stroke}%
\end{pgfscope}%
\begin{pgfscope}%
\pgfpathrectangle{\pgfqpoint{0.800000in}{0.528000in}}{\pgfqpoint{4.960000in}{3.696000in}}%
\pgfusepath{clip}%
\pgfsetbuttcap%
\pgfsetroundjoin%
\pgfsetlinewidth{0.501875pt}%
\definecolor{currentstroke}{rgb}{0.000000,0.000000,1.000000}%
\pgfsetstrokecolor{currentstroke}%
\pgfsetdash{}{0pt}%
\pgfpathmoveto{\pgfqpoint{4.524509in}{2.568887in}}%
\pgfpathlineto{\pgfqpoint{4.524509in}{2.685942in}}%
\pgfusepath{stroke}%
\end{pgfscope}%
\begin{pgfscope}%
\pgfpathrectangle{\pgfqpoint{0.800000in}{0.528000in}}{\pgfqpoint{4.960000in}{3.696000in}}%
\pgfusepath{clip}%
\pgfsetbuttcap%
\pgfsetroundjoin%
\pgfsetlinewidth{0.501875pt}%
\definecolor{currentstroke}{rgb}{0.000000,0.000000,1.000000}%
\pgfsetstrokecolor{currentstroke}%
\pgfsetdash{}{0pt}%
\pgfpathmoveto{\pgfqpoint{4.542545in}{2.334776in}}%
\pgfpathlineto{\pgfqpoint{4.542545in}{2.451831in}}%
\pgfusepath{stroke}%
\end{pgfscope}%
\begin{pgfscope}%
\pgfpathrectangle{\pgfqpoint{0.800000in}{0.528000in}}{\pgfqpoint{4.960000in}{3.696000in}}%
\pgfusepath{clip}%
\pgfsetbuttcap%
\pgfsetroundjoin%
\pgfsetlinewidth{0.501875pt}%
\definecolor{currentstroke}{rgb}{0.000000,0.000000,1.000000}%
\pgfsetstrokecolor{currentstroke}%
\pgfsetdash{}{0pt}%
\pgfpathmoveto{\pgfqpoint{4.560582in}{2.100665in}}%
\pgfpathlineto{\pgfqpoint{4.560582in}{2.217720in}}%
\pgfusepath{stroke}%
\end{pgfscope}%
\begin{pgfscope}%
\pgfpathrectangle{\pgfqpoint{0.800000in}{0.528000in}}{\pgfqpoint{4.960000in}{3.696000in}}%
\pgfusepath{clip}%
\pgfsetbuttcap%
\pgfsetroundjoin%
\pgfsetlinewidth{0.501875pt}%
\definecolor{currentstroke}{rgb}{0.000000,0.000000,1.000000}%
\pgfsetstrokecolor{currentstroke}%
\pgfsetdash{}{0pt}%
\pgfpathmoveto{\pgfqpoint{4.578618in}{1.925082in}}%
\pgfpathlineto{\pgfqpoint{4.578618in}{2.042137in}}%
\pgfusepath{stroke}%
\end{pgfscope}%
\begin{pgfscope}%
\pgfpathrectangle{\pgfqpoint{0.800000in}{0.528000in}}{\pgfqpoint{4.960000in}{3.696000in}}%
\pgfusepath{clip}%
\pgfsetbuttcap%
\pgfsetroundjoin%
\pgfsetlinewidth{0.501875pt}%
\definecolor{currentstroke}{rgb}{0.000000,0.000000,1.000000}%
\pgfsetstrokecolor{currentstroke}%
\pgfsetdash{}{0pt}%
\pgfpathmoveto{\pgfqpoint{4.596655in}{1.866554in}}%
\pgfpathlineto{\pgfqpoint{4.596655in}{1.983610in}}%
\pgfusepath{stroke}%
\end{pgfscope}%
\begin{pgfscope}%
\pgfpathrectangle{\pgfqpoint{0.800000in}{0.528000in}}{\pgfqpoint{4.960000in}{3.696000in}}%
\pgfusepath{clip}%
\pgfsetbuttcap%
\pgfsetroundjoin%
\pgfsetlinewidth{0.501875pt}%
\definecolor{currentstroke}{rgb}{0.000000,0.000000,1.000000}%
\pgfsetstrokecolor{currentstroke}%
\pgfsetdash{}{0pt}%
\pgfpathmoveto{\pgfqpoint{4.614691in}{1.866554in}}%
\pgfpathlineto{\pgfqpoint{4.614691in}{1.983610in}}%
\pgfusepath{stroke}%
\end{pgfscope}%
\begin{pgfscope}%
\pgfpathrectangle{\pgfqpoint{0.800000in}{0.528000in}}{\pgfqpoint{4.960000in}{3.696000in}}%
\pgfusepath{clip}%
\pgfsetbuttcap%
\pgfsetroundjoin%
\pgfsetlinewidth{0.501875pt}%
\definecolor{currentstroke}{rgb}{0.000000,0.000000,1.000000}%
\pgfsetstrokecolor{currentstroke}%
\pgfsetdash{}{0pt}%
\pgfpathmoveto{\pgfqpoint{4.632727in}{1.983610in}}%
\pgfpathlineto{\pgfqpoint{4.632727in}{2.100665in}}%
\pgfusepath{stroke}%
\end{pgfscope}%
\begin{pgfscope}%
\pgfpathrectangle{\pgfqpoint{0.800000in}{0.528000in}}{\pgfqpoint{4.960000in}{3.696000in}}%
\pgfusepath{clip}%
\pgfsetbuttcap%
\pgfsetroundjoin%
\pgfsetlinewidth{0.501875pt}%
\definecolor{currentstroke}{rgb}{0.000000,0.000000,1.000000}%
\pgfsetstrokecolor{currentstroke}%
\pgfsetdash{}{0pt}%
\pgfpathmoveto{\pgfqpoint{4.650764in}{2.159193in}}%
\pgfpathlineto{\pgfqpoint{4.650764in}{2.276248in}}%
\pgfusepath{stroke}%
\end{pgfscope}%
\begin{pgfscope}%
\pgfpathrectangle{\pgfqpoint{0.800000in}{0.528000in}}{\pgfqpoint{4.960000in}{3.696000in}}%
\pgfusepath{clip}%
\pgfsetbuttcap%
\pgfsetroundjoin%
\pgfsetlinewidth{0.501875pt}%
\definecolor{currentstroke}{rgb}{0.000000,0.000000,1.000000}%
\pgfsetstrokecolor{currentstroke}%
\pgfsetdash{}{0pt}%
\pgfpathmoveto{\pgfqpoint{4.668800in}{2.393304in}}%
\pgfpathlineto{\pgfqpoint{4.668800in}{2.510359in}}%
\pgfusepath{stroke}%
\end{pgfscope}%
\begin{pgfscope}%
\pgfpathrectangle{\pgfqpoint{0.800000in}{0.528000in}}{\pgfqpoint{4.960000in}{3.696000in}}%
\pgfusepath{clip}%
\pgfsetbuttcap%
\pgfsetroundjoin%
\pgfsetlinewidth{0.501875pt}%
\definecolor{currentstroke}{rgb}{0.000000,0.000000,1.000000}%
\pgfsetstrokecolor{currentstroke}%
\pgfsetdash{}{0pt}%
\pgfpathmoveto{\pgfqpoint{4.686836in}{2.627414in}}%
\pgfpathlineto{\pgfqpoint{4.686836in}{2.744470in}}%
\pgfusepath{stroke}%
\end{pgfscope}%
\begin{pgfscope}%
\pgfpathrectangle{\pgfqpoint{0.800000in}{0.528000in}}{\pgfqpoint{4.960000in}{3.696000in}}%
\pgfusepath{clip}%
\pgfsetbuttcap%
\pgfsetroundjoin%
\pgfsetlinewidth{0.501875pt}%
\definecolor{currentstroke}{rgb}{0.000000,0.000000,1.000000}%
\pgfsetstrokecolor{currentstroke}%
\pgfsetdash{}{0pt}%
\pgfpathmoveto{\pgfqpoint{4.704873in}{2.802998in}}%
\pgfpathlineto{\pgfqpoint{4.704873in}{2.920053in}}%
\pgfusepath{stroke}%
\end{pgfscope}%
\begin{pgfscope}%
\pgfpathrectangle{\pgfqpoint{0.800000in}{0.528000in}}{\pgfqpoint{4.960000in}{3.696000in}}%
\pgfusepath{clip}%
\pgfsetbuttcap%
\pgfsetroundjoin%
\pgfsetlinewidth{0.501875pt}%
\definecolor{currentstroke}{rgb}{0.000000,0.000000,1.000000}%
\pgfsetstrokecolor{currentstroke}%
\pgfsetdash{}{0pt}%
\pgfpathmoveto{\pgfqpoint{4.722909in}{2.861525in}}%
\pgfpathlineto{\pgfqpoint{4.722909in}{2.978581in}}%
\pgfusepath{stroke}%
\end{pgfscope}%
\begin{pgfscope}%
\pgfpathrectangle{\pgfqpoint{0.800000in}{0.528000in}}{\pgfqpoint{4.960000in}{3.696000in}}%
\pgfusepath{clip}%
\pgfsetbuttcap%
\pgfsetroundjoin%
\pgfsetlinewidth{0.501875pt}%
\definecolor{currentstroke}{rgb}{0.000000,0.000000,1.000000}%
\pgfsetstrokecolor{currentstroke}%
\pgfsetdash{}{0pt}%
\pgfpathmoveto{\pgfqpoint{4.740945in}{2.920053in}}%
\pgfpathlineto{\pgfqpoint{4.740945in}{3.037108in}}%
\pgfusepath{stroke}%
\end{pgfscope}%
\begin{pgfscope}%
\pgfpathrectangle{\pgfqpoint{0.800000in}{0.528000in}}{\pgfqpoint{4.960000in}{3.696000in}}%
\pgfusepath{clip}%
\pgfsetbuttcap%
\pgfsetroundjoin%
\pgfsetlinewidth{0.501875pt}%
\definecolor{currentstroke}{rgb}{0.000000,0.000000,1.000000}%
\pgfsetstrokecolor{currentstroke}%
\pgfsetdash{}{0pt}%
\pgfpathmoveto{\pgfqpoint{4.758982in}{2.802998in}}%
\pgfpathlineto{\pgfqpoint{4.758982in}{2.920053in}}%
\pgfusepath{stroke}%
\end{pgfscope}%
\begin{pgfscope}%
\pgfpathrectangle{\pgfqpoint{0.800000in}{0.528000in}}{\pgfqpoint{4.960000in}{3.696000in}}%
\pgfusepath{clip}%
\pgfsetbuttcap%
\pgfsetroundjoin%
\pgfsetlinewidth{0.501875pt}%
\definecolor{currentstroke}{rgb}{0.000000,0.000000,1.000000}%
\pgfsetstrokecolor{currentstroke}%
\pgfsetdash{}{0pt}%
\pgfpathmoveto{\pgfqpoint{4.777018in}{2.627414in}}%
\pgfpathlineto{\pgfqpoint{4.777018in}{2.744470in}}%
\pgfusepath{stroke}%
\end{pgfscope}%
\begin{pgfscope}%
\pgfpathrectangle{\pgfqpoint{0.800000in}{0.528000in}}{\pgfqpoint{4.960000in}{3.696000in}}%
\pgfusepath{clip}%
\pgfsetbuttcap%
\pgfsetroundjoin%
\pgfsetlinewidth{0.501875pt}%
\definecolor{currentstroke}{rgb}{0.000000,0.000000,1.000000}%
\pgfsetstrokecolor{currentstroke}%
\pgfsetdash{}{0pt}%
\pgfpathmoveto{\pgfqpoint{4.795055in}{2.451831in}}%
\pgfpathlineto{\pgfqpoint{4.795055in}{2.568887in}}%
\pgfusepath{stroke}%
\end{pgfscope}%
\begin{pgfscope}%
\pgfpathrectangle{\pgfqpoint{0.800000in}{0.528000in}}{\pgfqpoint{4.960000in}{3.696000in}}%
\pgfusepath{clip}%
\pgfsetbuttcap%
\pgfsetroundjoin%
\pgfsetlinewidth{0.501875pt}%
\definecolor{currentstroke}{rgb}{0.000000,0.000000,1.000000}%
\pgfsetstrokecolor{currentstroke}%
\pgfsetdash{}{0pt}%
\pgfpathmoveto{\pgfqpoint{4.813091in}{2.217720in}}%
\pgfpathlineto{\pgfqpoint{4.813091in}{2.334776in}}%
\pgfusepath{stroke}%
\end{pgfscope}%
\begin{pgfscope}%
\pgfpathrectangle{\pgfqpoint{0.800000in}{0.528000in}}{\pgfqpoint{4.960000in}{3.696000in}}%
\pgfusepath{clip}%
\pgfsetbuttcap%
\pgfsetroundjoin%
\pgfsetlinewidth{0.501875pt}%
\definecolor{currentstroke}{rgb}{0.000000,0.000000,1.000000}%
\pgfsetstrokecolor{currentstroke}%
\pgfsetdash{}{0pt}%
\pgfpathmoveto{\pgfqpoint{4.831127in}{2.042137in}}%
\pgfpathlineto{\pgfqpoint{4.831127in}{2.159193in}}%
\pgfusepath{stroke}%
\end{pgfscope}%
\begin{pgfscope}%
\pgfpathrectangle{\pgfqpoint{0.800000in}{0.528000in}}{\pgfqpoint{4.960000in}{3.696000in}}%
\pgfusepath{clip}%
\pgfsetbuttcap%
\pgfsetroundjoin%
\pgfsetlinewidth{0.501875pt}%
\definecolor{currentstroke}{rgb}{0.000000,0.000000,1.000000}%
\pgfsetstrokecolor{currentstroke}%
\pgfsetdash{}{0pt}%
\pgfpathmoveto{\pgfqpoint{4.849164in}{1.925082in}}%
\pgfpathlineto{\pgfqpoint{4.849164in}{2.042137in}}%
\pgfusepath{stroke}%
\end{pgfscope}%
\begin{pgfscope}%
\pgfpathrectangle{\pgfqpoint{0.800000in}{0.528000in}}{\pgfqpoint{4.960000in}{3.696000in}}%
\pgfusepath{clip}%
\pgfsetbuttcap%
\pgfsetroundjoin%
\pgfsetlinewidth{0.501875pt}%
\definecolor{currentstroke}{rgb}{0.000000,0.000000,1.000000}%
\pgfsetstrokecolor{currentstroke}%
\pgfsetdash{}{0pt}%
\pgfpathmoveto{\pgfqpoint{4.867200in}{1.925082in}}%
\pgfpathlineto{\pgfqpoint{4.867200in}{2.042137in}}%
\pgfusepath{stroke}%
\end{pgfscope}%
\begin{pgfscope}%
\pgfpathrectangle{\pgfqpoint{0.800000in}{0.528000in}}{\pgfqpoint{4.960000in}{3.696000in}}%
\pgfusepath{clip}%
\pgfsetbuttcap%
\pgfsetroundjoin%
\pgfsetlinewidth{0.501875pt}%
\definecolor{currentstroke}{rgb}{0.000000,0.000000,1.000000}%
\pgfsetstrokecolor{currentstroke}%
\pgfsetdash{}{0pt}%
\pgfpathmoveto{\pgfqpoint{4.885236in}{1.925082in}}%
\pgfpathlineto{\pgfqpoint{4.885236in}{2.042137in}}%
\pgfusepath{stroke}%
\end{pgfscope}%
\begin{pgfscope}%
\pgfpathrectangle{\pgfqpoint{0.800000in}{0.528000in}}{\pgfqpoint{4.960000in}{3.696000in}}%
\pgfusepath{clip}%
\pgfsetbuttcap%
\pgfsetroundjoin%
\pgfsetlinewidth{0.501875pt}%
\definecolor{currentstroke}{rgb}{0.000000,0.000000,1.000000}%
\pgfsetstrokecolor{currentstroke}%
\pgfsetdash{}{0pt}%
\pgfpathmoveto{\pgfqpoint{4.903273in}{2.100665in}}%
\pgfpathlineto{\pgfqpoint{4.903273in}{2.217720in}}%
\pgfusepath{stroke}%
\end{pgfscope}%
\begin{pgfscope}%
\pgfpathrectangle{\pgfqpoint{0.800000in}{0.528000in}}{\pgfqpoint{4.960000in}{3.696000in}}%
\pgfusepath{clip}%
\pgfsetbuttcap%
\pgfsetroundjoin%
\pgfsetlinewidth{0.501875pt}%
\definecolor{currentstroke}{rgb}{0.000000,0.000000,1.000000}%
\pgfsetstrokecolor{currentstroke}%
\pgfsetdash{}{0pt}%
\pgfpathmoveto{\pgfqpoint{4.921309in}{2.276248in}}%
\pgfpathlineto{\pgfqpoint{4.921309in}{2.393304in}}%
\pgfusepath{stroke}%
\end{pgfscope}%
\begin{pgfscope}%
\pgfpathrectangle{\pgfqpoint{0.800000in}{0.528000in}}{\pgfqpoint{4.960000in}{3.696000in}}%
\pgfusepath{clip}%
\pgfsetbuttcap%
\pgfsetroundjoin%
\pgfsetlinewidth{0.501875pt}%
\definecolor{currentstroke}{rgb}{0.000000,0.000000,1.000000}%
\pgfsetstrokecolor{currentstroke}%
\pgfsetdash{}{0pt}%
\pgfpathmoveto{\pgfqpoint{4.939345in}{2.510359in}}%
\pgfpathlineto{\pgfqpoint{4.939345in}{2.627414in}}%
\pgfusepath{stroke}%
\end{pgfscope}%
\begin{pgfscope}%
\pgfpathrectangle{\pgfqpoint{0.800000in}{0.528000in}}{\pgfqpoint{4.960000in}{3.696000in}}%
\pgfusepath{clip}%
\pgfsetbuttcap%
\pgfsetroundjoin%
\pgfsetlinewidth{0.501875pt}%
\definecolor{currentstroke}{rgb}{0.000000,0.000000,1.000000}%
\pgfsetstrokecolor{currentstroke}%
\pgfsetdash{}{0pt}%
\pgfpathmoveto{\pgfqpoint{4.957382in}{2.685942in}}%
\pgfpathlineto{\pgfqpoint{4.957382in}{2.802998in}}%
\pgfusepath{stroke}%
\end{pgfscope}%
\begin{pgfscope}%
\pgfpathrectangle{\pgfqpoint{0.800000in}{0.528000in}}{\pgfqpoint{4.960000in}{3.696000in}}%
\pgfusepath{clip}%
\pgfsetbuttcap%
\pgfsetroundjoin%
\pgfsetlinewidth{0.501875pt}%
\definecolor{currentstroke}{rgb}{0.000000,0.000000,1.000000}%
\pgfsetstrokecolor{currentstroke}%
\pgfsetdash{}{0pt}%
\pgfpathmoveto{\pgfqpoint{4.975418in}{2.802998in}}%
\pgfpathlineto{\pgfqpoint{4.975418in}{2.920053in}}%
\pgfusepath{stroke}%
\end{pgfscope}%
\begin{pgfscope}%
\pgfpathrectangle{\pgfqpoint{0.800000in}{0.528000in}}{\pgfqpoint{4.960000in}{3.696000in}}%
\pgfusepath{clip}%
\pgfsetbuttcap%
\pgfsetroundjoin%
\pgfsetlinewidth{0.501875pt}%
\definecolor{currentstroke}{rgb}{0.000000,0.000000,1.000000}%
\pgfsetstrokecolor{currentstroke}%
\pgfsetdash{}{0pt}%
\pgfpathmoveto{\pgfqpoint{4.993455in}{2.861525in}}%
\pgfpathlineto{\pgfqpoint{4.993455in}{2.978581in}}%
\pgfusepath{stroke}%
\end{pgfscope}%
\begin{pgfscope}%
\pgfpathrectangle{\pgfqpoint{0.800000in}{0.528000in}}{\pgfqpoint{4.960000in}{3.696000in}}%
\pgfusepath{clip}%
\pgfsetbuttcap%
\pgfsetroundjoin%
\pgfsetlinewidth{0.501875pt}%
\definecolor{currentstroke}{rgb}{0.000000,0.000000,1.000000}%
\pgfsetstrokecolor{currentstroke}%
\pgfsetdash{}{0pt}%
\pgfpathmoveto{\pgfqpoint{5.011491in}{2.861525in}}%
\pgfpathlineto{\pgfqpoint{5.011491in}{2.978581in}}%
\pgfusepath{stroke}%
\end{pgfscope}%
\begin{pgfscope}%
\pgfpathrectangle{\pgfqpoint{0.800000in}{0.528000in}}{\pgfqpoint{4.960000in}{3.696000in}}%
\pgfusepath{clip}%
\pgfsetbuttcap%
\pgfsetroundjoin%
\pgfsetlinewidth{0.501875pt}%
\definecolor{currentstroke}{rgb}{0.000000,0.000000,1.000000}%
\pgfsetstrokecolor{currentstroke}%
\pgfsetdash{}{0pt}%
\pgfpathmoveto{\pgfqpoint{5.029527in}{2.744470in}}%
\pgfpathlineto{\pgfqpoint{5.029527in}{2.861525in}}%
\pgfusepath{stroke}%
\end{pgfscope}%
\begin{pgfscope}%
\pgfpathrectangle{\pgfqpoint{0.800000in}{0.528000in}}{\pgfqpoint{4.960000in}{3.696000in}}%
\pgfusepath{clip}%
\pgfsetbuttcap%
\pgfsetroundjoin%
\pgfsetlinewidth{0.501875pt}%
\definecolor{currentstroke}{rgb}{0.000000,0.000000,1.000000}%
\pgfsetstrokecolor{currentstroke}%
\pgfsetdash{}{0pt}%
\pgfpathmoveto{\pgfqpoint{5.047564in}{2.568887in}}%
\pgfpathlineto{\pgfqpoint{5.047564in}{2.685942in}}%
\pgfusepath{stroke}%
\end{pgfscope}%
\begin{pgfscope}%
\pgfpathrectangle{\pgfqpoint{0.800000in}{0.528000in}}{\pgfqpoint{4.960000in}{3.696000in}}%
\pgfusepath{clip}%
\pgfsetbuttcap%
\pgfsetroundjoin%
\pgfsetlinewidth{0.501875pt}%
\definecolor{currentstroke}{rgb}{0.000000,0.000000,1.000000}%
\pgfsetstrokecolor{currentstroke}%
\pgfsetdash{}{0pt}%
\pgfpathmoveto{\pgfqpoint{5.065600in}{2.393304in}}%
\pgfpathlineto{\pgfqpoint{5.065600in}{2.510359in}}%
\pgfusepath{stroke}%
\end{pgfscope}%
\begin{pgfscope}%
\pgfpathrectangle{\pgfqpoint{0.800000in}{0.528000in}}{\pgfqpoint{4.960000in}{3.696000in}}%
\pgfusepath{clip}%
\pgfsetbuttcap%
\pgfsetroundjoin%
\pgfsetlinewidth{0.501875pt}%
\definecolor{currentstroke}{rgb}{0.000000,0.000000,1.000000}%
\pgfsetstrokecolor{currentstroke}%
\pgfsetdash{}{0pt}%
\pgfpathmoveto{\pgfqpoint{5.083636in}{2.159193in}}%
\pgfpathlineto{\pgfqpoint{5.083636in}{2.276248in}}%
\pgfusepath{stroke}%
\end{pgfscope}%
\begin{pgfscope}%
\pgfpathrectangle{\pgfqpoint{0.800000in}{0.528000in}}{\pgfqpoint{4.960000in}{3.696000in}}%
\pgfusepath{clip}%
\pgfsetbuttcap%
\pgfsetroundjoin%
\pgfsetlinewidth{0.501875pt}%
\definecolor{currentstroke}{rgb}{0.000000,0.000000,1.000000}%
\pgfsetstrokecolor{currentstroke}%
\pgfsetdash{}{0pt}%
\pgfpathmoveto{\pgfqpoint{5.101673in}{2.042137in}}%
\pgfpathlineto{\pgfqpoint{5.101673in}{2.159193in}}%
\pgfusepath{stroke}%
\end{pgfscope}%
\begin{pgfscope}%
\pgfpathrectangle{\pgfqpoint{0.800000in}{0.528000in}}{\pgfqpoint{4.960000in}{3.696000in}}%
\pgfusepath{clip}%
\pgfsetbuttcap%
\pgfsetroundjoin%
\pgfsetlinewidth{0.501875pt}%
\definecolor{currentstroke}{rgb}{0.000000,0.000000,1.000000}%
\pgfsetstrokecolor{currentstroke}%
\pgfsetdash{}{0pt}%
\pgfpathmoveto{\pgfqpoint{5.119709in}{1.925082in}}%
\pgfpathlineto{\pgfqpoint{5.119709in}{2.042137in}}%
\pgfusepath{stroke}%
\end{pgfscope}%
\begin{pgfscope}%
\pgfpathrectangle{\pgfqpoint{0.800000in}{0.528000in}}{\pgfqpoint{4.960000in}{3.696000in}}%
\pgfusepath{clip}%
\pgfsetbuttcap%
\pgfsetroundjoin%
\pgfsetlinewidth{0.501875pt}%
\definecolor{currentstroke}{rgb}{0.000000,0.000000,1.000000}%
\pgfsetstrokecolor{currentstroke}%
\pgfsetdash{}{0pt}%
\pgfpathmoveto{\pgfqpoint{5.137745in}{1.925082in}}%
\pgfpathlineto{\pgfqpoint{5.137745in}{2.042137in}}%
\pgfusepath{stroke}%
\end{pgfscope}%
\begin{pgfscope}%
\pgfpathrectangle{\pgfqpoint{0.800000in}{0.528000in}}{\pgfqpoint{4.960000in}{3.696000in}}%
\pgfusepath{clip}%
\pgfsetbuttcap%
\pgfsetroundjoin%
\pgfsetlinewidth{0.501875pt}%
\definecolor{currentstroke}{rgb}{0.000000,0.000000,1.000000}%
\pgfsetstrokecolor{currentstroke}%
\pgfsetdash{}{0pt}%
\pgfpathmoveto{\pgfqpoint{5.155782in}{2.042137in}}%
\pgfpathlineto{\pgfqpoint{5.155782in}{2.159193in}}%
\pgfusepath{stroke}%
\end{pgfscope}%
\begin{pgfscope}%
\pgfpathrectangle{\pgfqpoint{0.800000in}{0.528000in}}{\pgfqpoint{4.960000in}{3.696000in}}%
\pgfusepath{clip}%
\pgfsetbuttcap%
\pgfsetroundjoin%
\pgfsetlinewidth{0.501875pt}%
\definecolor{currentstroke}{rgb}{0.000000,0.000000,1.000000}%
\pgfsetstrokecolor{currentstroke}%
\pgfsetdash{}{0pt}%
\pgfpathmoveto{\pgfqpoint{5.173818in}{2.159193in}}%
\pgfpathlineto{\pgfqpoint{5.173818in}{2.276248in}}%
\pgfusepath{stroke}%
\end{pgfscope}%
\begin{pgfscope}%
\pgfpathrectangle{\pgfqpoint{0.800000in}{0.528000in}}{\pgfqpoint{4.960000in}{3.696000in}}%
\pgfusepath{clip}%
\pgfsetbuttcap%
\pgfsetroundjoin%
\pgfsetlinewidth{0.501875pt}%
\definecolor{currentstroke}{rgb}{0.000000,0.000000,1.000000}%
\pgfsetstrokecolor{currentstroke}%
\pgfsetdash{}{0pt}%
\pgfpathmoveto{\pgfqpoint{5.191855in}{2.334776in}}%
\pgfpathlineto{\pgfqpoint{5.191855in}{2.451831in}}%
\pgfusepath{stroke}%
\end{pgfscope}%
\begin{pgfscope}%
\pgfpathrectangle{\pgfqpoint{0.800000in}{0.528000in}}{\pgfqpoint{4.960000in}{3.696000in}}%
\pgfusepath{clip}%
\pgfsetbuttcap%
\pgfsetroundjoin%
\pgfsetlinewidth{0.501875pt}%
\definecolor{currentstroke}{rgb}{0.000000,0.000000,1.000000}%
\pgfsetstrokecolor{currentstroke}%
\pgfsetdash{}{0pt}%
\pgfpathmoveto{\pgfqpoint{5.209891in}{2.568887in}}%
\pgfpathlineto{\pgfqpoint{5.209891in}{2.685942in}}%
\pgfusepath{stroke}%
\end{pgfscope}%
\begin{pgfscope}%
\pgfpathrectangle{\pgfqpoint{0.800000in}{0.528000in}}{\pgfqpoint{4.960000in}{3.696000in}}%
\pgfusepath{clip}%
\pgfsetbuttcap%
\pgfsetroundjoin%
\pgfsetlinewidth{0.501875pt}%
\definecolor{currentstroke}{rgb}{0.000000,0.000000,1.000000}%
\pgfsetstrokecolor{currentstroke}%
\pgfsetdash{}{0pt}%
\pgfpathmoveto{\pgfqpoint{5.227927in}{2.685942in}}%
\pgfpathlineto{\pgfqpoint{5.227927in}{2.802998in}}%
\pgfusepath{stroke}%
\end{pgfscope}%
\begin{pgfscope}%
\pgfpathrectangle{\pgfqpoint{0.800000in}{0.528000in}}{\pgfqpoint{4.960000in}{3.696000in}}%
\pgfusepath{clip}%
\pgfsetbuttcap%
\pgfsetroundjoin%
\pgfsetlinewidth{0.501875pt}%
\definecolor{currentstroke}{rgb}{0.000000,0.000000,1.000000}%
\pgfsetstrokecolor{currentstroke}%
\pgfsetdash{}{0pt}%
\pgfpathmoveto{\pgfqpoint{5.245964in}{2.802998in}}%
\pgfpathlineto{\pgfqpoint{5.245964in}{2.920053in}}%
\pgfusepath{stroke}%
\end{pgfscope}%
\begin{pgfscope}%
\pgfpathrectangle{\pgfqpoint{0.800000in}{0.528000in}}{\pgfqpoint{4.960000in}{3.696000in}}%
\pgfusepath{clip}%
\pgfsetbuttcap%
\pgfsetroundjoin%
\pgfsetlinewidth{0.501875pt}%
\definecolor{currentstroke}{rgb}{0.000000,0.000000,1.000000}%
\pgfsetstrokecolor{currentstroke}%
\pgfsetdash{}{0pt}%
\pgfpathmoveto{\pgfqpoint{5.264000in}{2.861525in}}%
\pgfpathlineto{\pgfqpoint{5.264000in}{2.978581in}}%
\pgfusepath{stroke}%
\end{pgfscope}%
\begin{pgfscope}%
\pgfpathrectangle{\pgfqpoint{0.800000in}{0.528000in}}{\pgfqpoint{4.960000in}{3.696000in}}%
\pgfusepath{clip}%
\pgfsetbuttcap%
\pgfsetroundjoin%
\pgfsetlinewidth{0.501875pt}%
\definecolor{currentstroke}{rgb}{0.000000,0.000000,1.000000}%
\pgfsetstrokecolor{currentstroke}%
\pgfsetdash{}{0pt}%
\pgfpathmoveto{\pgfqpoint{5.282036in}{2.802998in}}%
\pgfpathlineto{\pgfqpoint{5.282036in}{2.920053in}}%
\pgfusepath{stroke}%
\end{pgfscope}%
\begin{pgfscope}%
\pgfpathrectangle{\pgfqpoint{0.800000in}{0.528000in}}{\pgfqpoint{4.960000in}{3.696000in}}%
\pgfusepath{clip}%
\pgfsetbuttcap%
\pgfsetroundjoin%
\pgfsetlinewidth{0.501875pt}%
\definecolor{currentstroke}{rgb}{0.000000,0.000000,1.000000}%
\pgfsetstrokecolor{currentstroke}%
\pgfsetdash{}{0pt}%
\pgfpathmoveto{\pgfqpoint{5.300073in}{2.685942in}}%
\pgfpathlineto{\pgfqpoint{5.300073in}{2.802998in}}%
\pgfusepath{stroke}%
\end{pgfscope}%
\begin{pgfscope}%
\pgfpathrectangle{\pgfqpoint{0.800000in}{0.528000in}}{\pgfqpoint{4.960000in}{3.696000in}}%
\pgfusepath{clip}%
\pgfsetbuttcap%
\pgfsetroundjoin%
\pgfsetlinewidth{0.501875pt}%
\definecolor{currentstroke}{rgb}{0.000000,0.000000,1.000000}%
\pgfsetstrokecolor{currentstroke}%
\pgfsetdash{}{0pt}%
\pgfpathmoveto{\pgfqpoint{5.318109in}{2.510359in}}%
\pgfpathlineto{\pgfqpoint{5.318109in}{2.627414in}}%
\pgfusepath{stroke}%
\end{pgfscope}%
\begin{pgfscope}%
\pgfpathrectangle{\pgfqpoint{0.800000in}{0.528000in}}{\pgfqpoint{4.960000in}{3.696000in}}%
\pgfusepath{clip}%
\pgfsetbuttcap%
\pgfsetroundjoin%
\pgfsetlinewidth{0.501875pt}%
\definecolor{currentstroke}{rgb}{0.000000,0.000000,1.000000}%
\pgfsetstrokecolor{currentstroke}%
\pgfsetdash{}{0pt}%
\pgfpathmoveto{\pgfqpoint{5.336145in}{2.276248in}}%
\pgfpathlineto{\pgfqpoint{5.336145in}{2.393304in}}%
\pgfusepath{stroke}%
\end{pgfscope}%
\begin{pgfscope}%
\pgfpathrectangle{\pgfqpoint{0.800000in}{0.528000in}}{\pgfqpoint{4.960000in}{3.696000in}}%
\pgfusepath{clip}%
\pgfsetbuttcap%
\pgfsetroundjoin%
\pgfsetlinewidth{0.501875pt}%
\definecolor{currentstroke}{rgb}{0.000000,0.000000,1.000000}%
\pgfsetstrokecolor{currentstroke}%
\pgfsetdash{}{0pt}%
\pgfpathmoveto{\pgfqpoint{5.354182in}{2.159193in}}%
\pgfpathlineto{\pgfqpoint{5.354182in}{2.276248in}}%
\pgfusepath{stroke}%
\end{pgfscope}%
\begin{pgfscope}%
\pgfpathrectangle{\pgfqpoint{0.800000in}{0.528000in}}{\pgfqpoint{4.960000in}{3.696000in}}%
\pgfusepath{clip}%
\pgfsetbuttcap%
\pgfsetroundjoin%
\pgfsetlinewidth{0.501875pt}%
\definecolor{currentstroke}{rgb}{0.000000,0.000000,1.000000}%
\pgfsetstrokecolor{currentstroke}%
\pgfsetdash{}{0pt}%
\pgfpathmoveto{\pgfqpoint{5.372218in}{1.983610in}}%
\pgfpathlineto{\pgfqpoint{5.372218in}{2.100665in}}%
\pgfusepath{stroke}%
\end{pgfscope}%
\begin{pgfscope}%
\pgfpathrectangle{\pgfqpoint{0.800000in}{0.528000in}}{\pgfqpoint{4.960000in}{3.696000in}}%
\pgfusepath{clip}%
\pgfsetbuttcap%
\pgfsetroundjoin%
\pgfsetlinewidth{0.501875pt}%
\definecolor{currentstroke}{rgb}{0.000000,0.000000,1.000000}%
\pgfsetstrokecolor{currentstroke}%
\pgfsetdash{}{0pt}%
\pgfpathmoveto{\pgfqpoint{5.390255in}{1.983610in}}%
\pgfpathlineto{\pgfqpoint{5.390255in}{2.100665in}}%
\pgfusepath{stroke}%
\end{pgfscope}%
\begin{pgfscope}%
\pgfpathrectangle{\pgfqpoint{0.800000in}{0.528000in}}{\pgfqpoint{4.960000in}{3.696000in}}%
\pgfusepath{clip}%
\pgfsetbuttcap%
\pgfsetroundjoin%
\pgfsetlinewidth{0.501875pt}%
\definecolor{currentstroke}{rgb}{0.000000,0.000000,1.000000}%
\pgfsetstrokecolor{currentstroke}%
\pgfsetdash{}{0pt}%
\pgfpathmoveto{\pgfqpoint{5.408291in}{1.983610in}}%
\pgfpathlineto{\pgfqpoint{5.408291in}{2.100665in}}%
\pgfusepath{stroke}%
\end{pgfscope}%
\begin{pgfscope}%
\pgfpathrectangle{\pgfqpoint{0.800000in}{0.528000in}}{\pgfqpoint{4.960000in}{3.696000in}}%
\pgfusepath{clip}%
\pgfsetbuttcap%
\pgfsetroundjoin%
\pgfsetlinewidth{0.501875pt}%
\definecolor{currentstroke}{rgb}{0.000000,0.000000,1.000000}%
\pgfsetstrokecolor{currentstroke}%
\pgfsetdash{}{0pt}%
\pgfpathmoveto{\pgfqpoint{5.426327in}{2.100665in}}%
\pgfpathlineto{\pgfqpoint{5.426327in}{2.217720in}}%
\pgfusepath{stroke}%
\end{pgfscope}%
\begin{pgfscope}%
\pgfpathrectangle{\pgfqpoint{0.800000in}{0.528000in}}{\pgfqpoint{4.960000in}{3.696000in}}%
\pgfusepath{clip}%
\pgfsetbuttcap%
\pgfsetroundjoin%
\pgfsetlinewidth{0.501875pt}%
\definecolor{currentstroke}{rgb}{0.000000,0.000000,1.000000}%
\pgfsetstrokecolor{currentstroke}%
\pgfsetdash{}{0pt}%
\pgfpathmoveto{\pgfqpoint{5.444364in}{2.276248in}}%
\pgfpathlineto{\pgfqpoint{5.444364in}{2.393304in}}%
\pgfusepath{stroke}%
\end{pgfscope}%
\begin{pgfscope}%
\pgfpathrectangle{\pgfqpoint{0.800000in}{0.528000in}}{\pgfqpoint{4.960000in}{3.696000in}}%
\pgfusepath{clip}%
\pgfsetbuttcap%
\pgfsetroundjoin%
\pgfsetlinewidth{0.501875pt}%
\definecolor{currentstroke}{rgb}{0.000000,0.000000,1.000000}%
\pgfsetstrokecolor{currentstroke}%
\pgfsetdash{}{0pt}%
\pgfpathmoveto{\pgfqpoint{5.462400in}{2.451831in}}%
\pgfpathlineto{\pgfqpoint{5.462400in}{2.568887in}}%
\pgfusepath{stroke}%
\end{pgfscope}%
\begin{pgfscope}%
\pgfpathrectangle{\pgfqpoint{0.800000in}{0.528000in}}{\pgfqpoint{4.960000in}{3.696000in}}%
\pgfusepath{clip}%
\pgfsetbuttcap%
\pgfsetroundjoin%
\pgfsetlinewidth{0.501875pt}%
\definecolor{currentstroke}{rgb}{0.000000,0.000000,1.000000}%
\pgfsetstrokecolor{currentstroke}%
\pgfsetdash{}{0pt}%
\pgfpathmoveto{\pgfqpoint{5.480436in}{2.627414in}}%
\pgfpathlineto{\pgfqpoint{5.480436in}{2.744470in}}%
\pgfusepath{stroke}%
\end{pgfscope}%
\begin{pgfscope}%
\pgfpathrectangle{\pgfqpoint{0.800000in}{0.528000in}}{\pgfqpoint{4.960000in}{3.696000in}}%
\pgfusepath{clip}%
\pgfsetbuttcap%
\pgfsetroundjoin%
\pgfsetlinewidth{0.501875pt}%
\definecolor{currentstroke}{rgb}{0.000000,0.000000,1.000000}%
\pgfsetstrokecolor{currentstroke}%
\pgfsetdash{}{0pt}%
\pgfpathmoveto{\pgfqpoint{5.498473in}{2.744470in}}%
\pgfpathlineto{\pgfqpoint{5.498473in}{2.861525in}}%
\pgfusepath{stroke}%
\end{pgfscope}%
\begin{pgfscope}%
\pgfpathrectangle{\pgfqpoint{0.800000in}{0.528000in}}{\pgfqpoint{4.960000in}{3.696000in}}%
\pgfusepath{clip}%
\pgfsetbuttcap%
\pgfsetroundjoin%
\pgfsetlinewidth{0.501875pt}%
\definecolor{currentstroke}{rgb}{0.000000,0.000000,1.000000}%
\pgfsetstrokecolor{currentstroke}%
\pgfsetdash{}{0pt}%
\pgfpathmoveto{\pgfqpoint{5.516509in}{2.802998in}}%
\pgfpathlineto{\pgfqpoint{5.516509in}{2.920053in}}%
\pgfusepath{stroke}%
\end{pgfscope}%
\begin{pgfscope}%
\pgfpathrectangle{\pgfqpoint{0.800000in}{0.528000in}}{\pgfqpoint{4.960000in}{3.696000in}}%
\pgfusepath{clip}%
\pgfsetbuttcap%
\pgfsetroundjoin%
\pgfsetlinewidth{0.501875pt}%
\definecolor{currentstroke}{rgb}{0.000000,0.000000,1.000000}%
\pgfsetstrokecolor{currentstroke}%
\pgfsetdash{}{0pt}%
\pgfpathmoveto{\pgfqpoint{5.534545in}{2.802998in}}%
\pgfpathlineto{\pgfqpoint{5.534545in}{2.920053in}}%
\pgfusepath{stroke}%
\end{pgfscope}%
\begin{pgfscope}%
\pgfpathrectangle{\pgfqpoint{0.800000in}{0.528000in}}{\pgfqpoint{4.960000in}{3.696000in}}%
\pgfusepath{clip}%
\pgfsetbuttcap%
\pgfsetroundjoin%
\definecolor{currentfill}{rgb}{0.000000,0.000000,1.000000}%
\pgfsetfillcolor{currentfill}%
\pgfsetlinewidth{0.501875pt}%
\definecolor{currentstroke}{rgb}{0.000000,0.000000,1.000000}%
\pgfsetstrokecolor{currentstroke}%
\pgfsetdash{}{0pt}%
\pgfsys@defobject{currentmarker}{\pgfqpoint{-0.027778in}{-0.000000in}}{\pgfqpoint{0.027778in}{0.000000in}}{%
\pgfpathmoveto{\pgfqpoint{0.027778in}{-0.000000in}}%
\pgfpathlineto{\pgfqpoint{-0.027778in}{0.000000in}}%
\pgfusepath{stroke,fill}%
}%
\begin{pgfscope}%
\pgfsys@transformshift{1.043491in}{3.856496in}%
\pgfsys@useobject{currentmarker}{}%
\end{pgfscope}%
\begin{pgfscope}%
\pgfsys@transformshift{1.061527in}{3.680913in}%
\pgfsys@useobject{currentmarker}{}%
\end{pgfscope}%
\begin{pgfscope}%
\pgfsys@transformshift{1.079564in}{3.212692in}%
\pgfsys@useobject{currentmarker}{}%
\end{pgfscope}%
\begin{pgfscope}%
\pgfsys@transformshift{1.097600in}{2.627414in}%
\pgfsys@useobject{currentmarker}{}%
\end{pgfscope}%
\begin{pgfscope}%
\pgfsys@transformshift{1.115636in}{1.983610in}%
\pgfsys@useobject{currentmarker}{}%
\end{pgfscope}%
\begin{pgfscope}%
\pgfsys@transformshift{1.133673in}{1.222749in}%
\pgfsys@useobject{currentmarker}{}%
\end{pgfscope}%
\begin{pgfscope}%
\pgfsys@transformshift{1.151709in}{0.813055in}%
\pgfsys@useobject{currentmarker}{}%
\end{pgfscope}%
\begin{pgfscope}%
\pgfsys@transformshift{1.169745in}{0.696000in}%
\pgfsys@useobject{currentmarker}{}%
\end{pgfscope}%
\begin{pgfscope}%
\pgfsys@transformshift{1.187782in}{0.813055in}%
\pgfsys@useobject{currentmarker}{}%
\end{pgfscope}%
\begin{pgfscope}%
\pgfsys@transformshift{1.205818in}{1.222749in}%
\pgfsys@useobject{currentmarker}{}%
\end{pgfscope}%
\begin{pgfscope}%
\pgfsys@transformshift{1.223855in}{1.983610in}%
\pgfsys@useobject{currentmarker}{}%
\end{pgfscope}%
\begin{pgfscope}%
\pgfsys@transformshift{1.241891in}{2.568887in}%
\pgfsys@useobject{currentmarker}{}%
\end{pgfscope}%
\begin{pgfscope}%
\pgfsys@transformshift{1.259927in}{3.154164in}%
\pgfsys@useobject{currentmarker}{}%
\end{pgfscope}%
\begin{pgfscope}%
\pgfsys@transformshift{1.277964in}{3.563858in}%
\pgfsys@useobject{currentmarker}{}%
\end{pgfscope}%
\begin{pgfscope}%
\pgfsys@transformshift{1.296000in}{3.739441in}%
\pgfsys@useobject{currentmarker}{}%
\end{pgfscope}%
\begin{pgfscope}%
\pgfsys@transformshift{1.314036in}{3.739441in}%
\pgfsys@useobject{currentmarker}{}%
\end{pgfscope}%
\begin{pgfscope}%
\pgfsys@transformshift{1.332073in}{3.446802in}%
\pgfsys@useobject{currentmarker}{}%
\end{pgfscope}%
\begin{pgfscope}%
\pgfsys@transformshift{1.350109in}{2.978581in}%
\pgfsys@useobject{currentmarker}{}%
\end{pgfscope}%
\begin{pgfscope}%
\pgfsys@transformshift{1.368145in}{2.393304in}%
\pgfsys@useobject{currentmarker}{}%
\end{pgfscope}%
\begin{pgfscope}%
\pgfsys@transformshift{1.386182in}{1.866554in}%
\pgfsys@useobject{currentmarker}{}%
\end{pgfscope}%
\begin{pgfscope}%
\pgfsys@transformshift{1.404218in}{1.164222in}%
\pgfsys@useobject{currentmarker}{}%
\end{pgfscope}%
\begin{pgfscope}%
\pgfsys@transformshift{1.422255in}{0.871583in}%
\pgfsys@useobject{currentmarker}{}%
\end{pgfscope}%
\begin{pgfscope}%
\pgfsys@transformshift{1.440291in}{0.871583in}%
\pgfsys@useobject{currentmarker}{}%
\end{pgfscope}%
\begin{pgfscope}%
\pgfsys@transformshift{1.458327in}{1.047166in}%
\pgfsys@useobject{currentmarker}{}%
\end{pgfscope}%
\begin{pgfscope}%
\pgfsys@transformshift{1.476364in}{1.456860in}%
\pgfsys@useobject{currentmarker}{}%
\end{pgfscope}%
\begin{pgfscope}%
\pgfsys@transformshift{1.494400in}{2.159193in}%
\pgfsys@useobject{currentmarker}{}%
\end{pgfscope}%
\begin{pgfscope}%
\pgfsys@transformshift{1.512436in}{2.744470in}%
\pgfsys@useobject{currentmarker}{}%
\end{pgfscope}%
\begin{pgfscope}%
\pgfsys@transformshift{1.530473in}{3.212692in}%
\pgfsys@useobject{currentmarker}{}%
\end{pgfscope}%
\begin{pgfscope}%
\pgfsys@transformshift{1.548509in}{3.505330in}%
\pgfsys@useobject{currentmarker}{}%
\end{pgfscope}%
\begin{pgfscope}%
\pgfsys@transformshift{1.566545in}{3.622386in}%
\pgfsys@useobject{currentmarker}{}%
\end{pgfscope}%
\begin{pgfscope}%
\pgfsys@transformshift{1.584582in}{3.563858in}%
\pgfsys@useobject{currentmarker}{}%
\end{pgfscope}%
\begin{pgfscope}%
\pgfsys@transformshift{1.602618in}{3.212692in}%
\pgfsys@useobject{currentmarker}{}%
\end{pgfscope}%
\begin{pgfscope}%
\pgfsys@transformshift{1.620655in}{2.744470in}%
\pgfsys@useobject{currentmarker}{}%
\end{pgfscope}%
\begin{pgfscope}%
\pgfsys@transformshift{1.638691in}{2.217720in}%
\pgfsys@useobject{currentmarker}{}%
\end{pgfscope}%
\begin{pgfscope}%
\pgfsys@transformshift{1.656727in}{1.573916in}%
\pgfsys@useobject{currentmarker}{}%
\end{pgfscope}%
\begin{pgfscope}%
\pgfsys@transformshift{1.674764in}{1.164222in}%
\pgfsys@useobject{currentmarker}{}%
\end{pgfscope}%
\begin{pgfscope}%
\pgfsys@transformshift{1.692800in}{0.988639in}%
\pgfsys@useobject{currentmarker}{}%
\end{pgfscope}%
\begin{pgfscope}%
\pgfsys@transformshift{1.710836in}{0.988639in}%
\pgfsys@useobject{currentmarker}{}%
\end{pgfscope}%
\begin{pgfscope}%
\pgfsys@transformshift{1.728873in}{1.281277in}%
\pgfsys@useobject{currentmarker}{}%
\end{pgfscope}%
\begin{pgfscope}%
\pgfsys@transformshift{1.746909in}{1.866554in}%
\pgfsys@useobject{currentmarker}{}%
\end{pgfscope}%
\begin{pgfscope}%
\pgfsys@transformshift{1.764945in}{2.393304in}%
\pgfsys@useobject{currentmarker}{}%
\end{pgfscope}%
\begin{pgfscope}%
\pgfsys@transformshift{1.782982in}{2.861525in}%
\pgfsys@useobject{currentmarker}{}%
\end{pgfscope}%
\begin{pgfscope}%
\pgfsys@transformshift{1.801018in}{3.271219in}%
\pgfsys@useobject{currentmarker}{}%
\end{pgfscope}%
\begin{pgfscope}%
\pgfsys@transformshift{1.819055in}{3.505330in}%
\pgfsys@useobject{currentmarker}{}%
\end{pgfscope}%
\begin{pgfscope}%
\pgfsys@transformshift{1.837091in}{3.505330in}%
\pgfsys@useobject{currentmarker}{}%
\end{pgfscope}%
\begin{pgfscope}%
\pgfsys@transformshift{1.855127in}{3.388275in}%
\pgfsys@useobject{currentmarker}{}%
\end{pgfscope}%
\begin{pgfscope}%
\pgfsys@transformshift{1.873164in}{3.037108in}%
\pgfsys@useobject{currentmarker}{}%
\end{pgfscope}%
\begin{pgfscope}%
\pgfsys@transformshift{1.891200in}{2.568887in}%
\pgfsys@useobject{currentmarker}{}%
\end{pgfscope}%
\begin{pgfscope}%
\pgfsys@transformshift{1.909236in}{2.100665in}%
\pgfsys@useobject{currentmarker}{}%
\end{pgfscope}%
\begin{pgfscope}%
\pgfsys@transformshift{1.927273in}{1.456860in}%
\pgfsys@useobject{currentmarker}{}%
\end{pgfscope}%
\begin{pgfscope}%
\pgfsys@transformshift{1.945309in}{1.164222in}%
\pgfsys@useobject{currentmarker}{}%
\end{pgfscope}%
\begin{pgfscope}%
\pgfsys@transformshift{1.963345in}{1.047166in}%
\pgfsys@useobject{currentmarker}{}%
\end{pgfscope}%
\begin{pgfscope}%
\pgfsys@transformshift{1.981382in}{1.164222in}%
\pgfsys@useobject{currentmarker}{}%
\end{pgfscope}%
\begin{pgfscope}%
\pgfsys@transformshift{1.999418in}{1.456860in}%
\pgfsys@useobject{currentmarker}{}%
\end{pgfscope}%
\begin{pgfscope}%
\pgfsys@transformshift{2.017455in}{2.042137in}%
\pgfsys@useobject{currentmarker}{}%
\end{pgfscope}%
\begin{pgfscope}%
\pgfsys@transformshift{2.035491in}{2.568887in}%
\pgfsys@useobject{currentmarker}{}%
\end{pgfscope}%
\begin{pgfscope}%
\pgfsys@transformshift{2.053527in}{2.978581in}%
\pgfsys@useobject{currentmarker}{}%
\end{pgfscope}%
\begin{pgfscope}%
\pgfsys@transformshift{2.071564in}{3.271219in}%
\pgfsys@useobject{currentmarker}{}%
\end{pgfscope}%
\begin{pgfscope}%
\pgfsys@transformshift{2.089600in}{3.446802in}%
\pgfsys@useobject{currentmarker}{}%
\end{pgfscope}%
\begin{pgfscope}%
\pgfsys@transformshift{2.107636in}{3.446802in}%
\pgfsys@useobject{currentmarker}{}%
\end{pgfscope}%
\begin{pgfscope}%
\pgfsys@transformshift{2.125673in}{3.212692in}%
\pgfsys@useobject{currentmarker}{}%
\end{pgfscope}%
\begin{pgfscope}%
\pgfsys@transformshift{2.143709in}{2.861525in}%
\pgfsys@useobject{currentmarker}{}%
\end{pgfscope}%
\begin{pgfscope}%
\pgfsys@transformshift{2.161745in}{2.393304in}%
\pgfsys@useobject{currentmarker}{}%
\end{pgfscope}%
\begin{pgfscope}%
\pgfsys@transformshift{2.179782in}{1.983610in}%
\pgfsys@useobject{currentmarker}{}%
\end{pgfscope}%
\begin{pgfscope}%
\pgfsys@transformshift{2.197818in}{1.398333in}%
\pgfsys@useobject{currentmarker}{}%
\end{pgfscope}%
\begin{pgfscope}%
\pgfsys@transformshift{2.215855in}{1.222749in}%
\pgfsys@useobject{currentmarker}{}%
\end{pgfscope}%
\begin{pgfscope}%
\pgfsys@transformshift{2.233891in}{1.164222in}%
\pgfsys@useobject{currentmarker}{}%
\end{pgfscope}%
\begin{pgfscope}%
\pgfsys@transformshift{2.251927in}{1.339805in}%
\pgfsys@useobject{currentmarker}{}%
\end{pgfscope}%
\begin{pgfscope}%
\pgfsys@transformshift{2.269964in}{1.690971in}%
\pgfsys@useobject{currentmarker}{}%
\end{pgfscope}%
\begin{pgfscope}%
\pgfsys@transformshift{2.288000in}{2.276248in}%
\pgfsys@useobject{currentmarker}{}%
\end{pgfscope}%
\begin{pgfscope}%
\pgfsys@transformshift{2.306036in}{2.685942in}%
\pgfsys@useobject{currentmarker}{}%
\end{pgfscope}%
\begin{pgfscope}%
\pgfsys@transformshift{2.324073in}{3.037108in}%
\pgfsys@useobject{currentmarker}{}%
\end{pgfscope}%
\begin{pgfscope}%
\pgfsys@transformshift{2.342109in}{3.271219in}%
\pgfsys@useobject{currentmarker}{}%
\end{pgfscope}%
\begin{pgfscope}%
\pgfsys@transformshift{2.360145in}{3.388275in}%
\pgfsys@useobject{currentmarker}{}%
\end{pgfscope}%
\begin{pgfscope}%
\pgfsys@transformshift{2.378182in}{3.271219in}%
\pgfsys@useobject{currentmarker}{}%
\end{pgfscope}%
\begin{pgfscope}%
\pgfsys@transformshift{2.396218in}{3.037108in}%
\pgfsys@useobject{currentmarker}{}%
\end{pgfscope}%
\begin{pgfscope}%
\pgfsys@transformshift{2.414255in}{2.685942in}%
\pgfsys@useobject{currentmarker}{}%
\end{pgfscope}%
\begin{pgfscope}%
\pgfsys@transformshift{2.432291in}{2.276248in}%
\pgfsys@useobject{currentmarker}{}%
\end{pgfscope}%
\begin{pgfscope}%
\pgfsys@transformshift{2.450327in}{1.690971in}%
\pgfsys@useobject{currentmarker}{}%
\end{pgfscope}%
\begin{pgfscope}%
\pgfsys@transformshift{2.468364in}{1.398333in}%
\pgfsys@useobject{currentmarker}{}%
\end{pgfscope}%
\begin{pgfscope}%
\pgfsys@transformshift{2.486400in}{1.222749in}%
\pgfsys@useobject{currentmarker}{}%
\end{pgfscope}%
\begin{pgfscope}%
\pgfsys@transformshift{2.504436in}{1.281277in}%
\pgfsys@useobject{currentmarker}{}%
\end{pgfscope}%
\begin{pgfscope}%
\pgfsys@transformshift{2.522473in}{1.515388in}%
\pgfsys@useobject{currentmarker}{}%
\end{pgfscope}%
\begin{pgfscope}%
\pgfsys@transformshift{2.540509in}{1.983610in}%
\pgfsys@useobject{currentmarker}{}%
\end{pgfscope}%
\begin{pgfscope}%
\pgfsys@transformshift{2.558545in}{2.393304in}%
\pgfsys@useobject{currentmarker}{}%
\end{pgfscope}%
\begin{pgfscope}%
\pgfsys@transformshift{2.576582in}{2.802998in}%
\pgfsys@useobject{currentmarker}{}%
\end{pgfscope}%
\begin{pgfscope}%
\pgfsys@transformshift{2.594618in}{3.095636in}%
\pgfsys@useobject{currentmarker}{}%
\end{pgfscope}%
\begin{pgfscope}%
\pgfsys@transformshift{2.612655in}{3.271219in}%
\pgfsys@useobject{currentmarker}{}%
\end{pgfscope}%
\begin{pgfscope}%
\pgfsys@transformshift{2.630691in}{3.271219in}%
\pgfsys@useobject{currentmarker}{}%
\end{pgfscope}%
\begin{pgfscope}%
\pgfsys@transformshift{2.648727in}{3.154164in}%
\pgfsys@useobject{currentmarker}{}%
\end{pgfscope}%
\begin{pgfscope}%
\pgfsys@transformshift{2.666764in}{2.861525in}%
\pgfsys@useobject{currentmarker}{}%
\end{pgfscope}%
\begin{pgfscope}%
\pgfsys@transformshift{2.684800in}{2.510359in}%
\pgfsys@useobject{currentmarker}{}%
\end{pgfscope}%
\begin{pgfscope}%
\pgfsys@transformshift{2.702836in}{2.159193in}%
\pgfsys@useobject{currentmarker}{}%
\end{pgfscope}%
\begin{pgfscope}%
\pgfsys@transformshift{2.720873in}{1.749499in}%
\pgfsys@useobject{currentmarker}{}%
\end{pgfscope}%
\begin{pgfscope}%
\pgfsys@transformshift{2.738909in}{1.398333in}%
\pgfsys@useobject{currentmarker}{}%
\end{pgfscope}%
\begin{pgfscope}%
\pgfsys@transformshift{2.756945in}{1.339805in}%
\pgfsys@useobject{currentmarker}{}%
\end{pgfscope}%
\begin{pgfscope}%
\pgfsys@transformshift{2.774982in}{1.398333in}%
\pgfsys@useobject{currentmarker}{}%
\end{pgfscope}%
\begin{pgfscope}%
\pgfsys@transformshift{2.793018in}{1.808026in}%
\pgfsys@useobject{currentmarker}{}%
\end{pgfscope}%
\begin{pgfscope}%
\pgfsys@transformshift{2.811055in}{2.159193in}%
\pgfsys@useobject{currentmarker}{}%
\end{pgfscope}%
\begin{pgfscope}%
\pgfsys@transformshift{2.829091in}{2.568887in}%
\pgfsys@useobject{currentmarker}{}%
\end{pgfscope}%
\begin{pgfscope}%
\pgfsys@transformshift{2.847127in}{2.861525in}%
\pgfsys@useobject{currentmarker}{}%
\end{pgfscope}%
\begin{pgfscope}%
\pgfsys@transformshift{2.865164in}{3.154164in}%
\pgfsys@useobject{currentmarker}{}%
\end{pgfscope}%
\begin{pgfscope}%
\pgfsys@transformshift{2.883200in}{3.212692in}%
\pgfsys@useobject{currentmarker}{}%
\end{pgfscope}%
\begin{pgfscope}%
\pgfsys@transformshift{2.901236in}{3.212692in}%
\pgfsys@useobject{currentmarker}{}%
\end{pgfscope}%
\begin{pgfscope}%
\pgfsys@transformshift{2.919273in}{3.037108in}%
\pgfsys@useobject{currentmarker}{}%
\end{pgfscope}%
\begin{pgfscope}%
\pgfsys@transformshift{2.937309in}{2.685942in}%
\pgfsys@useobject{currentmarker}{}%
\end{pgfscope}%
\begin{pgfscope}%
\pgfsys@transformshift{2.955345in}{2.393304in}%
\pgfsys@useobject{currentmarker}{}%
\end{pgfscope}%
\begin{pgfscope}%
\pgfsys@transformshift{2.973382in}{2.042137in}%
\pgfsys@useobject{currentmarker}{}%
\end{pgfscope}%
\begin{pgfscope}%
\pgfsys@transformshift{2.991418in}{1.749499in}%
\pgfsys@useobject{currentmarker}{}%
\end{pgfscope}%
\begin{pgfscope}%
\pgfsys@transformshift{3.009455in}{1.398333in}%
\pgfsys@useobject{currentmarker}{}%
\end{pgfscope}%
\begin{pgfscope}%
\pgfsys@transformshift{3.027491in}{1.398333in}%
\pgfsys@useobject{currentmarker}{}%
\end{pgfscope}%
\begin{pgfscope}%
\pgfsys@transformshift{3.045527in}{1.573916in}%
\pgfsys@useobject{currentmarker}{}%
\end{pgfscope}%
\begin{pgfscope}%
\pgfsys@transformshift{3.063564in}{1.983610in}%
\pgfsys@useobject{currentmarker}{}%
\end{pgfscope}%
\begin{pgfscope}%
\pgfsys@transformshift{3.081600in}{2.334776in}%
\pgfsys@useobject{currentmarker}{}%
\end{pgfscope}%
\begin{pgfscope}%
\pgfsys@transformshift{3.099636in}{2.627414in}%
\pgfsys@useobject{currentmarker}{}%
\end{pgfscope}%
\begin{pgfscope}%
\pgfsys@transformshift{3.117673in}{2.920053in}%
\pgfsys@useobject{currentmarker}{}%
\end{pgfscope}%
\begin{pgfscope}%
\pgfsys@transformshift{3.135709in}{3.095636in}%
\pgfsys@useobject{currentmarker}{}%
\end{pgfscope}%
\begin{pgfscope}%
\pgfsys@transformshift{3.153745in}{3.154164in}%
\pgfsys@useobject{currentmarker}{}%
\end{pgfscope}%
\begin{pgfscope}%
\pgfsys@transformshift{3.171782in}{3.095636in}%
\pgfsys@useobject{currentmarker}{}%
\end{pgfscope}%
\begin{pgfscope}%
\pgfsys@transformshift{3.189818in}{2.861525in}%
\pgfsys@useobject{currentmarker}{}%
\end{pgfscope}%
\begin{pgfscope}%
\pgfsys@transformshift{3.207855in}{2.568887in}%
\pgfsys@useobject{currentmarker}{}%
\end{pgfscope}%
\begin{pgfscope}%
\pgfsys@transformshift{3.225891in}{2.276248in}%
\pgfsys@useobject{currentmarker}{}%
\end{pgfscope}%
\begin{pgfscope}%
\pgfsys@transformshift{3.243927in}{1.925082in}%
\pgfsys@useobject{currentmarker}{}%
\end{pgfscope}%
\begin{pgfscope}%
\pgfsys@transformshift{3.261964in}{1.573916in}%
\pgfsys@useobject{currentmarker}{}%
\end{pgfscope}%
\begin{pgfscope}%
\pgfsys@transformshift{3.280000in}{1.456860in}%
\pgfsys@useobject{currentmarker}{}%
\end{pgfscope}%
\begin{pgfscope}%
\pgfsys@transformshift{3.298036in}{1.515388in}%
\pgfsys@useobject{currentmarker}{}%
\end{pgfscope}%
\begin{pgfscope}%
\pgfsys@transformshift{3.316073in}{1.690971in}%
\pgfsys@useobject{currentmarker}{}%
\end{pgfscope}%
\begin{pgfscope}%
\pgfsys@transformshift{3.334109in}{2.100665in}%
\pgfsys@useobject{currentmarker}{}%
\end{pgfscope}%
\begin{pgfscope}%
\pgfsys@transformshift{3.352145in}{2.451831in}%
\pgfsys@useobject{currentmarker}{}%
\end{pgfscope}%
\begin{pgfscope}%
\pgfsys@transformshift{3.370182in}{2.744470in}%
\pgfsys@useobject{currentmarker}{}%
\end{pgfscope}%
\begin{pgfscope}%
\pgfsys@transformshift{3.388218in}{2.978581in}%
\pgfsys@useobject{currentmarker}{}%
\end{pgfscope}%
\begin{pgfscope}%
\pgfsys@transformshift{3.406255in}{3.095636in}%
\pgfsys@useobject{currentmarker}{}%
\end{pgfscope}%
\begin{pgfscope}%
\pgfsys@transformshift{3.424291in}{3.095636in}%
\pgfsys@useobject{currentmarker}{}%
\end{pgfscope}%
\begin{pgfscope}%
\pgfsys@transformshift{3.442327in}{2.978581in}%
\pgfsys@useobject{currentmarker}{}%
\end{pgfscope}%
\begin{pgfscope}%
\pgfsys@transformshift{3.460364in}{2.744470in}%
\pgfsys@useobject{currentmarker}{}%
\end{pgfscope}%
\begin{pgfscope}%
\pgfsys@transformshift{3.478400in}{2.451831in}%
\pgfsys@useobject{currentmarker}{}%
\end{pgfscope}%
\begin{pgfscope}%
\pgfsys@transformshift{3.496436in}{2.159193in}%
\pgfsys@useobject{currentmarker}{}%
\end{pgfscope}%
\begin{pgfscope}%
\pgfsys@transformshift{3.514473in}{1.866554in}%
\pgfsys@useobject{currentmarker}{}%
\end{pgfscope}%
\begin{pgfscope}%
\pgfsys@transformshift{3.532509in}{1.573916in}%
\pgfsys@useobject{currentmarker}{}%
\end{pgfscope}%
\begin{pgfscope}%
\pgfsys@transformshift{3.550545in}{1.515388in}%
\pgfsys@useobject{currentmarker}{}%
\end{pgfscope}%
\begin{pgfscope}%
\pgfsys@transformshift{3.568582in}{1.573916in}%
\pgfsys@useobject{currentmarker}{}%
\end{pgfscope}%
\begin{pgfscope}%
\pgfsys@transformshift{3.586618in}{1.925082in}%
\pgfsys@useobject{currentmarker}{}%
\end{pgfscope}%
\begin{pgfscope}%
\pgfsys@transformshift{3.604655in}{2.217720in}%
\pgfsys@useobject{currentmarker}{}%
\end{pgfscope}%
\begin{pgfscope}%
\pgfsys@transformshift{3.622691in}{2.568887in}%
\pgfsys@useobject{currentmarker}{}%
\end{pgfscope}%
\begin{pgfscope}%
\pgfsys@transformshift{3.640727in}{2.802998in}%
\pgfsys@useobject{currentmarker}{}%
\end{pgfscope}%
\begin{pgfscope}%
\pgfsys@transformshift{3.658764in}{2.978581in}%
\pgfsys@useobject{currentmarker}{}%
\end{pgfscope}%
\begin{pgfscope}%
\pgfsys@transformshift{3.676800in}{3.037108in}%
\pgfsys@useobject{currentmarker}{}%
\end{pgfscope}%
\begin{pgfscope}%
\pgfsys@transformshift{3.694836in}{3.037108in}%
\pgfsys@useobject{currentmarker}{}%
\end{pgfscope}%
\begin{pgfscope}%
\pgfsys@transformshift{3.712873in}{2.861525in}%
\pgfsys@useobject{currentmarker}{}%
\end{pgfscope}%
\begin{pgfscope}%
\pgfsys@transformshift{3.730909in}{2.627414in}%
\pgfsys@useobject{currentmarker}{}%
\end{pgfscope}%
\begin{pgfscope}%
\pgfsys@transformshift{3.748945in}{2.334776in}%
\pgfsys@useobject{currentmarker}{}%
\end{pgfscope}%
\begin{pgfscope}%
\pgfsys@transformshift{3.766982in}{2.100665in}%
\pgfsys@useobject{currentmarker}{}%
\end{pgfscope}%
\begin{pgfscope}%
\pgfsys@transformshift{3.785018in}{1.866554in}%
\pgfsys@useobject{currentmarker}{}%
\end{pgfscope}%
\begin{pgfscope}%
\pgfsys@transformshift{3.803055in}{1.749499in}%
\pgfsys@useobject{currentmarker}{}%
\end{pgfscope}%
\begin{pgfscope}%
\pgfsys@transformshift{3.821091in}{1.749499in}%
\pgfsys@useobject{currentmarker}{}%
\end{pgfscope}%
\begin{pgfscope}%
\pgfsys@transformshift{3.839127in}{1.866554in}%
\pgfsys@useobject{currentmarker}{}%
\end{pgfscope}%
\begin{pgfscope}%
\pgfsys@transformshift{3.857164in}{2.100665in}%
\pgfsys@useobject{currentmarker}{}%
\end{pgfscope}%
\begin{pgfscope}%
\pgfsys@transformshift{3.875200in}{2.393304in}%
\pgfsys@useobject{currentmarker}{}%
\end{pgfscope}%
\begin{pgfscope}%
\pgfsys@transformshift{3.893236in}{2.627414in}%
\pgfsys@useobject{currentmarker}{}%
\end{pgfscope}%
\begin{pgfscope}%
\pgfsys@transformshift{3.911273in}{2.861525in}%
\pgfsys@useobject{currentmarker}{}%
\end{pgfscope}%
\begin{pgfscope}%
\pgfsys@transformshift{3.929309in}{2.978581in}%
\pgfsys@useobject{currentmarker}{}%
\end{pgfscope}%
\begin{pgfscope}%
\pgfsys@transformshift{3.947345in}{3.037108in}%
\pgfsys@useobject{currentmarker}{}%
\end{pgfscope}%
\begin{pgfscope}%
\pgfsys@transformshift{3.965382in}{2.920053in}%
\pgfsys@useobject{currentmarker}{}%
\end{pgfscope}%
\begin{pgfscope}%
\pgfsys@transformshift{3.983418in}{2.744470in}%
\pgfsys@useobject{currentmarker}{}%
\end{pgfscope}%
\begin{pgfscope}%
\pgfsys@transformshift{4.001455in}{2.510359in}%
\pgfsys@useobject{currentmarker}{}%
\end{pgfscope}%
\begin{pgfscope}%
\pgfsys@transformshift{4.019491in}{2.276248in}%
\pgfsys@useobject{currentmarker}{}%
\end{pgfscope}%
\begin{pgfscope}%
\pgfsys@transformshift{4.037527in}{1.983610in}%
\pgfsys@useobject{currentmarker}{}%
\end{pgfscope}%
\begin{pgfscope}%
\pgfsys@transformshift{4.055564in}{1.808026in}%
\pgfsys@useobject{currentmarker}{}%
\end{pgfscope}%
\begin{pgfscope}%
\pgfsys@transformshift{4.073600in}{1.749499in}%
\pgfsys@useobject{currentmarker}{}%
\end{pgfscope}%
\begin{pgfscope}%
\pgfsys@transformshift{4.091636in}{1.808026in}%
\pgfsys@useobject{currentmarker}{}%
\end{pgfscope}%
\begin{pgfscope}%
\pgfsys@transformshift{4.109673in}{1.983610in}%
\pgfsys@useobject{currentmarker}{}%
\end{pgfscope}%
\begin{pgfscope}%
\pgfsys@transformshift{4.127709in}{2.217720in}%
\pgfsys@useobject{currentmarker}{}%
\end{pgfscope}%
\begin{pgfscope}%
\pgfsys@transformshift{4.145745in}{2.451831in}%
\pgfsys@useobject{currentmarker}{}%
\end{pgfscope}%
\begin{pgfscope}%
\pgfsys@transformshift{4.163782in}{2.685942in}%
\pgfsys@useobject{currentmarker}{}%
\end{pgfscope}%
\begin{pgfscope}%
\pgfsys@transformshift{4.181818in}{2.861525in}%
\pgfsys@useobject{currentmarker}{}%
\end{pgfscope}%
\begin{pgfscope}%
\pgfsys@transformshift{4.199855in}{2.978581in}%
\pgfsys@useobject{currentmarker}{}%
\end{pgfscope}%
\begin{pgfscope}%
\pgfsys@transformshift{4.217891in}{2.978581in}%
\pgfsys@useobject{currentmarker}{}%
\end{pgfscope}%
\begin{pgfscope}%
\pgfsys@transformshift{4.235927in}{2.861525in}%
\pgfsys@useobject{currentmarker}{}%
\end{pgfscope}%
\begin{pgfscope}%
\pgfsys@transformshift{4.253964in}{2.627414in}%
\pgfsys@useobject{currentmarker}{}%
\end{pgfscope}%
\begin{pgfscope}%
\pgfsys@transformshift{4.272000in}{2.393304in}%
\pgfsys@useobject{currentmarker}{}%
\end{pgfscope}%
\begin{pgfscope}%
\pgfsys@transformshift{4.290036in}{2.159193in}%
\pgfsys@useobject{currentmarker}{}%
\end{pgfscope}%
\begin{pgfscope}%
\pgfsys@transformshift{4.308073in}{1.983610in}%
\pgfsys@useobject{currentmarker}{}%
\end{pgfscope}%
\begin{pgfscope}%
\pgfsys@transformshift{4.326109in}{1.808026in}%
\pgfsys@useobject{currentmarker}{}%
\end{pgfscope}%
\begin{pgfscope}%
\pgfsys@transformshift{4.344145in}{1.808026in}%
\pgfsys@useobject{currentmarker}{}%
\end{pgfscope}%
\begin{pgfscope}%
\pgfsys@transformshift{4.362182in}{1.925082in}%
\pgfsys@useobject{currentmarker}{}%
\end{pgfscope}%
\begin{pgfscope}%
\pgfsys@transformshift{4.380218in}{2.100665in}%
\pgfsys@useobject{currentmarker}{}%
\end{pgfscope}%
\begin{pgfscope}%
\pgfsys@transformshift{4.398255in}{2.334776in}%
\pgfsys@useobject{currentmarker}{}%
\end{pgfscope}%
\begin{pgfscope}%
\pgfsys@transformshift{4.416291in}{2.568887in}%
\pgfsys@useobject{currentmarker}{}%
\end{pgfscope}%
\begin{pgfscope}%
\pgfsys@transformshift{4.434327in}{2.744470in}%
\pgfsys@useobject{currentmarker}{}%
\end{pgfscope}%
\begin{pgfscope}%
\pgfsys@transformshift{4.452364in}{2.861525in}%
\pgfsys@useobject{currentmarker}{}%
\end{pgfscope}%
\begin{pgfscope}%
\pgfsys@transformshift{4.470400in}{2.920053in}%
\pgfsys@useobject{currentmarker}{}%
\end{pgfscope}%
\begin{pgfscope}%
\pgfsys@transformshift{4.488436in}{2.861525in}%
\pgfsys@useobject{currentmarker}{}%
\end{pgfscope}%
\begin{pgfscope}%
\pgfsys@transformshift{4.506473in}{2.744470in}%
\pgfsys@useobject{currentmarker}{}%
\end{pgfscope}%
\begin{pgfscope}%
\pgfsys@transformshift{4.524509in}{2.568887in}%
\pgfsys@useobject{currentmarker}{}%
\end{pgfscope}%
\begin{pgfscope}%
\pgfsys@transformshift{4.542545in}{2.334776in}%
\pgfsys@useobject{currentmarker}{}%
\end{pgfscope}%
\begin{pgfscope}%
\pgfsys@transformshift{4.560582in}{2.100665in}%
\pgfsys@useobject{currentmarker}{}%
\end{pgfscope}%
\begin{pgfscope}%
\pgfsys@transformshift{4.578618in}{1.925082in}%
\pgfsys@useobject{currentmarker}{}%
\end{pgfscope}%
\begin{pgfscope}%
\pgfsys@transformshift{4.596655in}{1.866554in}%
\pgfsys@useobject{currentmarker}{}%
\end{pgfscope}%
\begin{pgfscope}%
\pgfsys@transformshift{4.614691in}{1.866554in}%
\pgfsys@useobject{currentmarker}{}%
\end{pgfscope}%
\begin{pgfscope}%
\pgfsys@transformshift{4.632727in}{1.983610in}%
\pgfsys@useobject{currentmarker}{}%
\end{pgfscope}%
\begin{pgfscope}%
\pgfsys@transformshift{4.650764in}{2.159193in}%
\pgfsys@useobject{currentmarker}{}%
\end{pgfscope}%
\begin{pgfscope}%
\pgfsys@transformshift{4.668800in}{2.393304in}%
\pgfsys@useobject{currentmarker}{}%
\end{pgfscope}%
\begin{pgfscope}%
\pgfsys@transformshift{4.686836in}{2.627414in}%
\pgfsys@useobject{currentmarker}{}%
\end{pgfscope}%
\begin{pgfscope}%
\pgfsys@transformshift{4.704873in}{2.802998in}%
\pgfsys@useobject{currentmarker}{}%
\end{pgfscope}%
\begin{pgfscope}%
\pgfsys@transformshift{4.722909in}{2.861525in}%
\pgfsys@useobject{currentmarker}{}%
\end{pgfscope}%
\begin{pgfscope}%
\pgfsys@transformshift{4.740945in}{2.920053in}%
\pgfsys@useobject{currentmarker}{}%
\end{pgfscope}%
\begin{pgfscope}%
\pgfsys@transformshift{4.758982in}{2.802998in}%
\pgfsys@useobject{currentmarker}{}%
\end{pgfscope}%
\begin{pgfscope}%
\pgfsys@transformshift{4.777018in}{2.627414in}%
\pgfsys@useobject{currentmarker}{}%
\end{pgfscope}%
\begin{pgfscope}%
\pgfsys@transformshift{4.795055in}{2.451831in}%
\pgfsys@useobject{currentmarker}{}%
\end{pgfscope}%
\begin{pgfscope}%
\pgfsys@transformshift{4.813091in}{2.217720in}%
\pgfsys@useobject{currentmarker}{}%
\end{pgfscope}%
\begin{pgfscope}%
\pgfsys@transformshift{4.831127in}{2.042137in}%
\pgfsys@useobject{currentmarker}{}%
\end{pgfscope}%
\begin{pgfscope}%
\pgfsys@transformshift{4.849164in}{1.925082in}%
\pgfsys@useobject{currentmarker}{}%
\end{pgfscope}%
\begin{pgfscope}%
\pgfsys@transformshift{4.867200in}{1.925082in}%
\pgfsys@useobject{currentmarker}{}%
\end{pgfscope}%
\begin{pgfscope}%
\pgfsys@transformshift{4.885236in}{1.925082in}%
\pgfsys@useobject{currentmarker}{}%
\end{pgfscope}%
\begin{pgfscope}%
\pgfsys@transformshift{4.903273in}{2.100665in}%
\pgfsys@useobject{currentmarker}{}%
\end{pgfscope}%
\begin{pgfscope}%
\pgfsys@transformshift{4.921309in}{2.276248in}%
\pgfsys@useobject{currentmarker}{}%
\end{pgfscope}%
\begin{pgfscope}%
\pgfsys@transformshift{4.939345in}{2.510359in}%
\pgfsys@useobject{currentmarker}{}%
\end{pgfscope}%
\begin{pgfscope}%
\pgfsys@transformshift{4.957382in}{2.685942in}%
\pgfsys@useobject{currentmarker}{}%
\end{pgfscope}%
\begin{pgfscope}%
\pgfsys@transformshift{4.975418in}{2.802998in}%
\pgfsys@useobject{currentmarker}{}%
\end{pgfscope}%
\begin{pgfscope}%
\pgfsys@transformshift{4.993455in}{2.861525in}%
\pgfsys@useobject{currentmarker}{}%
\end{pgfscope}%
\begin{pgfscope}%
\pgfsys@transformshift{5.011491in}{2.861525in}%
\pgfsys@useobject{currentmarker}{}%
\end{pgfscope}%
\begin{pgfscope}%
\pgfsys@transformshift{5.029527in}{2.744470in}%
\pgfsys@useobject{currentmarker}{}%
\end{pgfscope}%
\begin{pgfscope}%
\pgfsys@transformshift{5.047564in}{2.568887in}%
\pgfsys@useobject{currentmarker}{}%
\end{pgfscope}%
\begin{pgfscope}%
\pgfsys@transformshift{5.065600in}{2.393304in}%
\pgfsys@useobject{currentmarker}{}%
\end{pgfscope}%
\begin{pgfscope}%
\pgfsys@transformshift{5.083636in}{2.159193in}%
\pgfsys@useobject{currentmarker}{}%
\end{pgfscope}%
\begin{pgfscope}%
\pgfsys@transformshift{5.101673in}{2.042137in}%
\pgfsys@useobject{currentmarker}{}%
\end{pgfscope}%
\begin{pgfscope}%
\pgfsys@transformshift{5.119709in}{1.925082in}%
\pgfsys@useobject{currentmarker}{}%
\end{pgfscope}%
\begin{pgfscope}%
\pgfsys@transformshift{5.137745in}{1.925082in}%
\pgfsys@useobject{currentmarker}{}%
\end{pgfscope}%
\begin{pgfscope}%
\pgfsys@transformshift{5.155782in}{2.042137in}%
\pgfsys@useobject{currentmarker}{}%
\end{pgfscope}%
\begin{pgfscope}%
\pgfsys@transformshift{5.173818in}{2.159193in}%
\pgfsys@useobject{currentmarker}{}%
\end{pgfscope}%
\begin{pgfscope}%
\pgfsys@transformshift{5.191855in}{2.334776in}%
\pgfsys@useobject{currentmarker}{}%
\end{pgfscope}%
\begin{pgfscope}%
\pgfsys@transformshift{5.209891in}{2.568887in}%
\pgfsys@useobject{currentmarker}{}%
\end{pgfscope}%
\begin{pgfscope}%
\pgfsys@transformshift{5.227927in}{2.685942in}%
\pgfsys@useobject{currentmarker}{}%
\end{pgfscope}%
\begin{pgfscope}%
\pgfsys@transformshift{5.245964in}{2.802998in}%
\pgfsys@useobject{currentmarker}{}%
\end{pgfscope}%
\begin{pgfscope}%
\pgfsys@transformshift{5.264000in}{2.861525in}%
\pgfsys@useobject{currentmarker}{}%
\end{pgfscope}%
\begin{pgfscope}%
\pgfsys@transformshift{5.282036in}{2.802998in}%
\pgfsys@useobject{currentmarker}{}%
\end{pgfscope}%
\begin{pgfscope}%
\pgfsys@transformshift{5.300073in}{2.685942in}%
\pgfsys@useobject{currentmarker}{}%
\end{pgfscope}%
\begin{pgfscope}%
\pgfsys@transformshift{5.318109in}{2.510359in}%
\pgfsys@useobject{currentmarker}{}%
\end{pgfscope}%
\begin{pgfscope}%
\pgfsys@transformshift{5.336145in}{2.276248in}%
\pgfsys@useobject{currentmarker}{}%
\end{pgfscope}%
\begin{pgfscope}%
\pgfsys@transformshift{5.354182in}{2.159193in}%
\pgfsys@useobject{currentmarker}{}%
\end{pgfscope}%
\begin{pgfscope}%
\pgfsys@transformshift{5.372218in}{1.983610in}%
\pgfsys@useobject{currentmarker}{}%
\end{pgfscope}%
\begin{pgfscope}%
\pgfsys@transformshift{5.390255in}{1.983610in}%
\pgfsys@useobject{currentmarker}{}%
\end{pgfscope}%
\begin{pgfscope}%
\pgfsys@transformshift{5.408291in}{1.983610in}%
\pgfsys@useobject{currentmarker}{}%
\end{pgfscope}%
\begin{pgfscope}%
\pgfsys@transformshift{5.426327in}{2.100665in}%
\pgfsys@useobject{currentmarker}{}%
\end{pgfscope}%
\begin{pgfscope}%
\pgfsys@transformshift{5.444364in}{2.276248in}%
\pgfsys@useobject{currentmarker}{}%
\end{pgfscope}%
\begin{pgfscope}%
\pgfsys@transformshift{5.462400in}{2.451831in}%
\pgfsys@useobject{currentmarker}{}%
\end{pgfscope}%
\begin{pgfscope}%
\pgfsys@transformshift{5.480436in}{2.627414in}%
\pgfsys@useobject{currentmarker}{}%
\end{pgfscope}%
\begin{pgfscope}%
\pgfsys@transformshift{5.498473in}{2.744470in}%
\pgfsys@useobject{currentmarker}{}%
\end{pgfscope}%
\begin{pgfscope}%
\pgfsys@transformshift{5.516509in}{2.802998in}%
\pgfsys@useobject{currentmarker}{}%
\end{pgfscope}%
\begin{pgfscope}%
\pgfsys@transformshift{5.534545in}{2.802998in}%
\pgfsys@useobject{currentmarker}{}%
\end{pgfscope}%
\end{pgfscope}%
\begin{pgfscope}%
\pgfpathrectangle{\pgfqpoint{0.800000in}{0.528000in}}{\pgfqpoint{4.960000in}{3.696000in}}%
\pgfusepath{clip}%
\pgfsetbuttcap%
\pgfsetroundjoin%
\definecolor{currentfill}{rgb}{0.000000,0.000000,1.000000}%
\pgfsetfillcolor{currentfill}%
\pgfsetlinewidth{0.501875pt}%
\definecolor{currentstroke}{rgb}{0.000000,0.000000,1.000000}%
\pgfsetstrokecolor{currentstroke}%
\pgfsetdash{}{0pt}%
\pgfsys@defobject{currentmarker}{\pgfqpoint{-0.027778in}{-0.000000in}}{\pgfqpoint{0.027778in}{0.000000in}}{%
\pgfpathmoveto{\pgfqpoint{0.027778in}{-0.000000in}}%
\pgfpathlineto{\pgfqpoint{-0.027778in}{0.000000in}}%
\pgfusepath{stroke,fill}%
}%
\begin{pgfscope}%
\pgfsys@transformshift{1.043491in}{3.973552in}%
\pgfsys@useobject{currentmarker}{}%
\end{pgfscope}%
\begin{pgfscope}%
\pgfsys@transformshift{1.061527in}{3.797969in}%
\pgfsys@useobject{currentmarker}{}%
\end{pgfscope}%
\begin{pgfscope}%
\pgfsys@transformshift{1.079564in}{3.329747in}%
\pgfsys@useobject{currentmarker}{}%
\end{pgfscope}%
\begin{pgfscope}%
\pgfsys@transformshift{1.097600in}{2.744470in}%
\pgfsys@useobject{currentmarker}{}%
\end{pgfscope}%
\begin{pgfscope}%
\pgfsys@transformshift{1.115636in}{2.100665in}%
\pgfsys@useobject{currentmarker}{}%
\end{pgfscope}%
\begin{pgfscope}%
\pgfsys@transformshift{1.133673in}{1.339805in}%
\pgfsys@useobject{currentmarker}{}%
\end{pgfscope}%
\begin{pgfscope}%
\pgfsys@transformshift{1.151709in}{0.930111in}%
\pgfsys@useobject{currentmarker}{}%
\end{pgfscope}%
\begin{pgfscope}%
\pgfsys@transformshift{1.169745in}{0.813055in}%
\pgfsys@useobject{currentmarker}{}%
\end{pgfscope}%
\begin{pgfscope}%
\pgfsys@transformshift{1.187782in}{0.930111in}%
\pgfsys@useobject{currentmarker}{}%
\end{pgfscope}%
\begin{pgfscope}%
\pgfsys@transformshift{1.205818in}{1.339805in}%
\pgfsys@useobject{currentmarker}{}%
\end{pgfscope}%
\begin{pgfscope}%
\pgfsys@transformshift{1.223855in}{2.100665in}%
\pgfsys@useobject{currentmarker}{}%
\end{pgfscope}%
\begin{pgfscope}%
\pgfsys@transformshift{1.241891in}{2.685942in}%
\pgfsys@useobject{currentmarker}{}%
\end{pgfscope}%
\begin{pgfscope}%
\pgfsys@transformshift{1.259927in}{3.271219in}%
\pgfsys@useobject{currentmarker}{}%
\end{pgfscope}%
\begin{pgfscope}%
\pgfsys@transformshift{1.277964in}{3.680913in}%
\pgfsys@useobject{currentmarker}{}%
\end{pgfscope}%
\begin{pgfscope}%
\pgfsys@transformshift{1.296000in}{3.856496in}%
\pgfsys@useobject{currentmarker}{}%
\end{pgfscope}%
\begin{pgfscope}%
\pgfsys@transformshift{1.314036in}{3.856496in}%
\pgfsys@useobject{currentmarker}{}%
\end{pgfscope}%
\begin{pgfscope}%
\pgfsys@transformshift{1.332073in}{3.563858in}%
\pgfsys@useobject{currentmarker}{}%
\end{pgfscope}%
\begin{pgfscope}%
\pgfsys@transformshift{1.350109in}{3.095636in}%
\pgfsys@useobject{currentmarker}{}%
\end{pgfscope}%
\begin{pgfscope}%
\pgfsys@transformshift{1.368145in}{2.510359in}%
\pgfsys@useobject{currentmarker}{}%
\end{pgfscope}%
\begin{pgfscope}%
\pgfsys@transformshift{1.386182in}{1.983610in}%
\pgfsys@useobject{currentmarker}{}%
\end{pgfscope}%
\begin{pgfscope}%
\pgfsys@transformshift{1.404218in}{1.281277in}%
\pgfsys@useobject{currentmarker}{}%
\end{pgfscope}%
\begin{pgfscope}%
\pgfsys@transformshift{1.422255in}{0.988639in}%
\pgfsys@useobject{currentmarker}{}%
\end{pgfscope}%
\begin{pgfscope}%
\pgfsys@transformshift{1.440291in}{0.988639in}%
\pgfsys@useobject{currentmarker}{}%
\end{pgfscope}%
\begin{pgfscope}%
\pgfsys@transformshift{1.458327in}{1.164222in}%
\pgfsys@useobject{currentmarker}{}%
\end{pgfscope}%
\begin{pgfscope}%
\pgfsys@transformshift{1.476364in}{1.573916in}%
\pgfsys@useobject{currentmarker}{}%
\end{pgfscope}%
\begin{pgfscope}%
\pgfsys@transformshift{1.494400in}{2.276248in}%
\pgfsys@useobject{currentmarker}{}%
\end{pgfscope}%
\begin{pgfscope}%
\pgfsys@transformshift{1.512436in}{2.861525in}%
\pgfsys@useobject{currentmarker}{}%
\end{pgfscope}%
\begin{pgfscope}%
\pgfsys@transformshift{1.530473in}{3.329747in}%
\pgfsys@useobject{currentmarker}{}%
\end{pgfscope}%
\begin{pgfscope}%
\pgfsys@transformshift{1.548509in}{3.622386in}%
\pgfsys@useobject{currentmarker}{}%
\end{pgfscope}%
\begin{pgfscope}%
\pgfsys@transformshift{1.566545in}{3.739441in}%
\pgfsys@useobject{currentmarker}{}%
\end{pgfscope}%
\begin{pgfscope}%
\pgfsys@transformshift{1.584582in}{3.680913in}%
\pgfsys@useobject{currentmarker}{}%
\end{pgfscope}%
\begin{pgfscope}%
\pgfsys@transformshift{1.602618in}{3.329747in}%
\pgfsys@useobject{currentmarker}{}%
\end{pgfscope}%
\begin{pgfscope}%
\pgfsys@transformshift{1.620655in}{2.861525in}%
\pgfsys@useobject{currentmarker}{}%
\end{pgfscope}%
\begin{pgfscope}%
\pgfsys@transformshift{1.638691in}{2.334776in}%
\pgfsys@useobject{currentmarker}{}%
\end{pgfscope}%
\begin{pgfscope}%
\pgfsys@transformshift{1.656727in}{1.690971in}%
\pgfsys@useobject{currentmarker}{}%
\end{pgfscope}%
\begin{pgfscope}%
\pgfsys@transformshift{1.674764in}{1.281277in}%
\pgfsys@useobject{currentmarker}{}%
\end{pgfscope}%
\begin{pgfscope}%
\pgfsys@transformshift{1.692800in}{1.105694in}%
\pgfsys@useobject{currentmarker}{}%
\end{pgfscope}%
\begin{pgfscope}%
\pgfsys@transformshift{1.710836in}{1.105694in}%
\pgfsys@useobject{currentmarker}{}%
\end{pgfscope}%
\begin{pgfscope}%
\pgfsys@transformshift{1.728873in}{1.398333in}%
\pgfsys@useobject{currentmarker}{}%
\end{pgfscope}%
\begin{pgfscope}%
\pgfsys@transformshift{1.746909in}{1.983610in}%
\pgfsys@useobject{currentmarker}{}%
\end{pgfscope}%
\begin{pgfscope}%
\pgfsys@transformshift{1.764945in}{2.510359in}%
\pgfsys@useobject{currentmarker}{}%
\end{pgfscope}%
\begin{pgfscope}%
\pgfsys@transformshift{1.782982in}{2.978581in}%
\pgfsys@useobject{currentmarker}{}%
\end{pgfscope}%
\begin{pgfscope}%
\pgfsys@transformshift{1.801018in}{3.388275in}%
\pgfsys@useobject{currentmarker}{}%
\end{pgfscope}%
\begin{pgfscope}%
\pgfsys@transformshift{1.819055in}{3.622386in}%
\pgfsys@useobject{currentmarker}{}%
\end{pgfscope}%
\begin{pgfscope}%
\pgfsys@transformshift{1.837091in}{3.622386in}%
\pgfsys@useobject{currentmarker}{}%
\end{pgfscope}%
\begin{pgfscope}%
\pgfsys@transformshift{1.855127in}{3.505330in}%
\pgfsys@useobject{currentmarker}{}%
\end{pgfscope}%
\begin{pgfscope}%
\pgfsys@transformshift{1.873164in}{3.154164in}%
\pgfsys@useobject{currentmarker}{}%
\end{pgfscope}%
\begin{pgfscope}%
\pgfsys@transformshift{1.891200in}{2.685942in}%
\pgfsys@useobject{currentmarker}{}%
\end{pgfscope}%
\begin{pgfscope}%
\pgfsys@transformshift{1.909236in}{2.217720in}%
\pgfsys@useobject{currentmarker}{}%
\end{pgfscope}%
\begin{pgfscope}%
\pgfsys@transformshift{1.927273in}{1.573916in}%
\pgfsys@useobject{currentmarker}{}%
\end{pgfscope}%
\begin{pgfscope}%
\pgfsys@transformshift{1.945309in}{1.281277in}%
\pgfsys@useobject{currentmarker}{}%
\end{pgfscope}%
\begin{pgfscope}%
\pgfsys@transformshift{1.963345in}{1.164222in}%
\pgfsys@useobject{currentmarker}{}%
\end{pgfscope}%
\begin{pgfscope}%
\pgfsys@transformshift{1.981382in}{1.281277in}%
\pgfsys@useobject{currentmarker}{}%
\end{pgfscope}%
\begin{pgfscope}%
\pgfsys@transformshift{1.999418in}{1.573916in}%
\pgfsys@useobject{currentmarker}{}%
\end{pgfscope}%
\begin{pgfscope}%
\pgfsys@transformshift{2.017455in}{2.159193in}%
\pgfsys@useobject{currentmarker}{}%
\end{pgfscope}%
\begin{pgfscope}%
\pgfsys@transformshift{2.035491in}{2.685942in}%
\pgfsys@useobject{currentmarker}{}%
\end{pgfscope}%
\begin{pgfscope}%
\pgfsys@transformshift{2.053527in}{3.095636in}%
\pgfsys@useobject{currentmarker}{}%
\end{pgfscope}%
\begin{pgfscope}%
\pgfsys@transformshift{2.071564in}{3.388275in}%
\pgfsys@useobject{currentmarker}{}%
\end{pgfscope}%
\begin{pgfscope}%
\pgfsys@transformshift{2.089600in}{3.563858in}%
\pgfsys@useobject{currentmarker}{}%
\end{pgfscope}%
\begin{pgfscope}%
\pgfsys@transformshift{2.107636in}{3.563858in}%
\pgfsys@useobject{currentmarker}{}%
\end{pgfscope}%
\begin{pgfscope}%
\pgfsys@transformshift{2.125673in}{3.329747in}%
\pgfsys@useobject{currentmarker}{}%
\end{pgfscope}%
\begin{pgfscope}%
\pgfsys@transformshift{2.143709in}{2.978581in}%
\pgfsys@useobject{currentmarker}{}%
\end{pgfscope}%
\begin{pgfscope}%
\pgfsys@transformshift{2.161745in}{2.510359in}%
\pgfsys@useobject{currentmarker}{}%
\end{pgfscope}%
\begin{pgfscope}%
\pgfsys@transformshift{2.179782in}{2.100665in}%
\pgfsys@useobject{currentmarker}{}%
\end{pgfscope}%
\begin{pgfscope}%
\pgfsys@transformshift{2.197818in}{1.515388in}%
\pgfsys@useobject{currentmarker}{}%
\end{pgfscope}%
\begin{pgfscope}%
\pgfsys@transformshift{2.215855in}{1.339805in}%
\pgfsys@useobject{currentmarker}{}%
\end{pgfscope}%
\begin{pgfscope}%
\pgfsys@transformshift{2.233891in}{1.281277in}%
\pgfsys@useobject{currentmarker}{}%
\end{pgfscope}%
\begin{pgfscope}%
\pgfsys@transformshift{2.251927in}{1.456860in}%
\pgfsys@useobject{currentmarker}{}%
\end{pgfscope}%
\begin{pgfscope}%
\pgfsys@transformshift{2.269964in}{1.808026in}%
\pgfsys@useobject{currentmarker}{}%
\end{pgfscope}%
\begin{pgfscope}%
\pgfsys@transformshift{2.288000in}{2.393304in}%
\pgfsys@useobject{currentmarker}{}%
\end{pgfscope}%
\begin{pgfscope}%
\pgfsys@transformshift{2.306036in}{2.802998in}%
\pgfsys@useobject{currentmarker}{}%
\end{pgfscope}%
\begin{pgfscope}%
\pgfsys@transformshift{2.324073in}{3.154164in}%
\pgfsys@useobject{currentmarker}{}%
\end{pgfscope}%
\begin{pgfscope}%
\pgfsys@transformshift{2.342109in}{3.388275in}%
\pgfsys@useobject{currentmarker}{}%
\end{pgfscope}%
\begin{pgfscope}%
\pgfsys@transformshift{2.360145in}{3.505330in}%
\pgfsys@useobject{currentmarker}{}%
\end{pgfscope}%
\begin{pgfscope}%
\pgfsys@transformshift{2.378182in}{3.388275in}%
\pgfsys@useobject{currentmarker}{}%
\end{pgfscope}%
\begin{pgfscope}%
\pgfsys@transformshift{2.396218in}{3.154164in}%
\pgfsys@useobject{currentmarker}{}%
\end{pgfscope}%
\begin{pgfscope}%
\pgfsys@transformshift{2.414255in}{2.802998in}%
\pgfsys@useobject{currentmarker}{}%
\end{pgfscope}%
\begin{pgfscope}%
\pgfsys@transformshift{2.432291in}{2.393304in}%
\pgfsys@useobject{currentmarker}{}%
\end{pgfscope}%
\begin{pgfscope}%
\pgfsys@transformshift{2.450327in}{1.808026in}%
\pgfsys@useobject{currentmarker}{}%
\end{pgfscope}%
\begin{pgfscope}%
\pgfsys@transformshift{2.468364in}{1.515388in}%
\pgfsys@useobject{currentmarker}{}%
\end{pgfscope}%
\begin{pgfscope}%
\pgfsys@transformshift{2.486400in}{1.339805in}%
\pgfsys@useobject{currentmarker}{}%
\end{pgfscope}%
\begin{pgfscope}%
\pgfsys@transformshift{2.504436in}{1.398333in}%
\pgfsys@useobject{currentmarker}{}%
\end{pgfscope}%
\begin{pgfscope}%
\pgfsys@transformshift{2.522473in}{1.632443in}%
\pgfsys@useobject{currentmarker}{}%
\end{pgfscope}%
\begin{pgfscope}%
\pgfsys@transformshift{2.540509in}{2.100665in}%
\pgfsys@useobject{currentmarker}{}%
\end{pgfscope}%
\begin{pgfscope}%
\pgfsys@transformshift{2.558545in}{2.510359in}%
\pgfsys@useobject{currentmarker}{}%
\end{pgfscope}%
\begin{pgfscope}%
\pgfsys@transformshift{2.576582in}{2.920053in}%
\pgfsys@useobject{currentmarker}{}%
\end{pgfscope}%
\begin{pgfscope}%
\pgfsys@transformshift{2.594618in}{3.212692in}%
\pgfsys@useobject{currentmarker}{}%
\end{pgfscope}%
\begin{pgfscope}%
\pgfsys@transformshift{2.612655in}{3.388275in}%
\pgfsys@useobject{currentmarker}{}%
\end{pgfscope}%
\begin{pgfscope}%
\pgfsys@transformshift{2.630691in}{3.388275in}%
\pgfsys@useobject{currentmarker}{}%
\end{pgfscope}%
\begin{pgfscope}%
\pgfsys@transformshift{2.648727in}{3.271219in}%
\pgfsys@useobject{currentmarker}{}%
\end{pgfscope}%
\begin{pgfscope}%
\pgfsys@transformshift{2.666764in}{2.978581in}%
\pgfsys@useobject{currentmarker}{}%
\end{pgfscope}%
\begin{pgfscope}%
\pgfsys@transformshift{2.684800in}{2.627414in}%
\pgfsys@useobject{currentmarker}{}%
\end{pgfscope}%
\begin{pgfscope}%
\pgfsys@transformshift{2.702836in}{2.276248in}%
\pgfsys@useobject{currentmarker}{}%
\end{pgfscope}%
\begin{pgfscope}%
\pgfsys@transformshift{2.720873in}{1.866554in}%
\pgfsys@useobject{currentmarker}{}%
\end{pgfscope}%
\begin{pgfscope}%
\pgfsys@transformshift{2.738909in}{1.515388in}%
\pgfsys@useobject{currentmarker}{}%
\end{pgfscope}%
\begin{pgfscope}%
\pgfsys@transformshift{2.756945in}{1.456860in}%
\pgfsys@useobject{currentmarker}{}%
\end{pgfscope}%
\begin{pgfscope}%
\pgfsys@transformshift{2.774982in}{1.515388in}%
\pgfsys@useobject{currentmarker}{}%
\end{pgfscope}%
\begin{pgfscope}%
\pgfsys@transformshift{2.793018in}{1.925082in}%
\pgfsys@useobject{currentmarker}{}%
\end{pgfscope}%
\begin{pgfscope}%
\pgfsys@transformshift{2.811055in}{2.276248in}%
\pgfsys@useobject{currentmarker}{}%
\end{pgfscope}%
\begin{pgfscope}%
\pgfsys@transformshift{2.829091in}{2.685942in}%
\pgfsys@useobject{currentmarker}{}%
\end{pgfscope}%
\begin{pgfscope}%
\pgfsys@transformshift{2.847127in}{2.978581in}%
\pgfsys@useobject{currentmarker}{}%
\end{pgfscope}%
\begin{pgfscope}%
\pgfsys@transformshift{2.865164in}{3.271219in}%
\pgfsys@useobject{currentmarker}{}%
\end{pgfscope}%
\begin{pgfscope}%
\pgfsys@transformshift{2.883200in}{3.329747in}%
\pgfsys@useobject{currentmarker}{}%
\end{pgfscope}%
\begin{pgfscope}%
\pgfsys@transformshift{2.901236in}{3.329747in}%
\pgfsys@useobject{currentmarker}{}%
\end{pgfscope}%
\begin{pgfscope}%
\pgfsys@transformshift{2.919273in}{3.154164in}%
\pgfsys@useobject{currentmarker}{}%
\end{pgfscope}%
\begin{pgfscope}%
\pgfsys@transformshift{2.937309in}{2.802998in}%
\pgfsys@useobject{currentmarker}{}%
\end{pgfscope}%
\begin{pgfscope}%
\pgfsys@transformshift{2.955345in}{2.510359in}%
\pgfsys@useobject{currentmarker}{}%
\end{pgfscope}%
\begin{pgfscope}%
\pgfsys@transformshift{2.973382in}{2.159193in}%
\pgfsys@useobject{currentmarker}{}%
\end{pgfscope}%
\begin{pgfscope}%
\pgfsys@transformshift{2.991418in}{1.866554in}%
\pgfsys@useobject{currentmarker}{}%
\end{pgfscope}%
\begin{pgfscope}%
\pgfsys@transformshift{3.009455in}{1.515388in}%
\pgfsys@useobject{currentmarker}{}%
\end{pgfscope}%
\begin{pgfscope}%
\pgfsys@transformshift{3.027491in}{1.515388in}%
\pgfsys@useobject{currentmarker}{}%
\end{pgfscope}%
\begin{pgfscope}%
\pgfsys@transformshift{3.045527in}{1.690971in}%
\pgfsys@useobject{currentmarker}{}%
\end{pgfscope}%
\begin{pgfscope}%
\pgfsys@transformshift{3.063564in}{2.100665in}%
\pgfsys@useobject{currentmarker}{}%
\end{pgfscope}%
\begin{pgfscope}%
\pgfsys@transformshift{3.081600in}{2.451831in}%
\pgfsys@useobject{currentmarker}{}%
\end{pgfscope}%
\begin{pgfscope}%
\pgfsys@transformshift{3.099636in}{2.744470in}%
\pgfsys@useobject{currentmarker}{}%
\end{pgfscope}%
\begin{pgfscope}%
\pgfsys@transformshift{3.117673in}{3.037108in}%
\pgfsys@useobject{currentmarker}{}%
\end{pgfscope}%
\begin{pgfscope}%
\pgfsys@transformshift{3.135709in}{3.212692in}%
\pgfsys@useobject{currentmarker}{}%
\end{pgfscope}%
\begin{pgfscope}%
\pgfsys@transformshift{3.153745in}{3.271219in}%
\pgfsys@useobject{currentmarker}{}%
\end{pgfscope}%
\begin{pgfscope}%
\pgfsys@transformshift{3.171782in}{3.212692in}%
\pgfsys@useobject{currentmarker}{}%
\end{pgfscope}%
\begin{pgfscope}%
\pgfsys@transformshift{3.189818in}{2.978581in}%
\pgfsys@useobject{currentmarker}{}%
\end{pgfscope}%
\begin{pgfscope}%
\pgfsys@transformshift{3.207855in}{2.685942in}%
\pgfsys@useobject{currentmarker}{}%
\end{pgfscope}%
\begin{pgfscope}%
\pgfsys@transformshift{3.225891in}{2.393304in}%
\pgfsys@useobject{currentmarker}{}%
\end{pgfscope}%
\begin{pgfscope}%
\pgfsys@transformshift{3.243927in}{2.042137in}%
\pgfsys@useobject{currentmarker}{}%
\end{pgfscope}%
\begin{pgfscope}%
\pgfsys@transformshift{3.261964in}{1.690971in}%
\pgfsys@useobject{currentmarker}{}%
\end{pgfscope}%
\begin{pgfscope}%
\pgfsys@transformshift{3.280000in}{1.573916in}%
\pgfsys@useobject{currentmarker}{}%
\end{pgfscope}%
\begin{pgfscope}%
\pgfsys@transformshift{3.298036in}{1.632443in}%
\pgfsys@useobject{currentmarker}{}%
\end{pgfscope}%
\begin{pgfscope}%
\pgfsys@transformshift{3.316073in}{1.808026in}%
\pgfsys@useobject{currentmarker}{}%
\end{pgfscope}%
\begin{pgfscope}%
\pgfsys@transformshift{3.334109in}{2.217720in}%
\pgfsys@useobject{currentmarker}{}%
\end{pgfscope}%
\begin{pgfscope}%
\pgfsys@transformshift{3.352145in}{2.568887in}%
\pgfsys@useobject{currentmarker}{}%
\end{pgfscope}%
\begin{pgfscope}%
\pgfsys@transformshift{3.370182in}{2.861525in}%
\pgfsys@useobject{currentmarker}{}%
\end{pgfscope}%
\begin{pgfscope}%
\pgfsys@transformshift{3.388218in}{3.095636in}%
\pgfsys@useobject{currentmarker}{}%
\end{pgfscope}%
\begin{pgfscope}%
\pgfsys@transformshift{3.406255in}{3.212692in}%
\pgfsys@useobject{currentmarker}{}%
\end{pgfscope}%
\begin{pgfscope}%
\pgfsys@transformshift{3.424291in}{3.212692in}%
\pgfsys@useobject{currentmarker}{}%
\end{pgfscope}%
\begin{pgfscope}%
\pgfsys@transformshift{3.442327in}{3.095636in}%
\pgfsys@useobject{currentmarker}{}%
\end{pgfscope}%
\begin{pgfscope}%
\pgfsys@transformshift{3.460364in}{2.861525in}%
\pgfsys@useobject{currentmarker}{}%
\end{pgfscope}%
\begin{pgfscope}%
\pgfsys@transformshift{3.478400in}{2.568887in}%
\pgfsys@useobject{currentmarker}{}%
\end{pgfscope}%
\begin{pgfscope}%
\pgfsys@transformshift{3.496436in}{2.276248in}%
\pgfsys@useobject{currentmarker}{}%
\end{pgfscope}%
\begin{pgfscope}%
\pgfsys@transformshift{3.514473in}{1.983610in}%
\pgfsys@useobject{currentmarker}{}%
\end{pgfscope}%
\begin{pgfscope}%
\pgfsys@transformshift{3.532509in}{1.690971in}%
\pgfsys@useobject{currentmarker}{}%
\end{pgfscope}%
\begin{pgfscope}%
\pgfsys@transformshift{3.550545in}{1.632443in}%
\pgfsys@useobject{currentmarker}{}%
\end{pgfscope}%
\begin{pgfscope}%
\pgfsys@transformshift{3.568582in}{1.690971in}%
\pgfsys@useobject{currentmarker}{}%
\end{pgfscope}%
\begin{pgfscope}%
\pgfsys@transformshift{3.586618in}{2.042137in}%
\pgfsys@useobject{currentmarker}{}%
\end{pgfscope}%
\begin{pgfscope}%
\pgfsys@transformshift{3.604655in}{2.334776in}%
\pgfsys@useobject{currentmarker}{}%
\end{pgfscope}%
\begin{pgfscope}%
\pgfsys@transformshift{3.622691in}{2.685942in}%
\pgfsys@useobject{currentmarker}{}%
\end{pgfscope}%
\begin{pgfscope}%
\pgfsys@transformshift{3.640727in}{2.920053in}%
\pgfsys@useobject{currentmarker}{}%
\end{pgfscope}%
\begin{pgfscope}%
\pgfsys@transformshift{3.658764in}{3.095636in}%
\pgfsys@useobject{currentmarker}{}%
\end{pgfscope}%
\begin{pgfscope}%
\pgfsys@transformshift{3.676800in}{3.154164in}%
\pgfsys@useobject{currentmarker}{}%
\end{pgfscope}%
\begin{pgfscope}%
\pgfsys@transformshift{3.694836in}{3.154164in}%
\pgfsys@useobject{currentmarker}{}%
\end{pgfscope}%
\begin{pgfscope}%
\pgfsys@transformshift{3.712873in}{2.978581in}%
\pgfsys@useobject{currentmarker}{}%
\end{pgfscope}%
\begin{pgfscope}%
\pgfsys@transformshift{3.730909in}{2.744470in}%
\pgfsys@useobject{currentmarker}{}%
\end{pgfscope}%
\begin{pgfscope}%
\pgfsys@transformshift{3.748945in}{2.451831in}%
\pgfsys@useobject{currentmarker}{}%
\end{pgfscope}%
\begin{pgfscope}%
\pgfsys@transformshift{3.766982in}{2.217720in}%
\pgfsys@useobject{currentmarker}{}%
\end{pgfscope}%
\begin{pgfscope}%
\pgfsys@transformshift{3.785018in}{1.983610in}%
\pgfsys@useobject{currentmarker}{}%
\end{pgfscope}%
\begin{pgfscope}%
\pgfsys@transformshift{3.803055in}{1.866554in}%
\pgfsys@useobject{currentmarker}{}%
\end{pgfscope}%
\begin{pgfscope}%
\pgfsys@transformshift{3.821091in}{1.866554in}%
\pgfsys@useobject{currentmarker}{}%
\end{pgfscope}%
\begin{pgfscope}%
\pgfsys@transformshift{3.839127in}{1.983610in}%
\pgfsys@useobject{currentmarker}{}%
\end{pgfscope}%
\begin{pgfscope}%
\pgfsys@transformshift{3.857164in}{2.217720in}%
\pgfsys@useobject{currentmarker}{}%
\end{pgfscope}%
\begin{pgfscope}%
\pgfsys@transformshift{3.875200in}{2.510359in}%
\pgfsys@useobject{currentmarker}{}%
\end{pgfscope}%
\begin{pgfscope}%
\pgfsys@transformshift{3.893236in}{2.744470in}%
\pgfsys@useobject{currentmarker}{}%
\end{pgfscope}%
\begin{pgfscope}%
\pgfsys@transformshift{3.911273in}{2.978581in}%
\pgfsys@useobject{currentmarker}{}%
\end{pgfscope}%
\begin{pgfscope}%
\pgfsys@transformshift{3.929309in}{3.095636in}%
\pgfsys@useobject{currentmarker}{}%
\end{pgfscope}%
\begin{pgfscope}%
\pgfsys@transformshift{3.947345in}{3.154164in}%
\pgfsys@useobject{currentmarker}{}%
\end{pgfscope}%
\begin{pgfscope}%
\pgfsys@transformshift{3.965382in}{3.037108in}%
\pgfsys@useobject{currentmarker}{}%
\end{pgfscope}%
\begin{pgfscope}%
\pgfsys@transformshift{3.983418in}{2.861525in}%
\pgfsys@useobject{currentmarker}{}%
\end{pgfscope}%
\begin{pgfscope}%
\pgfsys@transformshift{4.001455in}{2.627414in}%
\pgfsys@useobject{currentmarker}{}%
\end{pgfscope}%
\begin{pgfscope}%
\pgfsys@transformshift{4.019491in}{2.393304in}%
\pgfsys@useobject{currentmarker}{}%
\end{pgfscope}%
\begin{pgfscope}%
\pgfsys@transformshift{4.037527in}{2.100665in}%
\pgfsys@useobject{currentmarker}{}%
\end{pgfscope}%
\begin{pgfscope}%
\pgfsys@transformshift{4.055564in}{1.925082in}%
\pgfsys@useobject{currentmarker}{}%
\end{pgfscope}%
\begin{pgfscope}%
\pgfsys@transformshift{4.073600in}{1.866554in}%
\pgfsys@useobject{currentmarker}{}%
\end{pgfscope}%
\begin{pgfscope}%
\pgfsys@transformshift{4.091636in}{1.925082in}%
\pgfsys@useobject{currentmarker}{}%
\end{pgfscope}%
\begin{pgfscope}%
\pgfsys@transformshift{4.109673in}{2.100665in}%
\pgfsys@useobject{currentmarker}{}%
\end{pgfscope}%
\begin{pgfscope}%
\pgfsys@transformshift{4.127709in}{2.334776in}%
\pgfsys@useobject{currentmarker}{}%
\end{pgfscope}%
\begin{pgfscope}%
\pgfsys@transformshift{4.145745in}{2.568887in}%
\pgfsys@useobject{currentmarker}{}%
\end{pgfscope}%
\begin{pgfscope}%
\pgfsys@transformshift{4.163782in}{2.802998in}%
\pgfsys@useobject{currentmarker}{}%
\end{pgfscope}%
\begin{pgfscope}%
\pgfsys@transformshift{4.181818in}{2.978581in}%
\pgfsys@useobject{currentmarker}{}%
\end{pgfscope}%
\begin{pgfscope}%
\pgfsys@transformshift{4.199855in}{3.095636in}%
\pgfsys@useobject{currentmarker}{}%
\end{pgfscope}%
\begin{pgfscope}%
\pgfsys@transformshift{4.217891in}{3.095636in}%
\pgfsys@useobject{currentmarker}{}%
\end{pgfscope}%
\begin{pgfscope}%
\pgfsys@transformshift{4.235927in}{2.978581in}%
\pgfsys@useobject{currentmarker}{}%
\end{pgfscope}%
\begin{pgfscope}%
\pgfsys@transformshift{4.253964in}{2.744470in}%
\pgfsys@useobject{currentmarker}{}%
\end{pgfscope}%
\begin{pgfscope}%
\pgfsys@transformshift{4.272000in}{2.510359in}%
\pgfsys@useobject{currentmarker}{}%
\end{pgfscope}%
\begin{pgfscope}%
\pgfsys@transformshift{4.290036in}{2.276248in}%
\pgfsys@useobject{currentmarker}{}%
\end{pgfscope}%
\begin{pgfscope}%
\pgfsys@transformshift{4.308073in}{2.100665in}%
\pgfsys@useobject{currentmarker}{}%
\end{pgfscope}%
\begin{pgfscope}%
\pgfsys@transformshift{4.326109in}{1.925082in}%
\pgfsys@useobject{currentmarker}{}%
\end{pgfscope}%
\begin{pgfscope}%
\pgfsys@transformshift{4.344145in}{1.925082in}%
\pgfsys@useobject{currentmarker}{}%
\end{pgfscope}%
\begin{pgfscope}%
\pgfsys@transformshift{4.362182in}{2.042137in}%
\pgfsys@useobject{currentmarker}{}%
\end{pgfscope}%
\begin{pgfscope}%
\pgfsys@transformshift{4.380218in}{2.217720in}%
\pgfsys@useobject{currentmarker}{}%
\end{pgfscope}%
\begin{pgfscope}%
\pgfsys@transformshift{4.398255in}{2.451831in}%
\pgfsys@useobject{currentmarker}{}%
\end{pgfscope}%
\begin{pgfscope}%
\pgfsys@transformshift{4.416291in}{2.685942in}%
\pgfsys@useobject{currentmarker}{}%
\end{pgfscope}%
\begin{pgfscope}%
\pgfsys@transformshift{4.434327in}{2.861525in}%
\pgfsys@useobject{currentmarker}{}%
\end{pgfscope}%
\begin{pgfscope}%
\pgfsys@transformshift{4.452364in}{2.978581in}%
\pgfsys@useobject{currentmarker}{}%
\end{pgfscope}%
\begin{pgfscope}%
\pgfsys@transformshift{4.470400in}{3.037108in}%
\pgfsys@useobject{currentmarker}{}%
\end{pgfscope}%
\begin{pgfscope}%
\pgfsys@transformshift{4.488436in}{2.978581in}%
\pgfsys@useobject{currentmarker}{}%
\end{pgfscope}%
\begin{pgfscope}%
\pgfsys@transformshift{4.506473in}{2.861525in}%
\pgfsys@useobject{currentmarker}{}%
\end{pgfscope}%
\begin{pgfscope}%
\pgfsys@transformshift{4.524509in}{2.685942in}%
\pgfsys@useobject{currentmarker}{}%
\end{pgfscope}%
\begin{pgfscope}%
\pgfsys@transformshift{4.542545in}{2.451831in}%
\pgfsys@useobject{currentmarker}{}%
\end{pgfscope}%
\begin{pgfscope}%
\pgfsys@transformshift{4.560582in}{2.217720in}%
\pgfsys@useobject{currentmarker}{}%
\end{pgfscope}%
\begin{pgfscope}%
\pgfsys@transformshift{4.578618in}{2.042137in}%
\pgfsys@useobject{currentmarker}{}%
\end{pgfscope}%
\begin{pgfscope}%
\pgfsys@transformshift{4.596655in}{1.983610in}%
\pgfsys@useobject{currentmarker}{}%
\end{pgfscope}%
\begin{pgfscope}%
\pgfsys@transformshift{4.614691in}{1.983610in}%
\pgfsys@useobject{currentmarker}{}%
\end{pgfscope}%
\begin{pgfscope}%
\pgfsys@transformshift{4.632727in}{2.100665in}%
\pgfsys@useobject{currentmarker}{}%
\end{pgfscope}%
\begin{pgfscope}%
\pgfsys@transformshift{4.650764in}{2.276248in}%
\pgfsys@useobject{currentmarker}{}%
\end{pgfscope}%
\begin{pgfscope}%
\pgfsys@transformshift{4.668800in}{2.510359in}%
\pgfsys@useobject{currentmarker}{}%
\end{pgfscope}%
\begin{pgfscope}%
\pgfsys@transformshift{4.686836in}{2.744470in}%
\pgfsys@useobject{currentmarker}{}%
\end{pgfscope}%
\begin{pgfscope}%
\pgfsys@transformshift{4.704873in}{2.920053in}%
\pgfsys@useobject{currentmarker}{}%
\end{pgfscope}%
\begin{pgfscope}%
\pgfsys@transformshift{4.722909in}{2.978581in}%
\pgfsys@useobject{currentmarker}{}%
\end{pgfscope}%
\begin{pgfscope}%
\pgfsys@transformshift{4.740945in}{3.037108in}%
\pgfsys@useobject{currentmarker}{}%
\end{pgfscope}%
\begin{pgfscope}%
\pgfsys@transformshift{4.758982in}{2.920053in}%
\pgfsys@useobject{currentmarker}{}%
\end{pgfscope}%
\begin{pgfscope}%
\pgfsys@transformshift{4.777018in}{2.744470in}%
\pgfsys@useobject{currentmarker}{}%
\end{pgfscope}%
\begin{pgfscope}%
\pgfsys@transformshift{4.795055in}{2.568887in}%
\pgfsys@useobject{currentmarker}{}%
\end{pgfscope}%
\begin{pgfscope}%
\pgfsys@transformshift{4.813091in}{2.334776in}%
\pgfsys@useobject{currentmarker}{}%
\end{pgfscope}%
\begin{pgfscope}%
\pgfsys@transformshift{4.831127in}{2.159193in}%
\pgfsys@useobject{currentmarker}{}%
\end{pgfscope}%
\begin{pgfscope}%
\pgfsys@transformshift{4.849164in}{2.042137in}%
\pgfsys@useobject{currentmarker}{}%
\end{pgfscope}%
\begin{pgfscope}%
\pgfsys@transformshift{4.867200in}{2.042137in}%
\pgfsys@useobject{currentmarker}{}%
\end{pgfscope}%
\begin{pgfscope}%
\pgfsys@transformshift{4.885236in}{2.042137in}%
\pgfsys@useobject{currentmarker}{}%
\end{pgfscope}%
\begin{pgfscope}%
\pgfsys@transformshift{4.903273in}{2.217720in}%
\pgfsys@useobject{currentmarker}{}%
\end{pgfscope}%
\begin{pgfscope}%
\pgfsys@transformshift{4.921309in}{2.393304in}%
\pgfsys@useobject{currentmarker}{}%
\end{pgfscope}%
\begin{pgfscope}%
\pgfsys@transformshift{4.939345in}{2.627414in}%
\pgfsys@useobject{currentmarker}{}%
\end{pgfscope}%
\begin{pgfscope}%
\pgfsys@transformshift{4.957382in}{2.802998in}%
\pgfsys@useobject{currentmarker}{}%
\end{pgfscope}%
\begin{pgfscope}%
\pgfsys@transformshift{4.975418in}{2.920053in}%
\pgfsys@useobject{currentmarker}{}%
\end{pgfscope}%
\begin{pgfscope}%
\pgfsys@transformshift{4.993455in}{2.978581in}%
\pgfsys@useobject{currentmarker}{}%
\end{pgfscope}%
\begin{pgfscope}%
\pgfsys@transformshift{5.011491in}{2.978581in}%
\pgfsys@useobject{currentmarker}{}%
\end{pgfscope}%
\begin{pgfscope}%
\pgfsys@transformshift{5.029527in}{2.861525in}%
\pgfsys@useobject{currentmarker}{}%
\end{pgfscope}%
\begin{pgfscope}%
\pgfsys@transformshift{5.047564in}{2.685942in}%
\pgfsys@useobject{currentmarker}{}%
\end{pgfscope}%
\begin{pgfscope}%
\pgfsys@transformshift{5.065600in}{2.510359in}%
\pgfsys@useobject{currentmarker}{}%
\end{pgfscope}%
\begin{pgfscope}%
\pgfsys@transformshift{5.083636in}{2.276248in}%
\pgfsys@useobject{currentmarker}{}%
\end{pgfscope}%
\begin{pgfscope}%
\pgfsys@transformshift{5.101673in}{2.159193in}%
\pgfsys@useobject{currentmarker}{}%
\end{pgfscope}%
\begin{pgfscope}%
\pgfsys@transformshift{5.119709in}{2.042137in}%
\pgfsys@useobject{currentmarker}{}%
\end{pgfscope}%
\begin{pgfscope}%
\pgfsys@transformshift{5.137745in}{2.042137in}%
\pgfsys@useobject{currentmarker}{}%
\end{pgfscope}%
\begin{pgfscope}%
\pgfsys@transformshift{5.155782in}{2.159193in}%
\pgfsys@useobject{currentmarker}{}%
\end{pgfscope}%
\begin{pgfscope}%
\pgfsys@transformshift{5.173818in}{2.276248in}%
\pgfsys@useobject{currentmarker}{}%
\end{pgfscope}%
\begin{pgfscope}%
\pgfsys@transformshift{5.191855in}{2.451831in}%
\pgfsys@useobject{currentmarker}{}%
\end{pgfscope}%
\begin{pgfscope}%
\pgfsys@transformshift{5.209891in}{2.685942in}%
\pgfsys@useobject{currentmarker}{}%
\end{pgfscope}%
\begin{pgfscope}%
\pgfsys@transformshift{5.227927in}{2.802998in}%
\pgfsys@useobject{currentmarker}{}%
\end{pgfscope}%
\begin{pgfscope}%
\pgfsys@transformshift{5.245964in}{2.920053in}%
\pgfsys@useobject{currentmarker}{}%
\end{pgfscope}%
\begin{pgfscope}%
\pgfsys@transformshift{5.264000in}{2.978581in}%
\pgfsys@useobject{currentmarker}{}%
\end{pgfscope}%
\begin{pgfscope}%
\pgfsys@transformshift{5.282036in}{2.920053in}%
\pgfsys@useobject{currentmarker}{}%
\end{pgfscope}%
\begin{pgfscope}%
\pgfsys@transformshift{5.300073in}{2.802998in}%
\pgfsys@useobject{currentmarker}{}%
\end{pgfscope}%
\begin{pgfscope}%
\pgfsys@transformshift{5.318109in}{2.627414in}%
\pgfsys@useobject{currentmarker}{}%
\end{pgfscope}%
\begin{pgfscope}%
\pgfsys@transformshift{5.336145in}{2.393304in}%
\pgfsys@useobject{currentmarker}{}%
\end{pgfscope}%
\begin{pgfscope}%
\pgfsys@transformshift{5.354182in}{2.276248in}%
\pgfsys@useobject{currentmarker}{}%
\end{pgfscope}%
\begin{pgfscope}%
\pgfsys@transformshift{5.372218in}{2.100665in}%
\pgfsys@useobject{currentmarker}{}%
\end{pgfscope}%
\begin{pgfscope}%
\pgfsys@transformshift{5.390255in}{2.100665in}%
\pgfsys@useobject{currentmarker}{}%
\end{pgfscope}%
\begin{pgfscope}%
\pgfsys@transformshift{5.408291in}{2.100665in}%
\pgfsys@useobject{currentmarker}{}%
\end{pgfscope}%
\begin{pgfscope}%
\pgfsys@transformshift{5.426327in}{2.217720in}%
\pgfsys@useobject{currentmarker}{}%
\end{pgfscope}%
\begin{pgfscope}%
\pgfsys@transformshift{5.444364in}{2.393304in}%
\pgfsys@useobject{currentmarker}{}%
\end{pgfscope}%
\begin{pgfscope}%
\pgfsys@transformshift{5.462400in}{2.568887in}%
\pgfsys@useobject{currentmarker}{}%
\end{pgfscope}%
\begin{pgfscope}%
\pgfsys@transformshift{5.480436in}{2.744470in}%
\pgfsys@useobject{currentmarker}{}%
\end{pgfscope}%
\begin{pgfscope}%
\pgfsys@transformshift{5.498473in}{2.861525in}%
\pgfsys@useobject{currentmarker}{}%
\end{pgfscope}%
\begin{pgfscope}%
\pgfsys@transformshift{5.516509in}{2.920053in}%
\pgfsys@useobject{currentmarker}{}%
\end{pgfscope}%
\begin{pgfscope}%
\pgfsys@transformshift{5.534545in}{2.920053in}%
\pgfsys@useobject{currentmarker}{}%
\end{pgfscope}%
\end{pgfscope}%
\begin{pgfscope}%
\pgfpathrectangle{\pgfqpoint{0.800000in}{0.528000in}}{\pgfqpoint{4.960000in}{3.696000in}}%
\pgfusepath{clip}%
\pgfsetrectcap%
\pgfsetroundjoin%
\pgfsetlinewidth{0.501875pt}%
\definecolor{currentstroke}{rgb}{1.000000,0.000000,0.000000}%
\pgfsetstrokecolor{currentstroke}%
\pgfsetdash{}{0pt}%
\pgfpathmoveto{\pgfqpoint{1.025455in}{3.974417in}}%
\pgfpathlineto{\pgfqpoint{1.029968in}{4.020305in}}%
\pgfpathlineto{\pgfqpoint{1.034482in}{4.047586in}}%
\pgfpathlineto{\pgfqpoint{1.038995in}{4.056000in}}%
\pgfpathlineto{\pgfqpoint{1.043509in}{4.045502in}}%
\pgfpathlineto{\pgfqpoint{1.048023in}{4.016267in}}%
\pgfpathlineto{\pgfqpoint{1.052536in}{3.968683in}}%
\pgfpathlineto{\pgfqpoint{1.057050in}{3.903351in}}%
\pgfpathlineto{\pgfqpoint{1.066077in}{3.722840in}}%
\pgfpathlineto{\pgfqpoint{1.075104in}{3.483393in}}%
\pgfpathlineto{\pgfqpoint{1.088645in}{3.039054in}}%
\pgfpathlineto{\pgfqpoint{1.133781in}{1.416609in}}%
\pgfpathlineto{\pgfqpoint{1.142808in}{1.178572in}}%
\pgfpathlineto{\pgfqpoint{1.151835in}{0.998592in}}%
\pgfpathlineto{\pgfqpoint{1.156349in}{0.932968in}}%
\pgfpathlineto{\pgfqpoint{1.160863in}{0.884597in}}%
\pgfpathlineto{\pgfqpoint{1.165376in}{0.853988in}}%
\pgfpathlineto{\pgfqpoint{1.169890in}{0.841444in}}%
\pgfpathlineto{\pgfqpoint{1.174403in}{0.847057in}}%
\pgfpathlineto{\pgfqpoint{1.178917in}{0.870713in}}%
\pgfpathlineto{\pgfqpoint{1.183431in}{0.912088in}}%
\pgfpathlineto{\pgfqpoint{1.187944in}{0.970656in}}%
\pgfpathlineto{\pgfqpoint{1.196972in}{1.136294in}}%
\pgfpathlineto{\pgfqpoint{1.205999in}{1.359661in}}%
\pgfpathlineto{\pgfqpoint{1.219540in}{1.779249in}}%
\pgfpathlineto{\pgfqpoint{1.242108in}{2.591160in}}%
\pgfpathlineto{\pgfqpoint{1.260162in}{3.208000in}}%
\pgfpathlineto{\pgfqpoint{1.269189in}{3.466135in}}%
\pgfpathlineto{\pgfqpoint{1.278216in}{3.674641in}}%
\pgfpathlineto{\pgfqpoint{1.287244in}{3.824237in}}%
\pgfpathlineto{\pgfqpoint{1.291757in}{3.874786in}}%
\pgfpathlineto{\pgfqpoint{1.296271in}{3.908392in}}%
\pgfpathlineto{\pgfqpoint{1.300784in}{3.924715in}}%
\pgfpathlineto{\pgfqpoint{1.305298in}{3.923616in}}%
\pgfpathlineto{\pgfqpoint{1.309812in}{3.905156in}}%
\pgfpathlineto{\pgfqpoint{1.314325in}{3.869597in}}%
\pgfpathlineto{\pgfqpoint{1.318839in}{3.817396in}}%
\pgfpathlineto{\pgfqpoint{1.327866in}{3.665841in}}%
\pgfpathlineto{\pgfqpoint{1.336893in}{3.457801in}}%
\pgfpathlineto{\pgfqpoint{1.350434in}{3.061977in}}%
\pgfpathlineto{\pgfqpoint{1.373002in}{2.285745in}}%
\pgfpathlineto{\pgfqpoint{1.391057in}{1.688207in}}%
\pgfpathlineto{\pgfqpoint{1.404597in}{1.325677in}}%
\pgfpathlineto{\pgfqpoint{1.413625in}{1.145976in}}%
\pgfpathlineto{\pgfqpoint{1.422652in}{1.025499in}}%
\pgfpathlineto{\pgfqpoint{1.427165in}{0.989207in}}%
\pgfpathlineto{\pgfqpoint{1.431679in}{0.969439in}}%
\pgfpathlineto{\pgfqpoint{1.436193in}{0.966376in}}%
\pgfpathlineto{\pgfqpoint{1.440706in}{0.980005in}}%
\pgfpathlineto{\pgfqpoint{1.445220in}{1.010124in}}%
\pgfpathlineto{\pgfqpoint{1.449733in}{1.056338in}}%
\pgfpathlineto{\pgfqpoint{1.458761in}{1.194569in}}%
\pgfpathlineto{\pgfqpoint{1.467788in}{1.388011in}}%
\pgfpathlineto{\pgfqpoint{1.476815in}{1.627484in}}%
\pgfpathlineto{\pgfqpoint{1.490356in}{2.047979in}}%
\pgfpathlineto{\pgfqpoint{1.521951in}{3.081014in}}%
\pgfpathlineto{\pgfqpoint{1.535492in}{3.436546in}}%
\pgfpathlineto{\pgfqpoint{1.544519in}{3.615519in}}%
\pgfpathlineto{\pgfqpoint{1.553546in}{3.738459in}}%
\pgfpathlineto{\pgfqpoint{1.558060in}{3.777144in}}%
\pgfpathlineto{\pgfqpoint{1.562573in}{3.800042in}}%
\pgfpathlineto{\pgfqpoint{1.567087in}{3.806935in}}%
\pgfpathlineto{\pgfqpoint{1.571601in}{3.797788in}}%
\pgfpathlineto{\pgfqpoint{1.576114in}{3.772752in}}%
\pgfpathlineto{\pgfqpoint{1.580628in}{3.732160in}}%
\pgfpathlineto{\pgfqpoint{1.585142in}{3.676522in}}%
\pgfpathlineto{\pgfqpoint{1.594169in}{3.523006in}}%
\pgfpathlineto{\pgfqpoint{1.603196in}{3.319564in}}%
\pgfpathlineto{\pgfqpoint{1.616737in}{2.942318in}}%
\pgfpathlineto{\pgfqpoint{1.661873in}{1.566557in}}%
\pgfpathlineto{\pgfqpoint{1.670900in}{1.364999in}}%
\pgfpathlineto{\pgfqpoint{1.679927in}{1.212745in}}%
\pgfpathlineto{\pgfqpoint{1.684441in}{1.157301in}}%
\pgfpathlineto{\pgfqpoint{1.688954in}{1.116499in}}%
\pgfpathlineto{\pgfqpoint{1.693468in}{1.090768in}}%
\pgfpathlineto{\pgfqpoint{1.697982in}{1.080361in}}%
\pgfpathlineto{\pgfqpoint{1.702495in}{1.085356in}}%
\pgfpathlineto{\pgfqpoint{1.707009in}{1.105651in}}%
\pgfpathlineto{\pgfqpoint{1.711522in}{1.140970in}}%
\pgfpathlineto{\pgfqpoint{1.716036in}{1.190863in}}%
\pgfpathlineto{\pgfqpoint{1.725063in}{1.331755in}}%
\pgfpathlineto{\pgfqpoint{1.734090in}{1.521552in}}%
\pgfpathlineto{\pgfqpoint{1.747631in}{1.877805in}}%
\pgfpathlineto{\pgfqpoint{1.774713in}{2.704410in}}%
\pgfpathlineto{\pgfqpoint{1.788254in}{3.089483in}}%
\pgfpathlineto{\pgfqpoint{1.797281in}{3.308145in}}%
\pgfpathlineto{\pgfqpoint{1.806308in}{3.484645in}}%
\pgfpathlineto{\pgfqpoint{1.815335in}{3.611127in}}%
\pgfpathlineto{\pgfqpoint{1.819849in}{3.653788in}}%
\pgfpathlineto{\pgfqpoint{1.824363in}{3.682071in}}%
\pgfpathlineto{\pgfqpoint{1.828876in}{3.695693in}}%
\pgfpathlineto{\pgfqpoint{1.833390in}{3.694536in}}%
\pgfpathlineto{\pgfqpoint{1.837903in}{3.678656in}}%
\pgfpathlineto{\pgfqpoint{1.842417in}{3.648278in}}%
\pgfpathlineto{\pgfqpoint{1.846931in}{3.603792in}}%
\pgfpathlineto{\pgfqpoint{1.855958in}{3.474856in}}%
\pgfpathlineto{\pgfqpoint{1.864985in}{3.298065in}}%
\pgfpathlineto{\pgfqpoint{1.878526in}{2.961967in}}%
\pgfpathlineto{\pgfqpoint{1.901094in}{2.303410in}}%
\pgfpathlineto{\pgfqpoint{1.919148in}{1.796867in}}%
\pgfpathlineto{\pgfqpoint{1.932689in}{1.489795in}}%
\pgfpathlineto{\pgfqpoint{1.941716in}{1.337733in}}%
\pgfpathlineto{\pgfqpoint{1.950743in}{1.235947in}}%
\pgfpathlineto{\pgfqpoint{1.955257in}{1.205376in}}%
\pgfpathlineto{\pgfqpoint{1.959771in}{1.188823in}}%
\pgfpathlineto{\pgfqpoint{1.964284in}{1.186440in}}%
\pgfpathlineto{\pgfqpoint{1.968798in}{1.198214in}}%
\pgfpathlineto{\pgfqpoint{1.973311in}{1.223970in}}%
\pgfpathlineto{\pgfqpoint{1.977825in}{1.263372in}}%
\pgfpathlineto{\pgfqpoint{1.986852in}{1.380997in}}%
\pgfpathlineto{\pgfqpoint{1.995880in}{1.545400in}}%
\pgfpathlineto{\pgfqpoint{2.004907in}{1.748780in}}%
\pgfpathlineto{\pgfqpoint{2.018448in}{2.105682in}}%
\pgfpathlineto{\pgfqpoint{2.050043in}{2.981686in}}%
\pgfpathlineto{\pgfqpoint{2.063584in}{3.282851in}}%
\pgfpathlineto{\pgfqpoint{2.072611in}{3.434317in}}%
\pgfpathlineto{\pgfqpoint{2.081638in}{3.538212in}}%
\pgfpathlineto{\pgfqpoint{2.086152in}{3.570823in}}%
\pgfpathlineto{\pgfqpoint{2.090665in}{3.590040in}}%
\pgfpathlineto{\pgfqpoint{2.095179in}{3.595680in}}%
\pgfpathlineto{\pgfqpoint{2.099692in}{3.587716in}}%
\pgfpathlineto{\pgfqpoint{2.104206in}{3.566278in}}%
\pgfpathlineto{\pgfqpoint{2.108720in}{3.531651in}}%
\pgfpathlineto{\pgfqpoint{2.113233in}{3.484271in}}%
\pgfpathlineto{\pgfqpoint{2.122260in}{3.353713in}}%
\pgfpathlineto{\pgfqpoint{2.131288in}{3.180864in}}%
\pgfpathlineto{\pgfqpoint{2.144828in}{2.860582in}}%
\pgfpathlineto{\pgfqpoint{2.189965in}{1.694000in}}%
\pgfpathlineto{\pgfqpoint{2.198992in}{1.523332in}}%
\pgfpathlineto{\pgfqpoint{2.208019in}{1.394533in}}%
\pgfpathlineto{\pgfqpoint{2.212533in}{1.347691in}}%
\pgfpathlineto{\pgfqpoint{2.217046in}{1.313275in}}%
\pgfpathlineto{\pgfqpoint{2.221560in}{1.291646in}}%
\pgfpathlineto{\pgfqpoint{2.226073in}{1.283018in}}%
\pgfpathlineto{\pgfqpoint{2.230587in}{1.287452in}}%
\pgfpathlineto{\pgfqpoint{2.235101in}{1.304861in}}%
\pgfpathlineto{\pgfqpoint{2.239614in}{1.335009in}}%
\pgfpathlineto{\pgfqpoint{2.244128in}{1.377512in}}%
\pgfpathlineto{\pgfqpoint{2.253155in}{1.497353in}}%
\pgfpathlineto{\pgfqpoint{2.262182in}{1.658624in}}%
\pgfpathlineto{\pgfqpoint{2.275723in}{1.961102in}}%
\pgfpathlineto{\pgfqpoint{2.302805in}{2.662301in}}%
\pgfpathlineto{\pgfqpoint{2.316345in}{2.988684in}}%
\pgfpathlineto{\pgfqpoint{2.325373in}{3.173908in}}%
\pgfpathlineto{\pgfqpoint{2.334400in}{3.323315in}}%
\pgfpathlineto{\pgfqpoint{2.343427in}{3.430253in}}%
\pgfpathlineto{\pgfqpoint{2.347941in}{3.466255in}}%
\pgfpathlineto{\pgfqpoint{2.352454in}{3.490057in}}%
\pgfpathlineto{\pgfqpoint{2.356968in}{3.501421in}}%
\pgfpathlineto{\pgfqpoint{2.361481in}{3.500250in}}%
\pgfpathlineto{\pgfqpoint{2.365995in}{3.486593in}}%
\pgfpathlineto{\pgfqpoint{2.370509in}{3.460642in}}%
\pgfpathlineto{\pgfqpoint{2.375022in}{3.422731in}}%
\pgfpathlineto{\pgfqpoint{2.384050in}{3.313039in}}%
\pgfpathlineto{\pgfqpoint{2.393077in}{3.162804in}}%
\pgfpathlineto{\pgfqpoint{2.406618in}{2.877422in}}%
\pgfpathlineto{\pgfqpoint{2.429186in}{2.318702in}}%
\pgfpathlineto{\pgfqpoint{2.447240in}{1.889298in}}%
\pgfpathlineto{\pgfqpoint{2.460781in}{1.629202in}}%
\pgfpathlineto{\pgfqpoint{2.469808in}{1.500529in}}%
\pgfpathlineto{\pgfqpoint{2.478835in}{1.414536in}}%
\pgfpathlineto{\pgfqpoint{2.483349in}{1.388784in}}%
\pgfpathlineto{\pgfqpoint{2.487862in}{1.374926in}}%
\pgfpathlineto{\pgfqpoint{2.492376in}{1.373087in}}%
\pgfpathlineto{\pgfqpoint{2.496890in}{1.383255in}}%
\pgfpathlineto{\pgfqpoint{2.501403in}{1.405279in}}%
\pgfpathlineto{\pgfqpoint{2.505917in}{1.438872in}}%
\pgfpathlineto{\pgfqpoint{2.514944in}{1.538962in}}%
\pgfpathlineto{\pgfqpoint{2.523971in}{1.678684in}}%
\pgfpathlineto{\pgfqpoint{2.532998in}{1.851410in}}%
\pgfpathlineto{\pgfqpoint{2.546539in}{2.154336in}}%
\pgfpathlineto{\pgfqpoint{2.578134in}{2.897177in}}%
\pgfpathlineto{\pgfqpoint{2.591675in}{3.152288in}}%
\pgfpathlineto{\pgfqpoint{2.600703in}{3.280473in}}%
\pgfpathlineto{\pgfqpoint{2.609730in}{3.368272in}}%
\pgfpathlineto{\pgfqpoint{2.614243in}{3.395762in}}%
\pgfpathlineto{\pgfqpoint{2.618757in}{3.411888in}}%
\pgfpathlineto{\pgfqpoint{2.623271in}{3.416496in}}%
\pgfpathlineto{\pgfqpoint{2.627784in}{3.409567in}}%
\pgfpathlineto{\pgfqpoint{2.632298in}{3.391212in}}%
\pgfpathlineto{\pgfqpoint{2.636811in}{3.361675in}}%
\pgfpathlineto{\pgfqpoint{2.641325in}{3.321326in}}%
\pgfpathlineto{\pgfqpoint{2.650352in}{3.210294in}}%
\pgfpathlineto{\pgfqpoint{2.659379in}{3.063439in}}%
\pgfpathlineto{\pgfqpoint{2.672920in}{2.791520in}}%
\pgfpathlineto{\pgfqpoint{2.718056in}{1.802315in}}%
\pgfpathlineto{\pgfqpoint{2.727083in}{1.657804in}}%
\pgfpathlineto{\pgfqpoint{2.736111in}{1.548848in}}%
\pgfpathlineto{\pgfqpoint{2.740624in}{1.509274in}}%
\pgfpathlineto{\pgfqpoint{2.745138in}{1.480246in}}%
\pgfpathlineto{\pgfqpoint{2.749651in}{1.462067in}}%
\pgfpathlineto{\pgfqpoint{2.754165in}{1.454916in}}%
\pgfpathlineto{\pgfqpoint{2.758679in}{1.458844in}}%
\pgfpathlineto{\pgfqpoint{2.763192in}{1.473777in}}%
\pgfpathlineto{\pgfqpoint{2.767706in}{1.499509in}}%
\pgfpathlineto{\pgfqpoint{2.772219in}{1.535715in}}%
\pgfpathlineto{\pgfqpoint{2.781247in}{1.637650in}}%
\pgfpathlineto{\pgfqpoint{2.790274in}{1.774682in}}%
\pgfpathlineto{\pgfqpoint{2.803815in}{2.031501in}}%
\pgfpathlineto{\pgfqpoint{2.830896in}{2.626317in}}%
\pgfpathlineto{\pgfqpoint{2.844437in}{2.902955in}}%
\pgfpathlineto{\pgfqpoint{2.853464in}{3.059855in}}%
\pgfpathlineto{\pgfqpoint{2.862492in}{3.186325in}}%
\pgfpathlineto{\pgfqpoint{2.871519in}{3.276738in}}%
\pgfpathlineto{\pgfqpoint{2.876032in}{3.307120in}}%
\pgfpathlineto{\pgfqpoint{2.880546in}{3.327150in}}%
\pgfpathlineto{\pgfqpoint{2.885060in}{3.336628in}}%
\pgfpathlineto{\pgfqpoint{2.889573in}{3.335473in}}%
\pgfpathlineto{\pgfqpoint{2.894087in}{3.323730in}}%
\pgfpathlineto{\pgfqpoint{2.898600in}{3.301562in}}%
\pgfpathlineto{\pgfqpoint{2.903114in}{3.269255in}}%
\pgfpathlineto{\pgfqpoint{2.912141in}{3.175936in}}%
\pgfpathlineto{\pgfqpoint{2.921168in}{3.048269in}}%
\pgfpathlineto{\pgfqpoint{2.934709in}{2.805951in}}%
\pgfpathlineto{\pgfqpoint{2.957277in}{2.331934in}}%
\pgfpathlineto{\pgfqpoint{2.975332in}{1.967923in}}%
\pgfpathlineto{\pgfqpoint{2.984359in}{1.814200in}}%
\pgfpathlineto{\pgfqpoint{2.993386in}{1.688912in}}%
\pgfpathlineto{\pgfqpoint{3.002413in}{1.597647in}}%
\pgfpathlineto{\pgfqpoint{3.006927in}{1.566088in}}%
\pgfpathlineto{\pgfqpoint{3.011441in}{1.544397in}}%
\pgfpathlineto{\pgfqpoint{3.015954in}{1.532797in}}%
\pgfpathlineto{\pgfqpoint{3.020468in}{1.531392in}}%
\pgfpathlineto{\pgfqpoint{3.024981in}{1.540170in}}%
\pgfpathlineto{\pgfqpoint{3.029495in}{1.559002in}}%
\pgfpathlineto{\pgfqpoint{3.034009in}{1.587642in}}%
\pgfpathlineto{\pgfqpoint{3.043036in}{1.672809in}}%
\pgfpathlineto{\pgfqpoint{3.052063in}{1.791554in}}%
\pgfpathlineto{\pgfqpoint{3.061090in}{1.938245in}}%
\pgfpathlineto{\pgfqpoint{3.074631in}{2.195357in}}%
\pgfpathlineto{\pgfqpoint{3.106226in}{2.825277in}}%
\pgfpathlineto{\pgfqpoint{3.119767in}{3.041376in}}%
\pgfpathlineto{\pgfqpoint{3.128794in}{3.149856in}}%
\pgfpathlineto{\pgfqpoint{3.137821in}{3.224053in}}%
\pgfpathlineto{\pgfqpoint{3.142335in}{3.247224in}}%
\pgfpathlineto{\pgfqpoint{3.146849in}{3.260755in}}%
\pgfpathlineto{\pgfqpoint{3.151362in}{3.264515in}}%
\pgfpathlineto{\pgfqpoint{3.155876in}{3.258490in}}%
\pgfpathlineto{\pgfqpoint{3.160389in}{3.242776in}}%
\pgfpathlineto{\pgfqpoint{3.164903in}{3.217581in}}%
\pgfpathlineto{\pgfqpoint{3.169417in}{3.183222in}}%
\pgfpathlineto{\pgfqpoint{3.178444in}{3.088797in}}%
\pgfpathlineto{\pgfqpoint{3.187471in}{2.964026in}}%
\pgfpathlineto{\pgfqpoint{3.201012in}{2.733168in}}%
\pgfpathlineto{\pgfqpoint{3.246148in}{1.894373in}}%
\pgfpathlineto{\pgfqpoint{3.255175in}{1.772010in}}%
\pgfpathlineto{\pgfqpoint{3.264202in}{1.679841in}}%
\pgfpathlineto{\pgfqpoint{3.268716in}{1.646409in}}%
\pgfpathlineto{\pgfqpoint{3.273230in}{1.621925in}}%
\pgfpathlineto{\pgfqpoint{3.277743in}{1.606646in}}%
\pgfpathlineto{\pgfqpoint{3.282257in}{1.600724in}}%
\pgfpathlineto{\pgfqpoint{3.286770in}{1.604198in}}%
\pgfpathlineto{\pgfqpoint{3.291284in}{1.617004in}}%
\pgfpathlineto{\pgfqpoint{3.295798in}{1.638967in}}%
\pgfpathlineto{\pgfqpoint{3.300311in}{1.669809in}}%
\pgfpathlineto{\pgfqpoint{3.309338in}{1.756512in}}%
\pgfpathlineto{\pgfqpoint{3.318366in}{1.872947in}}%
\pgfpathlineto{\pgfqpoint{3.331906in}{2.090998in}}%
\pgfpathlineto{\pgfqpoint{3.381556in}{2.962950in}}%
\pgfpathlineto{\pgfqpoint{3.390583in}{3.070005in}}%
\pgfpathlineto{\pgfqpoint{3.399611in}{3.146446in}}%
\pgfpathlineto{\pgfqpoint{3.404124in}{3.172083in}}%
\pgfpathlineto{\pgfqpoint{3.408638in}{3.188938in}}%
\pgfpathlineto{\pgfqpoint{3.413151in}{3.196840in}}%
\pgfpathlineto{\pgfqpoint{3.417665in}{3.195724in}}%
\pgfpathlineto{\pgfqpoint{3.422179in}{3.185628in}}%
\pgfpathlineto{\pgfqpoint{3.426692in}{3.166693in}}%
\pgfpathlineto{\pgfqpoint{3.431206in}{3.139163in}}%
\pgfpathlineto{\pgfqpoint{3.440233in}{3.059774in}}%
\pgfpathlineto{\pgfqpoint{3.449260in}{2.951286in}}%
\pgfpathlineto{\pgfqpoint{3.462801in}{2.745533in}}%
\pgfpathlineto{\pgfqpoint{3.485369in}{2.343380in}}%
\pgfpathlineto{\pgfqpoint{3.503423in}{2.034805in}}%
\pgfpathlineto{\pgfqpoint{3.512451in}{1.904580in}}%
\pgfpathlineto{\pgfqpoint{3.521478in}{1.798516in}}%
\pgfpathlineto{\pgfqpoint{3.530505in}{1.721339in}}%
\pgfpathlineto{\pgfqpoint{3.535019in}{1.694696in}}%
\pgfpathlineto{\pgfqpoint{3.539532in}{1.676427in}}%
\pgfpathlineto{\pgfqpoint{3.544046in}{1.666718in}}%
\pgfpathlineto{\pgfqpoint{3.548559in}{1.665658in}}%
\pgfpathlineto{\pgfqpoint{3.553073in}{1.673234in}}%
\pgfpathlineto{\pgfqpoint{3.557587in}{1.689335in}}%
\pgfpathlineto{\pgfqpoint{3.562100in}{1.713751in}}%
\pgfpathlineto{\pgfqpoint{3.571127in}{1.786219in}}%
\pgfpathlineto{\pgfqpoint{3.580155in}{1.887137in}}%
\pgfpathlineto{\pgfqpoint{3.589182in}{2.011717in}}%
\pgfpathlineto{\pgfqpoint{3.602723in}{2.229943in}}%
\pgfpathlineto{\pgfqpoint{3.634318in}{2.764105in}}%
\pgfpathlineto{\pgfqpoint{3.647859in}{2.947156in}}%
\pgfpathlineto{\pgfqpoint{3.656886in}{3.038962in}}%
\pgfpathlineto{\pgfqpoint{3.665913in}{3.101661in}}%
\pgfpathlineto{\pgfqpoint{3.670427in}{3.121193in}}%
\pgfpathlineto{\pgfqpoint{3.674940in}{3.132544in}}%
\pgfpathlineto{\pgfqpoint{3.679454in}{3.135607in}}%
\pgfpathlineto{\pgfqpoint{3.683968in}{3.130371in}}%
\pgfpathlineto{\pgfqpoint{3.688481in}{3.116919in}}%
\pgfpathlineto{\pgfqpoint{3.692995in}{3.095429in}}%
\pgfpathlineto{\pgfqpoint{3.697508in}{3.066171in}}%
\pgfpathlineto{\pgfqpoint{3.706536in}{2.985869in}}%
\pgfpathlineto{\pgfqpoint{3.715563in}{2.879863in}}%
\pgfpathlineto{\pgfqpoint{3.729104in}{2.683866in}}%
\pgfpathlineto{\pgfqpoint{3.774240in}{1.972613in}}%
\pgfpathlineto{\pgfqpoint{3.783267in}{1.869004in}}%
\pgfpathlineto{\pgfqpoint{3.792294in}{1.791037in}}%
\pgfpathlineto{\pgfqpoint{3.796808in}{1.762793in}}%
\pgfpathlineto{\pgfqpoint{3.801321in}{1.742143in}}%
\pgfpathlineto{\pgfqpoint{3.805835in}{1.729304in}}%
\pgfpathlineto{\pgfqpoint{3.810349in}{1.724401in}}%
\pgfpathlineto{\pgfqpoint{3.814862in}{1.727469in}}%
\pgfpathlineto{\pgfqpoint{3.819376in}{1.738449in}}%
\pgfpathlineto{\pgfqpoint{3.823889in}{1.757194in}}%
\pgfpathlineto{\pgfqpoint{3.828403in}{1.783466in}}%
\pgfpathlineto{\pgfqpoint{3.837430in}{1.857212in}}%
\pgfpathlineto{\pgfqpoint{3.846457in}{1.956146in}}%
\pgfpathlineto{\pgfqpoint{3.859998in}{2.141280in}}%
\pgfpathlineto{\pgfqpoint{3.909648in}{2.880616in}}%
\pgfpathlineto{\pgfqpoint{3.918675in}{2.971236in}}%
\pgfpathlineto{\pgfqpoint{3.927702in}{3.035862in}}%
\pgfpathlineto{\pgfqpoint{3.932216in}{3.057496in}}%
\pgfpathlineto{\pgfqpoint{3.936729in}{3.071678in}}%
\pgfpathlineto{\pgfqpoint{3.941243in}{3.078265in}}%
\pgfpathlineto{\pgfqpoint{3.945757in}{3.077202in}}%
\pgfpathlineto{\pgfqpoint{3.950270in}{3.068524in}}%
\pgfpathlineto{\pgfqpoint{3.954784in}{3.052351in}}%
\pgfpathlineto{\pgfqpoint{3.959297in}{3.028892in}}%
\pgfpathlineto{\pgfqpoint{3.968325in}{2.961354in}}%
\pgfpathlineto{\pgfqpoint{3.977352in}{2.869164in}}%
\pgfpathlineto{\pgfqpoint{3.990893in}{2.694461in}}%
\pgfpathlineto{\pgfqpoint{4.013461in}{2.353277in}}%
\pgfpathlineto{\pgfqpoint{4.031515in}{2.091695in}}%
\pgfpathlineto{\pgfqpoint{4.040542in}{1.981378in}}%
\pgfpathlineto{\pgfqpoint{4.049570in}{1.891587in}}%
\pgfpathlineto{\pgfqpoint{4.058597in}{1.826325in}}%
\pgfpathlineto{\pgfqpoint{4.063110in}{1.803833in}}%
\pgfpathlineto{\pgfqpoint{4.067624in}{1.788447in}}%
\pgfpathlineto{\pgfqpoint{4.072138in}{1.780323in}}%
\pgfpathlineto{\pgfqpoint{4.076651in}{1.779535in}}%
\pgfpathlineto{\pgfqpoint{4.081165in}{1.786072in}}%
\pgfpathlineto{\pgfqpoint{4.085678in}{1.799836in}}%
\pgfpathlineto{\pgfqpoint{4.090192in}{1.820651in}}%
\pgfpathlineto{\pgfqpoint{4.099219in}{1.882313in}}%
\pgfpathlineto{\pgfqpoint{4.108246in}{1.968080in}}%
\pgfpathlineto{\pgfqpoint{4.117274in}{2.073881in}}%
\pgfpathlineto{\pgfqpoint{4.130814in}{2.259102in}}%
\pgfpathlineto{\pgfqpoint{4.162410in}{2.712061in}}%
\pgfpathlineto{\pgfqpoint{4.175950in}{2.867118in}}%
\pgfpathlineto{\pgfqpoint{4.184978in}{2.944811in}}%
\pgfpathlineto{\pgfqpoint{4.194005in}{2.997795in}}%
\pgfpathlineto{\pgfqpoint{4.198519in}{3.014257in}}%
\pgfpathlineto{\pgfqpoint{4.203032in}{3.023778in}}%
\pgfpathlineto{\pgfqpoint{4.207546in}{3.026270in}}%
\pgfpathlineto{\pgfqpoint{4.212059in}{3.021722in}}%
\pgfpathlineto{\pgfqpoint{4.216573in}{3.010208in}}%
\pgfpathlineto{\pgfqpoint{4.221087in}{2.991878in}}%
\pgfpathlineto{\pgfqpoint{4.225600in}{2.966964in}}%
\pgfpathlineto{\pgfqpoint{4.234627in}{2.898674in}}%
\pgfpathlineto{\pgfqpoint{4.243655in}{2.808611in}}%
\pgfpathlineto{\pgfqpoint{4.257195in}{2.642211in}}%
\pgfpathlineto{\pgfqpoint{4.302331in}{2.039109in}}%
\pgfpathlineto{\pgfqpoint{4.311359in}{1.951380in}}%
\pgfpathlineto{\pgfqpoint{4.320386in}{1.885427in}}%
\pgfpathlineto{\pgfqpoint{4.324899in}{1.861566in}}%
\pgfpathlineto{\pgfqpoint{4.329413in}{1.844151in}}%
\pgfpathlineto{\pgfqpoint{4.333927in}{1.833362in}}%
\pgfpathlineto{\pgfqpoint{4.338440in}{1.829306in}}%
\pgfpathlineto{\pgfqpoint{4.342954in}{1.832011in}}%
\pgfpathlineto{\pgfqpoint{4.347467in}{1.841425in}}%
\pgfpathlineto{\pgfqpoint{4.351981in}{1.857423in}}%
\pgfpathlineto{\pgfqpoint{4.356495in}{1.879801in}}%
\pgfpathlineto{\pgfqpoint{4.365522in}{1.942527in}}%
\pgfpathlineto{\pgfqpoint{4.374549in}{2.026589in}}%
\pgfpathlineto{\pgfqpoint{4.388090in}{2.183775in}}%
\pgfpathlineto{\pgfqpoint{4.437740in}{2.810662in}}%
\pgfpathlineto{\pgfqpoint{4.446767in}{2.887369in}}%
\pgfpathlineto{\pgfqpoint{4.455794in}{2.942007in}}%
\pgfpathlineto{\pgfqpoint{4.460308in}{2.960262in}}%
\pgfpathlineto{\pgfqpoint{4.464821in}{2.972194in}}%
\pgfpathlineto{\pgfqpoint{4.469335in}{2.977682in}}%
\pgfpathlineto{\pgfqpoint{4.473848in}{2.976683in}}%
\pgfpathlineto{\pgfqpoint{4.478362in}{2.969224in}}%
\pgfpathlineto{\pgfqpoint{4.482876in}{2.955412in}}%
\pgfpathlineto{\pgfqpoint{4.487389in}{2.935422in}}%
\pgfpathlineto{\pgfqpoint{4.496416in}{2.877967in}}%
\pgfpathlineto{\pgfqpoint{4.505444in}{2.799627in}}%
\pgfpathlineto{\pgfqpoint{4.518984in}{2.651288in}}%
\pgfpathlineto{\pgfqpoint{4.541552in}{2.361830in}}%
\pgfpathlineto{\pgfqpoint{4.559607in}{2.140086in}}%
\pgfpathlineto{\pgfqpoint{4.568634in}{2.046634in}}%
\pgfpathlineto{\pgfqpoint{4.577661in}{1.970621in}}%
\pgfpathlineto{\pgfqpoint{4.586689in}{1.915435in}}%
\pgfpathlineto{\pgfqpoint{4.591202in}{1.896447in}}%
\pgfpathlineto{\pgfqpoint{4.595716in}{1.883489in}}%
\pgfpathlineto{\pgfqpoint{4.600229in}{1.876693in}}%
\pgfpathlineto{\pgfqpoint{4.604743in}{1.876119in}}%
\pgfpathlineto{\pgfqpoint{4.609257in}{1.881757in}}%
\pgfpathlineto{\pgfqpoint{4.613770in}{1.893524in}}%
\pgfpathlineto{\pgfqpoint{4.618284in}{1.911268in}}%
\pgfpathlineto{\pgfqpoint{4.627311in}{1.963735in}}%
\pgfpathlineto{\pgfqpoint{4.636338in}{2.036624in}}%
\pgfpathlineto{\pgfqpoint{4.645365in}{2.126477in}}%
\pgfpathlineto{\pgfqpoint{4.658906in}{2.283684in}}%
\pgfpathlineto{\pgfqpoint{4.690501in}{2.667784in}}%
\pgfpathlineto{\pgfqpoint{4.704042in}{2.799128in}}%
\pgfpathlineto{\pgfqpoint{4.713069in}{2.864876in}}%
\pgfpathlineto{\pgfqpoint{4.722097in}{2.909649in}}%
\pgfpathlineto{\pgfqpoint{4.726610in}{2.923523in}}%
\pgfpathlineto{\pgfqpoint{4.731124in}{2.931510in}}%
\pgfpathlineto{\pgfqpoint{4.735637in}{2.933532in}}%
\pgfpathlineto{\pgfqpoint{4.740151in}{2.929585in}}%
\pgfpathlineto{\pgfqpoint{4.744665in}{2.919729in}}%
\pgfpathlineto{\pgfqpoint{4.749178in}{2.904096in}}%
\pgfpathlineto{\pgfqpoint{4.758205in}{2.856345in}}%
\pgfpathlineto{\pgfqpoint{4.767233in}{2.788646in}}%
\pgfpathlineto{\pgfqpoint{4.776260in}{2.704215in}}%
\pgfpathlineto{\pgfqpoint{4.789801in}{2.555032in}}%
\pgfpathlineto{\pgfqpoint{4.825910in}{2.138690in}}%
\pgfpathlineto{\pgfqpoint{4.834937in}{2.056359in}}%
\pgfpathlineto{\pgfqpoint{4.843964in}{1.990959in}}%
\pgfpathlineto{\pgfqpoint{4.852991in}{1.945394in}}%
\pgfpathlineto{\pgfqpoint{4.857505in}{1.930706in}}%
\pgfpathlineto{\pgfqpoint{4.862018in}{1.921642in}}%
\pgfpathlineto{\pgfqpoint{4.866532in}{1.918288in}}%
\pgfpathlineto{\pgfqpoint{4.871046in}{1.920669in}}%
\pgfpathlineto{\pgfqpoint{4.875559in}{1.928740in}}%
\pgfpathlineto{\pgfqpoint{4.880073in}{1.942393in}}%
\pgfpathlineto{\pgfqpoint{4.884586in}{1.961454in}}%
\pgfpathlineto{\pgfqpoint{4.893614in}{2.014805in}}%
\pgfpathlineto{\pgfqpoint{4.902641in}{2.086231in}}%
\pgfpathlineto{\pgfqpoint{4.916182in}{2.219688in}}%
\pgfpathlineto{\pgfqpoint{4.961318in}{2.712607in}}%
\pgfpathlineto{\pgfqpoint{4.970345in}{2.785873in}}%
\pgfpathlineto{\pgfqpoint{4.979372in}{2.841743in}}%
\pgfpathlineto{\pgfqpoint{4.983886in}{2.862349in}}%
\pgfpathlineto{\pgfqpoint{4.988399in}{2.877753in}}%
\pgfpathlineto{\pgfqpoint{4.992913in}{2.887791in}}%
\pgfpathlineto{\pgfqpoint{4.997427in}{2.892363in}}%
\pgfpathlineto{\pgfqpoint{5.001940in}{2.891432in}}%
\pgfpathlineto{\pgfqpoint{5.006454in}{2.885023in}}%
\pgfpathlineto{\pgfqpoint{5.010967in}{2.873227in}}%
\pgfpathlineto{\pgfqpoint{5.015481in}{2.856194in}}%
\pgfpathlineto{\pgfqpoint{5.024508in}{2.807317in}}%
\pgfpathlineto{\pgfqpoint{5.033535in}{2.740746in}}%
\pgfpathlineto{\pgfqpoint{5.047076in}{2.614793in}}%
\pgfpathlineto{\pgfqpoint{5.069644in}{2.369221in}}%
\pgfpathlineto{\pgfqpoint{5.087699in}{2.181248in}}%
\pgfpathlineto{\pgfqpoint{5.096726in}{2.102083in}}%
\pgfpathlineto{\pgfqpoint{5.105753in}{2.037733in}}%
\pgfpathlineto{\pgfqpoint{5.114780in}{1.991068in}}%
\pgfpathlineto{\pgfqpoint{5.119294in}{1.975040in}}%
\pgfpathlineto{\pgfqpoint{5.123807in}{1.964127in}}%
\pgfpathlineto{\pgfqpoint{5.128321in}{1.958443in}}%
\pgfpathlineto{\pgfqpoint{5.132835in}{1.958036in}}%
\pgfpathlineto{\pgfqpoint{5.137348in}{1.962898in}}%
\pgfpathlineto{\pgfqpoint{5.141862in}{1.972956in}}%
\pgfpathlineto{\pgfqpoint{5.146375in}{1.988082in}}%
\pgfpathlineto{\pgfqpoint{5.155403in}{2.032724in}}%
\pgfpathlineto{\pgfqpoint{5.164430in}{2.094669in}}%
\pgfpathlineto{\pgfqpoint{5.177971in}{2.213420in}}%
\pgfpathlineto{\pgfqpoint{5.196025in}{2.399977in}}%
\pgfpathlineto{\pgfqpoint{5.218593in}{2.630115in}}%
\pgfpathlineto{\pgfqpoint{5.232134in}{2.741371in}}%
\pgfpathlineto{\pgfqpoint{5.241161in}{2.797012in}}%
\pgfpathlineto{\pgfqpoint{5.250188in}{2.834845in}}%
\pgfpathlineto{\pgfqpoint{5.254702in}{2.846538in}}%
\pgfpathlineto{\pgfqpoint{5.259216in}{2.853235in}}%
\pgfpathlineto{\pgfqpoint{5.263729in}{2.854874in}}%
\pgfpathlineto{\pgfqpoint{5.268243in}{2.851449in}}%
\pgfpathlineto{\pgfqpoint{5.272756in}{2.843015in}}%
\pgfpathlineto{\pgfqpoint{5.277270in}{2.829682in}}%
\pgfpathlineto{\pgfqpoint{5.286297in}{2.789043in}}%
\pgfpathlineto{\pgfqpoint{5.295324in}{2.731503in}}%
\pgfpathlineto{\pgfqpoint{5.304352in}{2.659794in}}%
\pgfpathlineto{\pgfqpoint{5.317892in}{2.533168in}}%
\pgfpathlineto{\pgfqpoint{5.354001in}{2.180136in}}%
\pgfpathlineto{\pgfqpoint{5.363028in}{2.110402in}}%
\pgfpathlineto{\pgfqpoint{5.372056in}{2.055049in}}%
\pgfpathlineto{\pgfqpoint{5.381083in}{2.016537in}}%
\pgfpathlineto{\pgfqpoint{5.385597in}{2.004150in}}%
\pgfpathlineto{\pgfqpoint{5.390110in}{1.996535in}}%
\pgfpathlineto{\pgfqpoint{5.394624in}{1.993765in}}%
\pgfpathlineto{\pgfqpoint{5.399137in}{1.995857in}}%
\pgfpathlineto{\pgfqpoint{5.403651in}{2.002776in}}%
\pgfpathlineto{\pgfqpoint{5.408165in}{2.014427in}}%
\pgfpathlineto{\pgfqpoint{5.412678in}{2.030662in}}%
\pgfpathlineto{\pgfqpoint{5.421705in}{2.076039in}}%
\pgfpathlineto{\pgfqpoint{5.430733in}{2.136728in}}%
\pgfpathlineto{\pgfqpoint{5.444273in}{2.250037in}}%
\pgfpathlineto{\pgfqpoint{5.489409in}{2.668022in}}%
\pgfpathlineto{\pgfqpoint{5.498437in}{2.730063in}}%
\pgfpathlineto{\pgfqpoint{5.507464in}{2.777330in}}%
\pgfpathlineto{\pgfqpoint{5.516491in}{2.807739in}}%
\pgfpathlineto{\pgfqpoint{5.521005in}{2.816184in}}%
\pgfpathlineto{\pgfqpoint{5.525518in}{2.819991in}}%
\pgfpathlineto{\pgfqpoint{5.530032in}{2.819130in}}%
\pgfpathlineto{\pgfqpoint{5.534545in}{2.813625in}}%
\pgfpathlineto{\pgfqpoint{5.534545in}{2.813625in}}%
\pgfusepath{stroke}%
\end{pgfscope}%
\begin{pgfscope}%
\pgfpathrectangle{\pgfqpoint{0.800000in}{0.528000in}}{\pgfqpoint{4.960000in}{3.696000in}}%
\pgfusepath{clip}%
\pgfsetbuttcap%
\pgfsetroundjoin%
\definecolor{currentfill}{rgb}{0.000000,0.000000,1.000000}%
\pgfsetfillcolor{currentfill}%
\pgfsetlinewidth{0.501875pt}%
\definecolor{currentstroke}{rgb}{0.000000,0.000000,1.000000}%
\pgfsetstrokecolor{currentstroke}%
\pgfsetdash{}{0pt}%
\pgfsys@defobject{currentmarker}{\pgfqpoint{-0.027778in}{-0.000000in}}{\pgfqpoint{0.027778in}{0.000000in}}{%
\pgfpathmoveto{\pgfqpoint{0.027778in}{-0.000000in}}%
\pgfpathlineto{\pgfqpoint{-0.027778in}{0.000000in}}%
\pgfusepath{stroke,fill}%
}%
\begin{pgfscope}%
\pgfsys@transformshift{1.043491in}{3.915024in}%
\pgfsys@useobject{currentmarker}{}%
\end{pgfscope}%
\begin{pgfscope}%
\pgfsys@transformshift{1.061527in}{3.739441in}%
\pgfsys@useobject{currentmarker}{}%
\end{pgfscope}%
\begin{pgfscope}%
\pgfsys@transformshift{1.079564in}{3.271219in}%
\pgfsys@useobject{currentmarker}{}%
\end{pgfscope}%
\begin{pgfscope}%
\pgfsys@transformshift{1.097600in}{2.685942in}%
\pgfsys@useobject{currentmarker}{}%
\end{pgfscope}%
\begin{pgfscope}%
\pgfsys@transformshift{1.115636in}{2.042137in}%
\pgfsys@useobject{currentmarker}{}%
\end{pgfscope}%
\begin{pgfscope}%
\pgfsys@transformshift{1.133673in}{1.281277in}%
\pgfsys@useobject{currentmarker}{}%
\end{pgfscope}%
\begin{pgfscope}%
\pgfsys@transformshift{1.151709in}{0.871583in}%
\pgfsys@useobject{currentmarker}{}%
\end{pgfscope}%
\begin{pgfscope}%
\pgfsys@transformshift{1.169745in}{0.754528in}%
\pgfsys@useobject{currentmarker}{}%
\end{pgfscope}%
\begin{pgfscope}%
\pgfsys@transformshift{1.187782in}{0.871583in}%
\pgfsys@useobject{currentmarker}{}%
\end{pgfscope}%
\begin{pgfscope}%
\pgfsys@transformshift{1.205818in}{1.281277in}%
\pgfsys@useobject{currentmarker}{}%
\end{pgfscope}%
\begin{pgfscope}%
\pgfsys@transformshift{1.223855in}{2.042137in}%
\pgfsys@useobject{currentmarker}{}%
\end{pgfscope}%
\begin{pgfscope}%
\pgfsys@transformshift{1.241891in}{2.627414in}%
\pgfsys@useobject{currentmarker}{}%
\end{pgfscope}%
\begin{pgfscope}%
\pgfsys@transformshift{1.259927in}{3.212692in}%
\pgfsys@useobject{currentmarker}{}%
\end{pgfscope}%
\begin{pgfscope}%
\pgfsys@transformshift{1.277964in}{3.622386in}%
\pgfsys@useobject{currentmarker}{}%
\end{pgfscope}%
\begin{pgfscope}%
\pgfsys@transformshift{1.296000in}{3.797969in}%
\pgfsys@useobject{currentmarker}{}%
\end{pgfscope}%
\begin{pgfscope}%
\pgfsys@transformshift{1.314036in}{3.797969in}%
\pgfsys@useobject{currentmarker}{}%
\end{pgfscope}%
\begin{pgfscope}%
\pgfsys@transformshift{1.332073in}{3.505330in}%
\pgfsys@useobject{currentmarker}{}%
\end{pgfscope}%
\begin{pgfscope}%
\pgfsys@transformshift{1.350109in}{3.037108in}%
\pgfsys@useobject{currentmarker}{}%
\end{pgfscope}%
\begin{pgfscope}%
\pgfsys@transformshift{1.368145in}{2.451831in}%
\pgfsys@useobject{currentmarker}{}%
\end{pgfscope}%
\begin{pgfscope}%
\pgfsys@transformshift{1.386182in}{1.925082in}%
\pgfsys@useobject{currentmarker}{}%
\end{pgfscope}%
\begin{pgfscope}%
\pgfsys@transformshift{1.404218in}{1.222749in}%
\pgfsys@useobject{currentmarker}{}%
\end{pgfscope}%
\begin{pgfscope}%
\pgfsys@transformshift{1.422255in}{0.930111in}%
\pgfsys@useobject{currentmarker}{}%
\end{pgfscope}%
\begin{pgfscope}%
\pgfsys@transformshift{1.440291in}{0.930111in}%
\pgfsys@useobject{currentmarker}{}%
\end{pgfscope}%
\begin{pgfscope}%
\pgfsys@transformshift{1.458327in}{1.105694in}%
\pgfsys@useobject{currentmarker}{}%
\end{pgfscope}%
\begin{pgfscope}%
\pgfsys@transformshift{1.476364in}{1.515388in}%
\pgfsys@useobject{currentmarker}{}%
\end{pgfscope}%
\begin{pgfscope}%
\pgfsys@transformshift{1.494400in}{2.217720in}%
\pgfsys@useobject{currentmarker}{}%
\end{pgfscope}%
\begin{pgfscope}%
\pgfsys@transformshift{1.512436in}{2.802998in}%
\pgfsys@useobject{currentmarker}{}%
\end{pgfscope}%
\begin{pgfscope}%
\pgfsys@transformshift{1.530473in}{3.271219in}%
\pgfsys@useobject{currentmarker}{}%
\end{pgfscope}%
\begin{pgfscope}%
\pgfsys@transformshift{1.548509in}{3.563858in}%
\pgfsys@useobject{currentmarker}{}%
\end{pgfscope}%
\begin{pgfscope}%
\pgfsys@transformshift{1.566545in}{3.680913in}%
\pgfsys@useobject{currentmarker}{}%
\end{pgfscope}%
\begin{pgfscope}%
\pgfsys@transformshift{1.584582in}{3.622386in}%
\pgfsys@useobject{currentmarker}{}%
\end{pgfscope}%
\begin{pgfscope}%
\pgfsys@transformshift{1.602618in}{3.271219in}%
\pgfsys@useobject{currentmarker}{}%
\end{pgfscope}%
\begin{pgfscope}%
\pgfsys@transformshift{1.620655in}{2.802998in}%
\pgfsys@useobject{currentmarker}{}%
\end{pgfscope}%
\begin{pgfscope}%
\pgfsys@transformshift{1.638691in}{2.276248in}%
\pgfsys@useobject{currentmarker}{}%
\end{pgfscope}%
\begin{pgfscope}%
\pgfsys@transformshift{1.656727in}{1.632443in}%
\pgfsys@useobject{currentmarker}{}%
\end{pgfscope}%
\begin{pgfscope}%
\pgfsys@transformshift{1.674764in}{1.222749in}%
\pgfsys@useobject{currentmarker}{}%
\end{pgfscope}%
\begin{pgfscope}%
\pgfsys@transformshift{1.692800in}{1.047166in}%
\pgfsys@useobject{currentmarker}{}%
\end{pgfscope}%
\begin{pgfscope}%
\pgfsys@transformshift{1.710836in}{1.047166in}%
\pgfsys@useobject{currentmarker}{}%
\end{pgfscope}%
\begin{pgfscope}%
\pgfsys@transformshift{1.728873in}{1.339805in}%
\pgfsys@useobject{currentmarker}{}%
\end{pgfscope}%
\begin{pgfscope}%
\pgfsys@transformshift{1.746909in}{1.925082in}%
\pgfsys@useobject{currentmarker}{}%
\end{pgfscope}%
\begin{pgfscope}%
\pgfsys@transformshift{1.764945in}{2.451831in}%
\pgfsys@useobject{currentmarker}{}%
\end{pgfscope}%
\begin{pgfscope}%
\pgfsys@transformshift{1.782982in}{2.920053in}%
\pgfsys@useobject{currentmarker}{}%
\end{pgfscope}%
\begin{pgfscope}%
\pgfsys@transformshift{1.801018in}{3.329747in}%
\pgfsys@useobject{currentmarker}{}%
\end{pgfscope}%
\begin{pgfscope}%
\pgfsys@transformshift{1.819055in}{3.563858in}%
\pgfsys@useobject{currentmarker}{}%
\end{pgfscope}%
\begin{pgfscope}%
\pgfsys@transformshift{1.837091in}{3.563858in}%
\pgfsys@useobject{currentmarker}{}%
\end{pgfscope}%
\begin{pgfscope}%
\pgfsys@transformshift{1.855127in}{3.446802in}%
\pgfsys@useobject{currentmarker}{}%
\end{pgfscope}%
\begin{pgfscope}%
\pgfsys@transformshift{1.873164in}{3.095636in}%
\pgfsys@useobject{currentmarker}{}%
\end{pgfscope}%
\begin{pgfscope}%
\pgfsys@transformshift{1.891200in}{2.627414in}%
\pgfsys@useobject{currentmarker}{}%
\end{pgfscope}%
\begin{pgfscope}%
\pgfsys@transformshift{1.909236in}{2.159193in}%
\pgfsys@useobject{currentmarker}{}%
\end{pgfscope}%
\begin{pgfscope}%
\pgfsys@transformshift{1.927273in}{1.515388in}%
\pgfsys@useobject{currentmarker}{}%
\end{pgfscope}%
\begin{pgfscope}%
\pgfsys@transformshift{1.945309in}{1.222749in}%
\pgfsys@useobject{currentmarker}{}%
\end{pgfscope}%
\begin{pgfscope}%
\pgfsys@transformshift{1.963345in}{1.105694in}%
\pgfsys@useobject{currentmarker}{}%
\end{pgfscope}%
\begin{pgfscope}%
\pgfsys@transformshift{1.981382in}{1.222749in}%
\pgfsys@useobject{currentmarker}{}%
\end{pgfscope}%
\begin{pgfscope}%
\pgfsys@transformshift{1.999418in}{1.515388in}%
\pgfsys@useobject{currentmarker}{}%
\end{pgfscope}%
\begin{pgfscope}%
\pgfsys@transformshift{2.017455in}{2.100665in}%
\pgfsys@useobject{currentmarker}{}%
\end{pgfscope}%
\begin{pgfscope}%
\pgfsys@transformshift{2.035491in}{2.627414in}%
\pgfsys@useobject{currentmarker}{}%
\end{pgfscope}%
\begin{pgfscope}%
\pgfsys@transformshift{2.053527in}{3.037108in}%
\pgfsys@useobject{currentmarker}{}%
\end{pgfscope}%
\begin{pgfscope}%
\pgfsys@transformshift{2.071564in}{3.329747in}%
\pgfsys@useobject{currentmarker}{}%
\end{pgfscope}%
\begin{pgfscope}%
\pgfsys@transformshift{2.089600in}{3.505330in}%
\pgfsys@useobject{currentmarker}{}%
\end{pgfscope}%
\begin{pgfscope}%
\pgfsys@transformshift{2.107636in}{3.505330in}%
\pgfsys@useobject{currentmarker}{}%
\end{pgfscope}%
\begin{pgfscope}%
\pgfsys@transformshift{2.125673in}{3.271219in}%
\pgfsys@useobject{currentmarker}{}%
\end{pgfscope}%
\begin{pgfscope}%
\pgfsys@transformshift{2.143709in}{2.920053in}%
\pgfsys@useobject{currentmarker}{}%
\end{pgfscope}%
\begin{pgfscope}%
\pgfsys@transformshift{2.161745in}{2.451831in}%
\pgfsys@useobject{currentmarker}{}%
\end{pgfscope}%
\begin{pgfscope}%
\pgfsys@transformshift{2.179782in}{2.042137in}%
\pgfsys@useobject{currentmarker}{}%
\end{pgfscope}%
\begin{pgfscope}%
\pgfsys@transformshift{2.197818in}{1.456860in}%
\pgfsys@useobject{currentmarker}{}%
\end{pgfscope}%
\begin{pgfscope}%
\pgfsys@transformshift{2.215855in}{1.281277in}%
\pgfsys@useobject{currentmarker}{}%
\end{pgfscope}%
\begin{pgfscope}%
\pgfsys@transformshift{2.233891in}{1.222749in}%
\pgfsys@useobject{currentmarker}{}%
\end{pgfscope}%
\begin{pgfscope}%
\pgfsys@transformshift{2.251927in}{1.398333in}%
\pgfsys@useobject{currentmarker}{}%
\end{pgfscope}%
\begin{pgfscope}%
\pgfsys@transformshift{2.269964in}{1.749499in}%
\pgfsys@useobject{currentmarker}{}%
\end{pgfscope}%
\begin{pgfscope}%
\pgfsys@transformshift{2.288000in}{2.334776in}%
\pgfsys@useobject{currentmarker}{}%
\end{pgfscope}%
\begin{pgfscope}%
\pgfsys@transformshift{2.306036in}{2.744470in}%
\pgfsys@useobject{currentmarker}{}%
\end{pgfscope}%
\begin{pgfscope}%
\pgfsys@transformshift{2.324073in}{3.095636in}%
\pgfsys@useobject{currentmarker}{}%
\end{pgfscope}%
\begin{pgfscope}%
\pgfsys@transformshift{2.342109in}{3.329747in}%
\pgfsys@useobject{currentmarker}{}%
\end{pgfscope}%
\begin{pgfscope}%
\pgfsys@transformshift{2.360145in}{3.446802in}%
\pgfsys@useobject{currentmarker}{}%
\end{pgfscope}%
\begin{pgfscope}%
\pgfsys@transformshift{2.378182in}{3.329747in}%
\pgfsys@useobject{currentmarker}{}%
\end{pgfscope}%
\begin{pgfscope}%
\pgfsys@transformshift{2.396218in}{3.095636in}%
\pgfsys@useobject{currentmarker}{}%
\end{pgfscope}%
\begin{pgfscope}%
\pgfsys@transformshift{2.414255in}{2.744470in}%
\pgfsys@useobject{currentmarker}{}%
\end{pgfscope}%
\begin{pgfscope}%
\pgfsys@transformshift{2.432291in}{2.334776in}%
\pgfsys@useobject{currentmarker}{}%
\end{pgfscope}%
\begin{pgfscope}%
\pgfsys@transformshift{2.450327in}{1.749499in}%
\pgfsys@useobject{currentmarker}{}%
\end{pgfscope}%
\begin{pgfscope}%
\pgfsys@transformshift{2.468364in}{1.456860in}%
\pgfsys@useobject{currentmarker}{}%
\end{pgfscope}%
\begin{pgfscope}%
\pgfsys@transformshift{2.486400in}{1.281277in}%
\pgfsys@useobject{currentmarker}{}%
\end{pgfscope}%
\begin{pgfscope}%
\pgfsys@transformshift{2.504436in}{1.339805in}%
\pgfsys@useobject{currentmarker}{}%
\end{pgfscope}%
\begin{pgfscope}%
\pgfsys@transformshift{2.522473in}{1.573916in}%
\pgfsys@useobject{currentmarker}{}%
\end{pgfscope}%
\begin{pgfscope}%
\pgfsys@transformshift{2.540509in}{2.042137in}%
\pgfsys@useobject{currentmarker}{}%
\end{pgfscope}%
\begin{pgfscope}%
\pgfsys@transformshift{2.558545in}{2.451831in}%
\pgfsys@useobject{currentmarker}{}%
\end{pgfscope}%
\begin{pgfscope}%
\pgfsys@transformshift{2.576582in}{2.861525in}%
\pgfsys@useobject{currentmarker}{}%
\end{pgfscope}%
\begin{pgfscope}%
\pgfsys@transformshift{2.594618in}{3.154164in}%
\pgfsys@useobject{currentmarker}{}%
\end{pgfscope}%
\begin{pgfscope}%
\pgfsys@transformshift{2.612655in}{3.329747in}%
\pgfsys@useobject{currentmarker}{}%
\end{pgfscope}%
\begin{pgfscope}%
\pgfsys@transformshift{2.630691in}{3.329747in}%
\pgfsys@useobject{currentmarker}{}%
\end{pgfscope}%
\begin{pgfscope}%
\pgfsys@transformshift{2.648727in}{3.212692in}%
\pgfsys@useobject{currentmarker}{}%
\end{pgfscope}%
\begin{pgfscope}%
\pgfsys@transformshift{2.666764in}{2.920053in}%
\pgfsys@useobject{currentmarker}{}%
\end{pgfscope}%
\begin{pgfscope}%
\pgfsys@transformshift{2.684800in}{2.568887in}%
\pgfsys@useobject{currentmarker}{}%
\end{pgfscope}%
\begin{pgfscope}%
\pgfsys@transformshift{2.702836in}{2.217720in}%
\pgfsys@useobject{currentmarker}{}%
\end{pgfscope}%
\begin{pgfscope}%
\pgfsys@transformshift{2.720873in}{1.808026in}%
\pgfsys@useobject{currentmarker}{}%
\end{pgfscope}%
\begin{pgfscope}%
\pgfsys@transformshift{2.738909in}{1.456860in}%
\pgfsys@useobject{currentmarker}{}%
\end{pgfscope}%
\begin{pgfscope}%
\pgfsys@transformshift{2.756945in}{1.398333in}%
\pgfsys@useobject{currentmarker}{}%
\end{pgfscope}%
\begin{pgfscope}%
\pgfsys@transformshift{2.774982in}{1.456860in}%
\pgfsys@useobject{currentmarker}{}%
\end{pgfscope}%
\begin{pgfscope}%
\pgfsys@transformshift{2.793018in}{1.866554in}%
\pgfsys@useobject{currentmarker}{}%
\end{pgfscope}%
\begin{pgfscope}%
\pgfsys@transformshift{2.811055in}{2.217720in}%
\pgfsys@useobject{currentmarker}{}%
\end{pgfscope}%
\begin{pgfscope}%
\pgfsys@transformshift{2.829091in}{2.627414in}%
\pgfsys@useobject{currentmarker}{}%
\end{pgfscope}%
\begin{pgfscope}%
\pgfsys@transformshift{2.847127in}{2.920053in}%
\pgfsys@useobject{currentmarker}{}%
\end{pgfscope}%
\begin{pgfscope}%
\pgfsys@transformshift{2.865164in}{3.212692in}%
\pgfsys@useobject{currentmarker}{}%
\end{pgfscope}%
\begin{pgfscope}%
\pgfsys@transformshift{2.883200in}{3.271219in}%
\pgfsys@useobject{currentmarker}{}%
\end{pgfscope}%
\begin{pgfscope}%
\pgfsys@transformshift{2.901236in}{3.271219in}%
\pgfsys@useobject{currentmarker}{}%
\end{pgfscope}%
\begin{pgfscope}%
\pgfsys@transformshift{2.919273in}{3.095636in}%
\pgfsys@useobject{currentmarker}{}%
\end{pgfscope}%
\begin{pgfscope}%
\pgfsys@transformshift{2.937309in}{2.744470in}%
\pgfsys@useobject{currentmarker}{}%
\end{pgfscope}%
\begin{pgfscope}%
\pgfsys@transformshift{2.955345in}{2.451831in}%
\pgfsys@useobject{currentmarker}{}%
\end{pgfscope}%
\begin{pgfscope}%
\pgfsys@transformshift{2.973382in}{2.100665in}%
\pgfsys@useobject{currentmarker}{}%
\end{pgfscope}%
\begin{pgfscope}%
\pgfsys@transformshift{2.991418in}{1.808026in}%
\pgfsys@useobject{currentmarker}{}%
\end{pgfscope}%
\begin{pgfscope}%
\pgfsys@transformshift{3.009455in}{1.456860in}%
\pgfsys@useobject{currentmarker}{}%
\end{pgfscope}%
\begin{pgfscope}%
\pgfsys@transformshift{3.027491in}{1.456860in}%
\pgfsys@useobject{currentmarker}{}%
\end{pgfscope}%
\begin{pgfscope}%
\pgfsys@transformshift{3.045527in}{1.632443in}%
\pgfsys@useobject{currentmarker}{}%
\end{pgfscope}%
\begin{pgfscope}%
\pgfsys@transformshift{3.063564in}{2.042137in}%
\pgfsys@useobject{currentmarker}{}%
\end{pgfscope}%
\begin{pgfscope}%
\pgfsys@transformshift{3.081600in}{2.393304in}%
\pgfsys@useobject{currentmarker}{}%
\end{pgfscope}%
\begin{pgfscope}%
\pgfsys@transformshift{3.099636in}{2.685942in}%
\pgfsys@useobject{currentmarker}{}%
\end{pgfscope}%
\begin{pgfscope}%
\pgfsys@transformshift{3.117673in}{2.978581in}%
\pgfsys@useobject{currentmarker}{}%
\end{pgfscope}%
\begin{pgfscope}%
\pgfsys@transformshift{3.135709in}{3.154164in}%
\pgfsys@useobject{currentmarker}{}%
\end{pgfscope}%
\begin{pgfscope}%
\pgfsys@transformshift{3.153745in}{3.212692in}%
\pgfsys@useobject{currentmarker}{}%
\end{pgfscope}%
\begin{pgfscope}%
\pgfsys@transformshift{3.171782in}{3.154164in}%
\pgfsys@useobject{currentmarker}{}%
\end{pgfscope}%
\begin{pgfscope}%
\pgfsys@transformshift{3.189818in}{2.920053in}%
\pgfsys@useobject{currentmarker}{}%
\end{pgfscope}%
\begin{pgfscope}%
\pgfsys@transformshift{3.207855in}{2.627414in}%
\pgfsys@useobject{currentmarker}{}%
\end{pgfscope}%
\begin{pgfscope}%
\pgfsys@transformshift{3.225891in}{2.334776in}%
\pgfsys@useobject{currentmarker}{}%
\end{pgfscope}%
\begin{pgfscope}%
\pgfsys@transformshift{3.243927in}{1.983610in}%
\pgfsys@useobject{currentmarker}{}%
\end{pgfscope}%
\begin{pgfscope}%
\pgfsys@transformshift{3.261964in}{1.632443in}%
\pgfsys@useobject{currentmarker}{}%
\end{pgfscope}%
\begin{pgfscope}%
\pgfsys@transformshift{3.280000in}{1.515388in}%
\pgfsys@useobject{currentmarker}{}%
\end{pgfscope}%
\begin{pgfscope}%
\pgfsys@transformshift{3.298036in}{1.573916in}%
\pgfsys@useobject{currentmarker}{}%
\end{pgfscope}%
\begin{pgfscope}%
\pgfsys@transformshift{3.316073in}{1.749499in}%
\pgfsys@useobject{currentmarker}{}%
\end{pgfscope}%
\begin{pgfscope}%
\pgfsys@transformshift{3.334109in}{2.159193in}%
\pgfsys@useobject{currentmarker}{}%
\end{pgfscope}%
\begin{pgfscope}%
\pgfsys@transformshift{3.352145in}{2.510359in}%
\pgfsys@useobject{currentmarker}{}%
\end{pgfscope}%
\begin{pgfscope}%
\pgfsys@transformshift{3.370182in}{2.802998in}%
\pgfsys@useobject{currentmarker}{}%
\end{pgfscope}%
\begin{pgfscope}%
\pgfsys@transformshift{3.388218in}{3.037108in}%
\pgfsys@useobject{currentmarker}{}%
\end{pgfscope}%
\begin{pgfscope}%
\pgfsys@transformshift{3.406255in}{3.154164in}%
\pgfsys@useobject{currentmarker}{}%
\end{pgfscope}%
\begin{pgfscope}%
\pgfsys@transformshift{3.424291in}{3.154164in}%
\pgfsys@useobject{currentmarker}{}%
\end{pgfscope}%
\begin{pgfscope}%
\pgfsys@transformshift{3.442327in}{3.037108in}%
\pgfsys@useobject{currentmarker}{}%
\end{pgfscope}%
\begin{pgfscope}%
\pgfsys@transformshift{3.460364in}{2.802998in}%
\pgfsys@useobject{currentmarker}{}%
\end{pgfscope}%
\begin{pgfscope}%
\pgfsys@transformshift{3.478400in}{2.510359in}%
\pgfsys@useobject{currentmarker}{}%
\end{pgfscope}%
\begin{pgfscope}%
\pgfsys@transformshift{3.496436in}{2.217720in}%
\pgfsys@useobject{currentmarker}{}%
\end{pgfscope}%
\begin{pgfscope}%
\pgfsys@transformshift{3.514473in}{1.925082in}%
\pgfsys@useobject{currentmarker}{}%
\end{pgfscope}%
\begin{pgfscope}%
\pgfsys@transformshift{3.532509in}{1.632443in}%
\pgfsys@useobject{currentmarker}{}%
\end{pgfscope}%
\begin{pgfscope}%
\pgfsys@transformshift{3.550545in}{1.573916in}%
\pgfsys@useobject{currentmarker}{}%
\end{pgfscope}%
\begin{pgfscope}%
\pgfsys@transformshift{3.568582in}{1.632443in}%
\pgfsys@useobject{currentmarker}{}%
\end{pgfscope}%
\begin{pgfscope}%
\pgfsys@transformshift{3.586618in}{1.983610in}%
\pgfsys@useobject{currentmarker}{}%
\end{pgfscope}%
\begin{pgfscope}%
\pgfsys@transformshift{3.604655in}{2.276248in}%
\pgfsys@useobject{currentmarker}{}%
\end{pgfscope}%
\begin{pgfscope}%
\pgfsys@transformshift{3.622691in}{2.627414in}%
\pgfsys@useobject{currentmarker}{}%
\end{pgfscope}%
\begin{pgfscope}%
\pgfsys@transformshift{3.640727in}{2.861525in}%
\pgfsys@useobject{currentmarker}{}%
\end{pgfscope}%
\begin{pgfscope}%
\pgfsys@transformshift{3.658764in}{3.037108in}%
\pgfsys@useobject{currentmarker}{}%
\end{pgfscope}%
\begin{pgfscope}%
\pgfsys@transformshift{3.676800in}{3.095636in}%
\pgfsys@useobject{currentmarker}{}%
\end{pgfscope}%
\begin{pgfscope}%
\pgfsys@transformshift{3.694836in}{3.095636in}%
\pgfsys@useobject{currentmarker}{}%
\end{pgfscope}%
\begin{pgfscope}%
\pgfsys@transformshift{3.712873in}{2.920053in}%
\pgfsys@useobject{currentmarker}{}%
\end{pgfscope}%
\begin{pgfscope}%
\pgfsys@transformshift{3.730909in}{2.685942in}%
\pgfsys@useobject{currentmarker}{}%
\end{pgfscope}%
\begin{pgfscope}%
\pgfsys@transformshift{3.748945in}{2.393304in}%
\pgfsys@useobject{currentmarker}{}%
\end{pgfscope}%
\begin{pgfscope}%
\pgfsys@transformshift{3.766982in}{2.159193in}%
\pgfsys@useobject{currentmarker}{}%
\end{pgfscope}%
\begin{pgfscope}%
\pgfsys@transformshift{3.785018in}{1.925082in}%
\pgfsys@useobject{currentmarker}{}%
\end{pgfscope}%
\begin{pgfscope}%
\pgfsys@transformshift{3.803055in}{1.808026in}%
\pgfsys@useobject{currentmarker}{}%
\end{pgfscope}%
\begin{pgfscope}%
\pgfsys@transformshift{3.821091in}{1.808026in}%
\pgfsys@useobject{currentmarker}{}%
\end{pgfscope}%
\begin{pgfscope}%
\pgfsys@transformshift{3.839127in}{1.925082in}%
\pgfsys@useobject{currentmarker}{}%
\end{pgfscope}%
\begin{pgfscope}%
\pgfsys@transformshift{3.857164in}{2.159193in}%
\pgfsys@useobject{currentmarker}{}%
\end{pgfscope}%
\begin{pgfscope}%
\pgfsys@transformshift{3.875200in}{2.451831in}%
\pgfsys@useobject{currentmarker}{}%
\end{pgfscope}%
\begin{pgfscope}%
\pgfsys@transformshift{3.893236in}{2.685942in}%
\pgfsys@useobject{currentmarker}{}%
\end{pgfscope}%
\begin{pgfscope}%
\pgfsys@transformshift{3.911273in}{2.920053in}%
\pgfsys@useobject{currentmarker}{}%
\end{pgfscope}%
\begin{pgfscope}%
\pgfsys@transformshift{3.929309in}{3.037108in}%
\pgfsys@useobject{currentmarker}{}%
\end{pgfscope}%
\begin{pgfscope}%
\pgfsys@transformshift{3.947345in}{3.095636in}%
\pgfsys@useobject{currentmarker}{}%
\end{pgfscope}%
\begin{pgfscope}%
\pgfsys@transformshift{3.965382in}{2.978581in}%
\pgfsys@useobject{currentmarker}{}%
\end{pgfscope}%
\begin{pgfscope}%
\pgfsys@transformshift{3.983418in}{2.802998in}%
\pgfsys@useobject{currentmarker}{}%
\end{pgfscope}%
\begin{pgfscope}%
\pgfsys@transformshift{4.001455in}{2.568887in}%
\pgfsys@useobject{currentmarker}{}%
\end{pgfscope}%
\begin{pgfscope}%
\pgfsys@transformshift{4.019491in}{2.334776in}%
\pgfsys@useobject{currentmarker}{}%
\end{pgfscope}%
\begin{pgfscope}%
\pgfsys@transformshift{4.037527in}{2.042137in}%
\pgfsys@useobject{currentmarker}{}%
\end{pgfscope}%
\begin{pgfscope}%
\pgfsys@transformshift{4.055564in}{1.866554in}%
\pgfsys@useobject{currentmarker}{}%
\end{pgfscope}%
\begin{pgfscope}%
\pgfsys@transformshift{4.073600in}{1.808026in}%
\pgfsys@useobject{currentmarker}{}%
\end{pgfscope}%
\begin{pgfscope}%
\pgfsys@transformshift{4.091636in}{1.866554in}%
\pgfsys@useobject{currentmarker}{}%
\end{pgfscope}%
\begin{pgfscope}%
\pgfsys@transformshift{4.109673in}{2.042137in}%
\pgfsys@useobject{currentmarker}{}%
\end{pgfscope}%
\begin{pgfscope}%
\pgfsys@transformshift{4.127709in}{2.276248in}%
\pgfsys@useobject{currentmarker}{}%
\end{pgfscope}%
\begin{pgfscope}%
\pgfsys@transformshift{4.145745in}{2.510359in}%
\pgfsys@useobject{currentmarker}{}%
\end{pgfscope}%
\begin{pgfscope}%
\pgfsys@transformshift{4.163782in}{2.744470in}%
\pgfsys@useobject{currentmarker}{}%
\end{pgfscope}%
\begin{pgfscope}%
\pgfsys@transformshift{4.181818in}{2.920053in}%
\pgfsys@useobject{currentmarker}{}%
\end{pgfscope}%
\begin{pgfscope}%
\pgfsys@transformshift{4.199855in}{3.037108in}%
\pgfsys@useobject{currentmarker}{}%
\end{pgfscope}%
\begin{pgfscope}%
\pgfsys@transformshift{4.217891in}{3.037108in}%
\pgfsys@useobject{currentmarker}{}%
\end{pgfscope}%
\begin{pgfscope}%
\pgfsys@transformshift{4.235927in}{2.920053in}%
\pgfsys@useobject{currentmarker}{}%
\end{pgfscope}%
\begin{pgfscope}%
\pgfsys@transformshift{4.253964in}{2.685942in}%
\pgfsys@useobject{currentmarker}{}%
\end{pgfscope}%
\begin{pgfscope}%
\pgfsys@transformshift{4.272000in}{2.451831in}%
\pgfsys@useobject{currentmarker}{}%
\end{pgfscope}%
\begin{pgfscope}%
\pgfsys@transformshift{4.290036in}{2.217720in}%
\pgfsys@useobject{currentmarker}{}%
\end{pgfscope}%
\begin{pgfscope}%
\pgfsys@transformshift{4.308073in}{2.042137in}%
\pgfsys@useobject{currentmarker}{}%
\end{pgfscope}%
\begin{pgfscope}%
\pgfsys@transformshift{4.326109in}{1.866554in}%
\pgfsys@useobject{currentmarker}{}%
\end{pgfscope}%
\begin{pgfscope}%
\pgfsys@transformshift{4.344145in}{1.866554in}%
\pgfsys@useobject{currentmarker}{}%
\end{pgfscope}%
\begin{pgfscope}%
\pgfsys@transformshift{4.362182in}{1.983610in}%
\pgfsys@useobject{currentmarker}{}%
\end{pgfscope}%
\begin{pgfscope}%
\pgfsys@transformshift{4.380218in}{2.159193in}%
\pgfsys@useobject{currentmarker}{}%
\end{pgfscope}%
\begin{pgfscope}%
\pgfsys@transformshift{4.398255in}{2.393304in}%
\pgfsys@useobject{currentmarker}{}%
\end{pgfscope}%
\begin{pgfscope}%
\pgfsys@transformshift{4.416291in}{2.627414in}%
\pgfsys@useobject{currentmarker}{}%
\end{pgfscope}%
\begin{pgfscope}%
\pgfsys@transformshift{4.434327in}{2.802998in}%
\pgfsys@useobject{currentmarker}{}%
\end{pgfscope}%
\begin{pgfscope}%
\pgfsys@transformshift{4.452364in}{2.920053in}%
\pgfsys@useobject{currentmarker}{}%
\end{pgfscope}%
\begin{pgfscope}%
\pgfsys@transformshift{4.470400in}{2.978581in}%
\pgfsys@useobject{currentmarker}{}%
\end{pgfscope}%
\begin{pgfscope}%
\pgfsys@transformshift{4.488436in}{2.920053in}%
\pgfsys@useobject{currentmarker}{}%
\end{pgfscope}%
\begin{pgfscope}%
\pgfsys@transformshift{4.506473in}{2.802998in}%
\pgfsys@useobject{currentmarker}{}%
\end{pgfscope}%
\begin{pgfscope}%
\pgfsys@transformshift{4.524509in}{2.627414in}%
\pgfsys@useobject{currentmarker}{}%
\end{pgfscope}%
\begin{pgfscope}%
\pgfsys@transformshift{4.542545in}{2.393304in}%
\pgfsys@useobject{currentmarker}{}%
\end{pgfscope}%
\begin{pgfscope}%
\pgfsys@transformshift{4.560582in}{2.159193in}%
\pgfsys@useobject{currentmarker}{}%
\end{pgfscope}%
\begin{pgfscope}%
\pgfsys@transformshift{4.578618in}{1.983610in}%
\pgfsys@useobject{currentmarker}{}%
\end{pgfscope}%
\begin{pgfscope}%
\pgfsys@transformshift{4.596655in}{1.925082in}%
\pgfsys@useobject{currentmarker}{}%
\end{pgfscope}%
\begin{pgfscope}%
\pgfsys@transformshift{4.614691in}{1.925082in}%
\pgfsys@useobject{currentmarker}{}%
\end{pgfscope}%
\begin{pgfscope}%
\pgfsys@transformshift{4.632727in}{2.042137in}%
\pgfsys@useobject{currentmarker}{}%
\end{pgfscope}%
\begin{pgfscope}%
\pgfsys@transformshift{4.650764in}{2.217720in}%
\pgfsys@useobject{currentmarker}{}%
\end{pgfscope}%
\begin{pgfscope}%
\pgfsys@transformshift{4.668800in}{2.451831in}%
\pgfsys@useobject{currentmarker}{}%
\end{pgfscope}%
\begin{pgfscope}%
\pgfsys@transformshift{4.686836in}{2.685942in}%
\pgfsys@useobject{currentmarker}{}%
\end{pgfscope}%
\begin{pgfscope}%
\pgfsys@transformshift{4.704873in}{2.861525in}%
\pgfsys@useobject{currentmarker}{}%
\end{pgfscope}%
\begin{pgfscope}%
\pgfsys@transformshift{4.722909in}{2.920053in}%
\pgfsys@useobject{currentmarker}{}%
\end{pgfscope}%
\begin{pgfscope}%
\pgfsys@transformshift{4.740945in}{2.978581in}%
\pgfsys@useobject{currentmarker}{}%
\end{pgfscope}%
\begin{pgfscope}%
\pgfsys@transformshift{4.758982in}{2.861525in}%
\pgfsys@useobject{currentmarker}{}%
\end{pgfscope}%
\begin{pgfscope}%
\pgfsys@transformshift{4.777018in}{2.685942in}%
\pgfsys@useobject{currentmarker}{}%
\end{pgfscope}%
\begin{pgfscope}%
\pgfsys@transformshift{4.795055in}{2.510359in}%
\pgfsys@useobject{currentmarker}{}%
\end{pgfscope}%
\begin{pgfscope}%
\pgfsys@transformshift{4.813091in}{2.276248in}%
\pgfsys@useobject{currentmarker}{}%
\end{pgfscope}%
\begin{pgfscope}%
\pgfsys@transformshift{4.831127in}{2.100665in}%
\pgfsys@useobject{currentmarker}{}%
\end{pgfscope}%
\begin{pgfscope}%
\pgfsys@transformshift{4.849164in}{1.983610in}%
\pgfsys@useobject{currentmarker}{}%
\end{pgfscope}%
\begin{pgfscope}%
\pgfsys@transformshift{4.867200in}{1.983610in}%
\pgfsys@useobject{currentmarker}{}%
\end{pgfscope}%
\begin{pgfscope}%
\pgfsys@transformshift{4.885236in}{1.983610in}%
\pgfsys@useobject{currentmarker}{}%
\end{pgfscope}%
\begin{pgfscope}%
\pgfsys@transformshift{4.903273in}{2.159193in}%
\pgfsys@useobject{currentmarker}{}%
\end{pgfscope}%
\begin{pgfscope}%
\pgfsys@transformshift{4.921309in}{2.334776in}%
\pgfsys@useobject{currentmarker}{}%
\end{pgfscope}%
\begin{pgfscope}%
\pgfsys@transformshift{4.939345in}{2.568887in}%
\pgfsys@useobject{currentmarker}{}%
\end{pgfscope}%
\begin{pgfscope}%
\pgfsys@transformshift{4.957382in}{2.744470in}%
\pgfsys@useobject{currentmarker}{}%
\end{pgfscope}%
\begin{pgfscope}%
\pgfsys@transformshift{4.975418in}{2.861525in}%
\pgfsys@useobject{currentmarker}{}%
\end{pgfscope}%
\begin{pgfscope}%
\pgfsys@transformshift{4.993455in}{2.920053in}%
\pgfsys@useobject{currentmarker}{}%
\end{pgfscope}%
\begin{pgfscope}%
\pgfsys@transformshift{5.011491in}{2.920053in}%
\pgfsys@useobject{currentmarker}{}%
\end{pgfscope}%
\begin{pgfscope}%
\pgfsys@transformshift{5.029527in}{2.802998in}%
\pgfsys@useobject{currentmarker}{}%
\end{pgfscope}%
\begin{pgfscope}%
\pgfsys@transformshift{5.047564in}{2.627414in}%
\pgfsys@useobject{currentmarker}{}%
\end{pgfscope}%
\begin{pgfscope}%
\pgfsys@transformshift{5.065600in}{2.451831in}%
\pgfsys@useobject{currentmarker}{}%
\end{pgfscope}%
\begin{pgfscope}%
\pgfsys@transformshift{5.083636in}{2.217720in}%
\pgfsys@useobject{currentmarker}{}%
\end{pgfscope}%
\begin{pgfscope}%
\pgfsys@transformshift{5.101673in}{2.100665in}%
\pgfsys@useobject{currentmarker}{}%
\end{pgfscope}%
\begin{pgfscope}%
\pgfsys@transformshift{5.119709in}{1.983610in}%
\pgfsys@useobject{currentmarker}{}%
\end{pgfscope}%
\begin{pgfscope}%
\pgfsys@transformshift{5.137745in}{1.983610in}%
\pgfsys@useobject{currentmarker}{}%
\end{pgfscope}%
\begin{pgfscope}%
\pgfsys@transformshift{5.155782in}{2.100665in}%
\pgfsys@useobject{currentmarker}{}%
\end{pgfscope}%
\begin{pgfscope}%
\pgfsys@transformshift{5.173818in}{2.217720in}%
\pgfsys@useobject{currentmarker}{}%
\end{pgfscope}%
\begin{pgfscope}%
\pgfsys@transformshift{5.191855in}{2.393304in}%
\pgfsys@useobject{currentmarker}{}%
\end{pgfscope}%
\begin{pgfscope}%
\pgfsys@transformshift{5.209891in}{2.627414in}%
\pgfsys@useobject{currentmarker}{}%
\end{pgfscope}%
\begin{pgfscope}%
\pgfsys@transformshift{5.227927in}{2.744470in}%
\pgfsys@useobject{currentmarker}{}%
\end{pgfscope}%
\begin{pgfscope}%
\pgfsys@transformshift{5.245964in}{2.861525in}%
\pgfsys@useobject{currentmarker}{}%
\end{pgfscope}%
\begin{pgfscope}%
\pgfsys@transformshift{5.264000in}{2.920053in}%
\pgfsys@useobject{currentmarker}{}%
\end{pgfscope}%
\begin{pgfscope}%
\pgfsys@transformshift{5.282036in}{2.861525in}%
\pgfsys@useobject{currentmarker}{}%
\end{pgfscope}%
\begin{pgfscope}%
\pgfsys@transformshift{5.300073in}{2.744470in}%
\pgfsys@useobject{currentmarker}{}%
\end{pgfscope}%
\begin{pgfscope}%
\pgfsys@transformshift{5.318109in}{2.568887in}%
\pgfsys@useobject{currentmarker}{}%
\end{pgfscope}%
\begin{pgfscope}%
\pgfsys@transformshift{5.336145in}{2.334776in}%
\pgfsys@useobject{currentmarker}{}%
\end{pgfscope}%
\begin{pgfscope}%
\pgfsys@transformshift{5.354182in}{2.217720in}%
\pgfsys@useobject{currentmarker}{}%
\end{pgfscope}%
\begin{pgfscope}%
\pgfsys@transformshift{5.372218in}{2.042137in}%
\pgfsys@useobject{currentmarker}{}%
\end{pgfscope}%
\begin{pgfscope}%
\pgfsys@transformshift{5.390255in}{2.042137in}%
\pgfsys@useobject{currentmarker}{}%
\end{pgfscope}%
\begin{pgfscope}%
\pgfsys@transformshift{5.408291in}{2.042137in}%
\pgfsys@useobject{currentmarker}{}%
\end{pgfscope}%
\begin{pgfscope}%
\pgfsys@transformshift{5.426327in}{2.159193in}%
\pgfsys@useobject{currentmarker}{}%
\end{pgfscope}%
\begin{pgfscope}%
\pgfsys@transformshift{5.444364in}{2.334776in}%
\pgfsys@useobject{currentmarker}{}%
\end{pgfscope}%
\begin{pgfscope}%
\pgfsys@transformshift{5.462400in}{2.510359in}%
\pgfsys@useobject{currentmarker}{}%
\end{pgfscope}%
\begin{pgfscope}%
\pgfsys@transformshift{5.480436in}{2.685942in}%
\pgfsys@useobject{currentmarker}{}%
\end{pgfscope}%
\begin{pgfscope}%
\pgfsys@transformshift{5.498473in}{2.802998in}%
\pgfsys@useobject{currentmarker}{}%
\end{pgfscope}%
\begin{pgfscope}%
\pgfsys@transformshift{5.516509in}{2.861525in}%
\pgfsys@useobject{currentmarker}{}%
\end{pgfscope}%
\begin{pgfscope}%
\pgfsys@transformshift{5.534545in}{2.861525in}%
\pgfsys@useobject{currentmarker}{}%
\end{pgfscope}%
\end{pgfscope}%
\begin{pgfscope}%
\pgfsetrectcap%
\pgfsetmiterjoin%
\pgfsetlinewidth{0.803000pt}%
\definecolor{currentstroke}{rgb}{0.000000,0.000000,0.000000}%
\pgfsetstrokecolor{currentstroke}%
\pgfsetdash{}{0pt}%
\pgfpathmoveto{\pgfqpoint{0.800000in}{0.528000in}}%
\pgfpathlineto{\pgfqpoint{0.800000in}{4.224000in}}%
\pgfusepath{stroke}%
\end{pgfscope}%
\begin{pgfscope}%
\pgfsetrectcap%
\pgfsetmiterjoin%
\pgfsetlinewidth{0.803000pt}%
\definecolor{currentstroke}{rgb}{0.000000,0.000000,0.000000}%
\pgfsetstrokecolor{currentstroke}%
\pgfsetdash{}{0pt}%
\pgfpathmoveto{\pgfqpoint{5.760000in}{0.528000in}}%
\pgfpathlineto{\pgfqpoint{5.760000in}{4.224000in}}%
\pgfusepath{stroke}%
\end{pgfscope}%
\begin{pgfscope}%
\pgfsetrectcap%
\pgfsetmiterjoin%
\pgfsetlinewidth{0.803000pt}%
\definecolor{currentstroke}{rgb}{0.000000,0.000000,0.000000}%
\pgfsetstrokecolor{currentstroke}%
\pgfsetdash{}{0pt}%
\pgfpathmoveto{\pgfqpoint{0.800000in}{0.528000in}}%
\pgfpathlineto{\pgfqpoint{5.760000in}{0.528000in}}%
\pgfusepath{stroke}%
\end{pgfscope}%
\begin{pgfscope}%
\pgfsetrectcap%
\pgfsetmiterjoin%
\pgfsetlinewidth{0.803000pt}%
\definecolor{currentstroke}{rgb}{0.000000,0.000000,0.000000}%
\pgfsetstrokecolor{currentstroke}%
\pgfsetdash{}{0pt}%
\pgfpathmoveto{\pgfqpoint{0.800000in}{4.224000in}}%
\pgfpathlineto{\pgfqpoint{5.760000in}{4.224000in}}%
\pgfusepath{stroke}%
\end{pgfscope}%
\begin{pgfscope}%
\pgfsetbuttcap%
\pgfsetmiterjoin%
\definecolor{currentfill}{rgb}{1.000000,1.000000,1.000000}%
\pgfsetfillcolor{currentfill}%
\pgfsetfillopacity{0.800000}%
\pgfsetlinewidth{1.003750pt}%
\definecolor{currentstroke}{rgb}{0.800000,0.800000,0.800000}%
\pgfsetstrokecolor{currentstroke}%
\pgfsetstrokeopacity{0.800000}%
\pgfsetdash{}{0pt}%
\pgfpathmoveto{\pgfqpoint{4.349504in}{3.710883in}}%
\pgfpathlineto{\pgfqpoint{5.662778in}{3.710883in}}%
\pgfpathquadraticcurveto{\pgfqpoint{5.690556in}{3.710883in}}{\pgfqpoint{5.690556in}{3.738661in}}%
\pgfpathlineto{\pgfqpoint{5.690556in}{4.126778in}}%
\pgfpathquadraticcurveto{\pgfqpoint{5.690556in}{4.154556in}}{\pgfqpoint{5.662778in}{4.154556in}}%
\pgfpathlineto{\pgfqpoint{4.349504in}{4.154556in}}%
\pgfpathquadraticcurveto{\pgfqpoint{4.321726in}{4.154556in}}{\pgfqpoint{4.321726in}{4.126778in}}%
\pgfpathlineto{\pgfqpoint{4.321726in}{3.738661in}}%
\pgfpathquadraticcurveto{\pgfqpoint{4.321726in}{3.710883in}}{\pgfqpoint{4.349504in}{3.710883in}}%
\pgfpathclose%
\pgfusepath{stroke,fill}%
\end{pgfscope}%
\begin{pgfscope}%
\pgfsetrectcap%
\pgfsetroundjoin%
\pgfsetlinewidth{0.501875pt}%
\definecolor{currentstroke}{rgb}{1.000000,0.000000,0.000000}%
\pgfsetstrokecolor{currentstroke}%
\pgfsetdash{}{0pt}%
\pgfpathmoveto{\pgfqpoint{4.377282in}{4.043444in}}%
\pgfpathlineto{\pgfqpoint{4.655060in}{4.043444in}}%
\pgfusepath{stroke}%
\end{pgfscope}%
\begin{pgfscope}%
\definecolor{textcolor}{rgb}{0.000000,0.000000,0.000000}%
\pgfsetstrokecolor{textcolor}%
\pgfsetfillcolor{textcolor}%
\pgftext[x=4.766171in,y=3.994833in,left,base]{\color{textcolor}\rmfamily\fontsize{10.000000}{12.000000}\selectfont Best Fit [1.03]}%
\end{pgfscope}%
\begin{pgfscope}%
\pgfsetbuttcap%
\pgfsetroundjoin%
\pgfsetlinewidth{0.501875pt}%
\definecolor{currentstroke}{rgb}{0.000000,0.000000,1.000000}%
\pgfsetstrokecolor{currentstroke}%
\pgfsetdash{}{0pt}%
\pgfpathmoveto{\pgfqpoint{4.516171in}{3.772611in}}%
\pgfpathlineto{\pgfqpoint{4.516171in}{3.911500in}}%
\pgfusepath{stroke}%
\end{pgfscope}%
\begin{pgfscope}%
\pgfsetbuttcap%
\pgfsetroundjoin%
\definecolor{currentfill}{rgb}{0.000000,0.000000,1.000000}%
\pgfsetfillcolor{currentfill}%
\pgfsetlinewidth{0.501875pt}%
\definecolor{currentstroke}{rgb}{0.000000,0.000000,1.000000}%
\pgfsetstrokecolor{currentstroke}%
\pgfsetdash{}{0pt}%
\pgfsys@defobject{currentmarker}{\pgfqpoint{-0.027778in}{-0.000000in}}{\pgfqpoint{0.027778in}{0.000000in}}{%
\pgfpathmoveto{\pgfqpoint{0.027778in}{-0.000000in}}%
\pgfpathlineto{\pgfqpoint{-0.027778in}{0.000000in}}%
\pgfusepath{stroke,fill}%
}%
\begin{pgfscope}%
\pgfsys@transformshift{4.516171in}{3.772611in}%
\pgfsys@useobject{currentmarker}{}%
\end{pgfscope}%
\end{pgfscope}%
\begin{pgfscope}%
\pgfsetbuttcap%
\pgfsetroundjoin%
\definecolor{currentfill}{rgb}{0.000000,0.000000,1.000000}%
\pgfsetfillcolor{currentfill}%
\pgfsetlinewidth{0.501875pt}%
\definecolor{currentstroke}{rgb}{0.000000,0.000000,1.000000}%
\pgfsetstrokecolor{currentstroke}%
\pgfsetdash{}{0pt}%
\pgfsys@defobject{currentmarker}{\pgfqpoint{-0.027778in}{-0.000000in}}{\pgfqpoint{0.027778in}{0.000000in}}{%
\pgfpathmoveto{\pgfqpoint{0.027778in}{-0.000000in}}%
\pgfpathlineto{\pgfqpoint{-0.027778in}{0.000000in}}%
\pgfusepath{stroke,fill}%
}%
\begin{pgfscope}%
\pgfsys@transformshift{4.516171in}{3.911500in}%
\pgfsys@useobject{currentmarker}{}%
\end{pgfscope}%
\end{pgfscope}%
\begin{pgfscope}%
\pgfsetbuttcap%
\pgfsetroundjoin%
\definecolor{currentfill}{rgb}{0.000000,0.000000,1.000000}%
\pgfsetfillcolor{currentfill}%
\pgfsetlinewidth{0.501875pt}%
\definecolor{currentstroke}{rgb}{0.000000,0.000000,1.000000}%
\pgfsetstrokecolor{currentstroke}%
\pgfsetdash{}{0pt}%
\pgfsys@defobject{currentmarker}{\pgfqpoint{-0.027778in}{-0.000000in}}{\pgfqpoint{0.027778in}{0.000000in}}{%
\pgfpathmoveto{\pgfqpoint{0.027778in}{-0.000000in}}%
\pgfpathlineto{\pgfqpoint{-0.027778in}{0.000000in}}%
\pgfusepath{stroke,fill}%
}%
\begin{pgfscope}%
\pgfsys@transformshift{4.516171in}{3.842056in}%
\pgfsys@useobject{currentmarker}{}%
\end{pgfscope}%
\end{pgfscope}%
\begin{pgfscope}%
\definecolor{textcolor}{rgb}{0.000000,0.000000,0.000000}%
\pgfsetstrokecolor{textcolor}%
\pgfsetfillcolor{textcolor}%
\pgftext[x=4.766171in,y=3.793444in,left,base]{\color{textcolor}\rmfamily\fontsize{10.000000}{12.000000}\selectfont Data}%
\end{pgfscope}%
\end{pgfpicture}%
\makeatother%
\endgroup%
}
           \caption{Algorithmically Determined Best Fit}
           \label{fig:Best Fit}
        \end{center}
    \end{figure}
    
    \newpage\noindent
    As mentioned before, this line of best fit comes with a so-called "goodness" that we call 
    $\chi ^2$ where

    \begin{equation}
        \chi^2=\sum\limits_{i=0}^{n}\frac{y_i-f(t_i,A,B,\gamma,\omega,\alpha)}{u_i}
    \end{equation}

    \noindent
    This $\chi^2$, calculated on lines 52-53, when divided by \texttt{dof}, the "degrees of 
    freedom" of the data (i.e. the number of data points minus the number of fitted parameters 
    [line 54]) gives us a much more useful and convenient measure of the fit. The results of 
    these calculations are below in Table \ref{table:Goodness}.

    \begin{table}[H]
        \centering
        \begin{tabular}{c c c}
            \hline
            $\chi^2$ & \texttt{dof} & $\frac{\chi^2}{\texttt{dof}}$ \\
            \hline
            252.73 & 245 & 1.03 \\
            \hline
        \end{tabular}
        \caption{Goodness Parameters}
        \label{table:Goodness}
    \end{table}

    \noindent
    Our value for $\frac{\chi^2}{\texttt{dof}}$ is $\sim1.03$ and ideally this value is $\sim1$, so our 
    fit is fairly reasonable. 
    \newline
    The next thing to consider is the uncertainties of the actual parameters used in the final 
    fit. These can be extracted from the covariance matrix \texttt{pcov}, which comes out of 
    the calling of \texttt{curve\_fit} in line 47. We introduce a correction factor of 
    $\sqrt{\frac{\chi^2}{\texttt{dof}}}$ as $\frac{\chi^2}{\texttt{dof}}$ deviates from 1. These 
    calculations can be found in lines 62-63, resulting in Table \ref{table:Params and Uncertainties}.

    \begin{table}[H]
        \centering
        \begin{tabular}{c c c c c}
            \hline
            $A$ & $B$ & $\gamma$ & $\omega$ & $\alpha$ \\
            \hline
            0.28 & -0.028 & 0.28 & 21.46 & -2.81 \\
            \num{\pm6.4e-5} & \num{\pm2.47e-4} & \num{\pm4.61e-3} & \num{\pm4.66e-3} & \num{\pm8.87e-3} \\
            \hline
        \end{tabular}
        \caption{Parameter Values}
        \label{table:Params and Uncertainties}
    \end{table}

    \noindent
    The results of line 65, i.e. the uncertainties of the parameters without the correction 
    factor, can be seen below in Table \ref{table:Unweighted Param Uncertainties}.

    \begin{table}[H]
        \centering
        \begin{tabular}{c c c c c}
            \hline
            $u(A)$ & $u(B)$ & $u(\gamma)$ & $u(\omega)$ & $u(\alpha)$ \\
            \hline
            \num{6.33e-5} & \num{2.43e-4} & \num{4.54e-3} & \num{4.59e-3} & \num{8.73e-3} \\
            \hline
        \end{tabular}
        \caption{Uncertainties of Parameters}
        \label{table:Unweighted Param Uncertainties}
    \end{table}

    \noindent
    Finally, we consider the relationship between each parameter and their correlations as these 
    parameters do not exist isolated from everything else. They are all coupled in some way and 
    in order to see the degree to which they are correlated, we calculate the correlation matrix 
    [lines 68-75], displayed below in Table \ref{table:Correlation Matrix}. These values are in 
    the interval [-1, 1] and the matrix is symmetric, so we didn't show all of it. 

    \begin{table}[H]
        \centering
        \begin{tabular}{c c c c c c}
            \hline
             & A & B & $\gamma$ & $\omega$ & $\alpha$ \\
            \hline
            A & 1 \\
            B & 0.00039 & 1 \\
            $\gamma$ & 0.0027 & -0.77 & 1 \\
            $\omega$ & -0.043 & 0.023 & -0.015 & 1 \\
            $\alpha$ & -0.044 & 0.032 & -0.023 & 0.77 & 1 \\
        \end{tabular}
        \caption{Parameter Correlation Matrix}
        \label{table:Correlation Matrix}
    \end{table}

    \noindent
    From this table we can see that the strongest correlations are between $\gamma$ and B, and 
    between $\omega$ and $\alpha$, both of which have a correlation of $\sim0.77$. This is far greater 
    than any other relationship in the system as the rest are an order of magnitude less than 
    these two, at least. These high correlation values are significant when calculating uncertainties 
    as two highly correlated values require a more sophisticated method in order to more reliably 
    calculate their uncertainties. For now we can say that this is a fairly good fit to the data 
    and, excusing the two correlations mentioned earlier, the uncertainties in Table 
    \ref{table:Params and Uncertainties} are valid. 
    \newline
    \newline
    The second section of this activity was an introduction into a weighted linear least-squares 
    fit. The code for all of this is in Appendix 2 and 3.To begin with, we went back to a section 
    of CP1 and replotted the line of best fit for the data in LinearNoErrors.txt using the 
    \texttt{curve\_fit} function and got the result below in Figure \ref{fig:CP1c Plot} as well 
    as the tables below showing the parameters and their uncertainties using the same techniques 
    as for the first analysis. 

    \begin{figure}[H]
        \begin{center}
           \scalebox{.6}{%% Creator: Matplotlib, PGF backend
%%
%% To include the figure in your LaTeX document, write
%%   \input{<filename>.pgf}
%%
%% Make sure the required packages are loaded in your preamble
%%   \usepackage{pgf}
%%
%% Figures using additional raster images can only be included by \input if
%% they are in the same directory as the main LaTeX file. For loading figures
%% from other directories you can use the `import` package
%%   \usepackage{import}
%% and then include the figures with
%%   \import{<path to file>}{<filename>.pgf}
%%
%% Matplotlib used the following preamble
%%
\begingroup%
\makeatletter%
\begin{pgfpicture}%
\pgfpathrectangle{\pgfpointorigin}{\pgfqpoint{6.400000in}{4.800000in}}%
\pgfusepath{use as bounding box, clip}%
\begin{pgfscope}%
\pgfsetbuttcap%
\pgfsetmiterjoin%
\definecolor{currentfill}{rgb}{1.000000,1.000000,1.000000}%
\pgfsetfillcolor{currentfill}%
\pgfsetlinewidth{0.000000pt}%
\definecolor{currentstroke}{rgb}{1.000000,1.000000,1.000000}%
\pgfsetstrokecolor{currentstroke}%
\pgfsetdash{}{0pt}%
\pgfpathmoveto{\pgfqpoint{0.000000in}{0.000000in}}%
\pgfpathlineto{\pgfqpoint{6.400000in}{0.000000in}}%
\pgfpathlineto{\pgfqpoint{6.400000in}{4.800000in}}%
\pgfpathlineto{\pgfqpoint{0.000000in}{4.800000in}}%
\pgfpathclose%
\pgfusepath{fill}%
\end{pgfscope}%
\begin{pgfscope}%
\pgfsetbuttcap%
\pgfsetmiterjoin%
\definecolor{currentfill}{rgb}{1.000000,1.000000,1.000000}%
\pgfsetfillcolor{currentfill}%
\pgfsetlinewidth{0.000000pt}%
\definecolor{currentstroke}{rgb}{0.000000,0.000000,0.000000}%
\pgfsetstrokecolor{currentstroke}%
\pgfsetstrokeopacity{0.000000}%
\pgfsetdash{}{0pt}%
\pgfpathmoveto{\pgfqpoint{0.800000in}{0.528000in}}%
\pgfpathlineto{\pgfqpoint{5.760000in}{0.528000in}}%
\pgfpathlineto{\pgfqpoint{5.760000in}{4.224000in}}%
\pgfpathlineto{\pgfqpoint{0.800000in}{4.224000in}}%
\pgfpathclose%
\pgfusepath{fill}%
\end{pgfscope}%
\begin{pgfscope}%
\pgfsetbuttcap%
\pgfsetroundjoin%
\definecolor{currentfill}{rgb}{0.000000,0.000000,0.000000}%
\pgfsetfillcolor{currentfill}%
\pgfsetlinewidth{0.803000pt}%
\definecolor{currentstroke}{rgb}{0.000000,0.000000,0.000000}%
\pgfsetstrokecolor{currentstroke}%
\pgfsetdash{}{0pt}%
\pgfsys@defobject{currentmarker}{\pgfqpoint{0.000000in}{-0.048611in}}{\pgfqpoint{0.000000in}{0.000000in}}{%
\pgfpathmoveto{\pgfqpoint{0.000000in}{0.000000in}}%
\pgfpathlineto{\pgfqpoint{0.000000in}{-0.048611in}}%
\pgfusepath{stroke,fill}%
}%
\begin{pgfscope}%
\pgfsys@transformshift{0.800000in}{0.528000in}%
\pgfsys@useobject{currentmarker}{}%
\end{pgfscope}%
\end{pgfscope}%
\begin{pgfscope}%
\definecolor{textcolor}{rgb}{0.000000,0.000000,0.000000}%
\pgfsetstrokecolor{textcolor}%
\pgfsetfillcolor{textcolor}%
\pgftext[x=0.800000in,y=0.430778in,,top]{\color{textcolor}\rmfamily\fontsize{10.000000}{12.000000}\selectfont \(\displaystyle 0\)}%
\end{pgfscope}%
\begin{pgfscope}%
\pgfsetbuttcap%
\pgfsetroundjoin%
\definecolor{currentfill}{rgb}{0.000000,0.000000,0.000000}%
\pgfsetfillcolor{currentfill}%
\pgfsetlinewidth{0.803000pt}%
\definecolor{currentstroke}{rgb}{0.000000,0.000000,0.000000}%
\pgfsetstrokecolor{currentstroke}%
\pgfsetdash{}{0pt}%
\pgfsys@defobject{currentmarker}{\pgfqpoint{0.000000in}{-0.048611in}}{\pgfqpoint{0.000000in}{0.000000in}}{%
\pgfpathmoveto{\pgfqpoint{0.000000in}{0.000000in}}%
\pgfpathlineto{\pgfqpoint{0.000000in}{-0.048611in}}%
\pgfusepath{stroke,fill}%
}%
\begin{pgfscope}%
\pgfsys@transformshift{1.563077in}{0.528000in}%
\pgfsys@useobject{currentmarker}{}%
\end{pgfscope}%
\end{pgfscope}%
\begin{pgfscope}%
\definecolor{textcolor}{rgb}{0.000000,0.000000,0.000000}%
\pgfsetstrokecolor{textcolor}%
\pgfsetfillcolor{textcolor}%
\pgftext[x=1.563077in,y=0.430778in,,top]{\color{textcolor}\rmfamily\fontsize{10.000000}{12.000000}\selectfont \(\displaystyle 2\)}%
\end{pgfscope}%
\begin{pgfscope}%
\pgfsetbuttcap%
\pgfsetroundjoin%
\definecolor{currentfill}{rgb}{0.000000,0.000000,0.000000}%
\pgfsetfillcolor{currentfill}%
\pgfsetlinewidth{0.803000pt}%
\definecolor{currentstroke}{rgb}{0.000000,0.000000,0.000000}%
\pgfsetstrokecolor{currentstroke}%
\pgfsetdash{}{0pt}%
\pgfsys@defobject{currentmarker}{\pgfqpoint{0.000000in}{-0.048611in}}{\pgfqpoint{0.000000in}{0.000000in}}{%
\pgfpathmoveto{\pgfqpoint{0.000000in}{0.000000in}}%
\pgfpathlineto{\pgfqpoint{0.000000in}{-0.048611in}}%
\pgfusepath{stroke,fill}%
}%
\begin{pgfscope}%
\pgfsys@transformshift{2.326154in}{0.528000in}%
\pgfsys@useobject{currentmarker}{}%
\end{pgfscope}%
\end{pgfscope}%
\begin{pgfscope}%
\definecolor{textcolor}{rgb}{0.000000,0.000000,0.000000}%
\pgfsetstrokecolor{textcolor}%
\pgfsetfillcolor{textcolor}%
\pgftext[x=2.326154in,y=0.430778in,,top]{\color{textcolor}\rmfamily\fontsize{10.000000}{12.000000}\selectfont \(\displaystyle 4\)}%
\end{pgfscope}%
\begin{pgfscope}%
\pgfsetbuttcap%
\pgfsetroundjoin%
\definecolor{currentfill}{rgb}{0.000000,0.000000,0.000000}%
\pgfsetfillcolor{currentfill}%
\pgfsetlinewidth{0.803000pt}%
\definecolor{currentstroke}{rgb}{0.000000,0.000000,0.000000}%
\pgfsetstrokecolor{currentstroke}%
\pgfsetdash{}{0pt}%
\pgfsys@defobject{currentmarker}{\pgfqpoint{0.000000in}{-0.048611in}}{\pgfqpoint{0.000000in}{0.000000in}}{%
\pgfpathmoveto{\pgfqpoint{0.000000in}{0.000000in}}%
\pgfpathlineto{\pgfqpoint{0.000000in}{-0.048611in}}%
\pgfusepath{stroke,fill}%
}%
\begin{pgfscope}%
\pgfsys@transformshift{3.089231in}{0.528000in}%
\pgfsys@useobject{currentmarker}{}%
\end{pgfscope}%
\end{pgfscope}%
\begin{pgfscope}%
\definecolor{textcolor}{rgb}{0.000000,0.000000,0.000000}%
\pgfsetstrokecolor{textcolor}%
\pgfsetfillcolor{textcolor}%
\pgftext[x=3.089231in,y=0.430778in,,top]{\color{textcolor}\rmfamily\fontsize{10.000000}{12.000000}\selectfont \(\displaystyle 6\)}%
\end{pgfscope}%
\begin{pgfscope}%
\pgfsetbuttcap%
\pgfsetroundjoin%
\definecolor{currentfill}{rgb}{0.000000,0.000000,0.000000}%
\pgfsetfillcolor{currentfill}%
\pgfsetlinewidth{0.803000pt}%
\definecolor{currentstroke}{rgb}{0.000000,0.000000,0.000000}%
\pgfsetstrokecolor{currentstroke}%
\pgfsetdash{}{0pt}%
\pgfsys@defobject{currentmarker}{\pgfqpoint{0.000000in}{-0.048611in}}{\pgfqpoint{0.000000in}{0.000000in}}{%
\pgfpathmoveto{\pgfqpoint{0.000000in}{0.000000in}}%
\pgfpathlineto{\pgfqpoint{0.000000in}{-0.048611in}}%
\pgfusepath{stroke,fill}%
}%
\begin{pgfscope}%
\pgfsys@transformshift{3.852308in}{0.528000in}%
\pgfsys@useobject{currentmarker}{}%
\end{pgfscope}%
\end{pgfscope}%
\begin{pgfscope}%
\definecolor{textcolor}{rgb}{0.000000,0.000000,0.000000}%
\pgfsetstrokecolor{textcolor}%
\pgfsetfillcolor{textcolor}%
\pgftext[x=3.852308in,y=0.430778in,,top]{\color{textcolor}\rmfamily\fontsize{10.000000}{12.000000}\selectfont \(\displaystyle 8\)}%
\end{pgfscope}%
\begin{pgfscope}%
\pgfsetbuttcap%
\pgfsetroundjoin%
\definecolor{currentfill}{rgb}{0.000000,0.000000,0.000000}%
\pgfsetfillcolor{currentfill}%
\pgfsetlinewidth{0.803000pt}%
\definecolor{currentstroke}{rgb}{0.000000,0.000000,0.000000}%
\pgfsetstrokecolor{currentstroke}%
\pgfsetdash{}{0pt}%
\pgfsys@defobject{currentmarker}{\pgfqpoint{0.000000in}{-0.048611in}}{\pgfqpoint{0.000000in}{0.000000in}}{%
\pgfpathmoveto{\pgfqpoint{0.000000in}{0.000000in}}%
\pgfpathlineto{\pgfqpoint{0.000000in}{-0.048611in}}%
\pgfusepath{stroke,fill}%
}%
\begin{pgfscope}%
\pgfsys@transformshift{4.615385in}{0.528000in}%
\pgfsys@useobject{currentmarker}{}%
\end{pgfscope}%
\end{pgfscope}%
\begin{pgfscope}%
\definecolor{textcolor}{rgb}{0.000000,0.000000,0.000000}%
\pgfsetstrokecolor{textcolor}%
\pgfsetfillcolor{textcolor}%
\pgftext[x=4.615385in,y=0.430778in,,top]{\color{textcolor}\rmfamily\fontsize{10.000000}{12.000000}\selectfont \(\displaystyle 10\)}%
\end{pgfscope}%
\begin{pgfscope}%
\pgfsetbuttcap%
\pgfsetroundjoin%
\definecolor{currentfill}{rgb}{0.000000,0.000000,0.000000}%
\pgfsetfillcolor{currentfill}%
\pgfsetlinewidth{0.803000pt}%
\definecolor{currentstroke}{rgb}{0.000000,0.000000,0.000000}%
\pgfsetstrokecolor{currentstroke}%
\pgfsetdash{}{0pt}%
\pgfsys@defobject{currentmarker}{\pgfqpoint{0.000000in}{-0.048611in}}{\pgfqpoint{0.000000in}{0.000000in}}{%
\pgfpathmoveto{\pgfqpoint{0.000000in}{0.000000in}}%
\pgfpathlineto{\pgfqpoint{0.000000in}{-0.048611in}}%
\pgfusepath{stroke,fill}%
}%
\begin{pgfscope}%
\pgfsys@transformshift{5.378462in}{0.528000in}%
\pgfsys@useobject{currentmarker}{}%
\end{pgfscope}%
\end{pgfscope}%
\begin{pgfscope}%
\definecolor{textcolor}{rgb}{0.000000,0.000000,0.000000}%
\pgfsetstrokecolor{textcolor}%
\pgfsetfillcolor{textcolor}%
\pgftext[x=5.378462in,y=0.430778in,,top]{\color{textcolor}\rmfamily\fontsize{10.000000}{12.000000}\selectfont \(\displaystyle 12\)}%
\end{pgfscope}%
\begin{pgfscope}%
\definecolor{textcolor}{rgb}{0.000000,0.000000,0.000000}%
\pgfsetstrokecolor{textcolor}%
\pgfsetfillcolor{textcolor}%
\pgftext[x=3.280000in,y=0.251766in,,top]{\color{textcolor}\rmfamily\fontsize{10.000000}{12.000000}\selectfont x (m)}%
\end{pgfscope}%
\begin{pgfscope}%
\pgfsetbuttcap%
\pgfsetroundjoin%
\definecolor{currentfill}{rgb}{0.000000,0.000000,0.000000}%
\pgfsetfillcolor{currentfill}%
\pgfsetlinewidth{0.803000pt}%
\definecolor{currentstroke}{rgb}{0.000000,0.000000,0.000000}%
\pgfsetstrokecolor{currentstroke}%
\pgfsetdash{}{0pt}%
\pgfsys@defobject{currentmarker}{\pgfqpoint{-0.048611in}{0.000000in}}{\pgfqpoint{0.000000in}{0.000000in}}{%
\pgfpathmoveto{\pgfqpoint{0.000000in}{0.000000in}}%
\pgfpathlineto{\pgfqpoint{-0.048611in}{0.000000in}}%
\pgfusepath{stroke,fill}%
}%
\begin{pgfscope}%
\pgfsys@transformshift{0.800000in}{0.859843in}%
\pgfsys@useobject{currentmarker}{}%
\end{pgfscope}%
\end{pgfscope}%
\begin{pgfscope}%
\definecolor{textcolor}{rgb}{0.000000,0.000000,0.000000}%
\pgfsetstrokecolor{textcolor}%
\pgfsetfillcolor{textcolor}%
\pgftext[x=0.633333in,y=0.811618in,left,base]{\color{textcolor}\rmfamily\fontsize{10.000000}{12.000000}\selectfont \(\displaystyle 2\)}%
\end{pgfscope}%
\begin{pgfscope}%
\pgfsetbuttcap%
\pgfsetroundjoin%
\definecolor{currentfill}{rgb}{0.000000,0.000000,0.000000}%
\pgfsetfillcolor{currentfill}%
\pgfsetlinewidth{0.803000pt}%
\definecolor{currentstroke}{rgb}{0.000000,0.000000,0.000000}%
\pgfsetstrokecolor{currentstroke}%
\pgfsetdash{}{0pt}%
\pgfsys@defobject{currentmarker}{\pgfqpoint{-0.048611in}{0.000000in}}{\pgfqpoint{0.000000in}{0.000000in}}{%
\pgfpathmoveto{\pgfqpoint{0.000000in}{0.000000in}}%
\pgfpathlineto{\pgfqpoint{-0.048611in}{0.000000in}}%
\pgfusepath{stroke,fill}%
}%
\begin{pgfscope}%
\pgfsys@transformshift{0.800000in}{1.343377in}%
\pgfsys@useobject{currentmarker}{}%
\end{pgfscope}%
\end{pgfscope}%
\begin{pgfscope}%
\definecolor{textcolor}{rgb}{0.000000,0.000000,0.000000}%
\pgfsetstrokecolor{textcolor}%
\pgfsetfillcolor{textcolor}%
\pgftext[x=0.633333in,y=1.295152in,left,base]{\color{textcolor}\rmfamily\fontsize{10.000000}{12.000000}\selectfont \(\displaystyle 3\)}%
\end{pgfscope}%
\begin{pgfscope}%
\pgfsetbuttcap%
\pgfsetroundjoin%
\definecolor{currentfill}{rgb}{0.000000,0.000000,0.000000}%
\pgfsetfillcolor{currentfill}%
\pgfsetlinewidth{0.803000pt}%
\definecolor{currentstroke}{rgb}{0.000000,0.000000,0.000000}%
\pgfsetstrokecolor{currentstroke}%
\pgfsetdash{}{0pt}%
\pgfsys@defobject{currentmarker}{\pgfqpoint{-0.048611in}{0.000000in}}{\pgfqpoint{0.000000in}{0.000000in}}{%
\pgfpathmoveto{\pgfqpoint{0.000000in}{0.000000in}}%
\pgfpathlineto{\pgfqpoint{-0.048611in}{0.000000in}}%
\pgfusepath{stroke,fill}%
}%
\begin{pgfscope}%
\pgfsys@transformshift{0.800000in}{1.826910in}%
\pgfsys@useobject{currentmarker}{}%
\end{pgfscope}%
\end{pgfscope}%
\begin{pgfscope}%
\definecolor{textcolor}{rgb}{0.000000,0.000000,0.000000}%
\pgfsetstrokecolor{textcolor}%
\pgfsetfillcolor{textcolor}%
\pgftext[x=0.633333in,y=1.778685in,left,base]{\color{textcolor}\rmfamily\fontsize{10.000000}{12.000000}\selectfont \(\displaystyle 4\)}%
\end{pgfscope}%
\begin{pgfscope}%
\pgfsetbuttcap%
\pgfsetroundjoin%
\definecolor{currentfill}{rgb}{0.000000,0.000000,0.000000}%
\pgfsetfillcolor{currentfill}%
\pgfsetlinewidth{0.803000pt}%
\definecolor{currentstroke}{rgb}{0.000000,0.000000,0.000000}%
\pgfsetstrokecolor{currentstroke}%
\pgfsetdash{}{0pt}%
\pgfsys@defobject{currentmarker}{\pgfqpoint{-0.048611in}{0.000000in}}{\pgfqpoint{0.000000in}{0.000000in}}{%
\pgfpathmoveto{\pgfqpoint{0.000000in}{0.000000in}}%
\pgfpathlineto{\pgfqpoint{-0.048611in}{0.000000in}}%
\pgfusepath{stroke,fill}%
}%
\begin{pgfscope}%
\pgfsys@transformshift{0.800000in}{2.310444in}%
\pgfsys@useobject{currentmarker}{}%
\end{pgfscope}%
\end{pgfscope}%
\begin{pgfscope}%
\definecolor{textcolor}{rgb}{0.000000,0.000000,0.000000}%
\pgfsetstrokecolor{textcolor}%
\pgfsetfillcolor{textcolor}%
\pgftext[x=0.633333in,y=2.262219in,left,base]{\color{textcolor}\rmfamily\fontsize{10.000000}{12.000000}\selectfont \(\displaystyle 5\)}%
\end{pgfscope}%
\begin{pgfscope}%
\pgfsetbuttcap%
\pgfsetroundjoin%
\definecolor{currentfill}{rgb}{0.000000,0.000000,0.000000}%
\pgfsetfillcolor{currentfill}%
\pgfsetlinewidth{0.803000pt}%
\definecolor{currentstroke}{rgb}{0.000000,0.000000,0.000000}%
\pgfsetstrokecolor{currentstroke}%
\pgfsetdash{}{0pt}%
\pgfsys@defobject{currentmarker}{\pgfqpoint{-0.048611in}{0.000000in}}{\pgfqpoint{0.000000in}{0.000000in}}{%
\pgfpathmoveto{\pgfqpoint{0.000000in}{0.000000in}}%
\pgfpathlineto{\pgfqpoint{-0.048611in}{0.000000in}}%
\pgfusepath{stroke,fill}%
}%
\begin{pgfscope}%
\pgfsys@transformshift{0.800000in}{2.793978in}%
\pgfsys@useobject{currentmarker}{}%
\end{pgfscope}%
\end{pgfscope}%
\begin{pgfscope}%
\definecolor{textcolor}{rgb}{0.000000,0.000000,0.000000}%
\pgfsetstrokecolor{textcolor}%
\pgfsetfillcolor{textcolor}%
\pgftext[x=0.633333in,y=2.745752in,left,base]{\color{textcolor}\rmfamily\fontsize{10.000000}{12.000000}\selectfont \(\displaystyle 6\)}%
\end{pgfscope}%
\begin{pgfscope}%
\pgfsetbuttcap%
\pgfsetroundjoin%
\definecolor{currentfill}{rgb}{0.000000,0.000000,0.000000}%
\pgfsetfillcolor{currentfill}%
\pgfsetlinewidth{0.803000pt}%
\definecolor{currentstroke}{rgb}{0.000000,0.000000,0.000000}%
\pgfsetstrokecolor{currentstroke}%
\pgfsetdash{}{0pt}%
\pgfsys@defobject{currentmarker}{\pgfqpoint{-0.048611in}{0.000000in}}{\pgfqpoint{0.000000in}{0.000000in}}{%
\pgfpathmoveto{\pgfqpoint{0.000000in}{0.000000in}}%
\pgfpathlineto{\pgfqpoint{-0.048611in}{0.000000in}}%
\pgfusepath{stroke,fill}%
}%
\begin{pgfscope}%
\pgfsys@transformshift{0.800000in}{3.277511in}%
\pgfsys@useobject{currentmarker}{}%
\end{pgfscope}%
\end{pgfscope}%
\begin{pgfscope}%
\definecolor{textcolor}{rgb}{0.000000,0.000000,0.000000}%
\pgfsetstrokecolor{textcolor}%
\pgfsetfillcolor{textcolor}%
\pgftext[x=0.633333in,y=3.229286in,left,base]{\color{textcolor}\rmfamily\fontsize{10.000000}{12.000000}\selectfont \(\displaystyle 7\)}%
\end{pgfscope}%
\begin{pgfscope}%
\pgfsetbuttcap%
\pgfsetroundjoin%
\definecolor{currentfill}{rgb}{0.000000,0.000000,0.000000}%
\pgfsetfillcolor{currentfill}%
\pgfsetlinewidth{0.803000pt}%
\definecolor{currentstroke}{rgb}{0.000000,0.000000,0.000000}%
\pgfsetstrokecolor{currentstroke}%
\pgfsetdash{}{0pt}%
\pgfsys@defobject{currentmarker}{\pgfqpoint{-0.048611in}{0.000000in}}{\pgfqpoint{0.000000in}{0.000000in}}{%
\pgfpathmoveto{\pgfqpoint{0.000000in}{0.000000in}}%
\pgfpathlineto{\pgfqpoint{-0.048611in}{0.000000in}}%
\pgfusepath{stroke,fill}%
}%
\begin{pgfscope}%
\pgfsys@transformshift{0.800000in}{3.761045in}%
\pgfsys@useobject{currentmarker}{}%
\end{pgfscope}%
\end{pgfscope}%
\begin{pgfscope}%
\definecolor{textcolor}{rgb}{0.000000,0.000000,0.000000}%
\pgfsetstrokecolor{textcolor}%
\pgfsetfillcolor{textcolor}%
\pgftext[x=0.633333in,y=3.712819in,left,base]{\color{textcolor}\rmfamily\fontsize{10.000000}{12.000000}\selectfont \(\displaystyle 8\)}%
\end{pgfscope}%
\begin{pgfscope}%
\definecolor{textcolor}{rgb}{0.000000,0.000000,0.000000}%
\pgfsetstrokecolor{textcolor}%
\pgfsetfillcolor{textcolor}%
\pgftext[x=0.577777in,y=2.376000in,,bottom,rotate=90.000000]{\color{textcolor}\rmfamily\fontsize{10.000000}{12.000000}\selectfont y (m)}%
\end{pgfscope}%
\begin{pgfscope}%
\pgfpathrectangle{\pgfqpoint{0.800000in}{0.528000in}}{\pgfqpoint{4.960000in}{3.696000in}}%
\pgfusepath{clip}%
\pgfsetrectcap%
\pgfsetroundjoin%
\pgfsetlinewidth{1.003750pt}%
\definecolor{currentstroke}{rgb}{0.000000,0.000000,1.000000}%
\pgfsetstrokecolor{currentstroke}%
\pgfsetdash{}{0pt}%
\pgfpathmoveto{\pgfqpoint{1.181538in}{0.696000in}}%
\pgfpathlineto{\pgfqpoint{1.219692in}{0.724540in}}%
\pgfpathlineto{\pgfqpoint{1.257846in}{0.753081in}}%
\pgfpathlineto{\pgfqpoint{1.296000in}{0.781621in}}%
\pgfpathlineto{\pgfqpoint{1.334154in}{0.810161in}}%
\pgfpathlineto{\pgfqpoint{1.372308in}{0.838702in}}%
\pgfpathlineto{\pgfqpoint{1.410462in}{0.867242in}}%
\pgfpathlineto{\pgfqpoint{1.448615in}{0.895782in}}%
\pgfpathlineto{\pgfqpoint{1.486769in}{0.924322in}}%
\pgfpathlineto{\pgfqpoint{1.524923in}{0.952863in}}%
\pgfpathlineto{\pgfqpoint{1.563077in}{0.981403in}}%
\pgfpathlineto{\pgfqpoint{1.601231in}{1.009943in}}%
\pgfpathlineto{\pgfqpoint{1.639385in}{1.038484in}}%
\pgfpathlineto{\pgfqpoint{1.677538in}{1.067024in}}%
\pgfpathlineto{\pgfqpoint{1.715692in}{1.095564in}}%
\pgfpathlineto{\pgfqpoint{1.753846in}{1.124105in}}%
\pgfpathlineto{\pgfqpoint{1.792000in}{1.152645in}}%
\pgfpathlineto{\pgfqpoint{1.830154in}{1.181185in}}%
\pgfpathlineto{\pgfqpoint{1.868308in}{1.209726in}}%
\pgfpathlineto{\pgfqpoint{1.906462in}{1.238266in}}%
\pgfpathlineto{\pgfqpoint{1.944615in}{1.266806in}}%
\pgfpathlineto{\pgfqpoint{1.982769in}{1.295347in}}%
\pgfpathlineto{\pgfqpoint{2.020923in}{1.323887in}}%
\pgfpathlineto{\pgfqpoint{2.059077in}{1.352427in}}%
\pgfpathlineto{\pgfqpoint{2.097231in}{1.380967in}}%
\pgfpathlineto{\pgfqpoint{2.135385in}{1.409508in}}%
\pgfpathlineto{\pgfqpoint{2.173538in}{1.438048in}}%
\pgfpathlineto{\pgfqpoint{2.211692in}{1.466588in}}%
\pgfpathlineto{\pgfqpoint{2.249846in}{1.495129in}}%
\pgfpathlineto{\pgfqpoint{2.288000in}{1.523669in}}%
\pgfpathlineto{\pgfqpoint{2.326154in}{1.552209in}}%
\pgfpathlineto{\pgfqpoint{2.364308in}{1.580750in}}%
\pgfpathlineto{\pgfqpoint{2.402462in}{1.609290in}}%
\pgfpathlineto{\pgfqpoint{2.440615in}{1.637830in}}%
\pgfpathlineto{\pgfqpoint{2.478769in}{1.666371in}}%
\pgfpathlineto{\pgfqpoint{2.516923in}{1.694911in}}%
\pgfpathlineto{\pgfqpoint{2.555077in}{1.723451in}}%
\pgfpathlineto{\pgfqpoint{2.593231in}{1.751992in}}%
\pgfpathlineto{\pgfqpoint{2.631385in}{1.780532in}}%
\pgfpathlineto{\pgfqpoint{2.669538in}{1.809072in}}%
\pgfpathlineto{\pgfqpoint{2.707692in}{1.837612in}}%
\pgfpathlineto{\pgfqpoint{2.745846in}{1.866153in}}%
\pgfpathlineto{\pgfqpoint{2.784000in}{1.894693in}}%
\pgfpathlineto{\pgfqpoint{2.822154in}{1.923233in}}%
\pgfpathlineto{\pgfqpoint{2.860308in}{1.951774in}}%
\pgfpathlineto{\pgfqpoint{2.898462in}{1.980314in}}%
\pgfpathlineto{\pgfqpoint{2.936615in}{2.008854in}}%
\pgfpathlineto{\pgfqpoint{2.974769in}{2.037395in}}%
\pgfpathlineto{\pgfqpoint{3.012923in}{2.065935in}}%
\pgfpathlineto{\pgfqpoint{3.051077in}{2.094475in}}%
\pgfpathlineto{\pgfqpoint{3.089231in}{2.123016in}}%
\pgfpathlineto{\pgfqpoint{3.127385in}{2.151556in}}%
\pgfpathlineto{\pgfqpoint{3.165538in}{2.180096in}}%
\pgfpathlineto{\pgfqpoint{3.203692in}{2.208637in}}%
\pgfpathlineto{\pgfqpoint{3.241846in}{2.237177in}}%
\pgfpathlineto{\pgfqpoint{3.280000in}{2.265717in}}%
\pgfpathlineto{\pgfqpoint{3.318154in}{2.294257in}}%
\pgfpathlineto{\pgfqpoint{3.356308in}{2.322798in}}%
\pgfpathlineto{\pgfqpoint{3.394462in}{2.351338in}}%
\pgfpathlineto{\pgfqpoint{3.432615in}{2.379878in}}%
\pgfpathlineto{\pgfqpoint{3.470769in}{2.408419in}}%
\pgfpathlineto{\pgfqpoint{3.508923in}{2.436959in}}%
\pgfpathlineto{\pgfqpoint{3.547077in}{2.465499in}}%
\pgfpathlineto{\pgfqpoint{3.585231in}{2.494040in}}%
\pgfpathlineto{\pgfqpoint{3.623385in}{2.522580in}}%
\pgfpathlineto{\pgfqpoint{3.661538in}{2.551120in}}%
\pgfpathlineto{\pgfqpoint{3.699692in}{2.579661in}}%
\pgfpathlineto{\pgfqpoint{3.737846in}{2.608201in}}%
\pgfpathlineto{\pgfqpoint{3.776000in}{2.636741in}}%
\pgfpathlineto{\pgfqpoint{3.814154in}{2.665282in}}%
\pgfpathlineto{\pgfqpoint{3.852308in}{2.693822in}}%
\pgfpathlineto{\pgfqpoint{3.890462in}{2.722362in}}%
\pgfpathlineto{\pgfqpoint{3.928615in}{2.750902in}}%
\pgfpathlineto{\pgfqpoint{3.966769in}{2.779443in}}%
\pgfpathlineto{\pgfqpoint{4.004923in}{2.807983in}}%
\pgfpathlineto{\pgfqpoint{4.043077in}{2.836523in}}%
\pgfpathlineto{\pgfqpoint{4.081231in}{2.865064in}}%
\pgfpathlineto{\pgfqpoint{4.119385in}{2.893604in}}%
\pgfpathlineto{\pgfqpoint{4.157538in}{2.922144in}}%
\pgfpathlineto{\pgfqpoint{4.195692in}{2.950685in}}%
\pgfpathlineto{\pgfqpoint{4.233846in}{2.979225in}}%
\pgfpathlineto{\pgfqpoint{4.272000in}{3.007765in}}%
\pgfpathlineto{\pgfqpoint{4.310154in}{3.036306in}}%
\pgfpathlineto{\pgfqpoint{4.348308in}{3.064846in}}%
\pgfpathlineto{\pgfqpoint{4.386462in}{3.093386in}}%
\pgfpathlineto{\pgfqpoint{4.424615in}{3.121927in}}%
\pgfpathlineto{\pgfqpoint{4.462769in}{3.150467in}}%
\pgfpathlineto{\pgfqpoint{4.500923in}{3.179007in}}%
\pgfpathlineto{\pgfqpoint{4.539077in}{3.207547in}}%
\pgfpathlineto{\pgfqpoint{4.577231in}{3.236088in}}%
\pgfpathlineto{\pgfqpoint{4.615385in}{3.264628in}}%
\pgfpathlineto{\pgfqpoint{4.653538in}{3.293168in}}%
\pgfpathlineto{\pgfqpoint{4.691692in}{3.321709in}}%
\pgfpathlineto{\pgfqpoint{4.729846in}{3.350249in}}%
\pgfpathlineto{\pgfqpoint{4.768000in}{3.378789in}}%
\pgfpathlineto{\pgfqpoint{4.806154in}{3.407330in}}%
\pgfpathlineto{\pgfqpoint{4.844308in}{3.435870in}}%
\pgfpathlineto{\pgfqpoint{4.882462in}{3.464410in}}%
\pgfpathlineto{\pgfqpoint{4.920615in}{3.492951in}}%
\pgfpathlineto{\pgfqpoint{4.958769in}{3.521491in}}%
\pgfpathlineto{\pgfqpoint{4.996923in}{3.550031in}}%
\pgfpathlineto{\pgfqpoint{5.035077in}{3.578572in}}%
\pgfpathlineto{\pgfqpoint{5.073231in}{3.607112in}}%
\pgfpathlineto{\pgfqpoint{5.111385in}{3.635652in}}%
\pgfpathlineto{\pgfqpoint{5.149538in}{3.664192in}}%
\pgfpathlineto{\pgfqpoint{5.187692in}{3.692733in}}%
\pgfpathlineto{\pgfqpoint{5.225846in}{3.721273in}}%
\pgfpathlineto{\pgfqpoint{5.264000in}{3.749813in}}%
\pgfpathlineto{\pgfqpoint{5.302154in}{3.778354in}}%
\pgfpathlineto{\pgfqpoint{5.340308in}{3.806894in}}%
\pgfpathlineto{\pgfqpoint{5.378462in}{3.835434in}}%
\pgfpathlineto{\pgfqpoint{5.416615in}{3.863975in}}%
\pgfpathlineto{\pgfqpoint{5.454769in}{3.892515in}}%
\pgfpathlineto{\pgfqpoint{5.492923in}{3.921055in}}%
\pgfpathlineto{\pgfqpoint{5.531077in}{3.949596in}}%
\pgfusepath{stroke}%
\end{pgfscope}%
\begin{pgfscope}%
\pgfpathrectangle{\pgfqpoint{0.800000in}{0.528000in}}{\pgfqpoint{4.960000in}{3.696000in}}%
\pgfusepath{clip}%
\pgfsetrectcap%
\pgfsetroundjoin%
\pgfsetlinewidth{0.501875pt}%
\definecolor{currentstroke}{rgb}{1.000000,0.000000,0.000000}%
\pgfsetstrokecolor{currentstroke}%
\pgfsetdash{}{0pt}%
\pgfpathmoveto{\pgfqpoint{1.181538in}{0.696000in}}%
\pgfpathlineto{\pgfqpoint{5.378462in}{3.835434in}}%
\pgfpathlineto{\pgfqpoint{5.378462in}{3.835434in}}%
\pgfusepath{stroke}%
\end{pgfscope}%
\begin{pgfscope}%
\pgfpathrectangle{\pgfqpoint{0.800000in}{0.528000in}}{\pgfqpoint{4.960000in}{3.696000in}}%
\pgfusepath{clip}%
\pgfsetbuttcap%
\pgfsetroundjoin%
\definecolor{currentfill}{rgb}{0.000000,0.000000,1.000000}%
\pgfsetfillcolor{currentfill}%
\pgfsetlinewidth{0.501875pt}%
\definecolor{currentstroke}{rgb}{0.000000,0.000000,1.000000}%
\pgfsetstrokecolor{currentstroke}%
\pgfsetdash{}{0pt}%
\pgfsys@defobject{currentmarker}{\pgfqpoint{-0.027778in}{-0.027778in}}{\pgfqpoint{0.027778in}{0.027778in}}{%
\pgfpathmoveto{\pgfqpoint{0.000000in}{-0.027778in}}%
\pgfpathcurveto{\pgfqpoint{0.007367in}{-0.027778in}}{\pgfqpoint{0.014433in}{-0.024851in}}{\pgfqpoint{0.019642in}{-0.019642in}}%
\pgfpathcurveto{\pgfqpoint{0.024851in}{-0.014433in}}{\pgfqpoint{0.027778in}{-0.007367in}}{\pgfqpoint{0.027778in}{0.000000in}}%
\pgfpathcurveto{\pgfqpoint{0.027778in}{0.007367in}}{\pgfqpoint{0.024851in}{0.014433in}}{\pgfqpoint{0.019642in}{0.019642in}}%
\pgfpathcurveto{\pgfqpoint{0.014433in}{0.024851in}}{\pgfqpoint{0.007367in}{0.027778in}}{\pgfqpoint{0.000000in}{0.027778in}}%
\pgfpathcurveto{\pgfqpoint{-0.007367in}{0.027778in}}{\pgfqpoint{-0.014433in}{0.024851in}}{\pgfqpoint{-0.019642in}{0.019642in}}%
\pgfpathcurveto{\pgfqpoint{-0.024851in}{0.014433in}}{\pgfqpoint{-0.027778in}{0.007367in}}{\pgfqpoint{-0.027778in}{0.000000in}}%
\pgfpathcurveto{\pgfqpoint{-0.027778in}{-0.007367in}}{\pgfqpoint{-0.024851in}{-0.014433in}}{\pgfqpoint{-0.019642in}{-0.019642in}}%
\pgfpathcurveto{\pgfqpoint{-0.014433in}{-0.024851in}}{\pgfqpoint{-0.007367in}{-0.027778in}}{\pgfqpoint{0.000000in}{-0.027778in}}%
\pgfpathclose%
\pgfusepath{stroke,fill}%
}%
\begin{pgfscope}%
\pgfsys@transformshift{1.181538in}{0.859843in}%
\pgfsys@useobject{currentmarker}{}%
\end{pgfscope}%
\begin{pgfscope}%
\pgfsys@transformshift{1.563077in}{0.961386in}%
\pgfsys@useobject{currentmarker}{}%
\end{pgfscope}%
\begin{pgfscope}%
\pgfsys@transformshift{1.944615in}{1.203152in}%
\pgfsys@useobject{currentmarker}{}%
\end{pgfscope}%
\begin{pgfscope}%
\pgfsys@transformshift{2.326154in}{1.691521in}%
\pgfsys@useobject{currentmarker}{}%
\end{pgfscope}%
\begin{pgfscope}%
\pgfsys@transformshift{2.707692in}{1.701192in}%
\pgfsys@useobject{currentmarker}{}%
\end{pgfscope}%
\begin{pgfscope}%
\pgfsys@transformshift{3.089231in}{2.054171in}%
\pgfsys@useobject{currentmarker}{}%
\end{pgfscope}%
\begin{pgfscope}%
\pgfsys@transformshift{3.470769in}{2.228243in}%
\pgfsys@useobject{currentmarker}{}%
\end{pgfscope}%
\begin{pgfscope}%
\pgfsys@transformshift{3.852308in}{2.895520in}%
\pgfsys@useobject{currentmarker}{}%
\end{pgfscope}%
\begin{pgfscope}%
\pgfsys@transformshift{4.233846in}{2.793978in}%
\pgfsys@useobject{currentmarker}{}%
\end{pgfscope}%
\begin{pgfscope}%
\pgfsys@transformshift{4.615385in}{3.209816in}%
\pgfsys@useobject{currentmarker}{}%
\end{pgfscope}%
\begin{pgfscope}%
\pgfsys@transformshift{4.996923in}{3.533784in}%
\pgfsys@useobject{currentmarker}{}%
\end{pgfscope}%
\begin{pgfscope}%
\pgfsys@transformshift{5.378462in}{4.056000in}%
\pgfsys@useobject{currentmarker}{}%
\end{pgfscope}%
\end{pgfscope}%
\begin{pgfscope}%
\pgfsetrectcap%
\pgfsetmiterjoin%
\pgfsetlinewidth{0.803000pt}%
\definecolor{currentstroke}{rgb}{0.000000,0.000000,0.000000}%
\pgfsetstrokecolor{currentstroke}%
\pgfsetdash{}{0pt}%
\pgfpathmoveto{\pgfqpoint{0.800000in}{0.528000in}}%
\pgfpathlineto{\pgfqpoint{0.800000in}{4.224000in}}%
\pgfusepath{stroke}%
\end{pgfscope}%
\begin{pgfscope}%
\pgfsetrectcap%
\pgfsetmiterjoin%
\pgfsetlinewidth{0.803000pt}%
\definecolor{currentstroke}{rgb}{0.000000,0.000000,0.000000}%
\pgfsetstrokecolor{currentstroke}%
\pgfsetdash{}{0pt}%
\pgfpathmoveto{\pgfqpoint{5.760000in}{0.528000in}}%
\pgfpathlineto{\pgfqpoint{5.760000in}{4.224000in}}%
\pgfusepath{stroke}%
\end{pgfscope}%
\begin{pgfscope}%
\pgfsetrectcap%
\pgfsetmiterjoin%
\pgfsetlinewidth{0.803000pt}%
\definecolor{currentstroke}{rgb}{0.000000,0.000000,0.000000}%
\pgfsetstrokecolor{currentstroke}%
\pgfsetdash{}{0pt}%
\pgfpathmoveto{\pgfqpoint{0.800000in}{0.528000in}}%
\pgfpathlineto{\pgfqpoint{5.760000in}{0.528000in}}%
\pgfusepath{stroke}%
\end{pgfscope}%
\begin{pgfscope}%
\pgfsetrectcap%
\pgfsetmiterjoin%
\pgfsetlinewidth{0.803000pt}%
\definecolor{currentstroke}{rgb}{0.000000,0.000000,0.000000}%
\pgfsetstrokecolor{currentstroke}%
\pgfsetdash{}{0pt}%
\pgfpathmoveto{\pgfqpoint{0.800000in}{4.224000in}}%
\pgfpathlineto{\pgfqpoint{5.760000in}{4.224000in}}%
\pgfusepath{stroke}%
\end{pgfscope}%
\begin{pgfscope}%
\definecolor{textcolor}{rgb}{0.000000,0.000000,0.000000}%
\pgfsetstrokecolor{textcolor}%
\pgfsetfillcolor{textcolor}%
\pgftext[x=3.280000in,y=4.307333in,,base]{\color{textcolor}\rmfamily\fontsize{12.000000}{14.400000}\selectfont Comparison of Experimental Data with Theoretical Prediction}%
\end{pgfscope}%
\begin{pgfscope}%
\pgfsetbuttcap%
\pgfsetmiterjoin%
\definecolor{currentfill}{rgb}{1.000000,1.000000,1.000000}%
\pgfsetfillcolor{currentfill}%
\pgfsetfillopacity{0.800000}%
\pgfsetlinewidth{1.003750pt}%
\definecolor{currentstroke}{rgb}{0.800000,0.800000,0.800000}%
\pgfsetstrokecolor{currentstroke}%
\pgfsetstrokeopacity{0.800000}%
\pgfsetdash{}{0pt}%
\pgfpathmoveto{\pgfqpoint{0.897222in}{3.531871in}}%
\pgfpathlineto{\pgfqpoint{2.388351in}{3.531871in}}%
\pgfpathquadraticcurveto{\pgfqpoint{2.416128in}{3.531871in}}{\pgfqpoint{2.416128in}{3.559648in}}%
\pgfpathlineto{\pgfqpoint{2.416128in}{4.126778in}}%
\pgfpathquadraticcurveto{\pgfqpoint{2.416128in}{4.154556in}}{\pgfqpoint{2.388351in}{4.154556in}}%
\pgfpathlineto{\pgfqpoint{0.897222in}{4.154556in}}%
\pgfpathquadraticcurveto{\pgfqpoint{0.869444in}{4.154556in}}{\pgfqpoint{0.869444in}{4.126778in}}%
\pgfpathlineto{\pgfqpoint{0.869444in}{3.559648in}}%
\pgfpathquadraticcurveto{\pgfqpoint{0.869444in}{3.531871in}}{\pgfqpoint{0.897222in}{3.531871in}}%
\pgfpathclose%
\pgfusepath{stroke,fill}%
\end{pgfscope}%
\begin{pgfscope}%
\pgfsetrectcap%
\pgfsetroundjoin%
\pgfsetlinewidth{1.003750pt}%
\definecolor{currentstroke}{rgb}{0.000000,0.000000,1.000000}%
\pgfsetstrokecolor{currentstroke}%
\pgfsetdash{}{0pt}%
\pgfpathmoveto{\pgfqpoint{0.925000in}{4.050389in}}%
\pgfpathlineto{\pgfqpoint{1.202778in}{4.050389in}}%
\pgfusepath{stroke}%
\end{pgfscope}%
\begin{pgfscope}%
\definecolor{textcolor}{rgb}{0.000000,0.000000,0.000000}%
\pgfsetstrokecolor{textcolor}%
\pgfsetfillcolor{textcolor}%
\pgftext[x=1.313889in,y=4.001778in,left,base]{\color{textcolor}\rmfamily\fontsize{10.000000}{12.000000}\selectfont Best Fit}%
\end{pgfscope}%
\begin{pgfscope}%
\pgfsetrectcap%
\pgfsetroundjoin%
\pgfsetlinewidth{0.501875pt}%
\definecolor{currentstroke}{rgb}{1.000000,0.000000,0.000000}%
\pgfsetstrokecolor{currentstroke}%
\pgfsetdash{}{0pt}%
\pgfpathmoveto{\pgfqpoint{0.925000in}{3.856716in}}%
\pgfpathlineto{\pgfqpoint{1.202778in}{3.856716in}}%
\pgfusepath{stroke}%
\end{pgfscope}%
\begin{pgfscope}%
\definecolor{textcolor}{rgb}{0.000000,0.000000,0.000000}%
\pgfsetstrokecolor{textcolor}%
\pgfsetfillcolor{textcolor}%
\pgftext[x=1.313889in,y=3.808105in,left,base]{\color{textcolor}\rmfamily\fontsize{10.000000}{12.000000}\selectfont curve fit Best Fit}%
\end{pgfscope}%
\begin{pgfscope}%
\pgfsetbuttcap%
\pgfsetroundjoin%
\definecolor{currentfill}{rgb}{0.000000,0.000000,1.000000}%
\pgfsetfillcolor{currentfill}%
\pgfsetlinewidth{0.501875pt}%
\definecolor{currentstroke}{rgb}{0.000000,0.000000,1.000000}%
\pgfsetstrokecolor{currentstroke}%
\pgfsetdash{}{0pt}%
\pgfsys@defobject{currentmarker}{\pgfqpoint{-0.027778in}{-0.027778in}}{\pgfqpoint{0.027778in}{0.027778in}}{%
\pgfpathmoveto{\pgfqpoint{0.000000in}{-0.027778in}}%
\pgfpathcurveto{\pgfqpoint{0.007367in}{-0.027778in}}{\pgfqpoint{0.014433in}{-0.024851in}}{\pgfqpoint{0.019642in}{-0.019642in}}%
\pgfpathcurveto{\pgfqpoint{0.024851in}{-0.014433in}}{\pgfqpoint{0.027778in}{-0.007367in}}{\pgfqpoint{0.027778in}{0.000000in}}%
\pgfpathcurveto{\pgfqpoint{0.027778in}{0.007367in}}{\pgfqpoint{0.024851in}{0.014433in}}{\pgfqpoint{0.019642in}{0.019642in}}%
\pgfpathcurveto{\pgfqpoint{0.014433in}{0.024851in}}{\pgfqpoint{0.007367in}{0.027778in}}{\pgfqpoint{0.000000in}{0.027778in}}%
\pgfpathcurveto{\pgfqpoint{-0.007367in}{0.027778in}}{\pgfqpoint{-0.014433in}{0.024851in}}{\pgfqpoint{-0.019642in}{0.019642in}}%
\pgfpathcurveto{\pgfqpoint{-0.024851in}{0.014433in}}{\pgfqpoint{-0.027778in}{0.007367in}}{\pgfqpoint{-0.027778in}{0.000000in}}%
\pgfpathcurveto{\pgfqpoint{-0.027778in}{-0.007367in}}{\pgfqpoint{-0.024851in}{-0.014433in}}{\pgfqpoint{-0.019642in}{-0.019642in}}%
\pgfpathcurveto{\pgfqpoint{-0.014433in}{-0.024851in}}{\pgfqpoint{-0.007367in}{-0.027778in}}{\pgfqpoint{0.000000in}{-0.027778in}}%
\pgfpathclose%
\pgfusepath{stroke,fill}%
}%
\begin{pgfscope}%
\pgfsys@transformshift{1.063889in}{3.663043in}%
\pgfsys@useobject{currentmarker}{}%
\end{pgfscope}%
\end{pgfscope}%
\begin{pgfscope}%
\definecolor{textcolor}{rgb}{0.000000,0.000000,0.000000}%
\pgfsetstrokecolor{textcolor}%
\pgfsetfillcolor{textcolor}%
\pgftext[x=1.313889in,y=3.614432in,left,base]{\color{textcolor}\rmfamily\fontsize{10.000000}{12.000000}\selectfont Data}%
\end{pgfscope}%
\end{pgfpicture}%
\makeatother%
\endgroup%
}
           \caption{Unweighted Linear Least-Squares Fit for LinearNoErrors.txt}
           \label{fig:CP1c Plot}
        \end{center}
    \end{figure}

    \begin{table}[H]
        \begin{minipage}{0.5\textwidth}
            \centering
            \begin{tabular}{c c c}
                \hline
                $\chi^2$ & \texttt{dof} & $\frac{\chi^2}{\texttt{dof}}$ \\
                \hline
                0.99 & 10 & 0.099 \\
                \hline
            \end{tabular}
            \caption{CP1c Goodness Parameters}
            \label{table:CP1c Goodness}
        \end{minipage}
        \begin{minipage}{0.5\textwidth}
            \centering
            \begin{tabular}{c c}
                \hline
                m & c\\
                \hline
                0.59 & 1.07 \\
                $\pm0.026$ & $\pm0.19$ \\
                \hline
            \end{tabular}
            \caption{CP1c Parameter Values}
            \label{table:CP1c Params and Uncertainties}
        \end{minipage}
    \end{table}
        
    \begin{table}[H]
        \begin{minipage}{0.5\textwidth}
            \centering
            \begin{tabular}{c c}
                \hline
                $u(m)$ & $u(c)$ \\
                \hline
                0.084 & 0.62 \\
                \hline
            \end{tabular}
            \caption{CP1c Unweighted Uncertainties of Parameters}
            \label{table:CP1c Unweighted Param Uncertainties}
        \end{minipage}
        \vspace{12pt}
        \begin{minipage}{0.5\textwidth}
            \centering
            \begin{tabular}{c c c}
                \hline
                & m & c \\
                \hline
                m & 1 \\
                c & -0.88 & 1 \\
                \hline
            \end{tabular}
            \caption{CP1c Parameter Correlation Matrix}
            \label{table:CP1c Correlation Matrix}
        \end{minipage}
    \end{table}

    \begin{table}[H]
        \centering
        \begin{tabular}{cc}
            \hline
            m & c \\
            \hline
            0.59 & 1.07 \\
            $\pm0.026$ & $\pm0.19$ \\
            \hline
        \end{tabular}
        \caption{CP1c Old Uncertainties}
        \label{table:CP1c Old Uncertainties}
    \end{table}

    \noindent
    Looking at the values returned using the new form of analysis, we can see that they are 
    exactly the same as the results printed out in lines 49-52 of Appendix 3 [Table 
    \ref{table:CP1c Old Uncertainties}]. This confirms that \texttt{scipy} correctly calculates 
    uncertainties of parameters if the data has no uncertainty. Regarding the analysis of the 
    parameters, the value for $\frac{\chi^2}{\texttt{dof}}$ is quite different from 1, meaning 
    that the fit is not a very good one. This can be seen again in the uncertainties for m and c 
    in the fact that they are each $\sim10$\% of the value themselves [Table \ref{table:CP1c Params 
    and Uncertainties}]. Even worse is the fact that the correlation between each value is -0.88 
    [Table \ref{table:CP1c Correlation Matrix}], meaning these uncertainties are not as accurate 
    as they should be as a strong correlation between parameters requires more sophisticated 
    methods for calculating uncertainties. In terms of the look of the red line of best fit in 
    Figure \ref{fig:CP1c Plot}, it is exactly the same as the blue plot, our previous plot, 
    which further shows that the methods return the same results. The uncertainties don't seem 
    The code for these calculations is on lines 78-102 of Appendix 3.
    \newline
    \newline
    In order to properly understand what the correlation between parameters means, we use a contour 
    plot, plotting m against c with the corresponding $\frac{\chi^2}{\texttt{dof}}$ for each point. 
    The section of the code that plots the contour is on lines 49-59 of Appendix 2. Below, 
    [Figure \ref{fig:Weighted Linear}] is the actual plot for the weighted linear least-squares 
    fit, along with the contour plot corresponding to it [Figure \ref{fig:Weighted Contour}].
    
    \begin{figure}[H]
        \begin{center}
            \scalebox{.7}{%% Creator: Matplotlib, PGF backend
%%
%% To include the figure in your LaTeX document, write
%%   \input{<filename>.pgf}
%%
%% Make sure the required packages are loaded in your preamble
%%   \usepackage{pgf}
%%
%% Figures using additional raster images can only be included by \input if
%% they are in the same directory as the main LaTeX file. For loading figures
%% from other directories you can use the `import` package
%%   \usepackage{import}
%% and then include the figures with
%%   \import{<path to file>}{<filename>.pgf}
%%
%% Matplotlib used the following preamble
%%
\begingroup%
\makeatletter%
\begin{pgfpicture}%
\pgfpathrectangle{\pgfpointorigin}{\pgfqpoint{6.400000in}{4.800000in}}%
\pgfusepath{use as bounding box, clip}%
\begin{pgfscope}%
\pgfsetbuttcap%
\pgfsetmiterjoin%
\definecolor{currentfill}{rgb}{1.000000,1.000000,1.000000}%
\pgfsetfillcolor{currentfill}%
\pgfsetlinewidth{0.000000pt}%
\definecolor{currentstroke}{rgb}{1.000000,1.000000,1.000000}%
\pgfsetstrokecolor{currentstroke}%
\pgfsetdash{}{0pt}%
\pgfpathmoveto{\pgfqpoint{0.000000in}{0.000000in}}%
\pgfpathlineto{\pgfqpoint{6.400000in}{0.000000in}}%
\pgfpathlineto{\pgfqpoint{6.400000in}{4.800000in}}%
\pgfpathlineto{\pgfqpoint{0.000000in}{4.800000in}}%
\pgfpathclose%
\pgfusepath{fill}%
\end{pgfscope}%
\begin{pgfscope}%
\pgfsetbuttcap%
\pgfsetmiterjoin%
\definecolor{currentfill}{rgb}{1.000000,1.000000,1.000000}%
\pgfsetfillcolor{currentfill}%
\pgfsetlinewidth{0.000000pt}%
\definecolor{currentstroke}{rgb}{0.000000,0.000000,0.000000}%
\pgfsetstrokecolor{currentstroke}%
\pgfsetstrokeopacity{0.000000}%
\pgfsetdash{}{0pt}%
\pgfpathmoveto{\pgfqpoint{0.800000in}{0.528000in}}%
\pgfpathlineto{\pgfqpoint{5.760000in}{0.528000in}}%
\pgfpathlineto{\pgfqpoint{5.760000in}{4.224000in}}%
\pgfpathlineto{\pgfqpoint{0.800000in}{4.224000in}}%
\pgfpathclose%
\pgfusepath{fill}%
\end{pgfscope}%
\begin{pgfscope}%
\pgfsetbuttcap%
\pgfsetroundjoin%
\definecolor{currentfill}{rgb}{0.000000,0.000000,0.000000}%
\pgfsetfillcolor{currentfill}%
\pgfsetlinewidth{0.803000pt}%
\definecolor{currentstroke}{rgb}{0.000000,0.000000,0.000000}%
\pgfsetstrokecolor{currentstroke}%
\pgfsetdash{}{0pt}%
\pgfsys@defobject{currentmarker}{\pgfqpoint{0.000000in}{-0.048611in}}{\pgfqpoint{0.000000in}{0.000000in}}{%
\pgfpathmoveto{\pgfqpoint{0.000000in}{0.000000in}}%
\pgfpathlineto{\pgfqpoint{0.000000in}{-0.048611in}}%
\pgfusepath{stroke,fill}%
}%
\begin{pgfscope}%
\pgfsys@transformshift{1.435372in}{0.528000in}%
\pgfsys@useobject{currentmarker}{}%
\end{pgfscope}%
\end{pgfscope}%
\begin{pgfscope}%
\definecolor{textcolor}{rgb}{0.000000,0.000000,0.000000}%
\pgfsetstrokecolor{textcolor}%
\pgfsetfillcolor{textcolor}%
\pgftext[x=1.435372in,y=0.430778in,,top]{\color{textcolor}\rmfamily\fontsize{10.000000}{12.000000}\selectfont \(\displaystyle 2\)}%
\end{pgfscope}%
\begin{pgfscope}%
\pgfsetbuttcap%
\pgfsetroundjoin%
\definecolor{currentfill}{rgb}{0.000000,0.000000,0.000000}%
\pgfsetfillcolor{currentfill}%
\pgfsetlinewidth{0.803000pt}%
\definecolor{currentstroke}{rgb}{0.000000,0.000000,0.000000}%
\pgfsetstrokecolor{currentstroke}%
\pgfsetdash{}{0pt}%
\pgfsys@defobject{currentmarker}{\pgfqpoint{0.000000in}{-0.048611in}}{\pgfqpoint{0.000000in}{0.000000in}}{%
\pgfpathmoveto{\pgfqpoint{0.000000in}{0.000000in}}%
\pgfpathlineto{\pgfqpoint{0.000000in}{-0.048611in}}%
\pgfusepath{stroke,fill}%
}%
\begin{pgfscope}%
\pgfsys@transformshift{2.255207in}{0.528000in}%
\pgfsys@useobject{currentmarker}{}%
\end{pgfscope}%
\end{pgfscope}%
\begin{pgfscope}%
\definecolor{textcolor}{rgb}{0.000000,0.000000,0.000000}%
\pgfsetstrokecolor{textcolor}%
\pgfsetfillcolor{textcolor}%
\pgftext[x=2.255207in,y=0.430778in,,top]{\color{textcolor}\rmfamily\fontsize{10.000000}{12.000000}\selectfont \(\displaystyle 4\)}%
\end{pgfscope}%
\begin{pgfscope}%
\pgfsetbuttcap%
\pgfsetroundjoin%
\definecolor{currentfill}{rgb}{0.000000,0.000000,0.000000}%
\pgfsetfillcolor{currentfill}%
\pgfsetlinewidth{0.803000pt}%
\definecolor{currentstroke}{rgb}{0.000000,0.000000,0.000000}%
\pgfsetstrokecolor{currentstroke}%
\pgfsetdash{}{0pt}%
\pgfsys@defobject{currentmarker}{\pgfqpoint{0.000000in}{-0.048611in}}{\pgfqpoint{0.000000in}{0.000000in}}{%
\pgfpathmoveto{\pgfqpoint{0.000000in}{0.000000in}}%
\pgfpathlineto{\pgfqpoint{0.000000in}{-0.048611in}}%
\pgfusepath{stroke,fill}%
}%
\begin{pgfscope}%
\pgfsys@transformshift{3.075041in}{0.528000in}%
\pgfsys@useobject{currentmarker}{}%
\end{pgfscope}%
\end{pgfscope}%
\begin{pgfscope}%
\definecolor{textcolor}{rgb}{0.000000,0.000000,0.000000}%
\pgfsetstrokecolor{textcolor}%
\pgfsetfillcolor{textcolor}%
\pgftext[x=3.075041in,y=0.430778in,,top]{\color{textcolor}\rmfamily\fontsize{10.000000}{12.000000}\selectfont \(\displaystyle 6\)}%
\end{pgfscope}%
\begin{pgfscope}%
\pgfsetbuttcap%
\pgfsetroundjoin%
\definecolor{currentfill}{rgb}{0.000000,0.000000,0.000000}%
\pgfsetfillcolor{currentfill}%
\pgfsetlinewidth{0.803000pt}%
\definecolor{currentstroke}{rgb}{0.000000,0.000000,0.000000}%
\pgfsetstrokecolor{currentstroke}%
\pgfsetdash{}{0pt}%
\pgfsys@defobject{currentmarker}{\pgfqpoint{0.000000in}{-0.048611in}}{\pgfqpoint{0.000000in}{0.000000in}}{%
\pgfpathmoveto{\pgfqpoint{0.000000in}{0.000000in}}%
\pgfpathlineto{\pgfqpoint{0.000000in}{-0.048611in}}%
\pgfusepath{stroke,fill}%
}%
\begin{pgfscope}%
\pgfsys@transformshift{3.894876in}{0.528000in}%
\pgfsys@useobject{currentmarker}{}%
\end{pgfscope}%
\end{pgfscope}%
\begin{pgfscope}%
\definecolor{textcolor}{rgb}{0.000000,0.000000,0.000000}%
\pgfsetstrokecolor{textcolor}%
\pgfsetfillcolor{textcolor}%
\pgftext[x=3.894876in,y=0.430778in,,top]{\color{textcolor}\rmfamily\fontsize{10.000000}{12.000000}\selectfont \(\displaystyle 8\)}%
\end{pgfscope}%
\begin{pgfscope}%
\pgfsetbuttcap%
\pgfsetroundjoin%
\definecolor{currentfill}{rgb}{0.000000,0.000000,0.000000}%
\pgfsetfillcolor{currentfill}%
\pgfsetlinewidth{0.803000pt}%
\definecolor{currentstroke}{rgb}{0.000000,0.000000,0.000000}%
\pgfsetstrokecolor{currentstroke}%
\pgfsetdash{}{0pt}%
\pgfsys@defobject{currentmarker}{\pgfqpoint{0.000000in}{-0.048611in}}{\pgfqpoint{0.000000in}{0.000000in}}{%
\pgfpathmoveto{\pgfqpoint{0.000000in}{0.000000in}}%
\pgfpathlineto{\pgfqpoint{0.000000in}{-0.048611in}}%
\pgfusepath{stroke,fill}%
}%
\begin{pgfscope}%
\pgfsys@transformshift{4.714711in}{0.528000in}%
\pgfsys@useobject{currentmarker}{}%
\end{pgfscope}%
\end{pgfscope}%
\begin{pgfscope}%
\definecolor{textcolor}{rgb}{0.000000,0.000000,0.000000}%
\pgfsetstrokecolor{textcolor}%
\pgfsetfillcolor{textcolor}%
\pgftext[x=4.714711in,y=0.430778in,,top]{\color{textcolor}\rmfamily\fontsize{10.000000}{12.000000}\selectfont \(\displaystyle 10\)}%
\end{pgfscope}%
\begin{pgfscope}%
\pgfsetbuttcap%
\pgfsetroundjoin%
\definecolor{currentfill}{rgb}{0.000000,0.000000,0.000000}%
\pgfsetfillcolor{currentfill}%
\pgfsetlinewidth{0.803000pt}%
\definecolor{currentstroke}{rgb}{0.000000,0.000000,0.000000}%
\pgfsetstrokecolor{currentstroke}%
\pgfsetdash{}{0pt}%
\pgfsys@defobject{currentmarker}{\pgfqpoint{0.000000in}{-0.048611in}}{\pgfqpoint{0.000000in}{0.000000in}}{%
\pgfpathmoveto{\pgfqpoint{0.000000in}{0.000000in}}%
\pgfpathlineto{\pgfqpoint{0.000000in}{-0.048611in}}%
\pgfusepath{stroke,fill}%
}%
\begin{pgfscope}%
\pgfsys@transformshift{5.534545in}{0.528000in}%
\pgfsys@useobject{currentmarker}{}%
\end{pgfscope}%
\end{pgfscope}%
\begin{pgfscope}%
\definecolor{textcolor}{rgb}{0.000000,0.000000,0.000000}%
\pgfsetstrokecolor{textcolor}%
\pgfsetfillcolor{textcolor}%
\pgftext[x=5.534545in,y=0.430778in,,top]{\color{textcolor}\rmfamily\fontsize{10.000000}{12.000000}\selectfont \(\displaystyle 12\)}%
\end{pgfscope}%
\begin{pgfscope}%
\pgfsetbuttcap%
\pgfsetroundjoin%
\definecolor{currentfill}{rgb}{0.000000,0.000000,0.000000}%
\pgfsetfillcolor{currentfill}%
\pgfsetlinewidth{0.803000pt}%
\definecolor{currentstroke}{rgb}{0.000000,0.000000,0.000000}%
\pgfsetstrokecolor{currentstroke}%
\pgfsetdash{}{0pt}%
\pgfsys@defobject{currentmarker}{\pgfqpoint{-0.048611in}{0.000000in}}{\pgfqpoint{0.000000in}{0.000000in}}{%
\pgfpathmoveto{\pgfqpoint{0.000000in}{0.000000in}}%
\pgfpathlineto{\pgfqpoint{-0.048611in}{0.000000in}}%
\pgfusepath{stroke,fill}%
}%
\begin{pgfscope}%
\pgfsys@transformshift{0.800000in}{0.696000in}%
\pgfsys@useobject{currentmarker}{}%
\end{pgfscope}%
\end{pgfscope}%
\begin{pgfscope}%
\definecolor{textcolor}{rgb}{0.000000,0.000000,0.000000}%
\pgfsetstrokecolor{textcolor}%
\pgfsetfillcolor{textcolor}%
\pgftext[x=0.633333in,y=0.647775in,left,base]{\color{textcolor}\rmfamily\fontsize{10.000000}{12.000000}\selectfont \(\displaystyle 1\)}%
\end{pgfscope}%
\begin{pgfscope}%
\pgfsetbuttcap%
\pgfsetroundjoin%
\definecolor{currentfill}{rgb}{0.000000,0.000000,0.000000}%
\pgfsetfillcolor{currentfill}%
\pgfsetlinewidth{0.803000pt}%
\definecolor{currentstroke}{rgb}{0.000000,0.000000,0.000000}%
\pgfsetstrokecolor{currentstroke}%
\pgfsetdash{}{0pt}%
\pgfsys@defobject{currentmarker}{\pgfqpoint{-0.048611in}{0.000000in}}{\pgfqpoint{0.000000in}{0.000000in}}{%
\pgfpathmoveto{\pgfqpoint{0.000000in}{0.000000in}}%
\pgfpathlineto{\pgfqpoint{-0.048611in}{0.000000in}}%
\pgfusepath{stroke,fill}%
}%
\begin{pgfscope}%
\pgfsys@transformshift{0.800000in}{1.120779in}%
\pgfsys@useobject{currentmarker}{}%
\end{pgfscope}%
\end{pgfscope}%
\begin{pgfscope}%
\definecolor{textcolor}{rgb}{0.000000,0.000000,0.000000}%
\pgfsetstrokecolor{textcolor}%
\pgfsetfillcolor{textcolor}%
\pgftext[x=0.633333in,y=1.072553in,left,base]{\color{textcolor}\rmfamily\fontsize{10.000000}{12.000000}\selectfont \(\displaystyle 2\)}%
\end{pgfscope}%
\begin{pgfscope}%
\pgfsetbuttcap%
\pgfsetroundjoin%
\definecolor{currentfill}{rgb}{0.000000,0.000000,0.000000}%
\pgfsetfillcolor{currentfill}%
\pgfsetlinewidth{0.803000pt}%
\definecolor{currentstroke}{rgb}{0.000000,0.000000,0.000000}%
\pgfsetstrokecolor{currentstroke}%
\pgfsetdash{}{0pt}%
\pgfsys@defobject{currentmarker}{\pgfqpoint{-0.048611in}{0.000000in}}{\pgfqpoint{0.000000in}{0.000000in}}{%
\pgfpathmoveto{\pgfqpoint{0.000000in}{0.000000in}}%
\pgfpathlineto{\pgfqpoint{-0.048611in}{0.000000in}}%
\pgfusepath{stroke,fill}%
}%
\begin{pgfscope}%
\pgfsys@transformshift{0.800000in}{1.545558in}%
\pgfsys@useobject{currentmarker}{}%
\end{pgfscope}%
\end{pgfscope}%
\begin{pgfscope}%
\definecolor{textcolor}{rgb}{0.000000,0.000000,0.000000}%
\pgfsetstrokecolor{textcolor}%
\pgfsetfillcolor{textcolor}%
\pgftext[x=0.633333in,y=1.497332in,left,base]{\color{textcolor}\rmfamily\fontsize{10.000000}{12.000000}\selectfont \(\displaystyle 3\)}%
\end{pgfscope}%
\begin{pgfscope}%
\pgfsetbuttcap%
\pgfsetroundjoin%
\definecolor{currentfill}{rgb}{0.000000,0.000000,0.000000}%
\pgfsetfillcolor{currentfill}%
\pgfsetlinewidth{0.803000pt}%
\definecolor{currentstroke}{rgb}{0.000000,0.000000,0.000000}%
\pgfsetstrokecolor{currentstroke}%
\pgfsetdash{}{0pt}%
\pgfsys@defobject{currentmarker}{\pgfqpoint{-0.048611in}{0.000000in}}{\pgfqpoint{0.000000in}{0.000000in}}{%
\pgfpathmoveto{\pgfqpoint{0.000000in}{0.000000in}}%
\pgfpathlineto{\pgfqpoint{-0.048611in}{0.000000in}}%
\pgfusepath{stroke,fill}%
}%
\begin{pgfscope}%
\pgfsys@transformshift{0.800000in}{1.970336in}%
\pgfsys@useobject{currentmarker}{}%
\end{pgfscope}%
\end{pgfscope}%
\begin{pgfscope}%
\definecolor{textcolor}{rgb}{0.000000,0.000000,0.000000}%
\pgfsetstrokecolor{textcolor}%
\pgfsetfillcolor{textcolor}%
\pgftext[x=0.633333in,y=1.922111in,left,base]{\color{textcolor}\rmfamily\fontsize{10.000000}{12.000000}\selectfont \(\displaystyle 4\)}%
\end{pgfscope}%
\begin{pgfscope}%
\pgfsetbuttcap%
\pgfsetroundjoin%
\definecolor{currentfill}{rgb}{0.000000,0.000000,0.000000}%
\pgfsetfillcolor{currentfill}%
\pgfsetlinewidth{0.803000pt}%
\definecolor{currentstroke}{rgb}{0.000000,0.000000,0.000000}%
\pgfsetstrokecolor{currentstroke}%
\pgfsetdash{}{0pt}%
\pgfsys@defobject{currentmarker}{\pgfqpoint{-0.048611in}{0.000000in}}{\pgfqpoint{0.000000in}{0.000000in}}{%
\pgfpathmoveto{\pgfqpoint{0.000000in}{0.000000in}}%
\pgfpathlineto{\pgfqpoint{-0.048611in}{0.000000in}}%
\pgfusepath{stroke,fill}%
}%
\begin{pgfscope}%
\pgfsys@transformshift{0.800000in}{2.395115in}%
\pgfsys@useobject{currentmarker}{}%
\end{pgfscope}%
\end{pgfscope}%
\begin{pgfscope}%
\definecolor{textcolor}{rgb}{0.000000,0.000000,0.000000}%
\pgfsetstrokecolor{textcolor}%
\pgfsetfillcolor{textcolor}%
\pgftext[x=0.633333in,y=2.346890in,left,base]{\color{textcolor}\rmfamily\fontsize{10.000000}{12.000000}\selectfont \(\displaystyle 5\)}%
\end{pgfscope}%
\begin{pgfscope}%
\pgfsetbuttcap%
\pgfsetroundjoin%
\definecolor{currentfill}{rgb}{0.000000,0.000000,0.000000}%
\pgfsetfillcolor{currentfill}%
\pgfsetlinewidth{0.803000pt}%
\definecolor{currentstroke}{rgb}{0.000000,0.000000,0.000000}%
\pgfsetstrokecolor{currentstroke}%
\pgfsetdash{}{0pt}%
\pgfsys@defobject{currentmarker}{\pgfqpoint{-0.048611in}{0.000000in}}{\pgfqpoint{0.000000in}{0.000000in}}{%
\pgfpathmoveto{\pgfqpoint{0.000000in}{0.000000in}}%
\pgfpathlineto{\pgfqpoint{-0.048611in}{0.000000in}}%
\pgfusepath{stroke,fill}%
}%
\begin{pgfscope}%
\pgfsys@transformshift{0.800000in}{2.819894in}%
\pgfsys@useobject{currentmarker}{}%
\end{pgfscope}%
\end{pgfscope}%
\begin{pgfscope}%
\definecolor{textcolor}{rgb}{0.000000,0.000000,0.000000}%
\pgfsetstrokecolor{textcolor}%
\pgfsetfillcolor{textcolor}%
\pgftext[x=0.633333in,y=2.771669in,left,base]{\color{textcolor}\rmfamily\fontsize{10.000000}{12.000000}\selectfont \(\displaystyle 6\)}%
\end{pgfscope}%
\begin{pgfscope}%
\pgfsetbuttcap%
\pgfsetroundjoin%
\definecolor{currentfill}{rgb}{0.000000,0.000000,0.000000}%
\pgfsetfillcolor{currentfill}%
\pgfsetlinewidth{0.803000pt}%
\definecolor{currentstroke}{rgb}{0.000000,0.000000,0.000000}%
\pgfsetstrokecolor{currentstroke}%
\pgfsetdash{}{0pt}%
\pgfsys@defobject{currentmarker}{\pgfqpoint{-0.048611in}{0.000000in}}{\pgfqpoint{0.000000in}{0.000000in}}{%
\pgfpathmoveto{\pgfqpoint{0.000000in}{0.000000in}}%
\pgfpathlineto{\pgfqpoint{-0.048611in}{0.000000in}}%
\pgfusepath{stroke,fill}%
}%
\begin{pgfscope}%
\pgfsys@transformshift{0.800000in}{3.244673in}%
\pgfsys@useobject{currentmarker}{}%
\end{pgfscope}%
\end{pgfscope}%
\begin{pgfscope}%
\definecolor{textcolor}{rgb}{0.000000,0.000000,0.000000}%
\pgfsetstrokecolor{textcolor}%
\pgfsetfillcolor{textcolor}%
\pgftext[x=0.633333in,y=3.196447in,left,base]{\color{textcolor}\rmfamily\fontsize{10.000000}{12.000000}\selectfont \(\displaystyle 7\)}%
\end{pgfscope}%
\begin{pgfscope}%
\pgfsetbuttcap%
\pgfsetroundjoin%
\definecolor{currentfill}{rgb}{0.000000,0.000000,0.000000}%
\pgfsetfillcolor{currentfill}%
\pgfsetlinewidth{0.803000pt}%
\definecolor{currentstroke}{rgb}{0.000000,0.000000,0.000000}%
\pgfsetstrokecolor{currentstroke}%
\pgfsetdash{}{0pt}%
\pgfsys@defobject{currentmarker}{\pgfqpoint{-0.048611in}{0.000000in}}{\pgfqpoint{0.000000in}{0.000000in}}{%
\pgfpathmoveto{\pgfqpoint{0.000000in}{0.000000in}}%
\pgfpathlineto{\pgfqpoint{-0.048611in}{0.000000in}}%
\pgfusepath{stroke,fill}%
}%
\begin{pgfscope}%
\pgfsys@transformshift{0.800000in}{3.669451in}%
\pgfsys@useobject{currentmarker}{}%
\end{pgfscope}%
\end{pgfscope}%
\begin{pgfscope}%
\definecolor{textcolor}{rgb}{0.000000,0.000000,0.000000}%
\pgfsetstrokecolor{textcolor}%
\pgfsetfillcolor{textcolor}%
\pgftext[x=0.633333in,y=3.621226in,left,base]{\color{textcolor}\rmfamily\fontsize{10.000000}{12.000000}\selectfont \(\displaystyle 8\)}%
\end{pgfscope}%
\begin{pgfscope}%
\pgfsetbuttcap%
\pgfsetroundjoin%
\definecolor{currentfill}{rgb}{0.000000,0.000000,0.000000}%
\pgfsetfillcolor{currentfill}%
\pgfsetlinewidth{0.803000pt}%
\definecolor{currentstroke}{rgb}{0.000000,0.000000,0.000000}%
\pgfsetstrokecolor{currentstroke}%
\pgfsetdash{}{0pt}%
\pgfsys@defobject{currentmarker}{\pgfqpoint{-0.048611in}{0.000000in}}{\pgfqpoint{0.000000in}{0.000000in}}{%
\pgfpathmoveto{\pgfqpoint{0.000000in}{0.000000in}}%
\pgfpathlineto{\pgfqpoint{-0.048611in}{0.000000in}}%
\pgfusepath{stroke,fill}%
}%
\begin{pgfscope}%
\pgfsys@transformshift{0.800000in}{4.094230in}%
\pgfsys@useobject{currentmarker}{}%
\end{pgfscope}%
\end{pgfscope}%
\begin{pgfscope}%
\definecolor{textcolor}{rgb}{0.000000,0.000000,0.000000}%
\pgfsetstrokecolor{textcolor}%
\pgfsetfillcolor{textcolor}%
\pgftext[x=0.633333in,y=4.046005in,left,base]{\color{textcolor}\rmfamily\fontsize{10.000000}{12.000000}\selectfont \(\displaystyle 9\)}%
\end{pgfscope}%
\begin{pgfscope}%
\pgfpathrectangle{\pgfqpoint{0.800000in}{0.528000in}}{\pgfqpoint{4.960000in}{3.696000in}}%
\pgfusepath{clip}%
\pgfsetbuttcap%
\pgfsetroundjoin%
\pgfsetlinewidth{0.501875pt}%
\definecolor{currentstroke}{rgb}{0.000000,0.000000,1.000000}%
\pgfsetstrokecolor{currentstroke}%
\pgfsetdash{}{0pt}%
\pgfpathmoveto{\pgfqpoint{1.025455in}{0.696000in}}%
\pgfpathlineto{\pgfqpoint{1.025455in}{1.545558in}}%
\pgfusepath{stroke}%
\end{pgfscope}%
\begin{pgfscope}%
\pgfpathrectangle{\pgfqpoint{0.800000in}{0.528000in}}{\pgfqpoint{4.960000in}{3.696000in}}%
\pgfusepath{clip}%
\pgfsetbuttcap%
\pgfsetroundjoin%
\pgfsetlinewidth{0.501875pt}%
\definecolor{currentstroke}{rgb}{0.000000,0.000000,1.000000}%
\pgfsetstrokecolor{currentstroke}%
\pgfsetdash{}{0pt}%
\pgfpathmoveto{\pgfqpoint{1.435372in}{0.997593in}}%
\pgfpathlineto{\pgfqpoint{1.435372in}{1.422372in}}%
\pgfusepath{stroke}%
\end{pgfscope}%
\begin{pgfscope}%
\pgfpathrectangle{\pgfqpoint{0.800000in}{0.528000in}}{\pgfqpoint{4.960000in}{3.696000in}}%
\pgfusepath{clip}%
\pgfsetbuttcap%
\pgfsetroundjoin%
\pgfsetlinewidth{0.501875pt}%
\definecolor{currentstroke}{rgb}{0.000000,0.000000,1.000000}%
\pgfsetstrokecolor{currentstroke}%
\pgfsetdash{}{0pt}%
\pgfpathmoveto{\pgfqpoint{1.845289in}{1.379894in}}%
\pgfpathlineto{\pgfqpoint{1.845289in}{1.464850in}}%
\pgfusepath{stroke}%
\end{pgfscope}%
\begin{pgfscope}%
\pgfpathrectangle{\pgfqpoint{0.800000in}{0.528000in}}{\pgfqpoint{4.960000in}{3.696000in}}%
\pgfusepath{clip}%
\pgfsetbuttcap%
\pgfsetroundjoin%
\pgfsetlinewidth{0.501875pt}%
\definecolor{currentstroke}{rgb}{0.000000,0.000000,1.000000}%
\pgfsetstrokecolor{currentstroke}%
\pgfsetdash{}{0pt}%
\pgfpathmoveto{\pgfqpoint{2.255207in}{0.959363in}}%
\pgfpathlineto{\pgfqpoint{2.255207in}{2.743434in}}%
\pgfusepath{stroke}%
\end{pgfscope}%
\begin{pgfscope}%
\pgfpathrectangle{\pgfqpoint{0.800000in}{0.528000in}}{\pgfqpoint{4.960000in}{3.696000in}}%
\pgfusepath{clip}%
\pgfsetbuttcap%
\pgfsetroundjoin%
\pgfsetlinewidth{0.501875pt}%
\definecolor{currentstroke}{rgb}{0.000000,0.000000,1.000000}%
\pgfsetstrokecolor{currentstroke}%
\pgfsetdash{}{0pt}%
\pgfpathmoveto{\pgfqpoint{2.665124in}{1.770690in}}%
\pgfpathlineto{\pgfqpoint{2.665124in}{1.949097in}}%
\pgfusepath{stroke}%
\end{pgfscope}%
\begin{pgfscope}%
\pgfpathrectangle{\pgfqpoint{0.800000in}{0.528000in}}{\pgfqpoint{4.960000in}{3.696000in}}%
\pgfusepath{clip}%
\pgfsetbuttcap%
\pgfsetroundjoin%
\pgfsetlinewidth{0.501875pt}%
\definecolor{currentstroke}{rgb}{0.000000,0.000000,1.000000}%
\pgfsetstrokecolor{currentstroke}%
\pgfsetdash{}{0pt}%
\pgfpathmoveto{\pgfqpoint{3.075041in}{1.847150in}}%
\pgfpathlineto{\pgfqpoint{3.075041in}{2.492814in}}%
\pgfusepath{stroke}%
\end{pgfscope}%
\begin{pgfscope}%
\pgfpathrectangle{\pgfqpoint{0.800000in}{0.528000in}}{\pgfqpoint{4.960000in}{3.696000in}}%
\pgfusepath{clip}%
\pgfsetbuttcap%
\pgfsetroundjoin%
\pgfsetlinewidth{0.501875pt}%
\definecolor{currentstroke}{rgb}{0.000000,0.000000,1.000000}%
\pgfsetstrokecolor{currentstroke}%
\pgfsetdash{}{0pt}%
\pgfpathmoveto{\pgfqpoint{3.484959in}{1.983080in}}%
\pgfpathlineto{\pgfqpoint{3.484959in}{2.662726in}}%
\pgfusepath{stroke}%
\end{pgfscope}%
\begin{pgfscope}%
\pgfpathrectangle{\pgfqpoint{0.800000in}{0.528000in}}{\pgfqpoint{4.960000in}{3.696000in}}%
\pgfusepath{clip}%
\pgfsetbuttcap%
\pgfsetroundjoin%
\pgfsetlinewidth{0.501875pt}%
\definecolor{currentstroke}{rgb}{0.000000,0.000000,1.000000}%
\pgfsetstrokecolor{currentstroke}%
\pgfsetdash{}{0pt}%
\pgfpathmoveto{\pgfqpoint{3.894876in}{2.441841in}}%
\pgfpathlineto{\pgfqpoint{3.894876in}{3.376354in}}%
\pgfusepath{stroke}%
\end{pgfscope}%
\begin{pgfscope}%
\pgfpathrectangle{\pgfqpoint{0.800000in}{0.528000in}}{\pgfqpoint{4.960000in}{3.696000in}}%
\pgfusepath{clip}%
\pgfsetbuttcap%
\pgfsetroundjoin%
\pgfsetlinewidth{0.501875pt}%
\definecolor{currentstroke}{rgb}{0.000000,0.000000,1.000000}%
\pgfsetstrokecolor{currentstroke}%
\pgfsetdash{}{0pt}%
\pgfpathmoveto{\pgfqpoint{4.304793in}{2.730690in}}%
\pgfpathlineto{\pgfqpoint{4.304793in}{2.909097in}}%
\pgfusepath{stroke}%
\end{pgfscope}%
\begin{pgfscope}%
\pgfpathrectangle{\pgfqpoint{0.800000in}{0.528000in}}{\pgfqpoint{4.960000in}{3.696000in}}%
\pgfusepath{clip}%
\pgfsetbuttcap%
\pgfsetroundjoin%
\pgfsetlinewidth{0.501875pt}%
\definecolor{currentstroke}{rgb}{0.000000,0.000000,1.000000}%
\pgfsetstrokecolor{currentstroke}%
\pgfsetdash{}{0pt}%
\pgfpathmoveto{\pgfqpoint{4.714711in}{2.802903in}}%
\pgfpathlineto{\pgfqpoint{4.714711in}{3.567504in}}%
\pgfusepath{stroke}%
\end{pgfscope}%
\begin{pgfscope}%
\pgfpathrectangle{\pgfqpoint{0.800000in}{0.528000in}}{\pgfqpoint{4.960000in}{3.696000in}}%
\pgfusepath{clip}%
\pgfsetbuttcap%
\pgfsetroundjoin%
\pgfsetlinewidth{0.501875pt}%
\definecolor{currentstroke}{rgb}{0.000000,0.000000,1.000000}%
\pgfsetstrokecolor{currentstroke}%
\pgfsetdash{}{0pt}%
\pgfpathmoveto{\pgfqpoint{5.124628in}{3.384850in}}%
\pgfpathlineto{\pgfqpoint{5.124628in}{3.554761in}}%
\pgfusepath{stroke}%
\end{pgfscope}%
\begin{pgfscope}%
\pgfpathrectangle{\pgfqpoint{0.800000in}{0.528000in}}{\pgfqpoint{4.960000in}{3.696000in}}%
\pgfusepath{clip}%
\pgfsetbuttcap%
\pgfsetroundjoin%
\pgfsetlinewidth{0.501875pt}%
\definecolor{currentstroke}{rgb}{0.000000,0.000000,1.000000}%
\pgfsetstrokecolor{currentstroke}%
\pgfsetdash{}{0pt}%
\pgfpathmoveto{\pgfqpoint{5.534545in}{3.801133in}}%
\pgfpathlineto{\pgfqpoint{5.534545in}{4.056000in}}%
\pgfusepath{stroke}%
\end{pgfscope}%
\begin{pgfscope}%
\pgfpathrectangle{\pgfqpoint{0.800000in}{0.528000in}}{\pgfqpoint{4.960000in}{3.696000in}}%
\pgfusepath{clip}%
\pgfsetbuttcap%
\pgfsetroundjoin%
\definecolor{currentfill}{rgb}{0.000000,0.000000,1.000000}%
\pgfsetfillcolor{currentfill}%
\pgfsetlinewidth{0.501875pt}%
\definecolor{currentstroke}{rgb}{0.000000,0.000000,1.000000}%
\pgfsetstrokecolor{currentstroke}%
\pgfsetdash{}{0pt}%
\pgfsys@defobject{currentmarker}{\pgfqpoint{-0.027778in}{-0.000000in}}{\pgfqpoint{0.027778in}{0.000000in}}{%
\pgfpathmoveto{\pgfqpoint{0.027778in}{-0.000000in}}%
\pgfpathlineto{\pgfqpoint{-0.027778in}{0.000000in}}%
\pgfusepath{stroke,fill}%
}%
\begin{pgfscope}%
\pgfsys@transformshift{1.025455in}{0.696000in}%
\pgfsys@useobject{currentmarker}{}%
\end{pgfscope}%
\begin{pgfscope}%
\pgfsys@transformshift{1.435372in}{0.997593in}%
\pgfsys@useobject{currentmarker}{}%
\end{pgfscope}%
\begin{pgfscope}%
\pgfsys@transformshift{1.845289in}{1.379894in}%
\pgfsys@useobject{currentmarker}{}%
\end{pgfscope}%
\begin{pgfscope}%
\pgfsys@transformshift{2.255207in}{0.959363in}%
\pgfsys@useobject{currentmarker}{}%
\end{pgfscope}%
\begin{pgfscope}%
\pgfsys@transformshift{2.665124in}{1.770690in}%
\pgfsys@useobject{currentmarker}{}%
\end{pgfscope}%
\begin{pgfscope}%
\pgfsys@transformshift{3.075041in}{1.847150in}%
\pgfsys@useobject{currentmarker}{}%
\end{pgfscope}%
\begin{pgfscope}%
\pgfsys@transformshift{3.484959in}{1.983080in}%
\pgfsys@useobject{currentmarker}{}%
\end{pgfscope}%
\begin{pgfscope}%
\pgfsys@transformshift{3.894876in}{2.441841in}%
\pgfsys@useobject{currentmarker}{}%
\end{pgfscope}%
\begin{pgfscope}%
\pgfsys@transformshift{4.304793in}{2.730690in}%
\pgfsys@useobject{currentmarker}{}%
\end{pgfscope}%
\begin{pgfscope}%
\pgfsys@transformshift{4.714711in}{2.802903in}%
\pgfsys@useobject{currentmarker}{}%
\end{pgfscope}%
\begin{pgfscope}%
\pgfsys@transformshift{5.124628in}{3.384850in}%
\pgfsys@useobject{currentmarker}{}%
\end{pgfscope}%
\begin{pgfscope}%
\pgfsys@transformshift{5.534545in}{3.801133in}%
\pgfsys@useobject{currentmarker}{}%
\end{pgfscope}%
\end{pgfscope}%
\begin{pgfscope}%
\pgfpathrectangle{\pgfqpoint{0.800000in}{0.528000in}}{\pgfqpoint{4.960000in}{3.696000in}}%
\pgfusepath{clip}%
\pgfsetbuttcap%
\pgfsetroundjoin%
\definecolor{currentfill}{rgb}{0.000000,0.000000,1.000000}%
\pgfsetfillcolor{currentfill}%
\pgfsetlinewidth{0.501875pt}%
\definecolor{currentstroke}{rgb}{0.000000,0.000000,1.000000}%
\pgfsetstrokecolor{currentstroke}%
\pgfsetdash{}{0pt}%
\pgfsys@defobject{currentmarker}{\pgfqpoint{-0.027778in}{-0.000000in}}{\pgfqpoint{0.027778in}{0.000000in}}{%
\pgfpathmoveto{\pgfqpoint{0.027778in}{-0.000000in}}%
\pgfpathlineto{\pgfqpoint{-0.027778in}{0.000000in}}%
\pgfusepath{stroke,fill}%
}%
\begin{pgfscope}%
\pgfsys@transformshift{1.025455in}{1.545558in}%
\pgfsys@useobject{currentmarker}{}%
\end{pgfscope}%
\begin{pgfscope}%
\pgfsys@transformshift{1.435372in}{1.422372in}%
\pgfsys@useobject{currentmarker}{}%
\end{pgfscope}%
\begin{pgfscope}%
\pgfsys@transformshift{1.845289in}{1.464850in}%
\pgfsys@useobject{currentmarker}{}%
\end{pgfscope}%
\begin{pgfscope}%
\pgfsys@transformshift{2.255207in}{2.743434in}%
\pgfsys@useobject{currentmarker}{}%
\end{pgfscope}%
\begin{pgfscope}%
\pgfsys@transformshift{2.665124in}{1.949097in}%
\pgfsys@useobject{currentmarker}{}%
\end{pgfscope}%
\begin{pgfscope}%
\pgfsys@transformshift{3.075041in}{2.492814in}%
\pgfsys@useobject{currentmarker}{}%
\end{pgfscope}%
\begin{pgfscope}%
\pgfsys@transformshift{3.484959in}{2.662726in}%
\pgfsys@useobject{currentmarker}{}%
\end{pgfscope}%
\begin{pgfscope}%
\pgfsys@transformshift{3.894876in}{3.376354in}%
\pgfsys@useobject{currentmarker}{}%
\end{pgfscope}%
\begin{pgfscope}%
\pgfsys@transformshift{4.304793in}{2.909097in}%
\pgfsys@useobject{currentmarker}{}%
\end{pgfscope}%
\begin{pgfscope}%
\pgfsys@transformshift{4.714711in}{3.567504in}%
\pgfsys@useobject{currentmarker}{}%
\end{pgfscope}%
\begin{pgfscope}%
\pgfsys@transformshift{5.124628in}{3.554761in}%
\pgfsys@useobject{currentmarker}{}%
\end{pgfscope}%
\begin{pgfscope}%
\pgfsys@transformshift{5.534545in}{4.056000in}%
\pgfsys@useobject{currentmarker}{}%
\end{pgfscope}%
\end{pgfscope}%
\begin{pgfscope}%
\pgfpathrectangle{\pgfqpoint{0.800000in}{0.528000in}}{\pgfqpoint{4.960000in}{3.696000in}}%
\pgfusepath{clip}%
\pgfsetrectcap%
\pgfsetroundjoin%
\pgfsetlinewidth{0.501875pt}%
\definecolor{currentstroke}{rgb}{1.000000,0.000000,0.000000}%
\pgfsetstrokecolor{currentstroke}%
\pgfsetdash{}{0pt}%
\pgfpathmoveto{\pgfqpoint{1.025455in}{0.892493in}}%
\pgfpathlineto{\pgfqpoint{5.534545in}{3.723025in}}%
\pgfpathlineto{\pgfqpoint{5.534545in}{3.723025in}}%
\pgfusepath{stroke}%
\end{pgfscope}%
\begin{pgfscope}%
\pgfpathrectangle{\pgfqpoint{0.800000in}{0.528000in}}{\pgfqpoint{4.960000in}{3.696000in}}%
\pgfusepath{clip}%
\pgfsetbuttcap%
\pgfsetroundjoin%
\definecolor{currentfill}{rgb}{0.000000,0.000000,1.000000}%
\pgfsetfillcolor{currentfill}%
\pgfsetlinewidth{0.501875pt}%
\definecolor{currentstroke}{rgb}{0.000000,0.000000,1.000000}%
\pgfsetstrokecolor{currentstroke}%
\pgfsetdash{}{0pt}%
\pgfsys@defobject{currentmarker}{\pgfqpoint{-0.027778in}{-0.000000in}}{\pgfqpoint{0.027778in}{0.000000in}}{%
\pgfpathmoveto{\pgfqpoint{0.027778in}{-0.000000in}}%
\pgfpathlineto{\pgfqpoint{-0.027778in}{0.000000in}}%
\pgfusepath{stroke,fill}%
}%
\begin{pgfscope}%
\pgfsys@transformshift{1.025455in}{1.120779in}%
\pgfsys@useobject{currentmarker}{}%
\end{pgfscope}%
\begin{pgfscope}%
\pgfsys@transformshift{1.435372in}{1.209982in}%
\pgfsys@useobject{currentmarker}{}%
\end{pgfscope}%
\begin{pgfscope}%
\pgfsys@transformshift{1.845289in}{1.422372in}%
\pgfsys@useobject{currentmarker}{}%
\end{pgfscope}%
\begin{pgfscope}%
\pgfsys@transformshift{2.255207in}{1.851398in}%
\pgfsys@useobject{currentmarker}{}%
\end{pgfscope}%
\begin{pgfscope}%
\pgfsys@transformshift{2.665124in}{1.859894in}%
\pgfsys@useobject{currentmarker}{}%
\end{pgfscope}%
\begin{pgfscope}%
\pgfsys@transformshift{3.075041in}{2.169982in}%
\pgfsys@useobject{currentmarker}{}%
\end{pgfscope}%
\begin{pgfscope}%
\pgfsys@transformshift{3.484959in}{2.322903in}%
\pgfsys@useobject{currentmarker}{}%
\end{pgfscope}%
\begin{pgfscope}%
\pgfsys@transformshift{3.894876in}{2.909097in}%
\pgfsys@useobject{currentmarker}{}%
\end{pgfscope}%
\begin{pgfscope}%
\pgfsys@transformshift{4.304793in}{2.819894in}%
\pgfsys@useobject{currentmarker}{}%
\end{pgfscope}%
\begin{pgfscope}%
\pgfsys@transformshift{4.714711in}{3.185204in}%
\pgfsys@useobject{currentmarker}{}%
\end{pgfscope}%
\begin{pgfscope}%
\pgfsys@transformshift{5.124628in}{3.469805in}%
\pgfsys@useobject{currentmarker}{}%
\end{pgfscope}%
\begin{pgfscope}%
\pgfsys@transformshift{5.534545in}{3.928566in}%
\pgfsys@useobject{currentmarker}{}%
\end{pgfscope}%
\end{pgfscope}%
\begin{pgfscope}%
\pgfsetrectcap%
\pgfsetmiterjoin%
\pgfsetlinewidth{0.803000pt}%
\definecolor{currentstroke}{rgb}{0.000000,0.000000,0.000000}%
\pgfsetstrokecolor{currentstroke}%
\pgfsetdash{}{0pt}%
\pgfpathmoveto{\pgfqpoint{0.800000in}{0.528000in}}%
\pgfpathlineto{\pgfqpoint{0.800000in}{4.224000in}}%
\pgfusepath{stroke}%
\end{pgfscope}%
\begin{pgfscope}%
\pgfsetrectcap%
\pgfsetmiterjoin%
\pgfsetlinewidth{0.803000pt}%
\definecolor{currentstroke}{rgb}{0.000000,0.000000,0.000000}%
\pgfsetstrokecolor{currentstroke}%
\pgfsetdash{}{0pt}%
\pgfpathmoveto{\pgfqpoint{5.760000in}{0.528000in}}%
\pgfpathlineto{\pgfqpoint{5.760000in}{4.224000in}}%
\pgfusepath{stroke}%
\end{pgfscope}%
\begin{pgfscope}%
\pgfsetrectcap%
\pgfsetmiterjoin%
\pgfsetlinewidth{0.803000pt}%
\definecolor{currentstroke}{rgb}{0.000000,0.000000,0.000000}%
\pgfsetstrokecolor{currentstroke}%
\pgfsetdash{}{0pt}%
\pgfpathmoveto{\pgfqpoint{0.800000in}{0.528000in}}%
\pgfpathlineto{\pgfqpoint{5.760000in}{0.528000in}}%
\pgfusepath{stroke}%
\end{pgfscope}%
\begin{pgfscope}%
\pgfsetrectcap%
\pgfsetmiterjoin%
\pgfsetlinewidth{0.803000pt}%
\definecolor{currentstroke}{rgb}{0.000000,0.000000,0.000000}%
\pgfsetstrokecolor{currentstroke}%
\pgfsetdash{}{0pt}%
\pgfpathmoveto{\pgfqpoint{0.800000in}{4.224000in}}%
\pgfpathlineto{\pgfqpoint{5.760000in}{4.224000in}}%
\pgfusepath{stroke}%
\end{pgfscope}%
\begin{pgfscope}%
\pgfsetbuttcap%
\pgfsetmiterjoin%
\definecolor{currentfill}{rgb}{1.000000,1.000000,1.000000}%
\pgfsetfillcolor{currentfill}%
\pgfsetfillopacity{0.800000}%
\pgfsetlinewidth{1.003750pt}%
\definecolor{currentstroke}{rgb}{0.800000,0.800000,0.800000}%
\pgfsetstrokecolor{currentstroke}%
\pgfsetstrokeopacity{0.800000}%
\pgfsetdash{}{0pt}%
\pgfpathmoveto{\pgfqpoint{0.897222in}{3.725543in}}%
\pgfpathlineto{\pgfqpoint{1.840124in}{3.725543in}}%
\pgfpathquadraticcurveto{\pgfqpoint{1.867902in}{3.725543in}}{\pgfqpoint{1.867902in}{3.753321in}}%
\pgfpathlineto{\pgfqpoint{1.867902in}{4.126778in}}%
\pgfpathquadraticcurveto{\pgfqpoint{1.867902in}{4.154556in}}{\pgfqpoint{1.840124in}{4.154556in}}%
\pgfpathlineto{\pgfqpoint{0.897222in}{4.154556in}}%
\pgfpathquadraticcurveto{\pgfqpoint{0.869444in}{4.154556in}}{\pgfqpoint{0.869444in}{4.126778in}}%
\pgfpathlineto{\pgfqpoint{0.869444in}{3.753321in}}%
\pgfpathquadraticcurveto{\pgfqpoint{0.869444in}{3.725543in}}{\pgfqpoint{0.897222in}{3.725543in}}%
\pgfpathclose%
\pgfusepath{stroke,fill}%
\end{pgfscope}%
\begin{pgfscope}%
\pgfsetrectcap%
\pgfsetroundjoin%
\pgfsetlinewidth{0.501875pt}%
\definecolor{currentstroke}{rgb}{1.000000,0.000000,0.000000}%
\pgfsetstrokecolor{currentstroke}%
\pgfsetdash{}{0pt}%
\pgfpathmoveto{\pgfqpoint{0.925000in}{4.050389in}}%
\pgfpathlineto{\pgfqpoint{1.202778in}{4.050389in}}%
\pgfusepath{stroke}%
\end{pgfscope}%
\begin{pgfscope}%
\definecolor{textcolor}{rgb}{0.000000,0.000000,0.000000}%
\pgfsetstrokecolor{textcolor}%
\pgfsetfillcolor{textcolor}%
\pgftext[x=1.313889in,y=4.001778in,left,base]{\color{textcolor}\rmfamily\fontsize{10.000000}{12.000000}\selectfont Best Fit}%
\end{pgfscope}%
\begin{pgfscope}%
\pgfsetbuttcap%
\pgfsetroundjoin%
\pgfsetlinewidth{0.501875pt}%
\definecolor{currentstroke}{rgb}{0.000000,0.000000,1.000000}%
\pgfsetstrokecolor{currentstroke}%
\pgfsetdash{}{0pt}%
\pgfpathmoveto{\pgfqpoint{1.063889in}{3.787272in}}%
\pgfpathlineto{\pgfqpoint{1.063889in}{3.926161in}}%
\pgfusepath{stroke}%
\end{pgfscope}%
\begin{pgfscope}%
\pgfsetbuttcap%
\pgfsetroundjoin%
\definecolor{currentfill}{rgb}{0.000000,0.000000,1.000000}%
\pgfsetfillcolor{currentfill}%
\pgfsetlinewidth{0.501875pt}%
\definecolor{currentstroke}{rgb}{0.000000,0.000000,1.000000}%
\pgfsetstrokecolor{currentstroke}%
\pgfsetdash{}{0pt}%
\pgfsys@defobject{currentmarker}{\pgfqpoint{-0.027778in}{-0.000000in}}{\pgfqpoint{0.027778in}{0.000000in}}{%
\pgfpathmoveto{\pgfqpoint{0.027778in}{-0.000000in}}%
\pgfpathlineto{\pgfqpoint{-0.027778in}{0.000000in}}%
\pgfusepath{stroke,fill}%
}%
\begin{pgfscope}%
\pgfsys@transformshift{1.063889in}{3.787272in}%
\pgfsys@useobject{currentmarker}{}%
\end{pgfscope}%
\end{pgfscope}%
\begin{pgfscope}%
\pgfsetbuttcap%
\pgfsetroundjoin%
\definecolor{currentfill}{rgb}{0.000000,0.000000,1.000000}%
\pgfsetfillcolor{currentfill}%
\pgfsetlinewidth{0.501875pt}%
\definecolor{currentstroke}{rgb}{0.000000,0.000000,1.000000}%
\pgfsetstrokecolor{currentstroke}%
\pgfsetdash{}{0pt}%
\pgfsys@defobject{currentmarker}{\pgfqpoint{-0.027778in}{-0.000000in}}{\pgfqpoint{0.027778in}{0.000000in}}{%
\pgfpathmoveto{\pgfqpoint{0.027778in}{-0.000000in}}%
\pgfpathlineto{\pgfqpoint{-0.027778in}{0.000000in}}%
\pgfusepath{stroke,fill}%
}%
\begin{pgfscope}%
\pgfsys@transformshift{1.063889in}{3.926161in}%
\pgfsys@useobject{currentmarker}{}%
\end{pgfscope}%
\end{pgfscope}%
\begin{pgfscope}%
\pgfsetbuttcap%
\pgfsetroundjoin%
\definecolor{currentfill}{rgb}{0.000000,0.000000,1.000000}%
\pgfsetfillcolor{currentfill}%
\pgfsetlinewidth{0.501875pt}%
\definecolor{currentstroke}{rgb}{0.000000,0.000000,1.000000}%
\pgfsetstrokecolor{currentstroke}%
\pgfsetdash{}{0pt}%
\pgfsys@defobject{currentmarker}{\pgfqpoint{-0.027778in}{-0.000000in}}{\pgfqpoint{0.027778in}{0.000000in}}{%
\pgfpathmoveto{\pgfqpoint{0.027778in}{-0.000000in}}%
\pgfpathlineto{\pgfqpoint{-0.027778in}{0.000000in}}%
\pgfusepath{stroke,fill}%
}%
\begin{pgfscope}%
\pgfsys@transformshift{1.063889in}{3.856716in}%
\pgfsys@useobject{currentmarker}{}%
\end{pgfscope}%
\end{pgfscope}%
\begin{pgfscope}%
\definecolor{textcolor}{rgb}{0.000000,0.000000,0.000000}%
\pgfsetstrokecolor{textcolor}%
\pgfsetfillcolor{textcolor}%
\pgftext[x=1.313889in,y=3.808105in,left,base]{\color{textcolor}\rmfamily\fontsize{10.000000}{12.000000}\selectfont Data}%
\end{pgfscope}%
\end{pgfpicture}%
\makeatother%
\endgroup%
}
            \caption{Weighted Linear Least-Squares Fit for LinearWithErrors.txt}
            \label{fig:Weighted Linear}
        \end{center}
    \end{figure}
    
    \begin{figure}[H]
        \begin{center}
            \scalebox{.7}{%% Creator: Matplotlib, PGF backend
%%
%% To include the figure in your LaTeX document, write
%%   \input{<filename>.pgf}
%%
%% Make sure the required packages are loaded in your preamble
%%   \usepackage{pgf}
%%
%% Figures using additional raster images can only be included by \input if
%% they are in the same directory as the main LaTeX file. For loading figures
%% from other directories you can use the `import` package
%%   \usepackage{import}
%% and then include the figures with
%%   \import{<path to file>}{<filename>.pgf}
%%
%% Matplotlib used the following preamble
%%
\begingroup%
\makeatletter%
\begin{pgfpicture}%
\pgfpathrectangle{\pgfpointorigin}{\pgfqpoint{6.400000in}{4.800000in}}%
\pgfusepath{use as bounding box, clip}%
\begin{pgfscope}%
\pgfsetbuttcap%
\pgfsetmiterjoin%
\definecolor{currentfill}{rgb}{1.000000,1.000000,1.000000}%
\pgfsetfillcolor{currentfill}%
\pgfsetlinewidth{0.000000pt}%
\definecolor{currentstroke}{rgb}{1.000000,1.000000,1.000000}%
\pgfsetstrokecolor{currentstroke}%
\pgfsetdash{}{0pt}%
\pgfpathmoveto{\pgfqpoint{0.000000in}{0.000000in}}%
\pgfpathlineto{\pgfqpoint{6.400000in}{0.000000in}}%
\pgfpathlineto{\pgfqpoint{6.400000in}{4.800000in}}%
\pgfpathlineto{\pgfqpoint{0.000000in}{4.800000in}}%
\pgfpathclose%
\pgfusepath{fill}%
\end{pgfscope}%
\begin{pgfscope}%
\pgfsetbuttcap%
\pgfsetmiterjoin%
\definecolor{currentfill}{rgb}{1.000000,1.000000,1.000000}%
\pgfsetfillcolor{currentfill}%
\pgfsetlinewidth{0.000000pt}%
\definecolor{currentstroke}{rgb}{0.000000,0.000000,0.000000}%
\pgfsetstrokecolor{currentstroke}%
\pgfsetstrokeopacity{0.000000}%
\pgfsetdash{}{0pt}%
\pgfpathmoveto{\pgfqpoint{0.800000in}{0.528000in}}%
\pgfpathlineto{\pgfqpoint{4.768000in}{0.528000in}}%
\pgfpathlineto{\pgfqpoint{4.768000in}{4.224000in}}%
\pgfpathlineto{\pgfqpoint{0.800000in}{4.224000in}}%
\pgfpathclose%
\pgfusepath{fill}%
\end{pgfscope}%
\begin{pgfscope}%
\pgfpathrectangle{\pgfqpoint{0.800000in}{0.528000in}}{\pgfqpoint{3.968000in}{3.696000in}}%
\pgfusepath{clip}%
\pgfsetbuttcap%
\pgfsetroundjoin%
\definecolor{currentfill}{rgb}{0.267004,0.004874,0.329415}%
\pgfsetfillcolor{currentfill}%
\pgfsetlinewidth{0.000000pt}%
\definecolor{currentstroke}{rgb}{0.000000,0.000000,0.000000}%
\pgfsetstrokecolor{currentstroke}%
\pgfsetdash{}{0pt}%
\pgfpathmoveto{\pgfqpoint{3.065975in}{1.404021in}}%
\pgfpathlineto{\pgfqpoint{3.084606in}{1.402651in}}%
\pgfpathlineto{\pgfqpoint{3.105745in}{1.404310in}}%
\pgfpathlineto{\pgfqpoint{3.118974in}{1.424000in}}%
\pgfpathlineto{\pgfqpoint{3.121034in}{1.427402in}}%
\pgfpathlineto{\pgfqpoint{3.115011in}{1.461333in}}%
\pgfpathlineto{\pgfqpoint{3.102871in}{1.478346in}}%
\pgfpathlineto{\pgfqpoint{3.099919in}{1.498667in}}%
\pgfpathlineto{\pgfqpoint{3.084606in}{1.528641in}}%
\pgfpathlineto{\pgfqpoint{3.080479in}{1.536000in}}%
\pgfpathlineto{\pgfqpoint{3.075493in}{1.544488in}}%
\pgfpathlineto{\pgfqpoint{3.054920in}{1.573333in}}%
\pgfpathlineto{\pgfqpoint{3.049880in}{1.578321in}}%
\pgfpathlineto{\pgfqpoint{3.044525in}{1.587517in}}%
\pgfpathlineto{\pgfqpoint{3.026232in}{1.610667in}}%
\pgfpathlineto{\pgfqpoint{3.004444in}{1.637518in}}%
\pgfpathlineto{\pgfqpoint{2.995218in}{1.648000in}}%
\pgfpathlineto{\pgfqpoint{2.964364in}{1.682159in}}%
\pgfpathlineto{\pgfqpoint{2.962391in}{1.683496in}}%
\pgfpathlineto{\pgfqpoint{2.961231in}{1.685333in}}%
\pgfpathlineto{\pgfqpoint{2.924283in}{1.721839in}}%
\pgfpathlineto{\pgfqpoint{2.923360in}{1.722667in}}%
\pgfpathlineto{\pgfqpoint{2.884202in}{1.756920in}}%
\pgfpathlineto{\pgfqpoint{2.880281in}{1.760000in}}%
\pgfpathlineto{\pgfqpoint{2.844121in}{1.787728in}}%
\pgfpathlineto{\pgfqpoint{2.836732in}{1.790451in}}%
\pgfpathlineto{\pgfqpoint{2.829989in}{1.797333in}}%
\pgfpathlineto{\pgfqpoint{2.813110in}{1.805782in}}%
\pgfpathlineto{\pgfqpoint{2.804040in}{1.814561in}}%
\pgfpathlineto{\pgfqpoint{2.787371in}{1.819140in}}%
\pgfpathlineto{\pgfqpoint{2.769299in}{1.834667in}}%
\pgfpathlineto{\pgfqpoint{2.763960in}{1.837687in}}%
\pgfpathlineto{\pgfqpoint{2.733232in}{1.843379in}}%
\pgfpathlineto{\pgfqpoint{2.723879in}{1.850004in}}%
\pgfpathlineto{\pgfqpoint{2.704258in}{1.852943in}}%
\pgfpathlineto{\pgfqpoint{2.683798in}{1.841909in}}%
\pgfpathlineto{\pgfqpoint{2.679245in}{1.834667in}}%
\pgfpathlineto{\pgfqpoint{2.678116in}{1.802626in}}%
\pgfpathlineto{\pgfqpoint{2.679928in}{1.797333in}}%
\pgfpathlineto{\pgfqpoint{2.683798in}{1.786824in}}%
\pgfpathlineto{\pgfqpoint{2.694687in}{1.770142in}}%
\pgfpathlineto{\pgfqpoint{2.694717in}{1.760000in}}%
\pgfpathlineto{\pgfqpoint{2.702175in}{1.742883in}}%
\pgfpathlineto{\pgfqpoint{2.714510in}{1.722667in}}%
\pgfpathlineto{\pgfqpoint{2.718845in}{1.717978in}}%
\pgfpathlineto{\pgfqpoint{2.723879in}{1.707818in}}%
\pgfpathlineto{\pgfqpoint{2.739243in}{1.685333in}}%
\pgfpathlineto{\pgfqpoint{2.763960in}{1.650316in}}%
\pgfpathlineto{\pgfqpoint{2.765238in}{1.649191in}}%
\pgfpathlineto{\pgfqpoint{2.765742in}{1.648000in}}%
\pgfpathlineto{\pgfqpoint{2.770811in}{1.641618in}}%
\pgfpathlineto{\pgfqpoint{2.797506in}{1.610667in}}%
\pgfpathlineto{\pgfqpoint{2.800876in}{1.607719in}}%
\pgfpathlineto{\pgfqpoint{2.804040in}{1.603250in}}%
\pgfpathlineto{\pgfqpoint{2.832333in}{1.573333in}}%
\pgfpathlineto{\pgfqpoint{2.838667in}{1.568253in}}%
\pgfpathlineto{\pgfqpoint{2.844121in}{1.561126in}}%
\pgfpathlineto{\pgfqpoint{2.870306in}{1.536000in}}%
\pgfpathlineto{\pgfqpoint{2.878090in}{1.530307in}}%
\pgfpathlineto{\pgfqpoint{2.884202in}{1.522936in}}%
\pgfpathlineto{\pgfqpoint{2.912239in}{1.498667in}}%
\pgfpathlineto{\pgfqpoint{2.919275in}{1.494002in}}%
\pgfpathlineto{\pgfqpoint{2.924283in}{1.488447in}}%
\pgfpathlineto{\pgfqpoint{2.959247in}{1.461333in}}%
\pgfpathlineto{\pgfqpoint{2.962366in}{1.459472in}}%
\pgfpathlineto{\pgfqpoint{2.964364in}{1.457443in}}%
\pgfpathlineto{\pgfqpoint{2.980398in}{1.446398in}}%
\pgfpathlineto{\pgfqpoint{3.004444in}{1.432040in}}%
\pgfpathlineto{\pgfqpoint{3.020574in}{1.424000in}}%
\pgfpathlineto{\pgfqpoint{3.037918in}{1.417846in}}%
\pgfpathlineto{\pgfqpoint{3.044525in}{1.412373in}}%
\pgfusepath{fill}%
\end{pgfscope}%
\begin{pgfscope}%
\pgfpathrectangle{\pgfqpoint{0.800000in}{0.528000in}}{\pgfqpoint{3.968000in}{3.696000in}}%
\pgfusepath{clip}%
\pgfsetbuttcap%
\pgfsetroundjoin%
\definecolor{currentfill}{rgb}{0.267004,0.004874,0.329415}%
\pgfsetfillcolor{currentfill}%
\pgfsetlinewidth{0.000000pt}%
\definecolor{currentstroke}{rgb}{0.000000,0.000000,0.000000}%
\pgfsetstrokecolor{currentstroke}%
\pgfsetdash{}{0pt}%
\pgfpathmoveto{\pgfqpoint{3.044525in}{1.412373in}}%
\pgfpathlineto{\pgfqpoint{3.037918in}{1.417846in}}%
\pgfpathlineto{\pgfqpoint{3.020574in}{1.424000in}}%
\pgfpathlineto{\pgfqpoint{3.004444in}{1.432040in}}%
\pgfpathlineto{\pgfqpoint{2.980398in}{1.446398in}}%
\pgfpathlineto{\pgfqpoint{2.959247in}{1.461333in}}%
\pgfpathlineto{\pgfqpoint{2.924283in}{1.488447in}}%
\pgfpathlineto{\pgfqpoint{2.919275in}{1.494002in}}%
\pgfpathlineto{\pgfqpoint{2.912239in}{1.498667in}}%
\pgfpathlineto{\pgfqpoint{2.884202in}{1.522936in}}%
\pgfpathlineto{\pgfqpoint{2.878090in}{1.530307in}}%
\pgfpathlineto{\pgfqpoint{2.870306in}{1.536000in}}%
\pgfpathlineto{\pgfqpoint{2.844121in}{1.561126in}}%
\pgfpathlineto{\pgfqpoint{2.838667in}{1.568253in}}%
\pgfpathlineto{\pgfqpoint{2.832333in}{1.573333in}}%
\pgfpathlineto{\pgfqpoint{2.797506in}{1.610667in}}%
\pgfpathlineto{\pgfqpoint{2.763960in}{1.650316in}}%
\pgfpathlineto{\pgfqpoint{2.723879in}{1.707818in}}%
\pgfpathlineto{\pgfqpoint{2.718845in}{1.717978in}}%
\pgfpathlineto{\pgfqpoint{2.714510in}{1.722667in}}%
\pgfpathlineto{\pgfqpoint{2.702175in}{1.742883in}}%
\pgfpathlineto{\pgfqpoint{2.694717in}{1.760000in}}%
\pgfpathlineto{\pgfqpoint{2.694687in}{1.770142in}}%
\pgfpathlineto{\pgfqpoint{2.683798in}{1.786824in}}%
\pgfpathlineto{\pgfqpoint{2.678116in}{1.802626in}}%
\pgfpathlineto{\pgfqpoint{2.679245in}{1.834667in}}%
\pgfpathlineto{\pgfqpoint{2.683798in}{1.841909in}}%
\pgfpathlineto{\pgfqpoint{2.704258in}{1.852943in}}%
\pgfpathlineto{\pgfqpoint{2.723879in}{1.850004in}}%
\pgfpathlineto{\pgfqpoint{2.733232in}{1.843379in}}%
\pgfpathlineto{\pgfqpoint{2.763960in}{1.837687in}}%
\pgfpathlineto{\pgfqpoint{2.769299in}{1.834667in}}%
\pgfpathlineto{\pgfqpoint{2.787371in}{1.819140in}}%
\pgfpathlineto{\pgfqpoint{2.804040in}{1.814561in}}%
\pgfpathlineto{\pgfqpoint{2.813110in}{1.805782in}}%
\pgfpathlineto{\pgfqpoint{2.829989in}{1.797333in}}%
\pgfpathlineto{\pgfqpoint{2.836732in}{1.790451in}}%
\pgfpathlineto{\pgfqpoint{2.844121in}{1.787728in}}%
\pgfpathlineto{\pgfqpoint{2.884202in}{1.756920in}}%
\pgfpathlineto{\pgfqpoint{2.924283in}{1.721839in}}%
\pgfpathlineto{\pgfqpoint{2.964364in}{1.682159in}}%
\pgfpathlineto{\pgfqpoint{3.004444in}{1.637518in}}%
\pgfpathlineto{\pgfqpoint{3.044525in}{1.587517in}}%
\pgfpathlineto{\pgfqpoint{3.049880in}{1.578321in}}%
\pgfpathlineto{\pgfqpoint{3.054920in}{1.573333in}}%
\pgfpathlineto{\pgfqpoint{3.080479in}{1.536000in}}%
\pgfpathlineto{\pgfqpoint{3.084606in}{1.528641in}}%
\pgfpathlineto{\pgfqpoint{3.099919in}{1.498667in}}%
\pgfpathlineto{\pgfqpoint{3.102871in}{1.478346in}}%
\pgfpathlineto{\pgfqpoint{3.115011in}{1.461333in}}%
\pgfpathlineto{\pgfqpoint{3.121034in}{1.427402in}}%
\pgfpathlineto{\pgfqpoint{3.118974in}{1.424000in}}%
\pgfpathlineto{\pgfqpoint{3.105745in}{1.404310in}}%
\pgfpathlineto{\pgfqpoint{3.084606in}{1.402651in}}%
\pgfpathlineto{\pgfqpoint{3.065975in}{1.404021in}}%
\pgfpathmoveto{\pgfqpoint{2.923114in}{1.422911in}}%
\pgfpathlineto{\pgfqpoint{2.924283in}{1.421719in}}%
\pgfpathlineto{\pgfqpoint{2.968605in}{1.386667in}}%
\pgfpathlineto{\pgfqpoint{2.990477in}{1.373656in}}%
\pgfpathlineto{\pgfqpoint{3.004444in}{1.360895in}}%
\pgfpathlineto{\pgfqpoint{3.012694in}{1.357017in}}%
\pgfpathlineto{\pgfqpoint{3.021841in}{1.349333in}}%
\pgfpathlineto{\pgfqpoint{3.044525in}{1.334477in}}%
\pgfpathlineto{\pgfqpoint{3.084606in}{1.310405in}}%
\pgfpathlineto{\pgfqpoint{3.112481in}{1.300631in}}%
\pgfpathlineto{\pgfqpoint{3.124687in}{1.292369in}}%
\pgfpathlineto{\pgfqpoint{3.164768in}{1.278335in}}%
\pgfpathlineto{\pgfqpoint{3.169954in}{1.279498in}}%
\pgfpathlineto{\pgfqpoint{3.205145in}{1.274667in}}%
\pgfpathlineto{\pgfqpoint{3.231481in}{1.287193in}}%
\pgfpathlineto{\pgfqpoint{3.243945in}{1.312000in}}%
\pgfpathlineto{\pgfqpoint{3.241336in}{1.345987in}}%
\pgfpathlineto{\pgfqpoint{3.242173in}{1.351901in}}%
\pgfpathlineto{\pgfqpoint{3.224106in}{1.406063in}}%
\pgfpathlineto{\pgfqpoint{3.204848in}{1.444579in}}%
\pgfpathlineto{\pgfqpoint{3.197859in}{1.454823in}}%
\pgfpathlineto{\pgfqpoint{3.195847in}{1.461333in}}%
\pgfpathlineto{\pgfqpoint{3.182753in}{1.478085in}}%
\pgfpathlineto{\pgfqpoint{3.173358in}{1.498667in}}%
\pgfpathlineto{\pgfqpoint{3.169572in}{1.503142in}}%
\pgfpathlineto{\pgfqpoint{3.164768in}{1.511723in}}%
\pgfpathlineto{\pgfqpoint{3.153263in}{1.525284in}}%
\pgfpathlineto{\pgfqpoint{3.147969in}{1.536000in}}%
\pgfpathlineto{\pgfqpoint{3.137327in}{1.547774in}}%
\pgfpathlineto{\pgfqpoint{3.121445in}{1.573333in}}%
\pgfpathlineto{\pgfqpoint{3.103979in}{1.591378in}}%
\pgfpathlineto{\pgfqpoint{3.091718in}{1.610667in}}%
\pgfpathlineto{\pgfqpoint{3.084606in}{1.619311in}}%
\pgfpathlineto{\pgfqpoint{3.069456in}{1.633888in}}%
\pgfpathlineto{\pgfqpoint{3.059874in}{1.648000in}}%
\pgfpathlineto{\pgfqpoint{3.044525in}{1.665540in}}%
\pgfpathlineto{\pgfqpoint{3.033679in}{1.675230in}}%
\pgfpathlineto{\pgfqpoint{3.026337in}{1.685333in}}%
\pgfpathlineto{\pgfqpoint{3.015135in}{1.695291in}}%
\pgfpathlineto{\pgfqpoint{3.004444in}{1.708808in}}%
\pgfpathlineto{\pgfqpoint{2.996561in}{1.715324in}}%
\pgfpathlineto{\pgfqpoint{2.990838in}{1.722667in}}%
\pgfpathlineto{\pgfqpoint{2.976872in}{1.734318in}}%
\pgfpathlineto{\pgfqpoint{2.964364in}{1.749242in}}%
\pgfpathlineto{\pgfqpoint{2.953049in}{1.760000in}}%
\pgfpathlineto{\pgfqpoint{2.937401in}{1.772219in}}%
\pgfpathlineto{\pgfqpoint{2.924283in}{1.786963in}}%
\pgfpathlineto{\pgfqpoint{2.917920in}{1.791407in}}%
\pgfpathlineto{\pgfqpoint{2.912566in}{1.797333in}}%
\pgfpathlineto{\pgfqpoint{2.868879in}{1.834667in}}%
\pgfpathlineto{\pgfqpoint{2.844121in}{1.854715in}}%
\pgfpathlineto{\pgfqpoint{2.832667in}{1.861331in}}%
\pgfpathlineto{\pgfqpoint{2.821347in}{1.872000in}}%
\pgfpathlineto{\pgfqpoint{2.811010in}{1.878492in}}%
\pgfpathlineto{\pgfqpoint{2.804040in}{1.884956in}}%
\pgfpathlineto{\pgfqpoint{2.763960in}{1.912904in}}%
\pgfpathlineto{\pgfqpoint{2.739749in}{1.924116in}}%
\pgfpathlineto{\pgfqpoint{2.723879in}{1.936641in}}%
\pgfpathlineto{\pgfqpoint{2.715688in}{1.939037in}}%
\pgfpathlineto{\pgfqpoint{2.704236in}{1.946667in}}%
\pgfpathlineto{\pgfqpoint{2.683798in}{1.956928in}}%
\pgfpathlineto{\pgfqpoint{2.662831in}{1.966196in}}%
\pgfpathlineto{\pgfqpoint{2.643717in}{1.971850in}}%
\pgfpathlineto{\pgfqpoint{2.629769in}{1.971008in}}%
\pgfpathlineto{\pgfqpoint{2.603636in}{1.979126in}}%
\pgfpathlineto{\pgfqpoint{2.574880in}{1.973452in}}%
\pgfpathlineto{\pgfqpoint{2.556585in}{1.953160in}}%
\pgfpathlineto{\pgfqpoint{2.552797in}{1.919354in}}%
\pgfpathlineto{\pgfqpoint{2.563556in}{1.871500in}}%
\pgfpathlineto{\pgfqpoint{2.579959in}{1.834667in}}%
\pgfpathlineto{\pgfqpoint{2.589312in}{1.821324in}}%
\pgfpathlineto{\pgfqpoint{2.600599in}{1.794504in}}%
\pgfpathlineto{\pgfqpoint{2.603636in}{1.788393in}}%
\pgfpathlineto{\pgfqpoint{2.621208in}{1.760000in}}%
\pgfpathlineto{\pgfqpoint{2.631083in}{1.748232in}}%
\pgfpathlineto{\pgfqpoint{2.647042in}{1.719570in}}%
\pgfpathlineto{\pgfqpoint{2.702082in}{1.648000in}}%
\pgfpathlineto{\pgfqpoint{2.732933in}{1.610667in}}%
\pgfpathlineto{\pgfqpoint{2.776021in}{1.562099in}}%
\pgfpathlineto{\pgfqpoint{2.804040in}{1.532852in}}%
\pgfpathlineto{\pgfqpoint{2.844121in}{1.493464in}}%
\pgfpathlineto{\pgfqpoint{2.863140in}{1.479048in}}%
\pgfpathlineto{\pgfqpoint{2.884202in}{1.456449in}}%
\pgfpathlineto{\pgfqpoint{2.921620in}{1.424000in}}%
\pgfpathlineto{\pgfqpoint{2.921620in}{1.424000in}}%
\pgfusepath{fill}%
\end{pgfscope}%
\begin{pgfscope}%
\pgfpathrectangle{\pgfqpoint{0.800000in}{0.528000in}}{\pgfqpoint{3.968000in}{3.696000in}}%
\pgfusepath{clip}%
\pgfsetbuttcap%
\pgfsetroundjoin%
\definecolor{currentfill}{rgb}{0.267004,0.004874,0.329415}%
\pgfsetfillcolor{currentfill}%
\pgfsetlinewidth{0.000000pt}%
\definecolor{currentstroke}{rgb}{0.000000,0.000000,0.000000}%
\pgfsetstrokecolor{currentstroke}%
\pgfsetdash{}{0pt}%
\pgfpathmoveto{\pgfqpoint{2.921620in}{1.424000in}}%
\pgfpathlineto{\pgfqpoint{2.878855in}{1.461333in}}%
\pgfpathlineto{\pgfqpoint{2.863140in}{1.479048in}}%
\pgfpathlineto{\pgfqpoint{2.838768in}{1.498667in}}%
\pgfpathlineto{\pgfqpoint{2.800990in}{1.536000in}}%
\pgfpathlineto{\pgfqpoint{2.763960in}{1.575043in}}%
\pgfpathlineto{\pgfqpoint{2.723879in}{1.621234in}}%
\pgfpathlineto{\pgfqpoint{2.683798in}{1.670829in}}%
\pgfpathlineto{\pgfqpoint{2.677908in}{1.679847in}}%
\pgfpathlineto{\pgfqpoint{2.672729in}{1.685333in}}%
\pgfpathlineto{\pgfqpoint{2.643717in}{1.724465in}}%
\pgfpathlineto{\pgfqpoint{2.631083in}{1.748232in}}%
\pgfpathlineto{\pgfqpoint{2.621208in}{1.760000in}}%
\pgfpathlineto{\pgfqpoint{2.598385in}{1.797333in}}%
\pgfpathlineto{\pgfqpoint{2.589312in}{1.821324in}}%
\pgfpathlineto{\pgfqpoint{2.579959in}{1.834667in}}%
\pgfpathlineto{\pgfqpoint{2.563220in}{1.872312in}}%
\pgfpathlineto{\pgfqpoint{2.552797in}{1.919354in}}%
\pgfpathlineto{\pgfqpoint{2.556585in}{1.953160in}}%
\pgfpathlineto{\pgfqpoint{2.563556in}{1.961727in}}%
\pgfpathlineto{\pgfqpoint{2.574880in}{1.973452in}}%
\pgfpathlineto{\pgfqpoint{2.603636in}{1.979126in}}%
\pgfpathlineto{\pgfqpoint{2.629769in}{1.971008in}}%
\pgfpathlineto{\pgfqpoint{2.643717in}{1.971850in}}%
\pgfpathlineto{\pgfqpoint{2.662831in}{1.966196in}}%
\pgfpathlineto{\pgfqpoint{2.683798in}{1.956928in}}%
\pgfpathlineto{\pgfqpoint{2.704236in}{1.946667in}}%
\pgfpathlineto{\pgfqpoint{2.715688in}{1.939037in}}%
\pgfpathlineto{\pgfqpoint{2.723879in}{1.936641in}}%
\pgfpathlineto{\pgfqpoint{2.739749in}{1.924116in}}%
\pgfpathlineto{\pgfqpoint{2.769141in}{1.909333in}}%
\pgfpathlineto{\pgfqpoint{2.804040in}{1.884956in}}%
\pgfpathlineto{\pgfqpoint{2.811010in}{1.878492in}}%
\pgfpathlineto{\pgfqpoint{2.821347in}{1.872000in}}%
\pgfpathlineto{\pgfqpoint{2.832667in}{1.861331in}}%
\pgfpathlineto{\pgfqpoint{2.844121in}{1.854715in}}%
\pgfpathlineto{\pgfqpoint{2.884202in}{1.822085in}}%
\pgfpathlineto{\pgfqpoint{2.924283in}{1.786963in}}%
\pgfpathlineto{\pgfqpoint{2.937401in}{1.772219in}}%
\pgfpathlineto{\pgfqpoint{2.953049in}{1.760000in}}%
\pgfpathlineto{\pgfqpoint{2.964364in}{1.749242in}}%
\pgfpathlineto{\pgfqpoint{2.976872in}{1.734318in}}%
\pgfpathlineto{\pgfqpoint{2.990838in}{1.722667in}}%
\pgfpathlineto{\pgfqpoint{2.996561in}{1.715324in}}%
\pgfpathlineto{\pgfqpoint{3.004444in}{1.708808in}}%
\pgfpathlineto{\pgfqpoint{3.015135in}{1.695291in}}%
\pgfpathlineto{\pgfqpoint{3.026337in}{1.685333in}}%
\pgfpathlineto{\pgfqpoint{3.033679in}{1.675230in}}%
\pgfpathlineto{\pgfqpoint{3.044525in}{1.665540in}}%
\pgfpathlineto{\pgfqpoint{3.059874in}{1.648000in}}%
\pgfpathlineto{\pgfqpoint{3.069456in}{1.633888in}}%
\pgfpathlineto{\pgfqpoint{3.084606in}{1.619311in}}%
\pgfpathlineto{\pgfqpoint{3.091718in}{1.610667in}}%
\pgfpathlineto{\pgfqpoint{3.103979in}{1.591378in}}%
\pgfpathlineto{\pgfqpoint{3.124687in}{1.568979in}}%
\pgfpathlineto{\pgfqpoint{3.137327in}{1.547774in}}%
\pgfpathlineto{\pgfqpoint{3.147969in}{1.536000in}}%
\pgfpathlineto{\pgfqpoint{3.153263in}{1.525284in}}%
\pgfpathlineto{\pgfqpoint{3.164768in}{1.511723in}}%
\pgfpathlineto{\pgfqpoint{3.169572in}{1.503142in}}%
\pgfpathlineto{\pgfqpoint{3.173358in}{1.498667in}}%
\pgfpathlineto{\pgfqpoint{3.182753in}{1.478085in}}%
\pgfpathlineto{\pgfqpoint{3.195847in}{1.461333in}}%
\pgfpathlineto{\pgfqpoint{3.197859in}{1.454823in}}%
\pgfpathlineto{\pgfqpoint{3.204848in}{1.444579in}}%
\pgfpathlineto{\pgfqpoint{3.224106in}{1.406063in}}%
\pgfpathlineto{\pgfqpoint{3.242173in}{1.351901in}}%
\pgfpathlineto{\pgfqpoint{3.242369in}{1.349333in}}%
\pgfpathlineto{\pgfqpoint{3.241336in}{1.345987in}}%
\pgfpathlineto{\pgfqpoint{3.243945in}{1.312000in}}%
\pgfpathlineto{\pgfqpoint{3.231481in}{1.287193in}}%
\pgfpathlineto{\pgfqpoint{3.203858in}{1.274667in}}%
\pgfpathlineto{\pgfqpoint{3.169954in}{1.279498in}}%
\pgfpathlineto{\pgfqpoint{3.164768in}{1.278335in}}%
\pgfpathlineto{\pgfqpoint{3.159867in}{1.279231in}}%
\pgfpathlineto{\pgfqpoint{3.124687in}{1.292369in}}%
\pgfpathlineto{\pgfqpoint{3.112481in}{1.300631in}}%
\pgfpathlineto{\pgfqpoint{3.081914in}{1.312000in}}%
\pgfpathlineto{\pgfqpoint{3.021841in}{1.349333in}}%
\pgfpathlineto{\pgfqpoint{3.012694in}{1.357017in}}%
\pgfpathlineto{\pgfqpoint{3.004444in}{1.360895in}}%
\pgfpathlineto{\pgfqpoint{2.990477in}{1.373656in}}%
\pgfpathlineto{\pgfqpoint{2.964364in}{1.389763in}}%
\pgfpathlineto{\pgfqpoint{2.923114in}{1.422911in}}%
\pgfpathmoveto{\pgfqpoint{2.897702in}{1.399241in}}%
\pgfpathlineto{\pgfqpoint{2.910037in}{1.386667in}}%
\pgfpathlineto{\pgfqpoint{2.917936in}{1.380755in}}%
\pgfpathlineto{\pgfqpoint{2.924283in}{1.374435in}}%
\pgfpathlineto{\pgfqpoint{2.964364in}{1.341645in}}%
\pgfpathlineto{\pgfqpoint{2.983255in}{1.329596in}}%
\pgfpathlineto{\pgfqpoint{3.011069in}{1.305829in}}%
\pgfpathlineto{\pgfqpoint{3.056768in}{1.274667in}}%
\pgfpathlineto{\pgfqpoint{3.084606in}{1.256794in}}%
\pgfpathlineto{\pgfqpoint{3.136376in}{1.226445in}}%
\pgfpathlineto{\pgfqpoint{3.164768in}{1.213015in}}%
\pgfpathlineto{\pgfqpoint{3.175898in}{1.210367in}}%
\pgfpathlineto{\pgfqpoint{3.204848in}{1.196171in}}%
\pgfpathlineto{\pgfqpoint{3.211608in}{1.193704in}}%
\pgfpathlineto{\pgfqpoint{3.244929in}{1.186098in}}%
\pgfpathlineto{\pgfqpoint{3.263074in}{1.183099in}}%
\pgfpathlineto{\pgfqpoint{3.285010in}{1.184824in}}%
\pgfpathlineto{\pgfqpoint{3.318133in}{1.200000in}}%
\pgfpathlineto{\pgfqpoint{3.325091in}{1.209354in}}%
\pgfpathlineto{\pgfqpoint{3.333346in}{1.229644in}}%
\pgfpathlineto{\pgfqpoint{3.333499in}{1.237333in}}%
\pgfpathlineto{\pgfqpoint{3.331756in}{1.243542in}}%
\pgfpathlineto{\pgfqpoint{3.331756in}{1.274667in}}%
\pgfpathlineto{\pgfqpoint{3.322722in}{1.312000in}}%
\pgfpathlineto{\pgfqpoint{3.310643in}{1.335876in}}%
\pgfpathlineto{\pgfqpoint{3.307668in}{1.349333in}}%
\pgfpathlineto{\pgfqpoint{3.290656in}{1.386667in}}%
\pgfpathlineto{\pgfqpoint{3.285010in}{1.397150in}}%
\pgfpathlineto{\pgfqpoint{3.273938in}{1.413687in}}%
\pgfpathlineto{\pgfqpoint{3.269902in}{1.424000in}}%
\pgfpathlineto{\pgfqpoint{3.244929in}{1.465965in}}%
\pgfpathlineto{\pgfqpoint{3.230005in}{1.484765in}}%
\pgfpathlineto{\pgfqpoint{3.222686in}{1.498667in}}%
\pgfpathlineto{\pgfqpoint{3.196553in}{1.536000in}}%
\pgfpathlineto{\pgfqpoint{3.182133in}{1.552175in}}%
\pgfpathlineto{\pgfqpoint{3.164768in}{1.578299in}}%
\pgfpathlineto{\pgfqpoint{3.106889in}{1.648000in}}%
\pgfpathlineto{\pgfqpoint{3.096039in}{1.658649in}}%
\pgfpathlineto{\pgfqpoint{3.084606in}{1.673656in}}%
\pgfpathlineto{\pgfqpoint{3.074050in}{1.685333in}}%
\pgfpathlineto{\pgfqpoint{3.059302in}{1.699097in}}%
\pgfpathlineto{\pgfqpoint{3.039690in}{1.722667in}}%
\pgfpathlineto{\pgfqpoint{3.021627in}{1.738671in}}%
\pgfpathlineto{\pgfqpoint{3.003607in}{1.760000in}}%
\pgfpathlineto{\pgfqpoint{2.964364in}{1.798284in}}%
\pgfpathlineto{\pgfqpoint{2.924283in}{1.835401in}}%
\pgfpathlineto{\pgfqpoint{2.902515in}{1.851725in}}%
\pgfpathlineto{\pgfqpoint{2.882367in}{1.872000in}}%
\pgfpathlineto{\pgfqpoint{2.836453in}{1.909333in}}%
\pgfpathlineto{\pgfqpoint{2.804040in}{1.934268in}}%
\pgfpathlineto{\pgfqpoint{2.795834in}{1.939022in}}%
\pgfpathlineto{\pgfqpoint{2.787003in}{1.946667in}}%
\pgfpathlineto{\pgfqpoint{2.763960in}{1.963248in}}%
\pgfpathlineto{\pgfqpoint{2.749777in}{1.970790in}}%
\pgfpathlineto{\pgfqpoint{2.723879in}{1.990371in}}%
\pgfpathlineto{\pgfqpoint{2.683798in}{2.014584in}}%
\pgfpathlineto{\pgfqpoint{2.678550in}{2.016445in}}%
\pgfpathlineto{\pgfqpoint{2.670980in}{2.021333in}}%
\pgfpathlineto{\pgfqpoint{2.643717in}{2.035504in}}%
\pgfpathlineto{\pgfqpoint{2.624947in}{2.041183in}}%
\pgfpathlineto{\pgfqpoint{2.603636in}{2.053489in}}%
\pgfpathlineto{\pgfqpoint{2.598637in}{2.054010in}}%
\pgfpathlineto{\pgfqpoint{2.587031in}{2.058667in}}%
\pgfpathlineto{\pgfqpoint{2.563556in}{2.065873in}}%
\pgfpathlineto{\pgfqpoint{2.532717in}{2.067275in}}%
\pgfpathlineto{\pgfqpoint{2.523475in}{2.070231in}}%
\pgfpathlineto{\pgfqpoint{2.511256in}{2.070047in}}%
\pgfpathlineto{\pgfqpoint{2.482620in}{2.058667in}}%
\pgfpathlineto{\pgfqpoint{2.467326in}{2.036300in}}%
\pgfpathlineto{\pgfqpoint{2.464972in}{2.021333in}}%
\pgfpathlineto{\pgfqpoint{2.467708in}{2.006722in}}%
\pgfpathlineto{\pgfqpoint{2.465357in}{1.984000in}}%
\pgfpathlineto{\pgfqpoint{2.470333in}{1.971835in}}%
\pgfpathlineto{\pgfqpoint{2.473639in}{1.946667in}}%
\pgfpathlineto{\pgfqpoint{2.476935in}{1.940650in}}%
\pgfpathlineto{\pgfqpoint{2.486759in}{1.909333in}}%
\pgfpathlineto{\pgfqpoint{2.499826in}{1.887306in}}%
\pgfpathlineto{\pgfqpoint{2.504695in}{1.872000in}}%
\pgfpathlineto{\pgfqpoint{2.523860in}{1.834308in}}%
\pgfpathlineto{\pgfqpoint{2.563556in}{1.771121in}}%
\pgfpathlineto{\pgfqpoint{2.570919in}{1.760000in}}%
\pgfpathlineto{\pgfqpoint{2.597086in}{1.722667in}}%
\pgfpathlineto{\pgfqpoint{2.603636in}{1.713783in}}%
\pgfpathlineto{\pgfqpoint{2.617473in}{1.698221in}}%
\pgfpathlineto{\pgfqpoint{2.625428in}{1.685333in}}%
\pgfpathlineto{\pgfqpoint{2.654850in}{1.648000in}}%
\pgfpathlineto{\pgfqpoint{2.668617in}{1.633860in}}%
\pgfpathlineto{\pgfqpoint{2.685488in}{1.610667in}}%
\pgfpathlineto{\pgfqpoint{2.704249in}{1.592382in}}%
\pgfpathlineto{\pgfqpoint{2.723879in}{1.567447in}}%
\pgfpathlineto{\pgfqpoint{2.740833in}{1.551792in}}%
\pgfpathlineto{\pgfqpoint{2.753320in}{1.536000in}}%
\pgfpathlineto{\pgfqpoint{2.763960in}{1.524759in}}%
\pgfpathlineto{\pgfqpoint{2.778418in}{1.512134in}}%
\pgfpathlineto{\pgfqpoint{2.789632in}{1.498667in}}%
\pgfpathlineto{\pgfqpoint{2.827698in}{1.461333in}}%
\pgfpathlineto{\pgfqpoint{2.836380in}{1.454123in}}%
\pgfpathlineto{\pgfqpoint{2.844121in}{1.445642in}}%
\pgfpathlineto{\pgfqpoint{2.884202in}{1.409081in}}%
\pgfpathlineto{\pgfqpoint{2.884202in}{1.409081in}}%
\pgfusepath{fill}%
\end{pgfscope}%
\begin{pgfscope}%
\pgfpathrectangle{\pgfqpoint{0.800000in}{0.528000in}}{\pgfqpoint{3.968000in}{3.696000in}}%
\pgfusepath{clip}%
\pgfsetbuttcap%
\pgfsetroundjoin%
\definecolor{currentfill}{rgb}{0.267004,0.004874,0.329415}%
\pgfsetfillcolor{currentfill}%
\pgfsetlinewidth{0.000000pt}%
\definecolor{currentstroke}{rgb}{0.000000,0.000000,0.000000}%
\pgfsetstrokecolor{currentstroke}%
\pgfsetdash{}{0pt}%
\pgfpathmoveto{\pgfqpoint{2.884202in}{1.409081in}}%
\pgfpathlineto{\pgfqpoint{2.876654in}{1.416970in}}%
\pgfpathlineto{\pgfqpoint{2.867744in}{1.424000in}}%
\pgfpathlineto{\pgfqpoint{2.827698in}{1.461333in}}%
\pgfpathlineto{\pgfqpoint{2.789632in}{1.498667in}}%
\pgfpathlineto{\pgfqpoint{2.778418in}{1.512134in}}%
\pgfpathlineto{\pgfqpoint{2.763960in}{1.524759in}}%
\pgfpathlineto{\pgfqpoint{2.753320in}{1.536000in}}%
\pgfpathlineto{\pgfqpoint{2.740833in}{1.551792in}}%
\pgfpathlineto{\pgfqpoint{2.718569in}{1.573333in}}%
\pgfpathlineto{\pgfqpoint{2.704249in}{1.592382in}}%
\pgfpathlineto{\pgfqpoint{2.683798in}{1.612647in}}%
\pgfpathlineto{\pgfqpoint{2.668617in}{1.633860in}}%
\pgfpathlineto{\pgfqpoint{2.654850in}{1.648000in}}%
\pgfpathlineto{\pgfqpoint{2.650639in}{1.654447in}}%
\pgfpathlineto{\pgfqpoint{2.643717in}{1.661780in}}%
\pgfpathlineto{\pgfqpoint{2.625428in}{1.685333in}}%
\pgfpathlineto{\pgfqpoint{2.617473in}{1.698221in}}%
\pgfpathlineto{\pgfqpoint{2.597086in}{1.722667in}}%
\pgfpathlineto{\pgfqpoint{2.570919in}{1.760000in}}%
\pgfpathlineto{\pgfqpoint{2.568589in}{1.764689in}}%
\pgfpathlineto{\pgfqpoint{2.563556in}{1.771121in}}%
\pgfpathlineto{\pgfqpoint{2.523475in}{1.835022in}}%
\pgfpathlineto{\pgfqpoint{2.504695in}{1.872000in}}%
\pgfpathlineto{\pgfqpoint{2.499826in}{1.887306in}}%
\pgfpathlineto{\pgfqpoint{2.486759in}{1.909333in}}%
\pgfpathlineto{\pgfqpoint{2.483394in}{1.918530in}}%
\pgfpathlineto{\pgfqpoint{2.476935in}{1.940650in}}%
\pgfpathlineto{\pgfqpoint{2.473639in}{1.946667in}}%
\pgfpathlineto{\pgfqpoint{2.470333in}{1.971835in}}%
\pgfpathlineto{\pgfqpoint{2.465357in}{1.984000in}}%
\pgfpathlineto{\pgfqpoint{2.467708in}{2.006722in}}%
\pgfpathlineto{\pgfqpoint{2.464972in}{2.021333in}}%
\pgfpathlineto{\pgfqpoint{2.467326in}{2.036300in}}%
\pgfpathlineto{\pgfqpoint{2.483394in}{2.059325in}}%
\pgfpathlineto{\pgfqpoint{2.511256in}{2.070047in}}%
\pgfpathlineto{\pgfqpoint{2.523475in}{2.070231in}}%
\pgfpathlineto{\pgfqpoint{2.532717in}{2.067275in}}%
\pgfpathlineto{\pgfqpoint{2.563556in}{2.065873in}}%
\pgfpathlineto{\pgfqpoint{2.603636in}{2.053489in}}%
\pgfpathlineto{\pgfqpoint{2.624947in}{2.041183in}}%
\pgfpathlineto{\pgfqpoint{2.643717in}{2.035504in}}%
\pgfpathlineto{\pgfqpoint{2.702721in}{2.003708in}}%
\pgfpathlineto{\pgfqpoint{2.733374in}{1.984000in}}%
\pgfpathlineto{\pgfqpoint{2.749777in}{1.970790in}}%
\pgfpathlineto{\pgfqpoint{2.763960in}{1.963248in}}%
\pgfpathlineto{\pgfqpoint{2.804040in}{1.934268in}}%
\pgfpathlineto{\pgfqpoint{2.844121in}{1.903367in}}%
\pgfpathlineto{\pgfqpoint{2.884202in}{1.870477in}}%
\pgfpathlineto{\pgfqpoint{2.902515in}{1.851725in}}%
\pgfpathlineto{\pgfqpoint{2.925084in}{1.834667in}}%
\pgfpathlineto{\pgfqpoint{2.987502in}{1.775781in}}%
\pgfpathlineto{\pgfqpoint{3.004444in}{1.759170in}}%
\pgfpathlineto{\pgfqpoint{3.021627in}{1.738671in}}%
\pgfpathlineto{\pgfqpoint{3.044525in}{1.717598in}}%
\pgfpathlineto{\pgfqpoint{3.059302in}{1.699097in}}%
\pgfpathlineto{\pgfqpoint{3.074050in}{1.685333in}}%
\pgfpathlineto{\pgfqpoint{3.084606in}{1.673656in}}%
\pgfpathlineto{\pgfqpoint{3.096039in}{1.658649in}}%
\pgfpathlineto{\pgfqpoint{3.106889in}{1.648000in}}%
\pgfpathlineto{\pgfqpoint{3.138379in}{1.610667in}}%
\pgfpathlineto{\pgfqpoint{3.168668in}{1.573333in}}%
\pgfpathlineto{\pgfqpoint{3.182133in}{1.552175in}}%
\pgfpathlineto{\pgfqpoint{3.196553in}{1.536000in}}%
\pgfpathlineto{\pgfqpoint{3.199396in}{1.530921in}}%
\pgfpathlineto{\pgfqpoint{3.204848in}{1.524477in}}%
\pgfpathlineto{\pgfqpoint{3.222686in}{1.498667in}}%
\pgfpathlineto{\pgfqpoint{3.230005in}{1.484765in}}%
\pgfpathlineto{\pgfqpoint{3.247958in}{1.461333in}}%
\pgfpathlineto{\pgfqpoint{3.252489in}{1.454292in}}%
\pgfpathlineto{\pgfqpoint{3.269902in}{1.424000in}}%
\pgfpathlineto{\pgfqpoint{3.273938in}{1.413687in}}%
\pgfpathlineto{\pgfqpoint{3.285010in}{1.397150in}}%
\pgfpathlineto{\pgfqpoint{3.296100in}{1.376337in}}%
\pgfpathlineto{\pgfqpoint{3.307668in}{1.349333in}}%
\pgfpathlineto{\pgfqpoint{3.310643in}{1.335876in}}%
\pgfpathlineto{\pgfqpoint{3.322722in}{1.312000in}}%
\pgfpathlineto{\pgfqpoint{3.331756in}{1.274667in}}%
\pgfpathlineto{\pgfqpoint{3.331756in}{1.243542in}}%
\pgfpathlineto{\pgfqpoint{3.333499in}{1.237333in}}%
\pgfpathlineto{\pgfqpoint{3.333346in}{1.229644in}}%
\pgfpathlineto{\pgfqpoint{3.325091in}{1.209354in}}%
\pgfpathlineto{\pgfqpoint{3.318133in}{1.200000in}}%
\pgfpathlineto{\pgfqpoint{3.285010in}{1.184824in}}%
\pgfpathlineto{\pgfqpoint{3.263074in}{1.183099in}}%
\pgfpathlineto{\pgfqpoint{3.244929in}{1.186098in}}%
\pgfpathlineto{\pgfqpoint{3.204848in}{1.196171in}}%
\pgfpathlineto{\pgfqpoint{3.202343in}{1.197666in}}%
\pgfpathlineto{\pgfqpoint{3.195639in}{1.200000in}}%
\pgfpathlineto{\pgfqpoint{3.175898in}{1.210367in}}%
\pgfpathlineto{\pgfqpoint{3.164768in}{1.213015in}}%
\pgfpathlineto{\pgfqpoint{3.124687in}{1.232858in}}%
\pgfpathlineto{\pgfqpoint{3.117120in}{1.237333in}}%
\pgfpathlineto{\pgfqpoint{3.056768in}{1.274667in}}%
\pgfpathlineto{\pgfqpoint{3.050111in}{1.279869in}}%
\pgfpathlineto{\pgfqpoint{3.044525in}{1.282621in}}%
\pgfpathlineto{\pgfqpoint{3.002684in}{1.312000in}}%
\pgfpathlineto{\pgfqpoint{2.983255in}{1.329596in}}%
\pgfpathlineto{\pgfqpoint{2.964364in}{1.341645in}}%
\pgfpathlineto{\pgfqpoint{2.910037in}{1.386667in}}%
\pgfpathlineto{\pgfqpoint{2.897702in}{1.399241in}}%
\pgfpathmoveto{\pgfqpoint{2.896272in}{1.360576in}}%
\pgfpathlineto{\pgfqpoint{2.907561in}{1.349333in}}%
\pgfpathlineto{\pgfqpoint{2.916753in}{1.342320in}}%
\pgfpathlineto{\pgfqpoint{2.924283in}{1.334951in}}%
\pgfpathlineto{\pgfqpoint{2.964364in}{1.301900in}}%
\pgfpathlineto{\pgfqpoint{3.004444in}{1.270411in}}%
\pgfpathlineto{\pgfqpoint{3.025621in}{1.257058in}}%
\pgfpathlineto{\pgfqpoint{3.049773in}{1.237333in}}%
\pgfpathlineto{\pgfqpoint{3.071060in}{1.224716in}}%
\pgfpathlineto{\pgfqpoint{3.084606in}{1.213768in}}%
\pgfpathlineto{\pgfqpoint{3.094172in}{1.208910in}}%
\pgfpathlineto{\pgfqpoint{3.106132in}{1.200000in}}%
\pgfpathlineto{\pgfqpoint{3.117755in}{1.193543in}}%
\pgfpathlineto{\pgfqpoint{3.124687in}{1.188250in}}%
\pgfpathlineto{\pgfqpoint{3.168644in}{1.162667in}}%
\pgfpathlineto{\pgfqpoint{3.204848in}{1.144781in}}%
\pgfpathlineto{\pgfqpoint{3.250041in}{1.125333in}}%
\pgfpathlineto{\pgfqpoint{3.285010in}{1.114639in}}%
\pgfpathlineto{\pgfqpoint{3.311343in}{1.112528in}}%
\pgfpathlineto{\pgfqpoint{3.325091in}{1.108179in}}%
\pgfpathlineto{\pgfqpoint{3.344290in}{1.107451in}}%
\pgfpathlineto{\pgfqpoint{3.375607in}{1.115614in}}%
\pgfpathlineto{\pgfqpoint{3.386923in}{1.125333in}}%
\pgfpathlineto{\pgfqpoint{3.405253in}{1.154981in}}%
\pgfpathlineto{\pgfqpoint{3.407534in}{1.162667in}}%
\pgfpathlineto{\pgfqpoint{3.408165in}{1.200000in}}%
\pgfpathlineto{\pgfqpoint{3.407397in}{1.201997in}}%
\pgfpathlineto{\pgfqpoint{3.400001in}{1.242225in}}%
\pgfpathlineto{\pgfqpoint{3.388592in}{1.274667in}}%
\pgfpathlineto{\pgfqpoint{3.380754in}{1.289181in}}%
\pgfpathlineto{\pgfqpoint{3.373614in}{1.312000in}}%
\pgfpathlineto{\pgfqpoint{3.365172in}{1.329784in}}%
\pgfpathlineto{\pgfqpoint{3.357818in}{1.342484in}}%
\pgfpathlineto{\pgfqpoint{3.355510in}{1.349333in}}%
\pgfpathlineto{\pgfqpoint{3.343901in}{1.366854in}}%
\pgfpathlineto{\pgfqpoint{3.335168in}{1.386667in}}%
\pgfpathlineto{\pgfqpoint{3.312787in}{1.424000in}}%
\pgfpathlineto{\pgfqpoint{3.301257in}{1.439133in}}%
\pgfpathlineto{\pgfqpoint{3.285010in}{1.467351in}}%
\pgfpathlineto{\pgfqpoint{3.235984in}{1.536000in}}%
\pgfpathlineto{\pgfqpoint{3.204848in}{1.576908in}}%
\pgfpathlineto{\pgfqpoint{3.164768in}{1.625705in}}%
\pgfpathlineto{\pgfqpoint{3.153406in}{1.637417in}}%
\pgfpathlineto{\pgfqpoint{3.145706in}{1.648000in}}%
\pgfpathlineto{\pgfqpoint{3.113064in}{1.685333in}}%
\pgfpathlineto{\pgfqpoint{3.079163in}{1.722667in}}%
\pgfpathlineto{\pgfqpoint{3.036038in}{1.767905in}}%
\pgfpathlineto{\pgfqpoint{3.004444in}{1.799493in}}%
\pgfpathlineto{\pgfqpoint{2.984630in}{1.816211in}}%
\pgfpathlineto{\pgfqpoint{2.964364in}{1.837912in}}%
\pgfpathlineto{\pgfqpoint{2.924283in}{1.874715in}}%
\pgfpathlineto{\pgfqpoint{2.903883in}{1.890332in}}%
\pgfpathlineto{\pgfqpoint{2.884202in}{1.909943in}}%
\pgfpathlineto{\pgfqpoint{2.862065in}{1.926047in}}%
\pgfpathlineto{\pgfqpoint{2.839732in}{1.946667in}}%
\pgfpathlineto{\pgfqpoint{2.791543in}{1.984000in}}%
\pgfpathlineto{\pgfqpoint{2.775330in}{1.994591in}}%
\pgfpathlineto{\pgfqpoint{2.763960in}{2.004447in}}%
\pgfpathlineto{\pgfqpoint{2.723879in}{2.032640in}}%
\pgfpathlineto{\pgfqpoint{2.706193in}{2.042193in}}%
\pgfpathlineto{\pgfqpoint{2.683798in}{2.059271in}}%
\pgfpathlineto{\pgfqpoint{2.658150in}{2.072110in}}%
\pgfpathlineto{\pgfqpoint{2.643717in}{2.082486in}}%
\pgfpathlineto{\pgfqpoint{2.585590in}{2.112809in}}%
\pgfpathlineto{\pgfqpoint{2.563556in}{2.121932in}}%
\pgfpathlineto{\pgfqpoint{2.553499in}{2.123966in}}%
\pgfpathlineto{\pgfqpoint{2.523475in}{2.136627in}}%
\pgfpathlineto{\pgfqpoint{2.518112in}{2.138328in}}%
\pgfpathlineto{\pgfqpoint{2.483394in}{2.144811in}}%
\pgfpathlineto{\pgfqpoint{2.453487in}{2.142810in}}%
\pgfpathlineto{\pgfqpoint{2.443313in}{2.144505in}}%
\pgfpathlineto{\pgfqpoint{2.415871in}{2.133333in}}%
\pgfpathlineto{\pgfqpoint{2.408045in}{2.128850in}}%
\pgfpathlineto{\pgfqpoint{2.403232in}{2.121653in}}%
\pgfpathlineto{\pgfqpoint{2.391711in}{2.096000in}}%
\pgfpathlineto{\pgfqpoint{2.388124in}{2.072739in}}%
\pgfpathlineto{\pgfqpoint{2.389482in}{2.058667in}}%
\pgfpathlineto{\pgfqpoint{2.392724in}{2.048879in}}%
\pgfpathlineto{\pgfqpoint{2.395287in}{2.021333in}}%
\pgfpathlineto{\pgfqpoint{2.406177in}{1.984000in}}%
\pgfpathlineto{\pgfqpoint{2.418089in}{1.960505in}}%
\pgfpathlineto{\pgfqpoint{2.421799in}{1.946667in}}%
\pgfpathlineto{\pgfqpoint{2.438955in}{1.909333in}}%
\pgfpathlineto{\pgfqpoint{2.443313in}{1.901183in}}%
\pgfpathlineto{\pgfqpoint{2.459516in}{1.872000in}}%
\pgfpathlineto{\pgfqpoint{2.469036in}{1.858627in}}%
\pgfpathlineto{\pgfqpoint{2.483394in}{1.830841in}}%
\pgfpathlineto{\pgfqpoint{2.498200in}{1.811124in}}%
\pgfpathlineto{\pgfqpoint{2.505433in}{1.797333in}}%
\pgfpathlineto{\pgfqpoint{2.513139in}{1.787706in}}%
\pgfpathlineto{\pgfqpoint{2.530589in}{1.760000in}}%
\pgfpathlineto{\pgfqpoint{2.545051in}{1.742764in}}%
\pgfpathlineto{\pgfqpoint{2.563556in}{1.714660in}}%
\pgfpathlineto{\pgfqpoint{2.616090in}{1.648000in}}%
\pgfpathlineto{\pgfqpoint{2.659671in}{1.595806in}}%
\pgfpathlineto{\pgfqpoint{2.713821in}{1.536000in}}%
\pgfpathlineto{\pgfqpoint{2.723879in}{1.525192in}}%
\pgfpathlineto{\pgfqpoint{2.786381in}{1.461333in}}%
\pgfpathlineto{\pgfqpoint{2.795500in}{1.453378in}}%
\pgfpathlineto{\pgfqpoint{2.804040in}{1.443914in}}%
\pgfpathlineto{\pgfqpoint{2.815343in}{1.434528in}}%
\pgfpathlineto{\pgfqpoint{2.824952in}{1.424000in}}%
\pgfpathlineto{\pgfqpoint{2.844121in}{1.405913in}}%
\pgfpathlineto{\pgfqpoint{2.855308in}{1.397087in}}%
\pgfpathlineto{\pgfqpoint{2.865278in}{1.386667in}}%
\pgfpathlineto{\pgfqpoint{2.884202in}{1.369608in}}%
\pgfpathlineto{\pgfqpoint{2.884202in}{1.369608in}}%
\pgfusepath{fill}%
\end{pgfscope}%
\begin{pgfscope}%
\pgfpathrectangle{\pgfqpoint{0.800000in}{0.528000in}}{\pgfqpoint{3.968000in}{3.696000in}}%
\pgfusepath{clip}%
\pgfsetbuttcap%
\pgfsetroundjoin%
\definecolor{currentfill}{rgb}{0.268510,0.009605,0.335427}%
\pgfsetfillcolor{currentfill}%
\pgfsetlinewidth{0.000000pt}%
\definecolor{currentstroke}{rgb}{0.000000,0.000000,0.000000}%
\pgfsetstrokecolor{currentstroke}%
\pgfsetdash{}{0pt}%
\pgfpathmoveto{\pgfqpoint{2.884202in}{1.369608in}}%
\pgfpathlineto{\pgfqpoint{2.824952in}{1.424000in}}%
\pgfpathlineto{\pgfqpoint{2.815343in}{1.434528in}}%
\pgfpathlineto{\pgfqpoint{2.804040in}{1.443914in}}%
\pgfpathlineto{\pgfqpoint{2.795500in}{1.453378in}}%
\pgfpathlineto{\pgfqpoint{2.786381in}{1.461333in}}%
\pgfpathlineto{\pgfqpoint{2.749388in}{1.498667in}}%
\pgfpathlineto{\pgfqpoint{2.713821in}{1.536000in}}%
\pgfpathlineto{\pgfqpoint{2.659671in}{1.595806in}}%
\pgfpathlineto{\pgfqpoint{2.616090in}{1.648000in}}%
\pgfpathlineto{\pgfqpoint{2.603636in}{1.663324in}}%
\pgfpathlineto{\pgfqpoint{2.594285in}{1.676623in}}%
\pgfpathlineto{\pgfqpoint{2.586340in}{1.685333in}}%
\pgfpathlineto{\pgfqpoint{2.557534in}{1.722667in}}%
\pgfpathlineto{\pgfqpoint{2.545051in}{1.742764in}}%
\pgfpathlineto{\pgfqpoint{2.530589in}{1.760000in}}%
\pgfpathlineto{\pgfqpoint{2.523475in}{1.770292in}}%
\pgfpathlineto{\pgfqpoint{2.513139in}{1.787706in}}%
\pgfpathlineto{\pgfqpoint{2.505433in}{1.797333in}}%
\pgfpathlineto{\pgfqpoint{2.498200in}{1.811124in}}%
\pgfpathlineto{\pgfqpoint{2.480962in}{1.834667in}}%
\pgfpathlineto{\pgfqpoint{2.469036in}{1.858627in}}%
\pgfpathlineto{\pgfqpoint{2.459516in}{1.872000in}}%
\pgfpathlineto{\pgfqpoint{2.434787in}{1.917275in}}%
\pgfpathlineto{\pgfqpoint{2.421799in}{1.946667in}}%
\pgfpathlineto{\pgfqpoint{2.418089in}{1.960505in}}%
\pgfpathlineto{\pgfqpoint{2.405901in}{1.986486in}}%
\pgfpathlineto{\pgfqpoint{2.403232in}{1.993741in}}%
\pgfpathlineto{\pgfqpoint{2.395287in}{2.021333in}}%
\pgfpathlineto{\pgfqpoint{2.392724in}{2.048879in}}%
\pgfpathlineto{\pgfqpoint{2.389482in}{2.058667in}}%
\pgfpathlineto{\pgfqpoint{2.388124in}{2.072739in}}%
\pgfpathlineto{\pgfqpoint{2.391711in}{2.096000in}}%
\pgfpathlineto{\pgfqpoint{2.403232in}{2.121653in}}%
\pgfpathlineto{\pgfqpoint{2.408045in}{2.128850in}}%
\pgfpathlineto{\pgfqpoint{2.415871in}{2.133333in}}%
\pgfpathlineto{\pgfqpoint{2.443313in}{2.144505in}}%
\pgfpathlineto{\pgfqpoint{2.453487in}{2.142810in}}%
\pgfpathlineto{\pgfqpoint{2.483394in}{2.144811in}}%
\pgfpathlineto{\pgfqpoint{2.523475in}{2.136627in}}%
\pgfpathlineto{\pgfqpoint{2.532545in}{2.133333in}}%
\pgfpathlineto{\pgfqpoint{2.553499in}{2.123966in}}%
\pgfpathlineto{\pgfqpoint{2.563556in}{2.121932in}}%
\pgfpathlineto{\pgfqpoint{2.585590in}{2.112809in}}%
\pgfpathlineto{\pgfqpoint{2.618558in}{2.096000in}}%
\pgfpathlineto{\pgfqpoint{2.643717in}{2.082486in}}%
\pgfpathlineto{\pgfqpoint{2.658150in}{2.072110in}}%
\pgfpathlineto{\pgfqpoint{2.684715in}{2.058667in}}%
\pgfpathlineto{\pgfqpoint{2.706193in}{2.042193in}}%
\pgfpathlineto{\pgfqpoint{2.723879in}{2.032640in}}%
\pgfpathlineto{\pgfqpoint{2.763960in}{2.004447in}}%
\pgfpathlineto{\pgfqpoint{2.775330in}{1.994591in}}%
\pgfpathlineto{\pgfqpoint{2.804040in}{1.974647in}}%
\pgfpathlineto{\pgfqpoint{2.844121in}{1.943192in}}%
\pgfpathlineto{\pgfqpoint{2.862065in}{1.926047in}}%
\pgfpathlineto{\pgfqpoint{2.884902in}{1.909333in}}%
\pgfpathlineto{\pgfqpoint{2.903883in}{1.890332in}}%
\pgfpathlineto{\pgfqpoint{2.927263in}{1.872000in}}%
\pgfpathlineto{\pgfqpoint{2.967776in}{1.834667in}}%
\pgfpathlineto{\pgfqpoint{2.984630in}{1.816211in}}%
\pgfpathlineto{\pgfqpoint{3.006622in}{1.797333in}}%
\pgfpathlineto{\pgfqpoint{3.044525in}{1.759324in}}%
\pgfpathlineto{\pgfqpoint{3.113064in}{1.685333in}}%
\pgfpathlineto{\pgfqpoint{3.117889in}{1.679001in}}%
\pgfpathlineto{\pgfqpoint{3.124687in}{1.672328in}}%
\pgfpathlineto{\pgfqpoint{3.177211in}{1.610667in}}%
\pgfpathlineto{\pgfqpoint{3.215386in}{1.563518in}}%
\pgfpathlineto{\pgfqpoint{3.244929in}{1.523895in}}%
\pgfpathlineto{\pgfqpoint{3.255014in}{1.508060in}}%
\pgfpathlineto{\pgfqpoint{3.262941in}{1.498667in}}%
\pgfpathlineto{\pgfqpoint{3.289112in}{1.461333in}}%
\pgfpathlineto{\pgfqpoint{3.301257in}{1.439133in}}%
\pgfpathlineto{\pgfqpoint{3.312787in}{1.424000in}}%
\pgfpathlineto{\pgfqpoint{3.335168in}{1.386667in}}%
\pgfpathlineto{\pgfqpoint{3.343901in}{1.366854in}}%
\pgfpathlineto{\pgfqpoint{3.355510in}{1.349333in}}%
\pgfpathlineto{\pgfqpoint{3.357818in}{1.342484in}}%
\pgfpathlineto{\pgfqpoint{3.365172in}{1.329784in}}%
\pgfpathlineto{\pgfqpoint{3.373614in}{1.312000in}}%
\pgfpathlineto{\pgfqpoint{3.380754in}{1.289181in}}%
\pgfpathlineto{\pgfqpoint{3.388592in}{1.274667in}}%
\pgfpathlineto{\pgfqpoint{3.401030in}{1.237333in}}%
\pgfpathlineto{\pgfqpoint{3.408717in}{1.196773in}}%
\pgfpathlineto{\pgfqpoint{3.407534in}{1.162667in}}%
\pgfpathlineto{\pgfqpoint{3.405253in}{1.154981in}}%
\pgfpathlineto{\pgfqpoint{3.386923in}{1.125333in}}%
\pgfpathlineto{\pgfqpoint{3.375607in}{1.115614in}}%
\pgfpathlineto{\pgfqpoint{3.365172in}{1.112341in}}%
\pgfpathlineto{\pgfqpoint{3.344290in}{1.107451in}}%
\pgfpathlineto{\pgfqpoint{3.325091in}{1.108179in}}%
\pgfpathlineto{\pgfqpoint{3.311343in}{1.112528in}}%
\pgfpathlineto{\pgfqpoint{3.285010in}{1.114639in}}%
\pgfpathlineto{\pgfqpoint{3.242156in}{1.127916in}}%
\pgfpathlineto{\pgfqpoint{3.204848in}{1.144781in}}%
\pgfpathlineto{\pgfqpoint{3.159791in}{1.167303in}}%
\pgfpathlineto{\pgfqpoint{3.106132in}{1.200000in}}%
\pgfpathlineto{\pgfqpoint{3.094172in}{1.208910in}}%
\pgfpathlineto{\pgfqpoint{3.084606in}{1.213768in}}%
\pgfpathlineto{\pgfqpoint{3.071060in}{1.224716in}}%
\pgfpathlineto{\pgfqpoint{3.044525in}{1.240919in}}%
\pgfpathlineto{\pgfqpoint{3.025621in}{1.257058in}}%
\pgfpathlineto{\pgfqpoint{2.998986in}{1.274667in}}%
\pgfpathlineto{\pgfqpoint{2.952039in}{1.312000in}}%
\pgfpathlineto{\pgfqpoint{2.907561in}{1.349333in}}%
\pgfpathlineto{\pgfqpoint{2.896272in}{1.360576in}}%
\pgfpathmoveto{\pgfqpoint{3.405253in}{1.329862in}}%
\pgfpathlineto{\pgfqpoint{3.383011in}{1.370050in}}%
\pgfpathlineto{\pgfqpoint{3.349045in}{1.424000in}}%
\pgfpathlineto{\pgfqpoint{3.324715in}{1.461333in}}%
\pgfpathlineto{\pgfqpoint{3.307674in}{1.482443in}}%
\pgfpathlineto{\pgfqpoint{3.297774in}{1.498667in}}%
\pgfpathlineto{\pgfqpoint{3.270057in}{1.536000in}}%
\pgfpathlineto{\pgfqpoint{3.258761in}{1.548883in}}%
\pgfpathlineto{\pgfqpoint{3.241545in}{1.573333in}}%
\pgfpathlineto{\pgfqpoint{3.224683in}{1.591809in}}%
\pgfpathlineto{\pgfqpoint{3.204848in}{1.618358in}}%
\pgfpathlineto{\pgfqpoint{3.146924in}{1.685333in}}%
\pgfpathlineto{\pgfqpoint{3.136155in}{1.696015in}}%
\pgfpathlineto{\pgfqpoint{3.113195in}{1.722667in}}%
\pgfpathlineto{\pgfqpoint{3.099077in}{1.736146in}}%
\pgfpathlineto{\pgfqpoint{3.078280in}{1.760000in}}%
\pgfpathlineto{\pgfqpoint{3.061282in}{1.775608in}}%
\pgfpathlineto{\pgfqpoint{3.042064in}{1.797333in}}%
\pgfpathlineto{\pgfqpoint{3.022743in}{1.814378in}}%
\pgfpathlineto{\pgfqpoint{3.004414in}{1.834667in}}%
\pgfpathlineto{\pgfqpoint{2.983435in}{1.852431in}}%
\pgfpathlineto{\pgfqpoint{2.964364in}{1.872663in}}%
\pgfpathlineto{\pgfqpoint{2.924190in}{1.909333in}}%
\pgfpathlineto{\pgfqpoint{2.881247in}{1.946667in}}%
\pgfpathlineto{\pgfqpoint{2.860605in}{1.962020in}}%
\pgfpathlineto{\pgfqpoint{2.836122in}{1.984000in}}%
\pgfpathlineto{\pgfqpoint{2.817922in}{1.996930in}}%
\pgfpathlineto{\pgfqpoint{2.804040in}{2.009501in}}%
\pgfpathlineto{\pgfqpoint{2.796574in}{2.014378in}}%
\pgfpathlineto{\pgfqpoint{2.788550in}{2.021333in}}%
\pgfpathlineto{\pgfqpoint{2.723879in}{2.069003in}}%
\pgfpathlineto{\pgfqpoint{2.706034in}{2.079378in}}%
\pgfpathlineto{\pgfqpoint{2.681406in}{2.098228in}}%
\pgfpathlineto{\pgfqpoint{2.643717in}{2.121521in}}%
\pgfpathlineto{\pgfqpoint{2.635238in}{2.125436in}}%
\pgfpathlineto{\pgfqpoint{2.623484in}{2.133333in}}%
\pgfpathlineto{\pgfqpoint{2.603636in}{2.144814in}}%
\pgfpathlineto{\pgfqpoint{2.584558in}{2.152896in}}%
\pgfpathlineto{\pgfqpoint{2.563556in}{2.166040in}}%
\pgfpathlineto{\pgfqpoint{2.523475in}{2.183853in}}%
\pgfpathlineto{\pgfqpoint{2.503329in}{2.189235in}}%
\pgfpathlineto{\pgfqpoint{2.483394in}{2.198700in}}%
\pgfpathlineto{\pgfqpoint{2.441298in}{2.209877in}}%
\pgfpathlineto{\pgfqpoint{2.398105in}{2.212775in}}%
\pgfpathlineto{\pgfqpoint{2.363152in}{2.204074in}}%
\pgfpathlineto{\pgfqpoint{2.331863in}{2.170667in}}%
\pgfpathlineto{\pgfqpoint{2.328797in}{2.165333in}}%
\pgfpathlineto{\pgfqpoint{2.323071in}{2.132890in}}%
\pgfpathlineto{\pgfqpoint{2.326158in}{2.093125in}}%
\pgfpathlineto{\pgfqpoint{2.334781in}{2.058667in}}%
\pgfpathlineto{\pgfqpoint{2.342980in}{2.039878in}}%
\pgfpathlineto{\pgfqpoint{2.347171in}{2.021333in}}%
\pgfpathlineto{\pgfqpoint{2.363152in}{1.981139in}}%
\pgfpathlineto{\pgfqpoint{2.380269in}{1.946667in}}%
\pgfpathlineto{\pgfqpoint{2.388863in}{1.933282in}}%
\pgfpathlineto{\pgfqpoint{2.403232in}{1.903094in}}%
\pgfpathlineto{\pgfqpoint{2.416127in}{1.884011in}}%
\pgfpathlineto{\pgfqpoint{2.421805in}{1.872000in}}%
\pgfpathlineto{\pgfqpoint{2.446155in}{1.832020in}}%
\pgfpathlineto{\pgfqpoint{2.483394in}{1.777694in}}%
\pgfpathlineto{\pgfqpoint{2.491508in}{1.767558in}}%
\pgfpathlineto{\pgfqpoint{2.495993in}{1.760000in}}%
\pgfpathlineto{\pgfqpoint{2.523475in}{1.721977in}}%
\pgfpathlineto{\pgfqpoint{2.563556in}{1.671060in}}%
\pgfpathlineto{\pgfqpoint{2.574928in}{1.658593in}}%
\pgfpathlineto{\pgfqpoint{2.582377in}{1.648000in}}%
\pgfpathlineto{\pgfqpoint{2.592286in}{1.637427in}}%
\pgfpathlineto{\pgfqpoint{2.613435in}{1.610667in}}%
\pgfpathlineto{\pgfqpoint{2.627840in}{1.595877in}}%
\pgfpathlineto{\pgfqpoint{2.645503in}{1.573333in}}%
\pgfpathlineto{\pgfqpoint{2.664094in}{1.554980in}}%
\pgfpathlineto{\pgfqpoint{2.683798in}{1.531144in}}%
\pgfpathlineto{\pgfqpoint{2.701076in}{1.514760in}}%
\pgfpathlineto{\pgfqpoint{2.723879in}{1.488829in}}%
\pgfpathlineto{\pgfqpoint{2.738811in}{1.475242in}}%
\pgfpathlineto{\pgfqpoint{2.750873in}{1.461333in}}%
\pgfpathlineto{\pgfqpoint{2.763960in}{1.448113in}}%
\pgfpathlineto{\pgfqpoint{2.777328in}{1.436452in}}%
\pgfpathlineto{\pgfqpoint{2.788563in}{1.424000in}}%
\pgfpathlineto{\pgfqpoint{2.827697in}{1.386667in}}%
\pgfpathlineto{\pgfqpoint{2.836338in}{1.379417in}}%
\pgfpathlineto{\pgfqpoint{2.844121in}{1.371317in}}%
\pgfpathlineto{\pgfqpoint{2.884202in}{1.335165in}}%
\pgfpathlineto{\pgfqpoint{2.897873in}{1.324734in}}%
\pgfpathlineto{\pgfqpoint{2.918200in}{1.306334in}}%
\pgfpathlineto{\pgfqpoint{2.924283in}{1.300461in}}%
\pgfpathlineto{\pgfqpoint{2.939829in}{1.289147in}}%
\pgfpathlineto{\pgfqpoint{2.964364in}{1.267172in}}%
\pgfpathlineto{\pgfqpoint{3.016618in}{1.225994in}}%
\pgfpathlineto{\pgfqpoint{3.052053in}{1.200000in}}%
\pgfpathlineto{\pgfqpoint{3.084606in}{1.177150in}}%
\pgfpathlineto{\pgfqpoint{3.094347in}{1.171739in}}%
\pgfpathlineto{\pgfqpoint{3.106227in}{1.162667in}}%
\pgfpathlineto{\pgfqpoint{3.124687in}{1.150407in}}%
\pgfpathlineto{\pgfqpoint{3.141975in}{1.141436in}}%
\pgfpathlineto{\pgfqpoint{3.164768in}{1.125065in}}%
\pgfpathlineto{\pgfqpoint{3.190973in}{1.112409in}}%
\pgfpathlineto{\pgfqpoint{3.204848in}{1.103013in}}%
\pgfpathlineto{\pgfqpoint{3.216248in}{1.098618in}}%
\pgfpathlineto{\pgfqpoint{3.244929in}{1.082487in}}%
\pgfpathlineto{\pgfqpoint{3.256955in}{1.076799in}}%
\pgfpathlineto{\pgfqpoint{3.285010in}{1.065552in}}%
\pgfpathlineto{\pgfqpoint{3.327784in}{1.050667in}}%
\pgfpathlineto{\pgfqpoint{3.374863in}{1.041639in}}%
\pgfpathlineto{\pgfqpoint{3.405253in}{1.042315in}}%
\pgfpathlineto{\pgfqpoint{3.433884in}{1.050667in}}%
\pgfpathlineto{\pgfqpoint{3.445333in}{1.056944in}}%
\pgfpathlineto{\pgfqpoint{3.460970in}{1.073436in}}%
\pgfpathlineto{\pgfqpoint{3.466746in}{1.088000in}}%
\pgfpathlineto{\pgfqpoint{3.468541in}{1.109617in}}%
\pgfpathlineto{\pgfqpoint{3.473182in}{1.125333in}}%
\pgfpathlineto{\pgfqpoint{3.473715in}{1.136230in}}%
\pgfpathlineto{\pgfqpoint{3.469771in}{1.162667in}}%
\pgfpathlineto{\pgfqpoint{3.463735in}{1.179807in}}%
\pgfpathlineto{\pgfqpoint{3.460944in}{1.200000in}}%
\pgfpathlineto{\pgfqpoint{3.447625in}{1.239468in}}%
\pgfpathlineto{\pgfqpoint{3.445333in}{1.245634in}}%
\pgfpathlineto{\pgfqpoint{3.432672in}{1.274667in}}%
\pgfpathlineto{\pgfqpoint{3.422929in}{1.291132in}}%
\pgfpathlineto{\pgfqpoint{3.414724in}{1.312000in}}%
\pgfpathlineto{\pgfqpoint{3.411124in}{1.317469in}}%
\pgfpathlineto{\pgfqpoint{3.411124in}{1.317469in}}%
\pgfusepath{fill}%
\end{pgfscope}%
\begin{pgfscope}%
\pgfpathrectangle{\pgfqpoint{0.800000in}{0.528000in}}{\pgfqpoint{3.968000in}{3.696000in}}%
\pgfusepath{clip}%
\pgfsetbuttcap%
\pgfsetroundjoin%
\definecolor{currentfill}{rgb}{0.268510,0.009605,0.335427}%
\pgfsetfillcolor{currentfill}%
\pgfsetlinewidth{0.000000pt}%
\definecolor{currentstroke}{rgb}{0.000000,0.000000,0.000000}%
\pgfsetstrokecolor{currentstroke}%
\pgfsetdash{}{0pt}%
\pgfpathmoveto{\pgfqpoint{3.411124in}{1.317469in}}%
\pgfpathlineto{\pgfqpoint{3.414724in}{1.312000in}}%
\pgfpathlineto{\pgfqpoint{3.422929in}{1.291132in}}%
\pgfpathlineto{\pgfqpoint{3.432672in}{1.274667in}}%
\pgfpathlineto{\pgfqpoint{3.450978in}{1.232076in}}%
\pgfpathlineto{\pgfqpoint{3.460944in}{1.200000in}}%
\pgfpathlineto{\pgfqpoint{3.463735in}{1.179807in}}%
\pgfpathlineto{\pgfqpoint{3.469771in}{1.162667in}}%
\pgfpathlineto{\pgfqpoint{3.473715in}{1.136230in}}%
\pgfpathlineto{\pgfqpoint{3.473182in}{1.125333in}}%
\pgfpathlineto{\pgfqpoint{3.468541in}{1.109617in}}%
\pgfpathlineto{\pgfqpoint{3.466746in}{1.088000in}}%
\pgfpathlineto{\pgfqpoint{3.460970in}{1.073436in}}%
\pgfpathlineto{\pgfqpoint{3.445333in}{1.056944in}}%
\pgfpathlineto{\pgfqpoint{3.433884in}{1.050667in}}%
\pgfpathlineto{\pgfqpoint{3.405253in}{1.042315in}}%
\pgfpathlineto{\pgfqpoint{3.374863in}{1.041639in}}%
\pgfpathlineto{\pgfqpoint{3.365172in}{1.043128in}}%
\pgfpathlineto{\pgfqpoint{3.324280in}{1.051422in}}%
\pgfpathlineto{\pgfqpoint{3.256955in}{1.076799in}}%
\pgfpathlineto{\pgfqpoint{3.234088in}{1.088000in}}%
\pgfpathlineto{\pgfqpoint{3.216248in}{1.098618in}}%
\pgfpathlineto{\pgfqpoint{3.204848in}{1.103013in}}%
\pgfpathlineto{\pgfqpoint{3.190973in}{1.112409in}}%
\pgfpathlineto{\pgfqpoint{3.164340in}{1.125333in}}%
\pgfpathlineto{\pgfqpoint{3.141975in}{1.141436in}}%
\pgfpathlineto{\pgfqpoint{3.124687in}{1.150407in}}%
\pgfpathlineto{\pgfqpoint{3.084606in}{1.177150in}}%
\pgfpathlineto{\pgfqpoint{3.016618in}{1.225994in}}%
\pgfpathlineto{\pgfqpoint{2.964364in}{1.267172in}}%
\pgfpathlineto{\pgfqpoint{2.955284in}{1.274667in}}%
\pgfpathlineto{\pgfqpoint{2.939829in}{1.289147in}}%
\pgfpathlineto{\pgfqpoint{2.924283in}{1.300461in}}%
\pgfpathlineto{\pgfqpoint{2.897873in}{1.324734in}}%
\pgfpathlineto{\pgfqpoint{2.884202in}{1.335165in}}%
\pgfpathlineto{\pgfqpoint{2.827697in}{1.386667in}}%
\pgfpathlineto{\pgfqpoint{2.788563in}{1.424000in}}%
\pgfpathlineto{\pgfqpoint{2.777328in}{1.436452in}}%
\pgfpathlineto{\pgfqpoint{2.763960in}{1.448113in}}%
\pgfpathlineto{\pgfqpoint{2.750873in}{1.461333in}}%
\pgfpathlineto{\pgfqpoint{2.738811in}{1.475242in}}%
\pgfpathlineto{\pgfqpoint{2.714501in}{1.498667in}}%
\pgfpathlineto{\pgfqpoint{2.701076in}{1.514760in}}%
\pgfpathlineto{\pgfqpoint{2.679336in}{1.536000in}}%
\pgfpathlineto{\pgfqpoint{2.664094in}{1.554980in}}%
\pgfpathlineto{\pgfqpoint{2.643717in}{1.575352in}}%
\pgfpathlineto{\pgfqpoint{2.627840in}{1.595877in}}%
\pgfpathlineto{\pgfqpoint{2.603636in}{1.622205in}}%
\pgfpathlineto{\pgfqpoint{2.592286in}{1.637427in}}%
\pgfpathlineto{\pgfqpoint{2.582377in}{1.648000in}}%
\pgfpathlineto{\pgfqpoint{2.574928in}{1.658593in}}%
\pgfpathlineto{\pgfqpoint{2.563556in}{1.671060in}}%
\pgfpathlineto{\pgfqpoint{2.552240in}{1.685333in}}%
\pgfpathlineto{\pgfqpoint{2.521698in}{1.724322in}}%
\pgfpathlineto{\pgfqpoint{2.469830in}{1.797333in}}%
\pgfpathlineto{\pgfqpoint{2.443313in}{1.836517in}}%
\pgfpathlineto{\pgfqpoint{2.421805in}{1.872000in}}%
\pgfpathlineto{\pgfqpoint{2.416127in}{1.884011in}}%
\pgfpathlineto{\pgfqpoint{2.399623in}{1.909333in}}%
\pgfpathlineto{\pgfqpoint{2.388863in}{1.933282in}}%
\pgfpathlineto{\pgfqpoint{2.380269in}{1.946667in}}%
\pgfpathlineto{\pgfqpoint{2.360710in}{1.986274in}}%
\pgfpathlineto{\pgfqpoint{2.347171in}{2.021333in}}%
\pgfpathlineto{\pgfqpoint{2.342980in}{2.039878in}}%
\pgfpathlineto{\pgfqpoint{2.334781in}{2.058667in}}%
\pgfpathlineto{\pgfqpoint{2.325895in}{2.096000in}}%
\pgfpathlineto{\pgfqpoint{2.326287in}{2.098996in}}%
\pgfpathlineto{\pgfqpoint{2.323071in}{2.133567in}}%
\pgfpathlineto{\pgfqpoint{2.328797in}{2.165333in}}%
\pgfpathlineto{\pgfqpoint{2.331863in}{2.170667in}}%
\pgfpathlineto{\pgfqpoint{2.365653in}{2.205670in}}%
\pgfpathlineto{\pgfqpoint{2.398105in}{2.212775in}}%
\pgfpathlineto{\pgfqpoint{2.403232in}{2.212675in}}%
\pgfpathlineto{\pgfqpoint{2.444370in}{2.208985in}}%
\pgfpathlineto{\pgfqpoint{2.448723in}{2.208000in}}%
\pgfpathlineto{\pgfqpoint{2.483394in}{2.198700in}}%
\pgfpathlineto{\pgfqpoint{2.503329in}{2.189235in}}%
\pgfpathlineto{\pgfqpoint{2.523475in}{2.183853in}}%
\pgfpathlineto{\pgfqpoint{2.563556in}{2.166040in}}%
\pgfpathlineto{\pgfqpoint{2.584558in}{2.152896in}}%
\pgfpathlineto{\pgfqpoint{2.603636in}{2.144814in}}%
\pgfpathlineto{\pgfqpoint{2.643717in}{2.121521in}}%
\pgfpathlineto{\pgfqpoint{2.684766in}{2.096000in}}%
\pgfpathlineto{\pgfqpoint{2.706034in}{2.079378in}}%
\pgfpathlineto{\pgfqpoint{2.723879in}{2.069003in}}%
\pgfpathlineto{\pgfqpoint{2.788550in}{2.021333in}}%
\pgfpathlineto{\pgfqpoint{2.796574in}{2.014378in}}%
\pgfpathlineto{\pgfqpoint{2.804040in}{2.009501in}}%
\pgfpathlineto{\pgfqpoint{2.817922in}{1.996930in}}%
\pgfpathlineto{\pgfqpoint{2.844121in}{1.977588in}}%
\pgfpathlineto{\pgfqpoint{2.860605in}{1.962020in}}%
\pgfpathlineto{\pgfqpoint{2.884202in}{1.944185in}}%
\pgfpathlineto{\pgfqpoint{2.928245in}{1.905643in}}%
\pgfpathlineto{\pgfqpoint{2.965067in}{1.872000in}}%
\pgfpathlineto{\pgfqpoint{2.983435in}{1.852431in}}%
\pgfpathlineto{\pgfqpoint{3.004444in}{1.834637in}}%
\pgfpathlineto{\pgfqpoint{3.022743in}{1.814378in}}%
\pgfpathlineto{\pgfqpoint{3.044525in}{1.794871in}}%
\pgfpathlineto{\pgfqpoint{3.061282in}{1.775608in}}%
\pgfpathlineto{\pgfqpoint{3.084606in}{1.753407in}}%
\pgfpathlineto{\pgfqpoint{3.099077in}{1.736146in}}%
\pgfpathlineto{\pgfqpoint{3.124687in}{1.710199in}}%
\pgfpathlineto{\pgfqpoint{3.136155in}{1.696015in}}%
\pgfpathlineto{\pgfqpoint{3.146924in}{1.685333in}}%
\pgfpathlineto{\pgfqpoint{3.204848in}{1.618358in}}%
\pgfpathlineto{\pgfqpoint{3.224683in}{1.591809in}}%
\pgfpathlineto{\pgfqpoint{3.244929in}{1.569021in}}%
\pgfpathlineto{\pgfqpoint{3.258761in}{1.548883in}}%
\pgfpathlineto{\pgfqpoint{3.270057in}{1.536000in}}%
\pgfpathlineto{\pgfqpoint{3.275625in}{1.527258in}}%
\pgfpathlineto{\pgfqpoint{3.285010in}{1.516136in}}%
\pgfpathlineto{\pgfqpoint{3.297774in}{1.498667in}}%
\pgfpathlineto{\pgfqpoint{3.307674in}{1.482443in}}%
\pgfpathlineto{\pgfqpoint{3.325091in}{1.460775in}}%
\pgfpathlineto{\pgfqpoint{3.383011in}{1.370050in}}%
\pgfpathlineto{\pgfqpoint{3.405253in}{1.329862in}}%
\pgfpathmoveto{\pgfqpoint{3.428443in}{1.349333in}}%
\pgfpathlineto{\pgfqpoint{3.405253in}{1.387960in}}%
\pgfpathlineto{\pgfqpoint{3.389454in}{1.409285in}}%
\pgfpathlineto{\pgfqpoint{3.381194in}{1.424000in}}%
\pgfpathlineto{\pgfqpoint{3.337883in}{1.486751in}}%
\pgfpathlineto{\pgfqpoint{3.300723in}{1.536000in}}%
\pgfpathlineto{\pgfqpoint{3.271605in}{1.573333in}}%
\pgfpathlineto{\pgfqpoint{3.259415in}{1.586826in}}%
\pgfpathlineto{\pgfqpoint{3.241694in}{1.610667in}}%
\pgfpathlineto{\pgfqpoint{3.224529in}{1.628998in}}%
\pgfpathlineto{\pgfqpoint{3.204848in}{1.654250in}}%
\pgfpathlineto{\pgfqpoint{3.188998in}{1.670569in}}%
\pgfpathlineto{\pgfqpoint{3.171387in}{1.691499in}}%
\pgfpathlineto{\pgfqpoint{3.164768in}{1.699653in}}%
\pgfpathlineto{\pgfqpoint{3.152802in}{1.711521in}}%
\pgfpathlineto{\pgfqpoint{3.143791in}{1.722667in}}%
\pgfpathlineto{\pgfqpoint{3.134475in}{1.731784in}}%
\pgfpathlineto{\pgfqpoint{3.124687in}{1.743459in}}%
\pgfpathlineto{\pgfqpoint{3.115917in}{1.751831in}}%
\pgfpathlineto{\pgfqpoint{3.109069in}{1.760000in}}%
\pgfpathlineto{\pgfqpoint{3.096930in}{1.771479in}}%
\pgfpathlineto{\pgfqpoint{3.073225in}{1.797333in}}%
\pgfpathlineto{\pgfqpoint{3.058732in}{1.810566in}}%
\pgfpathlineto{\pgfqpoint{3.036161in}{1.834667in}}%
\pgfpathlineto{\pgfqpoint{2.964364in}{1.903448in}}%
\pgfpathlineto{\pgfqpoint{2.957918in}{1.909333in}}%
\pgfpathlineto{\pgfqpoint{2.940009in}{1.923982in}}%
\pgfpathlineto{\pgfqpoint{2.916475in}{1.946667in}}%
\pgfpathlineto{\pgfqpoint{2.844121in}{2.008407in}}%
\pgfpathlineto{\pgfqpoint{2.836249in}{2.014000in}}%
\pgfpathlineto{\pgfqpoint{2.828150in}{2.021333in}}%
\pgfpathlineto{\pgfqpoint{2.763960in}{2.071700in}}%
\pgfpathlineto{\pgfqpoint{2.699335in}{2.118861in}}%
\pgfpathlineto{\pgfqpoint{2.677936in}{2.133333in}}%
\pgfpathlineto{\pgfqpoint{2.656878in}{2.145592in}}%
\pgfpathlineto{\pgfqpoint{2.643717in}{2.155542in}}%
\pgfpathlineto{\pgfqpoint{2.603636in}{2.180274in}}%
\pgfpathlineto{\pgfqpoint{2.583966in}{2.189678in}}%
\pgfpathlineto{\pgfqpoint{2.563556in}{2.203138in}}%
\pgfpathlineto{\pgfqpoint{2.523475in}{2.223281in}}%
\pgfpathlineto{\pgfqpoint{2.506258in}{2.229297in}}%
\pgfpathlineto{\pgfqpoint{2.480140in}{2.242302in}}%
\pgfpathlineto{\pgfqpoint{2.472997in}{2.245333in}}%
\pgfpathlineto{\pgfqpoint{2.450785in}{2.252293in}}%
\pgfpathlineto{\pgfqpoint{2.443313in}{2.255980in}}%
\pgfpathlineto{\pgfqpoint{2.425876in}{2.261575in}}%
\pgfpathlineto{\pgfqpoint{2.403232in}{2.266504in}}%
\pgfpathlineto{\pgfqpoint{2.386175in}{2.266778in}}%
\pgfpathlineto{\pgfqpoint{2.363152in}{2.271587in}}%
\pgfpathlineto{\pgfqpoint{2.335952in}{2.270669in}}%
\pgfpathlineto{\pgfqpoint{2.323071in}{2.267057in}}%
\pgfpathlineto{\pgfqpoint{2.285935in}{2.245333in}}%
\pgfpathlineto{\pgfqpoint{2.282990in}{2.241675in}}%
\pgfpathlineto{\pgfqpoint{2.268925in}{2.208000in}}%
\pgfpathlineto{\pgfqpoint{2.265279in}{2.187164in}}%
\pgfpathlineto{\pgfqpoint{2.265981in}{2.170667in}}%
\pgfpathlineto{\pgfqpoint{2.269119in}{2.157747in}}%
\pgfpathlineto{\pgfqpoint{2.270280in}{2.133333in}}%
\pgfpathlineto{\pgfqpoint{2.273586in}{2.124574in}}%
\pgfpathlineto{\pgfqpoint{2.280238in}{2.093437in}}%
\pgfpathlineto{\pgfqpoint{2.282990in}{2.084258in}}%
\pgfpathlineto{\pgfqpoint{2.290524in}{2.065684in}}%
\pgfpathlineto{\pgfqpoint{2.291979in}{2.058667in}}%
\pgfpathlineto{\pgfqpoint{2.301931in}{2.038976in}}%
\pgfpathlineto{\pgfqpoint{2.307554in}{2.021333in}}%
\pgfpathlineto{\pgfqpoint{2.324578in}{1.984000in}}%
\pgfpathlineto{\pgfqpoint{2.338566in}{1.961099in}}%
\pgfpathlineto{\pgfqpoint{2.344733in}{1.946667in}}%
\pgfpathlineto{\pgfqpoint{2.368094in}{1.904730in}}%
\pgfpathlineto{\pgfqpoint{2.412685in}{1.834667in}}%
\pgfpathlineto{\pgfqpoint{2.443313in}{1.789805in}}%
\pgfpathlineto{\pgfqpoint{2.492698in}{1.722667in}}%
\pgfpathlineto{\pgfqpoint{2.523475in}{1.682728in}}%
\pgfpathlineto{\pgfqpoint{2.540503in}{1.663861in}}%
\pgfpathlineto{\pgfqpoint{2.551911in}{1.648000in}}%
\pgfpathlineto{\pgfqpoint{2.583257in}{1.610667in}}%
\pgfpathlineto{\pgfqpoint{2.592864in}{1.600633in}}%
\pgfpathlineto{\pgfqpoint{2.615501in}{1.573333in}}%
\pgfpathlineto{\pgfqpoint{2.683798in}{1.497913in}}%
\pgfpathlineto{\pgfqpoint{2.763960in}{1.416289in}}%
\pgfpathlineto{\pgfqpoint{2.804040in}{1.377561in}}%
\pgfpathlineto{\pgfqpoint{2.819984in}{1.364184in}}%
\pgfpathlineto{\pgfqpoint{2.844121in}{1.340183in}}%
\pgfpathlineto{\pgfqpoint{2.884202in}{1.304125in}}%
\pgfpathlineto{\pgfqpoint{2.901470in}{1.290751in}}%
\pgfpathlineto{\pgfqpoint{2.924283in}{1.269359in}}%
\pgfpathlineto{\pgfqpoint{2.964364in}{1.235859in}}%
\pgfpathlineto{\pgfqpoint{2.986186in}{1.220326in}}%
\pgfpathlineto{\pgfqpoint{3.009727in}{1.200000in}}%
\pgfpathlineto{\pgfqpoint{3.029828in}{1.186310in}}%
\pgfpathlineto{\pgfqpoint{3.044525in}{1.173707in}}%
\pgfpathlineto{\pgfqpoint{3.112555in}{1.125333in}}%
\pgfpathlineto{\pgfqpoint{3.124687in}{1.116989in}}%
\pgfpathlineto{\pgfqpoint{3.169561in}{1.088000in}}%
\pgfpathlineto{\pgfqpoint{3.192094in}{1.076120in}}%
\pgfpathlineto{\pgfqpoint{3.204848in}{1.067168in}}%
\pgfpathlineto{\pgfqpoint{3.216814in}{1.061812in}}%
\pgfpathlineto{\pgfqpoint{3.244929in}{1.044807in}}%
\pgfpathlineto{\pgfqpoint{3.259291in}{1.037289in}}%
\pgfpathlineto{\pgfqpoint{3.285010in}{1.025240in}}%
\pgfpathlineto{\pgfqpoint{3.294617in}{1.022282in}}%
\pgfpathlineto{\pgfqpoint{3.325091in}{1.007922in}}%
\pgfpathlineto{\pgfqpoint{3.335168in}{1.003947in}}%
\pgfpathlineto{\pgfqpoint{3.365172in}{0.994592in}}%
\pgfpathlineto{\pgfqpoint{3.382717in}{0.992342in}}%
\pgfpathlineto{\pgfqpoint{3.405253in}{0.985107in}}%
\pgfpathlineto{\pgfqpoint{3.445333in}{0.982033in}}%
\pgfpathlineto{\pgfqpoint{3.455622in}{0.985584in}}%
\pgfpathlineto{\pgfqpoint{3.485414in}{0.990869in}}%
\pgfpathlineto{\pgfqpoint{3.500907in}{0.998902in}}%
\pgfpathlineto{\pgfqpoint{3.514034in}{1.013333in}}%
\pgfpathlineto{\pgfqpoint{3.525495in}{1.035399in}}%
\pgfpathlineto{\pgfqpoint{3.529965in}{1.050667in}}%
\pgfpathlineto{\pgfqpoint{3.532226in}{1.081730in}}%
\pgfpathlineto{\pgfqpoint{3.526719in}{1.125333in}}%
\pgfpathlineto{\pgfqpoint{3.525495in}{1.129943in}}%
\pgfpathlineto{\pgfqpoint{3.517375in}{1.155103in}}%
\pgfpathlineto{\pgfqpoint{3.516455in}{1.162667in}}%
\pgfpathlineto{\pgfqpoint{3.507405in}{1.183150in}}%
\pgfpathlineto{\pgfqpoint{3.503175in}{1.200000in}}%
\pgfpathlineto{\pgfqpoint{3.485414in}{1.242857in}}%
\pgfpathlineto{\pgfqpoint{3.473704in}{1.263760in}}%
\pgfpathlineto{\pgfqpoint{3.469651in}{1.274667in}}%
\pgfpathlineto{\pgfqpoint{3.460674in}{1.288956in}}%
\pgfpathlineto{\pgfqpoint{3.445333in}{1.320616in}}%
\pgfpathlineto{\pgfqpoint{3.433614in}{1.338418in}}%
\pgfpathlineto{\pgfqpoint{3.433614in}{1.338418in}}%
\pgfusepath{fill}%
\end{pgfscope}%
\begin{pgfscope}%
\pgfpathrectangle{\pgfqpoint{0.800000in}{0.528000in}}{\pgfqpoint{3.968000in}{3.696000in}}%
\pgfusepath{clip}%
\pgfsetbuttcap%
\pgfsetroundjoin%
\definecolor{currentfill}{rgb}{0.268510,0.009605,0.335427}%
\pgfsetfillcolor{currentfill}%
\pgfsetlinewidth{0.000000pt}%
\definecolor{currentstroke}{rgb}{0.000000,0.000000,0.000000}%
\pgfsetstrokecolor{currentstroke}%
\pgfsetdash{}{0pt}%
\pgfpathmoveto{\pgfqpoint{3.433614in}{1.338418in}}%
\pgfpathlineto{\pgfqpoint{3.450254in}{1.312000in}}%
\pgfpathlineto{\pgfqpoint{3.460674in}{1.288956in}}%
\pgfpathlineto{\pgfqpoint{3.469651in}{1.274667in}}%
\pgfpathlineto{\pgfqpoint{3.473704in}{1.263760in}}%
\pgfpathlineto{\pgfqpoint{3.490142in}{1.232929in}}%
\pgfpathlineto{\pgfqpoint{3.503175in}{1.200000in}}%
\pgfpathlineto{\pgfqpoint{3.507405in}{1.183150in}}%
\pgfpathlineto{\pgfqpoint{3.516455in}{1.162667in}}%
\pgfpathlineto{\pgfqpoint{3.517375in}{1.155103in}}%
\pgfpathlineto{\pgfqpoint{3.526719in}{1.125333in}}%
\pgfpathlineto{\pgfqpoint{3.532226in}{1.081730in}}%
\pgfpathlineto{\pgfqpoint{3.529965in}{1.050667in}}%
\pgfpathlineto{\pgfqpoint{3.525495in}{1.035399in}}%
\pgfpathlineto{\pgfqpoint{3.514034in}{1.013333in}}%
\pgfpathlineto{\pgfqpoint{3.500907in}{0.998902in}}%
\pgfpathlineto{\pgfqpoint{3.485414in}{0.990869in}}%
\pgfpathlineto{\pgfqpoint{3.455622in}{0.985584in}}%
\pgfpathlineto{\pgfqpoint{3.445333in}{0.982033in}}%
\pgfpathlineto{\pgfqpoint{3.439368in}{0.981556in}}%
\pgfpathlineto{\pgfqpoint{3.405253in}{0.985107in}}%
\pgfpathlineto{\pgfqpoint{3.382717in}{0.992342in}}%
\pgfpathlineto{\pgfqpoint{3.365172in}{0.994592in}}%
\pgfpathlineto{\pgfqpoint{3.325091in}{1.007922in}}%
\pgfpathlineto{\pgfqpoint{3.312485in}{1.013333in}}%
\pgfpathlineto{\pgfqpoint{3.294617in}{1.022282in}}%
\pgfpathlineto{\pgfqpoint{3.285010in}{1.025240in}}%
\pgfpathlineto{\pgfqpoint{3.244929in}{1.044807in}}%
\pgfpathlineto{\pgfqpoint{3.234360in}{1.050667in}}%
\pgfpathlineto{\pgfqpoint{3.216814in}{1.061812in}}%
\pgfpathlineto{\pgfqpoint{3.204848in}{1.067168in}}%
\pgfpathlineto{\pgfqpoint{3.192094in}{1.076120in}}%
\pgfpathlineto{\pgfqpoint{3.164768in}{1.090854in}}%
\pgfpathlineto{\pgfqpoint{3.124687in}{1.116989in}}%
\pgfpathlineto{\pgfqpoint{3.112555in}{1.125333in}}%
\pgfpathlineto{\pgfqpoint{3.044525in}{1.173707in}}%
\pgfpathlineto{\pgfqpoint{3.029828in}{1.186310in}}%
\pgfpathlineto{\pgfqpoint{3.004444in}{1.204023in}}%
\pgfpathlineto{\pgfqpoint{2.986186in}{1.220326in}}%
\pgfpathlineto{\pgfqpoint{2.962588in}{1.237333in}}%
\pgfpathlineto{\pgfqpoint{2.918127in}{1.274667in}}%
\pgfpathlineto{\pgfqpoint{2.901470in}{1.290751in}}%
\pgfpathlineto{\pgfqpoint{2.875399in}{1.312000in}}%
\pgfpathlineto{\pgfqpoint{2.834255in}{1.349333in}}%
\pgfpathlineto{\pgfqpoint{2.819984in}{1.364184in}}%
\pgfpathlineto{\pgfqpoint{2.794564in}{1.386667in}}%
\pgfpathlineto{\pgfqpoint{2.723879in}{1.456396in}}%
\pgfpathlineto{\pgfqpoint{2.719081in}{1.461333in}}%
\pgfpathlineto{\pgfqpoint{2.678394in}{1.503701in}}%
\pgfpathlineto{\pgfqpoint{2.615501in}{1.573333in}}%
\pgfpathlineto{\pgfqpoint{2.610532in}{1.579757in}}%
\pgfpathlineto{\pgfqpoint{2.603636in}{1.586837in}}%
\pgfpathlineto{\pgfqpoint{2.592864in}{1.600633in}}%
\pgfpathlineto{\pgfqpoint{2.583257in}{1.610667in}}%
\pgfpathlineto{\pgfqpoint{2.575182in}{1.621496in}}%
\pgfpathlineto{\pgfqpoint{2.563556in}{1.633901in}}%
\pgfpathlineto{\pgfqpoint{2.551911in}{1.648000in}}%
\pgfpathlineto{\pgfqpoint{2.540503in}{1.663861in}}%
\pgfpathlineto{\pgfqpoint{2.515569in}{1.692697in}}%
\pgfpathlineto{\pgfqpoint{2.483394in}{1.734981in}}%
\pgfpathlineto{\pgfqpoint{2.473070in}{1.750384in}}%
\pgfpathlineto{\pgfqpoint{2.465003in}{1.760000in}}%
\pgfpathlineto{\pgfqpoint{2.427928in}{1.811664in}}%
\pgfpathlineto{\pgfqpoint{2.403232in}{1.849104in}}%
\pgfpathlineto{\pgfqpoint{2.394631in}{1.863989in}}%
\pgfpathlineto{\pgfqpoint{2.388665in}{1.872000in}}%
\pgfpathlineto{\pgfqpoint{2.363152in}{1.913332in}}%
\pgfpathlineto{\pgfqpoint{2.344733in}{1.946667in}}%
\pgfpathlineto{\pgfqpoint{2.338566in}{1.961099in}}%
\pgfpathlineto{\pgfqpoint{2.323071in}{1.987205in}}%
\pgfpathlineto{\pgfqpoint{2.307554in}{2.021333in}}%
\pgfpathlineto{\pgfqpoint{2.301931in}{2.038976in}}%
\pgfpathlineto{\pgfqpoint{2.291979in}{2.058667in}}%
\pgfpathlineto{\pgfqpoint{2.290524in}{2.065684in}}%
\pgfpathlineto{\pgfqpoint{2.279015in}{2.096000in}}%
\pgfpathlineto{\pgfqpoint{2.273586in}{2.124574in}}%
\pgfpathlineto{\pgfqpoint{2.270280in}{2.133333in}}%
\pgfpathlineto{\pgfqpoint{2.269119in}{2.157747in}}%
\pgfpathlineto{\pgfqpoint{2.265981in}{2.170667in}}%
\pgfpathlineto{\pgfqpoint{2.265279in}{2.187164in}}%
\pgfpathlineto{\pgfqpoint{2.268925in}{2.208000in}}%
\pgfpathlineto{\pgfqpoint{2.282990in}{2.241675in}}%
\pgfpathlineto{\pgfqpoint{2.285935in}{2.245333in}}%
\pgfpathlineto{\pgfqpoint{2.323071in}{2.267057in}}%
\pgfpathlineto{\pgfqpoint{2.335952in}{2.270669in}}%
\pgfpathlineto{\pgfqpoint{2.363152in}{2.271587in}}%
\pgfpathlineto{\pgfqpoint{2.386175in}{2.266778in}}%
\pgfpathlineto{\pgfqpoint{2.403232in}{2.266504in}}%
\pgfpathlineto{\pgfqpoint{2.425876in}{2.261575in}}%
\pgfpathlineto{\pgfqpoint{2.443313in}{2.255980in}}%
\pgfpathlineto{\pgfqpoint{2.450785in}{2.252293in}}%
\pgfpathlineto{\pgfqpoint{2.483394in}{2.241566in}}%
\pgfpathlineto{\pgfqpoint{2.506258in}{2.229297in}}%
\pgfpathlineto{\pgfqpoint{2.523475in}{2.223281in}}%
\pgfpathlineto{\pgfqpoint{2.563556in}{2.203138in}}%
\pgfpathlineto{\pgfqpoint{2.583966in}{2.189678in}}%
\pgfpathlineto{\pgfqpoint{2.603636in}{2.180274in}}%
\pgfpathlineto{\pgfqpoint{2.609453in}{2.176085in}}%
\pgfpathlineto{\pgfqpoint{2.619283in}{2.170667in}}%
\pgfpathlineto{\pgfqpoint{2.643717in}{2.155542in}}%
\pgfpathlineto{\pgfqpoint{2.656878in}{2.145592in}}%
\pgfpathlineto{\pgfqpoint{2.683798in}{2.129498in}}%
\pgfpathlineto{\pgfqpoint{2.731244in}{2.096000in}}%
\pgfpathlineto{\pgfqpoint{2.804040in}{2.040701in}}%
\pgfpathlineto{\pgfqpoint{2.873279in}{1.984000in}}%
\pgfpathlineto{\pgfqpoint{2.884202in}{1.974787in}}%
\pgfpathlineto{\pgfqpoint{2.924283in}{1.939811in}}%
\pgfpathlineto{\pgfqpoint{2.940009in}{1.923982in}}%
\pgfpathlineto{\pgfqpoint{2.964364in}{1.903448in}}%
\pgfpathlineto{\pgfqpoint{3.036161in}{1.834667in}}%
\pgfpathlineto{\pgfqpoint{3.058732in}{1.810566in}}%
\pgfpathlineto{\pgfqpoint{3.084606in}{1.785705in}}%
\pgfpathlineto{\pgfqpoint{3.096930in}{1.771479in}}%
\pgfpathlineto{\pgfqpoint{3.109069in}{1.760000in}}%
\pgfpathlineto{\pgfqpoint{3.115917in}{1.751831in}}%
\pgfpathlineto{\pgfqpoint{3.124687in}{1.743459in}}%
\pgfpathlineto{\pgfqpoint{3.134475in}{1.731784in}}%
\pgfpathlineto{\pgfqpoint{3.143791in}{1.722667in}}%
\pgfpathlineto{\pgfqpoint{3.152802in}{1.711521in}}%
\pgfpathlineto{\pgfqpoint{3.164768in}{1.699653in}}%
\pgfpathlineto{\pgfqpoint{3.188998in}{1.670569in}}%
\pgfpathlineto{\pgfqpoint{3.210212in}{1.648000in}}%
\pgfpathlineto{\pgfqpoint{3.224529in}{1.628998in}}%
\pgfpathlineto{\pgfqpoint{3.244929in}{1.606727in}}%
\pgfpathlineto{\pgfqpoint{3.259415in}{1.586826in}}%
\pgfpathlineto{\pgfqpoint{3.271605in}{1.573333in}}%
\pgfpathlineto{\pgfqpoint{3.276835in}{1.565719in}}%
\pgfpathlineto{\pgfqpoint{3.285010in}{1.556396in}}%
\pgfpathlineto{\pgfqpoint{3.337883in}{1.486751in}}%
\pgfpathlineto{\pgfqpoint{3.381194in}{1.424000in}}%
\pgfpathlineto{\pgfqpoint{3.389454in}{1.409285in}}%
\pgfpathlineto{\pgfqpoint{3.407344in}{1.384718in}}%
\pgfpathlineto{\pgfqpoint{3.428443in}{1.349333in}}%
\pgfpathmoveto{\pgfqpoint{2.803352in}{1.349333in}}%
\pgfpathlineto{\pgfqpoint{2.804040in}{1.348678in}}%
\pgfpathlineto{\pgfqpoint{2.824721in}{1.331263in}}%
\pgfpathlineto{\pgfqpoint{2.844121in}{1.311491in}}%
\pgfpathlineto{\pgfqpoint{2.885258in}{1.274667in}}%
\pgfpathlineto{\pgfqpoint{2.964364in}{1.207720in}}%
\pgfpathlineto{\pgfqpoint{2.973962in}{1.200000in}}%
\pgfpathlineto{\pgfqpoint{3.021248in}{1.162667in}}%
\pgfpathlineto{\pgfqpoint{3.034545in}{1.153371in}}%
\pgfpathlineto{\pgfqpoint{3.044525in}{1.144800in}}%
\pgfpathlineto{\pgfqpoint{3.056829in}{1.136793in}}%
\pgfpathlineto{\pgfqpoint{3.084606in}{1.115131in}}%
\pgfpathlineto{\pgfqpoint{3.102083in}{1.104279in}}%
\pgfpathlineto{\pgfqpoint{3.130199in}{1.082865in}}%
\pgfpathlineto{\pgfqpoint{3.180029in}{1.050667in}}%
\pgfpathlineto{\pgfqpoint{3.195504in}{1.041963in}}%
\pgfpathlineto{\pgfqpoint{3.204848in}{1.035226in}}%
\pgfpathlineto{\pgfqpoint{3.220190in}{1.027623in}}%
\pgfpathlineto{\pgfqpoint{3.244929in}{1.011415in}}%
\pgfpathlineto{\pgfqpoint{3.270301in}{0.999632in}}%
\pgfpathlineto{\pgfqpoint{3.285010in}{0.990384in}}%
\pgfpathlineto{\pgfqpoint{3.325091in}{0.970853in}}%
\pgfpathlineto{\pgfqpoint{3.350303in}{0.962151in}}%
\pgfpathlineto{\pgfqpoint{3.365172in}{0.954511in}}%
\pgfpathlineto{\pgfqpoint{3.407022in}{0.940315in}}%
\pgfpathlineto{\pgfqpoint{3.413125in}{0.938667in}}%
\pgfpathlineto{\pgfqpoint{3.445333in}{0.931371in}}%
\pgfpathlineto{\pgfqpoint{3.476037in}{0.929932in}}%
\pgfpathlineto{\pgfqpoint{3.485414in}{0.927865in}}%
\pgfpathlineto{\pgfqpoint{3.497123in}{0.927761in}}%
\pgfpathlineto{\pgfqpoint{3.525495in}{0.933152in}}%
\pgfpathlineto{\pgfqpoint{3.537830in}{0.938667in}}%
\pgfpathlineto{\pgfqpoint{3.565576in}{0.961079in}}%
\pgfpathlineto{\pgfqpoint{3.572346in}{0.969694in}}%
\pgfpathlineto{\pgfqpoint{3.574701in}{0.976000in}}%
\pgfpathlineto{\pgfqpoint{3.575830in}{0.985551in}}%
\pgfpathlineto{\pgfqpoint{3.583934in}{1.013333in}}%
\pgfpathlineto{\pgfqpoint{3.585491in}{1.032116in}}%
\pgfpathlineto{\pgfqpoint{3.583732in}{1.050667in}}%
\pgfpathlineto{\pgfqpoint{3.580067in}{1.064165in}}%
\pgfpathlineto{\pgfqpoint{3.577866in}{1.088000in}}%
\pgfpathlineto{\pgfqpoint{3.567484in}{1.127110in}}%
\pgfpathlineto{\pgfqpoint{3.565576in}{1.133171in}}%
\pgfpathlineto{\pgfqpoint{3.554846in}{1.162667in}}%
\pgfpathlineto{\pgfqpoint{3.545475in}{1.181277in}}%
\pgfpathlineto{\pgfqpoint{3.539247in}{1.200000in}}%
\pgfpathlineto{\pgfqpoint{3.522074in}{1.237333in}}%
\pgfpathlineto{\pgfqpoint{3.508931in}{1.259238in}}%
\pgfpathlineto{\pgfqpoint{3.502269in}{1.274667in}}%
\pgfpathlineto{\pgfqpoint{3.477542in}{1.319332in}}%
\pgfpathlineto{\pgfqpoint{3.435387in}{1.386667in}}%
\pgfpathlineto{\pgfqpoint{3.422965in}{1.403165in}}%
\pgfpathlineto{\pgfqpoint{3.405253in}{1.431740in}}%
\pgfpathlineto{\pgfqpoint{3.357213in}{1.498667in}}%
\pgfpathlineto{\pgfqpoint{3.343088in}{1.515430in}}%
\pgfpathlineto{\pgfqpoint{3.325091in}{1.541140in}}%
\pgfpathlineto{\pgfqpoint{3.269069in}{1.610667in}}%
\pgfpathlineto{\pgfqpoint{3.204848in}{1.686397in}}%
\pgfpathlineto{\pgfqpoint{3.186188in}{1.705285in}}%
\pgfpathlineto{\pgfqpoint{3.164768in}{1.730699in}}%
\pgfpathlineto{\pgfqpoint{3.149428in}{1.745712in}}%
\pgfpathlineto{\pgfqpoint{3.124687in}{1.773591in}}%
\pgfpathlineto{\pgfqpoint{3.112038in}{1.785552in}}%
\pgfpathlineto{\pgfqpoint{3.101832in}{1.797333in}}%
\pgfpathlineto{\pgfqpoint{3.044525in}{1.855262in}}%
\pgfpathlineto{\pgfqpoint{2.988245in}{1.909333in}}%
\pgfpathlineto{\pgfqpoint{2.975766in}{1.919954in}}%
\pgfpathlineto{\pgfqpoint{2.964364in}{1.931638in}}%
\pgfpathlineto{\pgfqpoint{2.955773in}{1.938665in}}%
\pgfpathlineto{\pgfqpoint{2.947823in}{1.946667in}}%
\pgfpathlineto{\pgfqpoint{2.935324in}{1.956951in}}%
\pgfpathlineto{\pgfqpoint{2.924283in}{1.967909in}}%
\pgfpathlineto{\pgfqpoint{2.862459in}{2.021333in}}%
\pgfpathlineto{\pgfqpoint{2.844121in}{2.036745in}}%
\pgfpathlineto{\pgfqpoint{2.770119in}{2.096000in}}%
\pgfpathlineto{\pgfqpoint{2.743622in}{2.114390in}}%
\pgfpathlineto{\pgfqpoint{2.720390in}{2.133333in}}%
\pgfpathlineto{\pgfqpoint{2.643717in}{2.186198in}}%
\pgfpathlineto{\pgfqpoint{2.629012in}{2.194303in}}%
\pgfpathlineto{\pgfqpoint{2.603636in}{2.212080in}}%
\pgfpathlineto{\pgfqpoint{2.563556in}{2.235772in}}%
\pgfpathlineto{\pgfqpoint{2.493628in}{2.273135in}}%
\pgfpathlineto{\pgfqpoint{2.472295in}{2.282667in}}%
\pgfpathlineto{\pgfqpoint{2.443313in}{2.295048in}}%
\pgfpathlineto{\pgfqpoint{2.422680in}{2.300781in}}%
\pgfpathlineto{\pgfqpoint{2.403232in}{2.309552in}}%
\pgfpathlineto{\pgfqpoint{2.362025in}{2.321049in}}%
\pgfpathlineto{\pgfqpoint{2.323071in}{2.325973in}}%
\pgfpathlineto{\pgfqpoint{2.279458in}{2.323290in}}%
\pgfpathlineto{\pgfqpoint{2.271216in}{2.320000in}}%
\pgfpathlineto{\pgfqpoint{2.242909in}{2.304645in}}%
\pgfpathlineto{\pgfqpoint{2.231177in}{2.293595in}}%
\pgfpathlineto{\pgfqpoint{2.225381in}{2.282667in}}%
\pgfpathlineto{\pgfqpoint{2.220219in}{2.261532in}}%
\pgfpathlineto{\pgfqpoint{2.213585in}{2.245333in}}%
\pgfpathlineto{\pgfqpoint{2.212038in}{2.236755in}}%
\pgfpathlineto{\pgfqpoint{2.212841in}{2.208000in}}%
\pgfpathlineto{\pgfqpoint{2.217935in}{2.184738in}}%
\pgfpathlineto{\pgfqpoint{2.218425in}{2.170667in}}%
\pgfpathlineto{\pgfqpoint{2.221965in}{2.152842in}}%
\pgfpathlineto{\pgfqpoint{2.241063in}{2.094281in}}%
\pgfpathlineto{\pgfqpoint{2.242909in}{2.089274in}}%
\pgfpathlineto{\pgfqpoint{2.253022in}{2.068086in}}%
\pgfpathlineto{\pgfqpoint{2.255846in}{2.058667in}}%
\pgfpathlineto{\pgfqpoint{2.273084in}{2.021333in}}%
\pgfpathlineto{\pgfqpoint{2.276693in}{2.015468in}}%
\pgfpathlineto{\pgfqpoint{2.292207in}{1.984000in}}%
\pgfpathlineto{\pgfqpoint{2.303753in}{1.966006in}}%
\pgfpathlineto{\pgfqpoint{2.312990in}{1.946667in}}%
\pgfpathlineto{\pgfqpoint{2.351356in}{1.882987in}}%
\pgfpathlineto{\pgfqpoint{2.363152in}{1.864985in}}%
\pgfpathlineto{\pgfqpoint{2.376482in}{1.847084in}}%
\pgfpathlineto{\pgfqpoint{2.383657in}{1.834667in}}%
\pgfpathlineto{\pgfqpoint{2.391906in}{1.824117in}}%
\pgfpathlineto{\pgfqpoint{2.409325in}{1.797333in}}%
\pgfpathlineto{\pgfqpoint{2.423937in}{1.779286in}}%
\pgfpathlineto{\pgfqpoint{2.443313in}{1.750773in}}%
\pgfpathlineto{\pgfqpoint{2.456550in}{1.734996in}}%
\pgfpathlineto{\pgfqpoint{2.464813in}{1.722667in}}%
\pgfpathlineto{\pgfqpoint{2.473111in}{1.713089in}}%
\pgfpathlineto{\pgfqpoint{2.493881in}{1.685333in}}%
\pgfpathlineto{\pgfqpoint{2.524372in}{1.647164in}}%
\pgfpathlineto{\pgfqpoint{2.563556in}{1.600973in}}%
\pgfpathlineto{\pgfqpoint{2.577544in}{1.586363in}}%
\pgfpathlineto{\pgfqpoint{2.587718in}{1.573333in}}%
\pgfpathlineto{\pgfqpoint{2.621025in}{1.536000in}}%
\pgfpathlineto{\pgfqpoint{2.631998in}{1.525085in}}%
\pgfpathlineto{\pgfqpoint{2.655272in}{1.498667in}}%
\pgfpathlineto{\pgfqpoint{2.669295in}{1.485157in}}%
\pgfpathlineto{\pgfqpoint{2.690532in}{1.461333in}}%
\pgfpathlineto{\pgfqpoint{2.707197in}{1.445795in}}%
\pgfpathlineto{\pgfqpoint{2.726888in}{1.424000in}}%
\pgfpathlineto{\pgfqpoint{2.764431in}{1.386667in}}%
\pgfpathlineto{\pgfqpoint{2.784891in}{1.368830in}}%
\pgfpathlineto{\pgfqpoint{2.784891in}{1.368830in}}%
\pgfusepath{fill}%
\end{pgfscope}%
\begin{pgfscope}%
\pgfpathrectangle{\pgfqpoint{0.800000in}{0.528000in}}{\pgfqpoint{3.968000in}{3.696000in}}%
\pgfusepath{clip}%
\pgfsetbuttcap%
\pgfsetroundjoin%
\definecolor{currentfill}{rgb}{0.269944,0.014625,0.341379}%
\pgfsetfillcolor{currentfill}%
\pgfsetlinewidth{0.000000pt}%
\definecolor{currentstroke}{rgb}{0.000000,0.000000,0.000000}%
\pgfsetstrokecolor{currentstroke}%
\pgfsetdash{}{0pt}%
\pgfpathmoveto{\pgfqpoint{2.784891in}{1.368830in}}%
\pgfpathlineto{\pgfqpoint{2.763960in}{1.387123in}}%
\pgfpathlineto{\pgfqpoint{2.723879in}{1.427016in}}%
\pgfpathlineto{\pgfqpoint{2.707197in}{1.445795in}}%
\pgfpathlineto{\pgfqpoint{2.683798in}{1.468315in}}%
\pgfpathlineto{\pgfqpoint{2.669295in}{1.485157in}}%
\pgfpathlineto{\pgfqpoint{2.643717in}{1.511053in}}%
\pgfpathlineto{\pgfqpoint{2.631998in}{1.525085in}}%
\pgfpathlineto{\pgfqpoint{2.621025in}{1.536000in}}%
\pgfpathlineto{\pgfqpoint{2.587718in}{1.573333in}}%
\pgfpathlineto{\pgfqpoint{2.577544in}{1.586363in}}%
\pgfpathlineto{\pgfqpoint{2.555283in}{1.610667in}}%
\pgfpathlineto{\pgfqpoint{2.523475in}{1.648253in}}%
\pgfpathlineto{\pgfqpoint{2.464813in}{1.722667in}}%
\pgfpathlineto{\pgfqpoint{2.456550in}{1.734996in}}%
\pgfpathlineto{\pgfqpoint{2.436425in}{1.760000in}}%
\pgfpathlineto{\pgfqpoint{2.423937in}{1.779286in}}%
\pgfpathlineto{\pgfqpoint{2.403232in}{1.806020in}}%
\pgfpathlineto{\pgfqpoint{2.391906in}{1.824117in}}%
\pgfpathlineto{\pgfqpoint{2.383657in}{1.834667in}}%
\pgfpathlineto{\pgfqpoint{2.376482in}{1.847084in}}%
\pgfpathlineto{\pgfqpoint{2.358526in}{1.872000in}}%
\pgfpathlineto{\pgfqpoint{2.335252in}{1.909333in}}%
\pgfpathlineto{\pgfqpoint{2.312990in}{1.946667in}}%
\pgfpathlineto{\pgfqpoint{2.303753in}{1.966006in}}%
\pgfpathlineto{\pgfqpoint{2.292207in}{1.984000in}}%
\pgfpathlineto{\pgfqpoint{2.264535in}{2.038523in}}%
\pgfpathlineto{\pgfqpoint{2.255846in}{2.058667in}}%
\pgfpathlineto{\pgfqpoint{2.253022in}{2.068086in}}%
\pgfpathlineto{\pgfqpoint{2.238512in}{2.100096in}}%
\pgfpathlineto{\pgfqpoint{2.221965in}{2.152842in}}%
\pgfpathlineto{\pgfqpoint{2.218425in}{2.170667in}}%
\pgfpathlineto{\pgfqpoint{2.217935in}{2.184738in}}%
\pgfpathlineto{\pgfqpoint{2.212841in}{2.208000in}}%
\pgfpathlineto{\pgfqpoint{2.212038in}{2.236755in}}%
\pgfpathlineto{\pgfqpoint{2.213585in}{2.245333in}}%
\pgfpathlineto{\pgfqpoint{2.220219in}{2.261532in}}%
\pgfpathlineto{\pgfqpoint{2.225381in}{2.282667in}}%
\pgfpathlineto{\pgfqpoint{2.231177in}{2.293595in}}%
\pgfpathlineto{\pgfqpoint{2.242909in}{2.304645in}}%
\pgfpathlineto{\pgfqpoint{2.279458in}{2.323290in}}%
\pgfpathlineto{\pgfqpoint{2.286646in}{2.323406in}}%
\pgfpathlineto{\pgfqpoint{2.323071in}{2.325973in}}%
\pgfpathlineto{\pgfqpoint{2.365957in}{2.320000in}}%
\pgfpathlineto{\pgfqpoint{2.403232in}{2.309552in}}%
\pgfpathlineto{\pgfqpoint{2.422680in}{2.300781in}}%
\pgfpathlineto{\pgfqpoint{2.443313in}{2.295048in}}%
\pgfpathlineto{\pgfqpoint{2.493628in}{2.273135in}}%
\pgfpathlineto{\pgfqpoint{2.563556in}{2.235772in}}%
\pgfpathlineto{\pgfqpoint{2.609993in}{2.208000in}}%
\pgfpathlineto{\pgfqpoint{2.629012in}{2.194303in}}%
\pgfpathlineto{\pgfqpoint{2.643717in}{2.186198in}}%
\pgfpathlineto{\pgfqpoint{2.720390in}{2.133333in}}%
\pgfpathlineto{\pgfqpoint{2.723879in}{2.130856in}}%
\pgfpathlineto{\pgfqpoint{2.743622in}{2.114390in}}%
\pgfpathlineto{\pgfqpoint{2.770119in}{2.096000in}}%
\pgfpathlineto{\pgfqpoint{2.844121in}{2.036745in}}%
\pgfpathlineto{\pgfqpoint{2.852404in}{2.029048in}}%
\pgfpathlineto{\pgfqpoint{2.862459in}{2.021333in}}%
\pgfpathlineto{\pgfqpoint{2.924283in}{1.967909in}}%
\pgfpathlineto{\pgfqpoint{2.935324in}{1.956951in}}%
\pgfpathlineto{\pgfqpoint{2.947823in}{1.946667in}}%
\pgfpathlineto{\pgfqpoint{2.955773in}{1.938665in}}%
\pgfpathlineto{\pgfqpoint{2.964364in}{1.931638in}}%
\pgfpathlineto{\pgfqpoint{2.975766in}{1.919954in}}%
\pgfpathlineto{\pgfqpoint{3.004444in}{1.894098in}}%
\pgfpathlineto{\pgfqpoint{3.015556in}{1.882350in}}%
\pgfpathlineto{\pgfqpoint{3.027319in}{1.872000in}}%
\pgfpathlineto{\pgfqpoint{3.065150in}{1.834667in}}%
\pgfpathlineto{\pgfqpoint{3.073999in}{1.824787in}}%
\pgfpathlineto{\pgfqpoint{3.084606in}{1.815102in}}%
\pgfpathlineto{\pgfqpoint{3.137449in}{1.760000in}}%
\pgfpathlineto{\pgfqpoint{3.149428in}{1.745712in}}%
\pgfpathlineto{\pgfqpoint{3.172078in}{1.722667in}}%
\pgfpathlineto{\pgfqpoint{3.186188in}{1.705285in}}%
\pgfpathlineto{\pgfqpoint{3.209951in}{1.680581in}}%
\pgfpathlineto{\pgfqpoint{3.269069in}{1.610667in}}%
\pgfpathlineto{\pgfqpoint{3.275510in}{1.601818in}}%
\pgfpathlineto{\pgfqpoint{3.285010in}{1.591302in}}%
\pgfpathlineto{\pgfqpoint{3.329089in}{1.536000in}}%
\pgfpathlineto{\pgfqpoint{3.343088in}{1.515430in}}%
\pgfpathlineto{\pgfqpoint{3.365172in}{1.487859in}}%
\pgfpathlineto{\pgfqpoint{3.375999in}{1.471418in}}%
\pgfpathlineto{\pgfqpoint{3.384213in}{1.461333in}}%
\pgfpathlineto{\pgfqpoint{3.410620in}{1.424000in}}%
\pgfpathlineto{\pgfqpoint{3.422965in}{1.403165in}}%
\pgfpathlineto{\pgfqpoint{3.435387in}{1.386667in}}%
\pgfpathlineto{\pgfqpoint{3.477542in}{1.319332in}}%
\pgfpathlineto{\pgfqpoint{3.502269in}{1.274667in}}%
\pgfpathlineto{\pgfqpoint{3.508931in}{1.259238in}}%
\pgfpathlineto{\pgfqpoint{3.525495in}{1.230079in}}%
\pgfpathlineto{\pgfqpoint{3.539247in}{1.200000in}}%
\pgfpathlineto{\pgfqpoint{3.545475in}{1.181277in}}%
\pgfpathlineto{\pgfqpoint{3.554846in}{1.162667in}}%
\pgfpathlineto{\pgfqpoint{3.569627in}{1.121560in}}%
\pgfpathlineto{\pgfqpoint{3.577866in}{1.088000in}}%
\pgfpathlineto{\pgfqpoint{3.580067in}{1.064165in}}%
\pgfpathlineto{\pgfqpoint{3.583732in}{1.050667in}}%
\pgfpathlineto{\pgfqpoint{3.585491in}{1.032116in}}%
\pgfpathlineto{\pgfqpoint{3.583934in}{1.013333in}}%
\pgfpathlineto{\pgfqpoint{3.575830in}{0.985551in}}%
\pgfpathlineto{\pgfqpoint{3.574701in}{0.976000in}}%
\pgfpathlineto{\pgfqpoint{3.572346in}{0.969694in}}%
\pgfpathlineto{\pgfqpoint{3.565576in}{0.961079in}}%
\pgfpathlineto{\pgfqpoint{3.537830in}{0.938667in}}%
\pgfpathlineto{\pgfqpoint{3.525495in}{0.933152in}}%
\pgfpathlineto{\pgfqpoint{3.497123in}{0.927761in}}%
\pgfpathlineto{\pgfqpoint{3.485414in}{0.927865in}}%
\pgfpathlineto{\pgfqpoint{3.476037in}{0.929932in}}%
\pgfpathlineto{\pgfqpoint{3.445333in}{0.931371in}}%
\pgfpathlineto{\pgfqpoint{3.402533in}{0.941200in}}%
\pgfpathlineto{\pgfqpoint{3.365172in}{0.954511in}}%
\pgfpathlineto{\pgfqpoint{3.350303in}{0.962151in}}%
\pgfpathlineto{\pgfqpoint{3.325091in}{0.970853in}}%
\pgfpathlineto{\pgfqpoint{3.285010in}{0.990384in}}%
\pgfpathlineto{\pgfqpoint{3.270301in}{0.999632in}}%
\pgfpathlineto{\pgfqpoint{3.241678in}{1.013333in}}%
\pgfpathlineto{\pgfqpoint{3.220190in}{1.027623in}}%
\pgfpathlineto{\pgfqpoint{3.204848in}{1.035226in}}%
\pgfpathlineto{\pgfqpoint{3.195504in}{1.041963in}}%
\pgfpathlineto{\pgfqpoint{3.164768in}{1.060233in}}%
\pgfpathlineto{\pgfqpoint{3.122737in}{1.088000in}}%
\pgfpathlineto{\pgfqpoint{3.102083in}{1.104279in}}%
\pgfpathlineto{\pgfqpoint{3.084606in}{1.115131in}}%
\pgfpathlineto{\pgfqpoint{3.021248in}{1.162667in}}%
\pgfpathlineto{\pgfqpoint{3.004444in}{1.175655in}}%
\pgfpathlineto{\pgfqpoint{2.924283in}{1.241020in}}%
\pgfpathlineto{\pgfqpoint{2.843569in}{1.312000in}}%
\pgfpathlineto{\pgfqpoint{2.824721in}{1.331263in}}%
\pgfpathlineto{\pgfqpoint{2.803352in}{1.349333in}}%
\pgfpathmoveto{\pgfqpoint{2.763960in}{1.360807in}}%
\pgfpathlineto{\pgfqpoint{2.775934in}{1.349333in}}%
\pgfpathlineto{\pgfqpoint{2.844121in}{1.285561in}}%
\pgfpathlineto{\pgfqpoint{2.898478in}{1.237333in}}%
\pgfpathlineto{\pgfqpoint{2.912429in}{1.226292in}}%
\pgfpathlineto{\pgfqpoint{2.924283in}{1.215066in}}%
\pgfpathlineto{\pgfqpoint{3.004444in}{1.149140in}}%
\pgfpathlineto{\pgfqpoint{3.019045in}{1.138933in}}%
\pgfpathlineto{\pgfqpoint{3.044525in}{1.117837in}}%
\pgfpathlineto{\pgfqpoint{3.086492in}{1.086244in}}%
\pgfpathlineto{\pgfqpoint{3.164768in}{1.032067in}}%
\pgfpathlineto{\pgfqpoint{3.177308in}{1.025014in}}%
\pgfpathlineto{\pgfqpoint{3.204848in}{1.005993in}}%
\pgfpathlineto{\pgfqpoint{3.254837in}{0.976000in}}%
\pgfpathlineto{\pgfqpoint{3.274396in}{0.966113in}}%
\pgfpathlineto{\pgfqpoint{3.285010in}{0.959153in}}%
\pgfpathlineto{\pgfqpoint{3.326930in}{0.936954in}}%
\pgfpathlineto{\pgfqpoint{3.365172in}{0.919739in}}%
\pgfpathlineto{\pgfqpoint{3.379973in}{0.915120in}}%
\pgfpathlineto{\pgfqpoint{3.410838in}{0.901333in}}%
\pgfpathlineto{\pgfqpoint{3.437369in}{0.893915in}}%
\pgfpathlineto{\pgfqpoint{3.445333in}{0.890437in}}%
\pgfpathlineto{\pgfqpoint{3.469529in}{0.886537in}}%
\pgfpathlineto{\pgfqpoint{3.485414in}{0.881389in}}%
\pgfpathlineto{\pgfqpoint{3.510817in}{0.877672in}}%
\pgfpathlineto{\pgfqpoint{3.525495in}{0.877543in}}%
\pgfpathlineto{\pgfqpoint{3.550062in}{0.878450in}}%
\pgfpathlineto{\pgfqpoint{3.565576in}{0.882213in}}%
\pgfpathlineto{\pgfqpoint{3.605657in}{0.902674in}}%
\pgfpathlineto{\pgfqpoint{3.626953in}{0.938667in}}%
\pgfpathlineto{\pgfqpoint{3.629815in}{0.961169in}}%
\pgfpathlineto{\pgfqpoint{3.634127in}{0.976000in}}%
\pgfpathlineto{\pgfqpoint{3.631846in}{1.000395in}}%
\pgfpathlineto{\pgfqpoint{3.632862in}{1.013333in}}%
\pgfpathlineto{\pgfqpoint{3.627738in}{1.033901in}}%
\pgfpathlineto{\pgfqpoint{3.626410in}{1.050667in}}%
\pgfpathlineto{\pgfqpoint{3.621230in}{1.073494in}}%
\pgfpathlineto{\pgfqpoint{3.604100in}{1.125333in}}%
\pgfpathlineto{\pgfqpoint{3.592231in}{1.150162in}}%
\pgfpathlineto{\pgfqpoint{3.588295in}{1.162667in}}%
\pgfpathlineto{\pgfqpoint{3.569307in}{1.203476in}}%
\pgfpathlineto{\pgfqpoint{3.565576in}{1.211590in}}%
\pgfpathlineto{\pgfqpoint{3.552254in}{1.237333in}}%
\pgfpathlineto{\pgfqpoint{3.542359in}{1.253041in}}%
\pgfpathlineto{\pgfqpoint{3.525495in}{1.285902in}}%
\pgfpathlineto{\pgfqpoint{3.509955in}{1.312000in}}%
\pgfpathlineto{\pgfqpoint{3.500265in}{1.325833in}}%
\pgfpathlineto{\pgfqpoint{3.485414in}{1.352233in}}%
\pgfpathlineto{\pgfqpoint{3.470500in}{1.372775in}}%
\pgfpathlineto{\pgfqpoint{3.462558in}{1.386667in}}%
\pgfpathlineto{\pgfqpoint{3.437341in}{1.424000in}}%
\pgfpathlineto{\pgfqpoint{3.423761in}{1.441239in}}%
\pgfpathlineto{\pgfqpoint{3.405253in}{1.469136in}}%
\pgfpathlineto{\pgfqpoint{3.391576in}{1.485928in}}%
\pgfpathlineto{\pgfqpoint{3.383187in}{1.498667in}}%
\pgfpathlineto{\pgfqpoint{3.375267in}{1.508070in}}%
\pgfpathlineto{\pgfqpoint{3.354816in}{1.536000in}}%
\pgfpathlineto{\pgfqpoint{3.325091in}{1.574120in}}%
\pgfpathlineto{\pgfqpoint{3.263446in}{1.648000in}}%
\pgfpathlineto{\pgfqpoint{3.254784in}{1.657179in}}%
\pgfpathlineto{\pgfqpoint{3.231172in}{1.685333in}}%
\pgfpathlineto{\pgfqpoint{3.218664in}{1.698202in}}%
\pgfpathlineto{\pgfqpoint{3.198070in}{1.722667in}}%
\pgfpathlineto{\pgfqpoint{3.181995in}{1.738713in}}%
\pgfpathlineto{\pgfqpoint{3.159276in}{1.765115in}}%
\pgfpathlineto{\pgfqpoint{3.084606in}{1.842435in}}%
\pgfpathlineto{\pgfqpoint{3.044525in}{1.882152in}}%
\pgfpathlineto{\pgfqpoint{3.029519in}{1.895355in}}%
\pgfpathlineto{\pgfqpoint{3.004444in}{1.920676in}}%
\pgfpathlineto{\pgfqpoint{2.989865in}{1.933086in}}%
\pgfpathlineto{\pgfqpoint{2.964364in}{1.958031in}}%
\pgfpathlineto{\pgfqpoint{2.949560in}{1.970212in}}%
\pgfpathlineto{\pgfqpoint{2.924283in}{1.994237in}}%
\pgfpathlineto{\pgfqpoint{2.908587in}{2.006713in}}%
\pgfpathlineto{\pgfqpoint{2.884202in}{2.029317in}}%
\pgfpathlineto{\pgfqpoint{2.866924in}{2.042573in}}%
\pgfpathlineto{\pgfqpoint{2.844121in}{2.063290in}}%
\pgfpathlineto{\pgfqpoint{2.804040in}{2.096178in}}%
\pgfpathlineto{\pgfqpoint{2.781447in}{2.112288in}}%
\pgfpathlineto{\pgfqpoint{2.756189in}{2.133333in}}%
\pgfpathlineto{\pgfqpoint{2.737587in}{2.146102in}}%
\pgfpathlineto{\pgfqpoint{2.723879in}{2.157584in}}%
\pgfpathlineto{\pgfqpoint{2.653081in}{2.208000in}}%
\pgfpathlineto{\pgfqpoint{2.643717in}{2.214485in}}%
\pgfpathlineto{\pgfqpoint{2.596383in}{2.245333in}}%
\pgfpathlineto{\pgfqpoint{2.575812in}{2.256749in}}%
\pgfpathlineto{\pgfqpoint{2.563556in}{2.265389in}}%
\pgfpathlineto{\pgfqpoint{2.551406in}{2.271350in}}%
\pgfpathlineto{\pgfqpoint{2.523475in}{2.288777in}}%
\pgfpathlineto{\pgfqpoint{2.501052in}{2.299115in}}%
\pgfpathlineto{\pgfqpoint{2.483394in}{2.309993in}}%
\pgfpathlineto{\pgfqpoint{2.443313in}{2.329291in}}%
\pgfpathlineto{\pgfqpoint{2.421393in}{2.336916in}}%
\pgfpathlineto{\pgfqpoint{2.403232in}{2.346118in}}%
\pgfpathlineto{\pgfqpoint{2.393683in}{2.348439in}}%
\pgfpathlineto{\pgfqpoint{2.358000in}{2.362131in}}%
\pgfpathlineto{\pgfqpoint{2.323071in}{2.370448in}}%
\pgfpathlineto{\pgfqpoint{2.298907in}{2.372159in}}%
\pgfpathlineto{\pgfqpoint{2.282990in}{2.375890in}}%
\pgfpathlineto{\pgfqpoint{2.260068in}{2.373316in}}%
\pgfpathlineto{\pgfqpoint{2.242909in}{2.374132in}}%
\pgfpathlineto{\pgfqpoint{2.199325in}{2.357333in}}%
\pgfpathlineto{\pgfqpoint{2.193529in}{2.348672in}}%
\pgfpathlineto{\pgfqpoint{2.171646in}{2.320000in}}%
\pgfpathlineto{\pgfqpoint{2.162586in}{2.282517in}}%
\pgfpathlineto{\pgfqpoint{2.163097in}{2.245333in}}%
\pgfpathlineto{\pgfqpoint{2.169401in}{2.208000in}}%
\pgfpathlineto{\pgfqpoint{2.170746in}{2.200550in}}%
\pgfpathlineto{\pgfqpoint{2.179357in}{2.170667in}}%
\pgfpathlineto{\pgfqpoint{2.186033in}{2.155023in}}%
\pgfpathlineto{\pgfqpoint{2.191788in}{2.133333in}}%
\pgfpathlineto{\pgfqpoint{2.206420in}{2.096000in}}%
\pgfpathlineto{\pgfqpoint{2.218458in}{2.073225in}}%
\pgfpathlineto{\pgfqpoint{2.223825in}{2.058667in}}%
\pgfpathlineto{\pgfqpoint{2.242909in}{2.019600in}}%
\pgfpathlineto{\pgfqpoint{2.256834in}{1.996971in}}%
\pgfpathlineto{\pgfqpoint{2.262776in}{1.984000in}}%
\pgfpathlineto{\pgfqpoint{2.284998in}{1.944796in}}%
\pgfpathlineto{\pgfqpoint{2.331308in}{1.872000in}}%
\pgfpathlineto{\pgfqpoint{2.344460in}{1.854590in}}%
\pgfpathlineto{\pgfqpoint{2.363152in}{1.825187in}}%
\pgfpathlineto{\pgfqpoint{2.375752in}{1.809070in}}%
\pgfpathlineto{\pgfqpoint{2.383098in}{1.797333in}}%
\pgfpathlineto{\pgfqpoint{2.426226in}{1.738582in}}%
\pgfpathlineto{\pgfqpoint{2.443313in}{1.716509in}}%
\pgfpathlineto{\pgfqpoint{2.458260in}{1.699256in}}%
\pgfpathlineto{\pgfqpoint{2.468133in}{1.685333in}}%
\pgfpathlineto{\pgfqpoint{2.498427in}{1.648000in}}%
\pgfpathlineto{\pgfqpoint{2.509961in}{1.635413in}}%
\pgfpathlineto{\pgfqpoint{2.529450in}{1.610667in}}%
\pgfpathlineto{\pgfqpoint{2.563556in}{1.570973in}}%
\pgfpathlineto{\pgfqpoint{2.603636in}{1.526317in}}%
\pgfpathlineto{\pgfqpoint{2.617991in}{1.512038in}}%
\pgfpathlineto{\pgfqpoint{2.629158in}{1.498667in}}%
\pgfpathlineto{\pgfqpoint{2.683798in}{1.440992in}}%
\pgfpathlineto{\pgfqpoint{2.712198in}{1.413120in}}%
\pgfpathlineto{\pgfqpoint{2.737635in}{1.386667in}}%
\pgfpathlineto{\pgfqpoint{2.737635in}{1.386667in}}%
\pgfusepath{fill}%
\end{pgfscope}%
\begin{pgfscope}%
\pgfpathrectangle{\pgfqpoint{0.800000in}{0.528000in}}{\pgfqpoint{3.968000in}{3.696000in}}%
\pgfusepath{clip}%
\pgfsetbuttcap%
\pgfsetroundjoin%
\definecolor{currentfill}{rgb}{0.269944,0.014625,0.341379}%
\pgfsetfillcolor{currentfill}%
\pgfsetlinewidth{0.000000pt}%
\definecolor{currentstroke}{rgb}{0.000000,0.000000,0.000000}%
\pgfsetstrokecolor{currentstroke}%
\pgfsetdash{}{0pt}%
\pgfpathmoveto{\pgfqpoint{2.737635in}{1.386667in}}%
\pgfpathlineto{\pgfqpoint{2.700458in}{1.424000in}}%
\pgfpathlineto{\pgfqpoint{2.692914in}{1.432491in}}%
\pgfpathlineto{\pgfqpoint{2.683798in}{1.440992in}}%
\pgfpathlineto{\pgfqpoint{2.629158in}{1.498667in}}%
\pgfpathlineto{\pgfqpoint{2.617991in}{1.512038in}}%
\pgfpathlineto{\pgfqpoint{2.594899in}{1.536000in}}%
\pgfpathlineto{\pgfqpoint{2.529450in}{1.610667in}}%
\pgfpathlineto{\pgfqpoint{2.523475in}{1.617722in}}%
\pgfpathlineto{\pgfqpoint{2.509961in}{1.635413in}}%
\pgfpathlineto{\pgfqpoint{2.498427in}{1.648000in}}%
\pgfpathlineto{\pgfqpoint{2.468133in}{1.685333in}}%
\pgfpathlineto{\pgfqpoint{2.458260in}{1.699256in}}%
\pgfpathlineto{\pgfqpoint{2.438520in}{1.722667in}}%
\pgfpathlineto{\pgfqpoint{2.403232in}{1.769468in}}%
\pgfpathlineto{\pgfqpoint{2.356519in}{1.834667in}}%
\pgfpathlineto{\pgfqpoint{2.344460in}{1.854590in}}%
\pgfpathlineto{\pgfqpoint{2.323071in}{1.884623in}}%
\pgfpathlineto{\pgfqpoint{2.307339in}{1.909333in}}%
\pgfpathlineto{\pgfqpoint{2.282990in}{1.948233in}}%
\pgfpathlineto{\pgfqpoint{2.262776in}{1.984000in}}%
\pgfpathlineto{\pgfqpoint{2.256834in}{1.996971in}}%
\pgfpathlineto{\pgfqpoint{2.241023in}{2.023090in}}%
\pgfpathlineto{\pgfqpoint{2.223825in}{2.058667in}}%
\pgfpathlineto{\pgfqpoint{2.218458in}{2.073225in}}%
\pgfpathlineto{\pgfqpoint{2.202828in}{2.104954in}}%
\pgfpathlineto{\pgfqpoint{2.191788in}{2.133333in}}%
\pgfpathlineto{\pgfqpoint{2.186033in}{2.155023in}}%
\pgfpathlineto{\pgfqpoint{2.179357in}{2.170667in}}%
\pgfpathlineto{\pgfqpoint{2.169401in}{2.208000in}}%
\pgfpathlineto{\pgfqpoint{2.169261in}{2.214067in}}%
\pgfpathlineto{\pgfqpoint{2.163097in}{2.245333in}}%
\pgfpathlineto{\pgfqpoint{2.162747in}{2.283834in}}%
\pgfpathlineto{\pgfqpoint{2.171646in}{2.320000in}}%
\pgfpathlineto{\pgfqpoint{2.202828in}{2.359941in}}%
\pgfpathlineto{\pgfqpoint{2.206546in}{2.360796in}}%
\pgfpathlineto{\pgfqpoint{2.242909in}{2.374132in}}%
\pgfpathlineto{\pgfqpoint{2.260068in}{2.373316in}}%
\pgfpathlineto{\pgfqpoint{2.282990in}{2.375890in}}%
\pgfpathlineto{\pgfqpoint{2.298907in}{2.372159in}}%
\pgfpathlineto{\pgfqpoint{2.323071in}{2.370448in}}%
\pgfpathlineto{\pgfqpoint{2.363152in}{2.360465in}}%
\pgfpathlineto{\pgfqpoint{2.403232in}{2.346118in}}%
\pgfpathlineto{\pgfqpoint{2.421393in}{2.336916in}}%
\pgfpathlineto{\pgfqpoint{2.443313in}{2.329291in}}%
\pgfpathlineto{\pgfqpoint{2.483394in}{2.309993in}}%
\pgfpathlineto{\pgfqpoint{2.501052in}{2.299115in}}%
\pgfpathlineto{\pgfqpoint{2.527281in}{2.286212in}}%
\pgfpathlineto{\pgfqpoint{2.534001in}{2.282667in}}%
\pgfpathlineto{\pgfqpoint{2.551406in}{2.271350in}}%
\pgfpathlineto{\pgfqpoint{2.563556in}{2.265389in}}%
\pgfpathlineto{\pgfqpoint{2.575812in}{2.256749in}}%
\pgfpathlineto{\pgfqpoint{2.603636in}{2.240870in}}%
\pgfpathlineto{\pgfqpoint{2.653081in}{2.208000in}}%
\pgfpathlineto{\pgfqpoint{2.723879in}{2.157584in}}%
\pgfpathlineto{\pgfqpoint{2.737587in}{2.146102in}}%
\pgfpathlineto{\pgfqpoint{2.763960in}{2.127462in}}%
\pgfpathlineto{\pgfqpoint{2.781447in}{2.112288in}}%
\pgfpathlineto{\pgfqpoint{2.804259in}{2.096000in}}%
\pgfpathlineto{\pgfqpoint{2.849605in}{2.058667in}}%
\pgfpathlineto{\pgfqpoint{2.866924in}{2.042573in}}%
\pgfpathlineto{\pgfqpoint{2.893367in}{2.021333in}}%
\pgfpathlineto{\pgfqpoint{2.908587in}{2.006713in}}%
\pgfpathlineto{\pgfqpoint{2.935666in}{1.984000in}}%
\pgfpathlineto{\pgfqpoint{2.949560in}{1.970212in}}%
\pgfpathlineto{\pgfqpoint{2.976610in}{1.946667in}}%
\pgfpathlineto{\pgfqpoint{2.989865in}{1.933086in}}%
\pgfpathlineto{\pgfqpoint{3.016298in}{1.909333in}}%
\pgfpathlineto{\pgfqpoint{3.029519in}{1.895355in}}%
\pgfpathlineto{\pgfqpoint{3.054819in}{1.872000in}}%
\pgfpathlineto{\pgfqpoint{3.128674in}{1.797333in}}%
\pgfpathlineto{\pgfqpoint{3.164768in}{1.759262in}}%
\pgfpathlineto{\pgfqpoint{3.181995in}{1.738713in}}%
\pgfpathlineto{\pgfqpoint{3.204848in}{1.715165in}}%
\pgfpathlineto{\pgfqpoint{3.218664in}{1.698202in}}%
\pgfpathlineto{\pgfqpoint{3.244929in}{1.669630in}}%
\pgfpathlineto{\pgfqpoint{3.263446in}{1.648000in}}%
\pgfpathlineto{\pgfqpoint{3.327588in}{1.571007in}}%
\pgfpathlineto{\pgfqpoint{3.383187in}{1.498667in}}%
\pgfpathlineto{\pgfqpoint{3.391576in}{1.485928in}}%
\pgfpathlineto{\pgfqpoint{3.410951in}{1.461333in}}%
\pgfpathlineto{\pgfqpoint{3.423761in}{1.441239in}}%
\pgfpathlineto{\pgfqpoint{3.445333in}{1.412364in}}%
\pgfpathlineto{\pgfqpoint{3.462558in}{1.386667in}}%
\pgfpathlineto{\pgfqpoint{3.470500in}{1.372775in}}%
\pgfpathlineto{\pgfqpoint{3.487294in}{1.349333in}}%
\pgfpathlineto{\pgfqpoint{3.500265in}{1.325833in}}%
\pgfpathlineto{\pgfqpoint{3.509955in}{1.312000in}}%
\pgfpathlineto{\pgfqpoint{3.532018in}{1.274667in}}%
\pgfpathlineto{\pgfqpoint{3.542359in}{1.253041in}}%
\pgfpathlineto{\pgfqpoint{3.552254in}{1.237333in}}%
\pgfpathlineto{\pgfqpoint{3.576435in}{1.189886in}}%
\pgfpathlineto{\pgfqpoint{3.588295in}{1.162667in}}%
\pgfpathlineto{\pgfqpoint{3.592231in}{1.150162in}}%
\pgfpathlineto{\pgfqpoint{3.605657in}{1.120792in}}%
\pgfpathlineto{\pgfqpoint{3.616541in}{1.088000in}}%
\pgfpathlineto{\pgfqpoint{3.621230in}{1.073494in}}%
\pgfpathlineto{\pgfqpoint{3.626410in}{1.050667in}}%
\pgfpathlineto{\pgfqpoint{3.627738in}{1.033901in}}%
\pgfpathlineto{\pgfqpoint{3.632862in}{1.013333in}}%
\pgfpathlineto{\pgfqpoint{3.631846in}{1.000395in}}%
\pgfpathlineto{\pgfqpoint{3.634127in}{0.976000in}}%
\pgfpathlineto{\pgfqpoint{3.629815in}{0.961169in}}%
\pgfpathlineto{\pgfqpoint{3.626953in}{0.938667in}}%
\pgfpathlineto{\pgfqpoint{3.604142in}{0.901333in}}%
\pgfpathlineto{\pgfqpoint{3.565576in}{0.882213in}}%
\pgfpathlineto{\pgfqpoint{3.550062in}{0.878450in}}%
\pgfpathlineto{\pgfqpoint{3.510817in}{0.877672in}}%
\pgfpathlineto{\pgfqpoint{3.485414in}{0.881389in}}%
\pgfpathlineto{\pgfqpoint{3.469529in}{0.886537in}}%
\pgfpathlineto{\pgfqpoint{3.445333in}{0.890437in}}%
\pgfpathlineto{\pgfqpoint{3.437369in}{0.893915in}}%
\pgfpathlineto{\pgfqpoint{3.405253in}{0.903114in}}%
\pgfpathlineto{\pgfqpoint{3.379973in}{0.915120in}}%
\pgfpathlineto{\pgfqpoint{3.365172in}{0.919739in}}%
\pgfpathlineto{\pgfqpoint{3.323586in}{0.938667in}}%
\pgfpathlineto{\pgfqpoint{3.285010in}{0.959153in}}%
\pgfpathlineto{\pgfqpoint{3.274396in}{0.966113in}}%
\pgfpathlineto{\pgfqpoint{3.244929in}{0.981573in}}%
\pgfpathlineto{\pgfqpoint{3.204848in}{1.005993in}}%
\pgfpathlineto{\pgfqpoint{3.137287in}{1.050667in}}%
\pgfpathlineto{\pgfqpoint{3.084090in}{1.088000in}}%
\pgfpathlineto{\pgfqpoint{3.034881in}{1.125333in}}%
\pgfpathlineto{\pgfqpoint{3.019045in}{1.138933in}}%
\pgfpathlineto{\pgfqpoint{3.004444in}{1.149140in}}%
\pgfpathlineto{\pgfqpoint{2.924283in}{1.215066in}}%
\pgfpathlineto{\pgfqpoint{2.912429in}{1.226292in}}%
\pgfpathlineto{\pgfqpoint{2.884202in}{1.249731in}}%
\pgfpathlineto{\pgfqpoint{2.871135in}{1.262496in}}%
\pgfpathlineto{\pgfqpoint{2.844121in}{1.285561in}}%
\pgfpathlineto{\pgfqpoint{2.775934in}{1.349333in}}%
\pgfpathlineto{\pgfqpoint{2.763960in}{1.360807in}}%
\pgfpathmoveto{\pgfqpoint{2.750114in}{1.349333in}}%
\pgfpathlineto{\pgfqpoint{2.763960in}{1.335872in}}%
\pgfpathlineto{\pgfqpoint{2.829192in}{1.274667in}}%
\pgfpathlineto{\pgfqpoint{2.857635in}{1.249921in}}%
\pgfpathlineto{\pgfqpoint{2.884202in}{1.225154in}}%
\pgfpathlineto{\pgfqpoint{2.898802in}{1.213599in}}%
\pgfpathlineto{\pgfqpoint{2.924283in}{1.190397in}}%
\pgfpathlineto{\pgfqpoint{2.940626in}{1.177890in}}%
\pgfpathlineto{\pgfqpoint{2.964364in}{1.156683in}}%
\pgfpathlineto{\pgfqpoint{3.044525in}{1.092722in}}%
\pgfpathlineto{\pgfqpoint{3.124687in}{1.033645in}}%
\pgfpathlineto{\pgfqpoint{3.194308in}{0.985818in}}%
\pgfpathlineto{\pgfqpoint{3.244929in}{0.954190in}}%
\pgfpathlineto{\pgfqpoint{3.255732in}{0.948729in}}%
\pgfpathlineto{\pgfqpoint{3.285010in}{0.930451in}}%
\pgfpathlineto{\pgfqpoint{3.305675in}{0.920582in}}%
\pgfpathlineto{\pgfqpoint{3.325091in}{0.908470in}}%
\pgfpathlineto{\pgfqpoint{3.392912in}{0.875495in}}%
\pgfpathlineto{\pgfqpoint{3.421692in}{0.864000in}}%
\pgfpathlineto{\pgfqpoint{3.461024in}{0.849385in}}%
\pgfpathlineto{\pgfqpoint{3.485414in}{0.842569in}}%
\pgfpathlineto{\pgfqpoint{3.500450in}{0.840672in}}%
\pgfpathlineto{\pgfqpoint{3.525495in}{0.833605in}}%
\pgfpathlineto{\pgfqpoint{3.565576in}{0.829807in}}%
\pgfpathlineto{\pgfqpoint{3.599328in}{0.832561in}}%
\pgfpathlineto{\pgfqpoint{3.605657in}{0.834419in}}%
\pgfpathlineto{\pgfqpoint{3.625615in}{0.845257in}}%
\pgfpathlineto{\pgfqpoint{3.645737in}{0.854116in}}%
\pgfpathlineto{\pgfqpoint{3.651596in}{0.858543in}}%
\pgfpathlineto{\pgfqpoint{3.667681in}{0.880894in}}%
\pgfpathlineto{\pgfqpoint{3.677705in}{0.908890in}}%
\pgfpathlineto{\pgfqpoint{3.681473in}{0.942714in}}%
\pgfpathlineto{\pgfqpoint{3.678782in}{0.982554in}}%
\pgfpathlineto{\pgfqpoint{3.672621in}{1.013333in}}%
\pgfpathlineto{\pgfqpoint{3.666117in}{1.032316in}}%
\pgfpathlineto{\pgfqpoint{3.662695in}{1.050667in}}%
\pgfpathlineto{\pgfqpoint{3.650477in}{1.088000in}}%
\pgfpathlineto{\pgfqpoint{3.628859in}{1.141054in}}%
\pgfpathlineto{\pgfqpoint{3.600371in}{1.200000in}}%
\pgfpathlineto{\pgfqpoint{3.587819in}{1.220718in}}%
\pgfpathlineto{\pgfqpoint{3.580177in}{1.237333in}}%
\pgfpathlineto{\pgfqpoint{3.559119in}{1.274667in}}%
\pgfpathlineto{\pgfqpoint{3.546125in}{1.293883in}}%
\pgfpathlineto{\pgfqpoint{3.536467in}{1.312000in}}%
\pgfpathlineto{\pgfqpoint{3.493359in}{1.379267in}}%
\pgfpathlineto{\pgfqpoint{3.462170in}{1.424000in}}%
\pgfpathlineto{\pgfqpoint{3.408026in}{1.498667in}}%
\pgfpathlineto{\pgfqpoint{3.405253in}{1.502308in}}%
\pgfpathlineto{\pgfqpoint{3.389351in}{1.521189in}}%
\pgfpathlineto{\pgfqpoint{3.379043in}{1.536000in}}%
\pgfpathlineto{\pgfqpoint{3.349447in}{1.573333in}}%
\pgfpathlineto{\pgfqpoint{3.338404in}{1.585734in}}%
\pgfpathlineto{\pgfqpoint{3.319198in}{1.610667in}}%
\pgfpathlineto{\pgfqpoint{3.303454in}{1.627846in}}%
\pgfpathlineto{\pgfqpoint{3.285010in}{1.651435in}}%
\pgfpathlineto{\pgfqpoint{3.267986in}{1.669477in}}%
\pgfpathlineto{\pgfqpoint{3.244929in}{1.697297in}}%
\pgfpathlineto{\pgfqpoint{3.231987in}{1.710612in}}%
\pgfpathlineto{\pgfqpoint{3.222169in}{1.722667in}}%
\pgfpathlineto{\pgfqpoint{3.164768in}{1.785113in}}%
\pgfpathlineto{\pgfqpoint{3.117304in}{1.834667in}}%
\pgfpathlineto{\pgfqpoint{3.100962in}{1.849902in}}%
\pgfpathlineto{\pgfqpoint{3.080462in}{1.872000in}}%
\pgfpathlineto{\pgfqpoint{3.003525in}{1.946667in}}%
\pgfpathlineto{\pgfqpoint{2.921685in}{2.021333in}}%
\pgfpathlineto{\pgfqpoint{2.901536in}{2.037479in}}%
\pgfpathlineto{\pgfqpoint{2.878681in}{2.058667in}}%
\pgfpathlineto{\pgfqpoint{2.859828in}{2.073297in}}%
\pgfpathlineto{\pgfqpoint{2.834135in}{2.096000in}}%
\pgfpathlineto{\pgfqpoint{2.817474in}{2.108512in}}%
\pgfpathlineto{\pgfqpoint{2.796263in}{2.126089in}}%
\pgfpathlineto{\pgfqpoint{2.787915in}{2.133333in}}%
\pgfpathlineto{\pgfqpoint{2.774454in}{2.143109in}}%
\pgfpathlineto{\pgfqpoint{2.763960in}{2.152218in}}%
\pgfpathlineto{\pgfqpoint{2.752556in}{2.160045in}}%
\pgfpathlineto{\pgfqpoint{2.723879in}{2.182846in}}%
\pgfpathlineto{\pgfqpoint{2.708075in}{2.193279in}}%
\pgfpathlineto{\pgfqpoint{2.683798in}{2.212449in}}%
\pgfpathlineto{\pgfqpoint{2.662794in}{2.225769in}}%
\pgfpathlineto{\pgfqpoint{2.636667in}{2.245333in}}%
\pgfpathlineto{\pgfqpoint{2.563556in}{2.292870in}}%
\pgfpathlineto{\pgfqpoint{2.544962in}{2.302681in}}%
\pgfpathlineto{\pgfqpoint{2.518347in}{2.320000in}}%
\pgfpathlineto{\pgfqpoint{2.443313in}{2.360240in}}%
\pgfpathlineto{\pgfqpoint{2.436653in}{2.363537in}}%
\pgfpathlineto{\pgfqpoint{2.362066in}{2.395678in}}%
\pgfpathlineto{\pgfqpoint{2.323071in}{2.408180in}}%
\pgfpathlineto{\pgfqpoint{2.301546in}{2.411951in}}%
\pgfpathlineto{\pgfqpoint{2.282990in}{2.418054in}}%
\pgfpathlineto{\pgfqpoint{2.252545in}{2.423025in}}%
\pgfpathlineto{\pgfqpoint{2.242909in}{2.423403in}}%
\pgfpathlineto{\pgfqpoint{2.212150in}{2.423317in}}%
\pgfpathlineto{\pgfqpoint{2.202828in}{2.421616in}}%
\pgfpathlineto{\pgfqpoint{2.182840in}{2.413382in}}%
\pgfpathlineto{\pgfqpoint{2.162747in}{2.407592in}}%
\pgfpathlineto{\pgfqpoint{2.154152in}{2.402673in}}%
\pgfpathlineto{\pgfqpoint{2.146471in}{2.394667in}}%
\pgfpathlineto{\pgfqpoint{2.122657in}{2.357333in}}%
\pgfpathlineto{\pgfqpoint{2.116487in}{2.320000in}}%
\pgfpathlineto{\pgfqpoint{2.117284in}{2.314986in}}%
\pgfpathlineto{\pgfqpoint{2.117349in}{2.282667in}}%
\pgfpathlineto{\pgfqpoint{2.122936in}{2.245083in}}%
\pgfpathlineto{\pgfqpoint{2.132944in}{2.208000in}}%
\pgfpathlineto{\pgfqpoint{2.141091in}{2.187828in}}%
\pgfpathlineto{\pgfqpoint{2.145376in}{2.170667in}}%
\pgfpathlineto{\pgfqpoint{2.150757in}{2.159498in}}%
\pgfpathlineto{\pgfqpoint{2.162747in}{2.126043in}}%
\pgfpathlineto{\pgfqpoint{2.194519in}{2.058667in}}%
\pgfpathlineto{\pgfqpoint{2.197557in}{2.053756in}}%
\pgfpathlineto{\pgfqpoint{2.214354in}{2.021333in}}%
\pgfpathlineto{\pgfqpoint{2.235346in}{1.984000in}}%
\pgfpathlineto{\pgfqpoint{2.257881in}{1.946667in}}%
\pgfpathlineto{\pgfqpoint{2.267846in}{1.932561in}}%
\pgfpathlineto{\pgfqpoint{2.282990in}{1.906317in}}%
\pgfpathlineto{\pgfqpoint{2.297906in}{1.885894in}}%
\pgfpathlineto{\pgfqpoint{2.306041in}{1.872000in}}%
\pgfpathlineto{\pgfqpoint{2.347376in}{1.812027in}}%
\pgfpathlineto{\pgfqpoint{2.385954in}{1.760000in}}%
\pgfpathlineto{\pgfqpoint{2.393533in}{1.750966in}}%
\pgfpathlineto{\pgfqpoint{2.414408in}{1.722667in}}%
\pgfpathlineto{\pgfqpoint{2.443949in}{1.684741in}}%
\pgfpathlineto{\pgfqpoint{2.523475in}{1.589644in}}%
\pgfpathlineto{\pgfqpoint{2.570555in}{1.536000in}}%
\pgfpathlineto{\pgfqpoint{2.586411in}{1.519955in}}%
\pgfpathlineto{\pgfqpoint{2.608240in}{1.494379in}}%
\pgfpathlineto{\pgfqpoint{2.683798in}{1.415252in}}%
\pgfpathlineto{\pgfqpoint{2.699181in}{1.400995in}}%
\pgfpathlineto{\pgfqpoint{2.723879in}{1.374992in}}%
\pgfpathlineto{\pgfqpoint{2.723879in}{1.374992in}}%
\pgfusepath{fill}%
\end{pgfscope}%
\begin{pgfscope}%
\pgfpathrectangle{\pgfqpoint{0.800000in}{0.528000in}}{\pgfqpoint{3.968000in}{3.696000in}}%
\pgfusepath{clip}%
\pgfsetbuttcap%
\pgfsetroundjoin%
\definecolor{currentfill}{rgb}{0.269944,0.014625,0.341379}%
\pgfsetfillcolor{currentfill}%
\pgfsetlinewidth{0.000000pt}%
\definecolor{currentstroke}{rgb}{0.000000,0.000000,0.000000}%
\pgfsetstrokecolor{currentstroke}%
\pgfsetdash{}{0pt}%
\pgfpathmoveto{\pgfqpoint{2.723879in}{1.374992in}}%
\pgfpathlineto{\pgfqpoint{2.643717in}{1.456672in}}%
\pgfpathlineto{\pgfqpoint{2.639308in}{1.461333in}}%
\pgfpathlineto{\pgfqpoint{2.603636in}{1.499337in}}%
\pgfpathlineto{\pgfqpoint{2.586411in}{1.519955in}}%
\pgfpathlineto{\pgfqpoint{2.563556in}{1.543809in}}%
\pgfpathlineto{\pgfqpoint{2.549897in}{1.560611in}}%
\pgfpathlineto{\pgfqpoint{2.523475in}{1.589644in}}%
\pgfpathlineto{\pgfqpoint{2.474174in}{1.648000in}}%
\pgfpathlineto{\pgfqpoint{2.443313in}{1.685535in}}%
\pgfpathlineto{\pgfqpoint{2.385954in}{1.760000in}}%
\pgfpathlineto{\pgfqpoint{2.347376in}{1.812027in}}%
\pgfpathlineto{\pgfqpoint{2.306041in}{1.872000in}}%
\pgfpathlineto{\pgfqpoint{2.297906in}{1.885894in}}%
\pgfpathlineto{\pgfqpoint{2.281008in}{1.909333in}}%
\pgfpathlineto{\pgfqpoint{2.267846in}{1.932561in}}%
\pgfpathlineto{\pgfqpoint{2.257881in}{1.946667in}}%
\pgfpathlineto{\pgfqpoint{2.226189in}{1.999574in}}%
\pgfpathlineto{\pgfqpoint{2.186802in}{2.073594in}}%
\pgfpathlineto{\pgfqpoint{2.159515in}{2.133333in}}%
\pgfpathlineto{\pgfqpoint{2.150757in}{2.159498in}}%
\pgfpathlineto{\pgfqpoint{2.145376in}{2.170667in}}%
\pgfpathlineto{\pgfqpoint{2.141091in}{2.187828in}}%
\pgfpathlineto{\pgfqpoint{2.132944in}{2.208000in}}%
\pgfpathlineto{\pgfqpoint{2.122667in}{2.246815in}}%
\pgfpathlineto{\pgfqpoint{2.117349in}{2.282667in}}%
\pgfpathlineto{\pgfqpoint{2.116537in}{2.325709in}}%
\pgfpathlineto{\pgfqpoint{2.122667in}{2.357358in}}%
\pgfpathlineto{\pgfqpoint{2.146471in}{2.394667in}}%
\pgfpathlineto{\pgfqpoint{2.154152in}{2.402673in}}%
\pgfpathlineto{\pgfqpoint{2.162747in}{2.407592in}}%
\pgfpathlineto{\pgfqpoint{2.182840in}{2.413382in}}%
\pgfpathlineto{\pgfqpoint{2.202828in}{2.421616in}}%
\pgfpathlineto{\pgfqpoint{2.212150in}{2.423317in}}%
\pgfpathlineto{\pgfqpoint{2.252545in}{2.423025in}}%
\pgfpathlineto{\pgfqpoint{2.282990in}{2.418054in}}%
\pgfpathlineto{\pgfqpoint{2.301546in}{2.411951in}}%
\pgfpathlineto{\pgfqpoint{2.323071in}{2.408180in}}%
\pgfpathlineto{\pgfqpoint{2.364574in}{2.394667in}}%
\pgfpathlineto{\pgfqpoint{2.436653in}{2.363537in}}%
\pgfpathlineto{\pgfqpoint{2.483394in}{2.339239in}}%
\pgfpathlineto{\pgfqpoint{2.523475in}{2.317158in}}%
\pgfpathlineto{\pgfqpoint{2.544962in}{2.302681in}}%
\pgfpathlineto{\pgfqpoint{2.563556in}{2.292870in}}%
\pgfpathlineto{\pgfqpoint{2.643717in}{2.240620in}}%
\pgfpathlineto{\pgfqpoint{2.662794in}{2.225769in}}%
\pgfpathlineto{\pgfqpoint{2.689853in}{2.208000in}}%
\pgfpathlineto{\pgfqpoint{2.708075in}{2.193279in}}%
\pgfpathlineto{\pgfqpoint{2.723879in}{2.182846in}}%
\pgfpathlineto{\pgfqpoint{2.787915in}{2.133333in}}%
\pgfpathlineto{\pgfqpoint{2.817474in}{2.108512in}}%
\pgfpathlineto{\pgfqpoint{2.844121in}{2.087805in}}%
\pgfpathlineto{\pgfqpoint{2.859828in}{2.073297in}}%
\pgfpathlineto{\pgfqpoint{2.884202in}{2.053983in}}%
\pgfpathlineto{\pgfqpoint{2.901536in}{2.037479in}}%
\pgfpathlineto{\pgfqpoint{2.924283in}{2.019057in}}%
\pgfpathlineto{\pgfqpoint{3.004444in}{1.945808in}}%
\pgfpathlineto{\pgfqpoint{3.084606in}{1.867887in}}%
\pgfpathlineto{\pgfqpoint{3.100962in}{1.849902in}}%
\pgfpathlineto{\pgfqpoint{3.124687in}{1.827118in}}%
\pgfpathlineto{\pgfqpoint{3.188095in}{1.760000in}}%
\pgfpathlineto{\pgfqpoint{3.195442in}{1.751238in}}%
\pgfpathlineto{\pgfqpoint{3.204848in}{1.741848in}}%
\pgfpathlineto{\pgfqpoint{3.255435in}{1.685333in}}%
\pgfpathlineto{\pgfqpoint{3.267986in}{1.669477in}}%
\pgfpathlineto{\pgfqpoint{3.287945in}{1.648000in}}%
\pgfpathlineto{\pgfqpoint{3.303454in}{1.627846in}}%
\pgfpathlineto{\pgfqpoint{3.325091in}{1.603522in}}%
\pgfpathlineto{\pgfqpoint{3.338404in}{1.585734in}}%
\pgfpathlineto{\pgfqpoint{3.349447in}{1.573333in}}%
\pgfpathlineto{\pgfqpoint{3.379043in}{1.536000in}}%
\pgfpathlineto{\pgfqpoint{3.389351in}{1.521189in}}%
\pgfpathlineto{\pgfqpoint{3.414626in}{1.489935in}}%
\pgfpathlineto{\pgfqpoint{3.462170in}{1.424000in}}%
\pgfpathlineto{\pgfqpoint{3.493359in}{1.379267in}}%
\pgfpathlineto{\pgfqpoint{3.536467in}{1.312000in}}%
\pgfpathlineto{\pgfqpoint{3.546125in}{1.293883in}}%
\pgfpathlineto{\pgfqpoint{3.565576in}{1.263415in}}%
\pgfpathlineto{\pgfqpoint{3.580177in}{1.237333in}}%
\pgfpathlineto{\pgfqpoint{3.587819in}{1.220718in}}%
\pgfpathlineto{\pgfqpoint{3.605657in}{1.189354in}}%
\pgfpathlineto{\pgfqpoint{3.618561in}{1.162667in}}%
\pgfpathlineto{\pgfqpoint{3.635391in}{1.125333in}}%
\pgfpathlineto{\pgfqpoint{3.637639in}{1.117790in}}%
\pgfpathlineto{\pgfqpoint{3.653167in}{1.081079in}}%
\pgfpathlineto{\pgfqpoint{3.662695in}{1.050667in}}%
\pgfpathlineto{\pgfqpoint{3.666117in}{1.032316in}}%
\pgfpathlineto{\pgfqpoint{3.672621in}{1.013333in}}%
\pgfpathlineto{\pgfqpoint{3.679283in}{0.976000in}}%
\pgfpathlineto{\pgfqpoint{3.681067in}{0.938667in}}%
\pgfpathlineto{\pgfqpoint{3.677705in}{0.908890in}}%
\pgfpathlineto{\pgfqpoint{3.667681in}{0.880894in}}%
\pgfpathlineto{\pgfqpoint{3.651596in}{0.858543in}}%
\pgfpathlineto{\pgfqpoint{3.645737in}{0.854116in}}%
\pgfpathlineto{\pgfqpoint{3.625615in}{0.845257in}}%
\pgfpathlineto{\pgfqpoint{3.605657in}{0.834419in}}%
\pgfpathlineto{\pgfqpoint{3.599328in}{0.832561in}}%
\pgfpathlineto{\pgfqpoint{3.562331in}{0.829689in}}%
\pgfpathlineto{\pgfqpoint{3.525495in}{0.833605in}}%
\pgfpathlineto{\pgfqpoint{3.500450in}{0.840672in}}%
\pgfpathlineto{\pgfqpoint{3.485414in}{0.842569in}}%
\pgfpathlineto{\pgfqpoint{3.461024in}{0.849385in}}%
\pgfpathlineto{\pgfqpoint{3.421692in}{0.864000in}}%
\pgfpathlineto{\pgfqpoint{3.356763in}{0.893501in}}%
\pgfpathlineto{\pgfqpoint{3.325091in}{0.908470in}}%
\pgfpathlineto{\pgfqpoint{3.305675in}{0.920582in}}%
\pgfpathlineto{\pgfqpoint{3.285010in}{0.930451in}}%
\pgfpathlineto{\pgfqpoint{3.204848in}{0.978994in}}%
\pgfpathlineto{\pgfqpoint{3.153740in}{1.013333in}}%
\pgfpathlineto{\pgfqpoint{3.084606in}{1.062655in}}%
\pgfpathlineto{\pgfqpoint{3.070229in}{1.074609in}}%
\pgfpathlineto{\pgfqpoint{3.044525in}{1.092722in}}%
\pgfpathlineto{\pgfqpoint{3.002793in}{1.125333in}}%
\pgfpathlineto{\pgfqpoint{2.957216in}{1.162667in}}%
\pgfpathlineto{\pgfqpoint{2.940626in}{1.177890in}}%
\pgfpathlineto{\pgfqpoint{2.913162in}{1.200000in}}%
\pgfpathlineto{\pgfqpoint{2.898802in}{1.213599in}}%
\pgfpathlineto{\pgfqpoint{2.870521in}{1.237333in}}%
\pgfpathlineto{\pgfqpoint{2.857635in}{1.249921in}}%
\pgfpathlineto{\pgfqpoint{2.837013in}{1.268045in}}%
\pgfpathlineto{\pgfqpoint{2.829192in}{1.274667in}}%
\pgfpathlineto{\pgfqpoint{2.763960in}{1.335872in}}%
\pgfpathlineto{\pgfqpoint{2.750114in}{1.349333in}}%
\pgfpathmoveto{\pgfqpoint{2.090428in}{2.238029in}}%
\pgfpathlineto{\pgfqpoint{2.100750in}{2.208000in}}%
\pgfpathlineto{\pgfqpoint{2.107327in}{2.193712in}}%
\pgfpathlineto{\pgfqpoint{2.114681in}{2.170667in}}%
\pgfpathlineto{\pgfqpoint{2.130755in}{2.133333in}}%
\pgfpathlineto{\pgfqpoint{2.141526in}{2.113567in}}%
\pgfpathlineto{\pgfqpoint{2.148643in}{2.096000in}}%
\pgfpathlineto{\pgfqpoint{2.167592in}{2.058667in}}%
\pgfpathlineto{\pgfqpoint{2.180474in}{2.037845in}}%
\pgfpathlineto{\pgfqpoint{2.188417in}{2.021333in}}%
\pgfpathlineto{\pgfqpoint{2.193913in}{2.013029in}}%
\pgfpathlineto{\pgfqpoint{2.210031in}{1.984000in}}%
\pgfpathlineto{\pgfqpoint{2.233039in}{1.946667in}}%
\pgfpathlineto{\pgfqpoint{2.242909in}{1.931171in}}%
\pgfpathlineto{\pgfqpoint{2.267689in}{1.895081in}}%
\pgfpathlineto{\pgfqpoint{2.282990in}{1.870207in}}%
\pgfpathlineto{\pgfqpoint{2.334987in}{1.797333in}}%
\pgfpathlineto{\pgfqpoint{2.347143in}{1.782423in}}%
\pgfpathlineto{\pgfqpoint{2.363152in}{1.759063in}}%
\pgfpathlineto{\pgfqpoint{2.380291in}{1.738632in}}%
\pgfpathlineto{\pgfqpoint{2.403232in}{1.707686in}}%
\pgfpathlineto{\pgfqpoint{2.478553in}{1.615176in}}%
\pgfpathlineto{\pgfqpoint{2.523475in}{1.563329in}}%
\pgfpathlineto{\pgfqpoint{2.537330in}{1.548905in}}%
\pgfpathlineto{\pgfqpoint{2.547822in}{1.536000in}}%
\pgfpathlineto{\pgfqpoint{2.603636in}{1.474791in}}%
\pgfpathlineto{\pgfqpoint{2.651775in}{1.424000in}}%
\pgfpathlineto{\pgfqpoint{2.667649in}{1.408958in}}%
\pgfpathlineto{\pgfqpoint{2.688168in}{1.386667in}}%
\pgfpathlineto{\pgfqpoint{2.725532in}{1.349333in}}%
\pgfpathlineto{\pgfqpoint{2.745112in}{1.331778in}}%
\pgfpathlineto{\pgfqpoint{2.763960in}{1.311976in}}%
\pgfpathlineto{\pgfqpoint{2.784654in}{1.293942in}}%
\pgfpathlineto{\pgfqpoint{2.804040in}{1.274162in}}%
\pgfpathlineto{\pgfqpoint{2.824753in}{1.256626in}}%
\pgfpathlineto{\pgfqpoint{2.844148in}{1.237333in}}%
\pgfpathlineto{\pgfqpoint{2.886133in}{1.200000in}}%
\pgfpathlineto{\pgfqpoint{2.906682in}{1.183606in}}%
\pgfpathlineto{\pgfqpoint{2.929497in}{1.162667in}}%
\pgfpathlineto{\pgfqpoint{2.974348in}{1.125333in}}%
\pgfpathlineto{\pgfqpoint{2.991020in}{1.112829in}}%
\pgfpathlineto{\pgfqpoint{3.004444in}{1.100924in}}%
\pgfpathlineto{\pgfqpoint{3.069007in}{1.050667in}}%
\pgfpathlineto{\pgfqpoint{3.084606in}{1.038814in}}%
\pgfpathlineto{\pgfqpoint{3.124687in}{1.009226in}}%
\pgfpathlineto{\pgfqpoint{3.145841in}{0.995704in}}%
\pgfpathlineto{\pgfqpoint{3.172238in}{0.976000in}}%
\pgfpathlineto{\pgfqpoint{3.244929in}{0.928412in}}%
\pgfpathlineto{\pgfqpoint{3.289396in}{0.901333in}}%
\pgfpathlineto{\pgfqpoint{3.312406in}{0.889518in}}%
\pgfpathlineto{\pgfqpoint{3.325091in}{0.881293in}}%
\pgfpathlineto{\pgfqpoint{3.337609in}{0.875660in}}%
\pgfpathlineto{\pgfqpoint{3.365172in}{0.859796in}}%
\pgfpathlineto{\pgfqpoint{3.389633in}{0.849451in}}%
\pgfpathlineto{\pgfqpoint{3.405253in}{0.840682in}}%
\pgfpathlineto{\pgfqpoint{3.445333in}{0.823093in}}%
\pgfpathlineto{\pgfqpoint{3.472413in}{0.814556in}}%
\pgfpathlineto{\pgfqpoint{3.485414in}{0.808626in}}%
\pgfpathlineto{\pgfqpoint{3.525495in}{0.796561in}}%
\pgfpathlineto{\pgfqpoint{3.532558in}{0.795912in}}%
\pgfpathlineto{\pgfqpoint{3.565576in}{0.788137in}}%
\pgfpathlineto{\pgfqpoint{3.605657in}{0.785301in}}%
\pgfpathlineto{\pgfqpoint{3.645737in}{0.789633in}}%
\pgfpathlineto{\pgfqpoint{3.673618in}{0.800698in}}%
\pgfpathlineto{\pgfqpoint{3.685818in}{0.809748in}}%
\pgfpathlineto{\pgfqpoint{3.701908in}{0.826667in}}%
\pgfpathlineto{\pgfqpoint{3.711266in}{0.840297in}}%
\pgfpathlineto{\pgfqpoint{3.719736in}{0.864000in}}%
\pgfpathlineto{\pgfqpoint{3.725184in}{0.901333in}}%
\pgfpathlineto{\pgfqpoint{3.723144in}{0.941233in}}%
\pgfpathlineto{\pgfqpoint{3.716736in}{0.976000in}}%
\pgfpathlineto{\pgfqpoint{3.709601in}{0.998153in}}%
\pgfpathlineto{\pgfqpoint{3.706979in}{1.013333in}}%
\pgfpathlineto{\pgfqpoint{3.694960in}{1.050667in}}%
\pgfpathlineto{\pgfqpoint{3.692127in}{1.056543in}}%
\pgfpathlineto{\pgfqpoint{3.677703in}{1.095558in}}%
\pgfpathlineto{\pgfqpoint{3.645737in}{1.164432in}}%
\pgfpathlineto{\pgfqpoint{3.632298in}{1.187482in}}%
\pgfpathlineto{\pgfqpoint{3.626758in}{1.200000in}}%
\pgfpathlineto{\pgfqpoint{3.605657in}{1.238806in}}%
\pgfpathlineto{\pgfqpoint{3.555417in}{1.321462in}}%
\pgfpathlineto{\pgfqpoint{3.511820in}{1.386667in}}%
\pgfpathlineto{\pgfqpoint{3.500792in}{1.400990in}}%
\pgfpathlineto{\pgfqpoint{3.485414in}{1.424979in}}%
\pgfpathlineto{\pgfqpoint{3.430755in}{1.498667in}}%
\pgfpathlineto{\pgfqpoint{3.394158in}{1.546334in}}%
\pgfpathlineto{\pgfqpoint{3.325091in}{1.630699in}}%
\pgfpathlineto{\pgfqpoint{3.310406in}{1.648000in}}%
\pgfpathlineto{\pgfqpoint{3.278371in}{1.685333in}}%
\pgfpathlineto{\pgfqpoint{3.262592in}{1.701785in}}%
\pgfpathlineto{\pgfqpoint{3.244929in}{1.723346in}}%
\pgfpathlineto{\pgfqpoint{3.164768in}{1.809689in}}%
\pgfpathlineto{\pgfqpoint{3.104011in}{1.872000in}}%
\pgfpathlineto{\pgfqpoint{3.084606in}{1.891466in}}%
\pgfpathlineto{\pgfqpoint{3.027794in}{1.946667in}}%
\pgfpathlineto{\pgfqpoint{3.015776in}{1.957221in}}%
\pgfpathlineto{\pgfqpoint{2.988105in}{1.984000in}}%
\pgfpathlineto{\pgfqpoint{2.975721in}{1.994578in}}%
\pgfpathlineto{\pgfqpoint{2.964364in}{2.005912in}}%
\pgfpathlineto{\pgfqpoint{2.884202in}{2.076945in}}%
\pgfpathlineto{\pgfqpoint{2.873042in}{2.085605in}}%
\pgfpathlineto{\pgfqpoint{2.844121in}{2.110937in}}%
\pgfpathlineto{\pgfqpoint{2.830809in}{2.120934in}}%
\pgfpathlineto{\pgfqpoint{2.804040in}{2.143933in}}%
\pgfpathlineto{\pgfqpoint{2.787914in}{2.155645in}}%
\pgfpathlineto{\pgfqpoint{2.763960in}{2.175952in}}%
\pgfpathlineto{\pgfqpoint{2.722416in}{2.208000in}}%
\pgfpathlineto{\pgfqpoint{2.643717in}{2.264829in}}%
\pgfpathlineto{\pgfqpoint{2.563556in}{2.318667in}}%
\pgfpathlineto{\pgfqpoint{2.537676in}{2.333227in}}%
\pgfpathlineto{\pgfqpoint{2.523475in}{2.343065in}}%
\pgfpathlineto{\pgfqpoint{2.483394in}{2.366449in}}%
\pgfpathlineto{\pgfqpoint{2.463317in}{2.375966in}}%
\pgfpathlineto{\pgfqpoint{2.443313in}{2.388223in}}%
\pgfpathlineto{\pgfqpoint{2.374647in}{2.421293in}}%
\pgfpathlineto{\pgfqpoint{2.347824in}{2.432000in}}%
\pgfpathlineto{\pgfqpoint{2.282990in}{2.454133in}}%
\pgfpathlineto{\pgfqpoint{2.268831in}{2.456145in}}%
\pgfpathlineto{\pgfqpoint{2.242909in}{2.463657in}}%
\pgfpathlineto{\pgfqpoint{2.236799in}{2.463642in}}%
\pgfpathlineto{\pgfqpoint{2.202828in}{2.468596in}}%
\pgfpathlineto{\pgfqpoint{2.162747in}{2.466340in}}%
\pgfpathlineto{\pgfqpoint{2.156452in}{2.463469in}}%
\pgfpathlineto{\pgfqpoint{2.122667in}{2.451830in}}%
\pgfpathlineto{\pgfqpoint{2.109917in}{2.443876in}}%
\pgfpathlineto{\pgfqpoint{2.099117in}{2.432000in}}%
\pgfpathlineto{\pgfqpoint{2.079274in}{2.397751in}}%
\pgfpathlineto{\pgfqpoint{2.073026in}{2.366238in}}%
\pgfpathlineto{\pgfqpoint{2.072856in}{2.357333in}}%
\pgfpathlineto{\pgfqpoint{2.073992in}{2.349329in}}%
\pgfpathlineto{\pgfqpoint{2.073775in}{2.320000in}}%
\pgfpathlineto{\pgfqpoint{2.080047in}{2.280302in}}%
\pgfpathlineto{\pgfqpoint{2.088519in}{2.245333in}}%
\pgfpathlineto{\pgfqpoint{2.088519in}{2.245333in}}%
\pgfusepath{fill}%
\end{pgfscope}%
\begin{pgfscope}%
\pgfpathrectangle{\pgfqpoint{0.800000in}{0.528000in}}{\pgfqpoint{3.968000in}{3.696000in}}%
\pgfusepath{clip}%
\pgfsetbuttcap%
\pgfsetroundjoin%
\definecolor{currentfill}{rgb}{0.269944,0.014625,0.341379}%
\pgfsetfillcolor{currentfill}%
\pgfsetlinewidth{0.000000pt}%
\definecolor{currentstroke}{rgb}{0.000000,0.000000,0.000000}%
\pgfsetstrokecolor{currentstroke}%
\pgfsetdash{}{0pt}%
\pgfpathmoveto{\pgfqpoint{2.088519in}{2.245333in}}%
\pgfpathlineto{\pgfqpoint{2.078229in}{2.286724in}}%
\pgfpathlineto{\pgfqpoint{2.073775in}{2.320000in}}%
\pgfpathlineto{\pgfqpoint{2.073992in}{2.349329in}}%
\pgfpathlineto{\pgfqpoint{2.072856in}{2.357333in}}%
\pgfpathlineto{\pgfqpoint{2.073026in}{2.366238in}}%
\pgfpathlineto{\pgfqpoint{2.079274in}{2.397751in}}%
\pgfpathlineto{\pgfqpoint{2.082586in}{2.404963in}}%
\pgfpathlineto{\pgfqpoint{2.099117in}{2.432000in}}%
\pgfpathlineto{\pgfqpoint{2.109917in}{2.443876in}}%
\pgfpathlineto{\pgfqpoint{2.122667in}{2.451830in}}%
\pgfpathlineto{\pgfqpoint{2.165326in}{2.466932in}}%
\pgfpathlineto{\pgfqpoint{2.202828in}{2.468596in}}%
\pgfpathlineto{\pgfqpoint{2.242909in}{2.463657in}}%
\pgfpathlineto{\pgfqpoint{2.268831in}{2.456145in}}%
\pgfpathlineto{\pgfqpoint{2.282990in}{2.454133in}}%
\pgfpathlineto{\pgfqpoint{2.323071in}{2.441507in}}%
\pgfpathlineto{\pgfqpoint{2.329785in}{2.438254in}}%
\pgfpathlineto{\pgfqpoint{2.363152in}{2.426068in}}%
\pgfpathlineto{\pgfqpoint{2.374647in}{2.421293in}}%
\pgfpathlineto{\pgfqpoint{2.443313in}{2.388223in}}%
\pgfpathlineto{\pgfqpoint{2.463317in}{2.375966in}}%
\pgfpathlineto{\pgfqpoint{2.489104in}{2.362652in}}%
\pgfpathlineto{\pgfqpoint{2.499079in}{2.357333in}}%
\pgfpathlineto{\pgfqpoint{2.523475in}{2.343065in}}%
\pgfpathlineto{\pgfqpoint{2.537676in}{2.333227in}}%
\pgfpathlineto{\pgfqpoint{2.568068in}{2.315797in}}%
\pgfpathlineto{\pgfqpoint{2.643717in}{2.264829in}}%
\pgfpathlineto{\pgfqpoint{2.723879in}{2.206918in}}%
\pgfpathlineto{\pgfqpoint{2.770606in}{2.170667in}}%
\pgfpathlineto{\pgfqpoint{2.787914in}{2.155645in}}%
\pgfpathlineto{\pgfqpoint{2.816965in}{2.133333in}}%
\pgfpathlineto{\pgfqpoint{2.830809in}{2.120934in}}%
\pgfpathlineto{\pgfqpoint{2.844121in}{2.110937in}}%
\pgfpathlineto{\pgfqpoint{2.905189in}{2.058667in}}%
\pgfpathlineto{\pgfqpoint{2.935114in}{2.031422in}}%
\pgfpathlineto{\pgfqpoint{2.964364in}{2.005912in}}%
\pgfpathlineto{\pgfqpoint{2.975721in}{1.994578in}}%
\pgfpathlineto{\pgfqpoint{3.004444in}{1.968834in}}%
\pgfpathlineto{\pgfqpoint{3.015776in}{1.957221in}}%
\pgfpathlineto{\pgfqpoint{3.044525in}{1.930692in}}%
\pgfpathlineto{\pgfqpoint{3.055294in}{1.919364in}}%
\pgfpathlineto{\pgfqpoint{3.084606in}{1.891466in}}%
\pgfpathlineto{\pgfqpoint{3.113493in}{1.861574in}}%
\pgfpathlineto{\pgfqpoint{3.132766in}{1.842192in}}%
\pgfpathlineto{\pgfqpoint{3.140671in}{1.834667in}}%
\pgfpathlineto{\pgfqpoint{3.211384in}{1.760000in}}%
\pgfpathlineto{\pgfqpoint{3.245539in}{1.722667in}}%
\pgfpathlineto{\pgfqpoint{3.262592in}{1.701785in}}%
\pgfpathlineto{\pgfqpoint{3.285010in}{1.677723in}}%
\pgfpathlineto{\pgfqpoint{3.341750in}{1.610667in}}%
\pgfpathlineto{\pgfqpoint{3.351371in}{1.597812in}}%
\pgfpathlineto{\pgfqpoint{3.372443in}{1.573333in}}%
\pgfpathlineto{\pgfqpoint{3.405253in}{1.532150in}}%
\pgfpathlineto{\pgfqpoint{3.458697in}{1.461333in}}%
\pgfpathlineto{\pgfqpoint{3.486119in}{1.424000in}}%
\pgfpathlineto{\pgfqpoint{3.500792in}{1.400990in}}%
\pgfpathlineto{\pgfqpoint{3.511820in}{1.386667in}}%
\pgfpathlineto{\pgfqpoint{3.561351in}{1.312000in}}%
\pgfpathlineto{\pgfqpoint{3.565576in}{1.305196in}}%
\pgfpathlineto{\pgfqpoint{3.607546in}{1.235574in}}%
\pgfpathlineto{\pgfqpoint{3.646658in}{1.162667in}}%
\pgfpathlineto{\pgfqpoint{3.657781in}{1.136552in}}%
\pgfpathlineto{\pgfqpoint{3.664080in}{1.125333in}}%
\pgfpathlineto{\pgfqpoint{3.680699in}{1.088000in}}%
\pgfpathlineto{\pgfqpoint{3.699786in}{1.037656in}}%
\pgfpathlineto{\pgfqpoint{3.706979in}{1.013333in}}%
\pgfpathlineto{\pgfqpoint{3.709601in}{0.998153in}}%
\pgfpathlineto{\pgfqpoint{3.716736in}{0.976000in}}%
\pgfpathlineto{\pgfqpoint{3.723059in}{0.936022in}}%
\pgfpathlineto{\pgfqpoint{3.724920in}{0.900422in}}%
\pgfpathlineto{\pgfqpoint{3.719736in}{0.864000in}}%
\pgfpathlineto{\pgfqpoint{3.711266in}{0.840297in}}%
\pgfpathlineto{\pgfqpoint{3.701908in}{0.826667in}}%
\pgfpathlineto{\pgfqpoint{3.685818in}{0.809748in}}%
\pgfpathlineto{\pgfqpoint{3.673618in}{0.800698in}}%
\pgfpathlineto{\pgfqpoint{3.643751in}{0.789333in}}%
\pgfpathlineto{\pgfqpoint{3.602096in}{0.786017in}}%
\pgfpathlineto{\pgfqpoint{3.559838in}{0.789333in}}%
\pgfpathlineto{\pgfqpoint{3.514986in}{0.799122in}}%
\pgfpathlineto{\pgfqpoint{3.485414in}{0.808626in}}%
\pgfpathlineto{\pgfqpoint{3.472413in}{0.814556in}}%
\pgfpathlineto{\pgfqpoint{3.437144in}{0.826667in}}%
\pgfpathlineto{\pgfqpoint{3.405253in}{0.840682in}}%
\pgfpathlineto{\pgfqpoint{3.389633in}{0.849451in}}%
\pgfpathlineto{\pgfqpoint{3.357291in}{0.864000in}}%
\pgfpathlineto{\pgfqpoint{3.337609in}{0.875660in}}%
\pgfpathlineto{\pgfqpoint{3.325091in}{0.881293in}}%
\pgfpathlineto{\pgfqpoint{3.312406in}{0.889518in}}%
\pgfpathlineto{\pgfqpoint{3.277757in}{0.908089in}}%
\pgfpathlineto{\pgfqpoint{3.204848in}{0.954217in}}%
\pgfpathlineto{\pgfqpoint{3.164768in}{0.981021in}}%
\pgfpathlineto{\pgfqpoint{3.145841in}{0.995704in}}%
\pgfpathlineto{\pgfqpoint{3.119096in}{1.013333in}}%
\pgfpathlineto{\pgfqpoint{3.069007in}{1.050667in}}%
\pgfpathlineto{\pgfqpoint{3.056072in}{1.061422in}}%
\pgfpathlineto{\pgfqpoint{3.044525in}{1.069375in}}%
\pgfpathlineto{\pgfqpoint{3.004444in}{1.100924in}}%
\pgfpathlineto{\pgfqpoint{2.991020in}{1.112829in}}%
\pgfpathlineto{\pgfqpoint{2.964364in}{1.133479in}}%
\pgfpathlineto{\pgfqpoint{2.924283in}{1.167057in}}%
\pgfpathlineto{\pgfqpoint{2.906682in}{1.183606in}}%
\pgfpathlineto{\pgfqpoint{2.884202in}{1.201677in}}%
\pgfpathlineto{\pgfqpoint{2.844121in}{1.237358in}}%
\pgfpathlineto{\pgfqpoint{2.824753in}{1.256626in}}%
\pgfpathlineto{\pgfqpoint{2.803503in}{1.274667in}}%
\pgfpathlineto{\pgfqpoint{2.784654in}{1.293942in}}%
\pgfpathlineto{\pgfqpoint{2.763935in}{1.312000in}}%
\pgfpathlineto{\pgfqpoint{2.745112in}{1.331778in}}%
\pgfpathlineto{\pgfqpoint{2.723879in}{1.350950in}}%
\pgfpathlineto{\pgfqpoint{2.683798in}{1.391064in}}%
\pgfpathlineto{\pgfqpoint{2.667649in}{1.408958in}}%
\pgfpathlineto{\pgfqpoint{2.643717in}{1.432338in}}%
\pgfpathlineto{\pgfqpoint{2.629700in}{1.448277in}}%
\pgfpathlineto{\pgfqpoint{2.603636in}{1.474791in}}%
\pgfpathlineto{\pgfqpoint{2.592257in}{1.488067in}}%
\pgfpathlineto{\pgfqpoint{2.573734in}{1.508148in}}%
\pgfpathlineto{\pgfqpoint{2.563556in}{1.518448in}}%
\pgfpathlineto{\pgfqpoint{2.482365in}{1.610667in}}%
\pgfpathlineto{\pgfqpoint{2.443313in}{1.657785in}}%
\pgfpathlineto{\pgfqpoint{2.431144in}{1.673998in}}%
\pgfpathlineto{\pgfqpoint{2.421117in}{1.685333in}}%
\pgfpathlineto{\pgfqpoint{2.391487in}{1.722667in}}%
\pgfpathlineto{\pgfqpoint{2.380291in}{1.738632in}}%
\pgfpathlineto{\pgfqpoint{2.362437in}{1.760000in}}%
\pgfpathlineto{\pgfqpoint{2.347143in}{1.782423in}}%
\pgfpathlineto{\pgfqpoint{2.323071in}{1.813701in}}%
\pgfpathlineto{\pgfqpoint{2.308125in}{1.834667in}}%
\pgfpathlineto{\pgfqpoint{2.281748in}{1.872000in}}%
\pgfpathlineto{\pgfqpoint{2.267689in}{1.895081in}}%
\pgfpathlineto{\pgfqpoint{2.257125in}{1.909333in}}%
\pgfpathlineto{\pgfqpoint{2.252213in}{1.918000in}}%
\pgfpathlineto{\pgfqpoint{2.233039in}{1.946667in}}%
\pgfpathlineto{\pgfqpoint{2.202828in}{1.996251in}}%
\pgfpathlineto{\pgfqpoint{2.180474in}{2.037845in}}%
\pgfpathlineto{\pgfqpoint{2.162747in}{2.068038in}}%
\pgfpathlineto{\pgfqpoint{2.148643in}{2.096000in}}%
\pgfpathlineto{\pgfqpoint{2.141526in}{2.113567in}}%
\pgfpathlineto{\pgfqpoint{2.128833in}{2.139077in}}%
\pgfpathlineto{\pgfqpoint{2.114681in}{2.170667in}}%
\pgfpathlineto{\pgfqpoint{2.107327in}{2.193712in}}%
\pgfpathlineto{\pgfqpoint{2.100750in}{2.208000in}}%
\pgfpathlineto{\pgfqpoint{2.090428in}{2.238029in}}%
\pgfpathmoveto{\pgfqpoint{2.064661in}{2.224696in}}%
\pgfpathlineto{\pgfqpoint{2.089335in}{2.164380in}}%
\pgfpathlineto{\pgfqpoint{2.122667in}{2.095499in}}%
\pgfpathlineto{\pgfqpoint{2.163679in}{2.021333in}}%
\pgfpathlineto{\pgfqpoint{2.178610in}{1.998775in}}%
\pgfpathlineto{\pgfqpoint{2.186460in}{1.984000in}}%
\pgfpathlineto{\pgfqpoint{2.192912in}{1.974764in}}%
\pgfpathlineto{\pgfqpoint{2.209669in}{1.946667in}}%
\pgfpathlineto{\pgfqpoint{2.223006in}{1.928128in}}%
\pgfpathlineto{\pgfqpoint{2.242909in}{1.896290in}}%
\pgfpathlineto{\pgfqpoint{2.259602in}{1.872000in}}%
\pgfpathlineto{\pgfqpoint{2.285519in}{1.834667in}}%
\pgfpathlineto{\pgfqpoint{2.301468in}{1.814545in}}%
\pgfpathlineto{\pgfqpoint{2.323071in}{1.783632in}}%
\pgfpathlineto{\pgfqpoint{2.340978in}{1.760000in}}%
\pgfpathlineto{\pgfqpoint{2.369596in}{1.722667in}}%
\pgfpathlineto{\pgfqpoint{2.384569in}{1.705282in}}%
\pgfpathlineto{\pgfqpoint{2.403232in}{1.680295in}}%
\pgfpathlineto{\pgfqpoint{2.418895in}{1.662589in}}%
\pgfpathlineto{\pgfqpoint{2.443313in}{1.631759in}}%
\pgfpathlineto{\pgfqpoint{2.523475in}{1.538541in}}%
\pgfpathlineto{\pgfqpoint{2.537193in}{1.523222in}}%
\pgfpathlineto{\pgfqpoint{2.563556in}{1.494226in}}%
\pgfpathlineto{\pgfqpoint{2.580588in}{1.477198in}}%
\pgfpathlineto{\pgfqpoint{2.603636in}{1.451241in}}%
\pgfpathlineto{\pgfqpoint{2.617936in}{1.437320in}}%
\pgfpathlineto{\pgfqpoint{2.643717in}{1.409350in}}%
\pgfpathlineto{\pgfqpoint{2.655789in}{1.397911in}}%
\pgfpathlineto{\pgfqpoint{2.683798in}{1.368534in}}%
\pgfpathlineto{\pgfqpoint{2.763960in}{1.290058in}}%
\pgfpathlineto{\pgfqpoint{2.792552in}{1.263966in}}%
\pgfpathlineto{\pgfqpoint{2.820407in}{1.237333in}}%
\pgfpathlineto{\pgfqpoint{2.884202in}{1.179977in}}%
\pgfpathlineto{\pgfqpoint{2.904129in}{1.162667in}}%
\pgfpathlineto{\pgfqpoint{2.964364in}{1.111488in}}%
\pgfpathlineto{\pgfqpoint{2.978333in}{1.101011in}}%
\pgfpathlineto{\pgfqpoint{3.004444in}{1.078667in}}%
\pgfpathlineto{\pgfqpoint{3.044525in}{1.046774in}}%
\pgfpathlineto{\pgfqpoint{3.064982in}{1.032388in}}%
\pgfpathlineto{\pgfqpoint{3.088225in}{1.013333in}}%
\pgfpathlineto{\pgfqpoint{3.109298in}{0.999000in}}%
\pgfpathlineto{\pgfqpoint{3.139466in}{0.976000in}}%
\pgfpathlineto{\pgfqpoint{3.164768in}{0.958043in}}%
\pgfpathlineto{\pgfqpoint{3.204848in}{0.930476in}}%
\pgfpathlineto{\pgfqpoint{3.223994in}{0.919166in}}%
\pgfpathlineto{\pgfqpoint{3.249336in}{0.901333in}}%
\pgfpathlineto{\pgfqpoint{3.271513in}{0.888761in}}%
\pgfpathlineto{\pgfqpoint{3.285010in}{0.879383in}}%
\pgfpathlineto{\pgfqpoint{3.295656in}{0.873916in}}%
\pgfpathlineto{\pgfqpoint{3.325091in}{0.855664in}}%
\pgfpathlineto{\pgfqpoint{3.345516in}{0.845692in}}%
\pgfpathlineto{\pgfqpoint{3.370222in}{0.831371in}}%
\pgfpathlineto{\pgfqpoint{3.378602in}{0.826667in}}%
\pgfpathlineto{\pgfqpoint{3.405253in}{0.813182in}}%
\pgfpathlineto{\pgfqpoint{3.449337in}{0.793063in}}%
\pgfpathlineto{\pgfqpoint{3.457514in}{0.789333in}}%
\pgfpathlineto{\pgfqpoint{3.485414in}{0.777981in}}%
\pgfpathlineto{\pgfqpoint{3.506023in}{0.770137in}}%
\pgfpathlineto{\pgfqpoint{3.565075in}{0.752467in}}%
\pgfpathlineto{\pgfqpoint{3.567638in}{0.752000in}}%
\pgfpathlineto{\pgfqpoint{3.605657in}{0.745382in}}%
\pgfpathlineto{\pgfqpoint{3.645737in}{0.743296in}}%
\pgfpathlineto{\pgfqpoint{3.688896in}{0.749133in}}%
\pgfpathlineto{\pgfqpoint{3.716065in}{0.761160in}}%
\pgfpathlineto{\pgfqpoint{3.737253in}{0.778758in}}%
\pgfpathlineto{\pgfqpoint{3.752640in}{0.801759in}}%
\pgfpathlineto{\pgfqpoint{3.761435in}{0.826667in}}%
\pgfpathlineto{\pgfqpoint{3.766597in}{0.864575in}}%
\pgfpathlineto{\pgfqpoint{3.765155in}{0.902101in}}%
\pgfpathlineto{\pgfqpoint{3.758957in}{0.938667in}}%
\pgfpathlineto{\pgfqpoint{3.751685in}{0.962685in}}%
\pgfpathlineto{\pgfqpoint{3.749557in}{0.976000in}}%
\pgfpathlineto{\pgfqpoint{3.743244in}{0.992156in}}%
\pgfpathlineto{\pgfqpoint{3.737888in}{1.013333in}}%
\pgfpathlineto{\pgfqpoint{3.724365in}{1.050667in}}%
\pgfpathlineto{\pgfqpoint{3.712420in}{1.075445in}}%
\pgfpathlineto{\pgfqpoint{3.708090in}{1.088000in}}%
\pgfpathlineto{\pgfqpoint{3.685818in}{1.135581in}}%
\pgfpathlineto{\pgfqpoint{3.671871in}{1.162667in}}%
\pgfpathlineto{\pgfqpoint{3.645737in}{1.210827in}}%
\pgfpathlineto{\pgfqpoint{3.605657in}{1.278676in}}%
\pgfpathlineto{\pgfqpoint{3.550867in}{1.363034in}}%
\pgfpathlineto{\pgfqpoint{3.507935in}{1.424000in}}%
\pgfpathlineto{\pgfqpoint{3.498339in}{1.436039in}}%
\pgfpathlineto{\pgfqpoint{3.471061in}{1.474703in}}%
\pgfpathlineto{\pgfqpoint{3.423673in}{1.536000in}}%
\pgfpathlineto{\pgfqpoint{3.415463in}{1.545510in}}%
\pgfpathlineto{\pgfqpoint{3.394012in}{1.573333in}}%
\pgfpathlineto{\pgfqpoint{3.332196in}{1.648000in}}%
\pgfpathlineto{\pgfqpoint{3.325091in}{1.656329in}}%
\pgfpathlineto{\pgfqpoint{3.244929in}{1.747117in}}%
\pgfpathlineto{\pgfqpoint{3.198644in}{1.797333in}}%
\pgfpathlineto{\pgfqpoint{3.182112in}{1.813488in}}%
\pgfpathlineto{\pgfqpoint{3.163334in}{1.834667in}}%
\pgfpathlineto{\pgfqpoint{3.144222in}{1.852862in}}%
\pgfpathlineto{\pgfqpoint{3.124687in}{1.874290in}}%
\pgfpathlineto{\pgfqpoint{3.105833in}{1.891771in}}%
\pgfpathlineto{\pgfqpoint{3.084606in}{1.914253in}}%
\pgfpathlineto{\pgfqpoint{3.066932in}{1.930204in}}%
\pgfpathlineto{\pgfqpoint{3.044525in}{1.953205in}}%
\pgfpathlineto{\pgfqpoint{3.027509in}{1.968150in}}%
\pgfpathlineto{\pgfqpoint{3.004444in}{1.991164in}}%
\pgfpathlineto{\pgfqpoint{2.987549in}{2.005596in}}%
\pgfpathlineto{\pgfqpoint{2.964364in}{2.028145in}}%
\pgfpathlineto{\pgfqpoint{2.947039in}{2.042530in}}%
\pgfpathlineto{\pgfqpoint{2.924283in}{2.064163in}}%
\pgfpathlineto{\pgfqpoint{2.844121in}{2.133370in}}%
\pgfpathlineto{\pgfqpoint{2.822075in}{2.150131in}}%
\pgfpathlineto{\pgfqpoint{2.798510in}{2.170667in}}%
\pgfpathlineto{\pgfqpoint{2.779226in}{2.184887in}}%
\pgfpathlineto{\pgfqpoint{2.751231in}{2.208000in}}%
\pgfpathlineto{\pgfqpoint{2.735755in}{2.219062in}}%
\pgfpathlineto{\pgfqpoint{2.723879in}{2.229074in}}%
\pgfpathlineto{\pgfqpoint{2.643717in}{2.288124in}}%
\pgfpathlineto{\pgfqpoint{2.623303in}{2.300985in}}%
\pgfpathlineto{\pgfqpoint{2.597489in}{2.320000in}}%
\pgfpathlineto{\pgfqpoint{2.523475in}{2.367693in}}%
\pgfpathlineto{\pgfqpoint{2.505059in}{2.377513in}}%
\pgfpathlineto{\pgfqpoint{2.478465in}{2.394667in}}%
\pgfpathlineto{\pgfqpoint{2.443313in}{2.414147in}}%
\pgfpathlineto{\pgfqpoint{2.430368in}{2.419942in}}%
\pgfpathlineto{\pgfqpoint{2.403232in}{2.435403in}}%
\pgfpathlineto{\pgfqpoint{2.378158in}{2.445978in}}%
\pgfpathlineto{\pgfqpoint{2.363152in}{2.454283in}}%
\pgfpathlineto{\pgfqpoint{2.351377in}{2.458366in}}%
\pgfpathlineto{\pgfqpoint{2.323071in}{2.471797in}}%
\pgfpathlineto{\pgfqpoint{2.295186in}{2.480694in}}%
\pgfpathlineto{\pgfqpoint{2.282990in}{2.486218in}}%
\pgfpathlineto{\pgfqpoint{2.242909in}{2.498367in}}%
\pgfpathlineto{\pgfqpoint{2.234936in}{2.499241in}}%
\pgfpathlineto{\pgfqpoint{2.202068in}{2.507375in}}%
\pgfpathlineto{\pgfqpoint{2.162747in}{2.510841in}}%
\pgfpathlineto{\pgfqpoint{2.117283in}{2.506667in}}%
\pgfpathlineto{\pgfqpoint{2.091546in}{2.498320in}}%
\pgfpathlineto{\pgfqpoint{2.082586in}{2.492822in}}%
\pgfpathlineto{\pgfqpoint{2.056148in}{2.469333in}}%
\pgfpathlineto{\pgfqpoint{2.050128in}{2.462233in}}%
\pgfpathlineto{\pgfqpoint{2.042505in}{2.446150in}}%
\pgfpathlineto{\pgfqpoint{2.037136in}{2.432000in}}%
\pgfpathlineto{\pgfqpoint{2.036965in}{2.426840in}}%
\pgfpathlineto{\pgfqpoint{2.031389in}{2.394667in}}%
\pgfpathlineto{\pgfqpoint{2.030902in}{2.368141in}}%
\pgfpathlineto{\pgfqpoint{2.037226in}{2.320000in}}%
\pgfpathlineto{\pgfqpoint{2.038506in}{2.316275in}}%
\pgfpathlineto{\pgfqpoint{2.045900in}{2.282667in}}%
\pgfpathlineto{\pgfqpoint{2.055160in}{2.257121in}}%
\pgfpathlineto{\pgfqpoint{2.057763in}{2.245333in}}%
\pgfpathlineto{\pgfqpoint{2.057763in}{2.245333in}}%
\pgfusepath{fill}%
\end{pgfscope}%
\begin{pgfscope}%
\pgfpathrectangle{\pgfqpoint{0.800000in}{0.528000in}}{\pgfqpoint{3.968000in}{3.696000in}}%
\pgfusepath{clip}%
\pgfsetbuttcap%
\pgfsetroundjoin%
\definecolor{currentfill}{rgb}{0.271305,0.019942,0.347269}%
\pgfsetfillcolor{currentfill}%
\pgfsetlinewidth{0.000000pt}%
\definecolor{currentstroke}{rgb}{0.000000,0.000000,0.000000}%
\pgfsetstrokecolor{currentstroke}%
\pgfsetdash{}{0pt}%
\pgfpathmoveto{\pgfqpoint{2.057763in}{2.245333in}}%
\pgfpathlineto{\pgfqpoint{2.055160in}{2.257121in}}%
\pgfpathlineto{\pgfqpoint{2.045900in}{2.282667in}}%
\pgfpathlineto{\pgfqpoint{2.035853in}{2.326196in}}%
\pgfpathlineto{\pgfqpoint{2.030902in}{2.368141in}}%
\pgfpathlineto{\pgfqpoint{2.031389in}{2.394667in}}%
\pgfpathlineto{\pgfqpoint{2.037136in}{2.432000in}}%
\pgfpathlineto{\pgfqpoint{2.042505in}{2.446150in}}%
\pgfpathlineto{\pgfqpoint{2.050128in}{2.462233in}}%
\pgfpathlineto{\pgfqpoint{2.056148in}{2.469333in}}%
\pgfpathlineto{\pgfqpoint{2.082586in}{2.492822in}}%
\pgfpathlineto{\pgfqpoint{2.091546in}{2.498320in}}%
\pgfpathlineto{\pgfqpoint{2.124132in}{2.508031in}}%
\pgfpathlineto{\pgfqpoint{2.162747in}{2.510841in}}%
\pgfpathlineto{\pgfqpoint{2.205404in}{2.506667in}}%
\pgfpathlineto{\pgfqpoint{2.282990in}{2.486218in}}%
\pgfpathlineto{\pgfqpoint{2.295186in}{2.480694in}}%
\pgfpathlineto{\pgfqpoint{2.328742in}{2.469333in}}%
\pgfpathlineto{\pgfqpoint{2.351377in}{2.458366in}}%
\pgfpathlineto{\pgfqpoint{2.363152in}{2.454283in}}%
\pgfpathlineto{\pgfqpoint{2.378158in}{2.445978in}}%
\pgfpathlineto{\pgfqpoint{2.409683in}{2.432000in}}%
\pgfpathlineto{\pgfqpoint{2.430368in}{2.419942in}}%
\pgfpathlineto{\pgfqpoint{2.443313in}{2.414147in}}%
\pgfpathlineto{\pgfqpoint{2.483394in}{2.391918in}}%
\pgfpathlineto{\pgfqpoint{2.505059in}{2.377513in}}%
\pgfpathlineto{\pgfqpoint{2.523475in}{2.367693in}}%
\pgfpathlineto{\pgfqpoint{2.603636in}{2.315938in}}%
\pgfpathlineto{\pgfqpoint{2.623303in}{2.300985in}}%
\pgfpathlineto{\pgfqpoint{2.651278in}{2.282667in}}%
\pgfpathlineto{\pgfqpoint{2.723879in}{2.229074in}}%
\pgfpathlineto{\pgfqpoint{2.735755in}{2.219062in}}%
\pgfpathlineto{\pgfqpoint{2.763960in}{2.198139in}}%
\pgfpathlineto{\pgfqpoint{2.779226in}{2.184887in}}%
\pgfpathlineto{\pgfqpoint{2.804040in}{2.166244in}}%
\pgfpathlineto{\pgfqpoint{2.822075in}{2.150131in}}%
\pgfpathlineto{\pgfqpoint{2.844164in}{2.133333in}}%
\pgfpathlineto{\pgfqpoint{2.930426in}{2.058667in}}%
\pgfpathlineto{\pgfqpoint{2.947039in}{2.042530in}}%
\pgfpathlineto{\pgfqpoint{2.971778in}{2.021333in}}%
\pgfpathlineto{\pgfqpoint{2.987549in}{2.005596in}}%
\pgfpathlineto{\pgfqpoint{3.012042in}{1.984000in}}%
\pgfpathlineto{\pgfqpoint{3.027509in}{1.968150in}}%
\pgfpathlineto{\pgfqpoint{3.051283in}{1.946667in}}%
\pgfpathlineto{\pgfqpoint{3.066932in}{1.930204in}}%
\pgfpathlineto{\pgfqpoint{3.089563in}{1.909333in}}%
\pgfpathlineto{\pgfqpoint{3.105833in}{1.891771in}}%
\pgfpathlineto{\pgfqpoint{3.126938in}{1.872000in}}%
\pgfpathlineto{\pgfqpoint{3.144222in}{1.852862in}}%
\pgfpathlineto{\pgfqpoint{3.164768in}{1.833181in}}%
\pgfpathlineto{\pgfqpoint{3.182112in}{1.813488in}}%
\pgfpathlineto{\pgfqpoint{3.204848in}{1.790729in}}%
\pgfpathlineto{\pgfqpoint{3.219514in}{1.773660in}}%
\pgfpathlineto{\pgfqpoint{3.244929in}{1.747117in}}%
\pgfpathlineto{\pgfqpoint{3.325091in}{1.656329in}}%
\pgfpathlineto{\pgfqpoint{3.332196in}{1.648000in}}%
\pgfpathlineto{\pgfqpoint{3.365172in}{1.608952in}}%
\pgfpathlineto{\pgfqpoint{3.445333in}{1.508348in}}%
\pgfpathlineto{\pgfqpoint{3.452772in}{1.498667in}}%
\pgfpathlineto{\pgfqpoint{3.485414in}{1.455273in}}%
\pgfpathlineto{\pgfqpoint{3.498339in}{1.436039in}}%
\pgfpathlineto{\pgfqpoint{3.507935in}{1.424000in}}%
\pgfpathlineto{\pgfqpoint{3.514478in}{1.413739in}}%
\pgfpathlineto{\pgfqpoint{3.534436in}{1.386667in}}%
\pgfpathlineto{\pgfqpoint{3.565576in}{1.340897in}}%
\pgfpathlineto{\pgfqpoint{3.584277in}{1.312000in}}%
\pgfpathlineto{\pgfqpoint{3.608176in}{1.274667in}}%
\pgfpathlineto{\pgfqpoint{3.620825in}{1.251462in}}%
\pgfpathlineto{\pgfqpoint{3.630316in}{1.237333in}}%
\pgfpathlineto{\pgfqpoint{3.658711in}{1.187916in}}%
\pgfpathlineto{\pgfqpoint{3.695421in}{1.116389in}}%
\pgfpathlineto{\pgfqpoint{3.708090in}{1.088000in}}%
\pgfpathlineto{\pgfqpoint{3.712420in}{1.075445in}}%
\pgfpathlineto{\pgfqpoint{3.725899in}{1.046530in}}%
\pgfpathlineto{\pgfqpoint{3.737888in}{1.013333in}}%
\pgfpathlineto{\pgfqpoint{3.743244in}{0.992156in}}%
\pgfpathlineto{\pgfqpoint{3.749557in}{0.976000in}}%
\pgfpathlineto{\pgfqpoint{3.751685in}{0.962685in}}%
\pgfpathlineto{\pgfqpoint{3.758957in}{0.938667in}}%
\pgfpathlineto{\pgfqpoint{3.765194in}{0.901333in}}%
\pgfpathlineto{\pgfqpoint{3.765980in}{0.857216in}}%
\pgfpathlineto{\pgfqpoint{3.761435in}{0.826667in}}%
\pgfpathlineto{\pgfqpoint{3.752640in}{0.801759in}}%
\pgfpathlineto{\pgfqpoint{3.737253in}{0.778758in}}%
\pgfpathlineto{\pgfqpoint{3.716065in}{0.761160in}}%
\pgfpathlineto{\pgfqpoint{3.685818in}{0.748489in}}%
\pgfpathlineto{\pgfqpoint{3.682006in}{0.748449in}}%
\pgfpathlineto{\pgfqpoint{3.645737in}{0.743296in}}%
\pgfpathlineto{\pgfqpoint{3.637775in}{0.744583in}}%
\pgfpathlineto{\pgfqpoint{3.605657in}{0.745382in}}%
\pgfpathlineto{\pgfqpoint{3.565075in}{0.752467in}}%
\pgfpathlineto{\pgfqpoint{3.506023in}{0.770137in}}%
\pgfpathlineto{\pgfqpoint{3.457514in}{0.789333in}}%
\pgfpathlineto{\pgfqpoint{3.396400in}{0.818421in}}%
\pgfpathlineto{\pgfqpoint{3.365172in}{0.833507in}}%
\pgfpathlineto{\pgfqpoint{3.345516in}{0.845692in}}%
\pgfpathlineto{\pgfqpoint{3.325091in}{0.855664in}}%
\pgfpathlineto{\pgfqpoint{3.285010in}{0.879383in}}%
\pgfpathlineto{\pgfqpoint{3.271513in}{0.888761in}}%
\pgfpathlineto{\pgfqpoint{3.244929in}{0.904061in}}%
\pgfpathlineto{\pgfqpoint{3.223994in}{0.919166in}}%
\pgfpathlineto{\pgfqpoint{3.192893in}{0.938667in}}%
\pgfpathlineto{\pgfqpoint{3.124687in}{0.986548in}}%
\pgfpathlineto{\pgfqpoint{3.109298in}{0.999000in}}%
\pgfpathlineto{\pgfqpoint{3.084606in}{1.016007in}}%
\pgfpathlineto{\pgfqpoint{3.064982in}{1.032388in}}%
\pgfpathlineto{\pgfqpoint{3.039610in}{1.050667in}}%
\pgfpathlineto{\pgfqpoint{2.993003in}{1.088000in}}%
\pgfpathlineto{\pgfqpoint{2.978333in}{1.101011in}}%
\pgfpathlineto{\pgfqpoint{2.964364in}{1.111488in}}%
\pgfpathlineto{\pgfqpoint{2.904129in}{1.162667in}}%
\pgfpathlineto{\pgfqpoint{2.884202in}{1.179977in}}%
\pgfpathlineto{\pgfqpoint{2.820407in}{1.237333in}}%
\pgfpathlineto{\pgfqpoint{2.792552in}{1.263966in}}%
\pgfpathlineto{\pgfqpoint{2.763960in}{1.290058in}}%
\pgfpathlineto{\pgfqpoint{2.683798in}{1.368534in}}%
\pgfpathlineto{\pgfqpoint{2.655789in}{1.397911in}}%
\pgfpathlineto{\pgfqpoint{2.629649in}{1.424000in}}%
\pgfpathlineto{\pgfqpoint{2.617936in}{1.437320in}}%
\pgfpathlineto{\pgfqpoint{2.594186in}{1.461333in}}%
\pgfpathlineto{\pgfqpoint{2.580588in}{1.477198in}}%
\pgfpathlineto{\pgfqpoint{2.559499in}{1.498667in}}%
\pgfpathlineto{\pgfqpoint{2.493097in}{1.573333in}}%
\pgfpathlineto{\pgfqpoint{2.443313in}{1.631759in}}%
\pgfpathlineto{\pgfqpoint{2.418895in}{1.662589in}}%
\pgfpathlineto{\pgfqpoint{2.399159in}{1.685333in}}%
\pgfpathlineto{\pgfqpoint{2.384569in}{1.705282in}}%
\pgfpathlineto{\pgfqpoint{2.363152in}{1.730945in}}%
\pgfpathlineto{\pgfqpoint{2.312889in}{1.797333in}}%
\pgfpathlineto{\pgfqpoint{2.301468in}{1.814545in}}%
\pgfpathlineto{\pgfqpoint{2.282990in}{1.838242in}}%
\pgfpathlineto{\pgfqpoint{2.234124in}{1.909333in}}%
\pgfpathlineto{\pgfqpoint{2.223006in}{1.928128in}}%
\pgfpathlineto{\pgfqpoint{2.202828in}{1.957507in}}%
\pgfpathlineto{\pgfqpoint{2.178610in}{1.998775in}}%
\pgfpathlineto{\pgfqpoint{2.162747in}{2.022973in}}%
\pgfpathlineto{\pgfqpoint{2.150328in}{2.047098in}}%
\pgfpathlineto{\pgfqpoint{2.142912in}{2.058667in}}%
\pgfpathlineto{\pgfqpoint{2.122132in}{2.096498in}}%
\pgfpathlineto{\pgfqpoint{2.085729in}{2.173595in}}%
\pgfpathlineto{\pgfqpoint{2.082586in}{2.180423in}}%
\pgfpathlineto{\pgfqpoint{2.064661in}{2.224696in}}%
\pgfpathmoveto{\pgfqpoint{2.044037in}{2.208000in}}%
\pgfpathlineto{\pgfqpoint{2.044974in}{2.205700in}}%
\pgfpathlineto{\pgfqpoint{2.061171in}{2.170667in}}%
\pgfpathlineto{\pgfqpoint{2.068486in}{2.157534in}}%
\pgfpathlineto{\pgfqpoint{2.082586in}{2.126373in}}%
\pgfpathlineto{\pgfqpoint{2.098737in}{2.096000in}}%
\pgfpathlineto{\pgfqpoint{2.107508in}{2.081880in}}%
\pgfpathlineto{\pgfqpoint{2.122667in}{2.052518in}}%
\pgfpathlineto{\pgfqpoint{2.135184in}{2.032992in}}%
\pgfpathlineto{\pgfqpoint{2.141186in}{2.021333in}}%
\pgfpathlineto{\pgfqpoint{2.163583in}{1.984000in}}%
\pgfpathlineto{\pgfqpoint{2.179053in}{1.961854in}}%
\pgfpathlineto{\pgfqpoint{2.187779in}{1.946667in}}%
\pgfpathlineto{\pgfqpoint{2.237922in}{1.872000in}}%
\pgfpathlineto{\pgfqpoint{2.242909in}{1.864928in}}%
\pgfpathlineto{\pgfqpoint{2.311759in}{1.770536in}}%
\pgfpathlineto{\pgfqpoint{2.363152in}{1.704732in}}%
\pgfpathlineto{\pgfqpoint{2.378737in}{1.685333in}}%
\pgfpathlineto{\pgfqpoint{2.409032in}{1.648000in}}%
\pgfpathlineto{\pgfqpoint{2.424690in}{1.630654in}}%
\pgfpathlineto{\pgfqpoint{2.443313in}{1.607007in}}%
\pgfpathlineto{\pgfqpoint{2.460051in}{1.588924in}}%
\pgfpathlineto{\pgfqpoint{2.483394in}{1.560759in}}%
\pgfpathlineto{\pgfqpoint{2.563556in}{1.471690in}}%
\pgfpathlineto{\pgfqpoint{2.573200in}{1.461333in}}%
\pgfpathlineto{\pgfqpoint{2.644107in}{1.386667in}}%
\pgfpathlineto{\pgfqpoint{2.663845in}{1.368082in}}%
\pgfpathlineto{\pgfqpoint{2.683798in}{1.346601in}}%
\pgfpathlineto{\pgfqpoint{2.702508in}{1.329427in}}%
\pgfpathlineto{\pgfqpoint{2.723879in}{1.307141in}}%
\pgfpathlineto{\pgfqpoint{2.763960in}{1.268638in}}%
\pgfpathlineto{\pgfqpoint{2.781324in}{1.253508in}}%
\pgfpathlineto{\pgfqpoint{2.804040in}{1.231079in}}%
\pgfpathlineto{\pgfqpoint{2.884202in}{1.158733in}}%
\pgfpathlineto{\pgfqpoint{2.966990in}{1.088000in}}%
\pgfpathlineto{\pgfqpoint{3.012959in}{1.050667in}}%
\pgfpathlineto{\pgfqpoint{3.030580in}{1.037678in}}%
\pgfpathlineto{\pgfqpoint{3.060487in}{1.013333in}}%
\pgfpathlineto{\pgfqpoint{3.124687in}{0.964828in}}%
\pgfpathlineto{\pgfqpoint{3.141152in}{0.954003in}}%
\pgfpathlineto{\pgfqpoint{3.177576in}{0.926736in}}%
\pgfpathlineto{\pgfqpoint{3.244929in}{0.881533in}}%
\pgfpathlineto{\pgfqpoint{3.256631in}{0.874899in}}%
\pgfpathlineto{\pgfqpoint{3.285010in}{0.855919in}}%
\pgfpathlineto{\pgfqpoint{3.333799in}{0.826667in}}%
\pgfpathlineto{\pgfqpoint{3.365172in}{0.808975in}}%
\pgfpathlineto{\pgfqpoint{3.379255in}{0.802451in}}%
\pgfpathlineto{\pgfqpoint{3.405253in}{0.787268in}}%
\pgfpathlineto{\pgfqpoint{3.430994in}{0.775977in}}%
\pgfpathlineto{\pgfqpoint{3.445333in}{0.767888in}}%
\pgfpathlineto{\pgfqpoint{3.457545in}{0.763374in}}%
\pgfpathlineto{\pgfqpoint{3.485414in}{0.749670in}}%
\pgfpathlineto{\pgfqpoint{3.512868in}{0.740239in}}%
\pgfpathlineto{\pgfqpoint{3.525495in}{0.734290in}}%
\pgfpathlineto{\pgfqpoint{3.541985in}{0.730027in}}%
\pgfpathlineto{\pgfqpoint{3.565576in}{0.720805in}}%
\pgfpathlineto{\pgfqpoint{3.611596in}{0.709134in}}%
\pgfpathlineto{\pgfqpoint{3.658265in}{0.702998in}}%
\pgfpathlineto{\pgfqpoint{3.685818in}{0.703240in}}%
\pgfpathlineto{\pgfqpoint{3.730147in}{0.710710in}}%
\pgfpathlineto{\pgfqpoint{3.738335in}{0.714667in}}%
\pgfpathlineto{\pgfqpoint{3.765980in}{0.731610in}}%
\pgfpathlineto{\pgfqpoint{3.777043in}{0.741695in}}%
\pgfpathlineto{\pgfqpoint{3.783790in}{0.752000in}}%
\pgfpathlineto{\pgfqpoint{3.800598in}{0.789333in}}%
\pgfpathlineto{\pgfqpoint{3.806255in}{0.826848in}}%
\pgfpathlineto{\pgfqpoint{3.805175in}{0.864000in}}%
\pgfpathlineto{\pgfqpoint{3.798274in}{0.908586in}}%
\pgfpathlineto{\pgfqpoint{3.785533in}{0.957787in}}%
\pgfpathlineto{\pgfqpoint{3.765980in}{1.014496in}}%
\pgfpathlineto{\pgfqpoint{3.750701in}{1.050667in}}%
\pgfpathlineto{\pgfqpoint{3.742594in}{1.066217in}}%
\pgfpathlineto{\pgfqpoint{3.731181in}{1.092920in}}%
\pgfpathlineto{\pgfqpoint{3.715616in}{1.125333in}}%
\pgfpathlineto{\pgfqpoint{3.705046in}{1.143243in}}%
\pgfpathlineto{\pgfqpoint{3.692229in}{1.168638in}}%
\pgfpathlineto{\pgfqpoint{3.675185in}{1.200000in}}%
\pgfpathlineto{\pgfqpoint{3.664064in}{1.217070in}}%
\pgfpathlineto{\pgfqpoint{3.645737in}{1.249798in}}%
\pgfpathlineto{\pgfqpoint{3.620487in}{1.288480in}}%
\pgfpathlineto{\pgfqpoint{3.605657in}{1.313578in}}%
\pgfpathlineto{\pgfqpoint{3.590162in}{1.334901in}}%
\pgfpathlineto{\pgfqpoint{3.581465in}{1.349333in}}%
\pgfpathlineto{\pgfqpoint{3.574901in}{1.358019in}}%
\pgfpathlineto{\pgfqpoint{3.555822in}{1.386667in}}%
\pgfpathlineto{\pgfqpoint{3.543098in}{1.403063in}}%
\pgfpathlineto{\pgfqpoint{3.525495in}{1.429338in}}%
\pgfpathlineto{\pgfqpoint{3.510859in}{1.447700in}}%
\pgfpathlineto{\pgfqpoint{3.494643in}{1.469929in}}%
\pgfpathlineto{\pgfqpoint{3.473517in}{1.498667in}}%
\pgfpathlineto{\pgfqpoint{3.461132in}{1.513383in}}%
\pgfpathlineto{\pgfqpoint{3.444821in}{1.536000in}}%
\pgfpathlineto{\pgfqpoint{3.427186in}{1.556430in}}%
\pgfpathlineto{\pgfqpoint{3.405253in}{1.584993in}}%
\pgfpathlineto{\pgfqpoint{3.352717in}{1.648000in}}%
\pgfpathlineto{\pgfqpoint{3.339928in}{1.661820in}}%
\pgfpathlineto{\pgfqpoint{3.320789in}{1.685333in}}%
\pgfpathlineto{\pgfqpoint{3.304006in}{1.703027in}}%
\pgfpathlineto{\pgfqpoint{3.285010in}{1.725941in}}%
\pgfpathlineto{\pgfqpoint{3.204848in}{1.812963in}}%
\pgfpathlineto{\pgfqpoint{3.174650in}{1.843871in}}%
\pgfpathlineto{\pgfqpoint{3.148030in}{1.872000in}}%
\pgfpathlineto{\pgfqpoint{3.136448in}{1.882955in}}%
\pgfpathlineto{\pgfqpoint{3.111073in}{1.909333in}}%
\pgfpathlineto{\pgfqpoint{3.073227in}{1.946667in}}%
\pgfpathlineto{\pgfqpoint{3.058631in}{1.959805in}}%
\pgfpathlineto{\pgfqpoint{3.034438in}{1.984000in}}%
\pgfpathlineto{\pgfqpoint{3.018995in}{1.997553in}}%
\pgfpathlineto{\pgfqpoint{2.994646in}{2.021333in}}%
\pgfpathlineto{\pgfqpoint{2.911788in}{2.096000in}}%
\pgfpathlineto{\pgfqpoint{2.897044in}{2.107962in}}%
\pgfpathlineto{\pgfqpoint{2.868574in}{2.133333in}}%
\pgfpathlineto{\pgfqpoint{2.804040in}{2.187164in}}%
\pgfpathlineto{\pgfqpoint{2.791556in}{2.196371in}}%
\pgfpathlineto{\pgfqpoint{2.763960in}{2.219373in}}%
\pgfpathlineto{\pgfqpoint{2.748190in}{2.230645in}}%
\pgfpathlineto{\pgfqpoint{2.723879in}{2.250692in}}%
\pgfpathlineto{\pgfqpoint{2.704188in}{2.264325in}}%
\pgfpathlineto{\pgfqpoint{2.681511in}{2.282667in}}%
\pgfpathlineto{\pgfqpoint{2.629393in}{2.320000in}}%
\pgfpathlineto{\pgfqpoint{2.603636in}{2.337976in}}%
\pgfpathlineto{\pgfqpoint{2.523475in}{2.391062in}}%
\pgfpathlineto{\pgfqpoint{2.517565in}{2.394667in}}%
\pgfpathlineto{\pgfqpoint{2.443313in}{2.438803in}}%
\pgfpathlineto{\pgfqpoint{2.403232in}{2.460551in}}%
\pgfpathlineto{\pgfqpoint{2.396742in}{2.463288in}}%
\pgfpathlineto{\pgfqpoint{2.363152in}{2.480696in}}%
\pgfpathlineto{\pgfqpoint{2.323071in}{2.499205in}}%
\pgfpathlineto{\pgfqpoint{2.317104in}{2.501109in}}%
\pgfpathlineto{\pgfqpoint{2.282990in}{2.515543in}}%
\pgfpathlineto{\pgfqpoint{2.259727in}{2.522331in}}%
\pgfpathlineto{\pgfqpoint{2.242909in}{2.529425in}}%
\pgfpathlineto{\pgfqpoint{2.202828in}{2.540926in}}%
\pgfpathlineto{\pgfqpoint{2.162747in}{2.548440in}}%
\pgfpathlineto{\pgfqpoint{2.122667in}{2.551135in}}%
\pgfpathlineto{\pgfqpoint{2.079583in}{2.546797in}}%
\pgfpathlineto{\pgfqpoint{2.050831in}{2.536245in}}%
\pgfpathlineto{\pgfqpoint{2.042505in}{2.530725in}}%
\pgfpathlineto{\pgfqpoint{2.016728in}{2.506667in}}%
\pgfpathlineto{\pgfqpoint{2.010571in}{2.499078in}}%
\pgfpathlineto{\pgfqpoint{1.997862in}{2.469333in}}%
\pgfpathlineto{\pgfqpoint{1.992207in}{2.441517in}}%
\pgfpathlineto{\pgfqpoint{1.991111in}{2.405205in}}%
\pgfpathlineto{\pgfqpoint{1.996821in}{2.357333in}}%
\pgfpathlineto{\pgfqpoint{2.004926in}{2.320000in}}%
\pgfpathlineto{\pgfqpoint{2.014037in}{2.293484in}}%
\pgfpathlineto{\pgfqpoint{2.016283in}{2.282667in}}%
\pgfpathlineto{\pgfqpoint{2.022114in}{2.264326in}}%
\pgfpathlineto{\pgfqpoint{2.042505in}{2.211824in}}%
\pgfpathlineto{\pgfqpoint{2.042505in}{2.211824in}}%
\pgfusepath{fill}%
\end{pgfscope}%
\begin{pgfscope}%
\pgfpathrectangle{\pgfqpoint{0.800000in}{0.528000in}}{\pgfqpoint{3.968000in}{3.696000in}}%
\pgfusepath{clip}%
\pgfsetbuttcap%
\pgfsetroundjoin%
\definecolor{currentfill}{rgb}{0.271305,0.019942,0.347269}%
\pgfsetfillcolor{currentfill}%
\pgfsetlinewidth{0.000000pt}%
\definecolor{currentstroke}{rgb}{0.000000,0.000000,0.000000}%
\pgfsetstrokecolor{currentstroke}%
\pgfsetdash{}{0pt}%
\pgfpathmoveto{\pgfqpoint{2.042505in}{2.211824in}}%
\pgfpathlineto{\pgfqpoint{2.022114in}{2.264326in}}%
\pgfpathlineto{\pgfqpoint{2.016283in}{2.282667in}}%
\pgfpathlineto{\pgfqpoint{2.014037in}{2.293484in}}%
\pgfpathlineto{\pgfqpoint{2.004926in}{2.320000in}}%
\pgfpathlineto{\pgfqpoint{1.996821in}{2.357333in}}%
\pgfpathlineto{\pgfqpoint{1.991111in}{2.405205in}}%
\pgfpathlineto{\pgfqpoint{1.992207in}{2.441517in}}%
\pgfpathlineto{\pgfqpoint{1.997862in}{2.469333in}}%
\pgfpathlineto{\pgfqpoint{2.002424in}{2.481501in}}%
\pgfpathlineto{\pgfqpoint{2.010571in}{2.499078in}}%
\pgfpathlineto{\pgfqpoint{2.016728in}{2.506667in}}%
\pgfpathlineto{\pgfqpoint{2.042505in}{2.530725in}}%
\pgfpathlineto{\pgfqpoint{2.050831in}{2.536245in}}%
\pgfpathlineto{\pgfqpoint{2.082586in}{2.547288in}}%
\pgfpathlineto{\pgfqpoint{2.086041in}{2.547218in}}%
\pgfpathlineto{\pgfqpoint{2.122667in}{2.551135in}}%
\pgfpathlineto{\pgfqpoint{2.129104in}{2.549996in}}%
\pgfpathlineto{\pgfqpoint{2.162747in}{2.548440in}}%
\pgfpathlineto{\pgfqpoint{2.207145in}{2.539979in}}%
\pgfpathlineto{\pgfqpoint{2.242909in}{2.529425in}}%
\pgfpathlineto{\pgfqpoint{2.259727in}{2.522331in}}%
\pgfpathlineto{\pgfqpoint{2.282990in}{2.515543in}}%
\pgfpathlineto{\pgfqpoint{2.338082in}{2.492684in}}%
\pgfpathlineto{\pgfqpoint{2.385821in}{2.469333in}}%
\pgfpathlineto{\pgfqpoint{2.403232in}{2.460551in}}%
\pgfpathlineto{\pgfqpoint{2.455008in}{2.432000in}}%
\pgfpathlineto{\pgfqpoint{2.523475in}{2.391062in}}%
\pgfpathlineto{\pgfqpoint{2.575084in}{2.357333in}}%
\pgfpathlineto{\pgfqpoint{2.643717in}{2.309948in}}%
\pgfpathlineto{\pgfqpoint{2.683798in}{2.281007in}}%
\pgfpathlineto{\pgfqpoint{2.704188in}{2.264325in}}%
\pgfpathlineto{\pgfqpoint{2.730765in}{2.245333in}}%
\pgfpathlineto{\pgfqpoint{2.748190in}{2.230645in}}%
\pgfpathlineto{\pgfqpoint{2.778159in}{2.208000in}}%
\pgfpathlineto{\pgfqpoint{2.791556in}{2.196371in}}%
\pgfpathlineto{\pgfqpoint{2.804040in}{2.187164in}}%
\pgfpathlineto{\pgfqpoint{2.884202in}{2.120023in}}%
\pgfpathlineto{\pgfqpoint{2.897044in}{2.107962in}}%
\pgfpathlineto{\pgfqpoint{2.924283in}{2.085062in}}%
\pgfpathlineto{\pgfqpoint{3.004444in}{2.012284in}}%
\pgfpathlineto{\pgfqpoint{3.018995in}{1.997553in}}%
\pgfpathlineto{\pgfqpoint{3.044525in}{1.974438in}}%
\pgfpathlineto{\pgfqpoint{3.058631in}{1.959805in}}%
\pgfpathlineto{\pgfqpoint{3.084606in}{1.935598in}}%
\pgfpathlineto{\pgfqpoint{3.164768in}{1.854878in}}%
\pgfpathlineto{\pgfqpoint{3.193477in}{1.824075in}}%
\pgfpathlineto{\pgfqpoint{3.219477in}{1.797333in}}%
\pgfpathlineto{\pgfqpoint{3.230791in}{1.784165in}}%
\pgfpathlineto{\pgfqpoint{3.254058in}{1.760000in}}%
\pgfpathlineto{\pgfqpoint{3.287932in}{1.722667in}}%
\pgfpathlineto{\pgfqpoint{3.304006in}{1.703027in}}%
\pgfpathlineto{\pgfqpoint{3.325091in}{1.680385in}}%
\pgfpathlineto{\pgfqpoint{3.339928in}{1.661820in}}%
\pgfpathlineto{\pgfqpoint{3.365172in}{1.633312in}}%
\pgfpathlineto{\pgfqpoint{3.384017in}{1.610667in}}%
\pgfpathlineto{\pgfqpoint{3.414721in}{1.573333in}}%
\pgfpathlineto{\pgfqpoint{3.427186in}{1.556430in}}%
\pgfpathlineto{\pgfqpoint{3.445333in}{1.535346in}}%
\pgfpathlineto{\pgfqpoint{3.461132in}{1.513383in}}%
\pgfpathlineto{\pgfqpoint{3.485414in}{1.483078in}}%
\pgfpathlineto{\pgfqpoint{3.510859in}{1.447700in}}%
\pgfpathlineto{\pgfqpoint{3.529393in}{1.424000in}}%
\pgfpathlineto{\pgfqpoint{3.543098in}{1.403063in}}%
\pgfpathlineto{\pgfqpoint{3.565576in}{1.372638in}}%
\pgfpathlineto{\pgfqpoint{3.590162in}{1.334901in}}%
\pgfpathlineto{\pgfqpoint{3.606704in}{1.312000in}}%
\pgfpathlineto{\pgfqpoint{3.620487in}{1.288480in}}%
\pgfpathlineto{\pgfqpoint{3.630244in}{1.274667in}}%
\pgfpathlineto{\pgfqpoint{3.635462in}{1.265095in}}%
\pgfpathlineto{\pgfqpoint{3.653348in}{1.237333in}}%
\pgfpathlineto{\pgfqpoint{3.664064in}{1.217070in}}%
\pgfpathlineto{\pgfqpoint{3.675185in}{1.200000in}}%
\pgfpathlineto{\pgfqpoint{3.696004in}{1.162667in}}%
\pgfpathlineto{\pgfqpoint{3.705046in}{1.143243in}}%
\pgfpathlineto{\pgfqpoint{3.715616in}{1.125333in}}%
\pgfpathlineto{\pgfqpoint{3.733990in}{1.088000in}}%
\pgfpathlineto{\pgfqpoint{3.742594in}{1.066217in}}%
\pgfpathlineto{\pgfqpoint{3.750701in}{1.050667in}}%
\pgfpathlineto{\pgfqpoint{3.766459in}{1.013333in}}%
\pgfpathlineto{\pgfqpoint{3.779379in}{0.976000in}}%
\pgfpathlineto{\pgfqpoint{3.785533in}{0.957787in}}%
\pgfpathlineto{\pgfqpoint{3.799464in}{0.901333in}}%
\pgfpathlineto{\pgfqpoint{3.799673in}{0.895384in}}%
\pgfpathlineto{\pgfqpoint{3.805175in}{0.864000in}}%
\pgfpathlineto{\pgfqpoint{3.806061in}{0.824667in}}%
\pgfpathlineto{\pgfqpoint{3.800598in}{0.789333in}}%
\pgfpathlineto{\pgfqpoint{3.795253in}{0.779266in}}%
\pgfpathlineto{\pgfqpoint{3.783790in}{0.752000in}}%
\pgfpathlineto{\pgfqpoint{3.777043in}{0.741695in}}%
\pgfpathlineto{\pgfqpoint{3.765980in}{0.731610in}}%
\pgfpathlineto{\pgfqpoint{3.730147in}{0.710710in}}%
\pgfpathlineto{\pgfqpoint{3.720184in}{0.709344in}}%
\pgfpathlineto{\pgfqpoint{3.685818in}{0.703240in}}%
\pgfpathlineto{\pgfqpoint{3.658265in}{0.702998in}}%
\pgfpathlineto{\pgfqpoint{3.611596in}{0.709134in}}%
\pgfpathlineto{\pgfqpoint{3.589341in}{0.714667in}}%
\pgfpathlineto{\pgfqpoint{3.565576in}{0.720805in}}%
\pgfpathlineto{\pgfqpoint{3.541985in}{0.730027in}}%
\pgfpathlineto{\pgfqpoint{3.525495in}{0.734290in}}%
\pgfpathlineto{\pgfqpoint{3.512868in}{0.740239in}}%
\pgfpathlineto{\pgfqpoint{3.480258in}{0.752000in}}%
\pgfpathlineto{\pgfqpoint{3.457545in}{0.763374in}}%
\pgfpathlineto{\pgfqpoint{3.445333in}{0.767888in}}%
\pgfpathlineto{\pgfqpoint{3.430994in}{0.775977in}}%
\pgfpathlineto{\pgfqpoint{3.401419in}{0.789333in}}%
\pgfpathlineto{\pgfqpoint{3.379255in}{0.802451in}}%
\pgfpathlineto{\pgfqpoint{3.365172in}{0.808975in}}%
\pgfpathlineto{\pgfqpoint{3.310646in}{0.840122in}}%
\pgfpathlineto{\pgfqpoint{3.272317in}{0.864000in}}%
\pgfpathlineto{\pgfqpoint{3.256631in}{0.874899in}}%
\pgfpathlineto{\pgfqpoint{3.244929in}{0.881533in}}%
\pgfpathlineto{\pgfqpoint{3.164768in}{0.935742in}}%
\pgfpathlineto{\pgfqpoint{3.141152in}{0.954003in}}%
\pgfpathlineto{\pgfqpoint{3.109696in}{0.976000in}}%
\pgfpathlineto{\pgfqpoint{3.096270in}{0.986864in}}%
\pgfpathlineto{\pgfqpoint{3.084606in}{0.994797in}}%
\pgfpathlineto{\pgfqpoint{3.004444in}{1.057446in}}%
\pgfpathlineto{\pgfqpoint{2.922647in}{1.125333in}}%
\pgfpathlineto{\pgfqpoint{2.838021in}{1.200000in}}%
\pgfpathlineto{\pgfqpoint{2.797338in}{1.237333in}}%
\pgfpathlineto{\pgfqpoint{2.781324in}{1.253508in}}%
\pgfpathlineto{\pgfqpoint{2.757657in}{1.274667in}}%
\pgfpathlineto{\pgfqpoint{2.718921in}{1.312000in}}%
\pgfpathlineto{\pgfqpoint{2.702508in}{1.329427in}}%
\pgfpathlineto{\pgfqpoint{2.681077in}{1.349333in}}%
\pgfpathlineto{\pgfqpoint{2.663845in}{1.368082in}}%
\pgfpathlineto{\pgfqpoint{2.643717in}{1.387065in}}%
\pgfpathlineto{\pgfqpoint{2.563556in}{1.471690in}}%
\pgfpathlineto{\pgfqpoint{2.483394in}{1.560759in}}%
\pgfpathlineto{\pgfqpoint{2.460051in}{1.588924in}}%
\pgfpathlineto{\pgfqpoint{2.440203in}{1.610667in}}%
\pgfpathlineto{\pgfqpoint{2.424690in}{1.630654in}}%
\pgfpathlineto{\pgfqpoint{2.403232in}{1.655038in}}%
\pgfpathlineto{\pgfqpoint{2.323071in}{1.755767in}}%
\pgfpathlineto{\pgfqpoint{2.282990in}{1.809337in}}%
\pgfpathlineto{\pgfqpoint{2.228571in}{1.885356in}}%
\pgfpathlineto{\pgfqpoint{2.187779in}{1.946667in}}%
\pgfpathlineto{\pgfqpoint{2.179053in}{1.961854in}}%
\pgfpathlineto{\pgfqpoint{2.162747in}{1.985365in}}%
\pgfpathlineto{\pgfqpoint{2.119089in}{2.058667in}}%
\pgfpathlineto{\pgfqpoint{2.107508in}{2.081880in}}%
\pgfpathlineto{\pgfqpoint{2.098737in}{2.096000in}}%
\pgfpathlineto{\pgfqpoint{2.078963in}{2.133333in}}%
\pgfpathlineto{\pgfqpoint{2.068486in}{2.157534in}}%
\pgfpathlineto{\pgfqpoint{2.061171in}{2.170667in}}%
\pgfpathlineto{\pgfqpoint{2.044037in}{2.208000in}}%
\pgfpathmoveto{\pgfqpoint{3.540397in}{0.700786in}}%
\pgfpathlineto{\pgfqpoint{3.565576in}{0.691991in}}%
\pgfpathlineto{\pgfqpoint{3.578188in}{0.689081in}}%
\pgfpathlineto{\pgfqpoint{3.613941in}{0.677333in}}%
\pgfpathlineto{\pgfqpoint{3.645737in}{0.670156in}}%
\pgfpathlineto{\pgfqpoint{3.675112in}{0.667361in}}%
\pgfpathlineto{\pgfqpoint{3.685818in}{0.664990in}}%
\pgfpathlineto{\pgfqpoint{3.700226in}{0.663913in}}%
\pgfpathlineto{\pgfqpoint{3.725899in}{0.665075in}}%
\pgfpathlineto{\pgfqpoint{3.765980in}{0.672979in}}%
\pgfpathlineto{\pgfqpoint{3.775845in}{0.677333in}}%
\pgfpathlineto{\pgfqpoint{3.795040in}{0.687599in}}%
\pgfpathlineto{\pgfqpoint{3.815153in}{0.706197in}}%
\pgfpathlineto{\pgfqpoint{3.829895in}{0.729799in}}%
\pgfpathlineto{\pgfqpoint{3.837570in}{0.752000in}}%
\pgfpathlineto{\pgfqpoint{3.844139in}{0.791199in}}%
\pgfpathlineto{\pgfqpoint{3.843444in}{0.826667in}}%
\pgfpathlineto{\pgfqpoint{3.837151in}{0.872374in}}%
\pgfpathlineto{\pgfqpoint{3.830158in}{0.901333in}}%
\pgfpathlineto{\pgfqpoint{3.824179in}{0.918210in}}%
\pgfpathlineto{\pgfqpoint{3.819542in}{0.938667in}}%
\pgfpathlineto{\pgfqpoint{3.815716in}{0.947660in}}%
\pgfpathlineto{\pgfqpoint{3.806061in}{0.978734in}}%
\pgfpathlineto{\pgfqpoint{3.795205in}{1.003222in}}%
\pgfpathlineto{\pgfqpoint{3.792011in}{1.013333in}}%
\pgfpathlineto{\pgfqpoint{3.775788in}{1.050667in}}%
\pgfpathlineto{\pgfqpoint{3.772454in}{1.056697in}}%
\pgfpathlineto{\pgfqpoint{3.758085in}{1.088000in}}%
\pgfpathlineto{\pgfqpoint{3.746909in}{1.107570in}}%
\pgfpathlineto{\pgfqpoint{3.738909in}{1.125333in}}%
\pgfpathlineto{\pgfqpoint{3.718799in}{1.162667in}}%
\pgfpathlineto{\pgfqpoint{3.706577in}{1.182002in}}%
\pgfpathlineto{\pgfqpoint{3.697340in}{1.200000in}}%
\pgfpathlineto{\pgfqpoint{3.674977in}{1.237333in}}%
\pgfpathlineto{\pgfqpoint{3.663568in}{1.253942in}}%
\pgfpathlineto{\pgfqpoint{3.645737in}{1.283975in}}%
\pgfpathlineto{\pgfqpoint{3.618483in}{1.323947in}}%
\pgfpathlineto{\pgfqpoint{3.602438in}{1.349333in}}%
\pgfpathlineto{\pgfqpoint{3.587208in}{1.369483in}}%
\pgfpathlineto{\pgfqpoint{3.565576in}{1.401656in}}%
\pgfpathlineto{\pgfqpoint{3.549374in}{1.424000in}}%
\pgfpathlineto{\pgfqpoint{3.522054in}{1.461333in}}%
\pgfpathlineto{\pgfqpoint{3.506180in}{1.480675in}}%
\pgfpathlineto{\pgfqpoint{3.485414in}{1.509180in}}%
\pgfpathlineto{\pgfqpoint{3.464284in}{1.536000in}}%
\pgfpathlineto{\pgfqpoint{3.434496in}{1.573333in}}%
\pgfpathlineto{\pgfqpoint{3.421308in}{1.588288in}}%
\pgfpathlineto{\pgfqpoint{3.404158in}{1.610667in}}%
\pgfpathlineto{\pgfqpoint{3.386344in}{1.630388in}}%
\pgfpathlineto{\pgfqpoint{3.365172in}{1.656717in}}%
\pgfpathlineto{\pgfqpoint{3.350949in}{1.672085in}}%
\pgfpathlineto{\pgfqpoint{3.325091in}{1.702850in}}%
\pgfpathlineto{\pgfqpoint{3.307503in}{1.722667in}}%
\pgfpathlineto{\pgfqpoint{3.239780in}{1.797333in}}%
\pgfpathlineto{\pgfqpoint{3.204839in}{1.834667in}}%
\pgfpathlineto{\pgfqpoint{3.185199in}{1.853698in}}%
\pgfpathlineto{\pgfqpoint{3.164768in}{1.876118in}}%
\pgfpathlineto{\pgfqpoint{3.147074in}{1.892853in}}%
\pgfpathlineto{\pgfqpoint{3.124687in}{1.916620in}}%
\pgfpathlineto{\pgfqpoint{3.044525in}{1.994803in}}%
\pgfpathlineto{\pgfqpoint{3.029860in}{2.007673in}}%
\pgfpathlineto{\pgfqpoint{3.004444in}{2.032512in}}%
\pgfpathlineto{\pgfqpoint{2.924283in}{2.105232in}}%
\pgfpathlineto{\pgfqpoint{2.908158in}{2.118314in}}%
\pgfpathlineto{\pgfqpoint{2.884202in}{2.140269in}}%
\pgfpathlineto{\pgfqpoint{2.803718in}{2.208000in}}%
\pgfpathlineto{\pgfqpoint{2.723879in}{2.271058in}}%
\pgfpathlineto{\pgfqpoint{2.708591in}{2.282667in}}%
\pgfpathlineto{\pgfqpoint{2.643717in}{2.330875in}}%
\pgfpathlineto{\pgfqpoint{2.626969in}{2.341733in}}%
\pgfpathlineto{\pgfqpoint{2.603636in}{2.359504in}}%
\pgfpathlineto{\pgfqpoint{2.581067in}{2.373645in}}%
\pgfpathlineto{\pgfqpoint{2.551374in}{2.394667in}}%
\pgfpathlineto{\pgfqpoint{2.483394in}{2.438089in}}%
\pgfpathlineto{\pgfqpoint{2.462130in}{2.449527in}}%
\pgfpathlineto{\pgfqpoint{2.430053in}{2.469333in}}%
\pgfpathlineto{\pgfqpoint{2.403232in}{2.484279in}}%
\pgfpathlineto{\pgfqpoint{2.361102in}{2.506667in}}%
\pgfpathlineto{\pgfqpoint{2.335215in}{2.517978in}}%
\pgfpathlineto{\pgfqpoint{2.323071in}{2.524741in}}%
\pgfpathlineto{\pgfqpoint{2.279748in}{2.544000in}}%
\pgfpathlineto{\pgfqpoint{2.242909in}{2.557951in}}%
\pgfpathlineto{\pgfqpoint{2.223365in}{2.563129in}}%
\pgfpathlineto{\pgfqpoint{2.202828in}{2.571162in}}%
\pgfpathlineto{\pgfqpoint{2.162084in}{2.581952in}}%
\pgfpathlineto{\pgfqpoint{2.114343in}{2.589087in}}%
\pgfpathlineto{\pgfqpoint{2.073830in}{2.589489in}}%
\pgfpathlineto{\pgfqpoint{2.042505in}{2.584416in}}%
\pgfpathlineto{\pgfqpoint{2.033543in}{2.581333in}}%
\pgfpathlineto{\pgfqpoint{2.011953in}{2.572458in}}%
\pgfpathlineto{\pgfqpoint{2.002424in}{2.565686in}}%
\pgfpathlineto{\pgfqpoint{1.980223in}{2.544000in}}%
\pgfpathlineto{\pgfqpoint{1.972704in}{2.534350in}}%
\pgfpathlineto{\pgfqpoint{1.960539in}{2.506667in}}%
\pgfpathlineto{\pgfqpoint{1.954376in}{2.476755in}}%
\pgfpathlineto{\pgfqpoint{1.954734in}{2.462245in}}%
\pgfpathlineto{\pgfqpoint{1.953762in}{2.432000in}}%
\pgfpathlineto{\pgfqpoint{1.955196in}{2.425343in}}%
\pgfpathlineto{\pgfqpoint{1.957844in}{2.394667in}}%
\pgfpathlineto{\pgfqpoint{1.966233in}{2.353710in}}%
\pgfpathlineto{\pgfqpoint{1.981624in}{2.302041in}}%
\pgfpathlineto{\pgfqpoint{2.002963in}{2.244831in}}%
\pgfpathlineto{\pgfqpoint{2.038757in}{2.167175in}}%
\pgfpathlineto{\pgfqpoint{2.042505in}{2.159182in}}%
\pgfpathlineto{\pgfqpoint{2.065430in}{2.117354in}}%
\pgfpathlineto{\pgfqpoint{2.082586in}{2.083912in}}%
\pgfpathlineto{\pgfqpoint{2.097136in}{2.058667in}}%
\pgfpathlineto{\pgfqpoint{2.122667in}{2.015606in}}%
\pgfpathlineto{\pgfqpoint{2.172377in}{1.937698in}}%
\pgfpathlineto{\pgfqpoint{2.217664in}{1.872000in}}%
\pgfpathlineto{\pgfqpoint{2.228261in}{1.858356in}}%
\pgfpathlineto{\pgfqpoint{2.246298in}{1.831510in}}%
\pgfpathlineto{\pgfqpoint{2.300132in}{1.760000in}}%
\pgfpathlineto{\pgfqpoint{2.310194in}{1.748006in}}%
\pgfpathlineto{\pgfqpoint{2.328963in}{1.722667in}}%
\pgfpathlineto{\pgfqpoint{2.344163in}{1.704979in}}%
\pgfpathlineto{\pgfqpoint{2.363152in}{1.679841in}}%
\pgfpathlineto{\pgfqpoint{2.443313in}{1.584083in}}%
\pgfpathlineto{\pgfqpoint{2.466985in}{1.558049in}}%
\pgfpathlineto{\pgfqpoint{2.485136in}{1.536000in}}%
\pgfpathlineto{\pgfqpoint{2.503262in}{1.517173in}}%
\pgfpathlineto{\pgfqpoint{2.523475in}{1.493442in}}%
\pgfpathlineto{\pgfqpoint{2.603636in}{1.407793in}}%
\pgfpathlineto{\pgfqpoint{2.683798in}{1.326152in}}%
\pgfpathlineto{\pgfqpoint{2.698163in}{1.312000in}}%
\pgfpathlineto{\pgfqpoint{2.775755in}{1.237333in}}%
\pgfpathlineto{\pgfqpoint{2.857292in}{1.162667in}}%
\pgfpathlineto{\pgfqpoint{2.871532in}{1.150865in}}%
\pgfpathlineto{\pgfqpoint{2.899696in}{1.125333in}}%
\pgfpathlineto{\pgfqpoint{2.964364in}{1.070251in}}%
\pgfpathlineto{\pgfqpoint{3.044525in}{1.005205in}}%
\pgfpathlineto{\pgfqpoint{3.124687in}{0.943940in}}%
\pgfpathlineto{\pgfqpoint{3.151003in}{0.925845in}}%
\pgfpathlineto{\pgfqpoint{3.164768in}{0.914963in}}%
\pgfpathlineto{\pgfqpoint{3.244929in}{0.859605in}}%
\pgfpathlineto{\pgfqpoint{3.266572in}{0.846826in}}%
\pgfpathlineto{\pgfqpoint{3.296495in}{0.826667in}}%
\pgfpathlineto{\pgfqpoint{3.325091in}{0.809143in}}%
\pgfpathlineto{\pgfqpoint{3.375981in}{0.779265in}}%
\pgfpathlineto{\pgfqpoint{3.445333in}{0.742839in}}%
\pgfpathlineto{\pgfqpoint{3.485414in}{0.723889in}}%
\pgfpathlineto{\pgfqpoint{3.492648in}{0.721404in}}%
\pgfpathlineto{\pgfqpoint{3.525495in}{0.706776in}}%
\pgfpathlineto{\pgfqpoint{3.525495in}{0.706776in}}%
\pgfusepath{fill}%
\end{pgfscope}%
\begin{pgfscope}%
\pgfpathrectangle{\pgfqpoint{0.800000in}{0.528000in}}{\pgfqpoint{3.968000in}{3.696000in}}%
\pgfusepath{clip}%
\pgfsetbuttcap%
\pgfsetroundjoin%
\definecolor{currentfill}{rgb}{0.271305,0.019942,0.347269}%
\pgfsetfillcolor{currentfill}%
\pgfsetlinewidth{0.000000pt}%
\definecolor{currentstroke}{rgb}{0.000000,0.000000,0.000000}%
\pgfsetstrokecolor{currentstroke}%
\pgfsetdash{}{0pt}%
\pgfpathmoveto{\pgfqpoint{3.525495in}{0.706776in}}%
\pgfpathlineto{\pgfqpoint{3.520019in}{0.709567in}}%
\pgfpathlineto{\pgfqpoint{3.485414in}{0.723889in}}%
\pgfpathlineto{\pgfqpoint{3.445333in}{0.742839in}}%
\pgfpathlineto{\pgfqpoint{3.375981in}{0.779265in}}%
\pgfpathlineto{\pgfqpoint{3.325091in}{0.809143in}}%
\pgfpathlineto{\pgfqpoint{3.314252in}{0.816571in}}%
\pgfpathlineto{\pgfqpoint{3.285010in}{0.833744in}}%
\pgfpathlineto{\pgfqpoint{3.266572in}{0.846826in}}%
\pgfpathlineto{\pgfqpoint{3.238437in}{0.864000in}}%
\pgfpathlineto{\pgfqpoint{3.164768in}{0.914963in}}%
\pgfpathlineto{\pgfqpoint{3.151003in}{0.925845in}}%
\pgfpathlineto{\pgfqpoint{3.124687in}{0.943940in}}%
\pgfpathlineto{\pgfqpoint{3.081979in}{0.976000in}}%
\pgfpathlineto{\pgfqpoint{3.034337in}{1.013333in}}%
\pgfpathlineto{\pgfqpoint{3.018807in}{1.026711in}}%
\pgfpathlineto{\pgfqpoint{2.988136in}{1.050667in}}%
\pgfpathlineto{\pgfqpoint{2.924283in}{1.104079in}}%
\pgfpathlineto{\pgfqpoint{2.844121in}{1.174409in}}%
\pgfpathlineto{\pgfqpoint{2.763960in}{1.248396in}}%
\pgfpathlineto{\pgfqpoint{2.683798in}{1.326152in}}%
\pgfpathlineto{\pgfqpoint{2.603636in}{1.407793in}}%
\pgfpathlineto{\pgfqpoint{2.588237in}{1.424000in}}%
\pgfpathlineto{\pgfqpoint{2.518733in}{1.498667in}}%
\pgfpathlineto{\pgfqpoint{2.503262in}{1.517173in}}%
\pgfpathlineto{\pgfqpoint{2.483394in}{1.537965in}}%
\pgfpathlineto{\pgfqpoint{2.466985in}{1.558049in}}%
\pgfpathlineto{\pgfqpoint{2.443313in}{1.584083in}}%
\pgfpathlineto{\pgfqpoint{2.358698in}{1.685333in}}%
\pgfpathlineto{\pgfqpoint{2.344163in}{1.704979in}}%
\pgfpathlineto{\pgfqpoint{2.323071in}{1.730186in}}%
\pgfpathlineto{\pgfqpoint{2.310194in}{1.748006in}}%
\pgfpathlineto{\pgfqpoint{2.300132in}{1.760000in}}%
\pgfpathlineto{\pgfqpoint{2.242909in}{1.836203in}}%
\pgfpathlineto{\pgfqpoint{2.228261in}{1.858356in}}%
\pgfpathlineto{\pgfqpoint{2.217664in}{1.872000in}}%
\pgfpathlineto{\pgfqpoint{2.166502in}{1.946667in}}%
\pgfpathlineto{\pgfqpoint{2.162747in}{1.952445in}}%
\pgfpathlineto{\pgfqpoint{2.114353in}{2.029077in}}%
\pgfpathlineto{\pgfqpoint{2.075757in}{2.096000in}}%
\pgfpathlineto{\pgfqpoint{2.065430in}{2.117354in}}%
\pgfpathlineto{\pgfqpoint{2.055808in}{2.133333in}}%
\pgfpathlineto{\pgfqpoint{2.051985in}{2.142164in}}%
\pgfpathlineto{\pgfqpoint{2.031707in}{2.180724in}}%
\pgfpathlineto{\pgfqpoint{2.002424in}{2.246226in}}%
\pgfpathlineto{\pgfqpoint{1.988762in}{2.282667in}}%
\pgfpathlineto{\pgfqpoint{1.976191in}{2.320000in}}%
\pgfpathlineto{\pgfqpoint{1.965197in}{2.359991in}}%
\pgfpathlineto{\pgfqpoint{1.957844in}{2.394667in}}%
\pgfpathlineto{\pgfqpoint{1.953762in}{2.432000in}}%
\pgfpathlineto{\pgfqpoint{1.954376in}{2.476755in}}%
\pgfpathlineto{\pgfqpoint{1.962343in}{2.511532in}}%
\pgfpathlineto{\pgfqpoint{1.972704in}{2.534350in}}%
\pgfpathlineto{\pgfqpoint{1.980223in}{2.544000in}}%
\pgfpathlineto{\pgfqpoint{2.002424in}{2.565686in}}%
\pgfpathlineto{\pgfqpoint{2.011953in}{2.572458in}}%
\pgfpathlineto{\pgfqpoint{2.042505in}{2.584416in}}%
\pgfpathlineto{\pgfqpoint{2.073830in}{2.589489in}}%
\pgfpathlineto{\pgfqpoint{2.114343in}{2.589087in}}%
\pgfpathlineto{\pgfqpoint{2.122667in}{2.588052in}}%
\pgfpathlineto{\pgfqpoint{2.164500in}{2.581333in}}%
\pgfpathlineto{\pgfqpoint{2.202828in}{2.571162in}}%
\pgfpathlineto{\pgfqpoint{2.223365in}{2.563129in}}%
\pgfpathlineto{\pgfqpoint{2.242909in}{2.557951in}}%
\pgfpathlineto{\pgfqpoint{2.285344in}{2.541807in}}%
\pgfpathlineto{\pgfqpoint{2.323071in}{2.524741in}}%
\pgfpathlineto{\pgfqpoint{2.335215in}{2.517978in}}%
\pgfpathlineto{\pgfqpoint{2.365430in}{2.504545in}}%
\pgfpathlineto{\pgfqpoint{2.443313in}{2.461902in}}%
\pgfpathlineto{\pgfqpoint{2.462130in}{2.449527in}}%
\pgfpathlineto{\pgfqpoint{2.493094in}{2.432000in}}%
\pgfpathlineto{\pgfqpoint{2.563556in}{2.386695in}}%
\pgfpathlineto{\pgfqpoint{2.581067in}{2.373645in}}%
\pgfpathlineto{\pgfqpoint{2.606689in}{2.357333in}}%
\pgfpathlineto{\pgfqpoint{2.626969in}{2.341733in}}%
\pgfpathlineto{\pgfqpoint{2.658551in}{2.320000in}}%
\pgfpathlineto{\pgfqpoint{2.723879in}{2.271058in}}%
\pgfpathlineto{\pgfqpoint{2.806452in}{2.205753in}}%
\pgfpathlineto{\pgfqpoint{2.892167in}{2.133333in}}%
\pgfpathlineto{\pgfqpoint{2.908158in}{2.118314in}}%
\pgfpathlineto{\pgfqpoint{2.934623in}{2.096000in}}%
\pgfpathlineto{\pgfqpoint{3.016367in}{2.021333in}}%
\pgfpathlineto{\pgfqpoint{3.029860in}{2.007673in}}%
\pgfpathlineto{\pgfqpoint{3.055775in}{1.984000in}}%
\pgfpathlineto{\pgfqpoint{3.069415in}{1.969850in}}%
\pgfpathlineto{\pgfqpoint{3.094278in}{1.946667in}}%
\pgfpathlineto{\pgfqpoint{3.131927in}{1.909333in}}%
\pgfpathlineto{\pgfqpoint{3.147074in}{1.892853in}}%
\pgfpathlineto{\pgfqpoint{3.168766in}{1.872000in}}%
\pgfpathlineto{\pgfqpoint{3.185199in}{1.853698in}}%
\pgfpathlineto{\pgfqpoint{3.204848in}{1.834656in}}%
\pgfpathlineto{\pgfqpoint{3.285010in}{1.747872in}}%
\pgfpathlineto{\pgfqpoint{3.340362in}{1.685333in}}%
\pgfpathlineto{\pgfqpoint{3.350949in}{1.672085in}}%
\pgfpathlineto{\pgfqpoint{3.372598in}{1.648000in}}%
\pgfpathlineto{\pgfqpoint{3.386344in}{1.630388in}}%
\pgfpathlineto{\pgfqpoint{3.405253in}{1.609343in}}%
\pgfpathlineto{\pgfqpoint{3.421308in}{1.588288in}}%
\pgfpathlineto{\pgfqpoint{3.445333in}{1.559906in}}%
\pgfpathlineto{\pgfqpoint{3.493551in}{1.498667in}}%
\pgfpathlineto{\pgfqpoint{3.506180in}{1.480675in}}%
\pgfpathlineto{\pgfqpoint{3.525495in}{1.456707in}}%
\pgfpathlineto{\pgfqpoint{3.576248in}{1.386667in}}%
\pgfpathlineto{\pgfqpoint{3.587208in}{1.369483in}}%
\pgfpathlineto{\pgfqpoint{3.605657in}{1.344577in}}%
\pgfpathlineto{\pgfqpoint{3.618483in}{1.323947in}}%
\pgfpathlineto{\pgfqpoint{3.627280in}{1.312000in}}%
\pgfpathlineto{\pgfqpoint{3.633797in}{1.300878in}}%
\pgfpathlineto{\pgfqpoint{3.651754in}{1.274667in}}%
\pgfpathlineto{\pgfqpoint{3.663568in}{1.253942in}}%
\pgfpathlineto{\pgfqpoint{3.678555in}{1.230568in}}%
\pgfpathlineto{\pgfqpoint{3.697340in}{1.200000in}}%
\pgfpathlineto{\pgfqpoint{3.706577in}{1.182002in}}%
\pgfpathlineto{\pgfqpoint{3.725899in}{1.149663in}}%
\pgfpathlineto{\pgfqpoint{3.738909in}{1.125333in}}%
\pgfpathlineto{\pgfqpoint{3.746909in}{1.107570in}}%
\pgfpathlineto{\pgfqpoint{3.760208in}{1.082624in}}%
\pgfpathlineto{\pgfqpoint{3.765980in}{1.071553in}}%
\pgfpathlineto{\pgfqpoint{3.807145in}{0.976000in}}%
\pgfpathlineto{\pgfqpoint{3.830158in}{0.901333in}}%
\pgfpathlineto{\pgfqpoint{3.838414in}{0.864000in}}%
\pgfpathlineto{\pgfqpoint{3.838595in}{0.856971in}}%
\pgfpathlineto{\pgfqpoint{3.843444in}{0.826667in}}%
\pgfpathlineto{\pgfqpoint{3.843902in}{0.789333in}}%
\pgfpathlineto{\pgfqpoint{3.843085in}{0.786486in}}%
\pgfpathlineto{\pgfqpoint{3.837570in}{0.752000in}}%
\pgfpathlineto{\pgfqpoint{3.829895in}{0.729799in}}%
\pgfpathlineto{\pgfqpoint{3.815153in}{0.706197in}}%
\pgfpathlineto{\pgfqpoint{3.795040in}{0.687599in}}%
\pgfpathlineto{\pgfqpoint{3.765980in}{0.672979in}}%
\pgfpathlineto{\pgfqpoint{3.760772in}{0.672483in}}%
\pgfpathlineto{\pgfqpoint{3.725899in}{0.665075in}}%
\pgfpathlineto{\pgfqpoint{3.700226in}{0.663913in}}%
\pgfpathlineto{\pgfqpoint{3.685818in}{0.664990in}}%
\pgfpathlineto{\pgfqpoint{3.675112in}{0.667361in}}%
\pgfpathlineto{\pgfqpoint{3.645737in}{0.670156in}}%
\pgfpathlineto{\pgfqpoint{3.605657in}{0.679218in}}%
\pgfpathlineto{\pgfqpoint{3.578188in}{0.689081in}}%
\pgfpathlineto{\pgfqpoint{3.565576in}{0.691991in}}%
\pgfpathlineto{\pgfqpoint{3.540397in}{0.700786in}}%
\pgfpathmoveto{\pgfqpoint{3.535018in}{0.677333in}}%
\pgfpathlineto{\pgfqpoint{3.605657in}{0.651055in}}%
\pgfpathlineto{\pgfqpoint{3.646909in}{0.638909in}}%
\pgfpathlineto{\pgfqpoint{3.685818in}{0.631297in}}%
\pgfpathlineto{\pgfqpoint{3.714519in}{0.629400in}}%
\pgfpathlineto{\pgfqpoint{3.725899in}{0.627265in}}%
\pgfpathlineto{\pgfqpoint{3.740223in}{0.626658in}}%
\pgfpathlineto{\pgfqpoint{3.776711in}{0.630004in}}%
\pgfpathlineto{\pgfqpoint{3.809392in}{0.640000in}}%
\pgfpathlineto{\pgfqpoint{3.846141in}{0.666390in}}%
\pgfpathlineto{\pgfqpoint{3.851788in}{0.672073in}}%
\pgfpathlineto{\pgfqpoint{3.855026in}{0.677333in}}%
\pgfpathlineto{\pgfqpoint{3.860411in}{0.690624in}}%
\pgfpathlineto{\pgfqpoint{3.872636in}{0.714667in}}%
\pgfpathlineto{\pgfqpoint{3.880490in}{0.757339in}}%
\pgfpathlineto{\pgfqpoint{3.880063in}{0.795070in}}%
\pgfpathlineto{\pgfqpoint{3.875945in}{0.826667in}}%
\pgfpathlineto{\pgfqpoint{3.870283in}{0.849153in}}%
\pgfpathlineto{\pgfqpoint{3.868425in}{0.864000in}}%
\pgfpathlineto{\pgfqpoint{3.863095in}{0.879791in}}%
\pgfpathlineto{\pgfqpoint{3.858474in}{0.901333in}}%
\pgfpathlineto{\pgfqpoint{3.846141in}{0.940067in}}%
\pgfpathlineto{\pgfqpoint{3.835193in}{0.965802in}}%
\pgfpathlineto{\pgfqpoint{3.832110in}{0.976000in}}%
\pgfpathlineto{\pgfqpoint{3.816445in}{1.013333in}}%
\pgfpathlineto{\pgfqpoint{3.812990in}{1.019788in}}%
\pgfpathlineto{\pgfqpoint{3.799359in}{1.050667in}}%
\pgfpathlineto{\pgfqpoint{3.788014in}{1.071191in}}%
\pgfpathlineto{\pgfqpoint{3.780688in}{1.088000in}}%
\pgfpathlineto{\pgfqpoint{3.756527in}{1.134138in}}%
\pgfpathlineto{\pgfqpoint{3.709462in}{1.215310in}}%
\pgfpathlineto{\pgfqpoint{3.671947in}{1.274667in}}%
\pgfpathlineto{\pgfqpoint{3.661446in}{1.289299in}}%
\pgfpathlineto{\pgfqpoint{3.645737in}{1.314876in}}%
\pgfpathlineto{\pgfqpoint{3.630687in}{1.335315in}}%
\pgfpathlineto{\pgfqpoint{3.621972in}{1.349333in}}%
\pgfpathlineto{\pgfqpoint{3.565576in}{1.428721in}}%
\pgfpathlineto{\pgfqpoint{3.483704in}{1.536000in}}%
\pgfpathlineto{\pgfqpoint{3.445333in}{1.583467in}}%
\pgfpathlineto{\pgfqpoint{3.422811in}{1.610667in}}%
\pgfpathlineto{\pgfqpoint{3.359586in}{1.685333in}}%
\pgfpathlineto{\pgfqpoint{3.293230in}{1.760000in}}%
\pgfpathlineto{\pgfqpoint{3.244929in}{1.812373in}}%
\pgfpathlineto{\pgfqpoint{3.164768in}{1.896067in}}%
\pgfpathlineto{\pgfqpoint{3.151651in}{1.909333in}}%
\pgfpathlineto{\pgfqpoint{3.076243in}{1.984000in}}%
\pgfpathlineto{\pgfqpoint{2.997272in}{2.058667in}}%
\pgfpathlineto{\pgfqpoint{2.980126in}{2.073348in}}%
\pgfpathlineto{\pgfqpoint{2.956317in}{2.096000in}}%
\pgfpathlineto{\pgfqpoint{2.939432in}{2.110111in}}%
\pgfpathlineto{\pgfqpoint{2.914303in}{2.133333in}}%
\pgfpathlineto{\pgfqpoint{2.898252in}{2.146420in}}%
\pgfpathlineto{\pgfqpoint{2.871164in}{2.170667in}}%
\pgfpathlineto{\pgfqpoint{2.856575in}{2.182267in}}%
\pgfpathlineto{\pgfqpoint{2.826827in}{2.208000in}}%
\pgfpathlineto{\pgfqpoint{2.804040in}{2.226851in}}%
\pgfpathlineto{\pgfqpoint{2.723879in}{2.290799in}}%
\pgfpathlineto{\pgfqpoint{2.643717in}{2.351036in}}%
\pgfpathlineto{\pgfqpoint{2.634905in}{2.357333in}}%
\pgfpathlineto{\pgfqpoint{2.563556in}{2.407256in}}%
\pgfpathlineto{\pgfqpoint{2.523475in}{2.434126in}}%
\pgfpathlineto{\pgfqpoint{2.500239in}{2.447690in}}%
\pgfpathlineto{\pgfqpoint{2.476665in}{2.463066in}}%
\pgfpathlineto{\pgfqpoint{2.467136in}{2.469333in}}%
\pgfpathlineto{\pgfqpoint{2.427470in}{2.491909in}}%
\pgfpathlineto{\pgfqpoint{2.403232in}{2.507198in}}%
\pgfpathlineto{\pgfqpoint{2.377402in}{2.519940in}}%
\pgfpathlineto{\pgfqpoint{2.363152in}{2.528605in}}%
\pgfpathlineto{\pgfqpoint{2.351902in}{2.533521in}}%
\pgfpathlineto{\pgfqpoint{2.323071in}{2.549098in}}%
\pgfpathlineto{\pgfqpoint{2.299104in}{2.559010in}}%
\pgfpathlineto{\pgfqpoint{2.282990in}{2.567561in}}%
\pgfpathlineto{\pgfqpoint{2.236650in}{2.587164in}}%
\pgfpathlineto{\pgfqpoint{2.202828in}{2.599050in}}%
\pgfpathlineto{\pgfqpoint{2.186060in}{2.603048in}}%
\pgfpathlineto{\pgfqpoint{2.162747in}{2.611462in}}%
\pgfpathlineto{\pgfqpoint{2.122667in}{2.620907in}}%
\pgfpathlineto{\pgfqpoint{2.082586in}{2.626107in}}%
\pgfpathlineto{\pgfqpoint{2.034634in}{2.625998in}}%
\pgfpathlineto{\pgfqpoint{2.000048in}{2.618667in}}%
\pgfpathlineto{\pgfqpoint{1.974678in}{2.607178in}}%
\pgfpathlineto{\pgfqpoint{1.962343in}{2.597841in}}%
\pgfpathlineto{\pgfqpoint{1.946141in}{2.581333in}}%
\pgfpathlineto{\pgfqpoint{1.936316in}{2.568243in}}%
\pgfpathlineto{\pgfqpoint{1.925515in}{2.544000in}}%
\pgfpathlineto{\pgfqpoint{1.917985in}{2.510651in}}%
\pgfpathlineto{\pgfqpoint{1.916787in}{2.469333in}}%
\pgfpathlineto{\pgfqpoint{1.917650in}{2.465037in}}%
\pgfpathlineto{\pgfqpoint{1.920178in}{2.432000in}}%
\pgfpathlineto{\pgfqpoint{1.928534in}{2.388825in}}%
\pgfpathlineto{\pgfqpoint{1.937381in}{2.357333in}}%
\pgfpathlineto{\pgfqpoint{1.943978in}{2.340227in}}%
\pgfpathlineto{\pgfqpoint{1.949315in}{2.320000in}}%
\pgfpathlineto{\pgfqpoint{1.953159in}{2.311445in}}%
\pgfpathlineto{\pgfqpoint{1.962757in}{2.282667in}}%
\pgfpathlineto{\pgfqpoint{1.974776in}{2.256913in}}%
\pgfpathlineto{\pgfqpoint{1.978775in}{2.245333in}}%
\pgfpathlineto{\pgfqpoint{1.986497in}{2.230498in}}%
\pgfpathlineto{\pgfqpoint{2.002424in}{2.193993in}}%
\pgfpathlineto{\pgfqpoint{2.014046in}{2.170667in}}%
\pgfpathlineto{\pgfqpoint{2.024072in}{2.153497in}}%
\pgfpathlineto{\pgfqpoint{2.033527in}{2.133333in}}%
\pgfpathlineto{\pgfqpoint{2.054239in}{2.096000in}}%
\pgfpathlineto{\pgfqpoint{2.064835in}{2.079466in}}%
\pgfpathlineto{\pgfqpoint{2.082586in}{2.047440in}}%
\pgfpathlineto{\pgfqpoint{2.122667in}{1.982742in}}%
\pgfpathlineto{\pgfqpoint{2.171845in}{1.909333in}}%
\pgfpathlineto{\pgfqpoint{2.184662in}{1.892413in}}%
\pgfpathlineto{\pgfqpoint{2.202828in}{1.865019in}}%
\pgfpathlineto{\pgfqpoint{2.216507in}{1.847408in}}%
\pgfpathlineto{\pgfqpoint{2.232608in}{1.825072in}}%
\pgfpathlineto{\pgfqpoint{2.252464in}{1.797333in}}%
\pgfpathlineto{\pgfqpoint{2.282990in}{1.756973in}}%
\pgfpathlineto{\pgfqpoint{2.299145in}{1.737714in}}%
\pgfpathlineto{\pgfqpoint{2.323071in}{1.706198in}}%
\pgfpathlineto{\pgfqpoint{2.339892in}{1.685333in}}%
\pgfpathlineto{\pgfqpoint{2.403232in}{1.608500in}}%
\pgfpathlineto{\pgfqpoint{2.483394in}{1.516823in}}%
\pgfpathlineto{\pgfqpoint{2.511057in}{1.487100in}}%
\pgfpathlineto{\pgfqpoint{2.533850in}{1.461333in}}%
\pgfpathlineto{\pgfqpoint{2.548150in}{1.446984in}}%
\pgfpathlineto{\pgfqpoint{2.568605in}{1.424000in}}%
\pgfpathlineto{\pgfqpoint{2.585668in}{1.407263in}}%
\pgfpathlineto{\pgfqpoint{2.608436in}{1.382196in}}%
\pgfpathlineto{\pgfqpoint{2.683798in}{1.306154in}}%
\pgfpathlineto{\pgfqpoint{2.763960in}{1.228806in}}%
\pgfpathlineto{\pgfqpoint{2.844121in}{1.154954in}}%
\pgfpathlineto{\pgfqpoint{2.860949in}{1.141007in}}%
\pgfpathlineto{\pgfqpoint{2.884202in}{1.119309in}}%
\pgfpathlineto{\pgfqpoint{2.924283in}{1.084501in}}%
\pgfpathlineto{\pgfqpoint{2.943970in}{1.069004in}}%
\pgfpathlineto{\pgfqpoint{2.965768in}{1.049359in}}%
\pgfpathlineto{\pgfqpoint{3.009832in}{1.013333in}}%
\pgfpathlineto{\pgfqpoint{3.029093in}{0.998959in}}%
\pgfpathlineto{\pgfqpoint{3.056908in}{0.976000in}}%
\pgfpathlineto{\pgfqpoint{3.072473in}{0.964698in}}%
\pgfpathlineto{\pgfqpoint{3.084606in}{0.954507in}}%
\pgfpathlineto{\pgfqpoint{3.164768in}{0.894704in}}%
\pgfpathlineto{\pgfqpoint{3.184159in}{0.882062in}}%
\pgfpathlineto{\pgfqpoint{3.208058in}{0.864000in}}%
\pgfpathlineto{\pgfqpoint{3.285010in}{0.812737in}}%
\pgfpathlineto{\pgfqpoint{3.300612in}{0.803866in}}%
\pgfpathlineto{\pgfqpoint{3.331610in}{0.783262in}}%
\pgfpathlineto{\pgfqpoint{3.405253in}{0.740738in}}%
\pgfpathlineto{\pgfqpoint{3.454565in}{0.714667in}}%
\pgfpathlineto{\pgfqpoint{3.485414in}{0.699579in}}%
\pgfpathlineto{\pgfqpoint{3.502245in}{0.693011in}}%
\pgfpathlineto{\pgfqpoint{3.525495in}{0.681161in}}%
\pgfpathlineto{\pgfqpoint{3.525495in}{0.681161in}}%
\pgfusepath{fill}%
\end{pgfscope}%
\begin{pgfscope}%
\pgfpathrectangle{\pgfqpoint{0.800000in}{0.528000in}}{\pgfqpoint{3.968000in}{3.696000in}}%
\pgfusepath{clip}%
\pgfsetbuttcap%
\pgfsetroundjoin%
\definecolor{currentfill}{rgb}{0.271305,0.019942,0.347269}%
\pgfsetfillcolor{currentfill}%
\pgfsetlinewidth{0.000000pt}%
\definecolor{currentstroke}{rgb}{0.000000,0.000000,0.000000}%
\pgfsetstrokecolor{currentstroke}%
\pgfsetdash{}{0pt}%
\pgfpathmoveto{\pgfqpoint{3.525495in}{0.681161in}}%
\pgfpathlineto{\pgfqpoint{3.502245in}{0.693011in}}%
\pgfpathlineto{\pgfqpoint{3.485414in}{0.699579in}}%
\pgfpathlineto{\pgfqpoint{3.434461in}{0.724793in}}%
\pgfpathlineto{\pgfqpoint{3.365172in}{0.763485in}}%
\pgfpathlineto{\pgfqpoint{3.321818in}{0.789333in}}%
\pgfpathlineto{\pgfqpoint{3.300612in}{0.803866in}}%
\pgfpathlineto{\pgfqpoint{3.285010in}{0.812737in}}%
\pgfpathlineto{\pgfqpoint{3.204848in}{0.866184in}}%
\pgfpathlineto{\pgfqpoint{3.184159in}{0.882062in}}%
\pgfpathlineto{\pgfqpoint{3.155724in}{0.901333in}}%
\pgfpathlineto{\pgfqpoint{3.138936in}{0.914606in}}%
\pgfpathlineto{\pgfqpoint{3.124687in}{0.924194in}}%
\pgfpathlineto{\pgfqpoint{3.044525in}{0.985656in}}%
\pgfpathlineto{\pgfqpoint{3.029093in}{0.998959in}}%
\pgfpathlineto{\pgfqpoint{3.004444in}{1.017652in}}%
\pgfpathlineto{\pgfqpoint{2.964190in}{1.050667in}}%
\pgfpathlineto{\pgfqpoint{2.943970in}{1.069004in}}%
\pgfpathlineto{\pgfqpoint{2.920237in}{1.088000in}}%
\pgfpathlineto{\pgfqpoint{2.877402in}{1.125333in}}%
\pgfpathlineto{\pgfqpoint{2.860949in}{1.141007in}}%
\pgfpathlineto{\pgfqpoint{2.835620in}{1.162667in}}%
\pgfpathlineto{\pgfqpoint{2.754987in}{1.237333in}}%
\pgfpathlineto{\pgfqpoint{2.677919in}{1.312000in}}%
\pgfpathlineto{\pgfqpoint{2.603636in}{1.387133in}}%
\pgfpathlineto{\pgfqpoint{2.585668in}{1.407263in}}%
\pgfpathlineto{\pgfqpoint{2.563556in}{1.429341in}}%
\pgfpathlineto{\pgfqpoint{2.548150in}{1.446984in}}%
\pgfpathlineto{\pgfqpoint{2.523475in}{1.472566in}}%
\pgfpathlineto{\pgfqpoint{2.492610in}{1.507251in}}%
\pgfpathlineto{\pgfqpoint{2.466377in}{1.536000in}}%
\pgfpathlineto{\pgfqpoint{2.456410in}{1.548199in}}%
\pgfpathlineto{\pgfqpoint{2.433590in}{1.573333in}}%
\pgfpathlineto{\pgfqpoint{2.370291in}{1.648000in}}%
\pgfpathlineto{\pgfqpoint{2.310019in}{1.722667in}}%
\pgfpathlineto{\pgfqpoint{2.299145in}{1.737714in}}%
\pgfpathlineto{\pgfqpoint{2.274693in}{1.767728in}}%
\pgfpathlineto{\pgfqpoint{2.224932in}{1.834667in}}%
\pgfpathlineto{\pgfqpoint{2.216507in}{1.847408in}}%
\pgfpathlineto{\pgfqpoint{2.197832in}{1.872000in}}%
\pgfpathlineto{\pgfqpoint{2.184662in}{1.892413in}}%
\pgfpathlineto{\pgfqpoint{2.162747in}{1.922665in}}%
\pgfpathlineto{\pgfqpoint{2.146664in}{1.946667in}}%
\pgfpathlineto{\pgfqpoint{2.121844in}{1.984000in}}%
\pgfpathlineto{\pgfqpoint{2.108059in}{2.007727in}}%
\pgfpathlineto{\pgfqpoint{2.098640in}{2.021333in}}%
\pgfpathlineto{\pgfqpoint{2.075810in}{2.058667in}}%
\pgfpathlineto{\pgfqpoint{2.064835in}{2.079466in}}%
\pgfpathlineto{\pgfqpoint{2.050595in}{2.103536in}}%
\pgfpathlineto{\pgfqpoint{2.033527in}{2.133333in}}%
\pgfpathlineto{\pgfqpoint{2.024072in}{2.153497in}}%
\pgfpathlineto{\pgfqpoint{2.014046in}{2.170667in}}%
\pgfpathlineto{\pgfqpoint{1.995589in}{2.208000in}}%
\pgfpathlineto{\pgfqpoint{1.986497in}{2.230498in}}%
\pgfpathlineto{\pgfqpoint{1.978775in}{2.245333in}}%
\pgfpathlineto{\pgfqpoint{1.974776in}{2.256913in}}%
\pgfpathlineto{\pgfqpoint{1.962343in}{2.283789in}}%
\pgfpathlineto{\pgfqpoint{1.928534in}{2.388825in}}%
\pgfpathlineto{\pgfqpoint{1.920178in}{2.432000in}}%
\pgfpathlineto{\pgfqpoint{1.916787in}{2.469333in}}%
\pgfpathlineto{\pgfqpoint{1.917985in}{2.510651in}}%
\pgfpathlineto{\pgfqpoint{1.925515in}{2.544000in}}%
\pgfpathlineto{\pgfqpoint{1.936316in}{2.568243in}}%
\pgfpathlineto{\pgfqpoint{1.946141in}{2.581333in}}%
\pgfpathlineto{\pgfqpoint{1.962343in}{2.597841in}}%
\pgfpathlineto{\pgfqpoint{1.974678in}{2.607178in}}%
\pgfpathlineto{\pgfqpoint{2.002424in}{2.619569in}}%
\pgfpathlineto{\pgfqpoint{2.034634in}{2.625998in}}%
\pgfpathlineto{\pgfqpoint{2.042505in}{2.626401in}}%
\pgfpathlineto{\pgfqpoint{2.049967in}{2.625617in}}%
\pgfpathlineto{\pgfqpoint{2.082586in}{2.626107in}}%
\pgfpathlineto{\pgfqpoint{2.088985in}{2.624627in}}%
\pgfpathlineto{\pgfqpoint{2.122667in}{2.620907in}}%
\pgfpathlineto{\pgfqpoint{2.162747in}{2.611462in}}%
\pgfpathlineto{\pgfqpoint{2.186060in}{2.603048in}}%
\pgfpathlineto{\pgfqpoint{2.202828in}{2.599050in}}%
\pgfpathlineto{\pgfqpoint{2.250726in}{2.581333in}}%
\pgfpathlineto{\pgfqpoint{2.282990in}{2.567561in}}%
\pgfpathlineto{\pgfqpoint{2.299104in}{2.559010in}}%
\pgfpathlineto{\pgfqpoint{2.333084in}{2.544000in}}%
\pgfpathlineto{\pgfqpoint{2.351902in}{2.533521in}}%
\pgfpathlineto{\pgfqpoint{2.363152in}{2.528605in}}%
\pgfpathlineto{\pgfqpoint{2.377402in}{2.519940in}}%
\pgfpathlineto{\pgfqpoint{2.404146in}{2.506667in}}%
\pgfpathlineto{\pgfqpoint{2.427470in}{2.491909in}}%
\pgfpathlineto{\pgfqpoint{2.443313in}{2.483742in}}%
\pgfpathlineto{\pgfqpoint{2.452284in}{2.477690in}}%
\pgfpathlineto{\pgfqpoint{2.483394in}{2.459447in}}%
\pgfpathlineto{\pgfqpoint{2.500239in}{2.447690in}}%
\pgfpathlineto{\pgfqpoint{2.526661in}{2.432000in}}%
\pgfpathlineto{\pgfqpoint{2.603636in}{2.379564in}}%
\pgfpathlineto{\pgfqpoint{2.685819in}{2.320000in}}%
\pgfpathlineto{\pgfqpoint{2.763960in}{2.259239in}}%
\pgfpathlineto{\pgfqpoint{2.781215in}{2.245333in}}%
\pgfpathlineto{\pgfqpoint{2.844121in}{2.193624in}}%
\pgfpathlineto{\pgfqpoint{2.856575in}{2.182267in}}%
\pgfpathlineto{\pgfqpoint{2.884202in}{2.159545in}}%
\pgfpathlineto{\pgfqpoint{2.898252in}{2.146420in}}%
\pgfpathlineto{\pgfqpoint{2.924283in}{2.124601in}}%
\pgfpathlineto{\pgfqpoint{2.939432in}{2.110111in}}%
\pgfpathlineto{\pgfqpoint{2.964364in}{2.088781in}}%
\pgfpathlineto{\pgfqpoint{2.980126in}{2.073348in}}%
\pgfpathlineto{\pgfqpoint{3.004444in}{2.052071in}}%
\pgfpathlineto{\pgfqpoint{3.084606in}{1.975929in}}%
\pgfpathlineto{\pgfqpoint{3.164768in}{1.896067in}}%
\pgfpathlineto{\pgfqpoint{3.195749in}{1.863524in}}%
\pgfpathlineto{\pgfqpoint{3.204848in}{1.854706in}}%
\pgfpathlineto{\pgfqpoint{3.293230in}{1.760000in}}%
\pgfpathlineto{\pgfqpoint{3.365172in}{1.678891in}}%
\pgfpathlineto{\pgfqpoint{3.422811in}{1.610667in}}%
\pgfpathlineto{\pgfqpoint{3.432164in}{1.598400in}}%
\pgfpathlineto{\pgfqpoint{3.453583in}{1.573333in}}%
\pgfpathlineto{\pgfqpoint{3.512631in}{1.498667in}}%
\pgfpathlineto{\pgfqpoint{3.534319in}{1.469552in}}%
\pgfpathlineto{\pgfqpoint{3.541075in}{1.461333in}}%
\pgfpathlineto{\pgfqpoint{3.595868in}{1.386667in}}%
\pgfpathlineto{\pgfqpoint{3.605657in}{1.372824in}}%
\pgfpathlineto{\pgfqpoint{3.647687in}{1.312000in}}%
\pgfpathlineto{\pgfqpoint{3.661446in}{1.289299in}}%
\pgfpathlineto{\pgfqpoint{3.676758in}{1.266227in}}%
\pgfpathlineto{\pgfqpoint{3.695705in}{1.237333in}}%
\pgfpathlineto{\pgfqpoint{3.725899in}{1.187497in}}%
\pgfpathlineto{\pgfqpoint{3.765980in}{1.116367in}}%
\pgfpathlineto{\pgfqpoint{3.780688in}{1.088000in}}%
\pgfpathlineto{\pgfqpoint{3.788014in}{1.071191in}}%
\pgfpathlineto{\pgfqpoint{3.799359in}{1.050667in}}%
\pgfpathlineto{\pgfqpoint{3.823884in}{0.996731in}}%
\pgfpathlineto{\pgfqpoint{3.846964in}{0.937901in}}%
\pgfpathlineto{\pgfqpoint{3.858474in}{0.901333in}}%
\pgfpathlineto{\pgfqpoint{3.863095in}{0.879791in}}%
\pgfpathlineto{\pgfqpoint{3.868425in}{0.864000in}}%
\pgfpathlineto{\pgfqpoint{3.870283in}{0.849153in}}%
\pgfpathlineto{\pgfqpoint{3.875945in}{0.826667in}}%
\pgfpathlineto{\pgfqpoint{3.880172in}{0.789333in}}%
\pgfpathlineto{\pgfqpoint{3.879783in}{0.752000in}}%
\pgfpathlineto{\pgfqpoint{3.877439in}{0.743819in}}%
\pgfpathlineto{\pgfqpoint{3.872636in}{0.714667in}}%
\pgfpathlineto{\pgfqpoint{3.860411in}{0.690624in}}%
\pgfpathlineto{\pgfqpoint{3.851788in}{0.672073in}}%
\pgfpathlineto{\pgfqpoint{3.846141in}{0.666390in}}%
\pgfpathlineto{\pgfqpoint{3.806061in}{0.638406in}}%
\pgfpathlineto{\pgfqpoint{3.776711in}{0.630004in}}%
\pgfpathlineto{\pgfqpoint{3.740223in}{0.626658in}}%
\pgfpathlineto{\pgfqpoint{3.725899in}{0.627265in}}%
\pgfpathlineto{\pgfqpoint{3.714519in}{0.629400in}}%
\pgfpathlineto{\pgfqpoint{3.685818in}{0.631297in}}%
\pgfpathlineto{\pgfqpoint{3.643087in}{0.640000in}}%
\pgfpathlineto{\pgfqpoint{3.605657in}{0.651055in}}%
\pgfpathlineto{\pgfqpoint{3.586111in}{0.659128in}}%
\pgfpathlineto{\pgfqpoint{3.565576in}{0.665130in}}%
\pgfpathlineto{\pgfqpoint{3.556948in}{0.669297in}}%
\pgfpathlineto{\pgfqpoint{3.535018in}{0.677333in}}%
\pgfpathmoveto{\pgfqpoint{3.856034in}{0.976000in}}%
\pgfpathlineto{\pgfqpoint{3.852813in}{0.982214in}}%
\pgfpathlineto{\pgfqpoint{3.839516in}{1.013333in}}%
\pgfpathlineto{\pgfqpoint{3.828386in}{1.034128in}}%
\pgfpathlineto{\pgfqpoint{3.821409in}{1.050667in}}%
\pgfpathlineto{\pgfqpoint{3.799300in}{1.094297in}}%
\pgfpathlineto{\pgfqpoint{3.755264in}{1.172648in}}%
\pgfpathlineto{\pgfqpoint{3.700826in}{1.260688in}}%
\pgfpathlineto{\pgfqpoint{3.645737in}{1.342819in}}%
\pgfpathlineto{\pgfqpoint{3.610901in}{1.391552in}}%
\pgfpathlineto{\pgfqpoint{3.587494in}{1.424000in}}%
\pgfpathlineto{\pgfqpoint{3.578112in}{1.435676in}}%
\pgfpathlineto{\pgfqpoint{3.559825in}{1.461333in}}%
\pgfpathlineto{\pgfqpoint{3.544937in}{1.479443in}}%
\pgfpathlineto{\pgfqpoint{3.525495in}{1.506054in}}%
\pgfpathlineto{\pgfqpoint{3.441378in}{1.610667in}}%
\pgfpathlineto{\pgfqpoint{3.377791in}{1.685333in}}%
\pgfpathlineto{\pgfqpoint{3.311618in}{1.760000in}}%
\pgfpathlineto{\pgfqpoint{3.242905in}{1.834667in}}%
\pgfpathlineto{\pgfqpoint{3.204848in}{1.874562in}}%
\pgfpathlineto{\pgfqpoint{3.186485in}{1.892229in}}%
\pgfpathlineto{\pgfqpoint{3.164768in}{1.915548in}}%
\pgfpathlineto{\pgfqpoint{3.084606in}{1.994870in}}%
\pgfpathlineto{\pgfqpoint{3.004444in}{2.070738in}}%
\pgfpathlineto{\pgfqpoint{2.990314in}{2.082838in}}%
\pgfpathlineto{\pgfqpoint{2.964364in}{2.107406in}}%
\pgfpathlineto{\pgfqpoint{2.949692in}{2.119667in}}%
\pgfpathlineto{\pgfqpoint{2.924283in}{2.143245in}}%
\pgfpathlineto{\pgfqpoint{2.908584in}{2.156044in}}%
\pgfpathlineto{\pgfqpoint{2.884202in}{2.178267in}}%
\pgfpathlineto{\pgfqpoint{2.866981in}{2.191960in}}%
\pgfpathlineto{\pgfqpoint{2.844121in}{2.212482in}}%
\pgfpathlineto{\pgfqpoint{2.824871in}{2.227403in}}%
\pgfpathlineto{\pgfqpoint{2.804040in}{2.245902in}}%
\pgfpathlineto{\pgfqpoint{2.782243in}{2.262364in}}%
\pgfpathlineto{\pgfqpoint{2.758345in}{2.282667in}}%
\pgfpathlineto{\pgfqpoint{2.683798in}{2.340365in}}%
\pgfpathlineto{\pgfqpoint{2.603636in}{2.399275in}}%
\pgfpathlineto{\pgfqpoint{2.582860in}{2.412648in}}%
\pgfpathlineto{\pgfqpoint{2.556422in}{2.432000in}}%
\pgfpathlineto{\pgfqpoint{2.536687in}{2.444307in}}%
\pgfpathlineto{\pgfqpoint{2.523475in}{2.453987in}}%
\pgfpathlineto{\pgfqpoint{2.513346in}{2.459899in}}%
\pgfpathlineto{\pgfqpoint{2.483394in}{2.479948in}}%
\pgfpathlineto{\pgfqpoint{2.465511in}{2.490010in}}%
\pgfpathlineto{\pgfqpoint{2.440442in}{2.506667in}}%
\pgfpathlineto{\pgfqpoint{2.416917in}{2.519413in}}%
\pgfpathlineto{\pgfqpoint{2.403232in}{2.528330in}}%
\pgfpathlineto{\pgfqpoint{2.392234in}{2.533755in}}%
\pgfpathlineto{\pgfqpoint{2.363152in}{2.550849in}}%
\pgfpathlineto{\pgfqpoint{2.323071in}{2.571803in}}%
\pgfpathlineto{\pgfqpoint{2.315985in}{2.574733in}}%
\pgfpathlineto{\pgfqpoint{2.282990in}{2.591313in}}%
\pgfpathlineto{\pgfqpoint{2.262297in}{2.599392in}}%
\pgfpathlineto{\pgfqpoint{2.242909in}{2.609180in}}%
\pgfpathlineto{\pgfqpoint{2.202828in}{2.625238in}}%
\pgfpathlineto{\pgfqpoint{2.177779in}{2.632668in}}%
\pgfpathlineto{\pgfqpoint{2.162747in}{2.638864in}}%
\pgfpathlineto{\pgfqpoint{2.122667in}{2.650356in}}%
\pgfpathlineto{\pgfqpoint{2.117249in}{2.650954in}}%
\pgfpathlineto{\pgfqpoint{2.082586in}{2.658642in}}%
\pgfpathlineto{\pgfqpoint{2.034863in}{2.663118in}}%
\pgfpathlineto{\pgfqpoint{2.002424in}{2.661471in}}%
\pgfpathlineto{\pgfqpoint{1.962343in}{2.652203in}}%
\pgfpathlineto{\pgfqpoint{1.938809in}{2.640588in}}%
\pgfpathlineto{\pgfqpoint{1.917427in}{2.623171in}}%
\pgfpathlineto{\pgfqpoint{1.914096in}{2.618667in}}%
\pgfpathlineto{\pgfqpoint{1.906397in}{2.603889in}}%
\pgfpathlineto{\pgfqpoint{1.892655in}{2.581333in}}%
\pgfpathlineto{\pgfqpoint{1.882895in}{2.543336in}}%
\pgfpathlineto{\pgfqpoint{1.881105in}{2.506667in}}%
\pgfpathlineto{\pgfqpoint{1.883931in}{2.469333in}}%
\pgfpathlineto{\pgfqpoint{1.899758in}{2.394667in}}%
\pgfpathlineto{\pgfqpoint{1.905518in}{2.379069in}}%
\pgfpathlineto{\pgfqpoint{1.910986in}{2.357333in}}%
\pgfpathlineto{\pgfqpoint{1.923918in}{2.320000in}}%
\pgfpathlineto{\pgfqpoint{1.935257in}{2.294770in}}%
\pgfpathlineto{\pgfqpoint{1.939276in}{2.282667in}}%
\pgfpathlineto{\pgfqpoint{1.957822in}{2.241122in}}%
\pgfpathlineto{\pgfqpoint{1.962343in}{2.230875in}}%
\pgfpathlineto{\pgfqpoint{1.983411in}{2.190290in}}%
\pgfpathlineto{\pgfqpoint{1.992337in}{2.170667in}}%
\pgfpathlineto{\pgfqpoint{2.022884in}{2.114276in}}%
\pgfpathlineto{\pgfqpoint{2.055713in}{2.058667in}}%
\pgfpathlineto{\pgfqpoint{2.066082in}{2.043294in}}%
\pgfpathlineto{\pgfqpoint{2.082586in}{2.014947in}}%
\pgfpathlineto{\pgfqpoint{2.102660in}{1.984000in}}%
\pgfpathlineto{\pgfqpoint{2.127135in}{1.946667in}}%
\pgfpathlineto{\pgfqpoint{2.141654in}{1.927020in}}%
\pgfpathlineto{\pgfqpoint{2.162747in}{1.895090in}}%
\pgfpathlineto{\pgfqpoint{2.179166in}{1.872000in}}%
\pgfpathlineto{\pgfqpoint{2.205939in}{1.834667in}}%
\pgfpathlineto{\pgfqpoint{2.221725in}{1.814934in}}%
\pgfpathlineto{\pgfqpoint{2.242909in}{1.785397in}}%
\pgfpathlineto{\pgfqpoint{2.323071in}{1.683072in}}%
\pgfpathlineto{\pgfqpoint{2.403232in}{1.587292in}}%
\pgfpathlineto{\pgfqpoint{2.483394in}{1.496045in}}%
\pgfpathlineto{\pgfqpoint{2.501125in}{1.477849in}}%
\pgfpathlineto{\pgfqpoint{2.523475in}{1.452395in}}%
\pgfpathlineto{\pgfqpoint{2.538150in}{1.437669in}}%
\pgfpathlineto{\pgfqpoint{2.563556in}{1.409696in}}%
\pgfpathlineto{\pgfqpoint{2.585610in}{1.386667in}}%
\pgfpathlineto{\pgfqpoint{2.658829in}{1.312000in}}%
\pgfpathlineto{\pgfqpoint{2.671310in}{1.300368in}}%
\pgfpathlineto{\pgfqpoint{2.696600in}{1.274667in}}%
\pgfpathlineto{\pgfqpoint{2.710387in}{1.262100in}}%
\pgfpathlineto{\pgfqpoint{2.735201in}{1.237333in}}%
\pgfpathlineto{\pgfqpoint{2.749898in}{1.224235in}}%
\pgfpathlineto{\pgfqpoint{2.774680in}{1.200000in}}%
\pgfpathlineto{\pgfqpoint{2.856472in}{1.125333in}}%
\pgfpathlineto{\pgfqpoint{2.871123in}{1.113151in}}%
\pgfpathlineto{\pgfqpoint{2.898896in}{1.088000in}}%
\pgfpathlineto{\pgfqpoint{2.964364in}{1.032148in}}%
\pgfpathlineto{\pgfqpoint{3.044525in}{0.966787in}}%
\pgfpathlineto{\pgfqpoint{3.061490in}{0.954469in}}%
\pgfpathlineto{\pgfqpoint{3.084606in}{0.935301in}}%
\pgfpathlineto{\pgfqpoint{3.129440in}{0.901333in}}%
\pgfpathlineto{\pgfqpoint{3.204848in}{0.846852in}}%
\pgfpathlineto{\pgfqpoint{3.289639in}{0.789333in}}%
\pgfpathlineto{\pgfqpoint{3.311340in}{0.776525in}}%
\pgfpathlineto{\pgfqpoint{3.325091in}{0.766907in}}%
\pgfpathlineto{\pgfqpoint{3.365172in}{0.742353in}}%
\pgfpathlineto{\pgfqpoint{3.384174in}{0.732367in}}%
\pgfpathlineto{\pgfqpoint{3.413055in}{0.714667in}}%
\pgfpathlineto{\pgfqpoint{3.488068in}{0.674861in}}%
\pgfpathlineto{\pgfqpoint{3.525495in}{0.657468in}}%
\pgfpathlineto{\pgfqpoint{3.538954in}{0.652536in}}%
\pgfpathlineto{\pgfqpoint{3.566150in}{0.639465in}}%
\pgfpathlineto{\pgfqpoint{3.605657in}{0.624708in}}%
\pgfpathlineto{\pgfqpoint{3.624046in}{0.619796in}}%
\pgfpathlineto{\pgfqpoint{3.645737in}{0.611454in}}%
\pgfpathlineto{\pgfqpoint{3.688612in}{0.600064in}}%
\pgfpathlineto{\pgfqpoint{3.725899in}{0.593874in}}%
\pgfpathlineto{\pgfqpoint{3.755356in}{0.592771in}}%
\pgfpathlineto{\pgfqpoint{3.765980in}{0.591129in}}%
\pgfpathlineto{\pgfqpoint{3.813877in}{0.595386in}}%
\pgfpathlineto{\pgfqpoint{3.846141in}{0.606516in}}%
\pgfpathlineto{\pgfqpoint{3.887624in}{0.640000in}}%
\pgfpathlineto{\pgfqpoint{3.901273in}{0.663314in}}%
\pgfpathlineto{\pgfqpoint{3.906034in}{0.677333in}}%
\pgfpathlineto{\pgfqpoint{3.909740in}{0.699239in}}%
\pgfpathlineto{\pgfqpoint{3.914151in}{0.714667in}}%
\pgfpathlineto{\pgfqpoint{3.914123in}{0.740655in}}%
\pgfpathlineto{\pgfqpoint{3.915505in}{0.752000in}}%
\pgfpathlineto{\pgfqpoint{3.912291in}{0.776282in}}%
\pgfpathlineto{\pgfqpoint{3.912178in}{0.789333in}}%
\pgfpathlineto{\pgfqpoint{3.910290in}{0.804248in}}%
\pgfpathlineto{\pgfqpoint{3.905477in}{0.826667in}}%
\pgfpathlineto{\pgfqpoint{3.901038in}{0.840467in}}%
\pgfpathlineto{\pgfqpoint{3.896263in}{0.864000in}}%
\pgfpathlineto{\pgfqpoint{3.884500in}{0.902938in}}%
\pgfpathlineto{\pgfqpoint{3.871079in}{0.938667in}}%
\pgfpathlineto{\pgfqpoint{3.863510in}{0.954845in}}%
\pgfpathlineto{\pgfqpoint{3.863510in}{0.954845in}}%
\pgfusepath{fill}%
\end{pgfscope}%
\begin{pgfscope}%
\pgfpathrectangle{\pgfqpoint{0.800000in}{0.528000in}}{\pgfqpoint{3.968000in}{3.696000in}}%
\pgfusepath{clip}%
\pgfsetbuttcap%
\pgfsetroundjoin%
\definecolor{currentfill}{rgb}{0.272594,0.025563,0.353093}%
\pgfsetfillcolor{currentfill}%
\pgfsetlinewidth{0.000000pt}%
\definecolor{currentstroke}{rgb}{0.000000,0.000000,0.000000}%
\pgfsetstrokecolor{currentstroke}%
\pgfsetdash{}{0pt}%
\pgfpathmoveto{\pgfqpoint{3.863510in}{0.954845in}}%
\pgfpathlineto{\pgfqpoint{3.871079in}{0.938667in}}%
\pgfpathlineto{\pgfqpoint{3.886222in}{0.897382in}}%
\pgfpathlineto{\pgfqpoint{3.896263in}{0.864000in}}%
\pgfpathlineto{\pgfqpoint{3.901038in}{0.840467in}}%
\pgfpathlineto{\pgfqpoint{3.905477in}{0.826667in}}%
\pgfpathlineto{\pgfqpoint{3.910290in}{0.804248in}}%
\pgfpathlineto{\pgfqpoint{3.912178in}{0.789333in}}%
\pgfpathlineto{\pgfqpoint{3.912291in}{0.776282in}}%
\pgfpathlineto{\pgfqpoint{3.915505in}{0.752000in}}%
\pgfpathlineto{\pgfqpoint{3.914123in}{0.740655in}}%
\pgfpathlineto{\pgfqpoint{3.914151in}{0.714667in}}%
\pgfpathlineto{\pgfqpoint{3.909740in}{0.699239in}}%
\pgfpathlineto{\pgfqpoint{3.906034in}{0.677333in}}%
\pgfpathlineto{\pgfqpoint{3.901273in}{0.663314in}}%
\pgfpathlineto{\pgfqpoint{3.886222in}{0.638215in}}%
\pgfpathlineto{\pgfqpoint{3.843578in}{0.605054in}}%
\pgfpathlineto{\pgfqpoint{3.813877in}{0.595386in}}%
\pgfpathlineto{\pgfqpoint{3.806061in}{0.594198in}}%
\pgfpathlineto{\pgfqpoint{3.797083in}{0.594304in}}%
\pgfpathlineto{\pgfqpoint{3.765980in}{0.591129in}}%
\pgfpathlineto{\pgfqpoint{3.755356in}{0.592771in}}%
\pgfpathlineto{\pgfqpoint{3.725899in}{0.593874in}}%
\pgfpathlineto{\pgfqpoint{3.678557in}{0.602667in}}%
\pgfpathlineto{\pgfqpoint{3.645737in}{0.611454in}}%
\pgfpathlineto{\pgfqpoint{3.624046in}{0.619796in}}%
\pgfpathlineto{\pgfqpoint{3.605657in}{0.624708in}}%
\pgfpathlineto{\pgfqpoint{3.564909in}{0.640000in}}%
\pgfpathlineto{\pgfqpoint{3.538954in}{0.652536in}}%
\pgfpathlineto{\pgfqpoint{3.525495in}{0.657468in}}%
\pgfpathlineto{\pgfqpoint{3.483214in}{0.677333in}}%
\pgfpathlineto{\pgfqpoint{3.405253in}{0.718926in}}%
\pgfpathlineto{\pgfqpoint{3.384174in}{0.732367in}}%
\pgfpathlineto{\pgfqpoint{3.359175in}{0.746414in}}%
\pgfpathlineto{\pgfqpoint{3.325091in}{0.766907in}}%
\pgfpathlineto{\pgfqpoint{3.311340in}{0.776525in}}%
\pgfpathlineto{\pgfqpoint{3.274377in}{0.799237in}}%
\pgfpathlineto{\pgfqpoint{3.204848in}{0.846852in}}%
\pgfpathlineto{\pgfqpoint{3.194799in}{0.854639in}}%
\pgfpathlineto{\pgfqpoint{3.164768in}{0.875438in}}%
\pgfpathlineto{\pgfqpoint{3.080304in}{0.938667in}}%
\pgfpathlineto{\pgfqpoint{3.061490in}{0.954469in}}%
\pgfpathlineto{\pgfqpoint{3.033049in}{0.976000in}}%
\pgfpathlineto{\pgfqpoint{3.018193in}{0.988806in}}%
\pgfpathlineto{\pgfqpoint{2.987114in}{1.013333in}}%
\pgfpathlineto{\pgfqpoint{2.924283in}{1.066045in}}%
\pgfpathlineto{\pgfqpoint{2.844121in}{1.136326in}}%
\pgfpathlineto{\pgfqpoint{2.830257in}{1.149752in}}%
\pgfpathlineto{\pgfqpoint{2.804040in}{1.172734in}}%
\pgfpathlineto{\pgfqpoint{2.723879in}{1.248145in}}%
\pgfpathlineto{\pgfqpoint{2.710387in}{1.262100in}}%
\pgfpathlineto{\pgfqpoint{2.683798in}{1.287172in}}%
\pgfpathlineto{\pgfqpoint{2.658829in}{1.312000in}}%
\pgfpathlineto{\pgfqpoint{2.585610in}{1.386667in}}%
\pgfpathlineto{\pgfqpoint{2.575598in}{1.397884in}}%
\pgfpathlineto{\pgfqpoint{2.550085in}{1.424000in}}%
\pgfpathlineto{\pgfqpoint{2.538150in}{1.437669in}}%
\pgfpathlineto{\pgfqpoint{2.515235in}{1.461333in}}%
\pgfpathlineto{\pgfqpoint{2.501125in}{1.477849in}}%
\pgfpathlineto{\pgfqpoint{2.468920in}{1.512149in}}%
\pgfpathlineto{\pgfqpoint{2.403232in}{1.587292in}}%
\pgfpathlineto{\pgfqpoint{2.383389in}{1.610667in}}%
\pgfpathlineto{\pgfqpoint{2.315621in}{1.692272in}}%
\pgfpathlineto{\pgfqpoint{2.242909in}{1.785397in}}%
\pgfpathlineto{\pgfqpoint{2.202828in}{1.838938in}}%
\pgfpathlineto{\pgfqpoint{2.152791in}{1.909333in}}%
\pgfpathlineto{\pgfqpoint{2.141654in}{1.927020in}}%
\pgfpathlineto{\pgfqpoint{2.122667in}{1.953384in}}%
\pgfpathlineto{\pgfqpoint{2.078515in}{2.021333in}}%
\pgfpathlineto{\pgfqpoint{2.066082in}{2.043294in}}%
\pgfpathlineto{\pgfqpoint{2.051361in}{2.066916in}}%
\pgfpathlineto{\pgfqpoint{2.033534in}{2.096000in}}%
\pgfpathlineto{\pgfqpoint{2.012428in}{2.133333in}}%
\pgfpathlineto{\pgfqpoint{2.009387in}{2.139818in}}%
\pgfpathlineto{\pgfqpoint{1.992337in}{2.170667in}}%
\pgfpathlineto{\pgfqpoint{1.983411in}{2.190290in}}%
\pgfpathlineto{\pgfqpoint{1.970362in}{2.215469in}}%
\pgfpathlineto{\pgfqpoint{1.955507in}{2.245333in}}%
\pgfpathlineto{\pgfqpoint{1.939276in}{2.282667in}}%
\pgfpathlineto{\pgfqpoint{1.935257in}{2.294770in}}%
\pgfpathlineto{\pgfqpoint{1.922263in}{2.324683in}}%
\pgfpathlineto{\pgfqpoint{1.910986in}{2.357333in}}%
\pgfpathlineto{\pgfqpoint{1.905518in}{2.379069in}}%
\pgfpathlineto{\pgfqpoint{1.899758in}{2.394667in}}%
\pgfpathlineto{\pgfqpoint{1.890524in}{2.432000in}}%
\pgfpathlineto{\pgfqpoint{1.889969in}{2.439253in}}%
\pgfpathlineto{\pgfqpoint{1.883998in}{2.471025in}}%
\pgfpathlineto{\pgfqpoint{1.881105in}{2.506667in}}%
\pgfpathlineto{\pgfqpoint{1.883476in}{2.545206in}}%
\pgfpathlineto{\pgfqpoint{1.892655in}{2.581333in}}%
\pgfpathlineto{\pgfqpoint{1.906397in}{2.603889in}}%
\pgfpathlineto{\pgfqpoint{1.917427in}{2.623171in}}%
\pgfpathlineto{\pgfqpoint{1.938809in}{2.640588in}}%
\pgfpathlineto{\pgfqpoint{1.962343in}{2.652203in}}%
\pgfpathlineto{\pgfqpoint{1.975831in}{2.656000in}}%
\pgfpathlineto{\pgfqpoint{2.002424in}{2.661471in}}%
\pgfpathlineto{\pgfqpoint{2.042505in}{2.662643in}}%
\pgfpathlineto{\pgfqpoint{2.048379in}{2.661472in}}%
\pgfpathlineto{\pgfqpoint{2.082586in}{2.658642in}}%
\pgfpathlineto{\pgfqpoint{2.130763in}{2.648458in}}%
\pgfpathlineto{\pgfqpoint{2.162747in}{2.638864in}}%
\pgfpathlineto{\pgfqpoint{2.177779in}{2.632668in}}%
\pgfpathlineto{\pgfqpoint{2.202828in}{2.625238in}}%
\pgfpathlineto{\pgfqpoint{2.242909in}{2.609180in}}%
\pgfpathlineto{\pgfqpoint{2.262297in}{2.599392in}}%
\pgfpathlineto{\pgfqpoint{2.289683in}{2.587567in}}%
\pgfpathlineto{\pgfqpoint{2.323071in}{2.571803in}}%
\pgfpathlineto{\pgfqpoint{2.375387in}{2.544000in}}%
\pgfpathlineto{\pgfqpoint{2.416917in}{2.519413in}}%
\pgfpathlineto{\pgfqpoint{2.443313in}{2.504986in}}%
\pgfpathlineto{\pgfqpoint{2.465511in}{2.490010in}}%
\pgfpathlineto{\pgfqpoint{2.483394in}{2.479948in}}%
\pgfpathlineto{\pgfqpoint{2.563556in}{2.427216in}}%
\pgfpathlineto{\pgfqpoint{2.582860in}{2.412648in}}%
\pgfpathlineto{\pgfqpoint{2.610017in}{2.394667in}}%
\pgfpathlineto{\pgfqpoint{2.683798in}{2.340365in}}%
\pgfpathlineto{\pgfqpoint{2.695389in}{2.330797in}}%
\pgfpathlineto{\pgfqpoint{2.723879in}{2.309710in}}%
\pgfpathlineto{\pgfqpoint{2.763960in}{2.278240in}}%
\pgfpathlineto{\pgfqpoint{2.782243in}{2.262364in}}%
\pgfpathlineto{\pgfqpoint{2.804726in}{2.245333in}}%
\pgfpathlineto{\pgfqpoint{2.824871in}{2.227403in}}%
\pgfpathlineto{\pgfqpoint{2.849393in}{2.208000in}}%
\pgfpathlineto{\pgfqpoint{2.866981in}{2.191960in}}%
\pgfpathlineto{\pgfqpoint{2.892931in}{2.170667in}}%
\pgfpathlineto{\pgfqpoint{2.908584in}{2.156044in}}%
\pgfpathlineto{\pgfqpoint{2.935404in}{2.133333in}}%
\pgfpathlineto{\pgfqpoint{2.949692in}{2.119667in}}%
\pgfpathlineto{\pgfqpoint{2.976870in}{2.096000in}}%
\pgfpathlineto{\pgfqpoint{2.990314in}{2.082838in}}%
\pgfpathlineto{\pgfqpoint{3.017383in}{2.058667in}}%
\pgfpathlineto{\pgfqpoint{3.095751in}{1.984000in}}%
\pgfpathlineto{\pgfqpoint{3.170869in}{1.909333in}}%
\pgfpathlineto{\pgfqpoint{3.186485in}{1.892229in}}%
\pgfpathlineto{\pgfqpoint{3.207311in}{1.872000in}}%
\pgfpathlineto{\pgfqpoint{3.244929in}{1.832524in}}%
\pgfpathlineto{\pgfqpoint{3.261889in}{1.813130in}}%
\pgfpathlineto{\pgfqpoint{3.285010in}{1.789305in}}%
\pgfpathlineto{\pgfqpoint{3.365172in}{1.699849in}}%
\pgfpathlineto{\pgfqpoint{3.441378in}{1.610667in}}%
\pgfpathlineto{\pgfqpoint{3.445333in}{1.605890in}}%
\pgfpathlineto{\pgfqpoint{3.531248in}{1.498667in}}%
\pgfpathlineto{\pgfqpoint{3.544937in}{1.479443in}}%
\pgfpathlineto{\pgfqpoint{3.565576in}{1.453679in}}%
\pgfpathlineto{\pgfqpoint{3.587494in}{1.424000in}}%
\pgfpathlineto{\pgfqpoint{3.645737in}{1.342819in}}%
\pgfpathlineto{\pgfqpoint{3.700826in}{1.260688in}}%
\pgfpathlineto{\pgfqpoint{3.755264in}{1.172648in}}%
\pgfpathlineto{\pgfqpoint{3.799300in}{1.094297in}}%
\pgfpathlineto{\pgfqpoint{3.821409in}{1.050667in}}%
\pgfpathlineto{\pgfqpoint{3.828386in}{1.034128in}}%
\pgfpathlineto{\pgfqpoint{3.839516in}{1.013333in}}%
\pgfpathlineto{\pgfqpoint{3.856034in}{0.976000in}}%
\pgfpathmoveto{\pgfqpoint{2.723879in}{2.328046in}}%
\pgfpathlineto{\pgfqpoint{2.706102in}{2.340775in}}%
\pgfpathlineto{\pgfqpoint{2.683798in}{2.359064in}}%
\pgfpathlineto{\pgfqpoint{2.661931in}{2.374299in}}%
\pgfpathlineto{\pgfqpoint{2.635838in}{2.394667in}}%
\pgfpathlineto{\pgfqpoint{2.563556in}{2.446112in}}%
\pgfpathlineto{\pgfqpoint{2.548623in}{2.455424in}}%
\pgfpathlineto{\pgfqpoint{2.523475in}{2.473532in}}%
\pgfpathlineto{\pgfqpoint{2.472310in}{2.506667in}}%
\pgfpathlineto{\pgfqpoint{2.403232in}{2.549058in}}%
\pgfpathlineto{\pgfqpoint{2.381013in}{2.560637in}}%
\pgfpathlineto{\pgfqpoint{2.363152in}{2.571869in}}%
\pgfpathlineto{\pgfqpoint{2.323071in}{2.593466in}}%
\pgfpathlineto{\pgfqpoint{2.304824in}{2.601670in}}%
\pgfpathlineto{\pgfqpoint{2.272708in}{2.618667in}}%
\pgfpathlineto{\pgfqpoint{2.242909in}{2.632435in}}%
\pgfpathlineto{\pgfqpoint{2.202828in}{2.649619in}}%
\pgfpathlineto{\pgfqpoint{2.197633in}{2.651160in}}%
\pgfpathlineto{\pgfqpoint{2.162747in}{2.664667in}}%
\pgfpathlineto{\pgfqpoint{2.138937in}{2.671155in}}%
\pgfpathlineto{\pgfqpoint{2.122667in}{2.677420in}}%
\pgfpathlineto{\pgfqpoint{2.082586in}{2.687875in}}%
\pgfpathlineto{\pgfqpoint{2.077204in}{2.688320in}}%
\pgfpathlineto{\pgfqpoint{2.042505in}{2.695029in}}%
\pgfpathlineto{\pgfqpoint{1.997574in}{2.697851in}}%
\pgfpathlineto{\pgfqpoint{1.955769in}{2.693333in}}%
\pgfpathlineto{\pgfqpoint{1.929910in}{2.686210in}}%
\pgfpathlineto{\pgfqpoint{1.904182in}{2.672842in}}%
\pgfpathlineto{\pgfqpoint{1.882182in}{2.653728in}}%
\pgfpathlineto{\pgfqpoint{1.867298in}{2.632530in}}%
\pgfpathlineto{\pgfqpoint{1.861328in}{2.618667in}}%
\pgfpathlineto{\pgfqpoint{1.854837in}{2.593197in}}%
\pgfpathlineto{\pgfqpoint{1.849143in}{2.574774in}}%
\pgfpathlineto{\pgfqpoint{1.847280in}{2.544000in}}%
\pgfpathlineto{\pgfqpoint{1.849743in}{2.499549in}}%
\pgfpathlineto{\pgfqpoint{1.854853in}{2.469333in}}%
\pgfpathlineto{\pgfqpoint{1.860477in}{2.449117in}}%
\pgfpathlineto{\pgfqpoint{1.863234in}{2.432000in}}%
\pgfpathlineto{\pgfqpoint{1.873701in}{2.394667in}}%
\pgfpathlineto{\pgfqpoint{1.888061in}{2.351857in}}%
\pgfpathlineto{\pgfqpoint{1.918355in}{2.279027in}}%
\pgfpathlineto{\pgfqpoint{1.933744in}{2.245333in}}%
\pgfpathlineto{\pgfqpoint{1.943428in}{2.227714in}}%
\pgfpathlineto{\pgfqpoint{1.952127in}{2.208000in}}%
\pgfpathlineto{\pgfqpoint{1.980806in}{2.153470in}}%
\pgfpathlineto{\pgfqpoint{2.013803in}{2.096000in}}%
\pgfpathlineto{\pgfqpoint{2.024702in}{2.079418in}}%
\pgfpathlineto{\pgfqpoint{2.042505in}{2.048454in}}%
\pgfpathlineto{\pgfqpoint{2.059664in}{2.021333in}}%
\pgfpathlineto{\pgfqpoint{2.084860in}{1.981881in}}%
\pgfpathlineto{\pgfqpoint{2.134440in}{1.909333in}}%
\pgfpathlineto{\pgfqpoint{2.146207in}{1.893927in}}%
\pgfpathlineto{\pgfqpoint{2.162747in}{1.869123in}}%
\pgfpathlineto{\pgfqpoint{2.244292in}{1.760000in}}%
\pgfpathlineto{\pgfqpoint{2.261266in}{1.739765in}}%
\pgfpathlineto{\pgfqpoint{2.282990in}{1.711076in}}%
\pgfpathlineto{\pgfqpoint{2.365384in}{1.610667in}}%
\pgfpathlineto{\pgfqpoint{2.382773in}{1.591610in}}%
\pgfpathlineto{\pgfqpoint{2.403232in}{1.566621in}}%
\pgfpathlineto{\pgfqpoint{2.483394in}{1.476601in}}%
\pgfpathlineto{\pgfqpoint{2.509874in}{1.448665in}}%
\pgfpathlineto{\pgfqpoint{2.531951in}{1.424000in}}%
\pgfpathlineto{\pgfqpoint{2.603636in}{1.348808in}}%
\pgfpathlineto{\pgfqpoint{2.623174in}{1.330198in}}%
\pgfpathlineto{\pgfqpoint{2.643717in}{1.308292in}}%
\pgfpathlineto{\pgfqpoint{2.661763in}{1.291475in}}%
\pgfpathlineto{\pgfqpoint{2.683798in}{1.268623in}}%
\pgfpathlineto{\pgfqpoint{2.700776in}{1.253148in}}%
\pgfpathlineto{\pgfqpoint{2.723879in}{1.229790in}}%
\pgfpathlineto{\pgfqpoint{2.740223in}{1.215224in}}%
\pgfpathlineto{\pgfqpoint{2.763960in}{1.191783in}}%
\pgfpathlineto{\pgfqpoint{2.844121in}{1.118199in}}%
\pgfpathlineto{\pgfqpoint{2.861251in}{1.103956in}}%
\pgfpathlineto{\pgfqpoint{2.884202in}{1.082603in}}%
\pgfpathlineto{\pgfqpoint{2.924283in}{1.047789in}}%
\pgfpathlineto{\pgfqpoint{2.944269in}{1.031949in}}%
\pgfpathlineto{\pgfqpoint{2.964900in}{1.013333in}}%
\pgfpathlineto{\pgfqpoint{3.057142in}{0.938667in}}%
\pgfpathlineto{\pgfqpoint{3.084606in}{0.917179in}}%
\pgfpathlineto{\pgfqpoint{3.164768in}{0.856703in}}%
\pgfpathlineto{\pgfqpoint{3.183561in}{0.844172in}}%
\pgfpathlineto{\pgfqpoint{3.206305in}{0.826667in}}%
\pgfpathlineto{\pgfqpoint{3.229093in}{0.811916in}}%
\pgfpathlineto{\pgfqpoint{3.260745in}{0.789333in}}%
\pgfpathlineto{\pgfqpoint{3.325091in}{0.746940in}}%
\pgfpathlineto{\pgfqpoint{3.346586in}{0.734688in}}%
\pgfpathlineto{\pgfqpoint{3.377726in}{0.714667in}}%
\pgfpathlineto{\pgfqpoint{3.405253in}{0.698451in}}%
\pgfpathlineto{\pgfqpoint{3.445333in}{0.675592in}}%
\pgfpathlineto{\pgfqpoint{3.470491in}{0.663433in}}%
\pgfpathlineto{\pgfqpoint{3.485414in}{0.654613in}}%
\pgfpathlineto{\pgfqpoint{3.537417in}{0.628895in}}%
\pgfpathlineto{\pgfqpoint{3.611192in}{0.597511in}}%
\pgfpathlineto{\pgfqpoint{3.645737in}{0.585487in}}%
\pgfpathlineto{\pgfqpoint{3.662908in}{0.581327in}}%
\pgfpathlineto{\pgfqpoint{3.685818in}{0.573158in}}%
\pgfpathlineto{\pgfqpoint{3.725899in}{0.563532in}}%
\pgfpathlineto{\pgfqpoint{3.775072in}{0.556864in}}%
\pgfpathlineto{\pgfqpoint{3.806061in}{0.556543in}}%
\pgfpathlineto{\pgfqpoint{3.849679in}{0.562038in}}%
\pgfpathlineto{\pgfqpoint{3.858780in}{0.565333in}}%
\pgfpathlineto{\pgfqpoint{3.886222in}{0.577287in}}%
\pgfpathlineto{\pgfqpoint{3.902153in}{0.587828in}}%
\pgfpathlineto{\pgfqpoint{3.921106in}{0.607508in}}%
\pgfpathlineto{\pgfqpoint{3.935177in}{0.631734in}}%
\pgfpathlineto{\pgfqpoint{3.944582in}{0.660307in}}%
\pgfpathlineto{\pgfqpoint{3.947165in}{0.677333in}}%
\pgfpathlineto{\pgfqpoint{3.947331in}{0.696920in}}%
\pgfpathlineto{\pgfqpoint{3.949575in}{0.714667in}}%
\pgfpathlineto{\pgfqpoint{3.949561in}{0.730336in}}%
\pgfpathlineto{\pgfqpoint{3.947224in}{0.752000in}}%
\pgfpathlineto{\pgfqpoint{3.943632in}{0.768141in}}%
\pgfpathlineto{\pgfqpoint{3.941407in}{0.789333in}}%
\pgfpathlineto{\pgfqpoint{3.932989in}{0.826667in}}%
\pgfpathlineto{\pgfqpoint{3.920549in}{0.869359in}}%
\pgfpathlineto{\pgfqpoint{3.908994in}{0.901333in}}%
\pgfpathlineto{\pgfqpoint{3.902251in}{0.916264in}}%
\pgfpathlineto{\pgfqpoint{3.894620in}{0.938667in}}%
\pgfpathlineto{\pgfqpoint{3.878625in}{0.976000in}}%
\pgfpathlineto{\pgfqpoint{3.868047in}{0.996404in}}%
\pgfpathlineto{\pgfqpoint{3.861132in}{1.013333in}}%
\pgfpathlineto{\pgfqpoint{3.839956in}{1.056428in}}%
\pgfpathlineto{\pgfqpoint{3.798524in}{1.132353in}}%
\pgfpathlineto{\pgfqpoint{3.758087in}{1.200000in}}%
\pgfpathlineto{\pgfqpoint{3.745581in}{1.218333in}}%
\pgfpathlineto{\pgfqpoint{3.725899in}{1.250850in}}%
\pgfpathlineto{\pgfqpoint{3.685116in}{1.312654in}}%
\pgfpathlineto{\pgfqpoint{3.632863in}{1.386667in}}%
\pgfpathlineto{\pgfqpoint{3.621367in}{1.401300in}}%
\pgfpathlineto{\pgfqpoint{3.605657in}{1.424335in}}%
\pgfpathlineto{\pgfqpoint{3.588653in}{1.445496in}}%
\pgfpathlineto{\pgfqpoint{3.565576in}{1.477146in}}%
\pgfpathlineto{\pgfqpoint{3.548910in}{1.498667in}}%
\pgfpathlineto{\pgfqpoint{3.519741in}{1.536000in}}%
\pgfpathlineto{\pgfqpoint{3.504478in}{1.553757in}}%
\pgfpathlineto{\pgfqpoint{3.485414in}{1.578660in}}%
\pgfpathlineto{\pgfqpoint{3.395577in}{1.685333in}}%
\pgfpathlineto{\pgfqpoint{3.363088in}{1.722667in}}%
\pgfpathlineto{\pgfqpoint{3.345229in}{1.741424in}}%
\pgfpathlineto{\pgfqpoint{3.325091in}{1.765053in}}%
\pgfpathlineto{\pgfqpoint{3.244929in}{1.851400in}}%
\pgfpathlineto{\pgfqpoint{3.164768in}{1.934099in}}%
\pgfpathlineto{\pgfqpoint{3.114595in}{1.984000in}}%
\pgfpathlineto{\pgfqpoint{3.099475in}{1.997850in}}%
\pgfpathlineto{\pgfqpoint{3.076172in}{2.021333in}}%
\pgfpathlineto{\pgfqpoint{3.060043in}{2.035788in}}%
\pgfpathlineto{\pgfqpoint{3.036906in}{2.058667in}}%
\pgfpathlineto{\pgfqpoint{3.020183in}{2.073326in}}%
\pgfpathlineto{\pgfqpoint{2.996750in}{2.096000in}}%
\pgfpathlineto{\pgfqpoint{2.913566in}{2.170667in}}%
\pgfpathlineto{\pgfqpoint{2.844121in}{2.230369in}}%
\pgfpathlineto{\pgfqpoint{2.826176in}{2.245333in}}%
\pgfpathlineto{\pgfqpoint{2.763960in}{2.296265in}}%
\pgfpathlineto{\pgfqpoint{2.728400in}{2.324211in}}%
\pgfpathlineto{\pgfqpoint{2.728400in}{2.324211in}}%
\pgfusepath{fill}%
\end{pgfscope}%
\begin{pgfscope}%
\pgfpathrectangle{\pgfqpoint{0.800000in}{0.528000in}}{\pgfqpoint{3.968000in}{3.696000in}}%
\pgfusepath{clip}%
\pgfsetbuttcap%
\pgfsetroundjoin%
\definecolor{currentfill}{rgb}{0.272594,0.025563,0.353093}%
\pgfsetfillcolor{currentfill}%
\pgfsetlinewidth{0.000000pt}%
\definecolor{currentstroke}{rgb}{0.000000,0.000000,0.000000}%
\pgfsetstrokecolor{currentstroke}%
\pgfsetdash{}{0pt}%
\pgfpathmoveto{\pgfqpoint{3.874053in}{0.528000in}}%
\pgfpathlineto{\pgfqpoint{3.886222in}{0.530621in}}%
\pgfpathlineto{\pgfqpoint{3.912231in}{0.541108in}}%
\pgfpathlineto{\pgfqpoint{3.935512in}{0.556756in}}%
\pgfpathlineto{\pgfqpoint{3.943473in}{0.565333in}}%
\pgfpathlineto{\pgfqpoint{3.968589in}{0.602667in}}%
\pgfpathlineto{\pgfqpoint{3.977369in}{0.629768in}}%
\pgfpathlineto{\pgfqpoint{3.978962in}{0.640000in}}%
\pgfpathlineto{\pgfqpoint{3.979169in}{0.651909in}}%
\pgfpathlineto{\pgfqpoint{3.982495in}{0.677333in}}%
\pgfpathlineto{\pgfqpoint{3.982657in}{0.699509in}}%
\pgfpathlineto{\pgfqpoint{3.981178in}{0.714667in}}%
\pgfpathlineto{\pgfqpoint{3.978767in}{0.726201in}}%
\pgfpathlineto{\pgfqpoint{3.976298in}{0.752000in}}%
\pgfpathlineto{\pgfqpoint{3.966384in}{0.797853in}}%
\pgfpathlineto{\pgfqpoint{3.955322in}{0.836970in}}%
\pgfpathlineto{\pgfqpoint{3.932262in}{0.901333in}}%
\pgfpathlineto{\pgfqpoint{3.926303in}{0.915854in}}%
\pgfpathlineto{\pgfqpoint{3.878779in}{1.020266in}}%
\pgfpathlineto{\pgfqpoint{3.842925in}{1.088000in}}%
\pgfpathlineto{\pgfqpoint{3.829483in}{1.109816in}}%
\pgfpathlineto{\pgfqpoint{3.815708in}{1.134319in}}%
\pgfpathlineto{\pgfqpoint{3.799672in}{1.162667in}}%
\pgfpathlineto{\pgfqpoint{3.786781in}{1.182042in}}%
\pgfpathlineto{\pgfqpoint{3.765980in}{1.216970in}}%
\pgfpathlineto{\pgfqpoint{3.742164in}{1.252483in}}%
\pgfpathlineto{\pgfqpoint{3.725899in}{1.278751in}}%
\pgfpathlineto{\pgfqpoint{3.677234in}{1.349333in}}%
\pgfpathlineto{\pgfqpoint{3.664097in}{1.366434in}}%
\pgfpathlineto{\pgfqpoint{3.645737in}{1.393350in}}%
\pgfpathlineto{\pgfqpoint{3.623012in}{1.424000in}}%
\pgfpathlineto{\pgfqpoint{3.565576in}{1.499851in}}%
\pgfpathlineto{\pgfqpoint{3.476109in}{1.610667in}}%
\pgfpathlineto{\pgfqpoint{3.445000in}{1.648000in}}%
\pgfpathlineto{\pgfqpoint{3.426736in}{1.668011in}}%
\pgfpathlineto{\pgfqpoint{3.405253in}{1.694035in}}%
\pgfpathlineto{\pgfqpoint{3.312817in}{1.797333in}}%
\pgfpathlineto{\pgfqpoint{3.299527in}{1.810855in}}%
\pgfpathlineto{\pgfqpoint{3.278299in}{1.834667in}}%
\pgfpathlineto{\pgfqpoint{3.204848in}{1.911672in}}%
\pgfpathlineto{\pgfqpoint{3.124687in}{1.992018in}}%
\pgfpathlineto{\pgfqpoint{3.108818in}{2.006552in}}%
\pgfpathlineto{\pgfqpoint{3.084606in}{2.030962in}}%
\pgfpathlineto{\pgfqpoint{3.069447in}{2.044547in}}%
\pgfpathlineto{\pgfqpoint{3.044525in}{2.069100in}}%
\pgfpathlineto{\pgfqpoint{3.029647in}{2.082142in}}%
\pgfpathlineto{\pgfqpoint{3.004444in}{2.106444in}}%
\pgfpathlineto{\pgfqpoint{2.924283in}{2.178788in}}%
\pgfpathlineto{\pgfqpoint{2.844121in}{2.248071in}}%
\pgfpathlineto{\pgfqpoint{2.802627in}{2.282667in}}%
\pgfpathlineto{\pgfqpoint{2.781364in}{2.298878in}}%
\pgfpathlineto{\pgfqpoint{2.756392in}{2.320000in}}%
\pgfpathlineto{\pgfqpoint{2.738317in}{2.333448in}}%
\pgfpathlineto{\pgfqpoint{2.708845in}{2.357333in}}%
\pgfpathlineto{\pgfqpoint{2.672724in}{2.384352in}}%
\pgfpathlineto{\pgfqpoint{2.643717in}{2.406823in}}%
\pgfpathlineto{\pgfqpoint{2.563556in}{2.464672in}}%
\pgfpathlineto{\pgfqpoint{2.556746in}{2.469333in}}%
\pgfpathlineto{\pgfqpoint{2.483394in}{2.518583in}}%
\pgfpathlineto{\pgfqpoint{2.441973in}{2.545249in}}%
\pgfpathlineto{\pgfqpoint{2.363152in}{2.592040in}}%
\pgfpathlineto{\pgfqpoint{2.314795in}{2.618667in}}%
\pgfpathlineto{\pgfqpoint{2.267593in}{2.641658in}}%
\pgfpathlineto{\pgfqpoint{2.240382in}{2.656000in}}%
\pgfpathlineto{\pgfqpoint{2.214245in}{2.666634in}}%
\pgfpathlineto{\pgfqpoint{2.202828in}{2.672482in}}%
\pgfpathlineto{\pgfqpoint{2.150158in}{2.693333in}}%
\pgfpathlineto{\pgfqpoint{2.122667in}{2.702956in}}%
\pgfpathlineto{\pgfqpoint{2.099186in}{2.708795in}}%
\pgfpathlineto{\pgfqpoint{2.082586in}{2.714743in}}%
\pgfpathlineto{\pgfqpoint{2.042505in}{2.724050in}}%
\pgfpathlineto{\pgfqpoint{2.035800in}{2.724421in}}%
\pgfpathlineto{\pgfqpoint{2.002424in}{2.730027in}}%
\pgfpathlineto{\pgfqpoint{1.956187in}{2.730667in}}%
\pgfpathlineto{\pgfqpoint{1.922263in}{2.726091in}}%
\pgfpathlineto{\pgfqpoint{1.912782in}{2.721836in}}%
\pgfpathlineto{\pgfqpoint{1.882182in}{2.710663in}}%
\pgfpathlineto{\pgfqpoint{1.870655in}{2.704070in}}%
\pgfpathlineto{\pgfqpoint{1.858556in}{2.693333in}}%
\pgfpathlineto{\pgfqpoint{1.842101in}{2.674497in}}%
\pgfpathlineto{\pgfqpoint{1.834390in}{2.663182in}}%
\pgfpathlineto{\pgfqpoint{1.831345in}{2.656000in}}%
\pgfpathlineto{\pgfqpoint{1.827985in}{2.642851in}}%
\pgfpathlineto{\pgfqpoint{1.819275in}{2.618667in}}%
\pgfpathlineto{\pgfqpoint{1.816417in}{2.605257in}}%
\pgfpathlineto{\pgfqpoint{1.814809in}{2.581333in}}%
\pgfpathlineto{\pgfqpoint{1.816324in}{2.557324in}}%
\pgfpathlineto{\pgfqpoint{1.815623in}{2.544000in}}%
\pgfpathlineto{\pgfqpoint{1.819312in}{2.522773in}}%
\pgfpathlineto{\pgfqpoint{1.820254in}{2.506667in}}%
\pgfpathlineto{\pgfqpoint{1.823559in}{2.486604in}}%
\pgfpathlineto{\pgfqpoint{1.837392in}{2.432000in}}%
\pgfpathlineto{\pgfqpoint{1.838675in}{2.428809in}}%
\pgfpathlineto{\pgfqpoint{1.852719in}{2.384777in}}%
\pgfpathlineto{\pgfqpoint{1.882182in}{2.311143in}}%
\pgfpathlineto{\pgfqpoint{1.895077in}{2.282667in}}%
\pgfpathlineto{\pgfqpoint{1.904102in}{2.265751in}}%
\pgfpathlineto{\pgfqpoint{1.912843in}{2.245333in}}%
\pgfpathlineto{\pgfqpoint{1.940723in}{2.190805in}}%
\pgfpathlineto{\pgfqpoint{1.972854in}{2.133333in}}%
\pgfpathlineto{\pgfqpoint{1.983905in}{2.116084in}}%
\pgfpathlineto{\pgfqpoint{2.002424in}{2.083292in}}%
\pgfpathlineto{\pgfqpoint{2.027335in}{2.044537in}}%
\pgfpathlineto{\pgfqpoint{2.042505in}{2.018929in}}%
\pgfpathlineto{\pgfqpoint{2.090703in}{1.946667in}}%
\pgfpathlineto{\pgfqpoint{2.103814in}{1.929106in}}%
\pgfpathlineto{\pgfqpoint{2.122667in}{1.900721in}}%
\pgfpathlineto{\pgfqpoint{2.151550in}{1.861570in}}%
\pgfpathlineto{\pgfqpoint{2.170534in}{1.834667in}}%
\pgfpathlineto{\pgfqpoint{2.202828in}{1.791588in}}%
\pgfpathlineto{\pgfqpoint{2.256614in}{1.722667in}}%
\pgfpathlineto{\pgfqpoint{2.268317in}{1.709000in}}%
\pgfpathlineto{\pgfqpoint{2.286378in}{1.685333in}}%
\pgfpathlineto{\pgfqpoint{2.302849in}{1.666498in}}%
\pgfpathlineto{\pgfqpoint{2.323071in}{1.640784in}}%
\pgfpathlineto{\pgfqpoint{2.412894in}{1.536000in}}%
\pgfpathlineto{\pgfqpoint{2.427133in}{1.520929in}}%
\pgfpathlineto{\pgfqpoint{2.445932in}{1.498667in}}%
\pgfpathlineto{\pgfqpoint{2.463659in}{1.480285in}}%
\pgfpathlineto{\pgfqpoint{2.483394in}{1.457443in}}%
\pgfpathlineto{\pgfqpoint{2.500571in}{1.439999in}}%
\pgfpathlineto{\pgfqpoint{2.523475in}{1.414397in}}%
\pgfpathlineto{\pgfqpoint{2.603636in}{1.330938in}}%
\pgfpathlineto{\pgfqpoint{2.622363in}{1.312000in}}%
\pgfpathlineto{\pgfqpoint{2.697800in}{1.237333in}}%
\pgfpathlineto{\pgfqpoint{2.776405in}{1.162667in}}%
\pgfpathlineto{\pgfqpoint{2.858520in}{1.088000in}}%
\pgfpathlineto{\pgfqpoint{2.924283in}{1.030536in}}%
\pgfpathlineto{\pgfqpoint{2.944539in}{1.013333in}}%
\pgfpathlineto{\pgfqpoint{3.004444in}{0.963364in}}%
\pgfpathlineto{\pgfqpoint{3.084606in}{0.899203in}}%
\pgfpathlineto{\pgfqpoint{3.105966in}{0.883896in}}%
\pgfpathlineto{\pgfqpoint{3.130743in}{0.864000in}}%
\pgfpathlineto{\pgfqpoint{3.204848in}{0.809731in}}%
\pgfpathlineto{\pgfqpoint{3.217769in}{0.801368in}}%
\pgfpathlineto{\pgfqpoint{3.244929in}{0.781467in}}%
\pgfpathlineto{\pgfqpoint{3.288200in}{0.752000in}}%
\pgfpathlineto{\pgfqpoint{3.365172in}{0.702758in}}%
\pgfpathlineto{\pgfqpoint{3.382334in}{0.693319in}}%
\pgfpathlineto{\pgfqpoint{3.406947in}{0.677333in}}%
\pgfpathlineto{\pgfqpoint{3.431410in}{0.664364in}}%
\pgfpathlineto{\pgfqpoint{3.445333in}{0.655542in}}%
\pgfpathlineto{\pgfqpoint{3.456319in}{0.650233in}}%
\pgfpathlineto{\pgfqpoint{3.485414in}{0.633561in}}%
\pgfpathlineto{\pgfqpoint{3.507577in}{0.623310in}}%
\pgfpathlineto{\pgfqpoint{3.525495in}{0.613131in}}%
\pgfpathlineto{\pgfqpoint{3.565576in}{0.594027in}}%
\pgfpathlineto{\pgfqpoint{3.587364in}{0.585628in}}%
\pgfpathlineto{\pgfqpoint{3.605657in}{0.576616in}}%
\pgfpathlineto{\pgfqpoint{3.645737in}{0.560787in}}%
\pgfpathlineto{\pgfqpoint{3.672433in}{0.552866in}}%
\pgfpathlineto{\pgfqpoint{3.685818in}{0.547445in}}%
\pgfpathlineto{\pgfqpoint{3.725899in}{0.536143in}}%
\pgfpathlineto{\pgfqpoint{3.733617in}{0.535189in}}%
\pgfpathlineto{\pgfqpoint{3.765980in}{0.528000in}}%
\pgfpathlineto{\pgfqpoint{3.846141in}{0.528000in}}%
\pgfpathmoveto{\pgfqpoint{2.728400in}{2.324211in}}%
\pgfpathlineto{\pgfqpoint{2.734060in}{2.320000in}}%
\pgfpathlineto{\pgfqpoint{2.804040in}{2.263709in}}%
\pgfpathlineto{\pgfqpoint{2.884202in}{2.196234in}}%
\pgfpathlineto{\pgfqpoint{2.964364in}{2.125536in}}%
\pgfpathlineto{\pgfqpoint{3.044525in}{2.051526in}}%
\pgfpathlineto{\pgfqpoint{3.060043in}{2.035788in}}%
\pgfpathlineto{\pgfqpoint{3.084606in}{2.013250in}}%
\pgfpathlineto{\pgfqpoint{3.099475in}{1.997850in}}%
\pgfpathlineto{\pgfqpoint{3.124687in}{1.974112in}}%
\pgfpathlineto{\pgfqpoint{3.138485in}{1.959519in}}%
\pgfpathlineto{\pgfqpoint{3.164768in}{1.934099in}}%
\pgfpathlineto{\pgfqpoint{3.244929in}{1.851400in}}%
\pgfpathlineto{\pgfqpoint{3.329653in}{1.760000in}}%
\pgfpathlineto{\pgfqpoint{3.345229in}{1.741424in}}%
\pgfpathlineto{\pgfqpoint{3.365172in}{1.720309in}}%
\pgfpathlineto{\pgfqpoint{3.445333in}{1.626935in}}%
\pgfpathlineto{\pgfqpoint{3.489760in}{1.573333in}}%
\pgfpathlineto{\pgfqpoint{3.504478in}{1.553757in}}%
\pgfpathlineto{\pgfqpoint{3.525495in}{1.528732in}}%
\pgfpathlineto{\pgfqpoint{3.577625in}{1.461333in}}%
\pgfpathlineto{\pgfqpoint{3.588653in}{1.445496in}}%
\pgfpathlineto{\pgfqpoint{3.605907in}{1.424000in}}%
\pgfpathlineto{\pgfqpoint{3.621367in}{1.401300in}}%
\pgfpathlineto{\pgfqpoint{3.645737in}{1.368714in}}%
\pgfpathlineto{\pgfqpoint{3.659408in}{1.349333in}}%
\pgfpathlineto{\pgfqpoint{3.685818in}{1.311615in}}%
\pgfpathlineto{\pgfqpoint{3.734626in}{1.237333in}}%
\pgfpathlineto{\pgfqpoint{3.745581in}{1.218333in}}%
\pgfpathlineto{\pgfqpoint{3.765980in}{1.187031in}}%
\pgfpathlineto{\pgfqpoint{3.780546in}{1.162667in}}%
\pgfpathlineto{\pgfqpoint{3.806061in}{1.118857in}}%
\pgfpathlineto{\pgfqpoint{3.822892in}{1.088000in}}%
\pgfpathlineto{\pgfqpoint{3.846141in}{1.044058in}}%
\pgfpathlineto{\pgfqpoint{3.861132in}{1.013333in}}%
\pgfpathlineto{\pgfqpoint{3.868047in}{0.996404in}}%
\pgfpathlineto{\pgfqpoint{3.880507in}{0.970676in}}%
\pgfpathlineto{\pgfqpoint{3.894620in}{0.938667in}}%
\pgfpathlineto{\pgfqpoint{3.902251in}{0.916264in}}%
\pgfpathlineto{\pgfqpoint{3.908994in}{0.901333in}}%
\pgfpathlineto{\pgfqpoint{3.922820in}{0.860755in}}%
\pgfpathlineto{\pgfqpoint{3.935360in}{0.818230in}}%
\pgfpathlineto{\pgfqpoint{3.941407in}{0.789333in}}%
\pgfpathlineto{\pgfqpoint{3.943632in}{0.768141in}}%
\pgfpathlineto{\pgfqpoint{3.947224in}{0.752000in}}%
\pgfpathlineto{\pgfqpoint{3.949561in}{0.730336in}}%
\pgfpathlineto{\pgfqpoint{3.949575in}{0.714667in}}%
\pgfpathlineto{\pgfqpoint{3.947331in}{0.696920in}}%
\pgfpathlineto{\pgfqpoint{3.947165in}{0.677333in}}%
\pgfpathlineto{\pgfqpoint{3.944582in}{0.660307in}}%
\pgfpathlineto{\pgfqpoint{3.935177in}{0.631734in}}%
\pgfpathlineto{\pgfqpoint{3.921106in}{0.607508in}}%
\pgfpathlineto{\pgfqpoint{3.902153in}{0.587828in}}%
\pgfpathlineto{\pgfqpoint{3.886222in}{0.577287in}}%
\pgfpathlineto{\pgfqpoint{3.849679in}{0.562038in}}%
\pgfpathlineto{\pgfqpoint{3.841786in}{0.561276in}}%
\pgfpathlineto{\pgfqpoint{3.806061in}{0.556543in}}%
\pgfpathlineto{\pgfqpoint{3.765980in}{0.557855in}}%
\pgfpathlineto{\pgfqpoint{3.759576in}{0.559369in}}%
\pgfpathlineto{\pgfqpoint{3.718355in}{0.565333in}}%
\pgfpathlineto{\pgfqpoint{3.685818in}{0.573158in}}%
\pgfpathlineto{\pgfqpoint{3.662908in}{0.581327in}}%
\pgfpathlineto{\pgfqpoint{3.645737in}{0.585487in}}%
\pgfpathlineto{\pgfqpoint{3.598540in}{0.602667in}}%
\pgfpathlineto{\pgfqpoint{3.537417in}{0.628895in}}%
\pgfpathlineto{\pgfqpoint{3.485414in}{0.654613in}}%
\pgfpathlineto{\pgfqpoint{3.470491in}{0.663433in}}%
\pgfpathlineto{\pgfqpoint{3.442266in}{0.677333in}}%
\pgfpathlineto{\pgfqpoint{3.365172in}{0.722100in}}%
\pgfpathlineto{\pgfqpoint{3.346586in}{0.734688in}}%
\pgfpathlineto{\pgfqpoint{3.317301in}{0.752000in}}%
\pgfpathlineto{\pgfqpoint{3.298859in}{0.764900in}}%
\pgfpathlineto{\pgfqpoint{3.285010in}{0.773073in}}%
\pgfpathlineto{\pgfqpoint{3.244929in}{0.799982in}}%
\pgfpathlineto{\pgfqpoint{3.229093in}{0.811916in}}%
\pgfpathlineto{\pgfqpoint{3.204848in}{0.827678in}}%
\pgfpathlineto{\pgfqpoint{3.183561in}{0.844172in}}%
\pgfpathlineto{\pgfqpoint{3.154937in}{0.864000in}}%
\pgfpathlineto{\pgfqpoint{3.084606in}{0.917179in}}%
\pgfpathlineto{\pgfqpoint{3.072490in}{0.927381in}}%
\pgfpathlineto{\pgfqpoint{3.044525in}{0.948583in}}%
\pgfpathlineto{\pgfqpoint{2.964364in}{1.013777in}}%
\pgfpathlineto{\pgfqpoint{2.944269in}{1.031949in}}%
\pgfpathlineto{\pgfqpoint{2.920957in}{1.050667in}}%
\pgfpathlineto{\pgfqpoint{2.836236in}{1.125333in}}%
\pgfpathlineto{\pgfqpoint{2.820452in}{1.140619in}}%
\pgfpathlineto{\pgfqpoint{2.795306in}{1.162667in}}%
\pgfpathlineto{\pgfqpoint{2.716066in}{1.237333in}}%
\pgfpathlineto{\pgfqpoint{2.700776in}{1.253148in}}%
\pgfpathlineto{\pgfqpoint{2.677669in}{1.274667in}}%
\pgfpathlineto{\pgfqpoint{2.661763in}{1.291475in}}%
\pgfpathlineto{\pgfqpoint{2.640034in}{1.312000in}}%
\pgfpathlineto{\pgfqpoint{2.623174in}{1.330198in}}%
\pgfpathlineto{\pgfqpoint{2.598409in}{1.354202in}}%
\pgfpathlineto{\pgfqpoint{2.523475in}{1.433045in}}%
\pgfpathlineto{\pgfqpoint{2.491193in}{1.468598in}}%
\pgfpathlineto{\pgfqpoint{2.472903in}{1.488895in}}%
\pgfpathlineto{\pgfqpoint{2.443313in}{1.521122in}}%
\pgfpathlineto{\pgfqpoint{2.430163in}{1.536000in}}%
\pgfpathlineto{\pgfqpoint{2.397421in}{1.573333in}}%
\pgfpathlineto{\pgfqpoint{2.382773in}{1.591610in}}%
\pgfpathlineto{\pgfqpoint{2.363152in}{1.613310in}}%
\pgfpathlineto{\pgfqpoint{2.282990in}{1.711076in}}%
\pgfpathlineto{\pgfqpoint{2.261266in}{1.739765in}}%
\pgfpathlineto{\pgfqpoint{2.242909in}{1.761796in}}%
\pgfpathlineto{\pgfqpoint{2.160663in}{1.872000in}}%
\pgfpathlineto{\pgfqpoint{2.146207in}{1.893927in}}%
\pgfpathlineto{\pgfqpoint{2.122667in}{1.926287in}}%
\pgfpathlineto{\pgfqpoint{2.108749in}{1.946667in}}%
\pgfpathlineto{\pgfqpoint{2.082586in}{1.985372in}}%
\pgfpathlineto{\pgfqpoint{2.036154in}{2.058667in}}%
\pgfpathlineto{\pgfqpoint{2.024702in}{2.079418in}}%
\pgfpathlineto{\pgfqpoint{2.013803in}{2.096000in}}%
\pgfpathlineto{\pgfqpoint{1.980806in}{2.153470in}}%
\pgfpathlineto{\pgfqpoint{1.952127in}{2.208000in}}%
\pgfpathlineto{\pgfqpoint{1.943428in}{2.227714in}}%
\pgfpathlineto{\pgfqpoint{1.933744in}{2.245333in}}%
\pgfpathlineto{\pgfqpoint{1.912146in}{2.292090in}}%
\pgfpathlineto{\pgfqpoint{1.885413in}{2.360343in}}%
\pgfpathlineto{\pgfqpoint{1.873701in}{2.394667in}}%
\pgfpathlineto{\pgfqpoint{1.870011in}{2.406003in}}%
\pgfpathlineto{\pgfqpoint{1.863234in}{2.432000in}}%
\pgfpathlineto{\pgfqpoint{1.860477in}{2.449117in}}%
\pgfpathlineto{\pgfqpoint{1.854853in}{2.469333in}}%
\pgfpathlineto{\pgfqpoint{1.849214in}{2.506667in}}%
\pgfpathlineto{\pgfqpoint{1.849534in}{2.513590in}}%
\pgfpathlineto{\pgfqpoint{1.847280in}{2.544000in}}%
\pgfpathlineto{\pgfqpoint{1.850526in}{2.581333in}}%
\pgfpathlineto{\pgfqpoint{1.854837in}{2.593197in}}%
\pgfpathlineto{\pgfqpoint{1.861328in}{2.618667in}}%
\pgfpathlineto{\pgfqpoint{1.867298in}{2.632530in}}%
\pgfpathlineto{\pgfqpoint{1.884257in}{2.656000in}}%
\pgfpathlineto{\pgfqpoint{1.904182in}{2.672842in}}%
\pgfpathlineto{\pgfqpoint{1.929910in}{2.686210in}}%
\pgfpathlineto{\pgfqpoint{1.962343in}{2.694895in}}%
\pgfpathlineto{\pgfqpoint{2.002424in}{2.697696in}}%
\pgfpathlineto{\pgfqpoint{2.006392in}{2.697029in}}%
\pgfpathlineto{\pgfqpoint{2.042505in}{2.695029in}}%
\pgfpathlineto{\pgfqpoint{2.090107in}{2.686328in}}%
\pgfpathlineto{\pgfqpoint{2.122667in}{2.677420in}}%
\pgfpathlineto{\pgfqpoint{2.138937in}{2.671155in}}%
\pgfpathlineto{\pgfqpoint{2.162747in}{2.664667in}}%
\pgfpathlineto{\pgfqpoint{2.214816in}{2.644834in}}%
\pgfpathlineto{\pgfqpoint{2.282990in}{2.613889in}}%
\pgfpathlineto{\pgfqpoint{2.304824in}{2.601670in}}%
\pgfpathlineto{\pgfqpoint{2.323071in}{2.593466in}}%
\pgfpathlineto{\pgfqpoint{2.363152in}{2.571869in}}%
\pgfpathlineto{\pgfqpoint{2.381013in}{2.560637in}}%
\pgfpathlineto{\pgfqpoint{2.411614in}{2.544000in}}%
\pgfpathlineto{\pgfqpoint{2.430265in}{2.531846in}}%
\pgfpathlineto{\pgfqpoint{2.443313in}{2.524777in}}%
\pgfpathlineto{\pgfqpoint{2.523475in}{2.473532in}}%
\pgfpathlineto{\pgfqpoint{2.603636in}{2.417921in}}%
\pgfpathlineto{\pgfqpoint{2.643717in}{2.388950in}}%
\pgfpathlineto{\pgfqpoint{2.661931in}{2.374299in}}%
\pgfpathlineto{\pgfqpoint{2.686044in}{2.357333in}}%
\pgfpathlineto{\pgfqpoint{2.706102in}{2.340775in}}%
\pgfpathlineto{\pgfqpoint{2.723879in}{2.328046in}}%
\pgfpathlineto{\pgfqpoint{2.723879in}{2.328046in}}%
\pgfusepath{fill}%
\end{pgfscope}%
\begin{pgfscope}%
\pgfpathrectangle{\pgfqpoint{0.800000in}{0.528000in}}{\pgfqpoint{3.968000in}{3.696000in}}%
\pgfusepath{clip}%
\pgfsetbuttcap%
\pgfsetroundjoin%
\definecolor{currentfill}{rgb}{0.272594,0.025563,0.353093}%
\pgfsetfillcolor{currentfill}%
\pgfsetlinewidth{0.000000pt}%
\definecolor{currentstroke}{rgb}{0.000000,0.000000,0.000000}%
\pgfsetstrokecolor{currentstroke}%
\pgfsetdash{}{0pt}%
\pgfpathmoveto{\pgfqpoint{3.764017in}{0.528000in}}%
\pgfpathlineto{\pgfqpoint{3.685818in}{0.547445in}}%
\pgfpathlineto{\pgfqpoint{3.672433in}{0.552866in}}%
\pgfpathlineto{\pgfqpoint{3.634178in}{0.565333in}}%
\pgfpathlineto{\pgfqpoint{3.605657in}{0.576616in}}%
\pgfpathlineto{\pgfqpoint{3.587364in}{0.585628in}}%
\pgfpathlineto{\pgfqpoint{3.565576in}{0.594027in}}%
\pgfpathlineto{\pgfqpoint{3.525495in}{0.613131in}}%
\pgfpathlineto{\pgfqpoint{3.507577in}{0.623310in}}%
\pgfpathlineto{\pgfqpoint{3.473632in}{0.640000in}}%
\pgfpathlineto{\pgfqpoint{3.431410in}{0.664364in}}%
\pgfpathlineto{\pgfqpoint{3.405253in}{0.678300in}}%
\pgfpathlineto{\pgfqpoint{3.382334in}{0.693319in}}%
\pgfpathlineto{\pgfqpoint{3.365172in}{0.702758in}}%
\pgfpathlineto{\pgfqpoint{3.277046in}{0.759418in}}%
\pgfpathlineto{\pgfqpoint{3.204848in}{0.809731in}}%
\pgfpathlineto{\pgfqpoint{3.124687in}{0.868516in}}%
\pgfpathlineto{\pgfqpoint{3.105966in}{0.883896in}}%
\pgfpathlineto{\pgfqpoint{3.081903in}{0.901333in}}%
\pgfpathlineto{\pgfqpoint{3.062271in}{0.917863in}}%
\pgfpathlineto{\pgfqpoint{3.034917in}{0.938667in}}%
\pgfpathlineto{\pgfqpoint{2.944539in}{1.013333in}}%
\pgfpathlineto{\pgfqpoint{2.913421in}{1.040549in}}%
\pgfpathlineto{\pgfqpoint{2.884202in}{1.065275in}}%
\pgfpathlineto{\pgfqpoint{2.804040in}{1.137110in}}%
\pgfpathlineto{\pgfqpoint{2.723879in}{1.212157in}}%
\pgfpathlineto{\pgfqpoint{2.643717in}{1.290501in}}%
\pgfpathlineto{\pgfqpoint{2.563556in}{1.372233in}}%
\pgfpathlineto{\pgfqpoint{2.549794in}{1.386667in}}%
\pgfpathlineto{\pgfqpoint{2.479830in}{1.461333in}}%
\pgfpathlineto{\pgfqpoint{2.463659in}{1.480285in}}%
\pgfpathlineto{\pgfqpoint{2.443313in}{1.501583in}}%
\pgfpathlineto{\pgfqpoint{2.427133in}{1.520929in}}%
\pgfpathlineto{\pgfqpoint{2.403232in}{1.546986in}}%
\pgfpathlineto{\pgfqpoint{2.373040in}{1.582544in}}%
\pgfpathlineto{\pgfqpoint{2.348490in}{1.610667in}}%
\pgfpathlineto{\pgfqpoint{2.317073in}{1.648000in}}%
\pgfpathlineto{\pgfqpoint{2.302849in}{1.666498in}}%
\pgfpathlineto{\pgfqpoint{2.282990in}{1.689524in}}%
\pgfpathlineto{\pgfqpoint{2.268317in}{1.709000in}}%
\pgfpathlineto{\pgfqpoint{2.242909in}{1.739980in}}%
\pgfpathlineto{\pgfqpoint{2.227309in}{1.760000in}}%
\pgfpathlineto{\pgfqpoint{2.170534in}{1.834667in}}%
\pgfpathlineto{\pgfqpoint{2.116560in}{1.909333in}}%
\pgfpathlineto{\pgfqpoint{2.103814in}{1.929106in}}%
\pgfpathlineto{\pgfqpoint{2.082586in}{1.958632in}}%
\pgfpathlineto{\pgfqpoint{2.065658in}{1.984000in}}%
\pgfpathlineto{\pgfqpoint{2.040937in}{2.021333in}}%
\pgfpathlineto{\pgfqpoint{2.027335in}{2.044537in}}%
\pgfpathlineto{\pgfqpoint{2.012464in}{2.068018in}}%
\pgfpathlineto{\pgfqpoint{1.994712in}{2.096000in}}%
\pgfpathlineto{\pgfqpoint{1.983905in}{2.116084in}}%
\pgfpathlineto{\pgfqpoint{1.972854in}{2.133333in}}%
\pgfpathlineto{\pgfqpoint{1.940723in}{2.190805in}}%
\pgfpathlineto{\pgfqpoint{1.912843in}{2.245333in}}%
\pgfpathlineto{\pgfqpoint{1.904102in}{2.265751in}}%
\pgfpathlineto{\pgfqpoint{1.895077in}{2.282667in}}%
\pgfpathlineto{\pgfqpoint{1.878248in}{2.320000in}}%
\pgfpathlineto{\pgfqpoint{1.869076in}{2.345126in}}%
\pgfpathlineto{\pgfqpoint{1.863323in}{2.357333in}}%
\pgfpathlineto{\pgfqpoint{1.849440in}{2.394667in}}%
\pgfpathlineto{\pgfqpoint{1.835459in}{2.438187in}}%
\pgfpathlineto{\pgfqpoint{1.823559in}{2.486604in}}%
\pgfpathlineto{\pgfqpoint{1.820254in}{2.506667in}}%
\pgfpathlineto{\pgfqpoint{1.819312in}{2.522773in}}%
\pgfpathlineto{\pgfqpoint{1.815623in}{2.544000in}}%
\pgfpathlineto{\pgfqpoint{1.816324in}{2.557324in}}%
\pgfpathlineto{\pgfqpoint{1.814809in}{2.581333in}}%
\pgfpathlineto{\pgfqpoint{1.816417in}{2.605257in}}%
\pgfpathlineto{\pgfqpoint{1.819275in}{2.618667in}}%
\pgfpathlineto{\pgfqpoint{1.827985in}{2.642851in}}%
\pgfpathlineto{\pgfqpoint{1.831345in}{2.656000in}}%
\pgfpathlineto{\pgfqpoint{1.834390in}{2.663182in}}%
\pgfpathlineto{\pgfqpoint{1.842101in}{2.674497in}}%
\pgfpathlineto{\pgfqpoint{1.858556in}{2.693333in}}%
\pgfpathlineto{\pgfqpoint{1.870655in}{2.704070in}}%
\pgfpathlineto{\pgfqpoint{1.882182in}{2.710663in}}%
\pgfpathlineto{\pgfqpoint{1.922263in}{2.726091in}}%
\pgfpathlineto{\pgfqpoint{1.962343in}{2.731302in}}%
\pgfpathlineto{\pgfqpoint{2.002424in}{2.730027in}}%
\pgfpathlineto{\pgfqpoint{2.051281in}{2.722492in}}%
\pgfpathlineto{\pgfqpoint{2.082586in}{2.714743in}}%
\pgfpathlineto{\pgfqpoint{2.099186in}{2.708795in}}%
\pgfpathlineto{\pgfqpoint{2.122667in}{2.702956in}}%
\pgfpathlineto{\pgfqpoint{2.170707in}{2.685920in}}%
\pgfpathlineto{\pgfqpoint{2.242909in}{2.654884in}}%
\pgfpathlineto{\pgfqpoint{2.267593in}{2.641658in}}%
\pgfpathlineto{\pgfqpoint{2.282990in}{2.635067in}}%
\pgfpathlineto{\pgfqpoint{2.293812in}{2.628747in}}%
\pgfpathlineto{\pgfqpoint{2.333962in}{2.608522in}}%
\pgfpathlineto{\pgfqpoint{2.403232in}{2.568627in}}%
\pgfpathlineto{\pgfqpoint{2.443954in}{2.544000in}}%
\pgfpathlineto{\pgfqpoint{2.523475in}{2.492006in}}%
\pgfpathlineto{\pgfqpoint{2.603636in}{2.436267in}}%
\pgfpathlineto{\pgfqpoint{2.683798in}{2.376636in}}%
\pgfpathlineto{\pgfqpoint{2.694766in}{2.367550in}}%
\pgfpathlineto{\pgfqpoint{2.723879in}{2.345696in}}%
\pgfpathlineto{\pgfqpoint{2.738317in}{2.333448in}}%
\pgfpathlineto{\pgfqpoint{2.763960in}{2.313993in}}%
\pgfpathlineto{\pgfqpoint{2.781364in}{2.298878in}}%
\pgfpathlineto{\pgfqpoint{2.815396in}{2.272090in}}%
\pgfpathlineto{\pgfqpoint{2.890872in}{2.208000in}}%
\pgfpathlineto{\pgfqpoint{2.974999in}{2.133333in}}%
\pgfpathlineto{\pgfqpoint{3.055525in}{2.058667in}}%
\pgfpathlineto{\pgfqpoint{3.069447in}{2.044547in}}%
\pgfpathlineto{\pgfqpoint{3.094548in}{2.021333in}}%
\pgfpathlineto{\pgfqpoint{3.108818in}{2.006552in}}%
\pgfpathlineto{\pgfqpoint{3.132797in}{1.984000in}}%
\pgfpathlineto{\pgfqpoint{3.147769in}{1.968167in}}%
\pgfpathlineto{\pgfqpoint{3.170310in}{1.946667in}}%
\pgfpathlineto{\pgfqpoint{3.244929in}{1.870125in}}%
\pgfpathlineto{\pgfqpoint{3.325091in}{1.783954in}}%
\pgfpathlineto{\pgfqpoint{3.412798in}{1.685333in}}%
\pgfpathlineto{\pgfqpoint{3.426736in}{1.668011in}}%
\pgfpathlineto{\pgfqpoint{3.445333in}{1.647606in}}%
\pgfpathlineto{\pgfqpoint{3.525495in}{1.550193in}}%
\pgfpathlineto{\pgfqpoint{3.568952in}{1.495522in}}%
\pgfpathlineto{\pgfqpoint{3.623012in}{1.424000in}}%
\pgfpathlineto{\pgfqpoint{3.631833in}{1.411049in}}%
\pgfpathlineto{\pgfqpoint{3.650616in}{1.386667in}}%
\pgfpathlineto{\pgfqpoint{3.664097in}{1.366434in}}%
\pgfpathlineto{\pgfqpoint{3.685818in}{1.337086in}}%
\pgfpathlineto{\pgfqpoint{3.703117in}{1.312000in}}%
\pgfpathlineto{\pgfqpoint{3.728653in}{1.274667in}}%
\pgfpathlineto{\pgfqpoint{3.742164in}{1.252483in}}%
\pgfpathlineto{\pgfqpoint{3.757486in}{1.229422in}}%
\pgfpathlineto{\pgfqpoint{3.776680in}{1.200000in}}%
\pgfpathlineto{\pgfqpoint{3.786781in}{1.182042in}}%
\pgfpathlineto{\pgfqpoint{3.806061in}{1.151895in}}%
\pgfpathlineto{\pgfqpoint{3.829483in}{1.109816in}}%
\pgfpathlineto{\pgfqpoint{3.846141in}{1.082049in}}%
\pgfpathlineto{\pgfqpoint{3.862796in}{1.050667in}}%
\pgfpathlineto{\pgfqpoint{3.886222in}{1.004920in}}%
\pgfpathlineto{\pgfqpoint{3.916751in}{0.938667in}}%
\pgfpathlineto{\pgfqpoint{3.919051in}{0.931912in}}%
\pgfpathlineto{\pgfqpoint{3.932262in}{0.901333in}}%
\pgfpathlineto{\pgfqpoint{3.940255in}{0.876996in}}%
\pgfpathlineto{\pgfqpoint{3.945922in}{0.864000in}}%
\pgfpathlineto{\pgfqpoint{3.958284in}{0.826667in}}%
\pgfpathlineto{\pgfqpoint{3.969505in}{0.786426in}}%
\pgfpathlineto{\pgfqpoint{3.976298in}{0.752000in}}%
\pgfpathlineto{\pgfqpoint{3.978767in}{0.726201in}}%
\pgfpathlineto{\pgfqpoint{3.981178in}{0.714667in}}%
\pgfpathlineto{\pgfqpoint{3.982657in}{0.699509in}}%
\pgfpathlineto{\pgfqpoint{3.982495in}{0.677333in}}%
\pgfpathlineto{\pgfqpoint{3.979169in}{0.651909in}}%
\pgfpathlineto{\pgfqpoint{3.978962in}{0.640000in}}%
\pgfpathlineto{\pgfqpoint{3.977369in}{0.629768in}}%
\pgfpathlineto{\pgfqpoint{3.966384in}{0.598084in}}%
\pgfpathlineto{\pgfqpoint{3.943473in}{0.565333in}}%
\pgfpathlineto{\pgfqpoint{3.935512in}{0.556756in}}%
\pgfpathlineto{\pgfqpoint{3.912231in}{0.541108in}}%
\pgfpathlineto{\pgfqpoint{3.884196in}{0.529887in}}%
\pgfpathlineto{\pgfqpoint{3.874053in}{0.528000in}}%
\pgfpathlineto{\pgfqpoint{3.969178in}{0.528000in}}%
\pgfpathlineto{\pgfqpoint{3.973946in}{0.535044in}}%
\pgfpathlineto{\pgfqpoint{3.996355in}{0.565333in}}%
\pgfpathlineto{\pgfqpoint{4.009662in}{0.602667in}}%
\pgfpathlineto{\pgfqpoint{4.009740in}{0.605717in}}%
\pgfpathlineto{\pgfqpoint{4.014366in}{0.640000in}}%
\pgfpathlineto{\pgfqpoint{4.013712in}{0.646750in}}%
\pgfpathlineto{\pgfqpoint{4.014128in}{0.677333in}}%
\pgfpathlineto{\pgfqpoint{4.010227in}{0.714667in}}%
\pgfpathlineto{\pgfqpoint{4.002470in}{0.755721in}}%
\pgfpathlineto{\pgfqpoint{3.993444in}{0.789333in}}%
\pgfpathlineto{\pgfqpoint{3.986621in}{0.808183in}}%
\pgfpathlineto{\pgfqpoint{3.981925in}{0.826667in}}%
\pgfpathlineto{\pgfqpoint{3.966384in}{0.870631in}}%
\pgfpathlineto{\pgfqpoint{3.916002in}{0.985595in}}%
\pgfpathlineto{\pgfqpoint{3.882466in}{1.050667in}}%
\pgfpathlineto{\pgfqpoint{3.869442in}{1.072370in}}%
\pgfpathlineto{\pgfqpoint{3.861651in}{1.088000in}}%
\pgfpathlineto{\pgfqpoint{3.840336in}{1.125333in}}%
\pgfpathlineto{\pgfqpoint{3.827421in}{1.145230in}}%
\pgfpathlineto{\pgfqpoint{3.813285in}{1.169396in}}%
\pgfpathlineto{\pgfqpoint{3.794675in}{1.200000in}}%
\pgfpathlineto{\pgfqpoint{3.765980in}{1.244715in}}%
\pgfpathlineto{\pgfqpoint{3.737778in}{1.285731in}}%
\pgfpathlineto{\pgfqpoint{3.720656in}{1.312000in}}%
\pgfpathlineto{\pgfqpoint{3.706312in}{1.331089in}}%
\pgfpathlineto{\pgfqpoint{3.685818in}{1.361364in}}%
\pgfpathlineto{\pgfqpoint{3.667456in}{1.386667in}}%
\pgfpathlineto{\pgfqpoint{3.605657in}{1.469503in}}%
\pgfpathlineto{\pgfqpoint{3.517878in}{1.580428in}}%
\pgfpathlineto{\pgfqpoint{3.445333in}{1.666809in}}%
\pgfpathlineto{\pgfqpoint{3.429346in}{1.685333in}}%
\pgfpathlineto{\pgfqpoint{3.356471in}{1.768104in}}%
\pgfpathlineto{\pgfqpoint{3.285010in}{1.845541in}}%
\pgfpathlineto{\pgfqpoint{3.204848in}{1.929083in}}%
\pgfpathlineto{\pgfqpoint{3.124687in}{2.009278in}}%
\pgfpathlineto{\pgfqpoint{3.112293in}{2.021333in}}%
\pgfpathlineto{\pgfqpoint{3.034034in}{2.096000in}}%
\pgfpathlineto{\pgfqpoint{3.018787in}{2.109359in}}%
\pgfpathlineto{\pgfqpoint{2.993659in}{2.133333in}}%
\pgfpathlineto{\pgfqpoint{2.978409in}{2.146416in}}%
\pgfpathlineto{\pgfqpoint{2.952396in}{2.170667in}}%
\pgfpathlineto{\pgfqpoint{2.924283in}{2.195684in}}%
\pgfpathlineto{\pgfqpoint{2.844121in}{2.264825in}}%
\pgfpathlineto{\pgfqpoint{2.763960in}{2.330985in}}%
\pgfpathlineto{\pgfqpoint{2.748233in}{2.342685in}}%
\pgfpathlineto{\pgfqpoint{2.723879in}{2.362970in}}%
\pgfpathlineto{\pgfqpoint{2.704751in}{2.376850in}}%
\pgfpathlineto{\pgfqpoint{2.683191in}{2.394667in}}%
\pgfpathlineto{\pgfqpoint{2.660757in}{2.410539in}}%
\pgfpathlineto{\pgfqpoint{2.633268in}{2.432000in}}%
\pgfpathlineto{\pgfqpoint{2.563556in}{2.482321in}}%
\pgfpathlineto{\pgfqpoint{2.523475in}{2.510232in}}%
\pgfpathlineto{\pgfqpoint{2.501724in}{2.523741in}}%
\pgfpathlineto{\pgfqpoint{2.472489in}{2.544000in}}%
\pgfpathlineto{\pgfqpoint{2.403232in}{2.587723in}}%
\pgfpathlineto{\pgfqpoint{2.350483in}{2.618667in}}%
\pgfpathlineto{\pgfqpoint{2.282725in}{2.656000in}}%
\pgfpathlineto{\pgfqpoint{2.256078in}{2.668266in}}%
\pgfpathlineto{\pgfqpoint{2.242909in}{2.675666in}}%
\pgfpathlineto{\pgfqpoint{2.229708in}{2.681037in}}%
\pgfpathlineto{\pgfqpoint{2.202828in}{2.694692in}}%
\pgfpathlineto{\pgfqpoint{2.175517in}{2.705228in}}%
\pgfpathlineto{\pgfqpoint{2.162747in}{2.711472in}}%
\pgfpathlineto{\pgfqpoint{2.111615in}{2.730667in}}%
\pgfpathlineto{\pgfqpoint{2.067513in}{2.744706in}}%
\pgfpathlineto{\pgfqpoint{2.042505in}{2.750856in}}%
\pgfpathlineto{\pgfqpoint{2.026443in}{2.753039in}}%
\pgfpathlineto{\pgfqpoint{2.002424in}{2.758909in}}%
\pgfpathlineto{\pgfqpoint{1.962343in}{2.763375in}}%
\pgfpathlineto{\pgfqpoint{1.956758in}{2.762798in}}%
\pgfpathlineto{\pgfqpoint{1.922263in}{2.762881in}}%
\pgfpathlineto{\pgfqpoint{1.892467in}{2.758420in}}%
\pgfpathlineto{\pgfqpoint{1.882182in}{2.755253in}}%
\pgfpathlineto{\pgfqpoint{1.838108in}{2.734386in}}%
\pgfpathlineto{\pgfqpoint{1.834102in}{2.730667in}}%
\pgfpathlineto{\pgfqpoint{1.802020in}{2.692043in}}%
\pgfpathlineto{\pgfqpoint{1.789150in}{2.656000in}}%
\pgfpathlineto{\pgfqpoint{1.784608in}{2.634886in}}%
\pgfpathlineto{\pgfqpoint{1.783415in}{2.618667in}}%
\pgfpathlineto{\pgfqpoint{1.784288in}{2.602150in}}%
\pgfpathlineto{\pgfqpoint{1.783065in}{2.581333in}}%
\pgfpathlineto{\pgfqpoint{1.784209in}{2.560590in}}%
\pgfpathlineto{\pgfqpoint{1.786646in}{2.544000in}}%
\pgfpathlineto{\pgfqpoint{1.789554in}{2.532388in}}%
\pgfpathlineto{\pgfqpoint{1.793181in}{2.506667in}}%
\pgfpathlineto{\pgfqpoint{1.802020in}{2.469252in}}%
\pgfpathlineto{\pgfqpoint{1.813660in}{2.432000in}}%
\pgfpathlineto{\pgfqpoint{1.821411in}{2.412728in}}%
\pgfpathlineto{\pgfqpoint{1.826745in}{2.394667in}}%
\pgfpathlineto{\pgfqpoint{1.831325in}{2.384629in}}%
\pgfpathlineto{\pgfqpoint{1.842101in}{2.354685in}}%
\pgfpathlineto{\pgfqpoint{1.857324in}{2.320000in}}%
\pgfpathlineto{\pgfqpoint{1.874437in}{2.282667in}}%
\pgfpathlineto{\pgfqpoint{1.902377in}{2.226522in}}%
\pgfpathlineto{\pgfqpoint{1.932815in}{2.170667in}}%
\pgfpathlineto{\pgfqpoint{1.943677in}{2.153280in}}%
\pgfpathlineto{\pgfqpoint{1.962343in}{2.119491in}}%
\pgfpathlineto{\pgfqpoint{1.976493in}{2.096000in}}%
\pgfpathlineto{\pgfqpoint{2.002424in}{2.053822in}}%
\pgfpathlineto{\pgfqpoint{2.023471in}{2.021333in}}%
\pgfpathlineto{\pgfqpoint{2.047907in}{1.984000in}}%
\pgfpathlineto{\pgfqpoint{2.061942in}{1.964772in}}%
\pgfpathlineto{\pgfqpoint{2.082586in}{1.933367in}}%
\pgfpathlineto{\pgfqpoint{2.099524in}{1.909333in}}%
\pgfpathlineto{\pgfqpoint{2.126048in}{1.872000in}}%
\pgfpathlineto{\pgfqpoint{2.141557in}{1.852262in}}%
\pgfpathlineto{\pgfqpoint{2.162747in}{1.822459in}}%
\pgfpathlineto{\pgfqpoint{2.210424in}{1.760000in}}%
\pgfpathlineto{\pgfqpoint{2.224673in}{1.743014in}}%
\pgfpathlineto{\pgfqpoint{2.242909in}{1.718641in}}%
\pgfpathlineto{\pgfqpoint{2.258776in}{1.700113in}}%
\pgfpathlineto{\pgfqpoint{2.282990in}{1.669279in}}%
\pgfpathlineto{\pgfqpoint{2.310775in}{1.636547in}}%
\pgfpathlineto{\pgfqpoint{2.331753in}{1.610667in}}%
\pgfpathlineto{\pgfqpoint{2.363424in}{1.573333in}}%
\pgfpathlineto{\pgfqpoint{2.381856in}{1.553422in}}%
\pgfpathlineto{\pgfqpoint{2.403232in}{1.527948in}}%
\pgfpathlineto{\pgfqpoint{2.435958in}{1.491816in}}%
\pgfpathlineto{\pgfqpoint{2.463239in}{1.461333in}}%
\pgfpathlineto{\pgfqpoint{2.532680in}{1.386667in}}%
\pgfpathlineto{\pgfqpoint{2.547676in}{1.371875in}}%
\pgfpathlineto{\pgfqpoint{2.568350in}{1.349333in}}%
\pgfpathlineto{\pgfqpoint{2.643717in}{1.272834in}}%
\pgfpathlineto{\pgfqpoint{2.662809in}{1.255117in}}%
\pgfpathlineto{\pgfqpoint{2.683798in}{1.233459in}}%
\pgfpathlineto{\pgfqpoint{2.763960in}{1.157043in}}%
\pgfpathlineto{\pgfqpoint{2.844121in}{1.083677in}}%
\pgfpathlineto{\pgfqpoint{2.924871in}{1.012786in}}%
\pgfpathlineto{\pgfqpoint{3.013812in}{0.938667in}}%
\pgfpathlineto{\pgfqpoint{3.084606in}{0.882243in}}%
\pgfpathlineto{\pgfqpoint{3.108225in}{0.864000in}}%
\pgfpathlineto{\pgfqpoint{3.124687in}{0.851339in}}%
\pgfpathlineto{\pgfqpoint{3.208416in}{0.789333in}}%
\pgfpathlineto{\pgfqpoint{3.285010in}{0.736222in}}%
\pgfpathlineto{\pgfqpoint{3.365172in}{0.683945in}}%
\pgfpathlineto{\pgfqpoint{3.394125in}{0.666968in}}%
\pgfpathlineto{\pgfqpoint{3.405253in}{0.659507in}}%
\pgfpathlineto{\pgfqpoint{3.418598in}{0.652431in}}%
\pgfpathlineto{\pgfqpoint{3.445333in}{0.635810in}}%
\pgfpathlineto{\pgfqpoint{3.468318in}{0.624075in}}%
\pgfpathlineto{\pgfqpoint{3.493266in}{0.609980in}}%
\pgfpathlineto{\pgfqpoint{3.525495in}{0.592495in}}%
\pgfpathlineto{\pgfqpoint{3.571155in}{0.570530in}}%
\pgfpathlineto{\pgfqpoint{3.605657in}{0.554478in}}%
\pgfpathlineto{\pgfqpoint{3.626098in}{0.547041in}}%
\pgfpathlineto{\pgfqpoint{3.645737in}{0.537835in}}%
\pgfpathlineto{\pgfqpoint{3.653745in}{0.535459in}}%
\pgfpathlineto{\pgfqpoint{3.672163in}{0.528000in}}%
\pgfpathlineto{\pgfqpoint{3.725899in}{0.528000in}}%
\pgfpathlineto{\pgfqpoint{3.725899in}{0.528000in}}%
\pgfusepath{fill}%
\end{pgfscope}%
\begin{pgfscope}%
\pgfpathrectangle{\pgfqpoint{0.800000in}{0.528000in}}{\pgfqpoint{3.968000in}{3.696000in}}%
\pgfusepath{clip}%
\pgfsetbuttcap%
\pgfsetroundjoin%
\definecolor{currentfill}{rgb}{0.272594,0.025563,0.353093}%
\pgfsetfillcolor{currentfill}%
\pgfsetlinewidth{0.000000pt}%
\definecolor{currentstroke}{rgb}{0.000000,0.000000,0.000000}%
\pgfsetstrokecolor{currentstroke}%
\pgfsetdash{}{0pt}%
\pgfpathmoveto{\pgfqpoint{3.672163in}{0.528000in}}%
\pgfpathlineto{\pgfqpoint{3.581703in}{0.565333in}}%
\pgfpathlineto{\pgfqpoint{3.485414in}{0.613612in}}%
\pgfpathlineto{\pgfqpoint{3.468318in}{0.624075in}}%
\pgfpathlineto{\pgfqpoint{3.438219in}{0.640000in}}%
\pgfpathlineto{\pgfqpoint{3.418598in}{0.652431in}}%
\pgfpathlineto{\pgfqpoint{3.405253in}{0.659507in}}%
\pgfpathlineto{\pgfqpoint{3.394125in}{0.666968in}}%
\pgfpathlineto{\pgfqpoint{3.365172in}{0.683945in}}%
\pgfpathlineto{\pgfqpoint{3.317319in}{0.714667in}}%
\pgfpathlineto{\pgfqpoint{3.244929in}{0.763668in}}%
\pgfpathlineto{\pgfqpoint{3.164768in}{0.821159in}}%
\pgfpathlineto{\pgfqpoint{3.157427in}{0.826667in}}%
\pgfpathlineto{\pgfqpoint{3.084606in}{0.882243in}}%
\pgfpathlineto{\pgfqpoint{3.004444in}{0.946259in}}%
\pgfpathlineto{\pgfqpoint{2.988048in}{0.960728in}}%
\pgfpathlineto{\pgfqpoint{2.964364in}{0.979390in}}%
\pgfpathlineto{\pgfqpoint{2.924228in}{1.013333in}}%
\pgfpathlineto{\pgfqpoint{2.904111in}{1.031878in}}%
\pgfpathlineto{\pgfqpoint{2.881314in}{1.050667in}}%
\pgfpathlineto{\pgfqpoint{2.798234in}{1.125333in}}%
\pgfpathlineto{\pgfqpoint{2.718527in}{1.200000in}}%
\pgfpathlineto{\pgfqpoint{2.641881in}{1.274667in}}%
\pgfpathlineto{\pgfqpoint{2.563556in}{1.354282in}}%
\pgfpathlineto{\pgfqpoint{2.547676in}{1.371875in}}%
\pgfpathlineto{\pgfqpoint{2.523475in}{1.396366in}}%
\pgfpathlineto{\pgfqpoint{2.443313in}{1.483186in}}%
\pgfpathlineto{\pgfqpoint{2.417947in}{1.512373in}}%
\pgfpathlineto{\pgfqpoint{2.396141in}{1.536000in}}%
\pgfpathlineto{\pgfqpoint{2.381856in}{1.553422in}}%
\pgfpathlineto{\pgfqpoint{2.363152in}{1.573649in}}%
\pgfpathlineto{\pgfqpoint{2.282990in}{1.669279in}}%
\pgfpathlineto{\pgfqpoint{2.239694in}{1.722667in}}%
\pgfpathlineto{\pgfqpoint{2.224673in}{1.743014in}}%
\pgfpathlineto{\pgfqpoint{2.202828in}{1.769801in}}%
\pgfpathlineto{\pgfqpoint{2.181817in}{1.797333in}}%
\pgfpathlineto{\pgfqpoint{2.153621in}{1.834667in}}%
\pgfpathlineto{\pgfqpoint{2.141557in}{1.852262in}}%
\pgfpathlineto{\pgfqpoint{2.122667in}{1.876694in}}%
\pgfpathlineto{\pgfqpoint{2.073358in}{1.946667in}}%
\pgfpathlineto{\pgfqpoint{2.061942in}{1.964772in}}%
\pgfpathlineto{\pgfqpoint{2.042505in}{1.992149in}}%
\pgfpathlineto{\pgfqpoint{1.994894in}{2.065681in}}%
\pgfpathlineto{\pgfqpoint{1.954143in}{2.133333in}}%
\pgfpathlineto{\pgfqpoint{1.943677in}{2.153280in}}%
\pgfpathlineto{\pgfqpoint{1.932815in}{2.170667in}}%
\pgfpathlineto{\pgfqpoint{1.902377in}{2.226522in}}%
\pgfpathlineto{\pgfqpoint{1.868064in}{2.295817in}}%
\pgfpathlineto{\pgfqpoint{1.840960in}{2.357333in}}%
\pgfpathlineto{\pgfqpoint{1.801986in}{2.469365in}}%
\pgfpathlineto{\pgfqpoint{1.793181in}{2.506667in}}%
\pgfpathlineto{\pgfqpoint{1.789554in}{2.532388in}}%
\pgfpathlineto{\pgfqpoint{1.786646in}{2.544000in}}%
\pgfpathlineto{\pgfqpoint{1.784209in}{2.560590in}}%
\pgfpathlineto{\pgfqpoint{1.783065in}{2.581333in}}%
\pgfpathlineto{\pgfqpoint{1.784288in}{2.602150in}}%
\pgfpathlineto{\pgfqpoint{1.783415in}{2.618667in}}%
\pgfpathlineto{\pgfqpoint{1.784608in}{2.634886in}}%
\pgfpathlineto{\pgfqpoint{1.789150in}{2.656000in}}%
\pgfpathlineto{\pgfqpoint{1.802662in}{2.693333in}}%
\pgfpathlineto{\pgfqpoint{1.842101in}{2.736820in}}%
\pgfpathlineto{\pgfqpoint{1.853798in}{2.741562in}}%
\pgfpathlineto{\pgfqpoint{1.882182in}{2.755253in}}%
\pgfpathlineto{\pgfqpoint{1.892467in}{2.758420in}}%
\pgfpathlineto{\pgfqpoint{1.922263in}{2.762881in}}%
\pgfpathlineto{\pgfqpoint{1.967484in}{2.763212in}}%
\pgfpathlineto{\pgfqpoint{2.002424in}{2.758909in}}%
\pgfpathlineto{\pgfqpoint{2.026443in}{2.753039in}}%
\pgfpathlineto{\pgfqpoint{2.042505in}{2.750856in}}%
\pgfpathlineto{\pgfqpoint{2.082586in}{2.740127in}}%
\pgfpathlineto{\pgfqpoint{2.128898in}{2.724862in}}%
\pgfpathlineto{\pgfqpoint{2.205704in}{2.693333in}}%
\pgfpathlineto{\pgfqpoint{2.229708in}{2.681037in}}%
\pgfpathlineto{\pgfqpoint{2.242909in}{2.675666in}}%
\pgfpathlineto{\pgfqpoint{2.256078in}{2.668266in}}%
\pgfpathlineto{\pgfqpoint{2.282990in}{2.655868in}}%
\pgfpathlineto{\pgfqpoint{2.307538in}{2.641532in}}%
\pgfpathlineto{\pgfqpoint{2.323071in}{2.634070in}}%
\pgfpathlineto{\pgfqpoint{2.383657in}{2.599567in}}%
\pgfpathlineto{\pgfqpoint{2.443313in}{2.562716in}}%
\pgfpathlineto{\pgfqpoint{2.483394in}{2.536972in}}%
\pgfpathlineto{\pgfqpoint{2.501724in}{2.523741in}}%
\pgfpathlineto{\pgfqpoint{2.528614in}{2.506667in}}%
\pgfpathlineto{\pgfqpoint{2.603636in}{2.453686in}}%
\pgfpathlineto{\pgfqpoint{2.683798in}{2.394209in}}%
\pgfpathlineto{\pgfqpoint{2.704751in}{2.376850in}}%
\pgfpathlineto{\pgfqpoint{2.730967in}{2.357333in}}%
\pgfpathlineto{\pgfqpoint{2.748233in}{2.342685in}}%
\pgfpathlineto{\pgfqpoint{2.777457in}{2.320000in}}%
\pgfpathlineto{\pgfqpoint{2.804040in}{2.298273in}}%
\pgfpathlineto{\pgfqpoint{2.884202in}{2.230632in}}%
\pgfpathlineto{\pgfqpoint{2.910196in}{2.208000in}}%
\pgfpathlineto{\pgfqpoint{2.964364in}{2.159971in}}%
\pgfpathlineto{\pgfqpoint{2.978409in}{2.146416in}}%
\pgfpathlineto{\pgfqpoint{3.004444in}{2.123484in}}%
\pgfpathlineto{\pgfqpoint{3.018787in}{2.109359in}}%
\pgfpathlineto{\pgfqpoint{3.044525in}{2.086213in}}%
\pgfpathlineto{\pgfqpoint{3.058752in}{2.071918in}}%
\pgfpathlineto{\pgfqpoint{3.084606in}{2.048148in}}%
\pgfpathlineto{\pgfqpoint{3.164768in}{1.969594in}}%
\pgfpathlineto{\pgfqpoint{3.244929in}{1.887736in}}%
\pgfpathlineto{\pgfqpoint{3.259918in}{1.872000in}}%
\pgfpathlineto{\pgfqpoint{3.329813in}{1.797333in}}%
\pgfpathlineto{\pgfqpoint{3.365172in}{1.758461in}}%
\pgfpathlineto{\pgfqpoint{3.382024in}{1.738363in}}%
\pgfpathlineto{\pgfqpoint{3.405253in}{1.713117in}}%
\pgfpathlineto{\pgfqpoint{3.492796in}{1.610667in}}%
\pgfpathlineto{\pgfqpoint{3.553530in}{1.536000in}}%
\pgfpathlineto{\pgfqpoint{3.611924in}{1.461333in}}%
\pgfpathlineto{\pgfqpoint{3.645737in}{1.416420in}}%
\pgfpathlineto{\pgfqpoint{3.694416in}{1.349333in}}%
\pgfpathlineto{\pgfqpoint{3.706312in}{1.331089in}}%
\pgfpathlineto{\pgfqpoint{3.725899in}{1.304348in}}%
\pgfpathlineto{\pgfqpoint{3.753176in}{1.262741in}}%
\pgfpathlineto{\pgfqpoint{3.778473in}{1.225697in}}%
\pgfpathlineto{\pgfqpoint{3.817850in}{1.162667in}}%
\pgfpathlineto{\pgfqpoint{3.827421in}{1.145230in}}%
\pgfpathlineto{\pgfqpoint{3.846141in}{1.115292in}}%
\pgfpathlineto{\pgfqpoint{3.886222in}{1.043522in}}%
\pgfpathlineto{\pgfqpoint{3.901812in}{1.013333in}}%
\pgfpathlineto{\pgfqpoint{3.926303in}{0.963704in}}%
\pgfpathlineto{\pgfqpoint{3.970529in}{0.860139in}}%
\pgfpathlineto{\pgfqpoint{3.981925in}{0.826667in}}%
\pgfpathlineto{\pgfqpoint{3.986621in}{0.808183in}}%
\pgfpathlineto{\pgfqpoint{3.993444in}{0.789333in}}%
\pgfpathlineto{\pgfqpoint{4.003471in}{0.749211in}}%
\pgfpathlineto{\pgfqpoint{4.010982in}{0.710459in}}%
\pgfpathlineto{\pgfqpoint{4.014128in}{0.677333in}}%
\pgfpathlineto{\pgfqpoint{4.014366in}{0.640000in}}%
\pgfpathlineto{\pgfqpoint{4.009248in}{0.600074in}}%
\pgfpathlineto{\pgfqpoint{4.006465in}{0.591430in}}%
\pgfpathlineto{\pgfqpoint{3.996355in}{0.565333in}}%
\pgfpathlineto{\pgfqpoint{3.969178in}{0.528000in}}%
\pgfpathlineto{\pgfqpoint{4.022604in}{0.528000in}}%
\pgfpathlineto{\pgfqpoint{4.031536in}{0.551352in}}%
\pgfpathlineto{\pgfqpoint{4.040092in}{0.571344in}}%
\pgfpathlineto{\pgfqpoint{4.045088in}{0.602667in}}%
\pgfpathlineto{\pgfqpoint{4.046098in}{0.640000in}}%
\pgfpathlineto{\pgfqpoint{4.042505in}{0.681097in}}%
\pgfpathlineto{\pgfqpoint{4.034373in}{0.726005in}}%
\pgfpathlineto{\pgfqpoint{4.017061in}{0.789333in}}%
\pgfpathlineto{\pgfqpoint{4.014149in}{0.796491in}}%
\pgfpathlineto{\pgfqpoint{4.004008in}{0.828955in}}%
\pgfpathlineto{\pgfqpoint{3.990281in}{0.864000in}}%
\pgfpathlineto{\pgfqpoint{3.982982in}{0.879461in}}%
\pgfpathlineto{\pgfqpoint{3.974866in}{0.901333in}}%
\pgfpathlineto{\pgfqpoint{3.958055in}{0.938667in}}%
\pgfpathlineto{\pgfqpoint{3.947547in}{0.958455in}}%
\pgfpathlineto{\pgfqpoint{3.939987in}{0.976000in}}%
\pgfpathlineto{\pgfqpoint{3.921132in}{1.013333in}}%
\pgfpathlineto{\pgfqpoint{3.908825in}{1.034386in}}%
\pgfpathlineto{\pgfqpoint{3.895548in}{1.059353in}}%
\pgfpathlineto{\pgfqpoint{3.873271in}{1.100064in}}%
\pgfpathlineto{\pgfqpoint{3.835551in}{1.162667in}}%
\pgfpathlineto{\pgfqpoint{3.824134in}{1.179501in}}%
\pgfpathlineto{\pgfqpoint{3.806061in}{1.209523in}}%
\pgfpathlineto{\pgfqpoint{3.763173in}{1.274667in}}%
\pgfpathlineto{\pgfqpoint{3.748023in}{1.295274in}}%
\pgfpathlineto{\pgfqpoint{3.725899in}{1.328362in}}%
\pgfpathlineto{\pgfqpoint{3.700457in}{1.362969in}}%
\pgfpathlineto{\pgfqpoint{3.681061in}{1.391097in}}%
\pgfpathlineto{\pgfqpoint{3.599383in}{1.498667in}}%
\pgfpathlineto{\pgfqpoint{3.584533in}{1.516324in}}%
\pgfpathlineto{\pgfqpoint{3.565576in}{1.541416in}}%
\pgfpathlineto{\pgfqpoint{3.477622in}{1.648000in}}%
\pgfpathlineto{\pgfqpoint{3.405253in}{1.731558in}}%
\pgfpathlineto{\pgfqpoint{3.372917in}{1.767214in}}%
\pgfpathlineto{\pgfqpoint{3.345974in}{1.797333in}}%
\pgfpathlineto{\pgfqpoint{3.276575in}{1.872000in}}%
\pgfpathlineto{\pgfqpoint{3.261225in}{1.887179in}}%
\pgfpathlineto{\pgfqpoint{3.240953in}{1.909333in}}%
\pgfpathlineto{\pgfqpoint{3.223264in}{1.926486in}}%
\pgfpathlineto{\pgfqpoint{3.202628in}{1.948735in}}%
\pgfpathlineto{\pgfqpoint{3.124687in}{2.026220in}}%
\pgfpathlineto{\pgfqpoint{3.044525in}{2.102881in}}%
\pgfpathlineto{\pgfqpoint{3.027679in}{2.117642in}}%
\pgfpathlineto{\pgfqpoint{3.004444in}{2.140089in}}%
\pgfpathlineto{\pgfqpoint{2.987356in}{2.154750in}}%
\pgfpathlineto{\pgfqpoint{2.964364in}{2.176561in}}%
\pgfpathlineto{\pgfqpoint{2.946614in}{2.191467in}}%
\pgfpathlineto{\pgfqpoint{2.924283in}{2.212305in}}%
\pgfpathlineto{\pgfqpoint{2.905445in}{2.227787in}}%
\pgfpathlineto{\pgfqpoint{2.884202in}{2.247330in}}%
\pgfpathlineto{\pgfqpoint{2.797871in}{2.320000in}}%
\pgfpathlineto{\pgfqpoint{2.723879in}{2.379516in}}%
\pgfpathlineto{\pgfqpoint{2.704457in}{2.394667in}}%
\pgfpathlineto{\pgfqpoint{2.643717in}{2.441205in}}%
\pgfpathlineto{\pgfqpoint{2.603636in}{2.470996in}}%
\pgfpathlineto{\pgfqpoint{2.581385in}{2.485940in}}%
\pgfpathlineto{\pgfqpoint{2.553506in}{2.506667in}}%
\pgfpathlineto{\pgfqpoint{2.483394in}{2.554623in}}%
\pgfpathlineto{\pgfqpoint{2.442814in}{2.581333in}}%
\pgfpathlineto{\pgfqpoint{2.418489in}{2.595544in}}%
\pgfpathlineto{\pgfqpoint{2.403232in}{2.605945in}}%
\pgfpathlineto{\pgfqpoint{2.318383in}{2.656000in}}%
\pgfpathlineto{\pgfqpoint{2.242909in}{2.696140in}}%
\pgfpathlineto{\pgfqpoint{2.217742in}{2.707225in}}%
\pgfpathlineto{\pgfqpoint{2.202828in}{2.715279in}}%
\pgfpathlineto{\pgfqpoint{2.191147in}{2.719786in}}%
\pgfpathlineto{\pgfqpoint{2.162747in}{2.733440in}}%
\pgfpathlineto{\pgfqpoint{2.136089in}{2.743169in}}%
\pgfpathlineto{\pgfqpoint{2.122667in}{2.749423in}}%
\pgfpathlineto{\pgfqpoint{2.082586in}{2.764070in}}%
\pgfpathlineto{\pgfqpoint{2.029918in}{2.779725in}}%
\pgfpathlineto{\pgfqpoint{2.002424in}{2.785784in}}%
\pgfpathlineto{\pgfqpoint{1.983672in}{2.787867in}}%
\pgfpathlineto{\pgfqpoint{1.962343in}{2.792480in}}%
\pgfpathlineto{\pgfqpoint{1.922263in}{2.795314in}}%
\pgfpathlineto{\pgfqpoint{1.909734in}{2.793663in}}%
\pgfpathlineto{\pgfqpoint{1.882182in}{2.792820in}}%
\pgfpathlineto{\pgfqpoint{1.862390in}{2.786898in}}%
\pgfpathlineto{\pgfqpoint{1.842101in}{2.782685in}}%
\pgfpathlineto{\pgfqpoint{1.806600in}{2.763734in}}%
\pgfpathlineto{\pgfqpoint{1.786964in}{2.744691in}}%
\pgfpathlineto{\pgfqpoint{1.777496in}{2.730667in}}%
\pgfpathlineto{\pgfqpoint{1.760039in}{2.693333in}}%
\pgfpathlineto{\pgfqpoint{1.753003in}{2.656000in}}%
\pgfpathlineto{\pgfqpoint{1.753348in}{2.647997in}}%
\pgfpathlineto{\pgfqpoint{1.751457in}{2.618667in}}%
\pgfpathlineto{\pgfqpoint{1.753955in}{2.581333in}}%
\pgfpathlineto{\pgfqpoint{1.755402in}{2.575244in}}%
\pgfpathlineto{\pgfqpoint{1.761939in}{2.533209in}}%
\pgfpathlineto{\pgfqpoint{1.778771in}{2.469333in}}%
\pgfpathlineto{\pgfqpoint{1.791026in}{2.432000in}}%
\pgfpathlineto{\pgfqpoint{1.794231in}{2.424745in}}%
\pgfpathlineto{\pgfqpoint{1.804731in}{2.394667in}}%
\pgfpathlineto{\pgfqpoint{1.815876in}{2.370239in}}%
\pgfpathlineto{\pgfqpoint{1.820431in}{2.357333in}}%
\pgfpathlineto{\pgfqpoint{1.842101in}{2.308991in}}%
\pgfpathlineto{\pgfqpoint{1.873681in}{2.245333in}}%
\pgfpathlineto{\pgfqpoint{1.882182in}{2.229278in}}%
\pgfpathlineto{\pgfqpoint{1.922263in}{2.157082in}}%
\pgfpathlineto{\pgfqpoint{1.936225in}{2.133333in}}%
\pgfpathlineto{\pgfqpoint{1.962343in}{2.089938in}}%
\pgfpathlineto{\pgfqpoint{2.011548in}{2.012835in}}%
\pgfpathlineto{\pgfqpoint{2.082586in}{1.909207in}}%
\pgfpathlineto{\pgfqpoint{2.165191in}{1.797333in}}%
\pgfpathlineto{\pgfqpoint{2.181510in}{1.777476in}}%
\pgfpathlineto{\pgfqpoint{2.202828in}{1.748926in}}%
\pgfpathlineto{\pgfqpoint{2.223697in}{1.722667in}}%
\pgfpathlineto{\pgfqpoint{2.284107in}{1.648000in}}%
\pgfpathlineto{\pgfqpoint{2.301761in}{1.628151in}}%
\pgfpathlineto{\pgfqpoint{2.323071in}{1.601789in}}%
\pgfpathlineto{\pgfqpoint{2.413069in}{1.498667in}}%
\pgfpathlineto{\pgfqpoint{2.427313in}{1.483763in}}%
\pgfpathlineto{\pgfqpoint{2.446648in}{1.461333in}}%
\pgfpathlineto{\pgfqpoint{2.523475in}{1.378866in}}%
\pgfpathlineto{\pgfqpoint{2.538874in}{1.363677in}}%
\pgfpathlineto{\pgfqpoint{2.563556in}{1.337158in}}%
\pgfpathlineto{\pgfqpoint{2.643717in}{1.256164in}}%
\pgfpathlineto{\pgfqpoint{2.662878in}{1.237333in}}%
\pgfpathlineto{\pgfqpoint{2.740471in}{1.162667in}}%
\pgfpathlineto{\pgfqpoint{2.763960in}{1.140579in}}%
\pgfpathlineto{\pgfqpoint{2.844121in}{1.067348in}}%
\pgfpathlineto{\pgfqpoint{2.924283in}{0.997089in}}%
\pgfpathlineto{\pgfqpoint{3.004444in}{0.929731in}}%
\pgfpathlineto{\pgfqpoint{3.086267in}{0.864000in}}%
\pgfpathlineto{\pgfqpoint{3.107860in}{0.848326in}}%
\pgfpathlineto{\pgfqpoint{3.135016in}{0.826667in}}%
\pgfpathlineto{\pgfqpoint{3.164768in}{0.804341in}}%
\pgfpathlineto{\pgfqpoint{3.244929in}{0.746255in}}%
\pgfpathlineto{\pgfqpoint{3.264989in}{0.733352in}}%
\pgfpathlineto{\pgfqpoint{3.290759in}{0.714667in}}%
\pgfpathlineto{\pgfqpoint{3.311301in}{0.701822in}}%
\pgfpathlineto{\pgfqpoint{3.325091in}{0.691863in}}%
\pgfpathlineto{\pgfqpoint{3.365172in}{0.665942in}}%
\pgfpathlineto{\pgfqpoint{3.382436in}{0.656081in}}%
\pgfpathlineto{\pgfqpoint{3.406574in}{0.640000in}}%
\pgfpathlineto{\pgfqpoint{3.485414in}{0.594293in}}%
\pgfpathlineto{\pgfqpoint{3.505772in}{0.584296in}}%
\pgfpathlineto{\pgfqpoint{3.539910in}{0.565333in}}%
\pgfpathlineto{\pgfqpoint{3.605657in}{0.533246in}}%
\pgfpathlineto{\pgfqpoint{3.617754in}{0.528000in}}%
\pgfpathlineto{\pgfqpoint{3.645737in}{0.528000in}}%
\pgfpathlineto{\pgfqpoint{3.645737in}{0.528000in}}%
\pgfusepath{fill}%
\end{pgfscope}%
\begin{pgfscope}%
\pgfpathrectangle{\pgfqpoint{0.800000in}{0.528000in}}{\pgfqpoint{3.968000in}{3.696000in}}%
\pgfusepath{clip}%
\pgfsetbuttcap%
\pgfsetroundjoin%
\definecolor{currentfill}{rgb}{0.273809,0.031497,0.358853}%
\pgfsetfillcolor{currentfill}%
\pgfsetlinewidth{0.000000pt}%
\definecolor{currentstroke}{rgb}{0.000000,0.000000,0.000000}%
\pgfsetstrokecolor{currentstroke}%
\pgfsetdash{}{0pt}%
\pgfpathmoveto{\pgfqpoint{3.617754in}{0.528000in}}%
\pgfpathlineto{\pgfqpoint{3.539910in}{0.565333in}}%
\pgfpathlineto{\pgfqpoint{3.525495in}{0.572645in}}%
\pgfpathlineto{\pgfqpoint{3.505772in}{0.584296in}}%
\pgfpathlineto{\pgfqpoint{3.470702in}{0.602667in}}%
\pgfpathlineto{\pgfqpoint{3.445333in}{0.617172in}}%
\pgfpathlineto{\pgfqpoint{3.405253in}{0.640782in}}%
\pgfpathlineto{\pgfqpoint{3.382436in}{0.656081in}}%
\pgfpathlineto{\pgfqpoint{3.365172in}{0.665942in}}%
\pgfpathlineto{\pgfqpoint{3.285010in}{0.718502in}}%
\pgfpathlineto{\pgfqpoint{3.264989in}{0.733352in}}%
\pgfpathlineto{\pgfqpoint{3.236877in}{0.752000in}}%
\pgfpathlineto{\pgfqpoint{3.164768in}{0.804341in}}%
\pgfpathlineto{\pgfqpoint{3.151916in}{0.814696in}}%
\pgfpathlineto{\pgfqpoint{3.124687in}{0.834450in}}%
\pgfpathlineto{\pgfqpoint{3.107860in}{0.848326in}}%
\pgfpathlineto{\pgfqpoint{3.075666in}{0.872328in}}%
\pgfpathlineto{\pgfqpoint{2.993664in}{0.938667in}}%
\pgfpathlineto{\pgfqpoint{2.957294in}{0.969415in}}%
\pgfpathlineto{\pgfqpoint{2.924283in}{0.997089in}}%
\pgfpathlineto{\pgfqpoint{2.844121in}{1.067348in}}%
\pgfpathlineto{\pgfqpoint{2.812778in}{1.096139in}}%
\pgfpathlineto{\pgfqpoint{2.804040in}{1.103587in}}%
\pgfpathlineto{\pgfqpoint{2.723879in}{1.178333in}}%
\pgfpathlineto{\pgfqpoint{2.701296in}{1.200000in}}%
\pgfpathlineto{\pgfqpoint{2.625180in}{1.274667in}}%
\pgfpathlineto{\pgfqpoint{2.551820in}{1.349333in}}%
\pgfpathlineto{\pgfqpoint{2.538874in}{1.363677in}}%
\pgfpathlineto{\pgfqpoint{2.516098in}{1.386667in}}%
\pgfpathlineto{\pgfqpoint{2.443313in}{1.464991in}}%
\pgfpathlineto{\pgfqpoint{2.427313in}{1.483763in}}%
\pgfpathlineto{\pgfqpoint{2.403232in}{1.509669in}}%
\pgfpathlineto{\pgfqpoint{2.315546in}{1.610667in}}%
\pgfpathlineto{\pgfqpoint{2.301761in}{1.628151in}}%
\pgfpathlineto{\pgfqpoint{2.282990in}{1.649351in}}%
\pgfpathlineto{\pgfqpoint{2.194155in}{1.760000in}}%
\pgfpathlineto{\pgfqpoint{2.181510in}{1.777476in}}%
\pgfpathlineto{\pgfqpoint{2.162747in}{1.800552in}}%
\pgfpathlineto{\pgfqpoint{2.082315in}{1.909586in}}%
\pgfpathlineto{\pgfqpoint{2.006006in}{2.021333in}}%
\pgfpathlineto{\pgfqpoint{2.002424in}{2.026862in}}%
\pgfpathlineto{\pgfqpoint{1.953415in}{2.104317in}}%
\pgfpathlineto{\pgfqpoint{1.914406in}{2.170667in}}%
\pgfpathlineto{\pgfqpoint{1.904003in}{2.190992in}}%
\pgfpathlineto{\pgfqpoint{1.893641in}{2.208000in}}%
\pgfpathlineto{\pgfqpoint{1.865551in}{2.260824in}}%
\pgfpathlineto{\pgfqpoint{1.832761in}{2.328700in}}%
\pgfpathlineto{\pgfqpoint{1.802020in}{2.401932in}}%
\pgfpathlineto{\pgfqpoint{1.769756in}{2.499386in}}%
\pgfpathlineto{\pgfqpoint{1.759516in}{2.544000in}}%
\pgfpathlineto{\pgfqpoint{1.752791in}{2.589854in}}%
\pgfpathlineto{\pgfqpoint{1.751457in}{2.618667in}}%
\pgfpathlineto{\pgfqpoint{1.753618in}{2.663751in}}%
\pgfpathlineto{\pgfqpoint{1.761939in}{2.698699in}}%
\pgfpathlineto{\pgfqpoint{1.777496in}{2.730667in}}%
\pgfpathlineto{\pgfqpoint{1.786964in}{2.744691in}}%
\pgfpathlineto{\pgfqpoint{1.806600in}{2.763734in}}%
\pgfpathlineto{\pgfqpoint{1.813597in}{2.768000in}}%
\pgfpathlineto{\pgfqpoint{1.842101in}{2.782685in}}%
\pgfpathlineto{\pgfqpoint{1.862390in}{2.786898in}}%
\pgfpathlineto{\pgfqpoint{1.882182in}{2.792820in}}%
\pgfpathlineto{\pgfqpoint{1.909734in}{2.793663in}}%
\pgfpathlineto{\pgfqpoint{1.922263in}{2.795314in}}%
\pgfpathlineto{\pgfqpoint{1.933044in}{2.795291in}}%
\pgfpathlineto{\pgfqpoint{1.962343in}{2.792480in}}%
\pgfpathlineto{\pgfqpoint{1.983672in}{2.787867in}}%
\pgfpathlineto{\pgfqpoint{2.002424in}{2.785784in}}%
\pgfpathlineto{\pgfqpoint{2.042505in}{2.776199in}}%
\pgfpathlineto{\pgfqpoint{2.048755in}{2.773822in}}%
\pgfpathlineto{\pgfqpoint{2.089076in}{2.761955in}}%
\pgfpathlineto{\pgfqpoint{2.122667in}{2.749423in}}%
\pgfpathlineto{\pgfqpoint{2.136089in}{2.743169in}}%
\pgfpathlineto{\pgfqpoint{2.168895in}{2.730667in}}%
\pgfpathlineto{\pgfqpoint{2.217742in}{2.707225in}}%
\pgfpathlineto{\pgfqpoint{2.248295in}{2.693333in}}%
\pgfpathlineto{\pgfqpoint{2.269935in}{2.681173in}}%
\pgfpathlineto{\pgfqpoint{2.282990in}{2.675164in}}%
\pgfpathlineto{\pgfqpoint{2.329973in}{2.649571in}}%
\pgfpathlineto{\pgfqpoint{2.403232in}{2.605945in}}%
\pgfpathlineto{\pgfqpoint{2.418489in}{2.595544in}}%
\pgfpathlineto{\pgfqpoint{2.444390in}{2.580331in}}%
\pgfpathlineto{\pgfqpoint{2.523475in}{2.527500in}}%
\pgfpathlineto{\pgfqpoint{2.563556in}{2.499664in}}%
\pgfpathlineto{\pgfqpoint{2.581385in}{2.485940in}}%
\pgfpathlineto{\pgfqpoint{2.605882in}{2.469333in}}%
\pgfpathlineto{\pgfqpoint{2.704457in}{2.394667in}}%
\pgfpathlineto{\pgfqpoint{2.723879in}{2.379516in}}%
\pgfpathlineto{\pgfqpoint{2.804040in}{2.314958in}}%
\pgfpathlineto{\pgfqpoint{2.886495in}{2.245333in}}%
\pgfpathlineto{\pgfqpoint{2.905445in}{2.227787in}}%
\pgfpathlineto{\pgfqpoint{2.929127in}{2.208000in}}%
\pgfpathlineto{\pgfqpoint{2.946614in}{2.191467in}}%
\pgfpathlineto{\pgfqpoint{2.970863in}{2.170667in}}%
\pgfpathlineto{\pgfqpoint{2.987356in}{2.154750in}}%
\pgfpathlineto{\pgfqpoint{3.011746in}{2.133333in}}%
\pgfpathlineto{\pgfqpoint{3.027679in}{2.117642in}}%
\pgfpathlineto{\pgfqpoint{3.051816in}{2.096000in}}%
\pgfpathlineto{\pgfqpoint{3.067590in}{2.080150in}}%
\pgfpathlineto{\pgfqpoint{3.091112in}{2.058667in}}%
\pgfpathlineto{\pgfqpoint{3.167516in}{1.984000in}}%
\pgfpathlineto{\pgfqpoint{3.244929in}{1.905223in}}%
\pgfpathlineto{\pgfqpoint{3.261225in}{1.887179in}}%
\pgfpathlineto{\pgfqpoint{3.285010in}{1.863105in}}%
\pgfpathlineto{\pgfqpoint{3.365172in}{1.776284in}}%
\pgfpathlineto{\pgfqpoint{3.390965in}{1.746692in}}%
\pgfpathlineto{\pgfqpoint{3.413092in}{1.722667in}}%
\pgfpathlineto{\pgfqpoint{3.485414in}{1.638789in}}%
\pgfpathlineto{\pgfqpoint{3.508854in}{1.610667in}}%
\pgfpathlineto{\pgfqpoint{3.569918in}{1.536000in}}%
\pgfpathlineto{\pgfqpoint{3.584533in}{1.516324in}}%
\pgfpathlineto{\pgfqpoint{3.605657in}{1.490611in}}%
\pgfpathlineto{\pgfqpoint{3.628119in}{1.461333in}}%
\pgfpathlineto{\pgfqpoint{3.685818in}{1.384569in}}%
\pgfpathlineto{\pgfqpoint{3.716629in}{1.340699in}}%
\pgfpathlineto{\pgfqpoint{3.737350in}{1.312000in}}%
\pgfpathlineto{\pgfqpoint{3.748023in}{1.295274in}}%
\pgfpathlineto{\pgfqpoint{3.765980in}{1.270477in}}%
\pgfpathlineto{\pgfqpoint{3.787838in}{1.237333in}}%
\pgfpathlineto{\pgfqpoint{3.812204in}{1.200000in}}%
\pgfpathlineto{\pgfqpoint{3.824134in}{1.179501in}}%
\pgfpathlineto{\pgfqpoint{3.846141in}{1.145345in}}%
\pgfpathlineto{\pgfqpoint{3.886222in}{1.077185in}}%
\pgfpathlineto{\pgfqpoint{3.908825in}{1.034386in}}%
\pgfpathlineto{\pgfqpoint{3.926303in}{1.003225in}}%
\pgfpathlineto{\pgfqpoint{3.939987in}{0.976000in}}%
\pgfpathlineto{\pgfqpoint{3.947547in}{0.958455in}}%
\pgfpathlineto{\pgfqpoint{3.960245in}{0.932949in}}%
\pgfpathlineto{\pgfqpoint{3.974866in}{0.901333in}}%
\pgfpathlineto{\pgfqpoint{3.982982in}{0.879461in}}%
\pgfpathlineto{\pgfqpoint{3.990281in}{0.864000in}}%
\pgfpathlineto{\pgfqpoint{4.006465in}{0.821655in}}%
\pgfpathlineto{\pgfqpoint{4.027705in}{0.752000in}}%
\pgfpathlineto{\pgfqpoint{4.036482in}{0.714667in}}%
\pgfpathlineto{\pgfqpoint{4.042815in}{0.673859in}}%
\pgfpathlineto{\pgfqpoint{4.046033in}{0.639523in}}%
\pgfpathlineto{\pgfqpoint{4.045088in}{0.602667in}}%
\pgfpathlineto{\pgfqpoint{4.038156in}{0.565333in}}%
\pgfpathlineto{\pgfqpoint{4.031536in}{0.551352in}}%
\pgfpathlineto{\pgfqpoint{4.022604in}{0.528000in}}%
\pgfpathlineto{\pgfqpoint{4.065167in}{0.528000in}}%
\pgfpathlineto{\pgfqpoint{4.070098in}{0.543396in}}%
\pgfpathlineto{\pgfqpoint{4.073675in}{0.565333in}}%
\pgfpathlineto{\pgfqpoint{4.074546in}{0.591415in}}%
\pgfpathlineto{\pgfqpoint{4.076322in}{0.612264in}}%
\pgfpathlineto{\pgfqpoint{4.073064in}{0.652633in}}%
\pgfpathlineto{\pgfqpoint{4.068895in}{0.677333in}}%
\pgfpathlineto{\pgfqpoint{4.064462in}{0.694021in}}%
\pgfpathlineto{\pgfqpoint{4.061126in}{0.714667in}}%
\pgfpathlineto{\pgfqpoint{4.057663in}{0.725023in}}%
\pgfpathlineto{\pgfqpoint{4.050086in}{0.755298in}}%
\pgfpathlineto{\pgfqpoint{4.039452in}{0.789333in}}%
\pgfpathlineto{\pgfqpoint{4.030386in}{0.811615in}}%
\pgfpathlineto{\pgfqpoint{4.025790in}{0.826667in}}%
\pgfpathlineto{\pgfqpoint{4.020013in}{0.839286in}}%
\pgfpathlineto{\pgfqpoint{4.006465in}{0.874761in}}%
\pgfpathlineto{\pgfqpoint{3.994741in}{0.901333in}}%
\pgfpathlineto{\pgfqpoint{3.977365in}{0.938667in}}%
\pgfpathlineto{\pgfqpoint{3.973622in}{0.945409in}}%
\pgfpathlineto{\pgfqpoint{3.952442in}{0.988986in}}%
\pgfpathlineto{\pgfqpoint{3.919086in}{1.050667in}}%
\pgfpathlineto{\pgfqpoint{3.907075in}{1.070090in}}%
\pgfpathlineto{\pgfqpoint{3.893384in}{1.094671in}}%
\pgfpathlineto{\pgfqpoint{3.875597in}{1.125333in}}%
\pgfpathlineto{\pgfqpoint{3.846141in}{1.173152in}}%
\pgfpathlineto{\pgfqpoint{3.828933in}{1.200000in}}%
\pgfpathlineto{\pgfqpoint{3.804828in}{1.237333in}}%
\pgfpathlineto{\pgfqpoint{3.789240in}{1.258999in}}%
\pgfpathlineto{\pgfqpoint{3.765980in}{1.294326in}}%
\pgfpathlineto{\pgfqpoint{3.742190in}{1.327174in}}%
\pgfpathlineto{\pgfqpoint{3.725899in}{1.351510in}}%
\pgfpathlineto{\pgfqpoint{3.710097in}{1.371948in}}%
\pgfpathlineto{\pgfqpoint{3.685818in}{1.406071in}}%
\pgfpathlineto{\pgfqpoint{3.644314in}{1.461333in}}%
\pgfpathlineto{\pgfqpoint{3.627523in}{1.481701in}}%
\pgfpathlineto{\pgfqpoint{3.605657in}{1.510755in}}%
\pgfpathlineto{\pgfqpoint{3.576665in}{1.546329in}}%
\pgfpathlineto{\pgfqpoint{3.555437in}{1.573333in}}%
\pgfpathlineto{\pgfqpoint{3.542007in}{1.588713in}}%
\pgfpathlineto{\pgfqpoint{3.524912in}{1.610667in}}%
\pgfpathlineto{\pgfqpoint{3.507008in}{1.630780in}}%
\pgfpathlineto{\pgfqpoint{3.485414in}{1.657373in}}%
\pgfpathlineto{\pgfqpoint{3.395730in}{1.760000in}}%
\pgfpathlineto{\pgfqpoint{3.381345in}{1.775065in}}%
\pgfpathlineto{\pgfqpoint{3.344476in}{1.816610in}}%
\pgfpathlineto{\pgfqpoint{3.285010in}{1.880130in}}%
\pgfpathlineto{\pgfqpoint{3.257010in}{1.909333in}}%
\pgfpathlineto{\pgfqpoint{3.183781in}{1.984000in}}%
\pgfpathlineto{\pgfqpoint{3.164768in}{2.003019in}}%
\pgfpathlineto{\pgfqpoint{3.084606in}{2.081066in}}%
\pgfpathlineto{\pgfqpoint{3.068847in}{2.096000in}}%
\pgfpathlineto{\pgfqpoint{2.988445in}{2.170667in}}%
\pgfpathlineto{\pgfqpoint{2.924283in}{2.228187in}}%
\pgfpathlineto{\pgfqpoint{2.893811in}{2.254284in}}%
\pgfpathlineto{\pgfqpoint{2.884202in}{2.263149in}}%
\pgfpathlineto{\pgfqpoint{2.804040in}{2.330952in}}%
\pgfpathlineto{\pgfqpoint{2.714887in}{2.403042in}}%
\pgfpathlineto{\pgfqpoint{2.627956in}{2.469333in}}%
\pgfpathlineto{\pgfqpoint{2.591580in}{2.495437in}}%
\pgfpathlineto{\pgfqpoint{2.563556in}{2.516372in}}%
\pgfpathlineto{\pgfqpoint{2.483394in}{2.571817in}}%
\pgfpathlineto{\pgfqpoint{2.403232in}{2.623812in}}%
\pgfpathlineto{\pgfqpoint{2.381738in}{2.635979in}}%
\pgfpathlineto{\pgfqpoint{2.349970in}{2.656000in}}%
\pgfpathlineto{\pgfqpoint{2.323071in}{2.671681in}}%
\pgfpathlineto{\pgfqpoint{2.280282in}{2.695855in}}%
\pgfpathlineto{\pgfqpoint{2.192057in}{2.740699in}}%
\pgfpathlineto{\pgfqpoint{2.116454in}{2.773787in}}%
\pgfpathlineto{\pgfqpoint{2.082586in}{2.786352in}}%
\pgfpathlineto{\pgfqpoint{2.066900in}{2.790723in}}%
\pgfpathlineto{\pgfqpoint{2.042505in}{2.799999in}}%
\pgfpathlineto{\pgfqpoint{2.002424in}{2.811191in}}%
\pgfpathlineto{\pgfqpoint{1.974077in}{2.816263in}}%
\pgfpathlineto{\pgfqpoint{1.962343in}{2.819549in}}%
\pgfpathlineto{\pgfqpoint{1.939635in}{2.821515in}}%
\pgfpathlineto{\pgfqpoint{1.922263in}{2.824791in}}%
\pgfpathlineto{\pgfqpoint{1.899699in}{2.826351in}}%
\pgfpathlineto{\pgfqpoint{1.861376in}{2.824713in}}%
\pgfpathlineto{\pgfqpoint{1.842101in}{2.821247in}}%
\pgfpathlineto{\pgfqpoint{1.796345in}{2.805333in}}%
\pgfpathlineto{\pgfqpoint{1.776036in}{2.792203in}}%
\pgfpathlineto{\pgfqpoint{1.761939in}{2.778555in}}%
\pgfpathlineto{\pgfqpoint{1.753225in}{2.768000in}}%
\pgfpathlineto{\pgfqpoint{1.746253in}{2.753389in}}%
\pgfpathlineto{\pgfqpoint{1.730381in}{2.722728in}}%
\pgfpathlineto{\pgfqpoint{1.723419in}{2.691880in}}%
\pgfpathlineto{\pgfqpoint{1.720853in}{2.655064in}}%
\pgfpathlineto{\pgfqpoint{1.722161in}{2.618385in}}%
\pgfpathlineto{\pgfqpoint{1.727221in}{2.581333in}}%
\pgfpathlineto{\pgfqpoint{1.733512in}{2.554855in}}%
\pgfpathlineto{\pgfqpoint{1.734942in}{2.544000in}}%
\pgfpathlineto{\pgfqpoint{1.738402in}{2.528591in}}%
\pgfpathlineto{\pgfqpoint{1.756116in}{2.469333in}}%
\pgfpathlineto{\pgfqpoint{1.757775in}{2.465455in}}%
\pgfpathlineto{\pgfqpoint{1.769449in}{2.432000in}}%
\pgfpathlineto{\pgfqpoint{1.778945in}{2.410506in}}%
\pgfpathlineto{\pgfqpoint{1.784353in}{2.394667in}}%
\pgfpathlineto{\pgfqpoint{1.789879in}{2.383358in}}%
\pgfpathlineto{\pgfqpoint{1.802020in}{2.353123in}}%
\pgfpathlineto{\pgfqpoint{1.842101in}{2.270457in}}%
\pgfpathlineto{\pgfqpoint{1.882182in}{2.196097in}}%
\pgfpathlineto{\pgfqpoint{1.896778in}{2.170667in}}%
\pgfpathlineto{\pgfqpoint{1.922263in}{2.127300in}}%
\pgfpathlineto{\pgfqpoint{1.968779in}{2.052673in}}%
\pgfpathlineto{\pgfqpoint{2.014496in}{1.984000in}}%
\pgfpathlineto{\pgfqpoint{2.025902in}{1.968535in}}%
\pgfpathlineto{\pgfqpoint{2.042505in}{1.943091in}}%
\pgfpathlineto{\pgfqpoint{2.093554in}{1.872000in}}%
\pgfpathlineto{\pgfqpoint{2.105889in}{1.856372in}}%
\pgfpathlineto{\pgfqpoint{2.122667in}{1.832436in}}%
\pgfpathlineto{\pgfqpoint{2.207700in}{1.722667in}}%
\pgfpathlineto{\pgfqpoint{2.268621in}{1.648000in}}%
\pgfpathlineto{\pgfqpoint{2.331678in}{1.573333in}}%
\pgfpathlineto{\pgfqpoint{2.403232in}{1.491861in}}%
\pgfpathlineto{\pgfqpoint{2.430922in}{1.461333in}}%
\pgfpathlineto{\pgfqpoint{2.500133in}{1.386667in}}%
\pgfpathlineto{\pgfqpoint{2.530071in}{1.355478in}}%
\pgfpathlineto{\pgfqpoint{2.535618in}{1.349333in}}%
\pgfpathlineto{\pgfqpoint{2.549236in}{1.335995in}}%
\pgfpathlineto{\pgfqpoint{2.571723in}{1.312000in}}%
\pgfpathlineto{\pgfqpoint{2.587431in}{1.296905in}}%
\pgfpathlineto{\pgfqpoint{2.608479in}{1.274667in}}%
\pgfpathlineto{\pgfqpoint{2.684065in}{1.200000in}}%
\pgfpathlineto{\pgfqpoint{2.704242in}{1.181709in}}%
\pgfpathlineto{\pgfqpoint{2.723879in}{1.161852in}}%
\pgfpathlineto{\pgfqpoint{2.743942in}{1.144022in}}%
\pgfpathlineto{\pgfqpoint{2.763960in}{1.124187in}}%
\pgfpathlineto{\pgfqpoint{2.844517in}{1.050667in}}%
\pgfpathlineto{\pgfqpoint{2.930034in}{0.976000in}}%
\pgfpathlineto{\pgfqpoint{2.948486in}{0.961211in}}%
\pgfpathlineto{\pgfqpoint{2.974281in}{0.938667in}}%
\pgfpathlineto{\pgfqpoint{3.004444in}{0.913664in}}%
\pgfpathlineto{\pgfqpoint{3.084606in}{0.849266in}}%
\pgfpathlineto{\pgfqpoint{3.098242in}{0.839368in}}%
\pgfpathlineto{\pgfqpoint{3.124687in}{0.818108in}}%
\pgfpathlineto{\pgfqpoint{3.142234in}{0.805678in}}%
\pgfpathlineto{\pgfqpoint{3.164768in}{0.787631in}}%
\pgfpathlineto{\pgfqpoint{3.186729in}{0.772456in}}%
\pgfpathlineto{\pgfqpoint{3.213501in}{0.752000in}}%
\pgfpathlineto{\pgfqpoint{3.285010in}{0.701653in}}%
\pgfpathlineto{\pgfqpoint{3.300632in}{0.691885in}}%
\pgfpathlineto{\pgfqpoint{3.335913in}{0.667253in}}%
\pgfpathlineto{\pgfqpoint{3.405253in}{0.623243in}}%
\pgfpathlineto{\pgfqpoint{3.456685in}{0.592093in}}%
\pgfpathlineto{\pgfqpoint{3.525495in}{0.553669in}}%
\pgfpathlineto{\pgfqpoint{3.575087in}{0.528000in}}%
\pgfpathlineto{\pgfqpoint{3.605657in}{0.528000in}}%
\pgfpathlineto{\pgfqpoint{3.605657in}{0.528000in}}%
\pgfusepath{fill}%
\end{pgfscope}%
\begin{pgfscope}%
\pgfpathrectangle{\pgfqpoint{0.800000in}{0.528000in}}{\pgfqpoint{3.968000in}{3.696000in}}%
\pgfusepath{clip}%
\pgfsetbuttcap%
\pgfsetroundjoin%
\definecolor{currentfill}{rgb}{0.273809,0.031497,0.358853}%
\pgfsetfillcolor{currentfill}%
\pgfsetlinewidth{0.000000pt}%
\definecolor{currentstroke}{rgb}{0.000000,0.000000,0.000000}%
\pgfsetstrokecolor{currentstroke}%
\pgfsetdash{}{0pt}%
\pgfpathmoveto{\pgfqpoint{3.575087in}{0.528000in}}%
\pgfpathlineto{\pgfqpoint{3.554733in}{0.538100in}}%
\pgfpathlineto{\pgfqpoint{3.485414in}{0.575742in}}%
\pgfpathlineto{\pgfqpoint{3.438978in}{0.602667in}}%
\pgfpathlineto{\pgfqpoint{3.365172in}{0.648378in}}%
\pgfpathlineto{\pgfqpoint{3.320781in}{0.677333in}}%
\pgfpathlineto{\pgfqpoint{3.300632in}{0.691885in}}%
\pgfpathlineto{\pgfqpoint{3.266290in}{0.714667in}}%
\pgfpathlineto{\pgfqpoint{3.244929in}{0.729578in}}%
\pgfpathlineto{\pgfqpoint{3.162521in}{0.789333in}}%
\pgfpathlineto{\pgfqpoint{3.142234in}{0.805678in}}%
\pgfpathlineto{\pgfqpoint{3.113645in}{0.826667in}}%
\pgfpathlineto{\pgfqpoint{3.084606in}{0.849266in}}%
\pgfpathlineto{\pgfqpoint{3.004444in}{0.913664in}}%
\pgfpathlineto{\pgfqpoint{2.990655in}{0.925822in}}%
\pgfpathlineto{\pgfqpoint{2.964364in}{0.946920in}}%
\pgfpathlineto{\pgfqpoint{2.948486in}{0.961211in}}%
\pgfpathlineto{\pgfqpoint{2.924283in}{0.980892in}}%
\pgfpathlineto{\pgfqpoint{2.844121in}{1.051018in}}%
\pgfpathlineto{\pgfqpoint{2.824530in}{1.069752in}}%
\pgfpathlineto{\pgfqpoint{2.803211in}{1.088000in}}%
\pgfpathlineto{\pgfqpoint{2.723025in}{1.162667in}}%
\pgfpathlineto{\pgfqpoint{2.704242in}{1.181709in}}%
\pgfpathlineto{\pgfqpoint{2.683798in}{1.200257in}}%
\pgfpathlineto{\pgfqpoint{2.603636in}{1.279521in}}%
\pgfpathlineto{\pgfqpoint{2.587431in}{1.296905in}}%
\pgfpathlineto{\pgfqpoint{2.563556in}{1.320348in}}%
\pgfpathlineto{\pgfqpoint{2.530071in}{1.355478in}}%
\pgfpathlineto{\pgfqpoint{2.511402in}{1.375421in}}%
\pgfpathlineto{\pgfqpoint{2.483394in}{1.404442in}}%
\pgfpathlineto{\pgfqpoint{2.465243in}{1.424000in}}%
\pgfpathlineto{\pgfqpoint{2.397146in}{1.498667in}}%
\pgfpathlineto{\pgfqpoint{2.323071in}{1.583342in}}%
\pgfpathlineto{\pgfqpoint{2.237788in}{1.685333in}}%
\pgfpathlineto{\pgfqpoint{2.202828in}{1.728797in}}%
\pgfpathlineto{\pgfqpoint{2.172092in}{1.768704in}}%
\pgfpathlineto{\pgfqpoint{2.149492in}{1.797333in}}%
\pgfpathlineto{\pgfqpoint{2.120986in}{1.834667in}}%
\pgfpathlineto{\pgfqpoint{2.105889in}{1.856372in}}%
\pgfpathlineto{\pgfqpoint{2.082586in}{1.887058in}}%
\pgfpathlineto{\pgfqpoint{2.066599in}{1.909333in}}%
\pgfpathlineto{\pgfqpoint{2.039990in}{1.946667in}}%
\pgfpathlineto{\pgfqpoint{2.025902in}{1.968535in}}%
\pgfpathlineto{\pgfqpoint{2.002424in}{2.001883in}}%
\pgfpathlineto{\pgfqpoint{1.989490in}{2.021333in}}%
\pgfpathlineto{\pgfqpoint{1.962343in}{2.062794in}}%
\pgfpathlineto{\pgfqpoint{1.941643in}{2.096000in}}%
\pgfpathlineto{\pgfqpoint{1.913826in}{2.141192in}}%
\pgfpathlineto{\pgfqpoint{1.875459in}{2.208000in}}%
\pgfpathlineto{\pgfqpoint{1.864871in}{2.229209in}}%
\pgfpathlineto{\pgfqpoint{1.855287in}{2.245333in}}%
\pgfpathlineto{\pgfqpoint{1.835800in}{2.282667in}}%
\pgfpathlineto{\pgfqpoint{1.825858in}{2.304870in}}%
\pgfpathlineto{\pgfqpoint{1.817629in}{2.320000in}}%
\pgfpathlineto{\pgfqpoint{1.800071in}{2.357333in}}%
\pgfpathlineto{\pgfqpoint{1.778945in}{2.410506in}}%
\pgfpathlineto{\pgfqpoint{1.769449in}{2.432000in}}%
\pgfpathlineto{\pgfqpoint{1.756116in}{2.469333in}}%
\pgfpathlineto{\pgfqpoint{1.749165in}{2.494768in}}%
\pgfpathlineto{\pgfqpoint{1.744716in}{2.506667in}}%
\pgfpathlineto{\pgfqpoint{1.734942in}{2.544000in}}%
\pgfpathlineto{\pgfqpoint{1.733512in}{2.554855in}}%
\pgfpathlineto{\pgfqpoint{1.727221in}{2.581333in}}%
\pgfpathlineto{\pgfqpoint{1.721859in}{2.626088in}}%
\pgfpathlineto{\pgfqpoint{1.720690in}{2.657088in}}%
\pgfpathlineto{\pgfqpoint{1.723735in}{2.693333in}}%
\pgfpathlineto{\pgfqpoint{1.730381in}{2.722728in}}%
\pgfpathlineto{\pgfqpoint{1.733574in}{2.730667in}}%
\pgfpathlineto{\pgfqpoint{1.753225in}{2.768000in}}%
\pgfpathlineto{\pgfqpoint{1.761939in}{2.778555in}}%
\pgfpathlineto{\pgfqpoint{1.776036in}{2.792203in}}%
\pgfpathlineto{\pgfqpoint{1.802020in}{2.808440in}}%
\pgfpathlineto{\pgfqpoint{1.806532in}{2.809536in}}%
\pgfpathlineto{\pgfqpoint{1.842101in}{2.821247in}}%
\pgfpathlineto{\pgfqpoint{1.861376in}{2.824713in}}%
\pgfpathlineto{\pgfqpoint{1.899699in}{2.826351in}}%
\pgfpathlineto{\pgfqpoint{1.922263in}{2.824791in}}%
\pgfpathlineto{\pgfqpoint{1.939635in}{2.821515in}}%
\pgfpathlineto{\pgfqpoint{1.962343in}{2.819549in}}%
\pgfpathlineto{\pgfqpoint{1.974077in}{2.816263in}}%
\pgfpathlineto{\pgfqpoint{2.002424in}{2.811191in}}%
\pgfpathlineto{\pgfqpoint{2.042505in}{2.799999in}}%
\pgfpathlineto{\pgfqpoint{2.066900in}{2.790723in}}%
\pgfpathlineto{\pgfqpoint{2.082586in}{2.786352in}}%
\pgfpathlineto{\pgfqpoint{2.130181in}{2.768000in}}%
\pgfpathlineto{\pgfqpoint{2.162747in}{2.753922in}}%
\pgfpathlineto{\pgfqpoint{2.212456in}{2.730667in}}%
\pgfpathlineto{\pgfqpoint{2.242909in}{2.715334in}}%
\pgfpathlineto{\pgfqpoint{2.284835in}{2.693333in}}%
\pgfpathlineto{\pgfqpoint{2.363152in}{2.648281in}}%
\pgfpathlineto{\pgfqpoint{2.381738in}{2.635979in}}%
\pgfpathlineto{\pgfqpoint{2.411300in}{2.618667in}}%
\pgfpathlineto{\pgfqpoint{2.453550in}{2.590868in}}%
\pgfpathlineto{\pgfqpoint{2.483394in}{2.571817in}}%
\pgfpathlineto{\pgfqpoint{2.524499in}{2.544000in}}%
\pgfpathlineto{\pgfqpoint{2.603636in}{2.487339in}}%
\pgfpathlineto{\pgfqpoint{2.627956in}{2.469333in}}%
\pgfpathlineto{\pgfqpoint{2.683798in}{2.427193in}}%
\pgfpathlineto{\pgfqpoint{2.763960in}{2.363809in}}%
\pgfpathlineto{\pgfqpoint{2.771884in}{2.357333in}}%
\pgfpathlineto{\pgfqpoint{2.861402in}{2.282667in}}%
\pgfpathlineto{\pgfqpoint{2.946998in}{2.208000in}}%
\pgfpathlineto{\pgfqpoint{2.964364in}{2.192506in}}%
\pgfpathlineto{\pgfqpoint{3.044525in}{2.118955in}}%
\pgfpathlineto{\pgfqpoint{3.124687in}{2.042424in}}%
\pgfpathlineto{\pgfqpoint{3.146180in}{2.021333in}}%
\pgfpathlineto{\pgfqpoint{3.220715in}{1.946667in}}%
\pgfpathlineto{\pgfqpoint{3.244929in}{1.921881in}}%
\pgfpathlineto{\pgfqpoint{3.327792in}{1.834667in}}%
\pgfpathlineto{\pgfqpoint{3.365172in}{1.794004in}}%
\pgfpathlineto{\pgfqpoint{3.381345in}{1.775065in}}%
\pgfpathlineto{\pgfqpoint{3.405253in}{1.749357in}}%
\pgfpathlineto{\pgfqpoint{3.493375in}{1.648000in}}%
\pgfpathlineto{\pgfqpoint{3.507008in}{1.630780in}}%
\pgfpathlineto{\pgfqpoint{3.525495in}{1.609964in}}%
\pgfpathlineto{\pgfqpoint{3.542007in}{1.588713in}}%
\pgfpathlineto{\pgfqpoint{3.565576in}{1.560869in}}%
\pgfpathlineto{\pgfqpoint{3.593752in}{1.524911in}}%
\pgfpathlineto{\pgfqpoint{3.615165in}{1.498667in}}%
\pgfpathlineto{\pgfqpoint{3.627523in}{1.481701in}}%
\pgfpathlineto{\pgfqpoint{3.645737in}{1.459468in}}%
\pgfpathlineto{\pgfqpoint{3.672408in}{1.424000in}}%
\pgfpathlineto{\pgfqpoint{3.727472in}{1.349333in}}%
\pgfpathlineto{\pgfqpoint{3.742190in}{1.327174in}}%
\pgfpathlineto{\pgfqpoint{3.765980in}{1.294326in}}%
\pgfpathlineto{\pgfqpoint{3.806061in}{1.235452in}}%
\pgfpathlineto{\pgfqpoint{3.862057in}{1.147842in}}%
\pgfpathlineto{\pgfqpoint{3.897709in}{1.088000in}}%
\pgfpathlineto{\pgfqpoint{3.907075in}{1.070090in}}%
\pgfpathlineto{\pgfqpoint{3.926303in}{1.037536in}}%
\pgfpathlineto{\pgfqpoint{3.939388in}{1.013333in}}%
\pgfpathlineto{\pgfqpoint{3.966384in}{0.961108in}}%
\pgfpathlineto{\pgfqpoint{4.011127in}{0.864000in}}%
\pgfpathlineto{\pgfqpoint{4.020013in}{0.839286in}}%
\pgfpathlineto{\pgfqpoint{4.025790in}{0.826667in}}%
\pgfpathlineto{\pgfqpoint{4.030386in}{0.811615in}}%
\pgfpathlineto{\pgfqpoint{4.039452in}{0.789333in}}%
\pgfpathlineto{\pgfqpoint{4.051382in}{0.752000in}}%
\pgfpathlineto{\pgfqpoint{4.068895in}{0.677333in}}%
\pgfpathlineto{\pgfqpoint{4.074144in}{0.640000in}}%
\pgfpathlineto{\pgfqpoint{4.076108in}{0.602667in}}%
\pgfpathlineto{\pgfqpoint{4.074546in}{0.591415in}}%
\pgfpathlineto{\pgfqpoint{4.073675in}{0.565333in}}%
\pgfpathlineto{\pgfqpoint{4.070098in}{0.543396in}}%
\pgfpathlineto{\pgfqpoint{4.065167in}{0.528000in}}%
\pgfpathlineto{\pgfqpoint{4.101427in}{0.528000in}}%
\pgfpathlineto{\pgfqpoint{4.102010in}{0.542330in}}%
\pgfpathlineto{\pgfqpoint{4.105288in}{0.565333in}}%
\pgfpathlineto{\pgfqpoint{4.105855in}{0.584756in}}%
\pgfpathlineto{\pgfqpoint{4.104606in}{0.602667in}}%
\pgfpathlineto{\pgfqpoint{4.102113in}{0.617092in}}%
\pgfpathlineto{\pgfqpoint{4.100504in}{0.640000in}}%
\pgfpathlineto{\pgfqpoint{4.097855in}{0.650459in}}%
\pgfpathlineto{\pgfqpoint{4.093761in}{0.677333in}}%
\pgfpathlineto{\pgfqpoint{4.084769in}{0.714667in}}%
\pgfpathlineto{\pgfqpoint{4.046545in}{0.826885in}}%
\pgfpathlineto{\pgfqpoint{4.030645in}{0.864000in}}%
\pgfpathlineto{\pgfqpoint{4.006465in}{0.917120in}}%
\pgfpathlineto{\pgfqpoint{3.996021in}{0.938667in}}%
\pgfpathlineto{\pgfqpoint{3.985919in}{0.956863in}}%
\pgfpathlineto{\pgfqpoint{3.977096in}{0.976000in}}%
\pgfpathlineto{\pgfqpoint{3.957256in}{1.013333in}}%
\pgfpathlineto{\pgfqpoint{3.946115in}{1.031787in}}%
\pgfpathlineto{\pgfqpoint{3.926303in}{1.068374in}}%
\pgfpathlineto{\pgfqpoint{3.904073in}{1.104627in}}%
\pgfpathlineto{\pgfqpoint{3.886222in}{1.135579in}}%
\pgfpathlineto{\pgfqpoint{3.844938in}{1.201121in}}%
\pgfpathlineto{\pgfqpoint{3.769757in}{1.312000in}}%
\pgfpathlineto{\pgfqpoint{3.751766in}{1.336094in}}%
\pgfpathlineto{\pgfqpoint{3.742973in}{1.349333in}}%
\pgfpathlineto{\pgfqpoint{3.685818in}{1.427151in}}%
\pgfpathlineto{\pgfqpoint{3.659566in}{1.461333in}}%
\pgfpathlineto{\pgfqpoint{3.601112in}{1.536000in}}%
\pgfpathlineto{\pgfqpoint{3.585339in}{1.554409in}}%
\pgfpathlineto{\pgfqpoint{3.565576in}{1.579844in}}%
\pgfpathlineto{\pgfqpoint{3.533433in}{1.618061in}}%
\pgfpathlineto{\pgfqpoint{3.508628in}{1.648000in}}%
\pgfpathlineto{\pgfqpoint{3.440561in}{1.727112in}}%
\pgfpathlineto{\pgfqpoint{3.365172in}{1.810821in}}%
\pgfpathlineto{\pgfqpoint{3.272802in}{1.909333in}}%
\pgfpathlineto{\pgfqpoint{3.200047in}{1.984000in}}%
\pgfpathlineto{\pgfqpoint{3.182544in}{2.000557in}}%
\pgfpathlineto{\pgfqpoint{3.162692in}{2.021333in}}%
\pgfpathlineto{\pgfqpoint{3.143657in}{2.039003in}}%
\pgfpathlineto{\pgfqpoint{3.123447in}{2.059822in}}%
\pgfpathlineto{\pgfqpoint{3.044525in}{2.134931in}}%
\pgfpathlineto{\pgfqpoint{2.953848in}{2.217795in}}%
\pgfpathlineto{\pgfqpoint{2.879881in}{2.282667in}}%
\pgfpathlineto{\pgfqpoint{2.791014in}{2.357333in}}%
\pgfpathlineto{\pgfqpoint{2.763960in}{2.379442in}}%
\pgfpathlineto{\pgfqpoint{2.683798in}{2.442987in}}%
\pgfpathlineto{\pgfqpoint{2.667816in}{2.454447in}}%
\pgfpathlineto{\pgfqpoint{2.643717in}{2.473752in}}%
\pgfpathlineto{\pgfqpoint{2.623542in}{2.487875in}}%
\pgfpathlineto{\pgfqpoint{2.599522in}{2.506667in}}%
\pgfpathlineto{\pgfqpoint{2.523475in}{2.560933in}}%
\pgfpathlineto{\pgfqpoint{2.483394in}{2.588543in}}%
\pgfpathlineto{\pgfqpoint{2.463964in}{2.600568in}}%
\pgfpathlineto{\pgfqpoint{2.438030in}{2.618667in}}%
\pgfpathlineto{\pgfqpoint{2.363152in}{2.665752in}}%
\pgfpathlineto{\pgfqpoint{2.344509in}{2.675968in}}%
\pgfpathlineto{\pgfqpoint{2.316721in}{2.693333in}}%
\pgfpathlineto{\pgfqpoint{2.282990in}{2.712349in}}%
\pgfpathlineto{\pgfqpoint{2.233982in}{2.738981in}}%
\pgfpathlineto{\pgfqpoint{2.162747in}{2.773944in}}%
\pgfpathlineto{\pgfqpoint{2.139194in}{2.783394in}}%
\pgfpathlineto{\pgfqpoint{2.122667in}{2.791609in}}%
\pgfpathlineto{\pgfqpoint{2.077506in}{2.810065in}}%
\pgfpathlineto{\pgfqpoint{2.014393in}{2.831518in}}%
\pgfpathlineto{\pgfqpoint{1.958823in}{2.845946in}}%
\pgfpathlineto{\pgfqpoint{1.910149in}{2.853950in}}%
\pgfpathlineto{\pgfqpoint{1.882182in}{2.855869in}}%
\pgfpathlineto{\pgfqpoint{1.854804in}{2.854499in}}%
\pgfpathlineto{\pgfqpoint{1.842101in}{2.855101in}}%
\pgfpathlineto{\pgfqpoint{1.797306in}{2.847058in}}%
\pgfpathlineto{\pgfqpoint{1.786588in}{2.842667in}}%
\pgfpathlineto{\pgfqpoint{1.761939in}{2.830759in}}%
\pgfpathlineto{\pgfqpoint{1.746186in}{2.820007in}}%
\pgfpathlineto{\pgfqpoint{1.731612in}{2.805333in}}%
\pgfpathlineto{\pgfqpoint{1.721859in}{2.792823in}}%
\pgfpathlineto{\pgfqpoint{1.712006in}{2.777177in}}%
\pgfpathlineto{\pgfqpoint{1.701111in}{2.749992in}}%
\pgfpathlineto{\pgfqpoint{1.696712in}{2.730667in}}%
\pgfpathlineto{\pgfqpoint{1.694276in}{2.704975in}}%
\pgfpathlineto{\pgfqpoint{1.691308in}{2.684457in}}%
\pgfpathlineto{\pgfqpoint{1.692184in}{2.656000in}}%
\pgfpathlineto{\pgfqpoint{1.695578in}{2.631521in}}%
\pgfpathlineto{\pgfqpoint{1.696024in}{2.618667in}}%
\pgfpathlineto{\pgfqpoint{1.700506in}{2.598778in}}%
\pgfpathlineto{\pgfqpoint{1.702648in}{2.581333in}}%
\pgfpathlineto{\pgfqpoint{1.711444in}{2.544000in}}%
\pgfpathlineto{\pgfqpoint{1.714063in}{2.536739in}}%
\pgfpathlineto{\pgfqpoint{1.721978in}{2.506667in}}%
\pgfpathlineto{\pgfqpoint{1.764204in}{2.394667in}}%
\pgfpathlineto{\pgfqpoint{1.776033in}{2.370461in}}%
\pgfpathlineto{\pgfqpoint{1.781174in}{2.357333in}}%
\pgfpathlineto{\pgfqpoint{1.802020in}{2.313339in}}%
\pgfpathlineto{\pgfqpoint{1.817712in}{2.282667in}}%
\pgfpathlineto{\pgfqpoint{1.842101in}{2.236570in}}%
\pgfpathlineto{\pgfqpoint{1.858113in}{2.208000in}}%
\pgfpathlineto{\pgfqpoint{1.882182in}{2.165935in}}%
\pgfpathlineto{\pgfqpoint{1.924721in}{2.096000in}}%
\pgfpathlineto{\pgfqpoint{1.939351in}{2.074584in}}%
\pgfpathlineto{\pgfqpoint{1.962343in}{2.037791in}}%
\pgfpathlineto{\pgfqpoint{2.002424in}{1.977972in}}%
\pgfpathlineto{\pgfqpoint{2.031852in}{1.936743in}}%
\pgfpathlineto{\pgfqpoint{2.050702in}{1.909333in}}%
\pgfpathlineto{\pgfqpoint{2.064057in}{1.892075in}}%
\pgfpathlineto{\pgfqpoint{2.082586in}{1.865457in}}%
\pgfpathlineto{\pgfqpoint{2.105666in}{1.834667in}}%
\pgfpathlineto{\pgfqpoint{2.162747in}{1.759855in}}%
\pgfpathlineto{\pgfqpoint{2.180259in}{1.738978in}}%
\pgfpathlineto{\pgfqpoint{2.202828in}{1.709657in}}%
\pgfpathlineto{\pgfqpoint{2.289095in}{1.604980in}}%
\pgfpathlineto{\pgfqpoint{2.363152in}{1.519672in}}%
\pgfpathlineto{\pgfqpoint{2.449510in}{1.424000in}}%
\pgfpathlineto{\pgfqpoint{2.465711in}{1.407529in}}%
\pgfpathlineto{\pgfqpoint{2.484169in}{1.386667in}}%
\pgfpathlineto{\pgfqpoint{2.503144in}{1.367730in}}%
\pgfpathlineto{\pgfqpoint{2.523475in}{1.345357in}}%
\pgfpathlineto{\pgfqpoint{2.540931in}{1.328260in}}%
\pgfpathlineto{\pgfqpoint{2.563556in}{1.304043in}}%
\pgfpathlineto{\pgfqpoint{2.592557in}{1.274667in}}%
\pgfpathlineto{\pgfqpoint{2.667981in}{1.200000in}}%
\pgfpathlineto{\pgfqpoint{2.723879in}{1.146292in}}%
\pgfpathlineto{\pgfqpoint{2.755283in}{1.117252in}}%
\pgfpathlineto{\pgfqpoint{2.786398in}{1.088000in}}%
\pgfpathlineto{\pgfqpoint{2.815880in}{1.061694in}}%
\pgfpathlineto{\pgfqpoint{2.844121in}{1.035617in}}%
\pgfpathlineto{\pgfqpoint{2.924283in}{0.965346in}}%
\pgfpathlineto{\pgfqpoint{2.939679in}{0.953007in}}%
\pgfpathlineto{\pgfqpoint{2.964364in}{0.931241in}}%
\pgfpathlineto{\pgfqpoint{2.981794in}{0.917568in}}%
\pgfpathlineto{\pgfqpoint{3.004444in}{0.897811in}}%
\pgfpathlineto{\pgfqpoint{3.024346in}{0.882537in}}%
\pgfpathlineto{\pgfqpoint{3.045924in}{0.864000in}}%
\pgfpathlineto{\pgfqpoint{3.141647in}{0.789333in}}%
\pgfpathlineto{\pgfqpoint{3.204848in}{0.742079in}}%
\pgfpathlineto{\pgfqpoint{3.285010in}{0.685045in}}%
\pgfpathlineto{\pgfqpoint{3.296353in}{0.677333in}}%
\pgfpathlineto{\pgfqpoint{3.365172in}{0.631385in}}%
\pgfpathlineto{\pgfqpoint{3.410315in}{0.602667in}}%
\pgfpathlineto{\pgfqpoint{3.485414in}{0.557724in}}%
\pgfpathlineto{\pgfqpoint{3.539023in}{0.528000in}}%
\pgfpathlineto{\pgfqpoint{3.565576in}{0.528000in}}%
\pgfpathlineto{\pgfqpoint{3.565576in}{0.528000in}}%
\pgfusepath{fill}%
\end{pgfscope}%
\begin{pgfscope}%
\pgfpathrectangle{\pgfqpoint{0.800000in}{0.528000in}}{\pgfqpoint{3.968000in}{3.696000in}}%
\pgfusepath{clip}%
\pgfsetbuttcap%
\pgfsetroundjoin%
\definecolor{currentfill}{rgb}{0.273809,0.031497,0.358853}%
\pgfsetfillcolor{currentfill}%
\pgfsetlinewidth{0.000000pt}%
\definecolor{currentstroke}{rgb}{0.000000,0.000000,0.000000}%
\pgfsetstrokecolor{currentstroke}%
\pgfsetdash{}{0pt}%
\pgfpathmoveto{\pgfqpoint{3.539023in}{0.528000in}}%
\pgfpathlineto{\pgfqpoint{3.485414in}{0.557724in}}%
\pgfpathlineto{\pgfqpoint{3.472492in}{0.565333in}}%
\pgfpathlineto{\pgfqpoint{3.395135in}{0.612090in}}%
\pgfpathlineto{\pgfqpoint{3.325091in}{0.657876in}}%
\pgfpathlineto{\pgfqpoint{3.242631in}{0.714667in}}%
\pgfpathlineto{\pgfqpoint{3.164768in}{0.771819in}}%
\pgfpathlineto{\pgfqpoint{3.084606in}{0.833328in}}%
\pgfpathlineto{\pgfqpoint{3.044525in}{0.865113in}}%
\pgfpathlineto{\pgfqpoint{3.024346in}{0.882537in}}%
\pgfpathlineto{\pgfqpoint{3.000207in}{0.901333in}}%
\pgfpathlineto{\pgfqpoint{2.981794in}{0.917568in}}%
\pgfpathlineto{\pgfqpoint{2.955611in}{0.938667in}}%
\pgfpathlineto{\pgfqpoint{2.939679in}{0.953007in}}%
\pgfpathlineto{\pgfqpoint{2.911976in}{0.976000in}}%
\pgfpathlineto{\pgfqpoint{2.827412in}{1.050667in}}%
\pgfpathlineto{\pgfqpoint{2.795516in}{1.080060in}}%
\pgfpathlineto{\pgfqpoint{2.763960in}{1.108687in}}%
\pgfpathlineto{\pgfqpoint{2.735394in}{1.136059in}}%
\pgfpathlineto{\pgfqpoint{2.706718in}{1.162667in}}%
\pgfpathlineto{\pgfqpoint{2.683798in}{1.184621in}}%
\pgfpathlineto{\pgfqpoint{2.592557in}{1.274667in}}%
\pgfpathlineto{\pgfqpoint{2.579079in}{1.289126in}}%
\pgfpathlineto{\pgfqpoint{2.555811in}{1.312000in}}%
\pgfpathlineto{\pgfqpoint{2.540931in}{1.328260in}}%
\pgfpathlineto{\pgfqpoint{2.519674in}{1.349333in}}%
\pgfpathlineto{\pgfqpoint{2.503144in}{1.367730in}}%
\pgfpathlineto{\pgfqpoint{2.483394in}{1.387489in}}%
\pgfpathlineto{\pgfqpoint{2.465711in}{1.407529in}}%
\pgfpathlineto{\pgfqpoint{2.443313in}{1.430706in}}%
\pgfpathlineto{\pgfqpoint{2.363152in}{1.519672in}}%
\pgfpathlineto{\pgfqpoint{2.348814in}{1.536000in}}%
\pgfpathlineto{\pgfqpoint{2.282990in}{1.612190in}}%
\pgfpathlineto{\pgfqpoint{2.192404in}{1.722667in}}%
\pgfpathlineto{\pgfqpoint{2.180259in}{1.738978in}}%
\pgfpathlineto{\pgfqpoint{2.162326in}{1.760393in}}%
\pgfpathlineto{\pgfqpoint{2.077749in}{1.872000in}}%
\pgfpathlineto{\pgfqpoint{2.064057in}{1.892075in}}%
\pgfpathlineto{\pgfqpoint{2.042505in}{1.920819in}}%
\pgfpathlineto{\pgfqpoint{2.016110in}{1.959414in}}%
\pgfpathlineto{\pgfqpoint{1.998268in}{1.984000in}}%
\pgfpathlineto{\pgfqpoint{1.984967in}{2.005072in}}%
\pgfpathlineto{\pgfqpoint{1.962343in}{2.037791in}}%
\pgfpathlineto{\pgfqpoint{1.922263in}{2.099970in}}%
\pgfpathlineto{\pgfqpoint{1.875889in}{2.176529in}}%
\pgfpathlineto{\pgfqpoint{1.832238in}{2.254521in}}%
\pgfpathlineto{\pgfqpoint{1.795761in}{2.325830in}}%
\pgfpathlineto{\pgfqpoint{1.761939in}{2.400157in}}%
\pgfpathlineto{\pgfqpoint{1.749043in}{2.432000in}}%
\pgfpathlineto{\pgfqpoint{1.742648in}{2.451364in}}%
\pgfpathlineto{\pgfqpoint{1.734959in}{2.469333in}}%
\pgfpathlineto{\pgfqpoint{1.732301in}{2.479060in}}%
\pgfpathlineto{\pgfqpoint{1.721859in}{2.507081in}}%
\pgfpathlineto{\pgfqpoint{1.702648in}{2.581333in}}%
\pgfpathlineto{\pgfqpoint{1.700506in}{2.598778in}}%
\pgfpathlineto{\pgfqpoint{1.696024in}{2.618667in}}%
\pgfpathlineto{\pgfqpoint{1.695578in}{2.631521in}}%
\pgfpathlineto{\pgfqpoint{1.692184in}{2.656000in}}%
\pgfpathlineto{\pgfqpoint{1.691992in}{2.693333in}}%
\pgfpathlineto{\pgfqpoint{1.694276in}{2.704975in}}%
\pgfpathlineto{\pgfqpoint{1.696712in}{2.730667in}}%
\pgfpathlineto{\pgfqpoint{1.701111in}{2.749992in}}%
\pgfpathlineto{\pgfqpoint{1.712006in}{2.777177in}}%
\pgfpathlineto{\pgfqpoint{1.721859in}{2.792823in}}%
\pgfpathlineto{\pgfqpoint{1.731612in}{2.805333in}}%
\pgfpathlineto{\pgfqpoint{1.746186in}{2.820007in}}%
\pgfpathlineto{\pgfqpoint{1.761939in}{2.830759in}}%
\pgfpathlineto{\pgfqpoint{1.797306in}{2.847058in}}%
\pgfpathlineto{\pgfqpoint{1.808639in}{2.848832in}}%
\pgfpathlineto{\pgfqpoint{1.842101in}{2.855101in}}%
\pgfpathlineto{\pgfqpoint{1.854804in}{2.854499in}}%
\pgfpathlineto{\pgfqpoint{1.882182in}{2.855869in}}%
\pgfpathlineto{\pgfqpoint{1.922263in}{2.852173in}}%
\pgfpathlineto{\pgfqpoint{1.971973in}{2.842667in}}%
\pgfpathlineto{\pgfqpoint{2.042505in}{2.822276in}}%
\pgfpathlineto{\pgfqpoint{2.089300in}{2.805333in}}%
\pgfpathlineto{\pgfqpoint{2.122667in}{2.791609in}}%
\pgfpathlineto{\pgfqpoint{2.139194in}{2.783394in}}%
\pgfpathlineto{\pgfqpoint{2.175100in}{2.768000in}}%
\pgfpathlineto{\pgfqpoint{2.245254in}{2.732851in}}%
\pgfpathlineto{\pgfqpoint{2.282990in}{2.712349in}}%
\pgfpathlineto{\pgfqpoint{2.295224in}{2.704729in}}%
\pgfpathlineto{\pgfqpoint{2.323071in}{2.689737in}}%
\pgfpathlineto{\pgfqpoint{2.344509in}{2.675968in}}%
\pgfpathlineto{\pgfqpoint{2.378894in}{2.656000in}}%
\pgfpathlineto{\pgfqpoint{2.403232in}{2.640860in}}%
\pgfpathlineto{\pgfqpoint{2.443313in}{2.615283in}}%
\pgfpathlineto{\pgfqpoint{2.463964in}{2.600568in}}%
\pgfpathlineto{\pgfqpoint{2.493892in}{2.581333in}}%
\pgfpathlineto{\pgfqpoint{2.563556in}{2.532648in}}%
\pgfpathlineto{\pgfqpoint{2.603636in}{2.503683in}}%
\pgfpathlineto{\pgfqpoint{2.623542in}{2.487875in}}%
\pgfpathlineto{\pgfqpoint{2.649493in}{2.469333in}}%
\pgfpathlineto{\pgfqpoint{2.667816in}{2.454447in}}%
\pgfpathlineto{\pgfqpoint{2.697843in}{2.432000in}}%
\pgfpathlineto{\pgfqpoint{2.763960in}{2.379442in}}%
\pgfpathlineto{\pgfqpoint{2.791014in}{2.357333in}}%
\pgfpathlineto{\pgfqpoint{2.844121in}{2.313158in}}%
\pgfpathlineto{\pgfqpoint{2.924283in}{2.244069in}}%
\pgfpathlineto{\pgfqpoint{3.005918in}{2.170667in}}%
\pgfpathlineto{\pgfqpoint{3.085793in}{2.096000in}}%
\pgfpathlineto{\pgfqpoint{3.164768in}{2.019289in}}%
\pgfpathlineto{\pgfqpoint{3.182544in}{2.000557in}}%
\pgfpathlineto{\pgfqpoint{3.204848in}{1.979177in}}%
\pgfpathlineto{\pgfqpoint{3.285010in}{1.896601in}}%
\pgfpathlineto{\pgfqpoint{3.308259in}{1.872000in}}%
\pgfpathlineto{\pgfqpoint{3.377461in}{1.797333in}}%
\pgfpathlineto{\pgfqpoint{3.389773in}{1.782915in}}%
\pgfpathlineto{\pgfqpoint{3.411252in}{1.760000in}}%
\pgfpathlineto{\pgfqpoint{3.485414in}{1.675332in}}%
\pgfpathlineto{\pgfqpoint{3.570906in}{1.573333in}}%
\pgfpathlineto{\pgfqpoint{3.585339in}{1.554409in}}%
\pgfpathlineto{\pgfqpoint{3.605657in}{1.530304in}}%
\pgfpathlineto{\pgfqpoint{3.693826in}{1.416541in}}%
\pgfpathlineto{\pgfqpoint{3.769757in}{1.312000in}}%
\pgfpathlineto{\pgfqpoint{3.806061in}{1.259235in}}%
\pgfpathlineto{\pgfqpoint{3.846141in}{1.199250in}}%
\pgfpathlineto{\pgfqpoint{3.892551in}{1.125333in}}%
\pgfpathlineto{\pgfqpoint{3.904073in}{1.104627in}}%
\pgfpathlineto{\pgfqpoint{3.926303in}{1.068374in}}%
\pgfpathlineto{\pgfqpoint{3.957256in}{1.013333in}}%
\pgfpathlineto{\pgfqpoint{3.977096in}{0.976000in}}%
\pgfpathlineto{\pgfqpoint{3.985919in}{0.956863in}}%
\pgfpathlineto{\pgfqpoint{3.996021in}{0.938667in}}%
\pgfpathlineto{\pgfqpoint{4.019886in}{0.888832in}}%
\pgfpathlineto{\pgfqpoint{4.046696in}{0.826527in}}%
\pgfpathlineto{\pgfqpoint{4.060525in}{0.789333in}}%
\pgfpathlineto{\pgfqpoint{4.066144in}{0.770255in}}%
\pgfpathlineto{\pgfqpoint{4.073321in}{0.752000in}}%
\pgfpathlineto{\pgfqpoint{4.075692in}{0.741815in}}%
\pgfpathlineto{\pgfqpoint{4.086626in}{0.707086in}}%
\pgfpathlineto{\pgfqpoint{4.093761in}{0.677333in}}%
\pgfpathlineto{\pgfqpoint{4.104606in}{0.602667in}}%
\pgfpathlineto{\pgfqpoint{4.105855in}{0.584756in}}%
\pgfpathlineto{\pgfqpoint{4.105288in}{0.565333in}}%
\pgfpathlineto{\pgfqpoint{4.102010in}{0.542330in}}%
\pgfpathlineto{\pgfqpoint{4.101427in}{0.528000in}}%
\pgfpathlineto{\pgfqpoint{4.133705in}{0.528000in}}%
\pgfpathlineto{\pgfqpoint{4.134309in}{0.565333in}}%
\pgfpathlineto{\pgfqpoint{4.133312in}{0.571486in}}%
\pgfpathlineto{\pgfqpoint{4.131362in}{0.602667in}}%
\pgfpathlineto{\pgfqpoint{4.125548in}{0.640000in}}%
\pgfpathlineto{\pgfqpoint{4.106441in}{0.714667in}}%
\pgfpathlineto{\pgfqpoint{4.094529in}{0.752000in}}%
\pgfpathlineto{\pgfqpoint{4.078000in}{0.797368in}}%
\pgfpathlineto{\pgfqpoint{4.046545in}{0.871204in}}%
\pgfpathlineto{\pgfqpoint{4.011421in}{0.943284in}}%
\pgfpathlineto{\pgfqpoint{3.994674in}{0.976000in}}%
\pgfpathlineto{\pgfqpoint{3.966384in}{1.027810in}}%
\pgfpathlineto{\pgfqpoint{3.953351in}{1.050667in}}%
\pgfpathlineto{\pgfqpoint{3.926303in}{1.096779in}}%
\pgfpathlineto{\pgfqpoint{3.900072in}{1.138234in}}%
\pgfpathlineto{\pgfqpoint{3.884977in}{1.163826in}}%
\pgfpathlineto{\pgfqpoint{3.811284in}{1.274667in}}%
\pgfpathlineto{\pgfqpoint{3.806061in}{1.282180in}}%
\pgfpathlineto{\pgfqpoint{3.731221in}{1.386667in}}%
\pgfpathlineto{\pgfqpoint{3.725899in}{1.393826in}}%
\pgfpathlineto{\pgfqpoint{3.645737in}{1.498881in}}%
\pgfpathlineto{\pgfqpoint{3.555033in}{1.610667in}}%
\pgfpathlineto{\pgfqpoint{3.541671in}{1.625733in}}%
\pgfpathlineto{\pgfqpoint{3.517770in}{1.655195in}}%
\pgfpathlineto{\pgfqpoint{3.426177in}{1.760000in}}%
\pgfpathlineto{\pgfqpoint{3.358484in}{1.834667in}}%
\pgfpathlineto{\pgfqpoint{3.342531in}{1.850911in}}%
\pgfpathlineto{\pgfqpoint{3.323825in}{1.872000in}}%
\pgfpathlineto{\pgfqpoint{3.305107in}{1.890719in}}%
\pgfpathlineto{\pgfqpoint{3.285010in}{1.912853in}}%
\pgfpathlineto{\pgfqpoint{3.204848in}{1.994844in}}%
\pgfpathlineto{\pgfqpoint{3.171555in}{2.027655in}}%
\pgfpathlineto{\pgfqpoint{3.140312in}{2.058667in}}%
\pgfpathlineto{\pgfqpoint{3.124687in}{2.073899in}}%
\pgfpathlineto{\pgfqpoint{3.044525in}{2.150085in}}%
\pgfpathlineto{\pgfqpoint{2.964364in}{2.223465in}}%
\pgfpathlineto{\pgfqpoint{2.884202in}{2.294104in}}%
\pgfpathlineto{\pgfqpoint{2.804040in}{2.362061in}}%
\pgfpathlineto{\pgfqpoint{2.784927in}{2.376864in}}%
\pgfpathlineto{\pgfqpoint{2.763960in}{2.395053in}}%
\pgfpathlineto{\pgfqpoint{2.742091in}{2.411630in}}%
\pgfpathlineto{\pgfqpoint{2.717672in}{2.432000in}}%
\pgfpathlineto{\pgfqpoint{2.677226in}{2.463212in}}%
\pgfpathlineto{\pgfqpoint{2.643717in}{2.489204in}}%
\pgfpathlineto{\pgfqpoint{2.620459in}{2.506667in}}%
\pgfpathlineto{\pgfqpoint{2.563556in}{2.548640in}}%
\pgfpathlineto{\pgfqpoint{2.543095in}{2.562275in}}%
\pgfpathlineto{\pgfqpoint{2.517399in}{2.581333in}}%
\pgfpathlineto{\pgfqpoint{2.443313in}{2.631569in}}%
\pgfpathlineto{\pgfqpoint{2.363152in}{2.682727in}}%
\pgfpathlineto{\pgfqpoint{2.345586in}{2.693333in}}%
\pgfpathlineto{\pgfqpoint{2.282366in}{2.730667in}}%
\pgfpathlineto{\pgfqpoint{2.256912in}{2.743709in}}%
\pgfpathlineto{\pgfqpoint{2.242909in}{2.752161in}}%
\pgfpathlineto{\pgfqpoint{2.190241in}{2.779724in}}%
\pgfpathlineto{\pgfqpoint{2.122667in}{2.811511in}}%
\pgfpathlineto{\pgfqpoint{2.098944in}{2.820571in}}%
\pgfpathlineto{\pgfqpoint{2.082586in}{2.828355in}}%
\pgfpathlineto{\pgfqpoint{2.042505in}{2.844002in}}%
\pgfpathlineto{\pgfqpoint{1.962343in}{2.868644in}}%
\pgfpathlineto{\pgfqpoint{1.922263in}{2.877822in}}%
\pgfpathlineto{\pgfqpoint{1.877635in}{2.884235in}}%
\pgfpathlineto{\pgfqpoint{1.835827in}{2.885844in}}%
\pgfpathlineto{\pgfqpoint{1.802020in}{2.883012in}}%
\pgfpathlineto{\pgfqpoint{1.767328in}{2.874981in}}%
\pgfpathlineto{\pgfqpoint{1.740094in}{2.863015in}}%
\pgfpathlineto{\pgfqpoint{1.721859in}{2.850714in}}%
\pgfpathlineto{\pgfqpoint{1.712626in}{2.842667in}}%
\pgfpathlineto{\pgfqpoint{1.681778in}{2.800466in}}%
\pgfpathlineto{\pgfqpoint{1.670446in}{2.768000in}}%
\pgfpathlineto{\pgfqpoint{1.665548in}{2.745784in}}%
\pgfpathlineto{\pgfqpoint{1.662576in}{2.711219in}}%
\pgfpathlineto{\pgfqpoint{1.662989in}{2.693333in}}%
\pgfpathlineto{\pgfqpoint{1.664992in}{2.677698in}}%
\pgfpathlineto{\pgfqpoint{1.665589in}{2.656000in}}%
\pgfpathlineto{\pgfqpoint{1.671104in}{2.618667in}}%
\pgfpathlineto{\pgfqpoint{1.673350in}{2.610816in}}%
\pgfpathlineto{\pgfqpoint{1.681778in}{2.570740in}}%
\pgfpathlineto{\pgfqpoint{1.689172in}{2.544000in}}%
\pgfpathlineto{\pgfqpoint{1.697392in}{2.521211in}}%
\pgfpathlineto{\pgfqpoint{1.701215in}{2.506667in}}%
\pgfpathlineto{\pgfqpoint{1.716685in}{2.464514in}}%
\pgfpathlineto{\pgfqpoint{1.733746in}{2.420927in}}%
\pgfpathlineto{\pgfqpoint{1.762533in}{2.356780in}}%
\pgfpathlineto{\pgfqpoint{1.802020in}{2.278534in}}%
\pgfpathlineto{\pgfqpoint{1.842101in}{2.205870in}}%
\pgfpathlineto{\pgfqpoint{1.889306in}{2.126697in}}%
\pgfpathlineto{\pgfqpoint{1.957296in}{2.021333in}}%
\pgfpathlineto{\pgfqpoint{1.975241in}{1.996013in}}%
\pgfpathlineto{\pgfqpoint{1.982823in}{1.984000in}}%
\pgfpathlineto{\pgfqpoint{2.008656in}{1.946667in}}%
\pgfpathlineto{\pgfqpoint{2.022672in}{1.928193in}}%
\pgfpathlineto{\pgfqpoint{2.042505in}{1.899384in}}%
\pgfpathlineto{\pgfqpoint{2.062642in}{1.872000in}}%
\pgfpathlineto{\pgfqpoint{2.122667in}{1.792164in}}%
\pgfpathlineto{\pgfqpoint{2.207392in}{1.685333in}}%
\pgfpathlineto{\pgfqpoint{2.223289in}{1.667058in}}%
\pgfpathlineto{\pgfqpoint{2.242909in}{1.642227in}}%
\pgfpathlineto{\pgfqpoint{2.333736in}{1.536000in}}%
\pgfpathlineto{\pgfqpoint{2.347400in}{1.521328in}}%
\pgfpathlineto{\pgfqpoint{2.383593in}{1.479626in}}%
\pgfpathlineto{\pgfqpoint{2.443313in}{1.414305in}}%
\pgfpathlineto{\pgfqpoint{2.469160in}{1.386667in}}%
\pgfpathlineto{\pgfqpoint{2.540428in}{1.312000in}}%
\pgfpathlineto{\pgfqpoint{2.614106in}{1.237333in}}%
\pgfpathlineto{\pgfqpoint{2.628855in}{1.223490in}}%
\pgfpathlineto{\pgfqpoint{2.651915in}{1.200000in}}%
\pgfpathlineto{\pgfqpoint{2.729623in}{1.125333in}}%
\pgfpathlineto{\pgfqpoint{2.747205in}{1.109727in}}%
\pgfpathlineto{\pgfqpoint{2.769586in}{1.088000in}}%
\pgfpathlineto{\pgfqpoint{2.787392in}{1.072493in}}%
\pgfpathlineto{\pgfqpoint{2.810336in}{1.050667in}}%
\pgfpathlineto{\pgfqpoint{2.827958in}{1.035612in}}%
\pgfpathlineto{\pgfqpoint{2.851909in}{1.013333in}}%
\pgfpathlineto{\pgfqpoint{2.868910in}{0.999089in}}%
\pgfpathlineto{\pgfqpoint{2.894345in}{0.976000in}}%
\pgfpathlineto{\pgfqpoint{2.981984in}{0.901333in}}%
\pgfpathlineto{\pgfqpoint{3.044525in}{0.849960in}}%
\pgfpathlineto{\pgfqpoint{3.141733in}{0.773455in}}%
\pgfpathlineto{\pgfqpoint{3.220922in}{0.714667in}}%
\pgfpathlineto{\pgfqpoint{3.285010in}{0.668963in}}%
\pgfpathlineto{\pgfqpoint{3.327111in}{0.640000in}}%
\pgfpathlineto{\pgfqpoint{3.349919in}{0.625793in}}%
\pgfpathlineto{\pgfqpoint{3.384191in}{0.602667in}}%
\pgfpathlineto{\pgfqpoint{3.445333in}{0.564067in}}%
\pgfpathlineto{\pgfqpoint{3.469599in}{0.550602in}}%
\pgfpathlineto{\pgfqpoint{3.485414in}{0.540388in}}%
\pgfpathlineto{\pgfqpoint{3.507039in}{0.528000in}}%
\pgfpathlineto{\pgfqpoint{3.525495in}{0.528000in}}%
\pgfpathlineto{\pgfqpoint{3.525495in}{0.528000in}}%
\pgfusepath{fill}%
\end{pgfscope}%
\begin{pgfscope}%
\pgfpathrectangle{\pgfqpoint{0.800000in}{0.528000in}}{\pgfqpoint{3.968000in}{3.696000in}}%
\pgfusepath{clip}%
\pgfsetbuttcap%
\pgfsetroundjoin%
\definecolor{currentfill}{rgb}{0.274952,0.037752,0.364543}%
\pgfsetfillcolor{currentfill}%
\pgfsetlinewidth{0.000000pt}%
\definecolor{currentstroke}{rgb}{0.000000,0.000000,0.000000}%
\pgfsetstrokecolor{currentstroke}%
\pgfsetdash{}{0pt}%
\pgfpathmoveto{\pgfqpoint{3.507039in}{0.528000in}}%
\pgfpathlineto{\pgfqpoint{3.443303in}{0.565333in}}%
\pgfpathlineto{\pgfqpoint{3.365172in}{0.614914in}}%
\pgfpathlineto{\pgfqpoint{3.349919in}{0.625793in}}%
\pgfpathlineto{\pgfqpoint{3.319836in}{0.644895in}}%
\pgfpathlineto{\pgfqpoint{3.244929in}{0.697317in}}%
\pgfpathlineto{\pgfqpoint{3.164768in}{0.756006in}}%
\pgfpathlineto{\pgfqpoint{3.084606in}{0.817916in}}%
\pgfpathlineto{\pgfqpoint{3.073631in}{0.826667in}}%
\pgfpathlineto{\pgfqpoint{2.981984in}{0.901333in}}%
\pgfpathlineto{\pgfqpoint{2.964364in}{0.916036in}}%
\pgfpathlineto{\pgfqpoint{2.884202in}{0.984815in}}%
\pgfpathlineto{\pgfqpoint{2.868910in}{0.999089in}}%
\pgfpathlineto{\pgfqpoint{2.844121in}{1.020237in}}%
\pgfpathlineto{\pgfqpoint{2.827958in}{1.035612in}}%
\pgfpathlineto{\pgfqpoint{2.804040in}{1.056358in}}%
\pgfpathlineto{\pgfqpoint{2.787392in}{1.072493in}}%
\pgfpathlineto{\pgfqpoint{2.763960in}{1.093187in}}%
\pgfpathlineto{\pgfqpoint{2.747205in}{1.109727in}}%
\pgfpathlineto{\pgfqpoint{2.723879in}{1.130732in}}%
\pgfpathlineto{\pgfqpoint{2.707390in}{1.147308in}}%
\pgfpathlineto{\pgfqpoint{2.683798in}{1.169001in}}%
\pgfpathlineto{\pgfqpoint{2.603636in}{1.247744in}}%
\pgfpathlineto{\pgfqpoint{2.504506in}{1.349333in}}%
\pgfpathlineto{\pgfqpoint{2.483394in}{1.371508in}}%
\pgfpathlineto{\pgfqpoint{2.400109in}{1.461333in}}%
\pgfpathlineto{\pgfqpoint{2.323071in}{1.548199in}}%
\pgfpathlineto{\pgfqpoint{2.238105in}{1.648000in}}%
\pgfpathlineto{\pgfqpoint{2.223289in}{1.667058in}}%
\pgfpathlineto{\pgfqpoint{2.202828in}{1.690950in}}%
\pgfpathlineto{\pgfqpoint{2.171409in}{1.730734in}}%
\pgfpathlineto{\pgfqpoint{2.147849in}{1.760000in}}%
\pgfpathlineto{\pgfqpoint{2.082586in}{1.845019in}}%
\pgfpathlineto{\pgfqpoint{2.035289in}{1.909333in}}%
\pgfpathlineto{\pgfqpoint{2.022672in}{1.928193in}}%
\pgfpathlineto{\pgfqpoint{2.002424in}{1.955575in}}%
\pgfpathlineto{\pgfqpoint{1.948840in}{2.033911in}}%
\pgfpathlineto{\pgfqpoint{1.885260in}{2.133333in}}%
\pgfpathlineto{\pgfqpoint{1.839401in}{2.210514in}}%
\pgfpathlineto{\pgfqpoint{1.799793in}{2.282667in}}%
\pgfpathlineto{\pgfqpoint{1.780877in}{2.320000in}}%
\pgfpathlineto{\pgfqpoint{1.761939in}{2.358070in}}%
\pgfpathlineto{\pgfqpoint{1.729262in}{2.432000in}}%
\pgfpathlineto{\pgfqpoint{1.727520in}{2.437273in}}%
\pgfpathlineto{\pgfqpoint{1.710273in}{2.480124in}}%
\pgfpathlineto{\pgfqpoint{1.689172in}{2.544000in}}%
\pgfpathlineto{\pgfqpoint{1.678907in}{2.581333in}}%
\pgfpathlineto{\pgfqpoint{1.665589in}{2.656000in}}%
\pgfpathlineto{\pgfqpoint{1.664992in}{2.677698in}}%
\pgfpathlineto{\pgfqpoint{1.662989in}{2.693333in}}%
\pgfpathlineto{\pgfqpoint{1.662576in}{2.711219in}}%
\pgfpathlineto{\pgfqpoint{1.665548in}{2.745784in}}%
\pgfpathlineto{\pgfqpoint{1.670446in}{2.768000in}}%
\pgfpathlineto{\pgfqpoint{1.684058in}{2.805333in}}%
\pgfpathlineto{\pgfqpoint{1.721859in}{2.850714in}}%
\pgfpathlineto{\pgfqpoint{1.740094in}{2.863015in}}%
\pgfpathlineto{\pgfqpoint{1.767328in}{2.874981in}}%
\pgfpathlineto{\pgfqpoint{1.802020in}{2.883012in}}%
\pgfpathlineto{\pgfqpoint{1.842101in}{2.885740in}}%
\pgfpathlineto{\pgfqpoint{1.885351in}{2.882952in}}%
\pgfpathlineto{\pgfqpoint{1.925124in}{2.877335in}}%
\pgfpathlineto{\pgfqpoint{1.979135in}{2.864360in}}%
\pgfpathlineto{\pgfqpoint{2.042505in}{2.844002in}}%
\pgfpathlineto{\pgfqpoint{2.082586in}{2.828355in}}%
\pgfpathlineto{\pgfqpoint{2.098944in}{2.820571in}}%
\pgfpathlineto{\pgfqpoint{2.136052in}{2.805333in}}%
\pgfpathlineto{\pgfqpoint{2.162747in}{2.792955in}}%
\pgfpathlineto{\pgfqpoint{2.212927in}{2.768000in}}%
\pgfpathlineto{\pgfqpoint{2.282990in}{2.730325in}}%
\pgfpathlineto{\pgfqpoint{2.306790in}{2.715502in}}%
\pgfpathlineto{\pgfqpoint{2.323071in}{2.706871in}}%
\pgfpathlineto{\pgfqpoint{2.405982in}{2.656000in}}%
\pgfpathlineto{\pgfqpoint{2.483394in}{2.604689in}}%
\pgfpathlineto{\pgfqpoint{2.523475in}{2.577143in}}%
\pgfpathlineto{\pgfqpoint{2.543095in}{2.562275in}}%
\pgfpathlineto{\pgfqpoint{2.569903in}{2.544000in}}%
\pgfpathlineto{\pgfqpoint{2.643717in}{2.489204in}}%
\pgfpathlineto{\pgfqpoint{2.677226in}{2.463212in}}%
\pgfpathlineto{\pgfqpoint{2.683798in}{2.458499in}}%
\pgfpathlineto{\pgfqpoint{2.698814in}{2.445987in}}%
\pgfpathlineto{\pgfqpoint{2.723879in}{2.427126in}}%
\pgfpathlineto{\pgfqpoint{2.742091in}{2.411630in}}%
\pgfpathlineto{\pgfqpoint{2.764431in}{2.394667in}}%
\pgfpathlineto{\pgfqpoint{2.784927in}{2.376864in}}%
\pgfpathlineto{\pgfqpoint{2.809690in}{2.357333in}}%
\pgfpathlineto{\pgfqpoint{2.827332in}{2.341695in}}%
\pgfpathlineto{\pgfqpoint{2.853978in}{2.320000in}}%
\pgfpathlineto{\pgfqpoint{2.939819in}{2.245333in}}%
\pgfpathlineto{\pgfqpoint{3.022289in}{2.170667in}}%
\pgfpathlineto{\pgfqpoint{3.101684in}{2.096000in}}%
\pgfpathlineto{\pgfqpoint{3.178267in}{2.021333in}}%
\pgfpathlineto{\pgfqpoint{3.210290in}{1.989069in}}%
\pgfpathlineto{\pgfqpoint{3.215578in}{1.984000in}}%
\pgfpathlineto{\pgfqpoint{3.288371in}{1.909333in}}%
\pgfpathlineto{\pgfqpoint{3.305107in}{1.890719in}}%
\pgfpathlineto{\pgfqpoint{3.325091in}{1.870655in}}%
\pgfpathlineto{\pgfqpoint{3.342531in}{1.850911in}}%
\pgfpathlineto{\pgfqpoint{3.365172in}{1.827429in}}%
\pgfpathlineto{\pgfqpoint{3.445333in}{1.738504in}}%
\pgfpathlineto{\pgfqpoint{3.470610in}{1.708877in}}%
\pgfpathlineto{\pgfqpoint{3.491850in}{1.685333in}}%
\pgfpathlineto{\pgfqpoint{3.555033in}{1.610667in}}%
\pgfpathlineto{\pgfqpoint{3.565576in}{1.597966in}}%
\pgfpathlineto{\pgfqpoint{3.646368in}{1.498079in}}%
\pgfpathlineto{\pgfqpoint{3.731221in}{1.386667in}}%
\pgfpathlineto{\pgfqpoint{3.765980in}{1.338893in}}%
\pgfpathlineto{\pgfqpoint{3.785040in}{1.312000in}}%
\pgfpathlineto{\pgfqpoint{3.836623in}{1.237333in}}%
\pgfpathlineto{\pgfqpoint{3.886222in}{1.161849in}}%
\pgfpathlineto{\pgfqpoint{3.914761in}{1.114583in}}%
\pgfpathlineto{\pgfqpoint{3.938353in}{1.076776in}}%
\pgfpathlineto{\pgfqpoint{3.983286in}{0.997590in}}%
\pgfpathlineto{\pgfqpoint{4.020982in}{0.925144in}}%
\pgfpathlineto{\pgfqpoint{4.052390in}{0.858556in}}%
\pgfpathlineto{\pgfqpoint{4.082202in}{0.785212in}}%
\pgfpathlineto{\pgfqpoint{4.098432in}{0.741004in}}%
\pgfpathlineto{\pgfqpoint{4.116900in}{0.677333in}}%
\pgfpathlineto{\pgfqpoint{4.118119in}{0.669334in}}%
\pgfpathlineto{\pgfqpoint{4.126707in}{0.632692in}}%
\pgfpathlineto{\pgfqpoint{4.131362in}{0.602667in}}%
\pgfpathlineto{\pgfqpoint{4.134770in}{0.557823in}}%
\pgfpathlineto{\pgfqpoint{4.133705in}{0.528000in}}%
\pgfpathlineto{\pgfqpoint{4.162879in}{0.528000in}}%
\pgfpathlineto{\pgfqpoint{4.163005in}{0.531524in}}%
\pgfpathlineto{\pgfqpoint{4.160931in}{0.565333in}}%
\pgfpathlineto{\pgfqpoint{4.148315in}{0.640000in}}%
\pgfpathlineto{\pgfqpoint{4.138857in}{0.677333in}}%
\pgfpathlineto{\pgfqpoint{4.135759in}{0.685765in}}%
\pgfpathlineto{\pgfqpoint{4.126707in}{0.717924in}}%
\pgfpathlineto{\pgfqpoint{4.100275in}{0.789333in}}%
\pgfpathlineto{\pgfqpoint{4.096087in}{0.798145in}}%
\pgfpathlineto{\pgfqpoint{4.083925in}{0.829183in}}%
\pgfpathlineto{\pgfqpoint{4.046545in}{0.908952in}}%
\pgfpathlineto{\pgfqpoint{4.006465in}{0.985818in}}%
\pgfpathlineto{\pgfqpoint{3.966384in}{1.056729in}}%
\pgfpathlineto{\pgfqpoint{3.925008in}{1.125333in}}%
\pgfpathlineto{\pgfqpoint{3.910036in}{1.147514in}}%
\pgfpathlineto{\pgfqpoint{3.901149in}{1.162667in}}%
\pgfpathlineto{\pgfqpoint{3.846141in}{1.246047in}}%
\pgfpathlineto{\pgfqpoint{3.800323in}{1.312000in}}%
\pgfpathlineto{\pgfqpoint{3.786059in}{1.330703in}}%
\pgfpathlineto{\pgfqpoint{3.765980in}{1.359619in}}%
\pgfpathlineto{\pgfqpoint{3.685818in}{1.466267in}}%
\pgfpathlineto{\pgfqpoint{3.600579in}{1.573333in}}%
\pgfpathlineto{\pgfqpoint{3.565576in}{1.615740in}}%
\pgfpathlineto{\pgfqpoint{3.538312in}{1.648000in}}%
\pgfpathlineto{\pgfqpoint{3.473977in}{1.722667in}}%
\pgfpathlineto{\pgfqpoint{3.460674in}{1.736956in}}%
\pgfpathlineto{\pgfqpoint{3.441101in}{1.760000in}}%
\pgfpathlineto{\pgfqpoint{3.365172in}{1.843483in}}%
\pgfpathlineto{\pgfqpoint{3.338512in}{1.872000in}}%
\pgfpathlineto{\pgfqpoint{3.267284in}{1.946667in}}%
\pgfpathlineto{\pgfqpoint{3.244929in}{1.969666in}}%
\pgfpathlineto{\pgfqpoint{3.155975in}{2.058667in}}%
\pgfpathlineto{\pgfqpoint{3.140277in}{2.073188in}}%
\pgfpathlineto{\pgfqpoint{3.117576in}{2.096000in}}%
\pgfpathlineto{\pgfqpoint{3.100884in}{2.111163in}}%
\pgfpathlineto{\pgfqpoint{3.078482in}{2.133333in}}%
\pgfpathlineto{\pgfqpoint{3.061135in}{2.148804in}}%
\pgfpathlineto{\pgfqpoint{3.038660in}{2.170667in}}%
\pgfpathlineto{\pgfqpoint{3.021022in}{2.186108in}}%
\pgfpathlineto{\pgfqpoint{2.998077in}{2.208000in}}%
\pgfpathlineto{\pgfqpoint{2.980540in}{2.223068in}}%
\pgfpathlineto{\pgfqpoint{2.956699in}{2.245333in}}%
\pgfpathlineto{\pgfqpoint{2.919408in}{2.278126in}}%
\pgfpathlineto{\pgfqpoint{2.914486in}{2.282667in}}%
\pgfpathlineto{\pgfqpoint{2.827395in}{2.357333in}}%
\pgfpathlineto{\pgfqpoint{2.793612in}{2.384953in}}%
\pgfpathlineto{\pgfqpoint{2.763960in}{2.409814in}}%
\pgfpathlineto{\pgfqpoint{2.736452in}{2.432000in}}%
\pgfpathlineto{\pgfqpoint{2.683798in}{2.473752in}}%
\pgfpathlineto{\pgfqpoint{2.663933in}{2.488164in}}%
\pgfpathlineto{\pgfqpoint{2.641040in}{2.506667in}}%
\pgfpathlineto{\pgfqpoint{2.563556in}{2.563974in}}%
\pgfpathlineto{\pgfqpoint{2.539337in}{2.581333in}}%
\pgfpathlineto{\pgfqpoint{2.475162in}{2.626334in}}%
\pgfpathlineto{\pgfqpoint{2.403232in}{2.673808in}}%
\pgfpathlineto{\pgfqpoint{2.323071in}{2.723774in}}%
\pgfpathlineto{\pgfqpoint{2.311302in}{2.730667in}}%
\pgfpathlineto{\pgfqpoint{2.242909in}{2.769926in}}%
\pgfpathlineto{\pgfqpoint{2.218084in}{2.782210in}}%
\pgfpathlineto{\pgfqpoint{2.202828in}{2.791126in}}%
\pgfpathlineto{\pgfqpoint{2.148702in}{2.818416in}}%
\pgfpathlineto{\pgfqpoint{2.071658in}{2.852846in}}%
\pgfpathlineto{\pgfqpoint{1.998932in}{2.880000in}}%
\pgfpathlineto{\pgfqpoint{1.922263in}{2.901398in}}%
\pgfpathlineto{\pgfqpoint{1.907302in}{2.903398in}}%
\pgfpathlineto{\pgfqpoint{1.882182in}{2.909165in}}%
\pgfpathlineto{\pgfqpoint{1.842101in}{2.913744in}}%
\pgfpathlineto{\pgfqpoint{1.802020in}{2.914087in}}%
\pgfpathlineto{\pgfqpoint{1.797492in}{2.913116in}}%
\pgfpathlineto{\pgfqpoint{1.761939in}{2.908629in}}%
\pgfpathlineto{\pgfqpoint{1.745928in}{2.902420in}}%
\pgfpathlineto{\pgfqpoint{1.721859in}{2.894924in}}%
\pgfpathlineto{\pgfqpoint{1.711277in}{2.889856in}}%
\pgfpathlineto{\pgfqpoint{1.697383in}{2.880000in}}%
\pgfpathlineto{\pgfqpoint{1.681778in}{2.866582in}}%
\pgfpathlineto{\pgfqpoint{1.669982in}{2.853654in}}%
\pgfpathlineto{\pgfqpoint{1.655304in}{2.829993in}}%
\pgfpathlineto{\pgfqpoint{1.645376in}{2.805333in}}%
\pgfpathlineto{\pgfqpoint{1.641697in}{2.791063in}}%
\pgfpathlineto{\pgfqpoint{1.637151in}{2.763765in}}%
\pgfpathlineto{\pgfqpoint{1.634504in}{2.730667in}}%
\pgfpathlineto{\pgfqpoint{1.635863in}{2.693333in}}%
\pgfpathlineto{\pgfqpoint{1.641697in}{2.648571in}}%
\pgfpathlineto{\pgfqpoint{1.647594in}{2.618667in}}%
\pgfpathlineto{\pgfqpoint{1.654785in}{2.593524in}}%
\pgfpathlineto{\pgfqpoint{1.657083in}{2.581333in}}%
\pgfpathlineto{\pgfqpoint{1.668175in}{2.544000in}}%
\pgfpathlineto{\pgfqpoint{1.681778in}{2.503494in}}%
\pgfpathlineto{\pgfqpoint{1.695033in}{2.469333in}}%
\pgfpathlineto{\pgfqpoint{1.703097in}{2.451858in}}%
\pgfpathlineto{\pgfqpoint{1.714076in}{2.424751in}}%
\pgfpathlineto{\pgfqpoint{1.730510in}{2.386608in}}%
\pgfpathlineto{\pgfqpoint{1.763085in}{2.320000in}}%
\pgfpathlineto{\pgfqpoint{1.789796in}{2.271280in}}%
\pgfpathlineto{\pgfqpoint{1.804283in}{2.243225in}}%
\pgfpathlineto{\pgfqpoint{1.846540in}{2.170667in}}%
\pgfpathlineto{\pgfqpoint{1.860036in}{2.150039in}}%
\pgfpathlineto{\pgfqpoint{1.882182in}{2.113013in}}%
\pgfpathlineto{\pgfqpoint{1.922263in}{2.050793in}}%
\pgfpathlineto{\pgfqpoint{1.993572in}{1.946667in}}%
\pgfpathlineto{\pgfqpoint{2.058653in}{1.856959in}}%
\pgfpathlineto{\pgfqpoint{2.133065in}{1.760000in}}%
\pgfpathlineto{\pgfqpoint{2.146084in}{1.744479in}}%
\pgfpathlineto{\pgfqpoint{2.162747in}{1.722293in}}%
\pgfpathlineto{\pgfqpoint{2.180328in}{1.701709in}}%
\pgfpathlineto{\pgfqpoint{2.202828in}{1.673108in}}%
\pgfpathlineto{\pgfqpoint{2.232332in}{1.638148in}}%
\pgfpathlineto{\pgfqpoint{2.254855in}{1.610667in}}%
\pgfpathlineto{\pgfqpoint{2.267676in}{1.596402in}}%
\pgfpathlineto{\pgfqpoint{2.300955in}{1.556600in}}%
\pgfpathlineto{\pgfqpoint{2.363152in}{1.486142in}}%
\pgfpathlineto{\pgfqpoint{2.454201in}{1.386667in}}%
\pgfpathlineto{\pgfqpoint{2.540273in}{1.296354in}}%
\pgfpathlineto{\pgfqpoint{2.603636in}{1.232300in}}%
\pgfpathlineto{\pgfqpoint{2.620909in}{1.216088in}}%
\pgfpathlineto{\pgfqpoint{2.643717in}{1.192750in}}%
\pgfpathlineto{\pgfqpoint{2.723879in}{1.115736in}}%
\pgfpathlineto{\pgfqpoint{2.753513in}{1.088000in}}%
\pgfpathlineto{\pgfqpoint{2.804040in}{1.041456in}}%
\pgfpathlineto{\pgfqpoint{2.819789in}{1.028002in}}%
\pgfpathlineto{\pgfqpoint{2.844121in}{1.005322in}}%
\pgfpathlineto{\pgfqpoint{2.860694in}{0.991437in}}%
\pgfpathlineto{\pgfqpoint{2.884202in}{0.969849in}}%
\pgfpathlineto{\pgfqpoint{2.969356in}{0.896683in}}%
\pgfpathlineto{\pgfqpoint{3.054775in}{0.826667in}}%
\pgfpathlineto{\pgfqpoint{3.084606in}{0.802883in}}%
\pgfpathlineto{\pgfqpoint{3.164768in}{0.740859in}}%
\pgfpathlineto{\pgfqpoint{3.180812in}{0.729611in}}%
\pgfpathlineto{\pgfqpoint{3.224355in}{0.696497in}}%
\pgfpathlineto{\pgfqpoint{3.304306in}{0.640000in}}%
\pgfpathlineto{\pgfqpoint{3.379669in}{0.589163in}}%
\pgfpathlineto{\pgfqpoint{3.445333in}{0.547731in}}%
\pgfpathlineto{\pgfqpoint{3.458608in}{0.540365in}}%
\pgfpathlineto{\pgfqpoint{3.477754in}{0.528000in}}%
\pgfpathlineto{\pgfqpoint{3.485414in}{0.528000in}}%
\pgfpathlineto{\pgfqpoint{3.485414in}{0.528000in}}%
\pgfusepath{fill}%
\end{pgfscope}%
\begin{pgfscope}%
\pgfpathrectangle{\pgfqpoint{0.800000in}{0.528000in}}{\pgfqpoint{3.968000in}{3.696000in}}%
\pgfusepath{clip}%
\pgfsetbuttcap%
\pgfsetroundjoin%
\definecolor{currentfill}{rgb}{0.274952,0.037752,0.364543}%
\pgfsetfillcolor{currentfill}%
\pgfsetlinewidth{0.000000pt}%
\definecolor{currentstroke}{rgb}{0.000000,0.000000,0.000000}%
\pgfsetstrokecolor{currentstroke}%
\pgfsetdash{}{0pt}%
\pgfpathmoveto{\pgfqpoint{3.477754in}{0.528000in}}%
\pgfpathlineto{\pgfqpoint{3.434392in}{0.555142in}}%
\pgfpathlineto{\pgfqpoint{3.405253in}{0.572757in}}%
\pgfpathlineto{\pgfqpoint{3.359250in}{0.602667in}}%
\pgfpathlineto{\pgfqpoint{3.325091in}{0.625692in}}%
\pgfpathlineto{\pgfqpoint{3.244929in}{0.681628in}}%
\pgfpathlineto{\pgfqpoint{3.199685in}{0.714667in}}%
\pgfpathlineto{\pgfqpoint{3.180812in}{0.729611in}}%
\pgfpathlineto{\pgfqpoint{3.150183in}{0.752000in}}%
\pgfpathlineto{\pgfqpoint{3.084606in}{0.802883in}}%
\pgfpathlineto{\pgfqpoint{3.071230in}{0.814208in}}%
\pgfpathlineto{\pgfqpoint{3.044525in}{0.834870in}}%
\pgfpathlineto{\pgfqpoint{2.963805in}{0.901333in}}%
\pgfpathlineto{\pgfqpoint{2.877231in}{0.976000in}}%
\pgfpathlineto{\pgfqpoint{2.860694in}{0.991437in}}%
\pgfpathlineto{\pgfqpoint{2.835210in}{1.013333in}}%
\pgfpathlineto{\pgfqpoint{2.799180in}{1.046140in}}%
\pgfpathlineto{\pgfqpoint{2.763960in}{1.078258in}}%
\pgfpathlineto{\pgfqpoint{2.674726in}{1.162667in}}%
\pgfpathlineto{\pgfqpoint{2.598609in}{1.237333in}}%
\pgfpathlineto{\pgfqpoint{2.523475in}{1.313620in}}%
\pgfpathlineto{\pgfqpoint{2.443313in}{1.398309in}}%
\pgfpathlineto{\pgfqpoint{2.351994in}{1.498667in}}%
\pgfpathlineto{\pgfqpoint{2.339326in}{1.513808in}}%
\pgfpathlineto{\pgfqpoint{2.300955in}{1.556600in}}%
\pgfpathlineto{\pgfqpoint{2.242909in}{1.624831in}}%
\pgfpathlineto{\pgfqpoint{2.214893in}{1.659238in}}%
\pgfpathlineto{\pgfqpoint{2.192833in}{1.685333in}}%
\pgfpathlineto{\pgfqpoint{2.180328in}{1.701709in}}%
\pgfpathlineto{\pgfqpoint{2.162447in}{1.722667in}}%
\pgfpathlineto{\pgfqpoint{2.133065in}{1.760000in}}%
\pgfpathlineto{\pgfqpoint{2.075483in}{1.834667in}}%
\pgfpathlineto{\pgfqpoint{2.020389in}{1.909333in}}%
\pgfpathlineto{\pgfqpoint{2.002424in}{1.934230in}}%
\pgfpathlineto{\pgfqpoint{1.962343in}{1.991344in}}%
\pgfpathlineto{\pgfqpoint{1.942068in}{2.021333in}}%
\pgfpathlineto{\pgfqpoint{1.917043in}{2.058667in}}%
\pgfpathlineto{\pgfqpoint{1.904462in}{2.079419in}}%
\pgfpathlineto{\pgfqpoint{1.882182in}{2.113013in}}%
\pgfpathlineto{\pgfqpoint{1.860036in}{2.150039in}}%
\pgfpathlineto{\pgfqpoint{1.842101in}{2.178162in}}%
\pgfpathlineto{\pgfqpoint{1.824692in}{2.208000in}}%
\pgfpathlineto{\pgfqpoint{1.802020in}{2.247343in}}%
\pgfpathlineto{\pgfqpoint{1.776525in}{2.296253in}}%
\pgfpathlineto{\pgfqpoint{1.761939in}{2.322297in}}%
\pgfpathlineto{\pgfqpoint{1.744747in}{2.357333in}}%
\pgfpathlineto{\pgfqpoint{1.721859in}{2.405930in}}%
\pgfpathlineto{\pgfqpoint{1.679915in}{2.508402in}}%
\pgfpathlineto{\pgfqpoint{1.662278in}{2.562163in}}%
\pgfpathlineto{\pgfqpoint{1.647594in}{2.618667in}}%
\pgfpathlineto{\pgfqpoint{1.639960in}{2.657618in}}%
\pgfpathlineto{\pgfqpoint{1.635207in}{2.699378in}}%
\pgfpathlineto{\pgfqpoint{1.634504in}{2.730667in}}%
\pgfpathlineto{\pgfqpoint{1.637078in}{2.768000in}}%
\pgfpathlineto{\pgfqpoint{1.645376in}{2.805333in}}%
\pgfpathlineto{\pgfqpoint{1.655304in}{2.829993in}}%
\pgfpathlineto{\pgfqpoint{1.669982in}{2.853654in}}%
\pgfpathlineto{\pgfqpoint{1.681778in}{2.866582in}}%
\pgfpathlineto{\pgfqpoint{1.697383in}{2.880000in}}%
\pgfpathlineto{\pgfqpoint{1.711277in}{2.889856in}}%
\pgfpathlineto{\pgfqpoint{1.721859in}{2.894924in}}%
\pgfpathlineto{\pgfqpoint{1.745928in}{2.902420in}}%
\pgfpathlineto{\pgfqpoint{1.761939in}{2.908629in}}%
\pgfpathlineto{\pgfqpoint{1.805218in}{2.914355in}}%
\pgfpathlineto{\pgfqpoint{1.846138in}{2.913573in}}%
\pgfpathlineto{\pgfqpoint{1.882182in}{2.909165in}}%
\pgfpathlineto{\pgfqpoint{1.907302in}{2.903398in}}%
\pgfpathlineto{\pgfqpoint{1.922263in}{2.901398in}}%
\pgfpathlineto{\pgfqpoint{1.939513in}{2.896067in}}%
\pgfpathlineto{\pgfqpoint{1.962343in}{2.891172in}}%
\pgfpathlineto{\pgfqpoint{1.970886in}{2.887957in}}%
\pgfpathlineto{\pgfqpoint{2.002424in}{2.878928in}}%
\pgfpathlineto{\pgfqpoint{2.029692in}{2.868065in}}%
\pgfpathlineto{\pgfqpoint{2.042505in}{2.864174in}}%
\pgfpathlineto{\pgfqpoint{2.095116in}{2.842667in}}%
\pgfpathlineto{\pgfqpoint{2.122667in}{2.830432in}}%
\pgfpathlineto{\pgfqpoint{2.174945in}{2.805333in}}%
\pgfpathlineto{\pgfqpoint{2.246316in}{2.768000in}}%
\pgfpathlineto{\pgfqpoint{2.268570in}{2.754568in}}%
\pgfpathlineto{\pgfqpoint{2.282990in}{2.747179in}}%
\pgfpathlineto{\pgfqpoint{2.363152in}{2.699318in}}%
\pgfpathlineto{\pgfqpoint{2.443313in}{2.647649in}}%
\pgfpathlineto{\pgfqpoint{2.486323in}{2.618667in}}%
\pgfpathlineto{\pgfqpoint{2.563556in}{2.563974in}}%
\pgfpathlineto{\pgfqpoint{2.643717in}{2.504657in}}%
\pgfpathlineto{\pgfqpoint{2.663933in}{2.488164in}}%
\pgfpathlineto{\pgfqpoint{2.689412in}{2.469333in}}%
\pgfpathlineto{\pgfqpoint{2.707604in}{2.454174in}}%
\pgfpathlineto{\pgfqpoint{2.736452in}{2.432000in}}%
\pgfpathlineto{\pgfqpoint{2.772374in}{2.402505in}}%
\pgfpathlineto{\pgfqpoint{2.804040in}{2.376877in}}%
\pgfpathlineto{\pgfqpoint{2.835965in}{2.349737in}}%
\pgfpathlineto{\pgfqpoint{2.856818in}{2.331826in}}%
\pgfpathlineto{\pgfqpoint{2.884202in}{2.309030in}}%
\pgfpathlineto{\pgfqpoint{2.964364in}{2.238504in}}%
\pgfpathlineto{\pgfqpoint{2.980540in}{2.223068in}}%
\pgfpathlineto{\pgfqpoint{3.004444in}{2.202217in}}%
\pgfpathlineto{\pgfqpoint{3.021022in}{2.186108in}}%
\pgfpathlineto{\pgfqpoint{3.044525in}{2.165238in}}%
\pgfpathlineto{\pgfqpoint{3.061135in}{2.148804in}}%
\pgfpathlineto{\pgfqpoint{3.084606in}{2.127557in}}%
\pgfpathlineto{\pgfqpoint{3.100884in}{2.111163in}}%
\pgfpathlineto{\pgfqpoint{3.124687in}{2.089168in}}%
\pgfpathlineto{\pgfqpoint{3.160141in}{2.054357in}}%
\pgfpathlineto{\pgfqpoint{3.164768in}{2.050062in}}%
\pgfpathlineto{\pgfqpoint{3.198935in}{2.015825in}}%
\pgfpathlineto{\pgfqpoint{3.218012in}{1.996261in}}%
\pgfpathlineto{\pgfqpoint{3.244929in}{1.969666in}}%
\pgfpathlineto{\pgfqpoint{3.267284in}{1.946667in}}%
\pgfpathlineto{\pgfqpoint{3.338512in}{1.872000in}}%
\pgfpathlineto{\pgfqpoint{3.369437in}{1.838640in}}%
\pgfpathlineto{\pgfqpoint{3.373305in}{1.834667in}}%
\pgfpathlineto{\pgfqpoint{3.445333in}{1.755251in}}%
\pgfpathlineto{\pgfqpoint{3.478942in}{1.716638in}}%
\pgfpathlineto{\pgfqpoint{3.506375in}{1.685333in}}%
\pgfpathlineto{\pgfqpoint{3.584270in}{1.593254in}}%
\pgfpathlineto{\pgfqpoint{3.645737in}{1.517168in}}%
\pgfpathlineto{\pgfqpoint{3.660348in}{1.498667in}}%
\pgfpathlineto{\pgfqpoint{3.725899in}{1.413666in}}%
\pgfpathlineto{\pgfqpoint{3.745971in}{1.386667in}}%
\pgfpathlineto{\pgfqpoint{3.806061in}{1.303861in}}%
\pgfpathlineto{\pgfqpoint{3.826355in}{1.274667in}}%
\pgfpathlineto{\pgfqpoint{3.852083in}{1.237333in}}%
\pgfpathlineto{\pgfqpoint{3.864654in}{1.217244in}}%
\pgfpathlineto{\pgfqpoint{3.886222in}{1.185820in}}%
\pgfpathlineto{\pgfqpoint{3.926303in}{1.123223in}}%
\pgfpathlineto{\pgfqpoint{3.974318in}{1.043277in}}%
\pgfpathlineto{\pgfqpoint{4.011854in}{0.976000in}}%
\pgfpathlineto{\pgfqpoint{4.022681in}{0.953771in}}%
\pgfpathlineto{\pgfqpoint{4.031378in}{0.938667in}}%
\pgfpathlineto{\pgfqpoint{4.050373in}{0.901333in}}%
\pgfpathlineto{\pgfqpoint{4.067925in}{0.864000in}}%
\pgfpathlineto{\pgfqpoint{4.086626in}{0.822719in}}%
\pgfpathlineto{\pgfqpoint{4.127842in}{0.714667in}}%
\pgfpathlineto{\pgfqpoint{4.148315in}{0.640000in}}%
\pgfpathlineto{\pgfqpoint{4.155842in}{0.602667in}}%
\pgfpathlineto{\pgfqpoint{4.156445in}{0.593032in}}%
\pgfpathlineto{\pgfqpoint{4.160931in}{0.565333in}}%
\pgfpathlineto{\pgfqpoint{4.162879in}{0.528000in}}%
\pgfpathlineto{\pgfqpoint{4.189291in}{0.528000in}}%
\pgfpathlineto{\pgfqpoint{4.188253in}{0.545340in}}%
\pgfpathlineto{\pgfqpoint{4.182101in}{0.588404in}}%
\pgfpathlineto{\pgfqpoint{4.169658in}{0.642673in}}%
\pgfpathlineto{\pgfqpoint{4.159943in}{0.677333in}}%
\pgfpathlineto{\pgfqpoint{4.151470in}{0.700399in}}%
\pgfpathlineto{\pgfqpoint{4.147562in}{0.714667in}}%
\pgfpathlineto{\pgfqpoint{4.141717in}{0.728648in}}%
\pgfpathlineto{\pgfqpoint{4.131856in}{0.756796in}}%
\pgfpathlineto{\pgfqpoint{4.119116in}{0.789333in}}%
\pgfpathlineto{\pgfqpoint{4.109146in}{0.810309in}}%
\pgfpathlineto{\pgfqpoint{4.102848in}{0.826667in}}%
\pgfpathlineto{\pgfqpoint{4.097592in}{0.836881in}}%
\pgfpathlineto{\pgfqpoint{4.085962in}{0.864000in}}%
\pgfpathlineto{\pgfqpoint{4.060230in}{0.914080in}}%
\pgfpathlineto{\pgfqpoint{4.046545in}{0.942368in}}%
\pgfpathlineto{\pgfqpoint{4.020366in}{0.988949in}}%
\pgfpathlineto{\pgfqpoint{4.006465in}{1.015403in}}%
\pgfpathlineto{\pgfqpoint{3.959988in}{1.093957in}}%
\pgfpathlineto{\pgfqpoint{3.892185in}{1.200000in}}%
\pgfpathlineto{\pgfqpoint{3.874045in}{1.225991in}}%
\pgfpathlineto{\pgfqpoint{3.866948in}{1.237333in}}%
\pgfpathlineto{\pgfqpoint{3.841426in}{1.274667in}}%
\pgfpathlineto{\pgfqpoint{3.826900in}{1.294078in}}%
\pgfpathlineto{\pgfqpoint{3.806061in}{1.324516in}}%
\pgfpathlineto{\pgfqpoint{3.788038in}{1.349333in}}%
\pgfpathlineto{\pgfqpoint{3.725899in}{1.432843in}}%
\pgfpathlineto{\pgfqpoint{3.645301in}{1.536000in}}%
\pgfpathlineto{\pgfqpoint{3.627718in}{1.556549in}}%
\pgfpathlineto{\pgfqpoint{3.605657in}{1.584537in}}%
\pgfpathlineto{\pgfqpoint{3.520900in}{1.685333in}}%
\pgfpathlineto{\pgfqpoint{3.445333in}{1.771283in}}%
\pgfpathlineto{\pgfqpoint{3.353121in}{1.872000in}}%
\pgfpathlineto{\pgfqpoint{3.282297in}{1.946667in}}%
\pgfpathlineto{\pgfqpoint{3.264046in}{1.964473in}}%
\pgfpathlineto{\pgfqpoint{3.244929in}{1.985050in}}%
\pgfpathlineto{\pgfqpoint{3.225733in}{2.003453in}}%
\pgfpathlineto{\pgfqpoint{3.204848in}{2.025382in}}%
\pgfpathlineto{\pgfqpoint{3.187080in}{2.042116in}}%
\pgfpathlineto{\pgfqpoint{3.164768in}{2.065021in}}%
\pgfpathlineto{\pgfqpoint{3.148081in}{2.080457in}}%
\pgfpathlineto{\pgfqpoint{3.124687in}{2.103977in}}%
\pgfpathlineto{\pgfqpoint{3.093974in}{2.133333in}}%
\pgfpathlineto{\pgfqpoint{3.044525in}{2.179864in}}%
\pgfpathlineto{\pgfqpoint{3.014029in}{2.208000in}}%
\pgfpathlineto{\pgfqpoint{2.964364in}{2.253102in}}%
\pgfpathlineto{\pgfqpoint{2.947703in}{2.267148in}}%
\pgfpathlineto{\pgfqpoint{2.924283in}{2.288744in}}%
\pgfpathlineto{\pgfqpoint{2.844121in}{2.358112in}}%
\pgfpathlineto{\pgfqpoint{2.822951in}{2.374948in}}%
\pgfpathlineto{\pgfqpoint{2.800428in}{2.394667in}}%
\pgfpathlineto{\pgfqpoint{2.780575in}{2.410143in}}%
\pgfpathlineto{\pgfqpoint{2.754754in}{2.432000in}}%
\pgfpathlineto{\pgfqpoint{2.683798in}{2.488406in}}%
\pgfpathlineto{\pgfqpoint{2.660194in}{2.506667in}}%
\pgfpathlineto{\pgfqpoint{2.603636in}{2.549702in}}%
\pgfpathlineto{\pgfqpoint{2.523475in}{2.607935in}}%
\pgfpathlineto{\pgfqpoint{2.443313in}{2.663287in}}%
\pgfpathlineto{\pgfqpoint{2.423908in}{2.675259in}}%
\pgfpathlineto{\pgfqpoint{2.397724in}{2.693333in}}%
\pgfpathlineto{\pgfqpoint{2.323071in}{2.740077in}}%
\pgfpathlineto{\pgfqpoint{2.275965in}{2.768000in}}%
\pgfpathlineto{\pgfqpoint{2.229836in}{2.793156in}}%
\pgfpathlineto{\pgfqpoint{2.202828in}{2.808714in}}%
\pgfpathlineto{\pgfqpoint{2.178729in}{2.820219in}}%
\pgfpathlineto{\pgfqpoint{2.162747in}{2.829257in}}%
\pgfpathlineto{\pgfqpoint{2.122667in}{2.848906in}}%
\pgfpathlineto{\pgfqpoint{2.099564in}{2.858481in}}%
\pgfpathlineto{\pgfqpoint{2.082586in}{2.867036in}}%
\pgfpathlineto{\pgfqpoint{2.034829in}{2.887150in}}%
\pgfpathlineto{\pgfqpoint{2.002424in}{2.899080in}}%
\pgfpathlineto{\pgfqpoint{1.987403in}{2.903342in}}%
\pgfpathlineto{\pgfqpoint{1.958363in}{2.913626in}}%
\pgfpathlineto{\pgfqpoint{1.922263in}{2.924176in}}%
\pgfpathlineto{\pgfqpoint{1.895158in}{2.929420in}}%
\pgfpathlineto{\pgfqpoint{1.882182in}{2.933129in}}%
\pgfpathlineto{\pgfqpoint{1.860936in}{2.937123in}}%
\pgfpathlineto{\pgfqpoint{1.815604in}{2.942014in}}%
\pgfpathlineto{\pgfqpoint{1.802020in}{2.942238in}}%
\pgfpathlineto{\pgfqpoint{1.787061in}{2.940733in}}%
\pgfpathlineto{\pgfqpoint{1.761939in}{2.940478in}}%
\pgfpathlineto{\pgfqpoint{1.741382in}{2.935519in}}%
\pgfpathlineto{\pgfqpoint{1.721859in}{2.932454in}}%
\pgfpathlineto{\pgfqpoint{1.709605in}{2.928747in}}%
\pgfpathlineto{\pgfqpoint{1.681778in}{2.915239in}}%
\pgfpathlineto{\pgfqpoint{1.641697in}{2.878484in}}%
\pgfpathlineto{\pgfqpoint{1.622760in}{2.842667in}}%
\pgfpathlineto{\pgfqpoint{1.616761in}{2.819440in}}%
\pgfpathlineto{\pgfqpoint{1.611870in}{2.805333in}}%
\pgfpathlineto{\pgfqpoint{1.607311in}{2.768000in}}%
\pgfpathlineto{\pgfqpoint{1.607513in}{2.725174in}}%
\pgfpathlineto{\pgfqpoint{1.612086in}{2.683581in}}%
\pgfpathlineto{\pgfqpoint{1.617251in}{2.656000in}}%
\pgfpathlineto{\pgfqpoint{1.622224in}{2.637862in}}%
\pgfpathlineto{\pgfqpoint{1.625669in}{2.618667in}}%
\pgfpathlineto{\pgfqpoint{1.637369in}{2.577302in}}%
\pgfpathlineto{\pgfqpoint{1.650523in}{2.535779in}}%
\pgfpathlineto{\pgfqpoint{1.676072in}{2.469333in}}%
\pgfpathlineto{\pgfqpoint{1.692170in}{2.432000in}}%
\pgfpathlineto{\pgfqpoint{1.701639in}{2.413166in}}%
\pgfpathlineto{\pgfqpoint{1.713492in}{2.386874in}}%
\pgfpathlineto{\pgfqpoint{1.727242in}{2.357333in}}%
\pgfpathlineto{\pgfqpoint{1.739031in}{2.335995in}}%
\pgfpathlineto{\pgfqpoint{1.746453in}{2.320000in}}%
\pgfpathlineto{\pgfqpoint{1.766175in}{2.282667in}}%
\pgfpathlineto{\pgfqpoint{1.779002in}{2.261226in}}%
\pgfpathlineto{\pgfqpoint{1.787118in}{2.245333in}}%
\pgfpathlineto{\pgfqpoint{1.808527in}{2.208000in}}%
\pgfpathlineto{\pgfqpoint{1.821072in}{2.188412in}}%
\pgfpathlineto{\pgfqpoint{1.842101in}{2.152369in}}%
\pgfpathlineto{\pgfqpoint{1.864874in}{2.117212in}}%
\pgfpathlineto{\pgfqpoint{1.882182in}{2.088769in}}%
\pgfpathlineto{\pgfqpoint{1.902026in}{2.058667in}}%
\pgfpathlineto{\pgfqpoint{1.926840in}{2.021333in}}%
\pgfpathlineto{\pgfqpoint{1.941208in}{2.001647in}}%
\pgfpathlineto{\pgfqpoint{1.962343in}{1.970011in}}%
\pgfpathlineto{\pgfqpoint{2.033031in}{1.872000in}}%
\pgfpathlineto{\pgfqpoint{2.042505in}{1.859284in}}%
\pgfpathlineto{\pgfqpoint{2.089503in}{1.797333in}}%
\pgfpathlineto{\pgfqpoint{2.103902in}{1.779855in}}%
\pgfpathlineto{\pgfqpoint{2.122667in}{1.754771in}}%
\pgfpathlineto{\pgfqpoint{2.148348in}{1.722667in}}%
\pgfpathlineto{\pgfqpoint{2.209155in}{1.648000in}}%
\pgfpathlineto{\pgfqpoint{2.224390in}{1.630750in}}%
\pgfpathlineto{\pgfqpoint{2.242909in}{1.607635in}}%
\pgfpathlineto{\pgfqpoint{2.277217in}{1.567956in}}%
\pgfpathlineto{\pgfqpoint{2.304744in}{1.536000in}}%
\pgfpathlineto{\pgfqpoint{2.370992in}{1.461333in}}%
\pgfpathlineto{\pgfqpoint{2.386194in}{1.445463in}}%
\pgfpathlineto{\pgfqpoint{2.414773in}{1.413250in}}%
\pgfpathlineto{\pgfqpoint{2.483394in}{1.340194in}}%
\pgfpathlineto{\pgfqpoint{2.583766in}{1.237333in}}%
\pgfpathlineto{\pgfqpoint{2.612962in}{1.208687in}}%
\pgfpathlineto{\pgfqpoint{2.643717in}{1.177945in}}%
\pgfpathlineto{\pgfqpoint{2.737812in}{1.088000in}}%
\pgfpathlineto{\pgfqpoint{2.819041in}{1.013333in}}%
\pgfpathlineto{\pgfqpoint{2.903418in}{0.938667in}}%
\pgfpathlineto{\pgfqpoint{2.991243in}{0.864000in}}%
\pgfpathlineto{\pgfqpoint{3.040897in}{0.823287in}}%
\pgfpathlineto{\pgfqpoint{3.044525in}{0.820151in}}%
\pgfpathlineto{\pgfqpoint{3.062776in}{0.806333in}}%
\pgfpathlineto{\pgfqpoint{3.084606in}{0.787929in}}%
\pgfpathlineto{\pgfqpoint{3.106085in}{0.772006in}}%
\pgfpathlineto{\pgfqpoint{3.130649in}{0.752000in}}%
\pgfpathlineto{\pgfqpoint{3.171672in}{0.721098in}}%
\pgfpathlineto{\pgfqpoint{3.204848in}{0.695942in}}%
\pgfpathlineto{\pgfqpoint{3.294403in}{0.631251in}}%
\pgfpathlineto{\pgfqpoint{3.365172in}{0.583184in}}%
\pgfpathlineto{\pgfqpoint{3.450910in}{0.528000in}}%
\pgfpathlineto{\pgfqpoint{3.450910in}{0.528000in}}%
\pgfusepath{fill}%
\end{pgfscope}%
\begin{pgfscope}%
\pgfpathrectangle{\pgfqpoint{0.800000in}{0.528000in}}{\pgfqpoint{3.968000in}{3.696000in}}%
\pgfusepath{clip}%
\pgfsetbuttcap%
\pgfsetroundjoin%
\definecolor{currentfill}{rgb}{0.274952,0.037752,0.364543}%
\pgfsetfillcolor{currentfill}%
\pgfsetlinewidth{0.000000pt}%
\definecolor{currentstroke}{rgb}{0.000000,0.000000,0.000000}%
\pgfsetstrokecolor{currentstroke}%
\pgfsetdash{}{0pt}%
\pgfpathmoveto{\pgfqpoint{3.450910in}{0.528000in}}%
\pgfpathlineto{\pgfqpoint{3.325091in}{0.610126in}}%
\pgfpathlineto{\pgfqpoint{3.281982in}{0.640000in}}%
\pgfpathlineto{\pgfqpoint{3.261540in}{0.655472in}}%
\pgfpathlineto{\pgfqpoint{3.230214in}{0.677333in}}%
\pgfpathlineto{\pgfqpoint{3.164768in}{0.725938in}}%
\pgfpathlineto{\pgfqpoint{3.149827in}{0.738083in}}%
\pgfpathlineto{\pgfqpoint{3.124687in}{0.756571in}}%
\pgfpathlineto{\pgfqpoint{3.106085in}{0.772006in}}%
\pgfpathlineto{\pgfqpoint{3.082853in}{0.789333in}}%
\pgfpathlineto{\pgfqpoint{3.062776in}{0.806333in}}%
\pgfpathlineto{\pgfqpoint{3.036552in}{0.826667in}}%
\pgfpathlineto{\pgfqpoint{2.946880in}{0.901333in}}%
\pgfpathlineto{\pgfqpoint{2.860818in}{0.976000in}}%
\pgfpathlineto{\pgfqpoint{2.844121in}{0.990787in}}%
\pgfpathlineto{\pgfqpoint{2.763960in}{1.063616in}}%
\pgfpathlineto{\pgfqpoint{2.683798in}{1.139147in}}%
\pgfpathlineto{\pgfqpoint{2.603636in}{1.217440in}}%
\pgfpathlineto{\pgfqpoint{2.583766in}{1.237333in}}%
\pgfpathlineto{\pgfqpoint{2.510501in}{1.312000in}}%
\pgfpathlineto{\pgfqpoint{2.439482in}{1.386667in}}%
\pgfpathlineto{\pgfqpoint{2.363152in}{1.470018in}}%
\pgfpathlineto{\pgfqpoint{2.272321in}{1.573333in}}%
\pgfpathlineto{\pgfqpoint{2.259690in}{1.588964in}}%
\pgfpathlineto{\pgfqpoint{2.240343in}{1.610667in}}%
\pgfpathlineto{\pgfqpoint{2.224390in}{1.630750in}}%
\pgfpathlineto{\pgfqpoint{2.202828in}{1.655637in}}%
\pgfpathlineto{\pgfqpoint{2.118537in}{1.760000in}}%
\pgfpathlineto{\pgfqpoint{2.103902in}{1.779855in}}%
\pgfpathlineto{\pgfqpoint{2.082586in}{1.806331in}}%
\pgfpathlineto{\pgfqpoint{2.061091in}{1.834667in}}%
\pgfpathlineto{\pgfqpoint{2.002424in}{1.913581in}}%
\pgfpathlineto{\pgfqpoint{1.952570in}{1.984000in}}%
\pgfpathlineto{\pgfqpoint{1.941208in}{2.001647in}}%
\pgfpathlineto{\pgfqpoint{1.922263in}{2.028143in}}%
\pgfpathlineto{\pgfqpoint{1.877480in}{2.096000in}}%
\pgfpathlineto{\pgfqpoint{1.864874in}{2.117212in}}%
\pgfpathlineto{\pgfqpoint{1.842101in}{2.152369in}}%
\pgfpathlineto{\pgfqpoint{1.802020in}{2.219230in}}%
\pgfpathlineto{\pgfqpoint{1.761939in}{2.290589in}}%
\pgfpathlineto{\pgfqpoint{1.721859in}{2.368403in}}%
\pgfpathlineto{\pgfqpoint{1.676072in}{2.469333in}}%
\pgfpathlineto{\pgfqpoint{1.667463in}{2.493333in}}%
\pgfpathlineto{\pgfqpoint{1.661482in}{2.506667in}}%
\pgfpathlineto{\pgfqpoint{1.647812in}{2.544000in}}%
\pgfpathlineto{\pgfqpoint{1.633366in}{2.589093in}}%
\pgfpathlineto{\pgfqpoint{1.625669in}{2.618667in}}%
\pgfpathlineto{\pgfqpoint{1.622224in}{2.637862in}}%
\pgfpathlineto{\pgfqpoint{1.617251in}{2.656000in}}%
\pgfpathlineto{\pgfqpoint{1.610967in}{2.693333in}}%
\pgfpathlineto{\pgfqpoint{1.607389in}{2.730667in}}%
\pgfpathlineto{\pgfqpoint{1.607854in}{2.736477in}}%
\pgfpathlineto{\pgfqpoint{1.607311in}{2.768000in}}%
\pgfpathlineto{\pgfqpoint{1.608617in}{2.774521in}}%
\pgfpathlineto{\pgfqpoint{1.611870in}{2.805333in}}%
\pgfpathlineto{\pgfqpoint{1.616761in}{2.819440in}}%
\pgfpathlineto{\pgfqpoint{1.622760in}{2.842667in}}%
\pgfpathlineto{\pgfqpoint{1.642806in}{2.880000in}}%
\pgfpathlineto{\pgfqpoint{1.685536in}{2.917333in}}%
\pgfpathlineto{\pgfqpoint{1.709605in}{2.928747in}}%
\pgfpathlineto{\pgfqpoint{1.721859in}{2.932454in}}%
\pgfpathlineto{\pgfqpoint{1.741382in}{2.935519in}}%
\pgfpathlineto{\pgfqpoint{1.761939in}{2.940478in}}%
\pgfpathlineto{\pgfqpoint{1.787061in}{2.940733in}}%
\pgfpathlineto{\pgfqpoint{1.802020in}{2.942238in}}%
\pgfpathlineto{\pgfqpoint{1.815604in}{2.942014in}}%
\pgfpathlineto{\pgfqpoint{1.860936in}{2.937123in}}%
\pgfpathlineto{\pgfqpoint{1.882182in}{2.933129in}}%
\pgfpathlineto{\pgfqpoint{1.895158in}{2.929420in}}%
\pgfpathlineto{\pgfqpoint{1.922263in}{2.924176in}}%
\pgfpathlineto{\pgfqpoint{1.962343in}{2.912772in}}%
\pgfpathlineto{\pgfqpoint{1.987403in}{2.903342in}}%
\pgfpathlineto{\pgfqpoint{2.002424in}{2.899080in}}%
\pgfpathlineto{\pgfqpoint{2.052059in}{2.880000in}}%
\pgfpathlineto{\pgfqpoint{2.082586in}{2.867036in}}%
\pgfpathlineto{\pgfqpoint{2.099564in}{2.858481in}}%
\pgfpathlineto{\pgfqpoint{2.135433in}{2.842667in}}%
\pgfpathlineto{\pgfqpoint{2.209002in}{2.805333in}}%
\pgfpathlineto{\pgfqpoint{2.254951in}{2.779216in}}%
\pgfpathlineto{\pgfqpoint{2.294444in}{2.757332in}}%
\pgfpathlineto{\pgfqpoint{2.363152in}{2.715270in}}%
\pgfpathlineto{\pgfqpoint{2.403232in}{2.689824in}}%
\pgfpathlineto{\pgfqpoint{2.423908in}{2.675259in}}%
\pgfpathlineto{\pgfqpoint{2.454017in}{2.656000in}}%
\pgfpathlineto{\pgfqpoint{2.523475in}{2.607935in}}%
\pgfpathlineto{\pgfqpoint{2.611193in}{2.544000in}}%
\pgfpathlineto{\pgfqpoint{2.708028in}{2.469333in}}%
\pgfpathlineto{\pgfqpoint{2.723879in}{2.456811in}}%
\pgfpathlineto{\pgfqpoint{2.804040in}{2.391693in}}%
\pgfpathlineto{\pgfqpoint{2.822951in}{2.374948in}}%
\pgfpathlineto{\pgfqpoint{2.845033in}{2.357333in}}%
\pgfpathlineto{\pgfqpoint{2.864927in}{2.339379in}}%
\pgfpathlineto{\pgfqpoint{2.888505in}{2.320000in}}%
\pgfpathlineto{\pgfqpoint{2.972967in}{2.245333in}}%
\pgfpathlineto{\pgfqpoint{3.004444in}{2.216811in}}%
\pgfpathlineto{\pgfqpoint{3.093974in}{2.133333in}}%
\pgfpathlineto{\pgfqpoint{3.124687in}{2.103977in}}%
\pgfpathlineto{\pgfqpoint{3.208885in}{2.021333in}}%
\pgfpathlineto{\pgfqpoint{3.225733in}{2.003453in}}%
\pgfpathlineto{\pgfqpoint{3.245959in}{1.984000in}}%
\pgfpathlineto{\pgfqpoint{3.264046in}{1.964473in}}%
\pgfpathlineto{\pgfqpoint{3.285010in}{1.943864in}}%
\pgfpathlineto{\pgfqpoint{3.302025in}{1.925181in}}%
\pgfpathlineto{\pgfqpoint{3.325091in}{1.901867in}}%
\pgfpathlineto{\pgfqpoint{3.405253in}{1.815585in}}%
\pgfpathlineto{\pgfqpoint{3.421804in}{1.797333in}}%
\pgfpathlineto{\pgfqpoint{3.501992in}{1.707225in}}%
\pgfpathlineto{\pgfqpoint{3.583954in}{1.610667in}}%
\pgfpathlineto{\pgfqpoint{3.645737in}{1.535455in}}%
\pgfpathlineto{\pgfqpoint{3.674790in}{1.498667in}}%
\pgfpathlineto{\pgfqpoint{3.732639in}{1.424000in}}%
\pgfpathlineto{\pgfqpoint{3.745901in}{1.405297in}}%
\pgfpathlineto{\pgfqpoint{3.765980in}{1.379558in}}%
\pgfpathlineto{\pgfqpoint{3.815030in}{1.312000in}}%
\pgfpathlineto{\pgfqpoint{3.826900in}{1.294078in}}%
\pgfpathlineto{\pgfqpoint{3.846141in}{1.267846in}}%
\pgfpathlineto{\pgfqpoint{3.916556in}{1.162667in}}%
\pgfpathlineto{\pgfqpoint{3.926303in}{1.147456in}}%
\pgfpathlineto{\pgfqpoint{3.966384in}{1.083349in}}%
\pgfpathlineto{\pgfqpoint{3.985730in}{1.050667in}}%
\pgfpathlineto{\pgfqpoint{4.007664in}{1.013333in}}%
\pgfpathlineto{\pgfqpoint{4.033940in}{0.964259in}}%
\pgfpathlineto{\pgfqpoint{4.048532in}{0.938667in}}%
\pgfpathlineto{\pgfqpoint{4.072972in}{0.888615in}}%
\pgfpathlineto{\pgfqpoint{4.086626in}{0.862553in}}%
\pgfpathlineto{\pgfqpoint{4.109146in}{0.810309in}}%
\pgfpathlineto{\pgfqpoint{4.119116in}{0.789333in}}%
\pgfpathlineto{\pgfqpoint{4.134075in}{0.752000in}}%
\pgfpathlineto{\pgfqpoint{4.141717in}{0.728648in}}%
\pgfpathlineto{\pgfqpoint{4.147562in}{0.714667in}}%
\pgfpathlineto{\pgfqpoint{4.151470in}{0.700399in}}%
\pgfpathlineto{\pgfqpoint{4.159943in}{0.677333in}}%
\pgfpathlineto{\pgfqpoint{4.171936in}{0.635205in}}%
\pgfpathlineto{\pgfqpoint{4.182101in}{0.588404in}}%
\pgfpathlineto{\pgfqpoint{4.188253in}{0.545340in}}%
\pgfpathlineto{\pgfqpoint{4.189291in}{0.528000in}}%
\pgfpathlineto{\pgfqpoint{4.214447in}{0.528000in}}%
\pgfpathlineto{\pgfqpoint{4.213162in}{0.533862in}}%
\pgfpathlineto{\pgfqpoint{4.206869in}{0.576335in}}%
\pgfpathlineto{\pgfqpoint{4.199813in}{0.609239in}}%
\pgfpathlineto{\pgfqpoint{4.191292in}{0.640000in}}%
\pgfpathlineto{\pgfqpoint{4.185201in}{0.657151in}}%
\pgfpathlineto{\pgfqpoint{4.179869in}{0.677333in}}%
\pgfpathlineto{\pgfqpoint{4.176281in}{0.686176in}}%
\pgfpathlineto{\pgfqpoint{4.166788in}{0.715847in}}%
\pgfpathlineto{\pgfqpoint{4.120590in}{0.826667in}}%
\pgfpathlineto{\pgfqpoint{4.109585in}{0.848051in}}%
\pgfpathlineto{\pgfqpoint{4.102773in}{0.864000in}}%
\pgfpathlineto{\pgfqpoint{4.084428in}{0.901333in}}%
\pgfpathlineto{\pgfqpoint{4.064657in}{0.938667in}}%
\pgfpathlineto{\pgfqpoint{4.044590in}{0.976000in}}%
\pgfpathlineto{\pgfqpoint{4.016982in}{1.023130in}}%
\pgfpathlineto{\pgfqpoint{4.001487in}{1.050667in}}%
\pgfpathlineto{\pgfqpoint{3.988280in}{1.071062in}}%
\pgfpathlineto{\pgfqpoint{3.966384in}{1.107961in}}%
\pgfpathlineto{\pgfqpoint{3.926303in}{1.170762in}}%
\pgfpathlineto{\pgfqpoint{3.855879in}{1.274667in}}%
\pgfpathlineto{\pgfqpoint{3.835781in}{1.302350in}}%
\pgfpathlineto{\pgfqpoint{3.829389in}{1.312000in}}%
\pgfpathlineto{\pgfqpoint{3.765980in}{1.398597in}}%
\pgfpathlineto{\pgfqpoint{3.746707in}{1.424000in}}%
\pgfpathlineto{\pgfqpoint{3.685818in}{1.502727in}}%
\pgfpathlineto{\pgfqpoint{3.653185in}{1.542937in}}%
\pgfpathlineto{\pgfqpoint{3.628807in}{1.573333in}}%
\pgfpathlineto{\pgfqpoint{3.605657in}{1.601569in}}%
\pgfpathlineto{\pgfqpoint{3.534853in}{1.685333in}}%
\pgfpathlineto{\pgfqpoint{3.469436in}{1.760000in}}%
\pgfpathlineto{\pgfqpoint{3.445333in}{1.787034in}}%
\pgfpathlineto{\pgfqpoint{3.365172in}{1.874585in}}%
\pgfpathlineto{\pgfqpoint{3.285010in}{1.958667in}}%
\pgfpathlineto{\pgfqpoint{3.260266in}{1.984000in}}%
\pgfpathlineto{\pgfqpoint{3.185907in}{2.058667in}}%
\pgfpathlineto{\pgfqpoint{3.109077in}{2.133333in}}%
\pgfpathlineto{\pgfqpoint{3.029564in}{2.208000in}}%
\pgfpathlineto{\pgfqpoint{2.947130in}{2.282667in}}%
\pgfpathlineto{\pgfqpoint{2.861511in}{2.357333in}}%
\pgfpathlineto{\pgfqpoint{2.804040in}{2.405881in}}%
\pgfpathlineto{\pgfqpoint{2.788775in}{2.417781in}}%
\pgfpathlineto{\pgfqpoint{2.763960in}{2.438949in}}%
\pgfpathlineto{\pgfqpoint{2.679135in}{2.506667in}}%
\pgfpathlineto{\pgfqpoint{2.580587in}{2.581333in}}%
\pgfpathlineto{\pgfqpoint{2.563556in}{2.593912in}}%
\pgfpathlineto{\pgfqpoint{2.476286in}{2.656000in}}%
\pgfpathlineto{\pgfqpoint{2.403232in}{2.705126in}}%
\pgfpathlineto{\pgfqpoint{2.363152in}{2.731191in}}%
\pgfpathlineto{\pgfqpoint{2.338872in}{2.745385in}}%
\pgfpathlineto{\pgfqpoint{2.303104in}{2.768000in}}%
\pgfpathlineto{\pgfqpoint{2.239483in}{2.805333in}}%
\pgfpathlineto{\pgfqpoint{2.153017in}{2.851731in}}%
\pgfpathlineto{\pgfqpoint{2.082586in}{2.885478in}}%
\pgfpathlineto{\pgfqpoint{2.058597in}{2.894989in}}%
\pgfpathlineto{\pgfqpoint{2.042505in}{2.902781in}}%
\pgfpathlineto{\pgfqpoint{1.999325in}{2.920220in}}%
\pgfpathlineto{\pgfqpoint{1.936171in}{2.941711in}}%
\pgfpathlineto{\pgfqpoint{1.879900in}{2.956792in}}%
\pgfpathlineto{\pgfqpoint{1.842101in}{2.963856in}}%
\pgfpathlineto{\pgfqpoint{1.814539in}{2.966327in}}%
\pgfpathlineto{\pgfqpoint{1.802020in}{2.968584in}}%
\pgfpathlineto{\pgfqpoint{1.776656in}{2.968374in}}%
\pgfpathlineto{\pgfqpoint{1.761939in}{2.969576in}}%
\pgfpathlineto{\pgfqpoint{1.746505in}{2.969043in}}%
\pgfpathlineto{\pgfqpoint{1.721859in}{2.965622in}}%
\pgfpathlineto{\pgfqpoint{1.681197in}{2.954667in}}%
\pgfpathlineto{\pgfqpoint{1.655681in}{2.941641in}}%
\pgfpathlineto{\pgfqpoint{1.641697in}{2.931245in}}%
\pgfpathlineto{\pgfqpoint{1.626834in}{2.917333in}}%
\pgfpathlineto{\pgfqpoint{1.615450in}{2.904448in}}%
\pgfpathlineto{\pgfqpoint{1.600647in}{2.880000in}}%
\pgfpathlineto{\pgfqpoint{1.590323in}{2.853185in}}%
\pgfpathlineto{\pgfqpoint{1.583348in}{2.822349in}}%
\pgfpathlineto{\pgfqpoint{1.580048in}{2.788089in}}%
\pgfpathlineto{\pgfqpoint{1.580494in}{2.750341in}}%
\pgfpathlineto{\pgfqpoint{1.582461in}{2.730667in}}%
\pgfpathlineto{\pgfqpoint{1.585366in}{2.715530in}}%
\pgfpathlineto{\pgfqpoint{1.587546in}{2.693333in}}%
\pgfpathlineto{\pgfqpoint{1.590313in}{2.682805in}}%
\pgfpathlineto{\pgfqpoint{1.594890in}{2.656000in}}%
\pgfpathlineto{\pgfqpoint{1.605125in}{2.615398in}}%
\pgfpathlineto{\pgfqpoint{1.615812in}{2.581333in}}%
\pgfpathlineto{\pgfqpoint{1.622700in}{2.563638in}}%
\pgfpathlineto{\pgfqpoint{1.628573in}{2.544000in}}%
\pgfpathlineto{\pgfqpoint{1.642396in}{2.506667in}}%
\pgfpathlineto{\pgfqpoint{1.658113in}{2.469333in}}%
\pgfpathlineto{\pgfqpoint{1.681778in}{2.416349in}}%
\pgfpathlineto{\pgfqpoint{1.750197in}{2.282667in}}%
\pgfpathlineto{\pgfqpoint{1.792976in}{2.208000in}}%
\pgfpathlineto{\pgfqpoint{1.842101in}{2.127811in}}%
\pgfpathlineto{\pgfqpoint{1.862667in}{2.096000in}}%
\pgfpathlineto{\pgfqpoint{1.912254in}{2.021333in}}%
\pgfpathlineto{\pgfqpoint{1.967782in}{1.941601in}}%
\pgfpathlineto{\pgfqpoint{2.046699in}{1.834667in}}%
\pgfpathlineto{\pgfqpoint{2.062124in}{1.815608in}}%
\pgfpathlineto{\pgfqpoint{2.082586in}{1.788025in}}%
\pgfpathlineto{\pgfqpoint{2.104619in}{1.760000in}}%
\pgfpathlineto{\pgfqpoint{2.169107in}{1.679410in}}%
\pgfpathlineto{\pgfqpoint{2.242909in}{1.591313in}}%
\pgfpathlineto{\pgfqpoint{2.269643in}{1.560901in}}%
\pgfpathlineto{\pgfqpoint{2.290565in}{1.536000in}}%
\pgfpathlineto{\pgfqpoint{2.363152in}{1.454320in}}%
\pgfpathlineto{\pgfqpoint{2.390868in}{1.424000in}}%
\pgfpathlineto{\pgfqpoint{2.460464in}{1.349333in}}%
\pgfpathlineto{\pgfqpoint{2.483394in}{1.325167in}}%
\pgfpathlineto{\pgfqpoint{2.568923in}{1.237333in}}%
\pgfpathlineto{\pgfqpoint{2.586097in}{1.220996in}}%
\pgfpathlineto{\pgfqpoint{2.606250in}{1.200000in}}%
\pgfpathlineto{\pgfqpoint{2.624948in}{1.182518in}}%
\pgfpathlineto{\pgfqpoint{2.657148in}{1.150157in}}%
\pgfpathlineto{\pgfqpoint{2.723879in}{1.086432in}}%
\pgfpathlineto{\pgfqpoint{2.804040in}{1.012334in}}%
\pgfpathlineto{\pgfqpoint{2.886753in}{0.938667in}}%
\pgfpathlineto{\pgfqpoint{2.974050in}{0.864000in}}%
\pgfpathlineto{\pgfqpoint{3.004444in}{0.838672in}}%
\pgfpathlineto{\pgfqpoint{3.084606in}{0.773704in}}%
\pgfpathlineto{\pgfqpoint{3.119102in}{0.746798in}}%
\pgfpathlineto{\pgfqpoint{3.124687in}{0.742152in}}%
\pgfpathlineto{\pgfqpoint{3.141273in}{0.730116in}}%
\pgfpathlineto{\pgfqpoint{3.164768in}{0.711212in}}%
\pgfpathlineto{\pgfqpoint{3.185406in}{0.696557in}}%
\pgfpathlineto{\pgfqpoint{3.209951in}{0.677333in}}%
\pgfpathlineto{\pgfqpoint{3.252283in}{0.646850in}}%
\pgfpathlineto{\pgfqpoint{3.285010in}{0.623080in}}%
\pgfpathlineto{\pgfqpoint{3.368740in}{0.565333in}}%
\pgfpathlineto{\pgfqpoint{3.426225in}{0.528000in}}%
\pgfpathlineto{\pgfqpoint{3.445333in}{0.528000in}}%
\pgfpathlineto{\pgfqpoint{3.445333in}{0.528000in}}%
\pgfusepath{fill}%
\end{pgfscope}%
\begin{pgfscope}%
\pgfpathrectangle{\pgfqpoint{0.800000in}{0.528000in}}{\pgfqpoint{3.968000in}{3.696000in}}%
\pgfusepath{clip}%
\pgfsetbuttcap%
\pgfsetroundjoin%
\definecolor{currentfill}{rgb}{0.274952,0.037752,0.364543}%
\pgfsetfillcolor{currentfill}%
\pgfsetlinewidth{0.000000pt}%
\definecolor{currentstroke}{rgb}{0.000000,0.000000,0.000000}%
\pgfsetstrokecolor{currentstroke}%
\pgfsetdash{}{0pt}%
\pgfpathmoveto{\pgfqpoint{3.426225in}{0.528000in}}%
\pgfpathlineto{\pgfqpoint{3.405253in}{0.541431in}}%
\pgfpathlineto{\pgfqpoint{3.365172in}{0.567678in}}%
\pgfpathlineto{\pgfqpoint{3.344016in}{0.582961in}}%
\pgfpathlineto{\pgfqpoint{3.314128in}{0.602667in}}%
\pgfpathlineto{\pgfqpoint{3.244929in}{0.651767in}}%
\pgfpathlineto{\pgfqpoint{3.229987in}{0.663416in}}%
\pgfpathlineto{\pgfqpoint{3.204848in}{0.681077in}}%
\pgfpathlineto{\pgfqpoint{3.185406in}{0.696557in}}%
\pgfpathlineto{\pgfqpoint{3.160278in}{0.714667in}}%
\pgfpathlineto{\pgfqpoint{3.141273in}{0.730116in}}%
\pgfpathlineto{\pgfqpoint{3.112146in}{0.752000in}}%
\pgfpathlineto{\pgfqpoint{3.084606in}{0.773704in}}%
\pgfpathlineto{\pgfqpoint{3.004444in}{0.838672in}}%
\pgfpathlineto{\pgfqpoint{2.974050in}{0.864000in}}%
\pgfpathlineto{\pgfqpoint{2.924283in}{0.906169in}}%
\pgfpathlineto{\pgfqpoint{2.836391in}{0.983201in}}%
\pgfpathlineto{\pgfqpoint{2.762234in}{1.050667in}}%
\pgfpathlineto{\pgfqpoint{2.657148in}{1.150157in}}%
\pgfpathlineto{\pgfqpoint{2.603636in}{1.202581in}}%
\pgfpathlineto{\pgfqpoint{2.586097in}{1.220996in}}%
\pgfpathlineto{\pgfqpoint{2.563556in}{1.242726in}}%
\pgfpathlineto{\pgfqpoint{2.532198in}{1.274667in}}%
\pgfpathlineto{\pgfqpoint{2.460464in}{1.349333in}}%
\pgfpathlineto{\pgfqpoint{2.390868in}{1.424000in}}%
\pgfpathlineto{\pgfqpoint{2.378500in}{1.438296in}}%
\pgfpathlineto{\pgfqpoint{2.356821in}{1.461333in}}%
\pgfpathlineto{\pgfqpoint{2.282990in}{1.544685in}}%
\pgfpathlineto{\pgfqpoint{2.251705in}{1.581526in}}%
\pgfpathlineto{\pgfqpoint{2.226525in}{1.610667in}}%
\pgfpathlineto{\pgfqpoint{2.202828in}{1.638774in}}%
\pgfpathlineto{\pgfqpoint{2.134249in}{1.722667in}}%
\pgfpathlineto{\pgfqpoint{2.122667in}{1.737146in}}%
\pgfpathlineto{\pgfqpoint{2.075361in}{1.797333in}}%
\pgfpathlineto{\pgfqpoint{2.062124in}{1.815608in}}%
\pgfpathlineto{\pgfqpoint{2.042505in}{1.840221in}}%
\pgfpathlineto{\pgfqpoint{2.018827in}{1.872000in}}%
\pgfpathlineto{\pgfqpoint{1.962343in}{1.949255in}}%
\pgfpathlineto{\pgfqpoint{1.932143in}{1.993203in}}%
\pgfpathlineto{\pgfqpoint{1.912254in}{2.021333in}}%
\pgfpathlineto{\pgfqpoint{1.882182in}{2.065989in}}%
\pgfpathlineto{\pgfqpoint{1.833495in}{2.141350in}}%
\pgfpathlineto{\pgfqpoint{1.781694in}{2.226933in}}%
\pgfpathlineto{\pgfqpoint{1.737641in}{2.305300in}}%
\pgfpathlineto{\pgfqpoint{1.710569in}{2.357333in}}%
\pgfpathlineto{\pgfqpoint{1.702026in}{2.376193in}}%
\pgfpathlineto{\pgfqpoint{1.689202in}{2.401582in}}%
\pgfpathlineto{\pgfqpoint{1.668936in}{2.443961in}}%
\pgfpathlineto{\pgfqpoint{1.641697in}{2.508524in}}%
\pgfpathlineto{\pgfqpoint{1.628573in}{2.544000in}}%
\pgfpathlineto{\pgfqpoint{1.622700in}{2.563638in}}%
\pgfpathlineto{\pgfqpoint{1.615812in}{2.581333in}}%
\pgfpathlineto{\pgfqpoint{1.601616in}{2.629094in}}%
\pgfpathlineto{\pgfqpoint{1.594890in}{2.656000in}}%
\pgfpathlineto{\pgfqpoint{1.582461in}{2.730667in}}%
\pgfpathlineto{\pgfqpoint{1.580494in}{2.750341in}}%
\pgfpathlineto{\pgfqpoint{1.580048in}{2.788089in}}%
\pgfpathlineto{\pgfqpoint{1.583348in}{2.822349in}}%
\pgfpathlineto{\pgfqpoint{1.590323in}{2.853185in}}%
\pgfpathlineto{\pgfqpoint{1.601616in}{2.881863in}}%
\pgfpathlineto{\pgfqpoint{1.615450in}{2.904448in}}%
\pgfpathlineto{\pgfqpoint{1.626834in}{2.917333in}}%
\pgfpathlineto{\pgfqpoint{1.641697in}{2.931245in}}%
\pgfpathlineto{\pgfqpoint{1.655681in}{2.941641in}}%
\pgfpathlineto{\pgfqpoint{1.682124in}{2.954990in}}%
\pgfpathlineto{\pgfqpoint{1.721859in}{2.965622in}}%
\pgfpathlineto{\pgfqpoint{1.746505in}{2.969043in}}%
\pgfpathlineto{\pgfqpoint{1.761939in}{2.969576in}}%
\pgfpathlineto{\pgfqpoint{1.776656in}{2.968374in}}%
\pgfpathlineto{\pgfqpoint{1.802020in}{2.968584in}}%
\pgfpathlineto{\pgfqpoint{1.814539in}{2.966327in}}%
\pgfpathlineto{\pgfqpoint{1.842101in}{2.963856in}}%
\pgfpathlineto{\pgfqpoint{1.888162in}{2.954667in}}%
\pgfpathlineto{\pgfqpoint{1.962343in}{2.933087in}}%
\pgfpathlineto{\pgfqpoint{2.006610in}{2.917333in}}%
\pgfpathlineto{\pgfqpoint{2.042505in}{2.902781in}}%
\pgfpathlineto{\pgfqpoint{2.058597in}{2.894989in}}%
\pgfpathlineto{\pgfqpoint{2.094231in}{2.880000in}}%
\pgfpathlineto{\pgfqpoint{2.122667in}{2.866564in}}%
\pgfpathlineto{\pgfqpoint{2.170398in}{2.842667in}}%
\pgfpathlineto{\pgfqpoint{2.202828in}{2.825406in}}%
\pgfpathlineto{\pgfqpoint{2.242909in}{2.803449in}}%
\pgfpathlineto{\pgfqpoint{2.265752in}{2.789277in}}%
\pgfpathlineto{\pgfqpoint{2.282990in}{2.780076in}}%
\pgfpathlineto{\pgfqpoint{2.363960in}{2.730667in}}%
\pgfpathlineto{\pgfqpoint{2.443313in}{2.678446in}}%
\pgfpathlineto{\pgfqpoint{2.529420in}{2.618667in}}%
\pgfpathlineto{\pgfqpoint{2.603636in}{2.564249in}}%
\pgfpathlineto{\pgfqpoint{2.630470in}{2.544000in}}%
\pgfpathlineto{\pgfqpoint{2.708797in}{2.483381in}}%
\pgfpathlineto{\pgfqpoint{2.772406in}{2.432000in}}%
\pgfpathlineto{\pgfqpoint{2.804040in}{2.405881in}}%
\pgfpathlineto{\pgfqpoint{2.884202in}{2.337876in}}%
\pgfpathlineto{\pgfqpoint{2.904737in}{2.320000in}}%
\pgfpathlineto{\pgfqpoint{2.988728in}{2.245333in}}%
\pgfpathlineto{\pgfqpoint{3.004444in}{2.231092in}}%
\pgfpathlineto{\pgfqpoint{3.084606in}{2.156639in}}%
\pgfpathlineto{\pgfqpoint{3.185907in}{2.058667in}}%
\pgfpathlineto{\pgfqpoint{3.204848in}{2.039923in}}%
\pgfpathlineto{\pgfqpoint{3.296579in}{1.946667in}}%
\pgfpathlineto{\pgfqpoint{3.325091in}{1.916983in}}%
\pgfpathlineto{\pgfqpoint{3.405253in}{1.831274in}}%
\pgfpathlineto{\pgfqpoint{3.436030in}{1.797333in}}%
\pgfpathlineto{\pgfqpoint{3.502371in}{1.722667in}}%
\pgfpathlineto{\pgfqpoint{3.571784in}{1.642218in}}%
\pgfpathlineto{\pgfqpoint{3.659114in}{1.536000in}}%
\pgfpathlineto{\pgfqpoint{3.670131in}{1.521388in}}%
\pgfpathlineto{\pgfqpoint{3.689036in}{1.498667in}}%
\pgfpathlineto{\pgfqpoint{3.704090in}{1.478353in}}%
\pgfpathlineto{\pgfqpoint{3.725899in}{1.451299in}}%
\pgfpathlineto{\pgfqpoint{3.806061in}{1.344555in}}%
\pgfpathlineto{\pgfqpoint{3.886222in}{1.230832in}}%
\pgfpathlineto{\pgfqpoint{3.906849in}{1.200000in}}%
\pgfpathlineto{\pgfqpoint{3.955473in}{1.125333in}}%
\pgfpathlineto{\pgfqpoint{4.006465in}{1.042202in}}%
\pgfpathlineto{\pgfqpoint{4.030824in}{0.998690in}}%
\pgfpathlineto{\pgfqpoint{4.046545in}{0.972410in}}%
\pgfpathlineto{\pgfqpoint{4.086626in}{0.896920in}}%
\pgfpathlineto{\pgfqpoint{4.126707in}{0.813001in}}%
\pgfpathlineto{\pgfqpoint{4.137130in}{0.789333in}}%
\pgfpathlineto{\pgfqpoint{4.144799in}{0.768852in}}%
\pgfpathlineto{\pgfqpoint{4.152594in}{0.752000in}}%
\pgfpathlineto{\pgfqpoint{4.155910in}{0.741868in}}%
\pgfpathlineto{\pgfqpoint{4.167244in}{0.714667in}}%
\pgfpathlineto{\pgfqpoint{4.199813in}{0.609239in}}%
\pgfpathlineto{\pgfqpoint{4.209169in}{0.565333in}}%
\pgfpathlineto{\pgfqpoint{4.214447in}{0.528000in}}%
\pgfpathlineto{\pgfqpoint{4.237871in}{0.528000in}}%
\pgfpathlineto{\pgfqpoint{4.232612in}{0.551979in}}%
\pgfpathlineto{\pgfqpoint{4.230918in}{0.565333in}}%
\pgfpathlineto{\pgfqpoint{4.225862in}{0.583024in}}%
\pgfpathlineto{\pgfqpoint{4.222029in}{0.602667in}}%
\pgfpathlineto{\pgfqpoint{4.218355in}{0.613365in}}%
\pgfpathlineto{\pgfqpoint{4.206869in}{0.654383in}}%
\pgfpathlineto{\pgfqpoint{4.173429in}{0.745814in}}%
\pgfpathlineto{\pgfqpoint{4.146101in}{0.808602in}}%
\pgfpathlineto{\pgfqpoint{4.113506in}{0.876296in}}%
\pgfpathlineto{\pgfqpoint{4.074938in}{0.949554in}}%
\pgfpathlineto{\pgfqpoint{4.038729in}{1.013333in}}%
\pgfpathlineto{\pgfqpoint{4.026745in}{1.032223in}}%
\pgfpathlineto{\pgfqpoint{4.006465in}{1.067315in}}%
\pgfpathlineto{\pgfqpoint{3.983200in}{1.103663in}}%
\pgfpathlineto{\pgfqpoint{3.966384in}{1.131578in}}%
\pgfpathlineto{\pgfqpoint{3.921513in}{1.200000in}}%
\pgfpathlineto{\pgfqpoint{3.907337in}{1.219668in}}%
\pgfpathlineto{\pgfqpoint{3.886222in}{1.251571in}}%
\pgfpathlineto{\pgfqpoint{3.816503in}{1.349333in}}%
\pgfpathlineto{\pgfqpoint{3.806061in}{1.363520in}}%
\pgfpathlineto{\pgfqpoint{3.725899in}{1.469206in}}%
\pgfpathlineto{\pgfqpoint{3.702651in}{1.498667in}}%
\pgfpathlineto{\pgfqpoint{3.642772in}{1.573333in}}%
\pgfpathlineto{\pgfqpoint{3.626167in}{1.592438in}}%
\pgfpathlineto{\pgfqpoint{3.605657in}{1.618122in}}%
\pgfpathlineto{\pgfqpoint{3.516234in}{1.722667in}}%
\pgfpathlineto{\pgfqpoint{3.501964in}{1.738082in}}%
\pgfpathlineto{\pgfqpoint{3.483478in}{1.760000in}}%
\pgfpathlineto{\pgfqpoint{3.465660in}{1.778933in}}%
\pgfpathlineto{\pgfqpoint{3.445333in}{1.802478in}}%
\pgfpathlineto{\pgfqpoint{3.415904in}{1.834667in}}%
\pgfpathlineto{\pgfqpoint{3.346284in}{1.909333in}}%
\pgfpathlineto{\pgfqpoint{3.274573in}{1.984000in}}%
\pgfpathlineto{\pgfqpoint{3.200601in}{2.058667in}}%
\pgfpathlineto{\pgfqpoint{3.101625in}{2.154815in}}%
\pgfpathlineto{\pgfqpoint{3.044525in}{2.208502in}}%
\pgfpathlineto{\pgfqpoint{2.963123in}{2.282667in}}%
\pgfpathlineto{\pgfqpoint{2.877989in}{2.357333in}}%
\pgfpathlineto{\pgfqpoint{2.789400in}{2.432000in}}%
\pgfpathlineto{\pgfqpoint{2.697021in}{2.506667in}}%
\pgfpathlineto{\pgfqpoint{2.649294in}{2.544000in}}%
\pgfpathlineto{\pgfqpoint{2.600212in}{2.581333in}}%
\pgfpathlineto{\pgfqpoint{2.563556in}{2.608406in}}%
\pgfpathlineto{\pgfqpoint{2.483394in}{2.665797in}}%
\pgfpathlineto{\pgfqpoint{2.403232in}{2.720228in}}%
\pgfpathlineto{\pgfqpoint{2.311477in}{2.778799in}}%
\pgfpathlineto{\pgfqpoint{2.242909in}{2.819325in}}%
\pgfpathlineto{\pgfqpoint{2.201759in}{2.842667in}}%
\pgfpathlineto{\pgfqpoint{2.162747in}{2.863344in}}%
\pgfpathlineto{\pgfqpoint{2.113549in}{2.888493in}}%
\pgfpathlineto{\pgfqpoint{2.042505in}{2.921247in}}%
\pgfpathlineto{\pgfqpoint{2.016995in}{2.930905in}}%
\pgfpathlineto{\pgfqpoint{2.002424in}{2.937678in}}%
\pgfpathlineto{\pgfqpoint{1.957459in}{2.954667in}}%
\pgfpathlineto{\pgfqpoint{1.922263in}{2.966208in}}%
\pgfpathlineto{\pgfqpoint{1.900325in}{2.971567in}}%
\pgfpathlineto{\pgfqpoint{1.882182in}{2.977607in}}%
\pgfpathlineto{\pgfqpoint{1.837174in}{2.987411in}}%
\pgfpathlineto{\pgfqpoint{1.799993in}{2.993888in}}%
\pgfpathlineto{\pgfqpoint{1.761939in}{2.996702in}}%
\pgfpathlineto{\pgfqpoint{1.718041in}{2.995556in}}%
\pgfpathlineto{\pgfqpoint{1.681778in}{2.989228in}}%
\pgfpathlineto{\pgfqpoint{1.654414in}{2.980155in}}%
\pgfpathlineto{\pgfqpoint{1.628756in}{2.966720in}}%
\pgfpathlineto{\pgfqpoint{1.613236in}{2.954667in}}%
\pgfpathlineto{\pgfqpoint{1.601616in}{2.943780in}}%
\pgfpathlineto{\pgfqpoint{1.589011in}{2.929074in}}%
\pgfpathlineto{\pgfqpoint{1.581970in}{2.917333in}}%
\pgfpathlineto{\pgfqpoint{1.563981in}{2.877722in}}%
\pgfpathlineto{\pgfqpoint{1.556539in}{2.842667in}}%
\pgfpathlineto{\pgfqpoint{1.556549in}{2.838023in}}%
\pgfpathlineto{\pgfqpoint{1.553637in}{2.805333in}}%
\pgfpathlineto{\pgfqpoint{1.554308in}{2.798601in}}%
\pgfpathlineto{\pgfqpoint{1.554547in}{2.768000in}}%
\pgfpathlineto{\pgfqpoint{1.559028in}{2.728332in}}%
\pgfpathlineto{\pgfqpoint{1.565011in}{2.693333in}}%
\pgfpathlineto{\pgfqpoint{1.588897in}{2.606820in}}%
\pgfpathlineto{\pgfqpoint{1.601616in}{2.566451in}}%
\pgfpathlineto{\pgfqpoint{1.641697in}{2.466184in}}%
\pgfpathlineto{\pgfqpoint{1.681778in}{2.381794in}}%
\pgfpathlineto{\pgfqpoint{1.761939in}{2.234683in}}%
\pgfpathlineto{\pgfqpoint{1.802020in}{2.167938in}}%
\pgfpathlineto{\pgfqpoint{1.872597in}{2.058667in}}%
\pgfpathlineto{\pgfqpoint{1.882182in}{2.044434in}}%
\pgfpathlineto{\pgfqpoint{1.923566in}{1.984000in}}%
\pgfpathlineto{\pgfqpoint{1.955280in}{1.940088in}}%
\pgfpathlineto{\pgfqpoint{1.977267in}{1.909333in}}%
\pgfpathlineto{\pgfqpoint{1.987865in}{1.895772in}}%
\pgfpathlineto{\pgfqpoint{2.004622in}{1.872000in}}%
\pgfpathlineto{\pgfqpoint{2.020742in}{1.851729in}}%
\pgfpathlineto{\pgfqpoint{2.042505in}{1.822066in}}%
\pgfpathlineto{\pgfqpoint{2.061618in}{1.797333in}}%
\pgfpathlineto{\pgfqpoint{2.122667in}{1.719717in}}%
\pgfpathlineto{\pgfqpoint{2.212708in}{1.610667in}}%
\pgfpathlineto{\pgfqpoint{2.282990in}{1.528868in}}%
\pgfpathlineto{\pgfqpoint{2.376974in}{1.424000in}}%
\pgfpathlineto{\pgfqpoint{2.446205in}{1.349333in}}%
\pgfpathlineto{\pgfqpoint{2.464132in}{1.331392in}}%
\pgfpathlineto{\pgfqpoint{2.483394in}{1.310239in}}%
\pgfpathlineto{\pgfqpoint{2.501963in}{1.291963in}}%
\pgfpathlineto{\pgfqpoint{2.523475in}{1.268939in}}%
\pgfpathlineto{\pgfqpoint{2.554637in}{1.237333in}}%
\pgfpathlineto{\pgfqpoint{2.603636in}{1.188374in}}%
\pgfpathlineto{\pgfqpoint{2.636776in}{1.156201in}}%
\pgfpathlineto{\pgfqpoint{2.668345in}{1.125333in}}%
\pgfpathlineto{\pgfqpoint{2.715821in}{1.080495in}}%
\pgfpathlineto{\pgfqpoint{2.747303in}{1.050667in}}%
\pgfpathlineto{\pgfqpoint{2.829033in}{0.976000in}}%
\pgfpathlineto{\pgfqpoint{2.877944in}{0.932838in}}%
\pgfpathlineto{\pgfqpoint{2.884202in}{0.927037in}}%
\pgfpathlineto{\pgfqpoint{2.964364in}{0.858046in}}%
\pgfpathlineto{\pgfqpoint{2.982558in}{0.843614in}}%
\pgfpathlineto{\pgfqpoint{3.004444in}{0.824464in}}%
\pgfpathlineto{\pgfqpoint{3.024995in}{0.808475in}}%
\pgfpathlineto{\pgfqpoint{3.047343in}{0.789333in}}%
\pgfpathlineto{\pgfqpoint{3.141923in}{0.714667in}}%
\pgfpathlineto{\pgfqpoint{3.204848in}{0.666803in}}%
\pgfpathlineto{\pgfqpoint{3.221332in}{0.655353in}}%
\pgfpathlineto{\pgfqpoint{3.258915in}{0.626973in}}%
\pgfpathlineto{\pgfqpoint{3.325091in}{0.580308in}}%
\pgfpathlineto{\pgfqpoint{3.357691in}{0.558366in}}%
\pgfpathlineto{\pgfqpoint{3.365172in}{0.552901in}}%
\pgfpathlineto{\pgfqpoint{3.402396in}{0.528000in}}%
\pgfpathlineto{\pgfqpoint{3.405253in}{0.528000in}}%
\pgfpathlineto{\pgfqpoint{3.405253in}{0.528000in}}%
\pgfusepath{fill}%
\end{pgfscope}%
\begin{pgfscope}%
\pgfpathrectangle{\pgfqpoint{0.800000in}{0.528000in}}{\pgfqpoint{3.968000in}{3.696000in}}%
\pgfusepath{clip}%
\pgfsetbuttcap%
\pgfsetroundjoin%
\definecolor{currentfill}{rgb}{0.276022,0.044167,0.370164}%
\pgfsetfillcolor{currentfill}%
\pgfsetlinewidth{0.000000pt}%
\definecolor{currentstroke}{rgb}{0.000000,0.000000,0.000000}%
\pgfsetstrokecolor{currentstroke}%
\pgfsetdash{}{0pt}%
\pgfpathmoveto{\pgfqpoint{3.402396in}{0.528000in}}%
\pgfpathlineto{\pgfqpoint{3.325091in}{0.580308in}}%
\pgfpathlineto{\pgfqpoint{3.241025in}{0.640000in}}%
\pgfpathlineto{\pgfqpoint{3.198767in}{0.671669in}}%
\pgfpathlineto{\pgfqpoint{3.164768in}{0.697087in}}%
\pgfpathlineto{\pgfqpoint{3.084606in}{0.759479in}}%
\pgfpathlineto{\pgfqpoint{3.001807in}{0.826667in}}%
\pgfpathlineto{\pgfqpoint{2.982558in}{0.843614in}}%
\pgfpathlineto{\pgfqpoint{2.957364in}{0.864000in}}%
\pgfpathlineto{\pgfqpoint{2.919345in}{0.896734in}}%
\pgfpathlineto{\pgfqpoint{2.884202in}{0.927037in}}%
\pgfpathlineto{\pgfqpoint{2.871013in}{0.938667in}}%
\pgfpathlineto{\pgfqpoint{2.787807in}{1.013333in}}%
\pgfpathlineto{\pgfqpoint{2.763960in}{1.035189in}}%
\pgfpathlineto{\pgfqpoint{2.668345in}{1.125333in}}%
\pgfpathlineto{\pgfqpoint{2.591941in}{1.200000in}}%
\pgfpathlineto{\pgfqpoint{2.563556in}{1.228319in}}%
\pgfpathlineto{\pgfqpoint{2.481707in}{1.312000in}}%
\pgfpathlineto{\pgfqpoint{2.464132in}{1.331392in}}%
\pgfpathlineto{\pgfqpoint{2.443313in}{1.352393in}}%
\pgfpathlineto{\pgfqpoint{2.411335in}{1.386667in}}%
\pgfpathlineto{\pgfqpoint{2.343103in}{1.461333in}}%
\pgfpathlineto{\pgfqpoint{2.276759in}{1.536000in}}%
\pgfpathlineto{\pgfqpoint{2.262069in}{1.553847in}}%
\pgfpathlineto{\pgfqpoint{2.242909in}{1.574990in}}%
\pgfpathlineto{\pgfqpoint{2.150703in}{1.685333in}}%
\pgfpathlineto{\pgfqpoint{2.082586in}{1.770322in}}%
\pgfpathlineto{\pgfqpoint{2.002424in}{1.874964in}}%
\pgfpathlineto{\pgfqpoint{1.977267in}{1.909333in}}%
\pgfpathlineto{\pgfqpoint{1.922263in}{1.985876in}}%
\pgfpathlineto{\pgfqpoint{1.847855in}{2.096000in}}%
\pgfpathlineto{\pgfqpoint{1.842101in}{2.104900in}}%
\pgfpathlineto{\pgfqpoint{1.797960in}{2.174449in}}%
\pgfpathlineto{\pgfqpoint{1.748214in}{2.258118in}}%
\pgfpathlineto{\pgfqpoint{1.705759in}{2.334996in}}%
\pgfpathlineto{\pgfqpoint{1.669683in}{2.405933in}}%
\pgfpathlineto{\pgfqpoint{1.640262in}{2.469333in}}%
\pgfpathlineto{\pgfqpoint{1.624677in}{2.506667in}}%
\pgfpathlineto{\pgfqpoint{1.608030in}{2.549974in}}%
\pgfpathlineto{\pgfqpoint{1.596273in}{2.581333in}}%
\pgfpathlineto{\pgfqpoint{1.584428in}{2.618667in}}%
\pgfpathlineto{\pgfqpoint{1.573909in}{2.656000in}}%
\pgfpathlineto{\pgfqpoint{1.572211in}{2.665944in}}%
\pgfpathlineto{\pgfqpoint{1.564720in}{2.696299in}}%
\pgfpathlineto{\pgfqpoint{1.557863in}{2.734087in}}%
\pgfpathlineto{\pgfqpoint{1.554547in}{2.768000in}}%
\pgfpathlineto{\pgfqpoint{1.553637in}{2.805333in}}%
\pgfpathlineto{\pgfqpoint{1.556539in}{2.842667in}}%
\pgfpathlineto{\pgfqpoint{1.564858in}{2.880000in}}%
\pgfpathlineto{\pgfqpoint{1.567982in}{2.886004in}}%
\pgfpathlineto{\pgfqpoint{1.581970in}{2.917333in}}%
\pgfpathlineto{\pgfqpoint{1.589011in}{2.929074in}}%
\pgfpathlineto{\pgfqpoint{1.601616in}{2.943780in}}%
\pgfpathlineto{\pgfqpoint{1.613236in}{2.954667in}}%
\pgfpathlineto{\pgfqpoint{1.628756in}{2.966720in}}%
\pgfpathlineto{\pgfqpoint{1.654414in}{2.980155in}}%
\pgfpathlineto{\pgfqpoint{1.681778in}{2.989228in}}%
\pgfpathlineto{\pgfqpoint{1.721859in}{2.995783in}}%
\pgfpathlineto{\pgfqpoint{1.725690in}{2.995569in}}%
\pgfpathlineto{\pgfqpoint{1.761939in}{2.996702in}}%
\pgfpathlineto{\pgfqpoint{1.811721in}{2.992000in}}%
\pgfpathlineto{\pgfqpoint{1.848858in}{2.985706in}}%
\pgfpathlineto{\pgfqpoint{1.882182in}{2.977607in}}%
\pgfpathlineto{\pgfqpoint{1.900325in}{2.971567in}}%
\pgfpathlineto{\pgfqpoint{1.922263in}{2.966208in}}%
\pgfpathlineto{\pgfqpoint{1.965102in}{2.952097in}}%
\pgfpathlineto{\pgfqpoint{2.042505in}{2.921247in}}%
\pgfpathlineto{\pgfqpoint{2.122667in}{2.883958in}}%
\pgfpathlineto{\pgfqpoint{2.130389in}{2.880000in}}%
\pgfpathlineto{\pgfqpoint{2.204310in}{2.841286in}}%
\pgfpathlineto{\pgfqpoint{2.282990in}{2.795901in}}%
\pgfpathlineto{\pgfqpoint{2.328826in}{2.768000in}}%
\pgfpathlineto{\pgfqpoint{2.372710in}{2.739570in}}%
\pgfpathlineto{\pgfqpoint{2.403232in}{2.720228in}}%
\pgfpathlineto{\pgfqpoint{2.483394in}{2.665797in}}%
\pgfpathlineto{\pgfqpoint{2.511803in}{2.645128in}}%
\pgfpathlineto{\pgfqpoint{2.534487in}{2.628924in}}%
\pgfpathlineto{\pgfqpoint{2.563556in}{2.608406in}}%
\pgfpathlineto{\pgfqpoint{2.579317in}{2.596014in}}%
\pgfpathlineto{\pgfqpoint{2.603636in}{2.578795in}}%
\pgfpathlineto{\pgfqpoint{2.623704in}{2.562692in}}%
\pgfpathlineto{\pgfqpoint{2.649294in}{2.544000in}}%
\pgfpathlineto{\pgfqpoint{2.743707in}{2.469333in}}%
\pgfpathlineto{\pgfqpoint{2.834146in}{2.394667in}}%
\pgfpathlineto{\pgfqpoint{2.844121in}{2.386272in}}%
\pgfpathlineto{\pgfqpoint{2.924283in}{2.317104in}}%
\pgfpathlineto{\pgfqpoint{3.004485in}{2.245333in}}%
\pgfpathlineto{\pgfqpoint{3.101625in}{2.154815in}}%
\pgfpathlineto{\pgfqpoint{3.164768in}{2.093998in}}%
\pgfpathlineto{\pgfqpoint{3.244929in}{2.014238in}}%
\pgfpathlineto{\pgfqpoint{3.346284in}{1.909333in}}%
\pgfpathlineto{\pgfqpoint{3.415904in}{1.834667in}}%
\pgfpathlineto{\pgfqpoint{3.429046in}{1.819496in}}%
\pgfpathlineto{\pgfqpoint{3.449978in}{1.797333in}}%
\pgfpathlineto{\pgfqpoint{3.465660in}{1.778933in}}%
\pgfpathlineto{\pgfqpoint{3.485414in}{1.757820in}}%
\pgfpathlineto{\pgfqpoint{3.516234in}{1.722667in}}%
\pgfpathlineto{\pgfqpoint{3.580417in}{1.648000in}}%
\pgfpathlineto{\pgfqpoint{3.645737in}{1.569701in}}%
\pgfpathlineto{\pgfqpoint{3.678230in}{1.528932in}}%
\pgfpathlineto{\pgfqpoint{3.702651in}{1.498667in}}%
\pgfpathlineto{\pgfqpoint{3.712145in}{1.485856in}}%
\pgfpathlineto{\pgfqpoint{3.748032in}{1.440717in}}%
\pgfpathlineto{\pgfqpoint{3.816503in}{1.349333in}}%
\pgfpathlineto{\pgfqpoint{3.846141in}{1.308645in}}%
\pgfpathlineto{\pgfqpoint{3.870052in}{1.274667in}}%
\pgfpathlineto{\pgfqpoint{3.926303in}{1.192802in}}%
\pgfpathlineto{\pgfqpoint{3.946086in}{1.162667in}}%
\pgfpathlineto{\pgfqpoint{3.970407in}{1.125333in}}%
\pgfpathlineto{\pgfqpoint{3.998109in}{1.080218in}}%
\pgfpathlineto{\pgfqpoint{4.016575in}{1.050667in}}%
\pgfpathlineto{\pgfqpoint{4.026745in}{1.032223in}}%
\pgfpathlineto{\pgfqpoint{4.046545in}{0.999755in}}%
\pgfpathlineto{\pgfqpoint{4.086626in}{0.927694in}}%
\pgfpathlineto{\pgfqpoint{4.126707in}{0.849255in}}%
\pgfpathlineto{\pgfqpoint{4.154587in}{0.789333in}}%
\pgfpathlineto{\pgfqpoint{4.157742in}{0.780907in}}%
\pgfpathlineto{\pgfqpoint{4.173429in}{0.745814in}}%
\pgfpathlineto{\pgfqpoint{4.199237in}{0.677333in}}%
\pgfpathlineto{\pgfqpoint{4.200744in}{0.671628in}}%
\pgfpathlineto{\pgfqpoint{4.211568in}{0.640000in}}%
\pgfpathlineto{\pgfqpoint{4.230918in}{0.565333in}}%
\pgfpathlineto{\pgfqpoint{4.232612in}{0.551979in}}%
\pgfpathlineto{\pgfqpoint{4.237871in}{0.528000in}}%
\pgfpathlineto{\pgfqpoint{4.260007in}{0.528000in}}%
\pgfpathlineto{\pgfqpoint{4.257346in}{0.537684in}}%
\pgfpathlineto{\pgfqpoint{4.252188in}{0.565333in}}%
\pgfpathlineto{\pgfqpoint{4.242327in}{0.602667in}}%
\pgfpathlineto{\pgfqpoint{4.230597in}{0.640000in}}%
\pgfpathlineto{\pgfqpoint{4.217720in}{0.677333in}}%
\pgfpathlineto{\pgfqpoint{4.201860in}{0.719332in}}%
\pgfpathlineto{\pgfqpoint{4.166788in}{0.799803in}}%
\pgfpathlineto{\pgfqpoint{4.132566in}{0.869457in}}%
\pgfpathlineto{\pgfqpoint{4.116352in}{0.901333in}}%
\pgfpathlineto{\pgfqpoint{4.086626in}{0.955983in}}%
\pgfpathlineto{\pgfqpoint{4.046545in}{1.025495in}}%
\pgfpathlineto{\pgfqpoint{4.031352in}{1.050667in}}%
\pgfpathlineto{\pgfqpoint{4.006465in}{1.091341in}}%
\pgfpathlineto{\pgfqpoint{3.984685in}{1.125333in}}%
\pgfpathlineto{\pgfqpoint{3.935604in}{1.200000in}}%
\pgfpathlineto{\pgfqpoint{3.884224in}{1.274667in}}%
\pgfpathlineto{\pgfqpoint{3.857427in}{1.312000in}}%
\pgfpathlineto{\pgfqpoint{3.794918in}{1.397046in}}%
\pgfpathlineto{\pgfqpoint{3.716266in}{1.498667in}}%
\pgfpathlineto{\pgfqpoint{3.645737in}{1.585988in}}%
\pgfpathlineto{\pgfqpoint{3.625229in}{1.610667in}}%
\pgfpathlineto{\pgfqpoint{3.548579in}{1.701165in}}%
\pgfpathlineto{\pgfqpoint{3.485414in}{1.772864in}}%
\pgfpathlineto{\pgfqpoint{3.463401in}{1.797333in}}%
\pgfpathlineto{\pgfqpoint{3.395109in}{1.872000in}}%
\pgfpathlineto{\pgfqpoint{3.380812in}{1.886569in}}%
\pgfpathlineto{\pgfqpoint{3.360224in}{1.909333in}}%
\pgfpathlineto{\pgfqpoint{3.285010in}{1.987754in}}%
\pgfpathlineto{\pgfqpoint{3.204848in}{2.068466in}}%
\pgfpathlineto{\pgfqpoint{3.099351in}{2.170667in}}%
\pgfpathlineto{\pgfqpoint{3.051912in}{2.214880in}}%
\pgfpathlineto{\pgfqpoint{3.019266in}{2.245333in}}%
\pgfpathlineto{\pgfqpoint{2.971000in}{2.288848in}}%
\pgfpathlineto{\pgfqpoint{2.964364in}{2.295123in}}%
\pgfpathlineto{\pgfqpoint{2.884202in}{2.365689in}}%
\pgfpathlineto{\pgfqpoint{2.786231in}{2.448588in}}%
\pgfpathlineto{\pgfqpoint{2.714564in}{2.506667in}}%
\pgfpathlineto{\pgfqpoint{2.683798in}{2.531017in}}%
\pgfpathlineto{\pgfqpoint{2.603636in}{2.592716in}}%
\pgfpathlineto{\pgfqpoint{2.587755in}{2.603874in}}%
\pgfpathlineto{\pgfqpoint{2.563556in}{2.622680in}}%
\pgfpathlineto{\pgfqpoint{2.542975in}{2.636830in}}%
\pgfpathlineto{\pgfqpoint{2.517617in}{2.656000in}}%
\pgfpathlineto{\pgfqpoint{2.483394in}{2.680186in}}%
\pgfpathlineto{\pgfqpoint{2.403232in}{2.735078in}}%
\pgfpathlineto{\pgfqpoint{2.381947in}{2.748174in}}%
\pgfpathlineto{\pgfqpoint{2.352554in}{2.768000in}}%
\pgfpathlineto{\pgfqpoint{2.264688in}{2.822381in}}%
\pgfpathlineto{\pgfqpoint{2.202828in}{2.857833in}}%
\pgfpathlineto{\pgfqpoint{2.187350in}{2.865582in}}%
\pgfpathlineto{\pgfqpoint{2.162682in}{2.880000in}}%
\pgfpathlineto{\pgfqpoint{2.122667in}{2.900512in}}%
\pgfpathlineto{\pgfqpoint{2.075718in}{2.923730in}}%
\pgfpathlineto{\pgfqpoint{1.999390in}{2.957493in}}%
\pgfpathlineto{\pgfqpoint{1.922263in}{2.986079in}}%
\pgfpathlineto{\pgfqpoint{1.917226in}{2.987309in}}%
\pgfpathlineto{\pgfqpoint{1.882182in}{2.998456in}}%
\pgfpathlineto{\pgfqpoint{1.855478in}{3.004460in}}%
\pgfpathlineto{\pgfqpoint{1.842101in}{3.008629in}}%
\pgfpathlineto{\pgfqpoint{1.802020in}{3.016577in}}%
\pgfpathlineto{\pgfqpoint{1.761939in}{3.021710in}}%
\pgfpathlineto{\pgfqpoint{1.753471in}{3.021445in}}%
\pgfpathlineto{\pgfqpoint{1.721859in}{3.023211in}}%
\pgfpathlineto{\pgfqpoint{1.681778in}{3.019913in}}%
\pgfpathlineto{\pgfqpoint{1.656970in}{3.015107in}}%
\pgfpathlineto{\pgfqpoint{1.627693in}{3.005044in}}%
\pgfpathlineto{\pgfqpoint{1.601616in}{2.990902in}}%
\pgfpathlineto{\pgfqpoint{1.561535in}{2.950891in}}%
\pgfpathlineto{\pgfqpoint{1.544343in}{2.917333in}}%
\pgfpathlineto{\pgfqpoint{1.538343in}{2.901602in}}%
\pgfpathlineto{\pgfqpoint{1.533318in}{2.880000in}}%
\pgfpathlineto{\pgfqpoint{1.528319in}{2.842667in}}%
\pgfpathlineto{\pgfqpoint{1.527812in}{2.805333in}}%
\pgfpathlineto{\pgfqpoint{1.536363in}{2.730667in}}%
\pgfpathlineto{\pgfqpoint{1.541149in}{2.711677in}}%
\pgfpathlineto{\pgfqpoint{1.544127in}{2.693333in}}%
\pgfpathlineto{\pgfqpoint{1.548054in}{2.680776in}}%
\pgfpathlineto{\pgfqpoint{1.553626in}{2.656000in}}%
\pgfpathlineto{\pgfqpoint{1.564800in}{2.618667in}}%
\pgfpathlineto{\pgfqpoint{1.577835in}{2.581333in}}%
\pgfpathlineto{\pgfqpoint{1.591832in}{2.544000in}}%
\pgfpathlineto{\pgfqpoint{1.594792in}{2.537643in}}%
\pgfpathlineto{\pgfqpoint{1.607010in}{2.506667in}}%
\pgfpathlineto{\pgfqpoint{1.617689in}{2.484305in}}%
\pgfpathlineto{\pgfqpoint{1.623565in}{2.469333in}}%
\pgfpathlineto{\pgfqpoint{1.629439in}{2.457915in}}%
\pgfpathlineto{\pgfqpoint{1.641697in}{2.429829in}}%
\pgfpathlineto{\pgfqpoint{1.681778in}{2.350661in}}%
\pgfpathlineto{\pgfqpoint{1.721859in}{2.277686in}}%
\pgfpathlineto{\pgfqpoint{1.763670in}{2.206388in}}%
\pgfpathlineto{\pgfqpoint{1.833574in}{2.096000in}}%
\pgfpathlineto{\pgfqpoint{1.842101in}{2.083106in}}%
\pgfpathlineto{\pgfqpoint{1.909824in}{1.984000in}}%
\pgfpathlineto{\pgfqpoint{1.963246in}{1.909333in}}%
\pgfpathlineto{\pgfqpoint{1.995925in}{1.865946in}}%
\pgfpathlineto{\pgfqpoint{2.019318in}{1.834667in}}%
\pgfpathlineto{\pgfqpoint{2.029296in}{1.822363in}}%
\pgfpathlineto{\pgfqpoint{2.061897in}{1.779271in}}%
\pgfpathlineto{\pgfqpoint{2.137229in}{1.685333in}}%
\pgfpathlineto{\pgfqpoint{2.148625in}{1.672179in}}%
\pgfpathlineto{\pgfqpoint{2.185513in}{1.626795in}}%
\pgfpathlineto{\pgfqpoint{2.242909in}{1.559520in}}%
\pgfpathlineto{\pgfqpoint{2.263379in}{1.536000in}}%
\pgfpathlineto{\pgfqpoint{2.329385in}{1.461333in}}%
\pgfpathlineto{\pgfqpoint{2.403232in}{1.380591in}}%
\pgfpathlineto{\pgfqpoint{2.437735in}{1.344137in}}%
\pgfpathlineto{\pgfqpoint{2.468087in}{1.312000in}}%
\pgfpathlineto{\pgfqpoint{2.540666in}{1.237333in}}%
\pgfpathlineto{\pgfqpoint{2.615497in}{1.162667in}}%
\pgfpathlineto{\pgfqpoint{2.643717in}{1.135074in}}%
\pgfpathlineto{\pgfqpoint{2.732372in}{1.050667in}}%
\pgfpathlineto{\pgfqpoint{2.813680in}{0.976000in}}%
\pgfpathlineto{\pgfqpoint{2.844121in}{0.948681in}}%
\pgfpathlineto{\pgfqpoint{2.941331in}{0.864000in}}%
\pgfpathlineto{\pgfqpoint{3.004444in}{0.810873in}}%
\pgfpathlineto{\pgfqpoint{3.030627in}{0.789333in}}%
\pgfpathlineto{\pgfqpoint{3.084606in}{0.745598in}}%
\pgfpathlineto{\pgfqpoint{3.103025in}{0.731823in}}%
\pgfpathlineto{\pgfqpoint{3.130006in}{0.709712in}}%
\pgfpathlineto{\pgfqpoint{3.222028in}{0.640000in}}%
\pgfpathlineto{\pgfqpoint{3.244929in}{0.623092in}}%
\pgfpathlineto{\pgfqpoint{3.325490in}{0.565333in}}%
\pgfpathlineto{\pgfqpoint{3.348878in}{0.550157in}}%
\pgfpathlineto{\pgfqpoint{3.380499in}{0.528000in}}%
\pgfpathlineto{\pgfqpoint{3.380499in}{0.528000in}}%
\pgfusepath{fill}%
\end{pgfscope}%
\begin{pgfscope}%
\pgfpathrectangle{\pgfqpoint{0.800000in}{0.528000in}}{\pgfqpoint{3.968000in}{3.696000in}}%
\pgfusepath{clip}%
\pgfsetbuttcap%
\pgfsetroundjoin%
\definecolor{currentfill}{rgb}{0.276022,0.044167,0.370164}%
\pgfsetfillcolor{currentfill}%
\pgfsetlinewidth{0.000000pt}%
\definecolor{currentstroke}{rgb}{0.000000,0.000000,0.000000}%
\pgfsetstrokecolor{currentstroke}%
\pgfsetdash{}{0pt}%
\pgfpathmoveto{\pgfqpoint{3.380499in}{0.528000in}}%
\pgfpathlineto{\pgfqpoint{3.323919in}{0.566425in}}%
\pgfpathlineto{\pgfqpoint{3.222028in}{0.640000in}}%
\pgfpathlineto{\pgfqpoint{3.190639in}{0.664097in}}%
\pgfpathlineto{\pgfqpoint{3.164768in}{0.682963in}}%
\pgfpathlineto{\pgfqpoint{3.076650in}{0.752000in}}%
\pgfpathlineto{\pgfqpoint{2.985534in}{0.826667in}}%
\pgfpathlineto{\pgfqpoint{2.964364in}{0.844408in}}%
\pgfpathlineto{\pgfqpoint{2.884202in}{0.913306in}}%
\pgfpathlineto{\pgfqpoint{2.855439in}{0.938667in}}%
\pgfpathlineto{\pgfqpoint{2.772668in}{1.013333in}}%
\pgfpathlineto{\pgfqpoint{2.732372in}{1.050667in}}%
\pgfpathlineto{\pgfqpoint{2.683798in}{1.096504in}}%
\pgfpathlineto{\pgfqpoint{2.653813in}{1.125333in}}%
\pgfpathlineto{\pgfqpoint{2.577789in}{1.200000in}}%
\pgfpathlineto{\pgfqpoint{2.563556in}{1.214200in}}%
\pgfpathlineto{\pgfqpoint{2.468087in}{1.312000in}}%
\pgfpathlineto{\pgfqpoint{2.437735in}{1.344137in}}%
\pgfpathlineto{\pgfqpoint{2.419269in}{1.364271in}}%
\pgfpathlineto{\pgfqpoint{2.397595in}{1.386667in}}%
\pgfpathlineto{\pgfqpoint{2.323071in}{1.468356in}}%
\pgfpathlineto{\pgfqpoint{2.296157in}{1.498667in}}%
\pgfpathlineto{\pgfqpoint{2.231034in}{1.573333in}}%
\pgfpathlineto{\pgfqpoint{2.162747in}{1.654167in}}%
\pgfpathlineto{\pgfqpoint{2.137229in}{1.685333in}}%
\pgfpathlineto{\pgfqpoint{2.061897in}{1.779271in}}%
\pgfpathlineto{\pgfqpoint{1.991095in}{1.872000in}}%
\pgfpathlineto{\pgfqpoint{1.979750in}{1.888214in}}%
\pgfpathlineto{\pgfqpoint{1.962343in}{1.910573in}}%
\pgfpathlineto{\pgfqpoint{1.930974in}{1.954781in}}%
\pgfpathlineto{\pgfqpoint{1.909824in}{1.984000in}}%
\pgfpathlineto{\pgfqpoint{1.882182in}{2.023460in}}%
\pgfpathlineto{\pgfqpoint{1.858473in}{2.058667in}}%
\pgfpathlineto{\pgfqpoint{1.802020in}{2.144895in}}%
\pgfpathlineto{\pgfqpoint{1.761939in}{2.209282in}}%
\pgfpathlineto{\pgfqpoint{1.740752in}{2.245333in}}%
\pgfpathlineto{\pgfqpoint{1.718990in}{2.282667in}}%
\pgfpathlineto{\pgfqpoint{1.698427in}{2.320000in}}%
\pgfpathlineto{\pgfqpoint{1.674702in}{2.363924in}}%
\pgfpathlineto{\pgfqpoint{1.640630in}{2.432000in}}%
\pgfpathlineto{\pgfqpoint{1.617689in}{2.484305in}}%
\pgfpathlineto{\pgfqpoint{1.601616in}{2.519796in}}%
\pgfpathlineto{\pgfqpoint{1.564800in}{2.618667in}}%
\pgfpathlineto{\pgfqpoint{1.553626in}{2.656000in}}%
\pgfpathlineto{\pgfqpoint{1.531674in}{2.758481in}}%
\pgfpathlineto{\pgfqpoint{1.527812in}{2.805333in}}%
\pgfpathlineto{\pgfqpoint{1.528388in}{2.811791in}}%
\pgfpathlineto{\pgfqpoint{1.528319in}{2.842667in}}%
\pgfpathlineto{\pgfqpoint{1.529929in}{2.850561in}}%
\pgfpathlineto{\pgfqpoint{1.533318in}{2.880000in}}%
\pgfpathlineto{\pgfqpoint{1.538343in}{2.901602in}}%
\pgfpathlineto{\pgfqpoint{1.544343in}{2.917333in}}%
\pgfpathlineto{\pgfqpoint{1.564148in}{2.954667in}}%
\pgfpathlineto{\pgfqpoint{1.603365in}{2.992000in}}%
\pgfpathlineto{\pgfqpoint{1.627693in}{3.005044in}}%
\pgfpathlineto{\pgfqpoint{1.656970in}{3.015107in}}%
\pgfpathlineto{\pgfqpoint{1.690364in}{3.021336in}}%
\pgfpathlineto{\pgfqpoint{1.721859in}{3.023211in}}%
\pgfpathlineto{\pgfqpoint{1.770830in}{3.021052in}}%
\pgfpathlineto{\pgfqpoint{1.818656in}{3.013838in}}%
\pgfpathlineto{\pgfqpoint{1.842101in}{3.008629in}}%
\pgfpathlineto{\pgfqpoint{1.855478in}{3.004460in}}%
\pgfpathlineto{\pgfqpoint{1.887130in}{2.996609in}}%
\pgfpathlineto{\pgfqpoint{1.932073in}{2.982862in}}%
\pgfpathlineto{\pgfqpoint{2.006011in}{2.954667in}}%
\pgfpathlineto{\pgfqpoint{2.042505in}{2.938751in}}%
\pgfpathlineto{\pgfqpoint{2.088844in}{2.917333in}}%
\pgfpathlineto{\pgfqpoint{2.162747in}{2.879966in}}%
\pgfpathlineto{\pgfqpoint{2.187350in}{2.865582in}}%
\pgfpathlineto{\pgfqpoint{2.202828in}{2.857833in}}%
\pgfpathlineto{\pgfqpoint{2.282990in}{2.811366in}}%
\pgfpathlineto{\pgfqpoint{2.292791in}{2.805333in}}%
\pgfpathlineto{\pgfqpoint{2.363152in}{2.761281in}}%
\pgfpathlineto{\pgfqpoint{2.381947in}{2.748174in}}%
\pgfpathlineto{\pgfqpoint{2.409764in}{2.730667in}}%
\pgfpathlineto{\pgfqpoint{2.483394in}{2.680186in}}%
\pgfpathlineto{\pgfqpoint{2.497744in}{2.669366in}}%
\pgfpathlineto{\pgfqpoint{2.523475in}{2.651845in}}%
\pgfpathlineto{\pgfqpoint{2.542975in}{2.636830in}}%
\pgfpathlineto{\pgfqpoint{2.568940in}{2.618667in}}%
\pgfpathlineto{\pgfqpoint{2.603636in}{2.592716in}}%
\pgfpathlineto{\pgfqpoint{2.683798in}{2.531017in}}%
\pgfpathlineto{\pgfqpoint{2.714564in}{2.506667in}}%
\pgfpathlineto{\pgfqpoint{2.786231in}{2.448588in}}%
\pgfpathlineto{\pgfqpoint{2.850433in}{2.394667in}}%
\pgfpathlineto{\pgfqpoint{2.936374in}{2.320000in}}%
\pgfpathlineto{\pgfqpoint{3.019266in}{2.245333in}}%
\pgfpathlineto{\pgfqpoint{3.099351in}{2.170667in}}%
\pgfpathlineto{\pgfqpoint{3.176848in}{2.096000in}}%
\pgfpathlineto{\pgfqpoint{3.251951in}{2.021333in}}%
\pgfpathlineto{\pgfqpoint{3.325091in}{1.946386in}}%
\pgfpathlineto{\pgfqpoint{3.429495in}{1.834667in}}%
\pgfpathlineto{\pgfqpoint{3.445333in}{1.817343in}}%
\pgfpathlineto{\pgfqpoint{3.529843in}{1.722667in}}%
\pgfpathlineto{\pgfqpoint{3.605657in}{1.634124in}}%
\pgfpathlineto{\pgfqpoint{3.689102in}{1.532941in}}%
\pgfpathlineto{\pgfqpoint{3.774347in}{1.424000in}}%
\pgfpathlineto{\pgfqpoint{3.830216in}{1.349333in}}%
\pgfpathlineto{\pgfqpoint{3.886222in}{1.271812in}}%
\pgfpathlineto{\pgfqpoint{3.935604in}{1.200000in}}%
\pgfpathlineto{\pgfqpoint{3.946949in}{1.181897in}}%
\pgfpathlineto{\pgfqpoint{3.966384in}{1.153739in}}%
\pgfpathlineto{\pgfqpoint{4.011566in}{1.083248in}}%
\pgfpathlineto{\pgfqpoint{4.062684in}{0.998301in}}%
\pgfpathlineto{\pgfqpoint{4.106446in}{0.920206in}}%
\pgfpathlineto{\pgfqpoint{4.143764in}{0.848113in}}%
\pgfpathlineto{\pgfqpoint{4.175405in}{0.781307in}}%
\pgfpathlineto{\pgfqpoint{4.206869in}{0.706258in}}%
\pgfpathlineto{\pgfqpoint{4.230597in}{0.640000in}}%
\pgfpathlineto{\pgfqpoint{4.233733in}{0.627690in}}%
\pgfpathlineto{\pgfqpoint{4.243038in}{0.599023in}}%
\pgfpathlineto{\pgfqpoint{4.252188in}{0.565333in}}%
\pgfpathlineto{\pgfqpoint{4.260007in}{0.528000in}}%
\pgfpathlineto{\pgfqpoint{4.281329in}{0.528000in}}%
\pgfpathlineto{\pgfqpoint{4.272113in}{0.565333in}}%
\pgfpathlineto{\pgfqpoint{4.261392in}{0.602667in}}%
\pgfpathlineto{\pgfqpoint{4.246949in}{0.646793in}}%
\pgfpathlineto{\pgfqpoint{4.220802in}{0.714667in}}%
\pgfpathlineto{\pgfqpoint{4.216510in}{0.723647in}}%
\pgfpathlineto{\pgfqpoint{4.204026in}{0.754648in}}%
\pgfpathlineto{\pgfqpoint{4.166788in}{0.833658in}}%
\pgfpathlineto{\pgfqpoint{4.126707in}{0.910858in}}%
\pgfpathlineto{\pgfqpoint{4.086626in}{0.982481in}}%
\pgfpathlineto{\pgfqpoint{4.045589in}{1.051558in}}%
\pgfpathlineto{\pgfqpoint{3.974528in}{1.162667in}}%
\pgfpathlineto{\pgfqpoint{3.955670in}{1.190020in}}%
\pgfpathlineto{\pgfqpoint{3.949415in}{1.200000in}}%
\pgfpathlineto{\pgfqpoint{3.897711in}{1.274667in}}%
\pgfpathlineto{\pgfqpoint{3.843928in}{1.349333in}}%
\pgfpathlineto{\pgfqpoint{3.811755in}{1.391970in}}%
\pgfpathlineto{\pgfqpoint{3.787629in}{1.424000in}}%
\pgfpathlineto{\pgfqpoint{3.765980in}{1.452244in}}%
\pgfpathlineto{\pgfqpoint{3.699673in}{1.536000in}}%
\pgfpathlineto{\pgfqpoint{3.638581in}{1.610667in}}%
\pgfpathlineto{\pgfqpoint{3.565576in}{1.696692in}}%
\pgfpathlineto{\pgfqpoint{3.476824in}{1.797333in}}%
\pgfpathlineto{\pgfqpoint{3.405253in}{1.875688in}}%
\pgfpathlineto{\pgfqpoint{3.373665in}{1.909333in}}%
\pgfpathlineto{\pgfqpoint{3.302155in}{1.984000in}}%
\pgfpathlineto{\pgfqpoint{3.255456in}{2.031138in}}%
\pgfpathlineto{\pgfqpoint{3.228526in}{2.058667in}}%
\pgfpathlineto{\pgfqpoint{3.204848in}{2.082250in}}%
\pgfpathlineto{\pgfqpoint{3.113740in}{2.170667in}}%
\pgfpathlineto{\pgfqpoint{3.084606in}{2.198290in}}%
\pgfpathlineto{\pgfqpoint{2.993169in}{2.282667in}}%
\pgfpathlineto{\pgfqpoint{2.964364in}{2.308627in}}%
\pgfpathlineto{\pgfqpoint{2.866065in}{2.394667in}}%
\pgfpathlineto{\pgfqpoint{2.812396in}{2.439783in}}%
\pgfpathlineto{\pgfqpoint{2.804040in}{2.447169in}}%
\pgfpathlineto{\pgfqpoint{2.723879in}{2.512874in}}%
\pgfpathlineto{\pgfqpoint{2.705417in}{2.526804in}}%
\pgfpathlineto{\pgfqpoint{2.678021in}{2.549381in}}%
\pgfpathlineto{\pgfqpoint{2.587376in}{2.618667in}}%
\pgfpathlineto{\pgfqpoint{2.523475in}{2.665754in}}%
\pgfpathlineto{\pgfqpoint{2.506281in}{2.677318in}}%
\pgfpathlineto{\pgfqpoint{2.483394in}{2.694511in}}%
\pgfpathlineto{\pgfqpoint{2.460643in}{2.709475in}}%
\pgfpathlineto{\pgfqpoint{2.430919in}{2.730667in}}%
\pgfpathlineto{\pgfqpoint{2.363152in}{2.775878in}}%
\pgfpathlineto{\pgfqpoint{2.317049in}{2.805333in}}%
\pgfpathlineto{\pgfqpoint{2.282990in}{2.826299in}}%
\pgfpathlineto{\pgfqpoint{2.202828in}{2.873534in}}%
\pgfpathlineto{\pgfqpoint{2.172929in}{2.889484in}}%
\pgfpathlineto{\pgfqpoint{2.162747in}{2.895608in}}%
\pgfpathlineto{\pgfqpoint{2.147402in}{2.903040in}}%
\pgfpathlineto{\pgfqpoint{2.122128in}{2.917333in}}%
\pgfpathlineto{\pgfqpoint{2.082586in}{2.936918in}}%
\pgfpathlineto{\pgfqpoint{2.042505in}{2.956156in}}%
\pgfpathlineto{\pgfqpoint{2.002424in}{2.973651in}}%
\pgfpathlineto{\pgfqpoint{1.957739in}{2.992000in}}%
\pgfpathlineto{\pgfqpoint{1.922263in}{3.004975in}}%
\pgfpathlineto{\pgfqpoint{1.902288in}{3.010728in}}%
\pgfpathlineto{\pgfqpoint{1.882182in}{3.018228in}}%
\pgfpathlineto{\pgfqpoint{1.872578in}{3.020388in}}%
\pgfpathlineto{\pgfqpoint{1.842101in}{3.029844in}}%
\pgfpathlineto{\pgfqpoint{1.802020in}{3.038739in}}%
\pgfpathlineto{\pgfqpoint{1.775760in}{3.042206in}}%
\pgfpathlineto{\pgfqpoint{1.761939in}{3.045216in}}%
\pgfpathlineto{\pgfqpoint{1.740006in}{3.046237in}}%
\pgfpathlineto{\pgfqpoint{1.721859in}{3.048638in}}%
\pgfpathlineto{\pgfqpoint{1.700497in}{3.049230in}}%
\pgfpathlineto{\pgfqpoint{1.681778in}{3.048123in}}%
\pgfpathlineto{\pgfqpoint{1.657304in}{3.043871in}}%
\pgfpathlineto{\pgfqpoint{1.641697in}{3.042416in}}%
\pgfpathlineto{\pgfqpoint{1.600953in}{3.029333in}}%
\pgfpathlineto{\pgfqpoint{1.555230in}{2.997873in}}%
\pgfpathlineto{\pgfqpoint{1.537682in}{2.976885in}}%
\pgfpathlineto{\pgfqpoint{1.521455in}{2.947480in}}%
\pgfpathlineto{\pgfqpoint{1.511192in}{2.917333in}}%
\pgfpathlineto{\pgfqpoint{1.506003in}{2.894392in}}%
\pgfpathlineto{\pgfqpoint{1.504439in}{2.880000in}}%
\pgfpathlineto{\pgfqpoint{1.504364in}{2.864081in}}%
\pgfpathlineto{\pgfqpoint{1.502366in}{2.842667in}}%
\pgfpathlineto{\pgfqpoint{1.503857in}{2.826276in}}%
\pgfpathlineto{\pgfqpoint{1.503896in}{2.805333in}}%
\pgfpathlineto{\pgfqpoint{1.508260in}{2.768000in}}%
\pgfpathlineto{\pgfqpoint{1.516335in}{2.725898in}}%
\pgfpathlineto{\pgfqpoint{1.523543in}{2.693333in}}%
\pgfpathlineto{\pgfqpoint{1.560141in}{2.580034in}}%
\pgfpathlineto{\pgfqpoint{1.574448in}{2.544000in}}%
\pgfpathlineto{\pgfqpoint{1.582666in}{2.526349in}}%
\pgfpathlineto{\pgfqpoint{1.590192in}{2.506667in}}%
\pgfpathlineto{\pgfqpoint{1.606867in}{2.469333in}}%
\pgfpathlineto{\pgfqpoint{1.618150in}{2.447401in}}%
\pgfpathlineto{\pgfqpoint{1.630507in}{2.421577in}}%
\pgfpathlineto{\pgfqpoint{1.644482in}{2.392072in}}%
\pgfpathlineto{\pgfqpoint{1.682952in}{2.320000in}}%
\pgfpathlineto{\pgfqpoint{1.704264in}{2.282667in}}%
\pgfpathlineto{\pgfqpoint{1.725823in}{2.245333in}}%
\pgfpathlineto{\pgfqpoint{1.753592in}{2.200225in}}%
\pgfpathlineto{\pgfqpoint{1.771475in}{2.170667in}}%
\pgfpathlineto{\pgfqpoint{1.802020in}{2.122773in}}%
\pgfpathlineto{\pgfqpoint{1.844348in}{2.058667in}}%
\pgfpathlineto{\pgfqpoint{1.875010in}{2.014653in}}%
\pgfpathlineto{\pgfqpoint{1.896156in}{1.984000in}}%
\pgfpathlineto{\pgfqpoint{1.949956in}{1.909333in}}%
\pgfpathlineto{\pgfqpoint{2.005747in}{1.834667in}}%
\pgfpathlineto{\pgfqpoint{2.021566in}{1.815163in}}%
\pgfpathlineto{\pgfqpoint{2.042505in}{1.787183in}}%
\pgfpathlineto{\pgfqpoint{2.123754in}{1.685333in}}%
\pgfpathlineto{\pgfqpoint{2.158310in}{1.643867in}}%
\pgfpathlineto{\pgfqpoint{2.186048in}{1.610667in}}%
\pgfpathlineto{\pgfqpoint{2.202828in}{1.590835in}}%
\pgfpathlineto{\pgfqpoint{2.282990in}{1.498261in}}%
\pgfpathlineto{\pgfqpoint{2.319348in}{1.457866in}}%
\pgfpathlineto{\pgfqpoint{2.349958in}{1.424000in}}%
\pgfpathlineto{\pgfqpoint{2.393313in}{1.377427in}}%
\pgfpathlineto{\pgfqpoint{2.419139in}{1.349333in}}%
\pgfpathlineto{\pgfqpoint{2.449361in}{1.317633in}}%
\pgfpathlineto{\pgfqpoint{2.454467in}{1.312000in}}%
\pgfpathlineto{\pgfqpoint{2.526695in}{1.237333in}}%
\pgfpathlineto{\pgfqpoint{2.603636in}{1.160357in}}%
\pgfpathlineto{\pgfqpoint{2.622435in}{1.142843in}}%
\pgfpathlineto{\pgfqpoint{2.643717in}{1.121269in}}%
\pgfpathlineto{\pgfqpoint{2.661723in}{1.104771in}}%
\pgfpathlineto{\pgfqpoint{2.683798in}{1.082807in}}%
\pgfpathlineto{\pgfqpoint{2.717822in}{1.050667in}}%
\pgfpathlineto{\pgfqpoint{2.798674in}{0.976000in}}%
\pgfpathlineto{\pgfqpoint{2.844121in}{0.935088in}}%
\pgfpathlineto{\pgfqpoint{2.863166in}{0.919072in}}%
\pgfpathlineto{\pgfqpoint{2.884202in}{0.899661in}}%
\pgfpathlineto{\pgfqpoint{2.904491in}{0.882898in}}%
\pgfpathlineto{\pgfqpoint{2.925298in}{0.864000in}}%
\pgfpathlineto{\pgfqpoint{3.014106in}{0.789333in}}%
\pgfpathlineto{\pgfqpoint{3.106608in}{0.714667in}}%
\pgfpathlineto{\pgfqpoint{3.164768in}{0.669268in}}%
\pgfpathlineto{\pgfqpoint{3.182510in}{0.656526in}}%
\pgfpathlineto{\pgfqpoint{3.204848in}{0.638721in}}%
\pgfpathlineto{\pgfqpoint{3.226901in}{0.623208in}}%
\pgfpathlineto{\pgfqpoint{3.253752in}{0.602667in}}%
\pgfpathlineto{\pgfqpoint{3.325091in}{0.551664in}}%
\pgfpathlineto{\pgfqpoint{3.340065in}{0.541948in}}%
\pgfpathlineto{\pgfqpoint{3.359157in}{0.528000in}}%
\pgfpathlineto{\pgfqpoint{3.365172in}{0.528000in}}%
\pgfpathlineto{\pgfqpoint{3.365172in}{0.528000in}}%
\pgfusepath{fill}%
\end{pgfscope}%
\begin{pgfscope}%
\pgfpathrectangle{\pgfqpoint{0.800000in}{0.528000in}}{\pgfqpoint{3.968000in}{3.696000in}}%
\pgfusepath{clip}%
\pgfsetbuttcap%
\pgfsetroundjoin%
\definecolor{currentfill}{rgb}{0.276022,0.044167,0.370164}%
\pgfsetfillcolor{currentfill}%
\pgfsetlinewidth{0.000000pt}%
\definecolor{currentstroke}{rgb}{0.000000,0.000000,0.000000}%
\pgfsetstrokecolor{currentstroke}%
\pgfsetdash{}{0pt}%
\pgfpathmoveto{\pgfqpoint{3.359157in}{0.528000in}}%
\pgfpathlineto{\pgfqpoint{3.325091in}{0.551664in}}%
\pgfpathlineto{\pgfqpoint{3.244929in}{0.609067in}}%
\pgfpathlineto{\pgfqpoint{3.226901in}{0.623208in}}%
\pgfpathlineto{\pgfqpoint{3.203165in}{0.640000in}}%
\pgfpathlineto{\pgfqpoint{3.182510in}{0.656526in}}%
\pgfpathlineto{\pgfqpoint{3.154356in}{0.677333in}}%
\pgfpathlineto{\pgfqpoint{3.059874in}{0.752000in}}%
\pgfpathlineto{\pgfqpoint{3.004444in}{0.797281in}}%
\pgfpathlineto{\pgfqpoint{2.924283in}{0.864866in}}%
\pgfpathlineto{\pgfqpoint{2.904491in}{0.882898in}}%
\pgfpathlineto{\pgfqpoint{2.882304in}{0.901333in}}%
\pgfpathlineto{\pgfqpoint{2.863166in}{0.919072in}}%
\pgfpathlineto{\pgfqpoint{2.840128in}{0.938667in}}%
\pgfpathlineto{\pgfqpoint{2.822193in}{0.955575in}}%
\pgfpathlineto{\pgfqpoint{2.798674in}{0.976000in}}%
\pgfpathlineto{\pgfqpoint{2.717822in}{1.050667in}}%
\pgfpathlineto{\pgfqpoint{2.701337in}{1.067004in}}%
\pgfpathlineto{\pgfqpoint{2.678371in}{1.088000in}}%
\pgfpathlineto{\pgfqpoint{2.661723in}{1.104771in}}%
\pgfpathlineto{\pgfqpoint{2.639537in}{1.125333in}}%
\pgfpathlineto{\pgfqpoint{2.622435in}{1.142843in}}%
\pgfpathlineto{\pgfqpoint{2.601298in}{1.162667in}}%
\pgfpathlineto{\pgfqpoint{2.523475in}{1.240600in}}%
\pgfpathlineto{\pgfqpoint{2.487117in}{1.278135in}}%
\pgfpathlineto{\pgfqpoint{2.483394in}{1.281799in}}%
\pgfpathlineto{\pgfqpoint{2.384308in}{1.386667in}}%
\pgfpathlineto{\pgfqpoint{2.316071in}{1.461333in}}%
\pgfpathlineto{\pgfqpoint{2.301452in}{1.478530in}}%
\pgfpathlineto{\pgfqpoint{2.280975in}{1.500544in}}%
\pgfpathlineto{\pgfqpoint{2.186048in}{1.610667in}}%
\pgfpathlineto{\pgfqpoint{2.122667in}{1.686667in}}%
\pgfpathlineto{\pgfqpoint{2.093672in}{1.722667in}}%
\pgfpathlineto{\pgfqpoint{2.034591in}{1.797333in}}%
\pgfpathlineto{\pgfqpoint{2.021566in}{1.815163in}}%
\pgfpathlineto{\pgfqpoint{2.002424in}{1.839041in}}%
\pgfpathlineto{\pgfqpoint{1.971636in}{1.880655in}}%
\pgfpathlineto{\pgfqpoint{1.949956in}{1.909333in}}%
\pgfpathlineto{\pgfqpoint{1.939141in}{1.925055in}}%
\pgfpathlineto{\pgfqpoint{1.922263in}{1.947059in}}%
\pgfpathlineto{\pgfqpoint{1.842101in}{2.062021in}}%
\pgfpathlineto{\pgfqpoint{1.819631in}{2.096000in}}%
\pgfpathlineto{\pgfqpoint{1.771475in}{2.170667in}}%
\pgfpathlineto{\pgfqpoint{1.721859in}{2.252123in}}%
\pgfpathlineto{\pgfqpoint{1.681778in}{2.322163in}}%
\pgfpathlineto{\pgfqpoint{1.662948in}{2.357333in}}%
\pgfpathlineto{\pgfqpoint{1.641697in}{2.397634in}}%
\pgfpathlineto{\pgfqpoint{1.618150in}{2.447401in}}%
\pgfpathlineto{\pgfqpoint{1.601616in}{2.480967in}}%
\pgfpathlineto{\pgfqpoint{1.590192in}{2.506667in}}%
\pgfpathlineto{\pgfqpoint{1.582666in}{2.526349in}}%
\pgfpathlineto{\pgfqpoint{1.574448in}{2.544000in}}%
\pgfpathlineto{\pgfqpoint{1.558422in}{2.584234in}}%
\pgfpathlineto{\pgfqpoint{1.534347in}{2.656000in}}%
\pgfpathlineto{\pgfqpoint{1.532114in}{2.665929in}}%
\pgfpathlineto{\pgfqpoint{1.521455in}{2.702223in}}%
\pgfpathlineto{\pgfqpoint{1.508260in}{2.768000in}}%
\pgfpathlineto{\pgfqpoint{1.505953in}{2.782439in}}%
\pgfpathlineto{\pgfqpoint{1.503896in}{2.805333in}}%
\pgfpathlineto{\pgfqpoint{1.503857in}{2.826276in}}%
\pgfpathlineto{\pgfqpoint{1.502366in}{2.842667in}}%
\pgfpathlineto{\pgfqpoint{1.504364in}{2.864081in}}%
\pgfpathlineto{\pgfqpoint{1.504439in}{2.880000in}}%
\pgfpathlineto{\pgfqpoint{1.506003in}{2.894392in}}%
\pgfpathlineto{\pgfqpoint{1.513027in}{2.925184in}}%
\pgfpathlineto{\pgfqpoint{1.524618in}{2.954667in}}%
\pgfpathlineto{\pgfqpoint{1.537682in}{2.976885in}}%
\pgfpathlineto{\pgfqpoint{1.555230in}{2.997873in}}%
\pgfpathlineto{\pgfqpoint{1.561535in}{3.003235in}}%
\pgfpathlineto{\pgfqpoint{1.602099in}{3.029783in}}%
\pgfpathlineto{\pgfqpoint{1.641697in}{3.042416in}}%
\pgfpathlineto{\pgfqpoint{1.657304in}{3.043871in}}%
\pgfpathlineto{\pgfqpoint{1.681778in}{3.048123in}}%
\pgfpathlineto{\pgfqpoint{1.700497in}{3.049230in}}%
\pgfpathlineto{\pgfqpoint{1.721859in}{3.048638in}}%
\pgfpathlineto{\pgfqpoint{1.740006in}{3.046237in}}%
\pgfpathlineto{\pgfqpoint{1.761939in}{3.045216in}}%
\pgfpathlineto{\pgfqpoint{1.775760in}{3.042206in}}%
\pgfpathlineto{\pgfqpoint{1.802020in}{3.038739in}}%
\pgfpathlineto{\pgfqpoint{1.809718in}{3.036503in}}%
\pgfpathlineto{\pgfqpoint{1.843871in}{3.029333in}}%
\pgfpathlineto{\pgfqpoint{1.922263in}{3.004975in}}%
\pgfpathlineto{\pgfqpoint{1.965437in}{2.989119in}}%
\pgfpathlineto{\pgfqpoint{2.045619in}{2.954667in}}%
\pgfpathlineto{\pgfqpoint{2.122667in}{2.917065in}}%
\pgfpathlineto{\pgfqpoint{2.172929in}{2.889484in}}%
\pgfpathlineto{\pgfqpoint{2.220556in}{2.863487in}}%
\pgfpathlineto{\pgfqpoint{2.282990in}{2.826299in}}%
\pgfpathlineto{\pgfqpoint{2.296125in}{2.817568in}}%
\pgfpathlineto{\pgfqpoint{2.335482in}{2.793773in}}%
\pgfpathlineto{\pgfqpoint{2.403232in}{2.749365in}}%
\pgfpathlineto{\pgfqpoint{2.485041in}{2.693333in}}%
\pgfpathlineto{\pgfqpoint{2.506281in}{2.677318in}}%
\pgfpathlineto{\pgfqpoint{2.536833in}{2.656000in}}%
\pgfpathlineto{\pgfqpoint{2.563556in}{2.636421in}}%
\pgfpathlineto{\pgfqpoint{2.643717in}{2.576000in}}%
\pgfpathlineto{\pgfqpoint{2.731541in}{2.506667in}}%
\pgfpathlineto{\pgfqpoint{2.770059in}{2.475015in}}%
\pgfpathlineto{\pgfqpoint{2.804040in}{2.447169in}}%
\pgfpathlineto{\pgfqpoint{2.822096in}{2.432000in}}%
\pgfpathlineto{\pgfqpoint{2.909210in}{2.357333in}}%
\pgfpathlineto{\pgfqpoint{2.951568in}{2.320000in}}%
\pgfpathlineto{\pgfqpoint{3.004444in}{2.272471in}}%
\pgfpathlineto{\pgfqpoint{3.034046in}{2.245333in}}%
\pgfpathlineto{\pgfqpoint{3.084606in}{2.198290in}}%
\pgfpathlineto{\pgfqpoint{3.113740in}{2.170667in}}%
\pgfpathlineto{\pgfqpoint{3.190865in}{2.096000in}}%
\pgfpathlineto{\pgfqpoint{3.265615in}{2.021333in}}%
\pgfpathlineto{\pgfqpoint{3.338165in}{1.946667in}}%
\pgfpathlineto{\pgfqpoint{3.429860in}{1.849080in}}%
\pgfpathlineto{\pgfqpoint{3.510104in}{1.760000in}}%
\pgfpathlineto{\pgfqpoint{3.525495in}{1.742615in}}%
\pgfpathlineto{\pgfqpoint{3.613620in}{1.640582in}}%
\pgfpathlineto{\pgfqpoint{3.699673in}{1.536000in}}%
\pgfpathlineto{\pgfqpoint{3.765980in}{1.452244in}}%
\pgfpathlineto{\pgfqpoint{3.787629in}{1.424000in}}%
\pgfpathlineto{\pgfqpoint{3.846141in}{1.346313in}}%
\pgfpathlineto{\pgfqpoint{3.926303in}{1.234045in}}%
\pgfpathlineto{\pgfqpoint{3.998964in}{1.125333in}}%
\pgfpathlineto{\pgfqpoint{4.006465in}{1.113626in}}%
\pgfpathlineto{\pgfqpoint{4.046545in}{1.049978in}}%
\pgfpathlineto{\pgfqpoint{4.090411in}{0.976000in}}%
\pgfpathlineto{\pgfqpoint{4.111353in}{0.938667in}}%
\pgfpathlineto{\pgfqpoint{4.131894in}{0.901333in}}%
\pgfpathlineto{\pgfqpoint{4.151329in}{0.864000in}}%
\pgfpathlineto{\pgfqpoint{4.170299in}{0.826667in}}%
\pgfpathlineto{\pgfqpoint{4.187946in}{0.789333in}}%
\pgfpathlineto{\pgfqpoint{4.206869in}{0.747983in}}%
\pgfpathlineto{\pgfqpoint{4.250678in}{0.636527in}}%
\pgfpathlineto{\pgfqpoint{4.272113in}{0.565333in}}%
\pgfpathlineto{\pgfqpoint{4.274322in}{0.553496in}}%
\pgfpathlineto{\pgfqpoint{4.281329in}{0.528000in}}%
\pgfpathlineto{\pgfqpoint{4.301365in}{0.528000in}}%
\pgfpathlineto{\pgfqpoint{4.287030in}{0.580413in}}%
\pgfpathlineto{\pgfqpoint{4.277194in}{0.611829in}}%
\pgfpathlineto{\pgfqpoint{4.253076in}{0.677333in}}%
\pgfpathlineto{\pgfqpoint{4.237715in}{0.714667in}}%
\pgfpathlineto{\pgfqpoint{4.228214in}{0.734549in}}%
\pgfpathlineto{\pgfqpoint{4.221299in}{0.752000in}}%
\pgfpathlineto{\pgfqpoint{4.204208in}{0.789333in}}%
\pgfpathlineto{\pgfqpoint{4.185737in}{0.826667in}}%
\pgfpathlineto{\pgfqpoint{4.166788in}{0.864353in}}%
\pgfpathlineto{\pgfqpoint{4.139610in}{0.913352in}}%
\pgfpathlineto{\pgfqpoint{4.126453in}{0.938667in}}%
\pgfpathlineto{\pgfqpoint{4.098046in}{0.986637in}}%
\pgfpathlineto{\pgfqpoint{4.078362in}{1.021031in}}%
\pgfpathlineto{\pgfqpoint{4.012858in}{1.125333in}}%
\pgfpathlineto{\pgfqpoint{3.957798in}{1.207997in}}%
\pgfpathlineto{\pgfqpoint{3.880970in}{1.316892in}}%
\pgfpathlineto{\pgfqpoint{3.800910in}{1.424000in}}%
\pgfpathlineto{\pgfqpoint{3.785854in}{1.442512in}}%
\pgfpathlineto{\pgfqpoint{3.765980in}{1.469065in}}%
\pgfpathlineto{\pgfqpoint{3.671742in}{1.586444in}}%
\pgfpathlineto{\pgfqpoint{3.588298in}{1.685333in}}%
\pgfpathlineto{\pgfqpoint{3.565576in}{1.711725in}}%
\pgfpathlineto{\pgfqpoint{3.485414in}{1.802419in}}%
\pgfpathlineto{\pgfqpoint{3.386833in}{1.909333in}}%
\pgfpathlineto{\pgfqpoint{3.315650in}{1.984000in}}%
\pgfpathlineto{\pgfqpoint{3.285010in}{2.015513in}}%
\pgfpathlineto{\pgfqpoint{3.204848in}{2.096032in}}%
\pgfpathlineto{\pgfqpoint{3.184859in}{2.114714in}}%
\pgfpathlineto{\pgfqpoint{3.164768in}{2.135213in}}%
\pgfpathlineto{\pgfqpoint{3.145614in}{2.152826in}}%
\pgfpathlineto{\pgfqpoint{3.124687in}{2.173780in}}%
\pgfpathlineto{\pgfqpoint{3.106049in}{2.190640in}}%
\pgfpathlineto{\pgfqpoint{3.084606in}{2.211739in}}%
\pgfpathlineto{\pgfqpoint{3.066160in}{2.228151in}}%
\pgfpathlineto{\pgfqpoint{3.044525in}{2.249096in}}%
\pgfpathlineto{\pgfqpoint{3.025941in}{2.265356in}}%
\pgfpathlineto{\pgfqpoint{3.004444in}{2.285857in}}%
\pgfpathlineto{\pgfqpoint{2.985388in}{2.302250in}}%
\pgfpathlineto{\pgfqpoint{2.964364in}{2.322028in}}%
\pgfpathlineto{\pgfqpoint{2.881696in}{2.394667in}}%
\pgfpathlineto{\pgfqpoint{2.861676in}{2.411018in}}%
\pgfpathlineto{\pgfqpoint{2.837955in}{2.432000in}}%
\pgfpathlineto{\pgfqpoint{2.804040in}{2.460493in}}%
\pgfpathlineto{\pgfqpoint{2.723879in}{2.526110in}}%
\pgfpathlineto{\pgfqpoint{2.691734in}{2.551392in}}%
\pgfpathlineto{\pgfqpoint{2.683798in}{2.558048in}}%
\pgfpathlineto{\pgfqpoint{2.594925in}{2.626781in}}%
\pgfpathlineto{\pgfqpoint{2.504125in}{2.693333in}}%
\pgfpathlineto{\pgfqpoint{2.446673in}{2.733796in}}%
\pgfpathlineto{\pgfqpoint{2.420213in}{2.752183in}}%
\pgfpathlineto{\pgfqpoint{2.339619in}{2.805333in}}%
\pgfpathlineto{\pgfqpoint{2.280605in}{2.842667in}}%
\pgfpathlineto{\pgfqpoint{2.202828in}{2.888718in}}%
\pgfpathlineto{\pgfqpoint{2.183132in}{2.898987in}}%
\pgfpathlineto{\pgfqpoint{2.151389in}{2.917333in}}%
\pgfpathlineto{\pgfqpoint{2.106942in}{2.940020in}}%
\pgfpathlineto{\pgfqpoint{2.079955in}{2.954667in}}%
\pgfpathlineto{\pgfqpoint{2.028013in}{2.978501in}}%
\pgfpathlineto{\pgfqpoint{2.000220in}{2.992000in}}%
\pgfpathlineto{\pgfqpoint{1.882182in}{3.037399in}}%
\pgfpathlineto{\pgfqpoint{1.842101in}{3.049519in}}%
\pgfpathlineto{\pgfqpoint{1.827021in}{3.052620in}}%
\pgfpathlineto{\pgfqpoint{1.795641in}{3.060725in}}%
\pgfpathlineto{\pgfqpoint{1.761939in}{3.067953in}}%
\pgfpathlineto{\pgfqpoint{1.721859in}{3.072904in}}%
\pgfpathlineto{\pgfqpoint{1.673268in}{3.074593in}}%
\pgfpathlineto{\pgfqpoint{1.636615in}{3.071400in}}%
\pgfpathlineto{\pgfqpoint{1.601616in}{3.063388in}}%
\pgfpathlineto{\pgfqpoint{1.575808in}{3.053373in}}%
\pgfpathlineto{\pgfqpoint{1.561535in}{3.045290in}}%
\pgfpathlineto{\pgfqpoint{1.530113in}{3.021268in}}%
\pgfpathlineto{\pgfqpoint{1.521455in}{3.011011in}}%
\pgfpathlineto{\pgfqpoint{1.507954in}{2.992000in}}%
\pgfpathlineto{\pgfqpoint{1.488194in}{2.948314in}}%
\pgfpathlineto{\pgfqpoint{1.481000in}{2.917333in}}%
\pgfpathlineto{\pgfqpoint{1.477679in}{2.876559in}}%
\pgfpathlineto{\pgfqpoint{1.477547in}{2.842667in}}%
\pgfpathlineto{\pgfqpoint{1.481374in}{2.800848in}}%
\pgfpathlineto{\pgfqpoint{1.487578in}{2.762221in}}%
\pgfpathlineto{\pgfqpoint{1.494695in}{2.730667in}}%
\pgfpathlineto{\pgfqpoint{1.500568in}{2.711212in}}%
\pgfpathlineto{\pgfqpoint{1.504434in}{2.693333in}}%
\pgfpathlineto{\pgfqpoint{1.515547in}{2.656000in}}%
\pgfpathlineto{\pgfqpoint{1.531577in}{2.609238in}}%
\pgfpathlineto{\pgfqpoint{1.561535in}{2.534395in}}%
\pgfpathlineto{\pgfqpoint{1.608972in}{2.432000in}}%
\pgfpathlineto{\pgfqpoint{1.620025in}{2.411813in}}%
\pgfpathlineto{\pgfqpoint{1.632814in}{2.386393in}}%
\pgfpathlineto{\pgfqpoint{1.647691in}{2.357333in}}%
\pgfpathlineto{\pgfqpoint{1.659769in}{2.336834in}}%
\pgfpathlineto{\pgfqpoint{1.673246in}{2.312053in}}%
\pgfpathlineto{\pgfqpoint{1.698523in}{2.267070in}}%
\pgfpathlineto{\pgfqpoint{1.734198in}{2.208000in}}%
\pgfpathlineto{\pgfqpoint{1.744801in}{2.192036in}}%
\pgfpathlineto{\pgfqpoint{1.761939in}{2.163449in}}%
\pgfpathlineto{\pgfqpoint{1.805688in}{2.096000in}}%
\pgfpathlineto{\pgfqpoint{1.830890in}{2.058667in}}%
\pgfpathlineto{\pgfqpoint{1.883078in}{1.983166in}}%
\pgfpathlineto{\pgfqpoint{1.964289in}{1.872000in}}%
\pgfpathlineto{\pgfqpoint{1.992730in}{1.834667in}}%
\pgfpathlineto{\pgfqpoint{2.050807in}{1.760000in}}%
\pgfpathlineto{\pgfqpoint{2.064745in}{1.743382in}}%
\pgfpathlineto{\pgfqpoint{2.082586in}{1.720058in}}%
\pgfpathlineto{\pgfqpoint{2.110944in}{1.685333in}}%
\pgfpathlineto{\pgfqpoint{2.172990in}{1.610667in}}%
\pgfpathlineto{\pgfqpoint{2.186532in}{1.595488in}}%
\pgfpathlineto{\pgfqpoint{2.211717in}{1.565054in}}%
\pgfpathlineto{\pgfqpoint{2.282990in}{1.483783in}}%
\pgfpathlineto{\pgfqpoint{2.312449in}{1.451440in}}%
\pgfpathlineto{\pgfqpoint{2.336832in}{1.424000in}}%
\pgfpathlineto{\pgfqpoint{2.363152in}{1.395180in}}%
\pgfpathlineto{\pgfqpoint{2.443313in}{1.309548in}}%
\pgfpathlineto{\pgfqpoint{2.550274in}{1.200000in}}%
\pgfpathlineto{\pgfqpoint{2.587775in}{1.162667in}}%
\pgfpathlineto{\pgfqpoint{2.664502in}{1.088000in}}%
\pgfpathlineto{\pgfqpoint{2.743682in}{1.013333in}}%
\pgfpathlineto{\pgfqpoint{2.784252in}{0.976000in}}%
\pgfpathlineto{\pgfqpoint{2.867489in}{0.901333in}}%
\pgfpathlineto{\pgfqpoint{2.953716in}{0.826667in}}%
\pgfpathlineto{\pgfqpoint{3.004444in}{0.783966in}}%
\pgfpathlineto{\pgfqpoint{3.089569in}{0.714667in}}%
\pgfpathlineto{\pgfqpoint{3.185574in}{0.640000in}}%
\pgfpathlineto{\pgfqpoint{3.244929in}{0.595425in}}%
\pgfpathlineto{\pgfqpoint{3.325091in}{0.537737in}}%
\pgfpathlineto{\pgfqpoint{3.339107in}{0.528000in}}%
\pgfpathlineto{\pgfqpoint{3.339107in}{0.528000in}}%
\pgfusepath{fill}%
\end{pgfscope}%
\begin{pgfscope}%
\pgfpathrectangle{\pgfqpoint{0.800000in}{0.528000in}}{\pgfqpoint{3.968000in}{3.696000in}}%
\pgfusepath{clip}%
\pgfsetbuttcap%
\pgfsetroundjoin%
\definecolor{currentfill}{rgb}{0.276022,0.044167,0.370164}%
\pgfsetfillcolor{currentfill}%
\pgfsetlinewidth{0.000000pt}%
\definecolor{currentstroke}{rgb}{0.000000,0.000000,0.000000}%
\pgfsetstrokecolor{currentstroke}%
\pgfsetdash{}{0pt}%
\pgfpathmoveto{\pgfqpoint{3.339107in}{0.528000in}}%
\pgfpathlineto{\pgfqpoint{3.325091in}{0.537737in}}%
\pgfpathlineto{\pgfqpoint{3.235208in}{0.602667in}}%
\pgfpathlineto{\pgfqpoint{3.204848in}{0.625357in}}%
\pgfpathlineto{\pgfqpoint{3.124687in}{0.686938in}}%
\pgfpathlineto{\pgfqpoint{3.043192in}{0.752000in}}%
\pgfpathlineto{\pgfqpoint{2.953716in}{0.826667in}}%
\pgfpathlineto{\pgfqpoint{2.867489in}{0.901333in}}%
\pgfpathlineto{\pgfqpoint{2.844121in}{0.921991in}}%
\pgfpathlineto{\pgfqpoint{2.743682in}{1.013333in}}%
\pgfpathlineto{\pgfqpoint{2.664502in}{1.088000in}}%
\pgfpathlineto{\pgfqpoint{2.643717in}{1.107954in}}%
\pgfpathlineto{\pgfqpoint{2.550274in}{1.200000in}}%
\pgfpathlineto{\pgfqpoint{2.476897in}{1.274667in}}%
\pgfpathlineto{\pgfqpoint{2.403232in}{1.351947in}}%
\pgfpathlineto{\pgfqpoint{2.371022in}{1.386667in}}%
\pgfpathlineto{\pgfqpoint{2.303102in}{1.461333in}}%
\pgfpathlineto{\pgfqpoint{2.282990in}{1.483783in}}%
\pgfpathlineto{\pgfqpoint{2.202828in}{1.575401in}}%
\pgfpathlineto{\pgfqpoint{2.172990in}{1.610667in}}%
\pgfpathlineto{\pgfqpoint{2.110944in}{1.685333in}}%
\pgfpathlineto{\pgfqpoint{2.099085in}{1.700701in}}%
\pgfpathlineto{\pgfqpoint{2.080480in}{1.722667in}}%
\pgfpathlineto{\pgfqpoint{2.050807in}{1.760000in}}%
\pgfpathlineto{\pgfqpoint{1.992730in}{1.834667in}}%
\pgfpathlineto{\pgfqpoint{1.980545in}{1.851621in}}%
\pgfpathlineto{\pgfqpoint{1.962343in}{1.874605in}}%
\pgfpathlineto{\pgfqpoint{1.882182in}{1.984435in}}%
\pgfpathlineto{\pgfqpoint{1.851137in}{2.029750in}}%
\pgfpathlineto{\pgfqpoint{1.830890in}{2.058667in}}%
\pgfpathlineto{\pgfqpoint{1.820189in}{2.075590in}}%
\pgfpathlineto{\pgfqpoint{1.802020in}{2.101576in}}%
\pgfpathlineto{\pgfqpoint{1.757335in}{2.170667in}}%
\pgfpathlineto{\pgfqpoint{1.729967in}{2.215552in}}%
\pgfpathlineto{\pgfqpoint{1.711543in}{2.245333in}}%
\pgfpathlineto{\pgfqpoint{1.681778in}{2.296225in}}%
\pgfpathlineto{\pgfqpoint{1.601616in}{2.447083in}}%
\pgfpathlineto{\pgfqpoint{1.557368in}{2.544000in}}%
\pgfpathlineto{\pgfqpoint{1.542440in}{2.581333in}}%
\pgfpathlineto{\pgfqpoint{1.528291in}{2.618667in}}%
\pgfpathlineto{\pgfqpoint{1.512881in}{2.663986in}}%
\pgfpathlineto{\pgfqpoint{1.504434in}{2.693333in}}%
\pgfpathlineto{\pgfqpoint{1.500568in}{2.711212in}}%
\pgfpathlineto{\pgfqpoint{1.494695in}{2.730667in}}%
\pgfpathlineto{\pgfqpoint{1.486623in}{2.768000in}}%
\pgfpathlineto{\pgfqpoint{1.480555in}{2.806095in}}%
\pgfpathlineto{\pgfqpoint{1.477547in}{2.842667in}}%
\pgfpathlineto{\pgfqpoint{1.477433in}{2.883671in}}%
\pgfpathlineto{\pgfqpoint{1.481374in}{2.919135in}}%
\pgfpathlineto{\pgfqpoint{1.490595in}{2.954667in}}%
\pgfpathlineto{\pgfqpoint{1.498781in}{2.970881in}}%
\pgfpathlineto{\pgfqpoint{1.507954in}{2.992000in}}%
\pgfpathlineto{\pgfqpoint{1.521455in}{3.011011in}}%
\pgfpathlineto{\pgfqpoint{1.530113in}{3.021268in}}%
\pgfpathlineto{\pgfqpoint{1.539598in}{3.029333in}}%
\pgfpathlineto{\pgfqpoint{1.561535in}{3.045290in}}%
\pgfpathlineto{\pgfqpoint{1.575808in}{3.053373in}}%
\pgfpathlineto{\pgfqpoint{1.601616in}{3.063388in}}%
\pgfpathlineto{\pgfqpoint{1.636615in}{3.071400in}}%
\pgfpathlineto{\pgfqpoint{1.673268in}{3.074593in}}%
\pgfpathlineto{\pgfqpoint{1.689359in}{3.073728in}}%
\pgfpathlineto{\pgfqpoint{1.727393in}{3.071821in}}%
\pgfpathlineto{\pgfqpoint{1.768358in}{3.066667in}}%
\pgfpathlineto{\pgfqpoint{1.802020in}{3.059883in}}%
\pgfpathlineto{\pgfqpoint{1.827021in}{3.052620in}}%
\pgfpathlineto{\pgfqpoint{1.842101in}{3.049519in}}%
\pgfpathlineto{\pgfqpoint{1.882182in}{3.037399in}}%
\pgfpathlineto{\pgfqpoint{1.932647in}{3.019660in}}%
\pgfpathlineto{\pgfqpoint{2.002424in}{2.991078in}}%
\pgfpathlineto{\pgfqpoint{2.054740in}{2.966063in}}%
\pgfpathlineto{\pgfqpoint{2.082586in}{2.953404in}}%
\pgfpathlineto{\pgfqpoint{2.106942in}{2.940020in}}%
\pgfpathlineto{\pgfqpoint{2.132775in}{2.926748in}}%
\pgfpathlineto{\pgfqpoint{2.162747in}{2.911248in}}%
\pgfpathlineto{\pgfqpoint{2.183132in}{2.898987in}}%
\pgfpathlineto{\pgfqpoint{2.217764in}{2.880000in}}%
\pgfpathlineto{\pgfqpoint{2.282990in}{2.841232in}}%
\pgfpathlineto{\pgfqpoint{2.305481in}{2.826283in}}%
\pgfpathlineto{\pgfqpoint{2.339619in}{2.805333in}}%
\pgfpathlineto{\pgfqpoint{2.420213in}{2.752183in}}%
\pgfpathlineto{\pgfqpoint{2.504125in}{2.693333in}}%
\pgfpathlineto{\pgfqpoint{2.523475in}{2.679448in}}%
\pgfpathlineto{\pgfqpoint{2.605655in}{2.618667in}}%
\pgfpathlineto{\pgfqpoint{2.701461in}{2.544000in}}%
\pgfpathlineto{\pgfqpoint{2.723879in}{2.526110in}}%
\pgfpathlineto{\pgfqpoint{2.804040in}{2.460493in}}%
\pgfpathlineto{\pgfqpoint{2.819736in}{2.446619in}}%
\pgfpathlineto{\pgfqpoint{2.844121in}{2.426802in}}%
\pgfpathlineto{\pgfqpoint{2.861676in}{2.411018in}}%
\pgfpathlineto{\pgfqpoint{2.884202in}{2.392516in}}%
\pgfpathlineto{\pgfqpoint{2.966618in}{2.320000in}}%
\pgfpathlineto{\pgfqpoint{2.985388in}{2.302250in}}%
\pgfpathlineto{\pgfqpoint{3.007934in}{2.282667in}}%
\pgfpathlineto{\pgfqpoint{3.025941in}{2.265356in}}%
\pgfpathlineto{\pgfqpoint{3.048575in}{2.245333in}}%
\pgfpathlineto{\pgfqpoint{3.066160in}{2.228151in}}%
\pgfpathlineto{\pgfqpoint{3.088566in}{2.208000in}}%
\pgfpathlineto{\pgfqpoint{3.106049in}{2.190640in}}%
\pgfpathlineto{\pgfqpoint{3.127932in}{2.170667in}}%
\pgfpathlineto{\pgfqpoint{3.145614in}{2.152826in}}%
\pgfpathlineto{\pgfqpoint{3.166696in}{2.133333in}}%
\pgfpathlineto{\pgfqpoint{3.184859in}{2.114714in}}%
\pgfpathlineto{\pgfqpoint{3.204880in}{2.096000in}}%
\pgfpathlineto{\pgfqpoint{3.223790in}{2.076309in}}%
\pgfpathlineto{\pgfqpoint{3.244929in}{2.056103in}}%
\pgfpathlineto{\pgfqpoint{3.279280in}{2.021333in}}%
\pgfpathlineto{\pgfqpoint{3.351494in}{1.946667in}}%
\pgfpathlineto{\pgfqpoint{3.421683in}{1.872000in}}%
\pgfpathlineto{\pgfqpoint{3.513123in}{1.771524in}}%
\pgfpathlineto{\pgfqpoint{3.565576in}{1.711725in}}%
\pgfpathlineto{\pgfqpoint{3.595773in}{1.676128in}}%
\pgfpathlineto{\pgfqpoint{3.620147in}{1.648000in}}%
\pgfpathlineto{\pgfqpoint{3.685818in}{1.569282in}}%
\pgfpathlineto{\pgfqpoint{3.772030in}{1.461333in}}%
\pgfpathlineto{\pgfqpoint{3.785854in}{1.442512in}}%
\pgfpathlineto{\pgfqpoint{3.806061in}{1.417251in}}%
\pgfpathlineto{\pgfqpoint{3.835863in}{1.377093in}}%
\pgfpathlineto{\pgfqpoint{3.857014in}{1.349333in}}%
\pgfpathlineto{\pgfqpoint{3.911086in}{1.274667in}}%
\pgfpathlineto{\pgfqpoint{3.966384in}{1.195327in}}%
\pgfpathlineto{\pgfqpoint{3.988166in}{1.162667in}}%
\pgfpathlineto{\pgfqpoint{4.036764in}{1.088000in}}%
\pgfpathlineto{\pgfqpoint{4.086626in}{1.007115in}}%
\pgfpathlineto{\pgfqpoint{4.112035in}{0.962334in}}%
\pgfpathlineto{\pgfqpoint{4.126707in}{0.938207in}}%
\pgfpathlineto{\pgfqpoint{4.153040in}{0.888528in}}%
\pgfpathlineto{\pgfqpoint{4.167159in}{0.863654in}}%
\pgfpathlineto{\pgfqpoint{4.206869in}{0.783595in}}%
\pgfpathlineto{\pgfqpoint{4.237715in}{0.714667in}}%
\pgfpathlineto{\pgfqpoint{4.253076in}{0.677333in}}%
\pgfpathlineto{\pgfqpoint{4.267045in}{0.640000in}}%
\pgfpathlineto{\pgfqpoint{4.280093in}{0.602667in}}%
\pgfpathlineto{\pgfqpoint{4.293459in}{0.559345in}}%
\pgfpathlineto{\pgfqpoint{4.301365in}{0.528000in}}%
\pgfpathlineto{\pgfqpoint{4.320932in}{0.528000in}}%
\pgfpathlineto{\pgfqpoint{4.319035in}{0.535523in}}%
\pgfpathlineto{\pgfqpoint{4.297936in}{0.602667in}}%
\pgfpathlineto{\pgfqpoint{4.284663in}{0.640000in}}%
\pgfpathlineto{\pgfqpoint{4.269729in}{0.677333in}}%
\pgfpathlineto{\pgfqpoint{4.246949in}{0.730734in}}%
\pgfpathlineto{\pgfqpoint{4.196167in}{0.836635in}}%
\pgfpathlineto{\pgfqpoint{4.161680in}{0.901333in}}%
\pgfpathlineto{\pgfqpoint{4.135541in}{0.946895in}}%
\pgfpathlineto{\pgfqpoint{4.119183in}{0.976000in}}%
\pgfpathlineto{\pgfqpoint{4.107086in}{0.995058in}}%
\pgfpathlineto{\pgfqpoint{4.086626in}{1.030181in}}%
\pgfpathlineto{\pgfqpoint{4.046545in}{1.094321in}}%
\pgfpathlineto{\pgfqpoint{4.001805in}{1.162667in}}%
\pgfpathlineto{\pgfqpoint{3.976454in}{1.200000in}}%
\pgfpathlineto{\pgfqpoint{3.920843in}{1.279752in}}%
\pgfpathlineto{\pgfqpoint{3.842241in}{1.386667in}}%
\pgfpathlineto{\pgfqpoint{3.813761in}{1.424000in}}%
\pgfpathlineto{\pgfqpoint{3.755418in}{1.498667in}}%
\pgfpathlineto{\pgfqpoint{3.742456in}{1.514089in}}%
\pgfpathlineto{\pgfqpoint{3.725239in}{1.536615in}}%
\pgfpathlineto{\pgfqpoint{3.632938in}{1.648000in}}%
\pgfpathlineto{\pgfqpoint{3.603143in}{1.682992in}}%
\pgfpathlineto{\pgfqpoint{3.601241in}{1.685333in}}%
\pgfpathlineto{\pgfqpoint{3.584912in}{1.703344in}}%
\pgfpathlineto{\pgfqpoint{3.565576in}{1.726538in}}%
\pgfpathlineto{\pgfqpoint{3.468919in}{1.834667in}}%
\pgfpathlineto{\pgfqpoint{3.400000in}{1.909333in}}%
\pgfpathlineto{\pgfqpoint{3.325091in}{1.987974in}}%
\pgfpathlineto{\pgfqpoint{3.292513in}{2.021333in}}%
\pgfpathlineto{\pgfqpoint{3.218117in}{2.096000in}}%
\pgfpathlineto{\pgfqpoint{3.141499in}{2.170667in}}%
\pgfpathlineto{\pgfqpoint{3.062489in}{2.245333in}}%
\pgfpathlineto{\pgfqpoint{2.980898in}{2.320000in}}%
\pgfpathlineto{\pgfqpoint{2.931324in}{2.363892in}}%
\pgfpathlineto{\pgfqpoint{2.896517in}{2.394667in}}%
\pgfpathlineto{\pgfqpoint{2.844121in}{2.439778in}}%
\pgfpathlineto{\pgfqpoint{2.827075in}{2.453456in}}%
\pgfpathlineto{\pgfqpoint{2.804040in}{2.473602in}}%
\pgfpathlineto{\pgfqpoint{2.784813in}{2.488757in}}%
\pgfpathlineto{\pgfqpoint{2.762259in}{2.508250in}}%
\pgfpathlineto{\pgfqpoint{2.670912in}{2.581333in}}%
\pgfpathlineto{\pgfqpoint{2.643717in}{2.602563in}}%
\pgfpathlineto{\pgfqpoint{2.563556in}{2.663513in}}%
\pgfpathlineto{\pgfqpoint{2.483394in}{2.721806in}}%
\pgfpathlineto{\pgfqpoint{2.454786in}{2.741353in}}%
\pgfpathlineto{\pgfqpoint{2.443313in}{2.749900in}}%
\pgfpathlineto{\pgfqpoint{2.361630in}{2.805333in}}%
\pgfpathlineto{\pgfqpoint{2.282990in}{2.855443in}}%
\pgfpathlineto{\pgfqpoint{2.202828in}{2.903540in}}%
\pgfpathlineto{\pgfqpoint{2.178632in}{2.917333in}}%
\pgfpathlineto{\pgfqpoint{2.122667in}{2.948239in}}%
\pgfpathlineto{\pgfqpoint{2.110288in}{2.954667in}}%
\pgfpathlineto{\pgfqpoint{2.036006in}{2.992000in}}%
\pgfpathlineto{\pgfqpoint{1.962343in}{3.025122in}}%
\pgfpathlineto{\pgfqpoint{1.900784in}{3.049339in}}%
\pgfpathlineto{\pgfqpoint{1.838303in}{3.070204in}}%
\pgfpathlineto{\pgfqpoint{1.782479in}{3.084868in}}%
\pgfpathlineto{\pgfqpoint{1.761939in}{3.088986in}}%
\pgfpathlineto{\pgfqpoint{1.747550in}{3.090597in}}%
\pgfpathlineto{\pgfqpoint{1.721859in}{3.095622in}}%
\pgfpathlineto{\pgfqpoint{1.713103in}{3.095844in}}%
\pgfpathlineto{\pgfqpoint{1.681778in}{3.099250in}}%
\pgfpathlineto{\pgfqpoint{1.641697in}{3.099009in}}%
\pgfpathlineto{\pgfqpoint{1.601616in}{3.093671in}}%
\pgfpathlineto{\pgfqpoint{1.584028in}{3.087617in}}%
\pgfpathlineto{\pgfqpoint{1.561535in}{3.081434in}}%
\pgfpathlineto{\pgfqpoint{1.550573in}{3.076878in}}%
\pgfpathlineto{\pgfqpoint{1.521455in}{3.058334in}}%
\pgfpathlineto{\pgfqpoint{1.505460in}{3.044231in}}%
\pgfpathlineto{\pgfqpoint{1.493222in}{3.029333in}}%
\pgfpathlineto{\pgfqpoint{1.474274in}{2.998613in}}%
\pgfpathlineto{\pgfqpoint{1.463862in}{2.970978in}}%
\pgfpathlineto{\pgfqpoint{1.460033in}{2.954667in}}%
\pgfpathlineto{\pgfqpoint{1.457257in}{2.932203in}}%
\pgfpathlineto{\pgfqpoint{1.453017in}{2.906413in}}%
\pgfpathlineto{\pgfqpoint{1.452742in}{2.880000in}}%
\pgfpathlineto{\pgfqpoint{1.459457in}{2.805333in}}%
\pgfpathlineto{\pgfqpoint{1.463345in}{2.788540in}}%
\pgfpathlineto{\pgfqpoint{1.466354in}{2.768000in}}%
\pgfpathlineto{\pgfqpoint{1.469560in}{2.756996in}}%
\pgfpathlineto{\pgfqpoint{1.475034in}{2.730667in}}%
\pgfpathlineto{\pgfqpoint{1.487000in}{2.688093in}}%
\pgfpathlineto{\pgfqpoint{1.510961in}{2.618667in}}%
\pgfpathlineto{\pgfqpoint{1.525330in}{2.581333in}}%
\pgfpathlineto{\pgfqpoint{1.541169in}{2.544000in}}%
\pgfpathlineto{\pgfqpoint{1.561535in}{2.498244in}}%
\pgfpathlineto{\pgfqpoint{1.601616in}{2.416525in}}%
\pgfpathlineto{\pgfqpoint{1.681778in}{2.271575in}}%
\pgfpathlineto{\pgfqpoint{1.721859in}{2.205153in}}%
\pgfpathlineto{\pgfqpoint{1.750835in}{2.160324in}}%
\pgfpathlineto{\pgfqpoint{1.776340in}{2.119920in}}%
\pgfpathlineto{\pgfqpoint{1.843056in}{2.021333in}}%
\pgfpathlineto{\pgfqpoint{1.869547in}{1.984000in}}%
\pgfpathlineto{\pgfqpoint{1.926146in}{1.905716in}}%
\pgfpathlineto{\pgfqpoint{2.008609in}{1.797333in}}%
\pgfpathlineto{\pgfqpoint{2.067876in}{1.722667in}}%
\pgfpathlineto{\pgfqpoint{2.128873in}{1.648000in}}%
\pgfpathlineto{\pgfqpoint{2.144116in}{1.630646in}}%
\pgfpathlineto{\pgfqpoint{2.162747in}{1.607511in}}%
\pgfpathlineto{\pgfqpoint{2.191985in}{1.573333in}}%
\pgfpathlineto{\pgfqpoint{2.256997in}{1.498667in}}%
\pgfpathlineto{\pgfqpoint{2.323706in}{1.424000in}}%
\pgfpathlineto{\pgfqpoint{2.358022in}{1.386667in}}%
\pgfpathlineto{\pgfqpoint{2.428099in}{1.312000in}}%
\pgfpathlineto{\pgfqpoint{2.500127in}{1.237333in}}%
\pgfpathlineto{\pgfqpoint{2.550048in}{1.187418in}}%
\pgfpathlineto{\pgfqpoint{2.574251in}{1.162667in}}%
\pgfpathlineto{\pgfqpoint{2.650633in}{1.088000in}}%
\pgfpathlineto{\pgfqpoint{2.729449in}{1.013333in}}%
\pgfpathlineto{\pgfqpoint{2.763960in}{0.981367in}}%
\pgfpathlineto{\pgfqpoint{2.852674in}{0.901333in}}%
\pgfpathlineto{\pgfqpoint{2.938486in}{0.826667in}}%
\pgfpathlineto{\pgfqpoint{2.994341in}{0.779923in}}%
\pgfpathlineto{\pgfqpoint{3.027522in}{0.752000in}}%
\pgfpathlineto{\pgfqpoint{3.120070in}{0.677333in}}%
\pgfpathlineto{\pgfqpoint{3.285010in}{0.552782in}}%
\pgfpathlineto{\pgfqpoint{3.319527in}{0.528000in}}%
\pgfpathlineto{\pgfqpoint{3.325091in}{0.528000in}}%
\pgfpathlineto{\pgfqpoint{3.325091in}{0.528000in}}%
\pgfusepath{fill}%
\end{pgfscope}%
\begin{pgfscope}%
\pgfpathrectangle{\pgfqpoint{0.800000in}{0.528000in}}{\pgfqpoint{3.968000in}{3.696000in}}%
\pgfusepath{clip}%
\pgfsetbuttcap%
\pgfsetroundjoin%
\definecolor{currentfill}{rgb}{0.277018,0.050344,0.375715}%
\pgfsetfillcolor{currentfill}%
\pgfsetlinewidth{0.000000pt}%
\definecolor{currentstroke}{rgb}{0.000000,0.000000,0.000000}%
\pgfsetstrokecolor{currentstroke}%
\pgfsetdash{}{0pt}%
\pgfpathmoveto{\pgfqpoint{3.319527in}{0.528000in}}%
\pgfpathlineto{\pgfqpoint{3.217327in}{0.602667in}}%
\pgfpathlineto{\pgfqpoint{3.166253in}{0.641384in}}%
\pgfpathlineto{\pgfqpoint{3.149145in}{0.654551in}}%
\pgfpathlineto{\pgfqpoint{3.073338in}{0.714667in}}%
\pgfpathlineto{\pgfqpoint{2.982584in}{0.789333in}}%
\pgfpathlineto{\pgfqpoint{2.924283in}{0.838800in}}%
\pgfpathlineto{\pgfqpoint{2.889637in}{0.869063in}}%
\pgfpathlineto{\pgfqpoint{2.884202in}{0.873553in}}%
\pgfpathlineto{\pgfqpoint{2.804040in}{0.944831in}}%
\pgfpathlineto{\pgfqpoint{2.723879in}{1.018510in}}%
\pgfpathlineto{\pgfqpoint{2.643717in}{1.094639in}}%
\pgfpathlineto{\pgfqpoint{2.563556in}{1.173268in}}%
\pgfpathlineto{\pgfqpoint{2.536917in}{1.200000in}}%
\pgfpathlineto{\pgfqpoint{2.463860in}{1.274667in}}%
\pgfpathlineto{\pgfqpoint{2.392825in}{1.349333in}}%
\pgfpathlineto{\pgfqpoint{2.363152in}{1.381109in}}%
\pgfpathlineto{\pgfqpoint{2.282990in}{1.469306in}}%
\pgfpathlineto{\pgfqpoint{2.191985in}{1.573333in}}%
\pgfpathlineto{\pgfqpoint{2.179401in}{1.588845in}}%
\pgfpathlineto{\pgfqpoint{2.160078in}{1.610667in}}%
\pgfpathlineto{\pgfqpoint{2.126254in}{1.651341in}}%
\pgfpathlineto{\pgfqpoint{2.098192in}{1.685333in}}%
\pgfpathlineto{\pgfqpoint{2.082586in}{1.704443in}}%
\pgfpathlineto{\pgfqpoint{2.002424in}{1.805299in}}%
\pgfpathlineto{\pgfqpoint{1.922263in}{1.910986in}}%
\pgfpathlineto{\pgfqpoint{1.842101in}{2.022715in}}%
\pgfpathlineto{\pgfqpoint{1.767628in}{2.133333in}}%
\pgfpathlineto{\pgfqpoint{1.717604in}{2.211963in}}%
\pgfpathlineto{\pgfqpoint{1.675218in}{2.282667in}}%
\pgfpathlineto{\pgfqpoint{1.664014in}{2.303454in}}%
\pgfpathlineto{\pgfqpoint{1.641697in}{2.341611in}}%
\pgfpathlineto{\pgfqpoint{1.612992in}{2.394667in}}%
\pgfpathlineto{\pgfqpoint{1.601616in}{2.416525in}}%
\pgfpathlineto{\pgfqpoint{1.575319in}{2.469333in}}%
\pgfpathlineto{\pgfqpoint{1.554538in}{2.513184in}}%
\pgfpathlineto{\pgfqpoint{1.521455in}{2.591288in}}%
\pgfpathlineto{\pgfqpoint{1.497715in}{2.656000in}}%
\pgfpathlineto{\pgfqpoint{1.481374in}{2.707879in}}%
\pgfpathlineto{\pgfqpoint{1.475034in}{2.730667in}}%
\pgfpathlineto{\pgfqpoint{1.459457in}{2.805333in}}%
\pgfpathlineto{\pgfqpoint{1.454742in}{2.842667in}}%
\pgfpathlineto{\pgfqpoint{1.454792in}{2.855240in}}%
\pgfpathlineto{\pgfqpoint{1.452742in}{2.880000in}}%
\pgfpathlineto{\pgfqpoint{1.454174in}{2.917333in}}%
\pgfpathlineto{\pgfqpoint{1.457257in}{2.932203in}}%
\pgfpathlineto{\pgfqpoint{1.460033in}{2.954667in}}%
\pgfpathlineto{\pgfqpoint{1.463862in}{2.970978in}}%
\pgfpathlineto{\pgfqpoint{1.474274in}{2.998613in}}%
\pgfpathlineto{\pgfqpoint{1.481374in}{3.011345in}}%
\pgfpathlineto{\pgfqpoint{1.493222in}{3.029333in}}%
\pgfpathlineto{\pgfqpoint{1.505460in}{3.044231in}}%
\pgfpathlineto{\pgfqpoint{1.521455in}{3.058334in}}%
\pgfpathlineto{\pgfqpoint{1.533436in}{3.066667in}}%
\pgfpathlineto{\pgfqpoint{1.550573in}{3.076878in}}%
\pgfpathlineto{\pgfqpoint{1.561535in}{3.081434in}}%
\pgfpathlineto{\pgfqpoint{1.584028in}{3.087617in}}%
\pgfpathlineto{\pgfqpoint{1.610599in}{3.095632in}}%
\pgfpathlineto{\pgfqpoint{1.646614in}{3.099420in}}%
\pgfpathlineto{\pgfqpoint{1.681778in}{3.099250in}}%
\pgfpathlineto{\pgfqpoint{1.721859in}{3.095622in}}%
\pgfpathlineto{\pgfqpoint{1.747550in}{3.090597in}}%
\pgfpathlineto{\pgfqpoint{1.761939in}{3.088986in}}%
\pgfpathlineto{\pgfqpoint{1.802020in}{3.079964in}}%
\pgfpathlineto{\pgfqpoint{1.849261in}{3.066667in}}%
\pgfpathlineto{\pgfqpoint{1.900784in}{3.049339in}}%
\pgfpathlineto{\pgfqpoint{1.962343in}{3.025122in}}%
\pgfpathlineto{\pgfqpoint{1.985920in}{3.013961in}}%
\pgfpathlineto{\pgfqpoint{2.002424in}{3.007486in}}%
\pgfpathlineto{\pgfqpoint{2.042505in}{2.988991in}}%
\pgfpathlineto{\pgfqpoint{2.065957in}{2.976511in}}%
\pgfpathlineto{\pgfqpoint{2.082586in}{2.968996in}}%
\pgfpathlineto{\pgfqpoint{2.122667in}{2.948239in}}%
\pgfpathlineto{\pgfqpoint{2.143049in}{2.936319in}}%
\pgfpathlineto{\pgfqpoint{2.178632in}{2.917333in}}%
\pgfpathlineto{\pgfqpoint{2.202828in}{2.903540in}}%
\pgfpathlineto{\pgfqpoint{2.243133in}{2.880000in}}%
\pgfpathlineto{\pgfqpoint{2.323071in}{2.830183in}}%
\pgfpathlineto{\pgfqpoint{2.363152in}{2.804349in}}%
\pgfpathlineto{\pgfqpoint{2.385342in}{2.788669in}}%
\pgfpathlineto{\pgfqpoint{2.416991in}{2.768000in}}%
\pgfpathlineto{\pgfqpoint{2.483394in}{2.721806in}}%
\pgfpathlineto{\pgfqpoint{2.500227in}{2.709013in}}%
\pgfpathlineto{\pgfqpoint{2.524423in}{2.692451in}}%
\pgfpathlineto{\pgfqpoint{2.622764in}{2.618667in}}%
\pgfpathlineto{\pgfqpoint{2.683798in}{2.571241in}}%
\pgfpathlineto{\pgfqpoint{2.699187in}{2.558334in}}%
\pgfpathlineto{\pgfqpoint{2.723879in}{2.539346in}}%
\pgfpathlineto{\pgfqpoint{2.809113in}{2.469333in}}%
\pgfpathlineto{\pgfqpoint{2.844121in}{2.439778in}}%
\pgfpathlineto{\pgfqpoint{2.939070in}{2.357333in}}%
\pgfpathlineto{\pgfqpoint{3.022029in}{2.282667in}}%
\pgfpathlineto{\pgfqpoint{3.044525in}{2.262028in}}%
\pgfpathlineto{\pgfqpoint{3.141499in}{2.170667in}}%
\pgfpathlineto{\pgfqpoint{3.172432in}{2.140473in}}%
\pgfpathlineto{\pgfqpoint{3.204848in}{2.109133in}}%
\pgfpathlineto{\pgfqpoint{3.292513in}{2.021333in}}%
\pgfpathlineto{\pgfqpoint{3.307643in}{2.005082in}}%
\pgfpathlineto{\pgfqpoint{3.328927in}{1.984000in}}%
\pgfpathlineto{\pgfqpoint{3.345622in}{1.965790in}}%
\pgfpathlineto{\pgfqpoint{3.365172in}{1.946302in}}%
\pgfpathlineto{\pgfqpoint{3.468919in}{1.834667in}}%
\pgfpathlineto{\pgfqpoint{3.536034in}{1.760000in}}%
\pgfpathlineto{\pgfqpoint{3.605657in}{1.680185in}}%
\pgfpathlineto{\pgfqpoint{3.632938in}{1.648000in}}%
\pgfpathlineto{\pgfqpoint{3.695176in}{1.573333in}}%
\pgfpathlineto{\pgfqpoint{3.765980in}{1.485328in}}%
\pgfpathlineto{\pgfqpoint{3.793410in}{1.449550in}}%
\pgfpathlineto{\pgfqpoint{3.813761in}{1.424000in}}%
\pgfpathlineto{\pgfqpoint{3.826790in}{1.405975in}}%
\pgfpathlineto{\pgfqpoint{3.846141in}{1.381471in}}%
\pgfpathlineto{\pgfqpoint{3.926303in}{1.272064in}}%
\pgfpathlineto{\pgfqpoint{3.950591in}{1.237333in}}%
\pgfpathlineto{\pgfqpoint{4.006465in}{1.155644in}}%
\pgfpathlineto{\pgfqpoint{4.050616in}{1.088000in}}%
\pgfpathlineto{\pgfqpoint{4.073979in}{1.050667in}}%
\pgfpathlineto{\pgfqpoint{4.121669in}{0.971307in}}%
\pgfpathlineto{\pgfqpoint{4.140667in}{0.938667in}}%
\pgfpathlineto{\pgfqpoint{4.149181in}{0.922266in}}%
\pgfpathlineto{\pgfqpoint{4.166788in}{0.891890in}}%
\pgfpathlineto{\pgfqpoint{4.181668in}{0.864000in}}%
\pgfpathlineto{\pgfqpoint{4.206869in}{0.815240in}}%
\pgfpathlineto{\pgfqpoint{4.259139in}{0.703313in}}%
\pgfpathlineto{\pgfqpoint{4.287030in}{0.633428in}}%
\pgfpathlineto{\pgfqpoint{4.297936in}{0.602667in}}%
\pgfpathlineto{\pgfqpoint{4.304124in}{0.581255in}}%
\pgfpathlineto{\pgfqpoint{4.310036in}{0.565333in}}%
\pgfpathlineto{\pgfqpoint{4.320932in}{0.528000in}}%
\pgfpathlineto{\pgfqpoint{4.339481in}{0.528000in}}%
\pgfpathlineto{\pgfqpoint{4.336370in}{0.536624in}}%
\pgfpathlineto{\pgfqpoint{4.327111in}{0.568859in}}%
\pgfpathlineto{\pgfqpoint{4.285987in}{0.678306in}}%
\pgfpathlineto{\pgfqpoint{4.246949in}{0.764458in}}%
\pgfpathlineto{\pgfqpoint{4.206869in}{0.844165in}}%
\pgfpathlineto{\pgfqpoint{4.166788in}{0.917721in}}%
\pgfpathlineto{\pgfqpoint{4.063920in}{1.088000in}}%
\pgfpathlineto{\pgfqpoint{4.046545in}{1.114981in}}%
\pgfpathlineto{\pgfqpoint{4.006465in}{1.175223in}}%
\pgfpathlineto{\pgfqpoint{3.989507in}{1.200000in}}%
\pgfpathlineto{\pgfqpoint{3.937221in}{1.274667in}}%
\pgfpathlineto{\pgfqpoint{3.926303in}{1.289871in}}%
\pgfpathlineto{\pgfqpoint{3.854880in}{1.386667in}}%
\pgfpathlineto{\pgfqpoint{3.826340in}{1.424000in}}%
\pgfpathlineto{\pgfqpoint{3.765980in}{1.501422in}}%
\pgfpathlineto{\pgfqpoint{3.738098in}{1.536000in}}%
\pgfpathlineto{\pgfqpoint{3.676886in}{1.610667in}}%
\pgfpathlineto{\pgfqpoint{3.662862in}{1.626618in}}%
\pgfpathlineto{\pgfqpoint{3.645728in}{1.648000in}}%
\pgfpathlineto{\pgfqpoint{3.613744in}{1.685333in}}%
\pgfpathlineto{\pgfqpoint{3.548593in}{1.760000in}}%
\pgfpathlineto{\pgfqpoint{3.461024in}{1.857385in}}%
\pgfpathlineto{\pgfqpoint{3.377463in}{1.946667in}}%
\pgfpathlineto{\pgfqpoint{3.341696in}{1.984000in}}%
\pgfpathlineto{\pgfqpoint{3.268660in}{2.058667in}}%
\pgfpathlineto{\pgfqpoint{3.193496in}{2.133333in}}%
\pgfpathlineto{\pgfqpoint{3.159723in}{2.165968in}}%
\pgfpathlineto{\pgfqpoint{3.124687in}{2.199812in}}%
\pgfpathlineto{\pgfqpoint{3.116044in}{2.208000in}}%
\pgfpathlineto{\pgfqpoint{3.036125in}{2.282667in}}%
\pgfpathlineto{\pgfqpoint{2.953542in}{2.357333in}}%
\pgfpathlineto{\pgfqpoint{2.924283in}{2.383228in}}%
\pgfpathlineto{\pgfqpoint{2.824187in}{2.469333in}}%
\pgfpathlineto{\pgfqpoint{2.733931in}{2.544000in}}%
\pgfpathlineto{\pgfqpoint{2.683798in}{2.584286in}}%
\pgfpathlineto{\pgfqpoint{2.663302in}{2.599576in}}%
\pgfpathlineto{\pgfqpoint{2.639872in}{2.618667in}}%
\pgfpathlineto{\pgfqpoint{2.603636in}{2.646424in}}%
\pgfpathlineto{\pgfqpoint{2.523475in}{2.706172in}}%
\pgfpathlineto{\pgfqpoint{2.508296in}{2.716528in}}%
\pgfpathlineto{\pgfqpoint{2.483394in}{2.735219in}}%
\pgfpathlineto{\pgfqpoint{2.462900in}{2.748911in}}%
\pgfpathlineto{\pgfqpoint{2.436770in}{2.768000in}}%
\pgfpathlineto{\pgfqpoint{2.394032in}{2.796764in}}%
\pgfpathlineto{\pgfqpoint{2.363152in}{2.817909in}}%
\pgfpathlineto{\pgfqpoint{2.282990in}{2.869578in}}%
\pgfpathlineto{\pgfqpoint{2.200009in}{2.919960in}}%
\pgfpathlineto{\pgfqpoint{2.100658in}{2.975166in}}%
\pgfpathlineto{\pgfqpoint{2.042505in}{3.004625in}}%
\pgfpathlineto{\pgfqpoint{1.990101in}{3.029333in}}%
\pgfpathlineto{\pgfqpoint{1.962343in}{3.041665in}}%
\pgfpathlineto{\pgfqpoint{1.943217in}{3.048852in}}%
\pgfpathlineto{\pgfqpoint{1.922263in}{3.058395in}}%
\pgfpathlineto{\pgfqpoint{1.882182in}{3.073705in}}%
\pgfpathlineto{\pgfqpoint{1.857754in}{3.081247in}}%
\pgfpathlineto{\pgfqpoint{1.842101in}{3.087334in}}%
\pgfpathlineto{\pgfqpoint{1.802020in}{3.099543in}}%
\pgfpathlineto{\pgfqpoint{1.753972in}{3.111422in}}%
\pgfpathlineto{\pgfqpoint{1.721859in}{3.117207in}}%
\pgfpathlineto{\pgfqpoint{1.698287in}{3.119378in}}%
\pgfpathlineto{\pgfqpoint{1.681778in}{3.122246in}}%
\pgfpathlineto{\pgfqpoint{1.660382in}{3.123929in}}%
\pgfpathlineto{\pgfqpoint{1.641697in}{3.124018in}}%
\pgfpathlineto{\pgfqpoint{1.620909in}{3.121971in}}%
\pgfpathlineto{\pgfqpoint{1.601616in}{3.121595in}}%
\pgfpathlineto{\pgfqpoint{1.585080in}{3.119402in}}%
\pgfpathlineto{\pgfqpoint{1.553508in}{3.111477in}}%
\pgfpathlineto{\pgfqpoint{1.521455in}{3.097444in}}%
\pgfpathlineto{\pgfqpoint{1.480005in}{3.065392in}}%
\pgfpathlineto{\pgfqpoint{1.450268in}{3.020973in}}%
\pgfpathlineto{\pgfqpoint{1.439177in}{2.990029in}}%
\pgfpathlineto{\pgfqpoint{1.431790in}{2.954667in}}%
\pgfpathlineto{\pgfqpoint{1.428763in}{2.917333in}}%
\pgfpathlineto{\pgfqpoint{1.429580in}{2.906423in}}%
\pgfpathlineto{\pgfqpoint{1.429331in}{2.880000in}}%
\pgfpathlineto{\pgfqpoint{1.432757in}{2.842667in}}%
\pgfpathlineto{\pgfqpoint{1.441293in}{2.792158in}}%
\pgfpathlineto{\pgfqpoint{1.446522in}{2.768000in}}%
\pgfpathlineto{\pgfqpoint{1.456416in}{2.730667in}}%
\pgfpathlineto{\pgfqpoint{1.467643in}{2.693333in}}%
\pgfpathlineto{\pgfqpoint{1.481374in}{2.652284in}}%
\pgfpathlineto{\pgfqpoint{1.509108in}{2.581333in}}%
\pgfpathlineto{\pgfqpoint{1.512922in}{2.573385in}}%
\pgfpathlineto{\pgfqpoint{1.527353in}{2.538506in}}%
\pgfpathlineto{\pgfqpoint{1.561535in}{2.465941in}}%
\pgfpathlineto{\pgfqpoint{1.578832in}{2.432000in}}%
\pgfpathlineto{\pgfqpoint{1.601616in}{2.388309in}}%
\pgfpathlineto{\pgfqpoint{1.618655in}{2.357333in}}%
\pgfpathlineto{\pgfqpoint{1.641697in}{2.316099in}}%
\pgfpathlineto{\pgfqpoint{1.668983in}{2.270749in}}%
\pgfpathlineto{\pgfqpoint{1.685592in}{2.241781in}}%
\pgfpathlineto{\pgfqpoint{1.730147in}{2.170667in}}%
\pgfpathlineto{\pgfqpoint{1.742541in}{2.152598in}}%
\pgfpathlineto{\pgfqpoint{1.761939in}{2.121762in}}%
\pgfpathlineto{\pgfqpoint{1.830232in}{2.021333in}}%
\pgfpathlineto{\pgfqpoint{1.885593in}{1.943489in}}%
\pgfpathlineto{\pgfqpoint{1.978855in}{1.819287in}}%
\pgfpathlineto{\pgfqpoint{2.055273in}{1.722667in}}%
\pgfpathlineto{\pgfqpoint{2.122667in}{1.640376in}}%
\pgfpathlineto{\pgfqpoint{2.147700in}{1.610667in}}%
\pgfpathlineto{\pgfqpoint{2.211620in}{1.536000in}}%
\pgfpathlineto{\pgfqpoint{2.226029in}{1.520277in}}%
\pgfpathlineto{\pgfqpoint{2.250160in}{1.491912in}}%
\pgfpathlineto{\pgfqpoint{2.323071in}{1.410982in}}%
\pgfpathlineto{\pgfqpoint{2.345438in}{1.386667in}}%
\pgfpathlineto{\pgfqpoint{2.415217in}{1.312000in}}%
\pgfpathlineto{\pgfqpoint{2.486932in}{1.237333in}}%
\pgfpathlineto{\pgfqpoint{2.504804in}{1.219942in}}%
\pgfpathlineto{\pgfqpoint{2.523560in}{1.200000in}}%
\pgfpathlineto{\pgfqpoint{2.543275in}{1.181109in}}%
\pgfpathlineto{\pgfqpoint{2.563556in}{1.159999in}}%
\pgfpathlineto{\pgfqpoint{2.643717in}{1.081644in}}%
\pgfpathlineto{\pgfqpoint{2.723879in}{1.005667in}}%
\pgfpathlineto{\pgfqpoint{2.760034in}{0.972343in}}%
\pgfpathlineto{\pgfqpoint{2.763960in}{0.968555in}}%
\pgfpathlineto{\pgfqpoint{2.844121in}{0.896059in}}%
\pgfpathlineto{\pgfqpoint{2.924283in}{0.825830in}}%
\pgfpathlineto{\pgfqpoint{2.945189in}{0.808807in}}%
\pgfpathlineto{\pgfqpoint{2.967137in}{0.789333in}}%
\pgfpathlineto{\pgfqpoint{3.004444in}{0.758111in}}%
\pgfpathlineto{\pgfqpoint{3.103935in}{0.677333in}}%
\pgfpathlineto{\pgfqpoint{3.158816in}{0.634457in}}%
\pgfpathlineto{\pgfqpoint{3.164768in}{0.629570in}}%
\pgfpathlineto{\pgfqpoint{3.249642in}{0.565333in}}%
\pgfpathlineto{\pgfqpoint{3.285010in}{0.539507in}}%
\pgfpathlineto{\pgfqpoint{3.301037in}{0.528000in}}%
\pgfpathlineto{\pgfqpoint{3.301037in}{0.528000in}}%
\pgfusepath{fill}%
\end{pgfscope}%
\begin{pgfscope}%
\pgfpathrectangle{\pgfqpoint{0.800000in}{0.528000in}}{\pgfqpoint{3.968000in}{3.696000in}}%
\pgfusepath{clip}%
\pgfsetbuttcap%
\pgfsetroundjoin%
\definecolor{currentfill}{rgb}{0.277018,0.050344,0.375715}%
\pgfsetfillcolor{currentfill}%
\pgfsetlinewidth{0.000000pt}%
\definecolor{currentstroke}{rgb}{0.000000,0.000000,0.000000}%
\pgfsetstrokecolor{currentstroke}%
\pgfsetdash{}{0pt}%
\pgfpathmoveto{\pgfqpoint{3.301037in}{0.528000in}}%
\pgfpathlineto{\pgfqpoint{3.226422in}{0.582572in}}%
\pgfpathlineto{\pgfqpoint{3.151382in}{0.640000in}}%
\pgfpathlineto{\pgfqpoint{3.057438in}{0.714667in}}%
\pgfpathlineto{\pgfqpoint{3.004444in}{0.758111in}}%
\pgfpathlineto{\pgfqpoint{2.987202in}{0.773273in}}%
\pgfpathlineto{\pgfqpoint{2.964364in}{0.791661in}}%
\pgfpathlineto{\pgfqpoint{2.945189in}{0.808807in}}%
\pgfpathlineto{\pgfqpoint{2.923317in}{0.826667in}}%
\pgfpathlineto{\pgfqpoint{2.903522in}{0.844662in}}%
\pgfpathlineto{\pgfqpoint{2.880413in}{0.864000in}}%
\pgfpathlineto{\pgfqpoint{2.796731in}{0.938667in}}%
\pgfpathlineto{\pgfqpoint{2.715706in}{1.013333in}}%
\pgfpathlineto{\pgfqpoint{2.637147in}{1.088000in}}%
\pgfpathlineto{\pgfqpoint{2.560880in}{1.162667in}}%
\pgfpathlineto{\pgfqpoint{2.543275in}{1.181109in}}%
\pgfpathlineto{\pgfqpoint{2.523475in}{1.200086in}}%
\pgfpathlineto{\pgfqpoint{2.485289in}{1.239098in}}%
\pgfpathlineto{\pgfqpoint{2.483394in}{1.240952in}}%
\pgfpathlineto{\pgfqpoint{2.450824in}{1.274667in}}%
\pgfpathlineto{\pgfqpoint{2.380094in}{1.349333in}}%
\pgfpathlineto{\pgfqpoint{2.335391in}{1.398143in}}%
\pgfpathlineto{\pgfqpoint{2.311233in}{1.424000in}}%
\pgfpathlineto{\pgfqpoint{2.282990in}{1.455166in}}%
\pgfpathlineto{\pgfqpoint{2.202828in}{1.546121in}}%
\pgfpathlineto{\pgfqpoint{2.116315in}{1.648000in}}%
\pgfpathlineto{\pgfqpoint{2.042505in}{1.738546in}}%
\pgfpathlineto{\pgfqpoint{2.025453in}{1.760000in}}%
\pgfpathlineto{\pgfqpoint{1.962343in}{1.840834in}}%
\pgfpathlineto{\pgfqpoint{1.938810in}{1.872000in}}%
\pgfpathlineto{\pgfqpoint{1.882182in}{1.948197in}}%
\pgfpathlineto{\pgfqpoint{1.802020in}{2.061912in}}%
\pgfpathlineto{\pgfqpoint{1.754304in}{2.133333in}}%
\pgfpathlineto{\pgfqpoint{1.742541in}{2.152598in}}%
\pgfpathlineto{\pgfqpoint{1.721859in}{2.183729in}}%
\pgfpathlineto{\pgfqpoint{1.706648in}{2.208000in}}%
\pgfpathlineto{\pgfqpoint{1.681778in}{2.248127in}}%
\pgfpathlineto{\pgfqpoint{1.654783in}{2.294855in}}%
\pgfpathlineto{\pgfqpoint{1.636800in}{2.324561in}}%
\pgfpathlineto{\pgfqpoint{1.594679in}{2.401128in}}%
\pgfpathlineto{\pgfqpoint{1.559830in}{2.469333in}}%
\pgfpathlineto{\pgfqpoint{1.542196in}{2.506667in}}%
\pgfpathlineto{\pgfqpoint{1.521455in}{2.552178in}}%
\pgfpathlineto{\pgfqpoint{1.479260in}{2.657969in}}%
\pgfpathlineto{\pgfqpoint{1.456416in}{2.730667in}}%
\pgfpathlineto{\pgfqpoint{1.453963in}{2.742468in}}%
\pgfpathlineto{\pgfqpoint{1.445895in}{2.772286in}}%
\pgfpathlineto{\pgfqpoint{1.437820in}{2.808568in}}%
\pgfpathlineto{\pgfqpoint{1.431442in}{2.851842in}}%
\pgfpathlineto{\pgfqpoint{1.429331in}{2.880000in}}%
\pgfpathlineto{\pgfqpoint{1.428949in}{2.928831in}}%
\pgfpathlineto{\pgfqpoint{1.431790in}{2.954667in}}%
\pgfpathlineto{\pgfqpoint{1.441293in}{2.997537in}}%
\pgfpathlineto{\pgfqpoint{1.450268in}{3.020973in}}%
\pgfpathlineto{\pgfqpoint{1.454930in}{3.029333in}}%
\pgfpathlineto{\pgfqpoint{1.481374in}{3.067245in}}%
\pgfpathlineto{\pgfqpoint{1.525747in}{3.100002in}}%
\pgfpathlineto{\pgfqpoint{1.553508in}{3.111477in}}%
\pgfpathlineto{\pgfqpoint{1.585080in}{3.119402in}}%
\pgfpathlineto{\pgfqpoint{1.601616in}{3.121595in}}%
\pgfpathlineto{\pgfqpoint{1.620909in}{3.121971in}}%
\pgfpathlineto{\pgfqpoint{1.641697in}{3.124018in}}%
\pgfpathlineto{\pgfqpoint{1.660382in}{3.123929in}}%
\pgfpathlineto{\pgfqpoint{1.681778in}{3.122246in}}%
\pgfpathlineto{\pgfqpoint{1.698287in}{3.119378in}}%
\pgfpathlineto{\pgfqpoint{1.721859in}{3.117207in}}%
\pgfpathlineto{\pgfqpoint{1.766416in}{3.108170in}}%
\pgfpathlineto{\pgfqpoint{1.808748in}{3.097733in}}%
\pgfpathlineto{\pgfqpoint{1.842101in}{3.087334in}}%
\pgfpathlineto{\pgfqpoint{1.857754in}{3.081247in}}%
\pgfpathlineto{\pgfqpoint{1.882182in}{3.073705in}}%
\pgfpathlineto{\pgfqpoint{1.922263in}{3.058395in}}%
\pgfpathlineto{\pgfqpoint{1.943217in}{3.048852in}}%
\pgfpathlineto{\pgfqpoint{1.962343in}{3.041665in}}%
\pgfpathlineto{\pgfqpoint{2.014113in}{3.018446in}}%
\pgfpathlineto{\pgfqpoint{2.100658in}{2.975166in}}%
\pgfpathlineto{\pgfqpoint{2.162747in}{2.941122in}}%
\pgfpathlineto{\pgfqpoint{2.204455in}{2.917333in}}%
\pgfpathlineto{\pgfqpoint{2.282990in}{2.869578in}}%
\pgfpathlineto{\pgfqpoint{2.363152in}{2.817909in}}%
\pgfpathlineto{\pgfqpoint{2.394032in}{2.796764in}}%
\pgfpathlineto{\pgfqpoint{2.403232in}{2.790984in}}%
\pgfpathlineto{\pgfqpoint{2.417067in}{2.780886in}}%
\pgfpathlineto{\pgfqpoint{2.443313in}{2.763500in}}%
\pgfpathlineto{\pgfqpoint{2.462900in}{2.748911in}}%
\pgfpathlineto{\pgfqpoint{2.489692in}{2.730667in}}%
\pgfpathlineto{\pgfqpoint{2.523475in}{2.706172in}}%
\pgfpathlineto{\pgfqpoint{2.603636in}{2.646424in}}%
\pgfpathlineto{\pgfqpoint{2.619584in}{2.633521in}}%
\pgfpathlineto{\pgfqpoint{2.643717in}{2.615712in}}%
\pgfpathlineto{\pgfqpoint{2.663302in}{2.599576in}}%
\pgfpathlineto{\pgfqpoint{2.687493in}{2.581333in}}%
\pgfpathlineto{\pgfqpoint{2.723879in}{2.552173in}}%
\pgfpathlineto{\pgfqpoint{2.824187in}{2.469333in}}%
\pgfpathlineto{\pgfqpoint{2.876281in}{2.424622in}}%
\pgfpathlineto{\pgfqpoint{2.884202in}{2.418153in}}%
\pgfpathlineto{\pgfqpoint{2.964364in}{2.347725in}}%
\pgfpathlineto{\pgfqpoint{2.999777in}{2.315652in}}%
\pgfpathlineto{\pgfqpoint{3.004444in}{2.311637in}}%
\pgfpathlineto{\pgfqpoint{3.084606in}{2.237687in}}%
\pgfpathlineto{\pgfqpoint{3.164768in}{2.161330in}}%
\pgfpathlineto{\pgfqpoint{3.198901in}{2.127793in}}%
\pgfpathlineto{\pgfqpoint{3.231353in}{2.096000in}}%
\pgfpathlineto{\pgfqpoint{3.276316in}{2.050569in}}%
\pgfpathlineto{\pgfqpoint{3.305434in}{2.021333in}}%
\pgfpathlineto{\pgfqpoint{3.352510in}{1.972206in}}%
\pgfpathlineto{\pgfqpoint{3.377463in}{1.946667in}}%
\pgfpathlineto{\pgfqpoint{3.461024in}{1.857385in}}%
\pgfpathlineto{\pgfqpoint{3.548593in}{1.760000in}}%
\pgfpathlineto{\pgfqpoint{3.613744in}{1.685333in}}%
\pgfpathlineto{\pgfqpoint{3.627535in}{1.668379in}}%
\pgfpathlineto{\pgfqpoint{3.645737in}{1.647989in}}%
\pgfpathlineto{\pgfqpoint{3.676886in}{1.610667in}}%
\pgfpathlineto{\pgfqpoint{3.738098in}{1.536000in}}%
\pgfpathlineto{\pgfqpoint{3.749679in}{1.520817in}}%
\pgfpathlineto{\pgfqpoint{3.768176in}{1.498667in}}%
\pgfpathlineto{\pgfqpoint{3.800967in}{1.456589in}}%
\pgfpathlineto{\pgfqpoint{3.826340in}{1.424000in}}%
\pgfpathlineto{\pgfqpoint{3.851143in}{1.391325in}}%
\pgfpathlineto{\pgfqpoint{3.875613in}{1.359216in}}%
\pgfpathlineto{\pgfqpoint{3.937221in}{1.274667in}}%
\pgfpathlineto{\pgfqpoint{3.948429in}{1.257943in}}%
\pgfpathlineto{\pgfqpoint{3.966384in}{1.233626in}}%
\pgfpathlineto{\pgfqpoint{4.039797in}{1.125333in}}%
\pgfpathlineto{\pgfqpoint{4.089518in}{1.047973in}}%
\pgfpathlineto{\pgfqpoint{4.133178in}{0.976000in}}%
\pgfpathlineto{\pgfqpoint{4.158751in}{0.931181in}}%
\pgfpathlineto{\pgfqpoint{4.175996in}{0.901333in}}%
\pgfpathlineto{\pgfqpoint{4.185925in}{0.881825in}}%
\pgfpathlineto{\pgfqpoint{4.206869in}{0.844165in}}%
\pgfpathlineto{\pgfqpoint{4.257732in}{0.741957in}}%
\pgfpathlineto{\pgfqpoint{4.287030in}{0.675729in}}%
\pgfpathlineto{\pgfqpoint{4.315272in}{0.602667in}}%
\pgfpathlineto{\pgfqpoint{4.317783in}{0.593978in}}%
\pgfpathlineto{\pgfqpoint{4.328326in}{0.565333in}}%
\pgfpathlineto{\pgfqpoint{4.339481in}{0.528000in}}%
\pgfpathlineto{\pgfqpoint{4.357559in}{0.528000in}}%
\pgfpathlineto{\pgfqpoint{4.349903in}{0.549229in}}%
\pgfpathlineto{\pgfqpoint{4.345391in}{0.565333in}}%
\pgfpathlineto{\pgfqpoint{4.327111in}{0.615893in}}%
\pgfpathlineto{\pgfqpoint{4.301988in}{0.677333in}}%
\pgfpathlineto{\pgfqpoint{4.297296in}{0.686895in}}%
\pgfpathlineto{\pgfqpoint{4.284749in}{0.716792in}}%
\pgfpathlineto{\pgfqpoint{4.246949in}{0.795004in}}%
\pgfpathlineto{\pgfqpoint{4.222276in}{0.841018in}}%
\pgfpathlineto{\pgfqpoint{4.206869in}{0.871145in}}%
\pgfpathlineto{\pgfqpoint{4.190011in}{0.901333in}}%
\pgfpathlineto{\pgfqpoint{4.166788in}{0.942319in}}%
\pgfpathlineto{\pgfqpoint{4.146743in}{0.976000in}}%
\pgfpathlineto{\pgfqpoint{4.100965in}{1.050667in}}%
\pgfpathlineto{\pgfqpoint{4.046545in}{1.134895in}}%
\pgfpathlineto{\pgfqpoint{3.976456in}{1.237333in}}%
\pgfpathlineto{\pgfqpoint{3.956225in}{1.265204in}}%
\pgfpathlineto{\pgfqpoint{3.940089in}{1.287507in}}%
\pgfpathlineto{\pgfqpoint{3.923037in}{1.312000in}}%
\pgfpathlineto{\pgfqpoint{3.891513in}{1.354261in}}%
\pgfpathlineto{\pgfqpoint{3.867314in}{1.386667in}}%
\pgfpathlineto{\pgfqpoint{3.841825in}{1.419979in}}%
\pgfpathlineto{\pgfqpoint{3.820573in}{1.447815in}}%
\pgfpathlineto{\pgfqpoint{3.750452in}{1.536000in}}%
\pgfpathlineto{\pgfqpoint{3.685818in}{1.614886in}}%
\pgfpathlineto{\pgfqpoint{3.652369in}{1.654177in}}%
\pgfpathlineto{\pgfqpoint{3.626018in}{1.685333in}}%
\pgfpathlineto{\pgfqpoint{3.540256in}{1.783584in}}%
\pgfpathlineto{\pgfqpoint{3.459914in}{1.872000in}}%
\pgfpathlineto{\pgfqpoint{3.390084in}{1.946667in}}%
\pgfpathlineto{\pgfqpoint{3.354466in}{1.984000in}}%
\pgfpathlineto{\pgfqpoint{3.281737in}{2.058667in}}%
\pgfpathlineto{\pgfqpoint{3.204848in}{2.135241in}}%
\pgfpathlineto{\pgfqpoint{3.168425in}{2.170667in}}%
\pgfpathlineto{\pgfqpoint{3.084606in}{2.250411in}}%
\pgfpathlineto{\pgfqpoint{3.004444in}{2.324317in}}%
\pgfpathlineto{\pgfqpoint{2.986021in}{2.340173in}}%
\pgfpathlineto{\pgfqpoint{2.964364in}{2.360423in}}%
\pgfpathlineto{\pgfqpoint{2.945106in}{2.376729in}}%
\pgfpathlineto{\pgfqpoint{2.901920in}{2.415497in}}%
\pgfpathlineto{\pgfqpoint{2.839261in}{2.469333in}}%
\pgfpathlineto{\pgfqpoint{2.749436in}{2.544000in}}%
\pgfpathlineto{\pgfqpoint{2.656090in}{2.618667in}}%
\pgfpathlineto{\pgfqpoint{2.603636in}{2.659363in}}%
\pgfpathlineto{\pgfqpoint{2.583046in}{2.674155in}}%
\pgfpathlineto{\pgfqpoint{2.558559in}{2.693333in}}%
\pgfpathlineto{\pgfqpoint{2.483394in}{2.748197in}}%
\pgfpathlineto{\pgfqpoint{2.448458in}{2.772793in}}%
\pgfpathlineto{\pgfqpoint{2.443313in}{2.776656in}}%
\pgfpathlineto{\pgfqpoint{2.425226in}{2.788486in}}%
\pgfpathlineto{\pgfqpoint{2.402051in}{2.805333in}}%
\pgfpathlineto{\pgfqpoint{2.363152in}{2.831418in}}%
\pgfpathlineto{\pgfqpoint{2.282990in}{2.883525in}}%
\pgfpathlineto{\pgfqpoint{2.260694in}{2.896566in}}%
\pgfpathlineto{\pgfqpoint{2.227892in}{2.917333in}}%
\pgfpathlineto{\pgfqpoint{2.159535in}{2.957659in}}%
\pgfpathlineto{\pgfqpoint{2.082586in}{2.999588in}}%
\pgfpathlineto{\pgfqpoint{2.061438in}{3.009635in}}%
\pgfpathlineto{\pgfqpoint{2.023503in}{3.029333in}}%
\pgfpathlineto{\pgfqpoint{1.922263in}{3.075120in}}%
\pgfpathlineto{\pgfqpoint{1.882182in}{3.090908in}}%
\pgfpathlineto{\pgfqpoint{1.871625in}{3.094167in}}%
\pgfpathlineto{\pgfqpoint{1.842101in}{3.105541in}}%
\pgfpathlineto{\pgfqpoint{1.761939in}{3.129058in}}%
\pgfpathlineto{\pgfqpoint{1.718728in}{3.138417in}}%
\pgfpathlineto{\pgfqpoint{1.681778in}{3.144560in}}%
\pgfpathlineto{\pgfqpoint{1.634125in}{3.148387in}}%
\pgfpathlineto{\pgfqpoint{1.601616in}{3.147811in}}%
\pgfpathlineto{\pgfqpoint{1.554120in}{3.141333in}}%
\pgfpathlineto{\pgfqpoint{1.521455in}{3.131431in}}%
\pgfpathlineto{\pgfqpoint{1.501493in}{3.122593in}}%
\pgfpathlineto{\pgfqpoint{1.473681in}{3.104000in}}%
\pgfpathlineto{\pgfqpoint{1.439667in}{3.066667in}}%
\pgfpathlineto{\pgfqpoint{1.438607in}{3.064165in}}%
\pgfpathlineto{\pgfqpoint{1.421494in}{3.029333in}}%
\pgfpathlineto{\pgfqpoint{1.416277in}{3.015301in}}%
\pgfpathlineto{\pgfqpoint{1.410784in}{2.992000in}}%
\pgfpathlineto{\pgfqpoint{1.405202in}{2.950950in}}%
\pgfpathlineto{\pgfqpoint{1.404658in}{2.917333in}}%
\pgfpathlineto{\pgfqpoint{1.407430in}{2.874209in}}%
\pgfpathlineto{\pgfqpoint{1.413673in}{2.831060in}}%
\pgfpathlineto{\pgfqpoint{1.427815in}{2.768000in}}%
\pgfpathlineto{\pgfqpoint{1.441293in}{2.720504in}}%
\pgfpathlineto{\pgfqpoint{1.453981in}{2.681515in}}%
\pgfpathlineto{\pgfqpoint{1.481374in}{2.609399in}}%
\pgfpathlineto{\pgfqpoint{1.509579in}{2.544000in}}%
\pgfpathlineto{\pgfqpoint{1.513420in}{2.536516in}}%
\pgfpathlineto{\pgfqpoint{1.526815in}{2.506667in}}%
\pgfpathlineto{\pgfqpoint{1.545190in}{2.469333in}}%
\pgfpathlineto{\pgfqpoint{1.566242in}{2.427616in}}%
\pgfpathlineto{\pgfqpoint{1.607196in}{2.352136in}}%
\pgfpathlineto{\pgfqpoint{1.647485in}{2.282667in}}%
\pgfpathlineto{\pgfqpoint{1.670096in}{2.245333in}}%
\pgfpathlineto{\pgfqpoint{1.721859in}{2.162931in}}%
\pgfpathlineto{\pgfqpoint{1.741291in}{2.133333in}}%
\pgfpathlineto{\pgfqpoint{1.791484in}{2.058667in}}%
\pgfpathlineto{\pgfqpoint{1.802020in}{2.043418in}}%
\pgfpathlineto{\pgfqpoint{1.870832in}{1.946667in}}%
\pgfpathlineto{\pgfqpoint{1.926123in}{1.872000in}}%
\pgfpathlineto{\pgfqpoint{1.957958in}{1.830582in}}%
\pgfpathlineto{\pgfqpoint{1.983653in}{1.797333in}}%
\pgfpathlineto{\pgfqpoint{2.002424in}{1.773351in}}%
\pgfpathlineto{\pgfqpoint{2.073193in}{1.685333in}}%
\pgfpathlineto{\pgfqpoint{2.112431in}{1.638466in}}%
\pgfpathlineto{\pgfqpoint{2.135322in}{1.610667in}}%
\pgfpathlineto{\pgfqpoint{2.202828in}{1.531771in}}%
\pgfpathlineto{\pgfqpoint{2.237027in}{1.493188in}}%
\pgfpathlineto{\pgfqpoint{2.265166in}{1.461333in}}%
\pgfpathlineto{\pgfqpoint{2.332855in}{1.386667in}}%
\pgfpathlineto{\pgfqpoint{2.403232in}{1.311100in}}%
\pgfpathlineto{\pgfqpoint{2.483394in}{1.227936in}}%
\pgfpathlineto{\pgfqpoint{2.510909in}{1.200000in}}%
\pgfpathlineto{\pgfqpoint{2.585790in}{1.125333in}}%
\pgfpathlineto{\pgfqpoint{2.633909in}{1.078864in}}%
\pgfpathlineto{\pgfqpoint{2.662866in}{1.050667in}}%
\pgfpathlineto{\pgfqpoint{2.713275in}{1.003456in}}%
\pgfpathlineto{\pgfqpoint{2.742302in}{0.976000in}}%
\pgfpathlineto{\pgfqpoint{2.793856in}{0.929181in}}%
\pgfpathlineto{\pgfqpoint{2.824281in}{0.901333in}}%
\pgfpathlineto{\pgfqpoint{2.875688in}{0.856069in}}%
\pgfpathlineto{\pgfqpoint{2.909004in}{0.826667in}}%
\pgfpathlineto{\pgfqpoint{2.964364in}{0.779189in}}%
\pgfpathlineto{\pgfqpoint{3.000855in}{0.748656in}}%
\pgfpathlineto{\pgfqpoint{3.004444in}{0.745501in}}%
\pgfpathlineto{\pgfqpoint{3.087799in}{0.677333in}}%
\pgfpathlineto{\pgfqpoint{3.124687in}{0.648064in}}%
\pgfpathlineto{\pgfqpoint{3.204848in}{0.586101in}}%
\pgfpathlineto{\pgfqpoint{3.282725in}{0.528000in}}%
\pgfpathlineto{\pgfqpoint{3.285010in}{0.528000in}}%
\pgfpathlineto{\pgfqpoint{3.285010in}{0.528000in}}%
\pgfusepath{fill}%
\end{pgfscope}%
\begin{pgfscope}%
\pgfpathrectangle{\pgfqpoint{0.800000in}{0.528000in}}{\pgfqpoint{3.968000in}{3.696000in}}%
\pgfusepath{clip}%
\pgfsetbuttcap%
\pgfsetroundjoin%
\definecolor{currentfill}{rgb}{0.277018,0.050344,0.375715}%
\pgfsetfillcolor{currentfill}%
\pgfsetlinewidth{0.000000pt}%
\definecolor{currentstroke}{rgb}{0.000000,0.000000,0.000000}%
\pgfsetstrokecolor{currentstroke}%
\pgfsetdash{}{0pt}%
\pgfpathmoveto{\pgfqpoint{3.282725in}{0.528000in}}%
\pgfpathlineto{\pgfqpoint{3.244929in}{0.555939in}}%
\pgfpathlineto{\pgfqpoint{3.164768in}{0.616808in}}%
\pgfpathlineto{\pgfqpoint{3.135003in}{0.640000in}}%
\pgfpathlineto{\pgfqpoint{3.084606in}{0.679875in}}%
\pgfpathlineto{\pgfqpoint{3.065077in}{0.696476in}}%
\pgfpathlineto{\pgfqpoint{3.041726in}{0.714667in}}%
\pgfpathlineto{\pgfqpoint{2.952463in}{0.789333in}}%
\pgfpathlineto{\pgfqpoint{2.924283in}{0.813428in}}%
\pgfpathlineto{\pgfqpoint{2.824281in}{0.901333in}}%
\pgfpathlineto{\pgfqpoint{2.742302in}{0.976000in}}%
\pgfpathlineto{\pgfqpoint{2.702278in}{1.013333in}}%
\pgfpathlineto{\pgfqpoint{2.624043in}{1.088000in}}%
\pgfpathlineto{\pgfqpoint{2.548085in}{1.162667in}}%
\pgfpathlineto{\pgfqpoint{2.523475in}{1.187283in}}%
\pgfpathlineto{\pgfqpoint{2.438074in}{1.274667in}}%
\pgfpathlineto{\pgfqpoint{2.363152in}{1.353843in}}%
\pgfpathlineto{\pgfqpoint{2.265166in}{1.461333in}}%
\pgfpathlineto{\pgfqpoint{2.183698in}{1.553818in}}%
\pgfpathlineto{\pgfqpoint{2.104076in}{1.648000in}}%
\pgfpathlineto{\pgfqpoint{2.042505in}{1.722870in}}%
\pgfpathlineto{\pgfqpoint{2.012994in}{1.760000in}}%
\pgfpathlineto{\pgfqpoint{1.954634in}{1.834667in}}%
\pgfpathlineto{\pgfqpoint{1.941587in}{1.852666in}}%
\pgfpathlineto{\pgfqpoint{1.922263in}{1.877134in}}%
\pgfpathlineto{\pgfqpoint{1.898331in}{1.909333in}}%
\pgfpathlineto{\pgfqpoint{1.842101in}{1.986247in}}%
\pgfpathlineto{\pgfqpoint{1.761939in}{2.102040in}}%
\pgfpathlineto{\pgfqpoint{1.716840in}{2.170667in}}%
\pgfpathlineto{\pgfqpoint{1.693222in}{2.208000in}}%
\pgfpathlineto{\pgfqpoint{1.641697in}{2.292512in}}%
\pgfpathlineto{\pgfqpoint{1.601616in}{2.362245in}}%
\pgfpathlineto{\pgfqpoint{1.561535in}{2.436820in}}%
\pgfpathlineto{\pgfqpoint{1.521455in}{2.518166in}}%
\pgfpathlineto{\pgfqpoint{1.477543in}{2.618667in}}%
\pgfpathlineto{\pgfqpoint{1.463399in}{2.656000in}}%
\pgfpathlineto{\pgfqpoint{1.450099in}{2.693333in}}%
\pgfpathlineto{\pgfqpoint{1.436708in}{2.734937in}}%
\pgfpathlineto{\pgfqpoint{1.423412in}{2.784655in}}%
\pgfpathlineto{\pgfqpoint{1.411949in}{2.842667in}}%
\pgfpathlineto{\pgfqpoint{1.406995in}{2.880000in}}%
\pgfpathlineto{\pgfqpoint{1.407047in}{2.885435in}}%
\pgfpathlineto{\pgfqpoint{1.404658in}{2.917333in}}%
\pgfpathlineto{\pgfqpoint{1.405611in}{2.954667in}}%
\pgfpathlineto{\pgfqpoint{1.406673in}{2.959753in}}%
\pgfpathlineto{\pgfqpoint{1.410784in}{2.992000in}}%
\pgfpathlineto{\pgfqpoint{1.416277in}{3.015301in}}%
\pgfpathlineto{\pgfqpoint{1.421494in}{3.029333in}}%
\pgfpathlineto{\pgfqpoint{1.441293in}{3.069052in}}%
\pgfpathlineto{\pgfqpoint{1.473681in}{3.104000in}}%
\pgfpathlineto{\pgfqpoint{1.481374in}{3.110022in}}%
\pgfpathlineto{\pgfqpoint{1.501493in}{3.122593in}}%
\pgfpathlineto{\pgfqpoint{1.528825in}{3.134468in}}%
\pgfpathlineto{\pgfqpoint{1.563574in}{3.143232in}}%
\pgfpathlineto{\pgfqpoint{1.601616in}{3.147811in}}%
\pgfpathlineto{\pgfqpoint{1.641697in}{3.147951in}}%
\pgfpathlineto{\pgfqpoint{1.647805in}{3.147022in}}%
\pgfpathlineto{\pgfqpoint{1.681778in}{3.144560in}}%
\pgfpathlineto{\pgfqpoint{1.726124in}{3.137360in}}%
\pgfpathlineto{\pgfqpoint{1.761939in}{3.129058in}}%
\pgfpathlineto{\pgfqpoint{1.782063in}{3.122744in}}%
\pgfpathlineto{\pgfqpoint{1.802020in}{3.118080in}}%
\pgfpathlineto{\pgfqpoint{1.812837in}{3.114075in}}%
\pgfpathlineto{\pgfqpoint{1.846338in}{3.104000in}}%
\pgfpathlineto{\pgfqpoint{1.962343in}{3.057949in}}%
\pgfpathlineto{\pgfqpoint{2.002424in}{3.039555in}}%
\pgfpathlineto{\pgfqpoint{2.023503in}{3.029333in}}%
\pgfpathlineto{\pgfqpoint{2.082586in}{2.999588in}}%
\pgfpathlineto{\pgfqpoint{2.096722in}{2.992000in}}%
\pgfpathlineto{\pgfqpoint{2.164731in}{2.954667in}}%
\pgfpathlineto{\pgfqpoint{2.242909in}{2.908308in}}%
\pgfpathlineto{\pgfqpoint{2.260694in}{2.896566in}}%
\pgfpathlineto{\pgfqpoint{2.288486in}{2.880000in}}%
\pgfpathlineto{\pgfqpoint{2.363152in}{2.831418in}}%
\pgfpathlineto{\pgfqpoint{2.378995in}{2.820090in}}%
\pgfpathlineto{\pgfqpoint{2.403232in}{2.804538in}}%
\pgfpathlineto{\pgfqpoint{2.448458in}{2.772793in}}%
\pgfpathlineto{\pgfqpoint{2.483394in}{2.748197in}}%
\pgfpathlineto{\pgfqpoint{2.563556in}{2.689639in}}%
\pgfpathlineto{\pgfqpoint{2.583046in}{2.674155in}}%
\pgfpathlineto{\pgfqpoint{2.607999in}{2.656000in}}%
\pgfpathlineto{\pgfqpoint{2.643717in}{2.628378in}}%
\pgfpathlineto{\pgfqpoint{2.749436in}{2.544000in}}%
\pgfpathlineto{\pgfqpoint{2.839261in}{2.469333in}}%
\pgfpathlineto{\pgfqpoint{2.884202in}{2.430919in}}%
\pgfpathlineto{\pgfqpoint{2.967803in}{2.357333in}}%
\pgfpathlineto{\pgfqpoint{3.004444in}{2.324317in}}%
\pgfpathlineto{\pgfqpoint{3.090003in}{2.245333in}}%
\pgfpathlineto{\pgfqpoint{3.168425in}{2.170667in}}%
\pgfpathlineto{\pgfqpoint{3.185833in}{2.152955in}}%
\pgfpathlineto{\pgfqpoint{3.206787in}{2.133333in}}%
\pgfpathlineto{\pgfqpoint{3.285010in}{2.055366in}}%
\pgfpathlineto{\pgfqpoint{3.365172in}{1.972870in}}%
\pgfpathlineto{\pgfqpoint{3.390084in}{1.946667in}}%
\pgfpathlineto{\pgfqpoint{3.459914in}{1.872000in}}%
\pgfpathlineto{\pgfqpoint{3.540256in}{1.783584in}}%
\pgfpathlineto{\pgfqpoint{3.626018in}{1.685333in}}%
\pgfpathlineto{\pgfqpoint{3.652369in}{1.654177in}}%
\pgfpathlineto{\pgfqpoint{3.669812in}{1.633091in}}%
\pgfpathlineto{\pgfqpoint{3.689341in}{1.610667in}}%
\pgfpathlineto{\pgfqpoint{3.720168in}{1.573333in}}%
\pgfpathlineto{\pgfqpoint{3.780391in}{1.498667in}}%
\pgfpathlineto{\pgfqpoint{3.790958in}{1.484599in}}%
\pgfpathlineto{\pgfqpoint{3.820573in}{1.447815in}}%
\pgfpathlineto{\pgfqpoint{3.895406in}{1.349333in}}%
\pgfpathlineto{\pgfqpoint{3.907603in}{1.331915in}}%
\pgfpathlineto{\pgfqpoint{3.926303in}{1.307504in}}%
\pgfpathlineto{\pgfqpoint{3.956225in}{1.265204in}}%
\pgfpathlineto{\pgfqpoint{3.976456in}{1.237333in}}%
\pgfpathlineto{\pgfqpoint{4.027854in}{1.162667in}}%
\pgfpathlineto{\pgfqpoint{4.080572in}{1.082361in}}%
\pgfpathlineto{\pgfqpoint{4.100965in}{1.050667in}}%
\pgfpathlineto{\pgfqpoint{4.146743in}{0.976000in}}%
\pgfpathlineto{\pgfqpoint{4.153511in}{0.963633in}}%
\pgfpathlineto{\pgfqpoint{4.171499in}{0.934278in}}%
\pgfpathlineto{\pgfqpoint{4.210807in}{0.864000in}}%
\pgfpathlineto{\pgfqpoint{4.235502in}{0.816004in}}%
\pgfpathlineto{\pgfqpoint{4.249855in}{0.789333in}}%
\pgfpathlineto{\pgfqpoint{4.267951in}{0.752000in}}%
\pgfpathlineto{\pgfqpoint{4.287030in}{0.711651in}}%
\pgfpathlineto{\pgfqpoint{4.335082in}{0.595242in}}%
\pgfpathlineto{\pgfqpoint{4.357559in}{0.528000in}}%
\pgfpathlineto{\pgfqpoint{4.375040in}{0.528000in}}%
\pgfpathlineto{\pgfqpoint{4.360215in}{0.571832in}}%
\pgfpathlineto{\pgfqpoint{4.327111in}{0.655203in}}%
\pgfpathlineto{\pgfqpoint{4.300586in}{0.714667in}}%
\pgfpathlineto{\pgfqpoint{4.296146in}{0.723158in}}%
\pgfpathlineto{\pgfqpoint{4.282961in}{0.752000in}}%
\pgfpathlineto{\pgfqpoint{4.264159in}{0.789333in}}%
\pgfpathlineto{\pgfqpoint{4.245008in}{0.826667in}}%
\pgfpathlineto{\pgfqpoint{4.224644in}{0.864000in}}%
\pgfpathlineto{\pgfqpoint{4.182332in}{0.938667in}}%
\pgfpathlineto{\pgfqpoint{4.137482in}{1.013333in}}%
\pgfpathlineto{\pgfqpoint{4.126707in}{1.030656in}}%
\pgfpathlineto{\pgfqpoint{4.086626in}{1.093648in}}%
\pgfpathlineto{\pgfqpoint{4.058046in}{1.136045in}}%
\pgfpathlineto{\pgfqpoint{4.030892in}{1.177247in}}%
\pgfpathlineto{\pgfqpoint{3.962545in}{1.274667in}}%
\pgfpathlineto{\pgfqpoint{3.935360in}{1.312000in}}%
\pgfpathlineto{\pgfqpoint{3.879749in}{1.386667in}}%
\pgfpathlineto{\pgfqpoint{3.865376in}{1.404583in}}%
\pgfpathlineto{\pgfqpoint{3.846141in}{1.430574in}}%
\pgfpathlineto{\pgfqpoint{3.753082in}{1.548014in}}%
\pgfpathlineto{\pgfqpoint{3.670003in}{1.648000in}}%
\pgfpathlineto{\pgfqpoint{3.641502in}{1.681388in}}%
\pgfpathlineto{\pgfqpoint{3.638293in}{1.685333in}}%
\pgfpathlineto{\pgfqpoint{3.565576in}{1.768743in}}%
\pgfpathlineto{\pgfqpoint{3.472249in}{1.872000in}}%
\pgfpathlineto{\pgfqpoint{3.402706in}{1.946667in}}%
\pgfpathlineto{\pgfqpoint{3.384649in}{1.964809in}}%
\pgfpathlineto{\pgfqpoint{3.365172in}{1.986043in}}%
\pgfpathlineto{\pgfqpoint{3.257146in}{2.096000in}}%
\pgfpathlineto{\pgfqpoint{3.212203in}{2.140184in}}%
\pgfpathlineto{\pgfqpoint{3.181260in}{2.170667in}}%
\pgfpathlineto{\pgfqpoint{3.103149in}{2.245333in}}%
\pgfpathlineto{\pgfqpoint{3.022647in}{2.320000in}}%
\pgfpathlineto{\pgfqpoint{2.939573in}{2.394667in}}%
\pgfpathlineto{\pgfqpoint{2.853725in}{2.469333in}}%
\pgfpathlineto{\pgfqpoint{2.804040in}{2.511430in}}%
\pgfpathlineto{\pgfqpoint{2.718951in}{2.581333in}}%
\pgfpathlineto{\pgfqpoint{2.678284in}{2.613531in}}%
\pgfpathlineto{\pgfqpoint{2.643717in}{2.640904in}}%
\pgfpathlineto{\pgfqpoint{2.612728in}{2.664469in}}%
\pgfpathlineto{\pgfqpoint{2.590615in}{2.681205in}}%
\pgfpathlineto{\pgfqpoint{2.563556in}{2.702261in}}%
\pgfpathlineto{\pgfqpoint{2.473791in}{2.768000in}}%
\pgfpathlineto{\pgfqpoint{2.433385in}{2.796086in}}%
\pgfpathlineto{\pgfqpoint{2.403232in}{2.817472in}}%
\pgfpathlineto{\pgfqpoint{2.387200in}{2.827733in}}%
\pgfpathlineto{\pgfqpoint{2.363152in}{2.844817in}}%
\pgfpathlineto{\pgfqpoint{2.323071in}{2.871208in}}%
\pgfpathlineto{\pgfqpoint{2.227748in}{2.931455in}}%
\pgfpathlineto{\pgfqpoint{2.162747in}{2.969811in}}%
\pgfpathlineto{\pgfqpoint{2.122667in}{2.992696in}}%
\pgfpathlineto{\pgfqpoint{2.071860in}{3.019343in}}%
\pgfpathlineto{\pgfqpoint{2.042505in}{3.035199in}}%
\pgfpathlineto{\pgfqpoint{2.019889in}{3.045601in}}%
\pgfpathlineto{\pgfqpoint{2.002424in}{3.054954in}}%
\pgfpathlineto{\pgfqpoint{1.962343in}{3.073794in}}%
\pgfpathlineto{\pgfqpoint{1.939731in}{3.082938in}}%
\pgfpathlineto{\pgfqpoint{1.922263in}{3.091337in}}%
\pgfpathlineto{\pgfqpoint{1.874974in}{3.110713in}}%
\pgfpathlineto{\pgfqpoint{1.802020in}{3.136311in}}%
\pgfpathlineto{\pgfqpoint{1.784913in}{3.141333in}}%
\pgfpathlineto{\pgfqpoint{1.761939in}{3.148048in}}%
\pgfpathlineto{\pgfqpoint{1.681778in}{3.165374in}}%
\pgfpathlineto{\pgfqpoint{1.668537in}{3.166333in}}%
\pgfpathlineto{\pgfqpoint{1.641697in}{3.170414in}}%
\pgfpathlineto{\pgfqpoint{1.601616in}{3.172208in}}%
\pgfpathlineto{\pgfqpoint{1.593653in}{3.171250in}}%
\pgfpathlineto{\pgfqpoint{1.561535in}{3.169838in}}%
\pgfpathlineto{\pgfqpoint{1.535330in}{3.165743in}}%
\pgfpathlineto{\pgfqpoint{1.521455in}{3.162005in}}%
\pgfpathlineto{\pgfqpoint{1.471897in}{3.141333in}}%
\pgfpathlineto{\pgfqpoint{1.433785in}{3.110994in}}%
\pgfpathlineto{\pgfqpoint{1.416990in}{3.089304in}}%
\pgfpathlineto{\pgfqpoint{1.401212in}{3.059195in}}%
\pgfpathlineto{\pgfqpoint{1.391205in}{3.029333in}}%
\pgfpathlineto{\pgfqpoint{1.389812in}{3.018714in}}%
\pgfpathlineto{\pgfqpoint{1.384192in}{2.992000in}}%
\pgfpathlineto{\pgfqpoint{1.381982in}{2.972579in}}%
\pgfpathlineto{\pgfqpoint{1.381592in}{2.954667in}}%
\pgfpathlineto{\pgfqpoint{1.382751in}{2.937471in}}%
\pgfpathlineto{\pgfqpoint{1.382455in}{2.917333in}}%
\pgfpathlineto{\pgfqpoint{1.386082in}{2.880000in}}%
\pgfpathlineto{\pgfqpoint{1.388583in}{2.868237in}}%
\pgfpathlineto{\pgfqpoint{1.393822in}{2.835783in}}%
\pgfpathlineto{\pgfqpoint{1.401212in}{2.799391in}}%
\pgfpathlineto{\pgfqpoint{1.412285in}{2.757686in}}%
\pgfpathlineto{\pgfqpoint{1.433155in}{2.693333in}}%
\pgfpathlineto{\pgfqpoint{1.435448in}{2.687889in}}%
\pgfpathlineto{\pgfqpoint{1.446811in}{2.656000in}}%
\pgfpathlineto{\pgfqpoint{1.461813in}{2.618667in}}%
\pgfpathlineto{\pgfqpoint{1.481374in}{2.572604in}}%
\pgfpathlineto{\pgfqpoint{1.549879in}{2.432000in}}%
\pgfpathlineto{\pgfqpoint{1.561535in}{2.410108in}}%
\pgfpathlineto{\pgfqpoint{1.601616in}{2.338078in}}%
\pgfpathlineto{\pgfqpoint{1.641697in}{2.270050in}}%
\pgfpathlineto{\pgfqpoint{1.656834in}{2.245333in}}%
\pgfpathlineto{\pgfqpoint{1.681778in}{2.205060in}}%
\pgfpathlineto{\pgfqpoint{1.703981in}{2.170667in}}%
\pgfpathlineto{\pgfqpoint{1.761939in}{2.083267in}}%
\pgfpathlineto{\pgfqpoint{1.831357in}{1.984000in}}%
\pgfpathlineto{\pgfqpoint{1.842101in}{1.969085in}}%
\pgfpathlineto{\pgfqpoint{1.913883in}{1.872000in}}%
\pgfpathlineto{\pgfqpoint{1.971335in}{1.797333in}}%
\pgfpathlineto{\pgfqpoint{1.984855in}{1.780969in}}%
\pgfpathlineto{\pgfqpoint{2.002424in}{1.757748in}}%
\pgfpathlineto{\pgfqpoint{2.035915in}{1.716529in}}%
\pgfpathlineto{\pgfqpoint{2.061091in}{1.685333in}}%
\pgfpathlineto{\pgfqpoint{2.082586in}{1.659146in}}%
\pgfpathlineto{\pgfqpoint{2.162747in}{1.564111in}}%
\pgfpathlineto{\pgfqpoint{2.252867in}{1.461333in}}%
\pgfpathlineto{\pgfqpoint{2.323071in}{1.383773in}}%
\pgfpathlineto{\pgfqpoint{2.425716in}{1.274667in}}%
\pgfpathlineto{\pgfqpoint{2.498263in}{1.200000in}}%
\pgfpathlineto{\pgfqpoint{2.572842in}{1.125333in}}%
\pgfpathlineto{\pgfqpoint{2.627376in}{1.072779in}}%
\pgfpathlineto{\pgfqpoint{2.649601in}{1.050667in}}%
\pgfpathlineto{\pgfqpoint{2.683798in}{1.018087in}}%
\pgfpathlineto{\pgfqpoint{2.769193in}{0.938667in}}%
\pgfpathlineto{\pgfqpoint{2.804040in}{0.906984in}}%
\pgfpathlineto{\pgfqpoint{2.894691in}{0.826667in}}%
\pgfpathlineto{\pgfqpoint{2.981992in}{0.752000in}}%
\pgfpathlineto{\pgfqpoint{3.036400in}{0.707099in}}%
\pgfpathlineto{\pgfqpoint{3.072486in}{0.677333in}}%
\pgfpathlineto{\pgfqpoint{3.144253in}{0.620891in}}%
\pgfpathlineto{\pgfqpoint{3.166564in}{0.602667in}}%
\pgfpathlineto{\pgfqpoint{3.204848in}{0.573380in}}%
\pgfpathlineto{\pgfqpoint{3.265570in}{0.528000in}}%
\pgfpathlineto{\pgfqpoint{3.265570in}{0.528000in}}%
\pgfusepath{fill}%
\end{pgfscope}%
\begin{pgfscope}%
\pgfpathrectangle{\pgfqpoint{0.800000in}{0.528000in}}{\pgfqpoint{3.968000in}{3.696000in}}%
\pgfusepath{clip}%
\pgfsetbuttcap%
\pgfsetroundjoin%
\definecolor{currentfill}{rgb}{0.277018,0.050344,0.375715}%
\pgfsetfillcolor{currentfill}%
\pgfsetlinewidth{0.000000pt}%
\definecolor{currentstroke}{rgb}{0.000000,0.000000,0.000000}%
\pgfsetstrokecolor{currentstroke}%
\pgfsetdash{}{0pt}%
\pgfpathmoveto{\pgfqpoint{3.265570in}{0.528000in}}%
\pgfpathlineto{\pgfqpoint{3.215530in}{0.565333in}}%
\pgfpathlineto{\pgfqpoint{3.164768in}{0.604045in}}%
\pgfpathlineto{\pgfqpoint{3.144253in}{0.620891in}}%
\pgfpathlineto{\pgfqpoint{3.119015in}{0.640000in}}%
\pgfpathlineto{\pgfqpoint{3.079094in}{0.672199in}}%
\pgfpathlineto{\pgfqpoint{3.044525in}{0.700072in}}%
\pgfpathlineto{\pgfqpoint{3.026824in}{0.714667in}}%
\pgfpathlineto{\pgfqpoint{2.937959in}{0.789333in}}%
\pgfpathlineto{\pgfqpoint{2.889383in}{0.831492in}}%
\pgfpathlineto{\pgfqpoint{2.868974in}{0.849816in}}%
\pgfpathlineto{\pgfqpoint{2.844121in}{0.871101in}}%
\pgfpathlineto{\pgfqpoint{2.810336in}{0.901333in}}%
\pgfpathlineto{\pgfqpoint{2.763960in}{0.943440in}}%
\pgfpathlineto{\pgfqpoint{2.726341in}{0.978293in}}%
\pgfpathlineto{\pgfqpoint{2.723879in}{0.980472in}}%
\pgfpathlineto{\pgfqpoint{2.688850in}{1.013333in}}%
\pgfpathlineto{\pgfqpoint{2.643717in}{1.056290in}}%
\pgfpathlineto{\pgfqpoint{2.610939in}{1.088000in}}%
\pgfpathlineto{\pgfqpoint{2.535289in}{1.162667in}}%
\pgfpathlineto{\pgfqpoint{2.491320in}{1.207383in}}%
\pgfpathlineto{\pgfqpoint{2.461744in}{1.237333in}}%
\pgfpathlineto{\pgfqpoint{2.425716in}{1.274667in}}%
\pgfpathlineto{\pgfqpoint{2.355065in}{1.349333in}}%
\pgfpathlineto{\pgfqpoint{2.282990in}{1.427713in}}%
\pgfpathlineto{\pgfqpoint{2.187123in}{1.536000in}}%
\pgfpathlineto{\pgfqpoint{2.122667in}{1.610995in}}%
\pgfpathlineto{\pgfqpoint{2.091837in}{1.648000in}}%
\pgfpathlineto{\pgfqpoint{2.030692in}{1.722667in}}%
\pgfpathlineto{\pgfqpoint{2.018943in}{1.738053in}}%
\pgfpathlineto{\pgfqpoint{2.000629in}{1.760000in}}%
\pgfpathlineto{\pgfqpoint{1.971335in}{1.797333in}}%
\pgfpathlineto{\pgfqpoint{1.913883in}{1.872000in}}%
\pgfpathlineto{\pgfqpoint{1.901223in}{1.889736in}}%
\pgfpathlineto{\pgfqpoint{1.882182in}{1.914208in}}%
\pgfpathlineto{\pgfqpoint{1.802020in}{2.025148in}}%
\pgfpathlineto{\pgfqpoint{1.778860in}{2.058667in}}%
\pgfpathlineto{\pgfqpoint{1.721859in}{2.143112in}}%
\pgfpathlineto{\pgfqpoint{1.656834in}{2.245333in}}%
\pgfpathlineto{\pgfqpoint{1.636946in}{2.278241in}}%
\pgfpathlineto{\pgfqpoint{1.624330in}{2.298843in}}%
\pgfpathlineto{\pgfqpoint{1.569864in}{2.394667in}}%
\pgfpathlineto{\pgfqpoint{1.530549in}{2.469333in}}%
\pgfpathlineto{\pgfqpoint{1.521455in}{2.487556in}}%
\pgfpathlineto{\pgfqpoint{1.494414in}{2.544000in}}%
\pgfpathlineto{\pgfqpoint{1.490859in}{2.552835in}}%
\pgfpathlineto{\pgfqpoint{1.474662in}{2.587585in}}%
\pgfpathlineto{\pgfqpoint{1.441293in}{2.670950in}}%
\pgfpathlineto{\pgfqpoint{1.412285in}{2.757686in}}%
\pgfpathlineto{\pgfqpoint{1.399667in}{2.805333in}}%
\pgfpathlineto{\pgfqpoint{1.386082in}{2.880000in}}%
\pgfpathlineto{\pgfqpoint{1.383897in}{2.896128in}}%
\pgfpathlineto{\pgfqpoint{1.382455in}{2.917333in}}%
\pgfpathlineto{\pgfqpoint{1.382751in}{2.937471in}}%
\pgfpathlineto{\pgfqpoint{1.381592in}{2.954667in}}%
\pgfpathlineto{\pgfqpoint{1.381982in}{2.972579in}}%
\pgfpathlineto{\pgfqpoint{1.384192in}{2.992000in}}%
\pgfpathlineto{\pgfqpoint{1.393012in}{3.036971in}}%
\pgfpathlineto{\pgfqpoint{1.404379in}{3.066667in}}%
\pgfpathlineto{\pgfqpoint{1.416990in}{3.089304in}}%
\pgfpathlineto{\pgfqpoint{1.433785in}{3.110994in}}%
\pgfpathlineto{\pgfqpoint{1.441293in}{3.118069in}}%
\pgfpathlineto{\pgfqpoint{1.477467in}{3.144972in}}%
\pgfpathlineto{\pgfqpoint{1.481374in}{3.146806in}}%
\pgfpathlineto{\pgfqpoint{1.490487in}{3.149822in}}%
\pgfpathlineto{\pgfqpoint{1.521455in}{3.162005in}}%
\pgfpathlineto{\pgfqpoint{1.535330in}{3.165743in}}%
\pgfpathlineto{\pgfqpoint{1.561535in}{3.169838in}}%
\pgfpathlineto{\pgfqpoint{1.608421in}{3.172328in}}%
\pgfpathlineto{\pgfqpoint{1.641697in}{3.170414in}}%
\pgfpathlineto{\pgfqpoint{1.721859in}{3.157753in}}%
\pgfpathlineto{\pgfqpoint{1.735253in}{3.153809in}}%
\pgfpathlineto{\pgfqpoint{1.767171in}{3.146207in}}%
\pgfpathlineto{\pgfqpoint{1.802020in}{3.136311in}}%
\pgfpathlineto{\pgfqpoint{1.826841in}{3.127119in}}%
\pgfpathlineto{\pgfqpoint{1.842101in}{3.122669in}}%
\pgfpathlineto{\pgfqpoint{1.891572in}{3.104000in}}%
\pgfpathlineto{\pgfqpoint{1.962343in}{3.073794in}}%
\pgfpathlineto{\pgfqpoint{2.042505in}{3.035199in}}%
\pgfpathlineto{\pgfqpoint{2.123890in}{2.992000in}}%
\pgfpathlineto{\pgfqpoint{2.227748in}{2.931455in}}%
\pgfpathlineto{\pgfqpoint{2.309408in}{2.880000in}}%
\pgfpathlineto{\pgfqpoint{2.366313in}{2.842667in}}%
\pgfpathlineto{\pgfqpoint{2.403232in}{2.817472in}}%
\pgfpathlineto{\pgfqpoint{2.483394in}{2.761175in}}%
\pgfpathlineto{\pgfqpoint{2.563556in}{2.702261in}}%
\pgfpathlineto{\pgfqpoint{2.643717in}{2.640904in}}%
\pgfpathlineto{\pgfqpoint{2.723879in}{2.577384in}}%
\pgfpathlineto{\pgfqpoint{2.809692in}{2.506667in}}%
\pgfpathlineto{\pgfqpoint{2.897009in}{2.432000in}}%
\pgfpathlineto{\pgfqpoint{2.981444in}{2.357333in}}%
\pgfpathlineto{\pgfqpoint{3.022647in}{2.320000in}}%
\pgfpathlineto{\pgfqpoint{3.103149in}{2.245333in}}%
\pgfpathlineto{\pgfqpoint{3.181260in}{2.170667in}}%
\pgfpathlineto{\pgfqpoint{3.219471in}{2.133333in}}%
\pgfpathlineto{\pgfqpoint{3.294303in}{2.058667in}}%
\pgfpathlineto{\pgfqpoint{3.367130in}{1.984000in}}%
\pgfpathlineto{\pgfqpoint{3.402706in}{1.946667in}}%
\pgfpathlineto{\pgfqpoint{3.472249in}{1.872000in}}%
\pgfpathlineto{\pgfqpoint{3.540033in}{1.797333in}}%
\pgfpathlineto{\pgfqpoint{3.608350in}{1.720158in}}%
\pgfpathlineto{\pgfqpoint{3.701344in}{1.610667in}}%
\pgfpathlineto{\pgfqpoint{3.725899in}{1.581146in}}%
\pgfpathlineto{\pgfqpoint{3.792606in}{1.498667in}}%
\pgfpathlineto{\pgfqpoint{3.806061in}{1.481726in}}%
\pgfpathlineto{\pgfqpoint{3.886222in}{1.378096in}}%
\pgfpathlineto{\pgfqpoint{3.915044in}{1.338846in}}%
\pgfpathlineto{\pgfqpoint{3.935360in}{1.312000in}}%
\pgfpathlineto{\pgfqpoint{3.947491in}{1.294403in}}%
\pgfpathlineto{\pgfqpoint{3.966384in}{1.269297in}}%
\pgfpathlineto{\pgfqpoint{3.988972in}{1.237333in}}%
\pgfpathlineto{\pgfqpoint{4.046545in}{1.154059in}}%
\pgfpathlineto{\pgfqpoint{4.073307in}{1.112927in}}%
\pgfpathlineto{\pgfqpoint{4.096017in}{1.079253in}}%
\pgfpathlineto{\pgfqpoint{4.160308in}{0.976000in}}%
\pgfpathlineto{\pgfqpoint{4.166788in}{0.965112in}}%
\pgfpathlineto{\pgfqpoint{4.206869in}{0.896243in}}%
\pgfpathlineto{\pgfqpoint{4.246949in}{0.822926in}}%
\pgfpathlineto{\pgfqpoint{4.287030in}{0.743471in}}%
\pgfpathlineto{\pgfqpoint{4.333586in}{0.640000in}}%
\pgfpathlineto{\pgfqpoint{4.348393in}{0.602667in}}%
\pgfpathlineto{\pgfqpoint{4.367192in}{0.551418in}}%
\pgfpathlineto{\pgfqpoint{4.375040in}{0.528000in}}%
\pgfpathlineto{\pgfqpoint{4.391840in}{0.528000in}}%
\pgfpathlineto{\pgfqpoint{4.378694in}{0.565333in}}%
\pgfpathlineto{\pgfqpoint{4.375383in}{0.572963in}}%
\pgfpathlineto{\pgfqpoint{4.364552in}{0.602667in}}%
\pgfpathlineto{\pgfqpoint{4.342188in}{0.654043in}}%
\pgfpathlineto{\pgfqpoint{4.327111in}{0.689565in}}%
\pgfpathlineto{\pgfqpoint{4.258781in}{0.826667in}}%
\pgfpathlineto{\pgfqpoint{4.217419in}{0.901333in}}%
\pgfpathlineto{\pgfqpoint{4.206869in}{0.919608in}}%
\pgfpathlineto{\pgfqpoint{4.166788in}{0.986961in}}%
\pgfpathlineto{\pgfqpoint{4.126707in}{1.051468in}}%
\pgfpathlineto{\pgfqpoint{4.053285in}{1.162667in}}%
\pgfpathlineto{\pgfqpoint{4.001488in}{1.237333in}}%
\pgfpathlineto{\pgfqpoint{3.974761in}{1.274667in}}%
\pgfpathlineto{\pgfqpoint{3.919994in}{1.349333in}}%
\pgfpathlineto{\pgfqpoint{3.891887in}{1.386667in}}%
\pgfpathlineto{\pgfqpoint{3.834156in}{1.461333in}}%
\pgfpathlineto{\pgfqpoint{3.765980in}{1.546762in}}%
\pgfpathlineto{\pgfqpoint{3.682141in}{1.648000in}}%
\pgfpathlineto{\pgfqpoint{3.650330in}{1.685333in}}%
\pgfpathlineto{\pgfqpoint{3.585229in}{1.760000in}}%
\pgfpathlineto{\pgfqpoint{3.518549in}{1.834667in}}%
\pgfpathlineto{\pgfqpoint{3.502971in}{1.851020in}}%
\pgfpathlineto{\pgfqpoint{3.479493in}{1.877515in}}%
\pgfpathlineto{\pgfqpoint{3.405253in}{1.956789in}}%
\pgfpathlineto{\pgfqpoint{3.372455in}{1.990784in}}%
\pgfpathlineto{\pgfqpoint{3.343214in}{2.021333in}}%
\pgfpathlineto{\pgfqpoint{3.296076in}{2.068974in}}%
\pgfpathlineto{\pgfqpoint{3.269684in}{2.096000in}}%
\pgfpathlineto{\pgfqpoint{3.218583in}{2.146127in}}%
\pgfpathlineto{\pgfqpoint{3.194094in}{2.170667in}}%
\pgfpathlineto{\pgfqpoint{3.116294in}{2.245333in}}%
\pgfpathlineto{\pgfqpoint{3.036119in}{2.320000in}}%
\pgfpathlineto{\pgfqpoint{2.953387in}{2.394667in}}%
\pgfpathlineto{\pgfqpoint{2.867900in}{2.469333in}}%
\pgfpathlineto{\pgfqpoint{2.813339in}{2.515328in}}%
\pgfpathlineto{\pgfqpoint{2.779434in}{2.544000in}}%
\pgfpathlineto{\pgfqpoint{2.706742in}{2.602705in}}%
\pgfpathlineto{\pgfqpoint{2.661244in}{2.639674in}}%
\pgfpathlineto{\pgfqpoint{2.591791in}{2.693333in}}%
\pgfpathlineto{\pgfqpoint{2.491433in}{2.768000in}}%
\pgfpathlineto{\pgfqpoint{2.439286in}{2.805333in}}%
\pgfpathlineto{\pgfqpoint{2.363152in}{2.857670in}}%
\pgfpathlineto{\pgfqpoint{2.348815in}{2.866646in}}%
\pgfpathlineto{\pgfqpoint{2.307021in}{2.894949in}}%
\pgfpathlineto{\pgfqpoint{2.242909in}{2.935511in}}%
\pgfpathlineto{\pgfqpoint{2.162747in}{2.983798in}}%
\pgfpathlineto{\pgfqpoint{2.081769in}{3.029333in}}%
\pgfpathlineto{\pgfqpoint{1.994485in}{3.074061in}}%
\pgfpathlineto{\pgfqpoint{1.922263in}{3.107350in}}%
\pgfpathlineto{\pgfqpoint{1.882182in}{3.124011in}}%
\pgfpathlineto{\pgfqpoint{1.837687in}{3.141333in}}%
\pgfpathlineto{\pgfqpoint{1.802020in}{3.153690in}}%
\pgfpathlineto{\pgfqpoint{1.781312in}{3.159378in}}%
\pgfpathlineto{\pgfqpoint{1.761939in}{3.166195in}}%
\pgfpathlineto{\pgfqpoint{1.751070in}{3.168542in}}%
\pgfpathlineto{\pgfqpoint{1.714695in}{3.178667in}}%
\pgfpathlineto{\pgfqpoint{1.672324in}{3.187473in}}%
\pgfpathlineto{\pgfqpoint{1.625533in}{3.193722in}}%
\pgfpathlineto{\pgfqpoint{1.601616in}{3.195091in}}%
\pgfpathlineto{\pgfqpoint{1.578245in}{3.194231in}}%
\pgfpathlineto{\pgfqpoint{1.561535in}{3.194896in}}%
\pgfpathlineto{\pgfqpoint{1.511082in}{3.188329in}}%
\pgfpathlineto{\pgfqpoint{1.479177in}{3.178667in}}%
\pgfpathlineto{\pgfqpoint{1.453986in}{3.166843in}}%
\pgfpathlineto{\pgfqpoint{1.441293in}{3.158551in}}%
\pgfpathlineto{\pgfqpoint{1.410677in}{3.132517in}}%
\pgfpathlineto{\pgfqpoint{1.401212in}{3.120399in}}%
\pgfpathlineto{\pgfqpoint{1.390245in}{3.104000in}}%
\pgfpathlineto{\pgfqpoint{1.383226in}{3.087247in}}%
\pgfpathlineto{\pgfqpoint{1.370145in}{3.058271in}}%
\pgfpathlineto{\pgfqpoint{1.363297in}{3.029333in}}%
\pgfpathlineto{\pgfqpoint{1.358991in}{2.990007in}}%
\pgfpathlineto{\pgfqpoint{1.358292in}{2.954667in}}%
\pgfpathlineto{\pgfqpoint{1.361131in}{2.913465in}}%
\pgfpathlineto{\pgfqpoint{1.365665in}{2.880000in}}%
\pgfpathlineto{\pgfqpoint{1.381666in}{2.805333in}}%
\pgfpathlineto{\pgfqpoint{1.391923in}{2.768000in}}%
\pgfpathlineto{\pgfqpoint{1.394302in}{2.761564in}}%
\pgfpathlineto{\pgfqpoint{1.403521in}{2.730667in}}%
\pgfpathlineto{\pgfqpoint{1.449081in}{2.611412in}}%
\pgfpathlineto{\pgfqpoint{1.481374in}{2.539858in}}%
\pgfpathlineto{\pgfqpoint{1.497598in}{2.506667in}}%
\pgfpathlineto{\pgfqpoint{1.521455in}{2.459330in}}%
\pgfpathlineto{\pgfqpoint{1.561535in}{2.384847in}}%
\pgfpathlineto{\pgfqpoint{1.601616in}{2.314817in}}%
\pgfpathlineto{\pgfqpoint{1.646092in}{2.241240in}}%
\pgfpathlineto{\pgfqpoint{1.718102in}{2.129834in}}%
\pgfpathlineto{\pgfqpoint{1.740803in}{2.096000in}}%
\pgfpathlineto{\pgfqpoint{1.802020in}{2.007812in}}%
\pgfpathlineto{\pgfqpoint{1.873696in}{1.909333in}}%
\pgfpathlineto{\pgfqpoint{1.901838in}{1.872000in}}%
\pgfpathlineto{\pgfqpoint{1.962343in}{1.793308in}}%
\pgfpathlineto{\pgfqpoint{1.988791in}{1.760000in}}%
\pgfpathlineto{\pgfqpoint{2.048988in}{1.685333in}}%
\pgfpathlineto{\pgfqpoint{2.063965in}{1.667989in}}%
\pgfpathlineto{\pgfqpoint{2.082586in}{1.644591in}}%
\pgfpathlineto{\pgfqpoint{2.116368in}{1.604800in}}%
\pgfpathlineto{\pgfqpoint{2.142945in}{1.573333in}}%
\pgfpathlineto{\pgfqpoint{2.187838in}{1.522037in}}%
\pgfpathlineto{\pgfqpoint{2.207634in}{1.498667in}}%
\pgfpathlineto{\pgfqpoint{2.240684in}{1.461333in}}%
\pgfpathlineto{\pgfqpoint{2.308462in}{1.386667in}}%
\pgfpathlineto{\pgfqpoint{2.352580in}{1.339486in}}%
\pgfpathlineto{\pgfqpoint{2.377942in}{1.312000in}}%
\pgfpathlineto{\pgfqpoint{2.413357in}{1.274667in}}%
\pgfpathlineto{\pgfqpoint{2.511299in}{1.174008in}}%
\pgfpathlineto{\pgfqpoint{2.563556in}{1.121891in}}%
\pgfpathlineto{\pgfqpoint{2.598138in}{1.088000in}}%
\pgfpathlineto{\pgfqpoint{2.683798in}{1.005810in}}%
\pgfpathlineto{\pgfqpoint{2.715584in}{0.976000in}}%
\pgfpathlineto{\pgfqpoint{2.763960in}{0.931235in}}%
\pgfpathlineto{\pgfqpoint{2.796814in}{0.901333in}}%
\pgfpathlineto{\pgfqpoint{2.844121in}{0.858864in}}%
\pgfpathlineto{\pgfqpoint{2.880595in}{0.826667in}}%
\pgfpathlineto{\pgfqpoint{2.933216in}{0.781012in}}%
\pgfpathlineto{\pgfqpoint{3.011922in}{0.714667in}}%
\pgfpathlineto{\pgfqpoint{3.072221in}{0.665797in}}%
\pgfpathlineto{\pgfqpoint{3.103692in}{0.640000in}}%
\pgfpathlineto{\pgfqpoint{3.164768in}{0.591807in}}%
\pgfpathlineto{\pgfqpoint{3.199054in}{0.565333in}}%
\pgfpathlineto{\pgfqpoint{3.248415in}{0.528000in}}%
\pgfpathlineto{\pgfqpoint{3.248415in}{0.528000in}}%
\pgfusepath{fill}%
\end{pgfscope}%
\begin{pgfscope}%
\pgfpathrectangle{\pgfqpoint{0.800000in}{0.528000in}}{\pgfqpoint{3.968000in}{3.696000in}}%
\pgfusepath{clip}%
\pgfsetbuttcap%
\pgfsetroundjoin%
\definecolor{currentfill}{rgb}{0.277941,0.056324,0.381191}%
\pgfsetfillcolor{currentfill}%
\pgfsetlinewidth{0.000000pt}%
\definecolor{currentstroke}{rgb}{0.000000,0.000000,0.000000}%
\pgfsetstrokecolor{currentstroke}%
\pgfsetdash{}{0pt}%
\pgfpathmoveto{\pgfqpoint{3.248415in}{0.528000in}}%
\pgfpathlineto{\pgfqpoint{3.084606in}{0.655258in}}%
\pgfpathlineto{\pgfqpoint{3.050657in}{0.683045in}}%
\pgfpathlineto{\pgfqpoint{3.029560in}{0.700728in}}%
\pgfpathlineto{\pgfqpoint{3.004444in}{0.720851in}}%
\pgfpathlineto{\pgfqpoint{2.987238in}{0.735973in}}%
\pgfpathlineto{\pgfqpoint{2.964364in}{0.754463in}}%
\pgfpathlineto{\pgfqpoint{2.880595in}{0.826667in}}%
\pgfpathlineto{\pgfqpoint{2.844121in}{0.858864in}}%
\pgfpathlineto{\pgfqpoint{2.755892in}{0.938667in}}%
\pgfpathlineto{\pgfqpoint{2.723879in}{0.968244in}}%
\pgfpathlineto{\pgfqpoint{2.636727in}{1.050667in}}%
\pgfpathlineto{\pgfqpoint{2.603636in}{1.082628in}}%
\pgfpathlineto{\pgfqpoint{2.511299in}{1.174008in}}%
\pgfpathlineto{\pgfqpoint{2.413357in}{1.274667in}}%
\pgfpathlineto{\pgfqpoint{2.342982in}{1.349333in}}%
\pgfpathlineto{\pgfqpoint{2.274367in}{1.424000in}}%
\pgfpathlineto{\pgfqpoint{2.202828in}{1.504129in}}%
\pgfpathlineto{\pgfqpoint{2.175097in}{1.536000in}}%
\pgfpathlineto{\pgfqpoint{2.111165in}{1.610667in}}%
\pgfpathlineto{\pgfqpoint{2.081020in}{1.646542in}}%
\pgfpathlineto{\pgfqpoint{2.079745in}{1.648000in}}%
\pgfpathlineto{\pgfqpoint{2.048988in}{1.685333in}}%
\pgfpathlineto{\pgfqpoint{1.988791in}{1.760000in}}%
\pgfpathlineto{\pgfqpoint{1.977894in}{1.774485in}}%
\pgfpathlineto{\pgfqpoint{1.959181in}{1.797333in}}%
\pgfpathlineto{\pgfqpoint{1.927046in}{1.839122in}}%
\pgfpathlineto{\pgfqpoint{1.901838in}{1.872000in}}%
\pgfpathlineto{\pgfqpoint{1.877294in}{1.904780in}}%
\pgfpathlineto{\pgfqpoint{1.854455in}{1.935159in}}%
\pgfpathlineto{\pgfqpoint{1.792430in}{2.021333in}}%
\pgfpathlineto{\pgfqpoint{1.780784in}{2.038887in}}%
\pgfpathlineto{\pgfqpoint{1.761939in}{2.064913in}}%
\pgfpathlineto{\pgfqpoint{1.691122in}{2.170667in}}%
\pgfpathlineto{\pgfqpoint{1.672864in}{2.199697in}}%
\pgfpathlineto{\pgfqpoint{1.667194in}{2.208000in}}%
\pgfpathlineto{\pgfqpoint{1.641697in}{2.248394in}}%
\pgfpathlineto{\pgfqpoint{1.620962in}{2.282667in}}%
\pgfpathlineto{\pgfqpoint{1.598534in}{2.320000in}}%
\pgfpathlineto{\pgfqpoint{1.577122in}{2.357333in}}%
\pgfpathlineto{\pgfqpoint{1.556041in}{2.394667in}}%
\pgfpathlineto{\pgfqpoint{1.535911in}{2.432000in}}%
\pgfpathlineto{\pgfqpoint{1.497598in}{2.506667in}}%
\pgfpathlineto{\pgfqpoint{1.492898in}{2.517401in}}%
\pgfpathlineto{\pgfqpoint{1.479376in}{2.544000in}}%
\pgfpathlineto{\pgfqpoint{1.462492in}{2.581333in}}%
\pgfpathlineto{\pgfqpoint{1.441293in}{2.630363in}}%
\pgfpathlineto{\pgfqpoint{1.403521in}{2.730667in}}%
\pgfpathlineto{\pgfqpoint{1.388229in}{2.780093in}}%
\pgfpathlineto{\pgfqpoint{1.372819in}{2.842667in}}%
\pgfpathlineto{\pgfqpoint{1.371542in}{2.852363in}}%
\pgfpathlineto{\pgfqpoint{1.365665in}{2.880000in}}%
\pgfpathlineto{\pgfqpoint{1.360547in}{2.917877in}}%
\pgfpathlineto{\pgfqpoint{1.358292in}{2.954667in}}%
\pgfpathlineto{\pgfqpoint{1.358927in}{2.994053in}}%
\pgfpathlineto{\pgfqpoint{1.363297in}{3.029333in}}%
\pgfpathlineto{\pgfqpoint{1.373243in}{3.066667in}}%
\pgfpathlineto{\pgfqpoint{1.390245in}{3.104000in}}%
\pgfpathlineto{\pgfqpoint{1.401212in}{3.120399in}}%
\pgfpathlineto{\pgfqpoint{1.410677in}{3.132517in}}%
\pgfpathlineto{\pgfqpoint{1.420032in}{3.141333in}}%
\pgfpathlineto{\pgfqpoint{1.441293in}{3.158551in}}%
\pgfpathlineto{\pgfqpoint{1.453986in}{3.166843in}}%
\pgfpathlineto{\pgfqpoint{1.481374in}{3.179584in}}%
\pgfpathlineto{\pgfqpoint{1.511082in}{3.188329in}}%
\pgfpathlineto{\pgfqpoint{1.521455in}{3.190192in}}%
\pgfpathlineto{\pgfqpoint{1.534804in}{3.191101in}}%
\pgfpathlineto{\pgfqpoint{1.561535in}{3.194896in}}%
\pgfpathlineto{\pgfqpoint{1.578245in}{3.194231in}}%
\pgfpathlineto{\pgfqpoint{1.601616in}{3.195091in}}%
\pgfpathlineto{\pgfqpoint{1.641697in}{3.191765in}}%
\pgfpathlineto{\pgfqpoint{1.681778in}{3.185638in}}%
\pgfpathlineto{\pgfqpoint{1.687555in}{3.184048in}}%
\pgfpathlineto{\pgfqpoint{1.721859in}{3.177142in}}%
\pgfpathlineto{\pgfqpoint{1.802020in}{3.153690in}}%
\pgfpathlineto{\pgfqpoint{1.844806in}{3.138814in}}%
\pgfpathlineto{\pgfqpoint{1.924597in}{3.106175in}}%
\pgfpathlineto{\pgfqpoint{1.962343in}{3.089134in}}%
\pgfpathlineto{\pgfqpoint{2.009307in}{3.066667in}}%
\pgfpathlineto{\pgfqpoint{2.042505in}{3.049805in}}%
\pgfpathlineto{\pgfqpoint{2.082586in}{3.028906in}}%
\pgfpathlineto{\pgfqpoint{2.122667in}{3.006634in}}%
\pgfpathlineto{\pgfqpoint{2.206266in}{2.957869in}}%
\pgfpathlineto{\pgfqpoint{2.242909in}{2.935511in}}%
\pgfpathlineto{\pgfqpoint{2.329743in}{2.880000in}}%
\pgfpathlineto{\pgfqpoint{2.363152in}{2.857670in}}%
\pgfpathlineto{\pgfqpoint{2.443313in}{2.802529in}}%
\pgfpathlineto{\pgfqpoint{2.463836in}{2.787116in}}%
\pgfpathlineto{\pgfqpoint{2.491433in}{2.768000in}}%
\pgfpathlineto{\pgfqpoint{2.591791in}{2.693333in}}%
\pgfpathlineto{\pgfqpoint{2.643717in}{2.653431in}}%
\pgfpathlineto{\pgfqpoint{2.734009in}{2.581333in}}%
\pgfpathlineto{\pgfqpoint{2.824054in}{2.506667in}}%
\pgfpathlineto{\pgfqpoint{2.911001in}{2.432000in}}%
\pgfpathlineto{\pgfqpoint{2.995085in}{2.357333in}}%
\pgfpathlineto{\pgfqpoint{3.036119in}{2.320000in}}%
\pgfpathlineto{\pgfqpoint{3.116294in}{2.245333in}}%
\pgfpathlineto{\pgfqpoint{3.194094in}{2.170667in}}%
\pgfpathlineto{\pgfqpoint{3.232156in}{2.133333in}}%
\pgfpathlineto{\pgfqpoint{3.306698in}{2.058667in}}%
\pgfpathlineto{\pgfqpoint{3.379249in}{1.984000in}}%
\pgfpathlineto{\pgfqpoint{3.428732in}{1.931204in}}%
\pgfpathlineto{\pgfqpoint{3.449939in}{1.909333in}}%
\pgfpathlineto{\pgfqpoint{3.525495in}{1.827005in}}%
\pgfpathlineto{\pgfqpoint{3.552093in}{1.797333in}}%
\pgfpathlineto{\pgfqpoint{3.617971in}{1.722667in}}%
\pgfpathlineto{\pgfqpoint{3.685818in}{1.643643in}}%
\pgfpathlineto{\pgfqpoint{3.774707in}{1.536000in}}%
\pgfpathlineto{\pgfqpoint{3.846141in}{1.446014in}}%
\pgfpathlineto{\pgfqpoint{3.926303in}{1.340849in}}%
\pgfpathlineto{\pgfqpoint{4.006465in}{1.230260in}}%
\pgfpathlineto{\pgfqpoint{4.027511in}{1.200000in}}%
\pgfpathlineto{\pgfqpoint{4.086626in}{1.112905in}}%
\pgfpathlineto{\pgfqpoint{4.127223in}{1.050667in}}%
\pgfpathlineto{\pgfqpoint{4.150469in}{1.013333in}}%
\pgfpathlineto{\pgfqpoint{4.195730in}{0.938667in}}%
\pgfpathlineto{\pgfqpoint{4.238480in}{0.864000in}}%
\pgfpathlineto{\pgfqpoint{4.246949in}{0.848540in}}%
\pgfpathlineto{\pgfqpoint{4.287030in}{0.772495in}}%
\pgfpathlineto{\pgfqpoint{4.332746in}{0.677333in}}%
\pgfpathlineto{\pgfqpoint{4.353590in}{0.627330in}}%
\pgfpathlineto{\pgfqpoint{4.367192in}{0.595780in}}%
\pgfpathlineto{\pgfqpoint{4.391840in}{0.528000in}}%
\pgfpathlineto{\pgfqpoint{4.408551in}{0.528000in}}%
\pgfpathlineto{\pgfqpoint{4.408189in}{0.528853in}}%
\pgfpathlineto{\pgfqpoint{4.394615in}{0.565333in}}%
\pgfpathlineto{\pgfqpoint{4.386721in}{0.583524in}}%
\pgfpathlineto{\pgfqpoint{4.379849in}{0.602667in}}%
\pgfpathlineto{\pgfqpoint{4.364272in}{0.640000in}}%
\pgfpathlineto{\pgfqpoint{4.347352in}{0.677333in}}%
\pgfpathlineto{\pgfqpoint{4.327111in}{0.720569in}}%
\pgfpathlineto{\pgfqpoint{4.303152in}{0.767016in}}%
\pgfpathlineto{\pgfqpoint{4.287030in}{0.799530in}}%
\pgfpathlineto{\pgfqpoint{4.246949in}{0.872943in}}%
\pgfpathlineto{\pgfqpoint{4.206869in}{0.942224in}}%
\pgfpathlineto{\pgfqpoint{4.163455in}{1.013333in}}%
\pgfpathlineto{\pgfqpoint{4.134614in}{1.058031in}}%
\pgfpathlineto{\pgfqpoint{4.115520in}{1.088000in}}%
\pgfpathlineto{\pgfqpoint{4.086626in}{1.131698in}}%
\pgfpathlineto{\pgfqpoint{4.065519in}{1.162667in}}%
\pgfpathlineto{\pgfqpoint{4.006465in}{1.247375in}}%
\pgfpathlineto{\pgfqpoint{3.986783in}{1.274667in}}%
\pgfpathlineto{\pgfqpoint{3.926303in}{1.356908in}}%
\pgfpathlineto{\pgfqpoint{3.903704in}{1.386667in}}%
\pgfpathlineto{\pgfqpoint{3.846141in}{1.461447in}}%
\pgfpathlineto{\pgfqpoint{3.756071in}{1.573333in}}%
\pgfpathlineto{\pgfqpoint{3.685818in}{1.657505in}}%
\pgfpathlineto{\pgfqpoint{3.662002in}{1.685333in}}%
\pgfpathlineto{\pgfqpoint{3.597158in}{1.760000in}}%
\pgfpathlineto{\pgfqpoint{3.525495in}{1.840178in}}%
\pgfpathlineto{\pgfqpoint{3.491187in}{1.877377in}}%
\pgfpathlineto{\pgfqpoint{3.461793in}{1.909333in}}%
\pgfpathlineto{\pgfqpoint{3.391367in}{1.984000in}}%
\pgfpathlineto{\pgfqpoint{3.340700in}{2.035873in}}%
\pgfpathlineto{\pgfqpoint{3.319093in}{2.058667in}}%
\pgfpathlineto{\pgfqpoint{3.285010in}{2.093207in}}%
\pgfpathlineto{\pgfqpoint{3.204848in}{2.172599in}}%
\pgfpathlineto{\pgfqpoint{3.089544in}{2.282667in}}%
\pgfpathlineto{\pgfqpoint{3.004444in}{2.361019in}}%
\pgfpathlineto{\pgfqpoint{2.924283in}{2.432593in}}%
\pgfpathlineto{\pgfqpoint{2.903479in}{2.449955in}}%
\pgfpathlineto{\pgfqpoint{2.882075in}{2.469333in}}%
\pgfpathlineto{\pgfqpoint{2.844121in}{2.501844in}}%
\pgfpathlineto{\pgfqpoint{2.748762in}{2.581333in}}%
\pgfpathlineto{\pgfqpoint{2.655771in}{2.656000in}}%
\pgfpathlineto{\pgfqpoint{2.583676in}{2.711925in}}%
\pgfpathlineto{\pgfqpoint{2.508390in}{2.768000in}}%
\pgfpathlineto{\pgfqpoint{2.443313in}{2.814992in}}%
\pgfpathlineto{\pgfqpoint{2.363152in}{2.870523in}}%
\pgfpathlineto{\pgfqpoint{2.333595in}{2.889802in}}%
\pgfpathlineto{\pgfqpoint{2.323071in}{2.897257in}}%
\pgfpathlineto{\pgfqpoint{2.233501in}{2.954667in}}%
\pgfpathlineto{\pgfqpoint{2.190364in}{2.980390in}}%
\pgfpathlineto{\pgfqpoint{2.146842in}{3.006815in}}%
\pgfpathlineto{\pgfqpoint{2.082586in}{3.042818in}}%
\pgfpathlineto{\pgfqpoint{2.038062in}{3.066667in}}%
\pgfpathlineto{\pgfqpoint{2.002424in}{3.084702in}}%
\pgfpathlineto{\pgfqpoint{1.962343in}{3.104448in}}%
\pgfpathlineto{\pgfqpoint{1.908097in}{3.128138in}}%
\pgfpathlineto{\pgfqpoint{1.879227in}{3.141333in}}%
\pgfpathlineto{\pgfqpoint{1.761939in}{3.183978in}}%
\pgfpathlineto{\pgfqpoint{1.753341in}{3.186676in}}%
\pgfpathlineto{\pgfqpoint{1.721859in}{3.195312in}}%
\pgfpathlineto{\pgfqpoint{1.703546in}{3.198943in}}%
\pgfpathlineto{\pgfqpoint{1.681778in}{3.204934in}}%
\pgfpathlineto{\pgfqpoint{1.641697in}{3.212471in}}%
\pgfpathlineto{\pgfqpoint{1.600029in}{3.217478in}}%
\pgfpathlineto{\pgfqpoint{1.561535in}{3.218881in}}%
\pgfpathlineto{\pgfqpoint{1.517502in}{3.216000in}}%
\pgfpathlineto{\pgfqpoint{1.481374in}{3.208839in}}%
\pgfpathlineto{\pgfqpoint{1.457289in}{3.201101in}}%
\pgfpathlineto{\pgfqpoint{1.441293in}{3.193654in}}%
\pgfpathlineto{\pgfqpoint{1.416429in}{3.178667in}}%
\pgfpathlineto{\pgfqpoint{1.401212in}{3.166623in}}%
\pgfpathlineto{\pgfqpoint{1.387924in}{3.153711in}}%
\pgfpathlineto{\pgfqpoint{1.378480in}{3.141333in}}%
\pgfpathlineto{\pgfqpoint{1.356704in}{3.104000in}}%
\pgfpathlineto{\pgfqpoint{1.347634in}{3.079239in}}%
\pgfpathlineto{\pgfqpoint{1.340473in}{3.048575in}}%
\pgfpathlineto{\pgfqpoint{1.336414in}{3.015023in}}%
\pgfpathlineto{\pgfqpoint{1.335791in}{2.992000in}}%
\pgfpathlineto{\pgfqpoint{1.337136in}{2.969649in}}%
\pgfpathlineto{\pgfqpoint{1.336839in}{2.954667in}}%
\pgfpathlineto{\pgfqpoint{1.337795in}{2.939070in}}%
\pgfpathlineto{\pgfqpoint{1.340577in}{2.917333in}}%
\pgfpathlineto{\pgfqpoint{1.343855in}{2.901241in}}%
\pgfpathlineto{\pgfqpoint{1.346505in}{2.880000in}}%
\pgfpathlineto{\pgfqpoint{1.354241in}{2.842667in}}%
\pgfpathlineto{\pgfqpoint{1.364475in}{2.802218in}}%
\pgfpathlineto{\pgfqpoint{1.374927in}{2.768000in}}%
\pgfpathlineto{\pgfqpoint{1.381660in}{2.749788in}}%
\pgfpathlineto{\pgfqpoint{1.391128in}{2.721274in}}%
\pgfpathlineto{\pgfqpoint{1.401212in}{2.691592in}}%
\pgfpathlineto{\pgfqpoint{1.447535in}{2.581333in}}%
\pgfpathlineto{\pgfqpoint{1.465120in}{2.544000in}}%
\pgfpathlineto{\pgfqpoint{1.484660in}{2.503606in}}%
\pgfpathlineto{\pgfqpoint{1.521943in}{2.432000in}}%
\pgfpathlineto{\pgfqpoint{1.542686in}{2.394667in}}%
\pgfpathlineto{\pgfqpoint{1.563600in}{2.357333in}}%
\pgfpathlineto{\pgfqpoint{1.591576in}{2.310648in}}%
\pgfpathlineto{\pgfqpoint{1.607860in}{2.282667in}}%
\pgfpathlineto{\pgfqpoint{1.630911in}{2.245333in}}%
\pgfpathlineto{\pgfqpoint{1.681778in}{2.165604in}}%
\pgfpathlineto{\pgfqpoint{1.754032in}{2.058667in}}%
\pgfpathlineto{\pgfqpoint{1.780309in}{2.021333in}}%
\pgfpathlineto{\pgfqpoint{1.842101in}{1.935752in}}%
\pgfpathlineto{\pgfqpoint{1.922263in}{1.829546in}}%
\pgfpathlineto{\pgfqpoint{1.947470in}{1.797333in}}%
\pgfpathlineto{\pgfqpoint{2.006754in}{1.722667in}}%
\pgfpathlineto{\pgfqpoint{2.039540in}{1.682571in}}%
\pgfpathlineto{\pgfqpoint{2.068106in}{1.648000in}}%
\pgfpathlineto{\pgfqpoint{2.109925in}{1.598798in}}%
\pgfpathlineto{\pgfqpoint{2.131050in}{1.573333in}}%
\pgfpathlineto{\pgfqpoint{2.145501in}{1.557269in}}%
\pgfpathlineto{\pgfqpoint{2.163071in}{1.536000in}}%
\pgfpathlineto{\pgfqpoint{2.195831in}{1.498667in}}%
\pgfpathlineto{\pgfqpoint{2.262546in}{1.424000in}}%
\pgfpathlineto{\pgfqpoint{2.309056in}{1.373612in}}%
\pgfpathlineto{\pgfqpoint{2.330898in}{1.349333in}}%
\pgfpathlineto{\pgfqpoint{2.363152in}{1.314728in}}%
\pgfpathlineto{\pgfqpoint{2.443313in}{1.230877in}}%
\pgfpathlineto{\pgfqpoint{2.547804in}{1.125333in}}%
\pgfpathlineto{\pgfqpoint{2.624164in}{1.050667in}}%
\pgfpathlineto{\pgfqpoint{2.702723in}{0.976000in}}%
\pgfpathlineto{\pgfqpoint{2.783642in}{0.901333in}}%
\pgfpathlineto{\pgfqpoint{2.867095in}{0.826667in}}%
\pgfpathlineto{\pgfqpoint{2.953277in}{0.752000in}}%
\pgfpathlineto{\pgfqpoint{3.004444in}{0.708799in}}%
\pgfpathlineto{\pgfqpoint{3.022720in}{0.694356in}}%
\pgfpathlineto{\pgfqpoint{3.044525in}{0.675578in}}%
\pgfpathlineto{\pgfqpoint{3.065349in}{0.659396in}}%
\pgfpathlineto{\pgfqpoint{3.088369in}{0.640000in}}%
\pgfpathlineto{\pgfqpoint{3.183286in}{0.565333in}}%
\pgfpathlineto{\pgfqpoint{3.232180in}{0.528000in}}%
\pgfpathlineto{\pgfqpoint{3.244929in}{0.528000in}}%
\pgfpathlineto{\pgfqpoint{3.244929in}{0.528000in}}%
\pgfusepath{fill}%
\end{pgfscope}%
\begin{pgfscope}%
\pgfpathrectangle{\pgfqpoint{0.800000in}{0.528000in}}{\pgfqpoint{3.968000in}{3.696000in}}%
\pgfusepath{clip}%
\pgfsetbuttcap%
\pgfsetroundjoin%
\definecolor{currentfill}{rgb}{0.277941,0.056324,0.381191}%
\pgfsetfillcolor{currentfill}%
\pgfsetlinewidth{0.000000pt}%
\definecolor{currentstroke}{rgb}{0.000000,0.000000,0.000000}%
\pgfsetstrokecolor{currentstroke}%
\pgfsetdash{}{0pt}%
\pgfpathmoveto{\pgfqpoint{3.232180in}{0.528000in}}%
\pgfpathlineto{\pgfqpoint{3.135362in}{0.602667in}}%
\pgfpathlineto{\pgfqpoint{3.065349in}{0.659396in}}%
\pgfpathlineto{\pgfqpoint{3.042402in}{0.677333in}}%
\pgfpathlineto{\pgfqpoint{3.004444in}{0.708799in}}%
\pgfpathlineto{\pgfqpoint{2.909832in}{0.789333in}}%
\pgfpathlineto{\pgfqpoint{2.825040in}{0.864000in}}%
\pgfpathlineto{\pgfqpoint{2.773899in}{0.910592in}}%
\pgfpathlineto{\pgfqpoint{2.742877in}{0.938667in}}%
\pgfpathlineto{\pgfqpoint{2.693499in}{0.985036in}}%
\pgfpathlineto{\pgfqpoint{2.663159in}{1.013333in}}%
\pgfpathlineto{\pgfqpoint{2.614311in}{1.060609in}}%
\pgfpathlineto{\pgfqpoint{2.585718in}{1.088000in}}%
\pgfpathlineto{\pgfqpoint{2.547804in}{1.125333in}}%
\pgfpathlineto{\pgfqpoint{2.473496in}{1.200000in}}%
\pgfpathlineto{\pgfqpoint{2.401109in}{1.274667in}}%
\pgfpathlineto{\pgfqpoint{2.365722in}{1.312000in}}%
\pgfpathlineto{\pgfqpoint{2.296511in}{1.386667in}}%
\pgfpathlineto{\pgfqpoint{2.228991in}{1.461333in}}%
\pgfpathlineto{\pgfqpoint{2.202828in}{1.490722in}}%
\pgfpathlineto{\pgfqpoint{2.122667in}{1.583143in}}%
\pgfpathlineto{\pgfqpoint{2.092215in}{1.619636in}}%
\pgfpathlineto{\pgfqpoint{2.068106in}{1.648000in}}%
\pgfpathlineto{\pgfqpoint{2.039540in}{1.682571in}}%
\pgfpathlineto{\pgfqpoint{2.037158in}{1.685333in}}%
\pgfpathlineto{\pgfqpoint{2.004954in}{1.725023in}}%
\pgfpathlineto{\pgfqpoint{1.976952in}{1.760000in}}%
\pgfpathlineto{\pgfqpoint{1.962343in}{1.778398in}}%
\pgfpathlineto{\pgfqpoint{1.882182in}{1.882064in}}%
\pgfpathlineto{\pgfqpoint{1.861784in}{1.909333in}}%
\pgfpathlineto{\pgfqpoint{1.802020in}{1.990722in}}%
\pgfpathlineto{\pgfqpoint{1.721859in}{2.105553in}}%
\pgfpathlineto{\pgfqpoint{1.703271in}{2.133333in}}%
\pgfpathlineto{\pgfqpoint{1.654486in}{2.208000in}}%
\pgfpathlineto{\pgfqpoint{1.601616in}{2.293043in}}%
\pgfpathlineto{\pgfqpoint{1.577483in}{2.334855in}}%
\pgfpathlineto{\pgfqpoint{1.561535in}{2.360979in}}%
\pgfpathlineto{\pgfqpoint{1.521455in}{2.432923in}}%
\pgfpathlineto{\pgfqpoint{1.502440in}{2.469333in}}%
\pgfpathlineto{\pgfqpoint{1.481374in}{2.510306in}}%
\pgfpathlineto{\pgfqpoint{1.441293in}{2.595334in}}%
\pgfpathlineto{\pgfqpoint{1.400526in}{2.693333in}}%
\pgfpathlineto{\pgfqpoint{1.363281in}{2.807336in}}%
\pgfpathlineto{\pgfqpoint{1.352183in}{2.851001in}}%
\pgfpathlineto{\pgfqpoint{1.346505in}{2.880000in}}%
\pgfpathlineto{\pgfqpoint{1.343855in}{2.901241in}}%
\pgfpathlineto{\pgfqpoint{1.340577in}{2.917333in}}%
\pgfpathlineto{\pgfqpoint{1.336839in}{2.954667in}}%
\pgfpathlineto{\pgfqpoint{1.337136in}{2.969649in}}%
\pgfpathlineto{\pgfqpoint{1.335791in}{2.992000in}}%
\pgfpathlineto{\pgfqpoint{1.336414in}{3.015023in}}%
\pgfpathlineto{\pgfqpoint{1.340473in}{3.048575in}}%
\pgfpathlineto{\pgfqpoint{1.347634in}{3.079239in}}%
\pgfpathlineto{\pgfqpoint{1.361131in}{3.113239in}}%
\pgfpathlineto{\pgfqpoint{1.378480in}{3.141333in}}%
\pgfpathlineto{\pgfqpoint{1.387924in}{3.153711in}}%
\pgfpathlineto{\pgfqpoint{1.401212in}{3.166623in}}%
\pgfpathlineto{\pgfqpoint{1.416429in}{3.178667in}}%
\pgfpathlineto{\pgfqpoint{1.441293in}{3.193654in}}%
\pgfpathlineto{\pgfqpoint{1.457289in}{3.201101in}}%
\pgfpathlineto{\pgfqpoint{1.487215in}{3.210559in}}%
\pgfpathlineto{\pgfqpoint{1.522135in}{3.216634in}}%
\pgfpathlineto{\pgfqpoint{1.561535in}{3.218881in}}%
\pgfpathlineto{\pgfqpoint{1.612548in}{3.216000in}}%
\pgfpathlineto{\pgfqpoint{1.646157in}{3.211845in}}%
\pgfpathlineto{\pgfqpoint{1.681778in}{3.204934in}}%
\pgfpathlineto{\pgfqpoint{1.703546in}{3.198943in}}%
\pgfpathlineto{\pgfqpoint{1.721859in}{3.195312in}}%
\pgfpathlineto{\pgfqpoint{1.761939in}{3.183978in}}%
\pgfpathlineto{\pgfqpoint{1.815561in}{3.166054in}}%
\pgfpathlineto{\pgfqpoint{1.882182in}{3.140163in}}%
\pgfpathlineto{\pgfqpoint{1.935248in}{3.116095in}}%
\pgfpathlineto{\pgfqpoint{1.963255in}{3.104000in}}%
\pgfpathlineto{\pgfqpoint{2.048008in}{3.061540in}}%
\pgfpathlineto{\pgfqpoint{2.146842in}{3.006815in}}%
\pgfpathlineto{\pgfqpoint{2.202828in}{2.973454in}}%
\pgfpathlineto{\pgfqpoint{2.214661in}{2.965688in}}%
\pgfpathlineto{\pgfqpoint{2.261894in}{2.936983in}}%
\pgfpathlineto{\pgfqpoint{2.348972in}{2.880000in}}%
\pgfpathlineto{\pgfqpoint{2.523475in}{2.756960in}}%
\pgfpathlineto{\pgfqpoint{2.538801in}{2.744943in}}%
\pgfpathlineto{\pgfqpoint{2.583676in}{2.711925in}}%
\pgfpathlineto{\pgfqpoint{2.655771in}{2.656000in}}%
\pgfpathlineto{\pgfqpoint{2.748762in}{2.581333in}}%
\pgfpathlineto{\pgfqpoint{2.804040in}{2.535640in}}%
\pgfpathlineto{\pgfqpoint{2.838416in}{2.506667in}}%
\pgfpathlineto{\pgfqpoint{2.884202in}{2.467505in}}%
\pgfpathlineto{\pgfqpoint{2.903479in}{2.449955in}}%
\pgfpathlineto{\pgfqpoint{2.924954in}{2.432000in}}%
\pgfpathlineto{\pgfqpoint{3.008493in}{2.357333in}}%
\pgfpathlineto{\pgfqpoint{3.089544in}{2.282667in}}%
\pgfpathlineto{\pgfqpoint{3.168277in}{2.208000in}}%
\pgfpathlineto{\pgfqpoint{3.244929in}{2.133246in}}%
\pgfpathlineto{\pgfqpoint{3.282223in}{2.096000in}}%
\pgfpathlineto{\pgfqpoint{3.365172in}{2.011325in}}%
\pgfpathlineto{\pgfqpoint{3.405253in}{1.969470in}}%
\pgfpathlineto{\pgfqpoint{3.496350in}{1.872000in}}%
\pgfpathlineto{\pgfqpoint{3.525495in}{1.840178in}}%
\pgfpathlineto{\pgfqpoint{3.605657in}{1.750349in}}%
\pgfpathlineto{\pgfqpoint{3.693867in}{1.648000in}}%
\pgfpathlineto{\pgfqpoint{3.765980in}{1.561245in}}%
\pgfpathlineto{\pgfqpoint{3.786450in}{1.536000in}}%
\pgfpathlineto{\pgfqpoint{3.846474in}{1.461024in}}%
\pgfpathlineto{\pgfqpoint{3.945084in}{1.331840in}}%
\pgfpathlineto{\pgfqpoint{4.013627in}{1.237333in}}%
\pgfpathlineto{\pgfqpoint{4.039885in}{1.200000in}}%
\pgfpathlineto{\pgfqpoint{4.098063in}{1.114680in}}%
\pgfpathlineto{\pgfqpoint{4.166788in}{1.007947in}}%
\pgfpathlineto{\pgfqpoint{4.186323in}{0.976000in}}%
\pgfpathlineto{\pgfqpoint{4.209007in}{0.938667in}}%
\pgfpathlineto{\pgfqpoint{4.230653in}{0.901333in}}%
\pgfpathlineto{\pgfqpoint{4.272444in}{0.826667in}}%
\pgfpathlineto{\pgfqpoint{4.292443in}{0.789333in}}%
\pgfpathlineto{\pgfqpoint{4.316060in}{0.741706in}}%
\pgfpathlineto{\pgfqpoint{4.332443in}{0.709700in}}%
\pgfpathlineto{\pgfqpoint{4.367192in}{0.633081in}}%
\pgfpathlineto{\pgfqpoint{4.394615in}{0.565333in}}%
\pgfpathlineto{\pgfqpoint{4.408551in}{0.528000in}}%
\pgfpathlineto{\pgfqpoint{4.424242in}{0.528000in}}%
\pgfpathlineto{\pgfqpoint{4.419439in}{0.539333in}}%
\pgfpathlineto{\pgfqpoint{4.407273in}{0.572853in}}%
\pgfpathlineto{\pgfqpoint{4.358026in}{0.685871in}}%
\pgfpathlineto{\pgfqpoint{4.325559in}{0.752000in}}%
\pgfpathlineto{\pgfqpoint{4.305935in}{0.789333in}}%
\pgfpathlineto{\pgfqpoint{4.265096in}{0.864000in}}%
\pgfpathlineto{\pgfqpoint{4.243888in}{0.901333in}}%
\pgfpathlineto{\pgfqpoint{4.221690in}{0.938667in}}%
\pgfpathlineto{\pgfqpoint{4.172402in}{1.018562in}}%
\pgfpathlineto{\pgfqpoint{4.152132in}{1.050667in}}%
\pgfpathlineto{\pgfqpoint{4.142206in}{1.065103in}}%
\pgfpathlineto{\pgfqpoint{4.126707in}{1.090020in}}%
\pgfpathlineto{\pgfqpoint{4.046545in}{1.207728in}}%
\pgfpathlineto{\pgfqpoint{4.025517in}{1.237333in}}%
\pgfpathlineto{\pgfqpoint{3.966384in}{1.318997in}}%
\pgfpathlineto{\pgfqpoint{3.886222in}{1.425036in}}%
\pgfpathlineto{\pgfqpoint{3.798194in}{1.536000in}}%
\pgfpathlineto{\pgfqpoint{3.783937in}{1.552727in}}%
\pgfpathlineto{\pgfqpoint{3.765980in}{1.575603in}}%
\pgfpathlineto{\pgfqpoint{3.673673in}{1.685333in}}%
\pgfpathlineto{\pgfqpoint{3.605657in}{1.763704in}}%
\pgfpathlineto{\pgfqpoint{3.508076in}{1.872000in}}%
\pgfpathlineto{\pgfqpoint{3.473647in}{1.909333in}}%
\pgfpathlineto{\pgfqpoint{3.403486in}{1.984000in}}%
\pgfpathlineto{\pgfqpoint{3.367599in}{2.021333in}}%
\pgfpathlineto{\pgfqpoint{3.285010in}{2.105326in}}%
\pgfpathlineto{\pgfqpoint{3.256887in}{2.133333in}}%
\pgfpathlineto{\pgfqpoint{3.164768in}{2.223272in}}%
\pgfpathlineto{\pgfqpoint{3.133083in}{2.253153in}}%
\pgfpathlineto{\pgfqpoint{3.102145in}{2.282667in}}%
\pgfpathlineto{\pgfqpoint{3.021393in}{2.357333in}}%
\pgfpathlineto{\pgfqpoint{2.980101in}{2.394667in}}%
\pgfpathlineto{\pgfqpoint{2.895571in}{2.469333in}}%
\pgfpathlineto{\pgfqpoint{2.826685in}{2.527759in}}%
\pgfpathlineto{\pgfqpoint{2.804040in}{2.547582in}}%
\pgfpathlineto{\pgfqpoint{2.784494in}{2.563127in}}%
\pgfpathlineto{\pgfqpoint{2.763514in}{2.581333in}}%
\pgfpathlineto{\pgfqpoint{2.670936in}{2.656000in}}%
\pgfpathlineto{\pgfqpoint{2.574758in}{2.730667in}}%
\pgfpathlineto{\pgfqpoint{2.403232in}{2.855528in}}%
\pgfpathlineto{\pgfqpoint{2.311971in}{2.917333in}}%
\pgfpathlineto{\pgfqpoint{2.247392in}{2.958842in}}%
\pgfpathlineto{\pgfqpoint{2.242909in}{2.961893in}}%
\pgfpathlineto{\pgfqpoint{2.223055in}{2.973507in}}%
\pgfpathlineto{\pgfqpoint{2.194137in}{2.992000in}}%
\pgfpathlineto{\pgfqpoint{2.108991in}{3.042072in}}%
\pgfpathlineto{\pgfqpoint{2.042505in}{3.078369in}}%
\pgfpathlineto{\pgfqpoint{2.024363in}{3.087102in}}%
\pgfpathlineto{\pgfqpoint{1.992792in}{3.104000in}}%
\pgfpathlineto{\pgfqpoint{1.882182in}{3.155454in}}%
\pgfpathlineto{\pgfqpoint{1.864390in}{3.162095in}}%
\pgfpathlineto{\pgfqpoint{1.824713in}{3.178667in}}%
\pgfpathlineto{\pgfqpoint{1.761939in}{3.200961in}}%
\pgfpathlineto{\pgfqpoint{1.711680in}{3.216000in}}%
\pgfpathlineto{\pgfqpoint{1.681778in}{3.223670in}}%
\pgfpathlineto{\pgfqpoint{1.601616in}{3.237916in}}%
\pgfpathlineto{\pgfqpoint{1.561535in}{3.241094in}}%
\pgfpathlineto{\pgfqpoint{1.547538in}{3.240296in}}%
\pgfpathlineto{\pgfqpoint{1.521455in}{3.240748in}}%
\pgfpathlineto{\pgfqpoint{1.497085in}{3.238699in}}%
\pgfpathlineto{\pgfqpoint{1.481374in}{3.235895in}}%
\pgfpathlineto{\pgfqpoint{1.434129in}{3.222673in}}%
\pgfpathlineto{\pgfqpoint{1.420553in}{3.216000in}}%
\pgfpathlineto{\pgfqpoint{1.401212in}{3.205169in}}%
\pgfpathlineto{\pgfqpoint{1.365416in}{3.174675in}}%
\pgfpathlineto{\pgfqpoint{1.361131in}{3.169059in}}%
\pgfpathlineto{\pgfqpoint{1.342926in}{3.141333in}}%
\pgfpathlineto{\pgfqpoint{1.325354in}{3.099991in}}%
\pgfpathlineto{\pgfqpoint{1.317810in}{3.066667in}}%
\pgfpathlineto{\pgfqpoint{1.313873in}{3.029333in}}%
\pgfpathlineto{\pgfqpoint{1.313449in}{2.992000in}}%
\pgfpathlineto{\pgfqpoint{1.314265in}{2.985679in}}%
\pgfpathlineto{\pgfqpoint{1.316654in}{2.950572in}}%
\pgfpathlineto{\pgfqpoint{1.321051in}{2.914813in}}%
\pgfpathlineto{\pgfqpoint{1.336534in}{2.842667in}}%
\pgfpathlineto{\pgfqpoint{1.346754in}{2.805333in}}%
\pgfpathlineto{\pgfqpoint{1.350363in}{2.795303in}}%
\pgfpathlineto{\pgfqpoint{1.361131in}{2.759346in}}%
\pgfpathlineto{\pgfqpoint{1.401212in}{2.653212in}}%
\pgfpathlineto{\pgfqpoint{1.469433in}{2.506667in}}%
\pgfpathlineto{\pgfqpoint{1.481374in}{2.483357in}}%
\pgfpathlineto{\pgfqpoint{1.508723in}{2.432000in}}%
\pgfpathlineto{\pgfqpoint{1.550696in}{2.357333in}}%
\pgfpathlineto{\pgfqpoint{1.568900in}{2.326859in}}%
\pgfpathlineto{\pgfqpoint{1.572642in}{2.320000in}}%
\pgfpathlineto{\pgfqpoint{1.583466in}{2.303094in}}%
\pgfpathlineto{\pgfqpoint{1.601616in}{2.272135in}}%
\pgfpathlineto{\pgfqpoint{1.618350in}{2.245333in}}%
\pgfpathlineto{\pgfqpoint{1.666245in}{2.170667in}}%
\pgfpathlineto{\pgfqpoint{1.721859in}{2.087693in}}%
\pgfpathlineto{\pgfqpoint{1.802020in}{1.974261in}}%
\pgfpathlineto{\pgfqpoint{1.822274in}{1.946667in}}%
\pgfpathlineto{\pgfqpoint{1.882182in}{1.866471in}}%
\pgfpathlineto{\pgfqpoint{1.973096in}{1.749985in}}%
\pgfpathlineto{\pgfqpoint{2.056466in}{1.648000in}}%
\pgfpathlineto{\pgfqpoint{2.085477in}{1.613360in}}%
\pgfpathlineto{\pgfqpoint{2.087634in}{1.610667in}}%
\pgfpathlineto{\pgfqpoint{2.120845in}{1.571636in}}%
\pgfpathlineto{\pgfqpoint{2.151608in}{1.536000in}}%
\pgfpathlineto{\pgfqpoint{2.217298in}{1.461333in}}%
\pgfpathlineto{\pgfqpoint{2.284560in}{1.386667in}}%
\pgfpathlineto{\pgfqpoint{2.320941in}{1.347350in}}%
\pgfpathlineto{\pgfqpoint{2.353975in}{1.312000in}}%
\pgfpathlineto{\pgfqpoint{2.425194in}{1.237333in}}%
\pgfpathlineto{\pgfqpoint{2.472177in}{1.189552in}}%
\pgfpathlineto{\pgfqpoint{2.498260in}{1.162667in}}%
\pgfpathlineto{\pgfqpoint{2.573299in}{1.088000in}}%
\pgfpathlineto{\pgfqpoint{2.650449in}{1.013333in}}%
\pgfpathlineto{\pgfqpoint{2.729862in}{0.938667in}}%
\pgfpathlineto{\pgfqpoint{2.811705in}{0.864000in}}%
\pgfpathlineto{\pgfqpoint{2.896162in}{0.789333in}}%
\pgfpathlineto{\pgfqpoint{2.939432in}{0.752000in}}%
\pgfpathlineto{\pgfqpoint{3.028195in}{0.677333in}}%
\pgfpathlineto{\pgfqpoint{3.120119in}{0.602667in}}%
\pgfpathlineto{\pgfqpoint{3.216180in}{0.528000in}}%
\pgfpathlineto{\pgfqpoint{3.216180in}{0.528000in}}%
\pgfusepath{fill}%
\end{pgfscope}%
\begin{pgfscope}%
\pgfpathrectangle{\pgfqpoint{0.800000in}{0.528000in}}{\pgfqpoint{3.968000in}{3.696000in}}%
\pgfusepath{clip}%
\pgfsetbuttcap%
\pgfsetroundjoin%
\definecolor{currentfill}{rgb}{0.277941,0.056324,0.381191}%
\pgfsetfillcolor{currentfill}%
\pgfsetlinewidth{0.000000pt}%
\definecolor{currentstroke}{rgb}{0.000000,0.000000,0.000000}%
\pgfsetstrokecolor{currentstroke}%
\pgfsetdash{}{0pt}%
\pgfpathmoveto{\pgfqpoint{3.216180in}{0.528000in}}%
\pgfpathlineto{\pgfqpoint{3.150429in}{0.578689in}}%
\pgfpathlineto{\pgfqpoint{3.073746in}{0.640000in}}%
\pgfpathlineto{\pgfqpoint{3.015880in}{0.687985in}}%
\pgfpathlineto{\pgfqpoint{2.983434in}{0.714667in}}%
\pgfpathlineto{\pgfqpoint{2.939432in}{0.752000in}}%
\pgfpathlineto{\pgfqpoint{2.853595in}{0.826667in}}%
\pgfpathlineto{\pgfqpoint{2.804040in}{0.870871in}}%
\pgfpathlineto{\pgfqpoint{2.723879in}{0.944195in}}%
\pgfpathlineto{\pgfqpoint{2.611601in}{1.050667in}}%
\pgfpathlineto{\pgfqpoint{2.568658in}{1.092753in}}%
\pgfpathlineto{\pgfqpoint{2.535525in}{1.125333in}}%
\pgfpathlineto{\pgfqpoint{2.461489in}{1.200000in}}%
\pgfpathlineto{\pgfqpoint{2.389361in}{1.274667in}}%
\pgfpathlineto{\pgfqpoint{2.339912in}{1.327687in}}%
\pgfpathlineto{\pgfqpoint{2.319021in}{1.349333in}}%
\pgfpathlineto{\pgfqpoint{2.284560in}{1.386667in}}%
\pgfpathlineto{\pgfqpoint{2.217298in}{1.461333in}}%
\pgfpathlineto{\pgfqpoint{2.174837in}{1.509928in}}%
\pgfpathlineto{\pgfqpoint{2.151608in}{1.536000in}}%
\pgfpathlineto{\pgfqpoint{2.120845in}{1.571636in}}%
\pgfpathlineto{\pgfqpoint{2.119323in}{1.573333in}}%
\pgfpathlineto{\pgfqpoint{2.085477in}{1.613360in}}%
\pgfpathlineto{\pgfqpoint{2.056466in}{1.648000in}}%
\pgfpathlineto{\pgfqpoint{2.015885in}{1.697872in}}%
\pgfpathlineto{\pgfqpoint{1.995151in}{1.722667in}}%
\pgfpathlineto{\pgfqpoint{1.922263in}{1.814581in}}%
\pgfpathlineto{\pgfqpoint{1.842101in}{1.919763in}}%
\pgfpathlineto{\pgfqpoint{1.761939in}{2.030181in}}%
\pgfpathlineto{\pgfqpoint{1.742044in}{2.058667in}}%
\pgfpathlineto{\pgfqpoint{1.681778in}{2.147080in}}%
\pgfpathlineto{\pgfqpoint{1.618350in}{2.245333in}}%
\pgfpathlineto{\pgfqpoint{1.601616in}{2.272135in}}%
\pgfpathlineto{\pgfqpoint{1.561535in}{2.338784in}}%
\pgfpathlineto{\pgfqpoint{1.481374in}{2.483357in}}%
\pgfpathlineto{\pgfqpoint{1.433095in}{2.581333in}}%
\pgfpathlineto{\pgfqpoint{1.412308in}{2.629002in}}%
\pgfpathlineto{\pgfqpoint{1.399222in}{2.657854in}}%
\pgfpathlineto{\pgfqpoint{1.371172in}{2.730667in}}%
\pgfpathlineto{\pgfqpoint{1.369018in}{2.738012in}}%
\pgfpathlineto{\pgfqpoint{1.358146in}{2.768000in}}%
\pgfpathlineto{\pgfqpoint{1.329188in}{2.872421in}}%
\pgfpathlineto{\pgfqpoint{1.320577in}{2.917333in}}%
\pgfpathlineto{\pgfqpoint{1.315855in}{2.954667in}}%
\pgfpathlineto{\pgfqpoint{1.313081in}{2.999423in}}%
\pgfpathlineto{\pgfqpoint{1.314082in}{3.035824in}}%
\pgfpathlineto{\pgfqpoint{1.318153in}{3.069366in}}%
\pgfpathlineto{\pgfqpoint{1.326823in}{3.104000in}}%
\pgfpathlineto{\pgfqpoint{1.331477in}{3.113712in}}%
\pgfpathlineto{\pgfqpoint{1.342926in}{3.141333in}}%
\pgfpathlineto{\pgfqpoint{1.365416in}{3.174675in}}%
\pgfpathlineto{\pgfqpoint{1.369524in}{3.178667in}}%
\pgfpathlineto{\pgfqpoint{1.401212in}{3.205169in}}%
\pgfpathlineto{\pgfqpoint{1.434129in}{3.222673in}}%
\pgfpathlineto{\pgfqpoint{1.441293in}{3.225150in}}%
\pgfpathlineto{\pgfqpoint{1.454217in}{3.228038in}}%
\pgfpathlineto{\pgfqpoint{1.481374in}{3.235895in}}%
\pgfpathlineto{\pgfqpoint{1.497085in}{3.238699in}}%
\pgfpathlineto{\pgfqpoint{1.521455in}{3.240748in}}%
\pgfpathlineto{\pgfqpoint{1.575210in}{3.240596in}}%
\pgfpathlineto{\pgfqpoint{1.601616in}{3.237916in}}%
\pgfpathlineto{\pgfqpoint{1.621190in}{3.234232in}}%
\pgfpathlineto{\pgfqpoint{1.641697in}{3.231930in}}%
\pgfpathlineto{\pgfqpoint{1.655109in}{3.228493in}}%
\pgfpathlineto{\pgfqpoint{1.681778in}{3.223670in}}%
\pgfpathlineto{\pgfqpoint{1.725850in}{3.212282in}}%
\pgfpathlineto{\pgfqpoint{1.802020in}{3.187247in}}%
\pgfpathlineto{\pgfqpoint{1.824713in}{3.178667in}}%
\pgfpathlineto{\pgfqpoint{1.842101in}{3.172065in}}%
\pgfpathlineto{\pgfqpoint{1.864390in}{3.162095in}}%
\pgfpathlineto{\pgfqpoint{1.882182in}{3.155454in}}%
\pgfpathlineto{\pgfqpoint{1.922263in}{3.137912in}}%
\pgfpathlineto{\pgfqpoint{1.972511in}{3.113471in}}%
\pgfpathlineto{\pgfqpoint{2.002424in}{3.099254in}}%
\pgfpathlineto{\pgfqpoint{2.024363in}{3.087102in}}%
\pgfpathlineto{\pgfqpoint{2.050222in}{3.073854in}}%
\pgfpathlineto{\pgfqpoint{2.082586in}{3.056707in}}%
\pgfpathlineto{\pgfqpoint{2.131086in}{3.029333in}}%
\pgfpathlineto{\pgfqpoint{2.202828in}{2.986783in}}%
\pgfpathlineto{\pgfqpoint{2.223055in}{2.973507in}}%
\pgfpathlineto{\pgfqpoint{2.254225in}{2.954667in}}%
\pgfpathlineto{\pgfqpoint{2.323071in}{2.910069in}}%
\pgfpathlineto{\pgfqpoint{2.341407in}{2.897080in}}%
\pgfpathlineto{\pgfqpoint{2.367824in}{2.880000in}}%
\pgfpathlineto{\pgfqpoint{2.474029in}{2.805333in}}%
\pgfpathlineto{\pgfqpoint{2.643717in}{2.677467in}}%
\pgfpathlineto{\pgfqpoint{2.723879in}{2.613660in}}%
\pgfpathlineto{\pgfqpoint{2.808284in}{2.544000in}}%
\pgfpathlineto{\pgfqpoint{2.847952in}{2.510235in}}%
\pgfpathlineto{\pgfqpoint{2.884202in}{2.479229in}}%
\pgfpathlineto{\pgfqpoint{2.930903in}{2.438166in}}%
\pgfpathlineto{\pgfqpoint{2.964364in}{2.408784in}}%
\pgfpathlineto{\pgfqpoint{3.062067in}{2.320000in}}%
\pgfpathlineto{\pgfqpoint{3.102145in}{2.282667in}}%
\pgfpathlineto{\pgfqpoint{3.180592in}{2.208000in}}%
\pgfpathlineto{\pgfqpoint{3.256887in}{2.133333in}}%
\pgfpathlineto{\pgfqpoint{3.308725in}{2.080756in}}%
\pgfpathlineto{\pgfqpoint{3.331171in}{2.058667in}}%
\pgfpathlineto{\pgfqpoint{3.365172in}{2.023846in}}%
\pgfpathlineto{\pgfqpoint{3.445333in}{1.939719in}}%
\pgfpathlineto{\pgfqpoint{3.542090in}{1.834667in}}%
\pgfpathlineto{\pgfqpoint{3.575701in}{1.797333in}}%
\pgfpathlineto{\pgfqpoint{3.645737in}{1.717863in}}%
\pgfpathlineto{\pgfqpoint{3.736803in}{1.610667in}}%
\pgfpathlineto{\pgfqpoint{3.806061in}{1.526263in}}%
\pgfpathlineto{\pgfqpoint{3.828118in}{1.498667in}}%
\pgfpathlineto{\pgfqpoint{3.889105in}{1.421315in}}%
\pgfpathlineto{\pgfqpoint{3.982928in}{1.296589in}}%
\pgfpathlineto{\pgfqpoint{4.062136in}{1.185479in}}%
\pgfpathlineto{\pgfqpoint{4.128047in}{1.088000in}}%
\pgfpathlineto{\pgfqpoint{4.157367in}{1.041892in}}%
\pgfpathlineto{\pgfqpoint{4.175941in}{1.013333in}}%
\pgfpathlineto{\pgfqpoint{4.186870in}{0.994706in}}%
\pgfpathlineto{\pgfqpoint{4.206869in}{0.963320in}}%
\pgfpathlineto{\pgfqpoint{4.246949in}{0.895999in}}%
\pgfpathlineto{\pgfqpoint{4.287030in}{0.824947in}}%
\pgfpathlineto{\pgfqpoint{4.327111in}{0.748886in}}%
\pgfpathlineto{\pgfqpoint{4.367192in}{0.665881in}}%
\pgfpathlineto{\pgfqpoint{4.394979in}{0.602667in}}%
\pgfpathlineto{\pgfqpoint{4.398059in}{0.594085in}}%
\pgfpathlineto{\pgfqpoint{4.410331in}{0.565333in}}%
\pgfpathlineto{\pgfqpoint{4.424242in}{0.528000in}}%
\pgfpathlineto{\pgfqpoint{4.439933in}{0.528000in}}%
\pgfpathlineto{\pgfqpoint{4.425253in}{0.565333in}}%
\pgfpathlineto{\pgfqpoint{4.407273in}{0.608705in}}%
\pgfpathlineto{\pgfqpoint{4.367192in}{0.695635in}}%
\pgfpathlineto{\pgfqpoint{4.338973in}{0.752000in}}%
\pgfpathlineto{\pgfqpoint{4.299076in}{0.826667in}}%
\pgfpathlineto{\pgfqpoint{4.256586in}{0.901333in}}%
\pgfpathlineto{\pgfqpoint{4.206869in}{0.983795in}}%
\pgfpathlineto{\pgfqpoint{4.179954in}{1.025597in}}%
\pgfpathlineto{\pgfqpoint{4.164587in}{1.050667in}}%
\pgfpathlineto{\pgfqpoint{4.140012in}{1.088000in}}%
\pgfpathlineto{\pgfqpoint{4.086626in}{1.167285in}}%
\pgfpathlineto{\pgfqpoint{4.006465in}{1.280356in}}%
\pgfpathlineto{\pgfqpoint{3.926303in}{1.387952in}}%
\pgfpathlineto{\pgfqpoint{3.839736in}{1.498667in}}%
\pgfpathlineto{\pgfqpoint{3.809755in}{1.536000in}}%
\pgfpathlineto{\pgfqpoint{3.748229in}{1.610667in}}%
\pgfpathlineto{\pgfqpoint{3.716962in}{1.648000in}}%
\pgfpathlineto{\pgfqpoint{3.645737in}{1.731025in}}%
\pgfpathlineto{\pgfqpoint{3.553691in}{1.834667in}}%
\pgfpathlineto{\pgfqpoint{3.485414in}{1.909423in}}%
\pgfpathlineto{\pgfqpoint{3.379253in}{2.021333in}}%
\pgfpathlineto{\pgfqpoint{3.285010in}{2.117319in}}%
\pgfpathlineto{\pgfqpoint{3.237723in}{2.163955in}}%
\pgfpathlineto{\pgfqpoint{3.204848in}{2.196441in}}%
\pgfpathlineto{\pgfqpoint{3.114746in}{2.282667in}}%
\pgfpathlineto{\pgfqpoint{3.034293in}{2.357333in}}%
\pgfpathlineto{\pgfqpoint{2.993156in}{2.394667in}}%
\pgfpathlineto{\pgfqpoint{2.908948in}{2.469333in}}%
\pgfpathlineto{\pgfqpoint{2.854307in}{2.516154in}}%
\pgfpathlineto{\pgfqpoint{2.821999in}{2.544000in}}%
\pgfpathlineto{\pgfqpoint{2.732095in}{2.618667in}}%
\pgfpathlineto{\pgfqpoint{2.541049in}{2.768000in}}%
\pgfpathlineto{\pgfqpoint{2.457273in}{2.829664in}}%
\pgfpathlineto{\pgfqpoint{2.385613in}{2.880000in}}%
\pgfpathlineto{\pgfqpoint{2.323071in}{2.922624in}}%
\pgfpathlineto{\pgfqpoint{2.302474in}{2.935482in}}%
\pgfpathlineto{\pgfqpoint{2.274160in}{2.954667in}}%
\pgfpathlineto{\pgfqpoint{2.207661in}{2.996501in}}%
\pgfpathlineto{\pgfqpoint{2.178564in}{3.014601in}}%
\pgfpathlineto{\pgfqpoint{2.122667in}{3.047491in}}%
\pgfpathlineto{\pgfqpoint{2.082586in}{3.070401in}}%
\pgfpathlineto{\pgfqpoint{2.034160in}{3.096227in}}%
\pgfpathlineto{\pgfqpoint{2.002424in}{3.113295in}}%
\pgfpathlineto{\pgfqpoint{1.982375in}{3.122658in}}%
\pgfpathlineto{\pgfqpoint{1.945828in}{3.141333in}}%
\pgfpathlineto{\pgfqpoint{1.842101in}{3.187609in}}%
\pgfpathlineto{\pgfqpoint{1.820081in}{3.195490in}}%
\pgfpathlineto{\pgfqpoint{1.802020in}{3.203270in}}%
\pgfpathlineto{\pgfqpoint{1.761939in}{3.217827in}}%
\pgfpathlineto{\pgfqpoint{1.671278in}{3.243553in}}%
\pgfpathlineto{\pgfqpoint{1.629150in}{3.253333in}}%
\pgfpathlineto{\pgfqpoint{1.601616in}{3.258140in}}%
\pgfpathlineto{\pgfqpoint{1.550702in}{3.263424in}}%
\pgfpathlineto{\pgfqpoint{1.521455in}{3.263889in}}%
\pgfpathlineto{\pgfqpoint{1.473564in}{3.260607in}}%
\pgfpathlineto{\pgfqpoint{1.438807in}{3.253333in}}%
\pgfpathlineto{\pgfqpoint{1.411389in}{3.243854in}}%
\pgfpathlineto{\pgfqpoint{1.385482in}{3.230652in}}%
\pgfpathlineto{\pgfqpoint{1.361131in}{3.212979in}}%
\pgfpathlineto{\pgfqpoint{1.343370in}{3.195210in}}%
\pgfpathlineto{\pgfqpoint{1.331025in}{3.178667in}}%
\pgfpathlineto{\pgfqpoint{1.321051in}{3.162197in}}%
\pgfpathlineto{\pgfqpoint{1.311168in}{3.141333in}}%
\pgfpathlineto{\pgfqpoint{1.307991in}{3.129169in}}%
\pgfpathlineto{\pgfqpoint{1.299634in}{3.104000in}}%
\pgfpathlineto{\pgfqpoint{1.296321in}{3.089701in}}%
\pgfpathlineto{\pgfqpoint{1.292139in}{3.056263in}}%
\pgfpathlineto{\pgfqpoint{1.291025in}{3.019967in}}%
\pgfpathlineto{\pgfqpoint{1.292411in}{2.992000in}}%
\pgfpathlineto{\pgfqpoint{1.302095in}{2.917333in}}%
\pgfpathlineto{\pgfqpoint{1.305711in}{2.903046in}}%
\pgfpathlineto{\pgfqpoint{1.309793in}{2.880000in}}%
\pgfpathlineto{\pgfqpoint{1.321051in}{2.835580in}}%
\pgfpathlineto{\pgfqpoint{1.330020in}{2.805333in}}%
\pgfpathlineto{\pgfqpoint{1.342285in}{2.768000in}}%
\pgfpathlineto{\pgfqpoint{1.361131in}{2.715932in}}%
\pgfpathlineto{\pgfqpoint{1.401524in}{2.618667in}}%
\pgfpathlineto{\pgfqpoint{1.419023in}{2.581333in}}%
\pgfpathlineto{\pgfqpoint{1.441293in}{2.535205in}}%
\pgfpathlineto{\pgfqpoint{1.481374in}{2.457920in}}%
\pgfpathlineto{\pgfqpoint{1.561535in}{2.317128in}}%
\pgfpathlineto{\pgfqpoint{1.582698in}{2.282667in}}%
\pgfpathlineto{\pgfqpoint{1.629710in}{2.208000in}}%
\pgfpathlineto{\pgfqpoint{1.681778in}{2.128869in}}%
\pgfpathlineto{\pgfqpoint{1.704289in}{2.096000in}}%
\pgfpathlineto{\pgfqpoint{1.761939in}{2.013524in}}%
\pgfpathlineto{\pgfqpoint{1.842101in}{1.904091in}}%
\pgfpathlineto{\pgfqpoint{1.866499in}{1.872000in}}%
\pgfpathlineto{\pgfqpoint{1.928918in}{1.791134in}}%
\pgfpathlineto{\pgfqpoint{2.014126in}{1.685333in}}%
\pgfpathlineto{\pgfqpoint{2.043843in}{1.649246in}}%
\pgfpathlineto{\pgfqpoint{2.044827in}{1.648000in}}%
\pgfpathlineto{\pgfqpoint{2.076185in}{1.610667in}}%
\pgfpathlineto{\pgfqpoint{2.140162in}{1.536000in}}%
\pgfpathlineto{\pgfqpoint{2.219344in}{1.445949in}}%
\pgfpathlineto{\pgfqpoint{2.307522in}{1.349333in}}%
\pgfpathlineto{\pgfqpoint{2.342353in}{1.312000in}}%
\pgfpathlineto{\pgfqpoint{2.413318in}{1.237333in}}%
\pgfpathlineto{\pgfqpoint{2.466028in}{1.183824in}}%
\pgfpathlineto{\pgfqpoint{2.486119in}{1.162667in}}%
\pgfpathlineto{\pgfqpoint{2.563556in}{1.085493in}}%
\pgfpathlineto{\pgfqpoint{2.677351in}{0.976000in}}%
\pgfpathlineto{\pgfqpoint{2.723879in}{0.932453in}}%
\pgfpathlineto{\pgfqpoint{2.804040in}{0.859138in}}%
\pgfpathlineto{\pgfqpoint{2.884202in}{0.787911in}}%
\pgfpathlineto{\pgfqpoint{2.904568in}{0.770970in}}%
\pgfpathlineto{\pgfqpoint{2.925588in}{0.752000in}}%
\pgfpathlineto{\pgfqpoint{3.013989in}{0.677333in}}%
\pgfpathlineto{\pgfqpoint{3.105531in}{0.602667in}}%
\pgfpathlineto{\pgfqpoint{3.152560in}{0.565333in}}%
\pgfpathlineto{\pgfqpoint{3.200474in}{0.528000in}}%
\pgfpathlineto{\pgfqpoint{3.204848in}{0.528000in}}%
\pgfpathlineto{\pgfqpoint{3.204848in}{0.528000in}}%
\pgfusepath{fill}%
\end{pgfscope}%
\begin{pgfscope}%
\pgfpathrectangle{\pgfqpoint{0.800000in}{0.528000in}}{\pgfqpoint{3.968000in}{3.696000in}}%
\pgfusepath{clip}%
\pgfsetbuttcap%
\pgfsetroundjoin%
\definecolor{currentfill}{rgb}{0.278791,0.062145,0.386592}%
\pgfsetfillcolor{currentfill}%
\pgfsetlinewidth{0.000000pt}%
\definecolor{currentstroke}{rgb}{0.000000,0.000000,0.000000}%
\pgfsetstrokecolor{currentstroke}%
\pgfsetdash{}{0pt}%
\pgfpathmoveto{\pgfqpoint{3.200474in}{0.528000in}}%
\pgfpathlineto{\pgfqpoint{3.164768in}{0.555724in}}%
\pgfpathlineto{\pgfqpoint{3.059351in}{0.640000in}}%
\pgfpathlineto{\pgfqpoint{3.009040in}{0.681614in}}%
\pgfpathlineto{\pgfqpoint{3.004444in}{0.685245in}}%
\pgfpathlineto{\pgfqpoint{2.924283in}{0.753117in}}%
\pgfpathlineto{\pgfqpoint{2.904568in}{0.770970in}}%
\pgfpathlineto{\pgfqpoint{2.882585in}{0.789333in}}%
\pgfpathlineto{\pgfqpoint{2.798672in}{0.864000in}}%
\pgfpathlineto{\pgfqpoint{2.717212in}{0.938667in}}%
\pgfpathlineto{\pgfqpoint{2.638042in}{1.013333in}}%
\pgfpathlineto{\pgfqpoint{2.561012in}{1.088000in}}%
\pgfpathlineto{\pgfqpoint{2.483394in}{1.165415in}}%
\pgfpathlineto{\pgfqpoint{2.427849in}{1.222929in}}%
\pgfpathlineto{\pgfqpoint{2.403232in}{1.247797in}}%
\pgfpathlineto{\pgfqpoint{2.307522in}{1.349333in}}%
\pgfpathlineto{\pgfqpoint{2.259573in}{1.402188in}}%
\pgfpathlineto{\pgfqpoint{2.219344in}{1.445949in}}%
\pgfpathlineto{\pgfqpoint{2.140162in}{1.536000in}}%
\pgfpathlineto{\pgfqpoint{2.097039in}{1.586796in}}%
\pgfpathlineto{\pgfqpoint{2.076185in}{1.610667in}}%
\pgfpathlineto{\pgfqpoint{2.043843in}{1.649246in}}%
\pgfpathlineto{\pgfqpoint{2.014126in}{1.685333in}}%
\pgfpathlineto{\pgfqpoint{1.983756in}{1.722667in}}%
\pgfpathlineto{\pgfqpoint{1.922263in}{1.799616in}}%
\pgfpathlineto{\pgfqpoint{1.829035in}{1.921503in}}%
\pgfpathlineto{\pgfqpoint{1.756352in}{2.021333in}}%
\pgfpathlineto{\pgfqpoint{1.730057in}{2.058667in}}%
\pgfpathlineto{\pgfqpoint{1.673641in}{2.140912in}}%
\pgfpathlineto{\pgfqpoint{1.601616in}{2.252016in}}%
\pgfpathlineto{\pgfqpoint{1.559791in}{2.320000in}}%
\pgfpathlineto{\pgfqpoint{1.537902in}{2.357333in}}%
\pgfpathlineto{\pgfqpoint{1.495530in}{2.432000in}}%
\pgfpathlineto{\pgfqpoint{1.481374in}{2.457920in}}%
\pgfpathlineto{\pgfqpoint{1.455815in}{2.506667in}}%
\pgfpathlineto{\pgfqpoint{1.419023in}{2.581333in}}%
\pgfpathlineto{\pgfqpoint{1.414019in}{2.593263in}}%
\pgfpathlineto{\pgfqpoint{1.401212in}{2.619382in}}%
\pgfpathlineto{\pgfqpoint{1.367847in}{2.699589in}}%
\pgfpathlineto{\pgfqpoint{1.352339in}{2.738856in}}%
\pgfpathlineto{\pgfqpoint{1.330020in}{2.805333in}}%
\pgfpathlineto{\pgfqpoint{1.318222in}{2.845301in}}%
\pgfpathlineto{\pgfqpoint{1.309793in}{2.880000in}}%
\pgfpathlineto{\pgfqpoint{1.296179in}{2.954667in}}%
\pgfpathlineto{\pgfqpoint{1.295485in}{2.968187in}}%
\pgfpathlineto{\pgfqpoint{1.292411in}{2.992000in}}%
\pgfpathlineto{\pgfqpoint{1.291273in}{3.029333in}}%
\pgfpathlineto{\pgfqpoint{1.293396in}{3.066667in}}%
\pgfpathlineto{\pgfqpoint{1.296321in}{3.089701in}}%
\pgfpathlineto{\pgfqpoint{1.299634in}{3.104000in}}%
\pgfpathlineto{\pgfqpoint{1.311168in}{3.141333in}}%
\pgfpathlineto{\pgfqpoint{1.321051in}{3.162197in}}%
\pgfpathlineto{\pgfqpoint{1.331025in}{3.178667in}}%
\pgfpathlineto{\pgfqpoint{1.343370in}{3.195210in}}%
\pgfpathlineto{\pgfqpoint{1.364806in}{3.216000in}}%
\pgfpathlineto{\pgfqpoint{1.385482in}{3.230652in}}%
\pgfpathlineto{\pgfqpoint{1.411389in}{3.243854in}}%
\pgfpathlineto{\pgfqpoint{1.441293in}{3.254108in}}%
\pgfpathlineto{\pgfqpoint{1.481374in}{3.261432in}}%
\pgfpathlineto{\pgfqpoint{1.490114in}{3.261474in}}%
\pgfpathlineto{\pgfqpoint{1.521455in}{3.263889in}}%
\pgfpathlineto{\pgfqpoint{1.561535in}{3.262535in}}%
\pgfpathlineto{\pgfqpoint{1.569902in}{3.261126in}}%
\pgfpathlineto{\pgfqpoint{1.601616in}{3.258140in}}%
\pgfpathlineto{\pgfqpoint{1.641697in}{3.251132in}}%
\pgfpathlineto{\pgfqpoint{1.721859in}{3.230456in}}%
\pgfpathlineto{\pgfqpoint{1.732984in}{3.226362in}}%
\pgfpathlineto{\pgfqpoint{1.766987in}{3.216000in}}%
\pgfpathlineto{\pgfqpoint{1.802020in}{3.203270in}}%
\pgfpathlineto{\pgfqpoint{1.820081in}{3.195490in}}%
\pgfpathlineto{\pgfqpoint{1.842101in}{3.187609in}}%
\pgfpathlineto{\pgfqpoint{1.898352in}{3.163605in}}%
\pgfpathlineto{\pgfqpoint{1.962343in}{3.133446in}}%
\pgfpathlineto{\pgfqpoint{1.982375in}{3.122658in}}%
\pgfpathlineto{\pgfqpoint{2.020128in}{3.104000in}}%
\pgfpathlineto{\pgfqpoint{2.162747in}{3.024063in}}%
\pgfpathlineto{\pgfqpoint{2.215187in}{2.992000in}}%
\pgfpathlineto{\pgfqpoint{2.255290in}{2.966199in}}%
\pgfpathlineto{\pgfqpoint{2.282990in}{2.949010in}}%
\pgfpathlineto{\pgfqpoint{2.302474in}{2.935482in}}%
\pgfpathlineto{\pgfqpoint{2.330902in}{2.917333in}}%
\pgfpathlineto{\pgfqpoint{2.363152in}{2.895477in}}%
\pgfpathlineto{\pgfqpoint{2.457273in}{2.829664in}}%
\pgfpathlineto{\pgfqpoint{2.541049in}{2.768000in}}%
\pgfpathlineto{\pgfqpoint{2.590358in}{2.730667in}}%
\pgfpathlineto{\pgfqpoint{2.671078in}{2.667848in}}%
\pgfpathlineto{\pgfqpoint{2.732095in}{2.618667in}}%
\pgfpathlineto{\pgfqpoint{2.821999in}{2.544000in}}%
\pgfpathlineto{\pgfqpoint{2.875191in}{2.498273in}}%
\pgfpathlineto{\pgfqpoint{2.908948in}{2.469333in}}%
\pgfpathlineto{\pgfqpoint{2.951382in}{2.432000in}}%
\pgfpathlineto{\pgfqpoint{3.044525in}{2.347990in}}%
\pgfpathlineto{\pgfqpoint{3.084606in}{2.310927in}}%
\pgfpathlineto{\pgfqpoint{3.192907in}{2.208000in}}%
\pgfpathlineto{\pgfqpoint{3.268929in}{2.133333in}}%
\pgfpathlineto{\pgfqpoint{3.315050in}{2.086647in}}%
\pgfpathlineto{\pgfqpoint{3.342952in}{2.058667in}}%
\pgfpathlineto{\pgfqpoint{3.391265in}{2.008305in}}%
\pgfpathlineto{\pgfqpoint{3.415102in}{1.984000in}}%
\pgfpathlineto{\pgfqpoint{3.485497in}{1.909333in}}%
\pgfpathlineto{\pgfqpoint{3.525495in}{1.865784in}}%
\pgfpathlineto{\pgfqpoint{3.620280in}{1.760000in}}%
\pgfpathlineto{\pgfqpoint{3.685818in}{1.684780in}}%
\pgfpathlineto{\pgfqpoint{3.779156in}{1.573333in}}%
\pgfpathlineto{\pgfqpoint{3.806061in}{1.540549in}}%
\pgfpathlineto{\pgfqpoint{3.869221in}{1.461333in}}%
\pgfpathlineto{\pgfqpoint{3.929707in}{1.383496in}}%
\pgfpathlineto{\pgfqpoint{4.019318in}{1.262695in}}%
\pgfpathlineto{\pgfqpoint{4.095556in}{1.154349in}}%
\pgfpathlineto{\pgfqpoint{4.166788in}{1.047230in}}%
\pgfpathlineto{\pgfqpoint{4.194832in}{1.002122in}}%
\pgfpathlineto{\pgfqpoint{4.218367in}{0.965290in}}%
\pgfpathlineto{\pgfqpoint{4.278171in}{0.864000in}}%
\pgfpathlineto{\pgfqpoint{4.321801in}{0.784387in}}%
\pgfpathlineto{\pgfqpoint{4.338973in}{0.752000in}}%
\pgfpathlineto{\pgfqpoint{4.367192in}{0.695635in}}%
\pgfpathlineto{\pgfqpoint{4.411774in}{0.598474in}}%
\pgfpathlineto{\pgfqpoint{4.439933in}{0.528000in}}%
\pgfpathlineto{\pgfqpoint{4.455112in}{0.528000in}}%
\pgfpathlineto{\pgfqpoint{4.435796in}{0.576099in}}%
\pgfpathlineto{\pgfqpoint{4.407273in}{0.640826in}}%
\pgfpathlineto{\pgfqpoint{4.367192in}{0.723231in}}%
\pgfpathlineto{\pgfqpoint{4.327111in}{0.799376in}}%
\pgfpathlineto{\pgfqpoint{4.287030in}{0.870875in}}%
\pgfpathlineto{\pgfqpoint{4.246949in}{0.938830in}}%
\pgfpathlineto{\pgfqpoint{4.176530in}{1.050667in}}%
\pgfpathlineto{\pgfqpoint{4.126707in}{1.126035in}}%
\pgfpathlineto{\pgfqpoint{4.046545in}{1.240977in}}%
\pgfpathlineto{\pgfqpoint{3.966384in}{1.350188in}}%
\pgfpathlineto{\pgfqpoint{3.880716in}{1.461333in}}%
\pgfpathlineto{\pgfqpoint{3.851111in}{1.498667in}}%
\pgfpathlineto{\pgfqpoint{3.790463in}{1.573333in}}%
\pgfpathlineto{\pgfqpoint{3.744494in}{1.627987in}}%
\pgfpathlineto{\pgfqpoint{3.725899in}{1.650942in}}%
\pgfpathlineto{\pgfqpoint{3.631638in}{1.760000in}}%
\pgfpathlineto{\pgfqpoint{3.598657in}{1.797333in}}%
\pgfpathlineto{\pgfqpoint{3.525495in}{1.878283in}}%
\pgfpathlineto{\pgfqpoint{3.485414in}{1.921598in}}%
\pgfpathlineto{\pgfqpoint{3.390908in}{2.021333in}}%
\pgfpathlineto{\pgfqpoint{3.318092in}{2.096000in}}%
\pgfpathlineto{\pgfqpoint{3.243353in}{2.170667in}}%
\pgfpathlineto{\pgfqpoint{3.204848in}{2.208346in}}%
\pgfpathlineto{\pgfqpoint{3.087414in}{2.320000in}}%
\pgfpathlineto{\pgfqpoint{3.044525in}{2.359666in}}%
\pgfpathlineto{\pgfqpoint{3.005255in}{2.395422in}}%
\pgfpathlineto{\pgfqpoint{3.004444in}{2.396189in}}%
\pgfpathlineto{\pgfqpoint{2.922325in}{2.469333in}}%
\pgfpathlineto{\pgfqpoint{2.860662in}{2.522074in}}%
\pgfpathlineto{\pgfqpoint{2.835713in}{2.544000in}}%
\pgfpathlineto{\pgfqpoint{2.746165in}{2.618667in}}%
\pgfpathlineto{\pgfqpoint{2.691248in}{2.662939in}}%
\pgfpathlineto{\pgfqpoint{2.683798in}{2.669218in}}%
\pgfpathlineto{\pgfqpoint{2.603636in}{2.732365in}}%
\pgfpathlineto{\pgfqpoint{2.560503in}{2.765157in}}%
\pgfpathlineto{\pgfqpoint{2.523475in}{2.793016in}}%
\pgfpathlineto{\pgfqpoint{2.443313in}{2.851577in}}%
\pgfpathlineto{\pgfqpoint{2.425601in}{2.863502in}}%
\pgfpathlineto{\pgfqpoint{2.402746in}{2.880453in}}%
\pgfpathlineto{\pgfqpoint{2.293236in}{2.954667in}}%
\pgfpathlineto{\pgfqpoint{2.235495in}{2.992000in}}%
\pgfpathlineto{\pgfqpoint{2.167564in}{3.033820in}}%
\pgfpathlineto{\pgfqpoint{2.162747in}{3.036965in}}%
\pgfpathlineto{\pgfqpoint{2.142806in}{3.048092in}}%
\pgfpathlineto{\pgfqpoint{2.112289in}{3.066667in}}%
\pgfpathlineto{\pgfqpoint{2.046012in}{3.104000in}}%
\pgfpathlineto{\pgfqpoint{2.042505in}{3.105950in}}%
\pgfpathlineto{\pgfqpoint{1.992238in}{3.131846in}}%
\pgfpathlineto{\pgfqpoint{1.962343in}{3.147602in}}%
\pgfpathlineto{\pgfqpoint{1.939901in}{3.157763in}}%
\pgfpathlineto{\pgfqpoint{1.922263in}{3.166993in}}%
\pgfpathlineto{\pgfqpoint{1.868047in}{3.191833in}}%
\pgfpathlineto{\pgfqpoint{1.802020in}{3.219105in}}%
\pgfpathlineto{\pgfqpoint{1.681778in}{3.259237in}}%
\pgfpathlineto{\pgfqpoint{1.672714in}{3.261776in}}%
\pgfpathlineto{\pgfqpoint{1.619341in}{3.274157in}}%
\pgfpathlineto{\pgfqpoint{1.588536in}{3.278483in}}%
\pgfpathlineto{\pgfqpoint{1.561535in}{3.283028in}}%
\pgfpathlineto{\pgfqpoint{1.521455in}{3.285980in}}%
\pgfpathlineto{\pgfqpoint{1.515969in}{3.285557in}}%
\pgfpathlineto{\pgfqpoint{1.481374in}{3.285390in}}%
\pgfpathlineto{\pgfqpoint{1.441293in}{3.280278in}}%
\pgfpathlineto{\pgfqpoint{1.424536in}{3.275058in}}%
\pgfpathlineto{\pgfqpoint{1.401212in}{3.269264in}}%
\pgfpathlineto{\pgfqpoint{1.388970in}{3.264737in}}%
\pgfpathlineto{\pgfqpoint{1.361131in}{3.249927in}}%
\pgfpathlineto{\pgfqpoint{1.321051in}{3.215133in}}%
\pgfpathlineto{\pgfqpoint{1.305315in}{3.193323in}}%
\pgfpathlineto{\pgfqpoint{1.297522in}{3.178667in}}%
\pgfpathlineto{\pgfqpoint{1.280970in}{3.137054in}}%
\pgfpathlineto{\pgfqpoint{1.273882in}{3.104000in}}%
\pgfpathlineto{\pgfqpoint{1.273608in}{3.097143in}}%
\pgfpathlineto{\pgfqpoint{1.270099in}{3.066667in}}%
\pgfpathlineto{\pgfqpoint{1.269741in}{3.029333in}}%
\pgfpathlineto{\pgfqpoint{1.270882in}{3.019937in}}%
\pgfpathlineto{\pgfqpoint{1.273461in}{2.985006in}}%
\pgfpathlineto{\pgfqpoint{1.276843in}{2.954667in}}%
\pgfpathlineto{\pgfqpoint{1.283614in}{2.917333in}}%
\pgfpathlineto{\pgfqpoint{1.302501in}{2.842667in}}%
\pgfpathlineto{\pgfqpoint{1.307064in}{2.829639in}}%
\pgfpathlineto{\pgfqpoint{1.313798in}{2.805333in}}%
\pgfpathlineto{\pgfqpoint{1.328968in}{2.760625in}}%
\pgfpathlineto{\pgfqpoint{1.361131in}{2.678899in}}%
\pgfpathlineto{\pgfqpoint{1.387615in}{2.618667in}}%
\pgfpathlineto{\pgfqpoint{1.392006in}{2.610092in}}%
\pgfpathlineto{\pgfqpoint{1.407893in}{2.575110in}}%
\pgfpathlineto{\pgfqpoint{1.442197in}{2.506667in}}%
\pgfpathlineto{\pgfqpoint{1.462181in}{2.469333in}}%
\pgfpathlineto{\pgfqpoint{1.503623in}{2.394667in}}%
\pgfpathlineto{\pgfqpoint{1.510144in}{2.384131in}}%
\pgfpathlineto{\pgfqpoint{1.525109in}{2.357333in}}%
\pgfpathlineto{\pgfqpoint{1.547514in}{2.320000in}}%
\pgfpathlineto{\pgfqpoint{1.601616in}{2.232847in}}%
\pgfpathlineto{\pgfqpoint{1.667025in}{2.133333in}}%
\pgfpathlineto{\pgfqpoint{1.688441in}{2.102207in}}%
\pgfpathlineto{\pgfqpoint{1.711429in}{2.068381in}}%
\pgfpathlineto{\pgfqpoint{1.771635in}{1.984000in}}%
\pgfpathlineto{\pgfqpoint{1.826813in}{1.909333in}}%
\pgfpathlineto{\pgfqpoint{1.887055in}{1.830128in}}%
\pgfpathlineto{\pgfqpoint{1.972361in}{1.722667in}}%
\pgfpathlineto{\pgfqpoint{2.042505in}{1.637345in}}%
\pgfpathlineto{\pgfqpoint{2.090596in}{1.580794in}}%
\pgfpathlineto{\pgfqpoint{2.122667in}{1.542986in}}%
\pgfpathlineto{\pgfqpoint{2.162747in}{1.496906in}}%
\pgfpathlineto{\pgfqpoint{2.261728in}{1.386667in}}%
\pgfpathlineto{\pgfqpoint{2.330731in}{1.312000in}}%
\pgfpathlineto{\pgfqpoint{2.403232in}{1.235559in}}%
\pgfpathlineto{\pgfqpoint{2.459879in}{1.178097in}}%
\pgfpathlineto{\pgfqpoint{2.483394in}{1.153589in}}%
\pgfpathlineto{\pgfqpoint{2.587336in}{1.050667in}}%
\pgfpathlineto{\pgfqpoint{2.634896in}{1.005117in}}%
\pgfpathlineto{\pgfqpoint{2.665150in}{0.976000in}}%
\pgfpathlineto{\pgfqpoint{2.714561in}{0.929988in}}%
\pgfpathlineto{\pgfqpoint{2.745165in}{0.901333in}}%
\pgfpathlineto{\pgfqpoint{2.786046in}{0.864000in}}%
\pgfpathlineto{\pgfqpoint{2.884202in}{0.776539in}}%
\pgfpathlineto{\pgfqpoint{2.918688in}{0.746788in}}%
\pgfpathlineto{\pgfqpoint{2.924283in}{0.741730in}}%
\pgfpathlineto{\pgfqpoint{3.004444in}{0.673633in}}%
\pgfpathlineto{\pgfqpoint{3.090942in}{0.602667in}}%
\pgfpathlineto{\pgfqpoint{3.137773in}{0.565333in}}%
\pgfpathlineto{\pgfqpoint{3.185483in}{0.528000in}}%
\pgfpathlineto{\pgfqpoint{3.185483in}{0.528000in}}%
\pgfusepath{fill}%
\end{pgfscope}%
\begin{pgfscope}%
\pgfpathrectangle{\pgfqpoint{0.800000in}{0.528000in}}{\pgfqpoint{3.968000in}{3.696000in}}%
\pgfusepath{clip}%
\pgfsetbuttcap%
\pgfsetroundjoin%
\definecolor{currentfill}{rgb}{0.278791,0.062145,0.386592}%
\pgfsetfillcolor{currentfill}%
\pgfsetlinewidth{0.000000pt}%
\definecolor{currentstroke}{rgb}{0.000000,0.000000,0.000000}%
\pgfsetstrokecolor{currentstroke}%
\pgfsetdash{}{0pt}%
\pgfpathmoveto{\pgfqpoint{3.185483in}{0.528000in}}%
\pgfpathlineto{\pgfqpoint{3.137773in}{0.565333in}}%
\pgfpathlineto{\pgfqpoint{3.044525in}{0.640352in}}%
\pgfpathlineto{\pgfqpoint{2.955888in}{0.714667in}}%
\pgfpathlineto{\pgfqpoint{2.869658in}{0.789333in}}%
\pgfpathlineto{\pgfqpoint{2.786046in}{0.864000in}}%
\pgfpathlineto{\pgfqpoint{2.683798in}{0.958369in}}%
\pgfpathlineto{\pgfqpoint{2.634896in}{1.005117in}}%
\pgfpathlineto{\pgfqpoint{2.603636in}{1.034818in}}%
\pgfpathlineto{\pgfqpoint{2.556323in}{1.081264in}}%
\pgfpathlineto{\pgfqpoint{2.523475in}{1.113446in}}%
\pgfpathlineto{\pgfqpoint{2.478816in}{1.158403in}}%
\pgfpathlineto{\pgfqpoint{2.443313in}{1.194292in}}%
\pgfpathlineto{\pgfqpoint{2.401527in}{1.237333in}}%
\pgfpathlineto{\pgfqpoint{2.323071in}{1.320171in}}%
\pgfpathlineto{\pgfqpoint{2.282990in}{1.363428in}}%
\pgfpathlineto{\pgfqpoint{2.194329in}{1.461333in}}%
\pgfpathlineto{\pgfqpoint{2.122667in}{1.542986in}}%
\pgfpathlineto{\pgfqpoint{2.082586in}{1.589826in}}%
\pgfpathlineto{\pgfqpoint{2.002424in}{1.685561in}}%
\pgfpathlineto{\pgfqpoint{1.912804in}{1.797333in}}%
\pgfpathlineto{\pgfqpoint{1.842101in}{1.889013in}}%
\pgfpathlineto{\pgfqpoint{1.761939in}{1.997406in}}%
\pgfpathlineto{\pgfqpoint{1.735980in}{2.034487in}}%
\pgfpathlineto{\pgfqpoint{1.711429in}{2.068381in}}%
\pgfpathlineto{\pgfqpoint{1.641697in}{2.170897in}}%
\pgfpathlineto{\pgfqpoint{1.570281in}{2.282667in}}%
\pgfpathlineto{\pgfqpoint{1.561535in}{2.296907in}}%
\pgfpathlineto{\pgfqpoint{1.521455in}{2.363621in}}%
\pgfpathlineto{\pgfqpoint{1.503623in}{2.394667in}}%
\pgfpathlineto{\pgfqpoint{1.462181in}{2.469333in}}%
\pgfpathlineto{\pgfqpoint{1.455632in}{2.482690in}}%
\pgfpathlineto{\pgfqpoint{1.441293in}{2.508443in}}%
\pgfpathlineto{\pgfqpoint{1.423423in}{2.544000in}}%
\pgfpathlineto{\pgfqpoint{1.401212in}{2.589305in}}%
\pgfpathlineto{\pgfqpoint{1.351190in}{2.702593in}}%
\pgfpathlineto{\pgfqpoint{1.321051in}{2.783756in}}%
\pgfpathlineto{\pgfqpoint{1.313798in}{2.805333in}}%
\pgfpathlineto{\pgfqpoint{1.302501in}{2.842667in}}%
\pgfpathlineto{\pgfqpoint{1.292369in}{2.880000in}}%
\pgfpathlineto{\pgfqpoint{1.290755in}{2.889115in}}%
\pgfpathlineto{\pgfqpoint{1.280970in}{2.931764in}}%
\pgfpathlineto{\pgfqpoint{1.276843in}{2.954667in}}%
\pgfpathlineto{\pgfqpoint{1.272154in}{2.992000in}}%
\pgfpathlineto{\pgfqpoint{1.269267in}{3.040234in}}%
\pgfpathlineto{\pgfqpoint{1.270099in}{3.066667in}}%
\pgfpathlineto{\pgfqpoint{1.274649in}{3.109888in}}%
\pgfpathlineto{\pgfqpoint{1.282969in}{3.143196in}}%
\pgfpathlineto{\pgfqpoint{1.297522in}{3.178667in}}%
\pgfpathlineto{\pgfqpoint{1.305315in}{3.193323in}}%
\pgfpathlineto{\pgfqpoint{1.321788in}{3.216000in}}%
\pgfpathlineto{\pgfqpoint{1.363208in}{3.251399in}}%
\pgfpathlineto{\pgfqpoint{1.388970in}{3.264737in}}%
\pgfpathlineto{\pgfqpoint{1.401212in}{3.269264in}}%
\pgfpathlineto{\pgfqpoint{1.450463in}{3.282125in}}%
\pgfpathlineto{\pgfqpoint{1.481374in}{3.285390in}}%
\pgfpathlineto{\pgfqpoint{1.526574in}{3.285898in}}%
\pgfpathlineto{\pgfqpoint{1.561535in}{3.283028in}}%
\pgfpathlineto{\pgfqpoint{1.641697in}{3.269185in}}%
\pgfpathlineto{\pgfqpoint{1.686382in}{3.257622in}}%
\pgfpathlineto{\pgfqpoint{1.721859in}{3.247367in}}%
\pgfpathlineto{\pgfqpoint{1.775155in}{3.228310in}}%
\pgfpathlineto{\pgfqpoint{1.809664in}{3.216000in}}%
\pgfpathlineto{\pgfqpoint{1.922263in}{3.166993in}}%
\pgfpathlineto{\pgfqpoint{1.939901in}{3.157763in}}%
\pgfpathlineto{\pgfqpoint{1.974621in}{3.141333in}}%
\pgfpathlineto{\pgfqpoint{2.122667in}{3.060732in}}%
\pgfpathlineto{\pgfqpoint{2.142806in}{3.048092in}}%
\pgfpathlineto{\pgfqpoint{2.175233in}{3.029333in}}%
\pgfpathlineto{\pgfqpoint{2.242909in}{2.987353in}}%
\pgfpathlineto{\pgfqpoint{2.263188in}{2.973555in}}%
\pgfpathlineto{\pgfqpoint{2.293236in}{2.954667in}}%
\pgfpathlineto{\pgfqpoint{2.323071in}{2.934844in}}%
\pgfpathlineto{\pgfqpoint{2.403391in}{2.880000in}}%
\pgfpathlineto{\pgfqpoint{2.448576in}{2.847569in}}%
\pgfpathlineto{\pgfqpoint{2.483394in}{2.822546in}}%
\pgfpathlineto{\pgfqpoint{2.506785in}{2.805333in}}%
\pgfpathlineto{\pgfqpoint{2.563556in}{2.762983in}}%
\pgfpathlineto{\pgfqpoint{2.582207in}{2.748039in}}%
\pgfpathlineto{\pgfqpoint{2.605815in}{2.730667in}}%
\pgfpathlineto{\pgfqpoint{2.700217in}{2.656000in}}%
\pgfpathlineto{\pgfqpoint{2.791320in}{2.581333in}}%
\pgfpathlineto{\pgfqpoint{2.860662in}{2.522074in}}%
\pgfpathlineto{\pgfqpoint{2.884202in}{2.502515in}}%
\pgfpathlineto{\pgfqpoint{2.923315in}{2.468432in}}%
\pgfpathlineto{\pgfqpoint{2.924283in}{2.467624in}}%
\pgfpathlineto{\pgfqpoint{2.943503in}{2.449902in}}%
\pgfpathlineto{\pgfqpoint{2.964584in}{2.432000in}}%
\pgfpathlineto{\pgfqpoint{3.047056in}{2.357333in}}%
\pgfpathlineto{\pgfqpoint{3.087414in}{2.320000in}}%
\pgfpathlineto{\pgfqpoint{3.166470in}{2.245333in}}%
\pgfpathlineto{\pgfqpoint{3.244929in}{2.169119in}}%
\pgfpathlineto{\pgfqpoint{3.325091in}{2.088931in}}%
\pgfpathlineto{\pgfqpoint{3.365172in}{2.047976in}}%
\pgfpathlineto{\pgfqpoint{3.461921in}{1.946667in}}%
\pgfpathlineto{\pgfqpoint{3.531245in}{1.872000in}}%
\pgfpathlineto{\pgfqpoint{3.605657in}{1.789475in}}%
\pgfpathlineto{\pgfqpoint{3.645737in}{1.743953in}}%
\pgfpathlineto{\pgfqpoint{3.728389in}{1.648000in}}%
\pgfpathlineto{\pgfqpoint{3.759655in}{1.610667in}}%
\pgfpathlineto{\pgfqpoint{3.820945in}{1.536000in}}%
\pgfpathlineto{\pgfqpoint{3.846141in}{1.504870in}}%
\pgfpathlineto{\pgfqpoint{3.909776in}{1.424000in}}%
\pgfpathlineto{\pgfqpoint{3.967029in}{1.349333in}}%
\pgfpathlineto{\pgfqpoint{3.994693in}{1.312000in}}%
\pgfpathlineto{\pgfqpoint{4.049166in}{1.237333in}}%
\pgfpathlineto{\pgfqpoint{4.075498in}{1.200000in}}%
\pgfpathlineto{\pgfqpoint{4.128008in}{1.124121in}}%
\pgfpathlineto{\pgfqpoint{4.206869in}{1.003334in}}%
\pgfpathlineto{\pgfqpoint{4.269122in}{0.901333in}}%
\pgfpathlineto{\pgfqpoint{4.311995in}{0.826667in}}%
\pgfpathlineto{\pgfqpoint{4.352299in}{0.752000in}}%
\pgfpathlineto{\pgfqpoint{4.371572in}{0.714667in}}%
\pgfpathlineto{\pgfqpoint{4.389783in}{0.677333in}}%
\pgfpathlineto{\pgfqpoint{4.407664in}{0.640000in}}%
\pgfpathlineto{\pgfqpoint{4.455112in}{0.528000in}}%
\pgfpathlineto{\pgfqpoint{4.469831in}{0.528000in}}%
\pgfpathlineto{\pgfqpoint{4.447354in}{0.582188in}}%
\pgfpathlineto{\pgfqpoint{4.421252in}{0.640000in}}%
\pgfpathlineto{\pgfqpoint{4.416685in}{0.648767in}}%
\pgfpathlineto{\pgfqpoint{4.403544in}{0.677333in}}%
\pgfpathlineto{\pgfqpoint{4.378738in}{0.725421in}}%
\pgfpathlineto{\pgfqpoint{4.364150in}{0.754833in}}%
\pgfpathlineto{\pgfqpoint{4.322546in}{0.830919in}}%
\pgfpathlineto{\pgfqpoint{4.259090in}{0.938667in}}%
\pgfpathlineto{\pgfqpoint{4.246949in}{0.958465in}}%
\pgfpathlineto{\pgfqpoint{4.206869in}{1.022217in}}%
\pgfpathlineto{\pgfqpoint{4.188365in}{1.050667in}}%
\pgfpathlineto{\pgfqpoint{4.138697in}{1.125333in}}%
\pgfpathlineto{\pgfqpoint{4.086626in}{1.200851in}}%
\pgfpathlineto{\pgfqpoint{4.060490in}{1.237333in}}%
\pgfpathlineto{\pgfqpoint{4.006258in}{1.312000in}}%
\pgfpathlineto{\pgfqpoint{3.978170in}{1.349333in}}%
\pgfpathlineto{\pgfqpoint{3.921151in}{1.424000in}}%
\pgfpathlineto{\pgfqpoint{3.891934in}{1.461333in}}%
\pgfpathlineto{\pgfqpoint{3.832136in}{1.536000in}}%
\pgfpathlineto{\pgfqpoint{3.765980in}{1.616495in}}%
\pgfpathlineto{\pgfqpoint{3.725899in}{1.663953in}}%
\pgfpathlineto{\pgfqpoint{3.642996in}{1.760000in}}%
\pgfpathlineto{\pgfqpoint{3.605657in}{1.802128in}}%
\pgfpathlineto{\pgfqpoint{3.508079in}{1.909333in}}%
\pgfpathlineto{\pgfqpoint{3.459968in}{1.960299in}}%
\pgfpathlineto{\pgfqpoint{3.438163in}{1.984000in}}%
\pgfpathlineto{\pgfqpoint{3.365172in}{2.059980in}}%
\pgfpathlineto{\pgfqpoint{3.325091in}{2.100744in}}%
\pgfpathlineto{\pgfqpoint{3.216912in}{2.208000in}}%
\pgfpathlineto{\pgfqpoint{3.124687in}{2.296403in}}%
\pgfpathlineto{\pgfqpoint{3.018495in}{2.394667in}}%
\pgfpathlineto{\pgfqpoint{2.949803in}{2.455771in}}%
\pgfpathlineto{\pgfqpoint{2.924283in}{2.478880in}}%
\pgfpathlineto{\pgfqpoint{2.892446in}{2.506667in}}%
\pgfpathlineto{\pgfqpoint{2.804040in}{2.582268in}}%
\pgfpathlineto{\pgfqpoint{2.763960in}{2.615611in}}%
\pgfpathlineto{\pgfqpoint{2.667889in}{2.693333in}}%
\pgfpathlineto{\pgfqpoint{2.620454in}{2.730667in}}%
\pgfpathlineto{\pgfqpoint{2.523475in}{2.804870in}}%
\pgfpathlineto{\pgfqpoint{2.419942in}{2.880000in}}%
\pgfpathlineto{\pgfqpoint{2.242909in}{2.999712in}}%
\pgfpathlineto{\pgfqpoint{2.223543in}{3.011294in}}%
\pgfpathlineto{\pgfqpoint{2.195928in}{3.029333in}}%
\pgfpathlineto{\pgfqpoint{2.127096in}{3.070793in}}%
\pgfpathlineto{\pgfqpoint{2.122667in}{3.073624in}}%
\pgfpathlineto{\pgfqpoint{2.102088in}{3.084832in}}%
\pgfpathlineto{\pgfqpoint{2.069669in}{3.104000in}}%
\pgfpathlineto{\pgfqpoint{2.001544in}{3.141333in}}%
\pgfpathlineto{\pgfqpoint{1.922263in}{3.181296in}}%
\pgfpathlineto{\pgfqpoint{1.882182in}{3.199901in}}%
\pgfpathlineto{\pgfqpoint{1.838423in}{3.219426in}}%
\pgfpathlineto{\pgfqpoint{1.759013in}{3.250607in}}%
\pgfpathlineto{\pgfqpoint{1.721859in}{3.263620in}}%
\pgfpathlineto{\pgfqpoint{1.699514in}{3.269854in}}%
\pgfpathlineto{\pgfqpoint{1.681778in}{3.276077in}}%
\pgfpathlineto{\pgfqpoint{1.625609in}{3.290667in}}%
\pgfpathlineto{\pgfqpoint{1.601616in}{3.295979in}}%
\pgfpathlineto{\pgfqpoint{1.521455in}{3.306729in}}%
\pgfpathlineto{\pgfqpoint{1.502961in}{3.307893in}}%
\pgfpathlineto{\pgfqpoint{1.481374in}{3.307797in}}%
\pgfpathlineto{\pgfqpoint{1.427723in}{3.303306in}}%
\pgfpathlineto{\pgfqpoint{1.401212in}{3.297367in}}%
\pgfpathlineto{\pgfqpoint{1.366811in}{3.285377in}}%
\pgfpathlineto{\pgfqpoint{1.361131in}{3.282439in}}%
\pgfpathlineto{\pgfqpoint{1.317285in}{3.253333in}}%
\pgfpathlineto{\pgfqpoint{1.280970in}{3.208007in}}%
\pgfpathlineto{\pgfqpoint{1.267227in}{3.178667in}}%
\pgfpathlineto{\pgfqpoint{1.260594in}{3.160312in}}%
\pgfpathlineto{\pgfqpoint{1.253354in}{3.129723in}}%
\pgfpathlineto{\pgfqpoint{1.250005in}{3.104000in}}%
\pgfpathlineto{\pgfqpoint{1.247758in}{3.060268in}}%
\pgfpathlineto{\pgfqpoint{1.249103in}{3.029333in}}%
\pgfpathlineto{\pgfqpoint{1.258671in}{2.954667in}}%
\pgfpathlineto{\pgfqpoint{1.266275in}{2.917333in}}%
\pgfpathlineto{\pgfqpoint{1.276710in}{2.876033in}}%
\pgfpathlineto{\pgfqpoint{1.286022in}{2.842667in}}%
\pgfpathlineto{\pgfqpoint{1.298165in}{2.805333in}}%
\pgfpathlineto{\pgfqpoint{1.311219in}{2.768000in}}%
\pgfpathlineto{\pgfqpoint{1.327691in}{2.724482in}}%
\pgfpathlineto{\pgfqpoint{1.361131in}{2.646062in}}%
\pgfpathlineto{\pgfqpoint{1.391458in}{2.581333in}}%
\pgfpathlineto{\pgfqpoint{1.429268in}{2.506667in}}%
\pgfpathlineto{\pgfqpoint{1.469787in}{2.432000in}}%
\pgfpathlineto{\pgfqpoint{1.512782in}{2.357333in}}%
\pgfpathlineto{\pgfqpoint{1.561535in}{2.277111in}}%
\pgfpathlineto{\pgfqpoint{1.630238in}{2.170667in}}%
\pgfpathlineto{\pgfqpoint{1.641697in}{2.153495in}}%
\pgfpathlineto{\pgfqpoint{1.706840in}{2.058667in}}%
\pgfpathlineto{\pgfqpoint{1.745128in}{2.005674in}}%
\pgfpathlineto{\pgfqpoint{1.761939in}{1.981444in}}%
\pgfpathlineto{\pgfqpoint{1.847188in}{1.867262in}}%
\pgfpathlineto{\pgfqpoint{1.931152in}{1.760000in}}%
\pgfpathlineto{\pgfqpoint{1.961610in}{1.721983in}}%
\pgfpathlineto{\pgfqpoint{1.991618in}{1.685333in}}%
\pgfpathlineto{\pgfqpoint{2.053764in}{1.610667in}}%
\pgfpathlineto{\pgfqpoint{2.122667in}{1.530061in}}%
\pgfpathlineto{\pgfqpoint{2.216574in}{1.424000in}}%
\pgfpathlineto{\pgfqpoint{2.250350in}{1.386667in}}%
\pgfpathlineto{\pgfqpoint{2.323071in}{1.307963in}}%
\pgfpathlineto{\pgfqpoint{2.426325in}{1.200000in}}%
\pgfpathlineto{\pgfqpoint{2.472915in}{1.152906in}}%
\pgfpathlineto{\pgfqpoint{2.499905in}{1.125333in}}%
\pgfpathlineto{\pgfqpoint{2.550373in}{1.075721in}}%
\pgfpathlineto{\pgfqpoint{2.575404in}{1.050667in}}%
\pgfpathlineto{\pgfqpoint{2.628897in}{0.999529in}}%
\pgfpathlineto{\pgfqpoint{2.652950in}{0.976000in}}%
\pgfpathlineto{\pgfqpoint{2.708512in}{0.924353in}}%
\pgfpathlineto{\pgfqpoint{2.732684in}{0.901333in}}%
\pgfpathlineto{\pgfqpoint{2.773420in}{0.864000in}}%
\pgfpathlineto{\pgfqpoint{2.856732in}{0.789333in}}%
\pgfpathlineto{\pgfqpoint{2.942646in}{0.714667in}}%
\pgfpathlineto{\pgfqpoint{2.996169in}{0.669625in}}%
\pgfpathlineto{\pgfqpoint{3.031359in}{0.640000in}}%
\pgfpathlineto{\pgfqpoint{3.102836in}{0.582314in}}%
\pgfpathlineto{\pgfqpoint{3.133965in}{0.556691in}}%
\pgfpathlineto{\pgfqpoint{3.170492in}{0.528000in}}%
\pgfpathlineto{\pgfqpoint{3.170492in}{0.528000in}}%
\pgfusepath{fill}%
\end{pgfscope}%
\begin{pgfscope}%
\pgfpathrectangle{\pgfqpoint{0.800000in}{0.528000in}}{\pgfqpoint{3.968000in}{3.696000in}}%
\pgfusepath{clip}%
\pgfsetbuttcap%
\pgfsetroundjoin%
\definecolor{currentfill}{rgb}{0.278791,0.062145,0.386592}%
\pgfsetfillcolor{currentfill}%
\pgfsetlinewidth{0.000000pt}%
\definecolor{currentstroke}{rgb}{0.000000,0.000000,0.000000}%
\pgfsetstrokecolor{currentstroke}%
\pgfsetdash{}{0pt}%
\pgfpathmoveto{\pgfqpoint{3.170492in}{0.528000in}}%
\pgfpathlineto{\pgfqpoint{2.964364in}{0.696123in}}%
\pgfpathlineto{\pgfqpoint{2.912510in}{0.741034in}}%
\pgfpathlineto{\pgfqpoint{2.884202in}{0.765168in}}%
\pgfpathlineto{\pgfqpoint{2.773420in}{0.864000in}}%
\pgfpathlineto{\pgfqpoint{2.683798in}{0.946834in}}%
\pgfpathlineto{\pgfqpoint{2.628897in}{0.999529in}}%
\pgfpathlineto{\pgfqpoint{2.603636in}{1.023215in}}%
\pgfpathlineto{\pgfqpoint{2.550373in}{1.075721in}}%
\pgfpathlineto{\pgfqpoint{2.523475in}{1.101776in}}%
\pgfpathlineto{\pgfqpoint{2.472915in}{1.152906in}}%
\pgfpathlineto{\pgfqpoint{2.443313in}{1.182553in}}%
\pgfpathlineto{\pgfqpoint{2.396495in}{1.231058in}}%
\pgfpathlineto{\pgfqpoint{2.363152in}{1.265587in}}%
\pgfpathlineto{\pgfqpoint{2.302801in}{1.330453in}}%
\pgfpathlineto{\pgfqpoint{2.282990in}{1.350992in}}%
\pgfpathlineto{\pgfqpoint{2.183184in}{1.461333in}}%
\pgfpathlineto{\pgfqpoint{2.117516in}{1.536000in}}%
\pgfpathlineto{\pgfqpoint{2.082586in}{1.576560in}}%
\pgfpathlineto{\pgfqpoint{1.991618in}{1.685333in}}%
\pgfpathlineto{\pgfqpoint{1.922263in}{1.771165in}}%
\pgfpathlineto{\pgfqpoint{1.876642in}{1.829506in}}%
\pgfpathlineto{\pgfqpoint{1.847188in}{1.867262in}}%
\pgfpathlineto{\pgfqpoint{1.787638in}{1.946667in}}%
\pgfpathlineto{\pgfqpoint{1.721859in}{2.037368in}}%
\pgfpathlineto{\pgfqpoint{1.706840in}{2.058667in}}%
\pgfpathlineto{\pgfqpoint{1.655297in}{2.133333in}}%
\pgfpathlineto{\pgfqpoint{1.601616in}{2.214150in}}%
\pgfpathlineto{\pgfqpoint{1.581718in}{2.245333in}}%
\pgfpathlineto{\pgfqpoint{1.535237in}{2.320000in}}%
\pgfpathlineto{\pgfqpoint{1.521455in}{2.342815in}}%
\pgfpathlineto{\pgfqpoint{1.481374in}{2.411480in}}%
\pgfpathlineto{\pgfqpoint{1.441293in}{2.483977in}}%
\pgfpathlineto{\pgfqpoint{1.401212in}{2.561552in}}%
\pgfpathlineto{\pgfqpoint{1.373723in}{2.618667in}}%
\pgfpathlineto{\pgfqpoint{1.356621in}{2.656000in}}%
\pgfpathlineto{\pgfqpoint{1.340692in}{2.693333in}}%
\pgfpathlineto{\pgfqpoint{1.321051in}{2.741844in}}%
\pgfpathlineto{\pgfqpoint{1.298165in}{2.805333in}}%
\pgfpathlineto{\pgfqpoint{1.294638in}{2.818065in}}%
\pgfpathlineto{\pgfqpoint{1.280970in}{2.860214in}}%
\pgfpathlineto{\pgfqpoint{1.266275in}{2.917333in}}%
\pgfpathlineto{\pgfqpoint{1.258671in}{2.954667in}}%
\pgfpathlineto{\pgfqpoint{1.256998in}{2.969672in}}%
\pgfpathlineto{\pgfqpoint{1.252828in}{2.992000in}}%
\pgfpathlineto{\pgfqpoint{1.252342in}{3.002668in}}%
\pgfpathlineto{\pgfqpoint{1.249103in}{3.029333in}}%
\pgfpathlineto{\pgfqpoint{1.247960in}{3.066667in}}%
\pgfpathlineto{\pgfqpoint{1.248809in}{3.074044in}}%
\pgfpathlineto{\pgfqpoint{1.250005in}{3.104000in}}%
\pgfpathlineto{\pgfqpoint{1.253354in}{3.129723in}}%
\pgfpathlineto{\pgfqpoint{1.260594in}{3.160312in}}%
\pgfpathlineto{\pgfqpoint{1.267227in}{3.178667in}}%
\pgfpathlineto{\pgfqpoint{1.285722in}{3.216000in}}%
\pgfpathlineto{\pgfqpoint{1.321051in}{3.256677in}}%
\pgfpathlineto{\pgfqpoint{1.366811in}{3.285377in}}%
\pgfpathlineto{\pgfqpoint{1.401212in}{3.297367in}}%
\pgfpathlineto{\pgfqpoint{1.427723in}{3.303306in}}%
\pgfpathlineto{\pgfqpoint{1.441293in}{3.305029in}}%
\pgfpathlineto{\pgfqpoint{1.457167in}{3.305453in}}%
\pgfpathlineto{\pgfqpoint{1.481374in}{3.307797in}}%
\pgfpathlineto{\pgfqpoint{1.502961in}{3.307893in}}%
\pgfpathlineto{\pgfqpoint{1.521455in}{3.306729in}}%
\pgfpathlineto{\pgfqpoint{1.536325in}{3.304518in}}%
\pgfpathlineto{\pgfqpoint{1.561535in}{3.302597in}}%
\pgfpathlineto{\pgfqpoint{1.571908in}{3.300328in}}%
\pgfpathlineto{\pgfqpoint{1.601616in}{3.295979in}}%
\pgfpathlineto{\pgfqpoint{1.646863in}{3.285855in}}%
\pgfpathlineto{\pgfqpoint{1.681778in}{3.276077in}}%
\pgfpathlineto{\pgfqpoint{1.699514in}{3.269854in}}%
\pgfpathlineto{\pgfqpoint{1.729596in}{3.260540in}}%
\pgfpathlineto{\pgfqpoint{1.761939in}{3.249744in}}%
\pgfpathlineto{\pgfqpoint{1.802020in}{3.234213in}}%
\pgfpathlineto{\pgfqpoint{1.846214in}{3.216000in}}%
\pgfpathlineto{\pgfqpoint{1.927562in}{3.178667in}}%
\pgfpathlineto{\pgfqpoint{1.962343in}{3.161348in}}%
\pgfpathlineto{\pgfqpoint{2.002424in}{3.140882in}}%
\pgfpathlineto{\pgfqpoint{2.042505in}{3.119103in}}%
\pgfpathlineto{\pgfqpoint{2.122667in}{3.073624in}}%
\pgfpathlineto{\pgfqpoint{2.151299in}{3.056003in}}%
\pgfpathlineto{\pgfqpoint{2.175549in}{3.041257in}}%
\pgfpathlineto{\pgfqpoint{2.202828in}{3.025102in}}%
\pgfpathlineto{\pgfqpoint{2.242909in}{2.999712in}}%
\pgfpathlineto{\pgfqpoint{2.323071in}{2.947064in}}%
\pgfpathlineto{\pgfqpoint{2.403232in}{2.891862in}}%
\pgfpathlineto{\pgfqpoint{2.432944in}{2.870342in}}%
\pgfpathlineto{\pgfqpoint{2.455536in}{2.854052in}}%
\pgfpathlineto{\pgfqpoint{2.483394in}{2.834365in}}%
\pgfpathlineto{\pgfqpoint{2.500410in}{2.821183in}}%
\pgfpathlineto{\pgfqpoint{2.526015in}{2.802967in}}%
\pgfpathlineto{\pgfqpoint{2.620454in}{2.730667in}}%
\pgfpathlineto{\pgfqpoint{2.714472in}{2.656000in}}%
\pgfpathlineto{\pgfqpoint{2.763960in}{2.615611in}}%
\pgfpathlineto{\pgfqpoint{2.782999in}{2.599068in}}%
\pgfpathlineto{\pgfqpoint{2.805146in}{2.581333in}}%
\pgfpathlineto{\pgfqpoint{2.892446in}{2.506667in}}%
\pgfpathlineto{\pgfqpoint{2.949803in}{2.455771in}}%
\pgfpathlineto{\pgfqpoint{2.977102in}{2.432000in}}%
\pgfpathlineto{\pgfqpoint{3.059292in}{2.357333in}}%
\pgfpathlineto{\pgfqpoint{3.099514in}{2.320000in}}%
\pgfpathlineto{\pgfqpoint{3.204848in}{2.219753in}}%
\pgfpathlineto{\pgfqpoint{3.292627in}{2.133333in}}%
\pgfpathlineto{\pgfqpoint{3.378410in}{2.046336in}}%
\pgfpathlineto{\pgfqpoint{3.445333in}{1.976452in}}%
\pgfpathlineto{\pgfqpoint{3.497337in}{1.920439in}}%
\pgfpathlineto{\pgfqpoint{3.525495in}{1.890496in}}%
\pgfpathlineto{\pgfqpoint{3.609929in}{1.797333in}}%
\pgfpathlineto{\pgfqpoint{3.645737in}{1.756880in}}%
\pgfpathlineto{\pgfqpoint{3.739401in}{1.648000in}}%
\pgfpathlineto{\pgfqpoint{3.785818in}{1.591811in}}%
\pgfpathlineto{\pgfqpoint{3.806061in}{1.568106in}}%
\pgfpathlineto{\pgfqpoint{3.891934in}{1.461333in}}%
\pgfpathlineto{\pgfqpoint{3.921151in}{1.424000in}}%
\pgfpathlineto{\pgfqpoint{3.978170in}{1.349333in}}%
\pgfpathlineto{\pgfqpoint{3.989464in}{1.333498in}}%
\pgfpathlineto{\pgfqpoint{4.006465in}{1.311720in}}%
\pgfpathlineto{\pgfqpoint{4.087229in}{1.200000in}}%
\pgfpathlineto{\pgfqpoint{4.113093in}{1.162667in}}%
\pgfpathlineto{\pgfqpoint{4.166788in}{1.083691in}}%
\pgfpathlineto{\pgfqpoint{4.236076in}{0.976000in}}%
\pgfpathlineto{\pgfqpoint{4.287030in}{0.892197in}}%
\pgfpathlineto{\pgfqpoint{4.303420in}{0.864000in}}%
\pgfpathlineto{\pgfqpoint{4.327111in}{0.822702in}}%
\pgfpathlineto{\pgfqpoint{4.345376in}{0.789333in}}%
\pgfpathlineto{\pgfqpoint{4.367192in}{0.748976in}}%
\pgfpathlineto{\pgfqpoint{4.438386in}{0.602667in}}%
\pgfpathlineto{\pgfqpoint{4.459481in}{0.554037in}}%
\pgfpathlineto{\pgfqpoint{4.469831in}{0.528000in}}%
\pgfpathlineto{\pgfqpoint{4.484550in}{0.528000in}}%
\pgfpathlineto{\pgfqpoint{4.468678in}{0.565333in}}%
\pgfpathlineto{\pgfqpoint{4.462073in}{0.579044in}}%
\pgfpathlineto{\pgfqpoint{4.447354in}{0.613359in}}%
\pgfpathlineto{\pgfqpoint{4.378329in}{0.752000in}}%
\pgfpathlineto{\pgfqpoint{4.367192in}{0.772706in}}%
\pgfpathlineto{\pgfqpoint{4.327111in}{0.844459in}}%
\pgfpathlineto{\pgfqpoint{4.287030in}{0.912610in}}%
\pgfpathlineto{\pgfqpoint{4.224291in}{1.013333in}}%
\pgfpathlineto{\pgfqpoint{4.166788in}{1.100919in}}%
\pgfpathlineto{\pgfqpoint{4.098436in}{1.200000in}}%
\pgfpathlineto{\pgfqpoint{4.041548in}{1.279321in}}%
\pgfpathlineto{\pgfqpoint{3.961057in}{1.386667in}}%
\pgfpathlineto{\pgfqpoint{3.929673in}{1.427139in}}%
\pgfpathlineto{\pgfqpoint{3.902899in}{1.461333in}}%
\pgfpathlineto{\pgfqpoint{3.834582in}{1.546767in}}%
\pgfpathlineto{\pgfqpoint{3.750414in}{1.648000in}}%
\pgfpathlineto{\pgfqpoint{3.685818in}{1.723666in}}%
\pgfpathlineto{\pgfqpoint{3.587426in}{1.834667in}}%
\pgfpathlineto{\pgfqpoint{3.553594in}{1.872000in}}%
\pgfpathlineto{\pgfqpoint{3.484740in}{1.946667in}}%
\pgfpathlineto{\pgfqpoint{3.445333in}{1.988389in}}%
\pgfpathlineto{\pgfqpoint{3.341109in}{2.096000in}}%
\pgfpathlineto{\pgfqpoint{3.244929in}{2.192065in}}%
\pgfpathlineto{\pgfqpoint{3.204848in}{2.231160in}}%
\pgfpathlineto{\pgfqpoint{3.111614in}{2.320000in}}%
\pgfpathlineto{\pgfqpoint{3.030870in}{2.394667in}}%
\pgfpathlineto{\pgfqpoint{2.976501in}{2.443306in}}%
\pgfpathlineto{\pgfqpoint{2.947758in}{2.469333in}}%
\pgfpathlineto{\pgfqpoint{2.862105in}{2.544000in}}%
\pgfpathlineto{\pgfqpoint{2.789410in}{2.605039in}}%
\pgfpathlineto{\pgfqpoint{2.763960in}{2.626799in}}%
\pgfpathlineto{\pgfqpoint{2.728452in}{2.656000in}}%
\pgfpathlineto{\pgfqpoint{2.682334in}{2.693333in}}%
\pgfpathlineto{\pgfqpoint{2.635093in}{2.730667in}}%
\pgfpathlineto{\pgfqpoint{2.537931in}{2.805333in}}%
\pgfpathlineto{\pgfqpoint{2.483394in}{2.846033in}}%
\pgfpathlineto{\pgfqpoint{2.440287in}{2.877182in}}%
\pgfpathlineto{\pgfqpoint{2.403232in}{2.903612in}}%
\pgfpathlineto{\pgfqpoint{2.305961in}{2.970603in}}%
\pgfpathlineto{\pgfqpoint{2.215518in}{3.029333in}}%
\pgfpathlineto{\pgfqpoint{2.082586in}{3.109706in}}%
\pgfpathlineto{\pgfqpoint{2.025817in}{3.141333in}}%
\pgfpathlineto{\pgfqpoint{1.955169in}{3.178667in}}%
\pgfpathlineto{\pgfqpoint{1.842101in}{3.232178in}}%
\pgfpathlineto{\pgfqpoint{1.791791in}{3.253333in}}%
\pgfpathlineto{\pgfqpoint{1.761939in}{3.265000in}}%
\pgfpathlineto{\pgfqpoint{1.741551in}{3.271676in}}%
\pgfpathlineto{\pgfqpoint{1.712646in}{3.282086in}}%
\pgfpathlineto{\pgfqpoint{1.681778in}{3.292782in}}%
\pgfpathlineto{\pgfqpoint{1.601616in}{3.313803in}}%
\pgfpathlineto{\pgfqpoint{1.588444in}{3.315731in}}%
\pgfpathlineto{\pgfqpoint{1.555201in}{3.322100in}}%
\pgfpathlineto{\pgfqpoint{1.521455in}{3.327118in}}%
\pgfpathlineto{\pgfqpoint{1.481374in}{3.329631in}}%
\pgfpathlineto{\pgfqpoint{1.434643in}{3.328000in}}%
\pgfpathlineto{\pgfqpoint{1.401212in}{3.323367in}}%
\pgfpathlineto{\pgfqpoint{1.393465in}{3.320784in}}%
\pgfpathlineto{\pgfqpoint{1.361131in}{3.311956in}}%
\pgfpathlineto{\pgfqpoint{1.345063in}{3.305634in}}%
\pgfpathlineto{\pgfqpoint{1.318176in}{3.290667in}}%
\pgfpathlineto{\pgfqpoint{1.297794in}{3.274996in}}%
\pgfpathlineto{\pgfqpoint{1.277165in}{3.253333in}}%
\pgfpathlineto{\pgfqpoint{1.262760in}{3.232961in}}%
\pgfpathlineto{\pgfqpoint{1.253868in}{3.216000in}}%
\pgfpathlineto{\pgfqpoint{1.238840in}{3.176758in}}%
\pgfpathlineto{\pgfqpoint{1.231201in}{3.141333in}}%
\pgfpathlineto{\pgfqpoint{1.227476in}{3.104000in}}%
\pgfpathlineto{\pgfqpoint{1.227104in}{3.066667in}}%
\pgfpathlineto{\pgfqpoint{1.228429in}{3.055061in}}%
\pgfpathlineto{\pgfqpoint{1.229458in}{3.029333in}}%
\pgfpathlineto{\pgfqpoint{1.234056in}{2.992000in}}%
\pgfpathlineto{\pgfqpoint{1.240889in}{2.953069in}}%
\pgfpathlineto{\pgfqpoint{1.251433in}{2.907512in}}%
\pgfpathlineto{\pgfqpoint{1.270284in}{2.842667in}}%
\pgfpathlineto{\pgfqpoint{1.283245in}{2.803214in}}%
\pgfpathlineto{\pgfqpoint{1.310880in}{2.730667in}}%
\pgfpathlineto{\pgfqpoint{1.313940in}{2.724043in}}%
\pgfpathlineto{\pgfqpoint{1.326328in}{2.693333in}}%
\pgfpathlineto{\pgfqpoint{1.342905in}{2.656000in}}%
\pgfpathlineto{\pgfqpoint{1.361131in}{2.616129in}}%
\pgfpathlineto{\pgfqpoint{1.378173in}{2.581333in}}%
\pgfpathlineto{\pgfqpoint{1.401212in}{2.535478in}}%
\pgfpathlineto{\pgfqpoint{1.481374in}{2.389755in}}%
\pgfpathlineto{\pgfqpoint{1.500642in}{2.357333in}}%
\pgfpathlineto{\pgfqpoint{1.546244in}{2.282667in}}%
\pgfpathlineto{\pgfqpoint{1.561535in}{2.258326in}}%
\pgfpathlineto{\pgfqpoint{1.601616in}{2.196280in}}%
\pgfpathlineto{\pgfqpoint{1.669340in}{2.096000in}}%
\pgfpathlineto{\pgfqpoint{1.721859in}{2.021194in}}%
\pgfpathlineto{\pgfqpoint{1.748958in}{1.984000in}}%
\pgfpathlineto{\pgfqpoint{1.809055in}{1.902781in}}%
\pgfpathlineto{\pgfqpoint{1.890482in}{1.797333in}}%
\pgfpathlineto{\pgfqpoint{1.920983in}{1.758808in}}%
\pgfpathlineto{\pgfqpoint{1.950155in}{1.722667in}}%
\pgfpathlineto{\pgfqpoint{1.980634in}{1.685333in}}%
\pgfpathlineto{\pgfqpoint{2.042734in}{1.610454in}}%
\pgfpathlineto{\pgfqpoint{2.139137in}{1.498667in}}%
\pgfpathlineto{\pgfqpoint{2.186330in}{1.445966in}}%
\pgfpathlineto{\pgfqpoint{2.205313in}{1.424000in}}%
\pgfpathlineto{\pgfqpoint{2.242909in}{1.382544in}}%
\pgfpathlineto{\pgfqpoint{2.343349in}{1.274667in}}%
\pgfpathlineto{\pgfqpoint{2.390641in}{1.225605in}}%
\pgfpathlineto{\pgfqpoint{2.414895in}{1.200000in}}%
\pgfpathlineto{\pgfqpoint{2.467013in}{1.147408in}}%
\pgfpathlineto{\pgfqpoint{2.488230in}{1.125333in}}%
\pgfpathlineto{\pgfqpoint{2.544423in}{1.070179in}}%
\pgfpathlineto{\pgfqpoint{2.563556in}{1.050588in}}%
\pgfpathlineto{\pgfqpoint{2.622898in}{0.993941in}}%
\pgfpathlineto{\pgfqpoint{2.643717in}{0.973304in}}%
\pgfpathlineto{\pgfqpoint{2.702463in}{0.918719in}}%
\pgfpathlineto{\pgfqpoint{2.723879in}{0.898087in}}%
\pgfpathlineto{\pgfqpoint{2.763960in}{0.861244in}}%
\pgfpathlineto{\pgfqpoint{2.849493in}{0.784330in}}%
\pgfpathlineto{\pgfqpoint{2.929404in}{0.714667in}}%
\pgfpathlineto{\pgfqpoint{2.989938in}{0.663822in}}%
\pgfpathlineto{\pgfqpoint{3.017786in}{0.640000in}}%
\pgfpathlineto{\pgfqpoint{3.063087in}{0.602667in}}%
\pgfpathlineto{\pgfqpoint{3.156051in}{0.528000in}}%
\pgfpathlineto{\pgfqpoint{3.164768in}{0.528000in}}%
\pgfpathlineto{\pgfqpoint{3.164768in}{0.528000in}}%
\pgfusepath{fill}%
\end{pgfscope}%
\begin{pgfscope}%
\pgfpathrectangle{\pgfqpoint{0.800000in}{0.528000in}}{\pgfqpoint{3.968000in}{3.696000in}}%
\pgfusepath{clip}%
\pgfsetbuttcap%
\pgfsetroundjoin%
\definecolor{currentfill}{rgb}{0.278791,0.062145,0.386592}%
\pgfsetfillcolor{currentfill}%
\pgfsetlinewidth{0.000000pt}%
\definecolor{currentstroke}{rgb}{0.000000,0.000000,0.000000}%
\pgfsetstrokecolor{currentstroke}%
\pgfsetdash{}{0pt}%
\pgfpathmoveto{\pgfqpoint{3.156051in}{0.528000in}}%
\pgfpathlineto{\pgfqpoint{3.063087in}{0.602667in}}%
\pgfpathlineto{\pgfqpoint{2.964364in}{0.684816in}}%
\pgfpathlineto{\pgfqpoint{2.906332in}{0.735280in}}%
\pgfpathlineto{\pgfqpoint{2.884202in}{0.753797in}}%
\pgfpathlineto{\pgfqpoint{2.802092in}{0.826667in}}%
\pgfpathlineto{\pgfqpoint{2.720387in}{0.901333in}}%
\pgfpathlineto{\pgfqpoint{2.662542in}{0.956201in}}%
\pgfpathlineto{\pgfqpoint{2.640894in}{0.976000in}}%
\pgfpathlineto{\pgfqpoint{2.583526in}{1.031934in}}%
\pgfpathlineto{\pgfqpoint{2.563476in}{1.050667in}}%
\pgfpathlineto{\pgfqpoint{2.505586in}{1.108671in}}%
\pgfpathlineto{\pgfqpoint{2.483394in}{1.130181in}}%
\pgfpathlineto{\pgfqpoint{2.428699in}{1.186387in}}%
\pgfpathlineto{\pgfqpoint{2.403232in}{1.212012in}}%
\pgfpathlineto{\pgfqpoint{2.352837in}{1.265059in}}%
\pgfpathlineto{\pgfqpoint{2.323071in}{1.296120in}}%
\pgfpathlineto{\pgfqpoint{2.239151in}{1.386667in}}%
\pgfpathlineto{\pgfqpoint{2.186330in}{1.445966in}}%
\pgfpathlineto{\pgfqpoint{2.162747in}{1.471799in}}%
\pgfpathlineto{\pgfqpoint{2.122667in}{1.517468in}}%
\pgfpathlineto{\pgfqpoint{2.042505in}{1.610724in}}%
\pgfpathlineto{\pgfqpoint{1.950155in}{1.722667in}}%
\pgfpathlineto{\pgfqpoint{1.920983in}{1.758808in}}%
\pgfpathlineto{\pgfqpoint{1.919983in}{1.760000in}}%
\pgfpathlineto{\pgfqpoint{1.870366in}{1.823661in}}%
\pgfpathlineto{\pgfqpoint{1.842101in}{1.859565in}}%
\pgfpathlineto{\pgfqpoint{1.695428in}{2.058667in}}%
\pgfpathlineto{\pgfqpoint{1.641697in}{2.136107in}}%
\pgfpathlineto{\pgfqpoint{1.618635in}{2.170667in}}%
\pgfpathlineto{\pgfqpoint{1.561535in}{2.258326in}}%
\pgfpathlineto{\pgfqpoint{1.457288in}{2.432000in}}%
\pgfpathlineto{\pgfqpoint{1.416388in}{2.506667in}}%
\pgfpathlineto{\pgfqpoint{1.411691in}{2.516427in}}%
\pgfpathlineto{\pgfqpoint{1.392650in}{2.551976in}}%
\pgfpathlineto{\pgfqpoint{1.358909in}{2.620737in}}%
\pgfpathlineto{\pgfqpoint{1.321051in}{2.705978in}}%
\pgfpathlineto{\pgfqpoint{1.280970in}{2.810037in}}%
\pgfpathlineto{\pgfqpoint{1.270284in}{2.842667in}}%
\pgfpathlineto{\pgfqpoint{1.259125in}{2.880000in}}%
\pgfpathlineto{\pgfqpoint{1.249127in}{2.917333in}}%
\pgfpathlineto{\pgfqpoint{1.240432in}{2.955093in}}%
\pgfpathlineto{\pgfqpoint{1.232787in}{2.999547in}}%
\pgfpathlineto{\pgfqpoint{1.229458in}{3.029333in}}%
\pgfpathlineto{\pgfqpoint{1.227476in}{3.104000in}}%
\pgfpathlineto{\pgfqpoint{1.227978in}{3.116026in}}%
\pgfpathlineto{\pgfqpoint{1.231201in}{3.141333in}}%
\pgfpathlineto{\pgfqpoint{1.240889in}{3.184053in}}%
\pgfpathlineto{\pgfqpoint{1.253868in}{3.216000in}}%
\pgfpathlineto{\pgfqpoint{1.262760in}{3.232961in}}%
\pgfpathlineto{\pgfqpoint{1.280970in}{3.257870in}}%
\pgfpathlineto{\pgfqpoint{1.297794in}{3.274996in}}%
\pgfpathlineto{\pgfqpoint{1.321051in}{3.292611in}}%
\pgfpathlineto{\pgfqpoint{1.345063in}{3.305634in}}%
\pgfpathlineto{\pgfqpoint{1.361131in}{3.311956in}}%
\pgfpathlineto{\pgfqpoint{1.405221in}{3.324266in}}%
\pgfpathlineto{\pgfqpoint{1.442084in}{3.328737in}}%
\pgfpathlineto{\pgfqpoint{1.482912in}{3.329432in}}%
\pgfpathlineto{\pgfqpoint{1.521455in}{3.327118in}}%
\pgfpathlineto{\pgfqpoint{1.561535in}{3.321617in}}%
\pgfpathlineto{\pgfqpoint{1.641697in}{3.304085in}}%
\pgfpathlineto{\pgfqpoint{1.652300in}{3.300543in}}%
\pgfpathlineto{\pgfqpoint{1.688191in}{3.290667in}}%
\pgfpathlineto{\pgfqpoint{1.791791in}{3.253333in}}%
\pgfpathlineto{\pgfqpoint{1.842101in}{3.232178in}}%
\pgfpathlineto{\pgfqpoint{1.853457in}{3.226578in}}%
\pgfpathlineto{\pgfqpoint{1.882182in}{3.214296in}}%
\pgfpathlineto{\pgfqpoint{1.933350in}{3.188994in}}%
\pgfpathlineto{\pgfqpoint{1.970831in}{3.170761in}}%
\pgfpathlineto{\pgfqpoint{2.066681in}{3.118814in}}%
\pgfpathlineto{\pgfqpoint{2.122667in}{3.086234in}}%
\pgfpathlineto{\pgfqpoint{2.135125in}{3.078271in}}%
\pgfpathlineto{\pgfqpoint{2.162747in}{3.062264in}}%
\pgfpathlineto{\pgfqpoint{2.207850in}{3.034011in}}%
\pgfpathlineto{\pgfqpoint{2.242909in}{3.011858in}}%
\pgfpathlineto{\pgfqpoint{2.329522in}{2.954667in}}%
\pgfpathlineto{\pgfqpoint{2.403232in}{2.903612in}}%
\pgfpathlineto{\pgfqpoint{2.440287in}{2.877182in}}%
\pgfpathlineto{\pgfqpoint{2.443313in}{2.875145in}}%
\pgfpathlineto{\pgfqpoint{2.462497in}{2.860535in}}%
\pgfpathlineto{\pgfqpoint{2.487933in}{2.842667in}}%
\pgfpathlineto{\pgfqpoint{2.537931in}{2.805333in}}%
\pgfpathlineto{\pgfqpoint{2.635093in}{2.730667in}}%
\pgfpathlineto{\pgfqpoint{2.683798in}{2.692170in}}%
\pgfpathlineto{\pgfqpoint{2.723879in}{2.659751in}}%
\pgfpathlineto{\pgfqpoint{2.818269in}{2.581333in}}%
\pgfpathlineto{\pgfqpoint{2.905260in}{2.506667in}}%
\pgfpathlineto{\pgfqpoint{2.956103in}{2.461639in}}%
\pgfpathlineto{\pgfqpoint{2.989620in}{2.432000in}}%
\pgfpathlineto{\pgfqpoint{3.084606in}{2.345245in}}%
\pgfpathlineto{\pgfqpoint{3.190142in}{2.245333in}}%
\pgfpathlineto{\pgfqpoint{3.285010in}{2.152430in}}%
\pgfpathlineto{\pgfqpoint{3.325091in}{2.112250in}}%
\pgfpathlineto{\pgfqpoint{3.413797in}{2.021333in}}%
\pgfpathlineto{\pgfqpoint{3.485414in}{1.945948in}}%
\pgfpathlineto{\pgfqpoint{3.525495in}{1.902708in}}%
\pgfpathlineto{\pgfqpoint{3.620878in}{1.797333in}}%
\pgfpathlineto{\pgfqpoint{3.653961in}{1.760000in}}%
\pgfpathlineto{\pgfqpoint{3.725899in}{1.676964in}}%
\pgfpathlineto{\pgfqpoint{3.812759in}{1.573333in}}%
\pgfpathlineto{\pgfqpoint{3.886222in}{1.482435in}}%
\pgfpathlineto{\pgfqpoint{3.966384in}{1.379687in}}%
\pgfpathlineto{\pgfqpoint{4.046545in}{1.272464in}}%
\pgfpathlineto{\pgfqpoint{4.126707in}{1.159796in}}%
\pgfpathlineto{\pgfqpoint{4.206869in}{1.040412in}}%
\pgfpathlineto{\pgfqpoint{4.248188in}{0.976000in}}%
\pgfpathlineto{\pgfqpoint{4.271130in}{0.938667in}}%
\pgfpathlineto{\pgfqpoint{4.315813in}{0.864000in}}%
\pgfpathlineto{\pgfqpoint{4.358144in}{0.789333in}}%
\pgfpathlineto{\pgfqpoint{4.397902in}{0.714667in}}%
\pgfpathlineto{\pgfqpoint{4.407273in}{0.696224in}}%
\pgfpathlineto{\pgfqpoint{4.434841in}{0.640000in}}%
\pgfpathlineto{\pgfqpoint{4.438370in}{0.631632in}}%
\pgfpathlineto{\pgfqpoint{4.452313in}{0.602667in}}%
\pgfpathlineto{\pgfqpoint{4.468678in}{0.565333in}}%
\pgfpathlineto{\pgfqpoint{4.484550in}{0.528000in}}%
\pgfpathlineto{\pgfqpoint{4.498579in}{0.528000in}}%
\pgfpathlineto{\pgfqpoint{4.482718in}{0.565333in}}%
\pgfpathlineto{\pgfqpoint{4.465734in}{0.602667in}}%
\pgfpathlineto{\pgfqpoint{4.447354in}{0.642065in}}%
\pgfpathlineto{\pgfqpoint{4.429764in}{0.677333in}}%
\pgfpathlineto{\pgfqpoint{4.407273in}{0.721470in}}%
\pgfpathlineto{\pgfqpoint{4.382593in}{0.766346in}}%
\pgfpathlineto{\pgfqpoint{4.367192in}{0.795612in}}%
\pgfpathlineto{\pgfqpoint{4.341369in}{0.839947in}}%
\pgfpathlineto{\pgfqpoint{4.327111in}{0.865752in}}%
\pgfpathlineto{\pgfqpoint{4.298717in}{0.912219in}}%
\pgfpathlineto{\pgfqpoint{4.283171in}{0.938667in}}%
\pgfpathlineto{\pgfqpoint{4.254823in}{0.983334in}}%
\pgfpathlineto{\pgfqpoint{4.235998in}{1.013333in}}%
\pgfpathlineto{\pgfqpoint{4.186860in}{1.088000in}}%
\pgfpathlineto{\pgfqpoint{4.166788in}{1.117872in}}%
\pgfpathlineto{\pgfqpoint{4.126707in}{1.175837in}}%
\pgfpathlineto{\pgfqpoint{4.109643in}{1.200000in}}%
\pgfpathlineto{\pgfqpoint{4.046545in}{1.287448in}}%
\pgfpathlineto{\pgfqpoint{3.966384in}{1.394027in}}%
\pgfpathlineto{\pgfqpoint{3.918975in}{1.454508in}}%
\pgfpathlineto{\pgfqpoint{3.913864in}{1.461333in}}%
\pgfpathlineto{\pgfqpoint{3.846141in}{1.545835in}}%
\pgfpathlineto{\pgfqpoint{3.744465in}{1.668040in}}%
\pgfpathlineto{\pgfqpoint{3.664801in}{1.760000in}}%
\pgfpathlineto{\pgfqpoint{3.598487in}{1.834667in}}%
\pgfpathlineto{\pgfqpoint{3.564768in}{1.872000in}}%
\pgfpathlineto{\pgfqpoint{3.485414in}{1.957619in}}%
\pgfpathlineto{\pgfqpoint{3.445333in}{1.999994in}}%
\pgfpathlineto{\pgfqpoint{3.352451in}{2.096000in}}%
\pgfpathlineto{\pgfqpoint{3.244929in}{2.203505in}}%
\pgfpathlineto{\pgfqpoint{3.201978in}{2.245333in}}%
\pgfpathlineto{\pgfqpoint{3.123713in}{2.320000in}}%
\pgfpathlineto{\pgfqpoint{3.043246in}{2.394667in}}%
\pgfpathlineto{\pgfqpoint{2.982517in}{2.448909in}}%
\pgfpathlineto{\pgfqpoint{2.960422in}{2.469333in}}%
\pgfpathlineto{\pgfqpoint{2.875072in}{2.544000in}}%
\pgfpathlineto{\pgfqpoint{2.831392in}{2.581333in}}%
\pgfpathlineto{\pgfqpoint{2.723879in}{2.670780in}}%
\pgfpathlineto{\pgfqpoint{2.689365in}{2.698519in}}%
\pgfpathlineto{\pgfqpoint{2.667584in}{2.715564in}}%
\pgfpathlineto{\pgfqpoint{2.643717in}{2.735174in}}%
\pgfpathlineto{\pgfqpoint{2.552976in}{2.805333in}}%
\pgfpathlineto{\pgfqpoint{2.452411in}{2.880000in}}%
\pgfpathlineto{\pgfqpoint{2.282990in}{2.997738in}}%
\pgfpathlineto{\pgfqpoint{2.234555in}{3.029333in}}%
\pgfpathlineto{\pgfqpoint{2.167684in}{3.071265in}}%
\pgfpathlineto{\pgfqpoint{2.162747in}{3.074537in}}%
\pgfpathlineto{\pgfqpoint{2.143154in}{3.085749in}}%
\pgfpathlineto{\pgfqpoint{2.113868in}{3.104000in}}%
\pgfpathlineto{\pgfqpoint{2.032068in}{3.151055in}}%
\pgfpathlineto{\pgfqpoint{1.962343in}{3.188338in}}%
\pgfpathlineto{\pgfqpoint{1.942652in}{3.197659in}}%
\pgfpathlineto{\pgfqpoint{1.907156in}{3.216000in}}%
\pgfpathlineto{\pgfqpoint{1.842101in}{3.246521in}}%
\pgfpathlineto{\pgfqpoint{1.761939in}{3.280051in}}%
\pgfpathlineto{\pgfqpoint{1.753507in}{3.282812in}}%
\pgfpathlineto{\pgfqpoint{1.714003in}{3.297984in}}%
\pgfpathlineto{\pgfqpoint{1.641697in}{3.320854in}}%
\pgfpathlineto{\pgfqpoint{1.596653in}{3.332623in}}%
\pgfpathlineto{\pgfqpoint{1.561535in}{3.339790in}}%
\pgfpathlineto{\pgfqpoint{1.537458in}{3.342906in}}%
\pgfpathlineto{\pgfqpoint{1.521455in}{3.346111in}}%
\pgfpathlineto{\pgfqpoint{1.502036in}{3.347245in}}%
\pgfpathlineto{\pgfqpoint{1.481374in}{3.349916in}}%
\pgfpathlineto{\pgfqpoint{1.464408in}{3.349530in}}%
\pgfpathlineto{\pgfqpoint{1.441293in}{3.350623in}}%
\pgfpathlineto{\pgfqpoint{1.418115in}{3.349589in}}%
\pgfpathlineto{\pgfqpoint{1.383188in}{3.344789in}}%
\pgfpathlineto{\pgfqpoint{1.361131in}{3.339355in}}%
\pgfpathlineto{\pgfqpoint{1.321051in}{3.324454in}}%
\pgfpathlineto{\pgfqpoint{1.272727in}{3.290667in}}%
\pgfpathlineto{\pgfqpoint{1.240889in}{3.250810in}}%
\pgfpathlineto{\pgfqpoint{1.228859in}{3.227205in}}%
\pgfpathlineto{\pgfqpoint{1.218755in}{3.199283in}}%
\pgfpathlineto{\pgfqpoint{1.213809in}{3.178667in}}%
\pgfpathlineto{\pgfqpoint{1.207092in}{3.135480in}}%
\pgfpathlineto{\pgfqpoint{1.205616in}{3.099521in}}%
\pgfpathlineto{\pgfqpoint{1.206851in}{3.066667in}}%
\pgfpathlineto{\pgfqpoint{1.216183in}{2.992000in}}%
\pgfpathlineto{\pgfqpoint{1.223645in}{2.954667in}}%
\pgfpathlineto{\pgfqpoint{1.227277in}{2.941988in}}%
\pgfpathlineto{\pgfqpoint{1.234541in}{2.911420in}}%
\pgfpathlineto{\pgfqpoint{1.242891in}{2.880000in}}%
\pgfpathlineto{\pgfqpoint{1.281757in}{2.767267in}}%
\pgfpathlineto{\pgfqpoint{1.312483in}{2.693333in}}%
\pgfpathlineto{\pgfqpoint{1.346779in}{2.618667in}}%
\pgfpathlineto{\pgfqpoint{1.384047in}{2.544000in}}%
\pgfpathlineto{\pgfqpoint{1.389969in}{2.533528in}}%
\pgfpathlineto{\pgfqpoint{1.405747in}{2.502443in}}%
\pgfpathlineto{\pgfqpoint{1.466481in}{2.394667in}}%
\pgfpathlineto{\pgfqpoint{1.486083in}{2.361720in}}%
\pgfpathlineto{\pgfqpoint{1.497443in}{2.342365in}}%
\pgfpathlineto{\pgfqpoint{1.561535in}{2.239925in}}%
\pgfpathlineto{\pgfqpoint{1.589989in}{2.197170in}}%
\pgfpathlineto{\pgfqpoint{1.607031in}{2.170667in}}%
\pgfpathlineto{\pgfqpoint{1.632303in}{2.133333in}}%
\pgfpathlineto{\pgfqpoint{1.684017in}{2.058667in}}%
\pgfpathlineto{\pgfqpoint{1.710759in}{2.021333in}}%
\pgfpathlineto{\pgfqpoint{1.765185in}{1.946667in}}%
\pgfpathlineto{\pgfqpoint{1.796953in}{1.904613in}}%
\pgfpathlineto{\pgfqpoint{1.821632in}{1.872000in}}%
\pgfpathlineto{\pgfqpoint{1.864090in}{1.817815in}}%
\pgfpathlineto{\pgfqpoint{1.882182in}{1.793876in}}%
\pgfpathlineto{\pgfqpoint{1.932171in}{1.731896in}}%
\pgfpathlineto{\pgfqpoint{1.962343in}{1.694248in}}%
\pgfpathlineto{\pgfqpoint{2.042505in}{1.598049in}}%
\pgfpathlineto{\pgfqpoint{2.128105in}{1.498667in}}%
\pgfpathlineto{\pgfqpoint{2.180358in}{1.440403in}}%
\pgfpathlineto{\pgfqpoint{2.202828in}{1.414688in}}%
\pgfpathlineto{\pgfqpoint{2.242909in}{1.370630in}}%
\pgfpathlineto{\pgfqpoint{2.332155in}{1.274667in}}%
\pgfpathlineto{\pgfqpoint{2.384787in}{1.220152in}}%
\pgfpathlineto{\pgfqpoint{2.403465in}{1.200000in}}%
\pgfpathlineto{\pgfqpoint{2.461111in}{1.141911in}}%
\pgfpathlineto{\pgfqpoint{2.483394in}{1.118767in}}%
\pgfpathlineto{\pgfqpoint{2.538473in}{1.064637in}}%
\pgfpathlineto{\pgfqpoint{2.563556in}{1.039442in}}%
\pgfpathlineto{\pgfqpoint{2.616898in}{0.988353in}}%
\pgfpathlineto{\pgfqpoint{2.643717in}{0.962220in}}%
\pgfpathlineto{\pgfqpoint{2.696414in}{0.913084in}}%
\pgfpathlineto{\pgfqpoint{2.723879in}{0.887064in}}%
\pgfpathlineto{\pgfqpoint{2.763960in}{0.850251in}}%
\pgfpathlineto{\pgfqpoint{2.873762in}{0.752000in}}%
\pgfpathlineto{\pgfqpoint{2.964364in}{0.673666in}}%
\pgfpathlineto{\pgfqpoint{3.025940in}{0.622689in}}%
\pgfpathlineto{\pgfqpoint{3.049342in}{0.602667in}}%
\pgfpathlineto{\pgfqpoint{3.141950in}{0.528000in}}%
\pgfpathlineto{\pgfqpoint{3.141950in}{0.528000in}}%
\pgfusepath{fill}%
\end{pgfscope}%
\begin{pgfscope}%
\pgfpathrectangle{\pgfqpoint{0.800000in}{0.528000in}}{\pgfqpoint{3.968000in}{3.696000in}}%
\pgfusepath{clip}%
\pgfsetbuttcap%
\pgfsetroundjoin%
\definecolor{currentfill}{rgb}{0.279566,0.067836,0.391917}%
\pgfsetfillcolor{currentfill}%
\pgfsetlinewidth{0.000000pt}%
\definecolor{currentstroke}{rgb}{0.000000,0.000000,0.000000}%
\pgfsetstrokecolor{currentstroke}%
\pgfsetdash{}{0pt}%
\pgfpathmoveto{\pgfqpoint{3.141950in}{0.528000in}}%
\pgfpathlineto{\pgfqpoint{3.044525in}{0.606607in}}%
\pgfpathlineto{\pgfqpoint{3.025940in}{0.622689in}}%
\pgfpathlineto{\pgfqpoint{3.004226in}{0.640000in}}%
\pgfpathlineto{\pgfqpoint{2.941780in}{0.693632in}}%
\pgfpathlineto{\pgfqpoint{2.916591in}{0.714667in}}%
\pgfpathlineto{\pgfqpoint{2.873762in}{0.752000in}}%
\pgfpathlineto{\pgfqpoint{2.763960in}{0.850251in}}%
\pgfpathlineto{\pgfqpoint{2.668643in}{0.938667in}}%
\pgfpathlineto{\pgfqpoint{2.616898in}{0.988353in}}%
\pgfpathlineto{\pgfqpoint{2.590452in}{1.013333in}}%
\pgfpathlineto{\pgfqpoint{2.538473in}{1.064637in}}%
\pgfpathlineto{\pgfqpoint{2.514259in}{1.088000in}}%
\pgfpathlineto{\pgfqpoint{2.461111in}{1.141911in}}%
\pgfpathlineto{\pgfqpoint{2.439944in}{1.162667in}}%
\pgfpathlineto{\pgfqpoint{2.384787in}{1.220152in}}%
\pgfpathlineto{\pgfqpoint{2.363152in}{1.241971in}}%
\pgfpathlineto{\pgfqpoint{2.323071in}{1.284277in}}%
\pgfpathlineto{\pgfqpoint{2.228293in}{1.386667in}}%
\pgfpathlineto{\pgfqpoint{2.180358in}{1.440403in}}%
\pgfpathlineto{\pgfqpoint{2.160977in}{1.461333in}}%
\pgfpathlineto{\pgfqpoint{2.122667in}{1.504875in}}%
\pgfpathlineto{\pgfqpoint{2.031846in}{1.610667in}}%
\pgfpathlineto{\pgfqpoint{1.962343in}{1.694248in}}%
\pgfpathlineto{\pgfqpoint{1.914939in}{1.753178in}}%
\pgfpathlineto{\pgfqpoint{1.882182in}{1.793876in}}%
\pgfpathlineto{\pgfqpoint{1.737846in}{1.984000in}}%
\pgfpathlineto{\pgfqpoint{1.721859in}{2.005943in}}%
\pgfpathlineto{\pgfqpoint{1.658047in}{2.096000in}}%
\pgfpathlineto{\pgfqpoint{1.641697in}{2.119621in}}%
\pgfpathlineto{\pgfqpoint{1.601616in}{2.178817in}}%
\pgfpathlineto{\pgfqpoint{1.574828in}{2.220382in}}%
\pgfpathlineto{\pgfqpoint{1.558032in}{2.245333in}}%
\pgfpathlineto{\pgfqpoint{1.534443in}{2.282667in}}%
\pgfpathlineto{\pgfqpoint{1.481374in}{2.369328in}}%
\pgfpathlineto{\pgfqpoint{1.401212in}{2.511026in}}%
\pgfpathlineto{\pgfqpoint{1.326972in}{2.661516in}}%
\pgfpathlineto{\pgfqpoint{1.312483in}{2.693333in}}%
\pgfpathlineto{\pgfqpoint{1.296704in}{2.730667in}}%
\pgfpathlineto{\pgfqpoint{1.280970in}{2.769375in}}%
\pgfpathlineto{\pgfqpoint{1.240889in}{2.887163in}}%
\pgfpathlineto{\pgfqpoint{1.223645in}{2.954667in}}%
\pgfpathlineto{\pgfqpoint{1.216183in}{2.992000in}}%
\pgfpathlineto{\pgfqpoint{1.210469in}{3.029333in}}%
\pgfpathlineto{\pgfqpoint{1.210122in}{3.038009in}}%
\pgfpathlineto{\pgfqpoint{1.206851in}{3.066667in}}%
\pgfpathlineto{\pgfqpoint{1.205779in}{3.104000in}}%
\pgfpathlineto{\pgfqpoint{1.207837in}{3.141333in}}%
\pgfpathlineto{\pgfqpoint{1.209604in}{3.149527in}}%
\pgfpathlineto{\pgfqpoint{1.213809in}{3.178667in}}%
\pgfpathlineto{\pgfqpoint{1.218755in}{3.199283in}}%
\pgfpathlineto{\pgfqpoint{1.228859in}{3.227205in}}%
\pgfpathlineto{\pgfqpoint{1.242363in}{3.253333in}}%
\pgfpathlineto{\pgfqpoint{1.280970in}{3.298191in}}%
\pgfpathlineto{\pgfqpoint{1.329152in}{3.328000in}}%
\pgfpathlineto{\pgfqpoint{1.361131in}{3.339355in}}%
\pgfpathlineto{\pgfqpoint{1.383188in}{3.344789in}}%
\pgfpathlineto{\pgfqpoint{1.418115in}{3.349589in}}%
\pgfpathlineto{\pgfqpoint{1.441293in}{3.350623in}}%
\pgfpathlineto{\pgfqpoint{1.464408in}{3.349530in}}%
\pgfpathlineto{\pgfqpoint{1.481374in}{3.349916in}}%
\pgfpathlineto{\pgfqpoint{1.502036in}{3.347245in}}%
\pgfpathlineto{\pgfqpoint{1.521455in}{3.346111in}}%
\pgfpathlineto{\pgfqpoint{1.537458in}{3.342906in}}%
\pgfpathlineto{\pgfqpoint{1.561535in}{3.339790in}}%
\pgfpathlineto{\pgfqpoint{1.614568in}{3.328000in}}%
\pgfpathlineto{\pgfqpoint{1.641697in}{3.320854in}}%
\pgfpathlineto{\pgfqpoint{1.681778in}{3.308615in}}%
\pgfpathlineto{\pgfqpoint{1.733776in}{3.290667in}}%
\pgfpathlineto{\pgfqpoint{1.842101in}{3.246521in}}%
\pgfpathlineto{\pgfqpoint{1.863525in}{3.235956in}}%
\pgfpathlineto{\pgfqpoint{1.882182in}{3.228036in}}%
\pgfpathlineto{\pgfqpoint{1.922263in}{3.208694in}}%
\pgfpathlineto{\pgfqpoint{1.942652in}{3.197659in}}%
\pgfpathlineto{\pgfqpoint{1.980650in}{3.178667in}}%
\pgfpathlineto{\pgfqpoint{2.049308in}{3.141333in}}%
\pgfpathlineto{\pgfqpoint{2.122667in}{3.098843in}}%
\pgfpathlineto{\pgfqpoint{2.167684in}{3.071265in}}%
\pgfpathlineto{\pgfqpoint{2.202828in}{3.049514in}}%
\pgfpathlineto{\pgfqpoint{2.239424in}{3.026088in}}%
\pgfpathlineto{\pgfqpoint{2.261792in}{3.011745in}}%
\pgfpathlineto{\pgfqpoint{2.346593in}{2.954667in}}%
\pgfpathlineto{\pgfqpoint{2.403232in}{2.915361in}}%
\pgfpathlineto{\pgfqpoint{2.447220in}{2.883639in}}%
\pgfpathlineto{\pgfqpoint{2.483394in}{2.857346in}}%
\pgfpathlineto{\pgfqpoint{2.563556in}{2.797332in}}%
\pgfpathlineto{\pgfqpoint{2.649384in}{2.730667in}}%
\pgfpathlineto{\pgfqpoint{2.710652in}{2.681014in}}%
\pgfpathlineto{\pgfqpoint{2.741900in}{2.656000in}}%
\pgfpathlineto{\pgfqpoint{2.795821in}{2.611011in}}%
\pgfpathlineto{\pgfqpoint{2.831392in}{2.581333in}}%
\pgfpathlineto{\pgfqpoint{2.924283in}{2.501247in}}%
\pgfpathlineto{\pgfqpoint{3.004444in}{2.429928in}}%
\pgfpathlineto{\pgfqpoint{3.044525in}{2.393501in}}%
\pgfpathlineto{\pgfqpoint{3.124687in}{2.319087in}}%
\pgfpathlineto{\pgfqpoint{3.240329in}{2.208000in}}%
\pgfpathlineto{\pgfqpoint{3.300526in}{2.147786in}}%
\pgfpathlineto{\pgfqpoint{3.325091in}{2.123755in}}%
\pgfpathlineto{\pgfqpoint{3.424907in}{2.021333in}}%
\pgfpathlineto{\pgfqpoint{3.495658in}{1.946667in}}%
\pgfpathlineto{\pgfqpoint{3.530426in}{1.909333in}}%
\pgfpathlineto{\pgfqpoint{3.605657in}{1.826702in}}%
\pgfpathlineto{\pgfqpoint{3.645737in}{1.781675in}}%
\pgfpathlineto{\pgfqpoint{3.744465in}{1.668040in}}%
\pgfpathlineto{\pgfqpoint{3.823553in}{1.573333in}}%
\pgfpathlineto{\pgfqpoint{3.886222in}{1.496310in}}%
\pgfpathlineto{\pgfqpoint{3.986184in}{1.368224in}}%
\pgfpathlineto{\pgfqpoint{4.055935in}{1.274667in}}%
\pgfpathlineto{\pgfqpoint{4.109643in}{1.200000in}}%
\pgfpathlineto{\pgfqpoint{4.126707in}{1.175837in}}%
\pgfpathlineto{\pgfqpoint{4.166788in}{1.117872in}}%
\pgfpathlineto{\pgfqpoint{4.186860in}{1.088000in}}%
\pgfpathlineto{\pgfqpoint{4.246949in}{0.996233in}}%
\pgfpathlineto{\pgfqpoint{4.349535in}{0.826667in}}%
\pgfpathlineto{\pgfqpoint{4.355336in}{0.815624in}}%
\pgfpathlineto{\pgfqpoint{4.370722in}{0.789333in}}%
\pgfpathlineto{\pgfqpoint{4.396065in}{0.741561in}}%
\pgfpathlineto{\pgfqpoint{4.414315in}{0.708107in}}%
\pgfpathlineto{\pgfqpoint{4.449225in}{0.638257in}}%
\pgfpathlineto{\pgfqpoint{4.487434in}{0.554333in}}%
\pgfpathlineto{\pgfqpoint{4.498579in}{0.528000in}}%
\pgfpathlineto{\pgfqpoint{4.512440in}{0.528000in}}%
\pgfpathlineto{\pgfqpoint{4.487434in}{0.584688in}}%
\pgfpathlineto{\pgfqpoint{4.438747in}{0.685350in}}%
\pgfpathlineto{\pgfqpoint{4.403567in}{0.752000in}}%
\pgfpathlineto{\pgfqpoint{4.377199in}{0.798655in}}%
\pgfpathlineto{\pgfqpoint{4.361788in}{0.826667in}}%
\pgfpathlineto{\pgfqpoint{4.339931in}{0.864000in}}%
\pgfpathlineto{\pgfqpoint{4.287030in}{0.951143in}}%
\pgfpathlineto{\pgfqpoint{4.261936in}{0.989959in}}%
\pgfpathlineto{\pgfqpoint{4.246949in}{1.014436in}}%
\pgfpathlineto{\pgfqpoint{4.166788in}{1.134253in}}%
\pgfpathlineto{\pgfqpoint{4.086626in}{1.247652in}}%
\pgfpathlineto{\pgfqpoint{4.006465in}{1.355788in}}%
\pgfpathlineto{\pgfqpoint{3.920715in}{1.466538in}}%
\pgfpathlineto{\pgfqpoint{3.834346in}{1.573333in}}%
\pgfpathlineto{\pgfqpoint{3.765980in}{1.655296in}}%
\pgfpathlineto{\pgfqpoint{3.675641in}{1.760000in}}%
\pgfpathlineto{\pgfqpoint{3.605657in}{1.838797in}}%
\pgfpathlineto{\pgfqpoint{3.565576in}{1.882862in}}%
\pgfpathlineto{\pgfqpoint{3.471486in}{1.984000in}}%
\pgfpathlineto{\pgfqpoint{3.421186in}{2.036175in}}%
\pgfpathlineto{\pgfqpoint{3.400123in}{2.058667in}}%
\pgfpathlineto{\pgfqpoint{3.325091in}{2.135180in}}%
\pgfpathlineto{\pgfqpoint{3.285010in}{2.175179in}}%
\pgfpathlineto{\pgfqpoint{3.174584in}{2.282667in}}%
\pgfpathlineto{\pgfqpoint{3.124687in}{2.330001in}}%
\pgfpathlineto{\pgfqpoint{3.014146in}{2.432000in}}%
\pgfpathlineto{\pgfqpoint{2.964364in}{2.476743in}}%
\pgfpathlineto{\pgfqpoint{2.884202in}{2.547166in}}%
\pgfpathlineto{\pgfqpoint{2.844121in}{2.581654in}}%
\pgfpathlineto{\pgfqpoint{2.755348in}{2.656000in}}%
\pgfpathlineto{\pgfqpoint{2.643717in}{2.746143in}}%
\pgfpathlineto{\pgfqpoint{2.544567in}{2.823020in}}%
\pgfpathlineto{\pgfqpoint{2.467884in}{2.880000in}}%
\pgfpathlineto{\pgfqpoint{2.416294in}{2.917333in}}%
\pgfpathlineto{\pgfqpoint{2.363152in}{2.955004in}}%
\pgfpathlineto{\pgfqpoint{2.323071in}{2.982442in}}%
\pgfpathlineto{\pgfqpoint{2.242909in}{3.035850in}}%
\pgfpathlineto{\pgfqpoint{2.222917in}{3.048045in}}%
\pgfpathlineto{\pgfqpoint{2.194757in}{3.066667in}}%
\pgfpathlineto{\pgfqpoint{2.134327in}{3.104000in}}%
\pgfpathlineto{\pgfqpoint{2.071258in}{3.141333in}}%
\pgfpathlineto{\pgfqpoint{1.922263in}{3.222079in}}%
\pgfpathlineto{\pgfqpoint{1.873594in}{3.245334in}}%
\pgfpathlineto{\pgfqpoint{1.842101in}{3.260477in}}%
\pgfpathlineto{\pgfqpoint{1.802020in}{3.278121in}}%
\pgfpathlineto{\pgfqpoint{1.753992in}{3.298070in}}%
\pgfpathlineto{\pgfqpoint{1.678804in}{3.325230in}}%
\pgfpathlineto{\pgfqpoint{1.641697in}{3.337049in}}%
\pgfpathlineto{\pgfqpoint{1.554184in}{3.358486in}}%
\pgfpathlineto{\pgfqpoint{1.519033in}{3.365333in}}%
\pgfpathlineto{\pgfqpoint{1.481374in}{3.369855in}}%
\pgfpathlineto{\pgfqpoint{1.434017in}{3.372111in}}%
\pgfpathlineto{\pgfqpoint{1.401212in}{3.370654in}}%
\pgfpathlineto{\pgfqpoint{1.360878in}{3.365097in}}%
\pgfpathlineto{\pgfqpoint{1.321051in}{3.353258in}}%
\pgfpathlineto{\pgfqpoint{1.302317in}{3.345449in}}%
\pgfpathlineto{\pgfqpoint{1.273339in}{3.328000in}}%
\pgfpathlineto{\pgfqpoint{1.255832in}{3.314082in}}%
\pgfpathlineto{\pgfqpoint{1.234488in}{3.290667in}}%
\pgfpathlineto{\pgfqpoint{1.208274in}{3.246379in}}%
\pgfpathlineto{\pgfqpoint{1.197535in}{3.216000in}}%
\pgfpathlineto{\pgfqpoint{1.189686in}{3.178667in}}%
\pgfpathlineto{\pgfqpoint{1.185936in}{3.141333in}}%
\pgfpathlineto{\pgfqpoint{1.186395in}{3.127908in}}%
\pgfpathlineto{\pgfqpoint{1.185481in}{3.104000in}}%
\pgfpathlineto{\pgfqpoint{1.186871in}{3.091018in}}%
\pgfpathlineto{\pgfqpoint{1.187714in}{3.066667in}}%
\pgfpathlineto{\pgfqpoint{1.192170in}{3.029333in}}%
\pgfpathlineto{\pgfqpoint{1.193681in}{3.022695in}}%
\pgfpathlineto{\pgfqpoint{1.200808in}{2.981406in}}%
\pgfpathlineto{\pgfqpoint{1.206763in}{2.954667in}}%
\pgfpathlineto{\pgfqpoint{1.240889in}{2.838930in}}%
\pgfpathlineto{\pgfqpoint{1.267458in}{2.768000in}}%
\pgfpathlineto{\pgfqpoint{1.271464in}{2.759146in}}%
\pgfpathlineto{\pgfqpoint{1.282528in}{2.730667in}}%
\pgfpathlineto{\pgfqpoint{1.298938in}{2.693333in}}%
\pgfpathlineto{\pgfqpoint{1.321051in}{2.644896in}}%
\pgfpathlineto{\pgfqpoint{1.361131in}{2.563689in}}%
\pgfpathlineto{\pgfqpoint{1.401212in}{2.488273in}}%
\pgfpathlineto{\pgfqpoint{1.441293in}{2.417211in}}%
\pgfpathlineto{\pgfqpoint{1.481374in}{2.349505in}}%
\pgfpathlineto{\pgfqpoint{1.522642in}{2.282667in}}%
\pgfpathlineto{\pgfqpoint{1.552476in}{2.236895in}}%
\pgfpathlineto{\pgfqpoint{1.570951in}{2.208000in}}%
\pgfpathlineto{\pgfqpoint{1.598074in}{2.167367in}}%
\pgfpathlineto{\pgfqpoint{1.621126in}{2.133333in}}%
\pgfpathlineto{\pgfqpoint{1.681778in}{2.046361in}}%
\pgfpathlineto{\pgfqpoint{1.699760in}{2.021333in}}%
\pgfpathlineto{\pgfqpoint{1.761939in}{1.936452in}}%
\pgfpathlineto{\pgfqpoint{1.802020in}{1.883357in}}%
\pgfpathlineto{\pgfqpoint{1.868790in}{1.797333in}}%
\pgfpathlineto{\pgfqpoint{1.908895in}{1.747549in}}%
\pgfpathlineto{\pgfqpoint{1.928408in}{1.722667in}}%
\pgfpathlineto{\pgfqpoint{1.962343in}{1.681065in}}%
\pgfpathlineto{\pgfqpoint{2.052780in}{1.573333in}}%
\pgfpathlineto{\pgfqpoint{2.122667in}{1.492572in}}%
\pgfpathlineto{\pgfqpoint{2.174386in}{1.434840in}}%
\pgfpathlineto{\pgfqpoint{2.202828in}{1.402739in}}%
\pgfpathlineto{\pgfqpoint{2.242909in}{1.358717in}}%
\pgfpathlineto{\pgfqpoint{2.323071in}{1.272530in}}%
\pgfpathlineto{\pgfqpoint{2.378932in}{1.214699in}}%
\pgfpathlineto{\pgfqpoint{2.403232in}{1.188958in}}%
\pgfpathlineto{\pgfqpoint{2.455209in}{1.136414in}}%
\pgfpathlineto{\pgfqpoint{2.483394in}{1.107559in}}%
\pgfpathlineto{\pgfqpoint{2.532523in}{1.059094in}}%
\pgfpathlineto{\pgfqpoint{2.563556in}{1.028296in}}%
\pgfpathlineto{\pgfqpoint{2.610899in}{0.982765in}}%
\pgfpathlineto{\pgfqpoint{2.643717in}{0.951136in}}%
\pgfpathlineto{\pgfqpoint{2.690365in}{0.907450in}}%
\pgfpathlineto{\pgfqpoint{2.723879in}{0.876042in}}%
\pgfpathlineto{\pgfqpoint{2.763960in}{0.839259in}}%
\pgfpathlineto{\pgfqpoint{2.861362in}{0.752000in}}%
\pgfpathlineto{\pgfqpoint{2.964364in}{0.662822in}}%
\pgfpathlineto{\pgfqpoint{3.019683in}{0.616861in}}%
\pgfpathlineto{\pgfqpoint{3.044525in}{0.595662in}}%
\pgfpathlineto{\pgfqpoint{3.127848in}{0.528000in}}%
\pgfpathlineto{\pgfqpoint{3.127848in}{0.528000in}}%
\pgfusepath{fill}%
\end{pgfscope}%
\begin{pgfscope}%
\pgfpathrectangle{\pgfqpoint{0.800000in}{0.528000in}}{\pgfqpoint{3.968000in}{3.696000in}}%
\pgfusepath{clip}%
\pgfsetbuttcap%
\pgfsetroundjoin%
\definecolor{currentfill}{rgb}{0.279566,0.067836,0.391917}%
\pgfsetfillcolor{currentfill}%
\pgfsetlinewidth{0.000000pt}%
\definecolor{currentstroke}{rgb}{0.000000,0.000000,0.000000}%
\pgfsetstrokecolor{currentstroke}%
\pgfsetdash{}{0pt}%
\pgfpathmoveto{\pgfqpoint{3.127848in}{0.528000in}}%
\pgfpathlineto{\pgfqpoint{3.081505in}{0.565333in}}%
\pgfpathlineto{\pgfqpoint{2.991387in}{0.640000in}}%
\pgfpathlineto{\pgfqpoint{2.935576in}{0.687853in}}%
\pgfpathlineto{\pgfqpoint{2.904048in}{0.714667in}}%
\pgfpathlineto{\pgfqpoint{2.861362in}{0.752000in}}%
\pgfpathlineto{\pgfqpoint{2.763960in}{0.839259in}}%
\pgfpathlineto{\pgfqpoint{2.656913in}{0.938667in}}%
\pgfpathlineto{\pgfqpoint{2.610899in}{0.982765in}}%
\pgfpathlineto{\pgfqpoint{2.578970in}{1.013333in}}%
\pgfpathlineto{\pgfqpoint{2.532523in}{1.059094in}}%
\pgfpathlineto{\pgfqpoint{2.503015in}{1.088000in}}%
\pgfpathlineto{\pgfqpoint{2.455209in}{1.136414in}}%
\pgfpathlineto{\pgfqpoint{2.428927in}{1.162667in}}%
\pgfpathlineto{\pgfqpoint{2.378932in}{1.214699in}}%
\pgfpathlineto{\pgfqpoint{2.356595in}{1.237333in}}%
\pgfpathlineto{\pgfqpoint{2.321055in}{1.274667in}}%
\pgfpathlineto{\pgfqpoint{2.242909in}{1.358717in}}%
\pgfpathlineto{\pgfqpoint{2.202828in}{1.402739in}}%
\pgfpathlineto{\pgfqpoint{2.093864in}{1.525495in}}%
\pgfpathlineto{\pgfqpoint{2.021141in}{1.610667in}}%
\pgfpathlineto{\pgfqpoint{1.958829in}{1.685333in}}%
\pgfpathlineto{\pgfqpoint{1.908895in}{1.747549in}}%
\pgfpathlineto{\pgfqpoint{1.882182in}{1.780386in}}%
\pgfpathlineto{\pgfqpoint{1.824155in}{1.855285in}}%
\pgfpathlineto{\pgfqpoint{1.802020in}{1.883357in}}%
\pgfpathlineto{\pgfqpoint{1.757535in}{1.942564in}}%
\pgfpathlineto{\pgfqpoint{1.737156in}{1.969751in}}%
\pgfpathlineto{\pgfqpoint{1.673025in}{2.058667in}}%
\pgfpathlineto{\pgfqpoint{1.621126in}{2.133333in}}%
\pgfpathlineto{\pgfqpoint{1.598074in}{2.167367in}}%
\pgfpathlineto{\pgfqpoint{1.595716in}{2.170667in}}%
\pgfpathlineto{\pgfqpoint{1.567525in}{2.213579in}}%
\pgfpathlineto{\pgfqpoint{1.546672in}{2.245333in}}%
\pgfpathlineto{\pgfqpoint{1.537586in}{2.260359in}}%
\pgfpathlineto{\pgfqpoint{1.521455in}{2.284566in}}%
\pgfpathlineto{\pgfqpoint{1.499538in}{2.320000in}}%
\pgfpathlineto{\pgfqpoint{1.454476in}{2.394667in}}%
\pgfpathlineto{\pgfqpoint{1.411668in}{2.469333in}}%
\pgfpathlineto{\pgfqpoint{1.371318in}{2.544000in}}%
\pgfpathlineto{\pgfqpoint{1.333655in}{2.618667in}}%
\pgfpathlineto{\pgfqpoint{1.298938in}{2.693333in}}%
\pgfpathlineto{\pgfqpoint{1.294118in}{2.705580in}}%
\pgfpathlineto{\pgfqpoint{1.280970in}{2.734483in}}%
\pgfpathlineto{\pgfqpoint{1.238887in}{2.844532in}}%
\pgfpathlineto{\pgfqpoint{1.216584in}{2.917333in}}%
\pgfpathlineto{\pgfqpoint{1.213952in}{2.929576in}}%
\pgfpathlineto{\pgfqpoint{1.206763in}{2.954667in}}%
\pgfpathlineto{\pgfqpoint{1.198485in}{2.992000in}}%
\pgfpathlineto{\pgfqpoint{1.187714in}{3.066667in}}%
\pgfpathlineto{\pgfqpoint{1.185936in}{3.141333in}}%
\pgfpathlineto{\pgfqpoint{1.186550in}{3.154614in}}%
\pgfpathlineto{\pgfqpoint{1.190938in}{3.187860in}}%
\pgfpathlineto{\pgfqpoint{1.200808in}{3.225980in}}%
\pgfpathlineto{\pgfqpoint{1.211818in}{3.253333in}}%
\pgfpathlineto{\pgfqpoint{1.240889in}{3.298463in}}%
\pgfpathlineto{\pgfqpoint{1.255832in}{3.314082in}}%
\pgfpathlineto{\pgfqpoint{1.280970in}{3.333342in}}%
\pgfpathlineto{\pgfqpoint{1.302317in}{3.345449in}}%
\pgfpathlineto{\pgfqpoint{1.321051in}{3.353258in}}%
\pgfpathlineto{\pgfqpoint{1.362035in}{3.365333in}}%
\pgfpathlineto{\pgfqpoint{1.401212in}{3.370654in}}%
\pgfpathlineto{\pgfqpoint{1.441293in}{3.371928in}}%
\pgfpathlineto{\pgfqpoint{1.447625in}{3.371231in}}%
\pgfpathlineto{\pgfqpoint{1.481374in}{3.369855in}}%
\pgfpathlineto{\pgfqpoint{1.521455in}{3.365041in}}%
\pgfpathlineto{\pgfqpoint{1.601616in}{3.348097in}}%
\pgfpathlineto{\pgfqpoint{1.617711in}{3.342991in}}%
\pgfpathlineto{\pgfqpoint{1.648672in}{3.334497in}}%
\pgfpathlineto{\pgfqpoint{1.687684in}{3.322499in}}%
\pgfpathlineto{\pgfqpoint{1.761939in}{3.294863in}}%
\pgfpathlineto{\pgfqpoint{1.842101in}{3.260477in}}%
\pgfpathlineto{\pgfqpoint{1.922263in}{3.222079in}}%
\pgfpathlineto{\pgfqpoint{2.005105in}{3.178667in}}%
\pgfpathlineto{\pgfqpoint{2.101269in}{3.123931in}}%
\pgfpathlineto{\pgfqpoint{2.194757in}{3.066667in}}%
\pgfpathlineto{\pgfqpoint{2.323071in}{2.982442in}}%
\pgfpathlineto{\pgfqpoint{2.339863in}{2.970308in}}%
\pgfpathlineto{\pgfqpoint{2.363630in}{2.954667in}}%
\pgfpathlineto{\pgfqpoint{2.403232in}{2.926695in}}%
\pgfpathlineto{\pgfqpoint{2.483394in}{2.868660in}}%
\pgfpathlineto{\pgfqpoint{2.567755in}{2.805333in}}%
\pgfpathlineto{\pgfqpoint{2.663173in}{2.730667in}}%
\pgfpathlineto{\pgfqpoint{2.717120in}{2.687038in}}%
\pgfpathlineto{\pgfqpoint{2.723879in}{2.681809in}}%
\pgfpathlineto{\pgfqpoint{2.840263in}{2.584927in}}%
\pgfpathlineto{\pgfqpoint{2.924283in}{2.512197in}}%
\pgfpathlineto{\pgfqpoint{2.968344in}{2.473041in}}%
\pgfpathlineto{\pgfqpoint{3.004444in}{2.440801in}}%
\pgfpathlineto{\pgfqpoint{3.044525in}{2.404366in}}%
\pgfpathlineto{\pgfqpoint{3.135274in}{2.320000in}}%
\pgfpathlineto{\pgfqpoint{3.244929in}{2.214657in}}%
\pgfpathlineto{\pgfqpoint{3.306350in}{2.153210in}}%
\pgfpathlineto{\pgfqpoint{3.348310in}{2.111706in}}%
\pgfpathlineto{\pgfqpoint{3.405253in}{2.053378in}}%
\pgfpathlineto{\pgfqpoint{3.458963in}{1.996696in}}%
\pgfpathlineto{\pgfqpoint{3.485414in}{1.969258in}}%
\pgfpathlineto{\pgfqpoint{3.575477in}{1.872000in}}%
\pgfpathlineto{\pgfqpoint{3.625549in}{1.815862in}}%
\pgfpathlineto{\pgfqpoint{3.645737in}{1.794000in}}%
\pgfpathlineto{\pgfqpoint{3.740319in}{1.685333in}}%
\pgfpathlineto{\pgfqpoint{3.806061in}{1.607654in}}%
\pgfpathlineto{\pgfqpoint{3.895008in}{1.498667in}}%
\pgfpathlineto{\pgfqpoint{3.966384in}{1.407998in}}%
\pgfpathlineto{\pgfqpoint{4.009277in}{1.351953in}}%
\pgfpathlineto{\pgfqpoint{4.023049in}{1.333886in}}%
\pgfpathlineto{\pgfqpoint{4.109458in}{1.216067in}}%
\pgfpathlineto{\pgfqpoint{4.172935in}{1.125333in}}%
\pgfpathlineto{\pgfqpoint{4.198251in}{1.088000in}}%
\pgfpathlineto{\pgfqpoint{4.247669in}{1.013333in}}%
\pgfpathlineto{\pgfqpoint{4.276911in}{0.966575in}}%
\pgfpathlineto{\pgfqpoint{4.294816in}{0.938667in}}%
\pgfpathlineto{\pgfqpoint{4.320889in}{0.895538in}}%
\pgfpathlineto{\pgfqpoint{4.339931in}{0.864000in}}%
\pgfpathlineto{\pgfqpoint{4.349160in}{0.847204in}}%
\pgfpathlineto{\pgfqpoint{4.367192in}{0.817163in}}%
\pgfpathlineto{\pgfqpoint{4.447354in}{0.668146in}}%
\pgfpathlineto{\pgfqpoint{4.487434in}{0.584688in}}%
\pgfpathlineto{\pgfqpoint{4.512440in}{0.528000in}}%
\pgfpathlineto{\pgfqpoint{4.526301in}{0.528000in}}%
\pgfpathlineto{\pgfqpoint{4.509495in}{0.565333in}}%
\pgfpathlineto{\pgfqpoint{4.502470in}{0.579338in}}%
\pgfpathlineto{\pgfqpoint{4.487434in}{0.612723in}}%
\pgfpathlineto{\pgfqpoint{4.447354in}{0.692711in}}%
\pgfpathlineto{\pgfqpoint{4.407273in}{0.767306in}}%
\pgfpathlineto{\pgfqpoint{4.306276in}{0.938667in}}%
\pgfpathlineto{\pgfqpoint{4.284386in}{0.973537in}}%
\pgfpathlineto{\pgfqpoint{4.282934in}{0.976000in}}%
\pgfpathlineto{\pgfqpoint{4.254171in}{1.020060in}}%
\pgfpathlineto{\pgfqpoint{4.234334in}{1.050667in}}%
\pgfpathlineto{\pgfqpoint{4.223476in}{1.066136in}}%
\pgfpathlineto{\pgfqpoint{4.206869in}{1.091895in}}%
\pgfpathlineto{\pgfqpoint{4.176995in}{1.134841in}}%
\pgfpathlineto{\pgfqpoint{4.141882in}{1.185866in}}%
\pgfpathlineto{\pgfqpoint{4.077770in}{1.274667in}}%
\pgfpathlineto{\pgfqpoint{4.050210in}{1.312000in}}%
\pgfpathlineto{\pgfqpoint{3.993541in}{1.386667in}}%
\pgfpathlineto{\pgfqpoint{3.948156in}{1.444355in}}%
\pgfpathlineto{\pgfqpoint{3.926303in}{1.472781in}}%
\pgfpathlineto{\pgfqpoint{3.841831in}{1.577348in}}%
\pgfpathlineto{\pgfqpoint{3.750947in}{1.685333in}}%
\pgfpathlineto{\pgfqpoint{3.685818in}{1.760722in}}%
\pgfpathlineto{\pgfqpoint{3.586150in}{1.872000in}}%
\pgfpathlineto{\pgfqpoint{3.485414in}{1.980898in}}%
\pgfpathlineto{\pgfqpoint{3.426940in}{2.041534in}}%
\pgfpathlineto{\pgfqpoint{3.405253in}{2.064687in}}%
\pgfpathlineto{\pgfqpoint{3.300588in}{2.170667in}}%
\pgfpathlineto{\pgfqpoint{3.204848in}{2.264552in}}%
\pgfpathlineto{\pgfqpoint{3.107079in}{2.357333in}}%
\pgfpathlineto{\pgfqpoint{3.066835in}{2.394667in}}%
\pgfpathlineto{\pgfqpoint{2.964364in}{2.487502in}}%
\pgfpathlineto{\pgfqpoint{2.856931in}{2.581333in}}%
\pgfpathlineto{\pgfqpoint{2.804040in}{2.626309in}}%
\pgfpathlineto{\pgfqpoint{2.723268in}{2.693333in}}%
\pgfpathlineto{\pgfqpoint{2.676963in}{2.730667in}}%
\pgfpathlineto{\pgfqpoint{2.563556in}{2.819480in}}%
\pgfpathlineto{\pgfqpoint{2.483358in}{2.880000in}}%
\pgfpathlineto{\pgfqpoint{2.431991in}{2.917333in}}%
\pgfpathlineto{\pgfqpoint{2.363152in}{2.966222in}}%
\pgfpathlineto{\pgfqpoint{2.346925in}{2.976885in}}%
\pgfpathlineto{\pgfqpoint{2.314929in}{2.999583in}}%
\pgfpathlineto{\pgfqpoint{2.213281in}{3.066667in}}%
\pgfpathlineto{\pgfqpoint{2.154056in}{3.104000in}}%
\pgfpathlineto{\pgfqpoint{2.082586in}{3.147140in}}%
\pgfpathlineto{\pgfqpoint{2.042505in}{3.170278in}}%
\pgfpathlineto{\pgfqpoint{1.959386in}{3.216000in}}%
\pgfpathlineto{\pgfqpoint{1.877868in}{3.257352in}}%
\pgfpathlineto{\pgfqpoint{1.798640in}{3.293815in}}%
\pgfpathlineto{\pgfqpoint{1.713928in}{3.328000in}}%
\pgfpathlineto{\pgfqpoint{1.681778in}{3.339586in}}%
\pgfpathlineto{\pgfqpoint{1.660828in}{3.345819in}}%
\pgfpathlineto{\pgfqpoint{1.641697in}{3.352819in}}%
\pgfpathlineto{\pgfqpoint{1.631084in}{3.355448in}}%
\pgfpathlineto{\pgfqpoint{1.599439in}{3.365333in}}%
\pgfpathlineto{\pgfqpoint{1.521455in}{3.382727in}}%
\pgfpathlineto{\pgfqpoint{1.481374in}{3.388696in}}%
\pgfpathlineto{\pgfqpoint{1.467004in}{3.389282in}}%
\pgfpathlineto{\pgfqpoint{1.441293in}{3.392111in}}%
\pgfpathlineto{\pgfqpoint{1.401212in}{3.392385in}}%
\pgfpathlineto{\pgfqpoint{1.388289in}{3.390630in}}%
\pgfpathlineto{\pgfqpoint{1.361131in}{3.388735in}}%
\pgfpathlineto{\pgfqpoint{1.339861in}{3.385145in}}%
\pgfpathlineto{\pgfqpoint{1.321051in}{3.380096in}}%
\pgfpathlineto{\pgfqpoint{1.280970in}{3.364934in}}%
\pgfpathlineto{\pgfqpoint{1.256793in}{3.350519in}}%
\pgfpathlineto{\pgfqpoint{1.240889in}{3.337987in}}%
\pgfpathlineto{\pgfqpoint{1.230236in}{3.328000in}}%
\pgfpathlineto{\pgfqpoint{1.200751in}{3.290613in}}%
\pgfpathlineto{\pgfqpoint{1.183653in}{3.253333in}}%
\pgfpathlineto{\pgfqpoint{1.170640in}{3.206767in}}%
\pgfpathlineto{\pgfqpoint{1.166156in}{3.173610in}}%
\pgfpathlineto{\pgfqpoint{1.164664in}{3.141333in}}%
\pgfpathlineto{\pgfqpoint{1.165599in}{3.104000in}}%
\pgfpathlineto{\pgfqpoint{1.174587in}{3.029333in}}%
\pgfpathlineto{\pgfqpoint{1.179173in}{3.009181in}}%
\pgfpathlineto{\pgfqpoint{1.181862in}{2.992000in}}%
\pgfpathlineto{\pgfqpoint{1.190607in}{2.954667in}}%
\pgfpathlineto{\pgfqpoint{1.192967in}{2.947363in}}%
\pgfpathlineto{\pgfqpoint{1.200808in}{2.916677in}}%
\pgfpathlineto{\pgfqpoint{1.238575in}{2.805333in}}%
\pgfpathlineto{\pgfqpoint{1.253465in}{2.768000in}}%
\pgfpathlineto{\pgfqpoint{1.261620in}{2.749977in}}%
\pgfpathlineto{\pgfqpoint{1.269063in}{2.730667in}}%
\pgfpathlineto{\pgfqpoint{1.302810in}{2.656000in}}%
\pgfpathlineto{\pgfqpoint{1.308767in}{2.644558in}}%
\pgfpathlineto{\pgfqpoint{1.321051in}{2.617687in}}%
\pgfpathlineto{\pgfqpoint{1.361131in}{2.539504in}}%
\pgfpathlineto{\pgfqpoint{1.378955in}{2.506667in}}%
\pgfpathlineto{\pgfqpoint{1.401212in}{2.466158in}}%
\pgfpathlineto{\pgfqpoint{1.428346in}{2.419940in}}%
\pgfpathlineto{\pgfqpoint{1.442470in}{2.394667in}}%
\pgfpathlineto{\pgfqpoint{1.465056in}{2.357333in}}%
\pgfpathlineto{\pgfqpoint{1.521455in}{2.266823in}}%
\pgfpathlineto{\pgfqpoint{1.662136in}{2.058667in}}%
\pgfpathlineto{\pgfqpoint{1.688760in}{2.021333in}}%
\pgfpathlineto{\pgfqpoint{1.743598in}{1.946667in}}%
\pgfpathlineto{\pgfqpoint{1.784508in}{1.893021in}}%
\pgfpathlineto{\pgfqpoint{1.802020in}{1.869193in}}%
\pgfpathlineto{\pgfqpoint{1.851537in}{1.806123in}}%
\pgfpathlineto{\pgfqpoint{1.882182in}{1.766896in}}%
\pgfpathlineto{\pgfqpoint{2.082586in}{1.526336in}}%
\pgfpathlineto{\pgfqpoint{2.172948in}{1.424000in}}%
\pgfpathlineto{\pgfqpoint{2.242909in}{1.346912in}}%
\pgfpathlineto{\pgfqpoint{2.345798in}{1.237333in}}%
\pgfpathlineto{\pgfqpoint{2.443313in}{1.136749in}}%
\pgfpathlineto{\pgfqpoint{2.487809in}{1.092112in}}%
\pgfpathlineto{\pgfqpoint{2.523475in}{1.056486in}}%
\pgfpathlineto{\pgfqpoint{2.567488in}{1.013333in}}%
\pgfpathlineto{\pgfqpoint{2.645183in}{0.938667in}}%
\pgfpathlineto{\pgfqpoint{2.704738in}{0.883505in}}%
\pgfpathlineto{\pgfqpoint{2.724987in}{0.864000in}}%
\pgfpathlineto{\pgfqpoint{2.844121in}{0.756268in}}%
\pgfpathlineto{\pgfqpoint{2.934693in}{0.677333in}}%
\pgfpathlineto{\pgfqpoint{2.992389in}{0.628770in}}%
\pgfpathlineto{\pgfqpoint{3.023093in}{0.602667in}}%
\pgfpathlineto{\pgfqpoint{3.068355in}{0.565333in}}%
\pgfpathlineto{\pgfqpoint{3.114359in}{0.528000in}}%
\pgfpathlineto{\pgfqpoint{3.124687in}{0.528000in}}%
\pgfpathlineto{\pgfqpoint{3.124687in}{0.528000in}}%
\pgfusepath{fill}%
\end{pgfscope}%
\begin{pgfscope}%
\pgfpathrectangle{\pgfqpoint{0.800000in}{0.528000in}}{\pgfqpoint{3.968000in}{3.696000in}}%
\pgfusepath{clip}%
\pgfsetbuttcap%
\pgfsetroundjoin%
\definecolor{currentfill}{rgb}{0.279566,0.067836,0.391917}%
\pgfsetfillcolor{currentfill}%
\pgfsetlinewidth{0.000000pt}%
\definecolor{currentstroke}{rgb}{0.000000,0.000000,0.000000}%
\pgfsetstrokecolor{currentstroke}%
\pgfsetdash{}{0pt}%
\pgfpathmoveto{\pgfqpoint{3.114359in}{0.528000in}}%
\pgfpathlineto{\pgfqpoint{3.068355in}{0.565333in}}%
\pgfpathlineto{\pgfqpoint{2.964364in}{0.651979in}}%
\pgfpathlineto{\pgfqpoint{2.924283in}{0.686253in}}%
\pgfpathlineto{\pgfqpoint{2.807042in}{0.789333in}}%
\pgfpathlineto{\pgfqpoint{2.763960in}{0.828266in}}%
\pgfpathlineto{\pgfqpoint{2.683798in}{0.902279in}}%
\pgfpathlineto{\pgfqpoint{2.643717in}{0.940052in}}%
\pgfpathlineto{\pgfqpoint{2.563556in}{1.017151in}}%
\pgfpathlineto{\pgfqpoint{2.523475in}{1.056486in}}%
\pgfpathlineto{\pgfqpoint{2.417911in}{1.162667in}}%
\pgfpathlineto{\pgfqpoint{2.373078in}{1.209246in}}%
\pgfpathlineto{\pgfqpoint{2.345798in}{1.237333in}}%
\pgfpathlineto{\pgfqpoint{2.310365in}{1.274667in}}%
\pgfpathlineto{\pgfqpoint{2.240683in}{1.349333in}}%
\pgfpathlineto{\pgfqpoint{2.202828in}{1.390790in}}%
\pgfpathlineto{\pgfqpoint{2.106779in}{1.498667in}}%
\pgfpathlineto{\pgfqpoint{2.041991in}{1.573333in}}%
\pgfpathlineto{\pgfqpoint{1.989618in}{1.636072in}}%
\pgfpathlineto{\pgfqpoint{1.962343in}{1.668311in}}%
\pgfpathlineto{\pgfqpoint{1.902851in}{1.741919in}}%
\pgfpathlineto{\pgfqpoint{1.882182in}{1.766896in}}%
\pgfpathlineto{\pgfqpoint{1.842101in}{1.817685in}}%
\pgfpathlineto{\pgfqpoint{1.761939in}{1.922084in}}%
\pgfpathlineto{\pgfqpoint{1.718396in}{1.980775in}}%
\pgfpathlineto{\pgfqpoint{1.703052in}{2.001517in}}%
\pgfpathlineto{\pgfqpoint{1.624642in}{2.111886in}}%
\pgfpathlineto{\pgfqpoint{1.559565in}{2.208000in}}%
\pgfpathlineto{\pgfqpoint{1.530319in}{2.253590in}}%
\pgfpathlineto{\pgfqpoint{1.511343in}{2.282667in}}%
\pgfpathlineto{\pgfqpoint{1.487863in}{2.320000in}}%
\pgfpathlineto{\pgfqpoint{1.441293in}{2.396679in}}%
\pgfpathlineto{\pgfqpoint{1.414342in}{2.444230in}}%
\pgfpathlineto{\pgfqpoint{1.397388in}{2.472895in}}%
\pgfpathlineto{\pgfqpoint{1.358718in}{2.544000in}}%
\pgfpathlineto{\pgfqpoint{1.339527in}{2.581333in}}%
\pgfpathlineto{\pgfqpoint{1.320558in}{2.618667in}}%
\pgfpathlineto{\pgfqpoint{1.302810in}{2.656000in}}%
\pgfpathlineto{\pgfqpoint{1.280970in}{2.703354in}}%
\pgfpathlineto{\pgfqpoint{1.253465in}{2.768000in}}%
\pgfpathlineto{\pgfqpoint{1.250371in}{2.776832in}}%
\pgfpathlineto{\pgfqpoint{1.238575in}{2.805333in}}%
\pgfpathlineto{\pgfqpoint{1.200604in}{2.917333in}}%
\pgfpathlineto{\pgfqpoint{1.181862in}{2.992000in}}%
\pgfpathlineto{\pgfqpoint{1.179173in}{3.009181in}}%
\pgfpathlineto{\pgfqpoint{1.174587in}{3.029333in}}%
\pgfpathlineto{\pgfqpoint{1.169052in}{3.066667in}}%
\pgfpathlineto{\pgfqpoint{1.168792in}{3.074178in}}%
\pgfpathlineto{\pgfqpoint{1.165779in}{3.108705in}}%
\pgfpathlineto{\pgfqpoint{1.164664in}{3.141333in}}%
\pgfpathlineto{\pgfqpoint{1.166815in}{3.178667in}}%
\pgfpathlineto{\pgfqpoint{1.172805in}{3.216000in}}%
\pgfpathlineto{\pgfqpoint{1.178116in}{3.232197in}}%
\pgfpathlineto{\pgfqpoint{1.183653in}{3.253333in}}%
\pgfpathlineto{\pgfqpoint{1.200808in}{3.290725in}}%
\pgfpathlineto{\pgfqpoint{1.230236in}{3.328000in}}%
\pgfpathlineto{\pgfqpoint{1.240889in}{3.337987in}}%
\pgfpathlineto{\pgfqpoint{1.256793in}{3.350519in}}%
\pgfpathlineto{\pgfqpoint{1.281831in}{3.365333in}}%
\pgfpathlineto{\pgfqpoint{1.321051in}{3.380096in}}%
\pgfpathlineto{\pgfqpoint{1.339861in}{3.385145in}}%
\pgfpathlineto{\pgfqpoint{1.361131in}{3.388735in}}%
\pgfpathlineto{\pgfqpoint{1.411749in}{3.392853in}}%
\pgfpathlineto{\pgfqpoint{1.441293in}{3.392111in}}%
\pgfpathlineto{\pgfqpoint{1.521455in}{3.382727in}}%
\pgfpathlineto{\pgfqpoint{1.536240in}{3.379105in}}%
\pgfpathlineto{\pgfqpoint{1.569101in}{3.372380in}}%
\pgfpathlineto{\pgfqpoint{1.601616in}{3.364796in}}%
\pgfpathlineto{\pgfqpoint{1.681778in}{3.339586in}}%
\pgfpathlineto{\pgfqpoint{1.727019in}{3.323193in}}%
\pgfpathlineto{\pgfqpoint{1.805673in}{3.290667in}}%
\pgfpathlineto{\pgfqpoint{1.959386in}{3.216000in}}%
\pgfpathlineto{\pgfqpoint{2.042505in}{3.170278in}}%
\pgfpathlineto{\pgfqpoint{2.061258in}{3.158800in}}%
\pgfpathlineto{\pgfqpoint{2.092310in}{3.141333in}}%
\pgfpathlineto{\pgfqpoint{2.162747in}{3.098682in}}%
\pgfpathlineto{\pgfqpoint{2.202828in}{3.073422in}}%
\pgfpathlineto{\pgfqpoint{2.282990in}{3.021031in}}%
\pgfpathlineto{\pgfqpoint{2.363152in}{2.966222in}}%
\pgfpathlineto{\pgfqpoint{2.443313in}{2.909196in}}%
\pgfpathlineto{\pgfqpoint{2.460511in}{2.896019in}}%
\pgfpathlineto{\pgfqpoint{2.483534in}{2.879870in}}%
\pgfpathlineto{\pgfqpoint{2.581904in}{2.805333in}}%
\pgfpathlineto{\pgfqpoint{2.683798in}{2.725215in}}%
\pgfpathlineto{\pgfqpoint{2.723879in}{2.692838in}}%
\pgfpathlineto{\pgfqpoint{2.813073in}{2.618667in}}%
\pgfpathlineto{\pgfqpoint{2.924283in}{2.522927in}}%
\pgfpathlineto{\pgfqpoint{2.974123in}{2.478424in}}%
\pgfpathlineto{\pgfqpoint{3.004444in}{2.451589in}}%
\pgfpathlineto{\pgfqpoint{3.044525in}{2.415183in}}%
\pgfpathlineto{\pgfqpoint{3.146787in}{2.320000in}}%
\pgfpathlineto{\pgfqpoint{3.244929in}{2.225622in}}%
\pgfpathlineto{\pgfqpoint{3.285010in}{2.186175in}}%
\pgfpathlineto{\pgfqpoint{3.374680in}{2.096000in}}%
\pgfpathlineto{\pgfqpoint{3.447046in}{2.021333in}}%
\pgfpathlineto{\pgfqpoint{3.485414in}{1.980898in}}%
\pgfpathlineto{\pgfqpoint{3.586150in}{1.872000in}}%
\pgfpathlineto{\pgfqpoint{3.631417in}{1.821328in}}%
\pgfpathlineto{\pgfqpoint{3.653374in}{1.797333in}}%
\pgfpathlineto{\pgfqpoint{3.725899in}{1.714561in}}%
\pgfpathlineto{\pgfqpoint{3.814087in}{1.610667in}}%
\pgfpathlineto{\pgfqpoint{3.886222in}{1.522793in}}%
\pgfpathlineto{\pgfqpoint{3.926303in}{1.472781in}}%
\pgfpathlineto{\pgfqpoint{3.993541in}{1.386667in}}%
\pgfpathlineto{\pgfqpoint{4.050210in}{1.312000in}}%
\pgfpathlineto{\pgfqpoint{4.097317in}{1.247291in}}%
\pgfpathlineto{\pgfqpoint{4.126707in}{1.207146in}}%
\pgfpathlineto{\pgfqpoint{4.246949in}{1.031533in}}%
\pgfpathlineto{\pgfqpoint{4.351710in}{0.864000in}}%
\pgfpathlineto{\pgfqpoint{4.367192in}{0.837807in}}%
\pgfpathlineto{\pgfqpoint{4.407273in}{0.767306in}}%
\pgfpathlineto{\pgfqpoint{4.447354in}{0.692711in}}%
\pgfpathlineto{\pgfqpoint{4.492301in}{0.602667in}}%
\pgfpathlineto{\pgfqpoint{4.509495in}{0.565333in}}%
\pgfpathlineto{\pgfqpoint{4.527515in}{0.528000in}}%
\pgfpathlineto{\pgfqpoint{4.539465in}{0.528000in}}%
\pgfpathlineto{\pgfqpoint{4.527515in}{0.554786in}}%
\pgfpathlineto{\pgfqpoint{4.505004in}{0.602667in}}%
\pgfpathlineto{\pgfqpoint{4.467675in}{0.677333in}}%
\pgfpathlineto{\pgfqpoint{4.460684in}{0.689750in}}%
\pgfpathlineto{\pgfqpoint{4.447354in}{0.716305in}}%
\pgfpathlineto{\pgfqpoint{4.427720in}{0.752000in}}%
\pgfpathlineto{\pgfqpoint{4.385359in}{0.826667in}}%
\pgfpathlineto{\pgfqpoint{4.340783in}{0.901333in}}%
\pgfpathlineto{\pgfqpoint{4.287030in}{0.987098in}}%
\pgfpathlineto{\pgfqpoint{4.260956in}{1.026380in}}%
\pgfpathlineto{\pgfqpoint{4.245607in}{1.050667in}}%
\pgfpathlineto{\pgfqpoint{4.220386in}{1.088000in}}%
\pgfpathlineto{\pgfqpoint{4.166788in}{1.165921in}}%
\pgfpathlineto{\pgfqpoint{4.086626in}{1.277332in}}%
\pgfpathlineto{\pgfqpoint{4.046545in}{1.330962in}}%
\pgfpathlineto{\pgfqpoint{3.966384in}{1.435342in}}%
\pgfpathlineto{\pgfqpoint{3.926303in}{1.486008in}}%
\pgfpathlineto{\pgfqpoint{3.846141in}{1.584690in}}%
\pgfpathlineto{\pgfqpoint{3.744147in}{1.705670in}}%
\pgfpathlineto{\pgfqpoint{3.663841in}{1.797333in}}%
\pgfpathlineto{\pgfqpoint{3.565576in}{1.906277in}}%
\pgfpathlineto{\pgfqpoint{3.485414in}{1.992177in}}%
\pgfpathlineto{\pgfqpoint{3.432694in}{2.046894in}}%
\pgfpathlineto{\pgfqpoint{3.405253in}{2.075774in}}%
\pgfpathlineto{\pgfqpoint{3.311633in}{2.170667in}}%
\pgfpathlineto{\pgfqpoint{3.204848in}{2.275488in}}%
\pgfpathlineto{\pgfqpoint{3.118716in}{2.357333in}}%
\pgfpathlineto{\pgfqpoint{3.078598in}{2.394667in}}%
\pgfpathlineto{\pgfqpoint{2.996691in}{2.469333in}}%
\pgfpathlineto{\pgfqpoint{2.938878in}{2.520261in}}%
\pgfpathlineto{\pgfqpoint{2.912431in}{2.544000in}}%
\pgfpathlineto{\pgfqpoint{2.855995in}{2.592393in}}%
\pgfpathlineto{\pgfqpoint{2.825654in}{2.618667in}}%
\pgfpathlineto{\pgfqpoint{2.781266in}{2.656000in}}%
\pgfpathlineto{\pgfqpoint{2.683798in}{2.735993in}}%
\pgfpathlineto{\pgfqpoint{2.596053in}{2.805333in}}%
\pgfpathlineto{\pgfqpoint{2.483394in}{2.890824in}}%
\pgfpathlineto{\pgfqpoint{2.445085in}{2.918984in}}%
\pgfpathlineto{\pgfqpoint{2.428713in}{2.930933in}}%
\pgfpathlineto{\pgfqpoint{2.342172in}{2.992000in}}%
\pgfpathlineto{\pgfqpoint{2.162747in}{3.110459in}}%
\pgfpathlineto{\pgfqpoint{2.112404in}{3.141333in}}%
\pgfpathlineto{\pgfqpoint{1.962343in}{3.226997in}}%
\pgfpathlineto{\pgfqpoint{1.911906in}{3.253333in}}%
\pgfpathlineto{\pgfqpoint{1.802020in}{3.305883in}}%
\pgfpathlineto{\pgfqpoint{1.750402in}{3.328000in}}%
\pgfpathlineto{\pgfqpoint{1.650595in}{3.365333in}}%
\pgfpathlineto{\pgfqpoint{1.636574in}{3.370105in}}%
\pgfpathlineto{\pgfqpoint{1.601616in}{3.380535in}}%
\pgfpathlineto{\pgfqpoint{1.582599in}{3.384953in}}%
\pgfpathlineto{\pgfqpoint{1.561535in}{3.391284in}}%
\pgfpathlineto{\pgfqpoint{1.551257in}{3.393093in}}%
\pgfpathlineto{\pgfqpoint{1.508133in}{3.402667in}}%
\pgfpathlineto{\pgfqpoint{1.475528in}{3.408111in}}%
\pgfpathlineto{\pgfqpoint{1.441293in}{3.411611in}}%
\pgfpathlineto{\pgfqpoint{1.389774in}{3.413321in}}%
\pgfpathlineto{\pgfqpoint{1.352446in}{3.410757in}}%
\pgfpathlineto{\pgfqpoint{1.310450in}{3.402667in}}%
\pgfpathlineto{\pgfqpoint{1.280970in}{3.393199in}}%
\pgfpathlineto{\pgfqpoint{1.230885in}{3.365333in}}%
\pgfpathlineto{\pgfqpoint{1.196345in}{3.332157in}}%
\pgfpathlineto{\pgfqpoint{1.180770in}{3.309331in}}%
\pgfpathlineto{\pgfqpoint{1.171241in}{3.290667in}}%
\pgfpathlineto{\pgfqpoint{1.157140in}{3.253333in}}%
\pgfpathlineto{\pgfqpoint{1.149267in}{3.216000in}}%
\pgfpathlineto{\pgfqpoint{1.148657in}{3.204757in}}%
\pgfpathlineto{\pgfqpoint{1.145417in}{3.178667in}}%
\pgfpathlineto{\pgfqpoint{1.145810in}{3.164772in}}%
\pgfpathlineto{\pgfqpoint{1.144816in}{3.141333in}}%
\pgfpathlineto{\pgfqpoint{1.146876in}{3.104000in}}%
\pgfpathlineto{\pgfqpoint{1.151142in}{3.066667in}}%
\pgfpathlineto{\pgfqpoint{1.152770in}{3.059255in}}%
\pgfpathlineto{\pgfqpoint{1.160727in}{3.012968in}}%
\pgfpathlineto{\pgfqpoint{1.174845in}{2.954667in}}%
\pgfpathlineto{\pgfqpoint{1.180851in}{2.936078in}}%
\pgfpathlineto{\pgfqpoint{1.185619in}{2.917333in}}%
\pgfpathlineto{\pgfqpoint{1.189463in}{2.906766in}}%
\pgfpathlineto{\pgfqpoint{1.200808in}{2.870260in}}%
\pgfpathlineto{\pgfqpoint{1.224760in}{2.805333in}}%
\pgfpathlineto{\pgfqpoint{1.240889in}{2.764868in}}%
\pgfpathlineto{\pgfqpoint{1.308121in}{2.618667in}}%
\pgfpathlineto{\pgfqpoint{1.346637in}{2.544000in}}%
\pgfpathlineto{\pgfqpoint{1.387659in}{2.469333in}}%
\pgfpathlineto{\pgfqpoint{1.406409in}{2.436841in}}%
\pgfpathlineto{\pgfqpoint{1.408991in}{2.432000in}}%
\pgfpathlineto{\pgfqpoint{1.430985in}{2.394667in}}%
\pgfpathlineto{\pgfqpoint{1.481374in}{2.312142in}}%
\pgfpathlineto{\pgfqpoint{1.508365in}{2.270474in}}%
\pgfpathlineto{\pgfqpoint{1.523951in}{2.245333in}}%
\pgfpathlineto{\pgfqpoint{1.548614in}{2.208000in}}%
\pgfpathlineto{\pgfqpoint{1.601616in}{2.129409in}}%
\pgfpathlineto{\pgfqpoint{1.681778in}{2.016049in}}%
\pgfpathlineto{\pgfqpoint{1.761939in}{1.907800in}}%
\pgfpathlineto{\pgfqpoint{1.811657in}{1.843643in}}%
\pgfpathlineto{\pgfqpoint{1.842101in}{1.804149in}}%
\pgfpathlineto{\pgfqpoint{2.031649in}{1.573333in}}%
\pgfpathlineto{\pgfqpoint{2.072368in}{1.526483in}}%
\pgfpathlineto{\pgfqpoint{2.096237in}{1.498667in}}%
\pgfpathlineto{\pgfqpoint{2.129040in}{1.461333in}}%
\pgfpathlineto{\pgfqpoint{2.202828in}{1.379179in}}%
\pgfpathlineto{\pgfqpoint{2.242909in}{1.335512in}}%
\pgfpathlineto{\pgfqpoint{2.335001in}{1.237333in}}%
\pgfpathlineto{\pgfqpoint{2.443487in}{1.125333in}}%
\pgfpathlineto{\pgfqpoint{2.483394in}{1.085258in}}%
\pgfpathlineto{\pgfqpoint{2.594902in}{0.976000in}}%
\pgfpathlineto{\pgfqpoint{2.658961in}{0.915532in}}%
\pgfpathlineto{\pgfqpoint{2.683798in}{0.891631in}}%
\pgfpathlineto{\pgfqpoint{2.723879in}{0.854396in}}%
\pgfpathlineto{\pgfqpoint{2.836938in}{0.752000in}}%
\pgfpathlineto{\pgfqpoint{2.884202in}{0.710292in}}%
\pgfpathlineto{\pgfqpoint{2.965708in}{0.640000in}}%
\pgfpathlineto{\pgfqpoint{3.028623in}{0.587855in}}%
\pgfpathlineto{\pgfqpoint{3.055204in}{0.565333in}}%
\pgfpathlineto{\pgfqpoint{3.101048in}{0.528000in}}%
\pgfpathlineto{\pgfqpoint{3.101048in}{0.528000in}}%
\pgfusepath{fill}%
\end{pgfscope}%
\begin{pgfscope}%
\pgfpathrectangle{\pgfqpoint{0.800000in}{0.528000in}}{\pgfqpoint{3.968000in}{3.696000in}}%
\pgfusepath{clip}%
\pgfsetbuttcap%
\pgfsetroundjoin%
\definecolor{currentfill}{rgb}{0.279566,0.067836,0.391917}%
\pgfsetfillcolor{currentfill}%
\pgfsetlinewidth{0.000000pt}%
\definecolor{currentstroke}{rgb}{0.000000,0.000000,0.000000}%
\pgfsetstrokecolor{currentstroke}%
\pgfsetdash{}{0pt}%
\pgfpathmoveto{\pgfqpoint{3.101048in}{0.528000in}}%
\pgfpathlineto{\pgfqpoint{3.028623in}{0.587855in}}%
\pgfpathlineto{\pgfqpoint{3.004444in}{0.607374in}}%
\pgfpathlineto{\pgfqpoint{2.944492in}{0.658824in}}%
\pgfpathlineto{\pgfqpoint{2.922119in}{0.677333in}}%
\pgfpathlineto{\pgfqpoint{2.804040in}{0.781384in}}%
\pgfpathlineto{\pgfqpoint{2.763960in}{0.817648in}}%
\pgfpathlineto{\pgfqpoint{2.673470in}{0.901333in}}%
\pgfpathlineto{\pgfqpoint{2.633936in}{0.938667in}}%
\pgfpathlineto{\pgfqpoint{2.556354in}{1.013333in}}%
\pgfpathlineto{\pgfqpoint{2.518276in}{1.050667in}}%
\pgfpathlineto{\pgfqpoint{2.406895in}{1.162667in}}%
\pgfpathlineto{\pgfqpoint{2.363152in}{1.207863in}}%
\pgfpathlineto{\pgfqpoint{2.264743in}{1.312000in}}%
\pgfpathlineto{\pgfqpoint{2.162220in}{1.424000in}}%
\pgfpathlineto{\pgfqpoint{2.082586in}{1.514279in}}%
\pgfpathlineto{\pgfqpoint{2.036575in}{1.567809in}}%
\pgfpathlineto{\pgfqpoint{1.999846in}{1.610667in}}%
\pgfpathlineto{\pgfqpoint{1.948736in}{1.672658in}}%
\pgfpathlineto{\pgfqpoint{1.922263in}{1.704308in}}%
\pgfpathlineto{\pgfqpoint{1.842101in}{1.804149in}}%
\pgfpathlineto{\pgfqpoint{1.760774in}{1.909333in}}%
\pgfpathlineto{\pgfqpoint{1.712227in}{1.975028in}}%
\pgfpathlineto{\pgfqpoint{1.681778in}{2.016049in}}%
\pgfpathlineto{\pgfqpoint{1.561535in}{2.188591in}}%
\pgfpathlineto{\pgfqpoint{1.521455in}{2.249205in}}%
\pgfpathlineto{\pgfqpoint{1.493418in}{2.293885in}}%
\pgfpathlineto{\pgfqpoint{1.469519in}{2.331042in}}%
\pgfpathlineto{\pgfqpoint{1.408991in}{2.432000in}}%
\pgfpathlineto{\pgfqpoint{1.392573in}{2.461286in}}%
\pgfpathlineto{\pgfqpoint{1.372535in}{2.496045in}}%
\pgfpathlineto{\pgfqpoint{1.326945in}{2.581333in}}%
\pgfpathlineto{\pgfqpoint{1.289843in}{2.656000in}}%
\pgfpathlineto{\pgfqpoint{1.255685in}{2.730667in}}%
\pgfpathlineto{\pgfqpoint{1.251776in}{2.740808in}}%
\pgfpathlineto{\pgfqpoint{1.238677in}{2.770061in}}%
\pgfpathlineto{\pgfqpoint{1.210604in}{2.842667in}}%
\pgfpathlineto{\pgfqpoint{1.197396in}{2.880000in}}%
\pgfpathlineto{\pgfqpoint{1.165239in}{2.992000in}}%
\pgfpathlineto{\pgfqpoint{1.164665in}{2.995667in}}%
\pgfpathlineto{\pgfqpoint{1.157259in}{3.029333in}}%
\pgfpathlineto{\pgfqpoint{1.146876in}{3.104000in}}%
\pgfpathlineto{\pgfqpoint{1.145543in}{3.118143in}}%
\pgfpathlineto{\pgfqpoint{1.144816in}{3.141333in}}%
\pgfpathlineto{\pgfqpoint{1.145417in}{3.178667in}}%
\pgfpathlineto{\pgfqpoint{1.150580in}{3.225451in}}%
\pgfpathlineto{\pgfqpoint{1.160727in}{3.264314in}}%
\pgfpathlineto{\pgfqpoint{1.171241in}{3.290667in}}%
\pgfpathlineto{\pgfqpoint{1.180770in}{3.309331in}}%
\pgfpathlineto{\pgfqpoint{1.200808in}{3.337109in}}%
\pgfpathlineto{\pgfqpoint{1.236225in}{3.369677in}}%
\pgfpathlineto{\pgfqpoint{1.240889in}{3.372571in}}%
\pgfpathlineto{\pgfqpoint{1.280970in}{3.393199in}}%
\pgfpathlineto{\pgfqpoint{1.321051in}{3.405489in}}%
\pgfpathlineto{\pgfqpoint{1.361131in}{3.411486in}}%
\pgfpathlineto{\pgfqpoint{1.401212in}{3.413246in}}%
\pgfpathlineto{\pgfqpoint{1.411560in}{3.412305in}}%
\pgfpathlineto{\pgfqpoint{1.441293in}{3.411611in}}%
\pgfpathlineto{\pgfqpoint{1.485296in}{3.406320in}}%
\pgfpathlineto{\pgfqpoint{1.521455in}{3.400393in}}%
\pgfpathlineto{\pgfqpoint{1.601616in}{3.380535in}}%
\pgfpathlineto{\pgfqpoint{1.650595in}{3.365333in}}%
\pgfpathlineto{\pgfqpoint{1.750402in}{3.328000in}}%
\pgfpathlineto{\pgfqpoint{1.770980in}{3.319579in}}%
\pgfpathlineto{\pgfqpoint{1.848524in}{3.284684in}}%
\pgfpathlineto{\pgfqpoint{1.934749in}{3.241703in}}%
\pgfpathlineto{\pgfqpoint{2.031461in}{3.188954in}}%
\pgfpathlineto{\pgfqpoint{2.112404in}{3.141333in}}%
\pgfpathlineto{\pgfqpoint{2.172939in}{3.104000in}}%
\pgfpathlineto{\pgfqpoint{2.237984in}{3.062080in}}%
\pgfpathlineto{\pgfqpoint{2.242909in}{3.059076in}}%
\pgfpathlineto{\pgfqpoint{2.323071in}{3.005220in}}%
\pgfpathlineto{\pgfqpoint{2.353986in}{2.983463in}}%
\pgfpathlineto{\pgfqpoint{2.363152in}{2.977440in}}%
\pgfpathlineto{\pgfqpoint{2.447417in}{2.917333in}}%
\pgfpathlineto{\pgfqpoint{2.497887in}{2.880000in}}%
\pgfpathlineto{\pgfqpoint{2.603636in}{2.799470in}}%
\pgfpathlineto{\pgfqpoint{2.621613in}{2.784745in}}%
\pgfpathlineto{\pgfqpoint{2.643813in}{2.768000in}}%
\pgfpathlineto{\pgfqpoint{2.763960in}{2.670434in}}%
\pgfpathlineto{\pgfqpoint{2.869368in}{2.581333in}}%
\pgfpathlineto{\pgfqpoint{2.938878in}{2.520261in}}%
\pgfpathlineto{\pgfqpoint{2.964364in}{2.498261in}}%
\pgfpathlineto{\pgfqpoint{3.020653in}{2.447098in}}%
\pgfpathlineto{\pgfqpoint{3.044525in}{2.426000in}}%
\pgfpathlineto{\pgfqpoint{3.124687in}{2.351753in}}%
\pgfpathlineto{\pgfqpoint{3.180983in}{2.297771in}}%
\pgfpathlineto{\pgfqpoint{3.204848in}{2.275488in}}%
\pgfpathlineto{\pgfqpoint{3.311633in}{2.170667in}}%
\pgfpathlineto{\pgfqpoint{3.348790in}{2.133333in}}%
\pgfpathlineto{\pgfqpoint{3.421790in}{2.058667in}}%
\pgfpathlineto{\pgfqpoint{3.470426in}{2.007373in}}%
\pgfpathlineto{\pgfqpoint{3.493123in}{1.984000in}}%
\pgfpathlineto{\pgfqpoint{3.565576in}{1.906277in}}%
\pgfpathlineto{\pgfqpoint{3.663841in}{1.797333in}}%
\pgfpathlineto{\pgfqpoint{3.744147in}{1.705670in}}%
\pgfpathlineto{\pgfqpoint{3.824511in}{1.610667in}}%
\pgfpathlineto{\pgfqpoint{3.886222in}{1.535976in}}%
\pgfpathlineto{\pgfqpoint{3.926303in}{1.486008in}}%
\pgfpathlineto{\pgfqpoint{4.006465in}{1.383828in}}%
\pgfpathlineto{\pgfqpoint{4.046545in}{1.330962in}}%
\pgfpathlineto{\pgfqpoint{4.126707in}{1.222117in}}%
\pgfpathlineto{\pgfqpoint{4.183538in}{1.140935in}}%
\pgfpathlineto{\pgfqpoint{4.206869in}{1.107891in}}%
\pgfpathlineto{\pgfqpoint{4.269983in}{1.013333in}}%
\pgfpathlineto{\pgfqpoint{4.287030in}{0.987098in}}%
\pgfpathlineto{\pgfqpoint{4.327111in}{0.923574in}}%
\pgfpathlineto{\pgfqpoint{4.367192in}{0.857736in}}%
\pgfpathlineto{\pgfqpoint{4.407273in}{0.788974in}}%
\pgfpathlineto{\pgfqpoint{4.448245in}{0.714667in}}%
\pgfpathlineto{\pgfqpoint{4.467675in}{0.677333in}}%
\pgfpathlineto{\pgfqpoint{4.505004in}{0.602667in}}%
\pgfpathlineto{\pgfqpoint{4.511505in}{0.587754in}}%
\pgfpathlineto{\pgfqpoint{4.527515in}{0.554786in}}%
\pgfpathlineto{\pgfqpoint{4.539465in}{0.528000in}}%
\pgfpathlineto{\pgfqpoint{4.552562in}{0.528000in}}%
\pgfpathlineto{\pgfqpoint{4.544687in}{0.543995in}}%
\pgfpathlineto{\pgfqpoint{4.527515in}{0.582265in}}%
\pgfpathlineto{\pgfqpoint{4.460097in}{0.714667in}}%
\pgfpathlineto{\pgfqpoint{4.442160in}{0.747162in}}%
\pgfpathlineto{\pgfqpoint{4.431576in}{0.766696in}}%
\pgfpathlineto{\pgfqpoint{4.374887in}{0.864000in}}%
\pgfpathlineto{\pgfqpoint{4.367192in}{0.876707in}}%
\pgfpathlineto{\pgfqpoint{4.327111in}{0.941803in}}%
\pgfpathlineto{\pgfqpoint{4.305212in}{0.976000in}}%
\pgfpathlineto{\pgfqpoint{4.246949in}{1.064814in}}%
\pgfpathlineto{\pgfqpoint{4.231256in}{1.088000in}}%
\pgfpathlineto{\pgfqpoint{4.166788in}{1.180948in}}%
\pgfpathlineto{\pgfqpoint{4.099034in}{1.274667in}}%
\pgfpathlineto{\pgfqpoint{4.071288in}{1.312000in}}%
\pgfpathlineto{\pgfqpoint{4.006465in}{1.397286in}}%
\pgfpathlineto{\pgfqpoint{3.960478in}{1.455832in}}%
\pgfpathlineto{\pgfqpoint{3.956337in}{1.461333in}}%
\pgfpathlineto{\pgfqpoint{3.886222in}{1.548535in}}%
\pgfpathlineto{\pgfqpoint{3.846141in}{1.597208in}}%
\pgfpathlineto{\pgfqpoint{3.765980in}{1.692290in}}%
\pgfpathlineto{\pgfqpoint{3.674308in}{1.797333in}}%
\pgfpathlineto{\pgfqpoint{3.605657in}{1.873938in}}%
\pgfpathlineto{\pgfqpoint{3.503634in}{1.984000in}}%
\pgfpathlineto{\pgfqpoint{3.405253in}{2.086861in}}%
\pgfpathlineto{\pgfqpoint{3.322679in}{2.170667in}}%
\pgfpathlineto{\pgfqpoint{3.244929in}{2.247465in}}%
\pgfpathlineto{\pgfqpoint{3.164768in}{2.324591in}}%
\pgfpathlineto{\pgfqpoint{3.107058in}{2.378246in}}%
\pgfpathlineto{\pgfqpoint{3.084606in}{2.399764in}}%
\pgfpathlineto{\pgfqpoint{3.004444in}{2.473015in}}%
\pgfpathlineto{\pgfqpoint{2.944680in}{2.525666in}}%
\pgfpathlineto{\pgfqpoint{2.924283in}{2.544372in}}%
\pgfpathlineto{\pgfqpoint{2.861844in}{2.597841in}}%
\pgfpathlineto{\pgfqpoint{2.838234in}{2.618667in}}%
\pgfpathlineto{\pgfqpoint{2.793994in}{2.656000in}}%
\pgfpathlineto{\pgfqpoint{2.703405in}{2.730667in}}%
\pgfpathlineto{\pgfqpoint{2.627862in}{2.790565in}}%
\pgfpathlineto{\pgfqpoint{2.603636in}{2.810208in}}%
\pgfpathlineto{\pgfqpoint{2.512415in}{2.880000in}}%
\pgfpathlineto{\pgfqpoint{2.403232in}{2.960208in}}%
\pgfpathlineto{\pgfqpoint{2.323071in}{3.016407in}}%
\pgfpathlineto{\pgfqpoint{2.228846in}{3.079765in}}%
\pgfpathlineto{\pgfqpoint{2.131731in}{3.141333in}}%
\pgfpathlineto{\pgfqpoint{2.002424in}{3.217562in}}%
\pgfpathlineto{\pgfqpoint{1.962343in}{3.239450in}}%
\pgfpathlineto{\pgfqpoint{1.882182in}{3.281207in}}%
\pgfpathlineto{\pgfqpoint{1.802020in}{3.319443in}}%
\pgfpathlineto{\pgfqpoint{1.761939in}{3.337097in}}%
\pgfpathlineto{\pgfqpoint{1.740455in}{3.345322in}}%
\pgfpathlineto{\pgfqpoint{1.721859in}{3.353666in}}%
\pgfpathlineto{\pgfqpoint{1.674758in}{3.371872in}}%
\pgfpathlineto{\pgfqpoint{1.601616in}{3.396243in}}%
\pgfpathlineto{\pgfqpoint{1.596097in}{3.397525in}}%
\pgfpathlineto{\pgfqpoint{1.554090in}{3.409602in}}%
\pgfpathlineto{\pgfqpoint{1.521455in}{3.417096in}}%
\pgfpathlineto{\pgfqpoint{1.500469in}{3.420453in}}%
\pgfpathlineto{\pgfqpoint{1.481374in}{3.424801in}}%
\pgfpathlineto{\pgfqpoint{1.441293in}{3.430364in}}%
\pgfpathlineto{\pgfqpoint{1.431200in}{3.430599in}}%
\pgfpathlineto{\pgfqpoint{1.401212in}{3.433328in}}%
\pgfpathlineto{\pgfqpoint{1.361131in}{3.433099in}}%
\pgfpathlineto{\pgfqpoint{1.321051in}{3.428886in}}%
\pgfpathlineto{\pgfqpoint{1.304343in}{3.424438in}}%
\pgfpathlineto{\pgfqpoint{1.280970in}{3.419615in}}%
\pgfpathlineto{\pgfqpoint{1.267358in}{3.415345in}}%
\pgfpathlineto{\pgfqpoint{1.238877in}{3.402667in}}%
\pgfpathlineto{\pgfqpoint{1.216028in}{3.388490in}}%
\pgfpathlineto{\pgfqpoint{1.200808in}{3.376110in}}%
\pgfpathlineto{\pgfqpoint{1.189602in}{3.365333in}}%
\pgfpathlineto{\pgfqpoint{1.176424in}{3.350712in}}%
\pgfpathlineto{\pgfqpoint{1.160727in}{3.327916in}}%
\pgfpathlineto{\pgfqpoint{1.148072in}{3.302455in}}%
\pgfpathlineto{\pgfqpoint{1.138057in}{3.274450in}}%
\pgfpathlineto{\pgfqpoint{1.130747in}{3.243925in}}%
\pgfpathlineto{\pgfqpoint{1.126868in}{3.216000in}}%
\pgfpathlineto{\pgfqpoint{1.124554in}{3.178667in}}%
\pgfpathlineto{\pgfqpoint{1.125431in}{3.136877in}}%
\pgfpathlineto{\pgfqpoint{1.129327in}{3.095914in}}%
\pgfpathlineto{\pgfqpoint{1.133838in}{3.066667in}}%
\pgfpathlineto{\pgfqpoint{1.138405in}{3.045875in}}%
\pgfpathlineto{\pgfqpoint{1.140887in}{3.029333in}}%
\pgfpathlineto{\pgfqpoint{1.144870in}{3.014563in}}%
\pgfpathlineto{\pgfqpoint{1.151973in}{2.983846in}}%
\pgfpathlineto{\pgfqpoint{1.160727in}{2.949577in}}%
\pgfpathlineto{\pgfqpoint{1.196400in}{2.842667in}}%
\pgfpathlineto{\pgfqpoint{1.226335in}{2.768000in}}%
\pgfpathlineto{\pgfqpoint{1.243276in}{2.728443in}}%
\pgfpathlineto{\pgfqpoint{1.280970in}{2.648135in}}%
\pgfpathlineto{\pgfqpoint{1.361131in}{2.495440in}}%
\pgfpathlineto{\pgfqpoint{1.401212in}{2.425397in}}%
\pgfpathlineto{\pgfqpoint{1.419556in}{2.394667in}}%
\pgfpathlineto{\pgfqpoint{1.465304in}{2.320000in}}%
\pgfpathlineto{\pgfqpoint{1.486187in}{2.287150in}}%
\pgfpathlineto{\pgfqpoint{1.488858in}{2.282667in}}%
\pgfpathlineto{\pgfqpoint{1.512995in}{2.245333in}}%
\pgfpathlineto{\pgfqpoint{1.562527in}{2.170667in}}%
\pgfpathlineto{\pgfqpoint{1.588226in}{2.133333in}}%
\pgfpathlineto{\pgfqpoint{1.641697in}{2.056881in}}%
\pgfpathlineto{\pgfqpoint{1.681778in}{2.001577in}}%
\pgfpathlineto{\pgfqpoint{1.761939in}{1.894172in}}%
\pgfpathlineto{\pgfqpoint{1.805408in}{1.837822in}}%
\pgfpathlineto{\pgfqpoint{1.822942in}{1.815179in}}%
\pgfpathlineto{\pgfqpoint{1.896945in}{1.722667in}}%
\pgfpathlineto{\pgfqpoint{1.962343in}{1.643042in}}%
\pgfpathlineto{\pgfqpoint{2.013153in}{1.583326in}}%
\pgfpathlineto{\pgfqpoint{2.042505in}{1.548545in}}%
\pgfpathlineto{\pgfqpoint{2.122667in}{1.456719in}}%
\pgfpathlineto{\pgfqpoint{2.219714in}{1.349333in}}%
\pgfpathlineto{\pgfqpoint{2.267832in}{1.297882in}}%
\pgfpathlineto{\pgfqpoint{2.288983in}{1.274667in}}%
\pgfpathlineto{\pgfqpoint{2.324204in}{1.237333in}}%
\pgfpathlineto{\pgfqpoint{2.403232in}{1.155450in}}%
\pgfpathlineto{\pgfqpoint{2.507433in}{1.050667in}}%
\pgfpathlineto{\pgfqpoint{2.603636in}{0.956913in}}%
\pgfpathlineto{\pgfqpoint{2.702107in}{0.864000in}}%
\pgfpathlineto{\pgfqpoint{2.804040in}{0.770858in}}%
\pgfpathlineto{\pgfqpoint{2.910072in}{0.677333in}}%
\pgfpathlineto{\pgfqpoint{3.004444in}{0.596799in}}%
\pgfpathlineto{\pgfqpoint{3.044525in}{0.563386in}}%
\pgfpathlineto{\pgfqpoint{3.087736in}{0.528000in}}%
\pgfpathlineto{\pgfqpoint{3.087736in}{0.528000in}}%
\pgfusepath{fill}%
\end{pgfscope}%
\begin{pgfscope}%
\pgfpathrectangle{\pgfqpoint{0.800000in}{0.528000in}}{\pgfqpoint{3.968000in}{3.696000in}}%
\pgfusepath{clip}%
\pgfsetbuttcap%
\pgfsetroundjoin%
\definecolor{currentfill}{rgb}{0.280267,0.073417,0.397163}%
\pgfsetfillcolor{currentfill}%
\pgfsetlinewidth{0.000000pt}%
\definecolor{currentstroke}{rgb}{0.000000,0.000000,0.000000}%
\pgfsetstrokecolor{currentstroke}%
\pgfsetdash{}{0pt}%
\pgfpathmoveto{\pgfqpoint{3.087736in}{0.528000in}}%
\pgfpathlineto{\pgfqpoint{2.884202in}{0.699821in}}%
\pgfpathlineto{\pgfqpoint{2.835020in}{0.743523in}}%
\pgfpathlineto{\pgfqpoint{2.804040in}{0.770858in}}%
\pgfpathlineto{\pgfqpoint{2.763960in}{0.807094in}}%
\pgfpathlineto{\pgfqpoint{2.662175in}{0.901333in}}%
\pgfpathlineto{\pgfqpoint{2.622757in}{0.938667in}}%
\pgfpathlineto{\pgfqpoint{2.523475in}{1.034802in}}%
\pgfpathlineto{\pgfqpoint{2.432848in}{1.125333in}}%
\pgfpathlineto{\pgfqpoint{2.323071in}{1.238523in}}%
\pgfpathlineto{\pgfqpoint{2.267832in}{1.297882in}}%
\pgfpathlineto{\pgfqpoint{2.242909in}{1.324112in}}%
\pgfpathlineto{\pgfqpoint{2.202828in}{1.367746in}}%
\pgfpathlineto{\pgfqpoint{2.118578in}{1.461333in}}%
\pgfpathlineto{\pgfqpoint{2.066697in}{1.521200in}}%
\pgfpathlineto{\pgfqpoint{2.042505in}{1.548545in}}%
\pgfpathlineto{\pgfqpoint{2.002424in}{1.595483in}}%
\pgfpathlineto{\pgfqpoint{1.922263in}{1.691514in}}%
\pgfpathlineto{\pgfqpoint{1.822942in}{1.815179in}}%
\pgfpathlineto{\pgfqpoint{1.750419in}{1.909333in}}%
\pgfpathlineto{\pgfqpoint{1.706057in}{1.969282in}}%
\pgfpathlineto{\pgfqpoint{1.681778in}{2.001577in}}%
\pgfpathlineto{\pgfqpoint{1.537663in}{2.208000in}}%
\pgfpathlineto{\pgfqpoint{1.521455in}{2.232446in}}%
\pgfpathlineto{\pgfqpoint{1.481374in}{2.294446in}}%
\pgfpathlineto{\pgfqpoint{1.375913in}{2.469333in}}%
\pgfpathlineto{\pgfqpoint{1.334555in}{2.544000in}}%
\pgfpathlineto{\pgfqpoint{1.321051in}{2.569302in}}%
\pgfpathlineto{\pgfqpoint{1.295684in}{2.618667in}}%
\pgfpathlineto{\pgfqpoint{1.291301in}{2.628290in}}%
\pgfpathlineto{\pgfqpoint{1.277087in}{2.656000in}}%
\pgfpathlineto{\pgfqpoint{1.259526in}{2.693333in}}%
\pgfpathlineto{\pgfqpoint{1.240889in}{2.733944in}}%
\pgfpathlineto{\pgfqpoint{1.208494in}{2.812492in}}%
\pgfpathlineto{\pgfqpoint{1.193958in}{2.849048in}}%
\pgfpathlineto{\pgfqpoint{1.170633in}{2.917333in}}%
\pgfpathlineto{\pgfqpoint{1.168736in}{2.924793in}}%
\pgfpathlineto{\pgfqpoint{1.159185in}{2.954667in}}%
\pgfpathlineto{\pgfqpoint{1.140887in}{3.029333in}}%
\pgfpathlineto{\pgfqpoint{1.138405in}{3.045875in}}%
\pgfpathlineto{\pgfqpoint{1.133838in}{3.066667in}}%
\pgfpathlineto{\pgfqpoint{1.128528in}{3.104000in}}%
\pgfpathlineto{\pgfqpoint{1.125482in}{3.145838in}}%
\pgfpathlineto{\pgfqpoint{1.125065in}{3.182782in}}%
\pgfpathlineto{\pgfqpoint{1.126868in}{3.216000in}}%
\pgfpathlineto{\pgfqpoint{1.132961in}{3.253333in}}%
\pgfpathlineto{\pgfqpoint{1.138057in}{3.274450in}}%
\pgfpathlineto{\pgfqpoint{1.148072in}{3.302455in}}%
\pgfpathlineto{\pgfqpoint{1.160775in}{3.328000in}}%
\pgfpathlineto{\pgfqpoint{1.176424in}{3.350712in}}%
\pgfpathlineto{\pgfqpoint{1.200808in}{3.376110in}}%
\pgfpathlineto{\pgfqpoint{1.216028in}{3.388490in}}%
\pgfpathlineto{\pgfqpoint{1.240889in}{3.403783in}}%
\pgfpathlineto{\pgfqpoint{1.267358in}{3.415345in}}%
\pgfpathlineto{\pgfqpoint{1.280970in}{3.419615in}}%
\pgfpathlineto{\pgfqpoint{1.331161in}{3.430583in}}%
\pgfpathlineto{\pgfqpoint{1.368064in}{3.433543in}}%
\pgfpathlineto{\pgfqpoint{1.401212in}{3.433328in}}%
\pgfpathlineto{\pgfqpoint{1.452914in}{3.429176in}}%
\pgfpathlineto{\pgfqpoint{1.481374in}{3.424801in}}%
\pgfpathlineto{\pgfqpoint{1.500469in}{3.420453in}}%
\pgfpathlineto{\pgfqpoint{1.521455in}{3.417096in}}%
\pgfpathlineto{\pgfqpoint{1.561535in}{3.407603in}}%
\pgfpathlineto{\pgfqpoint{1.611762in}{3.393216in}}%
\pgfpathlineto{\pgfqpoint{1.684641in}{3.368000in}}%
\pgfpathlineto{\pgfqpoint{1.721859in}{3.353666in}}%
\pgfpathlineto{\pgfqpoint{1.740455in}{3.345322in}}%
\pgfpathlineto{\pgfqpoint{1.782626in}{3.328000in}}%
\pgfpathlineto{\pgfqpoint{1.842101in}{3.300775in}}%
\pgfpathlineto{\pgfqpoint{1.922263in}{3.260786in}}%
\pgfpathlineto{\pgfqpoint{2.005156in}{3.216000in}}%
\pgfpathlineto{\pgfqpoint{2.082586in}{3.171140in}}%
\pgfpathlineto{\pgfqpoint{2.101594in}{3.159039in}}%
\pgfpathlineto{\pgfqpoint{2.131731in}{3.141333in}}%
\pgfpathlineto{\pgfqpoint{2.228846in}{3.079765in}}%
\pgfpathlineto{\pgfqpoint{2.304110in}{3.029333in}}%
\pgfpathlineto{\pgfqpoint{2.377865in}{2.978295in}}%
\pgfpathlineto{\pgfqpoint{2.462142in}{2.917333in}}%
\pgfpathlineto{\pgfqpoint{2.540305in}{2.858344in}}%
\pgfpathlineto{\pgfqpoint{2.572089in}{2.834718in}}%
\pgfpathlineto{\pgfqpoint{2.683798in}{2.746553in}}%
\pgfpathlineto{\pgfqpoint{2.723879in}{2.714034in}}%
\pgfpathlineto{\pgfqpoint{2.804040in}{2.647598in}}%
\pgfpathlineto{\pgfqpoint{2.861844in}{2.597841in}}%
\pgfpathlineto{\pgfqpoint{2.884202in}{2.579270in}}%
\pgfpathlineto{\pgfqpoint{2.944680in}{2.525666in}}%
\pgfpathlineto{\pgfqpoint{2.966894in}{2.506667in}}%
\pgfpathlineto{\pgfqpoint{3.026410in}{2.452460in}}%
\pgfpathlineto{\pgfqpoint{3.049569in}{2.432000in}}%
\pgfpathlineto{\pgfqpoint{3.130089in}{2.357333in}}%
\pgfpathlineto{\pgfqpoint{3.186650in}{2.303049in}}%
\pgfpathlineto{\pgfqpoint{3.208583in}{2.282667in}}%
\pgfpathlineto{\pgfqpoint{3.322679in}{2.170667in}}%
\pgfpathlineto{\pgfqpoint{3.381171in}{2.110902in}}%
\pgfpathlineto{\pgfqpoint{3.405253in}{2.086861in}}%
\pgfpathlineto{\pgfqpoint{3.503634in}{1.984000in}}%
\pgfpathlineto{\pgfqpoint{3.607417in}{1.872000in}}%
\pgfpathlineto{\pgfqpoint{3.685818in}{1.784345in}}%
\pgfpathlineto{\pgfqpoint{3.795053in}{1.658253in}}%
\pgfpathlineto{\pgfqpoint{3.865833in}{1.573333in}}%
\pgfpathlineto{\pgfqpoint{3.908962in}{1.519848in}}%
\pgfpathlineto{\pgfqpoint{3.928021in}{1.497066in}}%
\pgfpathlineto{\pgfqpoint{4.014666in}{1.386667in}}%
\pgfpathlineto{\pgfqpoint{4.046545in}{1.345031in}}%
\pgfpathlineto{\pgfqpoint{4.126707in}{1.237088in}}%
\pgfpathlineto{\pgfqpoint{4.166788in}{1.180948in}}%
\pgfpathlineto{\pgfqpoint{4.231256in}{1.088000in}}%
\pgfpathlineto{\pgfqpoint{4.252662in}{1.055988in}}%
\pgfpathlineto{\pgfqpoint{4.256430in}{1.050667in}}%
\pgfpathlineto{\pgfqpoint{4.283275in}{1.009836in}}%
\pgfpathlineto{\pgfqpoint{4.305212in}{0.976000in}}%
\pgfpathlineto{\pgfqpoint{4.313075in}{0.962926in}}%
\pgfpathlineto{\pgfqpoint{4.331937in}{0.934172in}}%
\pgfpathlineto{\pgfqpoint{4.397011in}{0.826667in}}%
\pgfpathlineto{\pgfqpoint{4.407273in}{0.809030in}}%
\pgfpathlineto{\pgfqpoint{4.447354in}{0.738091in}}%
\pgfpathlineto{\pgfqpoint{4.520540in}{0.596169in}}%
\pgfpathlineto{\pgfqpoint{4.535560in}{0.565333in}}%
\pgfpathlineto{\pgfqpoint{4.552562in}{0.528000in}}%
\pgfpathlineto{\pgfqpoint{4.565659in}{0.528000in}}%
\pgfpathlineto{\pgfqpoint{4.548117in}{0.565333in}}%
\pgfpathlineto{\pgfqpoint{4.527515in}{0.608143in}}%
\pgfpathlineto{\pgfqpoint{4.487434in}{0.686047in}}%
\pgfpathlineto{\pgfqpoint{4.386110in}{0.864000in}}%
\pgfpathlineto{\pgfqpoint{4.367192in}{0.895240in}}%
\pgfpathlineto{\pgfqpoint{4.327111in}{0.959048in}}%
\pgfpathlineto{\pgfqpoint{4.206869in}{1.139053in}}%
\pgfpathlineto{\pgfqpoint{4.166788in}{1.195975in}}%
\pgfpathlineto{\pgfqpoint{4.126707in}{1.251270in}}%
\pgfpathlineto{\pgfqpoint{4.083736in}{1.309308in}}%
\pgfpathlineto{\pgfqpoint{4.070750in}{1.326788in}}%
\pgfpathlineto{\pgfqpoint{3.996018in}{1.424000in}}%
\pgfpathlineto{\pgfqpoint{3.926303in}{1.511804in}}%
\pgfpathlineto{\pgfqpoint{3.880411in}{1.567920in}}%
\pgfpathlineto{\pgfqpoint{3.862797in}{1.588847in}}%
\pgfpathlineto{\pgfqpoint{3.845358in}{1.610667in}}%
\pgfpathlineto{\pgfqpoint{3.806061in}{1.657275in}}%
\pgfpathlineto{\pgfqpoint{3.717553in}{1.760000in}}%
\pgfpathlineto{\pgfqpoint{3.645737in}{1.840975in}}%
\pgfpathlineto{\pgfqpoint{3.549014in}{1.946667in}}%
\pgfpathlineto{\pgfqpoint{3.514145in}{1.984000in}}%
\pgfpathlineto{\pgfqpoint{3.407072in}{2.096000in}}%
\pgfpathlineto{\pgfqpoint{3.348344in}{2.154993in}}%
\pgfpathlineto{\pgfqpoint{3.325091in}{2.178940in}}%
\pgfpathlineto{\pgfqpoint{3.219455in}{2.282667in}}%
\pgfpathlineto{\pgfqpoint{3.152673in}{2.346068in}}%
\pgfpathlineto{\pgfqpoint{3.124687in}{2.372867in}}%
\pgfpathlineto{\pgfqpoint{3.084606in}{2.410184in}}%
\pgfpathlineto{\pgfqpoint{2.978460in}{2.506667in}}%
\pgfpathlineto{\pgfqpoint{2.909228in}{2.567310in}}%
\pgfpathlineto{\pgfqpoint{2.884202in}{2.589636in}}%
\pgfpathlineto{\pgfqpoint{2.825875in}{2.639004in}}%
\pgfpathlineto{\pgfqpoint{2.781075in}{2.677391in}}%
\pgfpathlineto{\pgfqpoint{2.716438in}{2.730667in}}%
\pgfpathlineto{\pgfqpoint{2.655699in}{2.779160in}}%
\pgfpathlineto{\pgfqpoint{2.623187in}{2.805333in}}%
\pgfpathlineto{\pgfqpoint{2.546608in}{2.864214in}}%
\pgfpathlineto{\pgfqpoint{2.509446in}{2.893067in}}%
\pgfpathlineto{\pgfqpoint{2.425825in}{2.954667in}}%
\pgfpathlineto{\pgfqpoint{2.344509in}{3.011968in}}%
\pgfpathlineto{\pgfqpoint{2.320519in}{3.029333in}}%
\pgfpathlineto{\pgfqpoint{2.265339in}{3.066667in}}%
\pgfpathlineto{\pgfqpoint{2.188893in}{3.116980in}}%
\pgfpathlineto{\pgfqpoint{2.089613in}{3.178667in}}%
\pgfpathlineto{\pgfqpoint{2.042505in}{3.206549in}}%
\pgfpathlineto{\pgfqpoint{1.959658in}{3.253333in}}%
\pgfpathlineto{\pgfqpoint{1.874273in}{3.298033in}}%
\pgfpathlineto{\pgfqpoint{1.802020in}{3.332759in}}%
\pgfpathlineto{\pgfqpoint{1.750737in}{3.354899in}}%
\pgfpathlineto{\pgfqpoint{1.721859in}{3.367728in}}%
\pgfpathlineto{\pgfqpoint{1.601616in}{3.411431in}}%
\pgfpathlineto{\pgfqpoint{1.561535in}{3.423250in}}%
\pgfpathlineto{\pgfqpoint{1.515643in}{3.434587in}}%
\pgfpathlineto{\pgfqpoint{1.481374in}{3.442241in}}%
\pgfpathlineto{\pgfqpoint{1.441293in}{3.448513in}}%
\pgfpathlineto{\pgfqpoint{1.386788in}{3.453435in}}%
\pgfpathlineto{\pgfqpoint{1.361131in}{3.453601in}}%
\pgfpathlineto{\pgfqpoint{1.310144in}{3.450159in}}%
\pgfpathlineto{\pgfqpoint{1.280970in}{3.444668in}}%
\pgfpathlineto{\pgfqpoint{1.246876in}{3.434423in}}%
\pgfpathlineto{\pgfqpoint{1.240889in}{3.431808in}}%
\pgfpathlineto{\pgfqpoint{1.190540in}{3.402667in}}%
\pgfpathlineto{\pgfqpoint{1.156598in}{3.369179in}}%
\pgfpathlineto{\pgfqpoint{1.153995in}{3.365333in}}%
\pgfpathlineto{\pgfqpoint{1.128420in}{3.320759in}}%
\pgfpathlineto{\pgfqpoint{1.117853in}{3.290667in}}%
\pgfpathlineto{\pgfqpoint{1.109877in}{3.253333in}}%
\pgfpathlineto{\pgfqpoint{1.109254in}{3.242722in}}%
\pgfpathlineto{\pgfqpoint{1.105862in}{3.216000in}}%
\pgfpathlineto{\pgfqpoint{1.106167in}{3.202513in}}%
\pgfpathlineto{\pgfqpoint{1.105059in}{3.178667in}}%
\pgfpathlineto{\pgfqpoint{1.106896in}{3.141333in}}%
\pgfpathlineto{\pgfqpoint{1.110931in}{3.104000in}}%
\pgfpathlineto{\pgfqpoint{1.116814in}{3.066667in}}%
\pgfpathlineto{\pgfqpoint{1.124514in}{3.029333in}}%
\pgfpathlineto{\pgfqpoint{1.153746in}{2.923836in}}%
\pgfpathlineto{\pgfqpoint{1.172757in}{2.868794in}}%
\pgfpathlineto{\pgfqpoint{1.200808in}{2.797035in}}%
\pgfpathlineto{\pgfqpoint{1.233174in}{2.723481in}}%
\pgfpathlineto{\pgfqpoint{1.246711in}{2.693333in}}%
\pgfpathlineto{\pgfqpoint{1.264791in}{2.656000in}}%
\pgfpathlineto{\pgfqpoint{1.302751in}{2.581333in}}%
\pgfpathlineto{\pgfqpoint{1.343222in}{2.506667in}}%
\pgfpathlineto{\pgfqpoint{1.349651in}{2.495973in}}%
\pgfpathlineto{\pgfqpoint{1.364167in}{2.469333in}}%
\pgfpathlineto{\pgfqpoint{1.386002in}{2.432000in}}%
\pgfpathlineto{\pgfqpoint{1.441293in}{2.340579in}}%
\pgfpathlineto{\pgfqpoint{1.481374in}{2.277122in}}%
\pgfpathlineto{\pgfqpoint{1.502153in}{2.245333in}}%
\pgfpathlineto{\pgfqpoint{1.561535in}{2.156568in}}%
\pgfpathlineto{\pgfqpoint{1.601616in}{2.098554in}}%
\pgfpathlineto{\pgfqpoint{1.641697in}{2.042357in}}%
\pgfpathlineto{\pgfqpoint{1.681778in}{1.987105in}}%
\pgfpathlineto{\pgfqpoint{1.740064in}{1.909333in}}%
\pgfpathlineto{\pgfqpoint{1.802020in}{1.828752in}}%
\pgfpathlineto{\pgfqpoint{1.900144in}{1.705936in}}%
\pgfpathlineto{\pgfqpoint{1.979358in}{1.610667in}}%
\pgfpathlineto{\pgfqpoint{2.042895in}{1.536000in}}%
\pgfpathlineto{\pgfqpoint{2.097004in}{1.474764in}}%
\pgfpathlineto{\pgfqpoint{2.122667in}{1.445220in}}%
\pgfpathlineto{\pgfqpoint{2.209229in}{1.349333in}}%
\pgfpathlineto{\pgfqpoint{2.262267in}{1.292698in}}%
\pgfpathlineto{\pgfqpoint{2.282990in}{1.269877in}}%
\pgfpathlineto{\pgfqpoint{2.323071in}{1.227604in}}%
\pgfpathlineto{\pgfqpoint{2.422217in}{1.125333in}}%
\pgfpathlineto{\pgfqpoint{2.459183in}{1.088000in}}%
\pgfpathlineto{\pgfqpoint{2.563556in}{0.984910in}}%
\pgfpathlineto{\pgfqpoint{2.650880in}{0.901333in}}%
\pgfpathlineto{\pgfqpoint{2.763960in}{0.796540in}}%
\pgfpathlineto{\pgfqpoint{2.855397in}{0.714667in}}%
\pgfpathlineto{\pgfqpoint{2.898026in}{0.677333in}}%
\pgfpathlineto{\pgfqpoint{3.004444in}{0.586409in}}%
\pgfpathlineto{\pgfqpoint{3.044525in}{0.553023in}}%
\pgfpathlineto{\pgfqpoint{3.074964in}{0.528000in}}%
\pgfpathlineto{\pgfqpoint{3.084606in}{0.528000in}}%
\pgfpathlineto{\pgfqpoint{3.084606in}{0.528000in}}%
\pgfusepath{fill}%
\end{pgfscope}%
\begin{pgfscope}%
\pgfpathrectangle{\pgfqpoint{0.800000in}{0.528000in}}{\pgfqpoint{3.968000in}{3.696000in}}%
\pgfusepath{clip}%
\pgfsetbuttcap%
\pgfsetroundjoin%
\definecolor{currentfill}{rgb}{0.280267,0.073417,0.397163}%
\pgfsetfillcolor{currentfill}%
\pgfsetlinewidth{0.000000pt}%
\definecolor{currentstroke}{rgb}{0.000000,0.000000,0.000000}%
\pgfsetstrokecolor{currentstroke}%
\pgfsetdash{}{0pt}%
\pgfpathmoveto{\pgfqpoint{3.074964in}{0.528000in}}%
\pgfpathlineto{\pgfqpoint{2.985167in}{0.602667in}}%
\pgfpathlineto{\pgfqpoint{2.932587in}{0.647735in}}%
\pgfpathlineto{\pgfqpoint{2.898026in}{0.677333in}}%
\pgfpathlineto{\pgfqpoint{2.804040in}{0.760331in}}%
\pgfpathlineto{\pgfqpoint{2.763960in}{0.796540in}}%
\pgfpathlineto{\pgfqpoint{2.650880in}{0.901333in}}%
\pgfpathlineto{\pgfqpoint{2.588224in}{0.961644in}}%
\pgfpathlineto{\pgfqpoint{2.563556in}{0.984910in}}%
\pgfpathlineto{\pgfqpoint{2.459183in}{1.088000in}}%
\pgfpathlineto{\pgfqpoint{2.422217in}{1.125333in}}%
\pgfpathlineto{\pgfqpoint{2.323071in}{1.227604in}}%
\pgfpathlineto{\pgfqpoint{2.242909in}{1.312712in}}%
\pgfpathlineto{\pgfqpoint{2.202828in}{1.356313in}}%
\pgfpathlineto{\pgfqpoint{2.108388in}{1.461333in}}%
\pgfpathlineto{\pgfqpoint{2.061025in}{1.515917in}}%
\pgfpathlineto{\pgfqpoint{2.042505in}{1.536451in}}%
\pgfpathlineto{\pgfqpoint{2.002424in}{1.583353in}}%
\pgfpathlineto{\pgfqpoint{1.900144in}{1.705936in}}%
\pgfpathlineto{\pgfqpoint{1.826833in}{1.797333in}}%
\pgfpathlineto{\pgfqpoint{1.761939in}{1.880544in}}%
\pgfpathlineto{\pgfqpoint{1.681778in}{1.987105in}}%
\pgfpathlineto{\pgfqpoint{1.641697in}{2.042357in}}%
\pgfpathlineto{\pgfqpoint{1.577551in}{2.133333in}}%
\pgfpathlineto{\pgfqpoint{1.551912in}{2.170667in}}%
\pgfpathlineto{\pgfqpoint{1.502153in}{2.245333in}}%
\pgfpathlineto{\pgfqpoint{1.481374in}{2.277122in}}%
\pgfpathlineto{\pgfqpoint{1.441293in}{2.340579in}}%
\pgfpathlineto{\pgfqpoint{1.343222in}{2.506667in}}%
\pgfpathlineto{\pgfqpoint{1.302751in}{2.581333in}}%
\pgfpathlineto{\pgfqpoint{1.264791in}{2.656000in}}%
\pgfpathlineto{\pgfqpoint{1.257762in}{2.671717in}}%
\pgfpathlineto{\pgfqpoint{1.240889in}{2.705913in}}%
\pgfpathlineto{\pgfqpoint{1.195135in}{2.810618in}}%
\pgfpathlineto{\pgfqpoint{1.167078in}{2.885916in}}%
\pgfpathlineto{\pgfqpoint{1.153746in}{2.923836in}}%
\pgfpathlineto{\pgfqpoint{1.133871in}{2.992000in}}%
\pgfpathlineto{\pgfqpoint{1.131785in}{3.002375in}}%
\pgfpathlineto{\pgfqpoint{1.124040in}{3.032494in}}%
\pgfpathlineto{\pgfqpoint{1.115933in}{3.071057in}}%
\pgfpathlineto{\pgfqpoint{1.109383in}{3.114491in}}%
\pgfpathlineto{\pgfqpoint{1.105667in}{3.155286in}}%
\pgfpathlineto{\pgfqpoint{1.105059in}{3.178667in}}%
\pgfpathlineto{\pgfqpoint{1.105862in}{3.216000in}}%
\pgfpathlineto{\pgfqpoint{1.111132in}{3.262196in}}%
\pgfpathlineto{\pgfqpoint{1.120646in}{3.299250in}}%
\pgfpathlineto{\pgfqpoint{1.132019in}{3.328000in}}%
\pgfpathlineto{\pgfqpoint{1.160727in}{3.373867in}}%
\pgfpathlineto{\pgfqpoint{1.195944in}{3.407198in}}%
\pgfpathlineto{\pgfqpoint{1.200808in}{3.410332in}}%
\pgfpathlineto{\pgfqpoint{1.246876in}{3.434423in}}%
\pgfpathlineto{\pgfqpoint{1.280970in}{3.444668in}}%
\pgfpathlineto{\pgfqpoint{1.321051in}{3.451286in}}%
\pgfpathlineto{\pgfqpoint{1.333336in}{3.451443in}}%
\pgfpathlineto{\pgfqpoint{1.361131in}{3.453601in}}%
\pgfpathlineto{\pgfqpoint{1.401212in}{3.452463in}}%
\pgfpathlineto{\pgfqpoint{1.412780in}{3.450775in}}%
\pgfpathlineto{\pgfqpoint{1.448747in}{3.446943in}}%
\pgfpathlineto{\pgfqpoint{1.491870in}{3.440000in}}%
\pgfpathlineto{\pgfqpoint{1.561535in}{3.423250in}}%
\pgfpathlineto{\pgfqpoint{1.608303in}{3.408896in}}%
\pgfpathlineto{\pgfqpoint{1.649366in}{3.395524in}}%
\pgfpathlineto{\pgfqpoint{1.727482in}{3.365333in}}%
\pgfpathlineto{\pgfqpoint{1.842101in}{3.313715in}}%
\pgfpathlineto{\pgfqpoint{1.888662in}{3.290667in}}%
\pgfpathlineto{\pgfqpoint{1.959658in}{3.253333in}}%
\pgfpathlineto{\pgfqpoint{2.002424in}{3.229490in}}%
\pgfpathlineto{\pgfqpoint{2.026022in}{3.216000in}}%
\pgfpathlineto{\pgfqpoint{2.089613in}{3.178667in}}%
\pgfpathlineto{\pgfqpoint{2.150259in}{3.141333in}}%
\pgfpathlineto{\pgfqpoint{2.208907in}{3.104000in}}%
\pgfpathlineto{\pgfqpoint{2.282990in}{3.054848in}}%
\pgfpathlineto{\pgfqpoint{2.298472in}{3.043754in}}%
\pgfpathlineto{\pgfqpoint{2.323071in}{3.027593in}}%
\pgfpathlineto{\pgfqpoint{2.373778in}{2.992000in}}%
\pgfpathlineto{\pgfqpoint{2.483394in}{2.912524in}}%
\pgfpathlineto{\pgfqpoint{2.575379in}{2.842667in}}%
\pgfpathlineto{\pgfqpoint{2.683798in}{2.757113in}}%
\pgfpathlineto{\pgfqpoint{2.723879in}{2.724622in}}%
\pgfpathlineto{\pgfqpoint{2.806585in}{2.656000in}}%
\pgfpathlineto{\pgfqpoint{2.867693in}{2.603289in}}%
\pgfpathlineto{\pgfqpoint{2.893741in}{2.581333in}}%
\pgfpathlineto{\pgfqpoint{2.950483in}{2.531071in}}%
\pgfpathlineto{\pgfqpoint{2.978460in}{2.506667in}}%
\pgfpathlineto{\pgfqpoint{3.032166in}{2.457821in}}%
\pgfpathlineto{\pgfqpoint{3.060894in}{2.432000in}}%
\pgfpathlineto{\pgfqpoint{3.164768in}{2.335065in}}%
\pgfpathlineto{\pgfqpoint{3.257872in}{2.245333in}}%
\pgfpathlineto{\pgfqpoint{3.295830in}{2.208000in}}%
\pgfpathlineto{\pgfqpoint{3.370417in}{2.133333in}}%
\pgfpathlineto{\pgfqpoint{3.407072in}{2.096000in}}%
\pgfpathlineto{\pgfqpoint{3.485414in}{2.014475in}}%
\pgfpathlineto{\pgfqpoint{3.583506in}{1.909333in}}%
\pgfpathlineto{\pgfqpoint{3.685818in}{1.796156in}}%
\pgfpathlineto{\pgfqpoint{3.782109in}{1.685333in}}%
\pgfpathlineto{\pgfqpoint{3.846141in}{1.609726in}}%
\pgfpathlineto{\pgfqpoint{3.886222in}{1.561092in}}%
\pgfpathlineto{\pgfqpoint{3.967968in}{1.459858in}}%
\pgfpathlineto{\pgfqpoint{4.070750in}{1.326788in}}%
\pgfpathlineto{\pgfqpoint{4.136876in}{1.237333in}}%
\pgfpathlineto{\pgfqpoint{4.163922in}{1.200000in}}%
\pgfpathlineto{\pgfqpoint{4.233808in}{1.100241in}}%
\pgfpathlineto{\pgfqpoint{4.300034in}{1.001221in}}%
\pgfpathlineto{\pgfqpoint{4.367192in}{0.895240in}}%
\pgfpathlineto{\pgfqpoint{4.386110in}{0.864000in}}%
\pgfpathlineto{\pgfqpoint{4.430140in}{0.789333in}}%
\pgfpathlineto{\pgfqpoint{4.471949in}{0.714667in}}%
\pgfpathlineto{\pgfqpoint{4.487434in}{0.686047in}}%
\pgfpathlineto{\pgfqpoint{4.511347in}{0.640000in}}%
\pgfpathlineto{\pgfqpoint{4.548117in}{0.565333in}}%
\pgfpathlineto{\pgfqpoint{4.553666in}{0.552359in}}%
\pgfpathlineto{\pgfqpoint{4.565659in}{0.528000in}}%
\pgfpathlineto{\pgfqpoint{4.578173in}{0.528000in}}%
\pgfpathlineto{\pgfqpoint{4.562646in}{0.560722in}}%
\pgfpathlineto{\pgfqpoint{4.555177in}{0.576901in}}%
\pgfpathlineto{\pgfqpoint{4.523541in}{0.640000in}}%
\pgfpathlineto{\pgfqpoint{4.503822in}{0.677333in}}%
\pgfpathlineto{\pgfqpoint{4.462886in}{0.752000in}}%
\pgfpathlineto{\pgfqpoint{4.419707in}{0.826667in}}%
\pgfpathlineto{\pgfqpoint{4.407273in}{0.847509in}}%
\pgfpathlineto{\pgfqpoint{4.367192in}{0.912958in}}%
\pgfpathlineto{\pgfqpoint{4.302718in}{1.013333in}}%
\pgfpathlineto{\pgfqpoint{4.246949in}{1.096423in}}%
\pgfpathlineto{\pgfqpoint{4.166788in}{1.210411in}}%
\pgfpathlineto{\pgfqpoint{4.122592in}{1.270834in}}%
\pgfpathlineto{\pgfqpoint{4.119910in}{1.274667in}}%
\pgfpathlineto{\pgfqpoint{4.086626in}{1.319300in}}%
\pgfpathlineto{\pgfqpoint{4.006222in}{1.424226in}}%
\pgfpathlineto{\pgfqpoint{3.916886in}{1.536000in}}%
\pgfpathlineto{\pgfqpoint{3.846141in}{1.621721in}}%
\pgfpathlineto{\pgfqpoint{3.806061in}{1.669192in}}%
\pgfpathlineto{\pgfqpoint{3.725899in}{1.762211in}}%
\pgfpathlineto{\pgfqpoint{3.627845in}{1.872000in}}%
\pgfpathlineto{\pgfqpoint{3.580514in}{1.923248in}}%
\pgfpathlineto{\pgfqpoint{3.559424in}{1.946667in}}%
\pgfpathlineto{\pgfqpoint{3.518606in}{1.990417in}}%
\pgfpathlineto{\pgfqpoint{3.417425in}{2.096000in}}%
\pgfpathlineto{\pgfqpoint{3.353924in}{2.160190in}}%
\pgfpathlineto{\pgfqpoint{3.325091in}{2.189525in}}%
\pgfpathlineto{\pgfqpoint{3.230326in}{2.282667in}}%
\pgfpathlineto{\pgfqpoint{3.191544in}{2.320000in}}%
\pgfpathlineto{\pgfqpoint{3.084606in}{2.420603in}}%
\pgfpathlineto{\pgfqpoint{2.990026in}{2.506667in}}%
\pgfpathlineto{\pgfqpoint{2.948086in}{2.544000in}}%
\pgfpathlineto{\pgfqpoint{2.844121in}{2.634384in}}%
\pgfpathlineto{\pgfqpoint{2.804040in}{2.668388in}}%
\pgfpathlineto{\pgfqpoint{2.723879in}{2.735035in}}%
\pgfpathlineto{\pgfqpoint{2.661668in}{2.784720in}}%
\pgfpathlineto{\pgfqpoint{2.636541in}{2.805333in}}%
\pgfpathlineto{\pgfqpoint{2.588899in}{2.842667in}}%
\pgfpathlineto{\pgfqpoint{2.483394in}{2.923135in}}%
\pgfpathlineto{\pgfqpoint{2.419283in}{2.969618in}}%
\pgfpathlineto{\pgfqpoint{2.388913in}{2.992000in}}%
\pgfpathlineto{\pgfqpoint{2.323071in}{3.038397in}}%
\pgfpathlineto{\pgfqpoint{2.305241in}{3.050060in}}%
\pgfpathlineto{\pgfqpoint{2.281999in}{3.066667in}}%
\pgfpathlineto{\pgfqpoint{2.168350in}{3.141333in}}%
\pgfpathlineto{\pgfqpoint{2.108462in}{3.178667in}}%
\pgfpathlineto{\pgfqpoint{2.042505in}{3.218405in}}%
\pgfpathlineto{\pgfqpoint{1.994224in}{3.245695in}}%
\pgfpathlineto{\pgfqpoint{1.962343in}{3.263859in}}%
\pgfpathlineto{\pgfqpoint{1.912542in}{3.290667in}}%
\pgfpathlineto{\pgfqpoint{1.839273in}{3.328000in}}%
\pgfpathlineto{\pgfqpoint{1.759110in}{3.365333in}}%
\pgfpathlineto{\pgfqpoint{1.681778in}{3.397523in}}%
\pgfpathlineto{\pgfqpoint{1.641697in}{3.412502in}}%
\pgfpathlineto{\pgfqpoint{1.619617in}{3.419434in}}%
\pgfpathlineto{\pgfqpoint{1.601616in}{3.426259in}}%
\pgfpathlineto{\pgfqpoint{1.557423in}{3.440000in}}%
\pgfpathlineto{\pgfqpoint{1.481374in}{3.458733in}}%
\pgfpathlineto{\pgfqpoint{1.464081in}{3.461226in}}%
\pgfpathlineto{\pgfqpoint{1.441293in}{3.466024in}}%
\pgfpathlineto{\pgfqpoint{1.430105in}{3.466912in}}%
\pgfpathlineto{\pgfqpoint{1.401212in}{3.471128in}}%
\pgfpathlineto{\pgfqpoint{1.356737in}{3.473241in}}%
\pgfpathlineto{\pgfqpoint{1.321051in}{3.472782in}}%
\pgfpathlineto{\pgfqpoint{1.280970in}{3.467929in}}%
\pgfpathlineto{\pgfqpoint{1.266396in}{3.463759in}}%
\pgfpathlineto{\pgfqpoint{1.240889in}{3.457933in}}%
\pgfpathlineto{\pgfqpoint{1.226729in}{3.453189in}}%
\pgfpathlineto{\pgfqpoint{1.198585in}{3.440000in}}%
\pgfpathlineto{\pgfqpoint{1.176206in}{3.425582in}}%
\pgfpathlineto{\pgfqpoint{1.160727in}{3.412604in}}%
\pgfpathlineto{\pgfqpoint{1.150644in}{3.402667in}}%
\pgfpathlineto{\pgfqpoint{1.137125in}{3.387318in}}%
\pgfpathlineto{\pgfqpoint{1.120646in}{3.362462in}}%
\pgfpathlineto{\pgfqpoint{1.109032in}{3.338818in}}%
\pgfpathlineto{\pgfqpoint{1.105145in}{3.328000in}}%
\pgfpathlineto{\pgfqpoint{1.100131in}{3.308891in}}%
\pgfpathlineto{\pgfqpoint{1.091742in}{3.280256in}}%
\pgfpathlineto{\pgfqpoint{1.087930in}{3.253333in}}%
\pgfpathlineto{\pgfqpoint{1.085391in}{3.216000in}}%
\pgfpathlineto{\pgfqpoint{1.086011in}{3.173595in}}%
\pgfpathlineto{\pgfqpoint{1.089637in}{3.132884in}}%
\pgfpathlineto{\pgfqpoint{1.096095in}{3.089535in}}%
\pgfpathlineto{\pgfqpoint{1.108948in}{3.029333in}}%
\pgfpathlineto{\pgfqpoint{1.111545in}{3.020856in}}%
\pgfpathlineto{\pgfqpoint{1.120646in}{2.984651in}}%
\pgfpathlineto{\pgfqpoint{1.151776in}{2.888337in}}%
\pgfpathlineto{\pgfqpoint{1.173829in}{2.830463in}}%
\pgfpathlineto{\pgfqpoint{1.200808in}{2.766094in}}%
\pgfpathlineto{\pgfqpoint{1.240889in}{2.679656in}}%
\pgfpathlineto{\pgfqpoint{1.314512in}{2.537909in}}%
\pgfpathlineto{\pgfqpoint{1.331602in}{2.506667in}}%
\pgfpathlineto{\pgfqpoint{1.352831in}{2.469333in}}%
\pgfpathlineto{\pgfqpoint{1.401212in}{2.387617in}}%
\pgfpathlineto{\pgfqpoint{1.467050in}{2.282667in}}%
\pgfpathlineto{\pgfqpoint{1.521455in}{2.199915in}}%
\pgfpathlineto{\pgfqpoint{1.601616in}{2.083836in}}%
\pgfpathlineto{\pgfqpoint{1.644566in}{2.024006in}}%
\pgfpathlineto{\pgfqpoint{1.646401in}{2.021333in}}%
\pgfpathlineto{\pgfqpoint{1.693718in}{1.957788in}}%
\pgfpathlineto{\pgfqpoint{1.721859in}{1.919699in}}%
\pgfpathlineto{\pgfqpoint{1.882182in}{1.715564in}}%
\pgfpathlineto{\pgfqpoint{1.969114in}{1.610667in}}%
\pgfpathlineto{\pgfqpoint{2.042505in}{1.524866in}}%
\pgfpathlineto{\pgfqpoint{2.091311in}{1.469460in}}%
\pgfpathlineto{\pgfqpoint{2.122667in}{1.433722in}}%
\pgfpathlineto{\pgfqpoint{2.202828in}{1.345065in}}%
\pgfpathlineto{\pgfqpoint{2.256702in}{1.287514in}}%
\pgfpathlineto{\pgfqpoint{2.282990in}{1.258978in}}%
\pgfpathlineto{\pgfqpoint{2.323071in}{1.216734in}}%
\pgfpathlineto{\pgfqpoint{2.411586in}{1.125333in}}%
\pgfpathlineto{\pgfqpoint{2.448447in}{1.088000in}}%
\pgfpathlineto{\pgfqpoint{2.561791in}{0.976000in}}%
\pgfpathlineto{\pgfqpoint{2.641762in}{0.899512in}}%
\pgfpathlineto{\pgfqpoint{2.679487in}{0.864000in}}%
\pgfpathlineto{\pgfqpoint{2.762245in}{0.787737in}}%
\pgfpathlineto{\pgfqpoint{2.782890in}{0.769633in}}%
\pgfpathlineto{\pgfqpoint{2.804040in}{0.749889in}}%
\pgfpathlineto{\pgfqpoint{2.864802in}{0.696597in}}%
\pgfpathlineto{\pgfqpoint{2.885980in}{0.677333in}}%
\pgfpathlineto{\pgfqpoint{3.004444in}{0.576020in}}%
\pgfpathlineto{\pgfqpoint{3.044525in}{0.542660in}}%
\pgfpathlineto{\pgfqpoint{3.062359in}{0.528000in}}%
\pgfpathlineto{\pgfqpoint{3.062359in}{0.528000in}}%
\pgfusepath{fill}%
\end{pgfscope}%
\begin{pgfscope}%
\pgfpathrectangle{\pgfqpoint{0.800000in}{0.528000in}}{\pgfqpoint{3.968000in}{3.696000in}}%
\pgfusepath{clip}%
\pgfsetbuttcap%
\pgfsetroundjoin%
\definecolor{currentfill}{rgb}{0.280267,0.073417,0.397163}%
\pgfsetfillcolor{currentfill}%
\pgfsetlinewidth{0.000000pt}%
\definecolor{currentstroke}{rgb}{0.000000,0.000000,0.000000}%
\pgfsetstrokecolor{currentstroke}%
\pgfsetdash{}{0pt}%
\pgfpathmoveto{\pgfqpoint{3.062359in}{0.528000in}}%
\pgfpathlineto{\pgfqpoint{2.964364in}{0.609840in}}%
\pgfpathlineto{\pgfqpoint{2.884202in}{0.678879in}}%
\pgfpathlineto{\pgfqpoint{2.823712in}{0.732990in}}%
\pgfpathlineto{\pgfqpoint{2.801699in}{0.752000in}}%
\pgfpathlineto{\pgfqpoint{2.760434in}{0.789333in}}%
\pgfpathlineto{\pgfqpoint{2.643717in}{0.897592in}}%
\pgfpathlineto{\pgfqpoint{2.582699in}{0.956498in}}%
\pgfpathlineto{\pgfqpoint{2.561791in}{0.976000in}}%
\pgfpathlineto{\pgfqpoint{2.448447in}{1.088000in}}%
\pgfpathlineto{\pgfqpoint{2.388769in}{1.149194in}}%
\pgfpathlineto{\pgfqpoint{2.350703in}{1.188405in}}%
\pgfpathlineto{\pgfqpoint{2.323071in}{1.216734in}}%
\pgfpathlineto{\pgfqpoint{2.233407in}{1.312000in}}%
\pgfpathlineto{\pgfqpoint{2.198917in}{1.349333in}}%
\pgfpathlineto{\pgfqpoint{2.122667in}{1.433722in}}%
\pgfpathlineto{\pgfqpoint{2.032876in}{1.536000in}}%
\pgfpathlineto{\pgfqpoint{2.000700in}{1.573333in}}%
\pgfpathlineto{\pgfqpoint{1.922263in}{1.666820in}}%
\pgfpathlineto{\pgfqpoint{1.842101in}{1.765187in}}%
\pgfpathlineto{\pgfqpoint{1.748720in}{1.884313in}}%
\pgfpathlineto{\pgfqpoint{1.673796in}{1.984000in}}%
\pgfpathlineto{\pgfqpoint{1.641697in}{2.027832in}}%
\pgfpathlineto{\pgfqpoint{1.561535in}{2.141081in}}%
\pgfpathlineto{\pgfqpoint{1.487690in}{2.251217in}}%
\pgfpathlineto{\pgfqpoint{1.467050in}{2.282667in}}%
\pgfpathlineto{\pgfqpoint{1.457750in}{2.297996in}}%
\pgfpathlineto{\pgfqpoint{1.441293in}{2.322805in}}%
\pgfpathlineto{\pgfqpoint{1.419896in}{2.357333in}}%
\pgfpathlineto{\pgfqpoint{1.374693in}{2.432000in}}%
\pgfpathlineto{\pgfqpoint{1.361131in}{2.455068in}}%
\pgfpathlineto{\pgfqpoint{1.321051in}{2.525602in}}%
\pgfpathlineto{\pgfqpoint{1.252494in}{2.656000in}}%
\pgfpathlineto{\pgfqpoint{1.240889in}{2.679656in}}%
\pgfpathlineto{\pgfqpoint{1.216901in}{2.730667in}}%
\pgfpathlineto{\pgfqpoint{1.199952in}{2.768000in}}%
\pgfpathlineto{\pgfqpoint{1.151776in}{2.888337in}}%
\pgfpathlineto{\pgfqpoint{1.129622in}{2.954667in}}%
\pgfpathlineto{\pgfqpoint{1.127942in}{2.961462in}}%
\pgfpathlineto{\pgfqpoint{1.118476in}{2.992000in}}%
\pgfpathlineto{\pgfqpoint{1.096095in}{3.089535in}}%
\pgfpathlineto{\pgfqpoint{1.088854in}{3.141333in}}%
\pgfpathlineto{\pgfqpoint{1.086119in}{3.183840in}}%
\pgfpathlineto{\pgfqpoint{1.086039in}{3.221098in}}%
\pgfpathlineto{\pgfqpoint{1.087930in}{3.253333in}}%
\pgfpathlineto{\pgfqpoint{1.094200in}{3.290667in}}%
\pgfpathlineto{\pgfqpoint{1.100131in}{3.308891in}}%
\pgfpathlineto{\pgfqpoint{1.109032in}{3.338818in}}%
\pgfpathlineto{\pgfqpoint{1.122245in}{3.365333in}}%
\pgfpathlineto{\pgfqpoint{1.137125in}{3.387318in}}%
\pgfpathlineto{\pgfqpoint{1.160727in}{3.412604in}}%
\pgfpathlineto{\pgfqpoint{1.176206in}{3.425582in}}%
\pgfpathlineto{\pgfqpoint{1.200808in}{3.441281in}}%
\pgfpathlineto{\pgfqpoint{1.226729in}{3.453189in}}%
\pgfpathlineto{\pgfqpoint{1.240889in}{3.457933in}}%
\pgfpathlineto{\pgfqpoint{1.289369in}{3.469510in}}%
\pgfpathlineto{\pgfqpoint{1.321051in}{3.472782in}}%
\pgfpathlineto{\pgfqpoint{1.365187in}{3.473555in}}%
\pgfpathlineto{\pgfqpoint{1.401212in}{3.471128in}}%
\pgfpathlineto{\pgfqpoint{1.481374in}{3.458733in}}%
\pgfpathlineto{\pgfqpoint{1.521455in}{3.449616in}}%
\pgfpathlineto{\pgfqpoint{1.529147in}{3.447165in}}%
\pgfpathlineto{\pgfqpoint{1.563237in}{3.438415in}}%
\pgfpathlineto{\pgfqpoint{1.641697in}{3.412502in}}%
\pgfpathlineto{\pgfqpoint{1.681778in}{3.397523in}}%
\pgfpathlineto{\pgfqpoint{1.721859in}{3.381196in}}%
\pgfpathlineto{\pgfqpoint{1.761939in}{3.364124in}}%
\pgfpathlineto{\pgfqpoint{1.802020in}{3.345658in}}%
\pgfpathlineto{\pgfqpoint{1.845129in}{3.325179in}}%
\pgfpathlineto{\pgfqpoint{1.934889in}{3.278906in}}%
\pgfpathlineto{\pgfqpoint{2.002424in}{3.241418in}}%
\pgfpathlineto{\pgfqpoint{2.018905in}{3.231351in}}%
\pgfpathlineto{\pgfqpoint{2.046535in}{3.216000in}}%
\pgfpathlineto{\pgfqpoint{2.116900in}{3.173295in}}%
\pgfpathlineto{\pgfqpoint{2.122667in}{3.169991in}}%
\pgfpathlineto{\pgfqpoint{2.140683in}{3.158114in}}%
\pgfpathlineto{\pgfqpoint{2.168350in}{3.141333in}}%
\pgfpathlineto{\pgfqpoint{2.281999in}{3.066667in}}%
\pgfpathlineto{\pgfqpoint{2.336032in}{3.029333in}}%
\pgfpathlineto{\pgfqpoint{2.388913in}{2.992000in}}%
\pgfpathlineto{\pgfqpoint{2.443313in}{2.952810in}}%
\pgfpathlineto{\pgfqpoint{2.491113in}{2.917333in}}%
\pgfpathlineto{\pgfqpoint{2.603636in}{2.831217in}}%
\pgfpathlineto{\pgfqpoint{2.661668in}{2.784720in}}%
\pgfpathlineto{\pgfqpoint{2.683798in}{2.767673in}}%
\pgfpathlineto{\pgfqpoint{2.729181in}{2.730667in}}%
\pgfpathlineto{\pgfqpoint{2.818666in}{2.656000in}}%
\pgfpathlineto{\pgfqpoint{2.873542in}{2.608737in}}%
\pgfpathlineto{\pgfqpoint{2.905560in}{2.581333in}}%
\pgfpathlineto{\pgfqpoint{2.956285in}{2.536475in}}%
\pgfpathlineto{\pgfqpoint{2.990026in}{2.506667in}}%
\pgfpathlineto{\pgfqpoint{3.044525in}{2.457413in}}%
\pgfpathlineto{\pgfqpoint{3.152276in}{2.357333in}}%
\pgfpathlineto{\pgfqpoint{3.217587in}{2.294532in}}%
\pgfpathlineto{\pgfqpoint{3.244929in}{2.268524in}}%
\pgfpathlineto{\pgfqpoint{3.343895in}{2.170667in}}%
\pgfpathlineto{\pgfqpoint{3.392287in}{2.121257in}}%
\pgfpathlineto{\pgfqpoint{3.417425in}{2.096000in}}%
\pgfpathlineto{\pgfqpoint{3.525495in}{1.983107in}}%
\pgfpathlineto{\pgfqpoint{3.580514in}{1.923248in}}%
\pgfpathlineto{\pgfqpoint{3.605657in}{1.896424in}}%
\pgfpathlineto{\pgfqpoint{3.654180in}{1.842530in}}%
\pgfpathlineto{\pgfqpoint{3.694845in}{1.797333in}}%
\pgfpathlineto{\pgfqpoint{3.765980in}{1.716053in}}%
\pgfpathlineto{\pgfqpoint{3.855377in}{1.610667in}}%
\pgfpathlineto{\pgfqpoint{3.887290in}{1.572339in}}%
\pgfpathlineto{\pgfqpoint{3.976838in}{1.461333in}}%
\pgfpathlineto{\pgfqpoint{4.046545in}{1.371967in}}%
\pgfpathlineto{\pgfqpoint{4.105945in}{1.292661in}}%
\pgfpathlineto{\pgfqpoint{4.126707in}{1.265439in}}%
\pgfpathlineto{\pgfqpoint{4.206869in}{1.154137in}}%
\pgfpathlineto{\pgfqpoint{4.287030in}{1.037070in}}%
\pgfpathlineto{\pgfqpoint{4.350964in}{0.938667in}}%
\pgfpathlineto{\pgfqpoint{4.407273in}{0.847509in}}%
\pgfpathlineto{\pgfqpoint{4.487434in}{0.707952in}}%
\pgfpathlineto{\pgfqpoint{4.527515in}{0.632170in}}%
\pgfpathlineto{\pgfqpoint{4.578173in}{0.528000in}}%
\pgfpathlineto{\pgfqpoint{4.590586in}{0.528000in}}%
\pgfpathlineto{\pgfqpoint{4.583178in}{0.542514in}}%
\pgfpathlineto{\pgfqpoint{4.567596in}{0.576179in}}%
\pgfpathlineto{\pgfqpoint{4.527515in}{0.654848in}}%
\pgfpathlineto{\pgfqpoint{4.447354in}{0.798815in}}%
\pgfpathlineto{\pgfqpoint{4.407273in}{0.865988in}}%
\pgfpathlineto{\pgfqpoint{4.385280in}{0.901333in}}%
\pgfpathlineto{\pgfqpoint{4.327111in}{0.992465in}}%
\pgfpathlineto{\pgfqpoint{4.263127in}{1.088000in}}%
\pgfpathlineto{\pgfqpoint{4.237405in}{1.125333in}}%
\pgfpathlineto{\pgfqpoint{4.184527in}{1.200000in}}%
\pgfpathlineto{\pgfqpoint{4.126707in}{1.279358in}}%
\pgfpathlineto{\pgfqpoint{4.086626in}{1.332704in}}%
\pgfpathlineto{\pgfqpoint{4.006465in}{1.436597in}}%
\pgfpathlineto{\pgfqpoint{3.926303in}{1.536953in}}%
\pgfpathlineto{\pgfqpoint{3.834062in}{1.648000in}}%
\pgfpathlineto{\pgfqpoint{3.765980in}{1.727709in}}%
\pgfpathlineto{\pgfqpoint{3.671636in}{1.834667in}}%
\pgfpathlineto{\pgfqpoint{3.622894in}{1.888056in}}%
\pgfpathlineto{\pgfqpoint{3.604128in}{1.909333in}}%
\pgfpathlineto{\pgfqpoint{3.548813in}{1.968386in}}%
\pgfpathlineto{\pgfqpoint{3.525495in}{1.993870in}}%
\pgfpathlineto{\pgfqpoint{3.427779in}{2.096000in}}%
\pgfpathlineto{\pgfqpoint{3.391323in}{2.133333in}}%
\pgfpathlineto{\pgfqpoint{3.317146in}{2.208000in}}%
\pgfpathlineto{\pgfqpoint{3.279400in}{2.245333in}}%
\pgfpathlineto{\pgfqpoint{3.163370in}{2.357333in}}%
\pgfpathlineto{\pgfqpoint{3.042842in}{2.469333in}}%
\pgfpathlineto{\pgfqpoint{2.924283in}{2.575308in}}%
\pgfpathlineto{\pgfqpoint{2.879391in}{2.614185in}}%
\pgfpathlineto{\pgfqpoint{2.844121in}{2.644643in}}%
\pgfpathlineto{\pgfqpoint{2.804040in}{2.678621in}}%
\pgfpathlineto{\pgfqpoint{2.695908in}{2.768000in}}%
\pgfpathlineto{\pgfqpoint{2.602419in}{2.842667in}}%
\pgfpathlineto{\pgfqpoint{2.483394in}{2.933558in}}%
\pgfpathlineto{\pgfqpoint{2.425669in}{2.975566in}}%
\pgfpathlineto{\pgfqpoint{2.400600in}{2.994452in}}%
\pgfpathlineto{\pgfqpoint{2.282990in}{3.076735in}}%
\pgfpathlineto{\pgfqpoint{2.202828in}{3.130173in}}%
\pgfpathlineto{\pgfqpoint{2.113895in}{3.186837in}}%
\pgfpathlineto{\pgfqpoint{2.042505in}{3.229851in}}%
\pgfpathlineto{\pgfqpoint{2.026639in}{3.238555in}}%
\pgfpathlineto{\pgfqpoint{2.002424in}{3.253346in}}%
\pgfpathlineto{\pgfqpoint{1.926898in}{3.294984in}}%
\pgfpathlineto{\pgfqpoint{1.904505in}{3.307207in}}%
\pgfpathlineto{\pgfqpoint{1.816835in}{3.351534in}}%
\pgfpathlineto{\pgfqpoint{1.737685in}{3.387925in}}%
\pgfpathlineto{\pgfqpoint{1.665717in}{3.417626in}}%
\pgfpathlineto{\pgfqpoint{1.601616in}{3.441029in}}%
\pgfpathlineto{\pgfqpoint{1.521455in}{3.465201in}}%
\pgfpathlineto{\pgfqpoint{1.470747in}{3.477333in}}%
\pgfpathlineto{\pgfqpoint{1.441293in}{3.483151in}}%
\pgfpathlineto{\pgfqpoint{1.361131in}{3.492423in}}%
\pgfpathlineto{\pgfqpoint{1.344013in}{3.493278in}}%
\pgfpathlineto{\pgfqpoint{1.304753in}{3.492514in}}%
\pgfpathlineto{\pgfqpoint{1.280970in}{3.490070in}}%
\pgfpathlineto{\pgfqpoint{1.240889in}{3.482755in}}%
\pgfpathlineto{\pgfqpoint{1.206720in}{3.471827in}}%
\pgfpathlineto{\pgfqpoint{1.200808in}{3.469111in}}%
\pgfpathlineto{\pgfqpoint{1.152070in}{3.440000in}}%
\pgfpathlineto{\pgfqpoint{1.115903in}{3.402667in}}%
\pgfpathlineto{\pgfqpoint{1.110943in}{3.393628in}}%
\pgfpathlineto{\pgfqpoint{1.089816in}{3.356717in}}%
\pgfpathlineto{\pgfqpoint{1.079602in}{3.328000in}}%
\pgfpathlineto{\pgfqpoint{1.071453in}{3.290667in}}%
\pgfpathlineto{\pgfqpoint{1.070887in}{3.281651in}}%
\pgfpathlineto{\pgfqpoint{1.067216in}{3.253333in}}%
\pgfpathlineto{\pgfqpoint{1.066162in}{3.216000in}}%
\pgfpathlineto{\pgfqpoint{1.067735in}{3.178667in}}%
\pgfpathlineto{\pgfqpoint{1.069301in}{3.168174in}}%
\pgfpathlineto{\pgfqpoint{1.072950in}{3.134240in}}%
\pgfpathlineto{\pgfqpoint{1.080566in}{3.086546in}}%
\pgfpathlineto{\pgfqpoint{1.085495in}{3.062075in}}%
\pgfpathlineto{\pgfqpoint{1.103893in}{2.992000in}}%
\pgfpathlineto{\pgfqpoint{1.120646in}{2.938320in}}%
\pgfpathlineto{\pgfqpoint{1.141427in}{2.880000in}}%
\pgfpathlineto{\pgfqpoint{1.146906in}{2.867126in}}%
\pgfpathlineto{\pgfqpoint{1.160727in}{2.830573in}}%
\pgfpathlineto{\pgfqpoint{1.187433in}{2.768000in}}%
\pgfpathlineto{\pgfqpoint{1.204236in}{2.730667in}}%
\pgfpathlineto{\pgfqpoint{1.222095in}{2.693333in}}%
\pgfpathlineto{\pgfqpoint{1.259495in}{2.618667in}}%
\pgfpathlineto{\pgfqpoint{1.266864in}{2.605528in}}%
\pgfpathlineto{\pgfqpoint{1.280970in}{2.577620in}}%
\pgfpathlineto{\pgfqpoint{1.321051in}{2.504888in}}%
\pgfpathlineto{\pgfqpoint{1.341638in}{2.469333in}}%
\pgfpathlineto{\pgfqpoint{1.386002in}{2.394667in}}%
\pgfpathlineto{\pgfqpoint{1.401212in}{2.369764in}}%
\pgfpathlineto{\pgfqpoint{1.441293in}{2.305974in}}%
\pgfpathlineto{\pgfqpoint{1.582698in}{2.096000in}}%
\pgfpathlineto{\pgfqpoint{1.641697in}{2.013725in}}%
\pgfpathlineto{\pgfqpoint{1.687549in}{1.952042in}}%
\pgfpathlineto{\pgfqpoint{1.713621in}{1.917006in}}%
\pgfpathlineto{\pgfqpoint{1.777177in}{1.834667in}}%
\pgfpathlineto{\pgfqpoint{1.806428in}{1.797333in}}%
\pgfpathlineto{\pgfqpoint{1.882182in}{1.703322in}}%
\pgfpathlineto{\pgfqpoint{1.962343in}{1.606722in}}%
\pgfpathlineto{\pgfqpoint{2.055279in}{1.498667in}}%
\pgfpathlineto{\pgfqpoint{2.122667in}{1.422297in}}%
\pgfpathlineto{\pgfqpoint{2.162747in}{1.377929in}}%
\pgfpathlineto{\pgfqpoint{2.258033in}{1.274667in}}%
\pgfpathlineto{\pgfqpoint{2.328696in}{1.200000in}}%
\pgfpathlineto{\pgfqpoint{2.383346in}{1.144144in}}%
\pgfpathlineto{\pgfqpoint{2.403232in}{1.123108in}}%
\pgfpathlineto{\pgfqpoint{2.443313in}{1.082594in}}%
\pgfpathlineto{\pgfqpoint{2.551218in}{0.976000in}}%
\pgfpathlineto{\pgfqpoint{2.603636in}{0.925445in}}%
\pgfpathlineto{\pgfqpoint{2.723879in}{0.812628in}}%
\pgfpathlineto{\pgfqpoint{2.844121in}{0.704027in}}%
\pgfpathlineto{\pgfqpoint{2.884202in}{0.668747in}}%
\pgfpathlineto{\pgfqpoint{3.004801in}{0.565333in}}%
\pgfpathlineto{\pgfqpoint{3.049753in}{0.528000in}}%
\pgfpathlineto{\pgfqpoint{3.049753in}{0.528000in}}%
\pgfusepath{fill}%
\end{pgfscope}%
\begin{pgfscope}%
\pgfpathrectangle{\pgfqpoint{0.800000in}{0.528000in}}{\pgfqpoint{3.968000in}{3.696000in}}%
\pgfusepath{clip}%
\pgfsetbuttcap%
\pgfsetroundjoin%
\definecolor{currentfill}{rgb}{0.280894,0.078907,0.402329}%
\pgfsetfillcolor{currentfill}%
\pgfsetlinewidth{0.000000pt}%
\definecolor{currentstroke}{rgb}{0.000000,0.000000,0.000000}%
\pgfsetstrokecolor{currentstroke}%
\pgfsetdash{}{0pt}%
\pgfpathmoveto{\pgfqpoint{3.049753in}{0.528000in}}%
\pgfpathlineto{\pgfqpoint{2.964364in}{0.599546in}}%
\pgfpathlineto{\pgfqpoint{2.917272in}{0.640000in}}%
\pgfpathlineto{\pgfqpoint{2.832168in}{0.714667in}}%
\pgfpathlineto{\pgfqpoint{2.777258in}{0.764387in}}%
\pgfpathlineto{\pgfqpoint{2.749321in}{0.789333in}}%
\pgfpathlineto{\pgfqpoint{2.643717in}{0.887365in}}%
\pgfpathlineto{\pgfqpoint{2.596770in}{0.932271in}}%
\pgfpathlineto{\pgfqpoint{2.551218in}{0.976000in}}%
\pgfpathlineto{\pgfqpoint{2.443313in}{1.082594in}}%
\pgfpathlineto{\pgfqpoint{2.363152in}{1.164179in}}%
\pgfpathlineto{\pgfqpoint{2.323071in}{1.205865in}}%
\pgfpathlineto{\pgfqpoint{2.223272in}{1.312000in}}%
\pgfpathlineto{\pgfqpoint{2.188876in}{1.349333in}}%
\pgfpathlineto{\pgfqpoint{2.088010in}{1.461333in}}%
\pgfpathlineto{\pgfqpoint{2.042505in}{1.513301in}}%
\pgfpathlineto{\pgfqpoint{1.946919in}{1.625034in}}%
\pgfpathlineto{\pgfqpoint{1.866440in}{1.722667in}}%
\pgfpathlineto{\pgfqpoint{1.802020in}{1.802915in}}%
\pgfpathlineto{\pgfqpoint{1.761939in}{1.854205in}}%
\pgfpathlineto{\pgfqpoint{1.681778in}{1.959501in}}%
\pgfpathlineto{\pgfqpoint{1.601616in}{2.069259in}}%
\pgfpathlineto{\pgfqpoint{1.521455in}{2.184368in}}%
\pgfpathlineto{\pgfqpoint{1.401212in}{2.369764in}}%
\pgfpathlineto{\pgfqpoint{1.299409in}{2.544000in}}%
\pgfpathlineto{\pgfqpoint{1.259495in}{2.618667in}}%
\pgfpathlineto{\pgfqpoint{1.222095in}{2.693333in}}%
\pgfpathlineto{\pgfqpoint{1.215916in}{2.707405in}}%
\pgfpathlineto{\pgfqpoint{1.200808in}{2.738213in}}%
\pgfpathlineto{\pgfqpoint{1.171219in}{2.805333in}}%
\pgfpathlineto{\pgfqpoint{1.155760in}{2.842667in}}%
\pgfpathlineto{\pgfqpoint{1.112745in}{2.962027in}}%
\pgfpathlineto{\pgfqpoint{1.093630in}{3.029333in}}%
\pgfpathlineto{\pgfqpoint{1.084555in}{3.066667in}}%
\pgfpathlineto{\pgfqpoint{1.077114in}{3.104000in}}%
\pgfpathlineto{\pgfqpoint{1.067735in}{3.178667in}}%
\pgfpathlineto{\pgfqpoint{1.066670in}{3.191610in}}%
\pgfpathlineto{\pgfqpoint{1.065942in}{3.229621in}}%
\pgfpathlineto{\pgfqpoint{1.067216in}{3.253333in}}%
\pgfpathlineto{\pgfqpoint{1.072532in}{3.298149in}}%
\pgfpathlineto{\pgfqpoint{1.080566in}{3.330971in}}%
\pgfpathlineto{\pgfqpoint{1.094049in}{3.365333in}}%
\pgfpathlineto{\pgfqpoint{1.120646in}{3.408794in}}%
\pgfpathlineto{\pgfqpoint{1.156563in}{3.443879in}}%
\pgfpathlineto{\pgfqpoint{1.160727in}{3.446659in}}%
\pgfpathlineto{\pgfqpoint{1.206720in}{3.471827in}}%
\pgfpathlineto{\pgfqpoint{1.240889in}{3.482755in}}%
\pgfpathlineto{\pgfqpoint{1.247674in}{3.483654in}}%
\pgfpathlineto{\pgfqpoint{1.280970in}{3.490070in}}%
\pgfpathlineto{\pgfqpoint{1.304753in}{3.492514in}}%
\pgfpathlineto{\pgfqpoint{1.344013in}{3.493278in}}%
\pgfpathlineto{\pgfqpoint{1.361131in}{3.492423in}}%
\pgfpathlineto{\pgfqpoint{1.375373in}{3.490598in}}%
\pgfpathlineto{\pgfqpoint{1.401212in}{3.488972in}}%
\pgfpathlineto{\pgfqpoint{1.411557in}{3.486969in}}%
\pgfpathlineto{\pgfqpoint{1.441293in}{3.483151in}}%
\pgfpathlineto{\pgfqpoint{1.484309in}{3.474599in}}%
\pgfpathlineto{\pgfqpoint{1.521455in}{3.465201in}}%
\pgfpathlineto{\pgfqpoint{1.541615in}{3.458779in}}%
\pgfpathlineto{\pgfqpoint{1.561535in}{3.453731in}}%
\pgfpathlineto{\pgfqpoint{1.572133in}{3.449872in}}%
\pgfpathlineto{\pgfqpoint{1.604482in}{3.440000in}}%
\pgfpathlineto{\pgfqpoint{1.721859in}{3.394665in}}%
\pgfpathlineto{\pgfqpoint{1.761939in}{3.377040in}}%
\pgfpathlineto{\pgfqpoint{1.842101in}{3.339055in}}%
\pgfpathlineto{\pgfqpoint{1.863981in}{3.328000in}}%
\pgfpathlineto{\pgfqpoint{1.926898in}{3.294984in}}%
\pgfpathlineto{\pgfqpoint{1.962343in}{3.275752in}}%
\pgfpathlineto{\pgfqpoint{1.977001in}{3.266986in}}%
\pgfpathlineto{\pgfqpoint{2.002446in}{3.253333in}}%
\pgfpathlineto{\pgfqpoint{2.042505in}{3.229851in}}%
\pgfpathlineto{\pgfqpoint{2.126970in}{3.178667in}}%
\pgfpathlineto{\pgfqpoint{2.185537in}{3.141333in}}%
\pgfpathlineto{\pgfqpoint{2.243314in}{3.103622in}}%
\pgfpathlineto{\pgfqpoint{2.351381in}{3.029333in}}%
\pgfpathlineto{\pgfqpoint{2.523475in}{2.903394in}}%
\pgfpathlineto{\pgfqpoint{2.563556in}{2.872783in}}%
\pgfpathlineto{\pgfqpoint{2.649566in}{2.805333in}}%
\pgfpathlineto{\pgfqpoint{2.710575in}{2.755608in}}%
\pgfpathlineto{\pgfqpoint{2.741537in}{2.730667in}}%
\pgfpathlineto{\pgfqpoint{2.844121in}{2.644643in}}%
\pgfpathlineto{\pgfqpoint{2.959777in}{2.544000in}}%
\pgfpathlineto{\pgfqpoint{3.004444in}{2.504111in}}%
\pgfpathlineto{\pgfqpoint{3.084606in}{2.431023in}}%
\pgfpathlineto{\pgfqpoint{3.124687in}{2.393760in}}%
\pgfpathlineto{\pgfqpoint{3.204848in}{2.317779in}}%
\pgfpathlineto{\pgfqpoint{3.244929in}{2.279053in}}%
\pgfpathlineto{\pgfqpoint{3.354450in}{2.170667in}}%
\pgfpathlineto{\pgfqpoint{3.416862in}{2.106814in}}%
\pgfpathlineto{\pgfqpoint{3.454921in}{2.067597in}}%
\pgfpathlineto{\pgfqpoint{3.492750in}{2.028166in}}%
\pgfpathlineto{\pgfqpoint{3.534756in}{1.984000in}}%
\pgfpathlineto{\pgfqpoint{3.585968in}{1.928328in}}%
\pgfpathlineto{\pgfqpoint{3.605657in}{1.907667in}}%
\pgfpathlineto{\pgfqpoint{3.659593in}{1.847572in}}%
\pgfpathlineto{\pgfqpoint{3.696068in}{1.806880in}}%
\pgfpathlineto{\pgfqpoint{3.732320in}{1.765981in}}%
\pgfpathlineto{\pgfqpoint{3.770344in}{1.722667in}}%
\pgfpathlineto{\pgfqpoint{3.846141in}{1.633674in}}%
\pgfpathlineto{\pgfqpoint{3.929505in}{1.533017in}}%
\pgfpathlineto{\pgfqpoint{4.016348in}{1.424000in}}%
\pgfpathlineto{\pgfqpoint{4.086626in}{1.332704in}}%
\pgfpathlineto{\pgfqpoint{4.166788in}{1.224631in}}%
\pgfpathlineto{\pgfqpoint{4.246949in}{1.111564in}}%
\pgfpathlineto{\pgfqpoint{4.385280in}{0.901333in}}%
\pgfpathlineto{\pgfqpoint{4.393029in}{0.888066in}}%
\pgfpathlineto{\pgfqpoint{4.408497in}{0.864000in}}%
\pgfpathlineto{\pgfqpoint{4.430815in}{0.826667in}}%
\pgfpathlineto{\pgfqpoint{4.474294in}{0.752000in}}%
\pgfpathlineto{\pgfqpoint{4.487434in}{0.728697in}}%
\pgfpathlineto{\pgfqpoint{4.527515in}{0.654848in}}%
\pgfpathlineto{\pgfqpoint{4.554382in}{0.602667in}}%
\pgfpathlineto{\pgfqpoint{4.590586in}{0.528000in}}%
\pgfpathlineto{\pgfqpoint{4.602999in}{0.528000in}}%
\pgfpathlineto{\pgfqpoint{4.579124in}{0.576071in}}%
\pgfpathlineto{\pgfqpoint{4.566440in}{0.602667in}}%
\pgfpathlineto{\pgfqpoint{4.546933in}{0.640000in}}%
\pgfpathlineto{\pgfqpoint{4.506553in}{0.714667in}}%
\pgfpathlineto{\pgfqpoint{4.463923in}{0.789333in}}%
\pgfpathlineto{\pgfqpoint{4.419214in}{0.864000in}}%
\pgfpathlineto{\pgfqpoint{4.367192in}{0.947049in}}%
\pgfpathlineto{\pgfqpoint{4.298941in}{1.050667in}}%
\pgfpathlineto{\pgfqpoint{4.246949in}{1.126631in}}%
\pgfpathlineto{\pgfqpoint{4.166788in}{1.238773in}}%
\pgfpathlineto{\pgfqpoint{4.084175in}{1.349333in}}%
\pgfpathlineto{\pgfqpoint{4.046545in}{1.398108in}}%
\pgfpathlineto{\pgfqpoint{3.966384in}{1.499717in}}%
\pgfpathlineto{\pgfqpoint{3.875349in}{1.610667in}}%
\pgfpathlineto{\pgfqpoint{3.806061in}{1.692695in}}%
\pgfpathlineto{\pgfqpoint{3.714896in}{1.797333in}}%
\pgfpathlineto{\pgfqpoint{3.665006in}{1.852615in}}%
\pgfpathlineto{\pgfqpoint{3.645737in}{1.874685in}}%
\pgfpathlineto{\pgfqpoint{3.591422in}{1.933408in}}%
\pgfpathlineto{\pgfqpoint{3.554288in}{1.973486in}}%
\pgfpathlineto{\pgfqpoint{3.525495in}{2.004597in}}%
\pgfpathlineto{\pgfqpoint{3.438132in}{2.096000in}}%
\pgfpathlineto{\pgfqpoint{3.401776in}{2.133333in}}%
\pgfpathlineto{\pgfqpoint{3.325091in}{2.210591in}}%
\pgfpathlineto{\pgfqpoint{3.285010in}{2.250194in}}%
\pgfpathlineto{\pgfqpoint{3.164768in}{2.366139in}}%
\pgfpathlineto{\pgfqpoint{3.044525in}{2.477863in}}%
\pgfpathlineto{\pgfqpoint{2.924283in}{2.585459in}}%
\pgfpathlineto{\pgfqpoint{2.842828in}{2.656000in}}%
\pgfpathlineto{\pgfqpoint{2.780076in}{2.708345in}}%
\pgfpathlineto{\pgfqpoint{2.753893in}{2.730667in}}%
\pgfpathlineto{\pgfqpoint{2.708407in}{2.768000in}}%
\pgfpathlineto{\pgfqpoint{2.603636in}{2.851861in}}%
\pgfpathlineto{\pgfqpoint{2.518844in}{2.917333in}}%
\pgfpathlineto{\pgfqpoint{2.403232in}{3.002927in}}%
\pgfpathlineto{\pgfqpoint{2.313202in}{3.066667in}}%
\pgfpathlineto{\pgfqpoint{2.242909in}{3.114573in}}%
\pgfpathlineto{\pgfqpoint{2.225782in}{3.125380in}}%
\pgfpathlineto{\pgfqpoint{2.202725in}{3.141333in}}%
\pgfpathlineto{\pgfqpoint{2.084726in}{3.216000in}}%
\pgfpathlineto{\pgfqpoint{2.002424in}{3.264760in}}%
\pgfpathlineto{\pgfqpoint{1.984776in}{3.274228in}}%
\pgfpathlineto{\pgfqpoint{1.956830in}{3.290667in}}%
\pgfpathlineto{\pgfqpoint{1.789060in}{3.377405in}}%
\pgfpathlineto{\pgfqpoint{1.721859in}{3.407868in}}%
\pgfpathlineto{\pgfqpoint{1.681778in}{3.424630in}}%
\pgfpathlineto{\pgfqpoint{1.640509in}{3.441107in}}%
\pgfpathlineto{\pgfqpoint{1.561535in}{3.468504in}}%
\pgfpathlineto{\pgfqpoint{1.554083in}{3.470392in}}%
\pgfpathlineto{\pgfqpoint{1.521455in}{3.480595in}}%
\pgfpathlineto{\pgfqpoint{1.441293in}{3.499576in}}%
\pgfpathlineto{\pgfqpoint{1.427054in}{3.501404in}}%
\pgfpathlineto{\pgfqpoint{1.392906in}{3.506930in}}%
\pgfpathlineto{\pgfqpoint{1.361131in}{3.511001in}}%
\pgfpathlineto{\pgfqpoint{1.321051in}{3.512887in}}%
\pgfpathlineto{\pgfqpoint{1.276618in}{3.510613in}}%
\pgfpathlineto{\pgfqpoint{1.240889in}{3.505882in}}%
\pgfpathlineto{\pgfqpoint{1.226849in}{3.501589in}}%
\pgfpathlineto{\pgfqpoint{1.200808in}{3.495075in}}%
\pgfpathlineto{\pgfqpoint{1.187029in}{3.490168in}}%
\pgfpathlineto{\pgfqpoint{1.160449in}{3.477333in}}%
\pgfpathlineto{\pgfqpoint{1.137265in}{3.461854in}}%
\pgfpathlineto{\pgfqpoint{1.113202in}{3.440000in}}%
\pgfpathlineto{\pgfqpoint{1.098672in}{3.423135in}}%
\pgfpathlineto{\pgfqpoint{1.080566in}{3.394486in}}%
\pgfpathlineto{\pgfqpoint{1.067572in}{3.365333in}}%
\pgfpathlineto{\pgfqpoint{1.063365in}{3.349311in}}%
\pgfpathlineto{\pgfqpoint{1.053567in}{3.315814in}}%
\pgfpathlineto{\pgfqpoint{1.049941in}{3.290667in}}%
\pgfpathlineto{\pgfqpoint{1.046808in}{3.247444in}}%
\pgfpathlineto{\pgfqpoint{1.047450in}{3.209512in}}%
\pgfpathlineto{\pgfqpoint{1.049987in}{3.178667in}}%
\pgfpathlineto{\pgfqpoint{1.061221in}{3.104000in}}%
\pgfpathlineto{\pgfqpoint{1.064920in}{3.089427in}}%
\pgfpathlineto{\pgfqpoint{1.071669in}{3.058380in}}%
\pgfpathlineto{\pgfqpoint{1.080566in}{3.021981in}}%
\pgfpathlineto{\pgfqpoint{1.106365in}{2.941365in}}%
\pgfpathlineto{\pgfqpoint{1.120646in}{2.899508in}}%
\pgfpathlineto{\pgfqpoint{1.142859in}{2.842667in}}%
\pgfpathlineto{\pgfqpoint{1.148188in}{2.830987in}}%
\pgfpathlineto{\pgfqpoint{1.160727in}{2.799818in}}%
\pgfpathlineto{\pgfqpoint{1.228540in}{2.656000in}}%
\pgfpathlineto{\pgfqpoint{1.240889in}{2.631816in}}%
\pgfpathlineto{\pgfqpoint{1.267578in}{2.581333in}}%
\pgfpathlineto{\pgfqpoint{1.308953in}{2.506667in}}%
\pgfpathlineto{\pgfqpoint{1.361131in}{2.417637in}}%
\pgfpathlineto{\pgfqpoint{1.401212in}{2.352253in}}%
\pgfpathlineto{\pgfqpoint{1.421711in}{2.320000in}}%
\pgfpathlineto{\pgfqpoint{1.481374in}{2.228423in}}%
\pgfpathlineto{\pgfqpoint{1.546214in}{2.133333in}}%
\pgfpathlineto{\pgfqpoint{1.572385in}{2.096000in}}%
\pgfpathlineto{\pgfqpoint{1.626067in}{2.021333in}}%
\pgfpathlineto{\pgfqpoint{1.681778in}{1.945823in}}%
\pgfpathlineto{\pgfqpoint{1.767069in}{1.834667in}}%
\pgfpathlineto{\pgfqpoint{1.802020in}{1.790338in}}%
\pgfpathlineto{\pgfqpoint{1.902471in}{1.666435in}}%
\pgfpathlineto{\pgfqpoint{1.980879in}{1.573333in}}%
\pgfpathlineto{\pgfqpoint{2.078016in}{1.461333in}}%
\pgfpathlineto{\pgfqpoint{2.162747in}{1.366939in}}%
\pgfpathlineto{\pgfqpoint{2.247802in}{1.274667in}}%
\pgfpathlineto{\pgfqpoint{2.323071in}{1.195192in}}%
\pgfpathlineto{\pgfqpoint{2.377924in}{1.139093in}}%
\pgfpathlineto{\pgfqpoint{2.403232in}{1.112721in}}%
\pgfpathlineto{\pgfqpoint{2.443313in}{1.072234in}}%
\pgfpathlineto{\pgfqpoint{2.540646in}{0.976000in}}%
\pgfpathlineto{\pgfqpoint{2.591445in}{0.927311in}}%
\pgfpathlineto{\pgfqpoint{2.618209in}{0.901333in}}%
\pgfpathlineto{\pgfqpoint{2.670946in}{0.852029in}}%
\pgfpathlineto{\pgfqpoint{2.697701in}{0.826667in}}%
\pgfpathlineto{\pgfqpoint{2.804040in}{0.729642in}}%
\pgfpathlineto{\pgfqpoint{2.853450in}{0.686022in}}%
\pgfpathlineto{\pgfqpoint{2.884202in}{0.658674in}}%
\pgfpathlineto{\pgfqpoint{2.992943in}{0.565333in}}%
\pgfpathlineto{\pgfqpoint{3.037519in}{0.528000in}}%
\pgfpathlineto{\pgfqpoint{3.044525in}{0.528000in}}%
\pgfpathlineto{\pgfqpoint{3.044525in}{0.528000in}}%
\pgfusepath{fill}%
\end{pgfscope}%
\begin{pgfscope}%
\pgfpathrectangle{\pgfqpoint{0.800000in}{0.528000in}}{\pgfqpoint{3.968000in}{3.696000in}}%
\pgfusepath{clip}%
\pgfsetbuttcap%
\pgfsetroundjoin%
\definecolor{currentfill}{rgb}{0.280894,0.078907,0.402329}%
\pgfsetfillcolor{currentfill}%
\pgfsetlinewidth{0.000000pt}%
\definecolor{currentstroke}{rgb}{0.000000,0.000000,0.000000}%
\pgfsetstrokecolor{currentstroke}%
\pgfsetdash{}{0pt}%
\pgfpathmoveto{\pgfqpoint{3.037519in}{0.528000in}}%
\pgfpathlineto{\pgfqpoint{2.924283in}{0.623874in}}%
\pgfpathlineto{\pgfqpoint{2.874078in}{0.667904in}}%
\pgfpathlineto{\pgfqpoint{2.844121in}{0.693929in}}%
\pgfpathlineto{\pgfqpoint{2.738208in}{0.789333in}}%
\pgfpathlineto{\pgfqpoint{2.643717in}{0.877138in}}%
\pgfpathlineto{\pgfqpoint{2.591445in}{0.927311in}}%
\pgfpathlineto{\pgfqpoint{2.552059in}{0.965291in}}%
\pgfpathlineto{\pgfqpoint{2.512912in}{1.003495in}}%
\pgfpathlineto{\pgfqpoint{2.474002in}{1.041919in}}%
\pgfpathlineto{\pgfqpoint{2.443313in}{1.072234in}}%
\pgfpathlineto{\pgfqpoint{2.354476in}{1.162667in}}%
\pgfpathlineto{\pgfqpoint{2.318471in}{1.200000in}}%
\pgfpathlineto{\pgfqpoint{2.213136in}{1.312000in}}%
\pgfpathlineto{\pgfqpoint{2.178835in}{1.349333in}}%
\pgfpathlineto{\pgfqpoint{2.082586in}{1.456165in}}%
\pgfpathlineto{\pgfqpoint{2.002424in}{1.548117in}}%
\pgfpathlineto{\pgfqpoint{1.902471in}{1.666435in}}%
\pgfpathlineto{\pgfqpoint{1.826334in}{1.760000in}}%
\pgfpathlineto{\pgfqpoint{1.761939in}{1.841244in}}%
\pgfpathlineto{\pgfqpoint{1.679660in}{1.948639in}}%
\pgfpathlineto{\pgfqpoint{1.593348in}{2.066368in}}%
\pgfpathlineto{\pgfqpoint{1.520253in}{2.170667in}}%
\pgfpathlineto{\pgfqpoint{1.481374in}{2.228423in}}%
\pgfpathlineto{\pgfqpoint{1.421711in}{2.320000in}}%
\pgfpathlineto{\pgfqpoint{1.401212in}{2.352253in}}%
\pgfpathlineto{\pgfqpoint{1.361131in}{2.417637in}}%
\pgfpathlineto{\pgfqpoint{1.321051in}{2.485558in}}%
\pgfpathlineto{\pgfqpoint{1.280970in}{2.556658in}}%
\pgfpathlineto{\pgfqpoint{1.240889in}{2.631816in}}%
\pgfpathlineto{\pgfqpoint{1.174914in}{2.768000in}}%
\pgfpathlineto{\pgfqpoint{1.171051in}{2.777616in}}%
\pgfpathlineto{\pgfqpoint{1.158296in}{2.805333in}}%
\pgfpathlineto{\pgfqpoint{1.114050in}{2.917333in}}%
\pgfpathlineto{\pgfqpoint{1.078450in}{3.029333in}}%
\pgfpathlineto{\pgfqpoint{1.061221in}{3.104000in}}%
\pgfpathlineto{\pgfqpoint{1.058864in}{3.121119in}}%
\pgfpathlineto{\pgfqpoint{1.054731in}{3.141333in}}%
\pgfpathlineto{\pgfqpoint{1.053721in}{3.153662in}}%
\pgfpathlineto{\pgfqpoint{1.049987in}{3.178667in}}%
\pgfpathlineto{\pgfqpoint{1.047315in}{3.216000in}}%
\pgfpathlineto{\pgfqpoint{1.047124in}{3.253333in}}%
\pgfpathlineto{\pgfqpoint{1.048036in}{3.260367in}}%
\pgfpathlineto{\pgfqpoint{1.049941in}{3.290667in}}%
\pgfpathlineto{\pgfqpoint{1.053567in}{3.315814in}}%
\pgfpathlineto{\pgfqpoint{1.056453in}{3.328000in}}%
\pgfpathlineto{\pgfqpoint{1.070817in}{3.374414in}}%
\pgfpathlineto{\pgfqpoint{1.085055in}{3.402667in}}%
\pgfpathlineto{\pgfqpoint{1.098672in}{3.423135in}}%
\pgfpathlineto{\pgfqpoint{1.120646in}{3.447510in}}%
\pgfpathlineto{\pgfqpoint{1.137265in}{3.461854in}}%
\pgfpathlineto{\pgfqpoint{1.160727in}{3.477500in}}%
\pgfpathlineto{\pgfqpoint{1.187029in}{3.490168in}}%
\pgfpathlineto{\pgfqpoint{1.200808in}{3.495075in}}%
\pgfpathlineto{\pgfqpoint{1.240889in}{3.505882in}}%
\pgfpathlineto{\pgfqpoint{1.284078in}{3.511771in}}%
\pgfpathlineto{\pgfqpoint{1.322937in}{3.512910in}}%
\pgfpathlineto{\pgfqpoint{1.361131in}{3.511001in}}%
\pgfpathlineto{\pgfqpoint{1.401212in}{3.506407in}}%
\pgfpathlineto{\pgfqpoint{1.481374in}{3.490874in}}%
\pgfpathlineto{\pgfqpoint{1.492343in}{3.487550in}}%
\pgfpathlineto{\pgfqpoint{1.532295in}{3.477333in}}%
\pgfpathlineto{\pgfqpoint{1.601616in}{3.455070in}}%
\pgfpathlineto{\pgfqpoint{1.643324in}{3.440000in}}%
\pgfpathlineto{\pgfqpoint{1.721859in}{3.407868in}}%
\pgfpathlineto{\pgfqpoint{1.806007in}{3.369046in}}%
\pgfpathlineto{\pgfqpoint{1.842101in}{3.351392in}}%
\pgfpathlineto{\pgfqpoint{1.857936in}{3.342750in}}%
\pgfpathlineto{\pgfqpoint{1.887819in}{3.328000in}}%
\pgfpathlineto{\pgfqpoint{1.962343in}{3.287645in}}%
\pgfpathlineto{\pgfqpoint{2.002424in}{3.264760in}}%
\pgfpathlineto{\pgfqpoint{2.084726in}{3.216000in}}%
\pgfpathlineto{\pgfqpoint{2.162747in}{3.167065in}}%
\pgfpathlineto{\pgfqpoint{2.242909in}{3.114573in}}%
\pgfpathlineto{\pgfqpoint{2.282990in}{3.087439in}}%
\pgfpathlineto{\pgfqpoint{2.366513in}{3.029333in}}%
\pgfpathlineto{\pgfqpoint{2.468993in}{2.954667in}}%
\pgfpathlineto{\pgfqpoint{2.683798in}{2.787995in}}%
\pgfpathlineto{\pgfqpoint{2.798692in}{2.693333in}}%
\pgfpathlineto{\pgfqpoint{2.886221in}{2.618667in}}%
\pgfpathlineto{\pgfqpoint{2.971135in}{2.544000in}}%
\pgfpathlineto{\pgfqpoint{3.012755in}{2.506667in}}%
\pgfpathlineto{\pgfqpoint{3.124687in}{2.403845in}}%
\pgfpathlineto{\pgfqpoint{3.213125in}{2.320000in}}%
\pgfpathlineto{\pgfqpoint{3.251757in}{2.282667in}}%
\pgfpathlineto{\pgfqpoint{3.327688in}{2.208000in}}%
\pgfpathlineto{\pgfqpoint{3.365172in}{2.170498in}}%
\pgfpathlineto{\pgfqpoint{3.485414in}{2.046848in}}%
\pgfpathlineto{\pgfqpoint{3.535629in}{1.993440in}}%
\pgfpathlineto{\pgfqpoint{3.579629in}{1.946667in}}%
\pgfpathlineto{\pgfqpoint{3.628328in}{1.893117in}}%
\pgfpathlineto{\pgfqpoint{3.648169in}{1.872000in}}%
\pgfpathlineto{\pgfqpoint{3.701461in}{1.811904in}}%
\pgfpathlineto{\pgfqpoint{3.737694in}{1.770986in}}%
\pgfpathlineto{\pgfqpoint{3.773707in}{1.729864in}}%
\pgfpathlineto{\pgfqpoint{3.812358in}{1.685333in}}%
\pgfpathlineto{\pgfqpoint{3.886222in}{1.597612in}}%
\pgfpathlineto{\pgfqpoint{3.936886in}{1.536000in}}%
\pgfpathlineto{\pgfqpoint{4.006465in}{1.449273in}}%
\pgfpathlineto{\pgfqpoint{4.068126in}{1.369435in}}%
\pgfpathlineto{\pgfqpoint{4.086626in}{1.346108in}}%
\pgfpathlineto{\pgfqpoint{4.170108in}{1.234241in}}%
\pgfpathlineto{\pgfqpoint{4.247860in}{1.125333in}}%
\pgfpathlineto{\pgfqpoint{4.287030in}{1.068248in}}%
\pgfpathlineto{\pgfqpoint{4.348397in}{0.976000in}}%
\pgfpathlineto{\pgfqpoint{4.407273in}{0.883383in}}%
\pgfpathlineto{\pgfqpoint{4.463923in}{0.789333in}}%
\pgfpathlineto{\pgfqpoint{4.506553in}{0.714667in}}%
\pgfpathlineto{\pgfqpoint{4.546933in}{0.640000in}}%
\pgfpathlineto{\pgfqpoint{4.553343in}{0.626724in}}%
\pgfpathlineto{\pgfqpoint{4.567596in}{0.600350in}}%
\pgfpathlineto{\pgfqpoint{4.602999in}{0.528000in}}%
\pgfpathlineto{\pgfqpoint{4.615028in}{0.528000in}}%
\pgfpathlineto{\pgfqpoint{4.577973in}{0.602667in}}%
\pgfpathlineto{\pgfqpoint{4.538455in}{0.677333in}}%
\pgfpathlineto{\pgfqpoint{4.527515in}{0.697244in}}%
\pgfpathlineto{\pgfqpoint{4.487434in}{0.767940in}}%
\pgfpathlineto{\pgfqpoint{4.447354in}{0.835597in}}%
\pgfpathlineto{\pgfqpoint{4.429931in}{0.864000in}}%
\pgfpathlineto{\pgfqpoint{4.383033in}{0.938667in}}%
\pgfpathlineto{\pgfqpoint{4.327111in}{1.024182in}}%
\pgfpathlineto{\pgfqpoint{4.257911in}{1.125333in}}%
\pgfpathlineto{\pgfqpoint{4.201218in}{1.205263in}}%
\pgfpathlineto{\pgfqpoint{4.112279in}{1.325439in}}%
\pgfpathlineto{\pgfqpoint{4.036241in}{1.424000in}}%
\pgfpathlineto{\pgfqpoint{3.989267in}{1.482648in}}%
\pgfpathlineto{\pgfqpoint{3.966384in}{1.511778in}}%
\pgfpathlineto{\pgfqpoint{3.882221in}{1.614394in}}%
\pgfpathlineto{\pgfqpoint{3.790032in}{1.722667in}}%
\pgfpathlineto{\pgfqpoint{3.743067in}{1.775991in}}%
\pgfpathlineto{\pgfqpoint{3.724921in}{1.797333in}}%
\pgfpathlineto{\pgfqpoint{3.670420in}{1.857657in}}%
\pgfpathlineto{\pgfqpoint{3.633761in}{1.898178in}}%
\pgfpathlineto{\pgfqpoint{3.596876in}{1.938488in}}%
\pgfpathlineto{\pgfqpoint{3.559762in}{1.978585in}}%
\pgfpathlineto{\pgfqpoint{3.519804in}{2.021333in}}%
\pgfpathlineto{\pgfqpoint{3.465554in}{2.077502in}}%
\pgfpathlineto{\pgfqpoint{3.445333in}{2.099123in}}%
\pgfpathlineto{\pgfqpoint{3.405253in}{2.140161in}}%
\pgfpathlineto{\pgfqpoint{3.285010in}{2.260346in}}%
\pgfpathlineto{\pgfqpoint{3.184630in}{2.357333in}}%
\pgfpathlineto{\pgfqpoint{3.145165in}{2.394667in}}%
\pgfpathlineto{\pgfqpoint{3.044525in}{2.487862in}}%
\pgfpathlineto{\pgfqpoint{2.940221in}{2.581333in}}%
\pgfpathlineto{\pgfqpoint{2.884202in}{2.630324in}}%
\pgfpathlineto{\pgfqpoint{2.804040in}{2.698872in}}%
\pgfpathlineto{\pgfqpoint{2.743394in}{2.748844in}}%
\pgfpathlineto{\pgfqpoint{2.720905in}{2.768000in}}%
\pgfpathlineto{\pgfqpoint{2.628070in}{2.842667in}}%
\pgfpathlineto{\pgfqpoint{2.549091in}{2.903861in}}%
\pgfpathlineto{\pgfqpoint{2.523475in}{2.924028in}}%
\pgfpathlineto{\pgfqpoint{2.432457in}{2.992000in}}%
\pgfpathlineto{\pgfqpoint{2.370730in}{3.036392in}}%
\pgfpathlineto{\pgfqpoint{2.363152in}{3.042086in}}%
\pgfpathlineto{\pgfqpoint{2.274348in}{3.104000in}}%
\pgfpathlineto{\pgfqpoint{2.209318in}{3.147378in}}%
\pgfpathlineto{\pgfqpoint{2.202828in}{3.151917in}}%
\pgfpathlineto{\pgfqpoint{2.122667in}{3.203450in}}%
\pgfpathlineto{\pgfqpoint{2.041500in}{3.253333in}}%
\pgfpathlineto{\pgfqpoint{1.962343in}{3.299156in}}%
\pgfpathlineto{\pgfqpoint{1.922263in}{3.321413in}}%
\pgfpathlineto{\pgfqpoint{1.838854in}{3.365333in}}%
\pgfpathlineto{\pgfqpoint{1.707279in}{3.426420in}}%
\pgfpathlineto{\pgfqpoint{1.676907in}{3.440000in}}%
\pgfpathlineto{\pgfqpoint{1.561535in}{3.482962in}}%
\pgfpathlineto{\pgfqpoint{1.481374in}{3.506399in}}%
\pgfpathlineto{\pgfqpoint{1.439522in}{3.516316in}}%
\pgfpathlineto{\pgfqpoint{1.389957in}{3.525150in}}%
\pgfpathlineto{\pgfqpoint{1.361131in}{3.528601in}}%
\pgfpathlineto{\pgfqpoint{1.280970in}{3.531475in}}%
\pgfpathlineto{\pgfqpoint{1.263813in}{3.530647in}}%
\pgfpathlineto{\pgfqpoint{1.240889in}{3.527856in}}%
\pgfpathlineto{\pgfqpoint{1.200808in}{3.519771in}}%
\pgfpathlineto{\pgfqpoint{1.167471in}{3.508385in}}%
\pgfpathlineto{\pgfqpoint{1.141400in}{3.495336in}}%
\pgfpathlineto{\pgfqpoint{1.115274in}{3.477333in}}%
\pgfpathlineto{\pgfqpoint{1.079087in}{3.440000in}}%
\pgfpathlineto{\pgfqpoint{1.077585in}{3.437224in}}%
\pgfpathlineto{\pgfqpoint{1.052018in}{3.391924in}}%
\pgfpathlineto{\pgfqpoint{1.040485in}{3.357596in}}%
\pgfpathlineto{\pgfqpoint{1.033941in}{3.328000in}}%
\pgfpathlineto{\pgfqpoint{1.029430in}{3.290667in}}%
\pgfpathlineto{\pgfqpoint{1.029551in}{3.280483in}}%
\pgfpathlineto{\pgfqpoint{1.028081in}{3.253333in}}%
\pgfpathlineto{\pgfqpoint{1.029351in}{3.216000in}}%
\pgfpathlineto{\pgfqpoint{1.030656in}{3.206845in}}%
\pgfpathlineto{\pgfqpoint{1.034003in}{3.172629in}}%
\pgfpathlineto{\pgfqpoint{1.040485in}{3.128991in}}%
\pgfpathlineto{\pgfqpoint{1.054093in}{3.066667in}}%
\pgfpathlineto{\pgfqpoint{1.059849in}{3.047370in}}%
\pgfpathlineto{\pgfqpoint{1.064060in}{3.029333in}}%
\pgfpathlineto{\pgfqpoint{1.080566in}{2.975163in}}%
\pgfpathlineto{\pgfqpoint{1.087348in}{2.954667in}}%
\pgfpathlineto{\pgfqpoint{1.139134in}{2.822554in}}%
\pgfpathlineto{\pgfqpoint{1.145920in}{2.805333in}}%
\pgfpathlineto{\pgfqpoint{1.180014in}{2.730667in}}%
\pgfpathlineto{\pgfqpoint{1.186792in}{2.717611in}}%
\pgfpathlineto{\pgfqpoint{1.200808in}{2.687594in}}%
\pgfpathlineto{\pgfqpoint{1.240889in}{2.609695in}}%
\pgfpathlineto{\pgfqpoint{1.341690in}{2.432000in}}%
\pgfpathlineto{\pgfqpoint{1.387488in}{2.357333in}}%
\pgfpathlineto{\pgfqpoint{1.441293in}{2.273255in}}%
\pgfpathlineto{\pgfqpoint{1.510132in}{2.170667in}}%
\pgfpathlineto{\pgfqpoint{1.563108in}{2.094535in}}%
\pgfpathlineto{\pgfqpoint{1.646903in}{1.979151in}}%
\pgfpathlineto{\pgfqpoint{1.744343in}{1.851057in}}%
\pgfpathlineto{\pgfqpoint{1.816463in}{1.760000in}}%
\pgfpathlineto{\pgfqpoint{1.862265in}{1.704115in}}%
\pgfpathlineto{\pgfqpoint{1.882182in}{1.679126in}}%
\pgfpathlineto{\pgfqpoint{1.970969in}{1.573333in}}%
\pgfpathlineto{\pgfqpoint{2.042505in}{1.490527in}}%
\pgfpathlineto{\pgfqpoint{2.134938in}{1.386667in}}%
\pgfpathlineto{\pgfqpoint{2.168793in}{1.349333in}}%
\pgfpathlineto{\pgfqpoint{2.242909in}{1.269191in}}%
\pgfpathlineto{\pgfqpoint{2.363152in}{1.143292in}}%
\pgfpathlineto{\pgfqpoint{2.410743in}{1.094996in}}%
\pgfpathlineto{\pgfqpoint{2.443313in}{1.061874in}}%
\pgfpathlineto{\pgfqpoint{2.530074in}{0.976000in}}%
\pgfpathlineto{\pgfqpoint{2.586120in}{0.922351in}}%
\pgfpathlineto{\pgfqpoint{2.607427in}{0.901333in}}%
\pgfpathlineto{\pgfqpoint{2.665582in}{0.847033in}}%
\pgfpathlineto{\pgfqpoint{2.686701in}{0.826667in}}%
\pgfpathlineto{\pgfqpoint{2.804040in}{0.719518in}}%
\pgfpathlineto{\pgfqpoint{2.868613in}{0.662813in}}%
\pgfpathlineto{\pgfqpoint{2.894097in}{0.640000in}}%
\pgfpathlineto{\pgfqpoint{2.951636in}{0.590812in}}%
\pgfpathlineto{\pgfqpoint{2.981102in}{0.565333in}}%
\pgfpathlineto{\pgfqpoint{3.025549in}{0.528000in}}%
\pgfpathlineto{\pgfqpoint{3.025549in}{0.528000in}}%
\pgfusepath{fill}%
\end{pgfscope}%
\begin{pgfscope}%
\pgfpathrectangle{\pgfqpoint{0.800000in}{0.528000in}}{\pgfqpoint{3.968000in}{3.696000in}}%
\pgfusepath{clip}%
\pgfsetbuttcap%
\pgfsetroundjoin%
\definecolor{currentfill}{rgb}{0.280894,0.078907,0.402329}%
\pgfsetfillcolor{currentfill}%
\pgfsetlinewidth{0.000000pt}%
\definecolor{currentstroke}{rgb}{0.000000,0.000000,0.000000}%
\pgfsetstrokecolor{currentstroke}%
\pgfsetdash{}{0pt}%
\pgfpathmoveto{\pgfqpoint{3.025549in}{0.528000in}}%
\pgfpathlineto{\pgfqpoint{2.924283in}{0.613827in}}%
\pgfpathlineto{\pgfqpoint{2.868613in}{0.662813in}}%
\pgfpathlineto{\pgfqpoint{2.844121in}{0.683831in}}%
\pgfpathlineto{\pgfqpoint{2.763960in}{0.755666in}}%
\pgfpathlineto{\pgfqpoint{2.723879in}{0.792278in}}%
\pgfpathlineto{\pgfqpoint{2.643717in}{0.866910in}}%
\pgfpathlineto{\pgfqpoint{2.586120in}{0.922351in}}%
\pgfpathlineto{\pgfqpoint{2.546753in}{0.960350in}}%
\pgfpathlineto{\pgfqpoint{2.507626in}{0.998571in}}%
\pgfpathlineto{\pgfqpoint{2.468735in}{1.037013in}}%
\pgfpathlineto{\pgfqpoint{2.443313in}{1.061874in}}%
\pgfpathlineto{\pgfqpoint{2.344395in}{1.162667in}}%
\pgfpathlineto{\pgfqpoint{2.308482in}{1.200000in}}%
\pgfpathlineto{\pgfqpoint{2.202828in}{1.312187in}}%
\pgfpathlineto{\pgfqpoint{2.101425in}{1.424000in}}%
\pgfpathlineto{\pgfqpoint{2.002424in}{1.536519in}}%
\pgfpathlineto{\pgfqpoint{1.908073in}{1.648000in}}%
\pgfpathlineto{\pgfqpoint{1.842101in}{1.728108in}}%
\pgfpathlineto{\pgfqpoint{1.793364in}{1.789270in}}%
\pgfpathlineto{\pgfqpoint{1.757165in}{1.834667in}}%
\pgfpathlineto{\pgfqpoint{1.681778in}{1.932778in}}%
\pgfpathlineto{\pgfqpoint{1.535998in}{2.133333in}}%
\pgfpathlineto{\pgfqpoint{1.481374in}{2.212815in}}%
\pgfpathlineto{\pgfqpoint{1.407550in}{2.325904in}}%
\pgfpathlineto{\pgfqpoint{1.387488in}{2.357333in}}%
\pgfpathlineto{\pgfqpoint{1.364194in}{2.394667in}}%
\pgfpathlineto{\pgfqpoint{1.317232in}{2.472890in}}%
\pgfpathlineto{\pgfqpoint{1.256201in}{2.581333in}}%
\pgfpathlineto{\pgfqpoint{1.240889in}{2.609695in}}%
\pgfpathlineto{\pgfqpoint{1.216848in}{2.656000in}}%
\pgfpathlineto{\pgfqpoint{1.180014in}{2.730667in}}%
\pgfpathlineto{\pgfqpoint{1.160727in}{2.771741in}}%
\pgfpathlineto{\pgfqpoint{1.127610in}{2.849153in}}%
\pgfpathlineto{\pgfqpoint{1.114813in}{2.880000in}}%
\pgfpathlineto{\pgfqpoint{1.072694in}{2.999332in}}%
\pgfpathlineto{\pgfqpoint{1.054093in}{3.066667in}}%
\pgfpathlineto{\pgfqpoint{1.052066in}{3.077454in}}%
\pgfpathlineto{\pgfqpoint{1.044777in}{3.107998in}}%
\pgfpathlineto{\pgfqpoint{1.038128in}{3.141333in}}%
\pgfpathlineto{\pgfqpoint{1.032813in}{3.178667in}}%
\pgfpathlineto{\pgfqpoint{1.028081in}{3.253333in}}%
\pgfpathlineto{\pgfqpoint{1.030039in}{3.300397in}}%
\pgfpathlineto{\pgfqpoint{1.033941in}{3.328000in}}%
\pgfpathlineto{\pgfqpoint{1.042517in}{3.365333in}}%
\pgfpathlineto{\pgfqpoint{1.052018in}{3.391924in}}%
\pgfpathlineto{\pgfqpoint{1.057235in}{3.402667in}}%
\pgfpathlineto{\pgfqpoint{1.080566in}{3.441944in}}%
\pgfpathlineto{\pgfqpoint{1.120646in}{3.481585in}}%
\pgfpathlineto{\pgfqpoint{1.141400in}{3.495336in}}%
\pgfpathlineto{\pgfqpoint{1.167471in}{3.508385in}}%
\pgfpathlineto{\pgfqpoint{1.200808in}{3.519771in}}%
\pgfpathlineto{\pgfqpoint{1.207363in}{3.520773in}}%
\pgfpathlineto{\pgfqpoint{1.240889in}{3.527856in}}%
\pgfpathlineto{\pgfqpoint{1.263813in}{3.530647in}}%
\pgfpathlineto{\pgfqpoint{1.280970in}{3.531475in}}%
\pgfpathlineto{\pgfqpoint{1.298362in}{3.530867in}}%
\pgfpathlineto{\pgfqpoint{1.321051in}{3.531504in}}%
\pgfpathlineto{\pgfqpoint{1.337209in}{3.529718in}}%
\pgfpathlineto{\pgfqpoint{1.361131in}{3.528601in}}%
\pgfpathlineto{\pgfqpoint{1.401212in}{3.523275in}}%
\pgfpathlineto{\pgfqpoint{1.446595in}{3.514667in}}%
\pgfpathlineto{\pgfqpoint{1.481374in}{3.506399in}}%
\pgfpathlineto{\pgfqpoint{1.601616in}{3.469111in}}%
\pgfpathlineto{\pgfqpoint{1.641697in}{3.454026in}}%
\pgfpathlineto{\pgfqpoint{1.651907in}{3.449510in}}%
\pgfpathlineto{\pgfqpoint{1.681778in}{3.438053in}}%
\pgfpathlineto{\pgfqpoint{1.721859in}{3.420684in}}%
\pgfpathlineto{\pgfqpoint{1.762114in}{3.402667in}}%
\pgfpathlineto{\pgfqpoint{1.842101in}{3.363730in}}%
\pgfpathlineto{\pgfqpoint{1.882182in}{3.342833in}}%
\pgfpathlineto{\pgfqpoint{1.962343in}{3.299156in}}%
\pgfpathlineto{\pgfqpoint{1.992551in}{3.281471in}}%
\pgfpathlineto{\pgfqpoint{2.002424in}{3.276174in}}%
\pgfpathlineto{\pgfqpoint{2.082586in}{3.228329in}}%
\pgfpathlineto{\pgfqpoint{2.102476in}{3.216000in}}%
\pgfpathlineto{\pgfqpoint{2.164618in}{3.176924in}}%
\pgfpathlineto{\pgfqpoint{2.282990in}{3.098143in}}%
\pgfpathlineto{\pgfqpoint{2.363152in}{3.042086in}}%
\pgfpathlineto{\pgfqpoint{2.415802in}{3.003708in}}%
\pgfpathlineto{\pgfqpoint{2.443313in}{2.984069in}}%
\pgfpathlineto{\pgfqpoint{2.532200in}{2.917333in}}%
\pgfpathlineto{\pgfqpoint{2.580528in}{2.880000in}}%
\pgfpathlineto{\pgfqpoint{2.683798in}{2.798149in}}%
\pgfpathlineto{\pgfqpoint{2.766135in}{2.730667in}}%
\pgfpathlineto{\pgfqpoint{2.827818in}{2.678147in}}%
\pgfpathlineto{\pgfqpoint{2.854387in}{2.656000in}}%
\pgfpathlineto{\pgfqpoint{2.964364in}{2.559996in}}%
\pgfpathlineto{\pgfqpoint{3.064757in}{2.469333in}}%
\pgfpathlineto{\pgfqpoint{3.164768in}{2.376214in}}%
\pgfpathlineto{\pgfqpoint{3.214224in}{2.328732in}}%
\pgfpathlineto{\pgfqpoint{3.253583in}{2.290727in}}%
\pgfpathlineto{\pgfqpoint{3.300238in}{2.245333in}}%
\pgfpathlineto{\pgfqpoint{3.350794in}{2.194608in}}%
\pgfpathlineto{\pgfqpoint{3.375117in}{2.170667in}}%
\pgfpathlineto{\pgfqpoint{3.448354in}{2.096000in}}%
\pgfpathlineto{\pgfqpoint{3.503345in}{2.038035in}}%
\pgfpathlineto{\pgfqpoint{3.540907in}{1.998356in}}%
\pgfpathlineto{\pgfqpoint{3.565576in}{1.972578in}}%
\pgfpathlineto{\pgfqpoint{3.657962in}{1.872000in}}%
\pgfpathlineto{\pgfqpoint{3.706854in}{1.816927in}}%
\pgfpathlineto{\pgfqpoint{3.725899in}{1.796228in}}%
\pgfpathlineto{\pgfqpoint{3.779061in}{1.734851in}}%
\pgfpathlineto{\pgfqpoint{3.814838in}{1.693509in}}%
\pgfpathlineto{\pgfqpoint{3.853890in}{1.648000in}}%
\pgfpathlineto{\pgfqpoint{3.926303in}{1.561002in}}%
\pgfpathlineto{\pgfqpoint{4.006933in}{1.461333in}}%
\pgfpathlineto{\pgfqpoint{4.046545in}{1.410825in}}%
\pgfpathlineto{\pgfqpoint{4.126707in}{1.306256in}}%
\pgfpathlineto{\pgfqpoint{4.206869in}{1.197408in}}%
\pgfpathlineto{\pgfqpoint{4.246949in}{1.140952in}}%
\pgfpathlineto{\pgfqpoint{4.309237in}{1.050667in}}%
\pgfpathlineto{\pgfqpoint{4.367192in}{0.963305in}}%
\pgfpathlineto{\pgfqpoint{4.429931in}{0.864000in}}%
\pgfpathlineto{\pgfqpoint{4.450742in}{0.829823in}}%
\pgfpathlineto{\pgfqpoint{4.459774in}{0.815098in}}%
\pgfpathlineto{\pgfqpoint{4.517842in}{0.714667in}}%
\pgfpathlineto{\pgfqpoint{4.561344in}{0.634176in}}%
\pgfpathlineto{\pgfqpoint{4.577973in}{0.602667in}}%
\pgfpathlineto{\pgfqpoint{4.600004in}{0.558187in}}%
\pgfpathlineto{\pgfqpoint{4.615028in}{0.528000in}}%
\pgfpathlineto{\pgfqpoint{4.626825in}{0.528000in}}%
\pgfpathlineto{\pgfqpoint{4.607677in}{0.567293in}}%
\pgfpathlineto{\pgfqpoint{4.589449in}{0.602667in}}%
\pgfpathlineto{\pgfqpoint{4.549629in}{0.677333in}}%
\pgfpathlineto{\pgfqpoint{4.541795in}{0.690635in}}%
\pgfpathlineto{\pgfqpoint{4.527515in}{0.717370in}}%
\pgfpathlineto{\pgfqpoint{4.463390in}{0.826667in}}%
\pgfpathlineto{\pgfqpoint{4.442988in}{0.859934in}}%
\pgfpathlineto{\pgfqpoint{4.431618in}{0.878657in}}%
\pgfpathlineto{\pgfqpoint{4.367192in}{0.979355in}}%
\pgfpathlineto{\pgfqpoint{4.327111in}{1.039438in}}%
\pgfpathlineto{\pgfqpoint{4.267963in}{1.125333in}}%
\pgfpathlineto{\pgfqpoint{4.206869in}{1.211081in}}%
\pgfpathlineto{\pgfqpoint{4.126707in}{1.319332in}}%
\pgfpathlineto{\pgfqpoint{4.046187in}{1.424000in}}%
\pgfpathlineto{\pgfqpoint{3.994929in}{1.487922in}}%
\pgfpathlineto{\pgfqpoint{3.966384in}{1.523839in}}%
\pgfpathlineto{\pgfqpoint{3.886222in}{1.621116in}}%
\pgfpathlineto{\pgfqpoint{3.799876in}{1.722667in}}%
\pgfpathlineto{\pgfqpoint{3.748441in}{1.780996in}}%
\pgfpathlineto{\pgfqpoint{3.712248in}{1.821951in}}%
\pgfpathlineto{\pgfqpoint{3.675833in}{1.862700in}}%
\pgfpathlineto{\pgfqpoint{3.639195in}{1.903240in}}%
\pgfpathlineto{\pgfqpoint{3.599575in}{1.946667in}}%
\pgfpathlineto{\pgfqpoint{3.525495in}{2.025867in}}%
\pgfpathlineto{\pgfqpoint{3.470871in}{2.082454in}}%
\pgfpathlineto{\pgfqpoint{3.432870in}{2.121725in}}%
\pgfpathlineto{\pgfqpoint{3.405253in}{2.150391in}}%
\pgfpathlineto{\pgfqpoint{3.310536in}{2.245333in}}%
\pgfpathlineto{\pgfqpoint{3.204848in}{2.348166in}}%
\pgfpathlineto{\pgfqpoint{3.116022in}{2.432000in}}%
\pgfpathlineto{\pgfqpoint{3.059699in}{2.483467in}}%
\pgfpathlineto{\pgfqpoint{3.034814in}{2.506667in}}%
\pgfpathlineto{\pgfqpoint{2.924283in}{2.605310in}}%
\pgfpathlineto{\pgfqpoint{2.822193in}{2.693333in}}%
\pgfpathlineto{\pgfqpoint{2.723879in}{2.775471in}}%
\pgfpathlineto{\pgfqpoint{2.640862in}{2.842667in}}%
\pgfpathlineto{\pgfqpoint{2.576918in}{2.892446in}}%
\pgfpathlineto{\pgfqpoint{2.545301in}{2.917333in}}%
\pgfpathlineto{\pgfqpoint{2.443313in}{2.994371in}}%
\pgfpathlineto{\pgfqpoint{2.376876in}{3.042117in}}%
\pgfpathlineto{\pgfqpoint{2.343032in}{3.066667in}}%
\pgfpathlineto{\pgfqpoint{2.162747in}{3.188769in}}%
\pgfpathlineto{\pgfqpoint{2.082586in}{3.239331in}}%
\pgfpathlineto{\pgfqpoint{1.997060in}{3.290667in}}%
\pgfpathlineto{\pgfqpoint{1.842101in}{3.375589in}}%
\pgfpathlineto{\pgfqpoint{1.822806in}{3.384694in}}%
\pgfpathlineto{\pgfqpoint{1.787681in}{3.402667in}}%
\pgfpathlineto{\pgfqpoint{1.681778in}{3.450922in}}%
\pgfpathlineto{\pgfqpoint{1.661645in}{3.458581in}}%
\pgfpathlineto{\pgfqpoint{1.641697in}{3.467404in}}%
\pgfpathlineto{\pgfqpoint{1.591732in}{3.486540in}}%
\pgfpathlineto{\pgfqpoint{1.517494in}{3.510978in}}%
\pgfpathlineto{\pgfqpoint{1.481374in}{3.521522in}}%
\pgfpathlineto{\pgfqpoint{1.401212in}{3.539633in}}%
\pgfpathlineto{\pgfqpoint{1.354957in}{3.546249in}}%
\pgfpathlineto{\pgfqpoint{1.318832in}{3.549934in}}%
\pgfpathlineto{\pgfqpoint{1.280970in}{3.551258in}}%
\pgfpathlineto{\pgfqpoint{1.236895in}{3.548279in}}%
\pgfpathlineto{\pgfqpoint{1.200808in}{3.542764in}}%
\pgfpathlineto{\pgfqpoint{1.185575in}{3.537811in}}%
\pgfpathlineto{\pgfqpoint{1.160727in}{3.531062in}}%
\pgfpathlineto{\pgfqpoint{1.148196in}{3.526339in}}%
\pgfpathlineto{\pgfqpoint{1.120646in}{3.512209in}}%
\pgfpathlineto{\pgfqpoint{1.077138in}{3.477333in}}%
\pgfpathlineto{\pgfqpoint{1.061011in}{3.458214in}}%
\pgfpathlineto{\pgfqpoint{1.049092in}{3.440000in}}%
\pgfpathlineto{\pgfqpoint{1.033373in}{3.409291in}}%
\pgfpathlineto{\pgfqpoint{1.027950in}{3.390991in}}%
\pgfpathlineto{\pgfqpoint{1.019657in}{3.365333in}}%
\pgfpathlineto{\pgfqpoint{1.011399in}{3.317758in}}%
\pgfpathlineto{\pgfqpoint{1.009703in}{3.290667in}}%
\pgfpathlineto{\pgfqpoint{1.009693in}{3.244681in}}%
\pgfpathlineto{\pgfqpoint{1.011889in}{3.216000in}}%
\pgfpathlineto{\pgfqpoint{1.022464in}{3.141333in}}%
\pgfpathlineto{\pgfqpoint{1.025826in}{3.127679in}}%
\pgfpathlineto{\pgfqpoint{1.030127in}{3.104000in}}%
\pgfpathlineto{\pgfqpoint{1.040485in}{3.061652in}}%
\pgfpathlineto{\pgfqpoint{1.066308in}{2.978719in}}%
\pgfpathlineto{\pgfqpoint{1.080566in}{2.936065in}}%
\pgfpathlineto{\pgfqpoint{1.117241in}{2.842667in}}%
\pgfpathlineto{\pgfqpoint{1.150419in}{2.768000in}}%
\pgfpathlineto{\pgfqpoint{1.160727in}{2.745998in}}%
\pgfpathlineto{\pgfqpoint{1.186358in}{2.693333in}}%
\pgfpathlineto{\pgfqpoint{1.200808in}{2.664565in}}%
\pgfpathlineto{\pgfqpoint{1.224837in}{2.618667in}}%
\pgfpathlineto{\pgfqpoint{1.265658in}{2.544000in}}%
\pgfpathlineto{\pgfqpoint{1.308644in}{2.469333in}}%
\pgfpathlineto{\pgfqpoint{1.330894in}{2.432000in}}%
\pgfpathlineto{\pgfqpoint{1.376961in}{2.357333in}}%
\pgfpathlineto{\pgfqpoint{1.424853in}{2.282667in}}%
\pgfpathlineto{\pgfqpoint{1.449401in}{2.245333in}}%
\pgfpathlineto{\pgfqpoint{1.500011in}{2.170667in}}%
\pgfpathlineto{\pgfqpoint{1.561535in}{2.082857in}}%
\pgfpathlineto{\pgfqpoint{1.641697in}{1.972992in}}%
\pgfpathlineto{\pgfqpoint{1.721859in}{1.867405in}}%
\pgfpathlineto{\pgfqpoint{1.806591in}{1.760000in}}%
\pgfpathlineto{\pgfqpoint{1.856702in}{1.698933in}}%
\pgfpathlineto{\pgfqpoint{1.882182in}{1.667425in}}%
\pgfpathlineto{\pgfqpoint{1.962343in}{1.571888in}}%
\pgfpathlineto{\pgfqpoint{2.058476in}{1.461333in}}%
\pgfpathlineto{\pgfqpoint{2.124988in}{1.386667in}}%
\pgfpathlineto{\pgfqpoint{2.162747in}{1.345135in}}%
\pgfpathlineto{\pgfqpoint{2.263054in}{1.237333in}}%
\pgfpathlineto{\pgfqpoint{2.310296in}{1.188101in}}%
\pgfpathlineto{\pgfqpoint{2.334313in}{1.162667in}}%
\pgfpathlineto{\pgfqpoint{2.444162in}{1.050667in}}%
\pgfpathlineto{\pgfqpoint{2.502340in}{0.993647in}}%
\pgfpathlineto{\pgfqpoint{2.541448in}{0.955408in}}%
\pgfpathlineto{\pgfqpoint{2.580795in}{0.917391in}}%
\pgfpathlineto{\pgfqpoint{2.620384in}{0.879600in}}%
\pgfpathlineto{\pgfqpoint{2.643717in}{0.856955in}}%
\pgfpathlineto{\pgfqpoint{2.683798in}{0.819436in}}%
\pgfpathlineto{\pgfqpoint{2.804040in}{0.709588in}}%
\pgfpathlineto{\pgfqpoint{2.884202in}{0.638583in}}%
\pgfpathlineto{\pgfqpoint{2.946130in}{0.585683in}}%
\pgfpathlineto{\pgfqpoint{2.969262in}{0.565333in}}%
\pgfpathlineto{\pgfqpoint{3.013578in}{0.528000in}}%
\pgfpathlineto{\pgfqpoint{3.013578in}{0.528000in}}%
\pgfusepath{fill}%
\end{pgfscope}%
\begin{pgfscope}%
\pgfpathrectangle{\pgfqpoint{0.800000in}{0.528000in}}{\pgfqpoint{3.968000in}{3.696000in}}%
\pgfusepath{clip}%
\pgfsetbuttcap%
\pgfsetroundjoin%
\definecolor{currentfill}{rgb}{0.280894,0.078907,0.402329}%
\pgfsetfillcolor{currentfill}%
\pgfsetlinewidth{0.000000pt}%
\definecolor{currentstroke}{rgb}{0.000000,0.000000,0.000000}%
\pgfsetstrokecolor{currentstroke}%
\pgfsetdash{}{0pt}%
\pgfpathmoveto{\pgfqpoint{3.013578in}{0.528000in}}%
\pgfpathlineto{\pgfqpoint{2.924283in}{0.603779in}}%
\pgfpathlineto{\pgfqpoint{2.863148in}{0.657723in}}%
\pgfpathlineto{\pgfqpoint{2.840221in}{0.677333in}}%
\pgfpathlineto{\pgfqpoint{2.723879in}{0.782370in}}%
\pgfpathlineto{\pgfqpoint{2.636267in}{0.864000in}}%
\pgfpathlineto{\pgfqpoint{2.580795in}{0.917391in}}%
\pgfpathlineto{\pgfqpoint{2.558087in}{0.938667in}}%
\pgfpathlineto{\pgfqpoint{2.502340in}{0.993647in}}%
\pgfpathlineto{\pgfqpoint{2.481688in}{1.013333in}}%
\pgfpathlineto{\pgfqpoint{2.443313in}{1.051515in}}%
\pgfpathlineto{\pgfqpoint{2.323071in}{1.174310in}}%
\pgfpathlineto{\pgfqpoint{2.272570in}{1.227627in}}%
\pgfpathlineto{\pgfqpoint{2.227984in}{1.274667in}}%
\pgfpathlineto{\pgfqpoint{2.158914in}{1.349333in}}%
\pgfpathlineto{\pgfqpoint{2.105997in}{1.408473in}}%
\pgfpathlineto{\pgfqpoint{2.082586in}{1.434065in}}%
\pgfpathlineto{\pgfqpoint{1.993256in}{1.536000in}}%
\pgfpathlineto{\pgfqpoint{1.922263in}{1.619325in}}%
\pgfpathlineto{\pgfqpoint{1.873978in}{1.677692in}}%
\pgfpathlineto{\pgfqpoint{1.836776in}{1.722667in}}%
\pgfpathlineto{\pgfqpoint{1.787843in}{1.784128in}}%
\pgfpathlineto{\pgfqpoint{1.761939in}{1.816227in}}%
\pgfpathlineto{\pgfqpoint{1.681778in}{1.919733in}}%
\pgfpathlineto{\pgfqpoint{1.633501in}{1.984000in}}%
\pgfpathlineto{\pgfqpoint{1.578939in}{2.058667in}}%
\pgfpathlineto{\pgfqpoint{1.521455in}{2.139554in}}%
\pgfpathlineto{\pgfqpoint{1.441293in}{2.257586in}}%
\pgfpathlineto{\pgfqpoint{1.321051in}{2.448425in}}%
\pgfpathlineto{\pgfqpoint{1.240889in}{2.588622in}}%
\pgfpathlineto{\pgfqpoint{1.200808in}{2.664565in}}%
\pgfpathlineto{\pgfqpoint{1.133543in}{2.805333in}}%
\pgfpathlineto{\pgfqpoint{1.130081in}{2.814121in}}%
\pgfpathlineto{\pgfqpoint{1.114969in}{2.847955in}}%
\pgfpathlineto{\pgfqpoint{1.080566in}{2.936065in}}%
\pgfpathlineto{\pgfqpoint{1.053034in}{3.017645in}}%
\pgfpathlineto{\pgfqpoint{1.038589in}{3.068432in}}%
\pgfpathlineto{\pgfqpoint{1.030127in}{3.104000in}}%
\pgfpathlineto{\pgfqpoint{1.016298in}{3.178667in}}%
\pgfpathlineto{\pgfqpoint{1.015241in}{3.192486in}}%
\pgfpathlineto{\pgfqpoint{1.011889in}{3.216000in}}%
\pgfpathlineto{\pgfqpoint{1.009556in}{3.253333in}}%
\pgfpathlineto{\pgfqpoint{1.010042in}{3.262311in}}%
\pgfpathlineto{\pgfqpoint{1.009703in}{3.290667in}}%
\pgfpathlineto{\pgfqpoint{1.012847in}{3.328000in}}%
\pgfpathlineto{\pgfqpoint{1.016074in}{3.342596in}}%
\pgfpathlineto{\pgfqpoint{1.019657in}{3.365333in}}%
\pgfpathlineto{\pgfqpoint{1.033373in}{3.409291in}}%
\pgfpathlineto{\pgfqpoint{1.049092in}{3.440000in}}%
\pgfpathlineto{\pgfqpoint{1.061011in}{3.458214in}}%
\pgfpathlineto{\pgfqpoint{1.080566in}{3.480870in}}%
\pgfpathlineto{\pgfqpoint{1.125020in}{3.514667in}}%
\pgfpathlineto{\pgfqpoint{1.148196in}{3.526339in}}%
\pgfpathlineto{\pgfqpoint{1.160727in}{3.531062in}}%
\pgfpathlineto{\pgfqpoint{1.200808in}{3.542764in}}%
\pgfpathlineto{\pgfqpoint{1.243620in}{3.549456in}}%
\pgfpathlineto{\pgfqpoint{1.280970in}{3.551258in}}%
\pgfpathlineto{\pgfqpoint{1.321051in}{3.549996in}}%
\pgfpathlineto{\pgfqpoint{1.368572in}{3.545070in}}%
\pgfpathlineto{\pgfqpoint{1.417706in}{3.536636in}}%
\pgfpathlineto{\pgfqpoint{1.481374in}{3.521522in}}%
\pgfpathlineto{\pgfqpoint{1.505322in}{3.514667in}}%
\pgfpathlineto{\pgfqpoint{1.561535in}{3.496954in}}%
\pgfpathlineto{\pgfqpoint{1.615976in}{3.477333in}}%
\pgfpathlineto{\pgfqpoint{1.681778in}{3.450922in}}%
\pgfpathlineto{\pgfqpoint{1.761939in}{3.415012in}}%
\pgfpathlineto{\pgfqpoint{1.787681in}{3.402667in}}%
\pgfpathlineto{\pgfqpoint{1.842101in}{3.375589in}}%
\pgfpathlineto{\pgfqpoint{1.931266in}{3.328000in}}%
\pgfpathlineto{\pgfqpoint{2.097283in}{3.229689in}}%
\pgfpathlineto{\pgfqpoint{2.127730in}{3.211284in}}%
\pgfpathlineto{\pgfqpoint{2.234789in}{3.141333in}}%
\pgfpathlineto{\pgfqpoint{2.403232in}{3.023664in}}%
\pgfpathlineto{\pgfqpoint{2.446495in}{2.992000in}}%
\pgfpathlineto{\pgfqpoint{2.563556in}{2.903291in}}%
\pgfpathlineto{\pgfqpoint{2.620394in}{2.858276in}}%
\pgfpathlineto{\pgfqpoint{2.661061in}{2.826512in}}%
\pgfpathlineto{\pgfqpoint{2.732927in}{2.768000in}}%
\pgfpathlineto{\pgfqpoint{2.791414in}{2.718906in}}%
\pgfpathlineto{\pgfqpoint{2.822193in}{2.693333in}}%
\pgfpathlineto{\pgfqpoint{2.865884in}{2.656000in}}%
\pgfpathlineto{\pgfqpoint{2.964364in}{2.569946in}}%
\pgfpathlineto{\pgfqpoint{3.084606in}{2.461134in}}%
\pgfpathlineto{\pgfqpoint{3.124687in}{2.423944in}}%
\pgfpathlineto{\pgfqpoint{3.244929in}{2.309570in}}%
\pgfpathlineto{\pgfqpoint{3.285010in}{2.270498in}}%
\pgfpathlineto{\pgfqpoint{3.385223in}{2.170667in}}%
\pgfpathlineto{\pgfqpoint{3.485414in}{2.067872in}}%
\pgfpathlineto{\pgfqpoint{3.565576in}{1.983334in}}%
\pgfpathlineto{\pgfqpoint{3.667755in}{1.872000in}}%
\pgfpathlineto{\pgfqpoint{3.712248in}{1.821951in}}%
\pgfpathlineto{\pgfqpoint{3.734581in}{1.797333in}}%
\pgfpathlineto{\pgfqpoint{3.784415in}{1.739838in}}%
\pgfpathlineto{\pgfqpoint{3.820172in}{1.698477in}}%
\pgfpathlineto{\pgfqpoint{3.855715in}{1.656918in}}%
\pgfpathlineto{\pgfqpoint{3.894954in}{1.610667in}}%
\pgfpathlineto{\pgfqpoint{3.926303in}{1.573027in}}%
\pgfpathlineto{\pgfqpoint{4.016565in}{1.461333in}}%
\pgfpathlineto{\pgfqpoint{4.046545in}{1.423542in}}%
\pgfpathlineto{\pgfqpoint{4.145334in}{1.294650in}}%
\pgfpathlineto{\pgfqpoint{4.214903in}{1.200000in}}%
\pgfpathlineto{\pgfqpoint{4.246949in}{1.155274in}}%
\pgfpathlineto{\pgfqpoint{4.327111in}{1.039438in}}%
\pgfpathlineto{\pgfqpoint{4.447354in}{0.853068in}}%
\pgfpathlineto{\pgfqpoint{4.549629in}{0.677333in}}%
\pgfpathlineto{\pgfqpoint{4.589449in}{0.602667in}}%
\pgfpathlineto{\pgfqpoint{4.595038in}{0.590895in}}%
\pgfpathlineto{\pgfqpoint{4.609571in}{0.563569in}}%
\pgfpathlineto{\pgfqpoint{4.626825in}{0.528000in}}%
\pgfpathlineto{\pgfqpoint{4.638622in}{0.528000in}}%
\pgfpathlineto{\pgfqpoint{4.607677in}{0.589565in}}%
\pgfpathlineto{\pgfqpoint{4.539835in}{0.714667in}}%
\pgfpathlineto{\pgfqpoint{4.487434in}{0.804455in}}%
\pgfpathlineto{\pgfqpoint{4.427647in}{0.901333in}}%
\pgfpathlineto{\pgfqpoint{4.403937in}{0.938667in}}%
\pgfpathlineto{\pgfqpoint{4.354763in}{1.013333in}}%
\pgfpathlineto{\pgfqpoint{4.327111in}{1.054475in}}%
\pgfpathlineto{\pgfqpoint{4.246949in}{1.169240in}}%
\pgfpathlineto{\pgfqpoint{4.206869in}{1.224621in}}%
\pgfpathlineto{\pgfqpoint{4.141934in}{1.312000in}}%
\pgfpathlineto{\pgfqpoint{4.102070in}{1.363718in}}%
\pgfpathlineto{\pgfqpoint{4.080711in}{1.392176in}}%
\pgfpathlineto{\pgfqpoint{3.996384in}{1.498667in}}%
\pgfpathlineto{\pgfqpoint{3.966045in}{1.536316in}}%
\pgfpathlineto{\pgfqpoint{3.873227in}{1.648000in}}%
\pgfpathlineto{\pgfqpoint{3.825506in}{1.703446in}}%
\pgfpathlineto{\pgfqpoint{3.789768in}{1.744824in}}%
\pgfpathlineto{\pgfqpoint{3.753814in}{1.786002in}}%
\pgfpathlineto{\pgfqpoint{3.717641in}{1.826975in}}%
\pgfpathlineto{\pgfqpoint{3.677548in}{1.872000in}}%
\pgfpathlineto{\pgfqpoint{3.643721in}{1.909333in}}%
\pgfpathlineto{\pgfqpoint{3.565576in}{1.993696in}}%
\pgfpathlineto{\pgfqpoint{3.513939in}{2.047903in}}%
\pgfpathlineto{\pgfqpoint{3.476188in}{2.087406in}}%
\pgfpathlineto{\pgfqpoint{3.438207in}{2.126695in}}%
\pgfpathlineto{\pgfqpoint{3.395329in}{2.170667in}}%
\pgfpathlineto{\pgfqpoint{3.282945in}{2.282667in}}%
\pgfpathlineto{\pgfqpoint{3.224707in}{2.338497in}}%
\pgfpathlineto{\pgfqpoint{3.185148in}{2.376317in}}%
\pgfpathlineto{\pgfqpoint{3.164768in}{2.396303in}}%
\pgfpathlineto{\pgfqpoint{3.044525in}{2.507818in}}%
\pgfpathlineto{\pgfqpoint{2.962738in}{2.581333in}}%
\pgfpathlineto{\pgfqpoint{2.901352in}{2.634641in}}%
\pgfpathlineto{\pgfqpoint{2.877380in}{2.656000in}}%
\pgfpathlineto{\pgfqpoint{2.763960in}{2.752142in}}%
\pgfpathlineto{\pgfqpoint{2.723879in}{2.785274in}}%
\pgfpathlineto{\pgfqpoint{2.643717in}{2.850245in}}%
\pgfpathlineto{\pgfqpoint{2.582699in}{2.897831in}}%
\pgfpathlineto{\pgfqpoint{2.558402in}{2.917333in}}%
\pgfpathlineto{\pgfqpoint{2.443313in}{3.004373in}}%
\pgfpathlineto{\pgfqpoint{2.383021in}{3.047841in}}%
\pgfpathlineto{\pgfqpoint{2.357644in}{3.066667in}}%
\pgfpathlineto{\pgfqpoint{2.291995in}{3.112388in}}%
\pgfpathlineto{\pgfqpoint{2.268847in}{3.128160in}}%
\pgfpathlineto{\pgfqpoint{2.242909in}{3.146396in}}%
\pgfpathlineto{\pgfqpoint{2.175546in}{3.190588in}}%
\pgfpathlineto{\pgfqpoint{2.136911in}{3.216000in}}%
\pgfpathlineto{\pgfqpoint{2.077665in}{3.253333in}}%
\pgfpathlineto{\pgfqpoint{1.884179in}{3.365333in}}%
\pgfpathlineto{\pgfqpoint{1.802020in}{3.407828in}}%
\pgfpathlineto{\pgfqpoint{1.761939in}{3.427273in}}%
\pgfpathlineto{\pgfqpoint{1.681778in}{3.463697in}}%
\pgfpathlineto{\pgfqpoint{1.671383in}{3.467651in}}%
\pgfpathlineto{\pgfqpoint{1.641697in}{3.480616in}}%
\pgfpathlineto{\pgfqpoint{1.521455in}{3.524219in}}%
\pgfpathlineto{\pgfqpoint{1.418161in}{3.552000in}}%
\pgfpathlineto{\pgfqpoint{1.401212in}{3.555758in}}%
\pgfpathlineto{\pgfqpoint{1.321051in}{3.567411in}}%
\pgfpathlineto{\pgfqpoint{1.302792in}{3.569007in}}%
\pgfpathlineto{\pgfqpoint{1.261625in}{3.570019in}}%
\pgfpathlineto{\pgfqpoint{1.223825in}{3.567894in}}%
\pgfpathlineto{\pgfqpoint{1.200808in}{3.564658in}}%
\pgfpathlineto{\pgfqpoint{1.149887in}{3.552000in}}%
\pgfpathlineto{\pgfqpoint{1.120646in}{3.539911in}}%
\pgfpathlineto{\pgfqpoint{1.079981in}{3.514667in}}%
\pgfpathlineto{\pgfqpoint{1.060099in}{3.496397in}}%
\pgfpathlineto{\pgfqpoint{1.040485in}{3.472683in}}%
\pgfpathlineto{\pgfqpoint{1.021493in}{3.440000in}}%
\pgfpathlineto{\pgfqpoint{1.014973in}{3.426430in}}%
\pgfpathlineto{\pgfqpoint{1.006528in}{3.402667in}}%
\pgfpathlineto{\pgfqpoint{0.997289in}{3.365333in}}%
\pgfpathlineto{\pgfqpoint{0.992458in}{3.328000in}}%
\pgfpathlineto{\pgfqpoint{0.992509in}{3.320646in}}%
\pgfpathlineto{\pgfqpoint{0.990776in}{3.290667in}}%
\pgfpathlineto{\pgfqpoint{0.991708in}{3.253333in}}%
\pgfpathlineto{\pgfqpoint{0.992686in}{3.246144in}}%
\pgfpathlineto{\pgfqpoint{0.994836in}{3.216000in}}%
\pgfpathlineto{\pgfqpoint{1.000404in}{3.175522in}}%
\pgfpathlineto{\pgfqpoint{1.015230in}{3.104000in}}%
\pgfpathlineto{\pgfqpoint{1.024877in}{3.066667in}}%
\pgfpathlineto{\pgfqpoint{1.040485in}{3.013961in}}%
\pgfpathlineto{\pgfqpoint{1.047584in}{2.992000in}}%
\pgfpathlineto{\pgfqpoint{1.094565in}{2.866960in}}%
\pgfpathlineto{\pgfqpoint{1.121530in}{2.804511in}}%
\pgfpathlineto{\pgfqpoint{1.160727in}{2.721494in}}%
\pgfpathlineto{\pgfqpoint{1.240889in}{2.568566in}}%
\pgfpathlineto{\pgfqpoint{1.280970in}{2.498086in}}%
\pgfpathlineto{\pgfqpoint{1.343208in}{2.394667in}}%
\pgfpathlineto{\pgfqpoint{1.401212in}{2.303143in}}%
\pgfpathlineto{\pgfqpoint{1.464413in}{2.208000in}}%
\pgfpathlineto{\pgfqpoint{1.489891in}{2.170667in}}%
\pgfpathlineto{\pgfqpoint{1.542281in}{2.096000in}}%
\pgfpathlineto{\pgfqpoint{1.582469in}{2.040832in}}%
\pgfpathlineto{\pgfqpoint{1.601616in}{2.013832in}}%
\pgfpathlineto{\pgfqpoint{1.681778in}{1.906813in}}%
\pgfpathlineto{\pgfqpoint{1.721859in}{1.855012in}}%
\pgfpathlineto{\pgfqpoint{1.802020in}{1.753680in}}%
\pgfpathlineto{\pgfqpoint{1.851138in}{1.693751in}}%
\pgfpathlineto{\pgfqpoint{1.882182in}{1.655725in}}%
\pgfpathlineto{\pgfqpoint{1.962343in}{1.560746in}}%
\pgfpathlineto{\pgfqpoint{2.048705in}{1.461333in}}%
\pgfpathlineto{\pgfqpoint{2.122667in}{1.378556in}}%
\pgfpathlineto{\pgfqpoint{2.162747in}{1.334583in}}%
\pgfpathlineto{\pgfqpoint{2.253156in}{1.237333in}}%
\pgfpathlineto{\pgfqpoint{2.305104in}{1.183265in}}%
\pgfpathlineto{\pgfqpoint{2.334885in}{1.151662in}}%
\pgfpathlineto{\pgfqpoint{2.443313in}{1.041513in}}%
\pgfpathlineto{\pgfqpoint{2.497053in}{0.988723in}}%
\pgfpathlineto{\pgfqpoint{2.536142in}{0.950466in}}%
\pgfpathlineto{\pgfqpoint{2.575470in}{0.912431in}}%
\pgfpathlineto{\pgfqpoint{2.615040in}{0.874622in}}%
\pgfpathlineto{\pgfqpoint{2.654853in}{0.837040in}}%
\pgfpathlineto{\pgfqpoint{2.683798in}{0.809613in}}%
\pgfpathlineto{\pgfqpoint{2.787572in}{0.714667in}}%
\pgfpathlineto{\pgfqpoint{2.857683in}{0.652632in}}%
\pgfpathlineto{\pgfqpoint{2.884202in}{0.628880in}}%
\pgfpathlineto{\pgfqpoint{2.940623in}{0.580554in}}%
\pgfpathlineto{\pgfqpoint{2.964364in}{0.559672in}}%
\pgfpathlineto{\pgfqpoint{3.004444in}{0.528000in}}%
\pgfpathlineto{\pgfqpoint{3.004444in}{0.528000in}}%
\pgfusepath{fill}%
\end{pgfscope}%
\begin{pgfscope}%
\pgfpathrectangle{\pgfqpoint{0.800000in}{0.528000in}}{\pgfqpoint{3.968000in}{3.696000in}}%
\pgfusepath{clip}%
\pgfsetbuttcap%
\pgfsetroundjoin%
\definecolor{currentfill}{rgb}{0.281446,0.084320,0.407414}%
\pgfsetfillcolor{currentfill}%
\pgfsetlinewidth{0.000000pt}%
\definecolor{currentstroke}{rgb}{0.000000,0.000000,0.000000}%
\pgfsetstrokecolor{currentstroke}%
\pgfsetdash{}{0pt}%
\pgfpathmoveto{\pgfqpoint{3.001743in}{0.528000in}}%
\pgfpathlineto{\pgfqpoint{2.940623in}{0.580554in}}%
\pgfpathlineto{\pgfqpoint{2.914356in}{0.602667in}}%
\pgfpathlineto{\pgfqpoint{2.857683in}{0.652632in}}%
\pgfpathlineto{\pgfqpoint{2.829284in}{0.677333in}}%
\pgfpathlineto{\pgfqpoint{2.723879in}{0.772572in}}%
\pgfpathlineto{\pgfqpoint{2.625854in}{0.864000in}}%
\pgfpathlineto{\pgfqpoint{2.575470in}{0.912431in}}%
\pgfpathlineto{\pgfqpoint{2.547869in}{0.938667in}}%
\pgfpathlineto{\pgfqpoint{2.497053in}{0.988723in}}%
\pgfpathlineto{\pgfqpoint{2.471660in}{1.013333in}}%
\pgfpathlineto{\pgfqpoint{2.434190in}{1.050667in}}%
\pgfpathlineto{\pgfqpoint{2.323071in}{1.163869in}}%
\pgfpathlineto{\pgfqpoint{2.267396in}{1.222809in}}%
\pgfpathlineto{\pgfqpoint{2.242909in}{1.248199in}}%
\pgfpathlineto{\pgfqpoint{2.149279in}{1.349333in}}%
\pgfpathlineto{\pgfqpoint{2.100713in}{1.403551in}}%
\pgfpathlineto{\pgfqpoint{2.077648in}{1.428600in}}%
\pgfpathlineto{\pgfqpoint{1.983658in}{1.536000in}}%
\pgfpathlineto{\pgfqpoint{1.919831in}{1.610667in}}%
\pgfpathlineto{\pgfqpoint{1.868618in}{1.672699in}}%
\pgfpathlineto{\pgfqpoint{1.833787in}{1.714922in}}%
\pgfpathlineto{\pgfqpoint{1.796931in}{1.760000in}}%
\pgfpathlineto{\pgfqpoint{1.761939in}{1.803871in}}%
\pgfpathlineto{\pgfqpoint{1.679846in}{1.909333in}}%
\pgfpathlineto{\pgfqpoint{1.631300in}{1.974316in}}%
\pgfpathlineto{\pgfqpoint{1.601616in}{2.013832in}}%
\pgfpathlineto{\pgfqpoint{1.542281in}{2.096000in}}%
\pgfpathlineto{\pgfqpoint{1.515809in}{2.133333in}}%
\pgfpathlineto{\pgfqpoint{1.464413in}{2.208000in}}%
\pgfpathlineto{\pgfqpoint{1.401212in}{2.303143in}}%
\pgfpathlineto{\pgfqpoint{1.297954in}{2.469333in}}%
\pgfpathlineto{\pgfqpoint{1.254691in}{2.544000in}}%
\pgfpathlineto{\pgfqpoint{1.213578in}{2.618667in}}%
\pgfpathlineto{\pgfqpoint{1.200808in}{2.642694in}}%
\pgfpathlineto{\pgfqpoint{1.170582in}{2.702512in}}%
\pgfpathlineto{\pgfqpoint{1.156195in}{2.730667in}}%
\pgfpathlineto{\pgfqpoint{1.138527in}{2.768000in}}%
\pgfpathlineto{\pgfqpoint{1.120646in}{2.806523in}}%
\pgfpathlineto{\pgfqpoint{1.080566in}{2.901895in}}%
\pgfpathlineto{\pgfqpoint{1.060680in}{2.954667in}}%
\pgfpathlineto{\pgfqpoint{1.056126in}{2.969236in}}%
\pgfpathlineto{\pgfqpoint{1.040485in}{3.013961in}}%
\pgfpathlineto{\pgfqpoint{1.019451in}{3.086259in}}%
\pgfpathlineto{\pgfqpoint{1.006100in}{3.146639in}}%
\pgfpathlineto{\pgfqpoint{0.999707in}{3.179316in}}%
\pgfpathlineto{\pgfqpoint{0.994836in}{3.216000in}}%
\pgfpathlineto{\pgfqpoint{0.991092in}{3.262007in}}%
\pgfpathlineto{\pgfqpoint{0.990776in}{3.290667in}}%
\pgfpathlineto{\pgfqpoint{0.992921in}{3.334971in}}%
\pgfpathlineto{\pgfqpoint{1.000404in}{3.379363in}}%
\pgfpathlineto{\pgfqpoint{1.006528in}{3.402667in}}%
\pgfpathlineto{\pgfqpoint{1.014973in}{3.426430in}}%
\pgfpathlineto{\pgfqpoint{1.021493in}{3.440000in}}%
\pgfpathlineto{\pgfqpoint{1.043861in}{3.477333in}}%
\pgfpathlineto{\pgfqpoint{1.060099in}{3.496397in}}%
\pgfpathlineto{\pgfqpoint{1.081942in}{3.515948in}}%
\pgfpathlineto{\pgfqpoint{1.129070in}{3.544154in}}%
\pgfpathlineto{\pgfqpoint{1.160727in}{3.555733in}}%
\pgfpathlineto{\pgfqpoint{1.165649in}{3.556584in}}%
\pgfpathlineto{\pgfqpoint{1.200808in}{3.564658in}}%
\pgfpathlineto{\pgfqpoint{1.223825in}{3.567894in}}%
\pgfpathlineto{\pgfqpoint{1.261625in}{3.570019in}}%
\pgfpathlineto{\pgfqpoint{1.302792in}{3.569007in}}%
\pgfpathlineto{\pgfqpoint{1.347522in}{3.564677in}}%
\pgfpathlineto{\pgfqpoint{1.401212in}{3.555758in}}%
\pgfpathlineto{\pgfqpoint{1.441293in}{3.546852in}}%
\pgfpathlineto{\pgfqpoint{1.550328in}{3.514667in}}%
\pgfpathlineto{\pgfqpoint{1.601616in}{3.496192in}}%
\pgfpathlineto{\pgfqpoint{1.615483in}{3.490249in}}%
\pgfpathlineto{\pgfqpoint{1.649494in}{3.477333in}}%
\pgfpathlineto{\pgfqpoint{1.761939in}{3.427273in}}%
\pgfpathlineto{\pgfqpoint{1.778894in}{3.418459in}}%
\pgfpathlineto{\pgfqpoint{1.812158in}{3.402667in}}%
\pgfpathlineto{\pgfqpoint{1.962343in}{3.321919in}}%
\pgfpathlineto{\pgfqpoint{2.007534in}{3.295426in}}%
\pgfpathlineto{\pgfqpoint{2.042505in}{3.274712in}}%
\pgfpathlineto{\pgfqpoint{2.077665in}{3.253333in}}%
\pgfpathlineto{\pgfqpoint{2.136911in}{3.216000in}}%
\pgfpathlineto{\pgfqpoint{2.202828in}{3.173211in}}%
\pgfpathlineto{\pgfqpoint{2.268847in}{3.128160in}}%
\pgfpathlineto{\pgfqpoint{2.304509in}{3.104000in}}%
\pgfpathlineto{\pgfqpoint{2.363152in}{3.062769in}}%
\pgfpathlineto{\pgfqpoint{2.428041in}{3.015109in}}%
\pgfpathlineto{\pgfqpoint{2.459920in}{2.992000in}}%
\pgfpathlineto{\pgfqpoint{2.563556in}{2.913369in}}%
\pgfpathlineto{\pgfqpoint{2.626153in}{2.863640in}}%
\pgfpathlineto{\pgfqpoint{2.653147in}{2.842667in}}%
\pgfpathlineto{\pgfqpoint{2.763960in}{2.752142in}}%
\pgfpathlineto{\pgfqpoint{2.884202in}{2.650125in}}%
\pgfpathlineto{\pgfqpoint{2.942651in}{2.598442in}}%
\pgfpathlineto{\pgfqpoint{2.964364in}{2.579896in}}%
\pgfpathlineto{\pgfqpoint{3.045785in}{2.506667in}}%
\pgfpathlineto{\pgfqpoint{3.105306in}{2.451281in}}%
\pgfpathlineto{\pgfqpoint{3.126739in}{2.432000in}}%
\pgfpathlineto{\pgfqpoint{3.205786in}{2.357333in}}%
\pgfpathlineto{\pgfqpoint{3.264029in}{2.300457in}}%
\pgfpathlineto{\pgfqpoint{3.285010in}{2.280650in}}%
\pgfpathlineto{\pgfqpoint{3.405253in}{2.160621in}}%
\pgfpathlineto{\pgfqpoint{3.504047in}{2.058667in}}%
\pgfpathlineto{\pgfqpoint{3.551463in}{2.008188in}}%
\pgfpathlineto{\pgfqpoint{3.574635in}{1.984000in}}%
\pgfpathlineto{\pgfqpoint{3.645737in}{1.907127in}}%
\pgfpathlineto{\pgfqpoint{3.685818in}{1.862843in}}%
\pgfpathlineto{\pgfqpoint{3.777043in}{1.760000in}}%
\pgfpathlineto{\pgfqpoint{3.825506in}{1.703446in}}%
\pgfpathlineto{\pgfqpoint{3.861030in}{1.661868in}}%
\pgfpathlineto{\pgfqpoint{3.886222in}{1.632585in}}%
\pgfpathlineto{\pgfqpoint{3.966384in}{1.535900in}}%
\pgfpathlineto{\pgfqpoint{4.017592in}{1.471698in}}%
\pgfpathlineto{\pgfqpoint{4.055749in}{1.424000in}}%
\pgfpathlineto{\pgfqpoint{4.126707in}{1.332131in}}%
\pgfpathlineto{\pgfqpoint{4.287030in}{1.112440in}}%
\pgfpathlineto{\pgfqpoint{4.354763in}{1.013333in}}%
\pgfpathlineto{\pgfqpoint{4.379510in}{0.976000in}}%
\pgfpathlineto{\pgfqpoint{4.427647in}{0.901333in}}%
\pgfpathlineto{\pgfqpoint{4.474002in}{0.826667in}}%
\pgfpathlineto{\pgfqpoint{4.527515in}{0.736253in}}%
\pgfpathlineto{\pgfqpoint{4.567596in}{0.664905in}}%
\pgfpathlineto{\pgfqpoint{4.638622in}{0.528000in}}%
\pgfpathlineto{\pgfqpoint{4.650293in}{0.528000in}}%
\pgfpathlineto{\pgfqpoint{4.647758in}{0.533053in}}%
\pgfpathlineto{\pgfqpoint{4.607677in}{0.611125in}}%
\pgfpathlineto{\pgfqpoint{4.567596in}{0.684769in}}%
\pgfpathlineto{\pgfqpoint{4.461446in}{0.864000in}}%
\pgfpathlineto{\pgfqpoint{4.407273in}{0.949134in}}%
\pgfpathlineto{\pgfqpoint{4.339585in}{1.050667in}}%
\pgfpathlineto{\pgfqpoint{4.313924in}{1.088000in}}%
\pgfpathlineto{\pgfqpoint{4.246949in}{1.182826in}}%
\pgfpathlineto{\pgfqpoint{4.086626in}{1.396806in}}%
\pgfpathlineto{\pgfqpoint{4.006101in}{1.498667in}}%
\pgfpathlineto{\pgfqpoint{3.966384in}{1.547439in}}%
\pgfpathlineto{\pgfqpoint{3.882896in}{1.648000in}}%
\pgfpathlineto{\pgfqpoint{3.830840in}{1.708414in}}%
\pgfpathlineto{\pgfqpoint{3.795122in}{1.749811in}}%
\pgfpathlineto{\pgfqpoint{3.759188in}{1.791007in}}%
\pgfpathlineto{\pgfqpoint{3.720744in}{1.834667in}}%
\pgfpathlineto{\pgfqpoint{3.645737in}{1.917603in}}%
\pgfpathlineto{\pgfqpoint{3.525495in}{2.046486in}}%
\pgfpathlineto{\pgfqpoint{3.478118in}{2.096000in}}%
\pgfpathlineto{\pgfqpoint{3.365172in}{2.211197in}}%
\pgfpathlineto{\pgfqpoint{3.244929in}{2.329461in}}%
\pgfpathlineto{\pgfqpoint{3.190409in}{2.381217in}}%
\pgfpathlineto{\pgfqpoint{3.164768in}{2.406008in}}%
\pgfpathlineto{\pgfqpoint{3.044525in}{2.517454in}}%
\pgfpathlineto{\pgfqpoint{2.924283in}{2.624925in}}%
\pgfpathlineto{\pgfqpoint{2.884202in}{2.659881in}}%
\pgfpathlineto{\pgfqpoint{2.801371in}{2.730667in}}%
\pgfpathlineto{\pgfqpoint{2.683798in}{2.827753in}}%
\pgfpathlineto{\pgfqpoint{2.563556in}{2.923223in}}%
\pgfpathlineto{\pgfqpoint{2.473345in}{2.992000in}}%
\pgfpathlineto{\pgfqpoint{2.363152in}{3.072868in}}%
\pgfpathlineto{\pgfqpoint{2.298193in}{3.118161in}}%
\pgfpathlineto{\pgfqpoint{2.265333in}{3.141333in}}%
\pgfpathlineto{\pgfqpoint{2.082586in}{3.261016in}}%
\pgfpathlineto{\pgfqpoint{2.042505in}{3.285684in}}%
\pgfpathlineto{\pgfqpoint{1.962343in}{3.333082in}}%
\pgfpathlineto{\pgfqpoint{1.922263in}{3.355742in}}%
\pgfpathlineto{\pgfqpoint{1.835241in}{3.402667in}}%
\pgfpathlineto{\pgfqpoint{1.681778in}{3.476472in}}%
\pgfpathlineto{\pgfqpoint{1.653184in}{3.488033in}}%
\pgfpathlineto{\pgfqpoint{1.641697in}{3.493351in}}%
\pgfpathlineto{\pgfqpoint{1.625286in}{3.499381in}}%
\pgfpathlineto{\pgfqpoint{1.587813in}{3.514667in}}%
\pgfpathlineto{\pgfqpoint{1.561535in}{3.524422in}}%
\pgfpathlineto{\pgfqpoint{1.441293in}{3.561747in}}%
\pgfpathlineto{\pgfqpoint{1.401212in}{3.571163in}}%
\pgfpathlineto{\pgfqpoint{1.384516in}{3.573782in}}%
\pgfpathlineto{\pgfqpoint{1.361131in}{3.578914in}}%
\pgfpathlineto{\pgfqpoint{1.321051in}{3.584696in}}%
\pgfpathlineto{\pgfqpoint{1.280970in}{3.588124in}}%
\pgfpathlineto{\pgfqpoint{1.240889in}{3.588709in}}%
\pgfpathlineto{\pgfqpoint{1.195792in}{3.584661in}}%
\pgfpathlineto{\pgfqpoint{1.160727in}{3.578595in}}%
\pgfpathlineto{\pgfqpoint{1.110175in}{3.561754in}}%
\pgfpathlineto{\pgfqpoint{1.080566in}{3.545534in}}%
\pgfpathlineto{\pgfqpoint{1.061801in}{3.532145in}}%
\pgfpathlineto{\pgfqpoint{1.040485in}{3.512437in}}%
\pgfpathlineto{\pgfqpoint{1.024092in}{3.492602in}}%
\pgfpathlineto{\pgfqpoint{1.014257in}{3.477333in}}%
\pgfpathlineto{\pgfqpoint{0.995417in}{3.440000in}}%
\pgfpathlineto{\pgfqpoint{0.993799in}{3.433848in}}%
\pgfpathlineto{\pgfqpoint{0.983754in}{3.402667in}}%
\pgfpathlineto{\pgfqpoint{0.980840in}{3.384443in}}%
\pgfpathlineto{\pgfqpoint{0.974673in}{3.351967in}}%
\pgfpathlineto{\pgfqpoint{0.973083in}{3.328000in}}%
\pgfpathlineto{\pgfqpoint{0.974522in}{3.253333in}}%
\pgfpathlineto{\pgfqpoint{0.989372in}{3.151609in}}%
\pgfpathlineto{\pgfqpoint{1.000404in}{3.103754in}}%
\pgfpathlineto{\pgfqpoint{1.022009in}{3.029333in}}%
\pgfpathlineto{\pgfqpoint{1.026717in}{3.016509in}}%
\pgfpathlineto{\pgfqpoint{1.040485in}{2.974302in}}%
\pgfpathlineto{\pgfqpoint{1.077920in}{2.877536in}}%
\pgfpathlineto{\pgfqpoint{1.101008in}{2.823626in}}%
\pgfpathlineto{\pgfqpoint{1.126634in}{2.768000in}}%
\pgfpathlineto{\pgfqpoint{1.150061in}{2.720731in}}%
\pgfpathlineto{\pgfqpoint{1.163224in}{2.693333in}}%
\pgfpathlineto{\pgfqpoint{1.182666in}{2.656000in}}%
\pgfpathlineto{\pgfqpoint{1.222930in}{2.581333in}}%
\pgfpathlineto{\pgfqpoint{1.240889in}{2.549047in}}%
\pgfpathlineto{\pgfqpoint{1.271091in}{2.497465in}}%
\pgfpathlineto{\pgfqpoint{1.287265in}{2.469333in}}%
\pgfpathlineto{\pgfqpoint{1.314046in}{2.425475in}}%
\pgfpathlineto{\pgfqpoint{1.348865in}{2.368759in}}%
\pgfpathlineto{\pgfqpoint{1.404311in}{2.282667in}}%
\pgfpathlineto{\pgfqpoint{1.441293in}{2.227315in}}%
\pgfpathlineto{\pgfqpoint{1.506014in}{2.133333in}}%
\pgfpathlineto{\pgfqpoint{1.544360in}{2.080002in}}%
\pgfpathlineto{\pgfqpoint{1.561535in}{2.055304in}}%
\pgfpathlineto{\pgfqpoint{1.601616in}{2.000701in}}%
\pgfpathlineto{\pgfqpoint{1.681778in}{1.894380in}}%
\pgfpathlineto{\pgfqpoint{1.728079in}{1.834667in}}%
\pgfpathlineto{\pgfqpoint{1.802020in}{1.741911in}}%
\pgfpathlineto{\pgfqpoint{1.863258in}{1.667706in}}%
\pgfpathlineto{\pgfqpoint{1.882182in}{1.644193in}}%
\pgfpathlineto{\pgfqpoint{1.933512in}{1.583812in}}%
\pgfpathlineto{\pgfqpoint{1.962343in}{1.549603in}}%
\pgfpathlineto{\pgfqpoint{2.042505in}{1.457447in}}%
\pgfpathlineto{\pgfqpoint{2.139645in}{1.349333in}}%
\pgfpathlineto{\pgfqpoint{2.187505in}{1.297728in}}%
\pgfpathlineto{\pgfqpoint{2.208367in}{1.274667in}}%
\pgfpathlineto{\pgfqpoint{2.282990in}{1.195490in}}%
\pgfpathlineto{\pgfqpoint{2.323071in}{1.153779in}}%
\pgfpathlineto{\pgfqpoint{2.443313in}{1.031543in}}%
\pgfpathlineto{\pgfqpoint{2.491767in}{0.983799in}}%
\pgfpathlineto{\pgfqpoint{2.530837in}{0.945524in}}%
\pgfpathlineto{\pgfqpoint{2.576319in}{0.901333in}}%
\pgfpathlineto{\pgfqpoint{2.683798in}{0.799791in}}%
\pgfpathlineto{\pgfqpoint{2.776744in}{0.714667in}}%
\pgfpathlineto{\pgfqpoint{2.818346in}{0.677333in}}%
\pgfpathlineto{\pgfqpoint{2.924283in}{0.584379in}}%
\pgfpathlineto{\pgfqpoint{2.976965in}{0.539738in}}%
\pgfpathlineto{\pgfqpoint{2.990347in}{0.528000in}}%
\pgfpathlineto{\pgfqpoint{2.990347in}{0.528000in}}%
\pgfusepath{fill}%
\end{pgfscope}%
\begin{pgfscope}%
\pgfpathrectangle{\pgfqpoint{0.800000in}{0.528000in}}{\pgfqpoint{3.968000in}{3.696000in}}%
\pgfusepath{clip}%
\pgfsetbuttcap%
\pgfsetroundjoin%
\definecolor{currentfill}{rgb}{0.281446,0.084320,0.407414}%
\pgfsetfillcolor{currentfill}%
\pgfsetlinewidth{0.000000pt}%
\definecolor{currentstroke}{rgb}{0.000000,0.000000,0.000000}%
\pgfsetstrokecolor{currentstroke}%
\pgfsetdash{}{0pt}%
\pgfpathmoveto{\pgfqpoint{2.990347in}{0.528000in}}%
\pgfpathlineto{\pgfqpoint{2.935117in}{0.575425in}}%
\pgfpathlineto{\pgfqpoint{2.903194in}{0.602667in}}%
\pgfpathlineto{\pgfqpoint{2.844121in}{0.654411in}}%
\pgfpathlineto{\pgfqpoint{2.723879in}{0.762773in}}%
\pgfpathlineto{\pgfqpoint{2.603636in}{0.875190in}}%
\pgfpathlineto{\pgfqpoint{2.499427in}{0.976000in}}%
\pgfpathlineto{\pgfqpoint{2.452934in}{1.022295in}}%
\pgfpathlineto{\pgfqpoint{2.424253in}{1.050667in}}%
\pgfpathlineto{\pgfqpoint{2.323071in}{1.153779in}}%
\pgfpathlineto{\pgfqpoint{2.242909in}{1.237703in}}%
\pgfpathlineto{\pgfqpoint{2.139645in}{1.349333in}}%
\pgfpathlineto{\pgfqpoint{2.095429in}{1.398629in}}%
\pgfpathlineto{\pgfqpoint{2.072275in}{1.424000in}}%
\pgfpathlineto{\pgfqpoint{1.974060in}{1.536000in}}%
\pgfpathlineto{\pgfqpoint{1.879005in}{1.648000in}}%
\pgfpathlineto{\pgfqpoint{1.828446in}{1.709948in}}%
\pgfpathlineto{\pgfqpoint{1.802020in}{1.741911in}}%
\pgfpathlineto{\pgfqpoint{1.721859in}{1.842618in}}%
\pgfpathlineto{\pgfqpoint{1.675212in}{1.903218in}}%
\pgfpathlineto{\pgfqpoint{1.641697in}{1.946817in}}%
\pgfpathlineto{\pgfqpoint{1.601616in}{2.000701in}}%
\pgfpathlineto{\pgfqpoint{1.521455in}{2.111402in}}%
\pgfpathlineto{\pgfqpoint{1.380134in}{2.320000in}}%
\pgfpathlineto{\pgfqpoint{1.321051in}{2.413646in}}%
\pgfpathlineto{\pgfqpoint{1.222930in}{2.581333in}}%
\pgfpathlineto{\pgfqpoint{1.182666in}{2.656000in}}%
\pgfpathlineto{\pgfqpoint{1.175850in}{2.670086in}}%
\pgfpathlineto{\pgfqpoint{1.160727in}{2.698332in}}%
\pgfpathlineto{\pgfqpoint{1.092767in}{2.842667in}}%
\pgfpathlineto{\pgfqpoint{1.061867in}{2.917333in}}%
\pgfpathlineto{\pgfqpoint{1.056641in}{2.932382in}}%
\pgfpathlineto{\pgfqpoint{1.040485in}{2.974302in}}%
\pgfpathlineto{\pgfqpoint{1.034207in}{2.992000in}}%
\pgfpathlineto{\pgfqpoint{1.000314in}{3.104084in}}%
\pgfpathlineto{\pgfqpoint{0.989372in}{3.151609in}}%
\pgfpathlineto{\pgfqpoint{0.978569in}{3.216000in}}%
\pgfpathlineto{\pgfqpoint{0.977432in}{3.231936in}}%
\pgfpathlineto{\pgfqpoint{0.974522in}{3.253333in}}%
\pgfpathlineto{\pgfqpoint{0.972561in}{3.290667in}}%
\pgfpathlineto{\pgfqpoint{0.973258in}{3.302714in}}%
\pgfpathlineto{\pgfqpoint{0.973083in}{3.328000in}}%
\pgfpathlineto{\pgfqpoint{0.976596in}{3.365333in}}%
\pgfpathlineto{\pgfqpoint{0.983754in}{3.402667in}}%
\pgfpathlineto{\pgfqpoint{1.000404in}{3.451384in}}%
\pgfpathlineto{\pgfqpoint{1.014257in}{3.477333in}}%
\pgfpathlineto{\pgfqpoint{1.024092in}{3.492602in}}%
\pgfpathlineto{\pgfqpoint{1.042617in}{3.514667in}}%
\pgfpathlineto{\pgfqpoint{1.061801in}{3.532145in}}%
\pgfpathlineto{\pgfqpoint{1.080566in}{3.545534in}}%
\pgfpathlineto{\pgfqpoint{1.110175in}{3.561754in}}%
\pgfpathlineto{\pgfqpoint{1.120646in}{3.565916in}}%
\pgfpathlineto{\pgfqpoint{1.160727in}{3.578595in}}%
\pgfpathlineto{\pgfqpoint{1.204092in}{3.586275in}}%
\pgfpathlineto{\pgfqpoint{1.241527in}{3.588739in}}%
\pgfpathlineto{\pgfqpoint{1.282315in}{3.588080in}}%
\pgfpathlineto{\pgfqpoint{1.326651in}{3.584117in}}%
\pgfpathlineto{\pgfqpoint{1.361131in}{3.578914in}}%
\pgfpathlineto{\pgfqpoint{1.449083in}{3.559256in}}%
\pgfpathlineto{\pgfqpoint{1.481374in}{3.550851in}}%
\pgfpathlineto{\pgfqpoint{1.601616in}{3.509525in}}%
\pgfpathlineto{\pgfqpoint{1.653184in}{3.488033in}}%
\pgfpathlineto{\pgfqpoint{1.683509in}{3.475720in}}%
\pgfpathlineto{\pgfqpoint{1.761939in}{3.439535in}}%
\pgfpathlineto{\pgfqpoint{1.839605in}{3.400342in}}%
\pgfpathlineto{\pgfqpoint{1.850645in}{3.394708in}}%
\pgfpathlineto{\pgfqpoint{1.922263in}{3.355742in}}%
\pgfpathlineto{\pgfqpoint{1.962343in}{3.333082in}}%
\pgfpathlineto{\pgfqpoint{2.042505in}{3.285684in}}%
\pgfpathlineto{\pgfqpoint{2.082586in}{3.261016in}}%
\pgfpathlineto{\pgfqpoint{2.162747in}{3.210007in}}%
\pgfpathlineto{\pgfqpoint{2.228906in}{3.165623in}}%
\pgfpathlineto{\pgfqpoint{2.265333in}{3.141333in}}%
\pgfpathlineto{\pgfqpoint{2.443313in}{3.014375in}}%
\pgfpathlineto{\pgfqpoint{2.526402in}{2.951941in}}%
\pgfpathlineto{\pgfqpoint{2.643717in}{2.860001in}}%
\pgfpathlineto{\pgfqpoint{2.683798in}{2.827753in}}%
\pgfpathlineto{\pgfqpoint{2.804040in}{2.728428in}}%
\pgfpathlineto{\pgfqpoint{2.888661in}{2.656000in}}%
\pgfpathlineto{\pgfqpoint{2.948029in}{2.603451in}}%
\pgfpathlineto{\pgfqpoint{2.973560in}{2.581333in}}%
\pgfpathlineto{\pgfqpoint{3.084606in}{2.480751in}}%
\pgfpathlineto{\pgfqpoint{3.176736in}{2.394667in}}%
\pgfpathlineto{\pgfqpoint{3.229949in}{2.343380in}}%
\pgfpathlineto{\pgfqpoint{3.254681in}{2.320000in}}%
\pgfpathlineto{\pgfqpoint{3.292992in}{2.282667in}}%
\pgfpathlineto{\pgfqpoint{3.407295in}{2.168764in}}%
\pgfpathlineto{\pgfqpoint{3.525495in}{2.046486in}}%
\pgfpathlineto{\pgfqpoint{3.618963in}{1.946667in}}%
\pgfpathlineto{\pgfqpoint{3.720744in}{1.834667in}}%
\pgfpathlineto{\pgfqpoint{3.765980in}{1.783552in}}%
\pgfpathlineto{\pgfqpoint{3.851173in}{1.685333in}}%
\pgfpathlineto{\pgfqpoint{3.886222in}{1.644054in}}%
\pgfpathlineto{\pgfqpoint{3.975723in}{1.536000in}}%
\pgfpathlineto{\pgfqpoint{4.023024in}{1.476757in}}%
\pgfpathlineto{\pgfqpoint{4.046545in}{1.447832in}}%
\pgfpathlineto{\pgfqpoint{4.126707in}{1.344930in}}%
\pgfpathlineto{\pgfqpoint{4.174231in}{1.281600in}}%
\pgfpathlineto{\pgfqpoint{4.207450in}{1.237333in}}%
\pgfpathlineto{\pgfqpoint{4.261387in}{1.162667in}}%
\pgfpathlineto{\pgfqpoint{4.327111in}{1.068900in}}%
\pgfpathlineto{\pgfqpoint{4.447354in}{0.886523in}}%
\pgfpathlineto{\pgfqpoint{4.507000in}{0.789333in}}%
\pgfpathlineto{\pgfqpoint{4.514026in}{0.776769in}}%
\pgfpathlineto{\pgfqpoint{4.531336in}{0.748441in}}%
\pgfpathlineto{\pgfqpoint{4.592132in}{0.640000in}}%
\pgfpathlineto{\pgfqpoint{4.631389in}{0.565333in}}%
\pgfpathlineto{\pgfqpoint{4.650293in}{0.528000in}}%
\pgfpathlineto{\pgfqpoint{4.661532in}{0.528000in}}%
\pgfpathlineto{\pgfqpoint{4.647758in}{0.555449in}}%
\pgfpathlineto{\pgfqpoint{4.623132in}{0.602667in}}%
\pgfpathlineto{\pgfqpoint{4.607677in}{0.631668in}}%
\pgfpathlineto{\pgfqpoint{4.582450in}{0.677333in}}%
\pgfpathlineto{\pgfqpoint{4.539652in}{0.752000in}}%
\pgfpathlineto{\pgfqpoint{4.487434in}{0.838749in}}%
\pgfpathlineto{\pgfqpoint{4.356123in}{1.040357in}}%
\pgfpathlineto{\pgfqpoint{4.327111in}{1.083325in}}%
\pgfpathlineto{\pgfqpoint{4.271117in}{1.162667in}}%
\pgfpathlineto{\pgfqpoint{4.244357in}{1.200000in}}%
\pgfpathlineto{\pgfqpoint{4.189242in}{1.274667in}}%
\pgfpathlineto{\pgfqpoint{4.146627in}{1.330554in}}%
\pgfpathlineto{\pgfqpoint{4.126707in}{1.357341in}}%
\pgfpathlineto{\pgfqpoint{4.086626in}{1.408977in}}%
\pgfpathlineto{\pgfqpoint{4.006465in}{1.509803in}}%
\pgfpathlineto{\pgfqpoint{3.966384in}{1.558974in}}%
\pgfpathlineto{\pgfqpoint{3.886222in}{1.655211in}}%
\pgfpathlineto{\pgfqpoint{3.836174in}{1.713383in}}%
\pgfpathlineto{\pgfqpoint{3.796113in}{1.760000in}}%
\pgfpathlineto{\pgfqpoint{3.696687in}{1.872000in}}%
\pgfpathlineto{\pgfqpoint{3.605657in}{1.971435in}}%
\pgfpathlineto{\pgfqpoint{3.509182in}{2.073861in}}%
\pgfpathlineto{\pgfqpoint{3.405253in}{2.180693in}}%
\pgfpathlineto{\pgfqpoint{3.285010in}{2.300279in}}%
\pgfpathlineto{\pgfqpoint{3.235191in}{2.348263in}}%
\pgfpathlineto{\pgfqpoint{3.195670in}{2.386117in}}%
\pgfpathlineto{\pgfqpoint{3.164768in}{2.415714in}}%
\pgfpathlineto{\pgfqpoint{3.066866in}{2.506667in}}%
\pgfpathlineto{\pgfqpoint{3.004444in}{2.563327in}}%
\pgfpathlineto{\pgfqpoint{2.884202in}{2.669424in}}%
\pgfpathlineto{\pgfqpoint{2.763960in}{2.771662in}}%
\pgfpathlineto{\pgfqpoint{2.677426in}{2.842667in}}%
\pgfpathlineto{\pgfqpoint{2.630827in}{2.880000in}}%
\pgfpathlineto{\pgfqpoint{2.523475in}{2.963889in}}%
\pgfpathlineto{\pgfqpoint{2.483394in}{2.994430in}}%
\pgfpathlineto{\pgfqpoint{2.385494in}{3.066667in}}%
\pgfpathlineto{\pgfqpoint{2.280349in}{3.141333in}}%
\pgfpathlineto{\pgfqpoint{2.225479in}{3.178667in}}%
\pgfpathlineto{\pgfqpoint{2.146848in}{3.230809in}}%
\pgfpathlineto{\pgfqpoint{2.042505in}{3.296418in}}%
\pgfpathlineto{\pgfqpoint{1.989700in}{3.328000in}}%
\pgfpathlineto{\pgfqpoint{1.922263in}{3.367019in}}%
\pgfpathlineto{\pgfqpoint{1.847371in}{3.407576in}}%
\pgfpathlineto{\pgfqpoint{1.842101in}{3.410598in}}%
\pgfpathlineto{\pgfqpoint{1.795791in}{3.434198in}}%
\pgfpathlineto{\pgfqpoint{1.761939in}{3.451274in}}%
\pgfpathlineto{\pgfqpoint{1.743042in}{3.459732in}}%
\pgfpathlineto{\pgfqpoint{1.706800in}{3.477333in}}%
\pgfpathlineto{\pgfqpoint{1.641697in}{3.506085in}}%
\pgfpathlineto{\pgfqpoint{1.620731in}{3.514667in}}%
\pgfpathlineto{\pgfqpoint{1.561535in}{3.537711in}}%
\pgfpathlineto{\pgfqpoint{1.550075in}{3.541325in}}%
\pgfpathlineto{\pgfqpoint{1.521284in}{3.552158in}}%
\pgfpathlineto{\pgfqpoint{1.481374in}{3.564806in}}%
\pgfpathlineto{\pgfqpoint{1.460760in}{3.570133in}}%
\pgfpathlineto{\pgfqpoint{1.441293in}{3.576358in}}%
\pgfpathlineto{\pgfqpoint{1.387886in}{3.589333in}}%
\pgfpathlineto{\pgfqpoint{1.361131in}{3.594864in}}%
\pgfpathlineto{\pgfqpoint{1.280970in}{3.605413in}}%
\pgfpathlineto{\pgfqpoint{1.200808in}{3.605663in}}%
\pgfpathlineto{\pgfqpoint{1.184725in}{3.604314in}}%
\pgfpathlineto{\pgfqpoint{1.160727in}{3.600493in}}%
\pgfpathlineto{\pgfqpoint{1.116997in}{3.589333in}}%
\pgfpathlineto{\pgfqpoint{1.091462in}{3.579184in}}%
\pgfpathlineto{\pgfqpoint{1.080566in}{3.573460in}}%
\pgfpathlineto{\pgfqpoint{1.040485in}{3.546765in}}%
\pgfpathlineto{\pgfqpoint{1.005849in}{3.509595in}}%
\pgfpathlineto{\pgfqpoint{1.000404in}{3.501142in}}%
\pgfpathlineto{\pgfqpoint{0.986745in}{3.477333in}}%
\pgfpathlineto{\pgfqpoint{0.978635in}{3.460277in}}%
\pgfpathlineto{\pgfqpoint{0.971467in}{3.440000in}}%
\pgfpathlineto{\pgfqpoint{0.960323in}{3.394986in}}%
\pgfpathlineto{\pgfqpoint{0.956265in}{3.361554in}}%
\pgfpathlineto{\pgfqpoint{0.954580in}{3.322650in}}%
\pgfpathlineto{\pgfqpoint{0.955370in}{3.286053in}}%
\pgfpathlineto{\pgfqpoint{0.957538in}{3.253333in}}%
\pgfpathlineto{\pgfqpoint{0.962581in}{3.213897in}}%
\pgfpathlineto{\pgfqpoint{0.970729in}{3.168974in}}%
\pgfpathlineto{\pgfqpoint{0.986319in}{3.104000in}}%
\pgfpathlineto{\pgfqpoint{0.989575in}{3.093913in}}%
\pgfpathlineto{\pgfqpoint{1.000404in}{3.054776in}}%
\pgfpathlineto{\pgfqpoint{1.021218in}{2.992000in}}%
\pgfpathlineto{\pgfqpoint{1.040485in}{2.939747in}}%
\pgfpathlineto{\pgfqpoint{1.080566in}{2.842593in}}%
\pgfpathlineto{\pgfqpoint{1.120646in}{2.756432in}}%
\pgfpathlineto{\pgfqpoint{1.200808in}{2.601684in}}%
\pgfpathlineto{\pgfqpoint{1.240889in}{2.530522in}}%
\pgfpathlineto{\pgfqpoint{1.280970in}{2.462371in}}%
\pgfpathlineto{\pgfqpoint{1.307608in}{2.419479in}}%
\pgfpathlineto{\pgfqpoint{1.324147in}{2.391782in}}%
\pgfpathlineto{\pgfqpoint{1.401212in}{2.272298in}}%
\pgfpathlineto{\pgfqpoint{1.459164in}{2.187313in}}%
\pgfpathlineto{\pgfqpoint{1.481374in}{2.154493in}}%
\pgfpathlineto{\pgfqpoint{1.561535in}{2.042130in}}%
\pgfpathlineto{\pgfqpoint{1.620005in}{1.963795in}}%
\pgfpathlineto{\pgfqpoint{1.641697in}{1.934339in}}%
\pgfpathlineto{\pgfqpoint{1.703389in}{1.854797in}}%
\pgfpathlineto{\pgfqpoint{1.721859in}{1.830423in}}%
\pgfpathlineto{\pgfqpoint{1.808092in}{1.722667in}}%
\pgfpathlineto{\pgfqpoint{1.857897in}{1.662714in}}%
\pgfpathlineto{\pgfqpoint{1.892899in}{1.620649in}}%
\pgfpathlineto{\pgfqpoint{1.932569in}{1.573333in}}%
\pgfpathlineto{\pgfqpoint{2.002424in}{1.492317in}}%
\pgfpathlineto{\pgfqpoint{2.042505in}{1.446811in}}%
\pgfpathlineto{\pgfqpoint{2.130011in}{1.349333in}}%
\pgfpathlineto{\pgfqpoint{2.182369in}{1.292943in}}%
\pgfpathlineto{\pgfqpoint{2.202828in}{1.270261in}}%
\pgfpathlineto{\pgfqpoint{2.323071in}{1.143734in}}%
\pgfpathlineto{\pgfqpoint{2.414316in}{1.050667in}}%
\pgfpathlineto{\pgfqpoint{2.527435in}{0.938667in}}%
\pgfpathlineto{\pgfqpoint{2.566005in}{0.901333in}}%
\pgfpathlineto{\pgfqpoint{2.644516in}{0.826667in}}%
\pgfpathlineto{\pgfqpoint{2.684484in}{0.789333in}}%
\pgfpathlineto{\pgfqpoint{2.765915in}{0.714667in}}%
\pgfpathlineto{\pgfqpoint{2.807409in}{0.677333in}}%
\pgfpathlineto{\pgfqpoint{2.924283in}{0.574699in}}%
\pgfpathlineto{\pgfqpoint{2.971438in}{0.534589in}}%
\pgfpathlineto{\pgfqpoint{2.978950in}{0.528000in}}%
\pgfpathlineto{\pgfqpoint{2.978950in}{0.528000in}}%
\pgfusepath{fill}%
\end{pgfscope}%
\begin{pgfscope}%
\pgfpathrectangle{\pgfqpoint{0.800000in}{0.528000in}}{\pgfqpoint{3.968000in}{3.696000in}}%
\pgfusepath{clip}%
\pgfsetbuttcap%
\pgfsetroundjoin%
\definecolor{currentfill}{rgb}{0.281446,0.084320,0.407414}%
\pgfsetfillcolor{currentfill}%
\pgfsetlinewidth{0.000000pt}%
\definecolor{currentstroke}{rgb}{0.000000,0.000000,0.000000}%
\pgfsetstrokecolor{currentstroke}%
\pgfsetdash{}{0pt}%
\pgfpathmoveto{\pgfqpoint{2.978950in}{0.528000in}}%
\pgfpathlineto{\pgfqpoint{2.909004in}{0.588435in}}%
\pgfpathlineto{\pgfqpoint{2.884202in}{0.609473in}}%
\pgfpathlineto{\pgfqpoint{2.763960in}{0.716432in}}%
\pgfpathlineto{\pgfqpoint{2.683798in}{0.789968in}}%
\pgfpathlineto{\pgfqpoint{2.643717in}{0.827415in}}%
\pgfpathlineto{\pgfqpoint{2.523475in}{0.942512in}}%
\pgfpathlineto{\pgfqpoint{2.403232in}{1.061815in}}%
\pgfpathlineto{\pgfqpoint{2.304842in}{1.162667in}}%
\pgfpathlineto{\pgfqpoint{2.257049in}{1.213171in}}%
\pgfpathlineto{\pgfqpoint{2.233742in}{1.237333in}}%
\pgfpathlineto{\pgfqpoint{2.162747in}{1.313481in}}%
\pgfpathlineto{\pgfqpoint{2.122667in}{1.357398in}}%
\pgfpathlineto{\pgfqpoint{2.029691in}{1.461333in}}%
\pgfpathlineto{\pgfqpoint{1.996884in}{1.498667in}}%
\pgfpathlineto{\pgfqpoint{1.922263in}{1.585438in}}%
\pgfpathlineto{\pgfqpoint{1.838620in}{1.685333in}}%
\pgfpathlineto{\pgfqpoint{1.802020in}{1.730141in}}%
\pgfpathlineto{\pgfqpoint{1.718520in}{1.834667in}}%
\pgfpathlineto{\pgfqpoint{1.669754in}{1.898133in}}%
\pgfpathlineto{\pgfqpoint{1.636321in}{1.941659in}}%
\pgfpathlineto{\pgfqpoint{1.604266in}{1.984000in}}%
\pgfpathlineto{\pgfqpoint{1.549519in}{2.058667in}}%
\pgfpathlineto{\pgfqpoint{1.506505in}{2.119408in}}%
\pgfpathlineto{\pgfqpoint{1.481374in}{2.154493in}}%
\pgfpathlineto{\pgfqpoint{1.361131in}{2.333703in}}%
\pgfpathlineto{\pgfqpoint{1.254780in}{2.506667in}}%
\pgfpathlineto{\pgfqpoint{1.212073in}{2.581333in}}%
\pgfpathlineto{\pgfqpoint{1.168168in}{2.662931in}}%
\pgfpathlineto{\pgfqpoint{1.152078in}{2.693333in}}%
\pgfpathlineto{\pgfqpoint{1.129561in}{2.738970in}}%
\pgfpathlineto{\pgfqpoint{1.115023in}{2.768000in}}%
\pgfpathlineto{\pgfqpoint{1.097614in}{2.805333in}}%
\pgfpathlineto{\pgfqpoint{1.080532in}{2.842667in}}%
\pgfpathlineto{\pgfqpoint{1.031488in}{2.963047in}}%
\pgfpathlineto{\pgfqpoint{1.008440in}{3.029333in}}%
\pgfpathlineto{\pgfqpoint{0.996695in}{3.066667in}}%
\pgfpathlineto{\pgfqpoint{0.970729in}{3.168974in}}%
\pgfpathlineto{\pgfqpoint{0.960323in}{3.231371in}}%
\pgfpathlineto{\pgfqpoint{0.957538in}{3.253333in}}%
\pgfpathlineto{\pgfqpoint{0.954772in}{3.290667in}}%
\pgfpathlineto{\pgfqpoint{0.954207in}{3.328000in}}%
\pgfpathlineto{\pgfqpoint{0.956506in}{3.368889in}}%
\pgfpathlineto{\pgfqpoint{0.962078in}{3.404301in}}%
\pgfpathlineto{\pgfqpoint{0.971467in}{3.440000in}}%
\pgfpathlineto{\pgfqpoint{0.978635in}{3.460277in}}%
\pgfpathlineto{\pgfqpoint{0.986745in}{3.477333in}}%
\pgfpathlineto{\pgfqpoint{1.005849in}{3.509595in}}%
\pgfpathlineto{\pgfqpoint{1.010040in}{3.514667in}}%
\pgfpathlineto{\pgfqpoint{1.043326in}{3.549354in}}%
\pgfpathlineto{\pgfqpoint{1.047034in}{3.552000in}}%
\pgfpathlineto{\pgfqpoint{1.091462in}{3.579184in}}%
\pgfpathlineto{\pgfqpoint{1.122442in}{3.591006in}}%
\pgfpathlineto{\pgfqpoint{1.160727in}{3.600493in}}%
\pgfpathlineto{\pgfqpoint{1.184725in}{3.604314in}}%
\pgfpathlineto{\pgfqpoint{1.200808in}{3.605663in}}%
\pgfpathlineto{\pgfqpoint{1.218366in}{3.605688in}}%
\pgfpathlineto{\pgfqpoint{1.240889in}{3.607075in}}%
\pgfpathlineto{\pgfqpoint{1.258503in}{3.605740in}}%
\pgfpathlineto{\pgfqpoint{1.280970in}{3.605413in}}%
\pgfpathlineto{\pgfqpoint{1.321051in}{3.601205in}}%
\pgfpathlineto{\pgfqpoint{1.331509in}{3.599075in}}%
\pgfpathlineto{\pgfqpoint{1.361131in}{3.594864in}}%
\pgfpathlineto{\pgfqpoint{1.405108in}{3.585704in}}%
\pgfpathlineto{\pgfqpoint{1.441293in}{3.576358in}}%
\pgfpathlineto{\pgfqpoint{1.460760in}{3.570133in}}%
\pgfpathlineto{\pgfqpoint{1.481374in}{3.564806in}}%
\pgfpathlineto{\pgfqpoint{1.521733in}{3.552000in}}%
\pgfpathlineto{\pgfqpoint{1.641697in}{3.506085in}}%
\pgfpathlineto{\pgfqpoint{1.681778in}{3.488699in}}%
\pgfpathlineto{\pgfqpoint{1.761939in}{3.451274in}}%
\pgfpathlineto{\pgfqpoint{1.795791in}{3.434198in}}%
\pgfpathlineto{\pgfqpoint{1.802020in}{3.431334in}}%
\pgfpathlineto{\pgfqpoint{1.847371in}{3.407576in}}%
\pgfpathlineto{\pgfqpoint{1.882182in}{3.389062in}}%
\pgfpathlineto{\pgfqpoint{1.897825in}{3.379904in}}%
\pgfpathlineto{\pgfqpoint{1.925204in}{3.365333in}}%
\pgfpathlineto{\pgfqpoint{1.997390in}{3.323311in}}%
\pgfpathlineto{\pgfqpoint{2.002424in}{3.320541in}}%
\pgfpathlineto{\pgfqpoint{2.042505in}{3.296418in}}%
\pgfpathlineto{\pgfqpoint{2.146848in}{3.230809in}}%
\pgfpathlineto{\pgfqpoint{2.225479in}{3.178667in}}%
\pgfpathlineto{\pgfqpoint{2.403232in}{3.053810in}}%
\pgfpathlineto{\pgfqpoint{2.443313in}{3.024378in}}%
\pgfpathlineto{\pgfqpoint{2.535435in}{2.954667in}}%
\pgfpathlineto{\pgfqpoint{2.643717in}{2.869757in}}%
\pgfpathlineto{\pgfqpoint{2.683798in}{2.837533in}}%
\pgfpathlineto{\pgfqpoint{2.768330in}{2.768000in}}%
\pgfpathlineto{\pgfqpoint{2.884202in}{2.669424in}}%
\pgfpathlineto{\pgfqpoint{3.004444in}{2.563327in}}%
\pgfpathlineto{\pgfqpoint{3.055404in}{2.516800in}}%
\pgfpathlineto{\pgfqpoint{3.084606in}{2.490410in}}%
\pgfpathlineto{\pgfqpoint{3.186978in}{2.394667in}}%
\pgfpathlineto{\pgfqpoint{3.244929in}{2.339214in}}%
\pgfpathlineto{\pgfqpoint{3.365172in}{2.221022in}}%
\pgfpathlineto{\pgfqpoint{3.451717in}{2.133333in}}%
\pgfpathlineto{\pgfqpoint{3.559010in}{2.021333in}}%
\pgfpathlineto{\pgfqpoint{3.617750in}{1.957931in}}%
\pgfpathlineto{\pgfqpoint{3.662777in}{1.909333in}}%
\pgfpathlineto{\pgfqpoint{3.696687in}{1.872000in}}%
\pgfpathlineto{\pgfqpoint{3.796113in}{1.760000in}}%
\pgfpathlineto{\pgfqpoint{3.836174in}{1.713383in}}%
\pgfpathlineto{\pgfqpoint{3.860544in}{1.685333in}}%
\pgfpathlineto{\pgfqpoint{3.892316in}{1.648000in}}%
\pgfpathlineto{\pgfqpoint{3.966384in}{1.558974in}}%
\pgfpathlineto{\pgfqpoint{4.046545in}{1.459966in}}%
\pgfpathlineto{\pgfqpoint{4.132861in}{1.349333in}}%
\pgfpathlineto{\pgfqpoint{4.179785in}{1.286773in}}%
\pgfpathlineto{\pgfqpoint{4.206869in}{1.251002in}}%
\pgfpathlineto{\pgfqpoint{4.271117in}{1.162667in}}%
\pgfpathlineto{\pgfqpoint{4.327111in}{1.083325in}}%
\pgfpathlineto{\pgfqpoint{4.371792in}{1.017618in}}%
\pgfpathlineto{\pgfqpoint{4.386932in}{0.994387in}}%
\pgfpathlineto{\pgfqpoint{4.407273in}{0.964508in}}%
\pgfpathlineto{\pgfqpoint{4.527515in}{0.772552in}}%
\pgfpathlineto{\pgfqpoint{4.567596in}{0.703741in}}%
\pgfpathlineto{\pgfqpoint{4.607677in}{0.631668in}}%
\pgfpathlineto{\pgfqpoint{4.661532in}{0.528000in}}%
\pgfpathlineto{\pgfqpoint{4.672771in}{0.528000in}}%
\pgfpathlineto{\pgfqpoint{4.651729in}{0.569033in}}%
\pgfpathlineto{\pgfqpoint{4.634080in}{0.602667in}}%
\pgfpathlineto{\pgfqpoint{4.613958in}{0.640000in}}%
\pgfpathlineto{\pgfqpoint{4.567596in}{0.722176in}}%
\pgfpathlineto{\pgfqpoint{4.505051in}{0.826667in}}%
\pgfpathlineto{\pgfqpoint{4.447354in}{0.918253in}}%
\pgfpathlineto{\pgfqpoint{4.307310in}{1.125333in}}%
\pgfpathlineto{\pgfqpoint{4.266900in}{1.181250in}}%
\pgfpathlineto{\pgfqpoint{4.246949in}{1.209510in}}%
\pgfpathlineto{\pgfqpoint{4.166788in}{1.317297in}}%
\pgfpathlineto{\pgfqpoint{4.118860in}{1.379357in}}%
\pgfpathlineto{\pgfqpoint{4.084390in}{1.424000in}}%
\pgfpathlineto{\pgfqpoint{4.006465in}{1.521371in}}%
\pgfpathlineto{\pgfqpoint{3.956305in}{1.582721in}}%
\pgfpathlineto{\pgfqpoint{3.869914in}{1.685333in}}%
\pgfpathlineto{\pgfqpoint{3.837931in}{1.722667in}}%
\pgfpathlineto{\pgfqpoint{3.765980in}{1.805040in}}%
\pgfpathlineto{\pgfqpoint{3.672265in}{1.909333in}}%
\pgfpathlineto{\pgfqpoint{3.565576in}{2.024577in}}%
\pgfpathlineto{\pgfqpoint{3.445333in}{2.149767in}}%
\pgfpathlineto{\pgfqpoint{3.350619in}{2.245333in}}%
\pgfpathlineto{\pgfqpoint{3.312913in}{2.282667in}}%
\pgfpathlineto{\pgfqpoint{3.204848in}{2.387421in}}%
\pgfpathlineto{\pgfqpoint{3.157758in}{2.432000in}}%
\pgfpathlineto{\pgfqpoint{3.044525in}{2.536725in}}%
\pgfpathlineto{\pgfqpoint{2.953098in}{2.618667in}}%
\pgfpathlineto{\pgfqpoint{2.910594in}{2.656000in}}%
\pgfpathlineto{\pgfqpoint{2.804040in}{2.747505in}}%
\pgfpathlineto{\pgfqpoint{2.683798in}{2.847147in}}%
\pgfpathlineto{\pgfqpoint{2.643103in}{2.880000in}}%
\pgfpathlineto{\pgfqpoint{2.523475in}{2.973573in}}%
\pgfpathlineto{\pgfqpoint{2.483394in}{3.004091in}}%
\pgfpathlineto{\pgfqpoint{2.399259in}{3.066667in}}%
\pgfpathlineto{\pgfqpoint{2.282990in}{3.149495in}}%
\pgfpathlineto{\pgfqpoint{2.240706in}{3.178667in}}%
\pgfpathlineto{\pgfqpoint{2.122667in}{3.256769in}}%
\pgfpathlineto{\pgfqpoint{2.042505in}{3.306953in}}%
\pgfpathlineto{\pgfqpoint{2.004550in}{3.329980in}}%
\pgfpathlineto{\pgfqpoint{1.992410in}{3.337327in}}%
\pgfpathlineto{\pgfqpoint{1.922263in}{3.377901in}}%
\pgfpathlineto{\pgfqpoint{1.905285in}{3.386853in}}%
\pgfpathlineto{\pgfqpoint{1.877929in}{3.402667in}}%
\pgfpathlineto{\pgfqpoint{1.710655in}{3.487769in}}%
\pgfpathlineto{\pgfqpoint{1.633789in}{3.522033in}}%
\pgfpathlineto{\pgfqpoint{1.558750in}{3.552000in}}%
\pgfpathlineto{\pgfqpoint{1.481374in}{3.578700in}}%
\pgfpathlineto{\pgfqpoint{1.472437in}{3.581010in}}%
\pgfpathlineto{\pgfqpoint{1.441293in}{3.590883in}}%
\pgfpathlineto{\pgfqpoint{1.361131in}{3.610210in}}%
\pgfpathlineto{\pgfqpoint{1.345803in}{3.612389in}}%
\pgfpathlineto{\pgfqpoint{1.321051in}{3.617429in}}%
\pgfpathlineto{\pgfqpoint{1.276695in}{3.622685in}}%
\pgfpathlineto{\pgfqpoint{1.239430in}{3.625308in}}%
\pgfpathlineto{\pgfqpoint{1.200808in}{3.625252in}}%
\pgfpathlineto{\pgfqpoint{1.160727in}{3.621537in}}%
\pgfpathlineto{\pgfqpoint{1.153223in}{3.619677in}}%
\pgfpathlineto{\pgfqpoint{1.120646in}{3.613391in}}%
\pgfpathlineto{\pgfqpoint{1.100928in}{3.607700in}}%
\pgfpathlineto{\pgfqpoint{1.080566in}{3.599657in}}%
\pgfpathlineto{\pgfqpoint{1.059877in}{3.589333in}}%
\pgfpathlineto{\pgfqpoint{1.040485in}{3.577524in}}%
\pgfpathlineto{\pgfqpoint{1.025177in}{3.566259in}}%
\pgfpathlineto{\pgfqpoint{1.000404in}{3.541809in}}%
\pgfpathlineto{\pgfqpoint{0.987869in}{3.526342in}}%
\pgfpathlineto{\pgfqpoint{0.980462in}{3.514667in}}%
\pgfpathlineto{\pgfqpoint{0.960323in}{3.476226in}}%
\pgfpathlineto{\pgfqpoint{0.948689in}{3.440000in}}%
\pgfpathlineto{\pgfqpoint{0.943511in}{3.418327in}}%
\pgfpathlineto{\pgfqpoint{0.938635in}{3.385535in}}%
\pgfpathlineto{\pgfqpoint{0.937223in}{3.365333in}}%
\pgfpathlineto{\pgfqpoint{0.937854in}{3.290667in}}%
\pgfpathlineto{\pgfqpoint{0.940275in}{3.271992in}}%
\pgfpathlineto{\pgfqpoint{0.941511in}{3.253333in}}%
\pgfpathlineto{\pgfqpoint{0.946950in}{3.216000in}}%
\pgfpathlineto{\pgfqpoint{0.949324in}{3.205755in}}%
\pgfpathlineto{\pgfqpoint{0.955225in}{3.173918in}}%
\pgfpathlineto{\pgfqpoint{0.962864in}{3.138966in}}%
\pgfpathlineto{\pgfqpoint{0.983294in}{3.066667in}}%
\pgfpathlineto{\pgfqpoint{0.987584in}{3.054725in}}%
\pgfpathlineto{\pgfqpoint{1.000404in}{3.014265in}}%
\pgfpathlineto{\pgfqpoint{1.036856in}{2.917333in}}%
\pgfpathlineto{\pgfqpoint{1.068895in}{2.842667in}}%
\pgfpathlineto{\pgfqpoint{1.103697in}{2.768000in}}%
\pgfpathlineto{\pgfqpoint{1.141047in}{2.693333in}}%
\pgfpathlineto{\pgfqpoint{1.180751in}{2.618667in}}%
\pgfpathlineto{\pgfqpoint{1.187833in}{2.606581in}}%
\pgfpathlineto{\pgfqpoint{1.201215in}{2.581333in}}%
\pgfpathlineto{\pgfqpoint{1.229273in}{2.533180in}}%
\pgfpathlineto{\pgfqpoint{1.244195in}{2.506667in}}%
\pgfpathlineto{\pgfqpoint{1.271926in}{2.460910in}}%
\pgfpathlineto{\pgfqpoint{1.289160in}{2.432000in}}%
\pgfpathlineto{\pgfqpoint{1.315656in}{2.389642in}}%
\pgfpathlineto{\pgfqpoint{1.335968in}{2.357333in}}%
\pgfpathlineto{\pgfqpoint{1.375888in}{2.296412in}}%
\pgfpathlineto{\pgfqpoint{1.401212in}{2.257450in}}%
\pgfpathlineto{\pgfqpoint{1.539946in}{2.058667in}}%
\pgfpathlineto{\pgfqpoint{1.601616in}{1.974892in}}%
\pgfpathlineto{\pgfqpoint{1.647395in}{1.914641in}}%
\pgfpathlineto{\pgfqpoint{1.664295in}{1.893049in}}%
\pgfpathlineto{\pgfqpoint{1.681778in}{1.869627in}}%
\pgfpathlineto{\pgfqpoint{1.721859in}{1.818584in}}%
\pgfpathlineto{\pgfqpoint{1.802020in}{1.718555in}}%
\pgfpathlineto{\pgfqpoint{1.852537in}{1.657721in}}%
\pgfpathlineto{\pgfqpoint{1.891538in}{1.610667in}}%
\pgfpathlineto{\pgfqpoint{1.962343in}{1.527669in}}%
\pgfpathlineto{\pgfqpoint{2.053343in}{1.424000in}}%
\pgfpathlineto{\pgfqpoint{2.122667in}{1.346915in}}%
\pgfpathlineto{\pgfqpoint{2.177232in}{1.288158in}}%
\pgfpathlineto{\pgfqpoint{2.202828in}{1.260140in}}%
\pgfpathlineto{\pgfqpoint{2.295171in}{1.162667in}}%
\pgfpathlineto{\pgfqpoint{2.404379in}{1.050667in}}%
\pgfpathlineto{\pgfqpoint{2.517476in}{0.938667in}}%
\pgfpathlineto{\pgfqpoint{2.599335in}{0.859993in}}%
\pgfpathlineto{\pgfqpoint{2.634414in}{0.826667in}}%
\pgfpathlineto{\pgfqpoint{2.723879in}{0.743491in}}%
\pgfpathlineto{\pgfqpoint{2.844121in}{0.635136in}}%
\pgfpathlineto{\pgfqpoint{2.965911in}{0.529441in}}%
\pgfpathlineto{\pgfqpoint{2.967553in}{0.528000in}}%
\pgfpathlineto{\pgfqpoint{2.967553in}{0.528000in}}%
\pgfusepath{fill}%
\end{pgfscope}%
\begin{pgfscope}%
\pgfpathrectangle{\pgfqpoint{0.800000in}{0.528000in}}{\pgfqpoint{3.968000in}{3.696000in}}%
\pgfusepath{clip}%
\pgfsetbuttcap%
\pgfsetroundjoin%
\definecolor{currentfill}{rgb}{0.281446,0.084320,0.407414}%
\pgfsetfillcolor{currentfill}%
\pgfsetlinewidth{0.000000pt}%
\definecolor{currentstroke}{rgb}{0.000000,0.000000,0.000000}%
\pgfsetstrokecolor{currentstroke}%
\pgfsetdash{}{0pt}%
\pgfpathmoveto{\pgfqpoint{2.967553in}{0.528000in}}%
\pgfpathlineto{\pgfqpoint{2.755474in}{0.714667in}}%
\pgfpathlineto{\pgfqpoint{2.643717in}{0.817895in}}%
\pgfpathlineto{\pgfqpoint{2.556018in}{0.901333in}}%
\pgfpathlineto{\pgfqpoint{2.517476in}{0.938667in}}%
\pgfpathlineto{\pgfqpoint{2.403232in}{1.051820in}}%
\pgfpathlineto{\pgfqpoint{2.282990in}{1.175349in}}%
\pgfpathlineto{\pgfqpoint{2.233124in}{1.228219in}}%
\pgfpathlineto{\pgfqpoint{2.189308in}{1.274667in}}%
\pgfpathlineto{\pgfqpoint{2.140236in}{1.328365in}}%
\pgfpathlineto{\pgfqpoint{2.108406in}{1.362617in}}%
\pgfpathlineto{\pgfqpoint{2.020306in}{1.461333in}}%
\pgfpathlineto{\pgfqpoint{1.987580in}{1.498667in}}%
\pgfpathlineto{\pgfqpoint{1.891538in}{1.610667in}}%
\pgfpathlineto{\pgfqpoint{1.852537in}{1.657721in}}%
\pgfpathlineto{\pgfqpoint{1.829350in}{1.685333in}}%
\pgfpathlineto{\pgfqpoint{1.798666in}{1.722667in}}%
\pgfpathlineto{\pgfqpoint{1.721859in}{1.818584in}}%
\pgfpathlineto{\pgfqpoint{1.641697in}{1.921868in}}%
\pgfpathlineto{\pgfqpoint{1.594801in}{1.984000in}}%
\pgfpathlineto{\pgfqpoint{1.539946in}{2.058667in}}%
\pgfpathlineto{\pgfqpoint{1.500702in}{2.114003in}}%
\pgfpathlineto{\pgfqpoint{1.481374in}{2.140532in}}%
\pgfpathlineto{\pgfqpoint{1.422012in}{2.227374in}}%
\pgfpathlineto{\pgfqpoint{1.401212in}{2.257450in}}%
\pgfpathlineto{\pgfqpoint{1.280970in}{2.445433in}}%
\pgfpathlineto{\pgfqpoint{1.180751in}{2.618667in}}%
\pgfpathlineto{\pgfqpoint{1.141047in}{2.693333in}}%
\pgfpathlineto{\pgfqpoint{1.134779in}{2.706497in}}%
\pgfpathlineto{\pgfqpoint{1.120646in}{2.733135in}}%
\pgfpathlineto{\pgfqpoint{1.080566in}{2.816882in}}%
\pgfpathlineto{\pgfqpoint{1.052567in}{2.880000in}}%
\pgfpathlineto{\pgfqpoint{1.036856in}{2.917333in}}%
\pgfpathlineto{\pgfqpoint{0.995171in}{3.029333in}}%
\pgfpathlineto{\pgfqpoint{0.962318in}{3.141333in}}%
\pgfpathlineto{\pgfqpoint{0.953916in}{3.178667in}}%
\pgfpathlineto{\pgfqpoint{0.941511in}{3.253333in}}%
\pgfpathlineto{\pgfqpoint{0.940275in}{3.271992in}}%
\pgfpathlineto{\pgfqpoint{0.937854in}{3.290667in}}%
\pgfpathlineto{\pgfqpoint{0.937756in}{3.306979in}}%
\pgfpathlineto{\pgfqpoint{0.936293in}{3.328000in}}%
\pgfpathlineto{\pgfqpoint{0.937266in}{3.343856in}}%
\pgfpathlineto{\pgfqpoint{0.937223in}{3.365333in}}%
\pgfpathlineto{\pgfqpoint{0.938635in}{3.385535in}}%
\pgfpathlineto{\pgfqpoint{0.943511in}{3.418327in}}%
\pgfpathlineto{\pgfqpoint{0.948689in}{3.440000in}}%
\pgfpathlineto{\pgfqpoint{0.961236in}{3.478183in}}%
\pgfpathlineto{\pgfqpoint{0.987869in}{3.526342in}}%
\pgfpathlineto{\pgfqpoint{1.009918in}{3.552000in}}%
\pgfpathlineto{\pgfqpoint{1.025177in}{3.566259in}}%
\pgfpathlineto{\pgfqpoint{1.040485in}{3.577524in}}%
\pgfpathlineto{\pgfqpoint{1.059877in}{3.589333in}}%
\pgfpathlineto{\pgfqpoint{1.080566in}{3.599657in}}%
\pgfpathlineto{\pgfqpoint{1.100928in}{3.607700in}}%
\pgfpathlineto{\pgfqpoint{1.120646in}{3.613391in}}%
\pgfpathlineto{\pgfqpoint{1.165431in}{3.622286in}}%
\pgfpathlineto{\pgfqpoint{1.200808in}{3.625252in}}%
\pgfpathlineto{\pgfqpoint{1.240889in}{3.625397in}}%
\pgfpathlineto{\pgfqpoint{1.285759in}{3.622206in}}%
\pgfpathlineto{\pgfqpoint{1.321051in}{3.617429in}}%
\pgfpathlineto{\pgfqpoint{1.410871in}{3.598330in}}%
\pgfpathlineto{\pgfqpoint{1.446407in}{3.589333in}}%
\pgfpathlineto{\pgfqpoint{1.521455in}{3.565345in}}%
\pgfpathlineto{\pgfqpoint{1.563320in}{3.550338in}}%
\pgfpathlineto{\pgfqpoint{1.650668in}{3.514667in}}%
\pgfpathlineto{\pgfqpoint{1.761939in}{3.462992in}}%
\pgfpathlineto{\pgfqpoint{1.777614in}{3.454600in}}%
\pgfpathlineto{\pgfqpoint{1.807656in}{3.440000in}}%
\pgfpathlineto{\pgfqpoint{1.962343in}{3.354906in}}%
\pgfpathlineto{\pgfqpoint{2.042505in}{3.306953in}}%
\pgfpathlineto{\pgfqpoint{2.127952in}{3.253333in}}%
\pgfpathlineto{\pgfqpoint{2.242909in}{3.177182in}}%
\pgfpathlineto{\pgfqpoint{2.310589in}{3.129707in}}%
\pgfpathlineto{\pgfqpoint{2.347424in}{3.104000in}}%
\pgfpathlineto{\pgfqpoint{2.547995in}{2.954667in}}%
\pgfpathlineto{\pgfqpoint{2.643717in}{2.879512in}}%
\pgfpathlineto{\pgfqpoint{2.689284in}{2.842667in}}%
\pgfpathlineto{\pgfqpoint{2.804040in}{2.747505in}}%
\pgfpathlineto{\pgfqpoint{2.855161in}{2.703616in}}%
\pgfpathlineto{\pgfqpoint{2.884202in}{2.678968in}}%
\pgfpathlineto{\pgfqpoint{3.004444in}{2.572940in}}%
\pgfpathlineto{\pgfqpoint{3.060537in}{2.521581in}}%
\pgfpathlineto{\pgfqpoint{3.084606in}{2.500069in}}%
\pgfpathlineto{\pgfqpoint{3.204848in}{2.387421in}}%
\pgfpathlineto{\pgfqpoint{3.325091in}{2.270684in}}%
\pgfpathlineto{\pgfqpoint{3.365172in}{2.230847in}}%
\pgfpathlineto{\pgfqpoint{3.485414in}{2.108516in}}%
\pgfpathlineto{\pgfqpoint{3.533222in}{2.058667in}}%
\pgfpathlineto{\pgfqpoint{3.605657in}{1.981798in}}%
\pgfpathlineto{\pgfqpoint{3.645737in}{1.938382in}}%
\pgfpathlineto{\pgfqpoint{3.739596in}{1.834667in}}%
\pgfpathlineto{\pgfqpoint{3.837931in}{1.722667in}}%
\pgfpathlineto{\pgfqpoint{3.876973in}{1.676718in}}%
\pgfpathlineto{\pgfqpoint{3.901607in}{1.648000in}}%
\pgfpathlineto{\pgfqpoint{3.994557in}{1.536000in}}%
\pgfpathlineto{\pgfqpoint{4.033887in}{1.486876in}}%
\pgfpathlineto{\pgfqpoint{4.078254in}{1.431799in}}%
\pgfpathlineto{\pgfqpoint{4.142243in}{1.349333in}}%
\pgfpathlineto{\pgfqpoint{4.185339in}{1.291946in}}%
\pgfpathlineto{\pgfqpoint{4.206869in}{1.263884in}}%
\pgfpathlineto{\pgfqpoint{4.287030in}{1.154003in}}%
\pgfpathlineto{\pgfqpoint{4.346403in}{1.068636in}}%
\pgfpathlineto{\pgfqpoint{4.367192in}{1.039123in}}%
\pgfpathlineto{\pgfqpoint{4.505051in}{0.826667in}}%
\pgfpathlineto{\pgfqpoint{4.512921in}{0.813073in}}%
\pgfpathlineto{\pgfqpoint{4.528623in}{0.788302in}}%
\pgfpathlineto{\pgfqpoint{4.593121in}{0.677333in}}%
\pgfpathlineto{\pgfqpoint{4.607677in}{0.651331in}}%
\pgfpathlineto{\pgfqpoint{4.647758in}{0.576870in}}%
\pgfpathlineto{\pgfqpoint{4.672771in}{0.528000in}}%
\pgfpathlineto{\pgfqpoint{4.684010in}{0.528000in}}%
\pgfpathlineto{\pgfqpoint{4.658839in}{0.575655in}}%
\pgfpathlineto{\pgfqpoint{4.645028in}{0.602667in}}%
\pgfpathlineto{\pgfqpoint{4.624526in}{0.640000in}}%
\pgfpathlineto{\pgfqpoint{4.582279in}{0.714667in}}%
\pgfpathlineto{\pgfqpoint{4.560469in}{0.752000in}}%
\pgfpathlineto{\pgfqpoint{4.515210in}{0.826667in}}%
\pgfpathlineto{\pgfqpoint{4.468149in}{0.901333in}}%
\pgfpathlineto{\pgfqpoint{4.429750in}{0.959603in}}%
\pgfpathlineto{\pgfqpoint{4.407273in}{0.994198in}}%
\pgfpathlineto{\pgfqpoint{4.263317in}{1.200000in}}%
\pgfpathlineto{\pgfqpoint{4.235983in}{1.237333in}}%
\pgfpathlineto{\pgfqpoint{4.166788in}{1.329542in}}%
\pgfpathlineto{\pgfqpoint{4.122902in}{1.386667in}}%
\pgfpathlineto{\pgfqpoint{4.046545in}{1.483230in}}%
\pgfpathlineto{\pgfqpoint{3.996067in}{1.545685in}}%
\pgfpathlineto{\pgfqpoint{3.910897in}{1.648000in}}%
\pgfpathlineto{\pgfqpoint{3.864282in}{1.702231in}}%
\pgfpathlineto{\pgfqpoint{3.846141in}{1.724050in}}%
\pgfpathlineto{\pgfqpoint{3.748921in}{1.834667in}}%
\pgfpathlineto{\pgfqpoint{3.715498in}{1.872000in}}%
\pgfpathlineto{\pgfqpoint{3.645737in}{1.948692in}}%
\pgfpathlineto{\pgfqpoint{3.605657in}{1.991853in}}%
\pgfpathlineto{\pgfqpoint{3.485414in}{2.118415in}}%
\pgfpathlineto{\pgfqpoint{3.397697in}{2.208000in}}%
\pgfpathlineto{\pgfqpoint{3.280708in}{2.324008in}}%
\pgfpathlineto{\pgfqpoint{3.164768in}{2.435015in}}%
\pgfpathlineto{\pgfqpoint{3.044525in}{2.546277in}}%
\pgfpathlineto{\pgfqpoint{2.984294in}{2.599897in}}%
\pgfpathlineto{\pgfqpoint{2.943240in}{2.636324in}}%
\pgfpathlineto{\pgfqpoint{2.921560in}{2.656000in}}%
\pgfpathlineto{\pgfqpoint{2.804040in}{2.757003in}}%
\pgfpathlineto{\pgfqpoint{2.683798in}{2.856578in}}%
\pgfpathlineto{\pgfqpoint{2.643717in}{2.888938in}}%
\pgfpathlineto{\pgfqpoint{2.560555in}{2.954667in}}%
\pgfpathlineto{\pgfqpoint{2.496015in}{3.003756in}}%
\pgfpathlineto{\pgfqpoint{2.462658in}{3.029333in}}%
\pgfpathlineto{\pgfqpoint{2.361366in}{3.104000in}}%
\pgfpathlineto{\pgfqpoint{2.242909in}{3.187116in}}%
\pgfpathlineto{\pgfqpoint{2.162747in}{3.240866in}}%
\pgfpathlineto{\pgfqpoint{2.076092in}{3.296716in}}%
\pgfpathlineto{\pgfqpoint{1.962343in}{3.365799in}}%
\pgfpathlineto{\pgfqpoint{1.897604in}{3.402667in}}%
\pgfpathlineto{\pgfqpoint{1.785603in}{3.462042in}}%
\pgfpathlineto{\pgfqpoint{1.756544in}{3.477333in}}%
\pgfpathlineto{\pgfqpoint{1.678178in}{3.514667in}}%
\pgfpathlineto{\pgfqpoint{1.591154in}{3.552000in}}%
\pgfpathlineto{\pgfqpoint{1.481374in}{3.592432in}}%
\pgfpathlineto{\pgfqpoint{1.441293in}{3.604730in}}%
\pgfpathlineto{\pgfqpoint{1.422643in}{3.609295in}}%
\pgfpathlineto{\pgfqpoint{1.401212in}{3.615836in}}%
\pgfpathlineto{\pgfqpoint{1.355366in}{3.626667in}}%
\pgfpathlineto{\pgfqpoint{1.312315in}{3.634803in}}%
\pgfpathlineto{\pgfqpoint{1.280970in}{3.639028in}}%
\pgfpathlineto{\pgfqpoint{1.200808in}{3.643589in}}%
\pgfpathlineto{\pgfqpoint{1.182987in}{3.643266in}}%
\pgfpathlineto{\pgfqpoint{1.160727in}{3.641408in}}%
\pgfpathlineto{\pgfqpoint{1.112929in}{3.633855in}}%
\pgfpathlineto{\pgfqpoint{1.080566in}{3.624371in}}%
\pgfpathlineto{\pgfqpoint{1.075258in}{3.621723in}}%
\pgfpathlineto{\pgfqpoint{1.040485in}{3.605808in}}%
\pgfpathlineto{\pgfqpoint{1.029524in}{3.599543in}}%
\pgfpathlineto{\pgfqpoint{1.007087in}{3.583108in}}%
\pgfpathlineto{\pgfqpoint{1.000404in}{3.576863in}}%
\pgfpathlineto{\pgfqpoint{0.970002in}{3.542985in}}%
\pgfpathlineto{\pgfqpoint{0.952922in}{3.514667in}}%
\pgfpathlineto{\pgfqpoint{0.942961in}{3.493505in}}%
\pgfpathlineto{\pgfqpoint{0.937273in}{3.477333in}}%
\pgfpathlineto{\pgfqpoint{0.925714in}{3.434904in}}%
\pgfpathlineto{\pgfqpoint{0.920242in}{3.394716in}}%
\pgfpathlineto{\pgfqpoint{0.918447in}{3.363661in}}%
\pgfpathlineto{\pgfqpoint{0.918510in}{3.328000in}}%
\pgfpathlineto{\pgfqpoint{0.920936in}{3.290667in}}%
\pgfpathlineto{\pgfqpoint{0.931725in}{3.216000in}}%
\pgfpathlineto{\pgfqpoint{0.958395in}{3.104000in}}%
\pgfpathlineto{\pgfqpoint{0.969893in}{3.066667in}}%
\pgfpathlineto{\pgfqpoint{0.987173in}{3.017010in}}%
\pgfpathlineto{\pgfqpoint{1.000404in}{2.979108in}}%
\pgfpathlineto{\pgfqpoint{1.024889in}{2.917333in}}%
\pgfpathlineto{\pgfqpoint{1.040485in}{2.879960in}}%
\pgfpathlineto{\pgfqpoint{1.110884in}{2.730667in}}%
\pgfpathlineto{\pgfqpoint{1.153590in}{2.649352in}}%
\pgfpathlineto{\pgfqpoint{1.170001in}{2.618667in}}%
\pgfpathlineto{\pgfqpoint{1.190815in}{2.581333in}}%
\pgfpathlineto{\pgfqpoint{1.240889in}{2.494969in}}%
\pgfpathlineto{\pgfqpoint{1.361131in}{2.303124in}}%
\pgfpathlineto{\pgfqpoint{1.503492in}{2.096000in}}%
\pgfpathlineto{\pgfqpoint{1.561535in}{2.016046in}}%
\pgfpathlineto{\pgfqpoint{1.641745in}{1.909333in}}%
\pgfpathlineto{\pgfqpoint{1.692431in}{1.844589in}}%
\pgfpathlineto{\pgfqpoint{1.729328in}{1.797333in}}%
\pgfpathlineto{\pgfqpoint{1.802020in}{1.707286in}}%
\pgfpathlineto{\pgfqpoint{1.850955in}{1.648000in}}%
\pgfpathlineto{\pgfqpoint{1.922263in}{1.563505in}}%
\pgfpathlineto{\pgfqpoint{2.010922in}{1.461333in}}%
\pgfpathlineto{\pgfqpoint{2.050631in}{1.416431in}}%
\pgfpathlineto{\pgfqpoint{2.162747in}{1.293131in}}%
\pgfpathlineto{\pgfqpoint{2.228144in}{1.223580in}}%
\pgfpathlineto{\pgfqpoint{2.265759in}{1.183951in}}%
\pgfpathlineto{\pgfqpoint{2.307311in}{1.140013in}}%
\pgfpathlineto{\pgfqpoint{2.403232in}{1.042148in}}%
\pgfpathlineto{\pgfqpoint{2.523475in}{0.923250in}}%
\pgfpathlineto{\pgfqpoint{2.563556in}{0.884521in}}%
\pgfpathlineto{\pgfqpoint{2.683798in}{0.771003in}}%
\pgfpathlineto{\pgfqpoint{2.804040in}{0.661421in}}%
\pgfpathlineto{\pgfqpoint{2.924283in}{0.555692in}}%
\pgfpathlineto{\pgfqpoint{2.956532in}{0.528000in}}%
\pgfpathlineto{\pgfqpoint{2.964364in}{0.528000in}}%
\pgfpathlineto{\pgfqpoint{2.964364in}{0.528000in}}%
\pgfusepath{fill}%
\end{pgfscope}%
\begin{pgfscope}%
\pgfpathrectangle{\pgfqpoint{0.800000in}{0.528000in}}{\pgfqpoint{3.968000in}{3.696000in}}%
\pgfusepath{clip}%
\pgfsetbuttcap%
\pgfsetroundjoin%
\definecolor{currentfill}{rgb}{0.281924,0.089666,0.412415}%
\pgfsetfillcolor{currentfill}%
\pgfsetlinewidth{0.000000pt}%
\definecolor{currentstroke}{rgb}{0.000000,0.000000,0.000000}%
\pgfsetstrokecolor{currentstroke}%
\pgfsetdash{}{0pt}%
\pgfpathmoveto{\pgfqpoint{2.956532in}{0.528000in}}%
\pgfpathlineto{\pgfqpoint{2.844121in}{0.625754in}}%
\pgfpathlineto{\pgfqpoint{2.795494in}{0.669372in}}%
\pgfpathlineto{\pgfqpoint{2.745117in}{0.714667in}}%
\pgfpathlineto{\pgfqpoint{2.643717in}{0.808401in}}%
\pgfpathlineto{\pgfqpoint{2.546133in}{0.901333in}}%
\pgfpathlineto{\pgfqpoint{2.507680in}{0.938667in}}%
\pgfpathlineto{\pgfqpoint{2.394795in}{1.050667in}}%
\pgfpathlineto{\pgfqpoint{2.282990in}{1.165279in}}%
\pgfpathlineto{\pgfqpoint{2.228144in}{1.223580in}}%
\pgfpathlineto{\pgfqpoint{2.179888in}{1.274667in}}%
\pgfpathlineto{\pgfqpoint{2.135117in}{1.323597in}}%
\pgfpathlineto{\pgfqpoint{2.111206in}{1.349333in}}%
\pgfpathlineto{\pgfqpoint{2.042505in}{1.425540in}}%
\pgfpathlineto{\pgfqpoint{1.989323in}{1.486464in}}%
\pgfpathlineto{\pgfqpoint{1.945930in}{1.536000in}}%
\pgfpathlineto{\pgfqpoint{1.850955in}{1.648000in}}%
\pgfpathlineto{\pgfqpoint{1.812425in}{1.695025in}}%
\pgfpathlineto{\pgfqpoint{1.789474in}{1.722667in}}%
\pgfpathlineto{\pgfqpoint{1.721859in}{1.806745in}}%
\pgfpathlineto{\pgfqpoint{1.641697in}{1.909397in}}%
\pgfpathlineto{\pgfqpoint{1.592253in}{1.975279in}}%
\pgfpathlineto{\pgfqpoint{1.557625in}{2.021333in}}%
\pgfpathlineto{\pgfqpoint{1.503492in}{2.096000in}}%
\pgfpathlineto{\pgfqpoint{1.463119in}{2.153663in}}%
\pgfpathlineto{\pgfqpoint{1.441293in}{2.184262in}}%
\pgfpathlineto{\pgfqpoint{1.302327in}{2.394667in}}%
\pgfpathlineto{\pgfqpoint{1.280970in}{2.428706in}}%
\pgfpathlineto{\pgfqpoint{1.240889in}{2.494969in}}%
\pgfpathlineto{\pgfqpoint{1.200808in}{2.563757in}}%
\pgfpathlineto{\pgfqpoint{1.160727in}{2.635672in}}%
\pgfpathlineto{\pgfqpoint{1.092371in}{2.768000in}}%
\pgfpathlineto{\pgfqpoint{1.076401in}{2.801454in}}%
\pgfpathlineto{\pgfqpoint{1.069483in}{2.815656in}}%
\pgfpathlineto{\pgfqpoint{1.040454in}{2.880028in}}%
\pgfpathlineto{\pgfqpoint{1.009793in}{2.954667in}}%
\pgfpathlineto{\pgfqpoint{0.995508in}{2.992000in}}%
\pgfpathlineto{\pgfqpoint{0.957643in}{3.106497in}}%
\pgfpathlineto{\pgfqpoint{0.939417in}{3.178667in}}%
\pgfpathlineto{\pgfqpoint{0.936802in}{3.194091in}}%
\pgfpathlineto{\pgfqpoint{0.931725in}{3.216000in}}%
\pgfpathlineto{\pgfqpoint{0.925179in}{3.257932in}}%
\pgfpathlineto{\pgfqpoint{0.920242in}{3.301217in}}%
\pgfpathlineto{\pgfqpoint{0.918510in}{3.328000in}}%
\pgfpathlineto{\pgfqpoint{0.918345in}{3.367101in}}%
\pgfpathlineto{\pgfqpoint{0.920836in}{3.402667in}}%
\pgfpathlineto{\pgfqpoint{0.926932in}{3.440000in}}%
\pgfpathlineto{\pgfqpoint{0.929787in}{3.448890in}}%
\pgfpathlineto{\pgfqpoint{0.937273in}{3.477333in}}%
\pgfpathlineto{\pgfqpoint{0.942961in}{3.493505in}}%
\pgfpathlineto{\pgfqpoint{0.952922in}{3.514667in}}%
\pgfpathlineto{\pgfqpoint{0.970002in}{3.542985in}}%
\pgfpathlineto{\pgfqpoint{0.977308in}{3.552000in}}%
\pgfpathlineto{\pgfqpoint{1.007087in}{3.583108in}}%
\pgfpathlineto{\pgfqpoint{1.029524in}{3.599543in}}%
\pgfpathlineto{\pgfqpoint{1.040485in}{3.605808in}}%
\pgfpathlineto{\pgfqpoint{1.087881in}{3.626667in}}%
\pgfpathlineto{\pgfqpoint{1.120646in}{3.635375in}}%
\pgfpathlineto{\pgfqpoint{1.131233in}{3.636528in}}%
\pgfpathlineto{\pgfqpoint{1.160727in}{3.641408in}}%
\pgfpathlineto{\pgfqpoint{1.182987in}{3.643266in}}%
\pgfpathlineto{\pgfqpoint{1.200808in}{3.643589in}}%
\pgfpathlineto{\pgfqpoint{1.217900in}{3.642587in}}%
\pgfpathlineto{\pgfqpoint{1.240889in}{3.642616in}}%
\pgfpathlineto{\pgfqpoint{1.255955in}{3.640700in}}%
\pgfpathlineto{\pgfqpoint{1.280970in}{3.639028in}}%
\pgfpathlineto{\pgfqpoint{1.326660in}{3.631892in}}%
\pgfpathlineto{\pgfqpoint{1.362664in}{3.625240in}}%
\pgfpathlineto{\pgfqpoint{1.401212in}{3.615836in}}%
\pgfpathlineto{\pgfqpoint{1.441293in}{3.604730in}}%
\pgfpathlineto{\pgfqpoint{1.490369in}{3.589333in}}%
\pgfpathlineto{\pgfqpoint{1.561535in}{3.563702in}}%
\pgfpathlineto{\pgfqpoint{1.570089in}{3.559967in}}%
\pgfpathlineto{\pgfqpoint{1.609537in}{3.544622in}}%
\pgfpathlineto{\pgfqpoint{1.685119in}{3.511554in}}%
\pgfpathlineto{\pgfqpoint{1.761939in}{3.474711in}}%
\pgfpathlineto{\pgfqpoint{1.837291in}{3.435520in}}%
\pgfpathlineto{\pgfqpoint{1.842101in}{3.433168in}}%
\pgfpathlineto{\pgfqpoint{1.887847in}{3.407943in}}%
\pgfpathlineto{\pgfqpoint{1.922263in}{3.388784in}}%
\pgfpathlineto{\pgfqpoint{2.002424in}{3.341852in}}%
\pgfpathlineto{\pgfqpoint{2.035523in}{3.321497in}}%
\pgfpathlineto{\pgfqpoint{2.042505in}{3.317488in}}%
\pgfpathlineto{\pgfqpoint{2.122667in}{3.266953in}}%
\pgfpathlineto{\pgfqpoint{2.162747in}{3.240866in}}%
\pgfpathlineto{\pgfqpoint{2.255148in}{3.178667in}}%
\pgfpathlineto{\pgfqpoint{2.323071in}{3.131269in}}%
\pgfpathlineto{\pgfqpoint{2.412488in}{3.066667in}}%
\pgfpathlineto{\pgfqpoint{2.462658in}{3.029333in}}%
\pgfpathlineto{\pgfqpoint{2.563556in}{2.952348in}}%
\pgfpathlineto{\pgfqpoint{2.654807in}{2.880000in}}%
\pgfpathlineto{\pgfqpoint{2.700834in}{2.842667in}}%
\pgfpathlineto{\pgfqpoint{2.804040in}{2.757003in}}%
\pgfpathlineto{\pgfqpoint{2.860386in}{2.708483in}}%
\pgfpathlineto{\pgfqpoint{2.884202in}{2.688512in}}%
\pgfpathlineto{\pgfqpoint{2.964364in}{2.618306in}}%
\pgfpathlineto{\pgfqpoint{3.025103in}{2.563242in}}%
\pgfpathlineto{\pgfqpoint{3.065670in}{2.526362in}}%
\pgfpathlineto{\pgfqpoint{3.087805in}{2.506667in}}%
\pgfpathlineto{\pgfqpoint{3.207355in}{2.394667in}}%
\pgfpathlineto{\pgfqpoint{3.325091in}{2.280485in}}%
\pgfpathlineto{\pgfqpoint{3.365172in}{2.240672in}}%
\pgfpathlineto{\pgfqpoint{3.485414in}{2.118415in}}%
\pgfpathlineto{\pgfqpoint{3.542662in}{2.058667in}}%
\pgfpathlineto{\pgfqpoint{3.647602in}{1.946667in}}%
\pgfpathlineto{\pgfqpoint{3.725899in}{1.860451in}}%
\pgfpathlineto{\pgfqpoint{3.782030in}{1.797333in}}%
\pgfpathlineto{\pgfqpoint{3.847336in}{1.722667in}}%
\pgfpathlineto{\pgfqpoint{3.899797in}{1.660644in}}%
\pgfpathlineto{\pgfqpoint{3.942227in}{1.610667in}}%
\pgfpathlineto{\pgfqpoint{4.034124in}{1.498667in}}%
\pgfpathlineto{\pgfqpoint{4.064015in}{1.461333in}}%
\pgfpathlineto{\pgfqpoint{4.126707in}{1.381758in}}%
\pgfpathlineto{\pgfqpoint{4.211830in}{1.270045in}}%
\pgfpathlineto{\pgfqpoint{4.290437in}{1.162667in}}%
\pgfpathlineto{\pgfqpoint{4.336634in}{1.096870in}}%
\pgfpathlineto{\pgfqpoint{4.369112in}{1.050667in}}%
\pgfpathlineto{\pgfqpoint{4.407273in}{0.994198in}}%
\pgfpathlineto{\pgfqpoint{4.468149in}{0.901333in}}%
\pgfpathlineto{\pgfqpoint{4.527515in}{0.806649in}}%
\pgfpathlineto{\pgfqpoint{4.567596in}{0.739879in}}%
\pgfpathlineto{\pgfqpoint{4.664651in}{0.565333in}}%
\pgfpathlineto{\pgfqpoint{4.671856in}{0.550446in}}%
\pgfpathlineto{\pgfqpoint{4.684010in}{0.528000in}}%
\pgfpathlineto{\pgfqpoint{4.694914in}{0.528000in}}%
\pgfpathlineto{\pgfqpoint{4.647758in}{0.617046in}}%
\pgfpathlineto{\pgfqpoint{4.548126in}{0.789333in}}%
\pgfpathlineto{\pgfqpoint{4.501842in}{0.864000in}}%
\pgfpathlineto{\pgfqpoint{4.466182in}{0.918871in}}%
\pgfpathlineto{\pgfqpoint{4.447354in}{0.948544in}}%
\pgfpathlineto{\pgfqpoint{4.287030in}{1.180356in}}%
\pgfpathlineto{\pgfqpoint{4.102832in}{1.424000in}}%
\pgfpathlineto{\pgfqpoint{4.073271in}{1.461333in}}%
\pgfpathlineto{\pgfqpoint{4.006465in}{1.544152in}}%
\pgfpathlineto{\pgfqpoint{3.920188in}{1.648000in}}%
\pgfpathlineto{\pgfqpoint{3.869411in}{1.707008in}}%
\pgfpathlineto{\pgfqpoint{3.846141in}{1.734576in}}%
\pgfpathlineto{\pgfqpoint{3.758246in}{1.834667in}}%
\pgfpathlineto{\pgfqpoint{3.719639in}{1.877831in}}%
\pgfpathlineto{\pgfqpoint{3.605657in}{2.001825in}}%
\pgfpathlineto{\pgfqpoint{3.516504in}{2.096000in}}%
\pgfpathlineto{\pgfqpoint{3.463500in}{2.150255in}}%
\pgfpathlineto{\pgfqpoint{3.425241in}{2.189285in}}%
\pgfpathlineto{\pgfqpoint{3.405253in}{2.210161in}}%
\pgfpathlineto{\pgfqpoint{3.285010in}{2.329267in}}%
\pgfpathlineto{\pgfqpoint{3.164768in}{2.444377in}}%
\pgfpathlineto{\pgfqpoint{3.044525in}{2.555575in}}%
\pgfpathlineto{\pgfqpoint{2.989463in}{2.604712in}}%
\pgfpathlineto{\pgfqpoint{2.948427in}{2.641156in}}%
\pgfpathlineto{\pgfqpoint{2.924283in}{2.662940in}}%
\pgfpathlineto{\pgfqpoint{2.802256in}{2.768000in}}%
\pgfpathlineto{\pgfqpoint{2.739481in}{2.819866in}}%
\pgfpathlineto{\pgfqpoint{2.712383in}{2.842667in}}%
\pgfpathlineto{\pgfqpoint{2.603636in}{2.930275in}}%
\pgfpathlineto{\pgfqpoint{2.523475in}{2.992909in}}%
\pgfpathlineto{\pgfqpoint{2.475516in}{3.029333in}}%
\pgfpathlineto{\pgfqpoint{2.363152in}{3.112363in}}%
\pgfpathlineto{\pgfqpoint{2.269457in}{3.178667in}}%
\pgfpathlineto{\pgfqpoint{2.162747in}{3.251076in}}%
\pgfpathlineto{\pgfqpoint{2.082586in}{3.302784in}}%
\pgfpathlineto{\pgfqpoint{2.002424in}{3.352359in}}%
\pgfpathlineto{\pgfqpoint{1.916933in}{3.402667in}}%
\pgfpathlineto{\pgfqpoint{1.741641in}{3.496241in}}%
\pgfpathlineto{\pgfqpoint{1.660219in}{3.534748in}}%
\pgfpathlineto{\pgfqpoint{1.601616in}{3.560156in}}%
\pgfpathlineto{\pgfqpoint{1.561535in}{3.576355in}}%
\pgfpathlineto{\pgfqpoint{1.521455in}{3.591716in}}%
\pgfpathlineto{\pgfqpoint{1.401212in}{3.630199in}}%
\pgfpathlineto{\pgfqpoint{1.395818in}{3.631691in}}%
\pgfpathlineto{\pgfqpoint{1.361131in}{3.640122in}}%
\pgfpathlineto{\pgfqpoint{1.272142in}{3.655777in}}%
\pgfpathlineto{\pgfqpoint{1.236322in}{3.659746in}}%
\pgfpathlineto{\pgfqpoint{1.200808in}{3.661828in}}%
\pgfpathlineto{\pgfqpoint{1.156683in}{3.660233in}}%
\pgfpathlineto{\pgfqpoint{1.120646in}{3.656309in}}%
\pgfpathlineto{\pgfqpoint{1.094378in}{3.651135in}}%
\pgfpathlineto{\pgfqpoint{1.080566in}{3.647171in}}%
\pgfpathlineto{\pgfqpoint{1.029609in}{3.626667in}}%
\pgfpathlineto{\pgfqpoint{1.011510in}{3.616322in}}%
\pgfpathlineto{\pgfqpoint{0.989231in}{3.599741in}}%
\pgfpathlineto{\pgfqpoint{0.978361in}{3.589333in}}%
\pgfpathlineto{\pgfqpoint{0.952300in}{3.559473in}}%
\pgfpathlineto{\pgfqpoint{0.947628in}{3.552000in}}%
\pgfpathlineto{\pgfqpoint{0.925311in}{3.509946in}}%
\pgfpathlineto{\pgfqpoint{0.914414in}{3.477333in}}%
\pgfpathlineto{\pgfqpoint{0.906438in}{3.440000in}}%
\pgfpathlineto{\pgfqpoint{0.905353in}{3.426131in}}%
\pgfpathlineto{\pgfqpoint{0.902082in}{3.402667in}}%
\pgfpathlineto{\pgfqpoint{0.900763in}{3.383478in}}%
\pgfpathlineto{\pgfqpoint{0.900716in}{3.365333in}}%
\pgfpathlineto{\pgfqpoint{0.901816in}{3.348170in}}%
\pgfpathlineto{\pgfqpoint{0.901851in}{3.328000in}}%
\pgfpathlineto{\pgfqpoint{0.905096in}{3.290667in}}%
\pgfpathlineto{\pgfqpoint{0.910139in}{3.253333in}}%
\pgfpathlineto{\pgfqpoint{0.911888in}{3.245551in}}%
\pgfpathlineto{\pgfqpoint{0.920242in}{3.199852in}}%
\pgfpathlineto{\pgfqpoint{0.934522in}{3.141333in}}%
\pgfpathlineto{\pgfqpoint{0.945158in}{3.104000in}}%
\pgfpathlineto{\pgfqpoint{0.960323in}{3.056002in}}%
\pgfpathlineto{\pgfqpoint{1.000404in}{2.947551in}}%
\pgfpathlineto{\pgfqpoint{1.049362in}{2.834398in}}%
\pgfpathlineto{\pgfqpoint{1.081442in}{2.767183in}}%
\pgfpathlineto{\pgfqpoint{1.120646in}{2.690353in}}%
\pgfpathlineto{\pgfqpoint{1.160727in}{2.616149in}}%
\pgfpathlineto{\pgfqpoint{1.180432in}{2.581333in}}%
\pgfpathlineto{\pgfqpoint{1.223786in}{2.506667in}}%
\pgfpathlineto{\pgfqpoint{1.280970in}{2.412787in}}%
\pgfpathlineto{\pgfqpoint{1.401212in}{2.228688in}}%
\pgfpathlineto{\pgfqpoint{1.561535in}{2.003496in}}%
\pgfpathlineto{\pgfqpoint{1.641697in}{1.897484in}}%
\pgfpathlineto{\pgfqpoint{1.690574in}{1.834667in}}%
\pgfpathlineto{\pgfqpoint{1.761939in}{1.745218in}}%
\pgfpathlineto{\pgfqpoint{1.842101in}{1.647429in}}%
\pgfpathlineto{\pgfqpoint{1.936705in}{1.536000in}}%
\pgfpathlineto{\pgfqpoint{1.984276in}{1.481762in}}%
\pgfpathlineto{\pgfqpoint{2.020393in}{1.440737in}}%
\pgfpathlineto{\pgfqpoint{2.042505in}{1.415256in}}%
\pgfpathlineto{\pgfqpoint{2.162747in}{1.282984in}}%
\pgfpathlineto{\pgfqpoint{2.223164in}{1.218941in}}%
\pgfpathlineto{\pgfqpoint{2.260762in}{1.179296in}}%
\pgfpathlineto{\pgfqpoint{2.282990in}{1.155483in}}%
\pgfpathlineto{\pgfqpoint{2.403232in}{1.032517in}}%
\pgfpathlineto{\pgfqpoint{2.497884in}{0.938667in}}%
\pgfpathlineto{\pgfqpoint{2.536247in}{0.901333in}}%
\pgfpathlineto{\pgfqpoint{2.643717in}{0.798908in}}%
\pgfpathlineto{\pgfqpoint{2.709329in}{0.738447in}}%
\pgfpathlineto{\pgfqpoint{2.749767in}{0.701447in}}%
\pgfpathlineto{\pgfqpoint{2.790442in}{0.664667in}}%
\pgfpathlineto{\pgfqpoint{2.831356in}{0.628109in}}%
\pgfpathlineto{\pgfqpoint{2.859696in}{0.602667in}}%
\pgfpathlineto{\pgfqpoint{2.945657in}{0.528000in}}%
\pgfpathlineto{\pgfqpoint{2.945657in}{0.528000in}}%
\pgfusepath{fill}%
\end{pgfscope}%
\begin{pgfscope}%
\pgfpathrectangle{\pgfqpoint{0.800000in}{0.528000in}}{\pgfqpoint{3.968000in}{3.696000in}}%
\pgfusepath{clip}%
\pgfsetbuttcap%
\pgfsetroundjoin%
\definecolor{currentfill}{rgb}{0.281924,0.089666,0.412415}%
\pgfsetfillcolor{currentfill}%
\pgfsetlinewidth{0.000000pt}%
\definecolor{currentstroke}{rgb}{0.000000,0.000000,0.000000}%
\pgfsetstrokecolor{currentstroke}%
\pgfsetdash{}{0pt}%
\pgfpathmoveto{\pgfqpoint{2.945657in}{0.528000in}}%
\pgfpathlineto{\pgfqpoint{2.844121in}{0.616372in}}%
\pgfpathlineto{\pgfqpoint{2.790442in}{0.664667in}}%
\pgfpathlineto{\pgfqpoint{2.749767in}{0.701447in}}%
\pgfpathlineto{\pgfqpoint{2.709329in}{0.738447in}}%
\pgfpathlineto{\pgfqpoint{2.669124in}{0.775665in}}%
\pgfpathlineto{\pgfqpoint{2.643717in}{0.798908in}}%
\pgfpathlineto{\pgfqpoint{2.523475in}{0.913688in}}%
\pgfpathlineto{\pgfqpoint{2.471520in}{0.964940in}}%
\pgfpathlineto{\pgfqpoint{2.422399in}{1.013333in}}%
\pgfpathlineto{\pgfqpoint{2.323071in}{1.114029in}}%
\pgfpathlineto{\pgfqpoint{2.240452in}{1.200000in}}%
\pgfpathlineto{\pgfqpoint{2.185778in}{1.258785in}}%
\pgfpathlineto{\pgfqpoint{2.162747in}{1.282984in}}%
\pgfpathlineto{\pgfqpoint{2.042505in}{1.415256in}}%
\pgfpathlineto{\pgfqpoint{1.839979in}{1.649976in}}%
\pgfpathlineto{\pgfqpoint{1.750017in}{1.760000in}}%
\pgfpathlineto{\pgfqpoint{1.681778in}{1.845879in}}%
\pgfpathlineto{\pgfqpoint{1.601616in}{1.949872in}}%
\pgfpathlineto{\pgfqpoint{1.553863in}{2.014187in}}%
\pgfpathlineto{\pgfqpoint{1.519496in}{2.060491in}}%
\pgfpathlineto{\pgfqpoint{1.441017in}{2.170667in}}%
\pgfpathlineto{\pgfqpoint{1.401212in}{2.228688in}}%
\pgfpathlineto{\pgfqpoint{1.340326in}{2.320000in}}%
\pgfpathlineto{\pgfqpoint{1.303323in}{2.378154in}}%
\pgfpathlineto{\pgfqpoint{1.280970in}{2.412787in}}%
\pgfpathlineto{\pgfqpoint{1.223786in}{2.506667in}}%
\pgfpathlineto{\pgfqpoint{1.215932in}{2.520754in}}%
\pgfpathlineto{\pgfqpoint{1.200808in}{2.545496in}}%
\pgfpathlineto{\pgfqpoint{1.139106in}{2.656000in}}%
\pgfpathlineto{\pgfqpoint{1.099965in}{2.730667in}}%
\pgfpathlineto{\pgfqpoint{1.063183in}{2.805333in}}%
\pgfpathlineto{\pgfqpoint{1.040485in}{2.854083in}}%
\pgfpathlineto{\pgfqpoint{1.009700in}{2.925992in}}%
\pgfpathlineto{\pgfqpoint{0.995710in}{2.959039in}}%
\pgfpathlineto{\pgfqpoint{0.967502in}{3.036020in}}%
\pgfpathlineto{\pgfqpoint{0.955038in}{3.071590in}}%
\pgfpathlineto{\pgfqpoint{0.934522in}{3.141333in}}%
\pgfpathlineto{\pgfqpoint{0.932148in}{3.152422in}}%
\pgfpathlineto{\pgfqpoint{0.924280in}{3.182428in}}%
\pgfpathlineto{\pgfqpoint{0.916726in}{3.216000in}}%
\pgfpathlineto{\pgfqpoint{0.905096in}{3.290667in}}%
\pgfpathlineto{\pgfqpoint{0.903242in}{3.306502in}}%
\pgfpathlineto{\pgfqpoint{0.901851in}{3.328000in}}%
\pgfpathlineto{\pgfqpoint{0.902082in}{3.402667in}}%
\pgfpathlineto{\pgfqpoint{0.908165in}{3.451249in}}%
\pgfpathlineto{\pgfqpoint{0.914414in}{3.477333in}}%
\pgfpathlineto{\pgfqpoint{0.927533in}{3.514667in}}%
\pgfpathlineto{\pgfqpoint{0.952300in}{3.559473in}}%
\pgfpathlineto{\pgfqpoint{0.960323in}{3.569562in}}%
\pgfpathlineto{\pgfqpoint{0.989231in}{3.599741in}}%
\pgfpathlineto{\pgfqpoint{1.011510in}{3.616322in}}%
\pgfpathlineto{\pgfqpoint{1.040485in}{3.632306in}}%
\pgfpathlineto{\pgfqpoint{1.049932in}{3.635466in}}%
\pgfpathlineto{\pgfqpoint{1.080566in}{3.647171in}}%
\pgfpathlineto{\pgfqpoint{1.094378in}{3.651135in}}%
\pgfpathlineto{\pgfqpoint{1.120646in}{3.656309in}}%
\pgfpathlineto{\pgfqpoint{1.163780in}{3.661157in}}%
\pgfpathlineto{\pgfqpoint{1.200808in}{3.661828in}}%
\pgfpathlineto{\pgfqpoint{1.240889in}{3.659752in}}%
\pgfpathlineto{\pgfqpoint{1.321051in}{3.648537in}}%
\pgfpathlineto{\pgfqpoint{1.339687in}{3.644026in}}%
\pgfpathlineto{\pgfqpoint{1.361131in}{3.640122in}}%
\pgfpathlineto{\pgfqpoint{1.413422in}{3.626667in}}%
\pgfpathlineto{\pgfqpoint{1.481374in}{3.605633in}}%
\pgfpathlineto{\pgfqpoint{1.493729in}{3.600841in}}%
\pgfpathlineto{\pgfqpoint{1.527688in}{3.589333in}}%
\pgfpathlineto{\pgfqpoint{1.601616in}{3.560156in}}%
\pgfpathlineto{\pgfqpoint{1.620622in}{3.552000in}}%
\pgfpathlineto{\pgfqpoint{1.681778in}{3.524792in}}%
\pgfpathlineto{\pgfqpoint{1.703207in}{3.514667in}}%
\pgfpathlineto{\pgfqpoint{1.761939in}{3.486044in}}%
\pgfpathlineto{\pgfqpoint{1.793592in}{3.469483in}}%
\pgfpathlineto{\pgfqpoint{1.802020in}{3.465462in}}%
\pgfpathlineto{\pgfqpoint{1.849866in}{3.440000in}}%
\pgfpathlineto{\pgfqpoint{2.017908in}{3.342423in}}%
\pgfpathlineto{\pgfqpoint{2.042541in}{3.328000in}}%
\pgfpathlineto{\pgfqpoint{2.162747in}{3.251076in}}%
\pgfpathlineto{\pgfqpoint{2.242909in}{3.196995in}}%
\pgfpathlineto{\pgfqpoint{2.323693in}{3.140753in}}%
\pgfpathlineto{\pgfqpoint{2.443313in}{3.053472in}}%
\pgfpathlineto{\pgfqpoint{2.524649in}{2.992000in}}%
\pgfpathlineto{\pgfqpoint{2.572636in}{2.954667in}}%
\pgfpathlineto{\pgfqpoint{2.683798in}{2.866009in}}%
\pgfpathlineto{\pgfqpoint{2.804040in}{2.766501in}}%
\pgfpathlineto{\pgfqpoint{2.865611in}{2.713350in}}%
\pgfpathlineto{\pgfqpoint{2.889439in}{2.693333in}}%
\pgfpathlineto{\pgfqpoint{3.004444in}{2.591785in}}%
\pgfpathlineto{\pgfqpoint{3.124687in}{2.481874in}}%
\pgfpathlineto{\pgfqpoint{3.191018in}{2.419118in}}%
\pgfpathlineto{\pgfqpoint{3.217174in}{2.394667in}}%
\pgfpathlineto{\pgfqpoint{3.332522in}{2.282667in}}%
\pgfpathlineto{\pgfqpoint{3.370152in}{2.245333in}}%
\pgfpathlineto{\pgfqpoint{3.485414in}{2.128313in}}%
\pgfpathlineto{\pgfqpoint{3.525495in}{2.086633in}}%
\pgfpathlineto{\pgfqpoint{3.622243in}{1.984000in}}%
\pgfpathlineto{\pgfqpoint{3.656803in}{1.946667in}}%
\pgfpathlineto{\pgfqpoint{3.758246in}{1.834667in}}%
\pgfpathlineto{\pgfqpoint{3.846141in}{1.734576in}}%
\pgfpathlineto{\pgfqpoint{3.926303in}{1.640745in}}%
\pgfpathlineto{\pgfqpoint{3.975292in}{1.581631in}}%
\pgfpathlineto{\pgfqpoint{4.013127in}{1.536000in}}%
\pgfpathlineto{\pgfqpoint{4.086626in}{1.444546in}}%
\pgfpathlineto{\pgfqpoint{4.132149in}{1.386667in}}%
\pgfpathlineto{\pgfqpoint{4.206869in}{1.288951in}}%
\pgfpathlineto{\pgfqpoint{4.287030in}{1.180356in}}%
\pgfpathlineto{\pgfqpoint{4.352689in}{1.088000in}}%
\pgfpathlineto{\pgfqpoint{4.389451in}{1.034067in}}%
\pgfpathlineto{\pgfqpoint{4.407273in}{1.008728in}}%
\pgfpathlineto{\pgfqpoint{4.487434in}{0.886707in}}%
\pgfpathlineto{\pgfqpoint{4.592589in}{0.714667in}}%
\pgfpathlineto{\pgfqpoint{4.635094in}{0.640000in}}%
\pgfpathlineto{\pgfqpoint{4.647758in}{0.617046in}}%
\pgfpathlineto{\pgfqpoint{4.687838in}{0.541686in}}%
\pgfpathlineto{\pgfqpoint{4.694914in}{0.528000in}}%
\pgfpathlineto{\pgfqpoint{4.705646in}{0.528000in}}%
\pgfpathlineto{\pgfqpoint{4.666080in}{0.602667in}}%
\pgfpathlineto{\pgfqpoint{4.624385in}{0.677333in}}%
\pgfpathlineto{\pgfqpoint{4.589991in}{0.735527in}}%
\pgfpathlineto{\pgfqpoint{4.567596in}{0.773825in}}%
\pgfpathlineto{\pgfqpoint{4.511672in}{0.864000in}}%
\pgfpathlineto{\pgfqpoint{4.472263in}{0.924536in}}%
\pgfpathlineto{\pgfqpoint{4.447354in}{0.963127in}}%
\pgfpathlineto{\pgfqpoint{4.309131in}{1.162667in}}%
\pgfpathlineto{\pgfqpoint{4.246949in}{1.247769in}}%
\pgfpathlineto{\pgfqpoint{4.166788in}{1.353826in}}%
\pgfpathlineto{\pgfqpoint{4.070671in}{1.476195in}}%
\pgfpathlineto{\pgfqpoint{3.991551in}{1.573333in}}%
\pgfpathlineto{\pgfqpoint{3.945294in}{1.628355in}}%
\pgfpathlineto{\pgfqpoint{3.910019in}{1.670165in}}%
\pgfpathlineto{\pgfqpoint{3.874540in}{1.711786in}}%
\pgfpathlineto{\pgfqpoint{3.833167in}{1.760000in}}%
\pgfpathlineto{\pgfqpoint{3.800521in}{1.797333in}}%
\pgfpathlineto{\pgfqpoint{3.700156in}{1.909333in}}%
\pgfpathlineto{\pgfqpoint{3.637987in}{1.976781in}}%
\pgfpathlineto{\pgfqpoint{3.596706in}{2.021333in}}%
\pgfpathlineto{\pgfqpoint{3.561544in}{2.058667in}}%
\pgfpathlineto{\pgfqpoint{3.485414in}{2.138035in}}%
\pgfpathlineto{\pgfqpoint{3.430216in}{2.193919in}}%
\pgfpathlineto{\pgfqpoint{3.405253in}{2.219657in}}%
\pgfpathlineto{\pgfqpoint{3.285010in}{2.338696in}}%
\pgfpathlineto{\pgfqpoint{3.164768in}{2.453740in}}%
\pgfpathlineto{\pgfqpoint{3.044525in}{2.564872in}}%
\pgfpathlineto{\pgfqpoint{2.994632in}{2.609527in}}%
\pgfpathlineto{\pgfqpoint{2.953615in}{2.645988in}}%
\pgfpathlineto{\pgfqpoint{2.924283in}{2.672173in}}%
\pgfpathlineto{\pgfqpoint{2.804040in}{2.775722in}}%
\pgfpathlineto{\pgfqpoint{2.744763in}{2.824786in}}%
\pgfpathlineto{\pgfqpoint{2.723500in}{2.843019in}}%
\pgfpathlineto{\pgfqpoint{2.603636in}{2.939661in}}%
\pgfpathlineto{\pgfqpoint{2.483394in}{3.032943in}}%
\pgfpathlineto{\pgfqpoint{2.387804in}{3.104000in}}%
\pgfpathlineto{\pgfqpoint{2.280765in}{3.180739in}}%
\pgfpathlineto{\pgfqpoint{2.162747in}{3.260990in}}%
\pgfpathlineto{\pgfqpoint{2.096474in}{3.303603in}}%
\pgfpathlineto{\pgfqpoint{2.058671in}{3.328000in}}%
\pgfpathlineto{\pgfqpoint{1.998296in}{3.365333in}}%
\pgfpathlineto{\pgfqpoint{1.842101in}{3.455094in}}%
\pgfpathlineto{\pgfqpoint{1.761939in}{3.497265in}}%
\pgfpathlineto{\pgfqpoint{1.681778in}{3.536443in}}%
\pgfpathlineto{\pgfqpoint{1.641697in}{3.554943in}}%
\pgfpathlineto{\pgfqpoint{1.521455in}{3.604330in}}%
\pgfpathlineto{\pgfqpoint{1.503735in}{3.610162in}}%
\pgfpathlineto{\pgfqpoint{1.481374in}{3.618835in}}%
\pgfpathlineto{\pgfqpoint{1.432361in}{3.634987in}}%
\pgfpathlineto{\pgfqpoint{1.361131in}{3.654627in}}%
\pgfpathlineto{\pgfqpoint{1.352713in}{3.656159in}}%
\pgfpathlineto{\pgfqpoint{1.320065in}{3.664000in}}%
\pgfpathlineto{\pgfqpoint{1.271959in}{3.672393in}}%
\pgfpathlineto{\pgfqpoint{1.240889in}{3.676104in}}%
\pgfpathlineto{\pgfqpoint{1.160727in}{3.679271in}}%
\pgfpathlineto{\pgfqpoint{1.144923in}{3.678721in}}%
\pgfpathlineto{\pgfqpoint{1.120646in}{3.676272in}}%
\pgfpathlineto{\pgfqpoint{1.075926in}{3.668322in}}%
\pgfpathlineto{\pgfqpoint{1.040485in}{3.656869in}}%
\pgfpathlineto{\pgfqpoint{1.018428in}{3.647212in}}%
\pgfpathlineto{\pgfqpoint{1.000404in}{3.636979in}}%
\pgfpathlineto{\pgfqpoint{0.971511in}{3.616246in}}%
\pgfpathlineto{\pgfqpoint{0.960323in}{3.605534in}}%
\pgfpathlineto{\pgfqpoint{0.945573in}{3.589333in}}%
\pgfpathlineto{\pgfqpoint{0.934793in}{3.575780in}}%
\pgfpathlineto{\pgfqpoint{0.919960in}{3.552000in}}%
\pgfpathlineto{\pgfqpoint{0.907911in}{3.526152in}}%
\pgfpathlineto{\pgfqpoint{0.903891in}{3.514667in}}%
\pgfpathlineto{\pgfqpoint{0.890726in}{3.467493in}}%
\pgfpathlineto{\pgfqpoint{0.886505in}{3.440000in}}%
\pgfpathlineto{\pgfqpoint{0.883181in}{3.399854in}}%
\pgfpathlineto{\pgfqpoint{0.883093in}{3.365333in}}%
\pgfpathlineto{\pgfqpoint{0.885515in}{3.323014in}}%
\pgfpathlineto{\pgfqpoint{0.890463in}{3.281071in}}%
\pgfpathlineto{\pgfqpoint{0.902417in}{3.216000in}}%
\pgfpathlineto{\pgfqpoint{0.910983in}{3.178667in}}%
\pgfpathlineto{\pgfqpoint{0.920827in}{3.140788in}}%
\pgfpathlineto{\pgfqpoint{0.944011in}{3.066667in}}%
\pgfpathlineto{\pgfqpoint{0.960323in}{3.020013in}}%
\pgfpathlineto{\pgfqpoint{1.001304in}{2.916495in}}%
\pgfpathlineto{\pgfqpoint{1.040485in}{2.829494in}}%
\pgfpathlineto{\pgfqpoint{1.080566in}{2.747385in}}%
\pgfpathlineto{\pgfqpoint{1.120646in}{2.670544in}}%
\pgfpathlineto{\pgfqpoint{1.160727in}{2.597804in}}%
\pgfpathlineto{\pgfqpoint{1.259129in}{2.432000in}}%
\pgfpathlineto{\pgfqpoint{1.306351in}{2.357333in}}%
\pgfpathlineto{\pgfqpoint{1.361131in}{2.273816in}}%
\pgfpathlineto{\pgfqpoint{1.420025in}{2.188190in}}%
\pgfpathlineto{\pgfqpoint{1.441293in}{2.156968in}}%
\pgfpathlineto{\pgfqpoint{1.490850in}{2.087173in}}%
\pgfpathlineto{\pgfqpoint{1.566718in}{1.984000in}}%
\pgfpathlineto{\pgfqpoint{1.614568in}{1.921397in}}%
\pgfpathlineto{\pgfqpoint{1.647919in}{1.877795in}}%
\pgfpathlineto{\pgfqpoint{1.681778in}{1.834033in}}%
\pgfpathlineto{\pgfqpoint{1.771091in}{1.722667in}}%
\pgfpathlineto{\pgfqpoint{1.842101in}{1.636650in}}%
\pgfpathlineto{\pgfqpoint{1.927480in}{1.536000in}}%
\pgfpathlineto{\pgfqpoint{1.979228in}{1.477061in}}%
\pgfpathlineto{\pgfqpoint{2.015328in}{1.436019in}}%
\pgfpathlineto{\pgfqpoint{2.058995in}{1.386667in}}%
\pgfpathlineto{\pgfqpoint{2.092688in}{1.349333in}}%
\pgfpathlineto{\pgfqpoint{2.196007in}{1.237333in}}%
\pgfpathlineto{\pgfqpoint{2.282990in}{1.145782in}}%
\pgfpathlineto{\pgfqpoint{2.403232in}{1.022886in}}%
\pgfpathlineto{\pgfqpoint{2.488087in}{0.938667in}}%
\pgfpathlineto{\pgfqpoint{2.526362in}{0.901333in}}%
\pgfpathlineto{\pgfqpoint{2.643804in}{0.789333in}}%
\pgfpathlineto{\pgfqpoint{2.704312in}{0.733775in}}%
\pgfpathlineto{\pgfqpoint{2.744734in}{0.696759in}}%
\pgfpathlineto{\pgfqpoint{2.785391in}{0.659962in}}%
\pgfpathlineto{\pgfqpoint{2.826287in}{0.623388in}}%
\pgfpathlineto{\pgfqpoint{2.849034in}{0.602667in}}%
\pgfpathlineto{\pgfqpoint{2.934782in}{0.528000in}}%
\pgfpathlineto{\pgfqpoint{2.934782in}{0.528000in}}%
\pgfusepath{fill}%
\end{pgfscope}%
\begin{pgfscope}%
\pgfpathrectangle{\pgfqpoint{0.800000in}{0.528000in}}{\pgfqpoint{3.968000in}{3.696000in}}%
\pgfusepath{clip}%
\pgfsetbuttcap%
\pgfsetroundjoin%
\definecolor{currentfill}{rgb}{0.281924,0.089666,0.412415}%
\pgfsetfillcolor{currentfill}%
\pgfsetlinewidth{0.000000pt}%
\definecolor{currentstroke}{rgb}{0.000000,0.000000,0.000000}%
\pgfsetstrokecolor{currentstroke}%
\pgfsetdash{}{0pt}%
\pgfpathmoveto{\pgfqpoint{2.934782in}{0.528000in}}%
\pgfpathlineto{\pgfqpoint{2.844121in}{0.606990in}}%
\pgfpathlineto{\pgfqpoint{2.785391in}{0.659962in}}%
\pgfpathlineto{\pgfqpoint{2.744734in}{0.696759in}}%
\pgfpathlineto{\pgfqpoint{2.704312in}{0.733775in}}%
\pgfpathlineto{\pgfqpoint{2.664124in}{0.771008in}}%
\pgfpathlineto{\pgfqpoint{2.643717in}{0.789414in}}%
\pgfpathlineto{\pgfqpoint{2.523475in}{0.904126in}}%
\pgfpathlineto{\pgfqpoint{2.466605in}{0.960362in}}%
\pgfpathlineto{\pgfqpoint{2.427769in}{0.998855in}}%
\pgfpathlineto{\pgfqpoint{2.403232in}{1.022886in}}%
\pgfpathlineto{\pgfqpoint{2.282990in}{1.145782in}}%
\pgfpathlineto{\pgfqpoint{2.196007in}{1.237333in}}%
\pgfpathlineto{\pgfqpoint{2.150951in}{1.285654in}}%
\pgfpathlineto{\pgfqpoint{2.042505in}{1.405032in}}%
\pgfpathlineto{\pgfqpoint{1.979228in}{1.477061in}}%
\pgfpathlineto{\pgfqpoint{1.943333in}{1.518293in}}%
\pgfpathlineto{\pgfqpoint{1.907642in}{1.559715in}}%
\pgfpathlineto{\pgfqpoint{1.863967in}{1.610667in}}%
\pgfpathlineto{\pgfqpoint{1.771091in}{1.722667in}}%
\pgfpathlineto{\pgfqpoint{1.679908in}{1.836408in}}%
\pgfpathlineto{\pgfqpoint{1.594861in}{1.946667in}}%
\pgfpathlineto{\pgfqpoint{1.548465in}{2.009159in}}%
\pgfpathlineto{\pgfqpoint{1.511623in}{2.058667in}}%
\pgfpathlineto{\pgfqpoint{1.441293in}{2.156968in}}%
\pgfpathlineto{\pgfqpoint{1.380384in}{2.245333in}}%
\pgfpathlineto{\pgfqpoint{1.355201in}{2.282667in}}%
\pgfpathlineto{\pgfqpoint{1.306351in}{2.357333in}}%
\pgfpathlineto{\pgfqpoint{1.280970in}{2.396869in}}%
\pgfpathlineto{\pgfqpoint{1.213652in}{2.506667in}}%
\pgfpathlineto{\pgfqpoint{1.191590in}{2.544000in}}%
\pgfpathlineto{\pgfqpoint{1.149031in}{2.618667in}}%
\pgfpathlineto{\pgfqpoint{1.128461in}{2.656000in}}%
\pgfpathlineto{\pgfqpoint{1.080566in}{2.747385in}}%
\pgfpathlineto{\pgfqpoint{1.017438in}{2.880000in}}%
\pgfpathlineto{\pgfqpoint{1.000404in}{2.918662in}}%
\pgfpathlineto{\pgfqpoint{0.968376in}{2.999501in}}%
\pgfpathlineto{\pgfqpoint{0.955030in}{3.034263in}}%
\pgfpathlineto{\pgfqpoint{0.931920in}{3.104000in}}%
\pgfpathlineto{\pgfqpoint{0.920242in}{3.143006in}}%
\pgfpathlineto{\pgfqpoint{0.902417in}{3.216000in}}%
\pgfpathlineto{\pgfqpoint{0.899473in}{3.233987in}}%
\pgfpathlineto{\pgfqpoint{0.895124in}{3.253333in}}%
\pgfpathlineto{\pgfqpoint{0.889302in}{3.290667in}}%
\pgfpathlineto{\pgfqpoint{0.885184in}{3.332678in}}%
\pgfpathlineto{\pgfqpoint{0.883093in}{3.365333in}}%
\pgfpathlineto{\pgfqpoint{0.883854in}{3.406106in}}%
\pgfpathlineto{\pgfqpoint{0.886505in}{3.440000in}}%
\pgfpathlineto{\pgfqpoint{0.893084in}{3.477333in}}%
\pgfpathlineto{\pgfqpoint{0.907911in}{3.526152in}}%
\pgfpathlineto{\pgfqpoint{0.920242in}{3.552504in}}%
\pgfpathlineto{\pgfqpoint{0.934793in}{3.575780in}}%
\pgfpathlineto{\pgfqpoint{0.945573in}{3.589333in}}%
\pgfpathlineto{\pgfqpoint{0.971511in}{3.616246in}}%
\pgfpathlineto{\pgfqpoint{1.000404in}{3.636979in}}%
\pgfpathlineto{\pgfqpoint{1.018428in}{3.647212in}}%
\pgfpathlineto{\pgfqpoint{1.040485in}{3.656869in}}%
\pgfpathlineto{\pgfqpoint{1.080566in}{3.669320in}}%
\pgfpathlineto{\pgfqpoint{1.087183in}{3.670163in}}%
\pgfpathlineto{\pgfqpoint{1.120646in}{3.676272in}}%
\pgfpathlineto{\pgfqpoint{1.160727in}{3.679271in}}%
\pgfpathlineto{\pgfqpoint{1.176420in}{3.678617in}}%
\pgfpathlineto{\pgfqpoint{1.200808in}{3.679031in}}%
\pgfpathlineto{\pgfqpoint{1.215228in}{3.677431in}}%
\pgfpathlineto{\pgfqpoint{1.240889in}{3.676104in}}%
\pgfpathlineto{\pgfqpoint{1.286946in}{3.669566in}}%
\pgfpathlineto{\pgfqpoint{1.321051in}{3.663825in}}%
\pgfpathlineto{\pgfqpoint{1.401212in}{3.643998in}}%
\pgfpathlineto{\pgfqpoint{1.457806in}{3.626667in}}%
\pgfpathlineto{\pgfqpoint{1.481374in}{3.618835in}}%
\pgfpathlineto{\pgfqpoint{1.521455in}{3.604330in}}%
\pgfpathlineto{\pgfqpoint{1.561535in}{3.589009in}}%
\pgfpathlineto{\pgfqpoint{1.696876in}{3.528730in}}%
\pgfpathlineto{\pgfqpoint{1.727299in}{3.514667in}}%
\pgfpathlineto{\pgfqpoint{1.803539in}{3.475919in}}%
\pgfpathlineto{\pgfqpoint{1.882182in}{3.433031in}}%
\pgfpathlineto{\pgfqpoint{1.927145in}{3.407214in}}%
\pgfpathlineto{\pgfqpoint{1.962343in}{3.386758in}}%
\pgfpathlineto{\pgfqpoint{1.998296in}{3.365333in}}%
\pgfpathlineto{\pgfqpoint{2.058671in}{3.328000in}}%
\pgfpathlineto{\pgfqpoint{2.122667in}{3.287319in}}%
\pgfpathlineto{\pgfqpoint{2.174195in}{3.253333in}}%
\pgfpathlineto{\pgfqpoint{2.283723in}{3.178667in}}%
\pgfpathlineto{\pgfqpoint{2.403232in}{3.092736in}}%
\pgfpathlineto{\pgfqpoint{2.488118in}{3.029333in}}%
\pgfpathlineto{\pgfqpoint{2.551128in}{2.980425in}}%
\pgfpathlineto{\pgfqpoint{2.584566in}{2.954667in}}%
\pgfpathlineto{\pgfqpoint{2.683798in}{2.875440in}}%
\pgfpathlineto{\pgfqpoint{2.768828in}{2.805333in}}%
\pgfpathlineto{\pgfqpoint{2.829068in}{2.753978in}}%
\pgfpathlineto{\pgfqpoint{2.856858in}{2.730667in}}%
\pgfpathlineto{\pgfqpoint{2.900023in}{2.693333in}}%
\pgfpathlineto{\pgfqpoint{3.004444in}{2.601060in}}%
\pgfpathlineto{\pgfqpoint{3.124687in}{2.491214in}}%
\pgfpathlineto{\pgfqpoint{3.164768in}{2.453740in}}%
\pgfpathlineto{\pgfqpoint{3.285010in}{2.338696in}}%
\pgfpathlineto{\pgfqpoint{3.353074in}{2.271398in}}%
\pgfpathlineto{\pgfqpoint{3.379629in}{2.245333in}}%
\pgfpathlineto{\pgfqpoint{3.489964in}{2.133333in}}%
\pgfpathlineto{\pgfqpoint{3.530140in}{2.091673in}}%
\pgfpathlineto{\pgfqpoint{3.645737in}{1.968686in}}%
\pgfpathlineto{\pgfqpoint{3.733989in}{1.872000in}}%
\pgfpathlineto{\pgfqpoint{3.774863in}{1.826392in}}%
\pgfpathlineto{\pgfqpoint{3.865517in}{1.722667in}}%
\pgfpathlineto{\pgfqpoint{3.910019in}{1.670165in}}%
\pgfpathlineto{\pgfqpoint{3.945294in}{1.628355in}}%
\pgfpathlineto{\pgfqpoint{3.980367in}{1.586358in}}%
\pgfpathlineto{\pgfqpoint{4.022185in}{1.536000in}}%
\pgfpathlineto{\pgfqpoint{4.086626in}{1.456182in}}%
\pgfpathlineto{\pgfqpoint{4.141250in}{1.386667in}}%
\pgfpathlineto{\pgfqpoint{4.206869in}{1.301234in}}%
\pgfpathlineto{\pgfqpoint{4.287030in}{1.193321in}}%
\pgfpathlineto{\pgfqpoint{4.367192in}{1.080900in}}%
\pgfpathlineto{\pgfqpoint{4.413689in}{1.013333in}}%
\pgfpathlineto{\pgfqpoint{4.463467in}{0.938667in}}%
\pgfpathlineto{\pgfqpoint{4.502365in}{0.877907in}}%
\pgfpathlineto{\pgfqpoint{4.527515in}{0.838933in}}%
\pgfpathlineto{\pgfqpoint{4.624385in}{0.677333in}}%
\pgfpathlineto{\pgfqpoint{4.666080in}{0.602667in}}%
\pgfpathlineto{\pgfqpoint{4.673058in}{0.588899in}}%
\pgfpathlineto{\pgfqpoint{4.687838in}{0.562442in}}%
\pgfpathlineto{\pgfqpoint{4.705646in}{0.528000in}}%
\pgfpathlineto{\pgfqpoint{4.716377in}{0.528000in}}%
\pgfpathlineto{\pgfqpoint{4.687838in}{0.581899in}}%
\pgfpathlineto{\pgfqpoint{4.607677in}{0.723606in}}%
\pgfpathlineto{\pgfqpoint{4.487434in}{0.916788in}}%
\pgfpathlineto{\pgfqpoint{4.345152in}{1.125333in}}%
\pgfpathlineto{\pgfqpoint{4.318479in}{1.162667in}}%
\pgfpathlineto{\pgfqpoint{4.246949in}{1.260090in}}%
\pgfpathlineto{\pgfqpoint{4.166788in}{1.365530in}}%
\pgfpathlineto{\pgfqpoint{4.086626in}{1.467544in}}%
\pgfpathlineto{\pgfqpoint{4.000685in}{1.573333in}}%
\pgfpathlineto{\pgfqpoint{3.950387in}{1.633099in}}%
\pgfpathlineto{\pgfqpoint{3.906594in}{1.685333in}}%
\pgfpathlineto{\pgfqpoint{3.874608in}{1.722667in}}%
\pgfpathlineto{\pgfqpoint{3.776484in}{1.834667in}}%
\pgfpathlineto{\pgfqpoint{3.725899in}{1.891032in}}%
\pgfpathlineto{\pgfqpoint{3.640802in}{1.984000in}}%
\pgfpathlineto{\pgfqpoint{3.586806in}{2.041108in}}%
\pgfpathlineto{\pgfqpoint{3.565576in}{2.064160in}}%
\pgfpathlineto{\pgfqpoint{3.445333in}{2.188592in}}%
\pgfpathlineto{\pgfqpoint{3.377365in}{2.256690in}}%
\pgfpathlineto{\pgfqpoint{3.351642in}{2.282667in}}%
\pgfpathlineto{\pgfqpoint{3.236812in}{2.394667in}}%
\pgfpathlineto{\pgfqpoint{3.118083in}{2.506667in}}%
\pgfpathlineto{\pgfqpoint{3.004444in}{2.610336in}}%
\pgfpathlineto{\pgfqpoint{2.938071in}{2.668843in}}%
\pgfpathlineto{\pgfqpoint{2.896746in}{2.705017in}}%
\pgfpathlineto{\pgfqpoint{2.867547in}{2.730667in}}%
\pgfpathlineto{\pgfqpoint{2.763960in}{2.818566in}}%
\pgfpathlineto{\pgfqpoint{2.723879in}{2.851836in}}%
\pgfpathlineto{\pgfqpoint{2.643506in}{2.917333in}}%
\pgfpathlineto{\pgfqpoint{2.596495in}{2.954667in}}%
\pgfpathlineto{\pgfqpoint{2.483394in}{3.042264in}}%
\pgfpathlineto{\pgfqpoint{2.400973in}{3.104000in}}%
\pgfpathlineto{\pgfqpoint{2.282990in}{3.188731in}}%
\pgfpathlineto{\pgfqpoint{2.188891in}{3.253333in}}%
\pgfpathlineto{\pgfqpoint{2.122667in}{3.297249in}}%
\pgfpathlineto{\pgfqpoint{2.074802in}{3.328000in}}%
\pgfpathlineto{\pgfqpoint{2.002424in}{3.373067in}}%
\pgfpathlineto{\pgfqpoint{1.933886in}{3.413493in}}%
\pgfpathlineto{\pgfqpoint{1.909146in}{3.427782in}}%
\pgfpathlineto{\pgfqpoint{1.882182in}{3.443731in}}%
\pgfpathlineto{\pgfqpoint{1.808796in}{3.483645in}}%
\pgfpathlineto{\pgfqpoint{1.802020in}{3.487535in}}%
\pgfpathlineto{\pgfqpoint{1.761939in}{3.508487in}}%
\pgfpathlineto{\pgfqpoint{1.673317in}{3.552000in}}%
\pgfpathlineto{\pgfqpoint{1.539158in}{3.610177in}}%
\pgfpathlineto{\pgfqpoint{1.481374in}{3.631781in}}%
\pgfpathlineto{\pgfqpoint{1.361131in}{3.668865in}}%
\pgfpathlineto{\pgfqpoint{1.321051in}{3.678287in}}%
\pgfpathlineto{\pgfqpoint{1.300090in}{3.681810in}}%
\pgfpathlineto{\pgfqpoint{1.280970in}{3.686153in}}%
\pgfpathlineto{\pgfqpoint{1.240889in}{3.692198in}}%
\pgfpathlineto{\pgfqpoint{1.231597in}{3.692679in}}%
\pgfpathlineto{\pgfqpoint{1.195079in}{3.695997in}}%
\pgfpathlineto{\pgfqpoint{1.160727in}{3.697427in}}%
\pgfpathlineto{\pgfqpoint{1.120646in}{3.695671in}}%
\pgfpathlineto{\pgfqpoint{1.113085in}{3.694291in}}%
\pgfpathlineto{\pgfqpoint{1.080566in}{3.690144in}}%
\pgfpathlineto{\pgfqpoint{1.057763in}{3.685239in}}%
\pgfpathlineto{\pgfqpoint{1.040485in}{3.679951in}}%
\pgfpathlineto{\pgfqpoint{0.999895in}{3.663526in}}%
\pgfpathlineto{\pgfqpoint{0.953922in}{3.632629in}}%
\pgfpathlineto{\pgfqpoint{0.947838in}{3.626667in}}%
\pgfpathlineto{\pgfqpoint{0.915682in}{3.589333in}}%
\pgfpathlineto{\pgfqpoint{0.911489in}{3.581180in}}%
\pgfpathlineto{\pgfqpoint{0.890613in}{3.542265in}}%
\pgfpathlineto{\pgfqpoint{0.880162in}{3.511665in}}%
\pgfpathlineto{\pgfqpoint{0.872446in}{3.477333in}}%
\pgfpathlineto{\pgfqpoint{0.871810in}{3.469555in}}%
\pgfpathlineto{\pgfqpoint{0.867624in}{3.440000in}}%
\pgfpathlineto{\pgfqpoint{0.867538in}{3.428242in}}%
\pgfpathlineto{\pgfqpoint{0.865798in}{3.402667in}}%
\pgfpathlineto{\pgfqpoint{0.866484in}{3.365333in}}%
\pgfpathlineto{\pgfqpoint{0.867833in}{3.353850in}}%
\pgfpathlineto{\pgfqpoint{0.869295in}{3.328000in}}%
\pgfpathlineto{\pgfqpoint{0.874976in}{3.285836in}}%
\pgfpathlineto{\pgfqpoint{0.880162in}{3.253103in}}%
\pgfpathlineto{\pgfqpoint{0.897318in}{3.178667in}}%
\pgfpathlineto{\pgfqpoint{0.907580in}{3.141333in}}%
\pgfpathlineto{\pgfqpoint{0.910653in}{3.132401in}}%
\pgfpathlineto{\pgfqpoint{0.920242in}{3.099564in}}%
\pgfpathlineto{\pgfqpoint{0.944674in}{3.029333in}}%
\pgfpathlineto{\pgfqpoint{0.960323in}{2.987834in}}%
\pgfpathlineto{\pgfqpoint{0.992881in}{2.910326in}}%
\pgfpathlineto{\pgfqpoint{1.009831in}{2.871219in}}%
\pgfpathlineto{\pgfqpoint{1.040765in}{2.805333in}}%
\pgfpathlineto{\pgfqpoint{1.120646in}{2.651101in}}%
\pgfpathlineto{\pgfqpoint{1.226248in}{2.469333in}}%
\pgfpathlineto{\pgfqpoint{1.280970in}{2.381728in}}%
\pgfpathlineto{\pgfqpoint{1.422364in}{2.170667in}}%
\pgfpathlineto{\pgfqpoint{1.481374in}{2.087549in}}%
\pgfpathlineto{\pgfqpoint{1.561535in}{1.978652in}}%
\pgfpathlineto{\pgfqpoint{1.609130in}{1.916332in}}%
\pgfpathlineto{\pgfqpoint{1.642988in}{1.872000in}}%
\pgfpathlineto{\pgfqpoint{1.721859in}{1.772345in}}%
\pgfpathlineto{\pgfqpoint{1.802020in}{1.673966in}}%
\pgfpathlineto{\pgfqpoint{1.886435in}{1.573333in}}%
\pgfpathlineto{\pgfqpoint{1.938303in}{1.513608in}}%
\pgfpathlineto{\pgfqpoint{1.983515in}{1.461333in}}%
\pgfpathlineto{\pgfqpoint{2.088128in}{1.344171in}}%
\pgfpathlineto{\pgfqpoint{2.202828in}{1.220307in}}%
\pgfpathlineto{\pgfqpoint{2.257522in}{1.162667in}}%
\pgfpathlineto{\pgfqpoint{2.366181in}{1.050667in}}%
\pgfpathlineto{\pgfqpoint{2.483394in}{0.933849in}}%
\pgfpathlineto{\pgfqpoint{2.523475in}{0.894801in}}%
\pgfpathlineto{\pgfqpoint{2.643717in}{0.780247in}}%
\pgfpathlineto{\pgfqpoint{2.699296in}{0.729102in}}%
\pgfpathlineto{\pgfqpoint{2.739700in}{0.692070in}}%
\pgfpathlineto{\pgfqpoint{2.780340in}{0.655257in}}%
\pgfpathlineto{\pgfqpoint{2.821218in}{0.618667in}}%
\pgfpathlineto{\pgfqpoint{2.844121in}{0.597781in}}%
\pgfpathlineto{\pgfqpoint{2.924283in}{0.528000in}}%
\pgfpathlineto{\pgfqpoint{2.924283in}{0.528000in}}%
\pgfusepath{fill}%
\end{pgfscope}%
\begin{pgfscope}%
\pgfpathrectangle{\pgfqpoint{0.800000in}{0.528000in}}{\pgfqpoint{3.968000in}{3.696000in}}%
\pgfusepath{clip}%
\pgfsetbuttcap%
\pgfsetroundjoin%
\definecolor{currentfill}{rgb}{0.281924,0.089666,0.412415}%
\pgfsetfillcolor{currentfill}%
\pgfsetlinewidth{0.000000pt}%
\definecolor{currentstroke}{rgb}{0.000000,0.000000,0.000000}%
\pgfsetstrokecolor{currentstroke}%
\pgfsetdash{}{0pt}%
\pgfpathmoveto{\pgfqpoint{2.923923in}{0.528000in}}%
\pgfpathlineto{\pgfqpoint{2.796740in}{0.640000in}}%
\pgfpathlineto{\pgfqpoint{2.739700in}{0.692070in}}%
\pgfpathlineto{\pgfqpoint{2.699296in}{0.729102in}}%
\pgfpathlineto{\pgfqpoint{2.659125in}{0.766352in}}%
\pgfpathlineto{\pgfqpoint{2.634054in}{0.789333in}}%
\pgfpathlineto{\pgfqpoint{2.516756in}{0.901333in}}%
\pgfpathlineto{\pgfqpoint{2.461691in}{0.955784in}}%
\pgfpathlineto{\pgfqpoint{2.422871in}{0.994293in}}%
\pgfpathlineto{\pgfqpoint{2.402236in}{1.014261in}}%
\pgfpathlineto{\pgfqpoint{2.282990in}{1.136081in}}%
\pgfpathlineto{\pgfqpoint{2.186870in}{1.237333in}}%
\pgfpathlineto{\pgfqpoint{2.152053in}{1.274667in}}%
\pgfpathlineto{\pgfqpoint{2.049814in}{1.386667in}}%
\pgfpathlineto{\pgfqpoint{1.962343in}{1.485434in}}%
\pgfpathlineto{\pgfqpoint{1.882182in}{1.578331in}}%
\pgfpathlineto{\pgfqpoint{1.842101in}{1.625871in}}%
\pgfpathlineto{\pgfqpoint{1.761901in}{1.722667in}}%
\pgfpathlineto{\pgfqpoint{1.721859in}{1.772345in}}%
\pgfpathlineto{\pgfqpoint{1.641697in}{1.873665in}}%
\pgfpathlineto{\pgfqpoint{1.557515in}{1.984000in}}%
\pgfpathlineto{\pgfqpoint{1.510331in}{2.048306in}}%
\pgfpathlineto{\pgfqpoint{1.475267in}{2.096000in}}%
\pgfpathlineto{\pgfqpoint{1.422364in}{2.170667in}}%
\pgfpathlineto{\pgfqpoint{1.396325in}{2.208000in}}%
\pgfpathlineto{\pgfqpoint{1.345747in}{2.282667in}}%
\pgfpathlineto{\pgfqpoint{1.290989in}{2.366666in}}%
\pgfpathlineto{\pgfqpoint{1.260710in}{2.413537in}}%
\pgfpathlineto{\pgfqpoint{1.200808in}{2.511235in}}%
\pgfpathlineto{\pgfqpoint{1.097978in}{2.693333in}}%
\pgfpathlineto{\pgfqpoint{1.059402in}{2.768000in}}%
\pgfpathlineto{\pgfqpoint{1.053716in}{2.780325in}}%
\pgfpathlineto{\pgfqpoint{1.040485in}{2.805922in}}%
\pgfpathlineto{\pgfqpoint{0.973815in}{2.954667in}}%
\pgfpathlineto{\pgfqpoint{0.957653in}{2.994487in}}%
\pgfpathlineto{\pgfqpoint{0.931325in}{3.066667in}}%
\pgfpathlineto{\pgfqpoint{0.918765in}{3.104000in}}%
\pgfpathlineto{\pgfqpoint{0.890132in}{3.206713in}}%
\pgfpathlineto{\pgfqpoint{0.880113in}{3.253333in}}%
\pgfpathlineto{\pgfqpoint{0.869295in}{3.328000in}}%
\pgfpathlineto{\pgfqpoint{0.865798in}{3.402667in}}%
\pgfpathlineto{\pgfqpoint{0.867624in}{3.440000in}}%
\pgfpathlineto{\pgfqpoint{0.873423in}{3.483610in}}%
\pgfpathlineto{\pgfqpoint{0.880952in}{3.514667in}}%
\pgfpathlineto{\pgfqpoint{0.895151in}{3.552000in}}%
\pgfpathlineto{\pgfqpoint{0.920242in}{3.595741in}}%
\pgfpathlineto{\pgfqpoint{0.953922in}{3.632629in}}%
\pgfpathlineto{\pgfqpoint{0.960323in}{3.637620in}}%
\pgfpathlineto{\pgfqpoint{1.000670in}{3.664000in}}%
\pgfpathlineto{\pgfqpoint{1.040485in}{3.679951in}}%
\pgfpathlineto{\pgfqpoint{1.057763in}{3.685239in}}%
\pgfpathlineto{\pgfqpoint{1.080566in}{3.690144in}}%
\pgfpathlineto{\pgfqpoint{1.126132in}{3.696224in}}%
\pgfpathlineto{\pgfqpoint{1.160727in}{3.697427in}}%
\pgfpathlineto{\pgfqpoint{1.200808in}{3.696094in}}%
\pgfpathlineto{\pgfqpoint{1.280970in}{3.686153in}}%
\pgfpathlineto{\pgfqpoint{1.300090in}{3.681810in}}%
\pgfpathlineto{\pgfqpoint{1.321051in}{3.678287in}}%
\pgfpathlineto{\pgfqpoint{1.378782in}{3.664000in}}%
\pgfpathlineto{\pgfqpoint{1.410873in}{3.655001in}}%
\pgfpathlineto{\pgfqpoint{1.481374in}{3.631781in}}%
\pgfpathlineto{\pgfqpoint{1.561535in}{3.601100in}}%
\pgfpathlineto{\pgfqpoint{1.589771in}{3.589333in}}%
\pgfpathlineto{\pgfqpoint{1.681778in}{3.548093in}}%
\pgfpathlineto{\pgfqpoint{1.749638in}{3.514667in}}%
\pgfpathlineto{\pgfqpoint{1.802020in}{3.487535in}}%
\pgfpathlineto{\pgfqpoint{1.820962in}{3.477333in}}%
\pgfpathlineto{\pgfqpoint{1.888686in}{3.440000in}}%
\pgfpathlineto{\pgfqpoint{2.082586in}{3.323099in}}%
\pgfpathlineto{\pgfqpoint{2.150017in}{3.278809in}}%
\pgfpathlineto{\pgfqpoint{2.188891in}{3.253333in}}%
\pgfpathlineto{\pgfqpoint{2.243950in}{3.216000in}}%
\pgfpathlineto{\pgfqpoint{2.363152in}{3.131547in}}%
\pgfpathlineto{\pgfqpoint{2.451098in}{3.066667in}}%
\pgfpathlineto{\pgfqpoint{2.563556in}{2.980524in}}%
\pgfpathlineto{\pgfqpoint{2.643717in}{2.917165in}}%
\pgfpathlineto{\pgfqpoint{2.689545in}{2.880000in}}%
\pgfpathlineto{\pgfqpoint{2.804040in}{2.784891in}}%
\pgfpathlineto{\pgfqpoint{2.924283in}{2.681405in}}%
\pgfpathlineto{\pgfqpoint{2.979156in}{2.632445in}}%
\pgfpathlineto{\pgfqpoint{3.004444in}{2.610336in}}%
\pgfpathlineto{\pgfqpoint{3.124687in}{2.500555in}}%
\pgfpathlineto{\pgfqpoint{3.164768in}{2.463102in}}%
\pgfpathlineto{\pgfqpoint{3.285010in}{2.348125in}}%
\pgfpathlineto{\pgfqpoint{3.325091in}{2.308915in}}%
\pgfpathlineto{\pgfqpoint{3.445333in}{2.188592in}}%
\pgfpathlineto{\pgfqpoint{3.565576in}{2.064160in}}%
\pgfpathlineto{\pgfqpoint{3.643047in}{1.981494in}}%
\pgfpathlineto{\pgfqpoint{3.675204in}{1.946667in}}%
\pgfpathlineto{\pgfqpoint{3.765980in}{1.846458in}}%
\pgfpathlineto{\pgfqpoint{3.825737in}{1.778328in}}%
\pgfpathlineto{\pgfqpoint{3.846141in}{1.755629in}}%
\pgfpathlineto{\pgfqpoint{3.938300in}{1.648000in}}%
\pgfpathlineto{\pgfqpoint{3.985443in}{1.591086in}}%
\pgfpathlineto{\pgfqpoint{4.020300in}{1.548887in}}%
\pgfpathlineto{\pgfqpoint{4.061543in}{1.498667in}}%
\pgfpathlineto{\pgfqpoint{4.126707in}{1.416984in}}%
\pgfpathlineto{\pgfqpoint{4.179278in}{1.349333in}}%
\pgfpathlineto{\pgfqpoint{4.246949in}{1.260090in}}%
\pgfpathlineto{\pgfqpoint{4.407273in}{1.036515in}}%
\pgfpathlineto{\pgfqpoint{4.473074in}{0.938667in}}%
\pgfpathlineto{\pgfqpoint{4.508421in}{0.883548in}}%
\pgfpathlineto{\pgfqpoint{4.527515in}{0.854486in}}%
\pgfpathlineto{\pgfqpoint{4.647758in}{0.654318in}}%
\pgfpathlineto{\pgfqpoint{4.716377in}{0.528000in}}%
\pgfpathlineto{\pgfqpoint{4.727109in}{0.528000in}}%
\pgfpathlineto{\pgfqpoint{4.707126in}{0.565333in}}%
\pgfpathlineto{\pgfqpoint{4.700398in}{0.577032in}}%
\pgfpathlineto{\pgfqpoint{4.686154in}{0.604235in}}%
\pgfpathlineto{\pgfqpoint{4.622859in}{0.714667in}}%
\pgfpathlineto{\pgfqpoint{4.567596in}{0.805969in}}%
\pgfpathlineto{\pgfqpoint{4.507012in}{0.901333in}}%
\pgfpathlineto{\pgfqpoint{4.447354in}{0.991442in}}%
\pgfpathlineto{\pgfqpoint{4.380817in}{1.088000in}}%
\pgfpathlineto{\pgfqpoint{4.327111in}{1.163615in}}%
\pgfpathlineto{\pgfqpoint{4.241335in}{1.279897in}}%
\pgfpathlineto{\pgfqpoint{4.159453in}{1.386667in}}%
\pgfpathlineto{\pgfqpoint{4.111410in}{1.447085in}}%
\pgfpathlineto{\pgfqpoint{4.070526in}{1.498667in}}%
\pgfpathlineto{\pgfqpoint{4.006465in}{1.577240in}}%
\pgfpathlineto{\pgfqpoint{3.955480in}{1.637843in}}%
\pgfpathlineto{\pgfqpoint{3.915609in}{1.685333in}}%
\pgfpathlineto{\pgfqpoint{3.873319in}{1.734686in}}%
\pgfpathlineto{\pgfqpoint{3.765980in}{1.856530in}}%
\pgfpathlineto{\pgfqpoint{3.703170in}{1.925496in}}%
\pgfpathlineto{\pgfqpoint{3.675919in}{1.955887in}}%
\pgfpathlineto{\pgfqpoint{3.565576in}{2.073747in}}%
\pgfpathlineto{\pgfqpoint{3.445333in}{2.198111in}}%
\pgfpathlineto{\pgfqpoint{3.382192in}{2.261187in}}%
\pgfpathlineto{\pgfqpoint{3.361201in}{2.282667in}}%
\pgfpathlineto{\pgfqpoint{3.244929in}{2.396241in}}%
\pgfpathlineto{\pgfqpoint{3.124687in}{2.509785in}}%
\pgfpathlineto{\pgfqpoint{3.004444in}{2.619579in}}%
\pgfpathlineto{\pgfqpoint{2.943082in}{2.673510in}}%
\pgfpathlineto{\pgfqpoint{2.901774in}{2.709701in}}%
\pgfpathlineto{\pgfqpoint{2.878235in}{2.730667in}}%
\pgfpathlineto{\pgfqpoint{2.763960in}{2.827714in}}%
\pgfpathlineto{\pgfqpoint{2.723879in}{2.860963in}}%
\pgfpathlineto{\pgfqpoint{2.603636in}{2.958306in}}%
\pgfpathlineto{\pgfqpoint{2.512514in}{3.029333in}}%
\pgfpathlineto{\pgfqpoint{2.403232in}{3.111686in}}%
\pgfpathlineto{\pgfqpoint{2.310717in}{3.178667in}}%
\pgfpathlineto{\pgfqpoint{2.109140in}{3.315401in}}%
\pgfpathlineto{\pgfqpoint{2.065303in}{3.344098in}}%
\pgfpathlineto{\pgfqpoint{1.962343in}{3.407526in}}%
\pgfpathlineto{\pgfqpoint{1.891378in}{3.448566in}}%
\pgfpathlineto{\pgfqpoint{1.866320in}{3.462559in}}%
\pgfpathlineto{\pgfqpoint{1.841004in}{3.477333in}}%
\pgfpathlineto{\pgfqpoint{1.681778in}{3.559416in}}%
\pgfpathlineto{\pgfqpoint{1.633389in}{3.581595in}}%
\pgfpathlineto{\pgfqpoint{1.601616in}{3.596181in}}%
\pgfpathlineto{\pgfqpoint{1.523460in}{3.628535in}}%
\pgfpathlineto{\pgfqpoint{1.516328in}{3.631442in}}%
\pgfpathlineto{\pgfqpoint{1.441293in}{3.658452in}}%
\pgfpathlineto{\pgfqpoint{1.401212in}{3.671218in}}%
\pgfpathlineto{\pgfqpoint{1.313234in}{3.694053in}}%
\pgfpathlineto{\pgfqpoint{1.280970in}{3.701379in}}%
\pgfpathlineto{\pgfqpoint{1.232511in}{3.709137in}}%
\pgfpathlineto{\pgfqpoint{1.200808in}{3.712441in}}%
\pgfpathlineto{\pgfqpoint{1.134030in}{3.713800in}}%
\pgfpathlineto{\pgfqpoint{1.107623in}{3.713464in}}%
\pgfpathlineto{\pgfqpoint{1.080566in}{3.710266in}}%
\pgfpathlineto{\pgfqpoint{1.037137in}{3.701333in}}%
\pgfpathlineto{\pgfqpoint{1.000404in}{3.688297in}}%
\pgfpathlineto{\pgfqpoint{0.982913in}{3.680292in}}%
\pgfpathlineto{\pgfqpoint{0.956048in}{3.664000in}}%
\pgfpathlineto{\pgfqpoint{0.936557in}{3.648804in}}%
\pgfpathlineto{\pgfqpoint{0.914753in}{3.626667in}}%
\pgfpathlineto{\pgfqpoint{0.888671in}{3.589333in}}%
\pgfpathlineto{\pgfqpoint{0.873449in}{3.558252in}}%
\pgfpathlineto{\pgfqpoint{0.868378in}{3.541024in}}%
\pgfpathlineto{\pgfqpoint{0.859963in}{3.514667in}}%
\pgfpathlineto{\pgfqpoint{0.856328in}{3.499533in}}%
\pgfpathlineto{\pgfqpoint{0.852872in}{3.477333in}}%
\pgfpathlineto{\pgfqpoint{0.849233in}{3.440000in}}%
\pgfpathlineto{\pgfqpoint{0.848436in}{3.394884in}}%
\pgfpathlineto{\pgfqpoint{0.850077in}{3.365333in}}%
\pgfpathlineto{\pgfqpoint{0.859113in}{3.290667in}}%
\pgfpathlineto{\pgfqpoint{0.862666in}{3.274370in}}%
\pgfpathlineto{\pgfqpoint{0.865991in}{3.253333in}}%
\pgfpathlineto{\pgfqpoint{0.875475in}{3.211634in}}%
\pgfpathlineto{\pgfqpoint{0.884766in}{3.174378in}}%
\pgfpathlineto{\pgfqpoint{0.906226in}{3.104000in}}%
\pgfpathlineto{\pgfqpoint{0.909871in}{3.094339in}}%
\pgfpathlineto{\pgfqpoint{0.920242in}{3.062502in}}%
\pgfpathlineto{\pgfqpoint{0.946938in}{2.992000in}}%
\pgfpathlineto{\pgfqpoint{0.950852in}{2.983178in}}%
\pgfpathlineto{\pgfqpoint{0.962999in}{2.952174in}}%
\pgfpathlineto{\pgfqpoint{1.000404in}{2.867656in}}%
\pgfpathlineto{\pgfqpoint{1.040485in}{2.784230in}}%
\pgfpathlineto{\pgfqpoint{1.107750in}{2.656000in}}%
\pgfpathlineto{\pgfqpoint{1.160727in}{2.562424in}}%
\pgfpathlineto{\pgfqpoint{1.216441in}{2.469333in}}%
\pgfpathlineto{\pgfqpoint{1.263126in}{2.394667in}}%
\pgfpathlineto{\pgfqpoint{1.321051in}{2.305495in}}%
\pgfpathlineto{\pgfqpoint{1.387078in}{2.208000in}}%
\pgfpathlineto{\pgfqpoint{1.441293in}{2.130499in}}%
\pgfpathlineto{\pgfqpoint{1.521455in}{2.020098in}}%
\pgfpathlineto{\pgfqpoint{1.570581in}{1.955093in}}%
\pgfpathlineto{\pgfqpoint{1.605111in}{1.909333in}}%
\pgfpathlineto{\pgfqpoint{1.681778in}{1.811304in}}%
\pgfpathlineto{\pgfqpoint{1.761939in}{1.711783in}}%
\pgfpathlineto{\pgfqpoint{1.845825in}{1.610667in}}%
\pgfpathlineto{\pgfqpoint{1.897616in}{1.550376in}}%
\pgfpathlineto{\pgfqpoint{1.941861in}{1.498667in}}%
\pgfpathlineto{\pgfqpoint{2.042505in}{1.384660in}}%
\pgfpathlineto{\pgfqpoint{2.082586in}{1.340408in}}%
\pgfpathlineto{\pgfqpoint{2.202828in}{1.210559in}}%
\pgfpathlineto{\pgfqpoint{2.248229in}{1.162667in}}%
\pgfpathlineto{\pgfqpoint{2.356894in}{1.050667in}}%
\pgfpathlineto{\pgfqpoint{2.431303in}{0.976000in}}%
\pgfpathlineto{\pgfqpoint{2.523475in}{0.885573in}}%
\pgfpathlineto{\pgfqpoint{2.643717in}{0.771082in}}%
\pgfpathlineto{\pgfqpoint{2.694279in}{0.724430in}}%
\pgfpathlineto{\pgfqpoint{2.745342in}{0.677333in}}%
\pgfpathlineto{\pgfqpoint{2.844121in}{0.588720in}}%
\pgfpathlineto{\pgfqpoint{2.913524in}{0.528000in}}%
\pgfpathlineto{\pgfqpoint{2.913524in}{0.528000in}}%
\pgfusepath{fill}%
\end{pgfscope}%
\begin{pgfscope}%
\pgfpathrectangle{\pgfqpoint{0.800000in}{0.528000in}}{\pgfqpoint{3.968000in}{3.696000in}}%
\pgfusepath{clip}%
\pgfsetbuttcap%
\pgfsetroundjoin%
\definecolor{currentfill}{rgb}{0.282327,0.094955,0.417331}%
\pgfsetfillcolor{currentfill}%
\pgfsetlinewidth{0.000000pt}%
\definecolor{currentstroke}{rgb}{0.000000,0.000000,0.000000}%
\pgfsetstrokecolor{currentstroke}%
\pgfsetdash{}{0pt}%
\pgfpathmoveto{\pgfqpoint{2.913524in}{0.528000in}}%
\pgfpathlineto{\pgfqpoint{2.804040in}{0.624360in}}%
\pgfpathlineto{\pgfqpoint{2.683798in}{0.733772in}}%
\pgfpathlineto{\pgfqpoint{2.563556in}{0.846979in}}%
\pgfpathlineto{\pgfqpoint{2.495798in}{0.912887in}}%
\pgfpathlineto{\pgfqpoint{2.456776in}{0.951207in}}%
\pgfpathlineto{\pgfqpoint{2.417973in}{0.989730in}}%
\pgfpathlineto{\pgfqpoint{2.393909in}{1.013333in}}%
\pgfpathlineto{\pgfqpoint{2.282990in}{1.126380in}}%
\pgfpathlineto{\pgfqpoint{2.162747in}{1.253361in}}%
\pgfpathlineto{\pgfqpoint{2.074486in}{1.349333in}}%
\pgfpathlineto{\pgfqpoint{2.023629in}{1.406417in}}%
\pgfpathlineto{\pgfqpoint{2.002424in}{1.429612in}}%
\pgfpathlineto{\pgfqpoint{1.909524in}{1.536000in}}%
\pgfpathlineto{\pgfqpoint{1.862159in}{1.592017in}}%
\pgfpathlineto{\pgfqpoint{1.842101in}{1.615092in}}%
\pgfpathlineto{\pgfqpoint{1.753052in}{1.722667in}}%
\pgfpathlineto{\pgfqpoint{1.721859in}{1.761013in}}%
\pgfpathlineto{\pgfqpoint{1.613932in}{1.897862in}}%
\pgfpathlineto{\pgfqpoint{1.548509in}{1.984000in}}%
\pgfpathlineto{\pgfqpoint{1.504953in}{2.043296in}}%
\pgfpathlineto{\pgfqpoint{1.472430in}{2.087669in}}%
\pgfpathlineto{\pgfqpoint{1.439268in}{2.133333in}}%
\pgfpathlineto{\pgfqpoint{1.387078in}{2.208000in}}%
\pgfpathlineto{\pgfqpoint{1.321051in}{2.305495in}}%
\pgfpathlineto{\pgfqpoint{1.216441in}{2.469333in}}%
\pgfpathlineto{\pgfqpoint{1.160727in}{2.562424in}}%
\pgfpathlineto{\pgfqpoint{1.120646in}{2.632672in}}%
\pgfpathlineto{\pgfqpoint{1.080566in}{2.706311in}}%
\pgfpathlineto{\pgfqpoint{1.040485in}{2.784230in}}%
\pgfpathlineto{\pgfqpoint{0.994687in}{2.880000in}}%
\pgfpathlineto{\pgfqpoint{0.978141in}{2.917333in}}%
\pgfpathlineto{\pgfqpoint{0.960323in}{2.958735in}}%
\pgfpathlineto{\pgfqpoint{0.918721in}{3.066667in}}%
\pgfpathlineto{\pgfqpoint{0.883652in}{3.178667in}}%
\pgfpathlineto{\pgfqpoint{0.872344in}{3.223282in}}%
\pgfpathlineto{\pgfqpoint{0.859113in}{3.290667in}}%
\pgfpathlineto{\pgfqpoint{0.853728in}{3.328000in}}%
\pgfpathlineto{\pgfqpoint{0.853045in}{3.340075in}}%
\pgfpathlineto{\pgfqpoint{0.850077in}{3.365333in}}%
\pgfpathlineto{\pgfqpoint{0.848456in}{3.402667in}}%
\pgfpathlineto{\pgfqpoint{0.849023in}{3.410995in}}%
\pgfpathlineto{\pgfqpoint{0.849233in}{3.440000in}}%
\pgfpathlineto{\pgfqpoint{0.850625in}{3.449821in}}%
\pgfpathlineto{\pgfqpoint{0.852872in}{3.477333in}}%
\pgfpathlineto{\pgfqpoint{0.856328in}{3.499533in}}%
\pgfpathlineto{\pgfqpoint{0.859963in}{3.514667in}}%
\pgfpathlineto{\pgfqpoint{0.873449in}{3.558252in}}%
\pgfpathlineto{\pgfqpoint{0.888671in}{3.589333in}}%
\pgfpathlineto{\pgfqpoint{0.920242in}{3.632814in}}%
\pgfpathlineto{\pgfqpoint{0.936557in}{3.648804in}}%
\pgfpathlineto{\pgfqpoint{0.960323in}{3.666997in}}%
\pgfpathlineto{\pgfqpoint{0.982913in}{3.680292in}}%
\pgfpathlineto{\pgfqpoint{1.000404in}{3.688297in}}%
\pgfpathlineto{\pgfqpoint{1.041768in}{3.702529in}}%
\pgfpathlineto{\pgfqpoint{1.080566in}{3.710266in}}%
\pgfpathlineto{\pgfqpoint{1.120646in}{3.714132in}}%
\pgfpathlineto{\pgfqpoint{1.134030in}{3.713800in}}%
\pgfpathlineto{\pgfqpoint{1.160727in}{3.714668in}}%
\pgfpathlineto{\pgfqpoint{1.173721in}{3.713436in}}%
\pgfpathlineto{\pgfqpoint{1.200808in}{3.712441in}}%
\pgfpathlineto{\pgfqpoint{1.246624in}{3.706675in}}%
\pgfpathlineto{\pgfqpoint{1.281184in}{3.701333in}}%
\pgfpathlineto{\pgfqpoint{1.361131in}{3.682616in}}%
\pgfpathlineto{\pgfqpoint{1.375997in}{3.677847in}}%
\pgfpathlineto{\pgfqpoint{1.406796in}{3.669202in}}%
\pgfpathlineto{\pgfqpoint{1.441293in}{3.658452in}}%
\pgfpathlineto{\pgfqpoint{1.561535in}{3.613176in}}%
\pgfpathlineto{\pgfqpoint{1.578715in}{3.605335in}}%
\pgfpathlineto{\pgfqpoint{1.616887in}{3.589333in}}%
\pgfpathlineto{\pgfqpoint{1.721859in}{3.539772in}}%
\pgfpathlineto{\pgfqpoint{1.738947in}{3.530584in}}%
\pgfpathlineto{\pgfqpoint{1.771114in}{3.514667in}}%
\pgfpathlineto{\pgfqpoint{1.922263in}{3.431142in}}%
\pgfpathlineto{\pgfqpoint{1.989481in}{3.390611in}}%
\pgfpathlineto{\pgfqpoint{2.031328in}{3.365333in}}%
\pgfpathlineto{\pgfqpoint{2.162747in}{3.280650in}}%
\pgfpathlineto{\pgfqpoint{2.257616in}{3.216000in}}%
\pgfpathlineto{\pgfqpoint{2.364110in}{3.140440in}}%
\pgfpathlineto{\pgfqpoint{2.483394in}{3.051585in}}%
\pgfpathlineto{\pgfqpoint{2.603636in}{2.958306in}}%
\pgfpathlineto{\pgfqpoint{2.670054in}{2.904531in}}%
\pgfpathlineto{\pgfqpoint{2.700671in}{2.880000in}}%
\pgfpathlineto{\pgfqpoint{2.804040in}{2.794060in}}%
\pgfpathlineto{\pgfqpoint{2.924283in}{2.690638in}}%
\pgfpathlineto{\pgfqpoint{2.984150in}{2.637096in}}%
\pgfpathlineto{\pgfqpoint{3.005457in}{2.618667in}}%
\pgfpathlineto{\pgfqpoint{3.128033in}{2.506667in}}%
\pgfpathlineto{\pgfqpoint{3.167977in}{2.469333in}}%
\pgfpathlineto{\pgfqpoint{3.285228in}{2.357333in}}%
\pgfpathlineto{\pgfqpoint{3.325091in}{2.318366in}}%
\pgfpathlineto{\pgfqpoint{3.445333in}{2.198111in}}%
\pgfpathlineto{\pgfqpoint{3.565576in}{2.073747in}}%
\pgfpathlineto{\pgfqpoint{3.629072in}{2.005810in}}%
\pgfpathlineto{\pgfqpoint{3.675919in}{1.955887in}}%
\pgfpathlineto{\pgfqpoint{3.765980in}{1.856530in}}%
\pgfpathlineto{\pgfqpoint{3.830710in}{1.782960in}}%
\pgfpathlineto{\pgfqpoint{3.873319in}{1.734686in}}%
\pgfpathlineto{\pgfqpoint{3.947241in}{1.648000in}}%
\pgfpathlineto{\pgfqpoint{3.990518in}{1.595813in}}%
\pgfpathlineto{\pgfqpoint{4.025357in}{1.553598in}}%
\pgfpathlineto{\pgfqpoint{4.046545in}{1.528337in}}%
\pgfpathlineto{\pgfqpoint{4.130231in}{1.424000in}}%
\pgfpathlineto{\pgfqpoint{4.166788in}{1.377234in}}%
\pgfpathlineto{\pgfqpoint{4.246949in}{1.272411in}}%
\pgfpathlineto{\pgfqpoint{4.294890in}{1.207321in}}%
\pgfpathlineto{\pgfqpoint{4.327799in}{1.162667in}}%
\pgfpathlineto{\pgfqpoint{4.367192in}{1.107349in}}%
\pgfpathlineto{\pgfqpoint{4.432475in}{1.013333in}}%
\pgfpathlineto{\pgfqpoint{4.487434in}{0.931425in}}%
\pgfpathlineto{\pgfqpoint{4.554682in}{0.826667in}}%
\pgfpathlineto{\pgfqpoint{4.607677in}{0.740269in}}%
\pgfpathlineto{\pgfqpoint{4.665986in}{0.640000in}}%
\pgfpathlineto{\pgfqpoint{4.707126in}{0.565333in}}%
\pgfpathlineto{\pgfqpoint{4.727919in}{0.528000in}}%
\pgfpathlineto{\pgfqpoint{4.737412in}{0.528000in}}%
\pgfpathlineto{\pgfqpoint{4.707148in}{0.583319in}}%
\pgfpathlineto{\pgfqpoint{4.687838in}{0.619191in}}%
\pgfpathlineto{\pgfqpoint{4.647758in}{0.689238in}}%
\pgfpathlineto{\pgfqpoint{4.527515in}{0.884394in}}%
\pgfpathlineto{\pgfqpoint{4.390006in}{1.088000in}}%
\pgfpathlineto{\pgfqpoint{4.327111in}{1.176012in}}%
\pgfpathlineto{\pgfqpoint{4.246949in}{1.284285in}}%
\pgfpathlineto{\pgfqpoint{4.166788in}{1.388842in}}%
\pgfpathlineto{\pgfqpoint{4.116611in}{1.451929in}}%
\pgfpathlineto{\pgfqpoint{4.079509in}{1.498667in}}%
\pgfpathlineto{\pgfqpoint{4.006465in}{1.587879in}}%
\pgfpathlineto{\pgfqpoint{3.956182in}{1.648000in}}%
\pgfpathlineto{\pgfqpoint{3.860130in}{1.760000in}}%
\pgfpathlineto{\pgfqpoint{3.761132in}{1.872000in}}%
\pgfpathlineto{\pgfqpoint{3.708028in}{1.930021in}}%
\pgfpathlineto{\pgfqpoint{3.658850in}{1.984000in}}%
\pgfpathlineto{\pgfqpoint{3.565576in}{2.083333in}}%
\pgfpathlineto{\pgfqpoint{3.444967in}{2.208000in}}%
\pgfpathlineto{\pgfqpoint{3.332757in}{2.320000in}}%
\pgfpathlineto{\pgfqpoint{3.244929in}{2.405324in}}%
\pgfpathlineto{\pgfqpoint{3.124687in}{2.518807in}}%
\pgfpathlineto{\pgfqpoint{3.004444in}{2.628541in}}%
\pgfpathlineto{\pgfqpoint{2.948092in}{2.678177in}}%
\pgfpathlineto{\pgfqpoint{2.906802in}{2.714384in}}%
\pgfpathlineto{\pgfqpoint{2.884202in}{2.734599in}}%
\pgfpathlineto{\pgfqpoint{2.763960in}{2.836862in}}%
\pgfpathlineto{\pgfqpoint{2.696674in}{2.891993in}}%
\pgfpathlineto{\pgfqpoint{2.666000in}{2.917333in}}%
\pgfpathlineto{\pgfqpoint{2.563556in}{2.999005in}}%
\pgfpathlineto{\pgfqpoint{2.475770in}{3.066667in}}%
\pgfpathlineto{\pgfqpoint{2.403232in}{3.120964in}}%
\pgfpathlineto{\pgfqpoint{2.323071in}{3.179449in}}%
\pgfpathlineto{\pgfqpoint{2.271283in}{3.216000in}}%
\pgfpathlineto{\pgfqpoint{2.162465in}{3.290667in}}%
\pgfpathlineto{\pgfqpoint{2.091799in}{3.336582in}}%
\pgfpathlineto{\pgfqpoint{2.082586in}{3.342844in}}%
\pgfpathlineto{\pgfqpoint{1.986981in}{3.402667in}}%
\pgfpathlineto{\pgfqpoint{1.842101in}{3.487162in}}%
\pgfpathlineto{\pgfqpoint{1.798129in}{3.511042in}}%
\pgfpathlineto{\pgfqpoint{1.761939in}{3.530268in}}%
\pgfpathlineto{\pgfqpoint{1.746564in}{3.537678in}}%
\pgfpathlineto{\pgfqpoint{1.719744in}{3.552000in}}%
\pgfpathlineto{\pgfqpoint{1.640751in}{3.590215in}}%
\pgfpathlineto{\pgfqpoint{1.558049in}{3.626667in}}%
\pgfpathlineto{\pgfqpoint{1.441293in}{3.671250in}}%
\pgfpathlineto{\pgfqpoint{1.321051in}{3.706888in}}%
\pgfpathlineto{\pgfqpoint{1.280970in}{3.715780in}}%
\pgfpathlineto{\pgfqpoint{1.190277in}{3.728857in}}%
\pgfpathlineto{\pgfqpoint{1.152931in}{3.731405in}}%
\pgfpathlineto{\pgfqpoint{1.120646in}{3.732206in}}%
\pgfpathlineto{\pgfqpoint{1.080566in}{3.729571in}}%
\pgfpathlineto{\pgfqpoint{1.068141in}{3.727094in}}%
\pgfpathlineto{\pgfqpoint{1.040485in}{3.723058in}}%
\pgfpathlineto{\pgfqpoint{1.021829in}{3.718710in}}%
\pgfpathlineto{\pgfqpoint{1.000404in}{3.711748in}}%
\pgfpathlineto{\pgfqpoint{0.965338in}{3.696662in}}%
\pgfpathlineto{\pgfqpoint{0.960323in}{3.693711in}}%
\pgfpathlineto{\pgfqpoint{0.918256in}{3.664000in}}%
\pgfpathlineto{\pgfqpoint{0.899986in}{3.645535in}}%
\pgfpathlineto{\pgfqpoint{0.880162in}{3.619352in}}%
\pgfpathlineto{\pgfqpoint{0.868590in}{3.600112in}}%
\pgfpathlineto{\pgfqpoint{0.856489in}{3.574050in}}%
\pgfpathlineto{\pgfqpoint{0.848806in}{3.552000in}}%
\pgfpathlineto{\pgfqpoint{0.839046in}{3.513703in}}%
\pgfpathlineto{\pgfqpoint{0.833815in}{3.477333in}}%
\pgfpathlineto{\pgfqpoint{0.831506in}{3.440000in}}%
\pgfpathlineto{\pgfqpoint{0.831723in}{3.402667in}}%
\pgfpathlineto{\pgfqpoint{0.832511in}{3.395616in}}%
\pgfpathlineto{\pgfqpoint{0.834084in}{3.365333in}}%
\pgfpathlineto{\pgfqpoint{0.840081in}{3.316727in}}%
\pgfpathlineto{\pgfqpoint{0.844304in}{3.290667in}}%
\pgfpathlineto{\pgfqpoint{0.860666in}{3.216000in}}%
\pgfpathlineto{\pgfqpoint{0.864932in}{3.201814in}}%
\pgfpathlineto{\pgfqpoint{0.872837in}{3.171844in}}%
\pgfpathlineto{\pgfqpoint{0.881424in}{3.141333in}}%
\pgfpathlineto{\pgfqpoint{0.920358in}{3.029225in}}%
\pgfpathlineto{\pgfqpoint{0.960323in}{2.932115in}}%
\pgfpathlineto{\pgfqpoint{0.983704in}{2.880000in}}%
\pgfpathlineto{\pgfqpoint{0.989024in}{2.869400in}}%
\pgfpathlineto{\pgfqpoint{1.001002in}{2.842667in}}%
\pgfpathlineto{\pgfqpoint{1.080566in}{2.686888in}}%
\pgfpathlineto{\pgfqpoint{1.183968in}{2.506667in}}%
\pgfpathlineto{\pgfqpoint{1.206633in}{2.469333in}}%
\pgfpathlineto{\pgfqpoint{1.253541in}{2.394667in}}%
\pgfpathlineto{\pgfqpoint{1.302077in}{2.320000in}}%
\pgfpathlineto{\pgfqpoint{1.326839in}{2.282667in}}%
\pgfpathlineto{\pgfqpoint{1.377831in}{2.208000in}}%
\pgfpathlineto{\pgfqpoint{1.441293in}{2.117830in}}%
\pgfpathlineto{\pgfqpoint{1.521455in}{2.008081in}}%
\pgfpathlineto{\pgfqpoint{1.567617in}{1.946667in}}%
\pgfpathlineto{\pgfqpoint{1.641697in}{1.850800in}}%
\pgfpathlineto{\pgfqpoint{1.721859in}{1.750105in}}%
\pgfpathlineto{\pgfqpoint{1.786879in}{1.671230in}}%
\pgfpathlineto{\pgfqpoint{1.818282in}{1.632853in}}%
\pgfpathlineto{\pgfqpoint{1.922263in}{1.510916in}}%
\pgfpathlineto{\pgfqpoint{1.979683in}{1.445182in}}%
\pgfpathlineto{\pgfqpoint{2.082586in}{1.330588in}}%
\pgfpathlineto{\pgfqpoint{2.202828in}{1.200811in}}%
\pgfpathlineto{\pgfqpoint{2.282990in}{1.116985in}}%
\pgfpathlineto{\pgfqpoint{2.403232in}{0.994673in}}%
\pgfpathlineto{\pgfqpoint{2.471406in}{0.927500in}}%
\pgfpathlineto{\pgfqpoint{2.497775in}{0.901333in}}%
\pgfpathlineto{\pgfqpoint{2.603636in}{0.799631in}}%
\pgfpathlineto{\pgfqpoint{2.723879in}{0.687760in}}%
\pgfpathlineto{\pgfqpoint{2.776522in}{0.640000in}}%
\pgfpathlineto{\pgfqpoint{2.884202in}{0.544450in}}%
\pgfpathlineto{\pgfqpoint{2.903125in}{0.528000in}}%
\pgfpathlineto{\pgfqpoint{2.903125in}{0.528000in}}%
\pgfusepath{fill}%
\end{pgfscope}%
\begin{pgfscope}%
\pgfpathrectangle{\pgfqpoint{0.800000in}{0.528000in}}{\pgfqpoint{3.968000in}{3.696000in}}%
\pgfusepath{clip}%
\pgfsetbuttcap%
\pgfsetroundjoin%
\definecolor{currentfill}{rgb}{0.282327,0.094955,0.417331}%
\pgfsetfillcolor{currentfill}%
\pgfsetlinewidth{0.000000pt}%
\definecolor{currentstroke}{rgb}{0.000000,0.000000,0.000000}%
\pgfsetstrokecolor{currentstroke}%
\pgfsetdash{}{0pt}%
\pgfpathmoveto{\pgfqpoint{2.903125in}{0.528000in}}%
\pgfpathlineto{\pgfqpoint{2.804040in}{0.615279in}}%
\pgfpathlineto{\pgfqpoint{2.683798in}{0.724628in}}%
\pgfpathlineto{\pgfqpoint{2.563556in}{0.837772in}}%
\pgfpathlineto{\pgfqpoint{2.523475in}{0.876344in}}%
\pgfpathlineto{\pgfqpoint{2.403232in}{0.994673in}}%
\pgfpathlineto{\pgfqpoint{2.282990in}{1.116985in}}%
\pgfpathlineto{\pgfqpoint{2.239085in}{1.162667in}}%
\pgfpathlineto{\pgfqpoint{2.133932in}{1.274667in}}%
\pgfpathlineto{\pgfqpoint{2.042505in}{1.374816in}}%
\pgfpathlineto{\pgfqpoint{1.962343in}{1.464879in}}%
\pgfpathlineto{\pgfqpoint{1.868660in}{1.573333in}}%
\pgfpathlineto{\pgfqpoint{1.836951in}{1.610667in}}%
\pgfpathlineto{\pgfqpoint{1.761939in}{1.700945in}}%
\pgfpathlineto{\pgfqpoint{1.681778in}{1.799940in}}%
\pgfpathlineto{\pgfqpoint{1.632278in}{1.863227in}}%
\pgfpathlineto{\pgfqpoint{1.596163in}{1.909333in}}%
\pgfpathlineto{\pgfqpoint{1.521455in}{2.008081in}}%
\pgfpathlineto{\pgfqpoint{1.457017in}{2.096000in}}%
\pgfpathlineto{\pgfqpoint{1.401212in}{2.174244in}}%
\pgfpathlineto{\pgfqpoint{1.321051in}{2.291336in}}%
\pgfpathlineto{\pgfqpoint{1.183968in}{2.506667in}}%
\pgfpathlineto{\pgfqpoint{1.160727in}{2.545269in}}%
\pgfpathlineto{\pgfqpoint{1.097562in}{2.656000in}}%
\pgfpathlineto{\pgfqpoint{1.078286in}{2.691210in}}%
\pgfpathlineto{\pgfqpoint{1.073462in}{2.699950in}}%
\pgfpathlineto{\pgfqpoint{1.019344in}{2.805333in}}%
\pgfpathlineto{\pgfqpoint{1.013714in}{2.817731in}}%
\pgfpathlineto{\pgfqpoint{1.000404in}{2.843944in}}%
\pgfpathlineto{\pgfqpoint{0.935228in}{2.992000in}}%
\pgfpathlineto{\pgfqpoint{0.920242in}{3.029538in}}%
\pgfpathlineto{\pgfqpoint{0.893687in}{3.104000in}}%
\pgfpathlineto{\pgfqpoint{0.890845in}{3.113951in}}%
\pgfpathlineto{\pgfqpoint{0.880162in}{3.145617in}}%
\pgfpathlineto{\pgfqpoint{0.854759in}{3.239661in}}%
\pgfpathlineto{\pgfqpoint{0.844304in}{3.290667in}}%
\pgfpathlineto{\pgfqpoint{0.837972in}{3.329964in}}%
\pgfpathlineto{\pgfqpoint{0.834084in}{3.365333in}}%
\pgfpathlineto{\pgfqpoint{0.831310in}{3.410837in}}%
\pgfpathlineto{\pgfqpoint{0.831594in}{3.447905in}}%
\pgfpathlineto{\pgfqpoint{0.833815in}{3.477333in}}%
\pgfpathlineto{\pgfqpoint{0.840081in}{3.518892in}}%
\pgfpathlineto{\pgfqpoint{0.848806in}{3.552000in}}%
\pgfpathlineto{\pgfqpoint{0.856489in}{3.574050in}}%
\pgfpathlineto{\pgfqpoint{0.868590in}{3.600112in}}%
\pgfpathlineto{\pgfqpoint{0.885109in}{3.626667in}}%
\pgfpathlineto{\pgfqpoint{0.899986in}{3.645535in}}%
\pgfpathlineto{\pgfqpoint{0.920242in}{3.665784in}}%
\pgfpathlineto{\pgfqpoint{0.965338in}{3.696662in}}%
\pgfpathlineto{\pgfqpoint{1.000404in}{3.711748in}}%
\pgfpathlineto{\pgfqpoint{1.021829in}{3.718710in}}%
\pgfpathlineto{\pgfqpoint{1.040485in}{3.723058in}}%
\pgfpathlineto{\pgfqpoint{1.089231in}{3.730596in}}%
\pgfpathlineto{\pgfqpoint{1.120646in}{3.732206in}}%
\pgfpathlineto{\pgfqpoint{1.160727in}{3.731659in}}%
\pgfpathlineto{\pgfqpoint{1.240889in}{3.723065in}}%
\pgfpathlineto{\pgfqpoint{1.259888in}{3.719030in}}%
\pgfpathlineto{\pgfqpoint{1.280970in}{3.715780in}}%
\pgfpathlineto{\pgfqpoint{1.321051in}{3.706888in}}%
\pgfpathlineto{\pgfqpoint{1.361131in}{3.696367in}}%
\pgfpathlineto{\pgfqpoint{1.481374in}{3.656929in}}%
\pgfpathlineto{\pgfqpoint{1.521455in}{3.641466in}}%
\pgfpathlineto{\pgfqpoint{1.532214in}{3.636689in}}%
\pgfpathlineto{\pgfqpoint{1.561535in}{3.625253in}}%
\pgfpathlineto{\pgfqpoint{1.721859in}{3.550962in}}%
\pgfpathlineto{\pgfqpoint{1.798129in}{3.511042in}}%
\pgfpathlineto{\pgfqpoint{1.802020in}{3.509123in}}%
\pgfpathlineto{\pgfqpoint{1.873455in}{3.469205in}}%
\pgfpathlineto{\pgfqpoint{1.898151in}{3.454874in}}%
\pgfpathlineto{\pgfqpoint{1.924838in}{3.440000in}}%
\pgfpathlineto{\pgfqpoint{1.996191in}{3.396861in}}%
\pgfpathlineto{\pgfqpoint{2.032297in}{3.374841in}}%
\pgfpathlineto{\pgfqpoint{2.122667in}{3.316862in}}%
\pgfpathlineto{\pgfqpoint{2.163457in}{3.290005in}}%
\pgfpathlineto{\pgfqpoint{2.282990in}{3.207823in}}%
\pgfpathlineto{\pgfqpoint{2.345766in}{3.162473in}}%
\pgfpathlineto{\pgfqpoint{2.375518in}{3.141333in}}%
\pgfpathlineto{\pgfqpoint{2.483394in}{3.060906in}}%
\pgfpathlineto{\pgfqpoint{2.572447in}{2.992000in}}%
\pgfpathlineto{\pgfqpoint{2.683798in}{2.902916in}}%
\pgfpathlineto{\pgfqpoint{2.723879in}{2.870090in}}%
\pgfpathlineto{\pgfqpoint{2.845456in}{2.768000in}}%
\pgfpathlineto{\pgfqpoint{2.973681in}{2.656000in}}%
\pgfpathlineto{\pgfqpoint{3.015405in}{2.618667in}}%
\pgfpathlineto{\pgfqpoint{3.124687in}{2.518807in}}%
\pgfpathlineto{\pgfqpoint{3.177573in}{2.469333in}}%
\pgfpathlineto{\pgfqpoint{3.294576in}{2.357333in}}%
\pgfpathlineto{\pgfqpoint{3.332757in}{2.320000in}}%
\pgfpathlineto{\pgfqpoint{3.445333in}{2.207629in}}%
\pgfpathlineto{\pgfqpoint{3.565576in}{2.083333in}}%
\pgfpathlineto{\pgfqpoint{3.633963in}{2.010366in}}%
\pgfpathlineto{\pgfqpoint{3.658850in}{1.984000in}}%
\pgfpathlineto{\pgfqpoint{3.765980in}{1.866603in}}%
\pgfpathlineto{\pgfqpoint{3.806061in}{1.821577in}}%
\pgfpathlineto{\pgfqpoint{3.892547in}{1.722667in}}%
\pgfpathlineto{\pgfqpoint{3.926303in}{1.683363in}}%
\pgfpathlineto{\pgfqpoint{4.018491in}{1.573333in}}%
\pgfpathlineto{\pgfqpoint{4.057833in}{1.525486in}}%
\pgfpathlineto{\pgfqpoint{4.139068in}{1.424000in}}%
\pgfpathlineto{\pgfqpoint{4.173005in}{1.380876in}}%
\pgfpathlineto{\pgfqpoint{4.254203in}{1.274667in}}%
\pgfpathlineto{\pgfqpoint{4.300183in}{1.212251in}}%
\pgfpathlineto{\pgfqpoint{4.336792in}{1.162667in}}%
\pgfpathlineto{\pgfqpoint{4.367192in}{1.120398in}}%
\pgfpathlineto{\pgfqpoint{4.447354in}{1.005263in}}%
\pgfpathlineto{\pgfqpoint{4.527515in}{0.884394in}}%
\pgfpathlineto{\pgfqpoint{4.587538in}{0.789333in}}%
\pgfpathlineto{\pgfqpoint{4.632740in}{0.714667in}}%
\pgfpathlineto{\pgfqpoint{4.665918in}{0.656915in}}%
\pgfpathlineto{\pgfqpoint{4.687838in}{0.619191in}}%
\pgfpathlineto{\pgfqpoint{4.727919in}{0.545879in}}%
\pgfpathlineto{\pgfqpoint{4.737412in}{0.528000in}}%
\pgfpathlineto{\pgfqpoint{4.747679in}{0.528000in}}%
\pgfpathlineto{\pgfqpoint{4.740849in}{0.540043in}}%
\pgfpathlineto{\pgfqpoint{4.727858in}{0.565333in}}%
\pgfpathlineto{\pgfqpoint{4.687838in}{0.637132in}}%
\pgfpathlineto{\pgfqpoint{4.647758in}{0.705970in}}%
\pgfpathlineto{\pgfqpoint{4.550060in}{0.864000in}}%
\pgfpathlineto{\pgfqpoint{4.487434in}{0.959542in}}%
\pgfpathlineto{\pgfqpoint{4.345786in}{1.162667in}}%
\pgfpathlineto{\pgfqpoint{4.287030in}{1.242814in}}%
\pgfpathlineto{\pgfqpoint{4.222940in}{1.326970in}}%
\pgfpathlineto{\pgfqpoint{4.205045in}{1.351032in}}%
\pgfpathlineto{\pgfqpoint{4.093871in}{1.491919in}}%
\pgfpathlineto{\pgfqpoint{4.006465in}{1.598517in}}%
\pgfpathlineto{\pgfqpoint{3.930179in}{1.688944in}}%
\pgfpathlineto{\pgfqpoint{3.901303in}{1.722667in}}%
\pgfpathlineto{\pgfqpoint{3.803402in}{1.834667in}}%
\pgfpathlineto{\pgfqpoint{3.749597in}{1.894074in}}%
\pgfpathlineto{\pgfqpoint{3.702174in}{1.946667in}}%
\pgfpathlineto{\pgfqpoint{3.638854in}{2.014922in}}%
\pgfpathlineto{\pgfqpoint{3.598019in}{2.058667in}}%
\pgfpathlineto{\pgfqpoint{3.562634in}{2.096000in}}%
\pgfpathlineto{\pgfqpoint{3.454018in}{2.208000in}}%
\pgfpathlineto{\pgfqpoint{3.372579in}{2.289566in}}%
\pgfpathlineto{\pgfqpoint{3.325091in}{2.336673in}}%
\pgfpathlineto{\pgfqpoint{3.274880in}{2.385231in}}%
\pgfpathlineto{\pgfqpoint{3.226507in}{2.432000in}}%
\pgfpathlineto{\pgfqpoint{3.124687in}{2.527829in}}%
\pgfpathlineto{\pgfqpoint{3.004444in}{2.637502in}}%
\pgfpathlineto{\pgfqpoint{2.953103in}{2.682845in}}%
\pgfpathlineto{\pgfqpoint{2.911830in}{2.719067in}}%
\pgfpathlineto{\pgfqpoint{2.884202in}{2.743501in}}%
\pgfpathlineto{\pgfqpoint{2.763960in}{2.845899in}}%
\pgfpathlineto{\pgfqpoint{2.677242in}{2.917333in}}%
\pgfpathlineto{\pgfqpoint{2.563556in}{3.008050in}}%
\pgfpathlineto{\pgfqpoint{2.483394in}{3.070106in}}%
\pgfpathlineto{\pgfqpoint{2.418619in}{3.118332in}}%
\pgfpathlineto{\pgfqpoint{2.388142in}{3.141333in}}%
\pgfpathlineto{\pgfqpoint{2.176501in}{3.290667in}}%
\pgfpathlineto{\pgfqpoint{2.120614in}{3.328000in}}%
\pgfpathlineto{\pgfqpoint{2.042505in}{3.378189in}}%
\pgfpathlineto{\pgfqpoint{1.962343in}{3.427687in}}%
\pgfpathlineto{\pgfqpoint{1.941721in}{3.440000in}}%
\pgfpathlineto{\pgfqpoint{1.878057in}{3.477333in}}%
\pgfpathlineto{\pgfqpoint{1.789067in}{3.526732in}}%
\pgfpathlineto{\pgfqpoint{1.698963in}{3.573326in}}%
\pgfpathlineto{\pgfqpoint{1.616771in}{3.612551in}}%
\pgfpathlineto{\pgfqpoint{1.561535in}{3.636864in}}%
\pgfpathlineto{\pgfqpoint{1.521455in}{3.653505in}}%
\pgfpathlineto{\pgfqpoint{1.471828in}{3.672891in}}%
\pgfpathlineto{\pgfqpoint{1.397959in}{3.698303in}}%
\pgfpathlineto{\pgfqpoint{1.348052in}{3.713516in}}%
\pgfpathlineto{\pgfqpoint{1.280970in}{3.730180in}}%
\pgfpathlineto{\pgfqpoint{1.273153in}{3.731385in}}%
\pgfpathlineto{\pgfqpoint{1.237987in}{3.738667in}}%
\pgfpathlineto{\pgfqpoint{1.193911in}{3.745091in}}%
\pgfpathlineto{\pgfqpoint{1.160727in}{3.748047in}}%
\pgfpathlineto{\pgfqpoint{1.108879in}{3.749627in}}%
\pgfpathlineto{\pgfqpoint{1.080566in}{3.748183in}}%
\pgfpathlineto{\pgfqpoint{1.036116in}{3.742736in}}%
\pgfpathlineto{\pgfqpoint{1.000404in}{3.734099in}}%
\pgfpathlineto{\pgfqpoint{0.991491in}{3.730365in}}%
\pgfpathlineto{\pgfqpoint{0.960323in}{3.718674in}}%
\pgfpathlineto{\pgfqpoint{0.948013in}{3.712800in}}%
\pgfpathlineto{\pgfqpoint{0.920242in}{3.695278in}}%
\pgfpathlineto{\pgfqpoint{0.902189in}{3.680816in}}%
\pgfpathlineto{\pgfqpoint{0.880162in}{3.657989in}}%
\pgfpathlineto{\pgfqpoint{0.851697in}{3.615846in}}%
\pgfpathlineto{\pgfqpoint{0.839133in}{3.588450in}}%
\pgfpathlineto{\pgfqpoint{0.827529in}{3.552000in}}%
\pgfpathlineto{\pgfqpoint{0.822486in}{3.531056in}}%
\pgfpathlineto{\pgfqpoint{0.819901in}{3.514667in}}%
\pgfpathlineto{\pgfqpoint{0.814437in}{3.440000in}}%
\pgfpathlineto{\pgfqpoint{0.814347in}{3.426637in}}%
\pgfpathlineto{\pgfqpoint{0.815561in}{3.402667in}}%
\pgfpathlineto{\pgfqpoint{0.826369in}{3.315228in}}%
\pgfpathlineto{\pgfqpoint{0.830115in}{3.290667in}}%
\pgfpathlineto{\pgfqpoint{0.840081in}{3.244398in}}%
\pgfpathlineto{\pgfqpoint{0.847171in}{3.216000in}}%
\pgfpathlineto{\pgfqpoint{0.881148in}{3.104000in}}%
\pgfpathlineto{\pgfqpoint{0.923519in}{2.992000in}}%
\pgfpathlineto{\pgfqpoint{0.990375in}{2.842667in}}%
\pgfpathlineto{\pgfqpoint{1.027558in}{2.768000in}}%
\pgfpathlineto{\pgfqpoint{1.044914in}{2.734792in}}%
\pgfpathlineto{\pgfqpoint{1.053053in}{2.718960in}}%
\pgfpathlineto{\pgfqpoint{1.108422in}{2.618667in}}%
\pgfpathlineto{\pgfqpoint{1.129866in}{2.581333in}}%
\pgfpathlineto{\pgfqpoint{1.174249in}{2.506667in}}%
\pgfpathlineto{\pgfqpoint{1.200808in}{2.463191in}}%
\pgfpathlineto{\pgfqpoint{1.268190in}{2.357333in}}%
\pgfpathlineto{\pgfqpoint{1.303844in}{2.303973in}}%
\pgfpathlineto{\pgfqpoint{1.321051in}{2.277456in}}%
\pgfpathlineto{\pgfqpoint{1.401212in}{2.161363in}}%
\pgfpathlineto{\pgfqpoint{1.481374in}{2.050074in}}%
\pgfpathlineto{\pgfqpoint{1.530496in}{1.984000in}}%
\pgfpathlineto{\pgfqpoint{1.601616in}{1.890836in}}%
\pgfpathlineto{\pgfqpoint{1.681778in}{1.788936in}}%
\pgfpathlineto{\pgfqpoint{1.765870in}{1.685333in}}%
\pgfpathlineto{\pgfqpoint{1.802020in}{1.641795in}}%
\pgfpathlineto{\pgfqpoint{1.922263in}{1.500612in}}%
\pgfpathlineto{\pgfqpoint{2.138837in}{1.259604in}}%
\pgfpathlineto{\pgfqpoint{2.242909in}{1.149276in}}%
\pgfpathlineto{\pgfqpoint{2.363152in}{1.025683in}}%
\pgfpathlineto{\pgfqpoint{2.450268in}{0.938667in}}%
\pgfpathlineto{\pgfqpoint{2.488285in}{0.901333in}}%
\pgfpathlineto{\pgfqpoint{2.604816in}{0.789333in}}%
\pgfpathlineto{\pgfqpoint{2.725311in}{0.677333in}}%
\pgfpathlineto{\pgfqpoint{2.850107in}{0.565333in}}%
\pgfpathlineto{\pgfqpoint{2.892725in}{0.528000in}}%
\pgfpathlineto{\pgfqpoint{2.892725in}{0.528000in}}%
\pgfusepath{fill}%
\end{pgfscope}%
\begin{pgfscope}%
\pgfpathrectangle{\pgfqpoint{0.800000in}{0.528000in}}{\pgfqpoint{3.968000in}{3.696000in}}%
\pgfusepath{clip}%
\pgfsetbuttcap%
\pgfsetroundjoin%
\definecolor{currentfill}{rgb}{0.282327,0.094955,0.417331}%
\pgfsetfillcolor{currentfill}%
\pgfsetlinewidth{0.000000pt}%
\definecolor{currentstroke}{rgb}{0.000000,0.000000,0.000000}%
\pgfsetstrokecolor{currentstroke}%
\pgfsetdash{}{0pt}%
\pgfpathmoveto{\pgfqpoint{2.892725in}{0.528000in}}%
\pgfpathlineto{\pgfqpoint{2.804040in}{0.606197in}}%
\pgfpathlineto{\pgfqpoint{2.683798in}{0.715484in}}%
\pgfpathlineto{\pgfqpoint{2.563556in}{0.828565in}}%
\pgfpathlineto{\pgfqpoint{2.523475in}{0.867116in}}%
\pgfpathlineto{\pgfqpoint{2.403232in}{0.985380in}}%
\pgfpathlineto{\pgfqpoint{2.282990in}{1.107628in}}%
\pgfpathlineto{\pgfqpoint{2.230142in}{1.162667in}}%
\pgfpathlineto{\pgfqpoint{2.122667in}{1.277051in}}%
\pgfpathlineto{\pgfqpoint{2.068757in}{1.336452in}}%
\pgfpathlineto{\pgfqpoint{2.023025in}{1.386667in}}%
\pgfpathlineto{\pgfqpoint{1.989689in}{1.424000in}}%
\pgfpathlineto{\pgfqpoint{1.891763in}{1.536000in}}%
\pgfpathlineto{\pgfqpoint{1.852168in}{1.582710in}}%
\pgfpathlineto{\pgfqpoint{1.828213in}{1.610667in}}%
\pgfpathlineto{\pgfqpoint{1.735353in}{1.722667in}}%
\pgfpathlineto{\pgfqpoint{1.705088in}{1.760000in}}%
\pgfpathlineto{\pgfqpoint{1.641697in}{1.839403in}}%
\pgfpathlineto{\pgfqpoint{1.587344in}{1.909333in}}%
\pgfpathlineto{\pgfqpoint{1.521455in}{1.996064in}}%
\pgfpathlineto{\pgfqpoint{1.441293in}{2.105162in}}%
\pgfpathlineto{\pgfqpoint{1.394643in}{2.170667in}}%
\pgfpathlineto{\pgfqpoint{1.342965in}{2.245333in}}%
\pgfpathlineto{\pgfqpoint{1.280970in}{2.337791in}}%
\pgfpathlineto{\pgfqpoint{1.160727in}{2.529080in}}%
\pgfpathlineto{\pgfqpoint{1.120646in}{2.597301in}}%
\pgfpathlineto{\pgfqpoint{1.040485in}{2.742988in}}%
\pgfpathlineto{\pgfqpoint{0.972722in}{2.880000in}}%
\pgfpathlineto{\pgfqpoint{0.957076in}{2.914309in}}%
\pgfpathlineto{\pgfqpoint{0.952119in}{2.924975in}}%
\pgfpathlineto{\pgfqpoint{0.920242in}{3.000203in}}%
\pgfpathlineto{\pgfqpoint{0.880162in}{3.106989in}}%
\pgfpathlineto{\pgfqpoint{0.854388in}{3.191994in}}%
\pgfpathlineto{\pgfqpoint{0.847171in}{3.216000in}}%
\pgfpathlineto{\pgfqpoint{0.837232in}{3.255987in}}%
\pgfpathlineto{\pgfqpoint{0.830115in}{3.290667in}}%
\pgfpathlineto{\pgfqpoint{0.818736in}{3.365333in}}%
\pgfpathlineto{\pgfqpoint{0.817872in}{3.381980in}}%
\pgfpathlineto{\pgfqpoint{0.815561in}{3.402667in}}%
\pgfpathlineto{\pgfqpoint{0.814437in}{3.440000in}}%
\pgfpathlineto{\pgfqpoint{0.815731in}{3.477333in}}%
\pgfpathlineto{\pgfqpoint{0.818164in}{3.494252in}}%
\pgfpathlineto{\pgfqpoint{0.819901in}{3.514667in}}%
\pgfpathlineto{\pgfqpoint{0.822486in}{3.531056in}}%
\pgfpathlineto{\pgfqpoint{0.827529in}{3.552000in}}%
\pgfpathlineto{\pgfqpoint{0.840081in}{3.591018in}}%
\pgfpathlineto{\pgfqpoint{0.851697in}{3.615846in}}%
\pgfpathlineto{\pgfqpoint{0.858205in}{3.626667in}}%
\pgfpathlineto{\pgfqpoint{0.885413in}{3.664000in}}%
\pgfpathlineto{\pgfqpoint{0.902189in}{3.680816in}}%
\pgfpathlineto{\pgfqpoint{0.929187in}{3.701333in}}%
\pgfpathlineto{\pgfqpoint{0.948013in}{3.712800in}}%
\pgfpathlineto{\pgfqpoint{0.960323in}{3.718674in}}%
\pgfpathlineto{\pgfqpoint{1.004040in}{3.735280in}}%
\pgfpathlineto{\pgfqpoint{1.040485in}{3.743404in}}%
\pgfpathlineto{\pgfqpoint{1.045998in}{3.743802in}}%
\pgfpathlineto{\pgfqpoint{1.080566in}{3.748183in}}%
\pgfpathlineto{\pgfqpoint{1.120646in}{3.749537in}}%
\pgfpathlineto{\pgfqpoint{1.131407in}{3.748689in}}%
\pgfpathlineto{\pgfqpoint{1.160727in}{3.748047in}}%
\pgfpathlineto{\pgfqpoint{1.205679in}{3.743204in}}%
\pgfpathlineto{\pgfqpoint{1.240889in}{3.738236in}}%
\pgfpathlineto{\pgfqpoint{1.321051in}{3.720592in}}%
\pgfpathlineto{\pgfqpoint{1.367659in}{3.707413in}}%
\pgfpathlineto{\pgfqpoint{1.401212in}{3.697448in}}%
\pgfpathlineto{\pgfqpoint{1.521455in}{3.653505in}}%
\pgfpathlineto{\pgfqpoint{1.568788in}{3.633422in}}%
\pgfpathlineto{\pgfqpoint{1.616771in}{3.612551in}}%
\pgfpathlineto{\pgfqpoint{1.698963in}{3.573326in}}%
\pgfpathlineto{\pgfqpoint{1.789067in}{3.526732in}}%
\pgfpathlineto{\pgfqpoint{1.878057in}{3.477333in}}%
\pgfpathlineto{\pgfqpoint{2.003529in}{3.402667in}}%
\pgfpathlineto{\pgfqpoint{2.122667in}{3.326668in}}%
\pgfpathlineto{\pgfqpoint{2.176501in}{3.290667in}}%
\pgfpathlineto{\pgfqpoint{2.284843in}{3.216000in}}%
\pgfpathlineto{\pgfqpoint{2.351459in}{3.167776in}}%
\pgfpathlineto{\pgfqpoint{2.388142in}{3.141333in}}%
\pgfpathlineto{\pgfqpoint{2.487874in}{3.066667in}}%
\pgfpathlineto{\pgfqpoint{2.603636in}{2.976436in}}%
\pgfpathlineto{\pgfqpoint{2.643717in}{2.944428in}}%
\pgfpathlineto{\pgfqpoint{2.763960in}{2.845899in}}%
\pgfpathlineto{\pgfqpoint{2.884202in}{2.743501in}}%
\pgfpathlineto{\pgfqpoint{2.953103in}{2.682845in}}%
\pgfpathlineto{\pgfqpoint{2.994137in}{2.646399in}}%
\pgfpathlineto{\pgfqpoint{3.025353in}{2.618667in}}%
\pgfpathlineto{\pgfqpoint{3.124687in}{2.527829in}}%
\pgfpathlineto{\pgfqpoint{3.175857in}{2.479663in}}%
\pgfpathlineto{\pgfqpoint{3.226507in}{2.432000in}}%
\pgfpathlineto{\pgfqpoint{3.325091in}{2.336673in}}%
\pgfpathlineto{\pgfqpoint{3.445333in}{2.216830in}}%
\pgfpathlineto{\pgfqpoint{3.490611in}{2.170667in}}%
\pgfpathlineto{\pgfqpoint{3.598019in}{2.058667in}}%
\pgfpathlineto{\pgfqpoint{3.645737in}{2.007776in}}%
\pgfpathlineto{\pgfqpoint{3.736253in}{1.909333in}}%
\pgfpathlineto{\pgfqpoint{3.786107in}{1.853415in}}%
\pgfpathlineto{\pgfqpoint{3.806061in}{1.831675in}}%
\pgfpathlineto{\pgfqpoint{3.901303in}{1.722667in}}%
\pgfpathlineto{\pgfqpoint{3.933372in}{1.685333in}}%
\pgfpathlineto{\pgfqpoint{4.027287in}{1.573333in}}%
\pgfpathlineto{\pgfqpoint{4.057981in}{1.536000in}}%
\pgfpathlineto{\pgfqpoint{4.126707in}{1.450811in}}%
\pgfpathlineto{\pgfqpoint{4.177254in}{1.386667in}}%
\pgfpathlineto{\pgfqpoint{4.246949in}{1.296058in}}%
\pgfpathlineto{\pgfqpoint{4.407273in}{1.076491in}}%
\pgfpathlineto{\pgfqpoint{4.487434in}{0.959542in}}%
\pgfpathlineto{\pgfqpoint{4.550060in}{0.864000in}}%
\pgfpathlineto{\pgfqpoint{4.586006in}{0.806481in}}%
\pgfpathlineto{\pgfqpoint{4.607677in}{0.772314in}}%
\pgfpathlineto{\pgfqpoint{4.707095in}{0.602667in}}%
\pgfpathlineto{\pgfqpoint{4.713897in}{0.589606in}}%
\pgfpathlineto{\pgfqpoint{4.727919in}{0.565219in}}%
\pgfpathlineto{\pgfqpoint{4.747679in}{0.528000in}}%
\pgfpathlineto{\pgfqpoint{4.757947in}{0.528000in}}%
\pgfpathlineto{\pgfqpoint{4.737793in}{0.565333in}}%
\pgfpathlineto{\pgfqpoint{4.687838in}{0.654117in}}%
\pgfpathlineto{\pgfqpoint{4.577257in}{0.835665in}}%
\pgfpathlineto{\pgfqpoint{4.547429in}{0.882784in}}%
\pgfpathlineto{\pgfqpoint{4.482767in}{0.980348in}}%
\pgfpathlineto{\pgfqpoint{4.407273in}{1.089509in}}%
\pgfpathlineto{\pgfqpoint{4.327111in}{1.200771in}}%
\pgfpathlineto{\pgfqpoint{4.278012in}{1.266266in}}%
\pgfpathlineto{\pgfqpoint{4.243780in}{1.312000in}}%
\pgfpathlineto{\pgfqpoint{4.206869in}{1.359933in}}%
\pgfpathlineto{\pgfqpoint{4.126707in}{1.461961in}}%
\pgfpathlineto{\pgfqpoint{4.036084in}{1.573333in}}%
\pgfpathlineto{\pgfqpoint{3.942057in}{1.685333in}}%
\pgfpathlineto{\pgfqpoint{3.910059in}{1.722667in}}%
\pgfpathlineto{\pgfqpoint{3.809338in}{1.837720in}}%
\pgfpathlineto{\pgfqpoint{3.765980in}{1.886208in}}%
\pgfpathlineto{\pgfqpoint{3.699204in}{1.959135in}}%
\pgfpathlineto{\pgfqpoint{3.676711in}{1.984000in}}%
\pgfpathlineto{\pgfqpoint{3.565576in}{2.102280in}}%
\pgfpathlineto{\pgfqpoint{3.445333in}{2.226018in}}%
\pgfpathlineto{\pgfqpoint{3.325091in}{2.345798in}}%
\pgfpathlineto{\pgfqpoint{3.274852in}{2.394667in}}%
\pgfpathlineto{\pgfqpoint{3.157078in}{2.506667in}}%
\pgfpathlineto{\pgfqpoint{3.100437in}{2.558745in}}%
\pgfpathlineto{\pgfqpoint{3.060018in}{2.595764in}}%
\pgfpathlineto{\pgfqpoint{3.035300in}{2.618667in}}%
\pgfpathlineto{\pgfqpoint{2.924283in}{2.717493in}}%
\pgfpathlineto{\pgfqpoint{2.884202in}{2.752402in}}%
\pgfpathlineto{\pgfqpoint{2.763960in}{2.854741in}}%
\pgfpathlineto{\pgfqpoint{2.642275in}{2.954667in}}%
\pgfpathlineto{\pgfqpoint{2.595406in}{2.992000in}}%
\pgfpathlineto{\pgfqpoint{2.483394in}{3.079110in}}%
\pgfpathlineto{\pgfqpoint{2.424059in}{3.123399in}}%
\pgfpathlineto{\pgfqpoint{2.400765in}{3.141333in}}%
\pgfpathlineto{\pgfqpoint{2.334497in}{3.189310in}}%
\pgfpathlineto{\pgfqpoint{2.297767in}{3.216000in}}%
\pgfpathlineto{\pgfqpoint{2.122667in}{3.336171in}}%
\pgfpathlineto{\pgfqpoint{2.056634in}{3.378494in}}%
\pgfpathlineto{\pgfqpoint{2.019067in}{3.402667in}}%
\pgfpathlineto{\pgfqpoint{1.860285in}{3.497729in}}%
\pgfpathlineto{\pgfqpoint{1.761550in}{3.552000in}}%
\pgfpathlineto{\pgfqpoint{1.673955in}{3.596620in}}%
\pgfpathlineto{\pgfqpoint{1.601616in}{3.630751in}}%
\pgfpathlineto{\pgfqpoint{1.450018in}{3.693206in}}%
\pgfpathlineto{\pgfqpoint{1.401212in}{3.710127in}}%
\pgfpathlineto{\pgfqpoint{1.280970in}{3.744275in}}%
\pgfpathlineto{\pgfqpoint{1.240889in}{3.752608in}}%
\pgfpathlineto{\pgfqpoint{1.160727in}{3.764012in}}%
\pgfpathlineto{\pgfqpoint{1.148153in}{3.764288in}}%
\pgfpathlineto{\pgfqpoint{1.120646in}{3.766456in}}%
\pgfpathlineto{\pgfqpoint{1.080566in}{3.766175in}}%
\pgfpathlineto{\pgfqpoint{1.068357in}{3.764628in}}%
\pgfpathlineto{\pgfqpoint{1.040485in}{3.762617in}}%
\pgfpathlineto{\pgfqpoint{1.018400in}{3.759238in}}%
\pgfpathlineto{\pgfqpoint{1.000404in}{3.755065in}}%
\pgfpathlineto{\pgfqpoint{0.951371in}{3.738667in}}%
\pgfpathlineto{\pgfqpoint{0.930787in}{3.728844in}}%
\pgfpathlineto{\pgfqpoint{0.920242in}{3.722422in}}%
\pgfpathlineto{\pgfqpoint{0.885165in}{3.696673in}}%
\pgfpathlineto{\pgfqpoint{0.880162in}{3.691657in}}%
\pgfpathlineto{\pgfqpoint{0.855919in}{3.664000in}}%
\pgfpathlineto{\pgfqpoint{0.834905in}{3.631487in}}%
\pgfpathlineto{\pgfqpoint{0.828710in}{3.616075in}}%
\pgfpathlineto{\pgfqpoint{0.817357in}{3.589333in}}%
\pgfpathlineto{\pgfqpoint{0.805722in}{3.546671in}}%
\pgfpathlineto{\pgfqpoint{0.800000in}{3.506746in}}%
\pgfpathlineto{\pgfqpoint{0.800000in}{3.397278in}}%
\pgfpathlineto{\pgfqpoint{0.803776in}{3.361817in}}%
\pgfpathlineto{\pgfqpoint{0.809058in}{3.328000in}}%
\pgfpathlineto{\pgfqpoint{0.827883in}{3.241972in}}%
\pgfpathlineto{\pgfqpoint{0.840081in}{3.194694in}}%
\pgfpathlineto{\pgfqpoint{0.856566in}{3.141333in}}%
\pgfpathlineto{\pgfqpoint{0.862612in}{3.124986in}}%
\pgfpathlineto{\pgfqpoint{0.869187in}{3.104000in}}%
\pgfpathlineto{\pgfqpoint{0.883846in}{3.063235in}}%
\pgfpathlineto{\pgfqpoint{0.920242in}{2.972996in}}%
\pgfpathlineto{\pgfqpoint{0.961740in}{2.880000in}}%
\pgfpathlineto{\pgfqpoint{1.040485in}{2.723521in}}%
\pgfpathlineto{\pgfqpoint{1.098564in}{2.618667in}}%
\pgfpathlineto{\pgfqpoint{1.120646in}{2.580148in}}%
\pgfpathlineto{\pgfqpoint{1.187565in}{2.469333in}}%
\pgfpathlineto{\pgfqpoint{1.240889in}{2.384970in}}%
\pgfpathlineto{\pgfqpoint{1.308433in}{2.282667in}}%
\pgfpathlineto{\pgfqpoint{1.361131in}{2.205526in}}%
\pgfpathlineto{\pgfqpoint{1.441293in}{2.092654in}}%
\pgfpathlineto{\pgfqpoint{1.493740in}{2.021333in}}%
\pgfpathlineto{\pgfqpoint{1.561535in}{1.931418in}}%
\pgfpathlineto{\pgfqpoint{1.641697in}{1.828281in}}%
\pgfpathlineto{\pgfqpoint{1.707515in}{1.746640in}}%
\pgfpathlineto{\pgfqpoint{1.741261in}{1.704594in}}%
\pgfpathlineto{\pgfqpoint{1.842101in}{1.583848in}}%
\pgfpathlineto{\pgfqpoint{1.915263in}{1.498667in}}%
\pgfpathlineto{\pgfqpoint{2.002424in}{1.399797in}}%
\pgfpathlineto{\pgfqpoint{2.064003in}{1.332024in}}%
\pgfpathlineto{\pgfqpoint{2.100881in}{1.291707in}}%
\pgfpathlineto{\pgfqpoint{2.150783in}{1.237333in}}%
\pgfpathlineto{\pgfqpoint{2.202828in}{1.181978in}}%
\pgfpathlineto{\pgfqpoint{2.323071in}{1.057095in}}%
\pgfpathlineto{\pgfqpoint{2.366165in}{1.013333in}}%
\pgfpathlineto{\pgfqpoint{2.483394in}{0.897002in}}%
\pgfpathlineto{\pgfqpoint{2.603636in}{0.781530in}}%
\pgfpathlineto{\pgfqpoint{2.723879in}{0.669775in}}%
\pgfpathlineto{\pgfqpoint{2.844121in}{0.561663in}}%
\pgfpathlineto{\pgfqpoint{2.884202in}{0.528000in}}%
\pgfpathlineto{\pgfqpoint{2.884202in}{0.528000in}}%
\pgfusepath{fill}%
\end{pgfscope}%
\begin{pgfscope}%
\pgfpathrectangle{\pgfqpoint{0.800000in}{0.528000in}}{\pgfqpoint{3.968000in}{3.696000in}}%
\pgfusepath{clip}%
\pgfsetbuttcap%
\pgfsetroundjoin%
\definecolor{currentfill}{rgb}{0.282656,0.100196,0.422160}%
\pgfsetfillcolor{currentfill}%
\pgfsetlinewidth{0.000000pt}%
\definecolor{currentstroke}{rgb}{0.000000,0.000000,0.000000}%
\pgfsetstrokecolor{currentstroke}%
\pgfsetdash{}{0pt}%
\pgfpathmoveto{\pgfqpoint{2.882405in}{0.528000in}}%
\pgfpathlineto{\pgfqpoint{2.821880in}{0.581950in}}%
\pgfpathlineto{\pgfqpoint{2.798060in}{0.602667in}}%
\pgfpathlineto{\pgfqpoint{2.675123in}{0.714667in}}%
\pgfpathlineto{\pgfqpoint{2.556188in}{0.826667in}}%
\pgfpathlineto{\pgfqpoint{2.517379in}{0.864000in}}%
\pgfpathlineto{\pgfqpoint{2.403232in}{0.976088in}}%
\pgfpathlineto{\pgfqpoint{2.282990in}{1.098270in}}%
\pgfpathlineto{\pgfqpoint{2.231587in}{1.152120in}}%
\pgfpathlineto{\pgfqpoint{2.185825in}{1.200000in}}%
\pgfpathlineto{\pgfqpoint{2.081658in}{1.312000in}}%
\pgfpathlineto{\pgfqpoint{2.042505in}{1.355127in}}%
\pgfpathlineto{\pgfqpoint{1.922263in}{1.490622in}}%
\pgfpathlineto{\pgfqpoint{1.842101in}{1.583848in}}%
\pgfpathlineto{\pgfqpoint{1.741261in}{1.704594in}}%
\pgfpathlineto{\pgfqpoint{1.666360in}{1.797333in}}%
\pgfpathlineto{\pgfqpoint{1.622151in}{1.853794in}}%
\pgfpathlineto{\pgfqpoint{1.588404in}{1.897027in}}%
\pgfpathlineto{\pgfqpoint{1.549898in}{1.946667in}}%
\pgfpathlineto{\pgfqpoint{1.481374in}{2.038020in}}%
\pgfpathlineto{\pgfqpoint{1.401212in}{2.148655in}}%
\pgfpathlineto{\pgfqpoint{1.258898in}{2.357333in}}%
\pgfpathlineto{\pgfqpoint{1.200808in}{2.448062in}}%
\pgfpathlineto{\pgfqpoint{1.142173in}{2.544000in}}%
\pgfpathlineto{\pgfqpoint{1.098564in}{2.618667in}}%
\pgfpathlineto{\pgfqpoint{1.056865in}{2.693333in}}%
\pgfpathlineto{\pgfqpoint{1.017218in}{2.768000in}}%
\pgfpathlineto{\pgfqpoint{1.000404in}{2.800697in}}%
\pgfpathlineto{\pgfqpoint{0.960323in}{2.883077in}}%
\pgfpathlineto{\pgfqpoint{0.897146in}{3.029333in}}%
\pgfpathlineto{\pgfqpoint{0.880162in}{3.073376in}}%
\pgfpathlineto{\pgfqpoint{0.856566in}{3.141333in}}%
\pgfpathlineto{\pgfqpoint{0.853163in}{3.153519in}}%
\pgfpathlineto{\pgfqpoint{0.840081in}{3.194694in}}%
\pgfpathlineto{\pgfqpoint{0.819996in}{3.272041in}}%
\pgfpathlineto{\pgfqpoint{0.809058in}{3.328000in}}%
\pgfpathlineto{\pgfqpoint{0.803233in}{3.368344in}}%
\pgfpathlineto{\pgfqpoint{0.800000in}{3.397278in}}%
\pgfpathlineto{\pgfqpoint{0.800000in}{3.301736in}}%
\pgfpathlineto{\pgfqpoint{0.811217in}{3.253333in}}%
\pgfpathlineto{\pgfqpoint{0.844170in}{3.141333in}}%
\pgfpathlineto{\pgfqpoint{0.885562in}{3.029333in}}%
\pgfpathlineto{\pgfqpoint{0.951180in}{2.880000in}}%
\pgfpathlineto{\pgfqpoint{0.987799in}{2.805333in}}%
\pgfpathlineto{\pgfqpoint{1.004884in}{2.772173in}}%
\pgfpathlineto{\pgfqpoint{1.006878in}{2.768000in}}%
\pgfpathlineto{\pgfqpoint{1.026643in}{2.730667in}}%
\pgfpathlineto{\pgfqpoint{1.067556in}{2.656000in}}%
\pgfpathlineto{\pgfqpoint{1.120646in}{2.563972in}}%
\pgfpathlineto{\pgfqpoint{1.178147in}{2.469333in}}%
\pgfpathlineto{\pgfqpoint{1.231405in}{2.385833in}}%
\pgfpathlineto{\pgfqpoint{1.249606in}{2.357333in}}%
\pgfpathlineto{\pgfqpoint{1.292465in}{2.293374in}}%
\pgfpathlineto{\pgfqpoint{1.324627in}{2.245333in}}%
\pgfpathlineto{\pgfqpoint{1.361131in}{2.192776in}}%
\pgfpathlineto{\pgfqpoint{1.441293in}{2.080564in}}%
\pgfpathlineto{\pgfqpoint{1.628010in}{1.834667in}}%
\pgfpathlineto{\pgfqpoint{1.721859in}{1.717708in}}%
\pgfpathlineto{\pgfqpoint{1.842235in}{1.573333in}}%
\pgfpathlineto{\pgfqpoint{1.922263in}{1.480705in}}%
\pgfpathlineto{\pgfqpoint{2.020281in}{1.370034in}}%
\pgfpathlineto{\pgfqpoint{2.122667in}{1.258073in}}%
\pgfpathlineto{\pgfqpoint{2.212255in}{1.162667in}}%
\pgfpathlineto{\pgfqpoint{2.264779in}{1.108370in}}%
\pgfpathlineto{\pgfqpoint{2.292542in}{1.079103in}}%
\pgfpathlineto{\pgfqpoint{2.403232in}{0.967108in}}%
\pgfpathlineto{\pgfqpoint{2.523475in}{0.849177in}}%
\pgfpathlineto{\pgfqpoint{2.575093in}{0.800080in}}%
\pgfpathlineto{\pgfqpoint{2.625606in}{0.752000in}}%
\pgfpathlineto{\pgfqpoint{2.683798in}{0.697780in}}%
\pgfpathlineto{\pgfqpoint{2.804040in}{0.588520in}}%
\pgfpathlineto{\pgfqpoint{2.858096in}{0.541017in}}%
\pgfpathlineto{\pgfqpoint{2.872441in}{0.528000in}}%
\pgfpathmoveto{\pgfqpoint{4.768000in}{0.528379in}}%
\pgfpathlineto{\pgfqpoint{4.727143in}{0.602667in}}%
\pgfpathlineto{\pgfqpoint{4.687838in}{0.670919in}}%
\pgfpathlineto{\pgfqpoint{4.607677in}{0.802864in}}%
\pgfpathlineto{\pgfqpoint{4.469185in}{1.013333in}}%
\pgfpathlineto{\pgfqpoint{4.443470in}{1.050667in}}%
\pgfpathlineto{\pgfqpoint{4.390578in}{1.125333in}}%
\pgfpathlineto{\pgfqpoint{4.327111in}{1.212614in}}%
\pgfpathlineto{\pgfqpoint{4.266520in}{1.292895in}}%
\pgfpathlineto{\pgfqpoint{4.233231in}{1.336555in}}%
\pgfpathlineto{\pgfqpoint{4.194786in}{1.386667in}}%
\pgfpathlineto{\pgfqpoint{4.148725in}{1.444508in}}%
\pgfpathlineto{\pgfqpoint{4.105736in}{1.498667in}}%
\pgfpathlineto{\pgfqpoint{4.013799in}{1.610667in}}%
\pgfpathlineto{\pgfqpoint{3.918815in}{1.722667in}}%
\pgfpathlineto{\pgfqpoint{3.820792in}{1.834667in}}%
\pgfpathlineto{\pgfqpoint{3.759282in}{1.903094in}}%
\pgfpathlineto{\pgfqpoint{3.719890in}{1.946667in}}%
\pgfpathlineto{\pgfqpoint{3.684527in}{1.985203in}}%
\pgfpathlineto{\pgfqpoint{3.565576in}{2.111532in}}%
\pgfpathlineto{\pgfqpoint{3.445333in}{2.235206in}}%
\pgfpathlineto{\pgfqpoint{3.325091in}{2.354923in}}%
\pgfpathlineto{\pgfqpoint{3.262969in}{2.415197in}}%
\pgfpathlineto{\pgfqpoint{3.164768in}{2.508466in}}%
\pgfpathlineto{\pgfqpoint{3.105218in}{2.563199in}}%
\pgfpathlineto{\pgfqpoint{3.064816in}{2.600233in}}%
\pgfpathlineto{\pgfqpoint{3.023245in}{2.638488in}}%
\pgfpathlineto{\pgfqpoint{2.919407in}{2.730667in}}%
\pgfpathlineto{\pgfqpoint{2.859386in}{2.782218in}}%
\pgfpathlineto{\pgfqpoint{2.817600in}{2.817964in}}%
\pgfpathlineto{\pgfqpoint{2.788860in}{2.842667in}}%
\pgfpathlineto{\pgfqpoint{2.683798in}{2.929806in}}%
\pgfpathlineto{\pgfqpoint{2.563556in}{3.026139in}}%
\pgfpathlineto{\pgfqpoint{2.443313in}{3.118521in}}%
\pgfpathlineto{\pgfqpoint{2.384882in}{3.161575in}}%
\pgfpathlineto{\pgfqpoint{2.362467in}{3.178667in}}%
\pgfpathlineto{\pgfqpoint{2.294782in}{3.226984in}}%
\pgfpathlineto{\pgfqpoint{2.258027in}{3.253333in}}%
\pgfpathlineto{\pgfqpoint{2.198296in}{3.294888in}}%
\pgfpathlineto{\pgfqpoint{2.082586in}{3.371946in}}%
\pgfpathlineto{\pgfqpoint{2.015105in}{3.414478in}}%
\pgfpathlineto{\pgfqpoint{1.974616in}{3.440000in}}%
\pgfpathlineto{\pgfqpoint{1.832414in}{3.523690in}}%
\pgfpathlineto{\pgfqpoint{1.736587in}{3.575615in}}%
\pgfpathlineto{\pgfqpoint{1.649433in}{3.619461in}}%
\pgfpathlineto{\pgfqpoint{1.601616in}{3.641848in}}%
\pgfpathlineto{\pgfqpoint{1.585218in}{3.648726in}}%
\pgfpathlineto{\pgfqpoint{1.552054in}{3.664000in}}%
\pgfpathlineto{\pgfqpoint{1.481374in}{3.693261in}}%
\pgfpathlineto{\pgfqpoint{1.441293in}{3.708514in}}%
\pgfpathlineto{\pgfqpoint{1.321051in}{3.747541in}}%
\pgfpathlineto{\pgfqpoint{1.232452in}{3.768141in}}%
\pgfpathlineto{\pgfqpoint{1.188827in}{3.776000in}}%
\pgfpathlineto{\pgfqpoint{1.156106in}{3.780304in}}%
\pgfpathlineto{\pgfqpoint{1.112698in}{3.783404in}}%
\pgfpathlineto{\pgfqpoint{1.080566in}{3.783648in}}%
\pgfpathlineto{\pgfqpoint{1.040485in}{3.781435in}}%
\pgfpathlineto{\pgfqpoint{0.999873in}{3.775505in}}%
\pgfpathlineto{\pgfqpoint{0.960323in}{3.764806in}}%
\pgfpathlineto{\pgfqpoint{0.940146in}{3.757461in}}%
\pgfpathlineto{\pgfqpoint{0.920242in}{3.748020in}}%
\pgfpathlineto{\pgfqpoint{0.903998in}{3.738667in}}%
\pgfpathlineto{\pgfqpoint{0.880162in}{3.722006in}}%
\pgfpathlineto{\pgfqpoint{0.868373in}{3.712314in}}%
\pgfpathlineto{\pgfqpoint{0.857653in}{3.701333in}}%
\pgfpathlineto{\pgfqpoint{0.832635in}{3.670936in}}%
\pgfpathlineto{\pgfqpoint{0.818335in}{3.646922in}}%
\pgfpathlineto{\pgfqpoint{0.808945in}{3.626667in}}%
\pgfpathlineto{\pgfqpoint{0.800000in}{3.602744in}}%
\pgfpathlineto{\pgfqpoint{0.800000in}{3.506746in}}%
\pgfpathlineto{\pgfqpoint{0.800674in}{3.514667in}}%
\pgfpathlineto{\pgfqpoint{0.807005in}{3.552000in}}%
\pgfpathlineto{\pgfqpoint{0.809957in}{3.561275in}}%
\pgfpathlineto{\pgfqpoint{0.817357in}{3.589333in}}%
\pgfpathlineto{\pgfqpoint{0.840081in}{3.640207in}}%
\pgfpathlineto{\pgfqpoint{0.855919in}{3.664000in}}%
\pgfpathlineto{\pgfqpoint{0.885165in}{3.696673in}}%
\pgfpathlineto{\pgfqpoint{0.890984in}{3.701333in}}%
\pgfpathlineto{\pgfqpoint{0.930787in}{3.728844in}}%
\pgfpathlineto{\pgfqpoint{0.960323in}{3.742580in}}%
\pgfpathlineto{\pgfqpoint{0.966244in}{3.744181in}}%
\pgfpathlineto{\pgfqpoint{1.000404in}{3.755065in}}%
\pgfpathlineto{\pgfqpoint{1.018400in}{3.759238in}}%
\pgfpathlineto{\pgfqpoint{1.040485in}{3.762617in}}%
\pgfpathlineto{\pgfqpoint{1.090566in}{3.766686in}}%
\pgfpathlineto{\pgfqpoint{1.120646in}{3.766456in}}%
\pgfpathlineto{\pgfqpoint{1.200808in}{3.759280in}}%
\pgfpathlineto{\pgfqpoint{1.219066in}{3.755673in}}%
\pgfpathlineto{\pgfqpoint{1.240889in}{3.752608in}}%
\pgfpathlineto{\pgfqpoint{1.285548in}{3.742931in}}%
\pgfpathlineto{\pgfqpoint{1.327579in}{3.732585in}}%
\pgfpathlineto{\pgfqpoint{1.401212in}{3.710127in}}%
\pgfpathlineto{\pgfqpoint{1.426766in}{3.701333in}}%
\pgfpathlineto{\pgfqpoint{1.481374in}{3.681258in}}%
\pgfpathlineto{\pgfqpoint{1.494035in}{3.675793in}}%
\pgfpathlineto{\pgfqpoint{1.524935in}{3.664000in}}%
\pgfpathlineto{\pgfqpoint{1.641697in}{3.612028in}}%
\pgfpathlineto{\pgfqpoint{1.688546in}{3.589333in}}%
\pgfpathlineto{\pgfqpoint{1.762438in}{3.551536in}}%
\pgfpathlineto{\pgfqpoint{1.860285in}{3.497729in}}%
\pgfpathlineto{\pgfqpoint{1.962343in}{3.437768in}}%
\pgfpathlineto{\pgfqpoint{2.019067in}{3.402667in}}%
\pgfpathlineto{\pgfqpoint{2.082586in}{3.362408in}}%
\pgfpathlineto{\pgfqpoint{2.150910in}{3.316974in}}%
\pgfpathlineto{\pgfqpoint{2.190521in}{3.290667in}}%
\pgfpathlineto{\pgfqpoint{2.244948in}{3.253333in}}%
\pgfpathlineto{\pgfqpoint{2.363152in}{3.168914in}}%
\pgfpathlineto{\pgfqpoint{2.424059in}{3.123399in}}%
\pgfpathlineto{\pgfqpoint{2.450621in}{3.104000in}}%
\pgfpathlineto{\pgfqpoint{2.499602in}{3.066667in}}%
\pgfpathlineto{\pgfqpoint{2.603636in}{2.985501in}}%
\pgfpathlineto{\pgfqpoint{2.653075in}{2.945951in}}%
\pgfpathlineto{\pgfqpoint{2.763960in}{2.854741in}}%
\pgfpathlineto{\pgfqpoint{2.884202in}{2.752402in}}%
\pgfpathlineto{\pgfqpoint{2.924283in}{2.717493in}}%
\pgfpathlineto{\pgfqpoint{3.044525in}{2.610338in}}%
\pgfpathlineto{\pgfqpoint{3.100437in}{2.558745in}}%
\pgfpathlineto{\pgfqpoint{3.140632in}{2.521519in}}%
\pgfpathlineto{\pgfqpoint{3.196765in}{2.469333in}}%
\pgfpathlineto{\pgfqpoint{3.285010in}{2.384858in}}%
\pgfpathlineto{\pgfqpoint{3.405253in}{2.266380in}}%
\pgfpathlineto{\pgfqpoint{3.463056in}{2.208000in}}%
\pgfpathlineto{\pgfqpoint{3.571560in}{2.096000in}}%
\pgfpathlineto{\pgfqpoint{3.616902in}{2.048192in}}%
\pgfpathlineto{\pgfqpoint{3.725899in}{1.930420in}}%
\pgfpathlineto{\pgfqpoint{3.845225in}{1.797333in}}%
\pgfpathlineto{\pgfqpoint{3.886222in}{1.750297in}}%
\pgfpathlineto{\pgfqpoint{4.000777in}{1.615964in}}%
\pgfpathlineto{\pgfqpoint{4.086626in}{1.511583in}}%
\pgfpathlineto{\pgfqpoint{4.166788in}{1.411258in}}%
\pgfpathlineto{\pgfqpoint{4.228085in}{1.331762in}}%
\pgfpathlineto{\pgfqpoint{4.261392in}{1.288120in}}%
\pgfpathlineto{\pgfqpoint{4.299924in}{1.237333in}}%
\pgfpathlineto{\pgfqpoint{4.328917in}{1.198318in}}%
\pgfpathlineto{\pgfqpoint{4.410406in}{1.085081in}}%
\pgfpathlineto{\pgfqpoint{4.487434in}{0.973411in}}%
\pgfpathlineto{\pgfqpoint{4.567596in}{0.851355in}}%
\pgfpathlineto{\pgfqpoint{4.687838in}{0.654117in}}%
\pgfpathlineto{\pgfqpoint{4.757947in}{0.528000in}}%
\pgfpathlineto{\pgfqpoint{4.768000in}{0.528000in}}%
\pgfpathlineto{\pgfqpoint{4.768000in}{0.528000in}}%
\pgfusepath{fill}%
\end{pgfscope}%
\begin{pgfscope}%
\pgfpathrectangle{\pgfqpoint{0.800000in}{0.528000in}}{\pgfqpoint{3.968000in}{3.696000in}}%
\pgfusepath{clip}%
\pgfsetbuttcap%
\pgfsetroundjoin%
\definecolor{currentfill}{rgb}{0.282656,0.100196,0.422160}%
\pgfsetfillcolor{currentfill}%
\pgfsetlinewidth{0.000000pt}%
\definecolor{currentstroke}{rgb}{0.000000,0.000000,0.000000}%
\pgfsetstrokecolor{currentstroke}%
\pgfsetdash{}{0pt}%
\pgfpathmoveto{\pgfqpoint{2.872441in}{0.528000in}}%
\pgfpathlineto{\pgfqpoint{2.816980in}{0.577386in}}%
\pgfpathlineto{\pgfqpoint{2.788277in}{0.602667in}}%
\pgfpathlineto{\pgfqpoint{2.683798in}{0.697780in}}%
\pgfpathlineto{\pgfqpoint{2.625606in}{0.752000in}}%
\pgfpathlineto{\pgfqpoint{2.523475in}{0.849177in}}%
\pgfpathlineto{\pgfqpoint{2.403232in}{0.967108in}}%
\pgfpathlineto{\pgfqpoint{2.341101in}{1.030127in}}%
\pgfpathlineto{\pgfqpoint{2.292542in}{1.079103in}}%
\pgfpathlineto{\pgfqpoint{2.202828in}{1.172576in}}%
\pgfpathlineto{\pgfqpoint{2.141984in}{1.237333in}}%
\pgfpathlineto{\pgfqpoint{2.038971in}{1.349333in}}%
\pgfpathlineto{\pgfqpoint{1.939251in}{1.461333in}}%
\pgfpathlineto{\pgfqpoint{1.842101in}{1.573491in}}%
\pgfpathlineto{\pgfqpoint{1.721859in}{1.717708in}}%
\pgfpathlineto{\pgfqpoint{1.541151in}{1.946667in}}%
\pgfpathlineto{\pgfqpoint{1.481374in}{2.025967in}}%
\pgfpathlineto{\pgfqpoint{1.401212in}{2.135946in}}%
\pgfpathlineto{\pgfqpoint{1.339243in}{2.224945in}}%
\pgfpathlineto{\pgfqpoint{1.321051in}{2.250576in}}%
\pgfpathlineto{\pgfqpoint{1.240889in}{2.370710in}}%
\pgfpathlineto{\pgfqpoint{1.120646in}{2.563972in}}%
\pgfpathlineto{\pgfqpoint{1.040485in}{2.704920in}}%
\pgfpathlineto{\pgfqpoint{1.000404in}{2.780589in}}%
\pgfpathlineto{\pgfqpoint{0.933837in}{2.917333in}}%
\pgfpathlineto{\pgfqpoint{0.901047in}{2.992000in}}%
\pgfpathlineto{\pgfqpoint{0.895579in}{3.006361in}}%
\pgfpathlineto{\pgfqpoint{0.880162in}{3.043098in}}%
\pgfpathlineto{\pgfqpoint{0.843326in}{3.144356in}}%
\pgfpathlineto{\pgfqpoint{0.832201in}{3.178667in}}%
\pgfpathlineto{\pgfqpoint{0.801955in}{3.292488in}}%
\pgfpathlineto{\pgfqpoint{0.800000in}{3.301736in}}%
\pgfpathlineto{\pgfqpoint{0.800000in}{3.246137in}}%
\pgfpathlineto{\pgfqpoint{0.828390in}{3.152223in}}%
\pgfpathlineto{\pgfqpoint{0.848014in}{3.096611in}}%
\pgfpathlineto{\pgfqpoint{0.880162in}{3.015152in}}%
\pgfpathlineto{\pgfqpoint{0.922964in}{2.917333in}}%
\pgfpathlineto{\pgfqpoint{1.000404in}{2.761012in}}%
\pgfpathlineto{\pgfqpoint{1.100850in}{2.581333in}}%
\pgfpathlineto{\pgfqpoint{1.122907in}{2.544000in}}%
\pgfpathlineto{\pgfqpoint{1.168728in}{2.469333in}}%
\pgfpathlineto{\pgfqpoint{1.216198in}{2.394667in}}%
\pgfpathlineto{\pgfqpoint{1.256174in}{2.334238in}}%
\pgfpathlineto{\pgfqpoint{1.286776in}{2.288075in}}%
\pgfpathlineto{\pgfqpoint{1.306058in}{2.259298in}}%
\pgfpathlineto{\pgfqpoint{1.367720in}{2.170667in}}%
\pgfpathlineto{\pgfqpoint{1.413406in}{2.107358in}}%
\pgfpathlineto{\pgfqpoint{1.448480in}{2.058667in}}%
\pgfpathlineto{\pgfqpoint{1.521455in}{1.961055in}}%
\pgfpathlineto{\pgfqpoint{1.601616in}{1.857187in}}%
\pgfpathlineto{\pgfqpoint{1.663259in}{1.780084in}}%
\pgfpathlineto{\pgfqpoint{1.697660in}{1.737460in}}%
\pgfpathlineto{\pgfqpoint{1.739930in}{1.685333in}}%
\pgfpathlineto{\pgfqpoint{1.833736in}{1.573333in}}%
\pgfpathlineto{\pgfqpoint{1.930554in}{1.461333in}}%
\pgfpathlineto{\pgfqpoint{1.968383in}{1.418374in}}%
\pgfpathlineto{\pgfqpoint{2.082586in}{1.292048in}}%
\pgfpathlineto{\pgfqpoint{2.147177in}{1.222830in}}%
\pgfpathlineto{\pgfqpoint{2.184618in}{1.183038in}}%
\pgfpathlineto{\pgfqpoint{2.239022in}{1.125333in}}%
\pgfpathlineto{\pgfqpoint{2.323071in}{1.038840in}}%
\pgfpathlineto{\pgfqpoint{2.363152in}{0.998270in}}%
\pgfpathlineto{\pgfqpoint{2.483394in}{0.879128in}}%
\pgfpathlineto{\pgfqpoint{2.603636in}{0.763775in}}%
\pgfpathlineto{\pgfqpoint{2.723879in}{0.652137in}}%
\pgfpathlineto{\pgfqpoint{2.844121in}{0.544141in}}%
\pgfpathlineto{\pgfqpoint{2.862478in}{0.528000in}}%
\pgfpathmoveto{\pgfqpoint{4.768000in}{0.546484in}}%
\pgfpathlineto{\pgfqpoint{4.687838in}{0.687102in}}%
\pgfpathlineto{\pgfqpoint{4.625438in}{0.789333in}}%
\pgfpathlineto{\pgfqpoint{4.567596in}{0.879954in}}%
\pgfpathlineto{\pgfqpoint{4.503685in}{0.976000in}}%
\pgfpathlineto{\pgfqpoint{4.465832in}{1.030545in}}%
\pgfpathlineto{\pgfqpoint{4.447354in}{1.057848in}}%
\pgfpathlineto{\pgfqpoint{4.367192in}{1.170029in}}%
\pgfpathlineto{\pgfqpoint{4.287030in}{1.278085in}}%
\pgfpathlineto{\pgfqpoint{4.232460in}{1.349333in}}%
\pgfpathlineto{\pgfqpoint{4.166788in}{1.433282in}}%
\pgfpathlineto{\pgfqpoint{4.114392in}{1.498667in}}%
\pgfpathlineto{\pgfqpoint{4.022348in}{1.610667in}}%
\pgfpathlineto{\pgfqpoint{3.926303in}{1.724085in}}%
\pgfpathlineto{\pgfqpoint{3.806061in}{1.860965in}}%
\pgfpathlineto{\pgfqpoint{3.744186in}{1.929634in}}%
\pgfpathlineto{\pgfqpoint{3.645737in}{2.036137in}}%
\pgfpathlineto{\pgfqpoint{3.525495in}{2.162432in}}%
\pgfpathlineto{\pgfqpoint{3.444402in}{2.245333in}}%
\pgfpathlineto{\pgfqpoint{3.386601in}{2.302627in}}%
\pgfpathlineto{\pgfqpoint{3.331709in}{2.357333in}}%
\pgfpathlineto{\pgfqpoint{3.084606in}{2.591460in}}%
\pgfpathlineto{\pgfqpoint{3.044525in}{2.627985in}}%
\pgfpathlineto{\pgfqpoint{2.924283in}{2.735183in}}%
\pgfpathlineto{\pgfqpoint{2.864264in}{2.786762in}}%
\pgfpathlineto{\pgfqpoint{2.822495in}{2.822523in}}%
\pgfpathlineto{\pgfqpoint{2.799386in}{2.842667in}}%
\pgfpathlineto{\pgfqpoint{2.683798in}{2.938610in}}%
\pgfpathlineto{\pgfqpoint{2.563556in}{3.034990in}}%
\pgfpathlineto{\pgfqpoint{2.501008in}{3.083073in}}%
\pgfpathlineto{\pgfqpoint{2.474332in}{3.104000in}}%
\pgfpathlineto{\pgfqpoint{2.363152in}{3.187131in}}%
\pgfpathlineto{\pgfqpoint{2.300284in}{3.232109in}}%
\pgfpathlineto{\pgfqpoint{2.271106in}{3.253333in}}%
\pgfpathlineto{\pgfqpoint{2.161249in}{3.329396in}}%
\pgfpathlineto{\pgfqpoint{2.042505in}{3.407293in}}%
\pgfpathlineto{\pgfqpoint{1.973279in}{3.450186in}}%
\pgfpathlineto{\pgfqpoint{1.962343in}{3.457271in}}%
\pgfpathlineto{\pgfqpoint{1.865710in}{3.514667in}}%
\pgfpathlineto{\pgfqpoint{1.721859in}{3.593769in}}%
\pgfpathlineto{\pgfqpoint{1.546757in}{3.677765in}}%
\pgfpathlineto{\pgfqpoint{1.474266in}{3.707954in}}%
\pgfpathlineto{\pgfqpoint{1.390542in}{3.738667in}}%
\pgfpathlineto{\pgfqpoint{1.321051in}{3.760570in}}%
\pgfpathlineto{\pgfqpoint{1.307847in}{3.763701in}}%
\pgfpathlineto{\pgfqpoint{1.262283in}{3.776000in}}%
\pgfpathlineto{\pgfqpoint{1.200808in}{3.788778in}}%
\pgfpathlineto{\pgfqpoint{1.120646in}{3.798833in}}%
\pgfpathlineto{\pgfqpoint{1.080566in}{3.800495in}}%
\pgfpathlineto{\pgfqpoint{1.065778in}{3.799560in}}%
\pgfpathlineto{\pgfqpoint{1.040485in}{3.799347in}}%
\pgfpathlineto{\pgfqpoint{1.022754in}{3.796818in}}%
\pgfpathlineto{\pgfqpoint{1.000404in}{3.794819in}}%
\pgfpathlineto{\pgfqpoint{0.983317in}{3.791916in}}%
\pgfpathlineto{\pgfqpoint{0.960323in}{3.786178in}}%
\pgfpathlineto{\pgfqpoint{0.920242in}{3.772138in}}%
\pgfpathlineto{\pgfqpoint{0.880162in}{3.750027in}}%
\pgfpathlineto{\pgfqpoint{0.851677in}{3.727865in}}%
\pgfpathlineto{\pgfqpoint{0.840081in}{3.715987in}}%
\pgfpathlineto{\pgfqpoint{0.827465in}{3.701333in}}%
\pgfpathlineto{\pgfqpoint{0.816128in}{3.686311in}}%
\pgfpathlineto{\pgfqpoint{0.800000in}{3.658548in}}%
\pgfpathlineto{\pgfqpoint{0.800000in}{3.602744in}}%
\pgfpathlineto{\pgfqpoint{0.808945in}{3.626667in}}%
\pgfpathlineto{\pgfqpoint{0.818335in}{3.646922in}}%
\pgfpathlineto{\pgfqpoint{0.832635in}{3.670936in}}%
\pgfpathlineto{\pgfqpoint{0.840081in}{3.680803in}}%
\pgfpathlineto{\pgfqpoint{0.868373in}{3.712314in}}%
\pgfpathlineto{\pgfqpoint{0.880162in}{3.722006in}}%
\pgfpathlineto{\pgfqpoint{0.903998in}{3.738667in}}%
\pgfpathlineto{\pgfqpoint{0.920242in}{3.748020in}}%
\pgfpathlineto{\pgfqpoint{0.940146in}{3.757461in}}%
\pgfpathlineto{\pgfqpoint{0.960323in}{3.764806in}}%
\pgfpathlineto{\pgfqpoint{1.002386in}{3.776000in}}%
\pgfpathlineto{\pgfqpoint{1.046373in}{3.781484in}}%
\pgfpathlineto{\pgfqpoint{1.080566in}{3.783648in}}%
\pgfpathlineto{\pgfqpoint{1.120646in}{3.782931in}}%
\pgfpathlineto{\pgfqpoint{1.164093in}{3.779135in}}%
\pgfpathlineto{\pgfqpoint{1.200808in}{3.774394in}}%
\pgfpathlineto{\pgfqpoint{1.280970in}{3.757932in}}%
\pgfpathlineto{\pgfqpoint{1.296698in}{3.753316in}}%
\pgfpathlineto{\pgfqpoint{1.328059in}{3.745194in}}%
\pgfpathlineto{\pgfqpoint{1.361131in}{3.735828in}}%
\pgfpathlineto{\pgfqpoint{1.481374in}{3.693261in}}%
\pgfpathlineto{\pgfqpoint{1.521455in}{3.676994in}}%
\pgfpathlineto{\pgfqpoint{1.601616in}{3.641848in}}%
\pgfpathlineto{\pgfqpoint{1.681778in}{3.603457in}}%
\pgfpathlineto{\pgfqpoint{1.709763in}{3.589333in}}%
\pgfpathlineto{\pgfqpoint{1.768679in}{3.558278in}}%
\pgfpathlineto{\pgfqpoint{1.802020in}{3.540453in}}%
\pgfpathlineto{\pgfqpoint{1.882182in}{3.495155in}}%
\pgfpathlineto{\pgfqpoint{1.974616in}{3.440000in}}%
\pgfpathlineto{\pgfqpoint{2.162747in}{3.318921in}}%
\pgfpathlineto{\pgfqpoint{2.258027in}{3.253333in}}%
\pgfpathlineto{\pgfqpoint{2.363152in}{3.178170in}}%
\pgfpathlineto{\pgfqpoint{2.429499in}{3.128466in}}%
\pgfpathlineto{\pgfqpoint{2.462476in}{3.104000in}}%
\pgfpathlineto{\pgfqpoint{2.559458in}{3.029333in}}%
\pgfpathlineto{\pgfqpoint{2.606736in}{2.992000in}}%
\pgfpathlineto{\pgfqpoint{2.653182in}{2.954667in}}%
\pgfpathlineto{\pgfqpoint{2.763960in}{2.863584in}}%
\pgfpathlineto{\pgfqpoint{2.884202in}{2.761304in}}%
\pgfpathlineto{\pgfqpoint{2.924283in}{2.726414in}}%
\pgfpathlineto{\pgfqpoint{3.045219in}{2.618667in}}%
\pgfpathlineto{\pgfqpoint{3.105218in}{2.563199in}}%
\pgfpathlineto{\pgfqpoint{3.145399in}{2.525959in}}%
\pgfpathlineto{\pgfqpoint{3.185358in}{2.488513in}}%
\pgfpathlineto{\pgfqpoint{3.206303in}{2.469333in}}%
\pgfpathlineto{\pgfqpoint{3.325091in}{2.354923in}}%
\pgfpathlineto{\pgfqpoint{3.445333in}{2.235206in}}%
\pgfpathlineto{\pgfqpoint{3.497358in}{2.181791in}}%
\pgfpathlineto{\pgfqpoint{3.544628in}{2.133333in}}%
\pgfpathlineto{\pgfqpoint{3.650880in}{2.021333in}}%
\pgfpathlineto{\pgfqpoint{3.753826in}{1.909333in}}%
\pgfpathlineto{\pgfqpoint{3.813996in}{1.842058in}}%
\pgfpathlineto{\pgfqpoint{3.853836in}{1.797333in}}%
\pgfpathlineto{\pgfqpoint{3.950743in}{1.685333in}}%
\pgfpathlineto{\pgfqpoint{4.046545in}{1.571313in}}%
\pgfpathlineto{\pgfqpoint{4.086626in}{1.522279in}}%
\pgfpathlineto{\pgfqpoint{4.166788in}{1.422467in}}%
\pgfpathlineto{\pgfqpoint{4.233231in}{1.336555in}}%
\pgfpathlineto{\pgfqpoint{4.266520in}{1.292895in}}%
\pgfpathlineto{\pgfqpoint{4.299620in}{1.249060in}}%
\pgfpathlineto{\pgfqpoint{4.336415in}{1.200000in}}%
\pgfpathlineto{\pgfqpoint{4.407273in}{1.101984in}}%
\pgfpathlineto{\pgfqpoint{4.544425in}{0.901333in}}%
\pgfpathlineto{\pgfqpoint{4.607677in}{0.802864in}}%
\pgfpathlineto{\pgfqpoint{4.661800in}{0.714667in}}%
\pgfpathlineto{\pgfqpoint{4.705696in}{0.640000in}}%
\pgfpathlineto{\pgfqpoint{4.747726in}{0.565333in}}%
\pgfpathlineto{\pgfqpoint{4.754286in}{0.552560in}}%
\pgfpathlineto{\pgfqpoint{4.768000in}{0.528379in}}%
\pgfpathlineto{\pgfqpoint{4.768000in}{0.528379in}}%
\pgfusepath{fill}%
\end{pgfscope}%
\begin{pgfscope}%
\pgfpathrectangle{\pgfqpoint{0.800000in}{0.528000in}}{\pgfqpoint{3.968000in}{3.696000in}}%
\pgfusepath{clip}%
\pgfsetbuttcap%
\pgfsetroundjoin%
\definecolor{currentfill}{rgb}{0.282656,0.100196,0.422160}%
\pgfsetfillcolor{currentfill}%
\pgfsetlinewidth{0.000000pt}%
\definecolor{currentstroke}{rgb}{0.000000,0.000000,0.000000}%
\pgfsetstrokecolor{currentstroke}%
\pgfsetdash{}{0pt}%
\pgfpathmoveto{\pgfqpoint{2.862478in}{0.528000in}}%
\pgfpathlineto{\pgfqpoint{2.804040in}{0.579740in}}%
\pgfpathlineto{\pgfqpoint{2.683798in}{0.688942in}}%
\pgfpathlineto{\pgfqpoint{2.630103in}{0.739319in}}%
\pgfpathlineto{\pgfqpoint{2.576683in}{0.789333in}}%
\pgfpathlineto{\pgfqpoint{2.523475in}{0.840259in}}%
\pgfpathlineto{\pgfqpoint{2.403232in}{0.958130in}}%
\pgfpathlineto{\pgfqpoint{2.336391in}{1.025741in}}%
\pgfpathlineto{\pgfqpoint{2.311491in}{1.050667in}}%
\pgfpathlineto{\pgfqpoint{2.202828in}{1.163174in}}%
\pgfpathlineto{\pgfqpoint{2.147177in}{1.222830in}}%
\pgfpathlineto{\pgfqpoint{2.098607in}{1.274667in}}%
\pgfpathlineto{\pgfqpoint{2.054495in}{1.323168in}}%
\pgfpathlineto{\pgfqpoint{2.030381in}{1.349333in}}%
\pgfpathlineto{\pgfqpoint{1.930554in}{1.461333in}}%
\pgfpathlineto{\pgfqpoint{1.833736in}{1.573333in}}%
\pgfpathlineto{\pgfqpoint{1.739930in}{1.685333in}}%
\pgfpathlineto{\pgfqpoint{1.648945in}{1.797333in}}%
\pgfpathlineto{\pgfqpoint{1.612024in}{1.844361in}}%
\pgfpathlineto{\pgfqpoint{1.578312in}{1.887627in}}%
\pgfpathlineto{\pgfqpoint{1.544789in}{1.931068in}}%
\pgfpathlineto{\pgfqpoint{1.504133in}{1.984000in}}%
\pgfpathlineto{\pgfqpoint{1.441293in}{2.068474in}}%
\pgfpathlineto{\pgfqpoint{1.361131in}{2.180027in}}%
\pgfpathlineto{\pgfqpoint{1.210614in}{2.403801in}}%
\pgfpathlineto{\pgfqpoint{1.179771in}{2.451595in}}%
\pgfpathlineto{\pgfqpoint{1.120646in}{2.547795in}}%
\pgfpathlineto{\pgfqpoint{1.016642in}{2.730667in}}%
\pgfpathlineto{\pgfqpoint{1.000404in}{2.761012in}}%
\pgfpathlineto{\pgfqpoint{0.977557in}{2.805333in}}%
\pgfpathlineto{\pgfqpoint{0.940683in}{2.880000in}}%
\pgfpathlineto{\pgfqpoint{0.934740in}{2.893504in}}%
\pgfpathlineto{\pgfqpoint{0.920242in}{2.923341in}}%
\pgfpathlineto{\pgfqpoint{0.874265in}{3.029333in}}%
\pgfpathlineto{\pgfqpoint{0.828390in}{3.152223in}}%
\pgfpathlineto{\pgfqpoint{0.806949in}{3.222472in}}%
\pgfpathlineto{\pgfqpoint{0.800000in}{3.246137in}}%
\pgfpathlineto{\pgfqpoint{0.800000in}{3.202975in}}%
\pgfpathlineto{\pgfqpoint{0.833771in}{3.104000in}}%
\pgfpathlineto{\pgfqpoint{0.840081in}{3.087452in}}%
\pgfpathlineto{\pgfqpoint{0.863219in}{3.029333in}}%
\pgfpathlineto{\pgfqpoint{0.868260in}{3.018248in}}%
\pgfpathlineto{\pgfqpoint{0.880162in}{2.988898in}}%
\pgfpathlineto{\pgfqpoint{0.948504in}{2.842667in}}%
\pgfpathlineto{\pgfqpoint{0.960323in}{2.819126in}}%
\pgfpathlineto{\pgfqpoint{1.000404in}{2.742324in}}%
\pgfpathlineto{\pgfqpoint{1.080566in}{2.599595in}}%
\pgfpathlineto{\pgfqpoint{1.183100in}{2.432000in}}%
\pgfpathlineto{\pgfqpoint{1.240889in}{2.342965in}}%
\pgfpathlineto{\pgfqpoint{1.392001in}{2.124754in}}%
\pgfpathlineto{\pgfqpoint{1.436156in}{2.063452in}}%
\pgfpathlineto{\pgfqpoint{1.495456in}{1.984000in}}%
\pgfpathlineto{\pgfqpoint{1.539760in}{1.926384in}}%
\pgfpathlineto{\pgfqpoint{1.581460in}{1.872000in}}%
\pgfpathlineto{\pgfqpoint{1.641697in}{1.795575in}}%
\pgfpathlineto{\pgfqpoint{1.692732in}{1.732870in}}%
\pgfpathlineto{\pgfqpoint{1.731331in}{1.685333in}}%
\pgfpathlineto{\pgfqpoint{1.825243in}{1.573333in}}%
\pgfpathlineto{\pgfqpoint{1.922263in}{1.460886in}}%
\pgfpathlineto{\pgfqpoint{1.962343in}{1.415596in}}%
\pgfpathlineto{\pgfqpoint{2.082586in}{1.282580in}}%
\pgfpathlineto{\pgfqpoint{2.142542in}{1.218513in}}%
\pgfpathlineto{\pgfqpoint{2.194675in}{1.162667in}}%
\pgfpathlineto{\pgfqpoint{2.282990in}{1.070804in}}%
\pgfpathlineto{\pgfqpoint{2.403232in}{0.949153in}}%
\pgfpathlineto{\pgfqpoint{2.523475in}{0.831342in}}%
\pgfpathlineto{\pgfqpoint{2.567321in}{0.789333in}}%
\pgfpathlineto{\pgfqpoint{2.686810in}{0.677333in}}%
\pgfpathlineto{\pgfqpoint{2.746259in}{0.623513in}}%
\pgfpathlineto{\pgfqpoint{2.768709in}{0.602667in}}%
\pgfpathlineto{\pgfqpoint{2.852515in}{0.528000in}}%
\pgfpathmoveto{\pgfqpoint{4.768000in}{0.564588in}}%
\pgfpathlineto{\pgfqpoint{4.703058in}{0.677333in}}%
\pgfpathlineto{\pgfqpoint{4.647758in}{0.768457in}}%
\pgfpathlineto{\pgfqpoint{4.527515in}{0.954062in}}%
\pgfpathlineto{\pgfqpoint{4.447354in}{1.070361in}}%
\pgfpathlineto{\pgfqpoint{4.367192in}{1.181907in}}%
\pgfpathlineto{\pgfqpoint{4.287030in}{1.289388in}}%
\pgfpathlineto{\pgfqpoint{4.241156in}{1.349333in}}%
\pgfpathlineto{\pgfqpoint{4.166788in}{1.444035in}}%
\pgfpathlineto{\pgfqpoint{4.111842in}{1.512512in}}%
\pgfpathlineto{\pgfqpoint{4.006465in}{1.639896in}}%
\pgfpathlineto{\pgfqpoint{3.926303in}{1.733884in}}%
\pgfpathlineto{\pgfqpoint{3.838084in}{1.834667in}}%
\pgfpathlineto{\pgfqpoint{3.798887in}{1.878682in}}%
\pgfpathlineto{\pgfqpoint{3.685818in}{2.002449in}}%
\pgfpathlineto{\pgfqpoint{3.565576in}{2.130035in}}%
\pgfpathlineto{\pgfqpoint{3.485414in}{2.212680in}}%
\pgfpathlineto{\pgfqpoint{3.429927in}{2.268317in}}%
\pgfpathlineto{\pgfqpoint{3.378577in}{2.320000in}}%
\pgfpathlineto{\pgfqpoint{3.293774in}{2.402830in}}%
\pgfpathlineto{\pgfqpoint{3.244929in}{2.450120in}}%
\pgfpathlineto{\pgfqpoint{3.124687in}{2.563261in}}%
\pgfpathlineto{\pgfqpoint{3.054154in}{2.627636in}}%
\pgfpathlineto{\pgfqpoint{3.004444in}{2.672779in}}%
\pgfpathlineto{\pgfqpoint{2.964364in}{2.708493in}}%
\pgfpathlineto{\pgfqpoint{2.844121in}{2.813290in}}%
\pgfpathlineto{\pgfqpoint{2.804040in}{2.847450in}}%
\pgfpathlineto{\pgfqpoint{2.683798in}{2.947413in}}%
\pgfpathlineto{\pgfqpoint{2.563556in}{3.043736in}}%
\pgfpathlineto{\pgfqpoint{2.506216in}{3.087925in}}%
\pgfpathlineto{\pgfqpoint{2.471206in}{3.115352in}}%
\pgfpathlineto{\pgfqpoint{2.363152in}{3.196074in}}%
\pgfpathlineto{\pgfqpoint{2.305786in}{3.237233in}}%
\pgfpathlineto{\pgfqpoint{2.279314in}{3.256757in}}%
\pgfpathlineto{\pgfqpoint{2.162747in}{3.337536in}}%
\pgfpathlineto{\pgfqpoint{2.064258in}{3.402667in}}%
\pgfpathlineto{\pgfqpoint{1.882182in}{3.515195in}}%
\pgfpathlineto{\pgfqpoint{1.705932in}{3.611832in}}%
\pgfpathlineto{\pgfqpoint{1.678149in}{3.626667in}}%
\pgfpathlineto{\pgfqpoint{1.601529in}{3.664081in}}%
\pgfpathlineto{\pgfqpoint{1.518297in}{3.701333in}}%
\pgfpathlineto{\pgfqpoint{1.436362in}{3.734074in}}%
\pgfpathlineto{\pgfqpoint{1.401212in}{3.747210in}}%
\pgfpathlineto{\pgfqpoint{1.280970in}{3.784793in}}%
\pgfpathlineto{\pgfqpoint{1.191115in}{3.804305in}}%
\pgfpathlineto{\pgfqpoint{1.160727in}{3.809865in}}%
\pgfpathlineto{\pgfqpoint{1.120646in}{3.814656in}}%
\pgfpathlineto{\pgfqpoint{1.076315in}{3.817292in}}%
\pgfpathlineto{\pgfqpoint{1.040485in}{3.817010in}}%
\pgfpathlineto{\pgfqpoint{0.996836in}{3.813333in}}%
\pgfpathlineto{\pgfqpoint{0.960323in}{3.806682in}}%
\pgfpathlineto{\pgfqpoint{0.949155in}{3.802931in}}%
\pgfpathlineto{\pgfqpoint{0.920242in}{3.794564in}}%
\pgfpathlineto{\pgfqpoint{0.906113in}{3.789161in}}%
\pgfpathlineto{\pgfqpoint{0.879568in}{3.776000in}}%
\pgfpathlineto{\pgfqpoint{0.835078in}{3.743326in}}%
\pgfpathlineto{\pgfqpoint{0.830622in}{3.738667in}}%
\pgfpathlineto{\pgfqpoint{0.800000in}{3.701333in}}%
\pgfpathlineto{\pgfqpoint{0.800000in}{3.658548in}}%
\pgfpathlineto{\pgfqpoint{0.802741in}{3.664000in}}%
\pgfpathlineto{\pgfqpoint{0.816128in}{3.686311in}}%
\pgfpathlineto{\pgfqpoint{0.827465in}{3.701333in}}%
\pgfpathlineto{\pgfqpoint{0.851677in}{3.727865in}}%
\pgfpathlineto{\pgfqpoint{0.880162in}{3.750027in}}%
\pgfpathlineto{\pgfqpoint{0.930019in}{3.776000in}}%
\pgfpathlineto{\pgfqpoint{0.960323in}{3.786178in}}%
\pgfpathlineto{\pgfqpoint{0.983317in}{3.791916in}}%
\pgfpathlineto{\pgfqpoint{1.000404in}{3.794819in}}%
\pgfpathlineto{\pgfqpoint{1.065778in}{3.799560in}}%
\pgfpathlineto{\pgfqpoint{1.094462in}{3.800390in}}%
\pgfpathlineto{\pgfqpoint{1.137559in}{3.797580in}}%
\pgfpathlineto{\pgfqpoint{1.177604in}{3.791720in}}%
\pgfpathlineto{\pgfqpoint{1.200808in}{3.788778in}}%
\pgfpathlineto{\pgfqpoint{1.245043in}{3.779869in}}%
\pgfpathlineto{\pgfqpoint{1.280970in}{3.771590in}}%
\pgfpathlineto{\pgfqpoint{1.390542in}{3.738667in}}%
\pgfpathlineto{\pgfqpoint{1.441293in}{3.720481in}}%
\pgfpathlineto{\pgfqpoint{1.490484in}{3.701333in}}%
\pgfpathlineto{\pgfqpoint{1.601616in}{3.652945in}}%
\pgfpathlineto{\pgfqpoint{1.656510in}{3.626667in}}%
\pgfpathlineto{\pgfqpoint{1.730233in}{3.589333in}}%
\pgfpathlineto{\pgfqpoint{1.805114in}{3.549118in}}%
\pgfpathlineto{\pgfqpoint{1.909538in}{3.489186in}}%
\pgfpathlineto{\pgfqpoint{2.002424in}{3.432562in}}%
\pgfpathlineto{\pgfqpoint{2.068827in}{3.389851in}}%
\pgfpathlineto{\pgfqpoint{2.107065in}{3.365333in}}%
\pgfpathlineto{\pgfqpoint{2.282990in}{3.244962in}}%
\pgfpathlineto{\pgfqpoint{2.345459in}{3.199520in}}%
\pgfpathlineto{\pgfqpoint{2.374622in}{3.178667in}}%
\pgfpathlineto{\pgfqpoint{2.443313in}{3.127505in}}%
\pgfpathlineto{\pgfqpoint{2.523475in}{3.066345in}}%
\pgfpathlineto{\pgfqpoint{2.570711in}{3.029333in}}%
\pgfpathlineto{\pgfqpoint{2.683798in}{2.938610in}}%
\pgfpathlineto{\pgfqpoint{2.804040in}{2.838748in}}%
\pgfpathlineto{\pgfqpoint{2.864264in}{2.786762in}}%
\pgfpathlineto{\pgfqpoint{2.905797in}{2.750781in}}%
\pgfpathlineto{\pgfqpoint{2.929415in}{2.730667in}}%
\pgfpathlineto{\pgfqpoint{3.054767in}{2.618667in}}%
\pgfpathlineto{\pgfqpoint{3.110000in}{2.567653in}}%
\pgfpathlineto{\pgfqpoint{3.150165in}{2.530398in}}%
\pgfpathlineto{\pgfqpoint{3.190109in}{2.492937in}}%
\pgfpathlineto{\pgfqpoint{3.215526in}{2.469333in}}%
\pgfpathlineto{\pgfqpoint{3.331709in}{2.357333in}}%
\pgfpathlineto{\pgfqpoint{3.445333in}{2.244394in}}%
\pgfpathlineto{\pgfqpoint{3.501987in}{2.186104in}}%
\pgfpathlineto{\pgfqpoint{3.553518in}{2.133333in}}%
\pgfpathlineto{\pgfqpoint{3.645737in}{2.036137in}}%
\pgfpathlineto{\pgfqpoint{3.708610in}{1.967896in}}%
\pgfpathlineto{\pgfqpoint{3.744186in}{1.929634in}}%
\pgfpathlineto{\pgfqpoint{3.846141in}{1.815830in}}%
\pgfpathlineto{\pgfqpoint{3.895107in}{1.760000in}}%
\pgfpathlineto{\pgfqpoint{3.991019in}{1.648000in}}%
\pgfpathlineto{\pgfqpoint{4.086626in}{1.532974in}}%
\pgfpathlineto{\pgfqpoint{4.174142in}{1.424000in}}%
\pgfpathlineto{\pgfqpoint{4.206869in}{1.382413in}}%
\pgfpathlineto{\pgfqpoint{4.289618in}{1.274667in}}%
\pgfpathlineto{\pgfqpoint{4.327111in}{1.224457in}}%
\pgfpathlineto{\pgfqpoint{4.407273in}{1.114459in}}%
\pgfpathlineto{\pgfqpoint{4.567596in}{0.879954in}}%
\pgfpathlineto{\pgfqpoint{4.625438in}{0.789333in}}%
\pgfpathlineto{\pgfqpoint{4.662531in}{0.728427in}}%
\pgfpathlineto{\pgfqpoint{4.687838in}{0.687102in}}%
\pgfpathlineto{\pgfqpoint{4.727919in}{0.618234in}}%
\pgfpathlineto{\pgfqpoint{4.768000in}{0.546484in}}%
\pgfpathlineto{\pgfqpoint{4.768000in}{0.546484in}}%
\pgfusepath{fill}%
\end{pgfscope}%
\begin{pgfscope}%
\pgfpathrectangle{\pgfqpoint{0.800000in}{0.528000in}}{\pgfqpoint{3.968000in}{3.696000in}}%
\pgfusepath{clip}%
\pgfsetbuttcap%
\pgfsetroundjoin%
\definecolor{currentfill}{rgb}{0.282656,0.100196,0.422160}%
\pgfsetfillcolor{currentfill}%
\pgfsetlinewidth{0.000000pt}%
\definecolor{currentstroke}{rgb}{0.000000,0.000000,0.000000}%
\pgfsetstrokecolor{currentstroke}%
\pgfsetdash{}{0pt}%
\pgfpathmoveto{\pgfqpoint{2.852515in}{0.528000in}}%
\pgfpathlineto{\pgfqpoint{2.603636in}{0.754897in}}%
\pgfpathlineto{\pgfqpoint{2.483394in}{0.870190in}}%
\pgfpathlineto{\pgfqpoint{2.363152in}{0.989273in}}%
\pgfpathlineto{\pgfqpoint{2.302662in}{1.050667in}}%
\pgfpathlineto{\pgfqpoint{2.194675in}{1.162667in}}%
\pgfpathlineto{\pgfqpoint{2.142542in}{1.218513in}}%
\pgfpathlineto{\pgfqpoint{2.089880in}{1.274667in}}%
\pgfpathlineto{\pgfqpoint{2.042505in}{1.326446in}}%
\pgfpathlineto{\pgfqpoint{1.954892in}{1.424000in}}%
\pgfpathlineto{\pgfqpoint{1.920135in}{1.463315in}}%
\pgfpathlineto{\pgfqpoint{1.802020in}{1.600652in}}%
\pgfpathlineto{\pgfqpoint{1.721859in}{1.696832in}}%
\pgfpathlineto{\pgfqpoint{1.658348in}{1.775510in}}%
\pgfpathlineto{\pgfqpoint{1.636206in}{1.802448in}}%
\pgfpathlineto{\pgfqpoint{1.552409in}{1.909333in}}%
\pgfpathlineto{\pgfqpoint{1.506441in}{1.970015in}}%
\pgfpathlineto{\pgfqpoint{1.467466in}{2.021333in}}%
\pgfpathlineto{\pgfqpoint{1.401212in}{2.111573in}}%
\pgfpathlineto{\pgfqpoint{1.250507in}{2.328959in}}%
\pgfpathlineto{\pgfqpoint{1.231393in}{2.357333in}}%
\pgfpathlineto{\pgfqpoint{1.200808in}{2.404260in}}%
\pgfpathlineto{\pgfqpoint{1.136367in}{2.506667in}}%
\pgfpathlineto{\pgfqpoint{1.080566in}{2.599595in}}%
\pgfpathlineto{\pgfqpoint{1.000404in}{2.742324in}}%
\pgfpathlineto{\pgfqpoint{0.960323in}{2.819126in}}%
\pgfpathlineto{\pgfqpoint{0.895425in}{2.954667in}}%
\pgfpathlineto{\pgfqpoint{0.891223in}{2.964970in}}%
\pgfpathlineto{\pgfqpoint{0.878792in}{2.992000in}}%
\pgfpathlineto{\pgfqpoint{0.830337in}{3.113076in}}%
\pgfpathlineto{\pgfqpoint{0.806130in}{3.184376in}}%
\pgfpathlineto{\pgfqpoint{0.800000in}{3.202975in}}%
\pgfpathlineto{\pgfqpoint{0.800000in}{3.166070in}}%
\pgfpathlineto{\pgfqpoint{0.822429in}{3.104000in}}%
\pgfpathlineto{\pgfqpoint{0.827404in}{3.092192in}}%
\pgfpathlineto{\pgfqpoint{0.840081in}{3.058609in}}%
\pgfpathlineto{\pgfqpoint{0.884661in}{2.954667in}}%
\pgfpathlineto{\pgfqpoint{0.902052in}{2.917333in}}%
\pgfpathlineto{\pgfqpoint{0.938357in}{2.842667in}}%
\pgfpathlineto{\pgfqpoint{0.957205in}{2.805333in}}%
\pgfpathlineto{\pgfqpoint{1.000404in}{2.724099in}}%
\pgfpathlineto{\pgfqpoint{1.059986in}{2.618667in}}%
\pgfpathlineto{\pgfqpoint{1.096049in}{2.558422in}}%
\pgfpathlineto{\pgfqpoint{1.120646in}{2.517088in}}%
\pgfpathlineto{\pgfqpoint{1.173964in}{2.432000in}}%
\pgfpathlineto{\pgfqpoint{1.222450in}{2.357333in}}%
\pgfpathlineto{\pgfqpoint{1.280970in}{2.270099in}}%
\pgfpathlineto{\pgfqpoint{1.361131in}{2.155270in}}%
\pgfpathlineto{\pgfqpoint{1.543906in}{1.909333in}}%
\pgfpathlineto{\pgfqpoint{1.631890in}{1.797333in}}%
\pgfpathlineto{\pgfqpoint{1.661932in}{1.760000in}}%
\pgfpathlineto{\pgfqpoint{1.753798in}{1.648000in}}%
\pgfpathlineto{\pgfqpoint{1.802020in}{1.590660in}}%
\pgfpathlineto{\pgfqpoint{1.882182in}{1.497076in}}%
\pgfpathlineto{\pgfqpoint{2.002424in}{1.361270in}}%
\pgfpathlineto{\pgfqpoint{2.082586in}{1.273165in}}%
\pgfpathlineto{\pgfqpoint{2.137908in}{1.214197in}}%
\pgfpathlineto{\pgfqpoint{2.186056in}{1.162667in}}%
\pgfpathlineto{\pgfqpoint{2.293833in}{1.050667in}}%
\pgfpathlineto{\pgfqpoint{2.346386in}{0.997717in}}%
\pgfpathlineto{\pgfqpoint{2.403232in}{0.940176in}}%
\pgfpathlineto{\pgfqpoint{2.519225in}{0.826667in}}%
\pgfpathlineto{\pgfqpoint{2.763960in}{0.598285in}}%
\pgfpathlineto{\pgfqpoint{2.842615in}{0.528000in}}%
\pgfpathlineto{\pgfqpoint{2.844121in}{0.528000in}}%
\pgfpathmoveto{\pgfqpoint{4.768000in}{0.581581in}}%
\pgfpathlineto{\pgfqpoint{4.712462in}{0.677333in}}%
\pgfpathlineto{\pgfqpoint{4.667150in}{0.752000in}}%
\pgfpathlineto{\pgfqpoint{4.607677in}{0.846169in}}%
\pgfpathlineto{\pgfqpoint{4.469940in}{1.050667in}}%
\pgfpathlineto{\pgfqpoint{4.428850in}{1.108098in}}%
\pgfpathlineto{\pgfqpoint{4.407273in}{1.138775in}}%
\pgfpathlineto{\pgfqpoint{4.327111in}{1.247679in}}%
\pgfpathlineto{\pgfqpoint{4.265012in}{1.328824in}}%
\pgfpathlineto{\pgfqpoint{4.246949in}{1.352941in}}%
\pgfpathlineto{\pgfqpoint{4.146216in}{1.480495in}}%
\pgfpathlineto{\pgfqpoint{4.046545in}{1.602152in}}%
\pgfpathlineto{\pgfqpoint{3.966384in}{1.697113in}}%
\pgfpathlineto{\pgfqpoint{3.900310in}{1.773122in}}%
\pgfpathlineto{\pgfqpoint{3.864234in}{1.814186in}}%
\pgfpathlineto{\pgfqpoint{3.813338in}{1.872000in}}%
\pgfpathlineto{\pgfqpoint{3.736445in}{1.956490in}}%
\pgfpathlineto{\pgfqpoint{3.685818in}{2.011765in}}%
\pgfpathlineto{\pgfqpoint{3.571090in}{2.133333in}}%
\pgfpathlineto{\pgfqpoint{3.498699in}{2.208000in}}%
\pgfpathlineto{\pgfqpoint{3.405253in}{2.302366in}}%
\pgfpathlineto{\pgfqpoint{3.337523in}{2.368913in}}%
\pgfpathlineto{\pgfqpoint{3.285010in}{2.420390in}}%
\pgfpathlineto{\pgfqpoint{3.164768in}{2.534697in}}%
\pgfpathlineto{\pgfqpoint{3.099255in}{2.594978in}}%
\pgfpathlineto{\pgfqpoint{3.044525in}{2.645358in}}%
\pgfpathlineto{\pgfqpoint{2.924283in}{2.752444in}}%
\pgfpathlineto{\pgfqpoint{2.874020in}{2.795849in}}%
\pgfpathlineto{\pgfqpoint{2.819743in}{2.842667in}}%
\pgfpathlineto{\pgfqpoint{2.723879in}{2.923164in}}%
\pgfpathlineto{\pgfqpoint{2.639588in}{2.992000in}}%
\pgfpathlineto{\pgfqpoint{2.576777in}{3.041649in}}%
\pgfpathlineto{\pgfqpoint{2.545429in}{3.066667in}}%
\pgfpathlineto{\pgfqpoint{2.443313in}{3.145336in}}%
\pgfpathlineto{\pgfqpoint{2.348100in}{3.216000in}}%
\pgfpathlineto{\pgfqpoint{2.242909in}{3.291507in}}%
\pgfpathlineto{\pgfqpoint{2.190116in}{3.328000in}}%
\pgfpathlineto{\pgfqpoint{2.078843in}{3.402667in}}%
\pgfpathlineto{\pgfqpoint{1.882182in}{3.524859in}}%
\pgfpathlineto{\pgfqpoint{1.835431in}{3.552000in}}%
\pgfpathlineto{\pgfqpoint{1.761939in}{3.593039in}}%
\pgfpathlineto{\pgfqpoint{1.578321in}{3.685698in}}%
\pgfpathlineto{\pgfqpoint{1.501826in}{3.719616in}}%
\pgfpathlineto{\pgfqpoint{1.441293in}{3.744168in}}%
\pgfpathlineto{\pgfqpoint{1.321051in}{3.786130in}}%
\pgfpathlineto{\pgfqpoint{1.280970in}{3.797780in}}%
\pgfpathlineto{\pgfqpoint{1.267514in}{3.800800in}}%
\pgfpathlineto{\pgfqpoint{1.240889in}{3.808254in}}%
\pgfpathlineto{\pgfqpoint{1.195500in}{3.818278in}}%
\pgfpathlineto{\pgfqpoint{1.146513in}{3.826573in}}%
\pgfpathlineto{\pgfqpoint{1.120646in}{3.829656in}}%
\pgfpathlineto{\pgfqpoint{1.040485in}{3.833787in}}%
\pgfpathlineto{\pgfqpoint{0.975908in}{3.827850in}}%
\pgfpathlineto{\pgfqpoint{0.948824in}{3.824044in}}%
\pgfpathlineto{\pgfqpoint{0.911523in}{3.813333in}}%
\pgfpathlineto{\pgfqpoint{0.880162in}{3.800326in}}%
\pgfpathlineto{\pgfqpoint{0.863351in}{3.791659in}}%
\pgfpathlineto{\pgfqpoint{0.839359in}{3.776000in}}%
\pgfpathlineto{\pgfqpoint{0.818698in}{3.758584in}}%
\pgfpathlineto{\pgfqpoint{0.800000in}{3.738667in}}%
\pgfpathlineto{\pgfqpoint{0.800000in}{3.702178in}}%
\pgfpathlineto{\pgfqpoint{0.835078in}{3.743326in}}%
\pgfpathlineto{\pgfqpoint{0.840081in}{3.747544in}}%
\pgfpathlineto{\pgfqpoint{0.880162in}{3.776353in}}%
\pgfpathlineto{\pgfqpoint{0.906113in}{3.789161in}}%
\pgfpathlineto{\pgfqpoint{0.920242in}{3.794564in}}%
\pgfpathlineto{\pgfqpoint{0.965955in}{3.808087in}}%
\pgfpathlineto{\pgfqpoint{1.001028in}{3.813914in}}%
\pgfpathlineto{\pgfqpoint{1.040485in}{3.817010in}}%
\pgfpathlineto{\pgfqpoint{1.084181in}{3.816701in}}%
\pgfpathlineto{\pgfqpoint{1.131741in}{3.813333in}}%
\pgfpathlineto{\pgfqpoint{1.160727in}{3.809865in}}%
\pgfpathlineto{\pgfqpoint{1.240889in}{3.794643in}}%
\pgfpathlineto{\pgfqpoint{1.256278in}{3.790335in}}%
\pgfpathlineto{\pgfqpoint{1.280970in}{3.784793in}}%
\pgfpathlineto{\pgfqpoint{1.321051in}{3.773599in}}%
\pgfpathlineto{\pgfqpoint{1.441293in}{3.732449in}}%
\pgfpathlineto{\pgfqpoint{1.481374in}{3.716577in}}%
\pgfpathlineto{\pgfqpoint{1.524034in}{3.698930in}}%
\pgfpathlineto{\pgfqpoint{1.601699in}{3.664000in}}%
\pgfpathlineto{\pgfqpoint{1.681778in}{3.624871in}}%
\pgfpathlineto{\pgfqpoint{1.857415in}{3.528931in}}%
\pgfpathlineto{\pgfqpoint{1.883069in}{3.514667in}}%
\pgfpathlineto{\pgfqpoint{2.002424in}{3.442215in}}%
\pgfpathlineto{\pgfqpoint{2.051236in}{3.410799in}}%
\pgfpathlineto{\pgfqpoint{2.082586in}{3.390812in}}%
\pgfpathlineto{\pgfqpoint{2.176714in}{3.328000in}}%
\pgfpathlineto{\pgfqpoint{2.284123in}{3.253333in}}%
\pgfpathlineto{\pgfqpoint{2.350940in}{3.204625in}}%
\pgfpathlineto{\pgfqpoint{2.386742in}{3.178667in}}%
\pgfpathlineto{\pgfqpoint{2.603636in}{3.012011in}}%
\pgfpathlineto{\pgfqpoint{2.643717in}{2.979904in}}%
\pgfpathlineto{\pgfqpoint{2.765419in}{2.880000in}}%
\pgfpathlineto{\pgfqpoint{2.884202in}{2.778745in}}%
\pgfpathlineto{\pgfqpoint{3.004444in}{2.672779in}}%
\pgfpathlineto{\pgfqpoint{3.124687in}{2.563261in}}%
\pgfpathlineto{\pgfqpoint{3.194859in}{2.497362in}}%
\pgfpathlineto{\pgfqpoint{3.224750in}{2.469333in}}%
\pgfpathlineto{\pgfqpoint{3.325091in}{2.372645in}}%
\pgfpathlineto{\pgfqpoint{3.391275in}{2.306981in}}%
\pgfpathlineto{\pgfqpoint{3.429927in}{2.268317in}}%
\pgfpathlineto{\pgfqpoint{3.468374in}{2.229461in}}%
\pgfpathlineto{\pgfqpoint{3.506617in}{2.190416in}}%
\pgfpathlineto{\pgfqpoint{3.562407in}{2.133333in}}%
\pgfpathlineto{\pgfqpoint{3.645737in}{2.045432in}}%
\pgfpathlineto{\pgfqpoint{3.713313in}{1.972277in}}%
\pgfpathlineto{\pgfqpoint{3.737185in}{1.946667in}}%
\pgfpathlineto{\pgfqpoint{3.838084in}{1.834667in}}%
\pgfpathlineto{\pgfqpoint{3.935970in}{1.722667in}}%
\pgfpathlineto{\pgfqpoint{3.984704in}{1.665064in}}%
\pgfpathlineto{\pgfqpoint{4.030897in}{1.610667in}}%
\pgfpathlineto{\pgfqpoint{4.126707in}{1.494133in}}%
\pgfpathlineto{\pgfqpoint{4.226212in}{1.368649in}}%
\pgfpathlineto{\pgfqpoint{4.298177in}{1.274667in}}%
\pgfpathlineto{\pgfqpoint{4.327111in}{1.236299in}}%
\pgfpathlineto{\pgfqpoint{4.408376in}{1.125333in}}%
\pgfpathlineto{\pgfqpoint{4.471241in}{1.035584in}}%
\pgfpathlineto{\pgfqpoint{4.487434in}{1.013076in}}%
\pgfpathlineto{\pgfqpoint{4.548954in}{0.921303in}}%
\pgfpathlineto{\pgfqpoint{4.567596in}{0.893920in}}%
\pgfpathlineto{\pgfqpoint{4.639352in}{0.781503in}}%
\pgfpathlineto{\pgfqpoint{4.657950in}{0.752000in}}%
\pgfpathlineto{\pgfqpoint{4.687838in}{0.702901in}}%
\pgfpathlineto{\pgfqpoint{4.746412in}{0.602667in}}%
\pgfpathlineto{\pgfqpoint{4.768000in}{0.564588in}}%
\pgfpathlineto{\pgfqpoint{4.768000in}{0.565333in}}%
\pgfusepath{fill}%
\end{pgfscope}%
\begin{pgfscope}%
\pgfpathrectangle{\pgfqpoint{0.800000in}{0.528000in}}{\pgfqpoint{3.968000in}{3.696000in}}%
\pgfusepath{clip}%
\pgfsetbuttcap%
\pgfsetroundjoin%
\definecolor{currentfill}{rgb}{0.282910,0.105393,0.426902}%
\pgfsetfillcolor{currentfill}%
\pgfsetlinewidth{0.000000pt}%
\definecolor{currentstroke}{rgb}{0.000000,0.000000,0.000000}%
\pgfsetstrokecolor{currentstroke}%
\pgfsetdash{}{0pt}%
\pgfpathmoveto{\pgfqpoint{2.842615in}{0.528000in}}%
\pgfpathlineto{\pgfqpoint{2.800636in}{0.565333in}}%
\pgfpathlineto{\pgfqpoint{2.677456in}{0.677333in}}%
\pgfpathlineto{\pgfqpoint{2.620772in}{0.730628in}}%
\pgfpathlineto{\pgfqpoint{2.597518in}{0.752000in}}%
\pgfpathlineto{\pgfqpoint{2.480671in}{0.864000in}}%
\pgfpathlineto{\pgfqpoint{2.363152in}{0.980275in}}%
\pgfpathlineto{\pgfqpoint{2.308007in}{1.036636in}}%
\pgfpathlineto{\pgfqpoint{2.257569in}{1.088000in}}%
\pgfpathlineto{\pgfqpoint{2.202828in}{1.144998in}}%
\pgfpathlineto{\pgfqpoint{2.081203in}{1.274667in}}%
\pgfpathlineto{\pgfqpoint{1.979674in}{1.386667in}}%
\pgfpathlineto{\pgfqpoint{1.880799in}{1.498667in}}%
\pgfpathlineto{\pgfqpoint{1.842101in}{1.543584in}}%
\pgfpathlineto{\pgfqpoint{1.721859in}{1.686393in}}%
\pgfpathlineto{\pgfqpoint{1.653437in}{1.770935in}}%
\pgfpathlineto{\pgfqpoint{1.631890in}{1.797333in}}%
\pgfpathlineto{\pgfqpoint{1.561535in}{1.886556in}}%
\pgfpathlineto{\pgfqpoint{1.481374in}{1.991180in}}%
\pgfpathlineto{\pgfqpoint{1.401212in}{2.099447in}}%
\pgfpathlineto{\pgfqpoint{1.338905in}{2.187297in}}%
\pgfpathlineto{\pgfqpoint{1.307270in}{2.232497in}}%
\pgfpathlineto{\pgfqpoint{1.272393in}{2.282667in}}%
\pgfpathlineto{\pgfqpoint{1.222450in}{2.357333in}}%
\pgfpathlineto{\pgfqpoint{1.197884in}{2.394667in}}%
\pgfpathlineto{\pgfqpoint{1.150304in}{2.469333in}}%
\pgfpathlineto{\pgfqpoint{1.104303in}{2.544000in}}%
\pgfpathlineto{\pgfqpoint{1.067490in}{2.606487in}}%
\pgfpathlineto{\pgfqpoint{1.040485in}{2.652266in}}%
\pgfpathlineto{\pgfqpoint{1.000404in}{2.724099in}}%
\pgfpathlineto{\pgfqpoint{0.938357in}{2.842667in}}%
\pgfpathlineto{\pgfqpoint{0.902052in}{2.917333in}}%
\pgfpathlineto{\pgfqpoint{0.895758in}{2.931861in}}%
\pgfpathlineto{\pgfqpoint{0.880162in}{2.964757in}}%
\pgfpathlineto{\pgfqpoint{0.849048in}{3.037686in}}%
\pgfpathlineto{\pgfqpoint{0.836786in}{3.066667in}}%
\pgfpathlineto{\pgfqpoint{0.800000in}{3.166070in}}%
\pgfpathlineto{\pgfqpoint{0.800000in}{3.133297in}}%
\pgfpathlineto{\pgfqpoint{0.825851in}{3.066667in}}%
\pgfpathlineto{\pgfqpoint{0.830037in}{3.057312in}}%
\pgfpathlineto{\pgfqpoint{0.841127in}{3.029333in}}%
\pgfpathlineto{\pgfqpoint{0.909659in}{2.880000in}}%
\pgfpathlineto{\pgfqpoint{0.947386in}{2.805333in}}%
\pgfpathlineto{\pgfqpoint{0.991848in}{2.722697in}}%
\pgfpathlineto{\pgfqpoint{1.007787in}{2.693333in}}%
\pgfpathlineto{\pgfqpoint{1.040485in}{2.635958in}}%
\pgfpathlineto{\pgfqpoint{1.095045in}{2.544000in}}%
\pgfpathlineto{\pgfqpoint{1.120646in}{2.502106in}}%
\pgfpathlineto{\pgfqpoint{1.200808in}{2.376615in}}%
\pgfpathlineto{\pgfqpoint{1.341537in}{2.170667in}}%
\pgfpathlineto{\pgfqpoint{1.401212in}{2.087700in}}%
\pgfpathlineto{\pgfqpoint{1.481374in}{1.979834in}}%
\pgfpathlineto{\pgfqpoint{1.535404in}{1.909333in}}%
\pgfpathlineto{\pgfqpoint{1.623490in}{1.797333in}}%
\pgfpathlineto{\pgfqpoint{1.653467in}{1.760000in}}%
\pgfpathlineto{\pgfqpoint{1.745436in}{1.648000in}}%
\pgfpathlineto{\pgfqpoint{1.776718in}{1.610667in}}%
\pgfpathlineto{\pgfqpoint{1.872473in}{1.498667in}}%
\pgfpathlineto{\pgfqpoint{1.931013in}{1.432151in}}%
\pgfpathlineto{\pgfqpoint{1.971152in}{1.386667in}}%
\pgfpathlineto{\pgfqpoint{2.016359in}{1.336354in}}%
\pgfpathlineto{\pgfqpoint{2.122667in}{1.220876in}}%
\pgfpathlineto{\pgfqpoint{2.177437in}{1.162667in}}%
\pgfpathlineto{\pgfqpoint{2.285004in}{1.050667in}}%
\pgfpathlineto{\pgfqpoint{2.341823in}{0.993467in}}%
\pgfpathlineto{\pgfqpoint{2.395980in}{0.938667in}}%
\pgfpathlineto{\pgfqpoint{2.643717in}{0.700068in}}%
\pgfpathlineto{\pgfqpoint{2.696376in}{0.651715in}}%
\pgfpathlineto{\pgfqpoint{2.749727in}{0.602667in}}%
\pgfpathlineto{\pgfqpoint{2.833052in}{0.528000in}}%
\pgfpathmoveto{\pgfqpoint{4.768000in}{0.598525in}}%
\pgfpathlineto{\pgfqpoint{4.727919in}{0.667125in}}%
\pgfpathlineto{\pgfqpoint{4.629210in}{0.826667in}}%
\pgfpathlineto{\pgfqpoint{4.567596in}{0.920824in}}%
\pgfpathlineto{\pgfqpoint{4.425570in}{1.125333in}}%
\pgfpathlineto{\pgfqpoint{4.385557in}{1.179773in}}%
\pgfpathlineto{\pgfqpoint{4.367192in}{1.205419in}}%
\pgfpathlineto{\pgfqpoint{4.287026in}{1.312000in}}%
\pgfpathlineto{\pgfqpoint{4.246949in}{1.363752in}}%
\pgfpathlineto{\pgfqpoint{4.166788in}{1.465376in}}%
\pgfpathlineto{\pgfqpoint{4.078868in}{1.573333in}}%
\pgfpathlineto{\pgfqpoint{4.046545in}{1.612341in}}%
\pgfpathlineto{\pgfqpoint{3.926303in}{1.753480in}}%
\pgfpathlineto{\pgfqpoint{3.846141in}{1.844712in}}%
\pgfpathlineto{\pgfqpoint{3.777830in}{1.920371in}}%
\pgfpathlineto{\pgfqpoint{3.725899in}{1.977684in}}%
\pgfpathlineto{\pgfqpoint{3.650597in}{2.058667in}}%
\pgfpathlineto{\pgfqpoint{3.565576in}{2.148025in}}%
\pgfpathlineto{\pgfqpoint{3.507405in}{2.208000in}}%
\pgfpathlineto{\pgfqpoint{3.396418in}{2.320000in}}%
\pgfpathlineto{\pgfqpoint{3.342067in}{2.373146in}}%
\pgfpathlineto{\pgfqpoint{3.285010in}{2.429192in}}%
\pgfpathlineto{\pgfqpoint{3.164167in}{2.544000in}}%
\pgfpathlineto{\pgfqpoint{3.103886in}{2.599292in}}%
\pgfpathlineto{\pgfqpoint{3.063446in}{2.636290in}}%
\pgfpathlineto{\pgfqpoint{3.042357in}{2.656000in}}%
\pgfpathlineto{\pgfqpoint{2.916342in}{2.768000in}}%
\pgfpathlineto{\pgfqpoint{2.804040in}{2.864599in}}%
\pgfpathlineto{\pgfqpoint{2.683798in}{2.964687in}}%
\pgfpathlineto{\pgfqpoint{2.563556in}{3.061228in}}%
\pgfpathlineto{\pgfqpoint{2.494639in}{3.114474in}}%
\pgfpathlineto{\pgfqpoint{2.459981in}{3.141333in}}%
\pgfpathlineto{\pgfqpoint{2.242909in}{3.300390in}}%
\pgfpathlineto{\pgfqpoint{2.200935in}{3.329763in}}%
\pgfpathlineto{\pgfqpoint{2.082586in}{3.409438in}}%
\pgfpathlineto{\pgfqpoint{2.015566in}{3.452241in}}%
\pgfpathlineto{\pgfqpoint{1.976341in}{3.477333in}}%
\pgfpathlineto{\pgfqpoint{1.825191in}{3.567751in}}%
\pgfpathlineto{\pgfqpoint{1.721859in}{3.624722in}}%
\pgfpathlineto{\pgfqpoint{1.694483in}{3.638501in}}%
\pgfpathlineto{\pgfqpoint{1.681778in}{3.645523in}}%
\pgfpathlineto{\pgfqpoint{1.601616in}{3.685341in}}%
\pgfpathlineto{\pgfqpoint{1.521455in}{3.722152in}}%
\pgfpathlineto{\pgfqpoint{1.479726in}{3.740201in}}%
\pgfpathlineto{\pgfqpoint{1.397268in}{3.772327in}}%
\pgfpathlineto{\pgfqpoint{1.345196in}{3.790843in}}%
\pgfpathlineto{\pgfqpoint{1.271362in}{3.813333in}}%
\pgfpathlineto{\pgfqpoint{1.240889in}{3.821447in}}%
\pgfpathlineto{\pgfqpoint{1.160727in}{3.838545in}}%
\pgfpathlineto{\pgfqpoint{1.114561in}{3.844998in}}%
\pgfpathlineto{\pgfqpoint{1.080566in}{3.848781in}}%
\pgfpathlineto{\pgfqpoint{1.040364in}{3.850554in}}%
\pgfpathlineto{\pgfqpoint{0.998977in}{3.849337in}}%
\pgfpathlineto{\pgfqpoint{0.960323in}{3.845219in}}%
\pgfpathlineto{\pgfqpoint{0.920242in}{3.836810in}}%
\pgfpathlineto{\pgfqpoint{0.900957in}{3.831297in}}%
\pgfpathlineto{\pgfqpoint{0.872612in}{3.820365in}}%
\pgfpathlineto{\pgfqpoint{0.840081in}{3.802737in}}%
\pgfpathlineto{\pgfqpoint{0.800000in}{3.771418in}}%
\pgfpathlineto{\pgfqpoint{0.800000in}{3.739031in}}%
\pgfpathlineto{\pgfqpoint{0.818698in}{3.758584in}}%
\pgfpathlineto{\pgfqpoint{0.840081in}{3.776548in}}%
\pgfpathlineto{\pgfqpoint{0.863351in}{3.791659in}}%
\pgfpathlineto{\pgfqpoint{0.889289in}{3.804831in}}%
\pgfpathlineto{\pgfqpoint{0.920242in}{3.816410in}}%
\pgfpathlineto{\pgfqpoint{0.960323in}{3.826190in}}%
\pgfpathlineto{\pgfqpoint{0.975908in}{3.827850in}}%
\pgfpathlineto{\pgfqpoint{1.000404in}{3.831731in}}%
\pgfpathlineto{\pgfqpoint{1.080566in}{3.832942in}}%
\pgfpathlineto{\pgfqpoint{1.099371in}{3.830849in}}%
\pgfpathlineto{\pgfqpoint{1.120646in}{3.829656in}}%
\pgfpathlineto{\pgfqpoint{1.160727in}{3.824298in}}%
\pgfpathlineto{\pgfqpoint{1.218078in}{3.813333in}}%
\pgfpathlineto{\pgfqpoint{1.240889in}{3.808254in}}%
\pgfpathlineto{\pgfqpoint{1.352795in}{3.776000in}}%
\pgfpathlineto{\pgfqpoint{1.401212in}{3.759142in}}%
\pgfpathlineto{\pgfqpoint{1.416510in}{3.752915in}}%
\pgfpathlineto{\pgfqpoint{1.455008in}{3.738667in}}%
\pgfpathlineto{\pgfqpoint{1.521455in}{3.711116in}}%
\pgfpathlineto{\pgfqpoint{1.543527in}{3.701333in}}%
\pgfpathlineto{\pgfqpoint{1.601616in}{3.674690in}}%
\pgfpathlineto{\pgfqpoint{1.623779in}{3.664000in}}%
\pgfpathlineto{\pgfqpoint{1.687549in}{3.632042in}}%
\pgfpathlineto{\pgfqpoint{1.721859in}{3.614404in}}%
\pgfpathlineto{\pgfqpoint{1.761939in}{3.593039in}}%
\pgfpathlineto{\pgfqpoint{1.842101in}{3.548232in}}%
\pgfpathlineto{\pgfqpoint{1.899304in}{3.514667in}}%
\pgfpathlineto{\pgfqpoint{1.962343in}{3.476692in}}%
\pgfpathlineto{\pgfqpoint{2.033472in}{3.431586in}}%
\pgfpathlineto{\pgfqpoint{2.057090in}{3.416252in}}%
\pgfpathlineto{\pgfqpoint{2.091359in}{3.394495in}}%
\pgfpathlineto{\pgfqpoint{2.202828in}{3.319301in}}%
\pgfpathlineto{\pgfqpoint{2.244095in}{3.290667in}}%
\pgfpathlineto{\pgfqpoint{2.363152in}{3.205018in}}%
\pgfpathlineto{\pgfqpoint{2.403232in}{3.175435in}}%
\pgfpathlineto{\pgfqpoint{2.497347in}{3.104000in}}%
\pgfpathlineto{\pgfqpoint{2.576777in}{3.041649in}}%
\pgfpathlineto{\pgfqpoint{2.619980in}{3.007223in}}%
\pgfpathlineto{\pgfqpoint{2.671040in}{2.966550in}}%
\pgfpathlineto{\pgfqpoint{2.763960in}{2.889784in}}%
\pgfpathlineto{\pgfqpoint{2.884202in}{2.787357in}}%
\pgfpathlineto{\pgfqpoint{3.004444in}{2.681447in}}%
\pgfpathlineto{\pgfqpoint{3.124687in}{2.571985in}}%
\pgfpathlineto{\pgfqpoint{3.164768in}{2.534697in}}%
\pgfpathlineto{\pgfqpoint{3.285010in}{2.420390in}}%
\pgfpathlineto{\pgfqpoint{3.337523in}{2.368913in}}%
\pgfpathlineto{\pgfqpoint{3.387497in}{2.320000in}}%
\pgfpathlineto{\pgfqpoint{3.445333in}{2.262185in}}%
\pgfpathlineto{\pgfqpoint{3.565576in}{2.139086in}}%
\pgfpathlineto{\pgfqpoint{3.662160in}{2.036631in}}%
\pgfpathlineto{\pgfqpoint{3.711470in}{1.984000in}}%
\pgfpathlineto{\pgfqpoint{3.773301in}{1.916153in}}%
\pgfpathlineto{\pgfqpoint{3.813338in}{1.872000in}}%
\pgfpathlineto{\pgfqpoint{3.864234in}{1.814186in}}%
\pgfpathlineto{\pgfqpoint{3.912129in}{1.760000in}}%
\pgfpathlineto{\pgfqpoint{4.014480in}{1.640534in}}%
\pgfpathlineto{\pgfqpoint{4.126707in}{1.504616in}}%
\pgfpathlineto{\pgfqpoint{4.306735in}{1.274667in}}%
\pgfpathlineto{\pgfqpoint{4.334861in}{1.237333in}}%
\pgfpathlineto{\pgfqpoint{4.407273in}{1.138775in}}%
\pgfpathlineto{\pgfqpoint{4.546753in}{0.938667in}}%
\pgfpathlineto{\pgfqpoint{4.571732in}{0.901333in}}%
\pgfpathlineto{\pgfqpoint{4.620206in}{0.826667in}}%
\pgfpathlineto{\pgfqpoint{4.667150in}{0.752000in}}%
\pgfpathlineto{\pgfqpoint{4.690167in}{0.714667in}}%
\pgfpathlineto{\pgfqpoint{4.734540in}{0.640000in}}%
\pgfpathlineto{\pgfqpoint{4.768000in}{0.581581in}}%
\pgfpathlineto{\pgfqpoint{4.768000in}{0.581581in}}%
\pgfusepath{fill}%
\end{pgfscope}%
\begin{pgfscope}%
\pgfpathrectangle{\pgfqpoint{0.800000in}{0.528000in}}{\pgfqpoint{3.968000in}{3.696000in}}%
\pgfusepath{clip}%
\pgfsetbuttcap%
\pgfsetroundjoin%
\definecolor{currentfill}{rgb}{0.282910,0.105393,0.426902}%
\pgfsetfillcolor{currentfill}%
\pgfsetlinewidth{0.000000pt}%
\definecolor{currentstroke}{rgb}{0.000000,0.000000,0.000000}%
\pgfsetstrokecolor{currentstroke}%
\pgfsetdash{}{0pt}%
\pgfpathmoveto{\pgfqpoint{2.833052in}{0.528000in}}%
\pgfpathlineto{\pgfqpoint{2.791157in}{0.565333in}}%
\pgfpathlineto{\pgfqpoint{2.683798in}{0.662908in}}%
\pgfpathlineto{\pgfqpoint{2.563556in}{0.775580in}}%
\pgfpathlineto{\pgfqpoint{2.443313in}{0.891866in}}%
\pgfpathlineto{\pgfqpoint{2.380388in}{0.954722in}}%
\pgfpathlineto{\pgfqpoint{2.323071in}{1.011838in}}%
\pgfpathlineto{\pgfqpoint{2.212958in}{1.125333in}}%
\pgfpathlineto{\pgfqpoint{2.162747in}{1.178175in}}%
\pgfpathlineto{\pgfqpoint{2.038513in}{1.312000in}}%
\pgfpathlineto{\pgfqpoint{1.937980in}{1.424000in}}%
\pgfpathlineto{\pgfqpoint{1.905089in}{1.461333in}}%
\pgfpathlineto{\pgfqpoint{1.802020in}{1.580668in}}%
\pgfpathlineto{\pgfqpoint{1.714402in}{1.685333in}}%
\pgfpathlineto{\pgfqpoint{1.681778in}{1.724999in}}%
\pgfpathlineto{\pgfqpoint{1.571794in}{1.862444in}}%
\pgfpathlineto{\pgfqpoint{1.506704in}{1.946667in}}%
\pgfpathlineto{\pgfqpoint{1.470709in}{1.993934in}}%
\pgfpathlineto{\pgfqpoint{1.395146in}{2.096000in}}%
\pgfpathlineto{\pgfqpoint{1.361131in}{2.143106in}}%
\pgfpathlineto{\pgfqpoint{1.280970in}{2.257268in}}%
\pgfpathlineto{\pgfqpoint{1.141245in}{2.469333in}}%
\pgfpathlineto{\pgfqpoint{1.080566in}{2.567933in}}%
\pgfpathlineto{\pgfqpoint{1.028953in}{2.656000in}}%
\pgfpathlineto{\pgfqpoint{0.991848in}{2.722697in}}%
\pgfpathlineto{\pgfqpoint{0.973458in}{2.755765in}}%
\pgfpathlineto{\pgfqpoint{0.928210in}{2.842667in}}%
\pgfpathlineto{\pgfqpoint{0.891655in}{2.917333in}}%
\pgfpathlineto{\pgfqpoint{0.880162in}{2.941791in}}%
\pgfpathlineto{\pgfqpoint{0.857474in}{2.992000in}}%
\pgfpathlineto{\pgfqpoint{0.852742in}{3.003794in}}%
\pgfpathlineto{\pgfqpoint{0.840081in}{3.031866in}}%
\pgfpathlineto{\pgfqpoint{0.800000in}{3.133297in}}%
\pgfpathlineto{\pgfqpoint{0.800000in}{3.103394in}}%
\pgfpathlineto{\pgfqpoint{0.851520in}{2.981345in}}%
\pgfpathlineto{\pgfqpoint{0.899606in}{2.880000in}}%
\pgfpathlineto{\pgfqpoint{0.937566in}{2.805333in}}%
\pgfpathlineto{\pgfqpoint{0.977600in}{2.730667in}}%
\pgfpathlineto{\pgfqpoint{1.013156in}{2.667878in}}%
\pgfpathlineto{\pgfqpoint{1.019569in}{2.656000in}}%
\pgfpathlineto{\pgfqpoint{1.063347in}{2.581333in}}%
\pgfpathlineto{\pgfqpoint{1.120646in}{2.487692in}}%
\pgfpathlineto{\pgfqpoint{1.180146in}{2.394667in}}%
\pgfpathlineto{\pgfqpoint{1.240889in}{2.303230in}}%
\pgfpathlineto{\pgfqpoint{1.321051in}{2.187369in}}%
\pgfpathlineto{\pgfqpoint{1.481374in}{1.968787in}}%
\pgfpathlineto{\pgfqpoint{1.540968in}{1.891158in}}%
\pgfpathlineto{\pgfqpoint{1.615090in}{1.797333in}}%
\pgfpathlineto{\pgfqpoint{1.655098in}{1.747518in}}%
\pgfpathlineto{\pgfqpoint{1.761939in}{1.618215in}}%
\pgfpathlineto{\pgfqpoint{1.962630in}{1.386667in}}%
\pgfpathlineto{\pgfqpoint{2.002424in}{1.342489in}}%
\pgfpathlineto{\pgfqpoint{2.122667in}{1.211755in}}%
\pgfpathlineto{\pgfqpoint{2.168817in}{1.162667in}}%
\pgfpathlineto{\pgfqpoint{2.282990in}{1.043920in}}%
\pgfpathlineto{\pgfqpoint{2.337260in}{0.989216in}}%
\pgfpathlineto{\pgfqpoint{2.387262in}{0.938667in}}%
\pgfpathlineto{\pgfqpoint{2.473088in}{0.854401in}}%
\pgfpathlineto{\pgfqpoint{2.523475in}{0.805310in}}%
\pgfpathlineto{\pgfqpoint{2.643717in}{0.691497in}}%
\pgfpathlineto{\pgfqpoint{2.699436in}{0.640000in}}%
\pgfpathlineto{\pgfqpoint{2.804040in}{0.545285in}}%
\pgfpathlineto{\pgfqpoint{2.823490in}{0.528000in}}%
\pgfpathmoveto{\pgfqpoint{4.768000in}{0.614700in}}%
\pgfpathlineto{\pgfqpoint{4.647758in}{0.811759in}}%
\pgfpathlineto{\pgfqpoint{4.589365in}{0.901333in}}%
\pgfpathlineto{\pgfqpoint{4.527515in}{0.992885in}}%
\pgfpathlineto{\pgfqpoint{4.367192in}{1.216787in}}%
\pgfpathlineto{\pgfqpoint{4.287030in}{1.322836in}}%
\pgfpathlineto{\pgfqpoint{4.224283in}{1.402887in}}%
\pgfpathlineto{\pgfqpoint{4.178319in}{1.461333in}}%
\pgfpathlineto{\pgfqpoint{4.138784in}{1.509916in}}%
\pgfpathlineto{\pgfqpoint{4.104101in}{1.552277in}}%
\pgfpathlineto{\pgfqpoint{4.056197in}{1.610667in}}%
\pgfpathlineto{\pgfqpoint{3.961304in}{1.722667in}}%
\pgfpathlineto{\pgfqpoint{3.863396in}{1.834667in}}%
\pgfpathlineto{\pgfqpoint{3.762804in}{1.946667in}}%
\pgfpathlineto{\pgfqpoint{3.708513in}{2.005139in}}%
\pgfpathlineto{\pgfqpoint{3.659033in}{2.058667in}}%
\pgfpathlineto{\pgfqpoint{3.552337in}{2.170667in}}%
\pgfpathlineto{\pgfqpoint{3.442620in}{2.282667in}}%
\pgfpathlineto{\pgfqpoint{3.325091in}{2.398964in}}%
\pgfpathlineto{\pgfqpoint{3.268082in}{2.453566in}}%
\pgfpathlineto{\pgfqpoint{3.212881in}{2.506667in}}%
\pgfpathlineto{\pgfqpoint{2.964364in}{2.734320in}}%
\pgfpathlineto{\pgfqpoint{2.924283in}{2.769651in}}%
\pgfpathlineto{\pgfqpoint{2.804040in}{2.873174in}}%
\pgfpathlineto{\pgfqpoint{2.683798in}{2.973207in}}%
\pgfpathlineto{\pgfqpoint{2.563556in}{3.069867in}}%
\pgfpathlineto{\pgfqpoint{2.471392in}{3.141333in}}%
\pgfpathlineto{\pgfqpoint{2.242909in}{3.309274in}}%
\pgfpathlineto{\pgfqpoint{2.202828in}{3.337320in}}%
\pgfpathlineto{\pgfqpoint{2.106542in}{3.402667in}}%
\pgfpathlineto{\pgfqpoint{2.042505in}{3.444768in}}%
\pgfpathlineto{\pgfqpoint{1.973757in}{3.487964in}}%
\pgfpathlineto{\pgfqpoint{1.931166in}{3.514667in}}%
\pgfpathlineto{\pgfqpoint{1.798184in}{3.592906in}}%
\pgfpathlineto{\pgfqpoint{1.721859in}{3.634725in}}%
\pgfpathlineto{\pgfqpoint{1.649977in}{3.671713in}}%
\pgfpathlineto{\pgfqpoint{1.641697in}{3.676199in}}%
\pgfpathlineto{\pgfqpoint{1.601616in}{3.695991in}}%
\pgfpathlineto{\pgfqpoint{1.508802in}{3.738667in}}%
\pgfpathlineto{\pgfqpoint{1.389078in}{3.787302in}}%
\pgfpathlineto{\pgfqpoint{1.313354in}{3.813333in}}%
\pgfpathlineto{\pgfqpoint{1.240889in}{3.834392in}}%
\pgfpathlineto{\pgfqpoint{1.226653in}{3.837407in}}%
\pgfpathlineto{\pgfqpoint{1.200808in}{3.844295in}}%
\pgfpathlineto{\pgfqpoint{1.160727in}{3.852683in}}%
\pgfpathlineto{\pgfqpoint{1.080566in}{3.863831in}}%
\pgfpathlineto{\pgfqpoint{1.000404in}{3.866339in}}%
\pgfpathlineto{\pgfqpoint{0.984248in}{3.865715in}}%
\pgfpathlineto{\pgfqpoint{0.960323in}{3.863338in}}%
\pgfpathlineto{\pgfqpoint{0.914889in}{3.855654in}}%
\pgfpathlineto{\pgfqpoint{0.880162in}{3.845371in}}%
\pgfpathlineto{\pgfqpoint{0.856109in}{3.835737in}}%
\pgfpathlineto{\pgfqpoint{0.840081in}{3.827522in}}%
\pgfpathlineto{\pgfqpoint{0.816928in}{3.813333in}}%
\pgfpathlineto{\pgfqpoint{0.800000in}{3.800773in}}%
\pgfpathlineto{\pgfqpoint{0.800000in}{3.771418in}}%
\pgfpathlineto{\pgfqpoint{0.804879in}{3.776000in}}%
\pgfpathlineto{\pgfqpoint{0.846784in}{3.807090in}}%
\pgfpathlineto{\pgfqpoint{0.880162in}{3.823389in}}%
\pgfpathlineto{\pgfqpoint{0.900957in}{3.831297in}}%
\pgfpathlineto{\pgfqpoint{0.931804in}{3.839898in}}%
\pgfpathlineto{\pgfqpoint{0.960323in}{3.845219in}}%
\pgfpathlineto{\pgfqpoint{1.000404in}{3.849563in}}%
\pgfpathlineto{\pgfqpoint{1.040595in}{3.850564in}}%
\pgfpathlineto{\pgfqpoint{1.080566in}{3.848781in}}%
\pgfpathlineto{\pgfqpoint{1.128046in}{3.843775in}}%
\pgfpathlineto{\pgfqpoint{1.176713in}{3.835777in}}%
\pgfpathlineto{\pgfqpoint{1.240889in}{3.821447in}}%
\pgfpathlineto{\pgfqpoint{1.284904in}{3.809669in}}%
\pgfpathlineto{\pgfqpoint{1.361131in}{3.785348in}}%
\pgfpathlineto{\pgfqpoint{1.401212in}{3.771074in}}%
\pgfpathlineto{\pgfqpoint{1.536210in}{3.715077in}}%
\pgfpathlineto{\pgfqpoint{1.567773in}{3.701333in}}%
\pgfpathlineto{\pgfqpoint{1.645505in}{3.664000in}}%
\pgfpathlineto{\pgfqpoint{1.721859in}{3.624722in}}%
\pgfpathlineto{\pgfqpoint{1.922263in}{3.510655in}}%
\pgfpathlineto{\pgfqpoint{1.991724in}{3.467366in}}%
\pgfpathlineto{\pgfqpoint{2.015566in}{3.452241in}}%
\pgfpathlineto{\pgfqpoint{2.042505in}{3.435526in}}%
\pgfpathlineto{\pgfqpoint{2.092800in}{3.402667in}}%
\pgfpathlineto{\pgfqpoint{2.203481in}{3.328000in}}%
\pgfpathlineto{\pgfqpoint{2.256635in}{3.290667in}}%
\pgfpathlineto{\pgfqpoint{2.363152in}{3.213961in}}%
\pgfpathlineto{\pgfqpoint{2.410607in}{3.178667in}}%
\pgfpathlineto{\pgfqpoint{2.523475in}{3.092536in}}%
\pgfpathlineto{\pgfqpoint{2.603886in}{3.029333in}}%
\pgfpathlineto{\pgfqpoint{2.650239in}{2.992000in}}%
\pgfpathlineto{\pgfqpoint{2.763960in}{2.898341in}}%
\pgfpathlineto{\pgfqpoint{2.884202in}{2.795969in}}%
\pgfpathlineto{\pgfqpoint{3.004444in}{2.690115in}}%
\pgfpathlineto{\pgfqpoint{3.124687in}{2.580710in}}%
\pgfpathlineto{\pgfqpoint{3.164768in}{2.543441in}}%
\pgfpathlineto{\pgfqpoint{3.285010in}{2.429192in}}%
\pgfpathlineto{\pgfqpoint{3.342067in}{2.373146in}}%
\pgfpathlineto{\pgfqpoint{3.396418in}{2.320000in}}%
\pgfpathlineto{\pgfqpoint{3.477662in}{2.238113in}}%
\pgfpathlineto{\pgfqpoint{3.525495in}{2.189467in}}%
\pgfpathlineto{\pgfqpoint{3.645737in}{2.063840in}}%
\pgfpathlineto{\pgfqpoint{3.725899in}{1.977684in}}%
\pgfpathlineto{\pgfqpoint{3.788151in}{1.909333in}}%
\pgfpathlineto{\pgfqpoint{3.888081in}{1.797333in}}%
\pgfpathlineto{\pgfqpoint{3.926303in}{1.753480in}}%
\pgfpathlineto{\pgfqpoint{4.047945in}{1.610667in}}%
\pgfpathlineto{\pgfqpoint{4.099400in}{1.547898in}}%
\pgfpathlineto{\pgfqpoint{4.139880in}{1.498667in}}%
\pgfpathlineto{\pgfqpoint{4.206869in}{1.414932in}}%
\pgfpathlineto{\pgfqpoint{4.269966in}{1.333439in}}%
\pgfpathlineto{\pgfqpoint{4.287030in}{1.311995in}}%
\pgfpathlineto{\pgfqpoint{4.371209in}{1.200000in}}%
\pgfpathlineto{\pgfqpoint{4.418033in}{1.135356in}}%
\pgfpathlineto{\pgfqpoint{4.452390in}{1.088000in}}%
\pgfpathlineto{\pgfqpoint{4.487434in}{1.038195in}}%
\pgfpathlineto{\pgfqpoint{4.567596in}{0.920824in}}%
\pgfpathlineto{\pgfqpoint{4.629210in}{0.826667in}}%
\pgfpathlineto{\pgfqpoint{4.687838in}{0.733382in}}%
\pgfpathlineto{\pgfqpoint{4.743864in}{0.640000in}}%
\pgfpathlineto{\pgfqpoint{4.768000in}{0.598525in}}%
\pgfpathlineto{\pgfqpoint{4.768000in}{0.602667in}}%
\pgfpathlineto{\pgfqpoint{4.768000in}{0.602667in}}%
\pgfusepath{fill}%
\end{pgfscope}%
\begin{pgfscope}%
\pgfpathrectangle{\pgfqpoint{0.800000in}{0.528000in}}{\pgfqpoint{3.968000in}{3.696000in}}%
\pgfusepath{clip}%
\pgfsetbuttcap%
\pgfsetroundjoin%
\definecolor{currentfill}{rgb}{0.282910,0.105393,0.426902}%
\pgfsetfillcolor{currentfill}%
\pgfsetlinewidth{0.000000pt}%
\definecolor{currentstroke}{rgb}{0.000000,0.000000,0.000000}%
\pgfsetstrokecolor{currentstroke}%
\pgfsetdash{}{0pt}%
\pgfpathmoveto{\pgfqpoint{2.823490in}{0.528000in}}%
\pgfpathlineto{\pgfqpoint{2.752482in}{0.591976in}}%
\pgfpathlineto{\pgfqpoint{2.699436in}{0.640000in}}%
\pgfpathlineto{\pgfqpoint{2.611441in}{0.721936in}}%
\pgfpathlineto{\pgfqpoint{2.563556in}{0.766972in}}%
\pgfpathlineto{\pgfqpoint{2.443313in}{0.883202in}}%
\pgfpathlineto{\pgfqpoint{2.375811in}{0.950458in}}%
\pgfpathlineto{\pgfqpoint{2.323071in}{1.003117in}}%
\pgfpathlineto{\pgfqpoint{2.202828in}{1.126840in}}%
\pgfpathlineto{\pgfqpoint{2.082586in}{1.254882in}}%
\pgfpathlineto{\pgfqpoint{1.962343in}{1.386987in}}%
\pgfpathlineto{\pgfqpoint{1.842101in}{1.524098in}}%
\pgfpathlineto{\pgfqpoint{1.641697in}{1.764096in}}%
\pgfpathlineto{\pgfqpoint{1.585409in}{1.834667in}}%
\pgfpathlineto{\pgfqpoint{1.521455in}{1.916389in}}%
\pgfpathlineto{\pgfqpoint{1.469845in}{1.984000in}}%
\pgfpathlineto{\pgfqpoint{1.401212in}{2.076105in}}%
\pgfpathlineto{\pgfqpoint{1.240889in}{2.303230in}}%
\pgfpathlineto{\pgfqpoint{1.180146in}{2.394667in}}%
\pgfpathlineto{\pgfqpoint{1.120646in}{2.487692in}}%
\pgfpathlineto{\pgfqpoint{1.019569in}{2.656000in}}%
\pgfpathlineto{\pgfqpoint{0.985631in}{2.716907in}}%
\pgfpathlineto{\pgfqpoint{0.960323in}{2.762230in}}%
\pgfpathlineto{\pgfqpoint{0.899606in}{2.880000in}}%
\pgfpathlineto{\pgfqpoint{0.863870in}{2.954667in}}%
\pgfpathlineto{\pgfqpoint{0.830525in}{3.029333in}}%
\pgfpathlineto{\pgfqpoint{0.811112in}{3.077017in}}%
\pgfpathlineto{\pgfqpoint{0.800000in}{3.103394in}}%
\pgfpathlineto{\pgfqpoint{0.800000in}{3.076471in}}%
\pgfpathlineto{\pgfqpoint{0.840081in}{2.983810in}}%
\pgfpathlineto{\pgfqpoint{0.880162in}{2.899071in}}%
\pgfpathlineto{\pgfqpoint{0.920242in}{2.819750in}}%
\pgfpathlineto{\pgfqpoint{0.988892in}{2.693333in}}%
\pgfpathlineto{\pgfqpoint{1.040485in}{2.604230in}}%
\pgfpathlineto{\pgfqpoint{1.160727in}{2.410869in}}%
\pgfpathlineto{\pgfqpoint{1.220885in}{2.320000in}}%
\pgfpathlineto{\pgfqpoint{1.280970in}{2.232240in}}%
\pgfpathlineto{\pgfqpoint{1.441293in}{2.010702in}}%
\pgfpathlineto{\pgfqpoint{1.521455in}{1.905531in}}%
\pgfpathlineto{\pgfqpoint{1.606691in}{1.797333in}}%
\pgfpathlineto{\pgfqpoint{1.641697in}{1.753847in}}%
\pgfpathlineto{\pgfqpoint{1.728710in}{1.648000in}}%
\pgfpathlineto{\pgfqpoint{1.761939in}{1.608288in}}%
\pgfpathlineto{\pgfqpoint{1.882182in}{1.468329in}}%
\pgfpathlineto{\pgfqpoint{1.962343in}{1.377771in}}%
\pgfpathlineto{\pgfqpoint{2.031475in}{1.301726in}}%
\pgfpathlineto{\pgfqpoint{2.082586in}{1.245740in}}%
\pgfpathlineto{\pgfqpoint{2.142469in}{1.181555in}}%
\pgfpathlineto{\pgfqpoint{2.242909in}{1.076385in}}%
\pgfpathlineto{\pgfqpoint{2.304439in}{1.013333in}}%
\pgfpathlineto{\pgfqpoint{2.416132in}{0.901333in}}%
\pgfpathlineto{\pgfqpoint{2.468622in}{0.850241in}}%
\pgfpathlineto{\pgfqpoint{2.523475in}{0.796684in}}%
\pgfpathlineto{\pgfqpoint{2.643717in}{0.682926in}}%
\pgfpathlineto{\pgfqpoint{2.690119in}{0.640000in}}%
\pgfpathlineto{\pgfqpoint{2.813927in}{0.528000in}}%
\pgfpathmoveto{\pgfqpoint{4.768000in}{0.630625in}}%
\pgfpathlineto{\pgfqpoint{4.670926in}{0.789333in}}%
\pgfpathlineto{\pgfqpoint{4.646414in}{0.827918in}}%
\pgfpathlineto{\pgfqpoint{4.567596in}{0.946941in}}%
\pgfpathlineto{\pgfqpoint{4.407273in}{1.174004in}}%
\pgfpathlineto{\pgfqpoint{4.360336in}{1.237333in}}%
\pgfpathlineto{\pgfqpoint{4.287030in}{1.333676in}}%
\pgfpathlineto{\pgfqpoint{4.086626in}{1.584074in}}%
\pgfpathlineto{\pgfqpoint{4.021169in}{1.661696in}}%
\pgfpathlineto{\pgfqpoint{3.969633in}{1.722667in}}%
\pgfpathlineto{\pgfqpoint{3.896203in}{1.806630in}}%
\pgfpathlineto{\pgfqpoint{3.846141in}{1.863520in}}%
\pgfpathlineto{\pgfqpoint{3.750073in}{1.969184in}}%
\pgfpathlineto{\pgfqpoint{3.702333in}{2.021333in}}%
\pgfpathlineto{\pgfqpoint{3.596796in}{2.133333in}}%
\pgfpathlineto{\pgfqpoint{3.560974in}{2.170667in}}%
\pgfpathlineto{\pgfqpoint{3.445333in}{2.288624in}}%
\pgfpathlineto{\pgfqpoint{3.325091in}{2.407500in}}%
\pgfpathlineto{\pgfqpoint{3.272655in}{2.457825in}}%
\pgfpathlineto{\pgfqpoint{3.221833in}{2.506667in}}%
\pgfpathlineto{\pgfqpoint{3.133318in}{2.589373in}}%
\pgfpathlineto{\pgfqpoint{3.084606in}{2.634469in}}%
\pgfpathlineto{\pgfqpoint{3.044525in}{2.670928in}}%
\pgfpathlineto{\pgfqpoint{2.924283in}{2.778009in}}%
\pgfpathlineto{\pgfqpoint{2.804040in}{2.881694in}}%
\pgfpathlineto{\pgfqpoint{2.716729in}{2.954667in}}%
\pgfpathlineto{\pgfqpoint{2.634915in}{3.021134in}}%
\pgfpathlineto{\pgfqpoint{2.578265in}{3.066667in}}%
\pgfpathlineto{\pgfqpoint{2.363152in}{3.231330in}}%
\pgfpathlineto{\pgfqpoint{2.281715in}{3.290667in}}%
\pgfpathlineto{\pgfqpoint{2.162747in}{3.373843in}}%
\pgfpathlineto{\pgfqpoint{2.120283in}{3.402667in}}%
\pgfpathlineto{\pgfqpoint{2.002424in}{3.479687in}}%
\pgfpathlineto{\pgfqpoint{1.931659in}{3.523419in}}%
\pgfpathlineto{\pgfqpoint{1.922263in}{3.529487in}}%
\pgfpathlineto{\pgfqpoint{1.842101in}{3.577291in}}%
\pgfpathlineto{\pgfqpoint{1.755020in}{3.626667in}}%
\pgfpathlineto{\pgfqpoint{1.681778in}{3.666026in}}%
\pgfpathlineto{\pgfqpoint{1.521455in}{3.744001in}}%
\pgfpathlineto{\pgfqpoint{1.368413in}{3.806550in}}%
\pgfpathlineto{\pgfqpoint{1.321051in}{3.822933in}}%
\pgfpathlineto{\pgfqpoint{1.200808in}{3.857509in}}%
\pgfpathlineto{\pgfqpoint{1.120646in}{3.873371in}}%
\pgfpathlineto{\pgfqpoint{1.080566in}{3.878776in}}%
\pgfpathlineto{\pgfqpoint{1.071091in}{3.879175in}}%
\pgfpathlineto{\pgfqpoint{1.040485in}{3.882122in}}%
\pgfpathlineto{\pgfqpoint{1.000404in}{3.883046in}}%
\pgfpathlineto{\pgfqpoint{0.960323in}{3.881091in}}%
\pgfpathlineto{\pgfqpoint{0.951201in}{3.879503in}}%
\pgfpathlineto{\pgfqpoint{0.920242in}{3.875681in}}%
\pgfpathlineto{\pgfqpoint{0.898196in}{3.871202in}}%
\pgfpathlineto{\pgfqpoint{0.880162in}{3.866073in}}%
\pgfpathlineto{\pgfqpoint{0.838798in}{3.850667in}}%
\pgfpathlineto{\pgfqpoint{0.800000in}{3.828026in}}%
\pgfpathlineto{\pgfqpoint{0.800000in}{3.800773in}}%
\pgfpathlineto{\pgfqpoint{0.816928in}{3.813333in}}%
\pgfpathlineto{\pgfqpoint{0.840081in}{3.827522in}}%
\pgfpathlineto{\pgfqpoint{0.856109in}{3.835737in}}%
\pgfpathlineto{\pgfqpoint{0.884199in}{3.846906in}}%
\pgfpathlineto{\pgfqpoint{0.920242in}{3.856741in}}%
\pgfpathlineto{\pgfqpoint{0.927761in}{3.857669in}}%
\pgfpathlineto{\pgfqpoint{0.960323in}{3.863338in}}%
\pgfpathlineto{\pgfqpoint{1.000404in}{3.866339in}}%
\pgfpathlineto{\pgfqpoint{1.040485in}{3.866346in}}%
\pgfpathlineto{\pgfqpoint{1.055741in}{3.864877in}}%
\pgfpathlineto{\pgfqpoint{1.080566in}{3.863831in}}%
\pgfpathlineto{\pgfqpoint{1.120646in}{3.859175in}}%
\pgfpathlineto{\pgfqpoint{1.128126in}{3.857633in}}%
\pgfpathlineto{\pgfqpoint{1.170386in}{3.850667in}}%
\pgfpathlineto{\pgfqpoint{1.200808in}{3.844295in}}%
\pgfpathlineto{\pgfqpoint{1.313354in}{3.813333in}}%
\pgfpathlineto{\pgfqpoint{1.324917in}{3.809732in}}%
\pgfpathlineto{\pgfqpoint{1.401212in}{3.782704in}}%
\pgfpathlineto{\pgfqpoint{1.481374in}{3.750512in}}%
\pgfpathlineto{\pgfqpoint{1.508802in}{3.738667in}}%
\pgfpathlineto{\pgfqpoint{1.561535in}{3.714895in}}%
\pgfpathlineto{\pgfqpoint{1.590311in}{3.701333in}}%
\pgfpathlineto{\pgfqpoint{1.641697in}{3.676199in}}%
\pgfpathlineto{\pgfqpoint{1.665708in}{3.664000in}}%
\pgfpathlineto{\pgfqpoint{1.736714in}{3.626667in}}%
\pgfpathlineto{\pgfqpoint{1.905423in}{3.530352in}}%
\pgfpathlineto{\pgfqpoint{2.002424in}{3.470382in}}%
\pgfpathlineto{\pgfqpoint{2.068798in}{3.427157in}}%
\pgfpathlineto{\pgfqpoint{2.106542in}{3.402667in}}%
\pgfpathlineto{\pgfqpoint{2.164142in}{3.364035in}}%
\pgfpathlineto{\pgfqpoint{2.282990in}{3.280858in}}%
\pgfpathlineto{\pgfqpoint{2.329287in}{3.247544in}}%
\pgfpathlineto{\pgfqpoint{2.443313in}{3.162713in}}%
\pgfpathlineto{\pgfqpoint{2.499669in}{3.119159in}}%
\pgfpathlineto{\pgfqpoint{2.541651in}{3.087069in}}%
\pgfpathlineto{\pgfqpoint{2.643717in}{3.005798in}}%
\pgfpathlineto{\pgfqpoint{2.763960in}{2.906897in}}%
\pgfpathlineto{\pgfqpoint{2.884202in}{2.804581in}}%
\pgfpathlineto{\pgfqpoint{3.004444in}{2.698610in}}%
\pgfpathlineto{\pgfqpoint{3.124687in}{2.589176in}}%
\pgfpathlineto{\pgfqpoint{3.173217in}{2.544000in}}%
\pgfpathlineto{\pgfqpoint{3.291007in}{2.432000in}}%
\pgfpathlineto{\pgfqpoint{3.346611in}{2.377378in}}%
\pgfpathlineto{\pgfqpoint{3.405335in}{2.320000in}}%
\pgfpathlineto{\pgfqpoint{3.525495in}{2.198386in}}%
\pgfpathlineto{\pgfqpoint{3.645737in}{2.072819in}}%
\pgfpathlineto{\pgfqpoint{3.728588in}{1.984000in}}%
\pgfpathlineto{\pgfqpoint{3.782358in}{1.924589in}}%
\pgfpathlineto{\pgfqpoint{3.830153in}{1.872000in}}%
\pgfpathlineto{\pgfqpoint{3.873519in}{1.822835in}}%
\pgfpathlineto{\pgfqpoint{3.909566in}{1.781743in}}%
\pgfpathlineto{\pgfqpoint{3.961304in}{1.722667in}}%
\pgfpathlineto{\pgfqpoint{4.016585in}{1.657426in}}%
\pgfpathlineto{\pgfqpoint{4.056197in}{1.610667in}}%
\pgfpathlineto{\pgfqpoint{4.104101in}{1.552277in}}%
\pgfpathlineto{\pgfqpoint{4.148232in}{1.498667in}}%
\pgfpathlineto{\pgfqpoint{4.237485in}{1.386667in}}%
\pgfpathlineto{\pgfqpoint{4.327111in}{1.270350in}}%
\pgfpathlineto{\pgfqpoint{4.407273in}{1.162602in}}%
\pgfpathlineto{\pgfqpoint{4.471216in}{1.072894in}}%
\pgfpathlineto{\pgfqpoint{4.487593in}{1.050519in}}%
\pgfpathlineto{\pgfqpoint{4.567596in}{0.934090in}}%
\pgfpathlineto{\pgfqpoint{4.647758in}{0.811759in}}%
\pgfpathlineto{\pgfqpoint{4.758221in}{0.630891in}}%
\pgfpathlineto{\pgfqpoint{4.768000in}{0.614700in}}%
\pgfpathlineto{\pgfqpoint{4.768000in}{0.614700in}}%
\pgfusepath{fill}%
\end{pgfscope}%
\begin{pgfscope}%
\pgfpathrectangle{\pgfqpoint{0.800000in}{0.528000in}}{\pgfqpoint{3.968000in}{3.696000in}}%
\pgfusepath{clip}%
\pgfsetbuttcap%
\pgfsetroundjoin%
\definecolor{currentfill}{rgb}{0.282910,0.105393,0.426902}%
\pgfsetfillcolor{currentfill}%
\pgfsetlinewidth{0.000000pt}%
\definecolor{currentstroke}{rgb}{0.000000,0.000000,0.000000}%
\pgfsetstrokecolor{currentstroke}%
\pgfsetdash{}{0pt}%
\pgfpathmoveto{\pgfqpoint{2.813927in}{0.528000in}}%
\pgfpathlineto{\pgfqpoint{2.563556in}{0.758364in}}%
\pgfpathlineto{\pgfqpoint{2.443313in}{0.874538in}}%
\pgfpathlineto{\pgfqpoint{2.371233in}{0.946194in}}%
\pgfpathlineto{\pgfqpoint{2.323071in}{0.994396in}}%
\pgfpathlineto{\pgfqpoint{2.202828in}{1.118012in}}%
\pgfpathlineto{\pgfqpoint{2.124005in}{1.201247in}}%
\pgfpathlineto{\pgfqpoint{2.082586in}{1.245740in}}%
\pgfpathlineto{\pgfqpoint{1.962343in}{1.377771in}}%
\pgfpathlineto{\pgfqpoint{1.882182in}{1.468329in}}%
\pgfpathlineto{\pgfqpoint{1.823599in}{1.536000in}}%
\pgfpathlineto{\pgfqpoint{1.721859in}{1.656227in}}%
\pgfpathlineto{\pgfqpoint{1.636714in}{1.760000in}}%
\pgfpathlineto{\pgfqpoint{1.587714in}{1.821717in}}%
\pgfpathlineto{\pgfqpoint{1.547679in}{1.872000in}}%
\pgfpathlineto{\pgfqpoint{1.461473in}{1.984000in}}%
\pgfpathlineto{\pgfqpoint{1.401212in}{2.064510in}}%
\pgfpathlineto{\pgfqpoint{1.351070in}{2.133333in}}%
\pgfpathlineto{\pgfqpoint{1.280970in}{2.232240in}}%
\pgfpathlineto{\pgfqpoint{1.220885in}{2.320000in}}%
\pgfpathlineto{\pgfqpoint{1.160727in}{2.410869in}}%
\pgfpathlineto{\pgfqpoint{1.099834in}{2.506667in}}%
\pgfpathlineto{\pgfqpoint{1.076680in}{2.544000in}}%
\pgfpathlineto{\pgfqpoint{1.031930in}{2.618667in}}%
\pgfpathlineto{\pgfqpoint{0.988892in}{2.693333in}}%
\pgfpathlineto{\pgfqpoint{0.952095in}{2.760336in}}%
\pgfpathlineto{\pgfqpoint{0.934875in}{2.791704in}}%
\pgfpathlineto{\pgfqpoint{0.889553in}{2.880000in}}%
\pgfpathlineto{\pgfqpoint{0.853572in}{2.954667in}}%
\pgfpathlineto{\pgfqpoint{0.819970in}{3.029333in}}%
\pgfpathlineto{\pgfqpoint{0.814602in}{3.042934in}}%
\pgfpathlineto{\pgfqpoint{0.800000in}{3.076471in}}%
\pgfpathlineto{\pgfqpoint{0.800000in}{3.051129in}}%
\pgfpathlineto{\pgfqpoint{0.845718in}{2.949416in}}%
\pgfpathlineto{\pgfqpoint{0.898689in}{2.842667in}}%
\pgfpathlineto{\pgfqpoint{0.938167in}{2.768000in}}%
\pgfpathlineto{\pgfqpoint{0.973197in}{2.705325in}}%
\pgfpathlineto{\pgfqpoint{0.979589in}{2.693333in}}%
\pgfpathlineto{\pgfqpoint{1.022828in}{2.618667in}}%
\pgfpathlineto{\pgfqpoint{1.044988in}{2.581333in}}%
\pgfpathlineto{\pgfqpoint{1.090850in}{2.506667in}}%
\pgfpathlineto{\pgfqpoint{1.131876in}{2.442460in}}%
\pgfpathlineto{\pgfqpoint{1.162407in}{2.394667in}}%
\pgfpathlineto{\pgfqpoint{1.212197in}{2.320000in}}%
\pgfpathlineto{\pgfqpoint{1.280970in}{2.220002in}}%
\pgfpathlineto{\pgfqpoint{1.441293in}{1.999624in}}%
\pgfpathlineto{\pgfqpoint{1.521455in}{1.894955in}}%
\pgfpathlineto{\pgfqpoint{1.601616in}{1.793328in}}%
\pgfpathlineto{\pgfqpoint{1.658905in}{1.722667in}}%
\pgfpathlineto{\pgfqpoint{1.751794in}{1.610667in}}%
\pgfpathlineto{\pgfqpoint{1.802020in}{1.551517in}}%
\pgfpathlineto{\pgfqpoint{1.882182in}{1.458837in}}%
\pgfpathlineto{\pgfqpoint{2.002424in}{1.324122in}}%
\pgfpathlineto{\pgfqpoint{2.116796in}{1.200000in}}%
\pgfpathlineto{\pgfqpoint{2.194809in}{1.117864in}}%
\pgfpathlineto{\pgfqpoint{2.242909in}{1.067626in}}%
\pgfpathlineto{\pgfqpoint{2.295858in}{1.013333in}}%
\pgfpathlineto{\pgfqpoint{2.407344in}{0.901333in}}%
\pgfpathlineto{\pgfqpoint{2.464155in}{0.846080in}}%
\pgfpathlineto{\pgfqpoint{2.522832in}{0.788734in}}%
\pgfpathlineto{\pgfqpoint{2.563556in}{0.749827in}}%
\pgfpathlineto{\pgfqpoint{2.683798in}{0.637336in}}%
\pgfpathlineto{\pgfqpoint{2.804365in}{0.528000in}}%
\pgfpathmoveto{\pgfqpoint{4.768000in}{0.646179in}}%
\pgfpathlineto{\pgfqpoint{4.703349in}{0.752000in}}%
\pgfpathlineto{\pgfqpoint{4.667562in}{0.807780in}}%
\pgfpathlineto{\pgfqpoint{4.647758in}{0.839221in}}%
\pgfpathlineto{\pgfqpoint{4.581775in}{0.938667in}}%
\pgfpathlineto{\pgfqpoint{4.527515in}{1.017855in}}%
\pgfpathlineto{\pgfqpoint{4.447354in}{1.130662in}}%
\pgfpathlineto{\pgfqpoint{4.396486in}{1.200000in}}%
\pgfpathlineto{\pgfqpoint{4.327111in}{1.292267in}}%
\pgfpathlineto{\pgfqpoint{4.246949in}{1.395811in}}%
\pgfpathlineto{\pgfqpoint{4.182628in}{1.476088in}}%
\pgfpathlineto{\pgfqpoint{4.134469in}{1.536000in}}%
\pgfpathlineto{\pgfqpoint{4.078668in}{1.603254in}}%
\pgfpathlineto{\pgfqpoint{4.041452in}{1.648000in}}%
\pgfpathlineto{\pgfqpoint{4.006465in}{1.689284in}}%
\pgfpathlineto{\pgfqpoint{3.886222in}{1.827728in}}%
\pgfpathlineto{\pgfqpoint{3.640793in}{2.096000in}}%
\pgfpathlineto{\pgfqpoint{3.586288in}{2.152626in}}%
\pgfpathlineto{\pgfqpoint{3.533239in}{2.208000in}}%
\pgfpathlineto{\pgfqpoint{3.452730in}{2.289557in}}%
\pgfpathlineto{\pgfqpoint{3.405253in}{2.337229in}}%
\pgfpathlineto{\pgfqpoint{3.336126in}{2.404945in}}%
\pgfpathlineto{\pgfqpoint{3.285010in}{2.454837in}}%
\pgfpathlineto{\pgfqpoint{3.217736in}{2.518671in}}%
\pgfpathlineto{\pgfqpoint{3.164768in}{2.568851in}}%
\pgfpathlineto{\pgfqpoint{3.097513in}{2.630689in}}%
\pgfpathlineto{\pgfqpoint{3.044525in}{2.679338in}}%
\pgfpathlineto{\pgfqpoint{2.924283in}{2.786367in}}%
\pgfpathlineto{\pgfqpoint{2.804040in}{2.890000in}}%
\pgfpathlineto{\pgfqpoint{2.681644in}{2.992000in}}%
\pgfpathlineto{\pgfqpoint{2.563556in}{3.086799in}}%
\pgfpathlineto{\pgfqpoint{2.435404in}{3.186034in}}%
\pgfpathlineto{\pgfqpoint{2.323071in}{3.269375in}}%
\pgfpathlineto{\pgfqpoint{2.282990in}{3.298405in}}%
\pgfpathlineto{\pgfqpoint{2.187934in}{3.365333in}}%
\pgfpathlineto{\pgfqpoint{2.002424in}{3.488755in}}%
\pgfpathlineto{\pgfqpoint{1.937585in}{3.528939in}}%
\pgfpathlineto{\pgfqpoint{1.900545in}{3.552000in}}%
\pgfpathlineto{\pgfqpoint{1.746060in}{3.641458in}}%
\pgfpathlineto{\pgfqpoint{1.681778in}{3.675932in}}%
\pgfpathlineto{\pgfqpoint{1.632462in}{3.701333in}}%
\pgfpathlineto{\pgfqpoint{1.556059in}{3.738667in}}%
\pgfpathlineto{\pgfqpoint{1.473128in}{3.776000in}}%
\pgfpathlineto{\pgfqpoint{1.361131in}{3.820706in}}%
\pgfpathlineto{\pgfqpoint{1.240889in}{3.859833in}}%
\pgfpathlineto{\pgfqpoint{1.200808in}{3.870412in}}%
\pgfpathlineto{\pgfqpoint{1.185253in}{3.873511in}}%
\pgfpathlineto{\pgfqpoint{1.153084in}{3.880881in}}%
\pgfpathlineto{\pgfqpoint{1.117685in}{3.888000in}}%
\pgfpathlineto{\pgfqpoint{1.080566in}{3.893415in}}%
\pgfpathlineto{\pgfqpoint{1.029550in}{3.898186in}}%
\pgfpathlineto{\pgfqpoint{1.000404in}{3.899054in}}%
\pgfpathlineto{\pgfqpoint{0.950009in}{3.897608in}}%
\pgfpathlineto{\pgfqpoint{0.914430in}{3.893414in}}%
\pgfpathlineto{\pgfqpoint{0.877257in}{3.885295in}}%
\pgfpathlineto{\pgfqpoint{0.840081in}{3.873157in}}%
\pgfpathlineto{\pgfqpoint{0.823420in}{3.866185in}}%
\pgfpathlineto{\pgfqpoint{0.800000in}{3.853742in}}%
\pgfpathlineto{\pgfqpoint{0.800000in}{3.828026in}}%
\pgfpathlineto{\pgfqpoint{0.841120in}{3.851635in}}%
\pgfpathlineto{\pgfqpoint{0.880162in}{3.866073in}}%
\pgfpathlineto{\pgfqpoint{0.898196in}{3.871202in}}%
\pgfpathlineto{\pgfqpoint{0.920242in}{3.875681in}}%
\pgfpathlineto{\pgfqpoint{0.967026in}{3.881757in}}%
\pgfpathlineto{\pgfqpoint{1.005593in}{3.883166in}}%
\pgfpathlineto{\pgfqpoint{1.040485in}{3.882122in}}%
\pgfpathlineto{\pgfqpoint{1.120646in}{3.873371in}}%
\pgfpathlineto{\pgfqpoint{1.227797in}{3.850667in}}%
\pgfpathlineto{\pgfqpoint{1.280970in}{3.835646in}}%
\pgfpathlineto{\pgfqpoint{1.298417in}{3.829584in}}%
\pgfpathlineto{\pgfqpoint{1.328355in}{3.820137in}}%
\pgfpathlineto{\pgfqpoint{1.368413in}{3.806550in}}%
\pgfpathlineto{\pgfqpoint{1.447032in}{3.776000in}}%
\pgfpathlineto{\pgfqpoint{1.533043in}{3.738667in}}%
\pgfpathlineto{\pgfqpoint{1.656945in}{3.678202in}}%
\pgfpathlineto{\pgfqpoint{1.685587in}{3.664000in}}%
\pgfpathlineto{\pgfqpoint{1.842101in}{3.577291in}}%
\pgfpathlineto{\pgfqpoint{1.922263in}{3.529487in}}%
\pgfpathlineto{\pgfqpoint{1.979658in}{3.493461in}}%
\pgfpathlineto{\pgfqpoint{2.006084in}{3.477333in}}%
\pgfpathlineto{\pgfqpoint{2.082586in}{3.427655in}}%
\pgfpathlineto{\pgfqpoint{2.175098in}{3.365333in}}%
\pgfpathlineto{\pgfqpoint{2.282990in}{3.289762in}}%
\pgfpathlineto{\pgfqpoint{2.333184in}{3.253333in}}%
\pgfpathlineto{\pgfqpoint{2.443313in}{3.171402in}}%
\pgfpathlineto{\pgfqpoint{2.504699in}{3.123844in}}%
\pgfpathlineto{\pgfqpoint{2.530873in}{3.104000in}}%
\pgfpathlineto{\pgfqpoint{2.643717in}{3.014300in}}%
\pgfpathlineto{\pgfqpoint{2.763960in}{2.915454in}}%
\pgfpathlineto{\pgfqpoint{2.884202in}{2.812945in}}%
\pgfpathlineto{\pgfqpoint{3.004444in}{2.707003in}}%
\pgfpathlineto{\pgfqpoint{3.124687in}{2.597622in}}%
\pgfpathlineto{\pgfqpoint{3.182244in}{2.544000in}}%
\pgfpathlineto{\pgfqpoint{3.285010in}{2.446318in}}%
\pgfpathlineto{\pgfqpoint{3.351155in}{2.381611in}}%
\pgfpathlineto{\pgfqpoint{3.405253in}{2.328656in}}%
\pgfpathlineto{\pgfqpoint{3.451248in}{2.282667in}}%
\pgfpathlineto{\pgfqpoint{3.565576in}{2.165903in}}%
\pgfpathlineto{\pgfqpoint{3.619588in}{2.108977in}}%
\pgfpathlineto{\pgfqpoint{3.667468in}{2.058667in}}%
\pgfpathlineto{\pgfqpoint{3.713070in}{2.009384in}}%
\pgfpathlineto{\pgfqpoint{3.765980in}{1.952345in}}%
\pgfpathlineto{\pgfqpoint{4.001572in}{1.685333in}}%
\pgfpathlineto{\pgfqpoint{4.046545in}{1.632083in}}%
\pgfpathlineto{\pgfqpoint{4.126707in}{1.535535in}}%
\pgfpathlineto{\pgfqpoint{4.186607in}{1.461333in}}%
\pgfpathlineto{\pgfqpoint{4.274914in}{1.349333in}}%
\pgfpathlineto{\pgfqpoint{4.303675in}{1.312000in}}%
\pgfpathlineto{\pgfqpoint{4.388060in}{1.200000in}}%
\pgfpathlineto{\pgfqpoint{4.428145in}{1.144774in}}%
\pgfpathlineto{\pgfqpoint{4.469449in}{1.088000in}}%
\pgfpathlineto{\pgfqpoint{4.495952in}{1.050667in}}%
\pgfpathlineto{\pgfqpoint{4.547730in}{0.976000in}}%
\pgfpathlineto{\pgfqpoint{4.573208in}{0.938667in}}%
\pgfpathlineto{\pgfqpoint{4.622786in}{0.864000in}}%
\pgfpathlineto{\pgfqpoint{4.670926in}{0.789333in}}%
\pgfpathlineto{\pgfqpoint{4.727919in}{0.697632in}}%
\pgfpathlineto{\pgfqpoint{4.768000in}{0.630625in}}%
\pgfpathlineto{\pgfqpoint{4.768000in}{0.640000in}}%
\pgfpathlineto{\pgfqpoint{4.768000in}{0.640000in}}%
\pgfusepath{fill}%
\end{pgfscope}%
\begin{pgfscope}%
\pgfpathrectangle{\pgfqpoint{0.800000in}{0.528000in}}{\pgfqpoint{3.968000in}{3.696000in}}%
\pgfusepath{clip}%
\pgfsetbuttcap%
\pgfsetroundjoin%
\definecolor{currentfill}{rgb}{0.283091,0.110553,0.431554}%
\pgfsetfillcolor{currentfill}%
\pgfsetlinewidth{0.000000pt}%
\definecolor{currentstroke}{rgb}{0.000000,0.000000,0.000000}%
\pgfsetstrokecolor{currentstroke}%
\pgfsetdash{}{0pt}%
\pgfpathmoveto{\pgfqpoint{2.804365in}{0.528000in}}%
\pgfpathlineto{\pgfqpoint{2.561275in}{0.752000in}}%
\pgfpathlineto{\pgfqpoint{2.443313in}{0.865874in}}%
\pgfpathlineto{\pgfqpoint{2.386222in}{0.922822in}}%
\pgfpathlineto{\pgfqpoint{2.332667in}{0.976000in}}%
\pgfpathlineto{\pgfqpoint{2.251689in}{1.058845in}}%
\pgfpathlineto{\pgfqpoint{2.202828in}{1.109234in}}%
\pgfpathlineto{\pgfqpoint{2.151970in}{1.162667in}}%
\pgfpathlineto{\pgfqpoint{2.042505in}{1.280134in}}%
\pgfpathlineto{\pgfqpoint{1.971905in}{1.358239in}}%
\pgfpathlineto{\pgfqpoint{1.922263in}{1.413471in}}%
\pgfpathlineto{\pgfqpoint{1.715344in}{1.654068in}}%
\pgfpathlineto{\pgfqpoint{1.601616in}{1.793328in}}%
\pgfpathlineto{\pgfqpoint{1.521455in}{1.894955in}}%
\pgfpathlineto{\pgfqpoint{1.441293in}{1.999624in}}%
\pgfpathlineto{\pgfqpoint{1.280970in}{2.220002in}}%
\pgfpathlineto{\pgfqpoint{1.200808in}{2.336931in}}%
\pgfpathlineto{\pgfqpoint{1.080566in}{2.523226in}}%
\pgfpathlineto{\pgfqpoint{1.022828in}{2.618667in}}%
\pgfpathlineto{\pgfqpoint{0.979589in}{2.693333in}}%
\pgfpathlineto{\pgfqpoint{0.945905in}{2.754570in}}%
\pgfpathlineto{\pgfqpoint{0.920242in}{2.801181in}}%
\pgfpathlineto{\pgfqpoint{0.879526in}{2.880000in}}%
\pgfpathlineto{\pgfqpoint{0.806884in}{3.035745in}}%
\pgfpathlineto{\pgfqpoint{0.800000in}{3.051129in}}%
\pgfpathlineto{\pgfqpoint{0.800000in}{3.026916in}}%
\pgfpathlineto{\pgfqpoint{0.840081in}{2.940508in}}%
\pgfpathlineto{\pgfqpoint{0.880162in}{2.859798in}}%
\pgfpathlineto{\pgfqpoint{0.920242in}{2.783572in}}%
\pgfpathlineto{\pgfqpoint{0.960323in}{2.710941in}}%
\pgfpathlineto{\pgfqpoint{1.058858in}{2.544000in}}%
\pgfpathlineto{\pgfqpoint{1.120646in}{2.445735in}}%
\pgfpathlineto{\pgfqpoint{1.240889in}{2.265450in}}%
\pgfpathlineto{\pgfqpoint{1.401212in}{2.042048in}}%
\pgfpathlineto{\pgfqpoint{1.601616in}{1.783210in}}%
\pgfpathlineto{\pgfqpoint{1.650669in}{1.722667in}}%
\pgfpathlineto{\pgfqpoint{1.743655in}{1.610667in}}%
\pgfpathlineto{\pgfqpoint{1.787307in}{1.559629in}}%
\pgfpathlineto{\pgfqpoint{1.839267in}{1.498667in}}%
\pgfpathlineto{\pgfqpoint{1.912818in}{1.415203in}}%
\pgfpathlineto{\pgfqpoint{1.962343in}{1.359362in}}%
\pgfpathlineto{\pgfqpoint{2.022572in}{1.293434in}}%
\pgfpathlineto{\pgfqpoint{2.073735in}{1.237333in}}%
\pgfpathlineto{\pgfqpoint{2.152740in}{1.153345in}}%
\pgfpathlineto{\pgfqpoint{2.202828in}{1.100456in}}%
\pgfpathlineto{\pgfqpoint{2.324018in}{0.976000in}}%
\pgfpathlineto{\pgfqpoint{2.443313in}{0.857425in}}%
\pgfpathlineto{\pgfqpoint{2.498943in}{0.803816in}}%
\pgfpathlineto{\pgfqpoint{2.552527in}{0.752000in}}%
\pgfpathlineto{\pgfqpoint{2.795160in}{0.528000in}}%
\pgfpathlineto{\pgfqpoint{2.804040in}{0.528000in}}%
\pgfpathmoveto{\pgfqpoint{4.768000in}{0.661200in}}%
\pgfpathlineto{\pgfqpoint{4.712206in}{0.752000in}}%
\pgfpathlineto{\pgfqpoint{4.673072in}{0.812912in}}%
\pgfpathlineto{\pgfqpoint{4.640276in}{0.864000in}}%
\pgfpathlineto{\pgfqpoint{4.607677in}{0.913028in}}%
\pgfpathlineto{\pgfqpoint{4.527515in}{1.029876in}}%
\pgfpathlineto{\pgfqpoint{4.447354in}{1.142095in}}%
\pgfpathlineto{\pgfqpoint{4.262300in}{1.386667in}}%
\pgfpathlineto{\pgfqpoint{4.166788in}{1.506406in}}%
\pgfpathlineto{\pgfqpoint{4.111895in}{1.573333in}}%
\pgfpathlineto{\pgfqpoint{4.017927in}{1.685333in}}%
\pgfpathlineto{\pgfqpoint{3.966384in}{1.745383in}}%
\pgfpathlineto{\pgfqpoint{3.886222in}{1.837069in}}%
\pgfpathlineto{\pgfqpoint{3.765980in}{1.970426in}}%
\pgfpathlineto{\pgfqpoint{3.719074in}{2.021333in}}%
\pgfpathlineto{\pgfqpoint{3.605657in}{2.141704in}}%
\pgfpathlineto{\pgfqpoint{3.533841in}{2.215774in}}%
\pgfpathlineto{\pgfqpoint{3.485414in}{2.265402in}}%
\pgfpathlineto{\pgfqpoint{3.365172in}{2.385389in}}%
\pgfpathlineto{\pgfqpoint{3.301284in}{2.447159in}}%
\pgfpathlineto{\pgfqpoint{3.244929in}{2.501737in}}%
\pgfpathlineto{\pgfqpoint{3.200297in}{2.544000in}}%
\pgfpathlineto{\pgfqpoint{3.079471in}{2.656000in}}%
\pgfpathlineto{\pgfqpoint{2.954661in}{2.768000in}}%
\pgfpathlineto{\pgfqpoint{2.844121in}{2.864152in}}%
\pgfpathlineto{\pgfqpoint{2.723879in}{2.965505in}}%
\pgfpathlineto{\pgfqpoint{2.683798in}{2.998556in}}%
\pgfpathlineto{\pgfqpoint{2.599614in}{3.066667in}}%
\pgfpathlineto{\pgfqpoint{2.514758in}{3.133214in}}%
\pgfpathlineto{\pgfqpoint{2.470875in}{3.167006in}}%
\pgfpathlineto{\pgfqpoint{2.443313in}{3.188458in}}%
\pgfpathlineto{\pgfqpoint{2.356748in}{3.253333in}}%
\pgfpathlineto{\pgfqpoint{2.242909in}{3.335668in}}%
\pgfpathlineto{\pgfqpoint{2.146393in}{3.402667in}}%
\pgfpathlineto{\pgfqpoint{1.962343in}{3.523246in}}%
\pgfpathlineto{\pgfqpoint{1.895220in}{3.564145in}}%
\pgfpathlineto{\pgfqpoint{1.853839in}{3.589333in}}%
\pgfpathlineto{\pgfqpoint{1.789509in}{3.626667in}}%
\pgfpathlineto{\pgfqpoint{1.628693in}{3.713446in}}%
\pgfpathlineto{\pgfqpoint{1.544401in}{3.754626in}}%
\pgfpathlineto{\pgfqpoint{1.481374in}{3.783227in}}%
\pgfpathlineto{\pgfqpoint{1.346405in}{3.836950in}}%
\pgfpathlineto{\pgfqpoint{1.309000in}{3.850667in}}%
\pgfpathlineto{\pgfqpoint{1.240889in}{3.872174in}}%
\pgfpathlineto{\pgfqpoint{1.181420in}{3.888000in}}%
\pgfpathlineto{\pgfqpoint{1.120646in}{3.901062in}}%
\pgfpathlineto{\pgfqpoint{1.080566in}{3.907560in}}%
\pgfpathlineto{\pgfqpoint{1.063103in}{3.909068in}}%
\pgfpathlineto{\pgfqpoint{1.040485in}{3.912230in}}%
\pgfpathlineto{\pgfqpoint{1.000404in}{3.914769in}}%
\pgfpathlineto{\pgfqpoint{0.988602in}{3.914340in}}%
\pgfpathlineto{\pgfqpoint{0.960323in}{3.914799in}}%
\pgfpathlineto{\pgfqpoint{0.920242in}{3.911853in}}%
\pgfpathlineto{\pgfqpoint{0.897802in}{3.908902in}}%
\pgfpathlineto{\pgfqpoint{0.880162in}{3.905335in}}%
\pgfpathlineto{\pgfqpoint{0.840081in}{3.894483in}}%
\pgfpathlineto{\pgfqpoint{0.807225in}{3.881270in}}%
\pgfpathlineto{\pgfqpoint{0.800000in}{3.877431in}}%
\pgfpathlineto{\pgfqpoint{0.800000in}{3.853742in}}%
\pgfpathlineto{\pgfqpoint{0.823420in}{3.866185in}}%
\pgfpathlineto{\pgfqpoint{0.840081in}{3.873157in}}%
\pgfpathlineto{\pgfqpoint{0.887655in}{3.888000in}}%
\pgfpathlineto{\pgfqpoint{0.920242in}{3.894178in}}%
\pgfpathlineto{\pgfqpoint{0.960323in}{3.898162in}}%
\pgfpathlineto{\pgfqpoint{0.971046in}{3.897988in}}%
\pgfpathlineto{\pgfqpoint{1.000404in}{3.899054in}}%
\pgfpathlineto{\pgfqpoint{1.040485in}{3.897342in}}%
\pgfpathlineto{\pgfqpoint{1.049205in}{3.896122in}}%
\pgfpathlineto{\pgfqpoint{1.080566in}{3.893415in}}%
\pgfpathlineto{\pgfqpoint{1.120646in}{3.887567in}}%
\pgfpathlineto{\pgfqpoint{1.200808in}{3.870412in}}%
\pgfpathlineto{\pgfqpoint{1.248127in}{3.857409in}}%
\pgfpathlineto{\pgfqpoint{1.280970in}{3.848025in}}%
\pgfpathlineto{\pgfqpoint{1.401212in}{3.805538in}}%
\pgfpathlineto{\pgfqpoint{1.450938in}{3.784984in}}%
\pgfpathlineto{\pgfqpoint{1.481374in}{3.772522in}}%
\pgfpathlineto{\pgfqpoint{1.641697in}{3.696728in}}%
\pgfpathlineto{\pgfqpoint{1.715285in}{3.657877in}}%
\pgfpathlineto{\pgfqpoint{1.746060in}{3.641458in}}%
\pgfpathlineto{\pgfqpoint{1.842101in}{3.586931in}}%
\pgfpathlineto{\pgfqpoint{1.922263in}{3.538832in}}%
\pgfpathlineto{\pgfqpoint{1.985559in}{3.498958in}}%
\pgfpathlineto{\pgfqpoint{2.020182in}{3.477333in}}%
\pgfpathlineto{\pgfqpoint{2.202828in}{3.355048in}}%
\pgfpathlineto{\pgfqpoint{2.293691in}{3.290667in}}%
\pgfpathlineto{\pgfqpoint{2.363152in}{3.239981in}}%
\pgfpathlineto{\pgfqpoint{2.445116in}{3.178667in}}%
\pgfpathlineto{\pgfqpoint{2.509729in}{3.128529in}}%
\pgfpathlineto{\pgfqpoint{2.541653in}{3.104000in}}%
\pgfpathlineto{\pgfqpoint{2.643717in}{3.022802in}}%
\pgfpathlineto{\pgfqpoint{2.726962in}{2.954667in}}%
\pgfpathlineto{\pgfqpoint{2.771642in}{2.917333in}}%
\pgfpathlineto{\pgfqpoint{2.884202in}{2.821285in}}%
\pgfpathlineto{\pgfqpoint{3.004444in}{2.715396in}}%
\pgfpathlineto{\pgfqpoint{3.124687in}{2.606068in}}%
\pgfpathlineto{\pgfqpoint{3.164768in}{2.568851in}}%
\pgfpathlineto{\pgfqpoint{3.285010in}{2.454837in}}%
\pgfpathlineto{\pgfqpoint{3.355700in}{2.385844in}}%
\pgfpathlineto{\pgfqpoint{3.405253in}{2.337229in}}%
\pgfpathlineto{\pgfqpoint{3.459779in}{2.282667in}}%
\pgfpathlineto{\pgfqpoint{3.569469in}{2.170667in}}%
\pgfpathlineto{\pgfqpoint{3.624035in}{2.113119in}}%
\pgfpathlineto{\pgfqpoint{3.675904in}{2.058667in}}%
\pgfpathlineto{\pgfqpoint{3.725899in}{2.004959in}}%
\pgfpathlineto{\pgfqpoint{3.846939in}{1.872000in}}%
\pgfpathlineto{\pgfqpoint{3.945486in}{1.760000in}}%
\pgfpathlineto{\pgfqpoint{4.046545in}{1.641954in}}%
\pgfpathlineto{\pgfqpoint{4.134469in}{1.536000in}}%
\pgfpathlineto{\pgfqpoint{4.182628in}{1.476088in}}%
\pgfpathlineto{\pgfqpoint{4.224627in}{1.424000in}}%
\pgfpathlineto{\pgfqpoint{4.254137in}{1.386667in}}%
\pgfpathlineto{\pgfqpoint{4.327111in}{1.292267in}}%
\pgfpathlineto{\pgfqpoint{4.487434in}{1.074714in}}%
\pgfpathlineto{\pgfqpoint{4.567596in}{0.959573in}}%
\pgfpathlineto{\pgfqpoint{4.631531in}{0.864000in}}%
\pgfpathlineto{\pgfqpoint{4.687838in}{0.776717in}}%
\pgfpathlineto{\pgfqpoint{4.768000in}{0.646179in}}%
\pgfpathlineto{\pgfqpoint{4.768000in}{0.646179in}}%
\pgfusepath{fill}%
\end{pgfscope}%
\begin{pgfscope}%
\pgfpathrectangle{\pgfqpoint{0.800000in}{0.528000in}}{\pgfqpoint{3.968000in}{3.696000in}}%
\pgfusepath{clip}%
\pgfsetbuttcap%
\pgfsetroundjoin%
\definecolor{currentfill}{rgb}{0.283091,0.110553,0.431554}%
\pgfsetfillcolor{currentfill}%
\pgfsetlinewidth{0.000000pt}%
\definecolor{currentstroke}{rgb}{0.000000,0.000000,0.000000}%
\pgfsetstrokecolor{currentstroke}%
\pgfsetdash{}{0pt}%
\pgfpathmoveto{\pgfqpoint{2.795160in}{0.528000in}}%
\pgfpathlineto{\pgfqpoint{2.671951in}{0.640000in}}%
\pgfpathlineto{\pgfqpoint{2.552527in}{0.752000in}}%
\pgfpathlineto{\pgfqpoint{2.436619in}{0.864000in}}%
\pgfpathlineto{\pgfqpoint{2.381783in}{0.918687in}}%
\pgfpathlineto{\pgfqpoint{2.323071in}{0.976955in}}%
\pgfpathlineto{\pgfqpoint{2.202828in}{1.100456in}}%
\pgfpathlineto{\pgfqpoint{2.143653in}{1.162667in}}%
\pgfpathlineto{\pgfqpoint{2.039226in}{1.274667in}}%
\pgfpathlineto{\pgfqpoint{1.985804in}{1.333852in}}%
\pgfpathlineto{\pgfqpoint{1.937938in}{1.386667in}}%
\pgfpathlineto{\pgfqpoint{1.839267in}{1.498667in}}%
\pgfpathlineto{\pgfqpoint{1.787307in}{1.559629in}}%
\pgfpathlineto{\pgfqpoint{1.743655in}{1.610667in}}%
\pgfpathlineto{\pgfqpoint{1.712320in}{1.648000in}}%
\pgfpathlineto{\pgfqpoint{1.620366in}{1.760000in}}%
\pgfpathlineto{\pgfqpoint{1.578271in}{1.812922in}}%
\pgfpathlineto{\pgfqpoint{1.531136in}{1.872000in}}%
\pgfpathlineto{\pgfqpoint{1.441293in}{1.988547in}}%
\pgfpathlineto{\pgfqpoint{1.377823in}{2.074214in}}%
\pgfpathlineto{\pgfqpoint{1.361131in}{2.096162in}}%
\pgfpathlineto{\pgfqpoint{1.280505in}{2.208433in}}%
\pgfpathlineto{\pgfqpoint{1.200808in}{2.324017in}}%
\pgfpathlineto{\pgfqpoint{1.058858in}{2.544000in}}%
\pgfpathlineto{\pgfqpoint{1.000404in}{2.641233in}}%
\pgfpathlineto{\pgfqpoint{0.920242in}{2.783572in}}%
\pgfpathlineto{\pgfqpoint{0.851330in}{2.917333in}}%
\pgfpathlineto{\pgfqpoint{0.815922in}{2.992000in}}%
\pgfpathlineto{\pgfqpoint{0.800000in}{3.026916in}}%
\pgfpathlineto{\pgfqpoint{0.800000in}{3.004546in}}%
\pgfpathlineto{\pgfqpoint{0.823400in}{2.954667in}}%
\pgfpathlineto{\pgfqpoint{0.860236in}{2.880000in}}%
\pgfpathlineto{\pgfqpoint{0.879571in}{2.842117in}}%
\pgfpathlineto{\pgfqpoint{0.899159in}{2.805333in}}%
\pgfpathlineto{\pgfqpoint{0.933524in}{2.743038in}}%
\pgfpathlineto{\pgfqpoint{0.940033in}{2.730667in}}%
\pgfpathlineto{\pgfqpoint{0.982733in}{2.656000in}}%
\pgfpathlineto{\pgfqpoint{1.017814in}{2.597550in}}%
\pgfpathlineto{\pgfqpoint{1.040485in}{2.559418in}}%
\pgfpathlineto{\pgfqpoint{1.096854in}{2.469333in}}%
\pgfpathlineto{\pgfqpoint{1.160727in}{2.371208in}}%
\pgfpathlineto{\pgfqpoint{1.220597in}{2.282667in}}%
\pgfpathlineto{\pgfqpoint{1.280970in}{2.196078in}}%
\pgfpathlineto{\pgfqpoint{1.361131in}{2.085015in}}%
\pgfpathlineto{\pgfqpoint{1.422232in}{2.003579in}}%
\pgfpathlineto{\pgfqpoint{1.465098in}{1.946667in}}%
\pgfpathlineto{\pgfqpoint{1.552389in}{1.834667in}}%
\pgfpathlineto{\pgfqpoint{1.644790in}{1.719785in}}%
\pgfpathlineto{\pgfqpoint{1.761939in}{1.579333in}}%
\pgfpathlineto{\pgfqpoint{1.842101in}{1.486133in}}%
\pgfpathlineto{\pgfqpoint{1.908407in}{1.411094in}}%
\pgfpathlineto{\pgfqpoint{1.963085in}{1.349333in}}%
\pgfpathlineto{\pgfqpoint{2.018121in}{1.289287in}}%
\pgfpathlineto{\pgfqpoint{2.065543in}{1.237333in}}%
\pgfpathlineto{\pgfqpoint{2.122667in}{1.176096in}}%
\pgfpathlineto{\pgfqpoint{2.242909in}{1.050126in}}%
\pgfpathlineto{\pgfqpoint{2.282990in}{1.009101in}}%
\pgfpathlineto{\pgfqpoint{2.403232in}{0.888452in}}%
\pgfpathlineto{\pgfqpoint{2.466270in}{0.826667in}}%
\pgfpathlineto{\pgfqpoint{2.563556in}{0.733153in}}%
\pgfpathlineto{\pgfqpoint{2.683798in}{0.620767in}}%
\pgfpathlineto{\pgfqpoint{2.785967in}{0.528000in}}%
\pgfpathmoveto{\pgfqpoint{4.768000in}{0.676221in}}%
\pgfpathlineto{\pgfqpoint{4.727919in}{0.741036in}}%
\pgfpathlineto{\pgfqpoint{4.673263in}{0.826667in}}%
\pgfpathlineto{\pgfqpoint{4.607677in}{0.925701in}}%
\pgfpathlineto{\pgfqpoint{4.547469in}{1.013333in}}%
\pgfpathlineto{\pgfqpoint{4.521341in}{1.050667in}}%
\pgfpathlineto{\pgfqpoint{4.447354in}{1.153527in}}%
\pgfpathlineto{\pgfqpoint{4.287030in}{1.365533in}}%
\pgfpathlineto{\pgfqpoint{4.206869in}{1.466869in}}%
\pgfpathlineto{\pgfqpoint{4.140126in}{1.548499in}}%
\pgfpathlineto{\pgfqpoint{4.089119in}{1.610667in}}%
\pgfpathlineto{\pgfqpoint{4.017205in}{1.695338in}}%
\pgfpathlineto{\pgfqpoint{3.966384in}{1.754854in}}%
\pgfpathlineto{\pgfqpoint{3.886222in}{1.846171in}}%
\pgfpathlineto{\pgfqpoint{3.761814in}{1.984000in}}%
\pgfpathlineto{\pgfqpoint{3.657380in}{2.096000in}}%
\pgfpathlineto{\pgfqpoint{3.586137in}{2.170667in}}%
\pgfpathlineto{\pgfqpoint{3.476839in}{2.282667in}}%
\pgfpathlineto{\pgfqpoint{3.422820in}{2.336363in}}%
\pgfpathlineto{\pgfqpoint{3.364434in}{2.394667in}}%
\pgfpathlineto{\pgfqpoint{3.305706in}{2.451277in}}%
\pgfpathlineto{\pgfqpoint{3.248553in}{2.506667in}}%
\pgfpathlineto{\pgfqpoint{3.164768in}{2.585641in}}%
\pgfpathlineto{\pgfqpoint{3.106492in}{2.639052in}}%
\pgfpathlineto{\pgfqpoint{3.047544in}{2.693333in}}%
\pgfpathlineto{\pgfqpoint{2.804040in}{2.906611in}}%
\pgfpathlineto{\pgfqpoint{2.683798in}{3.006810in}}%
\pgfpathlineto{\pgfqpoint{2.563213in}{3.104000in}}%
\pgfpathlineto{\pgfqpoint{2.443313in}{3.196870in}}%
\pgfpathlineto{\pgfqpoint{2.323071in}{3.286640in}}%
\pgfpathlineto{\pgfqpoint{2.282990in}{3.315632in}}%
\pgfpathlineto{\pgfqpoint{2.161301in}{3.401319in}}%
\pgfpathlineto{\pgfqpoint{2.122667in}{3.427610in}}%
\pgfpathlineto{\pgfqpoint{2.029403in}{3.489538in}}%
\pgfpathlineto{\pgfqpoint{1.922263in}{3.557334in}}%
\pgfpathlineto{\pgfqpoint{1.869474in}{3.589333in}}%
\pgfpathlineto{\pgfqpoint{1.802020in}{3.629116in}}%
\pgfpathlineto{\pgfqpoint{1.605108in}{3.735414in}}%
\pgfpathlineto{\pgfqpoint{1.521455in}{3.775782in}}%
\pgfpathlineto{\pgfqpoint{1.520971in}{3.776000in}}%
\pgfpathlineto{\pgfqpoint{1.436434in}{3.813333in}}%
\pgfpathlineto{\pgfqpoint{1.355431in}{3.845357in}}%
\pgfpathlineto{\pgfqpoint{1.321051in}{3.858180in}}%
\pgfpathlineto{\pgfqpoint{1.200808in}{3.895836in}}%
\pgfpathlineto{\pgfqpoint{1.120646in}{3.914535in}}%
\pgfpathlineto{\pgfqpoint{1.077001in}{3.922013in}}%
\pgfpathlineto{\pgfqpoint{1.038403in}{3.927273in}}%
\pgfpathlineto{\pgfqpoint{0.994805in}{3.930548in}}%
\pgfpathlineto{\pgfqpoint{0.960323in}{3.931075in}}%
\pgfpathlineto{\pgfqpoint{0.920242in}{3.929265in}}%
\pgfpathlineto{\pgfqpoint{0.878395in}{3.923688in}}%
\pgfpathlineto{\pgfqpoint{0.840081in}{3.914677in}}%
\pgfpathlineto{\pgfqpoint{0.800000in}{3.900042in}}%
\pgfpathlineto{\pgfqpoint{0.800000in}{3.877431in}}%
\pgfpathlineto{\pgfqpoint{0.807225in}{3.881270in}}%
\pgfpathlineto{\pgfqpoint{0.840081in}{3.894483in}}%
\pgfpathlineto{\pgfqpoint{0.849392in}{3.896673in}}%
\pgfpathlineto{\pgfqpoint{0.880162in}{3.905335in}}%
\pgfpathlineto{\pgfqpoint{0.897802in}{3.908902in}}%
\pgfpathlineto{\pgfqpoint{0.933122in}{3.913337in}}%
\pgfpathlineto{\pgfqpoint{0.960323in}{3.914799in}}%
\pgfpathlineto{\pgfqpoint{1.012121in}{3.914419in}}%
\pgfpathlineto{\pgfqpoint{1.040485in}{3.912230in}}%
\pgfpathlineto{\pgfqpoint{1.131797in}{3.898386in}}%
\pgfpathlineto{\pgfqpoint{1.181420in}{3.888000in}}%
\pgfpathlineto{\pgfqpoint{1.240889in}{3.872174in}}%
\pgfpathlineto{\pgfqpoint{1.257872in}{3.866486in}}%
\pgfpathlineto{\pgfqpoint{1.288113in}{3.857321in}}%
\pgfpathlineto{\pgfqpoint{1.321051in}{3.846655in}}%
\pgfpathlineto{\pgfqpoint{1.441293in}{3.800364in}}%
\pgfpathlineto{\pgfqpoint{1.481374in}{3.783227in}}%
\pgfpathlineto{\pgfqpoint{1.566896in}{3.743660in}}%
\pgfpathlineto{\pgfqpoint{1.628693in}{3.713446in}}%
\pgfpathlineto{\pgfqpoint{1.722757in}{3.664000in}}%
\pgfpathlineto{\pgfqpoint{1.802020in}{3.619588in}}%
\pgfpathlineto{\pgfqpoint{1.853839in}{3.589333in}}%
\pgfpathlineto{\pgfqpoint{1.922263in}{3.548177in}}%
\pgfpathlineto{\pgfqpoint{1.991461in}{3.504455in}}%
\pgfpathlineto{\pgfqpoint{2.034280in}{3.477333in}}%
\pgfpathlineto{\pgfqpoint{2.202828in}{3.363912in}}%
\pgfpathlineto{\pgfqpoint{2.282990in}{3.307019in}}%
\pgfpathlineto{\pgfqpoint{2.323071in}{3.278007in}}%
\pgfpathlineto{\pgfqpoint{2.406936in}{3.216000in}}%
\pgfpathlineto{\pgfqpoint{2.523475in}{3.126694in}}%
\pgfpathlineto{\pgfqpoint{2.646063in}{3.029333in}}%
\pgfpathlineto{\pgfqpoint{2.763960in}{2.932089in}}%
\pgfpathlineto{\pgfqpoint{2.825544in}{2.880000in}}%
\pgfpathlineto{\pgfqpoint{2.924283in}{2.794725in}}%
\pgfpathlineto{\pgfqpoint{3.044525in}{2.687749in}}%
\pgfpathlineto{\pgfqpoint{3.164768in}{2.577315in}}%
\pgfpathlineto{\pgfqpoint{3.285010in}{2.463355in}}%
\pgfpathlineto{\pgfqpoint{3.340533in}{2.409051in}}%
\pgfpathlineto{\pgfqpoint{3.393595in}{2.357333in}}%
\pgfpathlineto{\pgfqpoint{3.468309in}{2.282667in}}%
\pgfpathlineto{\pgfqpoint{3.577803in}{2.170667in}}%
\pgfpathlineto{\pgfqpoint{3.628482in}{2.117261in}}%
\pgfpathlineto{\pgfqpoint{3.685017in}{2.057920in}}%
\pgfpathlineto{\pgfqpoint{3.725899in}{2.013979in}}%
\pgfpathlineto{\pgfqpoint{3.846141in}{1.881974in}}%
\pgfpathlineto{\pgfqpoint{4.080954in}{1.610667in}}%
\pgfpathlineto{\pgfqpoint{4.135585in}{1.544269in}}%
\pgfpathlineto{\pgfqpoint{4.173067in}{1.498667in}}%
\pgfpathlineto{\pgfqpoint{4.206869in}{1.456726in}}%
\pgfpathlineto{\pgfqpoint{4.303274in}{1.334203in}}%
\pgfpathlineto{\pgfqpoint{4.399645in}{1.207105in}}%
\pgfpathlineto{\pgfqpoint{4.484670in}{1.090575in}}%
\pgfpathlineto{\pgfqpoint{4.560416in}{0.982688in}}%
\pgfpathlineto{\pgfqpoint{4.615522in}{0.901333in}}%
\pgfpathlineto{\pgfqpoint{4.647758in}{0.852575in}}%
\pgfpathlineto{\pgfqpoint{4.712206in}{0.752000in}}%
\pgfpathlineto{\pgfqpoint{4.768000in}{0.661200in}}%
\pgfpathlineto{\pgfqpoint{4.768000in}{0.661200in}}%
\pgfusepath{fill}%
\end{pgfscope}%
\begin{pgfscope}%
\pgfpathrectangle{\pgfqpoint{0.800000in}{0.528000in}}{\pgfqpoint{3.968000in}{3.696000in}}%
\pgfusepath{clip}%
\pgfsetbuttcap%
\pgfsetroundjoin%
\definecolor{currentfill}{rgb}{0.283091,0.110553,0.431554}%
\pgfsetfillcolor{currentfill}%
\pgfsetlinewidth{0.000000pt}%
\definecolor{currentstroke}{rgb}{0.000000,0.000000,0.000000}%
\pgfsetstrokecolor{currentstroke}%
\pgfsetdash{}{0pt}%
\pgfpathmoveto{\pgfqpoint{2.785967in}{0.528000in}}%
\pgfpathlineto{\pgfqpoint{2.683798in}{0.620767in}}%
\pgfpathlineto{\pgfqpoint{2.622845in}{0.677333in}}%
\pgfpathlineto{\pgfqpoint{2.523475in}{0.771389in}}%
\pgfpathlineto{\pgfqpoint{2.455223in}{0.837760in}}%
\pgfpathlineto{\pgfqpoint{2.403232in}{0.888452in}}%
\pgfpathlineto{\pgfqpoint{2.278847in}{1.013333in}}%
\pgfpathlineto{\pgfqpoint{2.223936in}{1.070328in}}%
\pgfpathlineto{\pgfqpoint{2.170693in}{1.125333in}}%
\pgfpathlineto{\pgfqpoint{2.082586in}{1.218950in}}%
\pgfpathlineto{\pgfqpoint{2.018121in}{1.289287in}}%
\pgfpathlineto{\pgfqpoint{1.962343in}{1.350157in}}%
\pgfpathlineto{\pgfqpoint{1.842101in}{1.486133in}}%
\pgfpathlineto{\pgfqpoint{1.641697in}{1.723569in}}%
\pgfpathlineto{\pgfqpoint{1.552389in}{1.834667in}}%
\pgfpathlineto{\pgfqpoint{1.521455in}{1.873802in}}%
\pgfpathlineto{\pgfqpoint{1.436525in}{1.984000in}}%
\pgfpathlineto{\pgfqpoint{1.401212in}{2.030940in}}%
\pgfpathlineto{\pgfqpoint{1.321051in}{2.139982in}}%
\pgfpathlineto{\pgfqpoint{1.272530in}{2.208000in}}%
\pgfpathlineto{\pgfqpoint{1.200808in}{2.311517in}}%
\pgfpathlineto{\pgfqpoint{1.080566in}{2.494934in}}%
\pgfpathlineto{\pgfqpoint{1.027144in}{2.581333in}}%
\pgfpathlineto{\pgfqpoint{0.989184in}{2.645549in}}%
\pgfpathlineto{\pgfqpoint{0.960982in}{2.693333in}}%
\pgfpathlineto{\pgfqpoint{0.933524in}{2.743038in}}%
\pgfpathlineto{\pgfqpoint{0.919184in}{2.768000in}}%
\pgfpathlineto{\pgfqpoint{0.823400in}{2.954667in}}%
\pgfpathlineto{\pgfqpoint{0.805721in}{2.992000in}}%
\pgfpathlineto{\pgfqpoint{0.800000in}{3.004546in}}%
\pgfpathlineto{\pgfqpoint{0.800000in}{2.982941in}}%
\pgfpathlineto{\pgfqpoint{0.880162in}{2.823278in}}%
\pgfpathlineto{\pgfqpoint{0.920242in}{2.749580in}}%
\pgfpathlineto{\pgfqpoint{1.018305in}{2.581333in}}%
\pgfpathlineto{\pgfqpoint{1.064503in}{2.506667in}}%
\pgfpathlineto{\pgfqpoint{1.120646in}{2.419064in}}%
\pgfpathlineto{\pgfqpoint{1.200808in}{2.299204in}}%
\pgfpathlineto{\pgfqpoint{1.264249in}{2.208000in}}%
\pgfpathlineto{\pgfqpoint{1.303164in}{2.154006in}}%
\pgfpathlineto{\pgfqpoint{1.344880in}{2.096000in}}%
\pgfpathlineto{\pgfqpoint{1.428438in}{1.984000in}}%
\pgfpathlineto{\pgfqpoint{1.514823in}{1.872000in}}%
\pgfpathlineto{\pgfqpoint{1.611394in}{1.750893in}}%
\pgfpathlineto{\pgfqpoint{1.721859in}{1.617227in}}%
\pgfpathlineto{\pgfqpoint{1.922263in}{1.385822in}}%
\pgfpathlineto{\pgfqpoint{2.042505in}{1.253383in}}%
\pgfpathlineto{\pgfqpoint{2.162747in}{1.124890in}}%
\pgfpathlineto{\pgfqpoint{2.219551in}{1.066243in}}%
\pgfpathlineto{\pgfqpoint{2.270566in}{1.013333in}}%
\pgfpathlineto{\pgfqpoint{2.344361in}{0.938667in}}%
\pgfpathlineto{\pgfqpoint{2.457660in}{0.826667in}}%
\pgfpathlineto{\pgfqpoint{2.574293in}{0.714667in}}%
\pgfpathlineto{\pgfqpoint{2.643717in}{0.649542in}}%
\pgfpathlineto{\pgfqpoint{2.763960in}{0.539502in}}%
\pgfpathlineto{\pgfqpoint{2.776775in}{0.528000in}}%
\pgfpathmoveto{\pgfqpoint{4.768000in}{0.690495in}}%
\pgfpathlineto{\pgfqpoint{4.632523in}{0.901333in}}%
\pgfpathlineto{\pgfqpoint{4.607139in}{0.939167in}}%
\pgfpathlineto{\pgfqpoint{4.527515in}{1.053777in}}%
\pgfpathlineto{\pgfqpoint{4.447354in}{1.164865in}}%
\pgfpathlineto{\pgfqpoint{4.393366in}{1.237333in}}%
\pgfpathlineto{\pgfqpoint{4.307737in}{1.349333in}}%
\pgfpathlineto{\pgfqpoint{4.278627in}{1.386667in}}%
\pgfpathlineto{\pgfqpoint{4.206869in}{1.476839in}}%
\pgfpathlineto{\pgfqpoint{4.144667in}{1.552729in}}%
\pgfpathlineto{\pgfqpoint{4.097095in}{1.610667in}}%
\pgfpathlineto{\pgfqpoint{4.039505in}{1.678776in}}%
\pgfpathlineto{\pgfqpoint{4.002253in}{1.722667in}}%
\pgfpathlineto{\pgfqpoint{3.966384in}{1.764175in}}%
\pgfpathlineto{\pgfqpoint{3.846141in}{1.900136in}}%
\pgfpathlineto{\pgfqpoint{3.765980in}{1.988357in}}%
\pgfpathlineto{\pgfqpoint{3.645737in}{2.117003in}}%
\pgfpathlineto{\pgfqpoint{3.594471in}{2.170667in}}%
\pgfpathlineto{\pgfqpoint{3.485370in}{2.282667in}}%
\pgfpathlineto{\pgfqpoint{3.427201in}{2.340444in}}%
\pgfpathlineto{\pgfqpoint{3.372886in}{2.394667in}}%
\pgfpathlineto{\pgfqpoint{3.124687in}{2.631013in}}%
\pgfpathlineto{\pgfqpoint{3.084606in}{2.667806in}}%
\pgfpathlineto{\pgfqpoint{2.964364in}{2.775946in}}%
\pgfpathlineto{\pgfqpoint{2.834186in}{2.889254in}}%
\pgfpathlineto{\pgfqpoint{2.711815in}{2.992000in}}%
\pgfpathlineto{\pgfqpoint{2.603636in}{3.080009in}}%
\pgfpathlineto{\pgfqpoint{2.563556in}{3.111943in}}%
\pgfpathlineto{\pgfqpoint{2.478111in}{3.178667in}}%
\pgfpathlineto{\pgfqpoint{2.403232in}{3.235588in}}%
\pgfpathlineto{\pgfqpoint{2.282990in}{3.324246in}}%
\pgfpathlineto{\pgfqpoint{2.162747in}{3.409009in}}%
\pgfpathlineto{\pgfqpoint{2.096871in}{3.453306in}}%
\pgfpathlineto{\pgfqpoint{2.061358in}{3.477333in}}%
\pgfpathlineto{\pgfqpoint{1.877009in}{3.594152in}}%
\pgfpathlineto{\pgfqpoint{1.761939in}{3.661417in}}%
\pgfpathlineto{\pgfqpoint{1.671273in}{3.711118in}}%
\pgfpathlineto{\pgfqpoint{1.601616in}{3.747064in}}%
\pgfpathlineto{\pgfqpoint{1.528378in}{3.782449in}}%
\pgfpathlineto{\pgfqpoint{1.499989in}{3.795994in}}%
\pgfpathlineto{\pgfqpoint{1.424140in}{3.829311in}}%
\pgfpathlineto{\pgfqpoint{1.361131in}{3.854687in}}%
\pgfpathlineto{\pgfqpoint{1.227290in}{3.900667in}}%
\pgfpathlineto{\pgfqpoint{1.160727in}{3.918707in}}%
\pgfpathlineto{\pgfqpoint{1.154801in}{3.919814in}}%
\pgfpathlineto{\pgfqpoint{1.117087in}{3.928649in}}%
\pgfpathlineto{\pgfqpoint{1.080566in}{3.935318in}}%
\pgfpathlineto{\pgfqpoint{1.000404in}{3.945028in}}%
\pgfpathlineto{\pgfqpoint{0.960323in}{3.946728in}}%
\pgfpathlineto{\pgfqpoint{0.942229in}{3.945813in}}%
\pgfpathlineto{\pgfqpoint{0.920242in}{3.945833in}}%
\pgfpathlineto{\pgfqpoint{0.864840in}{3.939604in}}%
\pgfpathlineto{\pgfqpoint{0.840081in}{3.934194in}}%
\pgfpathlineto{\pgfqpoint{0.800000in}{3.921785in}}%
\pgfpathlineto{\pgfqpoint{0.800000in}{3.900042in}}%
\pgfpathlineto{\pgfqpoint{0.840081in}{3.914677in}}%
\pgfpathlineto{\pgfqpoint{0.888348in}{3.925333in}}%
\pgfpathlineto{\pgfqpoint{0.924459in}{3.929261in}}%
\pgfpathlineto{\pgfqpoint{0.960323in}{3.931075in}}%
\pgfpathlineto{\pgfqpoint{1.000404in}{3.930194in}}%
\pgfpathlineto{\pgfqpoint{1.053258in}{3.925333in}}%
\pgfpathlineto{\pgfqpoint{1.085183in}{3.921032in}}%
\pgfpathlineto{\pgfqpoint{1.120646in}{3.914535in}}%
\pgfpathlineto{\pgfqpoint{1.228579in}{3.888000in}}%
\pgfpathlineto{\pgfqpoint{1.240889in}{3.884516in}}%
\pgfpathlineto{\pgfqpoint{1.361131in}{3.843476in}}%
\pgfpathlineto{\pgfqpoint{1.521916in}{3.775570in}}%
\pgfpathlineto{\pgfqpoint{1.605108in}{3.735414in}}%
\pgfpathlineto{\pgfqpoint{1.695771in}{3.688300in}}%
\pgfpathlineto{\pgfqpoint{1.802020in}{3.629116in}}%
\pgfpathlineto{\pgfqpoint{1.882182in}{3.581755in}}%
\pgfpathlineto{\pgfqpoint{1.949436in}{3.539977in}}%
\pgfpathlineto{\pgfqpoint{1.990173in}{3.514667in}}%
\pgfpathlineto{\pgfqpoint{2.162747in}{3.400376in}}%
\pgfpathlineto{\pgfqpoint{2.213055in}{3.365333in}}%
\pgfpathlineto{\pgfqpoint{2.323071in}{3.286640in}}%
\pgfpathlineto{\pgfqpoint{2.418047in}{3.216000in}}%
\pgfpathlineto{\pgfqpoint{2.523475in}{3.135142in}}%
\pgfpathlineto{\pgfqpoint{2.610002in}{3.066667in}}%
\pgfpathlineto{\pgfqpoint{2.683798in}{3.006810in}}%
\pgfpathlineto{\pgfqpoint{2.804040in}{2.906611in}}%
\pgfpathlineto{\pgfqpoint{2.924283in}{2.803082in}}%
\pgfpathlineto{\pgfqpoint{3.047544in}{2.693333in}}%
\pgfpathlineto{\pgfqpoint{3.169372in}{2.581333in}}%
\pgfpathlineto{\pgfqpoint{3.287562in}{2.469333in}}%
\pgfpathlineto{\pgfqpoint{3.344941in}{2.413156in}}%
\pgfpathlineto{\pgfqpoint{3.402261in}{2.357333in}}%
\pgfpathlineto{\pgfqpoint{3.657380in}{2.096000in}}%
\pgfpathlineto{\pgfqpoint{3.765980in}{1.979466in}}%
\pgfpathlineto{\pgfqpoint{3.837055in}{1.900870in}}%
\pgfpathlineto{\pgfqpoint{3.886222in}{1.846171in}}%
\pgfpathlineto{\pgfqpoint{3.966384in}{1.754854in}}%
\pgfpathlineto{\pgfqpoint{4.034921in}{1.674506in}}%
\pgfpathlineto{\pgfqpoint{4.086626in}{1.613649in}}%
\pgfpathlineto{\pgfqpoint{4.270464in}{1.386667in}}%
\pgfpathlineto{\pgfqpoint{4.357009in}{1.274667in}}%
\pgfpathlineto{\pgfqpoint{4.440670in}{1.162667in}}%
\pgfpathlineto{\pgfqpoint{4.467814in}{1.125333in}}%
\pgfpathlineto{\pgfqpoint{4.527515in}{1.041897in}}%
\pgfpathlineto{\pgfqpoint{4.687838in}{0.804155in}}%
\pgfpathlineto{\pgfqpoint{4.744275in}{0.714667in}}%
\pgfpathlineto{\pgfqpoint{4.768000in}{0.676221in}}%
\pgfpathlineto{\pgfqpoint{4.768000in}{0.677333in}}%
\pgfpathlineto{\pgfqpoint{4.768000in}{0.677333in}}%
\pgfusepath{fill}%
\end{pgfscope}%
\begin{pgfscope}%
\pgfpathrectangle{\pgfqpoint{0.800000in}{0.528000in}}{\pgfqpoint{3.968000in}{3.696000in}}%
\pgfusepath{clip}%
\pgfsetbuttcap%
\pgfsetroundjoin%
\definecolor{currentfill}{rgb}{0.283197,0.115680,0.436115}%
\pgfsetfillcolor{currentfill}%
\pgfsetlinewidth{0.000000pt}%
\definecolor{currentstroke}{rgb}{0.000000,0.000000,0.000000}%
\pgfsetstrokecolor{currentstroke}%
\pgfsetdash{}{0pt}%
\pgfpathmoveto{\pgfqpoint{2.776775in}{0.528000in}}%
\pgfpathlineto{\pgfqpoint{2.683798in}{0.612482in}}%
\pgfpathlineto{\pgfqpoint{2.629032in}{0.663655in}}%
\pgfpathlineto{\pgfqpoint{2.574293in}{0.714667in}}%
\pgfpathlineto{\pgfqpoint{2.483394in}{0.801644in}}%
\pgfpathlineto{\pgfqpoint{2.363152in}{0.919842in}}%
\pgfpathlineto{\pgfqpoint{2.242909in}{1.041648in}}%
\pgfpathlineto{\pgfqpoint{2.122667in}{1.167279in}}%
\pgfpathlineto{\pgfqpoint{2.057351in}{1.237333in}}%
\pgfpathlineto{\pgfqpoint{1.955050in}{1.349333in}}%
\pgfpathlineto{\pgfqpoint{1.855681in}{1.461333in}}%
\pgfpathlineto{\pgfqpoint{1.802020in}{1.523101in}}%
\pgfpathlineto{\pgfqpoint{1.721859in}{1.617227in}}%
\pgfpathlineto{\pgfqpoint{1.665188in}{1.685333in}}%
\pgfpathlineto{\pgfqpoint{1.574066in}{1.797333in}}%
\pgfpathlineto{\pgfqpoint{1.534827in}{1.847122in}}%
\pgfpathlineto{\pgfqpoint{1.496870in}{1.894899in}}%
\pgfpathlineto{\pgfqpoint{1.428438in}{1.984000in}}%
\pgfpathlineto{\pgfqpoint{1.397601in}{2.024697in}}%
\pgfpathlineto{\pgfqpoint{1.317559in}{2.133333in}}%
\pgfpathlineto{\pgfqpoint{1.264249in}{2.208000in}}%
\pgfpathlineto{\pgfqpoint{1.200808in}{2.299204in}}%
\pgfpathlineto{\pgfqpoint{1.064503in}{2.506667in}}%
\pgfpathlineto{\pgfqpoint{1.018305in}{2.581333in}}%
\pgfpathlineto{\pgfqpoint{0.983452in}{2.640210in}}%
\pgfpathlineto{\pgfqpoint{0.960323in}{2.678950in}}%
\pgfpathlineto{\pgfqpoint{0.880162in}{2.823278in}}%
\pgfpathlineto{\pgfqpoint{0.813530in}{2.954667in}}%
\pgfpathlineto{\pgfqpoint{0.800000in}{2.982941in}}%
\pgfpathlineto{\pgfqpoint{0.800000in}{2.962314in}}%
\pgfpathlineto{\pgfqpoint{0.840946in}{2.880000in}}%
\pgfpathlineto{\pgfqpoint{0.943052in}{2.693333in}}%
\pgfpathlineto{\pgfqpoint{0.977720in}{2.634871in}}%
\pgfpathlineto{\pgfqpoint{1.006305in}{2.586830in}}%
\pgfpathlineto{\pgfqpoint{1.021336in}{2.561837in}}%
\pgfpathlineto{\pgfqpoint{1.080566in}{2.467592in}}%
\pgfpathlineto{\pgfqpoint{1.160727in}{2.345858in}}%
\pgfpathlineto{\pgfqpoint{1.321051in}{2.117360in}}%
\pgfpathlineto{\pgfqpoint{1.401212in}{2.009231in}}%
\pgfpathlineto{\pgfqpoint{1.448802in}{1.946667in}}%
\pgfpathlineto{\pgfqpoint{1.521455in}{1.853394in}}%
\pgfpathlineto{\pgfqpoint{1.721859in}{1.607662in}}%
\pgfpathlineto{\pgfqpoint{1.782948in}{1.536000in}}%
\pgfpathlineto{\pgfqpoint{1.882182in}{1.421921in}}%
\pgfpathlineto{\pgfqpoint{1.947039in}{1.349333in}}%
\pgfpathlineto{\pgfqpoint{2.049159in}{1.237333in}}%
\pgfpathlineto{\pgfqpoint{2.102191in}{1.180927in}}%
\pgfpathlineto{\pgfqpoint{2.154230in}{1.125333in}}%
\pgfpathlineto{\pgfqpoint{2.225948in}{1.050667in}}%
\pgfpathlineto{\pgfqpoint{2.335951in}{0.938667in}}%
\pgfpathlineto{\pgfqpoint{2.449051in}{0.826667in}}%
\pgfpathlineto{\pgfqpoint{2.565473in}{0.714667in}}%
\pgfpathlineto{\pgfqpoint{2.624645in}{0.659568in}}%
\pgfpathlineto{\pgfqpoint{2.683798in}{0.604197in}}%
\pgfpathlineto{\pgfqpoint{2.767582in}{0.528000in}}%
\pgfpathmoveto{\pgfqpoint{4.768000in}{0.704709in}}%
\pgfpathlineto{\pgfqpoint{4.665846in}{0.864000in}}%
\pgfpathlineto{\pgfqpoint{4.627854in}{0.920127in}}%
\pgfpathlineto{\pgfqpoint{4.590093in}{0.976000in}}%
\pgfpathlineto{\pgfqpoint{4.549412in}{1.033729in}}%
\pgfpathlineto{\pgfqpoint{4.511242in}{1.088000in}}%
\pgfpathlineto{\pgfqpoint{4.478095in}{1.134032in}}%
\pgfpathlineto{\pgfqpoint{4.388206in}{1.255093in}}%
\pgfpathlineto{\pgfqpoint{4.286135in}{1.387500in}}%
\pgfpathlineto{\pgfqpoint{4.166788in}{1.536231in}}%
\pgfpathlineto{\pgfqpoint{4.096844in}{1.620184in}}%
\pgfpathlineto{\pgfqpoint{4.046545in}{1.680235in}}%
\pgfpathlineto{\pgfqpoint{3.827462in}{1.929268in}}%
\pgfpathlineto{\pgfqpoint{3.778007in}{1.984000in}}%
\pgfpathlineto{\pgfqpoint{3.697938in}{2.069956in}}%
\pgfpathlineto{\pgfqpoint{3.645737in}{2.125688in}}%
\pgfpathlineto{\pgfqpoint{3.577774in}{2.196638in}}%
\pgfpathlineto{\pgfqpoint{3.485414in}{2.290962in}}%
\pgfpathlineto{\pgfqpoint{3.419090in}{2.357333in}}%
\pgfpathlineto{\pgfqpoint{3.325091in}{2.449614in}}%
\pgfpathlineto{\pgfqpoint{3.265808in}{2.506667in}}%
\pgfpathlineto{\pgfqpoint{3.164768in}{2.602044in}}%
\pgfpathlineto{\pgfqpoint{3.106395in}{2.656000in}}%
\pgfpathlineto{\pgfqpoint{3.004444in}{2.748405in}}%
\pgfpathlineto{\pgfqpoint{2.884202in}{2.854281in}}%
\pgfpathlineto{\pgfqpoint{2.844121in}{2.888843in}}%
\pgfpathlineto{\pgfqpoint{2.721842in}{2.992000in}}%
\pgfpathlineto{\pgfqpoint{2.603636in}{3.088230in}}%
\pgfpathlineto{\pgfqpoint{2.563556in}{3.120146in}}%
\pgfpathlineto{\pgfqpoint{2.440269in}{3.216000in}}%
\pgfpathlineto{\pgfqpoint{2.363152in}{3.273912in}}%
\pgfpathlineto{\pgfqpoint{2.242909in}{3.361454in}}%
\pgfpathlineto{\pgfqpoint{2.122667in}{3.445091in}}%
\pgfpathlineto{\pgfqpoint{2.055607in}{3.489537in}}%
\pgfpathlineto{\pgfqpoint{2.017814in}{3.514667in}}%
\pgfpathlineto{\pgfqpoint{1.959762in}{3.552000in}}%
\pgfpathlineto{\pgfqpoint{1.761939in}{3.670766in}}%
\pgfpathlineto{\pgfqpoint{1.690874in}{3.709806in}}%
\pgfpathlineto{\pgfqpoint{1.681778in}{3.715042in}}%
\pgfpathlineto{\pgfqpoint{1.601616in}{3.756921in}}%
\pgfpathlineto{\pgfqpoint{1.588000in}{3.763317in}}%
\pgfpathlineto{\pgfqpoint{1.561535in}{3.777045in}}%
\pgfpathlineto{\pgfqpoint{1.400420in}{3.849929in}}%
\pgfpathlineto{\pgfqpoint{1.361131in}{3.865603in}}%
\pgfpathlineto{\pgfqpoint{1.315358in}{3.882698in}}%
\pgfpathlineto{\pgfqpoint{1.280970in}{3.895129in}}%
\pgfpathlineto{\pgfqpoint{1.160727in}{3.931280in}}%
\pgfpathlineto{\pgfqpoint{1.080566in}{3.948745in}}%
\pgfpathlineto{\pgfqpoint{1.067402in}{3.950405in}}%
\pgfpathlineto{\pgfqpoint{1.040485in}{3.955215in}}%
\pgfpathlineto{\pgfqpoint{1.000404in}{3.959861in}}%
\pgfpathlineto{\pgfqpoint{0.959999in}{3.962364in}}%
\pgfpathlineto{\pgfqpoint{0.919913in}{3.962360in}}%
\pgfpathlineto{\pgfqpoint{0.880162in}{3.959455in}}%
\pgfpathlineto{\pgfqpoint{0.840081in}{3.952956in}}%
\pgfpathlineto{\pgfqpoint{0.800000in}{3.942146in}}%
\pgfpathlineto{\pgfqpoint{0.800000in}{3.921785in}}%
\pgfpathlineto{\pgfqpoint{0.810293in}{3.925333in}}%
\pgfpathlineto{\pgfqpoint{0.840081in}{3.934194in}}%
\pgfpathlineto{\pgfqpoint{0.880162in}{3.941858in}}%
\pgfpathlineto{\pgfqpoint{0.960323in}{3.946728in}}%
\pgfpathlineto{\pgfqpoint{1.000404in}{3.945028in}}%
\pgfpathlineto{\pgfqpoint{1.040485in}{3.941120in}}%
\pgfpathlineto{\pgfqpoint{1.054705in}{3.938579in}}%
\pgfpathlineto{\pgfqpoint{1.080566in}{3.935318in}}%
\pgfpathlineto{\pgfqpoint{1.131794in}{3.925333in}}%
\pgfpathlineto{\pgfqpoint{1.200808in}{3.908139in}}%
\pgfpathlineto{\pgfqpoint{1.247447in}{3.894109in}}%
\pgfpathlineto{\pgfqpoint{1.280970in}{3.883622in}}%
\pgfpathlineto{\pgfqpoint{1.401212in}{3.838696in}}%
\pgfpathlineto{\pgfqpoint{1.460959in}{3.813333in}}%
\pgfpathlineto{\pgfqpoint{1.561535in}{3.766873in}}%
\pgfpathlineto{\pgfqpoint{1.618043in}{3.738667in}}%
\pgfpathlineto{\pgfqpoint{1.689423in}{3.701333in}}%
\pgfpathlineto{\pgfqpoint{1.761939in}{3.661417in}}%
\pgfpathlineto{\pgfqpoint{1.842101in}{3.614920in}}%
\pgfpathlineto{\pgfqpoint{1.922263in}{3.566360in}}%
\pgfpathlineto{\pgfqpoint{1.979071in}{3.530248in}}%
\pgfpathlineto{\pgfqpoint{2.004341in}{3.514667in}}%
\pgfpathlineto{\pgfqpoint{2.122667in}{3.436435in}}%
\pgfpathlineto{\pgfqpoint{2.189353in}{3.390115in}}%
\pgfpathlineto{\pgfqpoint{2.225234in}{3.365333in}}%
\pgfpathlineto{\pgfqpoint{2.329122in}{3.290667in}}%
\pgfpathlineto{\pgfqpoint{2.443313in}{3.205283in}}%
\pgfpathlineto{\pgfqpoint{2.483394in}{3.174617in}}%
\pgfpathlineto{\pgfqpoint{2.573540in}{3.104000in}}%
\pgfpathlineto{\pgfqpoint{2.643717in}{3.047717in}}%
\pgfpathlineto{\pgfqpoint{2.763960in}{2.948666in}}%
\pgfpathlineto{\pgfqpoint{2.804040in}{2.914917in}}%
\pgfpathlineto{\pgfqpoint{2.924283in}{2.811253in}}%
\pgfpathlineto{\pgfqpoint{2.973304in}{2.768000in}}%
\pgfpathlineto{\pgfqpoint{3.097485in}{2.656000in}}%
\pgfpathlineto{\pgfqpoint{3.164768in}{2.593843in}}%
\pgfpathlineto{\pgfqpoint{3.285010in}{2.480046in}}%
\pgfpathlineto{\pgfqpoint{3.410728in}{2.357333in}}%
\pgfpathlineto{\pgfqpoint{3.465834in}{2.301763in}}%
\pgfpathlineto{\pgfqpoint{3.522070in}{2.245333in}}%
\pgfpathlineto{\pgfqpoint{3.605657in}{2.159035in}}%
\pgfpathlineto{\pgfqpoint{3.725899in}{2.031667in}}%
\pgfpathlineto{\pgfqpoint{3.970010in}{1.760000in}}%
\pgfpathlineto{\pgfqpoint{4.021651in}{1.699479in}}%
\pgfpathlineto{\pgfqpoint{4.065727in}{1.648000in}}%
\pgfpathlineto{\pgfqpoint{4.166788in}{1.526295in}}%
\pgfpathlineto{\pgfqpoint{4.255666in}{1.415881in}}%
\pgfpathlineto{\pgfqpoint{4.336638in}{1.312000in}}%
\pgfpathlineto{\pgfqpoint{4.367192in}{1.272131in}}%
\pgfpathlineto{\pgfqpoint{4.452490in}{1.157882in}}%
\pgfpathlineto{\pgfqpoint{4.529725in}{1.050667in}}%
\pgfpathlineto{\pgfqpoint{4.581759in}{0.976000in}}%
\pgfpathlineto{\pgfqpoint{4.647758in}{0.878550in}}%
\pgfpathlineto{\pgfqpoint{4.768000in}{0.690495in}}%
\pgfpathlineto{\pgfqpoint{4.768000in}{0.690495in}}%
\pgfusepath{fill}%
\end{pgfscope}%
\begin{pgfscope}%
\pgfpathrectangle{\pgfqpoint{0.800000in}{0.528000in}}{\pgfqpoint{3.968000in}{3.696000in}}%
\pgfusepath{clip}%
\pgfsetbuttcap%
\pgfsetroundjoin%
\definecolor{currentfill}{rgb}{0.283197,0.115680,0.436115}%
\pgfsetfillcolor{currentfill}%
\pgfsetlinewidth{0.000000pt}%
\definecolor{currentstroke}{rgb}{0.000000,0.000000,0.000000}%
\pgfsetstrokecolor{currentstroke}%
\pgfsetdash{}{0pt}%
\pgfpathmoveto{\pgfqpoint{2.767582in}{0.528000in}}%
\pgfpathlineto{\pgfqpoint{2.683798in}{0.604197in}}%
\pgfpathlineto{\pgfqpoint{2.624645in}{0.659568in}}%
\pgfpathlineto{\pgfqpoint{2.563556in}{0.716479in}}%
\pgfpathlineto{\pgfqpoint{2.505441in}{0.772536in}}%
\pgfpathlineto{\pgfqpoint{2.449051in}{0.826667in}}%
\pgfpathlineto{\pgfqpoint{2.189932in}{1.088000in}}%
\pgfpathlineto{\pgfqpoint{2.139662in}{1.141163in}}%
\pgfpathlineto{\pgfqpoint{2.082586in}{1.201278in}}%
\pgfpathlineto{\pgfqpoint{2.027801in}{1.260971in}}%
\pgfpathlineto{\pgfqpoint{1.980796in}{1.312000in}}%
\pgfpathlineto{\pgfqpoint{1.947039in}{1.349333in}}%
\pgfpathlineto{\pgfqpoint{1.842101in}{1.467595in}}%
\pgfpathlineto{\pgfqpoint{1.773792in}{1.547040in}}%
\pgfpathlineto{\pgfqpoint{1.721859in}{1.607662in}}%
\pgfpathlineto{\pgfqpoint{1.641697in}{1.704092in}}%
\pgfpathlineto{\pgfqpoint{1.441293in}{1.956467in}}%
\pgfpathlineto{\pgfqpoint{1.361131in}{2.062737in}}%
\pgfpathlineto{\pgfqpoint{1.280970in}{2.172684in}}%
\pgfpathlineto{\pgfqpoint{1.229703in}{2.245333in}}%
\pgfpathlineto{\pgfqpoint{1.160727in}{2.345858in}}%
\pgfpathlineto{\pgfqpoint{1.103720in}{2.432000in}}%
\pgfpathlineto{\pgfqpoint{1.040485in}{2.531088in}}%
\pgfpathlineto{\pgfqpoint{0.943052in}{2.693333in}}%
\pgfpathlineto{\pgfqpoint{0.900892in}{2.768000in}}%
\pgfpathlineto{\pgfqpoint{0.867295in}{2.830682in}}%
\pgfpathlineto{\pgfqpoint{0.854116in}{2.855740in}}%
\pgfpathlineto{\pgfqpoint{0.840081in}{2.881708in}}%
\pgfpathlineto{\pgfqpoint{0.800000in}{2.962314in}}%
\pgfpathlineto{\pgfqpoint{0.800000in}{2.942626in}}%
\pgfpathlineto{\pgfqpoint{0.840081in}{2.863827in}}%
\pgfpathlineto{\pgfqpoint{0.920242in}{2.717387in}}%
\pgfpathlineto{\pgfqpoint{0.978165in}{2.618667in}}%
\pgfpathlineto{\pgfqpoint{1.030142in}{2.534366in}}%
\pgfpathlineto{\pgfqpoint{1.056752in}{2.491514in}}%
\pgfpathlineto{\pgfqpoint{1.120646in}{2.393142in}}%
\pgfpathlineto{\pgfqpoint{1.182201in}{2.302668in}}%
\pgfpathlineto{\pgfqpoint{1.221485in}{2.245333in}}%
\pgfpathlineto{\pgfqpoint{1.261237in}{2.189620in}}%
\pgfpathlineto{\pgfqpoint{1.301316in}{2.133333in}}%
\pgfpathlineto{\pgfqpoint{1.384072in}{2.021333in}}%
\pgfpathlineto{\pgfqpoint{1.469653in}{1.909333in}}%
\pgfpathlineto{\pgfqpoint{1.498836in}{1.872000in}}%
\pgfpathlineto{\pgfqpoint{1.588134in}{1.760000in}}%
\pgfpathlineto{\pgfqpoint{1.628745in}{1.710602in}}%
\pgfpathlineto{\pgfqpoint{1.680064in}{1.648000in}}%
\pgfpathlineto{\pgfqpoint{1.743086in}{1.573333in}}%
\pgfpathlineto{\pgfqpoint{1.842101in}{1.458428in}}%
\pgfpathlineto{\pgfqpoint{1.913292in}{1.378311in}}%
\pgfpathlineto{\pgfqpoint{1.962343in}{1.323443in}}%
\pgfpathlineto{\pgfqpoint{2.023481in}{1.256946in}}%
\pgfpathlineto{\pgfqpoint{2.075762in}{1.200000in}}%
\pgfpathlineto{\pgfqpoint{2.154060in}{1.117241in}}%
\pgfpathlineto{\pgfqpoint{2.202828in}{1.066071in}}%
\pgfpathlineto{\pgfqpoint{2.268040in}{0.999409in}}%
\pgfpathlineto{\pgfqpoint{2.323071in}{0.943155in}}%
\pgfpathlineto{\pgfqpoint{2.443313in}{0.823954in}}%
\pgfpathlineto{\pgfqpoint{2.563556in}{0.708342in}}%
\pgfpathlineto{\pgfqpoint{2.620258in}{0.655482in}}%
\pgfpathlineto{\pgfqpoint{2.676698in}{0.602667in}}%
\pgfpathlineto{\pgfqpoint{2.758597in}{0.528000in}}%
\pgfpathlineto{\pgfqpoint{2.763960in}{0.528000in}}%
\pgfpathmoveto{\pgfqpoint{4.768000in}{0.718707in}}%
\pgfpathlineto{\pgfqpoint{4.687838in}{0.843500in}}%
\pgfpathlineto{\pgfqpoint{4.624025in}{0.938667in}}%
\pgfpathlineto{\pgfqpoint{4.586044in}{0.993184in}}%
\pgfpathlineto{\pgfqpoint{4.546069in}{1.050667in}}%
\pgfpathlineto{\pgfqpoint{4.465127in}{1.162667in}}%
\pgfpathlineto{\pgfqpoint{4.381301in}{1.274667in}}%
\pgfpathlineto{\pgfqpoint{4.294691in}{1.386667in}}%
\pgfpathlineto{\pgfqpoint{4.246949in}{1.446848in}}%
\pgfpathlineto{\pgfqpoint{4.166788in}{1.545816in}}%
\pgfpathlineto{\pgfqpoint{4.101263in}{1.624300in}}%
\pgfpathlineto{\pgfqpoint{4.050180in}{1.685333in}}%
\pgfpathlineto{\pgfqpoint{3.977015in}{1.769903in}}%
\pgfpathlineto{\pgfqpoint{3.926303in}{1.828200in}}%
\pgfpathlineto{\pgfqpoint{3.682004in}{2.096000in}}%
\pgfpathlineto{\pgfqpoint{3.645737in}{2.134339in}}%
\pgfpathlineto{\pgfqpoint{3.525495in}{2.258664in}}%
\pgfpathlineto{\pgfqpoint{3.464816in}{2.320000in}}%
\pgfpathlineto{\pgfqpoint{3.351672in}{2.432000in}}%
\pgfpathlineto{\pgfqpoint{3.235249in}{2.544000in}}%
\pgfpathlineto{\pgfqpoint{3.115305in}{2.656000in}}%
\pgfpathlineto{\pgfqpoint{2.991575in}{2.768000in}}%
\pgfpathlineto{\pgfqpoint{2.906840in}{2.842667in}}%
\pgfpathlineto{\pgfqpoint{2.804040in}{2.931097in}}%
\pgfpathlineto{\pgfqpoint{2.763960in}{2.964922in}}%
\pgfpathlineto{\pgfqpoint{2.640648in}{3.066667in}}%
\pgfpathlineto{\pgfqpoint{2.523475in}{3.159893in}}%
\pgfpathlineto{\pgfqpoint{2.401954in}{3.253333in}}%
\pgfpathlineto{\pgfqpoint{2.313180in}{3.318787in}}%
\pgfpathlineto{\pgfqpoint{2.268038in}{3.351406in}}%
\pgfpathlineto{\pgfqpoint{2.222198in}{3.384625in}}%
\pgfpathlineto{\pgfqpoint{2.122667in}{3.453631in}}%
\pgfpathlineto{\pgfqpoint{2.061029in}{3.494588in}}%
\pgfpathlineto{\pgfqpoint{2.031288in}{3.514667in}}%
\pgfpathlineto{\pgfqpoint{1.922263in}{3.584413in}}%
\pgfpathlineto{\pgfqpoint{1.882182in}{3.609031in}}%
\pgfpathlineto{\pgfqpoint{1.789801in}{3.664000in}}%
\pgfpathlineto{\pgfqpoint{1.721859in}{3.702732in}}%
\pgfpathlineto{\pgfqpoint{1.561535in}{3.786878in}}%
\pgfpathlineto{\pgfqpoint{1.506488in}{3.813333in}}%
\pgfpathlineto{\pgfqpoint{1.441293in}{3.843032in}}%
\pgfpathlineto{\pgfqpoint{1.361131in}{3.876518in}}%
\pgfpathlineto{\pgfqpoint{1.331547in}{3.888000in}}%
\pgfpathlineto{\pgfqpoint{1.321051in}{3.892063in}}%
\pgfpathlineto{\pgfqpoint{1.189636in}{3.935739in}}%
\pgfpathlineto{\pgfqpoint{1.120646in}{3.953519in}}%
\pgfpathlineto{\pgfqpoint{1.077667in}{3.962667in}}%
\pgfpathlineto{\pgfqpoint{1.040485in}{3.968975in}}%
\pgfpathlineto{\pgfqpoint{0.960323in}{3.977175in}}%
\pgfpathlineto{\pgfqpoint{0.880162in}{3.976155in}}%
\pgfpathlineto{\pgfqpoint{0.832356in}{3.969862in}}%
\pgfpathlineto{\pgfqpoint{0.800000in}{3.962238in}}%
\pgfpathlineto{\pgfqpoint{0.800000in}{3.942146in}}%
\pgfpathlineto{\pgfqpoint{0.848525in}{3.954801in}}%
\pgfpathlineto{\pgfqpoint{0.880162in}{3.959455in}}%
\pgfpathlineto{\pgfqpoint{0.920242in}{3.962401in}}%
\pgfpathlineto{\pgfqpoint{0.960635in}{3.962376in}}%
\pgfpathlineto{\pgfqpoint{1.003687in}{3.959609in}}%
\pgfpathlineto{\pgfqpoint{1.040485in}{3.955215in}}%
\pgfpathlineto{\pgfqpoint{1.120646in}{3.940698in}}%
\pgfpathlineto{\pgfqpoint{1.133431in}{3.937241in}}%
\pgfpathlineto{\pgfqpoint{1.182750in}{3.925333in}}%
\pgfpathlineto{\pgfqpoint{1.240889in}{3.908253in}}%
\pgfpathlineto{\pgfqpoint{1.256586in}{3.902621in}}%
\pgfpathlineto{\pgfqpoint{1.301058in}{3.888000in}}%
\pgfpathlineto{\pgfqpoint{1.398643in}{3.850667in}}%
\pgfpathlineto{\pgfqpoint{1.441293in}{3.832494in}}%
\pgfpathlineto{\pgfqpoint{1.454935in}{3.826040in}}%
\pgfpathlineto{\pgfqpoint{1.484661in}{3.813333in}}%
\pgfpathlineto{\pgfqpoint{1.563621in}{3.776000in}}%
\pgfpathlineto{\pgfqpoint{1.721859in}{3.693211in}}%
\pgfpathlineto{\pgfqpoint{1.790971in}{3.653708in}}%
\pgfpathlineto{\pgfqpoint{1.802020in}{3.647676in}}%
\pgfpathlineto{\pgfqpoint{1.899596in}{3.589333in}}%
\pgfpathlineto{\pgfqpoint{2.082586in}{3.472093in}}%
\pgfpathlineto{\pgfqpoint{2.148641in}{3.426861in}}%
\pgfpathlineto{\pgfqpoint{2.184191in}{3.402667in}}%
\pgfpathlineto{\pgfqpoint{2.390728in}{3.253333in}}%
\pgfpathlineto{\pgfqpoint{2.488857in}{3.178667in}}%
\pgfpathlineto{\pgfqpoint{2.603636in}{3.088230in}}%
\pgfpathlineto{\pgfqpoint{2.723879in}{2.990319in}}%
\pgfpathlineto{\pgfqpoint{2.844121in}{2.888843in}}%
\pgfpathlineto{\pgfqpoint{2.964364in}{2.784064in}}%
\pgfpathlineto{\pgfqpoint{3.024198in}{2.730667in}}%
\pgfpathlineto{\pgfqpoint{3.124687in}{2.639198in}}%
\pgfpathlineto{\pgfqpoint{3.186905in}{2.581333in}}%
\pgfpathlineto{\pgfqpoint{3.285010in}{2.488299in}}%
\pgfpathlineto{\pgfqpoint{3.405253in}{2.371075in}}%
\pgfpathlineto{\pgfqpoint{3.530361in}{2.245333in}}%
\pgfpathlineto{\pgfqpoint{3.577774in}{2.196638in}}%
\pgfpathlineto{\pgfqpoint{3.685818in}{2.083251in}}%
\pgfpathlineto{\pgfqpoint{3.743564in}{2.021333in}}%
\pgfpathlineto{\pgfqpoint{3.846141in}{1.909218in}}%
\pgfpathlineto{\pgfqpoint{3.918636in}{1.827525in}}%
\pgfpathlineto{\pgfqpoint{3.966384in}{1.773319in}}%
\pgfpathlineto{\pgfqpoint{4.167584in}{1.535258in}}%
\pgfpathlineto{\pgfqpoint{4.287030in}{1.386361in}}%
\pgfpathlineto{\pgfqpoint{4.353865in}{1.299587in}}%
\pgfpathlineto{\pgfqpoint{4.388206in}{1.255093in}}%
\pgfpathlineto{\pgfqpoint{4.478095in}{1.134032in}}%
\pgfpathlineto{\pgfqpoint{4.537897in}{1.050667in}}%
\pgfpathlineto{\pgfqpoint{4.581058in}{0.988539in}}%
\pgfpathlineto{\pgfqpoint{4.615755in}{0.938667in}}%
\pgfpathlineto{\pgfqpoint{4.658749in}{0.874238in}}%
\pgfpathlineto{\pgfqpoint{4.690516in}{0.826667in}}%
\pgfpathlineto{\pgfqpoint{4.738387in}{0.752000in}}%
\pgfpathlineto{\pgfqpoint{4.768000in}{0.704709in}}%
\pgfpathlineto{\pgfqpoint{4.768000in}{0.714667in}}%
\pgfpathlineto{\pgfqpoint{4.768000in}{0.714667in}}%
\pgfusepath{fill}%
\end{pgfscope}%
\begin{pgfscope}%
\pgfpathrectangle{\pgfqpoint{0.800000in}{0.528000in}}{\pgfqpoint{3.968000in}{3.696000in}}%
\pgfusepath{clip}%
\pgfsetbuttcap%
\pgfsetroundjoin%
\definecolor{currentfill}{rgb}{0.283197,0.115680,0.436115}%
\pgfsetfillcolor{currentfill}%
\pgfsetlinewidth{0.000000pt}%
\definecolor{currentstroke}{rgb}{0.000000,0.000000,0.000000}%
\pgfsetstrokecolor{currentstroke}%
\pgfsetdash{}{0pt}%
\pgfpathmoveto{\pgfqpoint{2.758597in}{0.528000in}}%
\pgfpathlineto{\pgfqpoint{2.717437in}{0.565333in}}%
\pgfpathlineto{\pgfqpoint{2.596439in}{0.677333in}}%
\pgfpathlineto{\pgfqpoint{2.540615in}{0.730632in}}%
\pgfpathlineto{\pgfqpoint{2.483394in}{0.785037in}}%
\pgfpathlineto{\pgfqpoint{2.363152in}{0.902993in}}%
\pgfpathlineto{\pgfqpoint{2.242909in}{1.024691in}}%
\pgfpathlineto{\pgfqpoint{2.122667in}{1.150065in}}%
\pgfpathlineto{\pgfqpoint{2.060571in}{1.216827in}}%
\pgfpathlineto{\pgfqpoint{2.006703in}{1.274667in}}%
\pgfpathlineto{\pgfqpoint{1.939028in}{1.349333in}}%
\pgfpathlineto{\pgfqpoint{1.839558in}{1.461333in}}%
\pgfpathlineto{\pgfqpoint{1.802020in}{1.504521in}}%
\pgfpathlineto{\pgfqpoint{1.680064in}{1.648000in}}%
\pgfpathlineto{\pgfqpoint{1.618528in}{1.722667in}}%
\pgfpathlineto{\pgfqpoint{1.521455in}{1.843224in}}%
\pgfpathlineto{\pgfqpoint{1.439121in}{1.948690in}}%
\pgfpathlineto{\pgfqpoint{1.344687in}{2.073983in}}%
\pgfpathlineto{\pgfqpoint{1.274299in}{2.170667in}}%
\pgfpathlineto{\pgfqpoint{1.229335in}{2.234572in}}%
\pgfpathlineto{\pgfqpoint{1.195450in}{2.282667in}}%
\pgfpathlineto{\pgfqpoint{1.135689in}{2.371345in}}%
\pgfpathlineto{\pgfqpoint{1.117984in}{2.397146in}}%
\pgfpathlineto{\pgfqpoint{1.040485in}{2.517325in}}%
\pgfpathlineto{\pgfqpoint{0.920242in}{2.717387in}}%
\pgfpathlineto{\pgfqpoint{0.871224in}{2.805333in}}%
\pgfpathlineto{\pgfqpoint{0.840081in}{2.863827in}}%
\pgfpathlineto{\pgfqpoint{0.800000in}{2.942626in}}%
\pgfpathlineto{\pgfqpoint{0.800000in}{2.923490in}}%
\pgfpathlineto{\pgfqpoint{0.822374in}{2.880000in}}%
\pgfpathlineto{\pgfqpoint{0.862154in}{2.805333in}}%
\pgfpathlineto{\pgfqpoint{0.903776in}{2.730667in}}%
\pgfpathlineto{\pgfqpoint{0.960323in}{2.633806in}}%
\pgfpathlineto{\pgfqpoint{1.062606in}{2.469333in}}%
\pgfpathlineto{\pgfqpoint{1.086753in}{2.432000in}}%
\pgfpathlineto{\pgfqpoint{1.136367in}{2.357333in}}%
\pgfpathlineto{\pgfqpoint{1.200808in}{2.263171in}}%
\pgfpathlineto{\pgfqpoint{1.280970in}{2.150195in}}%
\pgfpathlineto{\pgfqpoint{1.461718in}{1.909333in}}%
\pgfpathlineto{\pgfqpoint{1.490843in}{1.872000in}}%
\pgfpathlineto{\pgfqpoint{1.580231in}{1.760000in}}%
\pgfpathlineto{\pgfqpoint{1.624295in}{1.706458in}}%
\pgfpathlineto{\pgfqpoint{1.672251in}{1.648000in}}%
\pgfpathlineto{\pgfqpoint{1.729394in}{1.580352in}}%
\pgfpathlineto{\pgfqpoint{1.766977in}{1.536000in}}%
\pgfpathlineto{\pgfqpoint{1.864556in}{1.424000in}}%
\pgfpathlineto{\pgfqpoint{1.964655in}{1.312000in}}%
\pgfpathlineto{\pgfqpoint{2.067785in}{1.200000in}}%
\pgfpathlineto{\pgfqpoint{2.149826in}{1.113298in}}%
\pgfpathlineto{\pgfqpoint{2.202828in}{1.057574in}}%
\pgfpathlineto{\pgfqpoint{2.263769in}{0.995430in}}%
\pgfpathlineto{\pgfqpoint{2.321121in}{0.936851in}}%
\pgfpathlineto{\pgfqpoint{2.363152in}{0.894777in}}%
\pgfpathlineto{\pgfqpoint{2.483394in}{0.776922in}}%
\pgfpathlineto{\pgfqpoint{2.548397in}{0.714667in}}%
\pgfpathlineto{\pgfqpoint{2.643717in}{0.625103in}}%
\pgfpathlineto{\pgfqpoint{2.708659in}{0.565333in}}%
\pgfpathlineto{\pgfqpoint{2.749747in}{0.528000in}}%
\pgfpathmoveto{\pgfqpoint{4.768000in}{0.732197in}}%
\pgfpathlineto{\pgfqpoint{4.707255in}{0.826667in}}%
\pgfpathlineto{\pgfqpoint{4.668999in}{0.883786in}}%
\pgfpathlineto{\pgfqpoint{4.632295in}{0.938667in}}%
\pgfpathlineto{\pgfqpoint{4.591031in}{0.997828in}}%
\pgfpathlineto{\pgfqpoint{4.554241in}{1.050667in}}%
\pgfpathlineto{\pgfqpoint{4.487434in}{1.143304in}}%
\pgfpathlineto{\pgfqpoint{4.407273in}{1.251018in}}%
\pgfpathlineto{\pgfqpoint{4.327111in}{1.355450in}}%
\pgfpathlineto{\pgfqpoint{4.261734in}{1.437771in}}%
\pgfpathlineto{\pgfqpoint{4.213193in}{1.498667in}}%
\pgfpathlineto{\pgfqpoint{4.121020in}{1.610667in}}%
\pgfpathlineto{\pgfqpoint{4.086626in}{1.651671in}}%
\pgfpathlineto{\pgfqpoint{3.961373in}{1.797333in}}%
\pgfpathlineto{\pgfqpoint{3.861859in}{1.909333in}}%
\pgfpathlineto{\pgfqpoint{3.759735in}{2.021333in}}%
\pgfpathlineto{\pgfqpoint{3.706520in}{2.077949in}}%
\pgfpathlineto{\pgfqpoint{3.654686in}{2.133333in}}%
\pgfpathlineto{\pgfqpoint{3.565576in}{2.226001in}}%
\pgfpathlineto{\pgfqpoint{3.510075in}{2.282667in}}%
\pgfpathlineto{\pgfqpoint{3.398166in}{2.394667in}}%
\pgfpathlineto{\pgfqpoint{3.360164in}{2.432000in}}%
\pgfpathlineto{\pgfqpoint{3.243945in}{2.544000in}}%
\pgfpathlineto{\pgfqpoint{3.124216in}{2.656000in}}%
\pgfpathlineto{\pgfqpoint{3.000710in}{2.768000in}}%
\pgfpathlineto{\pgfqpoint{2.940732in}{2.820655in}}%
\pgfpathlineto{\pgfqpoint{2.884202in}{2.870452in}}%
\pgfpathlineto{\pgfqpoint{2.816095in}{2.928562in}}%
\pgfpathlineto{\pgfqpoint{2.763960in}{2.972959in}}%
\pgfpathlineto{\pgfqpoint{2.643717in}{3.072255in}}%
\pgfpathlineto{\pgfqpoint{2.557489in}{3.141333in}}%
\pgfpathlineto{\pgfqpoint{2.473391in}{3.206683in}}%
\pgfpathlineto{\pgfqpoint{2.429345in}{3.240323in}}%
\pgfpathlineto{\pgfqpoint{2.385045in}{3.273726in}}%
\pgfpathlineto{\pgfqpoint{2.363151in}{3.290667in}}%
\pgfpathlineto{\pgfqpoint{2.242909in}{3.378226in}}%
\pgfpathlineto{\pgfqpoint{2.155009in}{3.440000in}}%
\pgfpathlineto{\pgfqpoint{2.037276in}{3.519537in}}%
\pgfpathlineto{\pgfqpoint{1.922263in}{3.593303in}}%
\pgfpathlineto{\pgfqpoint{1.842101in}{3.642276in}}%
\pgfpathlineto{\pgfqpoint{1.753729in}{3.693686in}}%
\pgfpathlineto{\pgfqpoint{1.721859in}{3.711969in}}%
\pgfpathlineto{\pgfqpoint{1.673321in}{3.738667in}}%
\pgfpathlineto{\pgfqpoint{1.600145in}{3.777370in}}%
\pgfpathlineto{\pgfqpoint{1.514601in}{3.819717in}}%
\pgfpathlineto{\pgfqpoint{1.441293in}{3.853459in}}%
\pgfpathlineto{\pgfqpoint{1.321051in}{3.902948in}}%
\pgfpathlineto{\pgfqpoint{1.274864in}{3.919646in}}%
\pgfpathlineto{\pgfqpoint{1.240889in}{3.931559in}}%
\pgfpathlineto{\pgfqpoint{1.120646in}{3.966169in}}%
\pgfpathlineto{\pgfqpoint{1.040485in}{3.982358in}}%
\pgfpathlineto{\pgfqpoint{1.023599in}{3.984272in}}%
\pgfpathlineto{\pgfqpoint{1.000404in}{3.988103in}}%
\pgfpathlineto{\pgfqpoint{0.960323in}{3.991954in}}%
\pgfpathlineto{\pgfqpoint{0.951693in}{3.991961in}}%
\pgfpathlineto{\pgfqpoint{0.912858in}{3.993122in}}%
\pgfpathlineto{\pgfqpoint{0.880162in}{3.992655in}}%
\pgfpathlineto{\pgfqpoint{0.870905in}{3.991378in}}%
\pgfpathlineto{\pgfqpoint{0.840081in}{3.988638in}}%
\pgfpathlineto{\pgfqpoint{0.816342in}{3.984778in}}%
\pgfpathlineto{\pgfqpoint{0.800000in}{3.980942in}}%
\pgfpathlineto{\pgfqpoint{0.800000in}{3.962238in}}%
\pgfpathlineto{\pgfqpoint{0.832356in}{3.969862in}}%
\pgfpathlineto{\pgfqpoint{0.850112in}{3.972010in}}%
\pgfpathlineto{\pgfqpoint{0.880162in}{3.976155in}}%
\pgfpathlineto{\pgfqpoint{0.894868in}{3.976365in}}%
\pgfpathlineto{\pgfqpoint{0.920242in}{3.978008in}}%
\pgfpathlineto{\pgfqpoint{0.935840in}{3.977195in}}%
\pgfpathlineto{\pgfqpoint{0.960323in}{3.977175in}}%
\pgfpathlineto{\pgfqpoint{1.000404in}{3.974056in}}%
\pgfpathlineto{\pgfqpoint{1.010791in}{3.972341in}}%
\pgfpathlineto{\pgfqpoint{1.040485in}{3.968975in}}%
\pgfpathlineto{\pgfqpoint{1.081225in}{3.962053in}}%
\pgfpathlineto{\pgfqpoint{1.120646in}{3.953519in}}%
\pgfpathlineto{\pgfqpoint{1.240889in}{3.920045in}}%
\pgfpathlineto{\pgfqpoint{1.361131in}{3.876518in}}%
\pgfpathlineto{\pgfqpoint{1.423487in}{3.850667in}}%
\pgfpathlineto{\pgfqpoint{1.481374in}{3.825024in}}%
\pgfpathlineto{\pgfqpoint{1.561535in}{3.786878in}}%
\pgfpathlineto{\pgfqpoint{1.655486in}{3.738667in}}%
\pgfpathlineto{\pgfqpoint{1.724333in}{3.701333in}}%
\pgfpathlineto{\pgfqpoint{1.922263in}{3.584413in}}%
\pgfpathlineto{\pgfqpoint{1.990418in}{3.540817in}}%
\pgfpathlineto{\pgfqpoint{2.031288in}{3.514667in}}%
\pgfpathlineto{\pgfqpoint{2.087689in}{3.477333in}}%
\pgfpathlineto{\pgfqpoint{2.162747in}{3.426126in}}%
\pgfpathlineto{\pgfqpoint{2.249265in}{3.365333in}}%
\pgfpathlineto{\pgfqpoint{2.323071in}{3.311846in}}%
\pgfpathlineto{\pgfqpoint{2.443313in}{3.221919in}}%
\pgfpathlineto{\pgfqpoint{2.563556in}{3.128350in}}%
\pgfpathlineto{\pgfqpoint{2.603636in}{3.096450in}}%
\pgfpathlineto{\pgfqpoint{2.723879in}{2.998391in}}%
\pgfpathlineto{\pgfqpoint{2.790567in}{2.942117in}}%
\pgfpathlineto{\pgfqpoint{2.844121in}{2.896912in}}%
\pgfpathlineto{\pgfqpoint{2.964364in}{2.792182in}}%
\pgfpathlineto{\pgfqpoint{3.018383in}{2.743650in}}%
\pgfpathlineto{\pgfqpoint{3.074497in}{2.693333in}}%
\pgfpathlineto{\pgfqpoint{3.164768in}{2.610246in}}%
\pgfpathlineto{\pgfqpoint{3.285010in}{2.496551in}}%
\pgfpathlineto{\pgfqpoint{3.405253in}{2.379379in}}%
\pgfpathlineto{\pgfqpoint{3.525495in}{2.258664in}}%
\pgfpathlineto{\pgfqpoint{3.646694in}{2.133333in}}%
\pgfpathlineto{\pgfqpoint{3.702229in}{2.073952in}}%
\pgfpathlineto{\pgfqpoint{3.751650in}{2.021333in}}%
\pgfpathlineto{\pgfqpoint{3.853950in}{1.909333in}}%
\pgfpathlineto{\pgfqpoint{3.893690in}{1.865044in}}%
\pgfpathlineto{\pgfqpoint{4.006465in}{1.736263in}}%
\pgfpathlineto{\pgfqpoint{4.206869in}{1.496778in}}%
\pgfpathlineto{\pgfqpoint{4.294691in}{1.386667in}}%
\pgfpathlineto{\pgfqpoint{4.327111in}{1.345250in}}%
\pgfpathlineto{\pgfqpoint{4.415338in}{1.229821in}}%
\pgfpathlineto{\pgfqpoint{4.492527in}{1.125333in}}%
\pgfpathlineto{\pgfqpoint{4.554416in}{1.038390in}}%
\pgfpathlineto{\pgfqpoint{4.586044in}{0.993184in}}%
\pgfpathlineto{\pgfqpoint{4.624025in}{0.938667in}}%
\pgfpathlineto{\pgfqpoint{4.663874in}{0.879012in}}%
\pgfpathlineto{\pgfqpoint{4.698886in}{0.826667in}}%
\pgfpathlineto{\pgfqpoint{4.746926in}{0.752000in}}%
\pgfpathlineto{\pgfqpoint{4.768000in}{0.718707in}}%
\pgfpathlineto{\pgfqpoint{4.768000in}{0.718707in}}%
\pgfusepath{fill}%
\end{pgfscope}%
\begin{pgfscope}%
\pgfpathrectangle{\pgfqpoint{0.800000in}{0.528000in}}{\pgfqpoint{3.968000in}{3.696000in}}%
\pgfusepath{clip}%
\pgfsetbuttcap%
\pgfsetroundjoin%
\definecolor{currentfill}{rgb}{0.283197,0.115680,0.436115}%
\pgfsetfillcolor{currentfill}%
\pgfsetlinewidth{0.000000pt}%
\definecolor{currentstroke}{rgb}{0.000000,0.000000,0.000000}%
\pgfsetstrokecolor{currentstroke}%
\pgfsetdash{}{0pt}%
\pgfpathmoveto{\pgfqpoint{2.749747in}{0.528000in}}%
\pgfpathlineto{\pgfqpoint{2.683798in}{0.588083in}}%
\pgfpathlineto{\pgfqpoint{2.563556in}{0.700260in}}%
\pgfpathlineto{\pgfqpoint{2.443313in}{0.815823in}}%
\pgfpathlineto{\pgfqpoint{2.394233in}{0.864000in}}%
\pgfpathlineto{\pgfqpoint{2.282285in}{0.976000in}}%
\pgfpathlineto{\pgfqpoint{2.242909in}{1.016213in}}%
\pgfpathlineto{\pgfqpoint{2.122667in}{1.141533in}}%
\pgfpathlineto{\pgfqpoint{2.056238in}{1.212791in}}%
\pgfpathlineto{\pgfqpoint{2.002424in}{1.270598in}}%
\pgfpathlineto{\pgfqpoint{1.922263in}{1.359077in}}%
\pgfpathlineto{\pgfqpoint{1.854610in}{1.435651in}}%
\pgfpathlineto{\pgfqpoint{1.802020in}{1.495347in}}%
\pgfpathlineto{\pgfqpoint{1.721859in}{1.588994in}}%
\pgfpathlineto{\pgfqpoint{1.641150in}{1.685333in}}%
\pgfpathlineto{\pgfqpoint{1.580231in}{1.760000in}}%
\pgfpathlineto{\pgfqpoint{1.490843in}{1.872000in}}%
\pgfpathlineto{\pgfqpoint{1.401212in}{1.987909in}}%
\pgfpathlineto{\pgfqpoint{1.318842in}{2.098057in}}%
\pgfpathlineto{\pgfqpoint{1.239458in}{2.208000in}}%
\pgfpathlineto{\pgfqpoint{1.177157in}{2.297970in}}%
\pgfpathlineto{\pgfqpoint{1.160727in}{2.321155in}}%
\pgfpathlineto{\pgfqpoint{1.080566in}{2.441511in}}%
\pgfpathlineto{\pgfqpoint{0.947126in}{2.656000in}}%
\pgfpathlineto{\pgfqpoint{0.920242in}{2.701847in}}%
\pgfpathlineto{\pgfqpoint{0.862154in}{2.805333in}}%
\pgfpathlineto{\pgfqpoint{0.840081in}{2.846062in}}%
\pgfpathlineto{\pgfqpoint{0.800000in}{2.923490in}}%
\pgfpathlineto{\pgfqpoint{0.800000in}{2.905230in}}%
\pgfpathlineto{\pgfqpoint{0.880162in}{2.756555in}}%
\pgfpathlineto{\pgfqpoint{0.989905in}{2.571554in}}%
\pgfpathlineto{\pgfqpoint{1.019470in}{2.524425in}}%
\pgfpathlineto{\pgfqpoint{1.054188in}{2.469333in}}%
\pgfpathlineto{\pgfqpoint{1.080566in}{2.428636in}}%
\pgfpathlineto{\pgfqpoint{1.160727in}{2.309303in}}%
\pgfpathlineto{\pgfqpoint{1.321051in}{2.084341in}}%
\pgfpathlineto{\pgfqpoint{1.401212in}{1.977510in}}%
\pgfpathlineto{\pgfqpoint{1.486882in}{1.866869in}}%
\pgfpathlineto{\pgfqpoint{1.602608in}{1.722667in}}%
\pgfpathlineto{\pgfqpoint{1.672164in}{1.639045in}}%
\pgfpathlineto{\pgfqpoint{1.721859in}{1.579661in}}%
\pgfpathlineto{\pgfqpoint{1.926880in}{1.345032in}}%
\pgfpathlineto{\pgfqpoint{2.042505in}{1.218587in}}%
\pgfpathlineto{\pgfqpoint{2.107989in}{1.148996in}}%
\pgfpathlineto{\pgfqpoint{2.165461in}{1.088000in}}%
\pgfpathlineto{\pgfqpoint{2.202828in}{1.049128in}}%
\pgfpathlineto{\pgfqpoint{2.323071in}{0.926653in}}%
\pgfpathlineto{\pgfqpoint{2.443313in}{0.807691in}}%
\pgfpathlineto{\pgfqpoint{2.563556in}{0.692178in}}%
\pgfpathlineto{\pgfqpoint{2.631574in}{0.628689in}}%
\pgfpathlineto{\pgfqpoint{2.683798in}{0.580050in}}%
\pgfpathlineto{\pgfqpoint{2.740896in}{0.528000in}}%
\pgfpathmoveto{\pgfqpoint{4.768000in}{0.745686in}}%
\pgfpathlineto{\pgfqpoint{4.715625in}{0.826667in}}%
\pgfpathlineto{\pgfqpoint{4.674124in}{0.888559in}}%
\pgfpathlineto{\pgfqpoint{4.640564in}{0.938667in}}%
\pgfpathlineto{\pgfqpoint{4.596017in}{1.002473in}}%
\pgfpathlineto{\pgfqpoint{4.562413in}{1.050667in}}%
\pgfpathlineto{\pgfqpoint{4.487434in}{1.154293in}}%
\pgfpathlineto{\pgfqpoint{4.407273in}{1.261514in}}%
\pgfpathlineto{\pgfqpoint{4.327111in}{1.365495in}}%
\pgfpathlineto{\pgfqpoint{4.126707in}{1.613315in}}%
\pgfpathlineto{\pgfqpoint{4.057147in}{1.695208in}}%
\pgfpathlineto{\pgfqpoint{4.006465in}{1.754592in}}%
\pgfpathlineto{\pgfqpoint{3.765980in}{2.023260in}}%
\pgfpathlineto{\pgfqpoint{3.697999in}{2.096000in}}%
\pgfpathlineto{\pgfqpoint{3.591125in}{2.208000in}}%
\pgfpathlineto{\pgfqpoint{3.521695in}{2.279127in}}%
\pgfpathlineto{\pgfqpoint{3.481411in}{2.320000in}}%
\pgfpathlineto{\pgfqpoint{3.365172in}{2.435297in}}%
\pgfpathlineto{\pgfqpoint{3.244929in}{2.551083in}}%
\pgfpathlineto{\pgfqpoint{3.124687in}{2.663520in}}%
\pgfpathlineto{\pgfqpoint{3.004444in}{2.772666in}}%
\pgfpathlineto{\pgfqpoint{2.945141in}{2.824762in}}%
\pgfpathlineto{\pgfqpoint{2.883375in}{2.879229in}}%
\pgfpathlineto{\pgfqpoint{2.839104in}{2.917333in}}%
\pgfpathlineto{\pgfqpoint{2.763960in}{2.980996in}}%
\pgfpathlineto{\pgfqpoint{2.643717in}{3.080244in}}%
\pgfpathlineto{\pgfqpoint{2.520393in}{3.178667in}}%
\pgfpathlineto{\pgfqpoint{2.403232in}{3.268680in}}%
\pgfpathlineto{\pgfqpoint{2.363152in}{3.298787in}}%
\pgfpathlineto{\pgfqpoint{2.272436in}{3.365333in}}%
\pgfpathlineto{\pgfqpoint{2.042505in}{3.524595in}}%
\pgfpathlineto{\pgfqpoint{1.942363in}{3.589333in}}%
\pgfpathlineto{\pgfqpoint{1.882182in}{3.627030in}}%
\pgfpathlineto{\pgfqpoint{1.681778in}{3.743512in}}%
\pgfpathlineto{\pgfqpoint{1.620916in}{3.776000in}}%
\pgfpathlineto{\pgfqpoint{1.492796in}{3.840027in}}%
\pgfpathlineto{\pgfqpoint{1.441293in}{3.863594in}}%
\pgfpathlineto{\pgfqpoint{1.368319in}{3.894695in}}%
\pgfpathlineto{\pgfqpoint{1.342174in}{3.905657in}}%
\pgfpathlineto{\pgfqpoint{1.274598in}{3.931268in}}%
\pgfpathlineto{\pgfqpoint{1.200808in}{3.955929in}}%
\pgfpathlineto{\pgfqpoint{1.152720in}{3.970125in}}%
\pgfpathlineto{\pgfqpoint{1.080566in}{3.987756in}}%
\pgfpathlineto{\pgfqpoint{1.036408in}{3.996203in}}%
\pgfpathlineto{\pgfqpoint{1.000404in}{4.002040in}}%
\pgfpathlineto{\pgfqpoint{0.952720in}{4.007082in}}%
\pgfpathlineto{\pgfqpoint{0.920242in}{4.008680in}}%
\pgfpathlineto{\pgfqpoint{0.871178in}{4.008368in}}%
\pgfpathlineto{\pgfqpoint{0.840081in}{4.005775in}}%
\pgfpathlineto{\pgfqpoint{0.800000in}{3.999617in}}%
\pgfpathlineto{\pgfqpoint{0.800000in}{3.980942in}}%
\pgfpathlineto{\pgfqpoint{0.816342in}{3.984778in}}%
\pgfpathlineto{\pgfqpoint{0.840081in}{3.988638in}}%
\pgfpathlineto{\pgfqpoint{0.880162in}{3.992655in}}%
\pgfpathlineto{\pgfqpoint{0.920242in}{3.993600in}}%
\pgfpathlineto{\pgfqpoint{0.969588in}{3.991370in}}%
\pgfpathlineto{\pgfqpoint{1.000404in}{3.988103in}}%
\pgfpathlineto{\pgfqpoint{1.090916in}{3.972308in}}%
\pgfpathlineto{\pgfqpoint{1.134227in}{3.962667in}}%
\pgfpathlineto{\pgfqpoint{1.200808in}{3.944173in}}%
\pgfpathlineto{\pgfqpoint{1.215545in}{3.939060in}}%
\pgfpathlineto{\pgfqpoint{1.259012in}{3.925333in}}%
\pgfpathlineto{\pgfqpoint{1.321051in}{3.902948in}}%
\pgfpathlineto{\pgfqpoint{1.362195in}{3.887010in}}%
\pgfpathlineto{\pgfqpoint{1.447431in}{3.850667in}}%
\pgfpathlineto{\pgfqpoint{1.602784in}{3.776000in}}%
\pgfpathlineto{\pgfqpoint{1.681778in}{3.734139in}}%
\pgfpathlineto{\pgfqpoint{1.753729in}{3.693686in}}%
\pgfpathlineto{\pgfqpoint{1.761939in}{3.689283in}}%
\pgfpathlineto{\pgfqpoint{1.842101in}{3.642276in}}%
\pgfpathlineto{\pgfqpoint{1.900406in}{3.606309in}}%
\pgfpathlineto{\pgfqpoint{1.928547in}{3.589333in}}%
\pgfpathlineto{\pgfqpoint{2.122667in}{3.462171in}}%
\pgfpathlineto{\pgfqpoint{2.208530in}{3.402667in}}%
\pgfpathlineto{\pgfqpoint{2.323071in}{3.320206in}}%
\pgfpathlineto{\pgfqpoint{2.412731in}{3.253333in}}%
\pgfpathlineto{\pgfqpoint{2.483394in}{3.199252in}}%
\pgfpathlineto{\pgfqpoint{2.604442in}{3.104000in}}%
\pgfpathlineto{\pgfqpoint{2.723879in}{3.006412in}}%
\pgfpathlineto{\pgfqpoint{2.795171in}{2.946405in}}%
\pgfpathlineto{\pgfqpoint{2.844121in}{2.904981in}}%
\pgfpathlineto{\pgfqpoint{2.964364in}{2.800301in}}%
\pgfpathlineto{\pgfqpoint{3.022766in}{2.747732in}}%
\pgfpathlineto{\pgfqpoint{3.084606in}{2.692310in}}%
\pgfpathlineto{\pgfqpoint{3.204848in}{2.580949in}}%
\pgfpathlineto{\pgfqpoint{3.325091in}{2.466153in}}%
\pgfpathlineto{\pgfqpoint{3.445333in}{2.347859in}}%
\pgfpathlineto{\pgfqpoint{3.565576in}{2.226001in}}%
\pgfpathlineto{\pgfqpoint{3.619005in}{2.170667in}}%
\pgfpathlineto{\pgfqpoint{3.725899in}{2.057835in}}%
\pgfpathlineto{\pgfqpoint{3.799491in}{1.977881in}}%
\pgfpathlineto{\pgfqpoint{3.846141in}{1.926782in}}%
\pgfpathlineto{\pgfqpoint{3.961373in}{1.797333in}}%
\pgfpathlineto{\pgfqpoint{4.058007in}{1.685333in}}%
\pgfpathlineto{\pgfqpoint{4.151974in}{1.573333in}}%
\pgfpathlineto{\pgfqpoint{4.182697in}{1.536000in}}%
\pgfpathlineto{\pgfqpoint{4.273074in}{1.424000in}}%
\pgfpathlineto{\pgfqpoint{4.367192in}{1.303636in}}%
\pgfpathlineto{\pgfqpoint{4.447354in}{1.197743in}}%
\pgfpathlineto{\pgfqpoint{4.528087in}{1.087467in}}%
\pgfpathlineto{\pgfqpoint{4.607677in}{0.974668in}}%
\pgfpathlineto{\pgfqpoint{4.668999in}{0.883786in}}%
\pgfpathlineto{\pgfqpoint{4.707255in}{0.826667in}}%
\pgfpathlineto{\pgfqpoint{4.768000in}{0.732197in}}%
\pgfpathlineto{\pgfqpoint{4.768000in}{0.732197in}}%
\pgfusepath{fill}%
\end{pgfscope}%
\begin{pgfscope}%
\pgfpathrectangle{\pgfqpoint{0.800000in}{0.528000in}}{\pgfqpoint{3.968000in}{3.696000in}}%
\pgfusepath{clip}%
\pgfsetbuttcap%
\pgfsetroundjoin%
\definecolor{currentfill}{rgb}{0.283229,0.120777,0.440584}%
\pgfsetfillcolor{currentfill}%
\pgfsetlinewidth{0.000000pt}%
\definecolor{currentstroke}{rgb}{0.000000,0.000000,0.000000}%
\pgfsetstrokecolor{currentstroke}%
\pgfsetdash{}{0pt}%
\pgfpathmoveto{\pgfqpoint{2.740896in}{0.528000in}}%
\pgfpathlineto{\pgfqpoint{2.651582in}{0.609992in}}%
\pgfpathlineto{\pgfqpoint{2.603636in}{0.654429in}}%
\pgfpathlineto{\pgfqpoint{2.483394in}{0.768807in}}%
\pgfpathlineto{\pgfqpoint{2.363152in}{0.886612in}}%
\pgfpathlineto{\pgfqpoint{2.311144in}{0.938667in}}%
\pgfpathlineto{\pgfqpoint{2.201343in}{1.050667in}}%
\pgfpathlineto{\pgfqpoint{2.145592in}{1.109354in}}%
\pgfpathlineto{\pgfqpoint{2.094726in}{1.162667in}}%
\pgfpathlineto{\pgfqpoint{2.014840in}{1.248898in}}%
\pgfpathlineto{\pgfqpoint{1.962343in}{1.305859in}}%
\pgfpathlineto{\pgfqpoint{1.727232in}{1.573333in}}%
\pgfpathlineto{\pgfqpoint{1.672164in}{1.639045in}}%
\pgfpathlineto{\pgfqpoint{1.633393in}{1.685333in}}%
\pgfpathlineto{\pgfqpoint{1.585174in}{1.744685in}}%
\pgfpathlineto{\pgfqpoint{1.542272in}{1.797333in}}%
\pgfpathlineto{\pgfqpoint{1.453783in}{1.909333in}}%
\pgfpathlineto{\pgfqpoint{1.368018in}{2.021333in}}%
\pgfpathlineto{\pgfqpoint{1.280970in}{2.138994in}}%
\pgfpathlineto{\pgfqpoint{1.200808in}{2.251405in}}%
\pgfpathlineto{\pgfqpoint{1.120646in}{2.368363in}}%
\pgfpathlineto{\pgfqpoint{1.040485in}{2.490631in}}%
\pgfpathlineto{\pgfqpoint{0.938429in}{2.656000in}}%
\pgfpathlineto{\pgfqpoint{0.916341in}{2.693333in}}%
\pgfpathlineto{\pgfqpoint{0.873701in}{2.768000in}}%
\pgfpathlineto{\pgfqpoint{0.832805in}{2.842667in}}%
\pgfpathlineto{\pgfqpoint{0.800000in}{2.905230in}}%
\pgfpathlineto{\pgfqpoint{0.800000in}{2.887385in}}%
\pgfpathlineto{\pgfqpoint{0.864893in}{2.768000in}}%
\pgfpathlineto{\pgfqpoint{0.920242in}{2.672017in}}%
\pgfpathlineto{\pgfqpoint{1.021894in}{2.506667in}}%
\pgfpathlineto{\pgfqpoint{1.059205in}{2.449437in}}%
\pgfpathlineto{\pgfqpoint{1.094871in}{2.394667in}}%
\pgfpathlineto{\pgfqpoint{1.160727in}{2.297503in}}%
\pgfpathlineto{\pgfqpoint{1.321051in}{2.073623in}}%
\pgfpathlineto{\pgfqpoint{1.401212in}{1.967262in}}%
\pgfpathlineto{\pgfqpoint{1.481374in}{1.863995in}}%
\pgfpathlineto{\pgfqpoint{1.564427in}{1.760000in}}%
\pgfpathlineto{\pgfqpoint{1.625636in}{1.685333in}}%
\pgfpathlineto{\pgfqpoint{1.721859in}{1.570429in}}%
\pgfpathlineto{\pgfqpoint{1.848768in}{1.424000in}}%
\pgfpathlineto{\pgfqpoint{1.948969in}{1.312000in}}%
\pgfpathlineto{\pgfqpoint{2.029126in}{1.224872in}}%
\pgfpathlineto{\pgfqpoint{2.082586in}{1.167033in}}%
\pgfpathlineto{\pgfqpoint{2.340213in}{0.901333in}}%
\pgfpathlineto{\pgfqpoint{2.429318in}{0.813631in}}%
\pgfpathlineto{\pgfqpoint{2.483394in}{0.760692in}}%
\pgfpathlineto{\pgfqpoint{2.531390in}{0.714667in}}%
\pgfpathlineto{\pgfqpoint{2.650573in}{0.602667in}}%
\pgfpathlineto{\pgfqpoint{2.732046in}{0.528000in}}%
\pgfpathmoveto{\pgfqpoint{4.768000in}{0.758828in}}%
\pgfpathlineto{\pgfqpoint{4.687838in}{0.880944in}}%
\pgfpathlineto{\pgfqpoint{4.527515in}{1.110300in}}%
\pgfpathlineto{\pgfqpoint{4.347553in}{1.349333in}}%
\pgfpathlineto{\pgfqpoint{4.304839in}{1.403255in}}%
\pgfpathlineto{\pgfqpoint{4.258963in}{1.461333in}}%
\pgfpathlineto{\pgfqpoint{4.206869in}{1.525669in}}%
\pgfpathlineto{\pgfqpoint{4.126707in}{1.622543in}}%
\pgfpathlineto{\pgfqpoint{4.061449in}{1.699216in}}%
\pgfpathlineto{\pgfqpoint{4.009615in}{1.760000in}}%
\pgfpathlineto{\pgfqpoint{3.911052in}{1.872000in}}%
\pgfpathlineto{\pgfqpoint{3.806061in}{1.988279in}}%
\pgfpathlineto{\pgfqpoint{3.740901in}{2.058667in}}%
\pgfpathlineto{\pgfqpoint{3.635108in}{2.170667in}}%
\pgfpathlineto{\pgfqpoint{3.583026in}{2.224254in}}%
\pgfpathlineto{\pgfqpoint{3.525495in}{2.283700in}}%
\pgfpathlineto{\pgfqpoint{3.405253in}{2.403998in}}%
\pgfpathlineto{\pgfqpoint{3.331808in}{2.475590in}}%
\pgfpathlineto{\pgfqpoint{3.285010in}{2.520866in}}%
\pgfpathlineto{\pgfqpoint{3.164768in}{2.634364in}}%
\pgfpathlineto{\pgfqpoint{3.100836in}{2.693333in}}%
\pgfpathlineto{\pgfqpoint{3.004444in}{2.780559in}}%
\pgfpathlineto{\pgfqpoint{2.949551in}{2.828869in}}%
\pgfpathlineto{\pgfqpoint{2.891587in}{2.880000in}}%
\pgfpathlineto{\pgfqpoint{2.670158in}{3.066667in}}%
\pgfpathlineto{\pgfqpoint{2.434170in}{3.253333in}}%
\pgfpathlineto{\pgfqpoint{2.202828in}{3.423327in}}%
\pgfpathlineto{\pgfqpoint{2.115169in}{3.484317in}}%
\pgfpathlineto{\pgfqpoint{2.002424in}{3.559500in}}%
\pgfpathlineto{\pgfqpoint{1.922263in}{3.610761in}}%
\pgfpathlineto{\pgfqpoint{1.863555in}{3.646650in}}%
\pgfpathlineto{\pgfqpoint{1.835806in}{3.664000in}}%
\pgfpathlineto{\pgfqpoint{1.744318in}{3.717747in}}%
\pgfpathlineto{\pgfqpoint{1.639048in}{3.776000in}}%
\pgfpathlineto{\pgfqpoint{1.471246in}{3.860100in}}%
\pgfpathlineto{\pgfqpoint{1.401212in}{3.891581in}}%
\pgfpathlineto{\pgfqpoint{1.255399in}{3.949151in}}%
\pgfpathlineto{\pgfqpoint{1.200808in}{3.967473in}}%
\pgfpathlineto{\pgfqpoint{1.082641in}{4.000000in}}%
\pgfpathlineto{\pgfqpoint{1.079833in}{4.000682in}}%
\pgfpathlineto{\pgfqpoint{1.040485in}{4.008684in}}%
\pgfpathlineto{\pgfqpoint{0.960323in}{4.020374in}}%
\pgfpathlineto{\pgfqpoint{0.920242in}{4.023404in}}%
\pgfpathlineto{\pgfqpoint{0.905081in}{4.023211in}}%
\pgfpathlineto{\pgfqpoint{0.880162in}{4.024148in}}%
\pgfpathlineto{\pgfqpoint{0.840081in}{4.022207in}}%
\pgfpathlineto{\pgfqpoint{0.800000in}{4.017085in}}%
\pgfpathlineto{\pgfqpoint{0.800000in}{3.999617in}}%
\pgfpathlineto{\pgfqpoint{0.800329in}{3.999694in}}%
\pgfpathlineto{\pgfqpoint{0.840081in}{4.005775in}}%
\pgfpathlineto{\pgfqpoint{0.880162in}{4.008617in}}%
\pgfpathlineto{\pgfqpoint{0.889054in}{4.008282in}}%
\pgfpathlineto{\pgfqpoint{0.920242in}{4.008680in}}%
\pgfpathlineto{\pgfqpoint{0.966210in}{4.005484in}}%
\pgfpathlineto{\pgfqpoint{1.013416in}{4.000000in}}%
\pgfpathlineto{\pgfqpoint{1.046076in}{3.994792in}}%
\pgfpathlineto{\pgfqpoint{1.080566in}{3.987756in}}%
\pgfpathlineto{\pgfqpoint{1.200808in}{3.955929in}}%
\pgfpathlineto{\pgfqpoint{1.321051in}{3.913834in}}%
\pgfpathlineto{\pgfqpoint{1.384972in}{3.888000in}}%
\pgfpathlineto{\pgfqpoint{1.441293in}{3.863594in}}%
\pgfpathlineto{\pgfqpoint{1.521455in}{3.826225in}}%
\pgfpathlineto{\pgfqpoint{1.556654in}{3.808786in}}%
\pgfpathlineto{\pgfqpoint{1.577307in}{3.798643in}}%
\pgfpathlineto{\pgfqpoint{1.669172in}{3.750408in}}%
\pgfpathlineto{\pgfqpoint{1.761939in}{3.698542in}}%
\pgfpathlineto{\pgfqpoint{1.842101in}{3.651262in}}%
\pgfpathlineto{\pgfqpoint{1.906124in}{3.611635in}}%
\pgfpathlineto{\pgfqpoint{1.942363in}{3.589333in}}%
\pgfpathlineto{\pgfqpoint{2.122667in}{3.470711in}}%
\pgfpathlineto{\pgfqpoint{2.220240in}{3.402667in}}%
\pgfpathlineto{\pgfqpoint{2.323810in}{3.328000in}}%
\pgfpathlineto{\pgfqpoint{2.443313in}{3.238226in}}%
\pgfpathlineto{\pgfqpoint{2.567718in}{3.141333in}}%
\pgfpathlineto{\pgfqpoint{2.651373in}{3.073798in}}%
\pgfpathlineto{\pgfqpoint{2.705845in}{3.029333in}}%
\pgfpathlineto{\pgfqpoint{2.804040in}{2.947203in}}%
\pgfpathlineto{\pgfqpoint{2.844121in}{2.913051in}}%
\pgfpathlineto{\pgfqpoint{2.967735in}{2.805333in}}%
\pgfpathlineto{\pgfqpoint{3.027149in}{2.751815in}}%
\pgfpathlineto{\pgfqpoint{3.084606in}{2.700265in}}%
\pgfpathlineto{\pgfqpoint{3.212907in}{2.581333in}}%
\pgfpathlineto{\pgfqpoint{3.330175in}{2.469333in}}%
\pgfpathlineto{\pgfqpoint{3.445333in}{2.356180in}}%
\pgfpathlineto{\pgfqpoint{3.565576in}{2.234374in}}%
\pgfpathlineto{\pgfqpoint{3.627056in}{2.170667in}}%
\pgfpathlineto{\pgfqpoint{3.733024in}{2.058667in}}%
\pgfpathlineto{\pgfqpoint{3.785168in}{2.001873in}}%
\pgfpathlineto{\pgfqpoint{3.836065in}{1.946667in}}%
\pgfpathlineto{\pgfqpoint{3.886222in}{1.891027in}}%
\pgfpathlineto{\pgfqpoint{4.001778in}{1.760000in}}%
\pgfpathlineto{\pgfqpoint{4.065835in}{1.685333in}}%
\pgfpathlineto{\pgfqpoint{4.166788in}{1.564985in}}%
\pgfpathlineto{\pgfqpoint{4.251216in}{1.461333in}}%
\pgfpathlineto{\pgfqpoint{4.317360in}{1.377584in}}%
\pgfpathlineto{\pgfqpoint{4.339718in}{1.349333in}}%
\pgfpathlineto{\pgfqpoint{4.425531in}{1.237333in}}%
\pgfpathlineto{\pgfqpoint{4.508561in}{1.125333in}}%
\pgfpathlineto{\pgfqpoint{4.535662in}{1.088000in}}%
\pgfpathlineto{\pgfqpoint{4.607677in}{0.986291in}}%
\pgfpathlineto{\pgfqpoint{4.739955in}{0.789333in}}%
\pgfpathlineto{\pgfqpoint{4.768000in}{0.745686in}}%
\pgfpathlineto{\pgfqpoint{4.768000in}{0.752000in}}%
\pgfpathlineto{\pgfqpoint{4.768000in}{0.752000in}}%
\pgfusepath{fill}%
\end{pgfscope}%
\begin{pgfscope}%
\pgfpathrectangle{\pgfqpoint{0.800000in}{0.528000in}}{\pgfqpoint{3.968000in}{3.696000in}}%
\pgfusepath{clip}%
\pgfsetbuttcap%
\pgfsetroundjoin%
\definecolor{currentfill}{rgb}{0.283229,0.120777,0.440584}%
\pgfsetfillcolor{currentfill}%
\pgfsetlinewidth{0.000000pt}%
\definecolor{currentstroke}{rgb}{0.000000,0.000000,0.000000}%
\pgfsetstrokecolor{currentstroke}%
\pgfsetdash{}{0pt}%
\pgfpathmoveto{\pgfqpoint{2.732046in}{0.528000in}}%
\pgfpathlineto{\pgfqpoint{2.483394in}{0.760692in}}%
\pgfpathlineto{\pgfqpoint{2.363152in}{0.878448in}}%
\pgfpathlineto{\pgfqpoint{2.303022in}{0.938667in}}%
\pgfpathlineto{\pgfqpoint{2.193399in}{1.050667in}}%
\pgfpathlineto{\pgfqpoint{2.141359in}{1.105411in}}%
\pgfpathlineto{\pgfqpoint{2.084805in}{1.164734in}}%
\pgfpathlineto{\pgfqpoint{2.042505in}{1.210018in}}%
\pgfpathlineto{\pgfqpoint{1.982977in}{1.274667in}}%
\pgfpathlineto{\pgfqpoint{1.879756in}{1.388926in}}%
\pgfpathlineto{\pgfqpoint{1.761939in}{1.523684in}}%
\pgfpathlineto{\pgfqpoint{1.561535in}{1.763575in}}%
\pgfpathlineto{\pgfqpoint{1.504642in}{1.834667in}}%
\pgfpathlineto{\pgfqpoint{1.417043in}{1.946667in}}%
\pgfpathlineto{\pgfqpoint{1.332170in}{2.058667in}}%
\pgfpathlineto{\pgfqpoint{1.250116in}{2.170667in}}%
\pgfpathlineto{\pgfqpoint{1.214656in}{2.220899in}}%
\pgfpathlineto{\pgfqpoint{1.189668in}{2.255710in}}%
\pgfpathlineto{\pgfqpoint{1.118260in}{2.359556in}}%
\pgfpathlineto{\pgfqpoint{1.040485in}{2.477548in}}%
\pgfpathlineto{\pgfqpoint{0.912292in}{2.685928in}}%
\pgfpathlineto{\pgfqpoint{0.892677in}{2.719009in}}%
\pgfpathlineto{\pgfqpoint{0.840081in}{2.812552in}}%
\pgfpathlineto{\pgfqpoint{0.800000in}{2.887385in}}%
\pgfpathlineto{\pgfqpoint{0.800000in}{2.870201in}}%
\pgfpathlineto{\pgfqpoint{0.856086in}{2.768000in}}%
\pgfpathlineto{\pgfqpoint{0.899086in}{2.693333in}}%
\pgfpathlineto{\pgfqpoint{0.922019in}{2.654345in}}%
\pgfpathlineto{\pgfqpoint{1.000404in}{2.527310in}}%
\pgfpathlineto{\pgfqpoint{1.137198in}{2.320000in}}%
\pgfpathlineto{\pgfqpoint{1.200808in}{2.228617in}}%
\pgfpathlineto{\pgfqpoint{1.380617in}{1.984000in}}%
\pgfpathlineto{\pgfqpoint{1.409166in}{1.946667in}}%
\pgfpathlineto{\pgfqpoint{1.496853in}{1.834667in}}%
\pgfpathlineto{\pgfqpoint{1.541829in}{1.778978in}}%
\pgfpathlineto{\pgfqpoint{1.587171in}{1.722667in}}%
\pgfpathlineto{\pgfqpoint{1.681778in}{1.608599in}}%
\pgfpathlineto{\pgfqpoint{1.808209in}{1.461333in}}%
\pgfpathlineto{\pgfqpoint{1.907484in}{1.349333in}}%
\pgfpathlineto{\pgfqpoint{1.975115in}{1.274667in}}%
\pgfpathlineto{\pgfqpoint{2.082586in}{1.158614in}}%
\pgfpathlineto{\pgfqpoint{2.149599in}{1.088000in}}%
\pgfpathlineto{\pgfqpoint{2.258099in}{0.976000in}}%
\pgfpathlineto{\pgfqpoint{2.369497in}{0.864000in}}%
\pgfpathlineto{\pgfqpoint{2.483994in}{0.752000in}}%
\pgfpathlineto{\pgfqpoint{2.603636in}{0.638347in}}%
\pgfpathlineto{\pgfqpoint{2.643717in}{0.601005in}}%
\pgfpathlineto{\pgfqpoint{2.723879in}{0.528000in}}%
\pgfpathmoveto{\pgfqpoint{4.768000in}{0.771663in}}%
\pgfpathlineto{\pgfqpoint{4.707328in}{0.864000in}}%
\pgfpathlineto{\pgfqpoint{4.647758in}{0.951775in}}%
\pgfpathlineto{\pgfqpoint{4.487434in}{1.175729in}}%
\pgfpathlineto{\pgfqpoint{4.407273in}{1.282206in}}%
\pgfpathlineto{\pgfqpoint{4.326260in}{1.386667in}}%
\pgfpathlineto{\pgfqpoint{4.258084in}{1.471705in}}%
\pgfpathlineto{\pgfqpoint{4.206277in}{1.536000in}}%
\pgfpathlineto{\pgfqpoint{4.113040in}{1.648000in}}%
\pgfpathlineto{\pgfqpoint{4.065751in}{1.703223in}}%
\pgfpathlineto{\pgfqpoint{4.017300in}{1.760000in}}%
\pgfpathlineto{\pgfqpoint{3.918904in}{1.872000in}}%
\pgfpathlineto{\pgfqpoint{3.817742in}{1.984000in}}%
\pgfpathlineto{\pgfqpoint{3.713867in}{2.096000in}}%
\pgfpathlineto{\pgfqpoint{3.605657in}{2.209697in}}%
\pgfpathlineto{\pgfqpoint{3.565576in}{2.250944in}}%
\pgfpathlineto{\pgfqpoint{3.445333in}{2.372350in}}%
\pgfpathlineto{\pgfqpoint{3.325091in}{2.490306in}}%
\pgfpathlineto{\pgfqpoint{3.204848in}{2.604873in}}%
\pgfpathlineto{\pgfqpoint{3.084606in}{2.716112in}}%
\pgfpathlineto{\pgfqpoint{3.027240in}{2.768000in}}%
\pgfpathlineto{\pgfqpoint{2.924283in}{2.859354in}}%
\pgfpathlineto{\pgfqpoint{2.804040in}{2.963053in}}%
\pgfpathlineto{\pgfqpoint{2.745355in}{3.012004in}}%
\pgfpathlineto{\pgfqpoint{2.683798in}{3.063523in}}%
\pgfpathlineto{\pgfqpoint{2.563556in}{3.160567in}}%
\pgfpathlineto{\pgfqpoint{2.523475in}{3.192220in}}%
\pgfpathlineto{\pgfqpoint{2.403232in}{3.284954in}}%
\pgfpathlineto{\pgfqpoint{2.345681in}{3.328000in}}%
\pgfpathlineto{\pgfqpoint{2.240510in}{3.404901in}}%
\pgfpathlineto{\pgfqpoint{2.122667in}{3.487464in}}%
\pgfpathlineto{\pgfqpoint{2.026729in}{3.552000in}}%
\pgfpathlineto{\pgfqpoint{1.842101in}{3.669062in}}%
\pgfpathlineto{\pgfqpoint{1.761939in}{3.716527in}}%
\pgfpathlineto{\pgfqpoint{1.681778in}{3.761944in}}%
\pgfpathlineto{\pgfqpoint{1.585681in}{3.813333in}}%
\pgfpathlineto{\pgfqpoint{1.449976in}{3.879912in}}%
\pgfpathlineto{\pgfqpoint{1.401212in}{3.901690in}}%
\pgfpathlineto{\pgfqpoint{1.328241in}{3.932031in}}%
\pgfpathlineto{\pgfqpoint{1.302609in}{3.942511in}}%
\pgfpathlineto{\pgfqpoint{1.236302in}{3.966939in}}%
\pgfpathlineto{\pgfqpoint{1.160727in}{3.991281in}}%
\pgfpathlineto{\pgfqpoint{1.116497in}{4.003865in}}%
\pgfpathlineto{\pgfqpoint{1.040485in}{4.021423in}}%
\pgfpathlineto{\pgfqpoint{1.025770in}{4.023627in}}%
\pgfpathlineto{\pgfqpoint{0.992056in}{4.029557in}}%
\pgfpathlineto{\pgfqpoint{0.960323in}{4.034371in}}%
\pgfpathlineto{\pgfqpoint{0.920242in}{4.038087in}}%
\pgfpathlineto{\pgfqpoint{0.877725in}{4.039603in}}%
\pgfpathlineto{\pgfqpoint{0.826829in}{4.037333in}}%
\pgfpathlineto{\pgfqpoint{0.800000in}{4.034528in}}%
\pgfpathlineto{\pgfqpoint{0.800000in}{4.017085in}}%
\pgfpathlineto{\pgfqpoint{0.854986in}{4.023449in}}%
\pgfpathlineto{\pgfqpoint{0.880162in}{4.024148in}}%
\pgfpathlineto{\pgfqpoint{0.960323in}{4.020374in}}%
\pgfpathlineto{\pgfqpoint{1.040485in}{4.008684in}}%
\pgfpathlineto{\pgfqpoint{1.082641in}{4.000000in}}%
\pgfpathlineto{\pgfqpoint{1.160727in}{3.979559in}}%
\pgfpathlineto{\pgfqpoint{1.174064in}{3.975089in}}%
\pgfpathlineto{\pgfqpoint{1.215277in}{3.962667in}}%
\pgfpathlineto{\pgfqpoint{1.280970in}{3.939792in}}%
\pgfpathlineto{\pgfqpoint{1.322179in}{3.924282in}}%
\pgfpathlineto{\pgfqpoint{1.409267in}{3.888000in}}%
\pgfpathlineto{\pgfqpoint{1.537063in}{3.827872in}}%
\pgfpathlineto{\pgfqpoint{1.567241in}{3.813333in}}%
\pgfpathlineto{\pgfqpoint{1.645199in}{3.772738in}}%
\pgfpathlineto{\pgfqpoint{1.744318in}{3.717747in}}%
\pgfpathlineto{\pgfqpoint{1.842101in}{3.660248in}}%
\pgfpathlineto{\pgfqpoint{1.911843in}{3.616961in}}%
\pgfpathlineto{\pgfqpoint{1.956180in}{3.589333in}}%
\pgfpathlineto{\pgfqpoint{2.115169in}{3.484317in}}%
\pgfpathlineto{\pgfqpoint{2.202828in}{3.423327in}}%
\pgfpathlineto{\pgfqpoint{2.283973in}{3.365333in}}%
\pgfpathlineto{\pgfqpoint{2.363152in}{3.306907in}}%
\pgfpathlineto{\pgfqpoint{2.486038in}{3.213537in}}%
\pgfpathlineto{\pgfqpoint{2.603636in}{3.120596in}}%
\pgfpathlineto{\pgfqpoint{2.723879in}{3.022453in}}%
\pgfpathlineto{\pgfqpoint{2.783045in}{2.972444in}}%
\pgfpathlineto{\pgfqpoint{2.844121in}{2.921008in}}%
\pgfpathlineto{\pgfqpoint{2.891587in}{2.880000in}}%
\pgfpathlineto{\pgfqpoint{3.018443in}{2.768000in}}%
\pgfpathlineto{\pgfqpoint{3.084606in}{2.708188in}}%
\pgfpathlineto{\pgfqpoint{3.204848in}{2.596902in}}%
\pgfpathlineto{\pgfqpoint{3.260759in}{2.544000in}}%
\pgfpathlineto{\pgfqpoint{3.376735in}{2.432000in}}%
\pgfpathlineto{\pgfqpoint{3.414684in}{2.394667in}}%
\pgfpathlineto{\pgfqpoint{3.537208in}{2.271756in}}%
\pgfpathlineto{\pgfqpoint{3.645737in}{2.159566in}}%
\pgfpathlineto{\pgfqpoint{3.765980in}{2.031723in}}%
\pgfpathlineto{\pgfqpoint{3.877677in}{1.909333in}}%
\pgfpathlineto{\pgfqpoint{3.944163in}{1.834667in}}%
\pgfpathlineto{\pgfqpoint{4.046545in}{1.717155in}}%
\pgfpathlineto{\pgfqpoint{4.114518in}{1.636647in}}%
\pgfpathlineto{\pgfqpoint{4.167776in}{1.573333in}}%
\pgfpathlineto{\pgfqpoint{4.258963in}{1.461333in}}%
\pgfpathlineto{\pgfqpoint{4.304839in}{1.403255in}}%
\pgfpathlineto{\pgfqpoint{4.347553in}{1.349333in}}%
\pgfpathlineto{\pgfqpoint{4.433456in}{1.237333in}}%
\pgfpathlineto{\pgfqpoint{4.516578in}{1.125333in}}%
\pgfpathlineto{\pgfqpoint{4.543620in}{1.088000in}}%
\pgfpathlineto{\pgfqpoint{4.607677in}{0.997856in}}%
\pgfpathlineto{\pgfqpoint{4.748260in}{0.789333in}}%
\pgfpathlineto{\pgfqpoint{4.768000in}{0.758828in}}%
\pgfpathlineto{\pgfqpoint{4.768000in}{0.758828in}}%
\pgfusepath{fill}%
\end{pgfscope}%
\begin{pgfscope}%
\pgfpathrectangle{\pgfqpoint{0.800000in}{0.528000in}}{\pgfqpoint{3.968000in}{3.696000in}}%
\pgfusepath{clip}%
\pgfsetbuttcap%
\pgfsetroundjoin%
\definecolor{currentfill}{rgb}{0.283229,0.120777,0.440584}%
\pgfsetfillcolor{currentfill}%
\pgfsetlinewidth{0.000000pt}%
\definecolor{currentstroke}{rgb}{0.000000,0.000000,0.000000}%
\pgfsetstrokecolor{currentstroke}%
\pgfsetdash{}{0pt}%
\pgfpathmoveto{\pgfqpoint{2.723220in}{0.528000in}}%
\pgfpathlineto{\pgfqpoint{2.601876in}{0.640000in}}%
\pgfpathlineto{\pgfqpoint{2.483394in}{0.752577in}}%
\pgfpathlineto{\pgfqpoint{2.363152in}{0.870283in}}%
\pgfpathlineto{\pgfqpoint{2.294900in}{0.938667in}}%
\pgfpathlineto{\pgfqpoint{2.185455in}{1.050667in}}%
\pgfpathlineto{\pgfqpoint{2.122667in}{1.116229in}}%
\pgfpathlineto{\pgfqpoint{2.009343in}{1.237333in}}%
\pgfpathlineto{\pgfqpoint{1.941164in}{1.312000in}}%
\pgfpathlineto{\pgfqpoint{1.835509in}{1.430141in}}%
\pgfpathlineto{\pgfqpoint{1.721859in}{1.561414in}}%
\pgfpathlineto{\pgfqpoint{1.641697in}{1.656540in}}%
\pgfpathlineto{\pgfqpoint{1.526581in}{1.797333in}}%
\pgfpathlineto{\pgfqpoint{1.429400in}{1.920411in}}%
\pgfpathlineto{\pgfqpoint{1.352263in}{2.021333in}}%
\pgfpathlineto{\pgfqpoint{1.306795in}{2.082722in}}%
\pgfpathlineto{\pgfqpoint{1.269232in}{2.133333in}}%
\pgfpathlineto{\pgfqpoint{1.200808in}{2.228617in}}%
\pgfpathlineto{\pgfqpoint{1.120646in}{2.344199in}}%
\pgfpathlineto{\pgfqpoint{1.000404in}{2.527310in}}%
\pgfpathlineto{\pgfqpoint{0.943710in}{2.618667in}}%
\pgfpathlineto{\pgfqpoint{0.899086in}{2.693333in}}%
\pgfpathlineto{\pgfqpoint{0.856086in}{2.768000in}}%
\pgfpathlineto{\pgfqpoint{0.814815in}{2.842667in}}%
\pgfpathlineto{\pgfqpoint{0.800000in}{2.870201in}}%
\pgfpathlineto{\pgfqpoint{0.800000in}{2.853483in}}%
\pgfpathlineto{\pgfqpoint{0.840081in}{2.780799in}}%
\pgfpathlineto{\pgfqpoint{0.890458in}{2.693333in}}%
\pgfpathlineto{\pgfqpoint{0.920242in}{2.643360in}}%
\pgfpathlineto{\pgfqpoint{0.981575in}{2.544000in}}%
\pgfpathlineto{\pgfqpoint{1.040485in}{2.452227in}}%
\pgfpathlineto{\pgfqpoint{1.181078in}{2.245333in}}%
\pgfpathlineto{\pgfqpoint{1.261379in}{2.133333in}}%
\pgfpathlineto{\pgfqpoint{1.302179in}{2.078422in}}%
\pgfpathlineto{\pgfqpoint{1.344498in}{2.021333in}}%
\pgfpathlineto{\pgfqpoint{1.384849in}{1.968758in}}%
\pgfpathlineto{\pgfqpoint{1.430343in}{1.909333in}}%
\pgfpathlineto{\pgfqpoint{1.521455in}{1.794061in}}%
\pgfpathlineto{\pgfqpoint{1.579469in}{1.722667in}}%
\pgfpathlineto{\pgfqpoint{1.681778in}{1.599563in}}%
\pgfpathlineto{\pgfqpoint{1.802020in}{1.459508in}}%
\pgfpathlineto{\pgfqpoint{1.922263in}{1.324261in}}%
\pgfpathlineto{\pgfqpoint{2.036092in}{1.200000in}}%
\pgfpathlineto{\pgfqpoint{2.141712in}{1.088000in}}%
\pgfpathlineto{\pgfqpoint{2.250037in}{0.976000in}}%
\pgfpathlineto{\pgfqpoint{2.363152in}{0.862174in}}%
\pgfpathlineto{\pgfqpoint{2.483394in}{0.744686in}}%
\pgfpathlineto{\pgfqpoint{2.603636in}{0.630520in}}%
\pgfpathlineto{\pgfqpoint{2.643717in}{0.593193in}}%
\pgfpathlineto{\pgfqpoint{2.714688in}{0.528000in}}%
\pgfpathmoveto{\pgfqpoint{4.768000in}{0.784499in}}%
\pgfpathlineto{\pgfqpoint{4.715473in}{0.864000in}}%
\pgfpathlineto{\pgfqpoint{4.647758in}{0.963374in}}%
\pgfpathlineto{\pgfqpoint{4.497626in}{1.172160in}}%
\pgfpathlineto{\pgfqpoint{4.449243in}{1.237333in}}%
\pgfpathlineto{\pgfqpoint{4.363224in}{1.349333in}}%
\pgfpathlineto{\pgfqpoint{4.304297in}{1.424000in}}%
\pgfpathlineto{\pgfqpoint{4.206869in}{1.544573in}}%
\pgfpathlineto{\pgfqpoint{4.086626in}{1.688392in}}%
\pgfpathlineto{\pgfqpoint{4.024986in}{1.760000in}}%
\pgfpathlineto{\pgfqpoint{3.926303in}{1.872493in}}%
\pgfpathlineto{\pgfqpoint{3.806061in}{2.005239in}}%
\pgfpathlineto{\pgfqpoint{3.756653in}{2.058667in}}%
\pgfpathlineto{\pgfqpoint{3.645737in}{2.176207in}}%
\pgfpathlineto{\pgfqpoint{3.572490in}{2.251773in}}%
\pgfpathlineto{\pgfqpoint{3.525495in}{2.299900in}}%
\pgfpathlineto{\pgfqpoint{3.405253in}{2.420100in}}%
\pgfpathlineto{\pgfqpoint{3.340125in}{2.483337in}}%
\pgfpathlineto{\pgfqpoint{3.285010in}{2.536872in}}%
\pgfpathlineto{\pgfqpoint{3.158587in}{2.656000in}}%
\pgfpathlineto{\pgfqpoint{3.036037in}{2.768000in}}%
\pgfpathlineto{\pgfqpoint{2.909620in}{2.880000in}}%
\pgfpathlineto{\pgfqpoint{2.812994in}{2.963007in}}%
\pgfpathlineto{\pgfqpoint{2.763960in}{3.004720in}}%
\pgfpathlineto{\pgfqpoint{2.723879in}{3.038225in}}%
\pgfpathlineto{\pgfqpoint{2.597641in}{3.141333in}}%
\pgfpathlineto{\pgfqpoint{2.483394in}{3.231455in}}%
\pgfpathlineto{\pgfqpoint{2.363152in}{3.323147in}}%
\pgfpathlineto{\pgfqpoint{2.242909in}{3.411255in}}%
\pgfpathlineto{\pgfqpoint{2.149291in}{3.477333in}}%
\pgfpathlineto{\pgfqpoint{1.962343in}{3.602491in}}%
\pgfpathlineto{\pgfqpoint{1.864536in}{3.664000in}}%
\pgfpathlineto{\pgfqpoint{1.694653in}{3.764007in}}%
\pgfpathlineto{\pgfqpoint{1.601616in}{3.814579in}}%
\pgfpathlineto{\pgfqpoint{1.441293in}{3.893776in}}%
\pgfpathlineto{\pgfqpoint{1.280970in}{3.961503in}}%
\pgfpathlineto{\pgfqpoint{1.200808in}{3.989986in}}%
\pgfpathlineto{\pgfqpoint{1.169688in}{4.000000in}}%
\pgfpathlineto{\pgfqpoint{1.120646in}{4.014415in}}%
\pgfpathlineto{\pgfqpoint{1.080566in}{4.024892in}}%
\pgfpathlineto{\pgfqpoint{1.023935in}{4.037333in}}%
\pgfpathlineto{\pgfqpoint{1.000404in}{4.041828in}}%
\pgfpathlineto{\pgfqpoint{0.920242in}{4.052035in}}%
\pgfpathlineto{\pgfqpoint{0.840081in}{4.054033in}}%
\pgfpathlineto{\pgfqpoint{0.800000in}{4.051066in}}%
\pgfpathlineto{\pgfqpoint{0.800000in}{4.034528in}}%
\pgfpathlineto{\pgfqpoint{0.840081in}{4.038562in}}%
\pgfpathlineto{\pgfqpoint{0.882344in}{4.039366in}}%
\pgfpathlineto{\pgfqpoint{0.928390in}{4.037333in}}%
\pgfpathlineto{\pgfqpoint{0.960323in}{4.034371in}}%
\pgfpathlineto{\pgfqpoint{1.040485in}{4.021423in}}%
\pgfpathlineto{\pgfqpoint{1.080566in}{4.012701in}}%
\pgfpathlineto{\pgfqpoint{1.130217in}{4.000000in}}%
\pgfpathlineto{\pgfqpoint{1.200808in}{3.978729in}}%
\pgfpathlineto{\pgfqpoint{1.248132in}{3.962667in}}%
\pgfpathlineto{\pgfqpoint{1.321051in}{3.935199in}}%
\pgfpathlineto{\pgfqpoint{1.345341in}{3.925333in}}%
\pgfpathlineto{\pgfqpoint{1.401212in}{3.901690in}}%
\pgfpathlineto{\pgfqpoint{1.438214in}{3.885132in}}%
\pgfpathlineto{\pgfqpoint{1.449976in}{3.879912in}}%
\pgfpathlineto{\pgfqpoint{1.521455in}{3.845842in}}%
\pgfpathlineto{\pgfqpoint{1.681778in}{3.761944in}}%
\pgfpathlineto{\pgfqpoint{1.761939in}{3.716527in}}%
\pgfpathlineto{\pgfqpoint{1.802020in}{3.693057in}}%
\pgfpathlineto{\pgfqpoint{1.882182in}{3.644450in}}%
\pgfpathlineto{\pgfqpoint{1.922263in}{3.619489in}}%
\pgfpathlineto{\pgfqpoint{2.002424in}{3.567986in}}%
\pgfpathlineto{\pgfqpoint{2.042505in}{3.541602in}}%
\pgfpathlineto{\pgfqpoint{2.137323in}{3.477333in}}%
\pgfpathlineto{\pgfqpoint{2.243625in}{3.402667in}}%
\pgfpathlineto{\pgfqpoint{2.323071in}{3.344755in}}%
\pgfpathlineto{\pgfqpoint{2.444822in}{3.253333in}}%
\pgfpathlineto{\pgfqpoint{2.563556in}{3.160567in}}%
\pgfpathlineto{\pgfqpoint{2.683798in}{3.063523in}}%
\pgfpathlineto{\pgfqpoint{2.804040in}{2.963053in}}%
\pgfpathlineto{\pgfqpoint{2.857479in}{2.917333in}}%
\pgfpathlineto{\pgfqpoint{2.964364in}{2.824081in}}%
\pgfpathlineto{\pgfqpoint{3.084606in}{2.716112in}}%
\pgfpathlineto{\pgfqpoint{3.204848in}{2.604873in}}%
\pgfpathlineto{\pgfqpoint{3.269148in}{2.544000in}}%
\pgfpathlineto{\pgfqpoint{3.365172in}{2.451367in}}%
\pgfpathlineto{\pgfqpoint{3.485414in}{2.332268in}}%
\pgfpathlineto{\pgfqpoint{3.534469in}{2.282667in}}%
\pgfpathlineto{\pgfqpoint{3.645737in}{2.167974in}}%
\pgfpathlineto{\pgfqpoint{3.765980in}{2.040185in}}%
\pgfpathlineto{\pgfqpoint{3.886222in}{1.908626in}}%
\pgfpathlineto{\pgfqpoint{4.006465in}{1.772489in}}%
\pgfpathlineto{\pgfqpoint{4.086626in}{1.679292in}}%
\pgfpathlineto{\pgfqpoint{4.153981in}{1.598738in}}%
\pgfpathlineto{\pgfqpoint{4.206869in}{1.535276in}}%
\pgfpathlineto{\pgfqpoint{4.327111in}{1.385584in}}%
\pgfpathlineto{\pgfqpoint{4.384314in}{1.312000in}}%
\pgfpathlineto{\pgfqpoint{4.469336in}{1.200000in}}%
\pgfpathlineto{\pgfqpoint{4.551578in}{1.088000in}}%
\pgfpathlineto{\pgfqpoint{4.630942in}{0.976000in}}%
\pgfpathlineto{\pgfqpoint{4.656789in}{0.938667in}}%
\pgfpathlineto{\pgfqpoint{4.715008in}{0.851974in}}%
\pgfpathlineto{\pgfqpoint{4.745473in}{0.805684in}}%
\pgfpathlineto{\pgfqpoint{4.768000in}{0.771663in}}%
\pgfpathlineto{\pgfqpoint{4.768000in}{0.771663in}}%
\pgfusepath{fill}%
\end{pgfscope}%
\begin{pgfscope}%
\pgfpathrectangle{\pgfqpoint{0.800000in}{0.528000in}}{\pgfqpoint{3.968000in}{3.696000in}}%
\pgfusepath{clip}%
\pgfsetbuttcap%
\pgfsetroundjoin%
\definecolor{currentfill}{rgb}{0.283229,0.120777,0.440584}%
\pgfsetfillcolor{currentfill}%
\pgfsetlinewidth{0.000000pt}%
\definecolor{currentstroke}{rgb}{0.000000,0.000000,0.000000}%
\pgfsetstrokecolor{currentstroke}%
\pgfsetdash{}{0pt}%
\pgfpathmoveto{\pgfqpoint{2.714688in}{0.528000in}}%
\pgfpathlineto{\pgfqpoint{2.593540in}{0.640000in}}%
\pgfpathlineto{\pgfqpoint{2.475825in}{0.752000in}}%
\pgfpathlineto{\pgfqpoint{2.361315in}{0.864000in}}%
\pgfpathlineto{\pgfqpoint{2.304472in}{0.921343in}}%
\pgfpathlineto{\pgfqpoint{2.250037in}{0.976000in}}%
\pgfpathlineto{\pgfqpoint{2.162747in}{1.065997in}}%
\pgfpathlineto{\pgfqpoint{2.036092in}{1.200000in}}%
\pgfpathlineto{\pgfqpoint{1.995680in}{1.243615in}}%
\pgfpathlineto{\pgfqpoint{1.882182in}{1.368910in}}%
\pgfpathlineto{\pgfqpoint{1.768018in}{1.498667in}}%
\pgfpathlineto{\pgfqpoint{1.672417in}{1.610667in}}%
\pgfpathlineto{\pgfqpoint{1.638677in}{1.650813in}}%
\pgfpathlineto{\pgfqpoint{1.518823in}{1.797333in}}%
\pgfpathlineto{\pgfqpoint{1.452021in}{1.881993in}}%
\pgfpathlineto{\pgfqpoint{1.401212in}{1.946766in}}%
\pgfpathlineto{\pgfqpoint{1.335036in}{2.034360in}}%
\pgfpathlineto{\pgfqpoint{1.288715in}{2.096000in}}%
\pgfpathlineto{\pgfqpoint{1.204870in}{2.211784in}}%
\pgfpathlineto{\pgfqpoint{1.181078in}{2.245333in}}%
\pgfpathlineto{\pgfqpoint{1.120646in}{2.332364in}}%
\pgfpathlineto{\pgfqpoint{0.981575in}{2.544000in}}%
\pgfpathlineto{\pgfqpoint{0.955015in}{2.586278in}}%
\pgfpathlineto{\pgfqpoint{0.880162in}{2.710915in}}%
\pgfpathlineto{\pgfqpoint{0.800000in}{2.853483in}}%
\pgfpathlineto{\pgfqpoint{0.800000in}{2.837116in}}%
\pgfpathlineto{\pgfqpoint{0.904228in}{2.656000in}}%
\pgfpathlineto{\pgfqpoint{0.960323in}{2.564594in}}%
\pgfpathlineto{\pgfqpoint{1.095658in}{2.357333in}}%
\pgfpathlineto{\pgfqpoint{1.121647in}{2.319068in}}%
\pgfpathlineto{\pgfqpoint{1.200808in}{2.206165in}}%
\pgfpathlineto{\pgfqpoint{1.280970in}{2.095785in}}%
\pgfpathlineto{\pgfqpoint{1.347024in}{2.008193in}}%
\pgfpathlineto{\pgfqpoint{1.393661in}{1.946667in}}%
\pgfpathlineto{\pgfqpoint{1.481374in}{1.834547in}}%
\pgfpathlineto{\pgfqpoint{1.541390in}{1.760000in}}%
\pgfpathlineto{\pgfqpoint{1.641697in}{1.638134in}}%
\pgfpathlineto{\pgfqpoint{1.696374in}{1.573333in}}%
\pgfpathlineto{\pgfqpoint{1.802020in}{1.450827in}}%
\pgfpathlineto{\pgfqpoint{1.925553in}{1.312000in}}%
\pgfpathlineto{\pgfqpoint{1.962343in}{1.271537in}}%
\pgfpathlineto{\pgfqpoint{2.082586in}{1.142047in}}%
\pgfpathlineto{\pgfqpoint{2.133826in}{1.088000in}}%
\pgfpathlineto{\pgfqpoint{2.242909in}{0.975077in}}%
\pgfpathlineto{\pgfqpoint{2.363152in}{0.854254in}}%
\pgfpathlineto{\pgfqpoint{2.483394in}{0.736813in}}%
\pgfpathlineto{\pgfqpoint{2.603636in}{0.622693in}}%
\pgfpathlineto{\pgfqpoint{2.665443in}{0.565333in}}%
\pgfpathlineto{\pgfqpoint{2.706155in}{0.528000in}}%
\pgfpathmoveto{\pgfqpoint{4.768000in}{0.796964in}}%
\pgfpathlineto{\pgfqpoint{4.607677in}{1.031746in}}%
\pgfpathlineto{\pgfqpoint{4.527515in}{1.142638in}}%
\pgfpathlineto{\pgfqpoint{4.341565in}{1.386667in}}%
\pgfpathlineto{\pgfqpoint{4.311990in}{1.424000in}}%
\pgfpathlineto{\pgfqpoint{4.221512in}{1.536000in}}%
\pgfpathlineto{\pgfqpoint{4.162791in}{1.606944in}}%
\pgfpathlineto{\pgfqpoint{4.126707in}{1.650152in}}%
\pgfpathlineto{\pgfqpoint{4.064860in}{1.722667in}}%
\pgfpathlineto{\pgfqpoint{3.966384in}{1.835946in}}%
\pgfpathlineto{\pgfqpoint{3.894112in}{1.916682in}}%
\pgfpathlineto{\pgfqpoint{3.846141in}{1.969911in}}%
\pgfpathlineto{\pgfqpoint{3.725899in}{2.099956in}}%
\pgfpathlineto{\pgfqpoint{3.605657in}{2.225963in}}%
\pgfpathlineto{\pgfqpoint{3.550387in}{2.282667in}}%
\pgfpathlineto{\pgfqpoint{3.439098in}{2.394667in}}%
\pgfpathlineto{\pgfqpoint{3.324756in}{2.506667in}}%
\pgfpathlineto{\pgfqpoint{3.265367in}{2.563037in}}%
\pgfpathlineto{\pgfqpoint{3.204848in}{2.620752in}}%
\pgfpathlineto{\pgfqpoint{3.164768in}{2.658165in}}%
\pgfpathlineto{\pgfqpoint{3.036448in}{2.775523in}}%
\pgfpathlineto{\pgfqpoint{2.918636in}{2.880000in}}%
\pgfpathlineto{\pgfqpoint{2.838343in}{2.949285in}}%
\pgfpathlineto{\pgfqpoint{2.788285in}{2.992000in}}%
\pgfpathlineto{\pgfqpoint{2.683798in}{3.079155in}}%
\pgfpathlineto{\pgfqpoint{2.643717in}{3.111958in}}%
\pgfpathlineto{\pgfqpoint{2.523475in}{3.208102in}}%
\pgfpathlineto{\pgfqpoint{2.403232in}{3.300912in}}%
\pgfpathlineto{\pgfqpoint{2.317113in}{3.365333in}}%
\pgfpathlineto{\pgfqpoint{2.082586in}{3.531377in}}%
\pgfpathlineto{\pgfqpoint{1.995690in}{3.589333in}}%
\pgfpathlineto{\pgfqpoint{1.802020in}{3.710672in}}%
\pgfpathlineto{\pgfqpoint{1.734115in}{3.750083in}}%
\pgfpathlineto{\pgfqpoint{1.689247in}{3.776000in}}%
\pgfpathlineto{\pgfqpoint{1.621053in}{3.813333in}}%
\pgfpathlineto{\pgfqpoint{1.549745in}{3.850667in}}%
\pgfpathlineto{\pgfqpoint{1.401212in}{3.921907in}}%
\pgfpathlineto{\pgfqpoint{1.393340in}{3.925333in}}%
\pgfpathlineto{\pgfqpoint{1.321051in}{3.956109in}}%
\pgfpathlineto{\pgfqpoint{1.240889in}{3.986964in}}%
\pgfpathlineto{\pgfqpoint{1.200808in}{4.001191in}}%
\pgfpathlineto{\pgfqpoint{1.120646in}{4.026103in}}%
\pgfpathlineto{\pgfqpoint{1.079496in}{4.037333in}}%
\pgfpathlineto{\pgfqpoint{1.000404in}{4.054527in}}%
\pgfpathlineto{\pgfqpoint{0.911246in}{4.066287in}}%
\pgfpathlineto{\pgfqpoint{0.880162in}{4.068890in}}%
\pgfpathlineto{\pgfqpoint{0.833926in}{4.068934in}}%
\pgfpathlineto{\pgfqpoint{0.800000in}{4.067431in}}%
\pgfpathlineto{\pgfqpoint{0.800000in}{4.051066in}}%
\pgfpathlineto{\pgfqpoint{0.815480in}{4.051752in}}%
\pgfpathlineto{\pgfqpoint{0.840081in}{4.054033in}}%
\pgfpathlineto{\pgfqpoint{0.880162in}{4.054219in}}%
\pgfpathlineto{\pgfqpoint{0.896795in}{4.052827in}}%
\pgfpathlineto{\pgfqpoint{0.920242in}{4.052035in}}%
\pgfpathlineto{\pgfqpoint{0.960323in}{4.047813in}}%
\pgfpathlineto{\pgfqpoint{0.969687in}{4.046055in}}%
\pgfpathlineto{\pgfqpoint{1.000404in}{4.041828in}}%
\pgfpathlineto{\pgfqpoint{1.044841in}{4.033276in}}%
\pgfpathlineto{\pgfqpoint{1.120646in}{4.014415in}}%
\pgfpathlineto{\pgfqpoint{1.169688in}{4.000000in}}%
\pgfpathlineto{\pgfqpoint{1.277792in}{3.962667in}}%
\pgfpathlineto{\pgfqpoint{1.321051in}{3.945654in}}%
\pgfpathlineto{\pgfqpoint{1.335862in}{3.939130in}}%
\pgfpathlineto{\pgfqpoint{1.370112in}{3.925333in}}%
\pgfpathlineto{\pgfqpoint{1.481374in}{3.874895in}}%
\pgfpathlineto{\pgfqpoint{1.550552in}{3.840436in}}%
\pgfpathlineto{\pgfqpoint{1.576361in}{3.827143in}}%
\pgfpathlineto{\pgfqpoint{1.603940in}{3.813333in}}%
\pgfpathlineto{\pgfqpoint{1.777554in}{3.715877in}}%
\pgfpathlineto{\pgfqpoint{1.803125in}{3.701333in}}%
\pgfpathlineto{\pgfqpoint{1.924611in}{3.626667in}}%
\pgfpathlineto{\pgfqpoint{2.049665in}{3.545331in}}%
\pgfpathlineto{\pgfqpoint{2.162747in}{3.468012in}}%
\pgfpathlineto{\pgfqpoint{2.203091in}{3.439755in}}%
\pgfpathlineto{\pgfqpoint{2.323071in}{3.352859in}}%
\pgfpathlineto{\pgfqpoint{2.406316in}{3.290667in}}%
\pgfpathlineto{\pgfqpoint{2.492289in}{3.224285in}}%
\pgfpathlineto{\pgfqpoint{2.550723in}{3.178667in}}%
\pgfpathlineto{\pgfqpoint{2.779039in}{2.992000in}}%
\pgfpathlineto{\pgfqpoint{2.866571in}{2.917333in}}%
\pgfpathlineto{\pgfqpoint{2.964364in}{2.831958in}}%
\pgfpathlineto{\pgfqpoint{3.084606in}{2.724035in}}%
\pgfpathlineto{\pgfqpoint{3.204848in}{2.612844in}}%
\pgfpathlineto{\pgfqpoint{3.261185in}{2.559141in}}%
\pgfpathlineto{\pgfqpoint{3.316431in}{2.506667in}}%
\pgfpathlineto{\pgfqpoint{3.405253in}{2.420100in}}%
\pgfpathlineto{\pgfqpoint{3.525495in}{2.299900in}}%
\pgfpathlineto{\pgfqpoint{3.651032in}{2.170667in}}%
\pgfpathlineto{\pgfqpoint{3.765980in}{2.048647in}}%
\pgfpathlineto{\pgfqpoint{3.893261in}{1.909333in}}%
\pgfpathlineto{\pgfqpoint{3.992496in}{1.797333in}}%
\pgfpathlineto{\pgfqpoint{4.046545in}{1.735095in}}%
\pgfpathlineto{\pgfqpoint{4.126707in}{1.640999in}}%
\pgfpathlineto{\pgfqpoint{4.183100in}{1.573333in}}%
\pgfpathlineto{\pgfqpoint{4.274459in}{1.461333in}}%
\pgfpathlineto{\pgfqpoint{4.313810in}{1.411610in}}%
\pgfpathlineto{\pgfqpoint{4.363224in}{1.349333in}}%
\pgfpathlineto{\pgfqpoint{4.449243in}{1.237333in}}%
\pgfpathlineto{\pgfqpoint{4.513918in}{1.150002in}}%
\pgfpathlineto{\pgfqpoint{4.559536in}{1.088000in}}%
\pgfpathlineto{\pgfqpoint{4.638993in}{0.976000in}}%
\pgfpathlineto{\pgfqpoint{4.664781in}{0.938667in}}%
\pgfpathlineto{\pgfqpoint{4.727919in}{0.845353in}}%
\pgfpathlineto{\pgfqpoint{4.768000in}{0.784499in}}%
\pgfpathlineto{\pgfqpoint{4.768000in}{0.789333in}}%
\pgfpathlineto{\pgfqpoint{4.768000in}{0.789333in}}%
\pgfusepath{fill}%
\end{pgfscope}%
\begin{pgfscope}%
\pgfpathrectangle{\pgfqpoint{0.800000in}{0.528000in}}{\pgfqpoint{3.968000in}{3.696000in}}%
\pgfusepath{clip}%
\pgfsetbuttcap%
\pgfsetroundjoin%
\definecolor{currentfill}{rgb}{0.283187,0.125848,0.444960}%
\pgfsetfillcolor{currentfill}%
\pgfsetlinewidth{0.000000pt}%
\definecolor{currentstroke}{rgb}{0.000000,0.000000,0.000000}%
\pgfsetstrokecolor{currentstroke}%
\pgfsetdash{}{0pt}%
\pgfpathmoveto{\pgfqpoint{2.706155in}{0.528000in}}%
\pgfpathlineto{\pgfqpoint{2.585204in}{0.640000in}}%
\pgfpathlineto{\pgfqpoint{2.467676in}{0.752000in}}%
\pgfpathlineto{\pgfqpoint{2.353346in}{0.864000in}}%
\pgfpathlineto{\pgfqpoint{2.300310in}{0.917466in}}%
\pgfpathlineto{\pgfqpoint{2.242006in}{0.976000in}}%
\pgfpathlineto{\pgfqpoint{2.202828in}{1.016200in}}%
\pgfpathlineto{\pgfqpoint{2.082586in}{1.142047in}}%
\pgfpathlineto{\pgfqpoint{1.959486in}{1.274667in}}%
\pgfpathlineto{\pgfqpoint{1.887599in}{1.354379in}}%
\pgfpathlineto{\pgfqpoint{1.842101in}{1.405329in}}%
\pgfpathlineto{\pgfqpoint{1.792838in}{1.461333in}}%
\pgfpathlineto{\pgfqpoint{1.696374in}{1.573333in}}%
\pgfpathlineto{\pgfqpoint{1.601616in}{1.686240in}}%
\pgfpathlineto{\pgfqpoint{1.481006in}{1.835009in}}%
\pgfpathlineto{\pgfqpoint{1.364976in}{1.984000in}}%
\pgfpathlineto{\pgfqpoint{1.280810in}{2.096000in}}%
\pgfpathlineto{\pgfqpoint{1.226423in}{2.170667in}}%
\pgfpathlineto{\pgfqpoint{1.160727in}{2.262979in}}%
\pgfpathlineto{\pgfqpoint{1.021200in}{2.469333in}}%
\pgfpathlineto{\pgfqpoint{0.991802in}{2.514679in}}%
\pgfpathlineto{\pgfqpoint{0.920242in}{2.629453in}}%
\pgfpathlineto{\pgfqpoint{0.860127in}{2.730667in}}%
\pgfpathlineto{\pgfqpoint{0.817659in}{2.805333in}}%
\pgfpathlineto{\pgfqpoint{0.800000in}{2.837116in}}%
\pgfpathlineto{\pgfqpoint{0.800000in}{2.821392in}}%
\pgfpathlineto{\pgfqpoint{0.895838in}{2.656000in}}%
\pgfpathlineto{\pgfqpoint{0.941640in}{2.581333in}}%
\pgfpathlineto{\pgfqpoint{0.971672in}{2.533429in}}%
\pgfpathlineto{\pgfqpoint{1.040485in}{2.427505in}}%
\pgfpathlineto{\pgfqpoint{1.200808in}{2.195360in}}%
\pgfpathlineto{\pgfqpoint{1.280970in}{2.085457in}}%
\pgfpathlineto{\pgfqpoint{1.357282in}{1.984000in}}%
\pgfpathlineto{\pgfqpoint{1.451963in}{1.862061in}}%
\pgfpathlineto{\pgfqpoint{1.564065in}{1.722667in}}%
\pgfpathlineto{\pgfqpoint{1.657182in}{1.610667in}}%
\pgfpathlineto{\pgfqpoint{1.761939in}{1.488063in}}%
\pgfpathlineto{\pgfqpoint{1.884235in}{1.349333in}}%
\pgfpathlineto{\pgfqpoint{1.956762in}{1.269468in}}%
\pgfpathlineto{\pgfqpoint{2.002424in}{1.219647in}}%
\pgfpathlineto{\pgfqpoint{2.055462in}{1.162667in}}%
\pgfpathlineto{\pgfqpoint{2.162747in}{1.049535in}}%
\pgfpathlineto{\pgfqpoint{2.282990in}{0.926471in}}%
\pgfpathlineto{\pgfqpoint{2.354098in}{0.855567in}}%
\pgfpathlineto{\pgfqpoint{2.403232in}{0.806830in}}%
\pgfpathlineto{\pgfqpoint{2.459528in}{0.752000in}}%
\pgfpathlineto{\pgfqpoint{2.576868in}{0.640000in}}%
\pgfpathlineto{\pgfqpoint{2.656977in}{0.565333in}}%
\pgfpathlineto{\pgfqpoint{2.697623in}{0.528000in}}%
\pgfpathmoveto{\pgfqpoint{4.768000in}{0.809206in}}%
\pgfpathlineto{\pgfqpoint{4.628553in}{1.013333in}}%
\pgfpathlineto{\pgfqpoint{4.547957in}{1.125333in}}%
\pgfpathlineto{\pgfqpoint{4.506734in}{1.180644in}}%
\pgfpathlineto{\pgfqpoint{4.464582in}{1.237333in}}%
\pgfpathlineto{\pgfqpoint{4.378521in}{1.349333in}}%
\pgfpathlineto{\pgfqpoint{4.287030in}{1.464848in}}%
\pgfpathlineto{\pgfqpoint{4.219304in}{1.547583in}}%
\pgfpathlineto{\pgfqpoint{4.166788in}{1.611495in}}%
\pgfpathlineto{\pgfqpoint{4.104388in}{1.685333in}}%
\pgfpathlineto{\pgfqpoint{4.006465in}{1.799007in}}%
\pgfpathlineto{\pgfqpoint{3.934737in}{1.879855in}}%
\pgfpathlineto{\pgfqpoint{3.886222in}{1.934197in}}%
\pgfpathlineto{\pgfqpoint{3.772200in}{2.058667in}}%
\pgfpathlineto{\pgfqpoint{3.685818in}{2.150522in}}%
\pgfpathlineto{\pgfqpoint{3.558346in}{2.282667in}}%
\pgfpathlineto{\pgfqpoint{3.445333in}{2.396497in}}%
\pgfpathlineto{\pgfqpoint{3.325091in}{2.514137in}}%
\pgfpathlineto{\pgfqpoint{3.254829in}{2.581333in}}%
\pgfpathlineto{\pgfqpoint{3.164768in}{2.665888in}}%
\pgfpathlineto{\pgfqpoint{3.044525in}{2.775948in}}%
\pgfpathlineto{\pgfqpoint{2.924283in}{2.882853in}}%
\pgfpathlineto{\pgfqpoint{2.863366in}{2.935259in}}%
\pgfpathlineto{\pgfqpoint{2.804040in}{2.986499in}}%
\pgfpathlineto{\pgfqpoint{2.753182in}{3.029333in}}%
\pgfpathlineto{\pgfqpoint{2.643717in}{3.119712in}}%
\pgfpathlineto{\pgfqpoint{2.523475in}{3.216042in}}%
\pgfpathlineto{\pgfqpoint{2.435860in}{3.283725in}}%
\pgfpathlineto{\pgfqpoint{2.377812in}{3.328000in}}%
\pgfpathlineto{\pgfqpoint{2.277105in}{3.402667in}}%
\pgfpathlineto{\pgfqpoint{2.162747in}{3.484372in}}%
\pgfpathlineto{\pgfqpoint{2.119181in}{3.514667in}}%
\pgfpathlineto{\pgfqpoint{2.002424in}{3.593315in}}%
\pgfpathlineto{\pgfqpoint{1.909931in}{3.652514in}}%
\pgfpathlineto{\pgfqpoint{1.882182in}{3.670369in}}%
\pgfpathlineto{\pgfqpoint{1.831854in}{3.701333in}}%
\pgfpathlineto{\pgfqpoint{1.761939in}{3.743210in}}%
\pgfpathlineto{\pgfqpoint{1.561535in}{3.854066in}}%
\pgfpathlineto{\pgfqpoint{1.387809in}{3.937818in}}%
\pgfpathlineto{\pgfqpoint{1.313797in}{3.969423in}}%
\pgfpathlineto{\pgfqpoint{1.234665in}{4.000000in}}%
\pgfpathlineto{\pgfqpoint{1.120646in}{4.037771in}}%
\pgfpathlineto{\pgfqpoint{1.040485in}{4.058614in}}%
\pgfpathlineto{\pgfqpoint{0.993443in}{4.068183in}}%
\pgfpathlineto{\pgfqpoint{0.958306in}{4.074667in}}%
\pgfpathlineto{\pgfqpoint{0.914086in}{4.080401in}}%
\pgfpathlineto{\pgfqpoint{0.880162in}{4.083093in}}%
\pgfpathlineto{\pgfqpoint{0.829529in}{4.084496in}}%
\pgfpathlineto{\pgfqpoint{0.800000in}{4.083263in}}%
\pgfpathlineto{\pgfqpoint{0.800000in}{4.067431in}}%
\pgfpathlineto{\pgfqpoint{0.845463in}{4.069653in}}%
\pgfpathlineto{\pgfqpoint{0.880162in}{4.068890in}}%
\pgfpathlineto{\pgfqpoint{0.960323in}{4.061107in}}%
\pgfpathlineto{\pgfqpoint{1.048057in}{4.044387in}}%
\pgfpathlineto{\pgfqpoint{1.080932in}{4.036992in}}%
\pgfpathlineto{\pgfqpoint{1.160727in}{4.014100in}}%
\pgfpathlineto{\pgfqpoint{1.171664in}{4.010187in}}%
\pgfpathlineto{\pgfqpoint{1.204173in}{4.000000in}}%
\pgfpathlineto{\pgfqpoint{1.280970in}{3.971977in}}%
\pgfpathlineto{\pgfqpoint{1.361131in}{3.939315in}}%
\pgfpathlineto{\pgfqpoint{1.441293in}{3.903536in}}%
\pgfpathlineto{\pgfqpoint{1.474330in}{3.888000in}}%
\pgfpathlineto{\pgfqpoint{1.521455in}{3.864931in}}%
\pgfpathlineto{\pgfqpoint{1.582781in}{3.833123in}}%
\pgfpathlineto{\pgfqpoint{1.621053in}{3.813333in}}%
\pgfpathlineto{\pgfqpoint{1.761939in}{3.734418in}}%
\pgfpathlineto{\pgfqpoint{1.856369in}{3.677290in}}%
\pgfpathlineto{\pgfqpoint{1.882182in}{3.661869in}}%
\pgfpathlineto{\pgfqpoint{1.975794in}{3.601862in}}%
\pgfpathlineto{\pgfqpoint{2.022805in}{3.570984in}}%
\pgfpathlineto{\pgfqpoint{2.052017in}{3.552000in}}%
\pgfpathlineto{\pgfqpoint{2.162747in}{3.476302in}}%
\pgfpathlineto{\pgfqpoint{2.242909in}{3.419326in}}%
\pgfpathlineto{\pgfqpoint{2.367358in}{3.328000in}}%
\pgfpathlineto{\pgfqpoint{2.483394in}{3.239380in}}%
\pgfpathlineto{\pgfqpoint{2.523475in}{3.208102in}}%
\pgfpathlineto{\pgfqpoint{2.643717in}{3.111958in}}%
\pgfpathlineto{\pgfqpoint{2.763960in}{3.012520in}}%
\pgfpathlineto{\pgfqpoint{2.884202in}{2.909964in}}%
\pgfpathlineto{\pgfqpoint{2.941931in}{2.859105in}}%
\pgfpathlineto{\pgfqpoint{3.004444in}{2.804235in}}%
\pgfpathlineto{\pgfqpoint{3.126732in}{2.693333in}}%
\pgfpathlineto{\pgfqpoint{3.246664in}{2.581333in}}%
\pgfpathlineto{\pgfqpoint{3.304920in}{2.525212in}}%
\pgfpathlineto{\pgfqpoint{3.365172in}{2.467437in}}%
\pgfpathlineto{\pgfqpoint{3.485414in}{2.348436in}}%
\pgfpathlineto{\pgfqpoint{3.550387in}{2.282667in}}%
\pgfpathlineto{\pgfqpoint{3.658819in}{2.170667in}}%
\pgfpathlineto{\pgfqpoint{3.765980in}{2.057109in}}%
\pgfpathlineto{\pgfqpoint{3.886222in}{1.925681in}}%
\pgfpathlineto{\pgfqpoint{3.948611in}{1.855445in}}%
\pgfpathlineto{\pgfqpoint{4.000236in}{1.797333in}}%
\pgfpathlineto{\pgfqpoint{4.074356in}{1.711237in}}%
\pgfpathlineto{\pgfqpoint{4.128523in}{1.648000in}}%
\pgfpathlineto{\pgfqpoint{4.190761in}{1.573333in}}%
\pgfpathlineto{\pgfqpoint{4.287030in}{1.455323in}}%
\pgfpathlineto{\pgfqpoint{4.399873in}{1.312000in}}%
\pgfpathlineto{\pgfqpoint{4.487434in}{1.196832in}}%
\pgfpathlineto{\pgfqpoint{4.540201in}{1.125333in}}%
\pgfpathlineto{\pgfqpoint{4.607677in}{1.031746in}}%
\pgfpathlineto{\pgfqpoint{4.687838in}{0.916737in}}%
\pgfpathlineto{\pgfqpoint{4.768000in}{0.796964in}}%
\pgfpathlineto{\pgfqpoint{4.768000in}{0.796964in}}%
\pgfusepath{fill}%
\end{pgfscope}%
\begin{pgfscope}%
\pgfpathrectangle{\pgfqpoint{0.800000in}{0.528000in}}{\pgfqpoint{3.968000in}{3.696000in}}%
\pgfusepath{clip}%
\pgfsetbuttcap%
\pgfsetroundjoin%
\definecolor{currentfill}{rgb}{0.283187,0.125848,0.444960}%
\pgfsetfillcolor{currentfill}%
\pgfsetlinewidth{0.000000pt}%
\definecolor{currentstroke}{rgb}{0.000000,0.000000,0.000000}%
\pgfsetstrokecolor{currentstroke}%
\pgfsetdash{}{0pt}%
\pgfpathmoveto{\pgfqpoint{2.697623in}{0.528000in}}%
\pgfpathlineto{\pgfqpoint{2.603636in}{0.614865in}}%
\pgfpathlineto{\pgfqpoint{2.537387in}{0.677333in}}%
\pgfpathlineto{\pgfqpoint{2.443313in}{0.767699in}}%
\pgfpathlineto{\pgfqpoint{2.383086in}{0.826667in}}%
\pgfpathlineto{\pgfqpoint{2.270945in}{0.938667in}}%
\pgfpathlineto{\pgfqpoint{2.152557in}{1.060158in}}%
\pgfpathlineto{\pgfqpoint{2.042505in}{1.176502in}}%
\pgfpathlineto{\pgfqpoint{1.917892in}{1.312000in}}%
\pgfpathlineto{\pgfqpoint{1.882182in}{1.351624in}}%
\pgfpathlineto{\pgfqpoint{1.761939in}{1.488063in}}%
\pgfpathlineto{\pgfqpoint{1.561535in}{1.725761in}}%
\pgfpathlineto{\pgfqpoint{1.441293in}{1.875616in}}%
\pgfpathlineto{\pgfqpoint{1.357282in}{1.984000in}}%
\pgfpathlineto{\pgfqpoint{1.292946in}{2.069822in}}%
\pgfpathlineto{\pgfqpoint{1.245673in}{2.133333in}}%
\pgfpathlineto{\pgfqpoint{1.160727in}{2.251682in}}%
\pgfpathlineto{\pgfqpoint{1.000404in}{2.488817in}}%
\pgfpathlineto{\pgfqpoint{0.941640in}{2.581333in}}%
\pgfpathlineto{\pgfqpoint{0.880162in}{2.682076in}}%
\pgfpathlineto{\pgfqpoint{0.830036in}{2.768000in}}%
\pgfpathlineto{\pgfqpoint{0.800000in}{2.821392in}}%
\pgfpathlineto{\pgfqpoint{0.800387in}{2.804973in}}%
\pgfpathlineto{\pgfqpoint{0.865121in}{2.693333in}}%
\pgfpathlineto{\pgfqpoint{0.920242in}{2.602488in}}%
\pgfpathlineto{\pgfqpoint{0.980817in}{2.506667in}}%
\pgfpathlineto{\pgfqpoint{1.040485in}{2.415599in}}%
\pgfpathlineto{\pgfqpoint{1.184011in}{2.208000in}}%
\pgfpathlineto{\pgfqpoint{1.265499in}{2.096000in}}%
\pgfpathlineto{\pgfqpoint{1.349710in}{1.984000in}}%
\pgfpathlineto{\pgfqpoint{1.441293in}{1.865997in}}%
\pgfpathlineto{\pgfqpoint{1.556525in}{1.722667in}}%
\pgfpathlineto{\pgfqpoint{1.618328in}{1.648000in}}%
\pgfpathlineto{\pgfqpoint{1.721859in}{1.525702in}}%
\pgfpathlineto{\pgfqpoint{1.788776in}{1.448997in}}%
\pgfpathlineto{\pgfqpoint{1.843290in}{1.386667in}}%
\pgfpathlineto{\pgfqpoint{1.944266in}{1.274667in}}%
\pgfpathlineto{\pgfqpoint{2.012937in}{1.200000in}}%
\pgfpathlineto{\pgfqpoint{2.122667in}{1.083311in}}%
\pgfpathlineto{\pgfqpoint{2.242909in}{0.959142in}}%
\pgfpathlineto{\pgfqpoint{2.363152in}{0.838413in}}%
\pgfpathlineto{\pgfqpoint{2.490066in}{0.714667in}}%
\pgfpathlineto{\pgfqpoint{2.546418in}{0.661371in}}%
\pgfpathlineto{\pgfqpoint{2.606015in}{0.604883in}}%
\pgfpathlineto{\pgfqpoint{2.648511in}{0.565333in}}%
\pgfpathlineto{\pgfqpoint{2.689090in}{0.528000in}}%
\pgfpathmoveto{\pgfqpoint{4.768000in}{0.821448in}}%
\pgfpathlineto{\pgfqpoint{4.647758in}{0.997238in}}%
\pgfpathlineto{\pgfqpoint{4.487434in}{1.217246in}}%
\pgfpathlineto{\pgfqpoint{4.287030in}{1.474163in}}%
\pgfpathlineto{\pgfqpoint{4.223556in}{1.551543in}}%
\pgfpathlineto{\pgfqpoint{4.174954in}{1.610667in}}%
\pgfpathlineto{\pgfqpoint{4.080123in}{1.722667in}}%
\pgfpathlineto{\pgfqpoint{3.982609in}{1.834667in}}%
\pgfpathlineto{\pgfqpoint{3.882644in}{1.946667in}}%
\pgfpathlineto{\pgfqpoint{3.846141in}{1.986817in}}%
\pgfpathlineto{\pgfqpoint{3.725899in}{2.116323in}}%
\pgfpathlineto{\pgfqpoint{3.602641in}{2.245333in}}%
\pgfpathlineto{\pgfqpoint{3.565576in}{2.283388in}}%
\pgfpathlineto{\pgfqpoint{3.445333in}{2.404326in}}%
\pgfpathlineto{\pgfqpoint{3.325091in}{2.521920in}}%
\pgfpathlineto{\pgfqpoint{3.262993in}{2.581333in}}%
\pgfpathlineto{\pgfqpoint{3.164768in}{2.673611in}}%
\pgfpathlineto{\pgfqpoint{3.044525in}{2.783627in}}%
\pgfpathlineto{\pgfqpoint{2.924283in}{2.890488in}}%
\pgfpathlineto{\pgfqpoint{2.867672in}{2.939270in}}%
\pgfpathlineto{\pgfqpoint{2.804040in}{2.994248in}}%
\pgfpathlineto{\pgfqpoint{2.683798in}{3.094694in}}%
\pgfpathlineto{\pgfqpoint{2.614153in}{3.151129in}}%
\pgfpathlineto{\pgfqpoint{2.563556in}{3.191994in}}%
\pgfpathlineto{\pgfqpoint{2.437381in}{3.290667in}}%
\pgfpathlineto{\pgfqpoint{2.323071in}{3.376818in}}%
\pgfpathlineto{\pgfqpoint{2.264671in}{3.419730in}}%
\pgfpathlineto{\pgfqpoint{2.162747in}{3.492410in}}%
\pgfpathlineto{\pgfqpoint{2.122667in}{3.520379in}}%
\pgfpathlineto{\pgfqpoint{2.020796in}{3.589333in}}%
\pgfpathlineto{\pgfqpoint{1.922263in}{3.653518in}}%
\pgfpathlineto{\pgfqpoint{1.867418in}{3.687582in}}%
\pgfpathlineto{\pgfqpoint{1.834517in}{3.708398in}}%
\pgfpathlineto{\pgfqpoint{1.720711in}{3.776000in}}%
\pgfpathlineto{\pgfqpoint{1.623284in}{3.830484in}}%
\pgfpathlineto{\pgfqpoint{1.521455in}{3.883847in}}%
\pgfpathlineto{\pgfqpoint{1.465744in}{3.910775in}}%
\pgfpathlineto{\pgfqpoint{1.436360in}{3.925333in}}%
\pgfpathlineto{\pgfqpoint{1.353635in}{3.962667in}}%
\pgfpathlineto{\pgfqpoint{1.240889in}{4.008278in}}%
\pgfpathlineto{\pgfqpoint{1.120646in}{4.048965in}}%
\pgfpathlineto{\pgfqpoint{1.000404in}{4.079682in}}%
\pgfpathlineto{\pgfqpoint{0.960323in}{4.087071in}}%
\pgfpathlineto{\pgfqpoint{0.880162in}{4.096992in}}%
\pgfpathlineto{\pgfqpoint{0.864402in}{4.097321in}}%
\pgfpathlineto{\pgfqpoint{0.840081in}{4.099023in}}%
\pgfpathlineto{\pgfqpoint{0.800000in}{4.098675in}}%
\pgfpathlineto{\pgfqpoint{0.800000in}{4.083263in}}%
\pgfpathlineto{\pgfqpoint{0.840081in}{4.084406in}}%
\pgfpathlineto{\pgfqpoint{0.849806in}{4.083726in}}%
\pgfpathlineto{\pgfqpoint{0.880162in}{4.083093in}}%
\pgfpathlineto{\pgfqpoint{0.924762in}{4.078877in}}%
\pgfpathlineto{\pgfqpoint{0.960323in}{4.074401in}}%
\pgfpathlineto{\pgfqpoint{1.040485in}{4.058614in}}%
\pgfpathlineto{\pgfqpoint{1.160727in}{4.025325in}}%
\pgfpathlineto{\pgfqpoint{1.313797in}{3.969423in}}%
\pgfpathlineto{\pgfqpoint{1.387809in}{3.937818in}}%
\pgfpathlineto{\pgfqpoint{1.441293in}{3.913297in}}%
\pgfpathlineto{\pgfqpoint{1.511703in}{3.878917in}}%
\pgfpathlineto{\pgfqpoint{1.537628in}{3.865731in}}%
\pgfpathlineto{\pgfqpoint{1.567994in}{3.850667in}}%
\pgfpathlineto{\pgfqpoint{1.721859in}{3.766423in}}%
\pgfpathlineto{\pgfqpoint{1.789127in}{3.726657in}}%
\pgfpathlineto{\pgfqpoint{1.831854in}{3.701333in}}%
\pgfpathlineto{\pgfqpoint{1.892287in}{3.664000in}}%
\pgfpathlineto{\pgfqpoint{2.082586in}{3.539634in}}%
\pgfpathlineto{\pgfqpoint{2.172725in}{3.477333in}}%
\pgfpathlineto{\pgfqpoint{2.403232in}{3.308806in}}%
\pgfpathlineto{\pgfqpoint{2.523528in}{3.216000in}}%
\pgfpathlineto{\pgfqpoint{2.643717in}{3.119712in}}%
\pgfpathlineto{\pgfqpoint{2.763960in}{3.020320in}}%
\pgfpathlineto{\pgfqpoint{2.884734in}{2.917333in}}%
\pgfpathlineto{\pgfqpoint{2.946212in}{2.863093in}}%
\pgfpathlineto{\pgfqpoint{3.004444in}{2.811931in}}%
\pgfpathlineto{\pgfqpoint{3.053306in}{2.768000in}}%
\pgfpathlineto{\pgfqpoint{3.175380in}{2.656000in}}%
\pgfpathlineto{\pgfqpoint{3.244929in}{2.590734in}}%
\pgfpathlineto{\pgfqpoint{3.371272in}{2.469333in}}%
\pgfpathlineto{\pgfqpoint{3.485414in}{2.356519in}}%
\pgfpathlineto{\pgfqpoint{3.525495in}{2.316100in}}%
\pgfpathlineto{\pgfqpoint{3.645737in}{2.192507in}}%
\pgfpathlineto{\pgfqpoint{3.772200in}{2.058667in}}%
\pgfpathlineto{\pgfqpoint{3.886222in}{1.934197in}}%
\pgfpathlineto{\pgfqpoint{3.952827in}{1.859372in}}%
\pgfpathlineto{\pgfqpoint{4.007928in}{1.797333in}}%
\pgfpathlineto{\pgfqpoint{4.104388in}{1.685333in}}%
\pgfpathlineto{\pgfqpoint{4.149319in}{1.631728in}}%
\pgfpathlineto{\pgfqpoint{4.198423in}{1.573333in}}%
\pgfpathlineto{\pgfqpoint{4.298293in}{1.450843in}}%
\pgfpathlineto{\pgfqpoint{4.407640in}{1.312000in}}%
\pgfpathlineto{\pgfqpoint{4.473822in}{1.224654in}}%
\pgfpathlineto{\pgfqpoint{4.520538in}{1.162667in}}%
\pgfpathlineto{\pgfqpoint{4.547957in}{1.125333in}}%
\pgfpathlineto{\pgfqpoint{4.628553in}{1.013333in}}%
\pgfpathlineto{\pgfqpoint{4.654855in}{0.976000in}}%
\pgfpathlineto{\pgfqpoint{4.706277in}{0.901333in}}%
\pgfpathlineto{\pgfqpoint{4.768000in}{0.809206in}}%
\pgfpathlineto{\pgfqpoint{4.768000in}{0.809206in}}%
\pgfusepath{fill}%
\end{pgfscope}%
\begin{pgfscope}%
\pgfpathrectangle{\pgfqpoint{0.800000in}{0.528000in}}{\pgfqpoint{3.968000in}{3.696000in}}%
\pgfusepath{clip}%
\pgfsetbuttcap%
\pgfsetroundjoin%
\definecolor{currentfill}{rgb}{0.283187,0.125848,0.444960}%
\pgfsetfillcolor{currentfill}%
\pgfsetlinewidth{0.000000pt}%
\definecolor{currentstroke}{rgb}{0.000000,0.000000,0.000000}%
\pgfsetstrokecolor{currentstroke}%
\pgfsetdash{}{0pt}%
\pgfpathmoveto{\pgfqpoint{2.689090in}{0.528000in}}%
\pgfpathlineto{\pgfqpoint{2.603636in}{0.607038in}}%
\pgfpathlineto{\pgfqpoint{2.546418in}{0.661371in}}%
\pgfpathlineto{\pgfqpoint{2.490066in}{0.714667in}}%
\pgfpathlineto{\pgfqpoint{2.403232in}{0.798925in}}%
\pgfpathlineto{\pgfqpoint{2.282990in}{0.918520in}}%
\pgfpathlineto{\pgfqpoint{2.153973in}{1.050667in}}%
\pgfpathlineto{\pgfqpoint{2.042505in}{1.168202in}}%
\pgfpathlineto{\pgfqpoint{1.922263in}{1.298817in}}%
\pgfpathlineto{\pgfqpoint{1.810351in}{1.424000in}}%
\pgfpathlineto{\pgfqpoint{1.713020in}{1.536000in}}%
\pgfpathlineto{\pgfqpoint{1.618328in}{1.648000in}}%
\pgfpathlineto{\pgfqpoint{1.521455in}{1.765729in}}%
\pgfpathlineto{\pgfqpoint{1.407307in}{1.909333in}}%
\pgfpathlineto{\pgfqpoint{1.321051in}{2.021537in}}%
\pgfpathlineto{\pgfqpoint{1.255683in}{2.109780in}}%
\pgfpathlineto{\pgfqpoint{1.210830in}{2.170667in}}%
\pgfpathlineto{\pgfqpoint{1.131214in}{2.282667in}}%
\pgfpathlineto{\pgfqpoint{1.054509in}{2.394667in}}%
\pgfpathlineto{\pgfqpoint{1.029577in}{2.432000in}}%
\pgfpathlineto{\pgfqpoint{0.980817in}{2.506667in}}%
\pgfpathlineto{\pgfqpoint{0.920242in}{2.602488in}}%
\pgfpathlineto{\pgfqpoint{0.821543in}{2.768000in}}%
\pgfpathlineto{\pgfqpoint{0.800000in}{2.805668in}}%
\pgfpathlineto{\pgfqpoint{0.800000in}{2.805333in}}%
\pgfpathlineto{\pgfqpoint{0.800000in}{2.790807in}}%
\pgfpathlineto{\pgfqpoint{0.856795in}{2.693333in}}%
\pgfpathlineto{\pgfqpoint{0.908908in}{2.608110in}}%
\pgfpathlineto{\pgfqpoint{0.938485in}{2.560992in}}%
\pgfpathlineto{\pgfqpoint{0.972748in}{2.506667in}}%
\pgfpathlineto{\pgfqpoint{1.040485in}{2.403693in}}%
\pgfpathlineto{\pgfqpoint{1.176270in}{2.208000in}}%
\pgfpathlineto{\pgfqpoint{1.240889in}{2.118932in}}%
\pgfpathlineto{\pgfqpoint{1.441293in}{1.856509in}}%
\pgfpathlineto{\pgfqpoint{1.549065in}{1.722667in}}%
\pgfpathlineto{\pgfqpoint{1.601616in}{1.658977in}}%
\pgfpathlineto{\pgfqpoint{1.705538in}{1.536000in}}%
\pgfpathlineto{\pgfqpoint{1.761939in}{1.470664in}}%
\pgfpathlineto{\pgfqpoint{1.869161in}{1.349333in}}%
\pgfpathlineto{\pgfqpoint{1.970801in}{1.237333in}}%
\pgfpathlineto{\pgfqpoint{2.082586in}{1.117445in}}%
\pgfpathlineto{\pgfqpoint{2.202828in}{0.992163in}}%
\pgfpathlineto{\pgfqpoint{2.255238in}{0.938667in}}%
\pgfpathlineto{\pgfqpoint{2.367030in}{0.826667in}}%
\pgfpathlineto{\pgfqpoint{2.483394in}{0.713235in}}%
\pgfpathlineto{\pgfqpoint{2.542302in}{0.657536in}}%
\pgfpathlineto{\pgfqpoint{2.603636in}{0.599310in}}%
\pgfpathlineto{\pgfqpoint{2.643717in}{0.562042in}}%
\pgfpathlineto{\pgfqpoint{2.680670in}{0.528000in}}%
\pgfpathlineto{\pgfqpoint{2.683798in}{0.528000in}}%
\pgfpathmoveto{\pgfqpoint{4.768000in}{0.833379in}}%
\pgfpathlineto{\pgfqpoint{4.687838in}{0.951088in}}%
\pgfpathlineto{\pgfqpoint{4.607677in}{1.064413in}}%
\pgfpathlineto{\pgfqpoint{4.422705in}{1.312000in}}%
\pgfpathlineto{\pgfqpoint{4.327111in}{1.433657in}}%
\pgfpathlineto{\pgfqpoint{4.213532in}{1.573333in}}%
\pgfpathlineto{\pgfqpoint{4.119544in}{1.685333in}}%
\pgfpathlineto{\pgfqpoint{4.022920in}{1.797333in}}%
\pgfpathlineto{\pgfqpoint{3.923879in}{1.909333in}}%
\pgfpathlineto{\pgfqpoint{3.851522in}{1.989012in}}%
\pgfpathlineto{\pgfqpoint{3.806061in}{2.038606in}}%
\pgfpathlineto{\pgfqpoint{3.682183in}{2.170667in}}%
\pgfpathlineto{\pgfqpoint{3.565576in}{2.291263in}}%
\pgfpathlineto{\pgfqpoint{3.445333in}{2.412155in}}%
\pgfpathlineto{\pgfqpoint{3.325091in}{2.529703in}}%
\pgfpathlineto{\pgfqpoint{3.271157in}{2.581333in}}%
\pgfpathlineto{\pgfqpoint{3.151790in}{2.693333in}}%
\pgfpathlineto{\pgfqpoint{3.057042in}{2.779659in}}%
\pgfpathlineto{\pgfqpoint{3.004444in}{2.827258in}}%
\pgfpathlineto{\pgfqpoint{2.884202in}{2.933036in}}%
\pgfpathlineto{\pgfqpoint{2.809478in}{2.997065in}}%
\pgfpathlineto{\pgfqpoint{2.763960in}{3.035731in}}%
\pgfpathlineto{\pgfqpoint{2.636172in}{3.141333in}}%
\pgfpathlineto{\pgfqpoint{2.543017in}{3.216000in}}%
\pgfpathlineto{\pgfqpoint{2.443313in}{3.293946in}}%
\pgfpathlineto{\pgfqpoint{2.349045in}{3.365333in}}%
\pgfpathlineto{\pgfqpoint{2.226644in}{3.455150in}}%
\pgfpathlineto{\pgfqpoint{2.122667in}{3.528401in}}%
\pgfpathlineto{\pgfqpoint{2.033173in}{3.589333in}}%
\pgfpathlineto{\pgfqpoint{1.919045in}{3.664000in}}%
\pgfpathlineto{\pgfqpoint{1.798697in}{3.738667in}}%
\pgfpathlineto{\pgfqpoint{1.681778in}{3.806945in}}%
\pgfpathlineto{\pgfqpoint{1.600122in}{3.852059in}}%
\pgfpathlineto{\pgfqpoint{1.509348in}{3.899276in}}%
\pgfpathlineto{\pgfqpoint{1.425449in}{3.940091in}}%
\pgfpathlineto{\pgfqpoint{1.361131in}{3.969312in}}%
\pgfpathlineto{\pgfqpoint{1.226230in}{4.023679in}}%
\pgfpathlineto{\pgfqpoint{1.189911in}{4.037333in}}%
\pgfpathlineto{\pgfqpoint{1.108496in}{4.063349in}}%
\pgfpathlineto{\pgfqpoint{1.070621in}{4.074667in}}%
\pgfpathlineto{\pgfqpoint{1.040485in}{4.082559in}}%
\pgfpathlineto{\pgfqpoint{0.948714in}{4.101186in}}%
\pgfpathlineto{\pgfqpoint{0.920242in}{4.106167in}}%
\pgfpathlineto{\pgfqpoint{0.863508in}{4.112000in}}%
\pgfpathlineto{\pgfqpoint{0.838311in}{4.113648in}}%
\pgfpathlineto{\pgfqpoint{0.800000in}{4.113971in}}%
\pgfpathlineto{\pgfqpoint{0.800000in}{4.098675in}}%
\pgfpathlineto{\pgfqpoint{0.813692in}{4.099246in}}%
\pgfpathlineto{\pgfqpoint{0.840081in}{4.099023in}}%
\pgfpathlineto{\pgfqpoint{0.920242in}{4.092917in}}%
\pgfpathlineto{\pgfqpoint{0.936738in}{4.090032in}}%
\pgfpathlineto{\pgfqpoint{0.960323in}{4.087071in}}%
\pgfpathlineto{\pgfqpoint{1.004607in}{4.078581in}}%
\pgfpathlineto{\pgfqpoint{1.040485in}{4.070768in}}%
\pgfpathlineto{\pgfqpoint{1.120646in}{4.048965in}}%
\pgfpathlineto{\pgfqpoint{1.160727in}{4.036550in}}%
\pgfpathlineto{\pgfqpoint{1.280970in}{3.992833in}}%
\pgfpathlineto{\pgfqpoint{1.330997in}{3.971931in}}%
\pgfpathlineto{\pgfqpoint{1.367535in}{3.956702in}}%
\pgfpathlineto{\pgfqpoint{1.441293in}{3.923058in}}%
\pgfpathlineto{\pgfqpoint{1.623284in}{3.830484in}}%
\pgfpathlineto{\pgfqpoint{1.721859in}{3.775349in}}%
\pgfpathlineto{\pgfqpoint{1.922263in}{3.653518in}}%
\pgfpathlineto{\pgfqpoint{2.020796in}{3.589333in}}%
\pgfpathlineto{\pgfqpoint{2.202828in}{3.464108in}}%
\pgfpathlineto{\pgfqpoint{2.288028in}{3.402667in}}%
\pgfpathlineto{\pgfqpoint{2.388267in}{3.328000in}}%
\pgfpathlineto{\pgfqpoint{2.485748in}{3.253333in}}%
\pgfpathlineto{\pgfqpoint{2.571142in}{3.185733in}}%
\pgfpathlineto{\pgfqpoint{2.626600in}{3.141333in}}%
\pgfpathlineto{\pgfqpoint{2.723879in}{3.061579in}}%
\pgfpathlineto{\pgfqpoint{2.850314in}{2.954667in}}%
\pgfpathlineto{\pgfqpoint{2.924283in}{2.890488in}}%
\pgfpathlineto{\pgfqpoint{3.044525in}{2.783627in}}%
\pgfpathlineto{\pgfqpoint{3.102814in}{2.730667in}}%
\pgfpathlineto{\pgfqpoint{3.223518in}{2.618667in}}%
\pgfpathlineto{\pgfqpoint{3.313260in}{2.532980in}}%
\pgfpathlineto{\pgfqpoint{3.365172in}{2.483090in}}%
\pgfpathlineto{\pgfqpoint{3.417316in}{2.432000in}}%
\pgfpathlineto{\pgfqpoint{3.529517in}{2.320000in}}%
\pgfpathlineto{\pgfqpoint{3.645737in}{2.200656in}}%
\pgfpathlineto{\pgfqpoint{3.765980in}{2.073557in}}%
\pgfpathlineto{\pgfqpoint{3.847513in}{1.985277in}}%
\pgfpathlineto{\pgfqpoint{3.886222in}{1.942713in}}%
\pgfpathlineto{\pgfqpoint{3.957043in}{1.863300in}}%
\pgfpathlineto{\pgfqpoint{4.006465in}{1.807578in}}%
\pgfpathlineto{\pgfqpoint{4.111966in}{1.685333in}}%
\pgfpathlineto{\pgfqpoint{4.166788in}{1.620432in}}%
\pgfpathlineto{\pgfqpoint{4.246949in}{1.523516in}}%
\pgfpathlineto{\pgfqpoint{4.356842in}{1.386667in}}%
\pgfpathlineto{\pgfqpoint{4.394834in}{1.337747in}}%
\pgfpathlineto{\pgfqpoint{4.443937in}{1.274667in}}%
\pgfpathlineto{\pgfqpoint{4.487434in}{1.217246in}}%
\pgfpathlineto{\pgfqpoint{4.567596in}{1.109079in}}%
\pgfpathlineto{\pgfqpoint{4.647758in}{0.997238in}}%
\pgfpathlineto{\pgfqpoint{4.727919in}{0.881174in}}%
\pgfpathlineto{\pgfqpoint{4.768000in}{0.821448in}}%
\pgfpathlineto{\pgfqpoint{4.768000in}{0.826667in}}%
\pgfpathlineto{\pgfqpoint{4.768000in}{0.826667in}}%
\pgfusepath{fill}%
\end{pgfscope}%
\begin{pgfscope}%
\pgfpathrectangle{\pgfqpoint{0.800000in}{0.528000in}}{\pgfqpoint{3.968000in}{3.696000in}}%
\pgfusepath{clip}%
\pgfsetbuttcap%
\pgfsetroundjoin%
\definecolor{currentfill}{rgb}{0.283187,0.125848,0.444960}%
\pgfsetfillcolor{currentfill}%
\pgfsetlinewidth{0.000000pt}%
\definecolor{currentstroke}{rgb}{0.000000,0.000000,0.000000}%
\pgfsetstrokecolor{currentstroke}%
\pgfsetdash{}{0pt}%
\pgfpathmoveto{\pgfqpoint{2.680670in}{0.528000in}}%
\pgfpathlineto{\pgfqpoint{2.560310in}{0.640000in}}%
\pgfpathlineto{\pgfqpoint{2.502667in}{0.695285in}}%
\pgfpathlineto{\pgfqpoint{2.443234in}{0.752000in}}%
\pgfpathlineto{\pgfqpoint{2.403232in}{0.791020in}}%
\pgfpathlineto{\pgfqpoint{2.282990in}{0.910568in}}%
\pgfpathlineto{\pgfqpoint{2.162747in}{1.033536in}}%
\pgfpathlineto{\pgfqpoint{2.082586in}{1.117445in}}%
\pgfpathlineto{\pgfqpoint{1.970801in}{1.237333in}}%
\pgfpathlineto{\pgfqpoint{1.869161in}{1.349333in}}%
\pgfpathlineto{\pgfqpoint{1.761939in}{1.470664in}}%
\pgfpathlineto{\pgfqpoint{1.705538in}{1.536000in}}%
\pgfpathlineto{\pgfqpoint{1.601616in}{1.658977in}}%
\pgfpathlineto{\pgfqpoint{1.549065in}{1.722667in}}%
\pgfpathlineto{\pgfqpoint{1.458653in}{1.834667in}}%
\pgfpathlineto{\pgfqpoint{1.370787in}{1.946667in}}%
\pgfpathlineto{\pgfqpoint{1.333555in}{1.995647in}}%
\pgfpathlineto{\pgfqpoint{1.285550in}{2.058667in}}%
\pgfpathlineto{\pgfqpoint{1.234702in}{2.127570in}}%
\pgfpathlineto{\pgfqpoint{1.200808in}{2.173750in}}%
\pgfpathlineto{\pgfqpoint{1.120646in}{2.286519in}}%
\pgfpathlineto{\pgfqpoint{0.960323in}{2.525982in}}%
\pgfpathlineto{\pgfqpoint{0.851031in}{2.703533in}}%
\pgfpathlineto{\pgfqpoint{0.834643in}{2.730667in}}%
\pgfpathlineto{\pgfqpoint{0.800000in}{2.790807in}}%
\pgfpathlineto{\pgfqpoint{0.800000in}{2.775964in}}%
\pgfpathlineto{\pgfqpoint{0.848470in}{2.693333in}}%
\pgfpathlineto{\pgfqpoint{0.893916in}{2.618667in}}%
\pgfpathlineto{\pgfqpoint{0.933371in}{2.556229in}}%
\pgfpathlineto{\pgfqpoint{0.964678in}{2.506667in}}%
\pgfpathlineto{\pgfqpoint{1.013741in}{2.432000in}}%
\pgfpathlineto{\pgfqpoint{1.080566in}{2.333229in}}%
\pgfpathlineto{\pgfqpoint{1.230253in}{2.123427in}}%
\pgfpathlineto{\pgfqpoint{1.277939in}{2.058667in}}%
\pgfpathlineto{\pgfqpoint{1.368380in}{1.939915in}}%
\pgfpathlineto{\pgfqpoint{1.481374in}{1.796704in}}%
\pgfpathlineto{\pgfqpoint{1.603200in}{1.648000in}}%
\pgfpathlineto{\pgfqpoint{1.698055in}{1.536000in}}%
\pgfpathlineto{\pgfqpoint{1.744583in}{1.482500in}}%
\pgfpathlineto{\pgfqpoint{1.795287in}{1.424000in}}%
\pgfpathlineto{\pgfqpoint{1.861656in}{1.349333in}}%
\pgfpathlineto{\pgfqpoint{1.967946in}{1.232115in}}%
\pgfpathlineto{\pgfqpoint{2.082586in}{1.109413in}}%
\pgfpathlineto{\pgfqpoint{2.210816in}{0.976000in}}%
\pgfpathlineto{\pgfqpoint{2.323071in}{0.862452in}}%
\pgfpathlineto{\pgfqpoint{2.363152in}{0.822692in}}%
\pgfpathlineto{\pgfqpoint{2.483394in}{0.705589in}}%
\pgfpathlineto{\pgfqpoint{2.552256in}{0.640000in}}%
\pgfpathlineto{\pgfqpoint{2.643717in}{0.554454in}}%
\pgfpathlineto{\pgfqpoint{2.672433in}{0.528000in}}%
\pgfpathmoveto{\pgfqpoint{4.768000in}{0.845079in}}%
\pgfpathlineto{\pgfqpoint{4.687838in}{0.962231in}}%
\pgfpathlineto{\pgfqpoint{4.607677in}{1.075049in}}%
\pgfpathlineto{\pgfqpoint{4.447354in}{1.289851in}}%
\pgfpathlineto{\pgfqpoint{4.246949in}{1.541894in}}%
\pgfpathlineto{\pgfqpoint{4.189902in}{1.610667in}}%
\pgfpathlineto{\pgfqpoint{4.086626in}{1.732547in}}%
\pgfpathlineto{\pgfqpoint{3.964746in}{1.872000in}}%
\pgfpathlineto{\pgfqpoint{3.897689in}{1.946667in}}%
\pgfpathlineto{\pgfqpoint{3.795071in}{2.058667in}}%
\pgfpathlineto{\pgfqpoint{3.744024in}{2.112883in}}%
\pgfpathlineto{\pgfqpoint{3.685818in}{2.174891in}}%
\pgfpathlineto{\pgfqpoint{3.565576in}{2.299138in}}%
\pgfpathlineto{\pgfqpoint{3.433168in}{2.432000in}}%
\pgfpathlineto{\pgfqpoint{3.356941in}{2.506667in}}%
\pgfpathlineto{\pgfqpoint{3.239970in}{2.618667in}}%
\pgfpathlineto{\pgfqpoint{3.119648in}{2.730667in}}%
\pgfpathlineto{\pgfqpoint{3.044525in}{2.798983in}}%
\pgfpathlineto{\pgfqpoint{2.924283in}{2.905756in}}%
\pgfpathlineto{\pgfqpoint{2.855423in}{2.965194in}}%
\pgfpathlineto{\pgfqpoint{2.804040in}{3.009430in}}%
\pgfpathlineto{\pgfqpoint{2.683798in}{3.110055in}}%
\pgfpathlineto{\pgfqpoint{2.563556in}{3.207443in}}%
\pgfpathlineto{\pgfqpoint{2.443313in}{3.301627in}}%
\pgfpathlineto{\pgfqpoint{2.323071in}{3.392544in}}%
\pgfpathlineto{\pgfqpoint{2.282990in}{3.422071in}}%
\pgfpathlineto{\pgfqpoint{2.162747in}{3.508487in}}%
\pgfpathlineto{\pgfqpoint{2.122667in}{3.536423in}}%
\pgfpathlineto{\pgfqpoint{2.002424in}{3.617981in}}%
\pgfpathlineto{\pgfqpoint{1.962343in}{3.644267in}}%
\pgfpathlineto{\pgfqpoint{1.873076in}{3.701333in}}%
\pgfpathlineto{\pgfqpoint{1.761939in}{3.769169in}}%
\pgfpathlineto{\pgfqpoint{1.721859in}{3.792639in}}%
\pgfpathlineto{\pgfqpoint{1.641697in}{3.838157in}}%
\pgfpathlineto{\pgfqpoint{1.549033in}{3.888000in}}%
\pgfpathlineto{\pgfqpoint{1.397476in}{3.962667in}}%
\pgfpathlineto{\pgfqpoint{1.255989in}{4.023268in}}%
\pgfpathlineto{\pgfqpoint{1.200808in}{4.044104in}}%
\pgfpathlineto{\pgfqpoint{1.067101in}{4.087208in}}%
\pgfpathlineto{\pgfqpoint{1.000404in}{4.103916in}}%
\pgfpathlineto{\pgfqpoint{0.959818in}{4.112470in}}%
\pgfpathlineto{\pgfqpoint{0.920242in}{4.119063in}}%
\pgfpathlineto{\pgfqpoint{0.840081in}{4.127405in}}%
\pgfpathlineto{\pgfqpoint{0.800000in}{4.128535in}}%
\pgfpathlineto{\pgfqpoint{0.800000in}{4.113971in}}%
\pgfpathlineto{\pgfqpoint{0.841570in}{4.113387in}}%
\pgfpathlineto{\pgfqpoint{0.881462in}{4.110789in}}%
\pgfpathlineto{\pgfqpoint{0.920242in}{4.106167in}}%
\pgfpathlineto{\pgfqpoint{1.000404in}{4.091799in}}%
\pgfpathlineto{\pgfqpoint{1.014760in}{4.088039in}}%
\pgfpathlineto{\pgfqpoint{1.040485in}{4.082559in}}%
\pgfpathlineto{\pgfqpoint{1.080566in}{4.072055in}}%
\pgfpathlineto{\pgfqpoint{1.200808in}{4.033583in}}%
\pgfpathlineto{\pgfqpoint{1.321051in}{3.986532in}}%
\pgfpathlineto{\pgfqpoint{1.375896in}{3.962667in}}%
\pgfpathlineto{\pgfqpoint{1.456193in}{3.925333in}}%
\pgfpathlineto{\pgfqpoint{1.602688in}{3.850667in}}%
\pgfpathlineto{\pgfqpoint{1.699262in}{3.797048in}}%
\pgfpathlineto{\pgfqpoint{1.808059in}{3.733042in}}%
\pgfpathlineto{\pgfqpoint{1.929219in}{3.657520in}}%
\pgfpathlineto{\pgfqpoint{2.042505in}{3.583121in}}%
\pgfpathlineto{\pgfqpoint{2.088430in}{3.552000in}}%
\pgfpathlineto{\pgfqpoint{2.202828in}{3.472162in}}%
\pgfpathlineto{\pgfqpoint{2.298688in}{3.402667in}}%
\pgfpathlineto{\pgfqpoint{2.543017in}{3.216000in}}%
\pgfpathlineto{\pgfqpoint{2.643717in}{3.135221in}}%
\pgfpathlineto{\pgfqpoint{2.771537in}{3.029333in}}%
\pgfpathlineto{\pgfqpoint{2.902247in}{2.917333in}}%
\pgfpathlineto{\pgfqpoint{3.164768in}{2.681334in}}%
\pgfpathlineto{\pgfqpoint{3.285010in}{2.568153in}}%
\pgfpathlineto{\pgfqpoint{3.405253in}{2.451706in}}%
\pgfpathlineto{\pgfqpoint{3.473622in}{2.383683in}}%
\pgfpathlineto{\pgfqpoint{3.525495in}{2.331935in}}%
\pgfpathlineto{\pgfqpoint{3.608109in}{2.247618in}}%
\pgfpathlineto{\pgfqpoint{3.652745in}{2.201473in}}%
\pgfpathlineto{\pgfqpoint{3.765980in}{2.081757in}}%
\pgfpathlineto{\pgfqpoint{3.821975in}{2.021333in}}%
\pgfpathlineto{\pgfqpoint{3.926303in}{1.906630in}}%
\pgfpathlineto{\pgfqpoint{3.997382in}{1.826207in}}%
\pgfpathlineto{\pgfqpoint{4.046545in}{1.770261in}}%
\pgfpathlineto{\pgfqpoint{4.246949in}{1.532809in}}%
\pgfpathlineto{\pgfqpoint{4.367192in}{1.383224in}}%
\pgfpathlineto{\pgfqpoint{4.479921in}{1.237333in}}%
\pgfpathlineto{\pgfqpoint{4.515842in}{1.189127in}}%
\pgfpathlineto{\pgfqpoint{4.563467in}{1.125333in}}%
\pgfpathlineto{\pgfqpoint{4.647758in}{1.008350in}}%
\pgfpathlineto{\pgfqpoint{4.727919in}{0.892840in}}%
\pgfpathlineto{\pgfqpoint{4.768000in}{0.833379in}}%
\pgfpathlineto{\pgfqpoint{4.768000in}{0.833379in}}%
\pgfusepath{fill}%
\end{pgfscope}%
\begin{pgfscope}%
\pgfpathrectangle{\pgfqpoint{0.800000in}{0.528000in}}{\pgfqpoint{3.968000in}{3.696000in}}%
\pgfusepath{clip}%
\pgfsetbuttcap%
\pgfsetroundjoin%
\definecolor{currentfill}{rgb}{0.283072,0.130895,0.449241}%
\pgfsetfillcolor{currentfill}%
\pgfsetlinewidth{0.000000pt}%
\definecolor{currentstroke}{rgb}{0.000000,0.000000,0.000000}%
\pgfsetstrokecolor{currentstroke}%
\pgfsetdash{}{0pt}%
\pgfpathmoveto{\pgfqpoint{2.672433in}{0.528000in}}%
\pgfpathlineto{\pgfqpoint{2.552256in}{0.640000in}}%
\pgfpathlineto{\pgfqpoint{2.478666in}{0.710263in}}%
\pgfpathlineto{\pgfqpoint{2.435355in}{0.752000in}}%
\pgfpathlineto{\pgfqpoint{2.397080in}{0.789333in}}%
\pgfpathlineto{\pgfqpoint{2.282990in}{0.902616in}}%
\pgfpathlineto{\pgfqpoint{2.162747in}{1.025536in}}%
\pgfpathlineto{\pgfqpoint{2.102936in}{1.088000in}}%
\pgfpathlineto{\pgfqpoint{1.997656in}{1.200000in}}%
\pgfpathlineto{\pgfqpoint{1.882182in}{1.326442in}}%
\pgfpathlineto{\pgfqpoint{1.761939in}{1.461964in}}%
\pgfpathlineto{\pgfqpoint{1.698055in}{1.536000in}}%
\pgfpathlineto{\pgfqpoint{1.601616in}{1.649900in}}%
\pgfpathlineto{\pgfqpoint{1.541604in}{1.722667in}}%
\pgfpathlineto{\pgfqpoint{1.441293in}{1.847020in}}%
\pgfpathlineto{\pgfqpoint{1.240889in}{2.108578in}}%
\pgfpathlineto{\pgfqpoint{1.064074in}{2.357333in}}%
\pgfpathlineto{\pgfqpoint{1.035453in}{2.399354in}}%
\pgfpathlineto{\pgfqpoint{0.960323in}{2.513437in}}%
\pgfpathlineto{\pgfqpoint{0.826381in}{2.730667in}}%
\pgfpathlineto{\pgfqpoint{0.800000in}{2.775964in}}%
\pgfpathlineto{\pgfqpoint{0.800000in}{2.761487in}}%
\pgfpathlineto{\pgfqpoint{0.920242in}{2.563729in}}%
\pgfpathlineto{\pgfqpoint{1.056302in}{2.357333in}}%
\pgfpathlineto{\pgfqpoint{1.097408in}{2.298355in}}%
\pgfpathlineto{\pgfqpoint{1.134302in}{2.245333in}}%
\pgfpathlineto{\pgfqpoint{1.160910in}{2.207829in}}%
\pgfpathlineto{\pgfqpoint{1.246201in}{2.091052in}}%
\pgfpathlineto{\pgfqpoint{1.355711in}{1.946667in}}%
\pgfpathlineto{\pgfqpoint{1.450120in}{1.826444in}}%
\pgfpathlineto{\pgfqpoint{1.564780in}{1.685333in}}%
\pgfpathlineto{\pgfqpoint{1.633673in}{1.603192in}}%
\pgfpathlineto{\pgfqpoint{1.681778in}{1.546270in}}%
\pgfpathlineto{\pgfqpoint{1.787885in}{1.424000in}}%
\pgfpathlineto{\pgfqpoint{1.854151in}{1.349333in}}%
\pgfpathlineto{\pgfqpoint{1.962343in}{1.230090in}}%
\pgfpathlineto{\pgfqpoint{2.095303in}{1.088000in}}%
\pgfpathlineto{\pgfqpoint{2.162747in}{1.017536in}}%
\pgfpathlineto{\pgfqpoint{2.282990in}{0.894859in}}%
\pgfpathlineto{\pgfqpoint{2.403232in}{0.775620in}}%
\pgfpathlineto{\pgfqpoint{2.474684in}{0.706554in}}%
\pgfpathlineto{\pgfqpoint{2.523475in}{0.659641in}}%
\pgfpathlineto{\pgfqpoint{2.583827in}{0.602667in}}%
\pgfpathlineto{\pgfqpoint{2.664196in}{0.528000in}}%
\pgfpathmoveto{\pgfqpoint{4.768000in}{0.856780in}}%
\pgfpathlineto{\pgfqpoint{4.711921in}{0.938667in}}%
\pgfpathlineto{\pgfqpoint{4.647758in}{1.029878in}}%
\pgfpathlineto{\pgfqpoint{4.567596in}{1.140301in}}%
\pgfpathlineto{\pgfqpoint{4.367192in}{1.402061in}}%
\pgfpathlineto{\pgfqpoint{4.163810in}{1.650774in}}%
\pgfpathlineto{\pgfqpoint{4.037912in}{1.797333in}}%
\pgfpathlineto{\pgfqpoint{3.938785in}{1.909333in}}%
\pgfpathlineto{\pgfqpoint{3.837116in}{2.021333in}}%
\pgfpathlineto{\pgfqpoint{3.785395in}{2.076751in}}%
\pgfpathlineto{\pgfqpoint{3.732798in}{2.133333in}}%
\pgfpathlineto{\pgfqpoint{3.625679in}{2.245333in}}%
\pgfpathlineto{\pgfqpoint{3.539528in}{2.333071in}}%
\pgfpathlineto{\pgfqpoint{3.485414in}{2.387920in}}%
\pgfpathlineto{\pgfqpoint{3.423438in}{2.448939in}}%
\pgfpathlineto{\pgfqpoint{3.364984in}{2.506667in}}%
\pgfpathlineto{\pgfqpoint{3.244929in}{2.621658in}}%
\pgfpathlineto{\pgfqpoint{3.124687in}{2.733673in}}%
\pgfpathlineto{\pgfqpoint{3.065291in}{2.787342in}}%
\pgfpathlineto{\pgfqpoint{3.004354in}{2.842667in}}%
\pgfpathlineto{\pgfqpoint{2.876798in}{2.954667in}}%
\pgfpathlineto{\pgfqpoint{2.775928in}{3.040481in}}%
\pgfpathlineto{\pgfqpoint{2.723879in}{3.084411in}}%
\pgfpathlineto{\pgfqpoint{2.603636in}{3.182992in}}%
\pgfpathlineto{\pgfqpoint{2.562505in}{3.216000in}}%
\pgfpathlineto{\pgfqpoint{2.443313in}{3.309307in}}%
\pgfpathlineto{\pgfqpoint{2.320007in}{3.402667in}}%
\pgfpathlineto{\pgfqpoint{2.202828in}{3.487949in}}%
\pgfpathlineto{\pgfqpoint{2.111707in}{3.552000in}}%
\pgfpathlineto{\pgfqpoint{1.922263in}{3.678408in}}%
\pgfpathlineto{\pgfqpoint{1.826323in}{3.738667in}}%
\pgfpathlineto{\pgfqpoint{1.663913in}{3.834027in}}%
\pgfpathlineto{\pgfqpoint{1.635090in}{3.850667in}}%
\pgfpathlineto{\pgfqpoint{1.466547in}{3.939144in}}%
\pgfpathlineto{\pgfqpoint{1.401212in}{3.970432in}}%
\pgfpathlineto{\pgfqpoint{1.240889in}{4.039399in}}%
\pgfpathlineto{\pgfqpoint{1.120646in}{4.082225in}}%
\pgfpathlineto{\pgfqpoint{1.000404in}{4.115856in}}%
\pgfpathlineto{\pgfqpoint{0.960323in}{4.124450in}}%
\pgfpathlineto{\pgfqpoint{0.920242in}{4.131681in}}%
\pgfpathlineto{\pgfqpoint{0.903349in}{4.133598in}}%
\pgfpathlineto{\pgfqpoint{0.880162in}{4.137359in}}%
\pgfpathlineto{\pgfqpoint{0.868123in}{4.138120in}}%
\pgfpathlineto{\pgfqpoint{0.840081in}{4.141256in}}%
\pgfpathlineto{\pgfqpoint{0.800000in}{4.143098in}}%
\pgfpathlineto{\pgfqpoint{0.800000in}{4.128535in}}%
\pgfpathlineto{\pgfqpoint{0.840081in}{4.127405in}}%
\pgfpathlineto{\pgfqpoint{0.854847in}{4.125754in}}%
\pgfpathlineto{\pgfqpoint{0.891274in}{4.122351in}}%
\pgfpathlineto{\pgfqpoint{0.920242in}{4.119063in}}%
\pgfpathlineto{\pgfqpoint{0.962081in}{4.112000in}}%
\pgfpathlineto{\pgfqpoint{1.000404in}{4.103916in}}%
\pgfpathlineto{\pgfqpoint{1.120646in}{4.071352in}}%
\pgfpathlineto{\pgfqpoint{1.160727in}{4.058118in}}%
\pgfpathlineto{\pgfqpoint{1.205861in}{4.042040in}}%
\pgfpathlineto{\pgfqpoint{1.255989in}{4.023268in}}%
\pgfpathlineto{\pgfqpoint{1.321051in}{3.996590in}}%
\pgfpathlineto{\pgfqpoint{1.487763in}{3.919382in}}%
\pgfpathlineto{\pgfqpoint{1.577524in}{3.873107in}}%
\pgfpathlineto{\pgfqpoint{1.681778in}{3.815769in}}%
\pgfpathlineto{\pgfqpoint{1.782206in}{3.757122in}}%
\pgfpathlineto{\pgfqpoint{1.882182in}{3.695665in}}%
\pgfpathlineto{\pgfqpoint{1.962343in}{3.644267in}}%
\pgfpathlineto{\pgfqpoint{2.045400in}{3.589333in}}%
\pgfpathlineto{\pgfqpoint{2.282990in}{3.422071in}}%
\pgfpathlineto{\pgfqpoint{2.408925in}{3.328000in}}%
\pgfpathlineto{\pgfqpoint{2.523475in}{3.239171in}}%
\pgfpathlineto{\pgfqpoint{2.645666in}{3.141333in}}%
\pgfpathlineto{\pgfqpoint{2.780511in}{3.029333in}}%
\pgfpathlineto{\pgfqpoint{2.884202in}{2.940656in}}%
\pgfpathlineto{\pgfqpoint{3.004444in}{2.834922in}}%
\pgfpathlineto{\pgfqpoint{3.061167in}{2.783501in}}%
\pgfpathlineto{\pgfqpoint{3.124687in}{2.726053in}}%
\pgfpathlineto{\pgfqpoint{3.164768in}{2.689057in}}%
\pgfpathlineto{\pgfqpoint{3.285010in}{2.575921in}}%
\pgfpathlineto{\pgfqpoint{3.405253in}{2.459520in}}%
\pgfpathlineto{\pgfqpoint{3.477745in}{2.387524in}}%
\pgfpathlineto{\pgfqpoint{3.525495in}{2.339795in}}%
\pgfpathlineto{\pgfqpoint{3.592930in}{2.270813in}}%
\pgfpathlineto{\pgfqpoint{3.645737in}{2.216688in}}%
\pgfpathlineto{\pgfqpoint{3.897689in}{1.946667in}}%
\pgfpathlineto{\pgfqpoint{3.947032in}{1.891308in}}%
\pgfpathlineto{\pgfqpoint{3.997705in}{1.834667in}}%
\pgfpathlineto{\pgfqpoint{4.062883in}{1.760000in}}%
\pgfpathlineto{\pgfqpoint{4.166788in}{1.638306in}}%
\pgfpathlineto{\pgfqpoint{4.287030in}{1.492791in}}%
\pgfpathlineto{\pgfqpoint{4.352679in}{1.410482in}}%
\pgfpathlineto{\pgfqpoint{4.401277in}{1.349333in}}%
\pgfpathlineto{\pgfqpoint{4.447354in}{1.289851in}}%
\pgfpathlineto{\pgfqpoint{4.527515in}{1.184098in}}%
\pgfpathlineto{\pgfqpoint{4.607677in}{1.075049in}}%
\pgfpathlineto{\pgfqpoint{4.768000in}{0.845079in}}%
\pgfpathlineto{\pgfqpoint{4.768000in}{0.845079in}}%
\pgfusepath{fill}%
\end{pgfscope}%
\begin{pgfscope}%
\pgfpathrectangle{\pgfqpoint{0.800000in}{0.528000in}}{\pgfqpoint{3.968000in}{3.696000in}}%
\pgfusepath{clip}%
\pgfsetbuttcap%
\pgfsetroundjoin%
\definecolor{currentfill}{rgb}{0.283072,0.130895,0.449241}%
\pgfsetfillcolor{currentfill}%
\pgfsetlinewidth{0.000000pt}%
\definecolor{currentstroke}{rgb}{0.000000,0.000000,0.000000}%
\pgfsetstrokecolor{currentstroke}%
\pgfsetdash{}{0pt}%
\pgfpathmoveto{\pgfqpoint{2.664196in}{0.528000in}}%
\pgfpathlineto{\pgfqpoint{2.544203in}{0.640000in}}%
\pgfpathlineto{\pgfqpoint{2.466036in}{0.714667in}}%
\pgfpathlineto{\pgfqpoint{2.351372in}{0.826667in}}%
\pgfpathlineto{\pgfqpoint{2.279695in}{0.898264in}}%
\pgfpathlineto{\pgfqpoint{2.239638in}{0.938667in}}%
\pgfpathlineto{\pgfqpoint{2.122667in}{1.059264in}}%
\pgfpathlineto{\pgfqpoint{2.059982in}{1.125333in}}%
\pgfpathlineto{\pgfqpoint{1.955689in}{1.237333in}}%
\pgfpathlineto{\pgfqpoint{1.917044in}{1.279527in}}%
\pgfpathlineto{\pgfqpoint{1.802020in}{1.407950in}}%
\pgfpathlineto{\pgfqpoint{1.561535in}{1.689263in}}%
\pgfpathlineto{\pgfqpoint{1.503721in}{1.760000in}}%
\pgfpathlineto{\pgfqpoint{1.414086in}{1.872000in}}%
\pgfpathlineto{\pgfqpoint{1.321051in}{1.991784in}}%
\pgfpathlineto{\pgfqpoint{1.240889in}{2.098224in}}%
\pgfpathlineto{\pgfqpoint{1.187865in}{2.170667in}}%
\pgfpathlineto{\pgfqpoint{1.107977in}{2.282667in}}%
\pgfpathlineto{\pgfqpoint{1.065973in}{2.343741in}}%
\pgfpathlineto{\pgfqpoint{1.030901in}{2.394667in}}%
\pgfpathlineto{\pgfqpoint{0.981207in}{2.469333in}}%
\pgfpathlineto{\pgfqpoint{0.920242in}{2.563729in}}%
\pgfpathlineto{\pgfqpoint{0.862888in}{2.656000in}}%
\pgfpathlineto{\pgfqpoint{0.811917in}{2.741767in}}%
\pgfpathlineto{\pgfqpoint{0.800000in}{2.761487in}}%
\pgfpathlineto{\pgfqpoint{0.800000in}{2.747433in}}%
\pgfpathlineto{\pgfqpoint{0.901168in}{2.581333in}}%
\pgfpathlineto{\pgfqpoint{0.960323in}{2.489176in}}%
\pgfpathlineto{\pgfqpoint{1.023184in}{2.394667in}}%
\pgfpathlineto{\pgfqpoint{1.100345in}{2.282667in}}%
\pgfpathlineto{\pgfqpoint{1.180316in}{2.170667in}}%
\pgfpathlineto{\pgfqpoint{1.221356in}{2.115139in}}%
\pgfpathlineto{\pgfqpoint{1.263004in}{2.058667in}}%
\pgfpathlineto{\pgfqpoint{1.348323in}{1.946667in}}%
\pgfpathlineto{\pgfqpoint{1.441293in}{1.828272in}}%
\pgfpathlineto{\pgfqpoint{1.561535in}{1.680337in}}%
\pgfpathlineto{\pgfqpoint{1.629593in}{1.599393in}}%
\pgfpathlineto{\pgfqpoint{1.683090in}{1.536000in}}%
\pgfpathlineto{\pgfqpoint{1.747768in}{1.461333in}}%
\pgfpathlineto{\pgfqpoint{1.846647in}{1.349333in}}%
\pgfpathlineto{\pgfqpoint{1.948264in}{1.237333in}}%
\pgfpathlineto{\pgfqpoint{2.029272in}{1.150341in}}%
\pgfpathlineto{\pgfqpoint{2.087670in}{1.088000in}}%
\pgfpathlineto{\pgfqpoint{2.142166in}{1.031496in}}%
\pgfpathlineto{\pgfqpoint{2.199017in}{0.972450in}}%
\pgfpathlineto{\pgfqpoint{2.242909in}{0.927604in}}%
\pgfpathlineto{\pgfqpoint{2.363152in}{0.807311in}}%
\pgfpathlineto{\pgfqpoint{2.483394in}{0.690296in}}%
\pgfpathlineto{\pgfqpoint{2.536149in}{0.640000in}}%
\pgfpathlineto{\pgfqpoint{2.655960in}{0.528000in}}%
\pgfpathmoveto{\pgfqpoint{4.768000in}{0.868290in}}%
\pgfpathlineto{\pgfqpoint{4.586129in}{1.125333in}}%
\pgfpathlineto{\pgfqpoint{4.502445in}{1.237333in}}%
\pgfpathlineto{\pgfqpoint{4.473968in}{1.274667in}}%
\pgfpathlineto{\pgfqpoint{4.386768in}{1.386667in}}%
\pgfpathlineto{\pgfqpoint{4.292656in}{1.503907in}}%
\pgfpathlineto{\pgfqpoint{4.246949in}{1.559849in}}%
\pgfpathlineto{\pgfqpoint{4.012606in}{1.834667in}}%
\pgfpathlineto{\pgfqpoint{3.946201in}{1.909333in}}%
\pgfpathlineto{\pgfqpoint{3.844687in}{2.021333in}}%
\pgfpathlineto{\pgfqpoint{3.775373in}{2.096000in}}%
\pgfpathlineto{\pgfqpoint{3.669269in}{2.208000in}}%
\pgfpathlineto{\pgfqpoint{3.597086in}{2.282667in}}%
\pgfpathlineto{\pgfqpoint{3.485414in}{2.395733in}}%
\pgfpathlineto{\pgfqpoint{3.427459in}{2.452684in}}%
\pgfpathlineto{\pgfqpoint{3.369005in}{2.510237in}}%
\pgfpathlineto{\pgfqpoint{3.325091in}{2.552794in}}%
\pgfpathlineto{\pgfqpoint{3.204848in}{2.666862in}}%
\pgfpathlineto{\pgfqpoint{3.084606in}{2.777796in}}%
\pgfpathlineto{\pgfqpoint{3.008425in}{2.846374in}}%
\pgfpathlineto{\pgfqpoint{2.964364in}{2.885647in}}%
\pgfpathlineto{\pgfqpoint{2.842277in}{2.992000in}}%
\pgfpathlineto{\pgfqpoint{2.763960in}{3.058461in}}%
\pgfpathlineto{\pgfqpoint{2.643717in}{3.157997in}}%
\pgfpathlineto{\pgfqpoint{2.588731in}{3.202116in}}%
\pgfpathlineto{\pgfqpoint{2.523475in}{3.254555in}}%
\pgfpathlineto{\pgfqpoint{2.443313in}{3.316987in}}%
\pgfpathlineto{\pgfqpoint{2.323071in}{3.408108in}}%
\pgfpathlineto{\pgfqpoint{2.228467in}{3.477333in}}%
\pgfpathlineto{\pgfqpoint{2.002424in}{3.634192in}}%
\pgfpathlineto{\pgfqpoint{1.899198in}{3.701333in}}%
\pgfpathlineto{\pgfqpoint{1.743805in}{3.796442in}}%
\pgfpathlineto{\pgfqpoint{1.715923in}{3.813333in}}%
\pgfpathlineto{\pgfqpoint{1.641697in}{3.855758in}}%
\pgfpathlineto{\pgfqpoint{1.445486in}{3.958761in}}%
\pgfpathlineto{\pgfqpoint{1.361131in}{3.998452in}}%
\pgfpathlineto{\pgfqpoint{1.332519in}{4.010682in}}%
\pgfpathlineto{\pgfqpoint{1.321051in}{4.016095in}}%
\pgfpathlineto{\pgfqpoint{1.270833in}{4.037333in}}%
\pgfpathlineto{\pgfqpoint{1.200808in}{4.064852in}}%
\pgfpathlineto{\pgfqpoint{1.164333in}{4.078025in}}%
\pgfpathlineto{\pgfqpoint{1.152423in}{4.082401in}}%
\pgfpathlineto{\pgfqpoint{1.080566in}{4.105651in}}%
\pgfpathlineto{\pgfqpoint{1.040485in}{4.117191in}}%
\pgfpathlineto{\pgfqpoint{0.960323in}{4.136530in}}%
\pgfpathlineto{\pgfqpoint{0.915425in}{4.144846in}}%
\pgfpathlineto{\pgfqpoint{0.880162in}{4.150507in}}%
\pgfpathlineto{\pgfqpoint{0.840081in}{4.154821in}}%
\pgfpathlineto{\pgfqpoint{0.800000in}{4.157227in}}%
\pgfpathlineto{\pgfqpoint{0.800000in}{4.143098in}}%
\pgfpathlineto{\pgfqpoint{0.806755in}{4.143041in}}%
\pgfpathlineto{\pgfqpoint{0.840081in}{4.141256in}}%
\pgfpathlineto{\pgfqpoint{0.920242in}{4.131681in}}%
\pgfpathlineto{\pgfqpoint{0.960323in}{4.124450in}}%
\pgfpathlineto{\pgfqpoint{1.015798in}{4.112000in}}%
\pgfpathlineto{\pgfqpoint{1.080566in}{4.094489in}}%
\pgfpathlineto{\pgfqpoint{1.096215in}{4.089243in}}%
\pgfpathlineto{\pgfqpoint{1.143385in}{4.074667in}}%
\pgfpathlineto{\pgfqpoint{1.200808in}{4.054478in}}%
\pgfpathlineto{\pgfqpoint{1.213604in}{4.049252in}}%
\pgfpathlineto{\pgfqpoint{1.246013in}{4.037333in}}%
\pgfpathlineto{\pgfqpoint{1.335604in}{4.000000in}}%
\pgfpathlineto{\pgfqpoint{1.466547in}{3.939144in}}%
\pgfpathlineto{\pgfqpoint{1.561535in}{3.890587in}}%
\pgfpathlineto{\pgfqpoint{1.641697in}{3.847043in}}%
\pgfpathlineto{\pgfqpoint{1.842101in}{3.728988in}}%
\pgfpathlineto{\pgfqpoint{1.931240in}{3.672362in}}%
\pgfpathlineto{\pgfqpoint{1.962343in}{3.652472in}}%
\pgfpathlineto{\pgfqpoint{2.057164in}{3.589333in}}%
\pgfpathlineto{\pgfqpoint{2.282990in}{3.429918in}}%
\pgfpathlineto{\pgfqpoint{2.403232in}{3.340023in}}%
\pgfpathlineto{\pgfqpoint{2.467395in}{3.290667in}}%
\pgfpathlineto{\pgfqpoint{2.563556in}{3.215167in}}%
\pgfpathlineto{\pgfqpoint{2.627305in}{3.163380in}}%
\pgfpathlineto{\pgfqpoint{2.683798in}{3.117603in}}%
\pgfpathlineto{\pgfqpoint{2.745111in}{3.066667in}}%
\pgfpathlineto{\pgfqpoint{2.844121in}{2.982818in}}%
\pgfpathlineto{\pgfqpoint{2.884202in}{2.948276in}}%
\pgfpathlineto{\pgfqpoint{3.004444in}{2.842586in}}%
\pgfpathlineto{\pgfqpoint{3.065291in}{2.787342in}}%
\pgfpathlineto{\pgfqpoint{3.127950in}{2.730667in}}%
\pgfpathlineto{\pgfqpoint{3.168368in}{2.693333in}}%
\pgfpathlineto{\pgfqpoint{3.287403in}{2.581333in}}%
\pgfpathlineto{\pgfqpoint{3.405253in}{2.467333in}}%
\pgfpathlineto{\pgfqpoint{3.481868in}{2.391364in}}%
\pgfpathlineto{\pgfqpoint{3.525495in}{2.347655in}}%
\pgfpathlineto{\pgfqpoint{3.597018in}{2.274621in}}%
\pgfpathlineto{\pgfqpoint{3.645737in}{2.224595in}}%
\pgfpathlineto{\pgfqpoint{3.725899in}{2.140646in}}%
\pgfpathlineto{\pgfqpoint{3.837116in}{2.021333in}}%
\pgfpathlineto{\pgfqpoint{3.938785in}{1.909333in}}%
\pgfpathlineto{\pgfqpoint{3.972162in}{1.872000in}}%
\pgfpathlineto{\pgfqpoint{4.070328in}{1.760000in}}%
\pgfpathlineto{\pgfqpoint{4.166788in}{1.647243in}}%
\pgfpathlineto{\pgfqpoint{4.289733in}{1.498667in}}%
\pgfpathlineto{\pgfqpoint{4.357019in}{1.414524in}}%
\pgfpathlineto{\pgfqpoint{4.408814in}{1.349333in}}%
\pgfpathlineto{\pgfqpoint{4.495015in}{1.237333in}}%
\pgfpathlineto{\pgfqpoint{4.578618in}{1.125333in}}%
\pgfpathlineto{\pgfqpoint{4.659538in}{1.013333in}}%
\pgfpathlineto{\pgfqpoint{4.727919in}{0.915546in}}%
\pgfpathlineto{\pgfqpoint{4.768000in}{0.856780in}}%
\pgfpathlineto{\pgfqpoint{4.768000in}{0.864000in}}%
\pgfpathlineto{\pgfqpoint{4.768000in}{0.864000in}}%
\pgfusepath{fill}%
\end{pgfscope}%
\begin{pgfscope}%
\pgfpathrectangle{\pgfqpoint{0.800000in}{0.528000in}}{\pgfqpoint{3.968000in}{3.696000in}}%
\pgfusepath{clip}%
\pgfsetbuttcap%
\pgfsetroundjoin%
\definecolor{currentfill}{rgb}{0.283072,0.130895,0.449241}%
\pgfsetfillcolor{currentfill}%
\pgfsetlinewidth{0.000000pt}%
\definecolor{currentstroke}{rgb}{0.000000,0.000000,0.000000}%
\pgfsetstrokecolor{currentstroke}%
\pgfsetdash{}{0pt}%
\pgfpathmoveto{\pgfqpoint{2.655960in}{0.528000in}}%
\pgfpathlineto{\pgfqpoint{2.563556in}{0.614080in}}%
\pgfpathlineto{\pgfqpoint{2.443313in}{0.728941in}}%
\pgfpathlineto{\pgfqpoint{2.381437in}{0.789333in}}%
\pgfpathlineto{\pgfqpoint{2.268913in}{0.901333in}}%
\pgfpathlineto{\pgfqpoint{2.159189in}{1.013333in}}%
\pgfpathlineto{\pgfqpoint{2.104364in}{1.070952in}}%
\pgfpathlineto{\pgfqpoint{2.052402in}{1.125333in}}%
\pgfpathlineto{\pgfqpoint{1.948264in}{1.237333in}}%
\pgfpathlineto{\pgfqpoint{1.914088in}{1.274667in}}%
\pgfpathlineto{\pgfqpoint{1.802020in}{1.399545in}}%
\pgfpathlineto{\pgfqpoint{1.721859in}{1.491077in}}%
\pgfpathlineto{\pgfqpoint{1.619797in}{1.610667in}}%
\pgfpathlineto{\pgfqpoint{1.521455in}{1.729058in}}%
\pgfpathlineto{\pgfqpoint{1.466148in}{1.797333in}}%
\pgfpathlineto{\pgfqpoint{1.377363in}{1.909333in}}%
\pgfpathlineto{\pgfqpoint{1.291122in}{2.021333in}}%
\pgfpathlineto{\pgfqpoint{1.263004in}{2.058667in}}%
\pgfpathlineto{\pgfqpoint{1.180316in}{2.170667in}}%
\pgfpathlineto{\pgfqpoint{1.100345in}{2.282667in}}%
\pgfpathlineto{\pgfqpoint{1.061299in}{2.339387in}}%
\pgfpathlineto{\pgfqpoint{1.023184in}{2.394667in}}%
\pgfpathlineto{\pgfqpoint{0.960323in}{2.489176in}}%
\pgfpathlineto{\pgfqpoint{0.901168in}{2.581333in}}%
\pgfpathlineto{\pgfqpoint{0.864261in}{2.641190in}}%
\pgfpathlineto{\pgfqpoint{0.832098in}{2.693333in}}%
\pgfpathlineto{\pgfqpoint{0.800000in}{2.747433in}}%
\pgfpathlineto{\pgfqpoint{0.801049in}{2.731644in}}%
\pgfpathlineto{\pgfqpoint{0.803540in}{2.727369in}}%
\pgfpathlineto{\pgfqpoint{0.880162in}{2.602061in}}%
\pgfpathlineto{\pgfqpoint{1.015466in}{2.394667in}}%
\pgfpathlineto{\pgfqpoint{1.080566in}{2.300004in}}%
\pgfpathlineto{\pgfqpoint{1.240889in}{2.078212in}}%
\pgfpathlineto{\pgfqpoint{1.321051in}{1.972390in}}%
\pgfpathlineto{\pgfqpoint{1.401212in}{1.869419in}}%
\pgfpathlineto{\pgfqpoint{1.468644in}{1.785477in}}%
\pgfpathlineto{\pgfqpoint{1.521455in}{1.720030in}}%
\pgfpathlineto{\pgfqpoint{1.590264in}{1.637426in}}%
\pgfpathlineto{\pgfqpoint{1.643899in}{1.573333in}}%
\pgfpathlineto{\pgfqpoint{1.707989in}{1.498667in}}%
\pgfpathlineto{\pgfqpoint{1.805987in}{1.386667in}}%
\pgfpathlineto{\pgfqpoint{1.877176in}{1.307338in}}%
\pgfpathlineto{\pgfqpoint{1.922263in}{1.257594in}}%
\pgfpathlineto{\pgfqpoint{1.988021in}{1.186584in}}%
\pgfpathlineto{\pgfqpoint{2.044822in}{1.125333in}}%
\pgfpathlineto{\pgfqpoint{2.162747in}{1.001883in}}%
\pgfpathlineto{\pgfqpoint{2.282990in}{0.879419in}}%
\pgfpathlineto{\pgfqpoint{2.411721in}{0.752000in}}%
\pgfpathlineto{\pgfqpoint{2.486248in}{0.679992in}}%
\pgfpathlineto{\pgfqpoint{2.528096in}{0.640000in}}%
\pgfpathlineto{\pgfqpoint{2.647723in}{0.528000in}}%
\pgfpathmoveto{\pgfqpoint{4.768000in}{0.879495in}}%
\pgfpathlineto{\pgfqpoint{4.607677in}{1.106222in}}%
\pgfpathlineto{\pgfqpoint{4.527515in}{1.214074in}}%
\pgfpathlineto{\pgfqpoint{4.447354in}{1.318843in}}%
\pgfpathlineto{\pgfqpoint{4.334643in}{1.461333in}}%
\pgfpathlineto{\pgfqpoint{4.230613in}{1.588550in}}%
\pgfpathlineto{\pgfqpoint{4.117294in}{1.722667in}}%
\pgfpathlineto{\pgfqpoint{4.067657in}{1.779664in}}%
\pgfpathlineto{\pgfqpoint{4.019922in}{1.834667in}}%
\pgfpathlineto{\pgfqpoint{3.953617in}{1.909333in}}%
\pgfpathlineto{\pgfqpoint{3.846141in}{2.027784in}}%
\pgfpathlineto{\pgfqpoint{3.782811in}{2.096000in}}%
\pgfpathlineto{\pgfqpoint{3.676862in}{2.208000in}}%
\pgfpathlineto{\pgfqpoint{3.623976in}{2.262397in}}%
\pgfpathlineto{\pgfqpoint{3.565576in}{2.322684in}}%
\pgfpathlineto{\pgfqpoint{3.494096in}{2.394667in}}%
\pgfpathlineto{\pgfqpoint{3.405253in}{2.482569in}}%
\pgfpathlineto{\pgfqpoint{3.333604in}{2.551929in}}%
\pgfpathlineto{\pgfqpoint{3.285010in}{2.598715in}}%
\pgfpathlineto{\pgfqpoint{3.164768in}{2.711690in}}%
\pgfpathlineto{\pgfqpoint{3.103529in}{2.768000in}}%
\pgfpathlineto{\pgfqpoint{3.004444in}{2.857484in}}%
\pgfpathlineto{\pgfqpoint{2.884202in}{2.963268in}}%
\pgfpathlineto{\pgfqpoint{2.826260in}{3.012697in}}%
\pgfpathlineto{\pgfqpoint{2.763208in}{3.066667in}}%
\pgfpathlineto{\pgfqpoint{2.683798in}{3.132699in}}%
\pgfpathlineto{\pgfqpoint{2.563556in}{3.230202in}}%
\pgfpathlineto{\pgfqpoint{2.438967in}{3.328000in}}%
\pgfpathlineto{\pgfqpoint{2.187254in}{3.514667in}}%
\pgfpathlineto{\pgfqpoint{2.080693in}{3.589333in}}%
\pgfpathlineto{\pgfqpoint{1.846467in}{3.742734in}}%
\pgfpathlineto{\pgfqpoint{1.822177in}{3.757442in}}%
\pgfpathlineto{\pgfqpoint{1.792650in}{3.776000in}}%
\pgfpathlineto{\pgfqpoint{1.681778in}{3.841616in}}%
\pgfpathlineto{\pgfqpoint{1.625525in}{3.872937in}}%
\pgfpathlineto{\pgfqpoint{1.599238in}{3.888000in}}%
\pgfpathlineto{\pgfqpoint{1.521455in}{3.929504in}}%
\pgfpathlineto{\pgfqpoint{1.361131in}{4.007876in}}%
\pgfpathlineto{\pgfqpoint{1.200808in}{4.075205in}}%
\pgfpathlineto{\pgfqpoint{1.073112in}{4.118943in}}%
\pgfpathlineto{\pgfqpoint{1.000404in}{4.139028in}}%
\pgfpathlineto{\pgfqpoint{0.956692in}{4.149333in}}%
\pgfpathlineto{\pgfqpoint{0.880162in}{4.163085in}}%
\pgfpathlineto{\pgfqpoint{0.800000in}{4.171031in}}%
\pgfpathlineto{\pgfqpoint{0.800000in}{4.157227in}}%
\pgfpathlineto{\pgfqpoint{0.807662in}{4.156470in}}%
\pgfpathlineto{\pgfqpoint{0.845157in}{4.154062in}}%
\pgfpathlineto{\pgfqpoint{0.887756in}{4.149333in}}%
\pgfpathlineto{\pgfqpoint{0.960323in}{4.136530in}}%
\pgfpathlineto{\pgfqpoint{1.000404in}{4.127442in}}%
\pgfpathlineto{\pgfqpoint{1.013061in}{4.123790in}}%
\pgfpathlineto{\pgfqpoint{1.058542in}{4.112000in}}%
\pgfpathlineto{\pgfqpoint{1.120646in}{4.092965in}}%
\pgfpathlineto{\pgfqpoint{1.173901in}{4.074667in}}%
\pgfpathlineto{\pgfqpoint{1.288880in}{4.029966in}}%
\pgfpathlineto{\pgfqpoint{1.361131in}{3.998452in}}%
\pgfpathlineto{\pgfqpoint{1.521455in}{3.920516in}}%
\pgfpathlineto{\pgfqpoint{1.721859in}{3.809907in}}%
\pgfpathlineto{\pgfqpoint{1.936342in}{3.677115in}}%
\pgfpathlineto{\pgfqpoint{1.974378in}{3.652791in}}%
\pgfpathlineto{\pgfqpoint{2.082586in}{3.580039in}}%
\pgfpathlineto{\pgfqpoint{2.145514in}{3.535948in}}%
\pgfpathlineto{\pgfqpoint{2.176271in}{3.514667in}}%
\pgfpathlineto{\pgfqpoint{2.294432in}{3.429342in}}%
\pgfpathlineto{\pgfqpoint{2.403232in}{3.347688in}}%
\pgfpathlineto{\pgfqpoint{2.458521in}{3.304832in}}%
\pgfpathlineto{\pgfqpoint{2.525014in}{3.253333in}}%
\pgfpathlineto{\pgfqpoint{2.643717in}{3.157997in}}%
\pgfpathlineto{\pgfqpoint{2.763960in}{3.058461in}}%
\pgfpathlineto{\pgfqpoint{2.822065in}{3.008789in}}%
\pgfpathlineto{\pgfqpoint{2.885575in}{2.954667in}}%
\pgfpathlineto{\pgfqpoint{3.012673in}{2.842667in}}%
\pgfpathlineto{\pgfqpoint{3.084606in}{2.777796in}}%
\pgfpathlineto{\pgfqpoint{3.216421in}{2.656000in}}%
\pgfpathlineto{\pgfqpoint{3.334205in}{2.544000in}}%
\pgfpathlineto{\pgfqpoint{3.448901in}{2.432000in}}%
\pgfpathlineto{\pgfqpoint{3.500518in}{2.380598in}}%
\pgfpathlineto{\pgfqpoint{3.605657in}{2.273885in}}%
\pgfpathlineto{\pgfqpoint{3.669269in}{2.208000in}}%
\pgfpathlineto{\pgfqpoint{3.775373in}{2.096000in}}%
\pgfpathlineto{\pgfqpoint{3.826573in}{2.040439in}}%
\pgfpathlineto{\pgfqpoint{3.878786in}{1.984000in}}%
\pgfpathlineto{\pgfqpoint{3.955229in}{1.898943in}}%
\pgfpathlineto{\pgfqpoint{4.006465in}{1.841636in}}%
\pgfpathlineto{\pgfqpoint{4.206869in}{1.608246in}}%
\pgfpathlineto{\pgfqpoint{4.328142in}{1.460374in}}%
\pgfpathlineto{\pgfqpoint{4.447354in}{1.309347in}}%
\pgfpathlineto{\pgfqpoint{4.558573in}{1.162667in}}%
\pgfpathlineto{\pgfqpoint{4.586129in}{1.125333in}}%
\pgfpathlineto{\pgfqpoint{4.667131in}{1.013333in}}%
\pgfpathlineto{\pgfqpoint{4.745361in}{0.901333in}}%
\pgfpathlineto{\pgfqpoint{4.768000in}{0.868290in}}%
\pgfpathlineto{\pgfqpoint{4.768000in}{0.868290in}}%
\pgfusepath{fill}%
\end{pgfscope}%
\begin{pgfscope}%
\pgfpathrectangle{\pgfqpoint{0.800000in}{0.528000in}}{\pgfqpoint{3.968000in}{3.696000in}}%
\pgfusepath{clip}%
\pgfsetbuttcap%
\pgfsetroundjoin%
\definecolor{currentfill}{rgb}{0.282884,0.135920,0.453427}%
\pgfsetfillcolor{currentfill}%
\pgfsetlinewidth{0.000000pt}%
\definecolor{currentstroke}{rgb}{0.000000,0.000000,0.000000}%
\pgfsetstrokecolor{currentstroke}%
\pgfsetdash{}{0pt}%
\pgfpathmoveto{\pgfqpoint{2.647723in}{0.528000in}}%
\pgfpathlineto{\pgfqpoint{2.563556in}{0.606463in}}%
\pgfpathlineto{\pgfqpoint{2.443313in}{0.721280in}}%
\pgfpathlineto{\pgfqpoint{2.373615in}{0.789333in}}%
\pgfpathlineto{\pgfqpoint{2.261256in}{0.901333in}}%
\pgfpathlineto{\pgfqpoint{2.151692in}{1.013333in}}%
\pgfpathlineto{\pgfqpoint{2.100372in}{1.067233in}}%
\pgfpathlineto{\pgfqpoint{2.042505in}{1.127794in}}%
\pgfpathlineto{\pgfqpoint{1.975230in}{1.200000in}}%
\pgfpathlineto{\pgfqpoint{1.872847in}{1.312000in}}%
\pgfpathlineto{\pgfqpoint{1.761939in}{1.436676in}}%
\pgfpathlineto{\pgfqpoint{1.641697in}{1.575928in}}%
\pgfpathlineto{\pgfqpoint{1.572734in}{1.658431in}}%
\pgfpathlineto{\pgfqpoint{1.519291in}{1.722667in}}%
\pgfpathlineto{\pgfqpoint{1.458789in}{1.797333in}}%
\pgfpathlineto{\pgfqpoint{1.361131in}{1.920604in}}%
\pgfpathlineto{\pgfqpoint{1.255536in}{2.058667in}}%
\pgfpathlineto{\pgfqpoint{1.172767in}{2.170667in}}%
\pgfpathlineto{\pgfqpoint{1.092713in}{2.282667in}}%
\pgfpathlineto{\pgfqpoint{1.056625in}{2.335033in}}%
\pgfpathlineto{\pgfqpoint{1.015466in}{2.394667in}}%
\pgfpathlineto{\pgfqpoint{0.941122in}{2.506667in}}%
\pgfpathlineto{\pgfqpoint{0.911888in}{2.551781in}}%
\pgfpathlineto{\pgfqpoint{0.840081in}{2.666833in}}%
\pgfpathlineto{\pgfqpoint{0.800000in}{2.733379in}}%
\pgfpathlineto{\pgfqpoint{0.800000in}{2.719897in}}%
\pgfpathlineto{\pgfqpoint{0.861885in}{2.618667in}}%
\pgfpathlineto{\pgfqpoint{0.920242in}{2.526793in}}%
\pgfpathlineto{\pgfqpoint{1.059075in}{2.320000in}}%
\pgfpathlineto{\pgfqpoint{1.138296in}{2.208000in}}%
\pgfpathlineto{\pgfqpoint{1.220232in}{2.096000in}}%
\pgfpathlineto{\pgfqpoint{1.264806in}{2.036389in}}%
\pgfpathlineto{\pgfqpoint{1.362486in}{1.909333in}}%
\pgfpathlineto{\pgfqpoint{1.421570in}{1.834667in}}%
\pgfpathlineto{\pgfqpoint{1.521455in}{1.711215in}}%
\pgfpathlineto{\pgfqpoint{1.586196in}{1.633637in}}%
\pgfpathlineto{\pgfqpoint{1.636626in}{1.573333in}}%
\pgfpathlineto{\pgfqpoint{1.733063in}{1.461333in}}%
\pgfpathlineto{\pgfqpoint{1.842101in}{1.337998in}}%
\pgfpathlineto{\pgfqpoint{1.967755in}{1.200000in}}%
\pgfpathlineto{\pgfqpoint{2.039821in}{1.122834in}}%
\pgfpathlineto{\pgfqpoint{2.082586in}{1.077602in}}%
\pgfpathlineto{\pgfqpoint{2.153105in}{1.004351in}}%
\pgfpathlineto{\pgfqpoint{2.202828in}{0.952940in}}%
\pgfpathlineto{\pgfqpoint{2.328075in}{0.826667in}}%
\pgfpathlineto{\pgfqpoint{2.443313in}{0.713649in}}%
\pgfpathlineto{\pgfqpoint{2.483394in}{0.675070in}}%
\pgfpathlineto{\pgfqpoint{2.603636in}{0.561411in}}%
\pgfpathlineto{\pgfqpoint{2.639628in}{0.528000in}}%
\pgfpathlineto{\pgfqpoint{2.643717in}{0.528000in}}%
\pgfpathmoveto{\pgfqpoint{4.768000in}{0.890700in}}%
\pgfpathlineto{\pgfqpoint{4.628429in}{1.088000in}}%
\pgfpathlineto{\pgfqpoint{4.545496in}{1.200000in}}%
\pgfpathlineto{\pgfqpoint{4.504674in}{1.253391in}}%
\pgfpathlineto{\pgfqpoint{4.459969in}{1.312000in}}%
\pgfpathlineto{\pgfqpoint{4.407273in}{1.379439in}}%
\pgfpathlineto{\pgfqpoint{4.327111in}{1.479689in}}%
\pgfpathlineto{\pgfqpoint{4.092478in}{1.760000in}}%
\pgfpathlineto{\pgfqpoint{4.017725in}{1.845155in}}%
\pgfpathlineto{\pgfqpoint{3.963426in}{1.906578in}}%
\pgfpathlineto{\pgfqpoint{3.926303in}{1.948011in}}%
\pgfpathlineto{\pgfqpoint{3.806061in}{2.079057in}}%
\pgfpathlineto{\pgfqpoint{3.722728in}{2.167713in}}%
\pgfpathlineto{\pgfqpoint{3.684455in}{2.208000in}}%
\pgfpathlineto{\pgfqpoint{3.612273in}{2.282667in}}%
\pgfpathlineto{\pgfqpoint{3.501713in}{2.394667in}}%
\pgfpathlineto{\pgfqpoint{3.388330in}{2.506667in}}%
\pgfpathlineto{\pgfqpoint{3.298045in}{2.593475in}}%
\pgfpathlineto{\pgfqpoint{3.244929in}{2.644255in}}%
\pgfpathlineto{\pgfqpoint{3.118176in}{2.761936in}}%
\pgfpathlineto{\pgfqpoint{3.070712in}{2.805333in}}%
\pgfpathlineto{\pgfqpoint{2.945258in}{2.917333in}}%
\pgfpathlineto{\pgfqpoint{2.682474in}{3.141333in}}%
\pgfpathlineto{\pgfqpoint{2.563556in}{3.237707in}}%
\pgfpathlineto{\pgfqpoint{2.443313in}{3.332224in}}%
\pgfpathlineto{\pgfqpoint{2.400180in}{3.365333in}}%
\pgfpathlineto{\pgfqpoint{2.282990in}{3.453073in}}%
\pgfpathlineto{\pgfqpoint{2.162747in}{3.539876in}}%
\pgfpathlineto{\pgfqpoint{2.109354in}{3.576933in}}%
\pgfpathlineto{\pgfqpoint{2.063232in}{3.608639in}}%
\pgfpathlineto{\pgfqpoint{2.016842in}{3.640096in}}%
\pgfpathlineto{\pgfqpoint{1.981525in}{3.664000in}}%
\pgfpathlineto{\pgfqpoint{1.882182in}{3.728527in}}%
\pgfpathlineto{\pgfqpoint{1.793712in}{3.783738in}}%
\pgfpathlineto{\pgfqpoint{1.681013in}{3.850667in}}%
\pgfpathlineto{\pgfqpoint{1.504352in}{3.946736in}}%
\pgfpathlineto{\pgfqpoint{1.474183in}{3.962667in}}%
\pgfpathlineto{\pgfqpoint{1.398221in}{4.000000in}}%
\pgfpathlineto{\pgfqpoint{1.316738in}{4.037333in}}%
\pgfpathlineto{\pgfqpoint{1.200808in}{4.085187in}}%
\pgfpathlineto{\pgfqpoint{1.055827in}{4.135043in}}%
\pgfpathlineto{\pgfqpoint{1.000404in}{4.150561in}}%
\pgfpathlineto{\pgfqpoint{0.920242in}{4.168616in}}%
\pgfpathlineto{\pgfqpoint{0.880162in}{4.175663in}}%
\pgfpathlineto{\pgfqpoint{0.869510in}{4.176745in}}%
\pgfpathlineto{\pgfqpoint{0.834460in}{4.181432in}}%
\pgfpathlineto{\pgfqpoint{0.800000in}{4.184834in}}%
\pgfpathlineto{\pgfqpoint{0.800000in}{4.171031in}}%
\pgfpathlineto{\pgfqpoint{0.880162in}{4.163085in}}%
\pgfpathlineto{\pgfqpoint{0.926431in}{4.155098in}}%
\pgfpathlineto{\pgfqpoint{0.961326in}{4.148399in}}%
\pgfpathlineto{\pgfqpoint{1.040485in}{4.128322in}}%
\pgfpathlineto{\pgfqpoint{1.094924in}{4.112000in}}%
\pgfpathlineto{\pgfqpoint{1.172117in}{4.085275in}}%
\pgfpathlineto{\pgfqpoint{1.202177in}{4.074667in}}%
\pgfpathlineto{\pgfqpoint{1.321051in}{4.025784in}}%
\pgfpathlineto{\pgfqpoint{1.393320in}{3.992649in}}%
\pgfpathlineto{\pgfqpoint{1.424825in}{3.978005in}}%
\pgfpathlineto{\pgfqpoint{1.521455in}{3.929504in}}%
\pgfpathlineto{\pgfqpoint{1.601616in}{3.886718in}}%
\pgfpathlineto{\pgfqpoint{1.802020in}{3.770337in}}%
\pgfpathlineto{\pgfqpoint{1.894101in}{3.712435in}}%
\pgfpathlineto{\pgfqpoint{1.941445in}{3.681867in}}%
\pgfpathlineto{\pgfqpoint{1.969499in}{3.664000in}}%
\pgfpathlineto{\pgfqpoint{2.082586in}{3.588045in}}%
\pgfpathlineto{\pgfqpoint{2.162747in}{3.532074in}}%
\pgfpathlineto{\pgfqpoint{2.282990in}{3.445452in}}%
\pgfpathlineto{\pgfqpoint{2.340543in}{3.402667in}}%
\pgfpathlineto{\pgfqpoint{2.443313in}{3.324667in}}%
\pgfpathlineto{\pgfqpoint{2.506573in}{3.274924in}}%
\pgfpathlineto{\pgfqpoint{2.563556in}{3.230202in}}%
\pgfpathlineto{\pgfqpoint{2.636073in}{3.171546in}}%
\pgfpathlineto{\pgfqpoint{2.683798in}{3.132699in}}%
\pgfpathlineto{\pgfqpoint{2.742220in}{3.083751in}}%
\pgfpathlineto{\pgfqpoint{2.804040in}{3.032122in}}%
\pgfpathlineto{\pgfqpoint{2.924283in}{2.928342in}}%
\pgfpathlineto{\pgfqpoint{2.979110in}{2.880000in}}%
\pgfpathlineto{\pgfqpoint{3.103529in}{2.768000in}}%
\pgfpathlineto{\pgfqpoint{3.184506in}{2.693333in}}%
\pgfpathlineto{\pgfqpoint{3.303190in}{2.581333in}}%
\pgfpathlineto{\pgfqpoint{3.418724in}{2.469333in}}%
\pgfpathlineto{\pgfqpoint{3.494096in}{2.394667in}}%
\pgfpathlineto{\pgfqpoint{3.605657in}{2.281775in}}%
\pgfpathlineto{\pgfqpoint{3.676862in}{2.208000in}}%
\pgfpathlineto{\pgfqpoint{3.782811in}{2.096000in}}%
\pgfpathlineto{\pgfqpoint{3.846141in}{2.027784in}}%
\pgfpathlineto{\pgfqpoint{3.953617in}{1.909333in}}%
\pgfpathlineto{\pgfqpoint{4.019922in}{1.834667in}}%
\pgfpathlineto{\pgfqpoint{4.126707in}{1.711685in}}%
\pgfpathlineto{\pgfqpoint{4.180759in}{1.648000in}}%
\pgfpathlineto{\pgfqpoint{4.273906in}{1.536000in}}%
\pgfpathlineto{\pgfqpoint{4.367192in}{1.420777in}}%
\pgfpathlineto{\pgfqpoint{4.481449in}{1.274667in}}%
\pgfpathlineto{\pgfqpoint{4.567596in}{1.160698in}}%
\pgfpathlineto{\pgfqpoint{4.648132in}{1.050667in}}%
\pgfpathlineto{\pgfqpoint{4.701141in}{0.976000in}}%
\pgfpathlineto{\pgfqpoint{4.768000in}{0.879495in}}%
\pgfpathlineto{\pgfqpoint{4.768000in}{0.879495in}}%
\pgfusepath{fill}%
\end{pgfscope}%
\begin{pgfscope}%
\pgfpathrectangle{\pgfqpoint{0.800000in}{0.528000in}}{\pgfqpoint{3.968000in}{3.696000in}}%
\pgfusepath{clip}%
\pgfsetbuttcap%
\pgfsetroundjoin%
\definecolor{currentfill}{rgb}{0.282884,0.135920,0.453427}%
\pgfsetfillcolor{currentfill}%
\pgfsetlinewidth{0.000000pt}%
\definecolor{currentstroke}{rgb}{0.000000,0.000000,0.000000}%
\pgfsetstrokecolor{currentstroke}%
\pgfsetdash{}{0pt}%
\pgfpathmoveto{\pgfqpoint{2.639628in}{0.528000in}}%
\pgfpathlineto{\pgfqpoint{2.520155in}{0.640000in}}%
\pgfpathlineto{\pgfqpoint{2.403232in}{0.752594in}}%
\pgfpathlineto{\pgfqpoint{2.325686in}{0.829103in}}%
\pgfpathlineto{\pgfqpoint{2.282990in}{0.871699in}}%
\pgfpathlineto{\pgfqpoint{2.216831in}{0.938667in}}%
\pgfpathlineto{\pgfqpoint{2.108315in}{1.050667in}}%
\pgfpathlineto{\pgfqpoint{2.039821in}{1.122834in}}%
\pgfpathlineto{\pgfqpoint{2.002364in}{1.162667in}}%
\pgfpathlineto{\pgfqpoint{1.928358in}{1.243011in}}%
\pgfpathlineto{\pgfqpoint{1.882182in}{1.293546in}}%
\pgfpathlineto{\pgfqpoint{1.782319in}{1.405649in}}%
\pgfpathlineto{\pgfqpoint{1.733063in}{1.461333in}}%
\pgfpathlineto{\pgfqpoint{1.656835in}{1.550101in}}%
\pgfpathlineto{\pgfqpoint{1.601616in}{1.614733in}}%
\pgfpathlineto{\pgfqpoint{1.533704in}{1.696743in}}%
\pgfpathlineto{\pgfqpoint{1.481374in}{1.760147in}}%
\pgfpathlineto{\pgfqpoint{1.413065in}{1.845707in}}%
\pgfpathlineto{\pgfqpoint{1.361131in}{1.911069in}}%
\pgfpathlineto{\pgfqpoint{1.248069in}{2.058667in}}%
\pgfpathlineto{\pgfqpoint{1.160727in}{2.176858in}}%
\pgfpathlineto{\pgfqpoint{1.080566in}{2.289111in}}%
\pgfpathlineto{\pgfqpoint{1.000404in}{2.405541in}}%
\pgfpathlineto{\pgfqpoint{0.861885in}{2.618667in}}%
\pgfpathlineto{\pgfqpoint{0.836663in}{2.659183in}}%
\pgfpathlineto{\pgfqpoint{0.800000in}{2.719897in}}%
\pgfpathlineto{\pgfqpoint{0.800000in}{2.706552in}}%
\pgfpathlineto{\pgfqpoint{0.853992in}{2.618667in}}%
\pgfpathlineto{\pgfqpoint{0.920242in}{2.514780in}}%
\pgfpathlineto{\pgfqpoint{1.040485in}{2.335869in}}%
\pgfpathlineto{\pgfqpoint{1.200808in}{2.112173in}}%
\pgfpathlineto{\pgfqpoint{1.384649in}{1.872000in}}%
\pgfpathlineto{\pgfqpoint{1.481374in}{1.751308in}}%
\pgfpathlineto{\pgfqpoint{1.535554in}{1.685333in}}%
\pgfpathlineto{\pgfqpoint{1.629420in}{1.573333in}}%
\pgfpathlineto{\pgfqpoint{1.725711in}{1.461333in}}%
\pgfpathlineto{\pgfqpoint{1.824682in}{1.349333in}}%
\pgfpathlineto{\pgfqpoint{1.925992in}{1.237333in}}%
\pgfpathlineto{\pgfqpoint{1.995067in}{1.162667in}}%
\pgfpathlineto{\pgfqpoint{2.100869in}{1.050667in}}%
\pgfpathlineto{\pgfqpoint{2.209228in}{0.938667in}}%
\pgfpathlineto{\pgfqpoint{2.263982in}{0.883628in}}%
\pgfpathlineto{\pgfqpoint{2.323071in}{0.824005in}}%
\pgfpathlineto{\pgfqpoint{2.443313in}{0.706204in}}%
\pgfpathlineto{\pgfqpoint{2.517930in}{0.634836in}}%
\pgfpathlineto{\pgfqpoint{2.563556in}{0.591549in}}%
\pgfpathlineto{\pgfqpoint{2.631667in}{0.528000in}}%
\pgfpathmoveto{\pgfqpoint{4.768000in}{0.901882in}}%
\pgfpathlineto{\pgfqpoint{4.687838in}{1.016136in}}%
\pgfpathlineto{\pgfqpoint{4.635887in}{1.088000in}}%
\pgfpathlineto{\pgfqpoint{4.552876in}{1.200000in}}%
\pgfpathlineto{\pgfqpoint{4.508962in}{1.257386in}}%
\pgfpathlineto{\pgfqpoint{4.467271in}{1.312000in}}%
\pgfpathlineto{\pgfqpoint{4.424913in}{1.365765in}}%
\pgfpathlineto{\pgfqpoint{4.379153in}{1.424000in}}%
\pgfpathlineto{\pgfqpoint{4.322442in}{1.494318in}}%
\pgfpathlineto{\pgfqpoint{4.287030in}{1.537913in}}%
\pgfpathlineto{\pgfqpoint{4.163828in}{1.685333in}}%
\pgfpathlineto{\pgfqpoint{4.093768in}{1.766652in}}%
\pgfpathlineto{\pgfqpoint{4.046545in}{1.821029in}}%
\pgfpathlineto{\pgfqpoint{3.934756in}{1.946667in}}%
\pgfpathlineto{\pgfqpoint{3.866743in}{2.021333in}}%
\pgfpathlineto{\pgfqpoint{3.762753in}{2.133333in}}%
\pgfpathlineto{\pgfqpoint{3.645737in}{2.255904in}}%
\pgfpathlineto{\pgfqpoint{3.574680in}{2.328480in}}%
\pgfpathlineto{\pgfqpoint{3.525495in}{2.378466in}}%
\pgfpathlineto{\pgfqpoint{3.396110in}{2.506667in}}%
\pgfpathlineto{\pgfqpoint{3.301991in}{2.597151in}}%
\pgfpathlineto{\pgfqpoint{3.240449in}{2.656000in}}%
\pgfpathlineto{\pgfqpoint{3.162599in}{2.728646in}}%
\pgfpathlineto{\pgfqpoint{3.119911in}{2.768000in}}%
\pgfpathlineto{\pgfqpoint{2.995871in}{2.880000in}}%
\pgfpathlineto{\pgfqpoint{2.911110in}{2.954667in}}%
\pgfpathlineto{\pgfqpoint{2.804040in}{3.046881in}}%
\pgfpathlineto{\pgfqpoint{2.736266in}{3.104000in}}%
\pgfpathlineto{\pgfqpoint{2.643717in}{3.180544in}}%
\pgfpathlineto{\pgfqpoint{2.563556in}{3.245213in}}%
\pgfpathlineto{\pgfqpoint{2.443313in}{3.339687in}}%
\pgfpathlineto{\pgfqpoint{2.403232in}{3.370533in}}%
\pgfpathlineto{\pgfqpoint{2.310954in}{3.440000in}}%
\pgfpathlineto{\pgfqpoint{2.082586in}{3.603626in}}%
\pgfpathlineto{\pgfqpoint{2.021740in}{3.644658in}}%
\pgfpathlineto{\pgfqpoint{1.993550in}{3.664000in}}%
\pgfpathlineto{\pgfqpoint{1.879058in}{3.738667in}}%
\pgfpathlineto{\pgfqpoint{1.681778in}{3.858590in}}%
\pgfpathlineto{\pgfqpoint{1.481374in}{3.968007in}}%
\pgfpathlineto{\pgfqpoint{1.417030in}{4.000000in}}%
\pgfpathlineto{\pgfqpoint{1.321051in}{4.044887in}}%
\pgfpathlineto{\pgfqpoint{1.160727in}{4.110501in}}%
\pgfpathlineto{\pgfqpoint{1.080566in}{4.138040in}}%
\pgfpathlineto{\pgfqpoint{1.038580in}{4.151107in}}%
\pgfpathlineto{\pgfqpoint{0.960323in}{4.171747in}}%
\pgfpathlineto{\pgfqpoint{0.947025in}{4.174280in}}%
\pgfpathlineto{\pgfqpoint{0.920242in}{4.180659in}}%
\pgfpathlineto{\pgfqpoint{0.878172in}{4.188520in}}%
\pgfpathlineto{\pgfqpoint{0.830988in}{4.195136in}}%
\pgfpathlineto{\pgfqpoint{0.800000in}{4.198044in}}%
\pgfpathlineto{\pgfqpoint{0.800000in}{4.184834in}}%
\pgfpathlineto{\pgfqpoint{0.840081in}{4.181146in}}%
\pgfpathlineto{\pgfqpoint{0.920242in}{4.168616in}}%
\pgfpathlineto{\pgfqpoint{1.040485in}{4.139453in}}%
\pgfpathlineto{\pgfqpoint{1.080566in}{4.127329in}}%
\pgfpathlineto{\pgfqpoint{1.092464in}{4.123083in}}%
\pgfpathlineto{\pgfqpoint{1.127290in}{4.112000in}}%
\pgfpathlineto{\pgfqpoint{1.200808in}{4.085187in}}%
\pgfpathlineto{\pgfqpoint{1.280970in}{4.052718in}}%
\pgfpathlineto{\pgfqpoint{1.316738in}{4.037333in}}%
\pgfpathlineto{\pgfqpoint{1.361131in}{4.017245in}}%
\pgfpathlineto{\pgfqpoint{1.441293in}{3.979037in}}%
\pgfpathlineto{\pgfqpoint{1.521455in}{3.938331in}}%
\pgfpathlineto{\pgfqpoint{1.614774in}{3.888000in}}%
\pgfpathlineto{\pgfqpoint{1.793712in}{3.783738in}}%
\pgfpathlineto{\pgfqpoint{1.882182in}{3.728527in}}%
\pgfpathlineto{\pgfqpoint{1.946547in}{3.686620in}}%
\pgfpathlineto{\pgfqpoint{1.981525in}{3.664000in}}%
\pgfpathlineto{\pgfqpoint{2.082586in}{3.595854in}}%
\pgfpathlineto{\pgfqpoint{2.202828in}{3.511399in}}%
\pgfpathlineto{\pgfqpoint{2.290696in}{3.447178in}}%
\pgfpathlineto{\pgfqpoint{2.350745in}{3.402667in}}%
\pgfpathlineto{\pgfqpoint{2.448751in}{3.328000in}}%
\pgfpathlineto{\pgfqpoint{2.543898in}{3.253333in}}%
\pgfpathlineto{\pgfqpoint{2.643717in}{3.173064in}}%
\pgfpathlineto{\pgfqpoint{2.771952in}{3.066667in}}%
\pgfpathlineto{\pgfqpoint{2.872157in}{2.980781in}}%
\pgfpathlineto{\pgfqpoint{2.924283in}{2.935763in}}%
\pgfpathlineto{\pgfqpoint{2.987491in}{2.880000in}}%
\pgfpathlineto{\pgfqpoint{3.084606in}{2.792748in}}%
\pgfpathlineto{\pgfqpoint{3.204848in}{2.681898in}}%
\pgfpathlineto{\pgfqpoint{3.271940in}{2.618667in}}%
\pgfpathlineto{\pgfqpoint{3.365172in}{2.529215in}}%
\pgfpathlineto{\pgfqpoint{3.435502in}{2.460176in}}%
\pgfpathlineto{\pgfqpoint{3.485414in}{2.410970in}}%
\pgfpathlineto{\pgfqpoint{3.612273in}{2.282667in}}%
\pgfpathlineto{\pgfqpoint{3.665908in}{2.226788in}}%
\pgfpathlineto{\pgfqpoint{3.725899in}{2.164459in}}%
\pgfpathlineto{\pgfqpoint{3.859406in}{2.021333in}}%
\pgfpathlineto{\pgfqpoint{3.893590in}{1.984000in}}%
\pgfpathlineto{\pgfqpoint{4.006465in}{1.858240in}}%
\pgfpathlineto{\pgfqpoint{4.089824in}{1.762978in}}%
\pgfpathlineto{\pgfqpoint{4.126707in}{1.720311in}}%
\pgfpathlineto{\pgfqpoint{4.188054in}{1.648000in}}%
\pgfpathlineto{\pgfqpoint{4.287030in}{1.528980in}}%
\pgfpathlineto{\pgfqpoint{4.341916in}{1.461333in}}%
\pgfpathlineto{\pgfqpoint{4.430866in}{1.349333in}}%
\pgfpathlineto{\pgfqpoint{4.517304in}{1.237333in}}%
\pgfpathlineto{\pgfqpoint{4.607677in}{1.116447in}}%
\pgfpathlineto{\pgfqpoint{4.687838in}{1.005556in}}%
\pgfpathlineto{\pgfqpoint{4.768000in}{0.890700in}}%
\pgfpathlineto{\pgfqpoint{4.768000in}{0.901333in}}%
\pgfpathlineto{\pgfqpoint{4.768000in}{0.901333in}}%
\pgfusepath{fill}%
\end{pgfscope}%
\begin{pgfscope}%
\pgfpathrectangle{\pgfqpoint{0.800000in}{0.528000in}}{\pgfqpoint{3.968000in}{3.696000in}}%
\pgfusepath{clip}%
\pgfsetbuttcap%
\pgfsetroundjoin%
\definecolor{currentfill}{rgb}{0.282884,0.135920,0.453427}%
\pgfsetfillcolor{currentfill}%
\pgfsetlinewidth{0.000000pt}%
\definecolor{currentstroke}{rgb}{0.000000,0.000000,0.000000}%
\pgfsetstrokecolor{currentstroke}%
\pgfsetdash{}{0pt}%
\pgfpathmoveto{\pgfqpoint{2.631667in}{0.528000in}}%
\pgfpathlineto{\pgfqpoint{2.512365in}{0.640000in}}%
\pgfpathlineto{\pgfqpoint{2.396197in}{0.752000in}}%
\pgfpathlineto{\pgfqpoint{2.358137in}{0.789333in}}%
\pgfpathlineto{\pgfqpoint{2.242909in}{0.904400in}}%
\pgfpathlineto{\pgfqpoint{2.172814in}{0.976000in}}%
\pgfpathlineto{\pgfqpoint{2.065325in}{1.088000in}}%
\pgfpathlineto{\pgfqpoint{1.960342in}{1.200000in}}%
\pgfpathlineto{\pgfqpoint{1.891967in}{1.274667in}}%
\pgfpathlineto{\pgfqpoint{1.791413in}{1.386667in}}%
\pgfpathlineto{\pgfqpoint{1.742467in}{1.442138in}}%
\pgfpathlineto{\pgfqpoint{1.641697in}{1.558892in}}%
\pgfpathlineto{\pgfqpoint{1.535554in}{1.685333in}}%
\pgfpathlineto{\pgfqpoint{1.481374in}{1.751308in}}%
\pgfpathlineto{\pgfqpoint{1.401212in}{1.851056in}}%
\pgfpathlineto{\pgfqpoint{1.321051in}{1.953276in}}%
\pgfpathlineto{\pgfqpoint{1.268945in}{2.021333in}}%
\pgfpathlineto{\pgfqpoint{1.185204in}{2.133333in}}%
\pgfpathlineto{\pgfqpoint{1.151497in}{2.179264in}}%
\pgfpathlineto{\pgfqpoint{1.071770in}{2.290859in}}%
\pgfpathlineto{\pgfqpoint{1.000043in}{2.394667in}}%
\pgfpathlineto{\pgfqpoint{0.840081in}{2.640986in}}%
\pgfpathlineto{\pgfqpoint{0.800000in}{2.706552in}}%
\pgfpathlineto{\pgfqpoint{0.800000in}{2.693213in}}%
\pgfpathlineto{\pgfqpoint{0.880162in}{2.564969in}}%
\pgfpathlineto{\pgfqpoint{0.949474in}{2.459228in}}%
\pgfpathlineto{\pgfqpoint{0.992569in}{2.394667in}}%
\pgfpathlineto{\pgfqpoint{1.027094in}{2.344860in}}%
\pgfpathlineto{\pgfqpoint{1.070151in}{2.282667in}}%
\pgfpathlineto{\pgfqpoint{1.096647in}{2.245333in}}%
\pgfpathlineto{\pgfqpoint{1.177838in}{2.133333in}}%
\pgfpathlineto{\pgfqpoint{1.261656in}{2.021333in}}%
\pgfpathlineto{\pgfqpoint{1.303268in}{1.967437in}}%
\pgfpathlineto{\pgfqpoint{1.348019in}{1.909333in}}%
\pgfpathlineto{\pgfqpoint{1.441293in}{1.791821in}}%
\pgfpathlineto{\pgfqpoint{1.497591in}{1.722667in}}%
\pgfpathlineto{\pgfqpoint{1.601616in}{1.597614in}}%
\pgfpathlineto{\pgfqpoint{1.654034in}{1.536000in}}%
\pgfpathlineto{\pgfqpoint{1.761939in}{1.411796in}}%
\pgfpathlineto{\pgfqpoint{1.817407in}{1.349333in}}%
\pgfpathlineto{\pgfqpoint{1.922263in}{1.233423in}}%
\pgfpathlineto{\pgfqpoint{1.987771in}{1.162667in}}%
\pgfpathlineto{\pgfqpoint{2.093423in}{1.050667in}}%
\pgfpathlineto{\pgfqpoint{2.202828in}{0.937475in}}%
\pgfpathlineto{\pgfqpoint{2.279154in}{0.860427in}}%
\pgfpathlineto{\pgfqpoint{2.323071in}{0.816518in}}%
\pgfpathlineto{\pgfqpoint{2.443313in}{0.698759in}}%
\pgfpathlineto{\pgfqpoint{2.514043in}{0.631215in}}%
\pgfpathlineto{\pgfqpoint{2.563556in}{0.584145in}}%
\pgfpathlineto{\pgfqpoint{2.623706in}{0.528000in}}%
\pgfpathmoveto{\pgfqpoint{4.768000in}{0.912632in}}%
\pgfpathlineto{\pgfqpoint{4.687838in}{1.026413in}}%
\pgfpathlineto{\pgfqpoint{4.503389in}{1.274667in}}%
\pgfpathlineto{\pgfqpoint{4.462930in}{1.326509in}}%
\pgfpathlineto{\pgfqpoint{4.416097in}{1.386667in}}%
\pgfpathlineto{\pgfqpoint{4.326346in}{1.498667in}}%
\pgfpathlineto{\pgfqpoint{4.256961in}{1.582659in}}%
\pgfpathlineto{\pgfqpoint{4.202643in}{1.648000in}}%
\pgfpathlineto{\pgfqpoint{4.139090in}{1.722667in}}%
\pgfpathlineto{\pgfqpoint{4.041868in}{1.834667in}}%
\pgfpathlineto{\pgfqpoint{3.971333in}{1.913943in}}%
\pgfpathlineto{\pgfqpoint{3.926303in}{1.964046in}}%
\pgfpathlineto{\pgfqpoint{3.798792in}{2.102770in}}%
\pgfpathlineto{\pgfqpoint{3.685818in}{2.222003in}}%
\pgfpathlineto{\pgfqpoint{3.627192in}{2.282667in}}%
\pgfpathlineto{\pgfqpoint{3.516945in}{2.394667in}}%
\pgfpathlineto{\pgfqpoint{3.403890in}{2.506667in}}%
\pgfpathlineto{\pgfqpoint{3.345410in}{2.562926in}}%
\pgfpathlineto{\pgfqpoint{3.285010in}{2.621279in}}%
\pgfpathlineto{\pgfqpoint{3.226452in}{2.676123in}}%
\pgfpathlineto{\pgfqpoint{3.164768in}{2.734104in}}%
\pgfpathlineto{\pgfqpoint{3.044525in}{2.843899in}}%
\pgfpathlineto{\pgfqpoint{2.983563in}{2.897883in}}%
\pgfpathlineto{\pgfqpoint{2.919622in}{2.954667in}}%
\pgfpathlineto{\pgfqpoint{2.789386in}{3.066667in}}%
\pgfpathlineto{\pgfqpoint{2.683798in}{3.154955in}}%
\pgfpathlineto{\pgfqpoint{2.562782in}{3.253333in}}%
\pgfpathlineto{\pgfqpoint{2.443313in}{3.347151in}}%
\pgfpathlineto{\pgfqpoint{2.403232in}{3.377983in}}%
\pgfpathlineto{\pgfqpoint{2.282990in}{3.468316in}}%
\pgfpathlineto{\pgfqpoint{2.162747in}{3.555379in}}%
\pgfpathlineto{\pgfqpoint{2.060330in}{3.626667in}}%
\pgfpathlineto{\pgfqpoint{1.962343in}{3.692611in}}%
\pgfpathlineto{\pgfqpoint{1.909461in}{3.726743in}}%
\pgfpathlineto{\pgfqpoint{1.861882in}{3.757092in}}%
\pgfpathlineto{\pgfqpoint{1.832543in}{3.776000in}}%
\pgfpathlineto{\pgfqpoint{1.721859in}{3.843462in}}%
\pgfpathlineto{\pgfqpoint{1.641697in}{3.890079in}}%
\pgfpathlineto{\pgfqpoint{1.447709in}{3.994024in}}%
\pgfpathlineto{\pgfqpoint{1.360137in}{4.036407in}}%
\pgfpathlineto{\pgfqpoint{1.321051in}{4.054234in}}%
\pgfpathlineto{\pgfqpoint{1.251177in}{4.084250in}}%
\pgfpathlineto{\pgfqpoint{1.213525in}{4.100155in}}%
\pgfpathlineto{\pgfqpoint{1.145461in}{4.126220in}}%
\pgfpathlineto{\pgfqpoint{1.078695in}{4.149333in}}%
\pgfpathlineto{\pgfqpoint{0.988429in}{4.175512in}}%
\pgfpathlineto{\pgfqpoint{0.945586in}{4.186667in}}%
\pgfpathlineto{\pgfqpoint{0.920242in}{4.192440in}}%
\pgfpathlineto{\pgfqpoint{0.840081in}{4.206485in}}%
\pgfpathlineto{\pgfqpoint{0.800000in}{4.211164in}}%
\pgfpathlineto{\pgfqpoint{0.800000in}{4.198044in}}%
\pgfpathlineto{\pgfqpoint{0.840081in}{4.193946in}}%
\pgfpathlineto{\pgfqpoint{0.888203in}{4.186667in}}%
\pgfpathlineto{\pgfqpoint{0.920242in}{4.180659in}}%
\pgfpathlineto{\pgfqpoint{1.000404in}{4.161661in}}%
\pgfpathlineto{\pgfqpoint{1.044345in}{4.149333in}}%
\pgfpathlineto{\pgfqpoint{1.120646in}{4.124668in}}%
\pgfpathlineto{\pgfqpoint{1.160727in}{4.110501in}}%
\pgfpathlineto{\pgfqpoint{1.280970in}{4.062384in}}%
\pgfpathlineto{\pgfqpoint{1.353232in}{4.029976in}}%
\pgfpathlineto{\pgfqpoint{1.384109in}{4.015931in}}%
\pgfpathlineto{\pgfqpoint{1.468667in}{3.974502in}}%
\pgfpathlineto{\pgfqpoint{1.562664in}{3.925333in}}%
\pgfpathlineto{\pgfqpoint{1.681778in}{3.858590in}}%
\pgfpathlineto{\pgfqpoint{1.768191in}{3.807510in}}%
\pgfpathlineto{\pgfqpoint{1.882182in}{3.736698in}}%
\pgfpathlineto{\pgfqpoint{2.122667in}{3.575813in}}%
\pgfpathlineto{\pgfqpoint{2.208937in}{3.514667in}}%
\pgfpathlineto{\pgfqpoint{2.458360in}{3.328000in}}%
\pgfpathlineto{\pgfqpoint{2.723879in}{3.114364in}}%
\pgfpathlineto{\pgfqpoint{2.844121in}{3.012647in}}%
\pgfpathlineto{\pgfqpoint{2.911110in}{2.954667in}}%
\pgfpathlineto{\pgfqpoint{3.004444in}{2.872380in}}%
\pgfpathlineto{\pgfqpoint{3.044525in}{2.836472in}}%
\pgfpathlineto{\pgfqpoint{3.164768in}{2.726698in}}%
\pgfpathlineto{\pgfqpoint{3.285010in}{2.613807in}}%
\pgfpathlineto{\pgfqpoint{3.357713in}{2.544000in}}%
\pgfpathlineto{\pgfqpoint{3.445333in}{2.458348in}}%
\pgfpathlineto{\pgfqpoint{3.565576in}{2.337979in}}%
\pgfpathlineto{\pgfqpoint{3.645737in}{2.255904in}}%
\pgfpathlineto{\pgfqpoint{3.765980in}{2.129906in}}%
\pgfpathlineto{\pgfqpoint{3.886222in}{2.000090in}}%
\pgfpathlineto{\pgfqpoint{4.006465in}{1.866542in}}%
\pgfpathlineto{\pgfqpoint{4.075785in}{1.787236in}}%
\pgfpathlineto{\pgfqpoint{4.126707in}{1.728740in}}%
\pgfpathlineto{\pgfqpoint{4.226648in}{1.610667in}}%
\pgfpathlineto{\pgfqpoint{4.327111in}{1.488707in}}%
\pgfpathlineto{\pgfqpoint{4.438217in}{1.349333in}}%
\pgfpathlineto{\pgfqpoint{4.527515in}{1.233666in}}%
\pgfpathlineto{\pgfqpoint{4.580840in}{1.162667in}}%
\pgfpathlineto{\pgfqpoint{4.662948in}{1.050667in}}%
\pgfpathlineto{\pgfqpoint{4.704717in}{0.991721in}}%
\pgfpathlineto{\pgfqpoint{4.742377in}{0.938667in}}%
\pgfpathlineto{\pgfqpoint{4.768000in}{0.901882in}}%
\pgfpathlineto{\pgfqpoint{4.768000in}{0.901882in}}%
\pgfusepath{fill}%
\end{pgfscope}%
\begin{pgfscope}%
\pgfpathrectangle{\pgfqpoint{0.800000in}{0.528000in}}{\pgfqpoint{3.968000in}{3.696000in}}%
\pgfusepath{clip}%
\pgfsetbuttcap%
\pgfsetroundjoin%
\definecolor{currentfill}{rgb}{0.282884,0.135920,0.453427}%
\pgfsetfillcolor{currentfill}%
\pgfsetlinewidth{0.000000pt}%
\definecolor{currentstroke}{rgb}{0.000000,0.000000,0.000000}%
\pgfsetstrokecolor{currentstroke}%
\pgfsetdash{}{0pt}%
\pgfpathmoveto{\pgfqpoint{2.623706in}{0.528000in}}%
\pgfpathlineto{\pgfqpoint{2.504575in}{0.640000in}}%
\pgfpathlineto{\pgfqpoint{2.388572in}{0.752000in}}%
\pgfpathlineto{\pgfqpoint{2.312877in}{0.826667in}}%
\pgfpathlineto{\pgfqpoint{2.201663in}{0.938667in}}%
\pgfpathlineto{\pgfqpoint{2.126154in}{1.016582in}}%
\pgfpathlineto{\pgfqpoint{2.082586in}{1.062011in}}%
\pgfpathlineto{\pgfqpoint{1.953095in}{1.200000in}}%
\pgfpathlineto{\pgfqpoint{1.898073in}{1.259865in}}%
\pgfpathlineto{\pgfqpoint{1.802020in}{1.366563in}}%
\pgfpathlineto{\pgfqpoint{1.718463in}{1.461333in}}%
\pgfpathlineto{\pgfqpoint{1.622214in}{1.573333in}}%
\pgfpathlineto{\pgfqpoint{1.578061in}{1.626059in}}%
\pgfpathlineto{\pgfqpoint{1.528272in}{1.685333in}}%
\pgfpathlineto{\pgfqpoint{1.456201in}{1.773886in}}%
\pgfpathlineto{\pgfqpoint{1.406951in}{1.834667in}}%
\pgfpathlineto{\pgfqpoint{1.361131in}{1.892604in}}%
\pgfpathlineto{\pgfqpoint{1.280970in}{1.995949in}}%
\pgfpathlineto{\pgfqpoint{1.096647in}{2.245333in}}%
\pgfpathlineto{\pgfqpoint{1.018168in}{2.357333in}}%
\pgfpathlineto{\pgfqpoint{0.942493in}{2.469333in}}%
\pgfpathlineto{\pgfqpoint{0.880162in}{2.564969in}}%
\pgfpathlineto{\pgfqpoint{0.800000in}{2.693213in}}%
\pgfpathlineto{\pgfqpoint{0.800000in}{2.680508in}}%
\pgfpathlineto{\pgfqpoint{0.862019in}{2.581333in}}%
\pgfpathlineto{\pgfqpoint{0.894530in}{2.530617in}}%
\pgfpathlineto{\pgfqpoint{0.960323in}{2.431064in}}%
\pgfpathlineto{\pgfqpoint{1.143130in}{2.170667in}}%
\pgfpathlineto{\pgfqpoint{1.226130in}{2.058667in}}%
\pgfpathlineto{\pgfqpoint{1.321051in}{1.934598in}}%
\pgfpathlineto{\pgfqpoint{1.429711in}{1.797333in}}%
\pgfpathlineto{\pgfqpoint{1.521455in}{1.684789in}}%
\pgfpathlineto{\pgfqpoint{1.646780in}{1.536000in}}%
\pgfpathlineto{\pgfqpoint{1.744028in}{1.424000in}}%
\pgfpathlineto{\pgfqpoint{1.843550in}{1.312000in}}%
\pgfpathlineto{\pgfqpoint{1.911480in}{1.237333in}}%
\pgfpathlineto{\pgfqpoint{2.015369in}{1.125333in}}%
\pgfpathlineto{\pgfqpoint{2.084404in}{1.052360in}}%
\pgfpathlineto{\pgfqpoint{2.122667in}{1.012360in}}%
\pgfpathlineto{\pgfqpoint{2.198420in}{0.934561in}}%
\pgfpathlineto{\pgfqpoint{2.242909in}{0.889282in}}%
\pgfpathlineto{\pgfqpoint{2.314034in}{0.818250in}}%
\pgfpathlineto{\pgfqpoint{2.363152in}{0.769439in}}%
\pgfpathlineto{\pgfqpoint{2.450791in}{0.684298in}}%
\pgfpathlineto{\pgfqpoint{2.496786in}{0.640000in}}%
\pgfpathlineto{\pgfqpoint{2.589837in}{0.552480in}}%
\pgfpathlineto{\pgfqpoint{2.615745in}{0.528000in}}%
\pgfpathmoveto{\pgfqpoint{4.768000in}{0.923382in}}%
\pgfpathlineto{\pgfqpoint{4.687838in}{1.036690in}}%
\pgfpathlineto{\pgfqpoint{4.510642in}{1.274667in}}%
\pgfpathlineto{\pgfqpoint{4.467109in}{1.330401in}}%
\pgfpathlineto{\pgfqpoint{4.423274in}{1.386667in}}%
\pgfpathlineto{\pgfqpoint{4.327111in}{1.506476in}}%
\pgfpathlineto{\pgfqpoint{4.260969in}{1.586392in}}%
\pgfpathlineto{\pgfqpoint{4.206869in}{1.651529in}}%
\pgfpathlineto{\pgfqpoint{4.146261in}{1.722667in}}%
\pgfpathlineto{\pgfqpoint{4.046545in}{1.837576in}}%
\pgfpathlineto{\pgfqpoint{3.975204in}{1.917549in}}%
\pgfpathlineto{\pgfqpoint{3.926303in}{1.972063in}}%
\pgfpathlineto{\pgfqpoint{3.806061in}{2.102761in}}%
\pgfpathlineto{\pgfqpoint{3.670755in}{2.245333in}}%
\pgfpathlineto{\pgfqpoint{3.561561in}{2.357333in}}%
\pgfpathlineto{\pgfqpoint{3.511931in}{2.407301in}}%
\pgfpathlineto{\pgfqpoint{3.405253in}{2.512753in}}%
\pgfpathlineto{\pgfqpoint{3.334451in}{2.581333in}}%
\pgfpathlineto{\pgfqpoint{3.244929in}{2.666551in}}%
\pgfpathlineto{\pgfqpoint{3.170466in}{2.735975in}}%
\pgfpathlineto{\pgfqpoint{3.124687in}{2.778320in}}%
\pgfpathlineto{\pgfqpoint{3.004444in}{2.887077in}}%
\pgfpathlineto{\pgfqpoint{2.946464in}{2.937994in}}%
\pgfpathlineto{\pgfqpoint{2.884202in}{2.992870in}}%
\pgfpathlineto{\pgfqpoint{2.801174in}{3.063997in}}%
\pgfpathlineto{\pgfqpoint{2.753840in}{3.104000in}}%
\pgfpathlineto{\pgfqpoint{2.643717in}{3.195195in}}%
\pgfpathlineto{\pgfqpoint{2.588712in}{3.239432in}}%
\pgfpathlineto{\pgfqpoint{2.523475in}{3.291975in}}%
\pgfpathlineto{\pgfqpoint{2.429389in}{3.365333in}}%
\pgfpathlineto{\pgfqpoint{2.202828in}{3.534270in}}%
\pgfpathlineto{\pgfqpoint{2.082586in}{3.619168in}}%
\pgfpathlineto{\pgfqpoint{2.031535in}{3.653782in}}%
\pgfpathlineto{\pgfqpoint{1.965175in}{3.698696in}}%
\pgfpathlineto{\pgfqpoint{1.882182in}{3.752611in}}%
\pgfpathlineto{\pgfqpoint{1.834997in}{3.782617in}}%
\pgfpathlineto{\pgfqpoint{1.718525in}{3.853771in}}%
\pgfpathlineto{\pgfqpoint{1.601616in}{3.921074in}}%
\pgfpathlineto{\pgfqpoint{1.521455in}{3.964743in}}%
\pgfpathlineto{\pgfqpoint{1.344329in}{4.052984in}}%
\pgfpathlineto{\pgfqpoint{1.280970in}{4.081468in}}%
\pgfpathlineto{\pgfqpoint{1.127608in}{4.142849in}}%
\pgfpathlineto{\pgfqpoint{1.080566in}{4.159067in}}%
\pgfpathlineto{\pgfqpoint{0.948723in}{4.197471in}}%
\pgfpathlineto{\pgfqpoint{0.880162in}{4.212185in}}%
\pgfpathlineto{\pgfqpoint{0.840081in}{4.219024in}}%
\pgfpathlineto{\pgfqpoint{0.800000in}{4.224000in}}%
\pgfpathlineto{\pgfqpoint{0.800000in}{4.211164in}}%
\pgfpathlineto{\pgfqpoint{0.840081in}{4.206485in}}%
\pgfpathlineto{\pgfqpoint{0.880162in}{4.200177in}}%
\pgfpathlineto{\pgfqpoint{0.891830in}{4.197536in}}%
\pgfpathlineto{\pgfqpoint{0.920242in}{4.192440in}}%
\pgfpathlineto{\pgfqpoint{0.960323in}{4.183300in}}%
\pgfpathlineto{\pgfqpoint{1.049786in}{4.157997in}}%
\pgfpathlineto{\pgfqpoint{1.080566in}{4.148750in}}%
\pgfpathlineto{\pgfqpoint{1.213525in}{4.100155in}}%
\pgfpathlineto{\pgfqpoint{1.280970in}{4.072051in}}%
\pgfpathlineto{\pgfqpoint{1.447709in}{3.994024in}}%
\pgfpathlineto{\pgfqpoint{1.538040in}{3.947218in}}%
\pgfpathlineto{\pgfqpoint{1.645304in}{3.888000in}}%
\pgfpathlineto{\pgfqpoint{1.842101in}{3.770062in}}%
\pgfpathlineto{\pgfqpoint{1.922263in}{3.718809in}}%
\pgfpathlineto{\pgfqpoint{2.005423in}{3.664000in}}%
\pgfpathlineto{\pgfqpoint{2.242909in}{3.497657in}}%
\pgfpathlineto{\pgfqpoint{2.299680in}{3.455546in}}%
\pgfpathlineto{\pgfqpoint{2.363152in}{3.408495in}}%
\pgfpathlineto{\pgfqpoint{2.432469in}{3.355233in}}%
\pgfpathlineto{\pgfqpoint{2.483394in}{3.315997in}}%
\pgfpathlineto{\pgfqpoint{2.609133in}{3.216000in}}%
\pgfpathlineto{\pgfqpoint{2.712586in}{3.130815in}}%
\pgfpathlineto{\pgfqpoint{2.763960in}{3.088152in}}%
\pgfpathlineto{\pgfqpoint{2.884202in}{2.985488in}}%
\pgfpathlineto{\pgfqpoint{3.004444in}{2.879828in}}%
\pgfpathlineto{\pgfqpoint{3.045890in}{2.842667in}}%
\pgfpathlineto{\pgfqpoint{3.168471in}{2.730667in}}%
\pgfpathlineto{\pgfqpoint{3.287750in}{2.618667in}}%
\pgfpathlineto{\pgfqpoint{3.345410in}{2.562926in}}%
\pgfpathlineto{\pgfqpoint{3.405253in}{2.505338in}}%
\pgfpathlineto{\pgfqpoint{3.525495in}{2.386099in}}%
\pgfpathlineto{\pgfqpoint{3.590745in}{2.320000in}}%
\pgfpathlineto{\pgfqpoint{3.699214in}{2.208000in}}%
\pgfpathlineto{\pgfqpoint{3.806061in}{2.094995in}}%
\pgfpathlineto{\pgfqpoint{3.926303in}{1.964046in}}%
\pgfpathlineto{\pgfqpoint{4.171040in}{1.685333in}}%
\pgfpathlineto{\pgfqpoint{4.233894in}{1.610667in}}%
\pgfpathlineto{\pgfqpoint{4.327111in}{1.497724in}}%
\pgfpathlineto{\pgfqpoint{4.447354in}{1.347049in}}%
\pgfpathlineto{\pgfqpoint{4.513250in}{1.261380in}}%
\pgfpathlineto{\pgfqpoint{4.560255in}{1.200000in}}%
\pgfpathlineto{\pgfqpoint{4.647758in}{1.081937in}}%
\pgfpathlineto{\pgfqpoint{4.768000in}{0.912632in}}%
\pgfpathlineto{\pgfqpoint{4.768000in}{0.912632in}}%
\pgfusepath{fill}%
\end{pgfscope}%
\begin{pgfscope}%
\pgfpathrectangle{\pgfqpoint{0.800000in}{0.528000in}}{\pgfqpoint{3.968000in}{3.696000in}}%
\pgfusepath{clip}%
\pgfsetbuttcap%
\pgfsetroundjoin%
\definecolor{currentfill}{rgb}{0.282623,0.140926,0.457517}%
\pgfsetfillcolor{currentfill}%
\pgfsetlinewidth{0.000000pt}%
\definecolor{currentstroke}{rgb}{0.000000,0.000000,0.000000}%
\pgfsetstrokecolor{currentstroke}%
\pgfsetdash{}{0pt}%
\pgfpathmoveto{\pgfqpoint{2.615745in}{0.528000in}}%
\pgfpathlineto{\pgfqpoint{2.523475in}{0.614586in}}%
\pgfpathlineto{\pgfqpoint{2.450791in}{0.684298in}}%
\pgfpathlineto{\pgfqpoint{2.403232in}{0.730200in}}%
\pgfpathlineto{\pgfqpoint{2.268033in}{0.864000in}}%
\pgfpathlineto{\pgfqpoint{2.179480in}{0.954252in}}%
\pgfpathlineto{\pgfqpoint{2.121731in}{1.013333in}}%
\pgfpathlineto{\pgfqpoint{2.002424in}{1.139125in}}%
\pgfpathlineto{\pgfqpoint{1.877368in}{1.274667in}}%
\pgfpathlineto{\pgfqpoint{1.806507in}{1.353513in}}%
\pgfpathlineto{\pgfqpoint{1.761939in}{1.403633in}}%
\pgfpathlineto{\pgfqpoint{1.641697in}{1.541939in}}%
\pgfpathlineto{\pgfqpoint{1.519497in}{1.687157in}}%
\pgfpathlineto{\pgfqpoint{1.395287in}{1.840185in}}%
\pgfpathlineto{\pgfqpoint{1.280970in}{1.986369in}}%
\pgfpathlineto{\pgfqpoint{1.215909in}{2.072733in}}%
\pgfpathlineto{\pgfqpoint{1.170472in}{2.133333in}}%
\pgfpathlineto{\pgfqpoint{1.143130in}{2.170667in}}%
\pgfpathlineto{\pgfqpoint{1.062756in}{2.282667in}}%
\pgfpathlineto{\pgfqpoint{0.985094in}{2.394667in}}%
\pgfpathlineto{\pgfqpoint{0.944845in}{2.454916in}}%
\pgfpathlineto{\pgfqpoint{0.910233in}{2.506667in}}%
\pgfpathlineto{\pgfqpoint{0.862019in}{2.581333in}}%
\pgfpathlineto{\pgfqpoint{0.800000in}{2.680508in}}%
\pgfpathlineto{\pgfqpoint{0.800000in}{2.667803in}}%
\pgfpathlineto{\pgfqpoint{0.854325in}{2.581333in}}%
\pgfpathlineto{\pgfqpoint{0.880162in}{2.541006in}}%
\pgfpathlineto{\pgfqpoint{0.960323in}{2.420081in}}%
\pgfpathlineto{\pgfqpoint{1.120646in}{2.191477in}}%
\pgfpathlineto{\pgfqpoint{1.200808in}{2.082707in}}%
\pgfpathlineto{\pgfqpoint{1.401212in}{1.823886in}}%
\pgfpathlineto{\pgfqpoint{1.513939in}{1.685333in}}%
\pgfpathlineto{\pgfqpoint{1.576299in}{1.610667in}}%
\pgfpathlineto{\pgfqpoint{1.681778in}{1.487103in}}%
\pgfpathlineto{\pgfqpoint{1.807048in}{1.344651in}}%
\pgfpathlineto{\pgfqpoint{1.922263in}{1.217710in}}%
\pgfpathlineto{\pgfqpoint{2.005451in}{1.128152in}}%
\pgfpathlineto{\pgfqpoint{2.047955in}{1.082923in}}%
\pgfpathlineto{\pgfqpoint{2.162747in}{0.963428in}}%
\pgfpathlineto{\pgfqpoint{2.252196in}{0.872651in}}%
\pgfpathlineto{\pgfqpoint{2.297837in}{0.826667in}}%
\pgfpathlineto{\pgfqpoint{2.373321in}{0.752000in}}%
\pgfpathlineto{\pgfqpoint{2.488996in}{0.640000in}}%
\pgfpathlineto{\pgfqpoint{2.585928in}{0.548839in}}%
\pgfpathlineto{\pgfqpoint{2.607784in}{0.528000in}}%
\pgfpathmoveto{\pgfqpoint{4.768000in}{0.934132in}}%
\pgfpathlineto{\pgfqpoint{4.711918in}{1.013333in}}%
\pgfpathlineto{\pgfqpoint{4.630470in}{1.125333in}}%
\pgfpathlineto{\pgfqpoint{4.546431in}{1.237333in}}%
\pgfpathlineto{\pgfqpoint{4.504933in}{1.290966in}}%
\pgfpathlineto{\pgfqpoint{4.459881in}{1.349333in}}%
\pgfpathlineto{\pgfqpoint{4.367192in}{1.465921in}}%
\pgfpathlineto{\pgfqpoint{4.246949in}{1.612333in}}%
\pgfpathlineto{\pgfqpoint{4.121352in}{1.760000in}}%
\pgfpathlineto{\pgfqpoint{4.056203in}{1.834667in}}%
\pgfpathlineto{\pgfqpoint{3.956474in}{1.946667in}}%
\pgfpathlineto{\pgfqpoint{3.902804in}{2.005888in}}%
\pgfpathlineto{\pgfqpoint{3.806061in}{2.110497in}}%
\pgfpathlineto{\pgfqpoint{3.678164in}{2.245333in}}%
\pgfpathlineto{\pgfqpoint{3.565576in}{2.360821in}}%
\pgfpathlineto{\pgfqpoint{3.490115in}{2.436379in}}%
\pgfpathlineto{\pgfqpoint{3.445333in}{2.480830in}}%
\pgfpathlineto{\pgfqpoint{3.325091in}{2.597691in}}%
\pgfpathlineto{\pgfqpoint{3.254334in}{2.664760in}}%
\pgfpathlineto{\pgfqpoint{3.204848in}{2.711456in}}%
\pgfpathlineto{\pgfqpoint{3.084606in}{2.822176in}}%
\pgfpathlineto{\pgfqpoint{2.964364in}{2.929900in}}%
\pgfpathlineto{\pgfqpoint{2.888626in}{2.996121in}}%
\pgfpathlineto{\pgfqpoint{2.844121in}{3.034676in}}%
\pgfpathlineto{\pgfqpoint{2.717959in}{3.141333in}}%
\pgfpathlineto{\pgfqpoint{2.627144in}{3.216000in}}%
\pgfpathlineto{\pgfqpoint{2.523475in}{3.299260in}}%
\pgfpathlineto{\pgfqpoint{2.403232in}{3.392882in}}%
\pgfpathlineto{\pgfqpoint{2.282990in}{3.483385in}}%
\pgfpathlineto{\pgfqpoint{2.202828in}{3.541862in}}%
\pgfpathlineto{\pgfqpoint{2.081484in}{3.627693in}}%
\pgfpathlineto{\pgfqpoint{1.962343in}{3.708318in}}%
\pgfpathlineto{\pgfqpoint{1.857920in}{3.776000in}}%
\pgfpathlineto{\pgfqpoint{1.798542in}{3.813333in}}%
\pgfpathlineto{\pgfqpoint{1.678823in}{3.885248in}}%
\pgfpathlineto{\pgfqpoint{1.629067in}{3.913569in}}%
\pgfpathlineto{\pgfqpoint{1.601616in}{3.929517in}}%
\pgfpathlineto{\pgfqpoint{1.425311in}{4.022447in}}%
\pgfpathlineto{\pgfqpoint{1.396207in}{4.037333in}}%
\pgfpathlineto{\pgfqpoint{1.317161in}{4.074667in}}%
\pgfpathlineto{\pgfqpoint{1.231614in}{4.112000in}}%
\pgfpathlineto{\pgfqpoint{1.120646in}{4.155393in}}%
\pgfpathlineto{\pgfqpoint{0.987983in}{4.198236in}}%
\pgfpathlineto{\pgfqpoint{0.920242in}{4.215479in}}%
\pgfpathlineto{\pgfqpoint{0.912571in}{4.216855in}}%
\pgfpathlineto{\pgfqpoint{0.880162in}{4.224000in}}%
\pgfpathlineto{\pgfqpoint{0.802066in}{4.224000in}}%
\pgfpathlineto{\pgfqpoint{0.846388in}{4.218125in}}%
\pgfpathlineto{\pgfqpoint{0.902201in}{4.207195in}}%
\pgfpathlineto{\pgfqpoint{0.920242in}{4.203960in}}%
\pgfpathlineto{\pgfqpoint{0.960323in}{4.194510in}}%
\pgfpathlineto{\pgfqpoint{1.004687in}{4.182677in}}%
\pgfpathlineto{\pgfqpoint{1.080566in}{4.159067in}}%
\pgfpathlineto{\pgfqpoint{1.117405in}{4.146314in}}%
\pgfpathlineto{\pgfqpoint{1.127608in}{4.142849in}}%
\pgfpathlineto{\pgfqpoint{1.208171in}{4.112000in}}%
\pgfpathlineto{\pgfqpoint{1.296233in}{4.074667in}}%
\pgfpathlineto{\pgfqpoint{1.401212in}{4.025862in}}%
\pgfpathlineto{\pgfqpoint{1.453248in}{4.000000in}}%
\pgfpathlineto{\pgfqpoint{1.525302in}{3.962667in}}%
\pgfpathlineto{\pgfqpoint{1.601616in}{3.921074in}}%
\pgfpathlineto{\pgfqpoint{1.802020in}{3.803064in}}%
\pgfpathlineto{\pgfqpoint{1.903815in}{3.738667in}}%
\pgfpathlineto{\pgfqpoint{2.082586in}{3.619168in}}%
\pgfpathlineto{\pgfqpoint{2.178074in}{3.552000in}}%
\pgfpathlineto{\pgfqpoint{2.282990in}{3.475937in}}%
\pgfpathlineto{\pgfqpoint{2.370881in}{3.409867in}}%
\pgfpathlineto{\pgfqpoint{2.429389in}{3.365333in}}%
\pgfpathlineto{\pgfqpoint{2.663870in}{3.178667in}}%
\pgfpathlineto{\pgfqpoint{2.763960in}{3.095518in}}%
\pgfpathlineto{\pgfqpoint{2.885202in}{2.992000in}}%
\pgfpathlineto{\pgfqpoint{3.012353in}{2.880000in}}%
\pgfpathlineto{\pgfqpoint{3.095107in}{2.805333in}}%
\pgfpathlineto{\pgfqpoint{3.216391in}{2.693333in}}%
\pgfpathlineto{\pgfqpoint{3.309884in}{2.604502in}}%
\pgfpathlineto{\pgfqpoint{3.365172in}{2.551719in}}%
\pgfpathlineto{\pgfqpoint{3.487194in}{2.432000in}}%
\pgfpathlineto{\pgfqpoint{3.605657in}{2.312450in}}%
\pgfpathlineto{\pgfqpoint{3.725899in}{2.187759in}}%
\pgfpathlineto{\pgfqpoint{3.812363in}{2.096000in}}%
\pgfpathlineto{\pgfqpoint{3.926303in}{1.972063in}}%
\pgfpathlineto{\pgfqpoint{4.006465in}{1.882807in}}%
\pgfpathlineto{\pgfqpoint{4.114133in}{1.760000in}}%
\pgfpathlineto{\pgfqpoint{4.190629in}{1.670206in}}%
\pgfpathlineto{\pgfqpoint{4.241140in}{1.610667in}}%
\pgfpathlineto{\pgfqpoint{4.333472in}{1.498667in}}%
\pgfpathlineto{\pgfqpoint{4.423274in}{1.386667in}}%
\pgfpathlineto{\pgfqpoint{4.467109in}{1.330401in}}%
\pgfpathlineto{\pgfqpoint{4.510642in}{1.274667in}}%
\pgfpathlineto{\pgfqpoint{4.550760in}{1.221652in}}%
\pgfpathlineto{\pgfqpoint{4.595498in}{1.162667in}}%
\pgfpathlineto{\pgfqpoint{4.677764in}{1.050667in}}%
\pgfpathlineto{\pgfqpoint{4.713684in}{1.000074in}}%
\pgfpathlineto{\pgfqpoint{4.757353in}{0.938667in}}%
\pgfpathlineto{\pgfqpoint{4.768000in}{0.923382in}}%
\pgfpathlineto{\pgfqpoint{4.768000in}{0.923382in}}%
\pgfusepath{fill}%
\end{pgfscope}%
\begin{pgfscope}%
\pgfpathrectangle{\pgfqpoint{0.800000in}{0.528000in}}{\pgfqpoint{3.968000in}{3.696000in}}%
\pgfusepath{clip}%
\pgfsetbuttcap%
\pgfsetroundjoin%
\definecolor{currentfill}{rgb}{0.282623,0.140926,0.457517}%
\pgfsetfillcolor{currentfill}%
\pgfsetlinewidth{0.000000pt}%
\definecolor{currentstroke}{rgb}{0.000000,0.000000,0.000000}%
\pgfsetstrokecolor{currentstroke}%
\pgfsetdash{}{0pt}%
\pgfpathmoveto{\pgfqpoint{2.607784in}{0.528000in}}%
\pgfpathlineto{\pgfqpoint{2.523475in}{0.607168in}}%
\pgfpathlineto{\pgfqpoint{2.446808in}{0.680589in}}%
\pgfpathlineto{\pgfqpoint{2.403232in}{0.722741in}}%
\pgfpathlineto{\pgfqpoint{2.282990in}{0.841477in}}%
\pgfpathlineto{\pgfqpoint{2.194614in}{0.931016in}}%
\pgfpathlineto{\pgfqpoint{2.150553in}{0.976000in}}%
\pgfpathlineto{\pgfqpoint{2.080529in}{1.048751in}}%
\pgfpathlineto{\pgfqpoint{2.042505in}{1.088669in}}%
\pgfpathlineto{\pgfqpoint{1.968224in}{1.168144in}}%
\pgfpathlineto{\pgfqpoint{1.922263in}{1.217710in}}%
\pgfpathlineto{\pgfqpoint{1.802020in}{1.350271in}}%
\pgfpathlineto{\pgfqpoint{1.736849in}{1.424000in}}%
\pgfpathlineto{\pgfqpoint{1.631334in}{1.545653in}}%
\pgfpathlineto{\pgfqpoint{1.513939in}{1.685333in}}%
\pgfpathlineto{\pgfqpoint{1.422573in}{1.797333in}}%
\pgfpathlineto{\pgfqpoint{1.333594in}{1.909333in}}%
\pgfpathlineto{\pgfqpoint{1.240889in}{2.029489in}}%
\pgfpathlineto{\pgfqpoint{1.135813in}{2.170667in}}%
\pgfpathlineto{\pgfqpoint{1.055361in}{2.282667in}}%
\pgfpathlineto{\pgfqpoint{0.977619in}{2.394667in}}%
\pgfpathlineto{\pgfqpoint{0.940215in}{2.450603in}}%
\pgfpathlineto{\pgfqpoint{0.902677in}{2.506667in}}%
\pgfpathlineto{\pgfqpoint{0.840081in}{2.603709in}}%
\pgfpathlineto{\pgfqpoint{0.800000in}{2.667803in}}%
\pgfpathlineto{\pgfqpoint{0.800000in}{2.655140in}}%
\pgfpathlineto{\pgfqpoint{0.960323in}{2.409098in}}%
\pgfpathlineto{\pgfqpoint{1.109323in}{2.197453in}}%
\pgfpathlineto{\pgfqpoint{1.155889in}{2.133333in}}%
\pgfpathlineto{\pgfqpoint{1.207274in}{2.064689in}}%
\pgfpathlineto{\pgfqpoint{1.240889in}{2.019937in}}%
\pgfpathlineto{\pgfqpoint{1.307482in}{1.934029in}}%
\pgfpathlineto{\pgfqpoint{1.358087in}{1.869164in}}%
\pgfpathlineto{\pgfqpoint{1.401212in}{1.815013in}}%
\pgfpathlineto{\pgfqpoint{1.506873in}{1.685333in}}%
\pgfpathlineto{\pgfqpoint{1.569140in}{1.610667in}}%
\pgfpathlineto{\pgfqpoint{1.664697in}{1.498667in}}%
\pgfpathlineto{\pgfqpoint{1.729670in}{1.424000in}}%
\pgfpathlineto{\pgfqpoint{1.842101in}{1.297786in}}%
\pgfpathlineto{\pgfqpoint{1.927215in}{1.204613in}}%
\pgfpathlineto{\pgfqpoint{1.965883in}{1.162667in}}%
\pgfpathlineto{\pgfqpoint{2.082586in}{1.038971in}}%
\pgfpathlineto{\pgfqpoint{2.216181in}{0.901333in}}%
\pgfpathlineto{\pgfqpoint{2.306362in}{0.811103in}}%
\pgfpathlineto{\pgfqpoint{2.365695in}{0.752000in}}%
\pgfpathlineto{\pgfqpoint{2.483394in}{0.637970in}}%
\pgfpathlineto{\pgfqpoint{2.560044in}{0.565333in}}%
\pgfpathlineto{\pgfqpoint{2.599947in}{0.528000in}}%
\pgfpathlineto{\pgfqpoint{2.603636in}{0.528000in}}%
\pgfpathmoveto{\pgfqpoint{4.768000in}{0.944639in}}%
\pgfpathlineto{\pgfqpoint{4.581947in}{1.200000in}}%
\pgfpathlineto{\pgfqpoint{4.487434in}{1.323269in}}%
\pgfpathlineto{\pgfqpoint{4.268986in}{1.593859in}}%
\pgfpathlineto{\pgfqpoint{4.224000in}{1.648000in}}%
\pgfpathlineto{\pgfqpoint{4.160603in}{1.722667in}}%
\pgfpathlineto{\pgfqpoint{4.063300in}{1.834667in}}%
\pgfpathlineto{\pgfqpoint{3.963713in}{1.946667in}}%
\pgfpathlineto{\pgfqpoint{3.891322in}{2.026084in}}%
\pgfpathlineto{\pgfqpoint{3.846141in}{2.075197in}}%
\pgfpathlineto{\pgfqpoint{3.721289in}{2.208000in}}%
\pgfpathlineto{\pgfqpoint{3.605657in}{2.327556in}}%
\pgfpathlineto{\pgfqpoint{3.532597in}{2.401282in}}%
\pgfpathlineto{\pgfqpoint{3.485414in}{2.448584in}}%
\pgfpathlineto{\pgfqpoint{3.349730in}{2.581333in}}%
\pgfpathlineto{\pgfqpoint{3.231999in}{2.693333in}}%
\pgfpathlineto{\pgfqpoint{3.137958in}{2.780362in}}%
\pgfpathlineto{\pgfqpoint{3.084606in}{2.829446in}}%
\pgfpathlineto{\pgfqpoint{2.964364in}{2.937131in}}%
\pgfpathlineto{\pgfqpoint{2.892573in}{2.999798in}}%
\pgfpathlineto{\pgfqpoint{2.844121in}{3.041868in}}%
\pgfpathlineto{\pgfqpoint{2.723879in}{3.143707in}}%
\pgfpathlineto{\pgfqpoint{2.640140in}{3.212668in}}%
\pgfpathlineto{\pgfqpoint{2.590055in}{3.253333in}}%
\pgfpathlineto{\pgfqpoint{2.350941in}{3.440000in}}%
\pgfpathlineto{\pgfqpoint{2.242909in}{3.520316in}}%
\pgfpathlineto{\pgfqpoint{2.162747in}{3.578113in}}%
\pgfpathlineto{\pgfqpoint{2.039761in}{3.664000in}}%
\pgfpathlineto{\pgfqpoint{1.908449in}{3.751533in}}%
\pgfpathlineto{\pgfqpoint{1.802020in}{3.819162in}}%
\pgfpathlineto{\pgfqpoint{1.721859in}{3.867998in}}%
\pgfpathlineto{\pgfqpoint{1.623631in}{3.925333in}}%
\pgfpathlineto{\pgfqpoint{1.556963in}{3.962667in}}%
\pgfpathlineto{\pgfqpoint{1.481374in}{4.003133in}}%
\pgfpathlineto{\pgfqpoint{1.305423in}{4.089223in}}%
\pgfpathlineto{\pgfqpoint{1.240889in}{4.117623in}}%
\pgfpathlineto{\pgfqpoint{1.103742in}{4.170921in}}%
\pgfpathlineto{\pgfqpoint{1.059539in}{4.186667in}}%
\pgfpathlineto{\pgfqpoint{1.000404in}{4.205281in}}%
\pgfpathlineto{\pgfqpoint{0.960323in}{4.216649in}}%
\pgfpathlineto{\pgfqpoint{0.931529in}{4.224000in}}%
\pgfpathlineto{\pgfqpoint{0.881012in}{4.224000in}}%
\pgfpathlineto{\pgfqpoint{0.960323in}{4.205580in}}%
\pgfpathlineto{\pgfqpoint{1.006688in}{4.192520in}}%
\pgfpathlineto{\pgfqpoint{1.040485in}{4.182579in}}%
\pgfpathlineto{\pgfqpoint{1.176956in}{4.134217in}}%
\pgfpathlineto{\pgfqpoint{1.240889in}{4.108185in}}%
\pgfpathlineto{\pgfqpoint{1.401212in}{4.034932in}}%
\pgfpathlineto{\pgfqpoint{1.561535in}{3.951634in}}%
\pgfpathlineto{\pgfqpoint{1.629067in}{3.913569in}}%
\pgfpathlineto{\pgfqpoint{1.674198in}{3.888000in}}%
\pgfpathlineto{\pgfqpoint{1.808862in}{3.806961in}}%
\pgfpathlineto{\pgfqpoint{1.922263in}{3.734694in}}%
\pgfpathlineto{\pgfqpoint{2.002424in}{3.681510in}}%
\pgfpathlineto{\pgfqpoint{2.122667in}{3.598891in}}%
\pgfpathlineto{\pgfqpoint{2.173838in}{3.562330in}}%
\pgfpathlineto{\pgfqpoint{2.240431in}{3.514667in}}%
\pgfpathlineto{\pgfqpoint{2.419209in}{3.380214in}}%
\pgfpathlineto{\pgfqpoint{2.483394in}{3.330871in}}%
\pgfpathlineto{\pgfqpoint{2.603636in}{3.235086in}}%
\pgfpathlineto{\pgfqpoint{2.723879in}{3.136420in}}%
\pgfpathlineto{\pgfqpoint{2.850320in}{3.029333in}}%
\pgfpathlineto{\pgfqpoint{2.950509in}{2.941762in}}%
\pgfpathlineto{\pgfqpoint{3.004444in}{2.894321in}}%
\pgfpathlineto{\pgfqpoint{3.124687in}{2.785604in}}%
\pgfpathlineto{\pgfqpoint{3.214426in}{2.702254in}}%
\pgfpathlineto{\pgfqpoint{3.263841in}{2.656000in}}%
\pgfpathlineto{\pgfqpoint{3.353281in}{2.570258in}}%
\pgfpathlineto{\pgfqpoint{3.405253in}{2.520130in}}%
\pgfpathlineto{\pgfqpoint{3.531971in}{2.394667in}}%
\pgfpathlineto{\pgfqpoint{3.586298in}{2.339302in}}%
\pgfpathlineto{\pgfqpoint{3.645737in}{2.278935in}}%
\pgfpathlineto{\pgfqpoint{3.765980in}{2.153170in}}%
\pgfpathlineto{\pgfqpoint{3.888677in}{2.021333in}}%
\pgfpathlineto{\pgfqpoint{3.989959in}{1.909333in}}%
\pgfpathlineto{\pgfqpoint{4.088971in}{1.797333in}}%
\pgfpathlineto{\pgfqpoint{4.159260in}{1.715655in}}%
\pgfpathlineto{\pgfqpoint{4.206869in}{1.659918in}}%
\pgfpathlineto{\pgfqpoint{4.407273in}{1.415921in}}%
\pgfpathlineto{\pgfqpoint{4.517894in}{1.274667in}}%
\pgfpathlineto{\pgfqpoint{4.567596in}{1.209517in}}%
\pgfpathlineto{\pgfqpoint{4.647758in}{1.101900in}}%
\pgfpathlineto{\pgfqpoint{4.727919in}{0.990925in}}%
\pgfpathlineto{\pgfqpoint{4.768000in}{0.934132in}}%
\pgfpathlineto{\pgfqpoint{4.768000in}{0.938667in}}%
\pgfpathlineto{\pgfqpoint{4.768000in}{0.938667in}}%
\pgfusepath{fill}%
\end{pgfscope}%
\begin{pgfscope}%
\pgfpathrectangle{\pgfqpoint{0.800000in}{0.528000in}}{\pgfqpoint{3.968000in}{3.696000in}}%
\pgfusepath{clip}%
\pgfsetbuttcap%
\pgfsetroundjoin%
\definecolor{currentfill}{rgb}{0.282623,0.140926,0.457517}%
\pgfsetfillcolor{currentfill}%
\pgfsetlinewidth{0.000000pt}%
\definecolor{currentstroke}{rgb}{0.000000,0.000000,0.000000}%
\pgfsetstrokecolor{currentstroke}%
\pgfsetdash{}{0pt}%
\pgfpathmoveto{\pgfqpoint{2.599947in}{0.528000in}}%
\pgfpathlineto{\pgfqpoint{2.523475in}{0.599830in}}%
\pgfpathlineto{\pgfqpoint{2.403232in}{0.715282in}}%
\pgfpathlineto{\pgfqpoint{2.282990in}{0.833976in}}%
\pgfpathlineto{\pgfqpoint{2.190808in}{0.927470in}}%
\pgfpathlineto{\pgfqpoint{2.143235in}{0.976000in}}%
\pgfpathlineto{\pgfqpoint{2.071434in}{1.050667in}}%
\pgfpathlineto{\pgfqpoint{1.962343in}{1.166471in}}%
\pgfpathlineto{\pgfqpoint{1.829297in}{1.312000in}}%
\pgfpathlineto{\pgfqpoint{1.721859in}{1.432900in}}%
\pgfpathlineto{\pgfqpoint{1.636675in}{1.531323in}}%
\pgfpathlineto{\pgfqpoint{1.596902in}{1.577724in}}%
\pgfpathlineto{\pgfqpoint{1.476053in}{1.722667in}}%
\pgfpathlineto{\pgfqpoint{1.385505in}{1.834667in}}%
\pgfpathlineto{\pgfqpoint{1.297345in}{1.946667in}}%
\pgfpathlineto{\pgfqpoint{1.257203in}{1.999196in}}%
\pgfpathlineto{\pgfqpoint{1.211650in}{2.058667in}}%
\pgfpathlineto{\pgfqpoint{1.183684in}{2.096000in}}%
\pgfpathlineto{\pgfqpoint{1.101418in}{2.208000in}}%
\pgfpathlineto{\pgfqpoint{1.021776in}{2.320000in}}%
\pgfpathlineto{\pgfqpoint{0.982178in}{2.377690in}}%
\pgfpathlineto{\pgfqpoint{0.944844in}{2.432000in}}%
\pgfpathlineto{\pgfqpoint{0.904899in}{2.492375in}}%
\pgfpathlineto{\pgfqpoint{0.870708in}{2.544000in}}%
\pgfpathlineto{\pgfqpoint{0.814717in}{2.632374in}}%
\pgfpathlineto{\pgfqpoint{0.800000in}{2.655140in}}%
\pgfpathlineto{\pgfqpoint{0.800000in}{2.643018in}}%
\pgfpathlineto{\pgfqpoint{0.937420in}{2.432000in}}%
\pgfpathlineto{\pgfqpoint{0.977676in}{2.373496in}}%
\pgfpathlineto{\pgfqpoint{1.014431in}{2.320000in}}%
\pgfpathlineto{\pgfqpoint{1.067281in}{2.245333in}}%
\pgfpathlineto{\pgfqpoint{1.148744in}{2.133333in}}%
\pgfpathlineto{\pgfqpoint{1.186667in}{2.082829in}}%
\pgfpathlineto{\pgfqpoint{1.232751in}{2.021333in}}%
\pgfpathlineto{\pgfqpoint{1.286393in}{1.951718in}}%
\pgfpathlineto{\pgfqpoint{1.321051in}{1.907008in}}%
\pgfpathlineto{\pgfqpoint{1.441293in}{1.756522in}}%
\pgfpathlineto{\pgfqpoint{1.563612in}{1.608732in}}%
\pgfpathlineto{\pgfqpoint{1.689937in}{1.461333in}}%
\pgfpathlineto{\pgfqpoint{1.755476in}{1.386667in}}%
\pgfpathlineto{\pgfqpoint{1.855912in}{1.274667in}}%
\pgfpathlineto{\pgfqpoint{1.923267in}{1.200935in}}%
\pgfpathlineto{\pgfqpoint{1.962343in}{1.158746in}}%
\pgfpathlineto{\pgfqpoint{2.082586in}{1.031398in}}%
\pgfpathlineto{\pgfqpoint{2.208764in}{0.901333in}}%
\pgfpathlineto{\pgfqpoint{2.263849in}{0.846171in}}%
\pgfpathlineto{\pgfqpoint{2.323071in}{0.786646in}}%
\pgfpathlineto{\pgfqpoint{2.443313in}{0.669207in}}%
\pgfpathlineto{\pgfqpoint{2.563556in}{0.554824in}}%
\pgfpathlineto{\pgfqpoint{2.592244in}{0.528000in}}%
\pgfpathmoveto{\pgfqpoint{4.768000in}{0.954969in}}%
\pgfpathlineto{\pgfqpoint{4.589104in}{1.200000in}}%
\pgfpathlineto{\pgfqpoint{4.503290in}{1.312000in}}%
\pgfpathlineto{\pgfqpoint{4.407273in}{1.433706in}}%
\pgfpathlineto{\pgfqpoint{4.354783in}{1.498667in}}%
\pgfpathlineto{\pgfqpoint{4.262406in}{1.610667in}}%
\pgfpathlineto{\pgfqpoint{4.184820in}{1.702130in}}%
\pgfpathlineto{\pgfqpoint{4.135521in}{1.760000in}}%
\pgfpathlineto{\pgfqpoint{4.037488in}{1.872000in}}%
\pgfpathlineto{\pgfqpoint{3.926303in}{1.995759in}}%
\pgfpathlineto{\pgfqpoint{3.799151in}{2.133333in}}%
\pgfpathlineto{\pgfqpoint{3.685818in}{2.252560in}}%
\pgfpathlineto{\pgfqpoint{3.565576in}{2.375686in}}%
\pgfpathlineto{\pgfqpoint{3.485414in}{2.455989in}}%
\pgfpathlineto{\pgfqpoint{3.357369in}{2.581333in}}%
\pgfpathlineto{\pgfqpoint{3.239803in}{2.693333in}}%
\pgfpathlineto{\pgfqpoint{3.141842in}{2.783979in}}%
\pgfpathlineto{\pgfqpoint{3.084606in}{2.836717in}}%
\pgfpathlineto{\pgfqpoint{2.952653in}{2.954667in}}%
\pgfpathlineto{\pgfqpoint{2.683798in}{3.184157in}}%
\pgfpathlineto{\pgfqpoint{2.563556in}{3.281929in}}%
\pgfpathlineto{\pgfqpoint{2.443313in}{3.376684in}}%
\pgfpathlineto{\pgfqpoint{2.383880in}{3.421975in}}%
\pgfpathlineto{\pgfqpoint{2.323071in}{3.468377in}}%
\pgfpathlineto{\pgfqpoint{2.260643in}{3.514667in}}%
\pgfpathlineto{\pgfqpoint{2.157603in}{3.589333in}}%
\pgfpathlineto{\pgfqpoint{1.922263in}{3.750233in}}%
\pgfpathlineto{\pgfqpoint{1.823891in}{3.813333in}}%
\pgfpathlineto{\pgfqpoint{1.640375in}{3.924102in}}%
\pgfpathlineto{\pgfqpoint{1.590325in}{3.952149in}}%
\pgfpathlineto{\pgfqpoint{1.546033in}{3.977107in}}%
\pgfpathlineto{\pgfqpoint{1.441293in}{4.032462in}}%
\pgfpathlineto{\pgfqpoint{1.401212in}{4.052549in}}%
\pgfpathlineto{\pgfqpoint{1.321051in}{4.091046in}}%
\pgfpathlineto{\pgfqpoint{1.240889in}{4.126927in}}%
\pgfpathlineto{\pgfqpoint{1.168511in}{4.156583in}}%
\pgfpathlineto{\pgfqpoint{1.141065in}{4.167648in}}%
\pgfpathlineto{\pgfqpoint{1.075100in}{4.191758in}}%
\pgfpathlineto{\pgfqpoint{1.000404in}{4.215934in}}%
\pgfpathlineto{\pgfqpoint{0.972641in}{4.224000in}}%
\pgfpathlineto{\pgfqpoint{0.931529in}{4.224000in}}%
\pgfpathlineto{\pgfqpoint{1.000404in}{4.205281in}}%
\pgfpathlineto{\pgfqpoint{1.040485in}{4.193005in}}%
\pgfpathlineto{\pgfqpoint{1.120646in}{4.165326in}}%
\pgfpathlineto{\pgfqpoint{1.200808in}{4.134261in}}%
\pgfpathlineto{\pgfqpoint{1.272089in}{4.103728in}}%
\pgfpathlineto{\pgfqpoint{1.305423in}{4.089223in}}%
\pgfpathlineto{\pgfqpoint{1.386515in}{4.051023in}}%
\pgfpathlineto{\pgfqpoint{1.481374in}{4.003133in}}%
\pgfpathlineto{\pgfqpoint{1.521455in}{3.981827in}}%
\pgfpathlineto{\pgfqpoint{1.601616in}{3.937827in}}%
\pgfpathlineto{\pgfqpoint{1.688251in}{3.888000in}}%
\pgfpathlineto{\pgfqpoint{1.802020in}{3.819162in}}%
\pgfpathlineto{\pgfqpoint{1.842101in}{3.793973in}}%
\pgfpathlineto{\pgfqpoint{1.928107in}{3.738667in}}%
\pgfpathlineto{\pgfqpoint{2.050042in}{3.656979in}}%
\pgfpathlineto{\pgfqpoint{2.162747in}{3.578113in}}%
\pgfpathlineto{\pgfqpoint{2.223540in}{3.533959in}}%
\pgfpathlineto{\pgfqpoint{2.282990in}{3.490793in}}%
\pgfpathlineto{\pgfqpoint{2.357612in}{3.434840in}}%
\pgfpathlineto{\pgfqpoint{2.403232in}{3.400331in}}%
\pgfpathlineto{\pgfqpoint{2.483394in}{3.338143in}}%
\pgfpathlineto{\pgfqpoint{2.603636in}{3.242398in}}%
\pgfpathlineto{\pgfqpoint{2.726712in}{3.141333in}}%
\pgfpathlineto{\pgfqpoint{2.858663in}{3.029333in}}%
\pgfpathlineto{\pgfqpoint{3.124687in}{2.792888in}}%
\pgfpathlineto{\pgfqpoint{3.192052in}{2.730667in}}%
\pgfpathlineto{\pgfqpoint{3.285010in}{2.643291in}}%
\pgfpathlineto{\pgfqpoint{3.405253in}{2.527508in}}%
\pgfpathlineto{\pgfqpoint{3.539350in}{2.394667in}}%
\pgfpathlineto{\pgfqpoint{3.613044in}{2.320000in}}%
\pgfpathlineto{\pgfqpoint{3.725899in}{2.203172in}}%
\pgfpathlineto{\pgfqpoint{3.817015in}{2.106204in}}%
\pgfpathlineto{\pgfqpoint{3.861420in}{2.058667in}}%
\pgfpathlineto{\pgfqpoint{3.966384in}{1.943703in}}%
\pgfpathlineto{\pgfqpoint{4.046545in}{1.853708in}}%
\pgfpathlineto{\pgfqpoint{4.166788in}{1.715446in}}%
\pgfpathlineto{\pgfqpoint{4.251624in}{1.615021in}}%
\pgfpathlineto{\pgfqpoint{4.287030in}{1.572719in}}%
\pgfpathlineto{\pgfqpoint{4.347680in}{1.498667in}}%
\pgfpathlineto{\pgfqpoint{4.447354in}{1.374365in}}%
\pgfpathlineto{\pgfqpoint{4.553636in}{1.237333in}}%
\pgfpathlineto{\pgfqpoint{4.592351in}{1.185725in}}%
\pgfpathlineto{\pgfqpoint{4.637750in}{1.125333in}}%
\pgfpathlineto{\pgfqpoint{4.722652in}{1.008427in}}%
\pgfpathlineto{\pgfqpoint{4.768000in}{0.944639in}}%
\pgfpathlineto{\pgfqpoint{4.768000in}{0.944639in}}%
\pgfusepath{fill}%
\end{pgfscope}%
\begin{pgfscope}%
\pgfpathrectangle{\pgfqpoint{0.800000in}{0.528000in}}{\pgfqpoint{3.968000in}{3.696000in}}%
\pgfusepath{clip}%
\pgfsetbuttcap%
\pgfsetroundjoin%
\definecolor{currentfill}{rgb}{0.282623,0.140926,0.457517}%
\pgfsetfillcolor{currentfill}%
\pgfsetlinewidth{0.000000pt}%
\definecolor{currentstroke}{rgb}{0.000000,0.000000,0.000000}%
\pgfsetstrokecolor{currentstroke}%
\pgfsetdash{}{0pt}%
\pgfpathmoveto{\pgfqpoint{2.592244in}{0.528000in}}%
\pgfpathlineto{\pgfqpoint{2.498493in}{0.616731in}}%
\pgfpathlineto{\pgfqpoint{2.443313in}{0.669207in}}%
\pgfpathlineto{\pgfqpoint{2.358228in}{0.752000in}}%
\pgfpathlineto{\pgfqpoint{2.242909in}{0.866737in}}%
\pgfpathlineto{\pgfqpoint{2.162747in}{0.948341in}}%
\pgfpathlineto{\pgfqpoint{2.028775in}{1.088000in}}%
\pgfpathlineto{\pgfqpoint{1.960414in}{1.160870in}}%
\pgfpathlineto{\pgfqpoint{1.922263in}{1.201997in}}%
\pgfpathlineto{\pgfqpoint{1.788714in}{1.349333in}}%
\pgfpathlineto{\pgfqpoint{1.755476in}{1.386667in}}%
\pgfpathlineto{\pgfqpoint{1.657611in}{1.498667in}}%
\pgfpathlineto{\pgfqpoint{1.561535in}{1.611198in}}%
\pgfpathlineto{\pgfqpoint{1.438462in}{1.760000in}}%
\pgfpathlineto{\pgfqpoint{1.348724in}{1.872000in}}%
\pgfpathlineto{\pgfqpoint{1.261376in}{1.984000in}}%
\pgfpathlineto{\pgfqpoint{1.212906in}{2.047398in}}%
\pgfpathlineto{\pgfqpoint{1.120646in}{2.171398in}}%
\pgfpathlineto{\pgfqpoint{1.056797in}{2.260527in}}%
\pgfpathlineto{\pgfqpoint{1.014431in}{2.320000in}}%
\pgfpathlineto{\pgfqpoint{0.977676in}{2.373496in}}%
\pgfpathlineto{\pgfqpoint{0.937420in}{2.432000in}}%
\pgfpathlineto{\pgfqpoint{0.900284in}{2.488076in}}%
\pgfpathlineto{\pgfqpoint{0.863204in}{2.544000in}}%
\pgfpathlineto{\pgfqpoint{0.824755in}{2.604391in}}%
\pgfpathlineto{\pgfqpoint{0.800000in}{2.643018in}}%
\pgfpathlineto{\pgfqpoint{0.800000in}{2.630895in}}%
\pgfpathlineto{\pgfqpoint{0.920242in}{2.446471in}}%
\pgfpathlineto{\pgfqpoint{1.060062in}{2.245333in}}%
\pgfpathlineto{\pgfqpoint{1.141600in}{2.133333in}}%
\pgfpathlineto{\pgfqpoint{1.182485in}{2.078933in}}%
\pgfpathlineto{\pgfqpoint{1.225679in}{2.021333in}}%
\pgfpathlineto{\pgfqpoint{1.265623in}{1.969705in}}%
\pgfpathlineto{\pgfqpoint{1.312224in}{1.909333in}}%
\pgfpathlineto{\pgfqpoint{1.401212in}{1.797268in}}%
\pgfpathlineto{\pgfqpoint{1.531476in}{1.638665in}}%
\pgfpathlineto{\pgfqpoint{1.650526in}{1.498667in}}%
\pgfpathlineto{\pgfqpoint{1.718407in}{1.420785in}}%
\pgfpathlineto{\pgfqpoint{1.761939in}{1.371451in}}%
\pgfpathlineto{\pgfqpoint{1.845758in}{1.278073in}}%
\pgfpathlineto{\pgfqpoint{1.885602in}{1.234148in}}%
\pgfpathlineto{\pgfqpoint{2.002424in}{1.108325in}}%
\pgfpathlineto{\pgfqpoint{2.082586in}{1.023826in}}%
\pgfpathlineto{\pgfqpoint{2.202828in}{0.899871in}}%
\pgfpathlineto{\pgfqpoint{2.275514in}{0.826667in}}%
\pgfpathlineto{\pgfqpoint{2.388968in}{0.714667in}}%
\pgfpathlineto{\pgfqpoint{2.584541in}{0.528000in}}%
\pgfpathmoveto{\pgfqpoint{4.768000in}{0.965299in}}%
\pgfpathlineto{\pgfqpoint{4.607677in}{1.184861in}}%
\pgfpathlineto{\pgfqpoint{4.415630in}{1.431784in}}%
\pgfpathlineto{\pgfqpoint{4.361887in}{1.498667in}}%
\pgfpathlineto{\pgfqpoint{4.269437in}{1.610667in}}%
\pgfpathlineto{\pgfqpoint{4.224163in}{1.664109in}}%
\pgfpathlineto{\pgfqpoint{4.174705in}{1.722667in}}%
\pgfpathlineto{\pgfqpoint{4.077494in}{1.834667in}}%
\pgfpathlineto{\pgfqpoint{3.966384in}{1.959383in}}%
\pgfpathlineto{\pgfqpoint{3.886222in}{2.047312in}}%
\pgfpathlineto{\pgfqpoint{3.765980in}{2.176173in}}%
\pgfpathlineto{\pgfqpoint{3.685818in}{2.260034in}}%
\pgfpathlineto{\pgfqpoint{3.554110in}{2.394667in}}%
\pgfpathlineto{\pgfqpoint{3.462656in}{2.485469in}}%
\pgfpathlineto{\pgfqpoint{3.403468in}{2.544000in}}%
\pgfpathlineto{\pgfqpoint{3.285010in}{2.657913in}}%
\pgfpathlineto{\pgfqpoint{3.207680in}{2.730667in}}%
\pgfpathlineto{\pgfqpoint{3.124687in}{2.807399in}}%
\pgfpathlineto{\pgfqpoint{3.084606in}{2.843952in}}%
\pgfpathlineto{\pgfqpoint{2.960870in}{2.954667in}}%
\pgfpathlineto{\pgfqpoint{2.875351in}{3.029333in}}%
\pgfpathlineto{\pgfqpoint{2.743793in}{3.141333in}}%
\pgfpathlineto{\pgfqpoint{2.643717in}{3.224267in}}%
\pgfpathlineto{\pgfqpoint{2.514747in}{3.328000in}}%
\pgfpathlineto{\pgfqpoint{2.403232in}{3.414886in}}%
\pgfpathlineto{\pgfqpoint{2.363152in}{3.445519in}}%
\pgfpathlineto{\pgfqpoint{2.242909in}{3.535104in}}%
\pgfpathlineto{\pgfqpoint{2.187441in}{3.575001in}}%
\pgfpathlineto{\pgfqpoint{2.122667in}{3.621582in}}%
\pgfpathlineto{\pgfqpoint{2.002424in}{3.704637in}}%
\pgfpathlineto{\pgfqpoint{1.761939in}{3.859795in}}%
\pgfpathlineto{\pgfqpoint{1.561535in}{3.976856in}}%
\pgfpathlineto{\pgfqpoint{1.481374in}{4.020180in}}%
\pgfpathlineto{\pgfqpoint{1.417627in}{4.052623in}}%
\pgfpathlineto{\pgfqpoint{1.374125in}{4.074667in}}%
\pgfpathlineto{\pgfqpoint{1.267335in}{4.124700in}}%
\pgfpathlineto{\pgfqpoint{1.200808in}{4.153354in}}%
\pgfpathlineto{\pgfqpoint{1.057953in}{4.207729in}}%
\pgfpathlineto{\pgfqpoint{1.008120in}{4.224000in}}%
\pgfpathlineto{\pgfqpoint{0.972641in}{4.224000in}}%
\pgfpathlineto{\pgfqpoint{1.040485in}{4.203272in}}%
\pgfpathlineto{\pgfqpoint{1.089273in}{4.186667in}}%
\pgfpathlineto{\pgfqpoint{1.200808in}{4.143880in}}%
\pgfpathlineto{\pgfqpoint{1.366524in}{4.069644in}}%
\pgfpathlineto{\pgfqpoint{1.441293in}{4.032462in}}%
\pgfpathlineto{\pgfqpoint{1.641697in}{3.923386in}}%
\pgfpathlineto{\pgfqpoint{1.721859in}{3.876104in}}%
\pgfpathlineto{\pgfqpoint{1.763956in}{3.850667in}}%
\pgfpathlineto{\pgfqpoint{1.882766in}{3.776000in}}%
\pgfpathlineto{\pgfqpoint{2.002424in}{3.696992in}}%
\pgfpathlineto{\pgfqpoint{2.104588in}{3.626667in}}%
\pgfpathlineto{\pgfqpoint{2.209574in}{3.552000in}}%
\pgfpathlineto{\pgfqpoint{2.457879in}{3.365333in}}%
\pgfpathlineto{\pgfqpoint{2.723879in}{3.150860in}}%
\pgfpathlineto{\pgfqpoint{2.844121in}{3.049060in}}%
\pgfpathlineto{\pgfqpoint{2.964364in}{2.944362in}}%
\pgfpathlineto{\pgfqpoint{3.084606in}{2.836717in}}%
\pgfpathlineto{\pgfqpoint{3.159660in}{2.768000in}}%
\pgfpathlineto{\pgfqpoint{3.244929in}{2.688523in}}%
\pgfpathlineto{\pgfqpoint{3.321722in}{2.615529in}}%
\pgfpathlineto{\pgfqpoint{3.365172in}{2.573812in}}%
\pgfpathlineto{\pgfqpoint{3.485414in}{2.455989in}}%
\pgfpathlineto{\pgfqpoint{3.574892in}{2.366011in}}%
\pgfpathlineto{\pgfqpoint{3.620325in}{2.320000in}}%
\pgfpathlineto{\pgfqpoint{3.728565in}{2.208000in}}%
\pgfpathlineto{\pgfqpoint{3.799151in}{2.133333in}}%
\pgfpathlineto{\pgfqpoint{3.902912in}{2.021333in}}%
\pgfpathlineto{\pgfqpoint{4.006465in}{1.906956in}}%
\pgfpathlineto{\pgfqpoint{4.135521in}{1.760000in}}%
\pgfpathlineto{\pgfqpoint{4.231078in}{1.648000in}}%
\pgfpathlineto{\pgfqpoint{4.327111in}{1.532638in}}%
\pgfpathlineto{\pgfqpoint{4.447354in}{1.383444in}}%
\pgfpathlineto{\pgfqpoint{4.560840in}{1.237333in}}%
\pgfpathlineto{\pgfqpoint{4.613206in}{1.167817in}}%
\pgfpathlineto{\pgfqpoint{4.647758in}{1.121637in}}%
\pgfpathlineto{\pgfqpoint{4.753121in}{0.976000in}}%
\pgfpathlineto{\pgfqpoint{4.768000in}{0.954969in}}%
\pgfpathlineto{\pgfqpoint{4.768000in}{0.954969in}}%
\pgfusepath{fill}%
\end{pgfscope}%
\begin{pgfscope}%
\pgfpathrectangle{\pgfqpoint{0.800000in}{0.528000in}}{\pgfqpoint{3.968000in}{3.696000in}}%
\pgfusepath{clip}%
\pgfsetbuttcap%
\pgfsetroundjoin%
\definecolor{currentfill}{rgb}{0.282290,0.145912,0.461510}%
\pgfsetfillcolor{currentfill}%
\pgfsetlinewidth{0.000000pt}%
\definecolor{currentstroke}{rgb}{0.000000,0.000000,0.000000}%
\pgfsetstrokecolor{currentstroke}%
\pgfsetdash{}{0pt}%
\pgfpathmoveto{\pgfqpoint{2.584541in}{0.528000in}}%
\pgfpathlineto{\pgfqpoint{2.483394in}{0.623514in}}%
\pgfpathlineto{\pgfqpoint{2.350839in}{0.752000in}}%
\pgfpathlineto{\pgfqpoint{2.238305in}{0.864000in}}%
\pgfpathlineto{\pgfqpoint{2.163851in}{0.939695in}}%
\pgfpathlineto{\pgfqpoint{2.122667in}{0.982129in}}%
\pgfpathlineto{\pgfqpoint{1.986480in}{1.125333in}}%
\pgfpathlineto{\pgfqpoint{1.882182in}{1.237881in}}%
\pgfpathlineto{\pgfqpoint{1.791043in}{1.339108in}}%
\pgfpathlineto{\pgfqpoint{1.748463in}{1.386667in}}%
\pgfpathlineto{\pgfqpoint{1.641697in}{1.508902in}}%
\pgfpathlineto{\pgfqpoint{1.555020in}{1.610667in}}%
\pgfpathlineto{\pgfqpoint{1.462014in}{1.722667in}}%
\pgfpathlineto{\pgfqpoint{1.361131in}{1.847446in}}%
\pgfpathlineto{\pgfqpoint{1.160727in}{2.107501in}}%
\pgfpathlineto{\pgfqpoint{0.981146in}{2.357333in}}%
\pgfpathlineto{\pgfqpoint{0.941988in}{2.414921in}}%
\pgfpathlineto{\pgfqpoint{0.904936in}{2.469333in}}%
\pgfpathlineto{\pgfqpoint{0.879773in}{2.507029in}}%
\pgfpathlineto{\pgfqpoint{0.800000in}{2.630895in}}%
\pgfpathlineto{\pgfqpoint{0.800067in}{2.618667in}}%
\pgfpathlineto{\pgfqpoint{0.960323in}{2.376887in}}%
\pgfpathlineto{\pgfqpoint{1.120646in}{2.152027in}}%
\pgfpathlineto{\pgfqpoint{1.200808in}{2.044711in}}%
\pgfpathlineto{\pgfqpoint{1.305224in}{1.909333in}}%
\pgfpathlineto{\pgfqpoint{1.401212in}{1.788682in}}%
\pgfpathlineto{\pgfqpoint{1.521455in}{1.642319in}}%
\pgfpathlineto{\pgfqpoint{1.643441in}{1.498667in}}%
\pgfpathlineto{\pgfqpoint{1.721859in}{1.408830in}}%
\pgfpathlineto{\pgfqpoint{1.842101in}{1.274139in}}%
\pgfpathlineto{\pgfqpoint{1.922263in}{1.186680in}}%
\pgfpathlineto{\pgfqpoint{2.049774in}{1.050667in}}%
\pgfpathlineto{\pgfqpoint{2.121969in}{0.975350in}}%
\pgfpathlineto{\pgfqpoint{2.162747in}{0.933404in}}%
\pgfpathlineto{\pgfqpoint{2.256197in}{0.839044in}}%
\pgfpathlineto{\pgfqpoint{2.305685in}{0.789333in}}%
\pgfpathlineto{\pgfqpoint{2.576838in}{0.528000in}}%
\pgfpathmoveto{\pgfqpoint{4.768000in}{0.975629in}}%
\pgfpathlineto{\pgfqpoint{4.631413in}{1.162667in}}%
\pgfpathlineto{\pgfqpoint{4.593285in}{1.213405in}}%
\pgfpathlineto{\pgfqpoint{4.487434in}{1.350527in}}%
\pgfpathlineto{\pgfqpoint{4.367192in}{1.500809in}}%
\pgfpathlineto{\pgfqpoint{4.245232in}{1.648000in}}%
\pgfpathlineto{\pgfqpoint{4.181666in}{1.722667in}}%
\pgfpathlineto{\pgfqpoint{4.075346in}{1.845174in}}%
\pgfpathlineto{\pgfqpoint{3.951130in}{1.984000in}}%
\pgfpathlineto{\pgfqpoint{3.865654in}{2.076842in}}%
\pgfpathlineto{\pgfqpoint{3.806061in}{2.141212in}}%
\pgfpathlineto{\pgfqpoint{3.671151in}{2.282667in}}%
\pgfpathlineto{\pgfqpoint{3.582439in}{2.373040in}}%
\pgfpathlineto{\pgfqpoint{3.524346in}{2.432000in}}%
\pgfpathlineto{\pgfqpoint{3.447203in}{2.508409in}}%
\pgfpathlineto{\pgfqpoint{3.405253in}{2.549489in}}%
\pgfpathlineto{\pgfqpoint{3.285010in}{2.665052in}}%
\pgfpathlineto{\pgfqpoint{3.215289in}{2.730667in}}%
\pgfpathlineto{\pgfqpoint{3.124687in}{2.814487in}}%
\pgfpathlineto{\pgfqpoint{3.048426in}{2.883634in}}%
\pgfpathlineto{\pgfqpoint{3.004444in}{2.923139in}}%
\pgfpathlineto{\pgfqpoint{2.883694in}{3.029333in}}%
\pgfpathlineto{\pgfqpoint{2.752334in}{3.141333in}}%
\pgfpathlineto{\pgfqpoint{2.643717in}{3.231395in}}%
\pgfpathlineto{\pgfqpoint{2.520701in}{3.330584in}}%
\pgfpathlineto{\pgfqpoint{2.403232in}{3.422132in}}%
\pgfpathlineto{\pgfqpoint{2.363152in}{3.452751in}}%
\pgfpathlineto{\pgfqpoint{2.242909in}{3.542499in}}%
\pgfpathlineto{\pgfqpoint{2.178311in}{3.589333in}}%
\pgfpathlineto{\pgfqpoint{2.072647in}{3.664000in}}%
\pgfpathlineto{\pgfqpoint{1.848706in}{3.813333in}}%
\pgfpathlineto{\pgfqpoint{1.789416in}{3.850667in}}%
\pgfpathlineto{\pgfqpoint{1.681778in}{3.916112in}}%
\pgfpathlineto{\pgfqpoint{1.601380in}{3.962887in}}%
\pgfpathlineto{\pgfqpoint{1.481374in}{4.028704in}}%
\pgfpathlineto{\pgfqpoint{1.423628in}{4.058213in}}%
\pgfpathlineto{\pgfqpoint{1.391919in}{4.074667in}}%
\pgfpathlineto{\pgfqpoint{1.240889in}{4.145535in}}%
\pgfpathlineto{\pgfqpoint{1.182587in}{4.169695in}}%
\pgfpathlineto{\pgfqpoint{1.141510in}{4.186667in}}%
\pgfpathlineto{\pgfqpoint{1.039888in}{4.224000in}}%
\pgfpathlineto{\pgfqpoint{1.008120in}{4.224000in}}%
\pgfpathlineto{\pgfqpoint{1.123286in}{4.184208in}}%
\pgfpathlineto{\pgfqpoint{1.210236in}{4.149333in}}%
\pgfpathlineto{\pgfqpoint{1.321051in}{4.100074in}}%
\pgfpathlineto{\pgfqpoint{1.374125in}{4.074667in}}%
\pgfpathlineto{\pgfqpoint{1.448561in}{4.037333in}}%
\pgfpathlineto{\pgfqpoint{1.624206in}{3.941626in}}%
\pgfpathlineto{\pgfqpoint{1.721859in}{3.884209in}}%
\pgfpathlineto{\pgfqpoint{1.951492in}{3.738667in}}%
\pgfpathlineto{\pgfqpoint{2.042505in}{3.677264in}}%
\pgfpathlineto{\pgfqpoint{2.096434in}{3.639566in}}%
\pgfpathlineto{\pgfqpoint{2.162747in}{3.593160in}}%
\pgfpathlineto{\pgfqpoint{2.282990in}{3.505609in}}%
\pgfpathlineto{\pgfqpoint{2.331615in}{3.469375in}}%
\pgfpathlineto{\pgfqpoint{2.443313in}{3.383943in}}%
\pgfpathlineto{\pgfqpoint{2.563556in}{3.289227in}}%
\pgfpathlineto{\pgfqpoint{2.648371in}{3.220335in}}%
\pgfpathlineto{\pgfqpoint{2.699025in}{3.178667in}}%
\pgfpathlineto{\pgfqpoint{2.964364in}{2.951593in}}%
\pgfpathlineto{\pgfqpoint{3.086018in}{2.842667in}}%
\pgfpathlineto{\pgfqpoint{3.164768in}{2.770520in}}%
\pgfpathlineto{\pgfqpoint{3.287020in}{2.656000in}}%
\pgfpathlineto{\pgfqpoint{3.405253in}{2.542264in}}%
\pgfpathlineto{\pgfqpoint{3.525495in}{2.423433in}}%
\pgfpathlineto{\pgfqpoint{3.617002in}{2.330568in}}%
\pgfpathlineto{\pgfqpoint{3.663918in}{2.282667in}}%
\pgfpathlineto{\pgfqpoint{3.771184in}{2.170667in}}%
\pgfpathlineto{\pgfqpoint{3.886222in}{2.047312in}}%
\pgfpathlineto{\pgfqpoint{3.966384in}{1.959383in}}%
\pgfpathlineto{\pgfqpoint{4.077494in}{1.834667in}}%
\pgfpathlineto{\pgfqpoint{4.174705in}{1.722667in}}%
\pgfpathlineto{\pgfqpoint{4.238155in}{1.648000in}}%
\pgfpathlineto{\pgfqpoint{4.345203in}{1.519148in}}%
\pgfpathlineto{\pgfqpoint{4.451847in}{1.386667in}}%
\pgfpathlineto{\pgfqpoint{4.539292in}{1.274667in}}%
\pgfpathlineto{\pgfqpoint{4.600877in}{1.193666in}}%
\pgfpathlineto{\pgfqpoint{4.647758in}{1.131284in}}%
\pgfpathlineto{\pgfqpoint{4.768000in}{0.965299in}}%
\pgfpathlineto{\pgfqpoint{4.768000in}{0.965299in}}%
\pgfusepath{fill}%
\end{pgfscope}%
\begin{pgfscope}%
\pgfpathrectangle{\pgfqpoint{0.800000in}{0.528000in}}{\pgfqpoint{3.968000in}{3.696000in}}%
\pgfusepath{clip}%
\pgfsetbuttcap%
\pgfsetroundjoin%
\definecolor{currentfill}{rgb}{0.282290,0.145912,0.461510}%
\pgfsetfillcolor{currentfill}%
\pgfsetlinewidth{0.000000pt}%
\definecolor{currentstroke}{rgb}{0.000000,0.000000,0.000000}%
\pgfsetstrokecolor{currentstroke}%
\pgfsetdash{}{0pt}%
\pgfpathmoveto{\pgfqpoint{2.576838in}{0.528000in}}%
\pgfpathlineto{\pgfqpoint{2.305685in}{0.789333in}}%
\pgfpathlineto{\pgfqpoint{2.217710in}{0.877861in}}%
\pgfpathlineto{\pgfqpoint{2.157620in}{0.938667in}}%
\pgfpathlineto{\pgfqpoint{2.042505in}{1.058304in}}%
\pgfpathlineto{\pgfqpoint{1.952920in}{1.153889in}}%
\pgfpathlineto{\pgfqpoint{1.909985in}{1.200000in}}%
\pgfpathlineto{\pgfqpoint{1.802020in}{1.318630in}}%
\pgfpathlineto{\pgfqpoint{1.714622in}{1.417259in}}%
\pgfpathlineto{\pgfqpoint{1.675851in}{1.461333in}}%
\pgfpathlineto{\pgfqpoint{1.611433in}{1.536000in}}%
\pgfpathlineto{\pgfqpoint{1.516708in}{1.648000in}}%
\pgfpathlineto{\pgfqpoint{1.454994in}{1.722667in}}%
\pgfpathlineto{\pgfqpoint{1.361131in}{1.838552in}}%
\pgfpathlineto{\pgfqpoint{1.280970in}{1.940284in}}%
\pgfpathlineto{\pgfqpoint{1.200808in}{2.044711in}}%
\pgfpathlineto{\pgfqpoint{1.026226in}{2.282667in}}%
\pgfpathlineto{\pgfqpoint{0.968671in}{2.365109in}}%
\pgfpathlineto{\pgfqpoint{0.922573in}{2.432000in}}%
\pgfpathlineto{\pgfqpoint{0.891054in}{2.479479in}}%
\pgfpathlineto{\pgfqpoint{0.848196in}{2.544000in}}%
\pgfpathlineto{\pgfqpoint{0.815287in}{2.595573in}}%
\pgfpathlineto{\pgfqpoint{0.800000in}{2.618772in}}%
\pgfpathlineto{\pgfqpoint{0.800000in}{2.618667in}}%
\pgfpathlineto{\pgfqpoint{0.800000in}{2.607176in}}%
\pgfpathlineto{\pgfqpoint{0.940858in}{2.394667in}}%
\pgfpathlineto{\pgfqpoint{1.019054in}{2.282667in}}%
\pgfpathlineto{\pgfqpoint{1.099870in}{2.170667in}}%
\pgfpathlineto{\pgfqpoint{1.141221in}{2.115165in}}%
\pgfpathlineto{\pgfqpoint{1.183225in}{2.058667in}}%
\pgfpathlineto{\pgfqpoint{1.223962in}{2.005567in}}%
\pgfpathlineto{\pgfqpoint{1.269045in}{1.946667in}}%
\pgfpathlineto{\pgfqpoint{1.361131in}{1.829822in}}%
\pgfpathlineto{\pgfqpoint{1.441293in}{1.730821in}}%
\pgfpathlineto{\pgfqpoint{1.541121in}{1.610667in}}%
\pgfpathlineto{\pgfqpoint{1.601616in}{1.539247in}}%
\pgfpathlineto{\pgfqpoint{1.701567in}{1.424000in}}%
\pgfpathlineto{\pgfqpoint{1.783433in}{1.332020in}}%
\pgfpathlineto{\pgfqpoint{1.842101in}{1.266480in}}%
\pgfpathlineto{\pgfqpoint{1.922263in}{1.179050in}}%
\pgfpathlineto{\pgfqpoint{2.042554in}{1.050667in}}%
\pgfpathlineto{\pgfqpoint{2.122667in}{0.967263in}}%
\pgfpathlineto{\pgfqpoint{2.242909in}{0.844738in}}%
\pgfpathlineto{\pgfqpoint{2.323071in}{0.764803in}}%
\pgfpathlineto{\pgfqpoint{2.451106in}{0.640000in}}%
\pgfpathlineto{\pgfqpoint{2.569135in}{0.528000in}}%
\pgfpathmoveto{\pgfqpoint{4.768000in}{0.985585in}}%
\pgfpathlineto{\pgfqpoint{4.687838in}{1.096261in}}%
\pgfpathlineto{\pgfqpoint{4.582016in}{1.237333in}}%
\pgfpathlineto{\pgfqpoint{4.553368in}{1.274667in}}%
\pgfpathlineto{\pgfqpoint{4.465781in}{1.386667in}}%
\pgfpathlineto{\pgfqpoint{4.423559in}{1.439170in}}%
\pgfpathlineto{\pgfqpoint{4.375836in}{1.498667in}}%
\pgfpathlineto{\pgfqpoint{4.314482in}{1.573333in}}%
\pgfpathlineto{\pgfqpoint{4.220498in}{1.685333in}}%
\pgfpathlineto{\pgfqpoint{4.113467in}{1.809665in}}%
\pgfpathlineto{\pgfqpoint{3.991892in}{1.946667in}}%
\pgfpathlineto{\pgfqpoint{3.886222in}{2.062725in}}%
\pgfpathlineto{\pgfqpoint{3.806061in}{2.148729in}}%
\pgfpathlineto{\pgfqpoint{3.678383in}{2.282667in}}%
\pgfpathlineto{\pgfqpoint{3.624529in}{2.337578in}}%
\pgfpathlineto{\pgfqpoint{3.565576in}{2.397894in}}%
\pgfpathlineto{\pgfqpoint{3.470262in}{2.492553in}}%
\pgfpathlineto{\pgfqpoint{3.418226in}{2.544000in}}%
\pgfpathlineto{\pgfqpoint{3.302028in}{2.656000in}}%
\pgfpathlineto{\pgfqpoint{3.182812in}{2.768000in}}%
\pgfpathlineto{\pgfqpoint{3.060382in}{2.880000in}}%
\pgfpathlineto{\pgfqpoint{2.804040in}{3.104828in}}%
\pgfpathlineto{\pgfqpoint{2.716242in}{3.178667in}}%
\pgfpathlineto{\pgfqpoint{2.603636in}{3.271153in}}%
\pgfpathlineto{\pgfqpoint{2.532853in}{3.328000in}}%
\pgfpathlineto{\pgfqpoint{2.437869in}{3.402667in}}%
\pgfpathlineto{\pgfqpoint{2.323071in}{3.490280in}}%
\pgfpathlineto{\pgfqpoint{2.282990in}{3.520268in}}%
\pgfpathlineto{\pgfqpoint{2.188555in}{3.589333in}}%
\pgfpathlineto{\pgfqpoint{2.082586in}{3.664660in}}%
\pgfpathlineto{\pgfqpoint{1.842101in}{3.825261in}}%
\pgfpathlineto{\pgfqpoint{1.801772in}{3.850898in}}%
\pgfpathlineto{\pgfqpoint{1.679847in}{3.925333in}}%
\pgfpathlineto{\pgfqpoint{1.561535in}{3.993440in}}%
\pgfpathlineto{\pgfqpoint{1.481173in}{4.037333in}}%
\pgfpathlineto{\pgfqpoint{1.321051in}{4.117929in}}%
\pgfpathlineto{\pgfqpoint{1.160727in}{4.188664in}}%
\pgfpathlineto{\pgfqpoint{1.067981in}{4.224000in}}%
\pgfpathlineto{\pgfqpoint{1.039888in}{4.224000in}}%
\pgfpathlineto{\pgfqpoint{1.040807in}{4.223700in}}%
\pgfpathlineto{\pgfqpoint{1.126669in}{4.192276in}}%
\pgfpathlineto{\pgfqpoint{1.160727in}{4.179140in}}%
\pgfpathlineto{\pgfqpoint{1.321051in}{4.109102in}}%
\pgfpathlineto{\pgfqpoint{1.481374in}{4.028704in}}%
\pgfpathlineto{\pgfqpoint{1.681778in}{3.916112in}}%
\pgfpathlineto{\pgfqpoint{1.761939in}{3.867675in}}%
\pgfpathlineto{\pgfqpoint{1.848706in}{3.813333in}}%
\pgfpathlineto{\pgfqpoint{1.963145in}{3.738667in}}%
\pgfpathlineto{\pgfqpoint{2.202828in}{3.571668in}}%
\pgfpathlineto{\pgfqpoint{2.330655in}{3.477333in}}%
\pgfpathlineto{\pgfqpoint{2.443313in}{3.391202in}}%
\pgfpathlineto{\pgfqpoint{2.570635in}{3.290667in}}%
\pgfpathlineto{\pgfqpoint{2.844121in}{3.063444in}}%
\pgfpathlineto{\pgfqpoint{2.968930in}{2.954667in}}%
\pgfpathlineto{\pgfqpoint{3.093790in}{2.842667in}}%
\pgfpathlineto{\pgfqpoint{3.189835in}{2.754015in}}%
\pgfpathlineto{\pgfqpoint{3.244929in}{2.702905in}}%
\pgfpathlineto{\pgfqpoint{3.333634in}{2.618667in}}%
\pgfpathlineto{\pgfqpoint{3.449017in}{2.506667in}}%
\pgfpathlineto{\pgfqpoint{3.505287in}{2.450510in}}%
\pgfpathlineto{\pgfqpoint{3.565576in}{2.390552in}}%
\pgfpathlineto{\pgfqpoint{3.658927in}{2.294952in}}%
\pgfpathlineto{\pgfqpoint{3.707133in}{2.245333in}}%
\pgfpathlineto{\pgfqpoint{3.813445in}{2.133333in}}%
\pgfpathlineto{\pgfqpoint{3.882912in}{2.058667in}}%
\pgfpathlineto{\pgfqpoint{3.984867in}{1.946667in}}%
\pgfpathlineto{\pgfqpoint{4.051623in}{1.872000in}}%
\pgfpathlineto{\pgfqpoint{4.149532in}{1.760000in}}%
\pgfpathlineto{\pgfqpoint{4.246949in}{1.645960in}}%
\pgfpathlineto{\pgfqpoint{4.333376in}{1.541836in}}%
\pgfpathlineto{\pgfqpoint{4.368938in}{1.498667in}}%
\pgfpathlineto{\pgfqpoint{4.429066in}{1.424000in}}%
\pgfpathlineto{\pgfqpoint{4.527515in}{1.299050in}}%
\pgfpathlineto{\pgfqpoint{4.631413in}{1.162667in}}%
\pgfpathlineto{\pgfqpoint{4.713988in}{1.050667in}}%
\pgfpathlineto{\pgfqpoint{4.768000in}{0.975629in}}%
\pgfpathlineto{\pgfqpoint{4.768000in}{0.976000in}}%
\pgfusepath{fill}%
\end{pgfscope}%
\begin{pgfscope}%
\pgfpathrectangle{\pgfqpoint{0.800000in}{0.528000in}}{\pgfqpoint{3.968000in}{3.696000in}}%
\pgfusepath{clip}%
\pgfsetbuttcap%
\pgfsetroundjoin%
\definecolor{currentfill}{rgb}{0.282290,0.145912,0.461510}%
\pgfsetfillcolor{currentfill}%
\pgfsetlinewidth{0.000000pt}%
\definecolor{currentstroke}{rgb}{0.000000,0.000000,0.000000}%
\pgfsetstrokecolor{currentstroke}%
\pgfsetdash{}{0pt}%
\pgfpathmoveto{\pgfqpoint{2.569135in}{0.528000in}}%
\pgfpathlineto{\pgfqpoint{2.298346in}{0.789333in}}%
\pgfpathlineto{\pgfqpoint{2.202828in}{0.885229in}}%
\pgfpathlineto{\pgfqpoint{2.078254in}{1.013333in}}%
\pgfpathlineto{\pgfqpoint{2.002424in}{1.093123in}}%
\pgfpathlineto{\pgfqpoint{1.868693in}{1.237333in}}%
\pgfpathlineto{\pgfqpoint{1.819966in}{1.291382in}}%
\pgfpathlineto{\pgfqpoint{1.761939in}{1.355613in}}%
\pgfpathlineto{\pgfqpoint{1.636511in}{1.498667in}}%
\pgfpathlineto{\pgfqpoint{1.572650in}{1.573333in}}%
\pgfpathlineto{\pgfqpoint{1.478692in}{1.685333in}}%
\pgfpathlineto{\pgfqpoint{1.393372in}{1.790031in}}%
\pgfpathlineto{\pgfqpoint{1.357254in}{1.834667in}}%
\pgfpathlineto{\pgfqpoint{1.291078in}{1.918749in}}%
\pgfpathlineto{\pgfqpoint{1.240047in}{1.984000in}}%
\pgfpathlineto{\pgfqpoint{1.174120in}{2.071141in}}%
\pgfpathlineto{\pgfqpoint{1.127310in}{2.133333in}}%
\pgfpathlineto{\pgfqpoint{1.040485in}{2.252515in}}%
\pgfpathlineto{\pgfqpoint{0.960323in}{2.366341in}}%
\pgfpathlineto{\pgfqpoint{0.880162in}{2.484354in}}%
\pgfpathlineto{\pgfqpoint{0.800000in}{2.607176in}}%
\pgfpathlineto{\pgfqpoint{0.800000in}{2.595585in}}%
\pgfpathlineto{\pgfqpoint{0.933611in}{2.394667in}}%
\pgfpathlineto{\pgfqpoint{1.011882in}{2.282667in}}%
\pgfpathlineto{\pgfqpoint{1.092771in}{2.170667in}}%
\pgfpathlineto{\pgfqpoint{1.153395in}{2.089170in}}%
\pgfpathlineto{\pgfqpoint{1.204464in}{2.021333in}}%
\pgfpathlineto{\pgfqpoint{1.262089in}{1.946667in}}%
\pgfpathlineto{\pgfqpoint{1.361131in}{1.821218in}}%
\pgfpathlineto{\pgfqpoint{1.601616in}{1.531167in}}%
\pgfpathlineto{\pgfqpoint{1.688926in}{1.430659in}}%
\pgfpathlineto{\pgfqpoint{1.727423in}{1.386667in}}%
\pgfpathlineto{\pgfqpoint{1.842101in}{1.258821in}}%
\pgfpathlineto{\pgfqpoint{1.945425in}{1.146908in}}%
\pgfpathlineto{\pgfqpoint{2.002424in}{1.085591in}}%
\pgfpathlineto{\pgfqpoint{2.082586in}{1.001450in}}%
\pgfpathlineto{\pgfqpoint{2.216583in}{0.864000in}}%
\pgfpathlineto{\pgfqpoint{2.306736in}{0.774119in}}%
\pgfpathlineto{\pgfqpoint{2.366651in}{0.714667in}}%
\pgfpathlineto{\pgfqpoint{2.443313in}{0.640241in}}%
\pgfpathlineto{\pgfqpoint{2.561499in}{0.528000in}}%
\pgfpathlineto{\pgfqpoint{2.563556in}{0.528000in}}%
\pgfpathmoveto{\pgfqpoint{4.768000in}{0.995527in}}%
\pgfpathlineto{\pgfqpoint{4.687838in}{1.105797in}}%
\pgfpathlineto{\pgfqpoint{4.607677in}{1.212872in}}%
\pgfpathlineto{\pgfqpoint{4.502219in}{1.349333in}}%
\pgfpathlineto{\pgfqpoint{4.407273in}{1.468503in}}%
\pgfpathlineto{\pgfqpoint{4.323883in}{1.570326in}}%
\pgfpathlineto{\pgfqpoint{4.287030in}{1.614732in}}%
\pgfpathlineto{\pgfqpoint{4.200511in}{1.716745in}}%
\pgfpathlineto{\pgfqpoint{4.163543in}{1.760000in}}%
\pgfpathlineto{\pgfqpoint{4.065494in}{1.872000in}}%
\pgfpathlineto{\pgfqpoint{3.998917in}{1.946667in}}%
\pgfpathlineto{\pgfqpoint{3.886222in}{2.070270in}}%
\pgfpathlineto{\pgfqpoint{3.806061in}{2.156246in}}%
\pgfpathlineto{\pgfqpoint{3.685616in}{2.282667in}}%
\pgfpathlineto{\pgfqpoint{3.589985in}{2.380069in}}%
\pgfpathlineto{\pgfqpoint{3.538790in}{2.432000in}}%
\pgfpathlineto{\pgfqpoint{3.425577in}{2.544000in}}%
\pgfpathlineto{\pgfqpoint{3.309532in}{2.656000in}}%
\pgfpathlineto{\pgfqpoint{3.190475in}{2.768000in}}%
\pgfpathlineto{\pgfqpoint{3.068211in}{2.880000in}}%
\pgfpathlineto{\pgfqpoint{2.804040in}{3.111818in}}%
\pgfpathlineto{\pgfqpoint{2.763960in}{3.145790in}}%
\pgfpathlineto{\pgfqpoint{2.634289in}{3.253333in}}%
\pgfpathlineto{\pgfqpoint{2.523475in}{3.342570in}}%
\pgfpathlineto{\pgfqpoint{2.398815in}{3.440000in}}%
\pgfpathlineto{\pgfqpoint{2.282990in}{3.527474in}}%
\pgfpathlineto{\pgfqpoint{2.162747in}{3.615261in}}%
\pgfpathlineto{\pgfqpoint{2.110123in}{3.652316in}}%
\pgfpathlineto{\pgfqpoint{2.042505in}{3.699870in}}%
\pgfpathlineto{\pgfqpoint{1.922263in}{3.780934in}}%
\pgfpathlineto{\pgfqpoint{1.681778in}{3.932084in}}%
\pgfpathlineto{\pgfqpoint{1.601616in}{3.978865in}}%
\pgfpathlineto{\pgfqpoint{1.512078in}{4.028600in}}%
\pgfpathlineto{\pgfqpoint{1.481374in}{4.045489in}}%
\pgfpathlineto{\pgfqpoint{1.304588in}{4.134000in}}%
\pgfpathlineto{\pgfqpoint{1.272628in}{4.149333in}}%
\pgfpathlineto{\pgfqpoint{1.160727in}{4.197925in}}%
\pgfpathlineto{\pgfqpoint{1.094375in}{4.224000in}}%
\pgfpathlineto{\pgfqpoint{1.067981in}{4.224000in}}%
\pgfpathlineto{\pgfqpoint{1.120646in}{4.204393in}}%
\pgfpathlineto{\pgfqpoint{1.160727in}{4.188664in}}%
\pgfpathlineto{\pgfqpoint{1.252688in}{4.149333in}}%
\pgfpathlineto{\pgfqpoint{1.333335in}{4.112000in}}%
\pgfpathlineto{\pgfqpoint{1.481374in}{4.037228in}}%
\pgfpathlineto{\pgfqpoint{1.681778in}{3.924201in}}%
\pgfpathlineto{\pgfqpoint{1.896778in}{3.789596in}}%
\pgfpathlineto{\pgfqpoint{1.943731in}{3.758664in}}%
\pgfpathlineto{\pgfqpoint{2.002424in}{3.719679in}}%
\pgfpathlineto{\pgfqpoint{2.059717in}{3.680032in}}%
\pgfpathlineto{\pgfqpoint{2.122667in}{3.636431in}}%
\pgfpathlineto{\pgfqpoint{2.253661in}{3.541985in}}%
\pgfpathlineto{\pgfqpoint{2.363152in}{3.459983in}}%
\pgfpathlineto{\pgfqpoint{2.485732in}{3.365333in}}%
\pgfpathlineto{\pgfqpoint{2.625542in}{3.253333in}}%
\pgfpathlineto{\pgfqpoint{2.723879in}{3.172320in}}%
\pgfpathlineto{\pgfqpoint{2.848604in}{3.066667in}}%
\pgfpathlineto{\pgfqpoint{2.976873in}{2.954667in}}%
\pgfpathlineto{\pgfqpoint{3.101562in}{2.842667in}}%
\pgfpathlineto{\pgfqpoint{3.379817in}{2.581333in}}%
\pgfpathlineto{\pgfqpoint{3.494104in}{2.469333in}}%
\pgfpathlineto{\pgfqpoint{3.586212in}{2.376555in}}%
\pgfpathlineto{\pgfqpoint{3.645737in}{2.316342in}}%
\pgfpathlineto{\pgfqpoint{3.738497in}{2.219735in}}%
\pgfpathlineto{\pgfqpoint{3.785366in}{2.170667in}}%
\pgfpathlineto{\pgfqpoint{3.889963in}{2.058667in}}%
\pgfpathlineto{\pgfqpoint{3.966384in}{1.974977in}}%
\pgfpathlineto{\pgfqpoint{4.091542in}{1.834667in}}%
\pgfpathlineto{\pgfqpoint{4.188627in}{1.722667in}}%
\pgfpathlineto{\pgfqpoint{4.246949in}{1.654171in}}%
\pgfpathlineto{\pgfqpoint{4.345259in}{1.536000in}}%
\pgfpathlineto{\pgfqpoint{4.447354in}{1.409885in}}%
\pgfpathlineto{\pgfqpoint{4.647758in}{1.150311in}}%
\pgfpathlineto{\pgfqpoint{4.768000in}{0.985585in}}%
\pgfpathlineto{\pgfqpoint{4.768000in}{0.985585in}}%
\pgfusepath{fill}%
\end{pgfscope}%
\begin{pgfscope}%
\pgfpathrectangle{\pgfqpoint{0.800000in}{0.528000in}}{\pgfqpoint{3.968000in}{3.696000in}}%
\pgfusepath{clip}%
\pgfsetbuttcap%
\pgfsetroundjoin%
\definecolor{currentfill}{rgb}{0.282290,0.145912,0.461510}%
\pgfsetfillcolor{currentfill}%
\pgfsetlinewidth{0.000000pt}%
\definecolor{currentstroke}{rgb}{0.000000,0.000000,0.000000}%
\pgfsetstrokecolor{currentstroke}%
\pgfsetdash{}{0pt}%
\pgfpathmoveto{\pgfqpoint{2.561499in}{0.528000in}}%
\pgfpathlineto{\pgfqpoint{2.443313in}{0.640241in}}%
\pgfpathlineto{\pgfqpoint{2.345661in}{0.735708in}}%
\pgfpathlineto{\pgfqpoint{2.282990in}{0.797303in}}%
\pgfpathlineto{\pgfqpoint{2.162747in}{0.918735in}}%
\pgfpathlineto{\pgfqpoint{2.057871in}{1.027646in}}%
\pgfpathlineto{\pgfqpoint{2.000155in}{1.088000in}}%
\pgfpathlineto{\pgfqpoint{1.882182in}{1.214932in}}%
\pgfpathlineto{\pgfqpoint{1.779628in}{1.328476in}}%
\pgfpathlineto{\pgfqpoint{1.721859in}{1.392962in}}%
\pgfpathlineto{\pgfqpoint{1.641697in}{1.484696in}}%
\pgfpathlineto{\pgfqpoint{1.534172in}{1.610667in}}%
\pgfpathlineto{\pgfqpoint{1.440964in}{1.722667in}}%
\pgfpathlineto{\pgfqpoint{1.361131in}{1.821218in}}%
\pgfpathlineto{\pgfqpoint{1.262089in}{1.946667in}}%
\pgfpathlineto{\pgfqpoint{1.219884in}{2.001768in}}%
\pgfpathlineto{\pgfqpoint{1.176199in}{2.058667in}}%
\pgfpathlineto{\pgfqpoint{1.137051in}{2.111280in}}%
\pgfpathlineto{\pgfqpoint{1.092771in}{2.170667in}}%
\pgfpathlineto{\pgfqpoint{1.065539in}{2.208000in}}%
\pgfpathlineto{\pgfqpoint{0.985512in}{2.320000in}}%
\pgfpathlineto{\pgfqpoint{0.908105in}{2.432000in}}%
\pgfpathlineto{\pgfqpoint{0.840081in}{2.533827in}}%
\pgfpathlineto{\pgfqpoint{0.800000in}{2.595585in}}%
\pgfpathlineto{\pgfqpoint{0.801086in}{2.582345in}}%
\pgfpathlineto{\pgfqpoint{0.804281in}{2.577346in}}%
\pgfpathlineto{\pgfqpoint{0.880162in}{2.462550in}}%
\pgfpathlineto{\pgfqpoint{0.960323in}{2.345713in}}%
\pgfpathlineto{\pgfqpoint{1.040485in}{2.232812in}}%
\pgfpathlineto{\pgfqpoint{1.120646in}{2.123373in}}%
\pgfpathlineto{\pgfqpoint{1.321051in}{1.862754in}}%
\pgfpathlineto{\pgfqpoint{1.561535in}{1.570048in}}%
\pgfpathlineto{\pgfqpoint{1.687631in}{1.424000in}}%
\pgfpathlineto{\pgfqpoint{1.775823in}{1.324932in}}%
\pgfpathlineto{\pgfqpoint{1.820794in}{1.274667in}}%
\pgfpathlineto{\pgfqpoint{1.904525in}{1.183478in}}%
\pgfpathlineto{\pgfqpoint{1.962343in}{1.120800in}}%
\pgfpathlineto{\pgfqpoint{2.100029in}{0.976000in}}%
\pgfpathlineto{\pgfqpoint{2.172620in}{0.901333in}}%
\pgfpathlineto{\pgfqpoint{2.283669in}{0.789333in}}%
\pgfpathlineto{\pgfqpoint{2.380995in}{0.693954in}}%
\pgfpathlineto{\pgfqpoint{2.443313in}{0.633186in}}%
\pgfpathlineto{\pgfqpoint{2.554038in}{0.528000in}}%
\pgfpathmoveto{\pgfqpoint{4.768000in}{1.005469in}}%
\pgfpathlineto{\pgfqpoint{4.687838in}{1.115333in}}%
\pgfpathlineto{\pgfqpoint{4.607677in}{1.222035in}}%
\pgfpathlineto{\pgfqpoint{4.509141in}{1.349333in}}%
\pgfpathlineto{\pgfqpoint{4.419923in}{1.461333in}}%
\pgfpathlineto{\pgfqpoint{4.327111in}{1.574907in}}%
\pgfpathlineto{\pgfqpoint{4.202548in}{1.722667in}}%
\pgfpathlineto{\pgfqpoint{4.105324in}{1.834667in}}%
\pgfpathlineto{\pgfqpoint{4.024166in}{1.925821in}}%
\pgfpathlineto{\pgfqpoint{3.966384in}{1.990382in}}%
\pgfpathlineto{\pgfqpoint{3.877059in}{2.087465in}}%
\pgfpathlineto{\pgfqpoint{3.834579in}{2.133333in}}%
\pgfpathlineto{\pgfqpoint{3.745966in}{2.226691in}}%
\pgfpathlineto{\pgfqpoint{3.685818in}{2.289732in}}%
\pgfpathlineto{\pgfqpoint{3.593758in}{2.383584in}}%
\pgfpathlineto{\pgfqpoint{3.545996in}{2.432000in}}%
\pgfpathlineto{\pgfqpoint{3.432928in}{2.544000in}}%
\pgfpathlineto{\pgfqpoint{3.317035in}{2.656000in}}%
\pgfpathlineto{\pgfqpoint{3.198137in}{2.768000in}}%
\pgfpathlineto{\pgfqpoint{3.076039in}{2.880000in}}%
\pgfpathlineto{\pgfqpoint{2.978033in}{2.967399in}}%
\pgfpathlineto{\pgfqpoint{2.924283in}{3.015047in}}%
\pgfpathlineto{\pgfqpoint{2.864846in}{3.066667in}}%
\pgfpathlineto{\pgfqpoint{2.763960in}{3.152767in}}%
\pgfpathlineto{\pgfqpoint{2.643036in}{3.253333in}}%
\pgfpathlineto{\pgfqpoint{2.503659in}{3.365333in}}%
\pgfpathlineto{\pgfqpoint{2.259600in}{3.552000in}}%
\pgfpathlineto{\pgfqpoint{2.157079in}{3.626667in}}%
\pgfpathlineto{\pgfqpoint{2.042505in}{3.707237in}}%
\pgfpathlineto{\pgfqpoint{1.802020in}{3.866076in}}%
\pgfpathlineto{\pgfqpoint{1.706024in}{3.925333in}}%
\pgfpathlineto{\pgfqpoint{1.601616in}{3.986921in}}%
\pgfpathlineto{\pgfqpoint{1.511703in}{4.037333in}}%
\pgfpathlineto{\pgfqpoint{1.440147in}{4.075734in}}%
\pgfpathlineto{\pgfqpoint{1.361131in}{4.115925in}}%
\pgfpathlineto{\pgfqpoint{1.193408in}{4.193560in}}%
\pgfpathlineto{\pgfqpoint{1.119442in}{4.224000in}}%
\pgfpathlineto{\pgfqpoint{1.094375in}{4.224000in}}%
\pgfpathlineto{\pgfqpoint{1.160727in}{4.197925in}}%
\pgfpathlineto{\pgfqpoint{1.240889in}{4.163640in}}%
\pgfpathlineto{\pgfqpoint{1.304588in}{4.134000in}}%
\pgfpathlineto{\pgfqpoint{1.351425in}{4.112000in}}%
\pgfpathlineto{\pgfqpoint{1.441293in}{4.066642in}}%
\pgfpathlineto{\pgfqpoint{1.641697in}{3.955736in}}%
\pgfpathlineto{\pgfqpoint{1.721859in}{3.907916in}}%
\pgfpathlineto{\pgfqpoint{1.814222in}{3.850667in}}%
\pgfpathlineto{\pgfqpoint{2.018216in}{3.716043in}}%
\pgfpathlineto{\pgfqpoint{2.082586in}{3.672000in}}%
\pgfpathlineto{\pgfqpoint{2.155706in}{3.620108in}}%
\pgfpathlineto{\pgfqpoint{2.202828in}{3.586430in}}%
\pgfpathlineto{\pgfqpoint{2.300133in}{3.514667in}}%
\pgfpathlineto{\pgfqpoint{2.541744in}{3.328000in}}%
\pgfpathlineto{\pgfqpoint{2.804040in}{3.111818in}}%
\pgfpathlineto{\pgfqpoint{2.924283in}{3.008021in}}%
\pgfpathlineto{\pgfqpoint{3.044525in}{2.901372in}}%
\pgfpathlineto{\pgfqpoint{3.137141in}{2.816934in}}%
\pgfpathlineto{\pgfqpoint{3.190475in}{2.768000in}}%
\pgfpathlineto{\pgfqpoint{3.463621in}{2.506667in}}%
\pgfpathlineto{\pgfqpoint{3.551512in}{2.418901in}}%
\pgfpathlineto{\pgfqpoint{3.612775in}{2.357333in}}%
\pgfpathlineto{\pgfqpoint{3.666435in}{2.301945in}}%
\pgfpathlineto{\pgfqpoint{3.725899in}{2.240751in}}%
\pgfpathlineto{\pgfqpoint{3.817386in}{2.143882in}}%
\pgfpathlineto{\pgfqpoint{3.862353in}{2.096000in}}%
\pgfpathlineto{\pgfqpoint{3.947146in}{2.003414in}}%
\pgfpathlineto{\pgfqpoint{4.006465in}{1.938274in}}%
\pgfpathlineto{\pgfqpoint{4.098433in}{1.834667in}}%
\pgfpathlineto{\pgfqpoint{4.206869in}{1.709488in}}%
\pgfpathlineto{\pgfqpoint{4.327111in}{1.566514in}}%
\pgfpathlineto{\pgfqpoint{4.560406in}{1.274667in}}%
\pgfpathlineto{\pgfqpoint{4.613322in}{1.205258in}}%
\pgfpathlineto{\pgfqpoint{4.647758in}{1.159825in}}%
\pgfpathlineto{\pgfqpoint{4.727919in}{1.051247in}}%
\pgfpathlineto{\pgfqpoint{4.768000in}{0.995527in}}%
\pgfpathlineto{\pgfqpoint{4.768000in}{0.995527in}}%
\pgfusepath{fill}%
\end{pgfscope}%
\begin{pgfscope}%
\pgfpathrectangle{\pgfqpoint{0.800000in}{0.528000in}}{\pgfqpoint{3.968000in}{3.696000in}}%
\pgfusepath{clip}%
\pgfsetbuttcap%
\pgfsetroundjoin%
\definecolor{currentfill}{rgb}{0.281887,0.150881,0.465405}%
\pgfsetfillcolor{currentfill}%
\pgfsetlinewidth{0.000000pt}%
\definecolor{currentstroke}{rgb}{0.000000,0.000000,0.000000}%
\pgfsetstrokecolor{currentstroke}%
\pgfsetdash{}{0pt}%
\pgfpathmoveto{\pgfqpoint{2.554038in}{0.528000in}}%
\pgfpathlineto{\pgfqpoint{2.436246in}{0.640000in}}%
\pgfpathlineto{\pgfqpoint{2.341913in}{0.732217in}}%
\pgfpathlineto{\pgfqpoint{2.282990in}{0.790009in}}%
\pgfpathlineto{\pgfqpoint{2.162747in}{0.911401in}}%
\pgfpathlineto{\pgfqpoint{2.072961in}{1.004368in}}%
\pgfpathlineto{\pgfqpoint{2.028538in}{1.050667in}}%
\pgfpathlineto{\pgfqpoint{1.941678in}{1.143418in}}%
\pgfpathlineto{\pgfqpoint{1.882182in}{1.207287in}}%
\pgfpathlineto{\pgfqpoint{1.794005in}{1.304534in}}%
\pgfpathlineto{\pgfqpoint{1.753665in}{1.349333in}}%
\pgfpathlineto{\pgfqpoint{1.641697in}{1.476730in}}%
\pgfpathlineto{\pgfqpoint{1.521455in}{1.617527in}}%
\pgfpathlineto{\pgfqpoint{1.401212in}{1.762926in}}%
\pgfpathlineto{\pgfqpoint{1.316886in}{1.868121in}}%
\pgfpathlineto{\pgfqpoint{1.280970in}{1.913485in}}%
\pgfpathlineto{\pgfqpoint{1.169173in}{2.058667in}}%
\pgfpathlineto{\pgfqpoint{1.120646in}{2.123373in}}%
\pgfpathlineto{\pgfqpoint{1.040485in}{2.232812in}}%
\pgfpathlineto{\pgfqpoint{0.900906in}{2.432000in}}%
\pgfpathlineto{\pgfqpoint{0.826124in}{2.544000in}}%
\pgfpathlineto{\pgfqpoint{0.800000in}{2.583993in}}%
\pgfpathlineto{\pgfqpoint{0.800000in}{2.572776in}}%
\pgfpathlineto{\pgfqpoint{0.880162in}{2.451948in}}%
\pgfpathlineto{\pgfqpoint{0.960323in}{2.335572in}}%
\pgfpathlineto{\pgfqpoint{1.040485in}{2.223092in}}%
\pgfpathlineto{\pgfqpoint{1.120646in}{2.114042in}}%
\pgfpathlineto{\pgfqpoint{1.321051in}{1.854132in}}%
\pgfpathlineto{\pgfqpoint{1.538710in}{1.589406in}}%
\pgfpathlineto{\pgfqpoint{1.583803in}{1.536000in}}%
\pgfpathlineto{\pgfqpoint{1.648155in}{1.461333in}}%
\pgfpathlineto{\pgfqpoint{1.761939in}{1.332365in}}%
\pgfpathlineto{\pgfqpoint{1.863801in}{1.220213in}}%
\pgfpathlineto{\pgfqpoint{1.922263in}{1.156343in}}%
\pgfpathlineto{\pgfqpoint{2.012695in}{1.060233in}}%
\pgfpathlineto{\pgfqpoint{2.057096in}{1.013333in}}%
\pgfpathlineto{\pgfqpoint{2.145290in}{0.922406in}}%
\pgfpathlineto{\pgfqpoint{2.202828in}{0.863285in}}%
\pgfpathlineto{\pgfqpoint{2.299259in}{0.767154in}}%
\pgfpathlineto{\pgfqpoint{2.352118in}{0.714667in}}%
\pgfpathlineto{\pgfqpoint{2.467821in}{0.602667in}}%
\pgfpathlineto{\pgfqpoint{2.546577in}{0.528000in}}%
\pgfpathmoveto{\pgfqpoint{4.768000in}{1.015336in}}%
\pgfpathlineto{\pgfqpoint{4.703720in}{1.102793in}}%
\pgfpathlineto{\pgfqpoint{4.659502in}{1.162667in}}%
\pgfpathlineto{\pgfqpoint{4.567596in}{1.283312in}}%
\pgfpathlineto{\pgfqpoint{4.456834in}{1.424000in}}%
\pgfpathlineto{\pgfqpoint{4.366084in}{1.536000in}}%
\pgfpathlineto{\pgfqpoint{4.278936in}{1.640460in}}%
\pgfpathlineto{\pgfqpoint{4.241247in}{1.685333in}}%
\pgfpathlineto{\pgfqpoint{4.144842in}{1.797333in}}%
\pgfpathlineto{\pgfqpoint{4.064303in}{1.888541in}}%
\pgfpathlineto{\pgfqpoint{4.006465in}{1.953691in}}%
\pgfpathlineto{\pgfqpoint{3.876352in}{2.096000in}}%
\pgfpathlineto{\pgfqpoint{3.765980in}{2.213529in}}%
\pgfpathlineto{\pgfqpoint{3.673942in}{2.308938in}}%
\pgfpathlineto{\pgfqpoint{3.626997in}{2.357333in}}%
\pgfpathlineto{\pgfqpoint{3.515863in}{2.469333in}}%
\pgfpathlineto{\pgfqpoint{3.402021in}{2.581333in}}%
\pgfpathlineto{\pgfqpoint{3.324539in}{2.656000in}}%
\pgfpathlineto{\pgfqpoint{3.204848in}{2.768860in}}%
\pgfpathlineto{\pgfqpoint{3.083867in}{2.880000in}}%
\pgfpathlineto{\pgfqpoint{2.981857in}{2.970961in}}%
\pgfpathlineto{\pgfqpoint{2.920205in}{3.025535in}}%
\pgfpathlineto{\pgfqpoint{2.872967in}{3.066667in}}%
\pgfpathlineto{\pgfqpoint{2.741431in}{3.178667in}}%
\pgfpathlineto{\pgfqpoint{2.483394in}{3.388412in}}%
\pgfpathlineto{\pgfqpoint{2.363152in}{3.481565in}}%
\pgfpathlineto{\pgfqpoint{2.242909in}{3.571530in}}%
\pgfpathlineto{\pgfqpoint{2.114846in}{3.664000in}}%
\pgfpathlineto{\pgfqpoint{1.882182in}{3.822298in}}%
\pgfpathlineto{\pgfqpoint{1.779298in}{3.888000in}}%
\pgfpathlineto{\pgfqpoint{1.622671in}{3.982278in}}%
\pgfpathlineto{\pgfqpoint{1.573173in}{4.010840in}}%
\pgfpathlineto{\pgfqpoint{1.526630in}{4.037333in}}%
\pgfpathlineto{\pgfqpoint{1.419877in}{4.094615in}}%
\pgfpathlineto{\pgfqpoint{1.321051in}{4.144121in}}%
\pgfpathlineto{\pgfqpoint{1.263756in}{4.170633in}}%
\pgfpathlineto{\pgfqpoint{1.229473in}{4.186667in}}%
\pgfpathlineto{\pgfqpoint{1.142232in}{4.224000in}}%
\pgfpathlineto{\pgfqpoint{1.119442in}{4.224000in}}%
\pgfpathlineto{\pgfqpoint{1.120646in}{4.223540in}}%
\pgfpathlineto{\pgfqpoint{1.257389in}{4.164703in}}%
\pgfpathlineto{\pgfqpoint{1.291670in}{4.149333in}}%
\pgfpathlineto{\pgfqpoint{1.442168in}{4.074667in}}%
\pgfpathlineto{\pgfqpoint{1.641697in}{3.963775in}}%
\pgfpathlineto{\pgfqpoint{1.763973in}{3.889894in}}%
\pgfpathlineto{\pgfqpoint{1.826305in}{3.850667in}}%
\pgfpathlineto{\pgfqpoint{2.002424in}{3.734721in}}%
\pgfpathlineto{\pgfqpoint{2.114686in}{3.656566in}}%
\pgfpathlineto{\pgfqpoint{2.162747in}{3.622628in}}%
\pgfpathlineto{\pgfqpoint{2.259600in}{3.552000in}}%
\pgfpathlineto{\pgfqpoint{2.380291in}{3.461368in}}%
\pgfpathlineto{\pgfqpoint{2.503659in}{3.365333in}}%
\pgfpathlineto{\pgfqpoint{2.603636in}{3.285383in}}%
\pgfpathlineto{\pgfqpoint{2.733123in}{3.178667in}}%
\pgfpathlineto{\pgfqpoint{2.864846in}{3.066667in}}%
\pgfpathlineto{\pgfqpoint{2.964364in}{2.979829in}}%
\pgfpathlineto{\pgfqpoint{3.059838in}{2.894263in}}%
\pgfpathlineto{\pgfqpoint{3.124687in}{2.835753in}}%
\pgfpathlineto{\pgfqpoint{3.204848in}{2.761769in}}%
\pgfpathlineto{\pgfqpoint{3.325091in}{2.648322in}}%
\pgfpathlineto{\pgfqpoint{3.405253in}{2.571022in}}%
\pgfpathlineto{\pgfqpoint{3.525495in}{2.452533in}}%
\pgfpathlineto{\pgfqpoint{3.656403in}{2.320000in}}%
\pgfpathlineto{\pgfqpoint{3.745966in}{2.226691in}}%
\pgfpathlineto{\pgfqpoint{3.806061in}{2.163762in}}%
\pgfpathlineto{\pgfqpoint{3.886222in}{2.077815in}}%
\pgfpathlineto{\pgfqpoint{4.006465in}{1.946086in}}%
\pgfpathlineto{\pgfqpoint{4.096782in}{1.844126in}}%
\pgfpathlineto{\pgfqpoint{4.137994in}{1.797333in}}%
\pgfpathlineto{\pgfqpoint{4.246949in}{1.670467in}}%
\pgfpathlineto{\pgfqpoint{4.367192in}{1.526190in}}%
\pgfpathlineto{\pgfqpoint{4.583964in}{1.252579in}}%
\pgfpathlineto{\pgfqpoint{4.624385in}{1.200000in}}%
\pgfpathlineto{\pgfqpoint{4.666811in}{1.143081in}}%
\pgfpathlineto{\pgfqpoint{4.707954in}{1.088000in}}%
\pgfpathlineto{\pgfqpoint{4.768000in}{1.005469in}}%
\pgfpathlineto{\pgfqpoint{4.768000in}{1.013333in}}%
\pgfpathlineto{\pgfqpoint{4.768000in}{1.013333in}}%
\pgfusepath{fill}%
\end{pgfscope}%
\begin{pgfscope}%
\pgfpathrectangle{\pgfqpoint{0.800000in}{0.528000in}}{\pgfqpoint{3.968000in}{3.696000in}}%
\pgfusepath{clip}%
\pgfsetbuttcap%
\pgfsetroundjoin%
\definecolor{currentfill}{rgb}{0.281887,0.150881,0.465405}%
\pgfsetfillcolor{currentfill}%
\pgfsetlinewidth{0.000000pt}%
\definecolor{currentstroke}{rgb}{0.000000,0.000000,0.000000}%
\pgfsetstrokecolor{currentstroke}%
\pgfsetdash{}{0pt}%
\pgfpathmoveto{\pgfqpoint{2.546577in}{0.528000in}}%
\pgfpathlineto{\pgfqpoint{2.428936in}{0.640000in}}%
\pgfpathlineto{\pgfqpoint{2.338164in}{0.728726in}}%
\pgfpathlineto{\pgfqpoint{2.276530in}{0.789333in}}%
\pgfpathlineto{\pgfqpoint{2.162747in}{0.904066in}}%
\pgfpathlineto{\pgfqpoint{2.069279in}{1.000939in}}%
\pgfpathlineto{\pgfqpoint{2.021531in}{1.050667in}}%
\pgfpathlineto{\pgfqpoint{1.937931in}{1.139928in}}%
\pgfpathlineto{\pgfqpoint{1.881863in}{1.200000in}}%
\pgfpathlineto{\pgfqpoint{1.790296in}{1.301080in}}%
\pgfpathlineto{\pgfqpoint{1.746810in}{1.349333in}}%
\pgfpathlineto{\pgfqpoint{1.641697in}{1.468764in}}%
\pgfpathlineto{\pgfqpoint{1.516163in}{1.615595in}}%
\pgfpathlineto{\pgfqpoint{1.396745in}{1.760000in}}%
\pgfpathlineto{\pgfqpoint{1.313001in}{1.864503in}}%
\pgfpathlineto{\pgfqpoint{1.267421in}{1.921954in}}%
\pgfpathlineto{\pgfqpoint{1.160727in}{2.060548in}}%
\pgfpathlineto{\pgfqpoint{1.078633in}{2.170667in}}%
\pgfpathlineto{\pgfqpoint{0.997623in}{2.282667in}}%
\pgfpathlineto{\pgfqpoint{0.919151in}{2.394667in}}%
\pgfpathlineto{\pgfqpoint{0.848750in}{2.498592in}}%
\pgfpathlineto{\pgfqpoint{0.800000in}{2.572776in}}%
\pgfpathlineto{\pgfqpoint{0.800000in}{2.561671in}}%
\pgfpathlineto{\pgfqpoint{0.868733in}{2.458688in}}%
\pgfpathlineto{\pgfqpoint{0.912119in}{2.394667in}}%
\pgfpathlineto{\pgfqpoint{0.947017in}{2.344939in}}%
\pgfpathlineto{\pgfqpoint{0.990661in}{2.282667in}}%
\pgfpathlineto{\pgfqpoint{1.026888in}{2.232669in}}%
\pgfpathlineto{\pgfqpoint{1.071741in}{2.170667in}}%
\pgfpathlineto{\pgfqpoint{1.124540in}{2.099626in}}%
\pgfpathlineto{\pgfqpoint{1.160727in}{2.051491in}}%
\pgfpathlineto{\pgfqpoint{1.377752in}{1.775481in}}%
\pgfpathlineto{\pgfqpoint{1.420509in}{1.722667in}}%
\pgfpathlineto{\pgfqpoint{1.521455in}{1.601291in}}%
\pgfpathlineto{\pgfqpoint{1.608990in}{1.498667in}}%
\pgfpathlineto{\pgfqpoint{1.721859in}{1.369668in}}%
\pgfpathlineto{\pgfqpoint{1.842101in}{1.235886in}}%
\pgfpathlineto{\pgfqpoint{1.934184in}{1.136437in}}%
\pgfpathlineto{\pgfqpoint{1.979268in}{1.088000in}}%
\pgfpathlineto{\pgfqpoint{2.085832in}{0.976000in}}%
\pgfpathlineto{\pgfqpoint{2.162747in}{0.896856in}}%
\pgfpathlineto{\pgfqpoint{2.282990in}{0.775793in}}%
\pgfpathlineto{\pgfqpoint{2.373479in}{0.686953in}}%
\pgfpathlineto{\pgfqpoint{2.421625in}{0.640000in}}%
\pgfpathlineto{\pgfqpoint{2.539116in}{0.528000in}}%
\pgfpathmoveto{\pgfqpoint{4.768000in}{1.024918in}}%
\pgfpathlineto{\pgfqpoint{4.567596in}{1.292148in}}%
\pgfpathlineto{\pgfqpoint{4.463645in}{1.424000in}}%
\pgfpathlineto{\pgfqpoint{4.367192in}{1.542890in}}%
\pgfpathlineto{\pgfqpoint{4.279645in}{1.648000in}}%
\pgfpathlineto{\pgfqpoint{4.184051in}{1.760000in}}%
\pgfpathlineto{\pgfqpoint{4.104268in}{1.851099in}}%
\pgfpathlineto{\pgfqpoint{4.046545in}{1.916650in}}%
\pgfpathlineto{\pgfqpoint{3.958492in}{2.013982in}}%
\pgfpathlineto{\pgfqpoint{3.917781in}{2.058667in}}%
\pgfpathlineto{\pgfqpoint{3.806061in}{2.178571in}}%
\pgfpathlineto{\pgfqpoint{3.725899in}{2.262724in}}%
\pgfpathlineto{\pgfqpoint{3.597401in}{2.394667in}}%
\pgfpathlineto{\pgfqpoint{3.485414in}{2.506775in}}%
\pgfpathlineto{\pgfqpoint{3.387447in}{2.602082in}}%
\pgfpathlineto{\pgfqpoint{3.325091in}{2.662452in}}%
\pgfpathlineto{\pgfqpoint{3.204848in}{2.775788in}}%
\pgfpathlineto{\pgfqpoint{3.084606in}{2.886241in}}%
\pgfpathlineto{\pgfqpoint{3.006410in}{2.956497in}}%
\pgfpathlineto{\pgfqpoint{2.964364in}{2.993857in}}%
\pgfpathlineto{\pgfqpoint{2.837739in}{3.104000in}}%
\pgfpathlineto{\pgfqpoint{2.705066in}{3.216000in}}%
\pgfpathlineto{\pgfqpoint{2.443313in}{3.426832in}}%
\pgfpathlineto{\pgfqpoint{2.323071in}{3.519038in}}%
\pgfpathlineto{\pgfqpoint{2.228425in}{3.589333in}}%
\pgfpathlineto{\pgfqpoint{2.002424in}{3.749457in}}%
\pgfpathlineto{\pgfqpoint{1.907198in}{3.813333in}}%
\pgfpathlineto{\pgfqpoint{1.802020in}{3.881410in}}%
\pgfpathlineto{\pgfqpoint{1.681778in}{3.955631in}}%
\pgfpathlineto{\pgfqpoint{1.601616in}{4.002945in}}%
\pgfpathlineto{\pgfqpoint{1.399644in}{4.113461in}}%
\pgfpathlineto{\pgfqpoint{1.321051in}{4.152739in}}%
\pgfpathlineto{\pgfqpoint{1.164543in}{4.224000in}}%
\pgfpathlineto{\pgfqpoint{1.142232in}{4.224000in}}%
\pgfpathlineto{\pgfqpoint{1.240889in}{4.181615in}}%
\pgfpathlineto{\pgfqpoint{1.419877in}{4.094615in}}%
\pgfpathlineto{\pgfqpoint{1.526630in}{4.037333in}}%
\pgfpathlineto{\pgfqpoint{1.641697in}{3.971608in}}%
\pgfpathlineto{\pgfqpoint{1.727121in}{3.920431in}}%
\pgfpathlineto{\pgfqpoint{1.850281in}{3.843048in}}%
\pgfpathlineto{\pgfqpoint{1.962343in}{3.769235in}}%
\pgfpathlineto{\pgfqpoint{2.061541in}{3.701333in}}%
\pgfpathlineto{\pgfqpoint{2.282990in}{3.541887in}}%
\pgfpathlineto{\pgfqpoint{2.403232in}{3.450818in}}%
\pgfpathlineto{\pgfqpoint{2.465190in}{3.402667in}}%
\pgfpathlineto{\pgfqpoint{2.563556in}{3.324779in}}%
\pgfpathlineto{\pgfqpoint{2.683798in}{3.226712in}}%
\pgfpathlineto{\pgfqpoint{2.763960in}{3.159745in}}%
\pgfpathlineto{\pgfqpoint{2.884202in}{3.056963in}}%
\pgfpathlineto{\pgfqpoint{3.004444in}{2.951344in}}%
\pgfpathlineto{\pgfqpoint{3.063642in}{2.897806in}}%
\pgfpathlineto{\pgfqpoint{3.124873in}{2.842667in}}%
\pgfpathlineto{\pgfqpoint{3.205304in}{2.768424in}}%
\pgfpathlineto{\pgfqpoint{3.245700in}{2.730667in}}%
\pgfpathlineto{\pgfqpoint{3.324539in}{2.656000in}}%
\pgfpathlineto{\pgfqpoint{3.405253in}{2.578200in}}%
\pgfpathlineto{\pgfqpoint{3.525495in}{2.459750in}}%
\pgfpathlineto{\pgfqpoint{3.663467in}{2.320000in}}%
\pgfpathlineto{\pgfqpoint{3.749700in}{2.230170in}}%
\pgfpathlineto{\pgfqpoint{3.806621in}{2.170667in}}%
\pgfpathlineto{\pgfqpoint{3.886222in}{2.085360in}}%
\pgfpathlineto{\pgfqpoint{4.012781in}{1.946667in}}%
\pgfpathlineto{\pgfqpoint{4.100525in}{1.847613in}}%
\pgfpathlineto{\pgfqpoint{4.144842in}{1.797333in}}%
\pgfpathlineto{\pgfqpoint{4.246949in}{1.678615in}}%
\pgfpathlineto{\pgfqpoint{4.367192in}{1.534650in}}%
\pgfpathlineto{\pgfqpoint{4.447354in}{1.435827in}}%
\pgfpathlineto{\pgfqpoint{4.545263in}{1.312000in}}%
\pgfpathlineto{\pgfqpoint{4.631333in}{1.200000in}}%
\pgfpathlineto{\pgfqpoint{4.670931in}{1.146918in}}%
\pgfpathlineto{\pgfqpoint{4.714972in}{1.088000in}}%
\pgfpathlineto{\pgfqpoint{4.768000in}{1.015336in}}%
\pgfpathlineto{\pgfqpoint{4.768000in}{1.015336in}}%
\pgfusepath{fill}%
\end{pgfscope}%
\begin{pgfscope}%
\pgfpathrectangle{\pgfqpoint{0.800000in}{0.528000in}}{\pgfqpoint{3.968000in}{3.696000in}}%
\pgfusepath{clip}%
\pgfsetbuttcap%
\pgfsetroundjoin%
\definecolor{currentfill}{rgb}{0.281887,0.150881,0.465405}%
\pgfsetfillcolor{currentfill}%
\pgfsetlinewidth{0.000000pt}%
\definecolor{currentstroke}{rgb}{0.000000,0.000000,0.000000}%
\pgfsetstrokecolor{currentstroke}%
\pgfsetdash{}{0pt}%
\pgfpathmoveto{\pgfqpoint{2.539116in}{0.528000in}}%
\pgfpathlineto{\pgfqpoint{2.421625in}{0.640000in}}%
\pgfpathlineto{\pgfqpoint{2.158367in}{0.901333in}}%
\pgfpathlineto{\pgfqpoint{2.065597in}{0.997509in}}%
\pgfpathlineto{\pgfqpoint{2.008928in}{1.056725in}}%
\pgfpathlineto{\pgfqpoint{1.962343in}{1.105995in}}%
\pgfpathlineto{\pgfqpoint{1.860073in}{1.216740in}}%
\pgfpathlineto{\pgfqpoint{1.802020in}{1.280069in}}%
\pgfpathlineto{\pgfqpoint{1.713716in}{1.379082in}}%
\pgfpathlineto{\pgfqpoint{1.673924in}{1.424000in}}%
\pgfpathlineto{\pgfqpoint{1.561535in}{1.554052in}}%
\pgfpathlineto{\pgfqpoint{1.329709in}{1.834667in}}%
\pgfpathlineto{\pgfqpoint{1.270530in}{1.909333in}}%
\pgfpathlineto{\pgfqpoint{1.183756in}{2.021333in}}%
\pgfpathlineto{\pgfqpoint{1.141229in}{2.077838in}}%
\pgfpathlineto{\pgfqpoint{1.099371in}{2.133333in}}%
\pgfpathlineto{\pgfqpoint{1.052571in}{2.196742in}}%
\pgfpathlineto{\pgfqpoint{0.960323in}{2.325431in}}%
\pgfpathlineto{\pgfqpoint{0.800000in}{2.561671in}}%
\pgfpathlineto{\pgfqpoint{0.800000in}{2.550566in}}%
\pgfpathlineto{\pgfqpoint{0.854132in}{2.469333in}}%
\pgfpathlineto{\pgfqpoint{0.911153in}{2.386200in}}%
\pgfpathlineto{\pgfqpoint{0.958414in}{2.318221in}}%
\pgfpathlineto{\pgfqpoint{1.000404in}{2.259288in}}%
\pgfpathlineto{\pgfqpoint{1.080566in}{2.149338in}}%
\pgfpathlineto{\pgfqpoint{1.263730in}{1.909333in}}%
\pgfpathlineto{\pgfqpoint{1.322080in}{1.835626in}}%
\pgfpathlineto{\pgfqpoint{1.352889in}{1.797333in}}%
\pgfpathlineto{\pgfqpoint{1.413691in}{1.722667in}}%
\pgfpathlineto{\pgfqpoint{1.513347in}{1.603114in}}%
\pgfpathlineto{\pgfqpoint{1.561535in}{1.546054in}}%
\pgfpathlineto{\pgfqpoint{1.667154in}{1.424000in}}%
\pgfpathlineto{\pgfqpoint{1.766415in}{1.312000in}}%
\pgfpathlineto{\pgfqpoint{1.856345in}{1.213268in}}%
\pgfpathlineto{\pgfqpoint{1.902653in}{1.162667in}}%
\pgfpathlineto{\pgfqpoint{2.007517in}{1.050667in}}%
\pgfpathlineto{\pgfqpoint{2.122667in}{0.930744in}}%
\pgfpathlineto{\pgfqpoint{2.214508in}{0.837546in}}%
\pgfpathlineto{\pgfqpoint{2.262293in}{0.789333in}}%
\pgfpathlineto{\pgfqpoint{2.375847in}{0.677333in}}%
\pgfpathlineto{\pgfqpoint{2.492212in}{0.565333in}}%
\pgfpathlineto{\pgfqpoint{2.531655in}{0.528000in}}%
\pgfpathmoveto{\pgfqpoint{4.768000in}{1.034500in}}%
\pgfpathlineto{\pgfqpoint{4.587954in}{1.274667in}}%
\pgfpathlineto{\pgfqpoint{4.545525in}{1.328776in}}%
\pgfpathlineto{\pgfqpoint{4.500240in}{1.386667in}}%
\pgfpathlineto{\pgfqpoint{4.440485in}{1.461333in}}%
\pgfpathlineto{\pgfqpoint{4.348776in}{1.573333in}}%
\pgfpathlineto{\pgfqpoint{4.246949in}{1.694625in}}%
\pgfpathlineto{\pgfqpoint{4.125999in}{1.834667in}}%
\pgfpathlineto{\pgfqpoint{4.035453in}{1.936334in}}%
\pgfpathlineto{\pgfqpoint{3.992770in}{1.984000in}}%
\pgfpathlineto{\pgfqpoint{3.924736in}{2.058667in}}%
\pgfpathlineto{\pgfqpoint{3.806061in}{2.185880in}}%
\pgfpathlineto{\pgfqpoint{3.700320in}{2.296174in}}%
\pgfpathlineto{\pgfqpoint{3.641218in}{2.357333in}}%
\pgfpathlineto{\pgfqpoint{3.525495in}{2.474055in}}%
\pgfpathlineto{\pgfqpoint{3.405253in}{2.592259in}}%
\pgfpathlineto{\pgfqpoint{3.325091in}{2.669416in}}%
\pgfpathlineto{\pgfqpoint{3.204848in}{2.782715in}}%
\pgfpathlineto{\pgfqpoint{3.084606in}{2.893133in}}%
\pgfpathlineto{\pgfqpoint{2.989505in}{2.978084in}}%
\pgfpathlineto{\pgfqpoint{2.924283in}{3.035951in}}%
\pgfpathlineto{\pgfqpoint{2.802203in}{3.141333in}}%
\pgfpathlineto{\pgfqpoint{2.668370in}{3.253333in}}%
\pgfpathlineto{\pgfqpoint{2.403232in}{3.464923in}}%
\pgfpathlineto{\pgfqpoint{2.282990in}{3.556186in}}%
\pgfpathlineto{\pgfqpoint{2.187045in}{3.626667in}}%
\pgfpathlineto{\pgfqpoint{2.082472in}{3.701439in}}%
\pgfpathlineto{\pgfqpoint{1.962343in}{3.784021in}}%
\pgfpathlineto{\pgfqpoint{1.861563in}{3.850667in}}%
\pgfpathlineto{\pgfqpoint{1.679375in}{3.964905in}}%
\pgfpathlineto{\pgfqpoint{1.555306in}{4.037333in}}%
\pgfpathlineto{\pgfqpoint{1.379837in}{4.131910in}}%
\pgfpathlineto{\pgfqpoint{1.280970in}{4.180535in}}%
\pgfpathlineto{\pgfqpoint{1.222621in}{4.206984in}}%
\pgfpathlineto{\pgfqpoint{1.185243in}{4.224000in}}%
\pgfpathlineto{\pgfqpoint{1.164543in}{4.224000in}}%
\pgfpathlineto{\pgfqpoint{1.249085in}{4.186667in}}%
\pgfpathlineto{\pgfqpoint{1.402455in}{4.112000in}}%
\pgfpathlineto{\pgfqpoint{1.492658in}{4.064156in}}%
\pgfpathlineto{\pgfqpoint{1.606646in}{4.000000in}}%
\pgfpathlineto{\pgfqpoint{1.802020in}{3.881410in}}%
\pgfpathlineto{\pgfqpoint{1.882182in}{3.829777in}}%
\pgfpathlineto{\pgfqpoint{2.002424in}{3.749457in}}%
\pgfpathlineto{\pgfqpoint{2.125176in}{3.664000in}}%
\pgfpathlineto{\pgfqpoint{2.242909in}{3.578724in}}%
\pgfpathlineto{\pgfqpoint{2.363152in}{3.488605in}}%
\pgfpathlineto{\pgfqpoint{2.483394in}{3.395489in}}%
\pgfpathlineto{\pgfqpoint{2.534628in}{3.354945in}}%
\pgfpathlineto{\pgfqpoint{2.659933in}{3.253333in}}%
\pgfpathlineto{\pgfqpoint{2.928658in}{3.025258in}}%
\pgfpathlineto{\pgfqpoint{3.050175in}{2.917333in}}%
\pgfpathlineto{\pgfqpoint{3.148493in}{2.827508in}}%
\pgfpathlineto{\pgfqpoint{3.208974in}{2.771843in}}%
\pgfpathlineto{\pgfqpoint{3.253073in}{2.730667in}}%
\pgfpathlineto{\pgfqpoint{3.348050in}{2.640052in}}%
\pgfpathlineto{\pgfqpoint{3.409297in}{2.581333in}}%
\pgfpathlineto{\pgfqpoint{3.525495in}{2.466967in}}%
\pgfpathlineto{\pgfqpoint{3.645737in}{2.345466in}}%
\pgfpathlineto{\pgfqpoint{3.778160in}{2.208000in}}%
\pgfpathlineto{\pgfqpoint{3.848596in}{2.133333in}}%
\pgfpathlineto{\pgfqpoint{3.966384in}{2.005529in}}%
\pgfpathlineto{\pgfqpoint{4.068056in}{1.892036in}}%
\pgfpathlineto{\pgfqpoint{4.126707in}{1.826001in}}%
\pgfpathlineto{\pgfqpoint{4.216196in}{1.722667in}}%
\pgfpathlineto{\pgfqpoint{4.310919in}{1.610667in}}%
\pgfpathlineto{\pgfqpoint{4.407273in}{1.493938in}}%
\pgfpathlineto{\pgfqpoint{4.527515in}{1.343565in}}%
\pgfpathlineto{\pgfqpoint{4.727919in}{1.079925in}}%
\pgfpathlineto{\pgfqpoint{4.768000in}{1.024918in}}%
\pgfpathlineto{\pgfqpoint{4.768000in}{1.024918in}}%
\pgfusepath{fill}%
\end{pgfscope}%
\begin{pgfscope}%
\pgfpathrectangle{\pgfqpoint{0.800000in}{0.528000in}}{\pgfqpoint{3.968000in}{3.696000in}}%
\pgfusepath{clip}%
\pgfsetbuttcap%
\pgfsetroundjoin%
\definecolor{currentfill}{rgb}{0.281412,0.155834,0.469201}%
\pgfsetfillcolor{currentfill}%
\pgfsetlinewidth{0.000000pt}%
\definecolor{currentstroke}{rgb}{0.000000,0.000000,0.000000}%
\pgfsetstrokecolor{currentstroke}%
\pgfsetdash{}{0pt}%
\pgfpathmoveto{\pgfqpoint{2.531655in}{0.528000in}}%
\pgfpathlineto{\pgfqpoint{2.443313in}{0.612040in}}%
\pgfpathlineto{\pgfqpoint{2.323071in}{0.729030in}}%
\pgfpathlineto{\pgfqpoint{2.188070in}{0.864000in}}%
\pgfpathlineto{\pgfqpoint{2.078845in}{0.976000in}}%
\pgfpathlineto{\pgfqpoint{2.002424in}{1.056037in}}%
\pgfpathlineto{\pgfqpoint{1.868201in}{1.200000in}}%
\pgfpathlineto{\pgfqpoint{1.782878in}{1.294170in}}%
\pgfpathlineto{\pgfqpoint{1.733099in}{1.349333in}}%
\pgfpathlineto{\pgfqpoint{1.634516in}{1.461333in}}%
\pgfpathlineto{\pgfqpoint{1.538362in}{1.573333in}}%
\pgfpathlineto{\pgfqpoint{1.475450in}{1.648000in}}%
\pgfpathlineto{\pgfqpoint{1.383194in}{1.760000in}}%
\pgfpathlineto{\pgfqpoint{1.328015in}{1.828179in}}%
\pgfpathlineto{\pgfqpoint{1.234456in}{1.946667in}}%
\pgfpathlineto{\pgfqpoint{1.148458in}{2.058667in}}%
\pgfpathlineto{\pgfqpoint{1.087681in}{2.139961in}}%
\pgfpathlineto{\pgfqpoint{1.037422in}{2.208000in}}%
\pgfpathlineto{\pgfqpoint{0.983700in}{2.282667in}}%
\pgfpathlineto{\pgfqpoint{0.905086in}{2.394667in}}%
\pgfpathlineto{\pgfqpoint{0.829087in}{2.506667in}}%
\pgfpathlineto{\pgfqpoint{0.800000in}{2.550566in}}%
\pgfpathlineto{\pgfqpoint{0.800000in}{2.539644in}}%
\pgfpathlineto{\pgfqpoint{0.976738in}{2.282667in}}%
\pgfpathlineto{\pgfqpoint{1.034665in}{2.202579in}}%
\pgfpathlineto{\pgfqpoint{1.085498in}{2.133333in}}%
\pgfpathlineto{\pgfqpoint{1.170021in}{2.021333in}}%
\pgfpathlineto{\pgfqpoint{1.256930in}{1.909333in}}%
\pgfpathlineto{\pgfqpoint{1.316081in}{1.834667in}}%
\pgfpathlineto{\pgfqpoint{1.406873in}{1.722667in}}%
\pgfpathlineto{\pgfqpoint{1.500052in}{1.610667in}}%
\pgfpathlineto{\pgfqpoint{1.580563in}{1.516390in}}%
\pgfpathlineto{\pgfqpoint{1.627789in}{1.461333in}}%
\pgfpathlineto{\pgfqpoint{1.706336in}{1.372208in}}%
\pgfpathlineto{\pgfqpoint{1.761939in}{1.309375in}}%
\pgfpathlineto{\pgfqpoint{1.852617in}{1.209795in}}%
\pgfpathlineto{\pgfqpoint{1.895779in}{1.162667in}}%
\pgfpathlineto{\pgfqpoint{2.002424in}{1.048703in}}%
\pgfpathlineto{\pgfqpoint{2.122667in}{0.923594in}}%
\pgfpathlineto{\pgfqpoint{2.210789in}{0.834082in}}%
\pgfpathlineto{\pgfqpoint{2.255174in}{0.789333in}}%
\pgfpathlineto{\pgfqpoint{2.368586in}{0.677333in}}%
\pgfpathlineto{\pgfqpoint{2.484802in}{0.565333in}}%
\pgfpathlineto{\pgfqpoint{2.524194in}{0.528000in}}%
\pgfpathmoveto{\pgfqpoint{4.768000in}{1.044083in}}%
\pgfpathlineto{\pgfqpoint{4.607677in}{1.257970in}}%
\pgfpathlineto{\pgfqpoint{4.407273in}{1.510515in}}%
\pgfpathlineto{\pgfqpoint{4.165385in}{1.797333in}}%
\pgfpathlineto{\pgfqpoint{4.075562in}{1.899027in}}%
\pgfpathlineto{\pgfqpoint{4.033251in}{1.946667in}}%
\pgfpathlineto{\pgfqpoint{3.962347in}{2.025093in}}%
\pgfpathlineto{\pgfqpoint{3.827277in}{2.170667in}}%
\pgfpathlineto{\pgfqpoint{3.720716in}{2.282667in}}%
\pgfpathlineto{\pgfqpoint{3.645737in}{2.359907in}}%
\pgfpathlineto{\pgfqpoint{3.525495in}{2.481080in}}%
\pgfpathlineto{\pgfqpoint{3.385173in}{2.618667in}}%
\pgfpathlineto{\pgfqpoint{3.267818in}{2.730667in}}%
\pgfpathlineto{\pgfqpoint{3.147451in}{2.842667in}}%
\pgfpathlineto{\pgfqpoint{2.884202in}{3.077712in}}%
\pgfpathlineto{\pgfqpoint{2.763960in}{3.180625in}}%
\pgfpathlineto{\pgfqpoint{2.659037in}{3.267603in}}%
\pgfpathlineto{\pgfqpoint{2.603636in}{3.313237in}}%
\pgfpathlineto{\pgfqpoint{2.483394in}{3.409462in}}%
\pgfpathlineto{\pgfqpoint{2.403232in}{3.471975in}}%
\pgfpathlineto{\pgfqpoint{2.282990in}{3.563201in}}%
\pgfpathlineto{\pgfqpoint{2.162747in}{3.651408in}}%
\pgfpathlineto{\pgfqpoint{2.039420in}{3.738667in}}%
\pgfpathlineto{\pgfqpoint{1.903822in}{3.830510in}}%
\pgfpathlineto{\pgfqpoint{1.802020in}{3.896497in}}%
\pgfpathlineto{\pgfqpoint{1.569183in}{4.037333in}}%
\pgfpathlineto{\pgfqpoint{1.502685in}{4.074667in}}%
\pgfpathlineto{\pgfqpoint{1.360055in}{4.150336in}}%
\pgfpathlineto{\pgfqpoint{1.280970in}{4.189165in}}%
\pgfpathlineto{\pgfqpoint{1.205532in}{4.224000in}}%
\pgfpathlineto{\pgfqpoint{1.185243in}{4.224000in}}%
\pgfpathlineto{\pgfqpoint{1.240889in}{4.199167in}}%
\pgfpathlineto{\pgfqpoint{1.302734in}{4.169606in}}%
\pgfpathlineto{\pgfqpoint{1.345005in}{4.149333in}}%
\pgfpathlineto{\pgfqpoint{1.481374in}{4.078404in}}%
\pgfpathlineto{\pgfqpoint{1.533939in}{4.048962in}}%
\pgfpathlineto{\pgfqpoint{1.583711in}{4.020655in}}%
\pgfpathlineto{\pgfqpoint{1.620003in}{4.000000in}}%
\pgfpathlineto{\pgfqpoint{1.803652in}{3.888000in}}%
\pgfpathlineto{\pgfqpoint{1.931392in}{3.804830in}}%
\pgfpathlineto{\pgfqpoint{2.042505in}{3.729217in}}%
\pgfpathlineto{\pgfqpoint{2.135177in}{3.664000in}}%
\pgfpathlineto{\pgfqpoint{2.260672in}{3.572788in}}%
\pgfpathlineto{\pgfqpoint{2.363152in}{3.495645in}}%
\pgfpathlineto{\pgfqpoint{2.483394in}{3.402567in}}%
\pgfpathlineto{\pgfqpoint{2.576808in}{3.328000in}}%
\pgfpathlineto{\pgfqpoint{2.683798in}{3.240618in}}%
\pgfpathlineto{\pgfqpoint{2.758045in}{3.178667in}}%
\pgfpathlineto{\pgfqpoint{2.845861in}{3.104000in}}%
\pgfpathlineto{\pgfqpoint{2.974201in}{2.992000in}}%
\pgfpathlineto{\pgfqpoint{3.071249in}{2.904892in}}%
\pgfpathlineto{\pgfqpoint{3.124687in}{2.856645in}}%
\pgfpathlineto{\pgfqpoint{3.212645in}{2.775262in}}%
\pgfpathlineto{\pgfqpoint{3.260445in}{2.730667in}}%
\pgfpathlineto{\pgfqpoint{3.351786in}{2.643531in}}%
\pgfpathlineto{\pgfqpoint{3.405253in}{2.592259in}}%
\pgfpathlineto{\pgfqpoint{3.492608in}{2.506667in}}%
\pgfpathlineto{\pgfqpoint{3.605657in}{2.393555in}}%
\pgfpathlineto{\pgfqpoint{3.725899in}{2.270006in}}%
\pgfpathlineto{\pgfqpoint{3.832247in}{2.157725in}}%
\pgfpathlineto{\pgfqpoint{3.890233in}{2.096000in}}%
\pgfpathlineto{\pgfqpoint{4.006465in}{1.968866in}}%
\pgfpathlineto{\pgfqpoint{4.108011in}{1.854586in}}%
\pgfpathlineto{\pgfqpoint{4.166788in}{1.787847in}}%
\pgfpathlineto{\pgfqpoint{4.254850in}{1.685333in}}%
\pgfpathlineto{\pgfqpoint{4.348776in}{1.573333in}}%
\pgfpathlineto{\pgfqpoint{4.408936in}{1.500216in}}%
\pgfpathlineto{\pgfqpoint{4.447354in}{1.452819in}}%
\pgfpathlineto{\pgfqpoint{4.559021in}{1.312000in}}%
\pgfpathlineto{\pgfqpoint{4.647758in}{1.196657in}}%
\pgfpathlineto{\pgfqpoint{4.756235in}{1.050667in}}%
\pgfpathlineto{\pgfqpoint{4.768000in}{1.034500in}}%
\pgfpathlineto{\pgfqpoint{4.768000in}{1.034500in}}%
\pgfusepath{fill}%
\end{pgfscope}%
\begin{pgfscope}%
\pgfpathrectangle{\pgfqpoint{0.800000in}{0.528000in}}{\pgfqpoint{3.968000in}{3.696000in}}%
\pgfusepath{clip}%
\pgfsetbuttcap%
\pgfsetroundjoin%
\definecolor{currentfill}{rgb}{0.281412,0.155834,0.469201}%
\pgfsetfillcolor{currentfill}%
\pgfsetlinewidth{0.000000pt}%
\definecolor{currentstroke}{rgb}{0.000000,0.000000,0.000000}%
\pgfsetstrokecolor{currentstroke}%
\pgfsetdash{}{0pt}%
\pgfpathmoveto{\pgfqpoint{2.524194in}{0.528000in}}%
\pgfpathlineto{\pgfqpoint{2.443313in}{0.604992in}}%
\pgfpathlineto{\pgfqpoint{2.323071in}{0.721944in}}%
\pgfpathlineto{\pgfqpoint{2.202828in}{0.841914in}}%
\pgfpathlineto{\pgfqpoint{2.071952in}{0.976000in}}%
\pgfpathlineto{\pgfqpoint{2.000565in}{1.050667in}}%
\pgfpathlineto{\pgfqpoint{1.882182in}{1.177363in}}%
\pgfpathlineto{\pgfqpoint{1.747732in}{1.325234in}}%
\pgfpathlineto{\pgfqpoint{1.627789in}{1.461333in}}%
\pgfpathlineto{\pgfqpoint{1.521455in}{1.585263in}}%
\pgfpathlineto{\pgfqpoint{1.280970in}{1.878785in}}%
\pgfpathlineto{\pgfqpoint{1.080566in}{2.139984in}}%
\pgfpathlineto{\pgfqpoint{0.898054in}{2.394667in}}%
\pgfpathlineto{\pgfqpoint{0.821982in}{2.506667in}}%
\pgfpathlineto{\pgfqpoint{0.800000in}{2.539644in}}%
\pgfpathlineto{\pgfqpoint{0.800000in}{2.528986in}}%
\pgfpathlineto{\pgfqpoint{0.960323in}{2.295930in}}%
\pgfpathlineto{\pgfqpoint{1.040485in}{2.185057in}}%
\pgfpathlineto{\pgfqpoint{1.120646in}{2.077373in}}%
\pgfpathlineto{\pgfqpoint{1.321051in}{1.820115in}}%
\pgfpathlineto{\pgfqpoint{1.430960in}{1.685333in}}%
\pgfpathlineto{\pgfqpoint{1.505873in}{1.596154in}}%
\pgfpathlineto{\pgfqpoint{1.561535in}{1.530233in}}%
\pgfpathlineto{\pgfqpoint{1.653615in}{1.424000in}}%
\pgfpathlineto{\pgfqpoint{1.761939in}{1.301904in}}%
\pgfpathlineto{\pgfqpoint{1.854539in}{1.200000in}}%
\pgfpathlineto{\pgfqpoint{1.962343in}{1.083902in}}%
\pgfpathlineto{\pgfqpoint{2.101106in}{0.938667in}}%
\pgfpathlineto{\pgfqpoint{2.210893in}{0.826667in}}%
\pgfpathlineto{\pgfqpoint{2.326949in}{0.711055in}}%
\pgfpathlineto{\pgfqpoint{2.443313in}{0.598066in}}%
\pgfpathlineto{\pgfqpoint{2.516938in}{0.528000in}}%
\pgfpathlineto{\pgfqpoint{2.523475in}{0.528000in}}%
\pgfpathmoveto{\pgfqpoint{4.768000in}{1.053560in}}%
\pgfpathlineto{\pgfqpoint{4.658801in}{1.200000in}}%
\pgfpathlineto{\pgfqpoint{4.587119in}{1.292852in}}%
\pgfpathlineto{\pgfqpoint{4.543294in}{1.349333in}}%
\pgfpathlineto{\pgfqpoint{4.484078in}{1.424000in}}%
\pgfpathlineto{\pgfqpoint{4.393094in}{1.536000in}}%
\pgfpathlineto{\pgfqpoint{4.287030in}{1.663250in}}%
\pgfpathlineto{\pgfqpoint{4.166788in}{1.803413in}}%
\pgfpathlineto{\pgfqpoint{4.073414in}{1.909333in}}%
\pgfpathlineto{\pgfqpoint{3.966384in}{2.028057in}}%
\pgfpathlineto{\pgfqpoint{3.834163in}{2.170667in}}%
\pgfpathlineto{\pgfqpoint{3.725899in}{2.284520in}}%
\pgfpathlineto{\pgfqpoint{3.645737in}{2.366970in}}%
\pgfpathlineto{\pgfqpoint{3.506776in}{2.506667in}}%
\pgfpathlineto{\pgfqpoint{3.392399in}{2.618667in}}%
\pgfpathlineto{\pgfqpoint{3.275191in}{2.730667in}}%
\pgfpathlineto{\pgfqpoint{3.154977in}{2.842667in}}%
\pgfpathlineto{\pgfqpoint{3.073071in}{2.917333in}}%
\pgfpathlineto{\pgfqpoint{2.947413in}{3.029333in}}%
\pgfpathlineto{\pgfqpoint{2.683798in}{3.254493in}}%
\pgfpathlineto{\pgfqpoint{2.593947in}{3.328000in}}%
\pgfpathlineto{\pgfqpoint{2.483394in}{3.416355in}}%
\pgfpathlineto{\pgfqpoint{2.425812in}{3.461032in}}%
\pgfpathlineto{\pgfqpoint{2.363152in}{3.509725in}}%
\pgfpathlineto{\pgfqpoint{2.242909in}{3.600014in}}%
\pgfpathlineto{\pgfqpoint{2.188805in}{3.639729in}}%
\pgfpathlineto{\pgfqpoint{2.082586in}{3.715643in}}%
\pgfpathlineto{\pgfqpoint{1.962343in}{3.798621in}}%
\pgfpathlineto{\pgfqpoint{1.842101in}{3.878232in}}%
\pgfpathlineto{\pgfqpoint{1.721859in}{3.954243in}}%
\pgfpathlineto{\pgfqpoint{1.633209in}{4.007906in}}%
\pgfpathlineto{\pgfqpoint{1.517216in}{4.074667in}}%
\pgfpathlineto{\pgfqpoint{1.340691in}{4.168372in}}%
\pgfpathlineto{\pgfqpoint{1.240889in}{4.216552in}}%
\pgfpathlineto{\pgfqpoint{1.224573in}{4.224000in}}%
\pgfpathlineto{\pgfqpoint{1.205532in}{4.224000in}}%
\pgfpathlineto{\pgfqpoint{1.362017in}{4.149333in}}%
\pgfpathlineto{\pgfqpoint{1.451078in}{4.102886in}}%
\pgfpathlineto{\pgfqpoint{1.561535in}{4.041746in}}%
\pgfpathlineto{\pgfqpoint{1.681778in}{3.971079in}}%
\pgfpathlineto{\pgfqpoint{1.735368in}{3.937917in}}%
\pgfpathlineto{\pgfqpoint{1.783311in}{3.907906in}}%
\pgfpathlineto{\pgfqpoint{1.842101in}{3.870767in}}%
\pgfpathlineto{\pgfqpoint{1.903822in}{3.830510in}}%
\pgfpathlineto{\pgfqpoint{2.002424in}{3.764083in}}%
\pgfpathlineto{\pgfqpoint{2.092717in}{3.701333in}}%
\pgfpathlineto{\pgfqpoint{2.333829in}{3.524688in}}%
\pgfpathlineto{\pgfqpoint{2.403232in}{3.471975in}}%
\pgfpathlineto{\pgfqpoint{2.508797in}{3.388995in}}%
\pgfpathlineto{\pgfqpoint{2.563556in}{3.345614in}}%
\pgfpathlineto{\pgfqpoint{2.683798in}{3.247571in}}%
\pgfpathlineto{\pgfqpoint{2.743743in}{3.197169in}}%
\pgfpathlineto{\pgfqpoint{2.810232in}{3.141333in}}%
\pgfpathlineto{\pgfqpoint{2.939617in}{3.029333in}}%
\pgfpathlineto{\pgfqpoint{3.065439in}{2.917333in}}%
\pgfpathlineto{\pgfqpoint{3.346378in}{2.656000in}}%
\pgfpathlineto{\pgfqpoint{3.611539in}{2.394667in}}%
\pgfpathlineto{\pgfqpoint{3.725899in}{2.277289in}}%
\pgfpathlineto{\pgfqpoint{3.806061in}{2.193189in}}%
\pgfpathlineto{\pgfqpoint{3.931537in}{2.058667in}}%
\pgfpathlineto{\pgfqpoint{4.046545in}{1.931855in}}%
\pgfpathlineto{\pgfqpoint{4.166788in}{1.795721in}}%
\pgfpathlineto{\pgfqpoint{4.255009in}{1.692841in}}%
\pgfpathlineto{\pgfqpoint{4.293211in}{1.648000in}}%
\pgfpathlineto{\pgfqpoint{4.386349in}{1.536000in}}%
\pgfpathlineto{\pgfqpoint{4.416940in}{1.498667in}}%
\pgfpathlineto{\pgfqpoint{4.507009in}{1.386667in}}%
\pgfpathlineto{\pgfqpoint{4.600080in}{1.267591in}}%
\pgfpathlineto{\pgfqpoint{4.652051in}{1.200000in}}%
\pgfpathlineto{\pgfqpoint{4.735794in}{1.088000in}}%
\pgfpathlineto{\pgfqpoint{4.768000in}{1.044083in}}%
\pgfpathlineto{\pgfqpoint{4.768000in}{1.050667in}}%
\pgfpathlineto{\pgfqpoint{4.768000in}{1.050667in}}%
\pgfusepath{fill}%
\end{pgfscope}%
\begin{pgfscope}%
\pgfpathrectangle{\pgfqpoint{0.800000in}{0.528000in}}{\pgfqpoint{3.968000in}{3.696000in}}%
\pgfusepath{clip}%
\pgfsetbuttcap%
\pgfsetroundjoin%
\definecolor{currentfill}{rgb}{0.281412,0.155834,0.469201}%
\pgfsetfillcolor{currentfill}%
\pgfsetlinewidth{0.000000pt}%
\definecolor{currentstroke}{rgb}{0.000000,0.000000,0.000000}%
\pgfsetstrokecolor{currentstroke}%
\pgfsetdash{}{0pt}%
\pgfpathmoveto{\pgfqpoint{2.516938in}{0.528000in}}%
\pgfpathlineto{\pgfqpoint{2.438529in}{0.602667in}}%
\pgfpathlineto{\pgfqpoint{2.323071in}{0.714858in}}%
\pgfpathlineto{\pgfqpoint{2.202828in}{0.834790in}}%
\pgfpathlineto{\pgfqpoint{2.065059in}{0.976000in}}%
\pgfpathlineto{\pgfqpoint{1.993758in}{1.050667in}}%
\pgfpathlineto{\pgfqpoint{1.882182in}{1.169933in}}%
\pgfpathlineto{\pgfqpoint{1.752880in}{1.312000in}}%
\pgfpathlineto{\pgfqpoint{1.641697in}{1.437613in}}%
\pgfpathlineto{\pgfqpoint{1.396440in}{1.727112in}}%
\pgfpathlineto{\pgfqpoint{1.279541in}{1.872000in}}%
\pgfpathlineto{\pgfqpoint{1.195712in}{1.979254in}}%
\pgfpathlineto{\pgfqpoint{1.160727in}{2.024511in}}%
\pgfpathlineto{\pgfqpoint{1.051064in}{2.170667in}}%
\pgfpathlineto{\pgfqpoint{0.960323in}{2.295930in}}%
\pgfpathlineto{\pgfqpoint{0.800000in}{2.528986in}}%
\pgfpathlineto{\pgfqpoint{0.800000in}{2.518328in}}%
\pgfpathlineto{\pgfqpoint{0.936359in}{2.320000in}}%
\pgfpathlineto{\pgfqpoint{0.994116in}{2.239477in}}%
\pgfpathlineto{\pgfqpoint{1.044172in}{2.170667in}}%
\pgfpathlineto{\pgfqpoint{1.108578in}{2.084758in}}%
\pgfpathlineto{\pgfqpoint{1.160727in}{2.015709in}}%
\pgfpathlineto{\pgfqpoint{1.272889in}{1.872000in}}%
\pgfpathlineto{\pgfqpoint{1.368330in}{1.753295in}}%
\pgfpathlineto{\pgfqpoint{1.486549in}{1.610667in}}%
\pgfpathlineto{\pgfqpoint{1.573051in}{1.509393in}}%
\pgfpathlineto{\pgfqpoint{1.614333in}{1.461333in}}%
\pgfpathlineto{\pgfqpoint{1.698955in}{1.365333in}}%
\pgfpathlineto{\pgfqpoint{1.746176in}{1.312000in}}%
\pgfpathlineto{\pgfqpoint{1.847709in}{1.200000in}}%
\pgfpathlineto{\pgfqpoint{1.962343in}{1.076702in}}%
\pgfpathlineto{\pgfqpoint{2.094169in}{0.938667in}}%
\pgfpathlineto{\pgfqpoint{2.216298in}{0.814120in}}%
\pgfpathlineto{\pgfqpoint{2.323071in}{0.707951in}}%
\pgfpathlineto{\pgfqpoint{2.443313in}{0.591200in}}%
\pgfpathlineto{\pgfqpoint{2.509704in}{0.528000in}}%
\pgfpathmoveto{\pgfqpoint{4.768000in}{1.062807in}}%
\pgfpathlineto{\pgfqpoint{4.665552in}{1.200000in}}%
\pgfpathlineto{\pgfqpoint{4.637091in}{1.237333in}}%
\pgfpathlineto{\pgfqpoint{4.550021in}{1.349333in}}%
\pgfpathlineto{\pgfqpoint{4.490793in}{1.424000in}}%
\pgfpathlineto{\pgfqpoint{4.399838in}{1.536000in}}%
\pgfpathlineto{\pgfqpoint{4.306570in}{1.648000in}}%
\pgfpathlineto{\pgfqpoint{4.226849in}{1.741277in}}%
\pgfpathlineto{\pgfqpoint{4.178737in}{1.797333in}}%
\pgfpathlineto{\pgfqpoint{4.080195in}{1.909333in}}%
\pgfpathlineto{\pgfqpoint{3.966384in}{2.035420in}}%
\pgfpathlineto{\pgfqpoint{3.841048in}{2.170667in}}%
\pgfpathlineto{\pgfqpoint{3.725899in}{2.291607in}}%
\pgfpathlineto{\pgfqpoint{3.645737in}{2.374032in}}%
\pgfpathlineto{\pgfqpoint{3.513861in}{2.506667in}}%
\pgfpathlineto{\pgfqpoint{3.399624in}{2.618667in}}%
\pgfpathlineto{\pgfqpoint{3.282564in}{2.730667in}}%
\pgfpathlineto{\pgfqpoint{3.162504in}{2.842667in}}%
\pgfpathlineto{\pgfqpoint{3.080703in}{2.917333in}}%
\pgfpathlineto{\pgfqpoint{2.955209in}{3.029333in}}%
\pgfpathlineto{\pgfqpoint{2.683798in}{3.261268in}}%
\pgfpathlineto{\pgfqpoint{2.624227in}{3.309846in}}%
\pgfpathlineto{\pgfqpoint{2.563556in}{3.359448in}}%
\pgfpathlineto{\pgfqpoint{2.443313in}{3.454699in}}%
\pgfpathlineto{\pgfqpoint{2.403232in}{3.485853in}}%
\pgfpathlineto{\pgfqpoint{2.282990in}{3.577232in}}%
\pgfpathlineto{\pgfqpoint{2.154693in}{3.671502in}}%
\pgfpathlineto{\pgfqpoint{2.042505in}{3.750859in}}%
\pgfpathlineto{\pgfqpoint{1.951318in}{3.813333in}}%
\pgfpathlineto{\pgfqpoint{1.745069in}{3.946953in}}%
\pgfpathlineto{\pgfqpoint{1.696882in}{3.976736in}}%
\pgfpathlineto{\pgfqpoint{1.641697in}{4.010467in}}%
\pgfpathlineto{\pgfqpoint{1.561535in}{4.057353in}}%
\pgfpathlineto{\pgfqpoint{1.499772in}{4.091804in}}%
\pgfpathlineto{\pgfqpoint{1.463771in}{4.112000in}}%
\pgfpathlineto{\pgfqpoint{1.393877in}{4.149333in}}%
\pgfpathlineto{\pgfqpoint{1.320803in}{4.186667in}}%
\pgfpathlineto{\pgfqpoint{1.243413in}{4.224000in}}%
\pgfpathlineto{\pgfqpoint{1.224573in}{4.224000in}}%
\pgfpathlineto{\pgfqpoint{1.340691in}{4.168372in}}%
\pgfpathlineto{\pgfqpoint{1.441293in}{4.116206in}}%
\pgfpathlineto{\pgfqpoint{1.521455in}{4.072326in}}%
\pgfpathlineto{\pgfqpoint{1.721859in}{3.954243in}}%
\pgfpathlineto{\pgfqpoint{1.826894in}{3.888000in}}%
\pgfpathlineto{\pgfqpoint{1.922263in}{3.825545in}}%
\pgfpathlineto{\pgfqpoint{1.976183in}{3.788891in}}%
\pgfpathlineto{\pgfqpoint{2.042505in}{3.743730in}}%
\pgfpathlineto{\pgfqpoint{2.113961in}{3.693225in}}%
\pgfpathlineto{\pgfqpoint{2.162747in}{3.658576in}}%
\pgfpathlineto{\pgfqpoint{2.257292in}{3.589333in}}%
\pgfpathlineto{\pgfqpoint{2.363152in}{3.509725in}}%
\pgfpathlineto{\pgfqpoint{2.483394in}{3.416355in}}%
\pgfpathlineto{\pgfqpoint{2.603636in}{3.320166in}}%
\pgfpathlineto{\pgfqpoint{2.662943in}{3.271241in}}%
\pgfpathlineto{\pgfqpoint{2.729966in}{3.216000in}}%
\pgfpathlineto{\pgfqpoint{2.818200in}{3.141333in}}%
\pgfpathlineto{\pgfqpoint{2.947413in}{3.029333in}}%
\pgfpathlineto{\pgfqpoint{3.044525in}{2.943066in}}%
\pgfpathlineto{\pgfqpoint{3.164768in}{2.833671in}}%
\pgfpathlineto{\pgfqpoint{3.285010in}{2.721409in}}%
\pgfpathlineto{\pgfqpoint{3.405253in}{2.606235in}}%
\pgfpathlineto{\pgfqpoint{3.485414in}{2.527813in}}%
\pgfpathlineto{\pgfqpoint{3.618487in}{2.394667in}}%
\pgfpathlineto{\pgfqpoint{3.727682in}{2.282667in}}%
\pgfpathlineto{\pgfqpoint{3.806061in}{2.200498in}}%
\pgfpathlineto{\pgfqpoint{3.938294in}{2.058667in}}%
\pgfpathlineto{\pgfqpoint{4.046545in}{1.939457in}}%
\pgfpathlineto{\pgfqpoint{4.172081in}{1.797333in}}%
\pgfpathlineto{\pgfqpoint{4.258714in}{1.696292in}}%
\pgfpathlineto{\pgfqpoint{4.299891in}{1.648000in}}%
\pgfpathlineto{\pgfqpoint{4.407273in}{1.518730in}}%
\pgfpathlineto{\pgfqpoint{4.607677in}{1.266826in}}%
\pgfpathlineto{\pgfqpoint{4.714945in}{1.125333in}}%
\pgfpathlineto{\pgfqpoint{4.742611in}{1.088000in}}%
\pgfpathlineto{\pgfqpoint{4.768000in}{1.053560in}}%
\pgfpathlineto{\pgfqpoint{4.768000in}{1.053560in}}%
\pgfusepath{fill}%
\end{pgfscope}%
\begin{pgfscope}%
\pgfpathrectangle{\pgfqpoint{0.800000in}{0.528000in}}{\pgfqpoint{3.968000in}{3.696000in}}%
\pgfusepath{clip}%
\pgfsetbuttcap%
\pgfsetroundjoin%
\definecolor{currentfill}{rgb}{0.281412,0.155834,0.469201}%
\pgfsetfillcolor{currentfill}%
\pgfsetlinewidth{0.000000pt}%
\definecolor{currentstroke}{rgb}{0.000000,0.000000,0.000000}%
\pgfsetstrokecolor{currentstroke}%
\pgfsetdash{}{0pt}%
\pgfpathmoveto{\pgfqpoint{2.509704in}{0.528000in}}%
\pgfpathlineto{\pgfqpoint{2.431390in}{0.602667in}}%
\pgfpathlineto{\pgfqpoint{2.316259in}{0.714667in}}%
\pgfpathlineto{\pgfqpoint{2.202828in}{0.827666in}}%
\pgfpathlineto{\pgfqpoint{2.082586in}{0.950626in}}%
\pgfpathlineto{\pgfqpoint{1.951730in}{1.088000in}}%
\pgfpathlineto{\pgfqpoint{1.881137in}{1.163640in}}%
\pgfpathlineto{\pgfqpoint{1.746176in}{1.312000in}}%
\pgfpathlineto{\pgfqpoint{1.641697in}{1.429880in}}%
\pgfpathlineto{\pgfqpoint{1.537519in}{1.550963in}}%
\pgfpathlineto{\pgfqpoint{1.481374in}{1.616821in}}%
\pgfpathlineto{\pgfqpoint{1.393459in}{1.722667in}}%
\pgfpathlineto{\pgfqpoint{1.302697in}{1.834667in}}%
\pgfpathlineto{\pgfqpoint{1.259527in}{1.889360in}}%
\pgfpathlineto{\pgfqpoint{1.214183in}{1.946667in}}%
\pgfpathlineto{\pgfqpoint{1.120646in}{2.068360in}}%
\pgfpathlineto{\pgfqpoint{0.936359in}{2.320000in}}%
\pgfpathlineto{\pgfqpoint{0.898500in}{2.374415in}}%
\pgfpathlineto{\pgfqpoint{0.858374in}{2.432000in}}%
\pgfpathlineto{\pgfqpoint{0.820452in}{2.488383in}}%
\pgfpathlineto{\pgfqpoint{0.800000in}{2.518328in}}%
\pgfpathlineto{\pgfqpoint{0.800669in}{2.506667in}}%
\pgfpathlineto{\pgfqpoint{0.982928in}{2.245333in}}%
\pgfpathlineto{\pgfqpoint{1.022600in}{2.191341in}}%
\pgfpathlineto{\pgfqpoint{1.065139in}{2.133333in}}%
\pgfpathlineto{\pgfqpoint{1.104642in}{2.081092in}}%
\pgfpathlineto{\pgfqpoint{1.149737in}{2.021333in}}%
\pgfpathlineto{\pgfqpoint{1.221757in}{1.928846in}}%
\pgfpathlineto{\pgfqpoint{1.266238in}{1.872000in}}%
\pgfpathlineto{\pgfqpoint{1.361131in}{1.753990in}}%
\pgfpathlineto{\pgfqpoint{1.481374in}{1.608846in}}%
\pgfpathlineto{\pgfqpoint{1.569295in}{1.505895in}}%
\pgfpathlineto{\pgfqpoint{1.607605in}{1.461333in}}%
\pgfpathlineto{\pgfqpoint{1.695265in}{1.361896in}}%
\pgfpathlineto{\pgfqpoint{1.739471in}{1.312000in}}%
\pgfpathlineto{\pgfqpoint{1.842101in}{1.198704in}}%
\pgfpathlineto{\pgfqpoint{1.980146in}{1.050667in}}%
\pgfpathlineto{\pgfqpoint{2.051273in}{0.976000in}}%
\pgfpathlineto{\pgfqpoint{2.162747in}{0.861247in}}%
\pgfpathlineto{\pgfqpoint{2.282990in}{0.740606in}}%
\pgfpathlineto{\pgfqpoint{2.424251in}{0.602667in}}%
\pgfpathlineto{\pgfqpoint{2.502470in}{0.528000in}}%
\pgfpathmoveto{\pgfqpoint{4.768000in}{1.072054in}}%
\pgfpathlineto{\pgfqpoint{4.672302in}{1.200000in}}%
\pgfpathlineto{\pgfqpoint{4.615083in}{1.274667in}}%
\pgfpathlineto{\pgfqpoint{4.527314in}{1.386667in}}%
\pgfpathlineto{\pgfqpoint{4.447354in}{1.486012in}}%
\pgfpathlineto{\pgfqpoint{4.344565in}{1.610667in}}%
\pgfpathlineto{\pgfqpoint{4.266124in}{1.703193in}}%
\pgfpathlineto{\pgfqpoint{4.217764in}{1.760000in}}%
\pgfpathlineto{\pgfqpoint{4.120003in}{1.872000in}}%
\pgfpathlineto{\pgfqpoint{4.006465in}{1.998794in}}%
\pgfpathlineto{\pgfqpoint{3.902772in}{2.111415in}}%
\pgfpathlineto{\pgfqpoint{3.846141in}{2.172521in}}%
\pgfpathlineto{\pgfqpoint{3.705275in}{2.320000in}}%
\pgfpathlineto{\pgfqpoint{3.595523in}{2.432000in}}%
\pgfpathlineto{\pgfqpoint{3.520945in}{2.506667in}}%
\pgfpathlineto{\pgfqpoint{3.405253in}{2.620172in}}%
\pgfpathlineto{\pgfqpoint{3.285010in}{2.735193in}}%
\pgfpathlineto{\pgfqpoint{3.164768in}{2.847379in}}%
\pgfpathlineto{\pgfqpoint{3.084606in}{2.920615in}}%
\pgfpathlineto{\pgfqpoint{2.963005in}{3.029333in}}%
\pgfpathlineto{\pgfqpoint{2.859970in}{3.118762in}}%
\pgfpathlineto{\pgfqpoint{2.804040in}{3.167054in}}%
\pgfpathlineto{\pgfqpoint{2.683798in}{3.268043in}}%
\pgfpathlineto{\pgfqpoint{2.606953in}{3.331089in}}%
\pgfpathlineto{\pgfqpoint{2.555760in}{3.372595in}}%
\pgfpathlineto{\pgfqpoint{2.423056in}{3.477333in}}%
\pgfpathlineto{\pgfqpoint{2.202828in}{3.643495in}}%
\pgfpathlineto{\pgfqpoint{2.082586in}{3.729927in}}%
\pgfpathlineto{\pgfqpoint{1.962175in}{3.813333in}}%
\pgfpathlineto{\pgfqpoint{1.721859in}{3.969326in}}%
\pgfpathlineto{\pgfqpoint{1.521455in}{4.087970in}}%
\pgfpathlineto{\pgfqpoint{1.321051in}{4.194740in}}%
\pgfpathlineto{\pgfqpoint{1.261042in}{4.224000in}}%
\pgfpathlineto{\pgfqpoint{1.243413in}{4.224000in}}%
\pgfpathlineto{\pgfqpoint{1.401212in}{4.145547in}}%
\pgfpathlineto{\pgfqpoint{1.610618in}{4.028949in}}%
\pgfpathlineto{\pgfqpoint{1.721859in}{3.961881in}}%
\pgfpathlineto{\pgfqpoint{1.962343in}{3.805921in}}%
\pgfpathlineto{\pgfqpoint{2.059930in}{3.738667in}}%
\pgfpathlineto{\pgfqpoint{2.165082in}{3.664000in}}%
\pgfpathlineto{\pgfqpoint{2.282990in}{3.577232in}}%
\pgfpathlineto{\pgfqpoint{2.342036in}{3.532332in}}%
\pgfpathlineto{\pgfqpoint{2.403232in}{3.485853in}}%
\pgfpathlineto{\pgfqpoint{2.473506in}{3.430790in}}%
\pgfpathlineto{\pgfqpoint{2.523475in}{3.391498in}}%
\pgfpathlineto{\pgfqpoint{2.648178in}{3.290667in}}%
\pgfpathlineto{\pgfqpoint{2.924283in}{3.056485in}}%
\pgfpathlineto{\pgfqpoint{3.021279in}{2.970347in}}%
\pgfpathlineto{\pgfqpoint{3.084606in}{2.913809in}}%
\pgfpathlineto{\pgfqpoint{3.204848in}{2.803498in}}%
\pgfpathlineto{\pgfqpoint{3.285010in}{2.728360in}}%
\pgfpathlineto{\pgfqpoint{3.405253in}{2.613223in}}%
\pgfpathlineto{\pgfqpoint{3.485414in}{2.534826in}}%
\pgfpathlineto{\pgfqpoint{3.625435in}{2.394667in}}%
\pgfpathlineto{\pgfqpoint{3.734499in}{2.282667in}}%
\pgfpathlineto{\pgfqpoint{3.806061in}{2.207807in}}%
\pgfpathlineto{\pgfqpoint{3.945051in}{2.058667in}}%
\pgfpathlineto{\pgfqpoint{4.048616in}{1.944738in}}%
\pgfpathlineto{\pgfqpoint{4.178737in}{1.797333in}}%
\pgfpathlineto{\pgfqpoint{4.262419in}{1.699742in}}%
\pgfpathlineto{\pgfqpoint{4.306570in}{1.648000in}}%
\pgfpathlineto{\pgfqpoint{4.407273in}{1.526945in}}%
\pgfpathlineto{\pgfqpoint{4.637091in}{1.237333in}}%
\pgfpathlineto{\pgfqpoint{4.691322in}{1.165911in}}%
\pgfpathlineto{\pgfqpoint{4.727919in}{1.117110in}}%
\pgfpathlineto{\pgfqpoint{4.768000in}{1.062807in}}%
\pgfpathlineto{\pgfqpoint{4.768000in}{1.062807in}}%
\pgfusepath{fill}%
\end{pgfscope}%
\begin{pgfscope}%
\pgfpathrectangle{\pgfqpoint{0.800000in}{0.528000in}}{\pgfqpoint{3.968000in}{3.696000in}}%
\pgfusepath{clip}%
\pgfsetbuttcap%
\pgfsetroundjoin%
\definecolor{currentfill}{rgb}{0.280868,0.160771,0.472899}%
\pgfsetfillcolor{currentfill}%
\pgfsetlinewidth{0.000000pt}%
\definecolor{currentstroke}{rgb}{0.000000,0.000000,0.000000}%
\pgfsetstrokecolor{currentstroke}%
\pgfsetdash{}{0pt}%
\pgfpathmoveto{\pgfqpoint{2.502470in}{0.528000in}}%
\pgfpathlineto{\pgfqpoint{2.234082in}{0.789333in}}%
\pgfpathlineto{\pgfqpoint{2.122667in}{0.902146in}}%
\pgfpathlineto{\pgfqpoint{2.002424in}{1.027139in}}%
\pgfpathlineto{\pgfqpoint{1.910038in}{1.125333in}}%
\pgfpathlineto{\pgfqpoint{1.802020in}{1.242629in}}%
\pgfpathlineto{\pgfqpoint{1.673016in}{1.386667in}}%
\pgfpathlineto{\pgfqpoint{1.561535in}{1.514706in}}%
\pgfpathlineto{\pgfqpoint{1.441293in}{1.656774in}}%
\pgfpathlineto{\pgfqpoint{1.207426in}{1.946667in}}%
\pgfpathlineto{\pgfqpoint{1.120646in}{2.059346in}}%
\pgfpathlineto{\pgfqpoint{0.920242in}{2.333021in}}%
\pgfpathlineto{\pgfqpoint{0.840081in}{2.448535in}}%
\pgfpathlineto{\pgfqpoint{0.800000in}{2.507670in}}%
\pgfpathlineto{\pgfqpoint{0.800000in}{2.506667in}}%
\pgfpathlineto{\pgfqpoint{0.800000in}{2.497386in}}%
\pgfpathlineto{\pgfqpoint{0.976122in}{2.245333in}}%
\pgfpathlineto{\pgfqpoint{1.030672in}{2.170667in}}%
\pgfpathlineto{\pgfqpoint{1.120646in}{2.050607in}}%
\pgfpathlineto{\pgfqpoint{1.230041in}{1.909333in}}%
\pgfpathlineto{\pgfqpoint{1.289312in}{1.834667in}}%
\pgfpathlineto{\pgfqpoint{1.380202in}{1.722667in}}%
\pgfpathlineto{\pgfqpoint{1.481374in}{1.601053in}}%
\pgfpathlineto{\pgfqpoint{1.568664in}{1.498667in}}%
\pgfpathlineto{\pgfqpoint{1.681778in}{1.369245in}}%
\pgfpathlineto{\pgfqpoint{1.938202in}{1.088000in}}%
\pgfpathlineto{\pgfqpoint{2.008730in}{1.013333in}}%
\pgfpathlineto{\pgfqpoint{2.122667in}{0.895163in}}%
\pgfpathlineto{\pgfqpoint{2.202828in}{0.813765in}}%
\pgfpathlineto{\pgfqpoint{2.340240in}{0.677333in}}%
\pgfpathlineto{\pgfqpoint{2.495236in}{0.528000in}}%
\pgfpathmoveto{\pgfqpoint{4.768000in}{1.081302in}}%
\pgfpathlineto{\pgfqpoint{4.687838in}{1.188422in}}%
\pgfpathlineto{\pgfqpoint{4.592688in}{1.312000in}}%
\pgfpathlineto{\pgfqpoint{4.504035in}{1.424000in}}%
\pgfpathlineto{\pgfqpoint{4.407273in}{1.543153in}}%
\pgfpathlineto{\pgfqpoint{4.287030in}{1.686962in}}%
\pgfpathlineto{\pgfqpoint{4.198578in}{1.789611in}}%
\pgfpathlineto{\pgfqpoint{4.159504in}{1.834667in}}%
\pgfpathlineto{\pgfqpoint{4.086626in}{1.917118in}}%
\pgfpathlineto{\pgfqpoint{3.958564in}{2.058667in}}%
\pgfpathlineto{\pgfqpoint{3.869126in}{2.154743in}}%
\pgfpathlineto{\pgfqpoint{3.813040in}{2.214501in}}%
\pgfpathlineto{\pgfqpoint{3.765980in}{2.264082in}}%
\pgfpathlineto{\pgfqpoint{3.661324in}{2.371851in}}%
\pgfpathlineto{\pgfqpoint{3.602516in}{2.432000in}}%
\pgfpathlineto{\pgfqpoint{3.525495in}{2.509115in}}%
\pgfpathlineto{\pgfqpoint{3.405253in}{2.626980in}}%
\pgfpathlineto{\pgfqpoint{3.285010in}{2.741967in}}%
\pgfpathlineto{\pgfqpoint{3.164768in}{2.854118in}}%
\pgfpathlineto{\pgfqpoint{3.084606in}{2.927332in}}%
\pgfpathlineto{\pgfqpoint{2.964364in}{3.034852in}}%
\pgfpathlineto{\pgfqpoint{2.863726in}{3.122261in}}%
\pgfpathlineto{\pgfqpoint{2.798379in}{3.178667in}}%
\pgfpathlineto{\pgfqpoint{2.504610in}{3.420239in}}%
\pgfpathlineto{\pgfqpoint{2.383654in}{3.514667in}}%
\pgfpathlineto{\pgfqpoint{2.282990in}{3.591213in}}%
\pgfpathlineto{\pgfqpoint{2.026805in}{3.776000in}}%
\pgfpathlineto{\pgfqpoint{1.826002in}{3.910338in}}%
\pgfpathlineto{\pgfqpoint{1.761939in}{3.951675in}}%
\pgfpathlineto{\pgfqpoint{1.706616in}{3.985802in}}%
\pgfpathlineto{\pgfqpoint{1.658181in}{4.015354in}}%
\pgfpathlineto{\pgfqpoint{1.601616in}{4.049495in}}%
\pgfpathlineto{\pgfqpoint{1.521455in}{4.095758in}}%
\pgfpathlineto{\pgfqpoint{1.481374in}{4.118255in}}%
\pgfpathlineto{\pgfqpoint{1.401212in}{4.161613in}}%
\pgfpathlineto{\pgfqpoint{1.321051in}{4.202930in}}%
\pgfpathlineto{\pgfqpoint{1.278671in}{4.224000in}}%
\pgfpathlineto{\pgfqpoint{1.261042in}{4.224000in}}%
\pgfpathlineto{\pgfqpoint{1.390574in}{4.159242in}}%
\pgfpathlineto{\pgfqpoint{1.481374in}{4.110436in}}%
\pgfpathlineto{\pgfqpoint{1.681778in}{3.993948in}}%
\pgfpathlineto{\pgfqpoint{1.922263in}{3.840118in}}%
\pgfpathlineto{\pgfqpoint{2.008347in}{3.781517in}}%
\pgfpathlineto{\pgfqpoint{2.070118in}{3.738667in}}%
\pgfpathlineto{\pgfqpoint{2.167907in}{3.668806in}}%
\pgfpathlineto{\pgfqpoint{2.225729in}{3.626667in}}%
\pgfpathlineto{\pgfqpoint{2.443313in}{3.461581in}}%
\pgfpathlineto{\pgfqpoint{2.564798in}{3.365333in}}%
\pgfpathlineto{\pgfqpoint{2.701487in}{3.253333in}}%
\pgfpathlineto{\pgfqpoint{2.964364in}{3.028139in}}%
\pgfpathlineto{\pgfqpoint{3.088219in}{2.917333in}}%
\pgfpathlineto{\pgfqpoint{3.210175in}{2.805333in}}%
\pgfpathlineto{\pgfqpoint{3.289789in}{2.730667in}}%
\pgfpathlineto{\pgfqpoint{3.406803in}{2.618667in}}%
\pgfpathlineto{\pgfqpoint{3.485414in}{2.541838in}}%
\pgfpathlineto{\pgfqpoint{3.605657in}{2.421784in}}%
\pgfpathlineto{\pgfqpoint{3.714924in}{2.309777in}}%
\pgfpathlineto{\pgfqpoint{3.765980in}{2.256982in}}%
\pgfpathlineto{\pgfqpoint{4.019849in}{1.984000in}}%
\pgfpathlineto{\pgfqpoint{4.088599in}{1.907496in}}%
\pgfpathlineto{\pgfqpoint{4.217764in}{1.760000in}}%
\pgfpathlineto{\pgfqpoint{4.301536in}{1.661512in}}%
\pgfpathlineto{\pgfqpoint{4.344565in}{1.610667in}}%
\pgfpathlineto{\pgfqpoint{4.447354in}{1.486012in}}%
\pgfpathlineto{\pgfqpoint{4.647758in}{1.232271in}}%
\pgfpathlineto{\pgfqpoint{4.768000in}{1.072054in}}%
\pgfpathlineto{\pgfqpoint{4.768000in}{1.072054in}}%
\pgfusepath{fill}%
\end{pgfscope}%
\begin{pgfscope}%
\pgfpathrectangle{\pgfqpoint{0.800000in}{0.528000in}}{\pgfqpoint{3.968000in}{3.696000in}}%
\pgfusepath{clip}%
\pgfsetbuttcap%
\pgfsetroundjoin%
\definecolor{currentfill}{rgb}{0.280868,0.160771,0.472899}%
\pgfsetfillcolor{currentfill}%
\pgfsetlinewidth{0.000000pt}%
\definecolor{currentstroke}{rgb}{0.000000,0.000000,0.000000}%
\pgfsetstrokecolor{currentstroke}%
\pgfsetdash{}{0pt}%
\pgfpathmoveto{\pgfqpoint{2.495236in}{0.528000in}}%
\pgfpathlineto{\pgfqpoint{2.227171in}{0.789333in}}%
\pgfpathlineto{\pgfqpoint{2.116656in}{0.901333in}}%
\pgfpathlineto{\pgfqpoint{2.002424in}{1.019951in}}%
\pgfpathlineto{\pgfqpoint{1.903315in}{1.125333in}}%
\pgfpathlineto{\pgfqpoint{1.789597in}{1.248905in}}%
\pgfpathlineto{\pgfqpoint{1.666393in}{1.386667in}}%
\pgfpathlineto{\pgfqpoint{1.561535in}{1.506943in}}%
\pgfpathlineto{\pgfqpoint{1.441293in}{1.648728in}}%
\pgfpathlineto{\pgfqpoint{1.337517in}{1.775338in}}%
\pgfpathlineto{\pgfqpoint{1.285812in}{1.839177in}}%
\pgfpathlineto{\pgfqpoint{1.240889in}{1.895573in}}%
\pgfpathlineto{\pgfqpoint{1.058400in}{2.133333in}}%
\pgfpathlineto{\pgfqpoint{1.018578in}{2.187595in}}%
\pgfpathlineto{\pgfqpoint{0.976122in}{2.245333in}}%
\pgfpathlineto{\pgfqpoint{0.937856in}{2.299073in}}%
\pgfpathlineto{\pgfqpoint{0.896301in}{2.357333in}}%
\pgfpathlineto{\pgfqpoint{0.852027in}{2.420873in}}%
\pgfpathlineto{\pgfqpoint{0.800000in}{2.497386in}}%
\pgfpathlineto{\pgfqpoint{0.800000in}{2.487141in}}%
\pgfpathlineto{\pgfqpoint{0.960323in}{2.257780in}}%
\pgfpathlineto{\pgfqpoint{1.136388in}{2.021333in}}%
\pgfpathlineto{\pgfqpoint{1.194101in}{1.946667in}}%
\pgfpathlineto{\pgfqpoint{1.287479in}{1.828604in}}%
\pgfpathlineto{\pgfqpoint{1.404285in}{1.685333in}}%
\pgfpathlineto{\pgfqpoint{1.490927in}{1.582232in}}%
\pgfpathlineto{\pgfqpoint{1.530016in}{1.536000in}}%
\pgfpathlineto{\pgfqpoint{1.641697in}{1.407172in}}%
\pgfpathlineto{\pgfqpoint{1.761939in}{1.272091in}}%
\pgfpathlineto{\pgfqpoint{1.896593in}{1.125333in}}%
\pgfpathlineto{\pgfqpoint{1.983577in}{1.033112in}}%
\pgfpathlineto{\pgfqpoint{2.042505in}{0.970915in}}%
\pgfpathlineto{\pgfqpoint{2.183188in}{0.826667in}}%
\pgfpathlineto{\pgfqpoint{2.448829in}{0.565333in}}%
\pgfpathlineto{\pgfqpoint{2.488002in}{0.528000in}}%
\pgfpathmoveto{\pgfqpoint{4.768000in}{1.090463in}}%
\pgfpathlineto{\pgfqpoint{4.567596in}{1.352541in}}%
\pgfpathlineto{\pgfqpoint{4.466104in}{1.478798in}}%
\pgfpathlineto{\pgfqpoint{4.419719in}{1.536000in}}%
\pgfpathlineto{\pgfqpoint{4.344174in}{1.626560in}}%
\pgfpathlineto{\pgfqpoint{4.287030in}{1.694652in}}%
\pgfpathlineto{\pgfqpoint{4.198705in}{1.797333in}}%
\pgfpathlineto{\pgfqpoint{4.086626in}{1.924522in}}%
\pgfpathlineto{\pgfqpoint{3.965321in}{2.058667in}}%
\pgfpathlineto{\pgfqpoint{3.872741in}{2.158109in}}%
\pgfpathlineto{\pgfqpoint{3.826081in}{2.208000in}}%
\pgfpathlineto{\pgfqpoint{3.754952in}{2.282667in}}%
\pgfpathlineto{\pgfqpoint{3.645737in}{2.395204in}}%
\pgfpathlineto{\pgfqpoint{3.549499in}{2.491692in}}%
\pgfpathlineto{\pgfqpoint{3.485414in}{2.555558in}}%
\pgfpathlineto{\pgfqpoint{3.343316in}{2.693333in}}%
\pgfpathlineto{\pgfqpoint{3.224668in}{2.805333in}}%
\pgfpathlineto{\pgfqpoint{2.950313in}{3.053580in}}%
\pgfpathlineto{\pgfqpoint{2.884202in}{3.111678in}}%
\pgfpathlineto{\pgfqpoint{2.783923in}{3.197262in}}%
\pgfpathlineto{\pgfqpoint{2.717782in}{3.253333in}}%
\pgfpathlineto{\pgfqpoint{2.581468in}{3.365333in}}%
\pgfpathlineto{\pgfqpoint{2.456809in}{3.464763in}}%
\pgfpathlineto{\pgfqpoint{2.343847in}{3.552000in}}%
\pgfpathlineto{\pgfqpoint{2.242909in}{3.627991in}}%
\pgfpathlineto{\pgfqpoint{1.982950in}{3.813333in}}%
\pgfpathlineto{\pgfqpoint{1.871921in}{3.888000in}}%
\pgfpathlineto{\pgfqpoint{1.756263in}{3.962667in}}%
\pgfpathlineto{\pgfqpoint{1.544762in}{4.090290in}}%
\pgfpathlineto{\pgfqpoint{1.439127in}{4.149333in}}%
\pgfpathlineto{\pgfqpoint{1.295242in}{4.224000in}}%
\pgfpathlineto{\pgfqpoint{1.278671in}{4.224000in}}%
\pgfpathlineto{\pgfqpoint{1.280970in}{4.222900in}}%
\pgfpathlineto{\pgfqpoint{1.353155in}{4.186667in}}%
\pgfpathlineto{\pgfqpoint{1.409443in}{4.157000in}}%
\pgfpathlineto{\pgfqpoint{1.464832in}{4.127408in}}%
\pgfpathlineto{\pgfqpoint{1.566354in}{4.070178in}}%
\pgfpathlineto{\pgfqpoint{1.684253in}{4.000000in}}%
\pgfpathlineto{\pgfqpoint{1.896593in}{3.864090in}}%
\pgfpathlineto{\pgfqpoint{1.962343in}{3.820327in}}%
\pgfpathlineto{\pgfqpoint{2.082586in}{3.737069in}}%
\pgfpathlineto{\pgfqpoint{2.334852in}{3.552000in}}%
\pgfpathlineto{\pgfqpoint{2.603636in}{3.340622in}}%
\pgfpathlineto{\pgfqpoint{2.683798in}{3.274818in}}%
\pgfpathlineto{\pgfqpoint{2.804040in}{3.173864in}}%
\pgfpathlineto{\pgfqpoint{2.905241in}{3.086264in}}%
\pgfpathlineto{\pgfqpoint{2.970597in}{3.029333in}}%
\pgfpathlineto{\pgfqpoint{3.095613in}{2.917333in}}%
\pgfpathlineto{\pgfqpoint{3.217422in}{2.805333in}}%
\pgfpathlineto{\pgfqpoint{3.296940in}{2.730667in}}%
\pgfpathlineto{\pgfqpoint{3.413815in}{2.618667in}}%
\pgfpathlineto{\pgfqpoint{3.490203in}{2.544000in}}%
\pgfpathlineto{\pgfqpoint{3.605657in}{2.428834in}}%
\pgfpathlineto{\pgfqpoint{3.685818in}{2.347140in}}%
\pgfpathlineto{\pgfqpoint{3.819348in}{2.208000in}}%
\pgfpathlineto{\pgfqpoint{3.926303in}{2.093757in}}%
\pgfpathlineto{\pgfqpoint{4.006465in}{2.006170in}}%
\pgfpathlineto{\pgfqpoint{4.126741in}{1.872000in}}%
\pgfpathlineto{\pgfqpoint{4.206869in}{1.780277in}}%
\pgfpathlineto{\pgfqpoint{4.327111in}{1.639467in}}%
\pgfpathlineto{\pgfqpoint{4.567596in}{1.344094in}}%
\pgfpathlineto{\pgfqpoint{4.679052in}{1.200000in}}%
\pgfpathlineto{\pgfqpoint{4.768000in}{1.081302in}}%
\pgfpathlineto{\pgfqpoint{4.768000in}{1.088000in}}%
\pgfpathlineto{\pgfqpoint{4.768000in}{1.088000in}}%
\pgfusepath{fill}%
\end{pgfscope}%
\begin{pgfscope}%
\pgfpathrectangle{\pgfqpoint{0.800000in}{0.528000in}}{\pgfqpoint{3.968000in}{3.696000in}}%
\pgfusepath{clip}%
\pgfsetbuttcap%
\pgfsetroundjoin%
\definecolor{currentfill}{rgb}{0.280868,0.160771,0.472899}%
\pgfsetfillcolor{currentfill}%
\pgfsetlinewidth{0.000000pt}%
\definecolor{currentstroke}{rgb}{0.000000,0.000000,0.000000}%
\pgfsetstrokecolor{currentstroke}%
\pgfsetdash{}{0pt}%
\pgfpathmoveto{\pgfqpoint{2.488002in}{0.528000in}}%
\pgfpathlineto{\pgfqpoint{2.220259in}{0.789333in}}%
\pgfpathlineto{\pgfqpoint{2.109873in}{0.901333in}}%
\pgfpathlineto{\pgfqpoint{2.001896in}{1.013333in}}%
\pgfpathlineto{\pgfqpoint{1.908681in}{1.112683in}}%
\pgfpathlineto{\pgfqpoint{1.861993in}{1.162667in}}%
\pgfpathlineto{\pgfqpoint{1.747464in}{1.288150in}}%
\pgfpathlineto{\pgfqpoint{1.626955in}{1.424000in}}%
\pgfpathlineto{\pgfqpoint{1.561535in}{1.499180in}}%
\pgfpathlineto{\pgfqpoint{1.435358in}{1.648000in}}%
\pgfpathlineto{\pgfqpoint{1.361131in}{1.737833in}}%
\pgfpathlineto{\pgfqpoint{1.252936in}{1.872000in}}%
\pgfpathlineto{\pgfqpoint{1.180299in}{1.964897in}}%
\pgfpathlineto{\pgfqpoint{1.136388in}{2.021333in}}%
\pgfpathlineto{\pgfqpoint{1.040485in}{2.148347in}}%
\pgfpathlineto{\pgfqpoint{0.942454in}{2.282667in}}%
\pgfpathlineto{\pgfqpoint{0.885847in}{2.362629in}}%
\pgfpathlineto{\pgfqpoint{0.837489in}{2.432000in}}%
\pgfpathlineto{\pgfqpoint{0.800000in}{2.487141in}}%
\pgfpathlineto{\pgfqpoint{0.800000in}{2.476895in}}%
\pgfpathlineto{\pgfqpoint{0.935690in}{2.282667in}}%
\pgfpathlineto{\pgfqpoint{1.017276in}{2.170667in}}%
\pgfpathlineto{\pgfqpoint{1.101245in}{2.058667in}}%
\pgfpathlineto{\pgfqpoint{1.187530in}{1.946667in}}%
\pgfpathlineto{\pgfqpoint{1.280970in}{1.828555in}}%
\pgfpathlineto{\pgfqpoint{1.401212in}{1.681114in}}%
\pgfpathlineto{\pgfqpoint{1.505016in}{1.558021in}}%
\pgfpathlineto{\pgfqpoint{1.561535in}{1.491623in}}%
\pgfpathlineto{\pgfqpoint{1.686142in}{1.349333in}}%
\pgfpathlineto{\pgfqpoint{1.802020in}{1.220724in}}%
\pgfpathlineto{\pgfqpoint{1.905125in}{1.109370in}}%
\pgfpathlineto{\pgfqpoint{1.962343in}{1.047971in}}%
\pgfpathlineto{\pgfqpoint{2.042505in}{0.963929in}}%
\pgfpathlineto{\pgfqpoint{2.176320in}{0.826667in}}%
\pgfpathlineto{\pgfqpoint{2.250657in}{0.752000in}}%
\pgfpathlineto{\pgfqpoint{2.364338in}{0.640000in}}%
\pgfpathlineto{\pgfqpoint{2.443313in}{0.563778in}}%
\pgfpathlineto{\pgfqpoint{2.480846in}{0.528000in}}%
\pgfpathlineto{\pgfqpoint{2.483394in}{0.528000in}}%
\pgfpathmoveto{\pgfqpoint{4.768000in}{1.099398in}}%
\pgfpathlineto{\pgfqpoint{4.567596in}{1.360824in}}%
\pgfpathlineto{\pgfqpoint{4.333119in}{1.648000in}}%
\pgfpathlineto{\pgfqpoint{4.237611in}{1.760000in}}%
\pgfpathlineto{\pgfqpoint{4.126707in}{1.886873in}}%
\pgfpathlineto{\pgfqpoint{4.003984in}{2.023644in}}%
\pgfpathlineto{\pgfqpoint{3.867957in}{2.170667in}}%
\pgfpathlineto{\pgfqpoint{3.761769in}{2.282667in}}%
\pgfpathlineto{\pgfqpoint{3.645737in}{2.402083in}}%
\pgfpathlineto{\pgfqpoint{3.553097in}{2.495043in}}%
\pgfpathlineto{\pgfqpoint{3.504049in}{2.544000in}}%
\pgfpathlineto{\pgfqpoint{3.231915in}{2.805333in}}%
\pgfpathlineto{\pgfqpoint{2.943381in}{3.066667in}}%
\pgfpathlineto{\pgfqpoint{2.829559in}{3.165103in}}%
\pgfpathlineto{\pgfqpoint{2.763960in}{3.221278in}}%
\pgfpathlineto{\pgfqpoint{2.681013in}{3.290667in}}%
\pgfpathlineto{\pgfqpoint{2.543486in}{3.402667in}}%
\pgfpathlineto{\pgfqpoint{2.303625in}{3.589333in}}%
\pgfpathlineto{\pgfqpoint{2.202828in}{3.664456in}}%
\pgfpathlineto{\pgfqpoint{1.938537in}{3.850667in}}%
\pgfpathlineto{\pgfqpoint{1.842101in}{3.914800in}}%
\pgfpathlineto{\pgfqpoint{1.787877in}{3.949493in}}%
\pgfpathlineto{\pgfqpoint{1.721859in}{3.991596in}}%
\pgfpathlineto{\pgfqpoint{1.667947in}{4.024450in}}%
\pgfpathlineto{\pgfqpoint{1.619262in}{4.053770in}}%
\pgfpathlineto{\pgfqpoint{1.561535in}{4.088167in}}%
\pgfpathlineto{\pgfqpoint{1.481374in}{4.133800in}}%
\pgfpathlineto{\pgfqpoint{1.383938in}{4.186667in}}%
\pgfpathlineto{\pgfqpoint{1.311653in}{4.224000in}}%
\pgfpathlineto{\pgfqpoint{1.295242in}{4.224000in}}%
\pgfpathlineto{\pgfqpoint{1.368800in}{4.186667in}}%
\pgfpathlineto{\pgfqpoint{1.444275in}{4.146555in}}%
\pgfpathlineto{\pgfqpoint{1.561535in}{4.080587in}}%
\pgfpathlineto{\pgfqpoint{1.614362in}{4.049206in}}%
\pgfpathlineto{\pgfqpoint{1.663064in}{4.019902in}}%
\pgfpathlineto{\pgfqpoint{1.721859in}{3.984173in}}%
\pgfpathlineto{\pgfqpoint{1.783197in}{3.945134in}}%
\pgfpathlineto{\pgfqpoint{1.842101in}{3.907540in}}%
\pgfpathlineto{\pgfqpoint{1.928053in}{3.850667in}}%
\pgfpathlineto{\pgfqpoint{2.162747in}{3.686633in}}%
\pgfpathlineto{\pgfqpoint{2.221067in}{3.643655in}}%
\pgfpathlineto{\pgfqpoint{2.282990in}{3.598047in}}%
\pgfpathlineto{\pgfqpoint{2.403232in}{3.506461in}}%
\pgfpathlineto{\pgfqpoint{2.485585in}{3.442041in}}%
\pgfpathlineto{\pgfqpoint{2.535087in}{3.402667in}}%
\pgfpathlineto{\pgfqpoint{2.643717in}{3.314633in}}%
\pgfpathlineto{\pgfqpoint{2.763960in}{3.214615in}}%
\pgfpathlineto{\pgfqpoint{2.867481in}{3.125759in}}%
\pgfpathlineto{\pgfqpoint{2.924283in}{3.076757in}}%
\pgfpathlineto{\pgfqpoint{3.061752in}{2.954667in}}%
\pgfpathlineto{\pgfqpoint{3.184458in}{2.842667in}}%
\pgfpathlineto{\pgfqpoint{3.304091in}{2.730667in}}%
\pgfpathlineto{\pgfqpoint{3.420828in}{2.618667in}}%
\pgfpathlineto{\pgfqpoint{3.497126in}{2.544000in}}%
\pgfpathlineto{\pgfqpoint{3.609399in}{2.432000in}}%
\pgfpathlineto{\pgfqpoint{3.685818in}{2.354214in}}%
\pgfpathlineto{\pgfqpoint{3.826081in}{2.208000in}}%
\pgfpathlineto{\pgfqpoint{3.930896in}{2.096000in}}%
\pgfpathlineto{\pgfqpoint{4.006465in}{2.013547in}}%
\pgfpathlineto{\pgfqpoint{4.133293in}{1.872000in}}%
\pgfpathlineto{\pgfqpoint{4.206869in}{1.787938in}}%
\pgfpathlineto{\pgfqpoint{4.327111in}{1.647403in}}%
\pgfpathlineto{\pgfqpoint{4.431505in}{1.521237in}}%
\pgfpathlineto{\pgfqpoint{4.487434in}{1.452931in}}%
\pgfpathlineto{\pgfqpoint{4.607677in}{1.301356in}}%
\pgfpathlineto{\pgfqpoint{4.713972in}{1.162667in}}%
\pgfpathlineto{\pgfqpoint{4.768000in}{1.090463in}}%
\pgfpathlineto{\pgfqpoint{4.768000in}{1.090463in}}%
\pgfusepath{fill}%
\end{pgfscope}%
\begin{pgfscope}%
\pgfpathrectangle{\pgfqpoint{0.800000in}{0.528000in}}{\pgfqpoint{3.968000in}{3.696000in}}%
\pgfusepath{clip}%
\pgfsetbuttcap%
\pgfsetroundjoin%
\definecolor{currentfill}{rgb}{0.280868,0.160771,0.472899}%
\pgfsetfillcolor{currentfill}%
\pgfsetlinewidth{0.000000pt}%
\definecolor{currentstroke}{rgb}{0.000000,0.000000,0.000000}%
\pgfsetstrokecolor{currentstroke}%
\pgfsetdash{}{0pt}%
\pgfpathmoveto{\pgfqpoint{2.480846in}{0.528000in}}%
\pgfpathlineto{\pgfqpoint{2.363152in}{0.641153in}}%
\pgfpathlineto{\pgfqpoint{2.282990in}{0.719866in}}%
\pgfpathlineto{\pgfqpoint{2.162747in}{0.840399in}}%
\pgfpathlineto{\pgfqpoint{2.030931in}{0.976000in}}%
\pgfpathlineto{\pgfqpoint{1.922263in}{1.090572in}}%
\pgfpathlineto{\pgfqpoint{1.842101in}{1.176983in}}%
\pgfpathlineto{\pgfqpoint{1.707318in}{1.325544in}}%
\pgfpathlineto{\pgfqpoint{1.587812in}{1.461333in}}%
\pgfpathlineto{\pgfqpoint{1.481374in}{1.585466in}}%
\pgfpathlineto{\pgfqpoint{1.361131in}{1.729754in}}%
\pgfpathlineto{\pgfqpoint{1.240889in}{1.878802in}}%
\pgfpathlineto{\pgfqpoint{1.040485in}{2.139293in}}%
\pgfpathlineto{\pgfqpoint{0.935690in}{2.282667in}}%
\pgfpathlineto{\pgfqpoint{0.909016in}{2.320000in}}%
\pgfpathlineto{\pgfqpoint{0.830701in}{2.432000in}}%
\pgfpathlineto{\pgfqpoint{0.800000in}{2.476895in}}%
\pgfpathlineto{\pgfqpoint{0.800000in}{2.466750in}}%
\pgfpathlineto{\pgfqpoint{0.902295in}{2.320000in}}%
\pgfpathlineto{\pgfqpoint{0.946465in}{2.258242in}}%
\pgfpathlineto{\pgfqpoint{1.040485in}{2.130341in}}%
\pgfpathlineto{\pgfqpoint{1.240889in}{1.870465in}}%
\pgfpathlineto{\pgfqpoint{1.361131in}{1.721705in}}%
\pgfpathlineto{\pgfqpoint{1.485051in}{1.573333in}}%
\pgfpathlineto{\pgfqpoint{1.561535in}{1.484081in}}%
\pgfpathlineto{\pgfqpoint{1.681778in}{1.346817in}}%
\pgfpathlineto{\pgfqpoint{1.814352in}{1.200000in}}%
\pgfpathlineto{\pgfqpoint{1.922263in}{1.083482in}}%
\pgfpathlineto{\pgfqpoint{2.002424in}{0.998782in}}%
\pgfpathlineto{\pgfqpoint{2.132745in}{0.864000in}}%
\pgfpathlineto{\pgfqpoint{2.255528in}{0.740246in}}%
\pgfpathlineto{\pgfqpoint{2.363152in}{0.634409in}}%
\pgfpathlineto{\pgfqpoint{2.443313in}{0.557086in}}%
\pgfpathlineto{\pgfqpoint{2.473825in}{0.528000in}}%
\pgfpathmoveto{\pgfqpoint{4.768000in}{1.108333in}}%
\pgfpathlineto{\pgfqpoint{4.583212in}{1.349333in}}%
\pgfpathlineto{\pgfqpoint{4.487434in}{1.469184in}}%
\pgfpathlineto{\pgfqpoint{4.233030in}{1.772965in}}%
\pgfpathlineto{\pgfqpoint{4.113337in}{1.909333in}}%
\pgfpathlineto{\pgfqpoint{4.046538in}{1.984007in}}%
\pgfpathlineto{\pgfqpoint{3.909503in}{2.133333in}}%
\pgfpathlineto{\pgfqpoint{3.804195in}{2.245333in}}%
\pgfpathlineto{\pgfqpoint{3.725899in}{2.326860in}}%
\pgfpathlineto{\pgfqpoint{3.585929in}{2.469333in}}%
\pgfpathlineto{\pgfqpoint{3.301645in}{2.746161in}}%
\pgfpathlineto{\pgfqpoint{3.239161in}{2.805333in}}%
\pgfpathlineto{\pgfqpoint{3.141302in}{2.895476in}}%
\pgfpathlineto{\pgfqpoint{3.076643in}{2.954667in}}%
\pgfpathlineto{\pgfqpoint{2.950982in}{3.066667in}}%
\pgfpathlineto{\pgfqpoint{2.821857in}{3.178667in}}%
\pgfpathlineto{\pgfqpoint{2.723879in}{3.261615in}}%
\pgfpathlineto{\pgfqpoint{2.598137in}{3.365333in}}%
\pgfpathlineto{\pgfqpoint{2.457908in}{3.477333in}}%
\pgfpathlineto{\pgfqpoint{2.242909in}{3.641636in}}%
\pgfpathlineto{\pgfqpoint{2.184507in}{3.684268in}}%
\pgfpathlineto{\pgfqpoint{2.122667in}{3.729428in}}%
\pgfpathlineto{\pgfqpoint{2.048673in}{3.781745in}}%
\pgfpathlineto{\pgfqpoint{1.998848in}{3.816664in}}%
\pgfpathlineto{\pgfqpoint{1.882182in}{3.895602in}}%
\pgfpathlineto{\pgfqpoint{1.779154in}{3.962667in}}%
\pgfpathlineto{\pgfqpoint{1.681778in}{4.023756in}}%
\pgfpathlineto{\pgfqpoint{1.624162in}{4.058333in}}%
\pgfpathlineto{\pgfqpoint{1.575208in}{4.087402in}}%
\pgfpathlineto{\pgfqpoint{1.521455in}{4.118918in}}%
\pgfpathlineto{\pgfqpoint{1.327612in}{4.224000in}}%
\pgfpathlineto{\pgfqpoint{1.311653in}{4.224000in}}%
\pgfpathlineto{\pgfqpoint{1.361131in}{4.198660in}}%
\pgfpathlineto{\pgfqpoint{1.453302in}{4.149333in}}%
\pgfpathlineto{\pgfqpoint{1.523290in}{4.110290in}}%
\pgfpathlineto{\pgfqpoint{1.647390in}{4.037333in}}%
\pgfpathlineto{\pgfqpoint{1.842101in}{3.914800in}}%
\pgfpathlineto{\pgfqpoint{1.938537in}{3.850667in}}%
\pgfpathlineto{\pgfqpoint{2.162747in}{3.693612in}}%
\pgfpathlineto{\pgfqpoint{2.269835in}{3.614414in}}%
\pgfpathlineto{\pgfqpoint{2.323071in}{3.574657in}}%
\pgfpathlineto{\pgfqpoint{2.449377in}{3.477333in}}%
\pgfpathlineto{\pgfqpoint{2.563556in}{3.386560in}}%
\pgfpathlineto{\pgfqpoint{2.683798in}{3.288368in}}%
\pgfpathlineto{\pgfqpoint{2.814093in}{3.178667in}}%
\pgfpathlineto{\pgfqpoint{3.110402in}{2.917333in}}%
\pgfpathlineto{\pgfqpoint{3.231915in}{2.805333in}}%
\pgfpathlineto{\pgfqpoint{3.504049in}{2.544000in}}%
\pgfpathlineto{\pgfqpoint{3.616192in}{2.432000in}}%
\pgfpathlineto{\pgfqpoint{3.689570in}{2.357333in}}%
\pgfpathlineto{\pgfqpoint{3.806061in}{2.236262in}}%
\pgfpathlineto{\pgfqpoint{3.913589in}{2.121491in}}%
\pgfpathlineto{\pgfqpoint{3.971921in}{2.058667in}}%
\pgfpathlineto{\pgfqpoint{4.086626in}{1.931926in}}%
\pgfpathlineto{\pgfqpoint{4.333119in}{1.648000in}}%
\pgfpathlineto{\pgfqpoint{4.417900in}{1.545899in}}%
\pgfpathlineto{\pgfqpoint{4.456889in}{1.498667in}}%
\pgfpathlineto{\pgfqpoint{4.527515in}{1.411219in}}%
\pgfpathlineto{\pgfqpoint{4.635015in}{1.274667in}}%
\pgfpathlineto{\pgfqpoint{4.727919in}{1.153047in}}%
\pgfpathlineto{\pgfqpoint{4.768000in}{1.099398in}}%
\pgfpathlineto{\pgfqpoint{4.768000in}{1.099398in}}%
\pgfusepath{fill}%
\end{pgfscope}%
\begin{pgfscope}%
\pgfpathrectangle{\pgfqpoint{0.800000in}{0.528000in}}{\pgfqpoint{3.968000in}{3.696000in}}%
\pgfusepath{clip}%
\pgfsetbuttcap%
\pgfsetroundjoin%
\definecolor{currentfill}{rgb}{0.280255,0.165693,0.476498}%
\pgfsetfillcolor{currentfill}%
\pgfsetlinewidth{0.000000pt}%
\definecolor{currentstroke}{rgb}{0.000000,0.000000,0.000000}%
\pgfsetstrokecolor{currentstroke}%
\pgfsetdash{}{0pt}%
\pgfpathmoveto{\pgfqpoint{2.473825in}{0.528000in}}%
\pgfpathlineto{\pgfqpoint{2.357417in}{0.640000in}}%
\pgfpathlineto{\pgfqpoint{2.281304in}{0.714667in}}%
\pgfpathlineto{\pgfqpoint{2.162747in}{0.833450in}}%
\pgfpathlineto{\pgfqpoint{2.024232in}{0.976000in}}%
\pgfpathlineto{\pgfqpoint{1.918031in}{1.088000in}}%
\pgfpathlineto{\pgfqpoint{1.842101in}{1.169743in}}%
\pgfpathlineto{\pgfqpoint{1.712907in}{1.312000in}}%
\pgfpathlineto{\pgfqpoint{1.601616in}{1.437923in}}%
\pgfpathlineto{\pgfqpoint{1.357769in}{1.725799in}}%
\pgfpathlineto{\pgfqpoint{1.239667in}{1.872000in}}%
\pgfpathlineto{\pgfqpoint{1.139026in}{2.001120in}}%
\pgfpathlineto{\pgfqpoint{1.094612in}{2.058667in}}%
\pgfpathlineto{\pgfqpoint{1.040485in}{2.130341in}}%
\pgfpathlineto{\pgfqpoint{0.960323in}{2.239161in}}%
\pgfpathlineto{\pgfqpoint{0.880162in}{2.351163in}}%
\pgfpathlineto{\pgfqpoint{0.800000in}{2.466750in}}%
\pgfpathlineto{\pgfqpoint{0.800000in}{2.456887in}}%
\pgfpathlineto{\pgfqpoint{0.880162in}{2.341699in}}%
\pgfpathlineto{\pgfqpoint{0.960323in}{2.230066in}}%
\pgfpathlineto{\pgfqpoint{1.145351in}{1.984000in}}%
\pgfpathlineto{\pgfqpoint{1.240889in}{1.862337in}}%
\pgfpathlineto{\pgfqpoint{1.361131in}{1.713865in}}%
\pgfpathlineto{\pgfqpoint{1.481374in}{1.569978in}}%
\pgfpathlineto{\pgfqpoint{1.586931in}{1.447655in}}%
\pgfpathlineto{\pgfqpoint{1.641697in}{1.384688in}}%
\pgfpathlineto{\pgfqpoint{1.750124in}{1.263661in}}%
\pgfpathlineto{\pgfqpoint{1.807712in}{1.200000in}}%
\pgfpathlineto{\pgfqpoint{1.922263in}{1.076459in}}%
\pgfpathlineto{\pgfqpoint{2.029466in}{0.963855in}}%
\pgfpathlineto{\pgfqpoint{2.089527in}{0.901333in}}%
\pgfpathlineto{\pgfqpoint{2.202828in}{0.786101in}}%
\pgfpathlineto{\pgfqpoint{2.298264in}{0.691561in}}%
\pgfpathlineto{\pgfqpoint{2.363152in}{0.627695in}}%
\pgfpathlineto{\pgfqpoint{2.443313in}{0.550394in}}%
\pgfpathlineto{\pgfqpoint{2.466805in}{0.528000in}}%
\pgfpathmoveto{\pgfqpoint{4.768000in}{1.117268in}}%
\pgfpathlineto{\pgfqpoint{4.589753in}{1.349333in}}%
\pgfpathlineto{\pgfqpoint{4.487434in}{1.477183in}}%
\pgfpathlineto{\pgfqpoint{4.246949in}{1.764389in}}%
\pgfpathlineto{\pgfqpoint{4.119929in}{1.909333in}}%
\pgfpathlineto{\pgfqpoint{4.046545in}{1.991188in}}%
\pgfpathlineto{\pgfqpoint{3.916154in}{2.133333in}}%
\pgfpathlineto{\pgfqpoint{3.806061in}{2.250352in}}%
\pgfpathlineto{\pgfqpoint{3.725899in}{2.333763in}}%
\pgfpathlineto{\pgfqpoint{3.592765in}{2.469333in}}%
\pgfpathlineto{\pgfqpoint{3.501930in}{2.559384in}}%
\pgfpathlineto{\pgfqpoint{3.441865in}{2.618667in}}%
\pgfpathlineto{\pgfqpoint{3.364629in}{2.693333in}}%
\pgfpathlineto{\pgfqpoint{3.244929in}{2.806679in}}%
\pgfpathlineto{\pgfqpoint{3.144900in}{2.898828in}}%
\pgfpathlineto{\pgfqpoint{3.084088in}{2.954667in}}%
\pgfpathlineto{\pgfqpoint{2.958584in}{3.066667in}}%
\pgfpathlineto{\pgfqpoint{2.829622in}{3.178667in}}%
\pgfpathlineto{\pgfqpoint{2.696976in}{3.290667in}}%
\pgfpathlineto{\pgfqpoint{2.466439in}{3.477333in}}%
\pgfpathlineto{\pgfqpoint{2.221895in}{3.664000in}}%
\pgfpathlineto{\pgfqpoint{2.119480in}{3.738667in}}%
\pgfpathlineto{\pgfqpoint{2.002424in}{3.821126in}}%
\pgfpathlineto{\pgfqpoint{1.904136in}{3.888000in}}%
\pgfpathlineto{\pgfqpoint{1.721859in}{4.006267in}}%
\pgfpathlineto{\pgfqpoint{1.505984in}{4.134923in}}%
\pgfpathlineto{\pgfqpoint{1.481341in}{4.149364in}}%
\pgfpathlineto{\pgfqpoint{1.361131in}{4.214580in}}%
\pgfpathlineto{\pgfqpoint{1.342964in}{4.224000in}}%
\pgfpathlineto{\pgfqpoint{1.327612in}{4.224000in}}%
\pgfpathlineto{\pgfqpoint{1.481374in}{4.141573in}}%
\pgfpathlineto{\pgfqpoint{1.696881in}{4.014068in}}%
\pgfpathlineto{\pgfqpoint{1.761939in}{3.973677in}}%
\pgfpathlineto{\pgfqpoint{1.882182in}{3.895602in}}%
\pgfpathlineto{\pgfqpoint{1.956650in}{3.845364in}}%
\pgfpathlineto{\pgfqpoint{2.003663in}{3.813333in}}%
\pgfpathlineto{\pgfqpoint{2.262978in}{3.626667in}}%
\pgfpathlineto{\pgfqpoint{2.505146in}{3.440000in}}%
\pgfpathlineto{\pgfqpoint{2.603636in}{3.360879in}}%
\pgfpathlineto{\pgfqpoint{2.689041in}{3.290667in}}%
\pgfpathlineto{\pgfqpoint{2.821857in}{3.178667in}}%
\pgfpathlineto{\pgfqpoint{3.084606in}{2.947482in}}%
\pgfpathlineto{\pgfqpoint{3.181636in}{2.858379in}}%
\pgfpathlineto{\pgfqpoint{3.244929in}{2.799951in}}%
\pgfpathlineto{\pgfqpoint{3.325091in}{2.724311in}}%
\pgfpathlineto{\pgfqpoint{3.445333in}{2.608474in}}%
\pgfpathlineto{\pgfqpoint{3.537286in}{2.517650in}}%
\pgfpathlineto{\pgfqpoint{3.585929in}{2.469333in}}%
\pgfpathlineto{\pgfqpoint{3.691241in}{2.362384in}}%
\pgfpathlineto{\pgfqpoint{3.732526in}{2.320000in}}%
\pgfpathlineto{\pgfqpoint{3.846141in}{2.201022in}}%
\pgfpathlineto{\pgfqpoint{3.978491in}{2.058667in}}%
\pgfpathlineto{\pgfqpoint{4.086626in}{1.939330in}}%
\pgfpathlineto{\pgfqpoint{4.179245in}{1.834667in}}%
\pgfpathlineto{\pgfqpoint{4.287030in}{1.710032in}}%
\pgfpathlineto{\pgfqpoint{4.407273in}{1.567056in}}%
\pgfpathlineto{\pgfqpoint{4.641660in}{1.274667in}}%
\pgfpathlineto{\pgfqpoint{4.727919in}{1.161962in}}%
\pgfpathlineto{\pgfqpoint{4.768000in}{1.108333in}}%
\pgfpathlineto{\pgfqpoint{4.768000in}{1.108333in}}%
\pgfusepath{fill}%
\end{pgfscope}%
\begin{pgfscope}%
\pgfpathrectangle{\pgfqpoint{0.800000in}{0.528000in}}{\pgfqpoint{3.968000in}{3.696000in}}%
\pgfusepath{clip}%
\pgfsetbuttcap%
\pgfsetroundjoin%
\definecolor{currentfill}{rgb}{0.280255,0.165693,0.476498}%
\pgfsetfillcolor{currentfill}%
\pgfsetlinewidth{0.000000pt}%
\definecolor{currentstroke}{rgb}{0.000000,0.000000,0.000000}%
\pgfsetstrokecolor{currentstroke}%
\pgfsetdash{}{0pt}%
\pgfpathmoveto{\pgfqpoint{2.466805in}{0.528000in}}%
\pgfpathlineto{\pgfqpoint{2.350531in}{0.640000in}}%
\pgfpathlineto{\pgfqpoint{2.274504in}{0.714667in}}%
\pgfpathlineto{\pgfqpoint{2.160838in}{0.828445in}}%
\pgfpathlineto{\pgfqpoint{2.042505in}{0.949958in}}%
\pgfpathlineto{\pgfqpoint{1.935357in}{1.062864in}}%
\pgfpathlineto{\pgfqpoint{1.876585in}{1.125333in}}%
\pgfpathlineto{\pgfqpoint{1.802020in}{1.206218in}}%
\pgfpathlineto{\pgfqpoint{1.673060in}{1.349333in}}%
\pgfpathlineto{\pgfqpoint{1.586931in}{1.447655in}}%
\pgfpathlineto{\pgfqpoint{1.542454in}{1.498667in}}%
\pgfpathlineto{\pgfqpoint{1.441293in}{1.617471in}}%
\pgfpathlineto{\pgfqpoint{1.200808in}{1.912880in}}%
\pgfpathlineto{\pgfqpoint{1.080566in}{2.068427in}}%
\pgfpathlineto{\pgfqpoint{0.976473in}{2.208000in}}%
\pgfpathlineto{\pgfqpoint{0.949217in}{2.245333in}}%
\pgfpathlineto{\pgfqpoint{0.869126in}{2.357333in}}%
\pgfpathlineto{\pgfqpoint{0.800000in}{2.456887in}}%
\pgfpathlineto{\pgfqpoint{0.800000in}{2.447023in}}%
\pgfpathlineto{\pgfqpoint{0.880162in}{2.332235in}}%
\pgfpathlineto{\pgfqpoint{0.960323in}{2.220970in}}%
\pgfpathlineto{\pgfqpoint{1.138820in}{1.984000in}}%
\pgfpathlineto{\pgfqpoint{1.240889in}{1.854209in}}%
\pgfpathlineto{\pgfqpoint{1.347443in}{1.722667in}}%
\pgfpathlineto{\pgfqpoint{1.423492in}{1.631420in}}%
\pgfpathlineto{\pgfqpoint{1.481374in}{1.562407in}}%
\pgfpathlineto{\pgfqpoint{1.583358in}{1.444326in}}%
\pgfpathlineto{\pgfqpoint{1.641697in}{1.377381in}}%
\pgfpathlineto{\pgfqpoint{1.746602in}{1.260381in}}%
\pgfpathlineto{\pgfqpoint{1.802020in}{1.198992in}}%
\pgfpathlineto{\pgfqpoint{1.939949in}{1.050667in}}%
\pgfpathlineto{\pgfqpoint{2.046660in}{0.938667in}}%
\pgfpathlineto{\pgfqpoint{2.162747in}{0.819734in}}%
\pgfpathlineto{\pgfqpoint{2.242909in}{0.739272in}}%
\pgfpathlineto{\pgfqpoint{2.382054in}{0.602667in}}%
\pgfpathlineto{\pgfqpoint{2.459785in}{0.528000in}}%
\pgfpathmoveto{\pgfqpoint{4.768000in}{1.126174in}}%
\pgfpathlineto{\pgfqpoint{4.536880in}{1.424000in}}%
\pgfpathlineto{\pgfqpoint{4.476446in}{1.498667in}}%
\pgfpathlineto{\pgfqpoint{4.383929in}{1.610667in}}%
\pgfpathlineto{\pgfqpoint{4.287030in}{1.725335in}}%
\pgfpathlineto{\pgfqpoint{4.159505in}{1.872000in}}%
\pgfpathlineto{\pgfqpoint{4.046545in}{1.998377in}}%
\pgfpathlineto{\pgfqpoint{3.922804in}{2.133333in}}%
\pgfpathlineto{\pgfqpoint{3.806061in}{2.257279in}}%
\pgfpathlineto{\pgfqpoint{3.698197in}{2.368864in}}%
\pgfpathlineto{\pgfqpoint{3.636572in}{2.432000in}}%
\pgfpathlineto{\pgfqpoint{3.524819in}{2.544000in}}%
\pgfpathlineto{\pgfqpoint{3.427259in}{2.639164in}}%
\pgfpathlineto{\pgfqpoint{3.365172in}{2.699454in}}%
\pgfpathlineto{\pgfqpoint{3.244929in}{2.813272in}}%
\pgfpathlineto{\pgfqpoint{3.148499in}{2.902180in}}%
\pgfpathlineto{\pgfqpoint{3.084606in}{2.960761in}}%
\pgfpathlineto{\pgfqpoint{2.964364in}{3.068229in}}%
\pgfpathlineto{\pgfqpoint{2.837387in}{3.178667in}}%
\pgfpathlineto{\pgfqpoint{2.704910in}{3.290667in}}%
\pgfpathlineto{\pgfqpoint{2.443313in}{3.502221in}}%
\pgfpathlineto{\pgfqpoint{2.370337in}{3.558693in}}%
\pgfpathlineto{\pgfqpoint{2.323071in}{3.595047in}}%
\pgfpathlineto{\pgfqpoint{2.202828in}{3.684888in}}%
\pgfpathlineto{\pgfqpoint{2.162747in}{3.714208in}}%
\pgfpathlineto{\pgfqpoint{2.042505in}{3.800114in}}%
\pgfpathlineto{\pgfqpoint{2.002424in}{3.828056in}}%
\pgfpathlineto{\pgfqpoint{1.882182in}{3.909759in}}%
\pgfpathlineto{\pgfqpoint{1.761939in}{3.988144in}}%
\pgfpathlineto{\pgfqpoint{1.706302in}{4.022843in}}%
\pgfpathlineto{\pgfqpoint{1.641697in}{4.062989in}}%
\pgfpathlineto{\pgfqpoint{1.585040in}{4.096560in}}%
\pgfpathlineto{\pgfqpoint{1.559644in}{4.112000in}}%
\pgfpathlineto{\pgfqpoint{1.441293in}{4.179239in}}%
\pgfpathlineto{\pgfqpoint{1.401212in}{4.201059in}}%
\pgfpathlineto{\pgfqpoint{1.358316in}{4.224000in}}%
\pgfpathlineto{\pgfqpoint{1.342964in}{4.224000in}}%
\pgfpathlineto{\pgfqpoint{1.481395in}{4.149333in}}%
\pgfpathlineto{\pgfqpoint{1.681778in}{4.031165in}}%
\pgfpathlineto{\pgfqpoint{1.922263in}{3.875857in}}%
\pgfpathlineto{\pgfqpoint{2.013612in}{3.813333in}}%
\pgfpathlineto{\pgfqpoint{2.242909in}{3.648458in}}%
\pgfpathlineto{\pgfqpoint{2.370553in}{3.552000in}}%
\pgfpathlineto{\pgfqpoint{2.483394in}{3.463982in}}%
\pgfpathlineto{\pgfqpoint{2.560285in}{3.402667in}}%
\pgfpathlineto{\pgfqpoint{2.651899in}{3.328000in}}%
\pgfpathlineto{\pgfqpoint{2.785828in}{3.216000in}}%
\pgfpathlineto{\pgfqpoint{3.044525in}{2.990299in}}%
\pgfpathlineto{\pgfqpoint{3.165905in}{2.880000in}}%
\pgfpathlineto{\pgfqpoint{3.265379in}{2.787048in}}%
\pgfpathlineto{\pgfqpoint{3.325531in}{2.730667in}}%
\pgfpathlineto{\pgfqpoint{3.445333in}{2.615294in}}%
\pgfpathlineto{\pgfqpoint{3.540798in}{2.520921in}}%
\pgfpathlineto{\pgfqpoint{3.605657in}{2.456384in}}%
\pgfpathlineto{\pgfqpoint{3.739194in}{2.320000in}}%
\pgfpathlineto{\pgfqpoint{3.847243in}{2.206974in}}%
\pgfpathlineto{\pgfqpoint{3.985061in}{2.058667in}}%
\pgfpathlineto{\pgfqpoint{4.086684in}{1.946667in}}%
\pgfpathlineto{\pgfqpoint{4.185759in}{1.834667in}}%
\pgfpathlineto{\pgfqpoint{4.287030in}{1.717722in}}%
\pgfpathlineto{\pgfqpoint{4.408634in}{1.573333in}}%
\pgfpathlineto{\pgfqpoint{4.500274in}{1.461333in}}%
\pgfpathlineto{\pgfqpoint{4.589753in}{1.349333in}}%
\pgfpathlineto{\pgfqpoint{4.687838in}{1.223223in}}%
\pgfpathlineto{\pgfqpoint{4.768000in}{1.117268in}}%
\pgfpathlineto{\pgfqpoint{4.768000in}{1.125333in}}%
\pgfpathlineto{\pgfqpoint{4.768000in}{1.125333in}}%
\pgfusepath{fill}%
\end{pgfscope}%
\begin{pgfscope}%
\pgfpathrectangle{\pgfqpoint{0.800000in}{0.528000in}}{\pgfqpoint{3.968000in}{3.696000in}}%
\pgfusepath{clip}%
\pgfsetbuttcap%
\pgfsetroundjoin%
\definecolor{currentfill}{rgb}{0.280255,0.165693,0.476498}%
\pgfsetfillcolor{currentfill}%
\pgfsetlinewidth{0.000000pt}%
\definecolor{currentstroke}{rgb}{0.000000,0.000000,0.000000}%
\pgfsetstrokecolor{currentstroke}%
\pgfsetdash{}{0pt}%
\pgfpathmoveto{\pgfqpoint{2.459785in}{0.528000in}}%
\pgfpathlineto{\pgfqpoint{2.343644in}{0.640000in}}%
\pgfpathlineto{\pgfqpoint{2.046660in}{0.938667in}}%
\pgfpathlineto{\pgfqpoint{1.939949in}{1.050667in}}%
\pgfpathlineto{\pgfqpoint{1.857268in}{1.139461in}}%
\pgfpathlineto{\pgfqpoint{1.795842in}{1.205755in}}%
\pgfpathlineto{\pgfqpoint{1.666579in}{1.349333in}}%
\pgfpathlineto{\pgfqpoint{1.583358in}{1.444326in}}%
\pgfpathlineto{\pgfqpoint{1.535951in}{1.498667in}}%
\pgfpathlineto{\pgfqpoint{1.440472in}{1.610667in}}%
\pgfpathlineto{\pgfqpoint{1.347443in}{1.722667in}}%
\pgfpathlineto{\pgfqpoint{1.240889in}{1.854209in}}%
\pgfpathlineto{\pgfqpoint{1.138820in}{1.984000in}}%
\pgfpathlineto{\pgfqpoint{1.083239in}{2.056176in}}%
\pgfpathlineto{\pgfqpoint{0.997270in}{2.170667in}}%
\pgfpathlineto{\pgfqpoint{0.800000in}{2.447023in}}%
\pgfpathlineto{\pgfqpoint{0.800000in}{2.437160in}}%
\pgfpathlineto{\pgfqpoint{0.855766in}{2.357333in}}%
\pgfpathlineto{\pgfqpoint{0.908958in}{2.282667in}}%
\pgfpathlineto{\pgfqpoint{1.000404in}{2.157672in}}%
\pgfpathlineto{\pgfqpoint{1.103497in}{2.021333in}}%
\pgfpathlineto{\pgfqpoint{1.200808in}{1.896482in}}%
\pgfpathlineto{\pgfqpoint{1.434083in}{1.610667in}}%
\pgfpathlineto{\pgfqpoint{1.521455in}{1.507953in}}%
\pgfpathlineto{\pgfqpoint{1.627066in}{1.386667in}}%
\pgfpathlineto{\pgfqpoint{1.726815in}{1.274667in}}%
\pgfpathlineto{\pgfqpoint{1.802020in}{1.191933in}}%
\pgfpathlineto{\pgfqpoint{1.933332in}{1.050667in}}%
\pgfpathlineto{\pgfqpoint{2.042505in}{0.936056in}}%
\pgfpathlineto{\pgfqpoint{2.186184in}{0.789333in}}%
\pgfpathlineto{\pgfqpoint{2.452765in}{0.528000in}}%
\pgfpathmoveto{\pgfqpoint{4.768000in}{1.134818in}}%
\pgfpathlineto{\pgfqpoint{4.567596in}{1.393735in}}%
\pgfpathlineto{\pgfqpoint{4.447354in}{1.542094in}}%
\pgfpathlineto{\pgfqpoint{4.359107in}{1.648000in}}%
\pgfpathlineto{\pgfqpoint{4.246949in}{1.779307in}}%
\pgfpathlineto{\pgfqpoint{4.126707in}{1.916346in}}%
\pgfpathlineto{\pgfqpoint{4.020246in}{2.034170in}}%
\pgfpathlineto{\pgfqpoint{3.963946in}{2.096000in}}%
\pgfpathlineto{\pgfqpoint{3.846141in}{2.222020in}}%
\pgfpathlineto{\pgfqpoint{3.752530in}{2.320000in}}%
\pgfpathlineto{\pgfqpoint{3.643365in}{2.432000in}}%
\pgfpathlineto{\pgfqpoint{3.525495in}{2.550019in}}%
\pgfpathlineto{\pgfqpoint{3.430796in}{2.642459in}}%
\pgfpathlineto{\pgfqpoint{3.365172in}{2.706080in}}%
\pgfpathlineto{\pgfqpoint{3.244929in}{2.819866in}}%
\pgfpathlineto{\pgfqpoint{3.152097in}{2.905532in}}%
\pgfpathlineto{\pgfqpoint{3.091403in}{2.960998in}}%
\pgfpathlineto{\pgfqpoint{3.044525in}{3.003421in}}%
\pgfpathlineto{\pgfqpoint{2.924283in}{3.109967in}}%
\pgfpathlineto{\pgfqpoint{2.801469in}{3.216000in}}%
\pgfpathlineto{\pgfqpoint{2.523475in}{3.445478in}}%
\pgfpathlineto{\pgfqpoint{2.417942in}{3.528368in}}%
\pgfpathlineto{\pgfqpoint{2.363152in}{3.571077in}}%
\pgfpathlineto{\pgfqpoint{2.240344in}{3.664000in}}%
\pgfpathlineto{\pgfqpoint{2.122667in}{3.750028in}}%
\pgfpathlineto{\pgfqpoint{2.002424in}{3.834987in}}%
\pgfpathlineto{\pgfqpoint{1.882182in}{3.916838in}}%
\pgfpathlineto{\pgfqpoint{1.754637in}{4.000000in}}%
\pgfpathlineto{\pgfqpoint{1.634588in}{4.074667in}}%
\pgfpathlineto{\pgfqpoint{1.440444in}{4.187457in}}%
\pgfpathlineto{\pgfqpoint{1.372907in}{4.224000in}}%
\pgfpathlineto{\pgfqpoint{1.358316in}{4.224000in}}%
\pgfpathlineto{\pgfqpoint{1.360106in}{4.223045in}}%
\pgfpathlineto{\pgfqpoint{1.364731in}{4.220647in}}%
\pgfpathlineto{\pgfqpoint{1.481374in}{4.156896in}}%
\pgfpathlineto{\pgfqpoint{1.561535in}{4.110906in}}%
\pgfpathlineto{\pgfqpoint{1.761939in}{3.988144in}}%
\pgfpathlineto{\pgfqpoint{1.858662in}{3.925333in}}%
\pgfpathlineto{\pgfqpoint{1.969665in}{3.850667in}}%
\pgfpathlineto{\pgfqpoint{2.101332in}{3.758539in}}%
\pgfpathlineto{\pgfqpoint{2.202828in}{3.684888in}}%
\pgfpathlineto{\pgfqpoint{2.330551in}{3.589333in}}%
\pgfpathlineto{\pgfqpoint{2.443313in}{3.502221in}}%
\pgfpathlineto{\pgfqpoint{2.500965in}{3.456367in}}%
\pgfpathlineto{\pgfqpoint{2.568509in}{3.402667in}}%
\pgfpathlineto{\pgfqpoint{2.704910in}{3.290667in}}%
\pgfpathlineto{\pgfqpoint{2.804040in}{3.207178in}}%
\pgfpathlineto{\pgfqpoint{2.934605in}{3.094386in}}%
\pgfpathlineto{\pgfqpoint{3.049950in}{2.992000in}}%
\pgfpathlineto{\pgfqpoint{3.173030in}{2.880000in}}%
\pgfpathlineto{\pgfqpoint{3.268951in}{2.790375in}}%
\pgfpathlineto{\pgfqpoint{3.332473in}{2.730667in}}%
\pgfpathlineto{\pgfqpoint{3.448776in}{2.618667in}}%
\pgfpathlineto{\pgfqpoint{3.544310in}{2.524192in}}%
\pgfpathlineto{\pgfqpoint{3.605657in}{2.463251in}}%
\pgfpathlineto{\pgfqpoint{3.745862in}{2.320000in}}%
\pgfpathlineto{\pgfqpoint{3.852824in}{2.208000in}}%
\pgfpathlineto{\pgfqpoint{3.966384in}{2.086194in}}%
\pgfpathlineto{\pgfqpoint{4.093137in}{1.946667in}}%
\pgfpathlineto{\pgfqpoint{4.180654in}{1.847583in}}%
\pgfpathlineto{\pgfqpoint{4.224827in}{1.797333in}}%
\pgfpathlineto{\pgfqpoint{4.327111in}{1.678240in}}%
\pgfpathlineto{\pgfqpoint{4.447354in}{1.534295in}}%
\pgfpathlineto{\pgfqpoint{4.567596in}{1.385673in}}%
\pgfpathlineto{\pgfqpoint{4.687838in}{1.231829in}}%
\pgfpathlineto{\pgfqpoint{4.768000in}{1.126174in}}%
\pgfpathlineto{\pgfqpoint{4.768000in}{1.126174in}}%
\pgfusepath{fill}%
\end{pgfscope}%
\begin{pgfscope}%
\pgfpathrectangle{\pgfqpoint{0.800000in}{0.528000in}}{\pgfqpoint{3.968000in}{3.696000in}}%
\pgfusepath{clip}%
\pgfsetbuttcap%
\pgfsetroundjoin%
\definecolor{currentfill}{rgb}{0.279574,0.170599,0.479997}%
\pgfsetfillcolor{currentfill}%
\pgfsetlinewidth{0.000000pt}%
\definecolor{currentstroke}{rgb}{0.000000,0.000000,0.000000}%
\pgfsetstrokecolor{currentstroke}%
\pgfsetdash{}{0pt}%
\pgfpathmoveto{\pgfqpoint{2.452765in}{0.528000in}}%
\pgfpathlineto{\pgfqpoint{2.336757in}{0.640000in}}%
\pgfpathlineto{\pgfqpoint{2.076146in}{0.901333in}}%
\pgfpathlineto{\pgfqpoint{1.962343in}{1.019931in}}%
\pgfpathlineto{\pgfqpoint{1.828958in}{1.162667in}}%
\pgfpathlineto{\pgfqpoint{1.721859in}{1.280167in}}%
\pgfpathlineto{\pgfqpoint{1.615765in}{1.399845in}}%
\pgfpathlineto{\pgfqpoint{1.561535in}{1.461455in}}%
\pgfpathlineto{\pgfqpoint{1.465672in}{1.573333in}}%
\pgfpathlineto{\pgfqpoint{1.361131in}{1.698187in}}%
\pgfpathlineto{\pgfqpoint{1.132288in}{1.984000in}}%
\pgfpathlineto{\pgfqpoint{1.074869in}{2.058667in}}%
\pgfpathlineto{\pgfqpoint{0.990756in}{2.170667in}}%
\pgfpathlineto{\pgfqpoint{0.920242in}{2.267023in}}%
\pgfpathlineto{\pgfqpoint{0.840081in}{2.379607in}}%
\pgfpathlineto{\pgfqpoint{0.800000in}{2.437160in}}%
\pgfpathlineto{\pgfqpoint{0.800000in}{2.427466in}}%
\pgfpathlineto{\pgfqpoint{0.984242in}{2.170667in}}%
\pgfpathlineto{\pgfqpoint{1.080566in}{2.042746in}}%
\pgfpathlineto{\pgfqpoint{1.184227in}{1.909333in}}%
\pgfpathlineto{\pgfqpoint{1.280970in}{1.788275in}}%
\pgfpathlineto{\pgfqpoint{1.401212in}{1.642167in}}%
\pgfpathlineto{\pgfqpoint{1.490994in}{1.536000in}}%
\pgfpathlineto{\pgfqpoint{1.601616in}{1.408259in}}%
\pgfpathlineto{\pgfqpoint{1.856973in}{1.125333in}}%
\pgfpathlineto{\pgfqpoint{1.962343in}{1.012931in}}%
\pgfpathlineto{\pgfqpoint{2.042505in}{0.929250in}}%
\pgfpathlineto{\pgfqpoint{2.179468in}{0.789333in}}%
\pgfpathlineto{\pgfqpoint{2.445744in}{0.528000in}}%
\pgfpathmoveto{\pgfqpoint{4.768000in}{1.143461in}}%
\pgfpathlineto{\pgfqpoint{4.567596in}{1.401767in}}%
\pgfpathlineto{\pgfqpoint{4.458755in}{1.536000in}}%
\pgfpathlineto{\pgfqpoint{4.359791in}{1.654894in}}%
\pgfpathlineto{\pgfqpoint{4.237777in}{1.797333in}}%
\pgfpathlineto{\pgfqpoint{4.166788in}{1.878423in}}%
\pgfpathlineto{\pgfqpoint{4.038754in}{2.021333in}}%
\pgfpathlineto{\pgfqpoint{3.966384in}{2.100402in}}%
\pgfpathlineto{\pgfqpoint{3.830595in}{2.245333in}}%
\pgfpathlineto{\pgfqpoint{3.723116in}{2.357333in}}%
\pgfpathlineto{\pgfqpoint{3.645737in}{2.436363in}}%
\pgfpathlineto{\pgfqpoint{3.525495in}{2.556690in}}%
\pgfpathlineto{\pgfqpoint{3.414709in}{2.664808in}}%
\pgfpathlineto{\pgfqpoint{3.365172in}{2.712707in}}%
\pgfpathlineto{\pgfqpoint{3.227533in}{2.842667in}}%
\pgfpathlineto{\pgfqpoint{3.105761in}{2.954667in}}%
\pgfpathlineto{\pgfqpoint{2.844121in}{3.186014in}}%
\pgfpathlineto{\pgfqpoint{2.720780in}{3.290667in}}%
\pgfpathlineto{\pgfqpoint{2.443313in}{3.515610in}}%
\pgfpathlineto{\pgfqpoint{2.162747in}{3.727806in}}%
\pgfpathlineto{\pgfqpoint{2.042505in}{3.813981in}}%
\pgfpathlineto{\pgfqpoint{1.802020in}{3.976621in}}%
\pgfpathlineto{\pgfqpoint{1.706815in}{4.037333in}}%
\pgfpathlineto{\pgfqpoint{1.601616in}{4.101873in}}%
\pgfpathlineto{\pgfqpoint{1.561535in}{4.125673in}}%
\pgfpathlineto{\pgfqpoint{1.481374in}{4.171998in}}%
\pgfpathlineto{\pgfqpoint{1.387327in}{4.224000in}}%
\pgfpathlineto{\pgfqpoint{1.372907in}{4.224000in}}%
\pgfpathlineto{\pgfqpoint{1.441863in}{4.186667in}}%
\pgfpathlineto{\pgfqpoint{1.561535in}{4.118305in}}%
\pgfpathlineto{\pgfqpoint{1.681778in}{4.045749in}}%
\pgfpathlineto{\pgfqpoint{1.802020in}{3.969567in}}%
\pgfpathlineto{\pgfqpoint{1.869352in}{3.925333in}}%
\pgfpathlineto{\pgfqpoint{1.962343in}{3.862629in}}%
\pgfpathlineto{\pgfqpoint{2.038598in}{3.809694in}}%
\pgfpathlineto{\pgfqpoint{2.082586in}{3.778764in}}%
\pgfpathlineto{\pgfqpoint{2.202828in}{3.691699in}}%
\pgfpathlineto{\pgfqpoint{2.323071in}{3.601720in}}%
\pgfpathlineto{\pgfqpoint{2.417942in}{3.528368in}}%
\pgfpathlineto{\pgfqpoint{2.483498in}{3.477333in}}%
\pgfpathlineto{\pgfqpoint{2.763960in}{3.247789in}}%
\pgfpathlineto{\pgfqpoint{2.845119in}{3.178667in}}%
\pgfpathlineto{\pgfqpoint{2.973495in}{3.066667in}}%
\pgfpathlineto{\pgfqpoint{3.260428in}{2.805333in}}%
\pgfpathlineto{\pgfqpoint{3.547821in}{2.527463in}}%
\pgfpathlineto{\pgfqpoint{3.606415in}{2.469333in}}%
\pgfpathlineto{\pgfqpoint{3.725899in}{2.347568in}}%
\pgfpathlineto{\pgfqpoint{3.859371in}{2.208000in}}%
\pgfpathlineto{\pgfqpoint{3.966384in}{2.093358in}}%
\pgfpathlineto{\pgfqpoint{4.099590in}{1.946667in}}%
\pgfpathlineto{\pgfqpoint{4.184198in}{1.850884in}}%
\pgfpathlineto{\pgfqpoint{4.231302in}{1.797333in}}%
\pgfpathlineto{\pgfqpoint{4.329515in}{1.683094in}}%
\pgfpathlineto{\pgfqpoint{4.452373in}{1.536000in}}%
\pgfpathlineto{\pgfqpoint{4.543322in}{1.424000in}}%
\pgfpathlineto{\pgfqpoint{4.632111in}{1.312000in}}%
\pgfpathlineto{\pgfqpoint{4.727919in}{1.187859in}}%
\pgfpathlineto{\pgfqpoint{4.768000in}{1.134818in}}%
\pgfpathlineto{\pgfqpoint{4.768000in}{1.134818in}}%
\pgfusepath{fill}%
\end{pgfscope}%
\begin{pgfscope}%
\pgfpathrectangle{\pgfqpoint{0.800000in}{0.528000in}}{\pgfqpoint{3.968000in}{3.696000in}}%
\pgfusepath{clip}%
\pgfsetbuttcap%
\pgfsetroundjoin%
\definecolor{currentfill}{rgb}{0.279574,0.170599,0.479997}%
\pgfsetfillcolor{currentfill}%
\pgfsetlinewidth{0.000000pt}%
\definecolor{currentstroke}{rgb}{0.000000,0.000000,0.000000}%
\pgfsetstrokecolor{currentstroke}%
\pgfsetdash{}{0pt}%
\pgfpathmoveto{\pgfqpoint{2.445744in}{0.528000in}}%
\pgfpathlineto{\pgfqpoint{2.363152in}{0.607551in}}%
\pgfpathlineto{\pgfqpoint{2.242909in}{0.725774in}}%
\pgfpathlineto{\pgfqpoint{2.142563in}{0.826667in}}%
\pgfpathlineto{\pgfqpoint{2.033437in}{0.938667in}}%
\pgfpathlineto{\pgfqpoint{1.959118in}{1.016338in}}%
\pgfpathlineto{\pgfqpoint{1.822460in}{1.162667in}}%
\pgfpathlineto{\pgfqpoint{1.720298in}{1.274667in}}%
\pgfpathlineto{\pgfqpoint{1.612183in}{1.396509in}}%
\pgfpathlineto{\pgfqpoint{1.555271in}{1.461333in}}%
\pgfpathlineto{\pgfqpoint{1.481374in}{1.547266in}}%
\pgfpathlineto{\pgfqpoint{1.361131in}{1.690347in}}%
\pgfpathlineto{\pgfqpoint{1.120646in}{1.990603in}}%
\pgfpathlineto{\pgfqpoint{1.012030in}{2.133333in}}%
\pgfpathlineto{\pgfqpoint{0.920242in}{2.257907in}}%
\pgfpathlineto{\pgfqpoint{0.822901in}{2.394667in}}%
\pgfpathlineto{\pgfqpoint{0.800000in}{2.427466in}}%
\pgfpathlineto{\pgfqpoint{0.800000in}{2.417957in}}%
\pgfpathlineto{\pgfqpoint{0.977727in}{2.170667in}}%
\pgfpathlineto{\pgfqpoint{1.061962in}{2.058667in}}%
\pgfpathlineto{\pgfqpoint{1.120646in}{1.982221in}}%
\pgfpathlineto{\pgfqpoint{1.240889in}{1.829970in}}%
\pgfpathlineto{\pgfqpoint{1.496911in}{1.521528in}}%
\pgfpathlineto{\pgfqpoint{1.614180in}{1.386667in}}%
\pgfpathlineto{\pgfqpoint{1.721859in}{1.265850in}}%
\pgfpathlineto{\pgfqpoint{1.815962in}{1.162667in}}%
\pgfpathlineto{\pgfqpoint{1.922263in}{1.048429in}}%
\pgfpathlineto{\pgfqpoint{2.062958in}{0.901333in}}%
\pgfpathlineto{\pgfqpoint{2.361316in}{0.602667in}}%
\pgfpathlineto{\pgfqpoint{2.438856in}{0.528000in}}%
\pgfpathlineto{\pgfqpoint{2.443313in}{0.528000in}}%
\pgfpathmoveto{\pgfqpoint{4.768000in}{1.152104in}}%
\pgfpathlineto{\pgfqpoint{4.586039in}{1.386667in}}%
\pgfpathlineto{\pgfqpoint{4.509261in}{1.481664in}}%
\pgfpathlineto{\pgfqpoint{4.465137in}{1.536000in}}%
\pgfpathlineto{\pgfqpoint{4.367192in}{1.653667in}}%
\pgfpathlineto{\pgfqpoint{4.244251in}{1.797333in}}%
\pgfpathlineto{\pgfqpoint{4.137017in}{1.918936in}}%
\pgfpathlineto{\pgfqpoint{4.079002in}{1.984000in}}%
\pgfpathlineto{\pgfqpoint{3.966384in}{2.107377in}}%
\pgfpathlineto{\pgfqpoint{3.837182in}{2.245333in}}%
\pgfpathlineto{\pgfqpoint{3.725899in}{2.361270in}}%
\pgfpathlineto{\pgfqpoint{3.645737in}{2.443067in}}%
\pgfpathlineto{\pgfqpoint{3.507258in}{2.581333in}}%
\pgfpathlineto{\pgfqpoint{3.234611in}{2.842667in}}%
\pgfpathlineto{\pgfqpoint{3.112980in}{2.954667in}}%
\pgfpathlineto{\pgfqpoint{2.816693in}{3.216000in}}%
\pgfpathlineto{\pgfqpoint{2.723879in}{3.294601in}}%
\pgfpathlineto{\pgfqpoint{2.619774in}{3.380365in}}%
\pgfpathlineto{\pgfqpoint{2.563556in}{3.426400in}}%
\pgfpathlineto{\pgfqpoint{2.469185in}{3.501432in}}%
\pgfpathlineto{\pgfqpoint{2.403232in}{3.553500in}}%
\pgfpathlineto{\pgfqpoint{2.122667in}{3.763604in}}%
\pgfpathlineto{\pgfqpoint{1.999788in}{3.850667in}}%
\pgfpathlineto{\pgfqpoint{1.761939in}{4.009584in}}%
\pgfpathlineto{\pgfqpoint{1.658287in}{4.074667in}}%
\pgfpathlineto{\pgfqpoint{1.561535in}{4.133042in}}%
\pgfpathlineto{\pgfqpoint{1.468718in}{4.186667in}}%
\pgfpathlineto{\pgfqpoint{1.401212in}{4.224000in}}%
\pgfpathlineto{\pgfqpoint{1.387327in}{4.224000in}}%
\pgfpathlineto{\pgfqpoint{1.455290in}{4.186667in}}%
\pgfpathlineto{\pgfqpoint{1.561535in}{4.125673in}}%
\pgfpathlineto{\pgfqpoint{1.646612in}{4.074667in}}%
\pgfpathlineto{\pgfqpoint{1.842101in}{3.950413in}}%
\pgfpathlineto{\pgfqpoint{1.935312in}{3.888000in}}%
\pgfpathlineto{\pgfqpoint{2.043420in}{3.813333in}}%
\pgfpathlineto{\pgfqpoint{2.298952in}{3.626667in}}%
\pgfpathlineto{\pgfqpoint{2.403232in}{3.546842in}}%
\pgfpathlineto{\pgfqpoint{2.508655in}{3.463530in}}%
\pgfpathlineto{\pgfqpoint{2.563556in}{3.419827in}}%
\pgfpathlineto{\pgfqpoint{2.683798in}{3.321455in}}%
\pgfpathlineto{\pgfqpoint{2.785293in}{3.235871in}}%
\pgfpathlineto{\pgfqpoint{2.852637in}{3.178667in}}%
\pgfpathlineto{\pgfqpoint{2.980861in}{3.066667in}}%
\pgfpathlineto{\pgfqpoint{3.267460in}{2.805333in}}%
\pgfpathlineto{\pgfqpoint{3.538292in}{2.544000in}}%
\pgfpathlineto{\pgfqpoint{3.613060in}{2.469333in}}%
\pgfpathlineto{\pgfqpoint{3.725899in}{2.354470in}}%
\pgfpathlineto{\pgfqpoint{3.865919in}{2.208000in}}%
\pgfpathlineto{\pgfqpoint{3.970443in}{2.096000in}}%
\pgfpathlineto{\pgfqpoint{4.086626in}{1.968338in}}%
\pgfpathlineto{\pgfqpoint{4.206869in}{1.832873in}}%
\pgfpathlineto{\pgfqpoint{4.302112in}{1.722667in}}%
\pgfpathlineto{\pgfqpoint{4.407273in}{1.598178in}}%
\pgfpathlineto{\pgfqpoint{4.527515in}{1.451684in}}%
\pgfpathlineto{\pgfqpoint{4.727919in}{1.196483in}}%
\pgfpathlineto{\pgfqpoint{4.768000in}{1.143461in}}%
\pgfpathlineto{\pgfqpoint{4.768000in}{1.143461in}}%
\pgfusepath{fill}%
\end{pgfscope}%
\begin{pgfscope}%
\pgfpathrectangle{\pgfqpoint{0.800000in}{0.528000in}}{\pgfqpoint{3.968000in}{3.696000in}}%
\pgfusepath{clip}%
\pgfsetbuttcap%
\pgfsetroundjoin%
\definecolor{currentfill}{rgb}{0.279574,0.170599,0.479997}%
\pgfsetfillcolor{currentfill}%
\pgfsetlinewidth{0.000000pt}%
\definecolor{currentstroke}{rgb}{0.000000,0.000000,0.000000}%
\pgfsetstrokecolor{currentstroke}%
\pgfsetdash{}{0pt}%
\pgfpathmoveto{\pgfqpoint{2.438856in}{0.528000in}}%
\pgfpathlineto{\pgfqpoint{2.361316in}{0.602667in}}%
\pgfpathlineto{\pgfqpoint{2.242909in}{0.719026in}}%
\pgfpathlineto{\pgfqpoint{2.162747in}{0.799420in}}%
\pgfpathlineto{\pgfqpoint{2.026882in}{0.938667in}}%
\pgfpathlineto{\pgfqpoint{1.955486in}{1.013333in}}%
\pgfpathlineto{\pgfqpoint{1.842101in}{1.134319in}}%
\pgfpathlineto{\pgfqpoint{1.713915in}{1.274667in}}%
\pgfpathlineto{\pgfqpoint{1.641697in}{1.355463in}}%
\pgfpathlineto{\pgfqpoint{1.516574in}{1.498667in}}%
\pgfpathlineto{\pgfqpoint{1.421305in}{1.610667in}}%
\pgfpathlineto{\pgfqpoint{1.342650in}{1.705452in}}%
\pgfpathlineto{\pgfqpoint{1.297589in}{1.760000in}}%
\pgfpathlineto{\pgfqpoint{1.200808in}{1.880192in}}%
\pgfpathlineto{\pgfqpoint{1.090512in}{2.021333in}}%
\pgfpathlineto{\pgfqpoint{1.000404in}{2.140124in}}%
\pgfpathlineto{\pgfqpoint{0.895805in}{2.282667in}}%
\pgfpathlineto{\pgfqpoint{0.857680in}{2.336393in}}%
\pgfpathlineto{\pgfqpoint{0.816262in}{2.394667in}}%
\pgfpathlineto{\pgfqpoint{0.800000in}{2.417957in}}%
\pgfpathlineto{\pgfqpoint{0.800000in}{2.408448in}}%
\pgfpathlineto{\pgfqpoint{0.960323in}{2.185366in}}%
\pgfpathlineto{\pgfqpoint{1.142051in}{1.946667in}}%
\pgfpathlineto{\pgfqpoint{1.200955in}{1.871863in}}%
\pgfpathlineto{\pgfqpoint{1.321650in}{1.722667in}}%
\pgfpathlineto{\pgfqpoint{1.426684in}{1.597060in}}%
\pgfpathlineto{\pgfqpoint{1.481374in}{1.532232in}}%
\pgfpathlineto{\pgfqpoint{1.607737in}{1.386667in}}%
\pgfpathlineto{\pgfqpoint{1.721859in}{1.258766in}}%
\pgfpathlineto{\pgfqpoint{2.002424in}{0.957290in}}%
\pgfpathlineto{\pgfqpoint{2.107141in}{0.849538in}}%
\pgfpathlineto{\pgfqpoint{2.166035in}{0.789333in}}%
\pgfpathlineto{\pgfqpoint{2.261340in}{0.694501in}}%
\pgfpathlineto{\pgfqpoint{2.323071in}{0.633358in}}%
\pgfpathlineto{\pgfqpoint{2.432037in}{0.528000in}}%
\pgfpathmoveto{\pgfqpoint{4.768000in}{1.160747in}}%
\pgfpathlineto{\pgfqpoint{4.592443in}{1.386667in}}%
\pgfpathlineto{\pgfqpoint{4.512910in}{1.485063in}}%
\pgfpathlineto{\pgfqpoint{4.471520in}{1.536000in}}%
\pgfpathlineto{\pgfqpoint{4.367192in}{1.661168in}}%
\pgfpathlineto{\pgfqpoint{4.246949in}{1.801565in}}%
\pgfpathlineto{\pgfqpoint{4.140486in}{1.922167in}}%
\pgfpathlineto{\pgfqpoint{4.079498in}{1.990640in}}%
\pgfpathlineto{\pgfqpoint{3.948777in}{2.133333in}}%
\pgfpathlineto{\pgfqpoint{3.879014in}{2.208000in}}%
\pgfpathlineto{\pgfqpoint{3.765980in}{2.326624in}}%
\pgfpathlineto{\pgfqpoint{3.626350in}{2.469333in}}%
\pgfpathlineto{\pgfqpoint{3.514027in}{2.581333in}}%
\pgfpathlineto{\pgfqpoint{3.421617in}{2.671243in}}%
\pgfpathlineto{\pgfqpoint{3.360241in}{2.730667in}}%
\pgfpathlineto{\pgfqpoint{3.281524in}{2.805333in}}%
\pgfpathlineto{\pgfqpoint{3.161033in}{2.917333in}}%
\pgfpathlineto{\pgfqpoint{3.037486in}{3.029333in}}%
\pgfpathlineto{\pgfqpoint{2.736237in}{3.290667in}}%
\pgfpathlineto{\pgfqpoint{2.483394in}{3.497071in}}%
\pgfpathlineto{\pgfqpoint{2.385840in}{3.573134in}}%
\pgfpathlineto{\pgfqpoint{2.316568in}{3.626667in}}%
\pgfpathlineto{\pgfqpoint{2.062518in}{3.813333in}}%
\pgfpathlineto{\pgfqpoint{1.955581in}{3.888000in}}%
\pgfpathlineto{\pgfqpoint{1.842101in}{3.964496in}}%
\pgfpathlineto{\pgfqpoint{1.601616in}{4.116511in}}%
\pgfpathlineto{\pgfqpoint{1.521455in}{4.163894in}}%
\pgfpathlineto{\pgfqpoint{1.441293in}{4.209597in}}%
\pgfpathlineto{\pgfqpoint{1.415312in}{4.224000in}}%
\pgfpathlineto{\pgfqpoint{1.401716in}{4.224000in}}%
\pgfpathlineto{\pgfqpoint{1.521455in}{4.156539in}}%
\pgfpathlineto{\pgfqpoint{1.561535in}{4.133042in}}%
\pgfpathlineto{\pgfqpoint{1.658287in}{4.074667in}}%
\pgfpathlineto{\pgfqpoint{1.861283in}{3.944799in}}%
\pgfpathlineto{\pgfqpoint{1.962343in}{3.876466in}}%
\pgfpathlineto{\pgfqpoint{2.042505in}{3.820746in}}%
\pgfpathlineto{\pgfqpoint{2.162747in}{3.734605in}}%
\pgfpathlineto{\pgfqpoint{2.443313in}{3.522151in}}%
\pgfpathlineto{\pgfqpoint{2.563556in}{3.426400in}}%
\pgfpathlineto{\pgfqpoint{2.683869in}{3.328000in}}%
\pgfpathlineto{\pgfqpoint{2.788974in}{3.239300in}}%
\pgfpathlineto{\pgfqpoint{2.844121in}{3.192501in}}%
\pgfpathlineto{\pgfqpoint{2.934654in}{3.113660in}}%
\pgfpathlineto{\pgfqpoint{2.996442in}{3.059213in}}%
\pgfpathlineto{\pgfqpoint{3.044525in}{3.016501in}}%
\pgfpathlineto{\pgfqpoint{3.124687in}{2.944026in}}%
\pgfpathlineto{\pgfqpoint{3.244929in}{2.833054in}}%
\pgfpathlineto{\pgfqpoint{3.365172in}{2.719334in}}%
\pgfpathlineto{\pgfqpoint{3.477020in}{2.610848in}}%
\pgfpathlineto{\pgfqpoint{3.525495in}{2.563361in}}%
\pgfpathlineto{\pgfqpoint{3.656640in}{2.432000in}}%
\pgfpathlineto{\pgfqpoint{3.765980in}{2.319882in}}%
\pgfpathlineto{\pgfqpoint{3.907506in}{2.170667in}}%
\pgfpathlineto{\pgfqpoint{4.011209in}{2.058667in}}%
\pgfpathlineto{\pgfqpoint{4.086626in}{1.975541in}}%
\pgfpathlineto{\pgfqpoint{4.211681in}{1.834667in}}%
\pgfpathlineto{\pgfqpoint{4.308510in}{1.722667in}}%
\pgfpathlineto{\pgfqpoint{4.407273in}{1.605913in}}%
\pgfpathlineto{\pgfqpoint{4.527515in}{1.459700in}}%
\pgfpathlineto{\pgfqpoint{4.760026in}{1.162667in}}%
\pgfpathlineto{\pgfqpoint{4.768000in}{1.152104in}}%
\pgfpathlineto{\pgfqpoint{4.768000in}{1.152104in}}%
\pgfusepath{fill}%
\end{pgfscope}%
\begin{pgfscope}%
\pgfpathrectangle{\pgfqpoint{0.800000in}{0.528000in}}{\pgfqpoint{3.968000in}{3.696000in}}%
\pgfusepath{clip}%
\pgfsetbuttcap%
\pgfsetroundjoin%
\definecolor{currentfill}{rgb}{0.279574,0.170599,0.479997}%
\pgfsetfillcolor{currentfill}%
\pgfsetlinewidth{0.000000pt}%
\definecolor{currentstroke}{rgb}{0.000000,0.000000,0.000000}%
\pgfsetstrokecolor{currentstroke}%
\pgfsetdash{}{0pt}%
\pgfpathmoveto{\pgfqpoint{2.432037in}{0.528000in}}%
\pgfpathlineto{\pgfqpoint{2.354582in}{0.602667in}}%
\pgfpathlineto{\pgfqpoint{2.240568in}{0.714667in}}%
\pgfpathlineto{\pgfqpoint{2.162747in}{0.792648in}}%
\pgfpathlineto{\pgfqpoint{2.020328in}{0.938667in}}%
\pgfpathlineto{\pgfqpoint{1.917760in}{1.046473in}}%
\pgfpathlineto{\pgfqpoint{1.878667in}{1.088000in}}%
\pgfpathlineto{\pgfqpoint{1.761939in}{1.214584in}}%
\pgfpathlineto{\pgfqpoint{1.510243in}{1.498667in}}%
\pgfpathlineto{\pgfqpoint{1.401212in}{1.626968in}}%
\pgfpathlineto{\pgfqpoint{1.280970in}{1.772535in}}%
\pgfpathlineto{\pgfqpoint{1.066099in}{2.045192in}}%
\pgfpathlineto{\pgfqpoint{1.027157in}{2.096000in}}%
\pgfpathlineto{\pgfqpoint{0.943655in}{2.208000in}}%
\pgfpathlineto{\pgfqpoint{0.862466in}{2.320000in}}%
\pgfpathlineto{\pgfqpoint{0.805909in}{2.400170in}}%
\pgfpathlineto{\pgfqpoint{0.800000in}{2.408448in}}%
\pgfpathlineto{\pgfqpoint{0.800000in}{2.398939in}}%
\pgfpathlineto{\pgfqpoint{0.937179in}{2.208000in}}%
\pgfpathlineto{\pgfqpoint{1.020741in}{2.096000in}}%
\pgfpathlineto{\pgfqpoint{1.080566in}{2.017499in}}%
\pgfpathlineto{\pgfqpoint{1.200808in}{1.864145in}}%
\pgfpathlineto{\pgfqpoint{1.321051in}{1.715748in}}%
\pgfpathlineto{\pgfqpoint{1.568627in}{1.424000in}}%
\pgfpathlineto{\pgfqpoint{1.656032in}{1.325353in}}%
\pgfpathlineto{\pgfqpoint{1.701148in}{1.274667in}}%
\pgfpathlineto{\pgfqpoint{1.808720in}{1.156426in}}%
\pgfpathlineto{\pgfqpoint{1.942534in}{1.013333in}}%
\pgfpathlineto{\pgfqpoint{2.065544in}{0.885460in}}%
\pgfpathlineto{\pgfqpoint{2.122667in}{0.826539in}}%
\pgfpathlineto{\pgfqpoint{2.242909in}{0.705756in}}%
\pgfpathlineto{\pgfqpoint{2.355422in}{0.595467in}}%
\pgfpathlineto{\pgfqpoint{2.403232in}{0.549079in}}%
\pgfpathlineto{\pgfqpoint{2.425218in}{0.528000in}}%
\pgfpathmoveto{\pgfqpoint{4.768000in}{1.169178in}}%
\pgfpathlineto{\pgfqpoint{4.657845in}{1.312000in}}%
\pgfpathlineto{\pgfqpoint{4.567596in}{1.425807in}}%
\pgfpathlineto{\pgfqpoint{4.477902in}{1.536000in}}%
\pgfpathlineto{\pgfqpoint{4.376840in}{1.656987in}}%
\pgfpathlineto{\pgfqpoint{4.321307in}{1.722667in}}%
\pgfpathlineto{\pgfqpoint{4.234353in}{1.822933in}}%
\pgfpathlineto{\pgfqpoint{4.191586in}{1.872000in}}%
\pgfpathlineto{\pgfqpoint{4.086626in}{1.989788in}}%
\pgfpathlineto{\pgfqpoint{3.955246in}{2.133333in}}%
\pgfpathlineto{\pgfqpoint{3.880759in}{2.213088in}}%
\pgfpathlineto{\pgfqpoint{3.742767in}{2.357333in}}%
\pgfpathlineto{\pgfqpoint{3.482834in}{2.618667in}}%
\pgfpathlineto{\pgfqpoint{3.365172in}{2.732540in}}%
\pgfpathlineto{\pgfqpoint{3.285010in}{2.808583in}}%
\pgfpathlineto{\pgfqpoint{3.164768in}{2.920406in}}%
\pgfpathlineto{\pgfqpoint{3.044525in}{3.029575in}}%
\pgfpathlineto{\pgfqpoint{2.941765in}{3.120284in}}%
\pgfpathlineto{\pgfqpoint{2.875194in}{3.178667in}}%
\pgfpathlineto{\pgfqpoint{2.603636in}{3.406971in}}%
\pgfpathlineto{\pgfqpoint{2.476895in}{3.508614in}}%
\pgfpathlineto{\pgfqpoint{2.421878in}{3.552000in}}%
\pgfpathlineto{\pgfqpoint{2.323071in}{3.628370in}}%
\pgfpathlineto{\pgfqpoint{2.072067in}{3.813333in}}%
\pgfpathlineto{\pgfqpoint{1.953151in}{3.896563in}}%
\pgfpathlineto{\pgfqpoint{1.842101in}{3.971379in}}%
\pgfpathlineto{\pgfqpoint{1.601616in}{4.123693in}}%
\pgfpathlineto{\pgfqpoint{1.521455in}{4.171249in}}%
\pgfpathlineto{\pgfqpoint{1.428907in}{4.224000in}}%
\pgfpathlineto{\pgfqpoint{1.415312in}{4.224000in}}%
\pgfpathlineto{\pgfqpoint{1.521455in}{4.163894in}}%
\pgfpathlineto{\pgfqpoint{1.561535in}{4.140411in}}%
\pgfpathlineto{\pgfqpoint{1.641697in}{4.092085in}}%
\pgfpathlineto{\pgfqpoint{1.729543in}{4.037333in}}%
\pgfpathlineto{\pgfqpoint{1.844846in}{3.962667in}}%
\pgfpathlineto{\pgfqpoint{1.962343in}{3.883384in}}%
\pgfpathlineto{\pgfqpoint{2.217023in}{3.701333in}}%
\pgfpathlineto{\pgfqpoint{2.323071in}{3.621740in}}%
\pgfpathlineto{\pgfqpoint{2.429539in}{3.539170in}}%
\pgfpathlineto{\pgfqpoint{2.483394in}{3.497071in}}%
\pgfpathlineto{\pgfqpoint{2.563556in}{3.432973in}}%
\pgfpathlineto{\pgfqpoint{2.691602in}{3.328000in}}%
\pgfpathlineto{\pgfqpoint{2.964364in}{3.094303in}}%
\pgfpathlineto{\pgfqpoint{3.084606in}{2.986964in}}%
\pgfpathlineto{\pgfqpoint{3.164768in}{2.913910in}}%
\pgfpathlineto{\pgfqpoint{3.285010in}{2.802059in}}%
\pgfpathlineto{\pgfqpoint{3.405253in}{2.687447in}}%
\pgfpathlineto{\pgfqpoint{3.499946in}{2.594869in}}%
\pgfpathlineto{\pgfqpoint{3.565576in}{2.530263in}}%
\pgfpathlineto{\pgfqpoint{3.699873in}{2.394667in}}%
\pgfpathlineto{\pgfqpoint{3.772356in}{2.320000in}}%
\pgfpathlineto{\pgfqpoint{3.886222in}{2.200348in}}%
\pgfpathlineto{\pgfqpoint{3.983305in}{2.096000in}}%
\pgfpathlineto{\pgfqpoint{4.086626in}{1.982743in}}%
\pgfpathlineto{\pgfqpoint{4.218021in}{1.834667in}}%
\pgfpathlineto{\pgfqpoint{4.302360in}{1.736946in}}%
\pgfpathlineto{\pgfqpoint{4.346699in}{1.685333in}}%
\pgfpathlineto{\pgfqpoint{4.447354in}{1.565344in}}%
\pgfpathlineto{\pgfqpoint{4.567596in}{1.417830in}}%
\pgfpathlineto{\pgfqpoint{4.768000in}{1.160747in}}%
\pgfpathlineto{\pgfqpoint{4.768000in}{1.162667in}}%
\pgfusepath{fill}%
\end{pgfscope}%
\begin{pgfscope}%
\pgfpathrectangle{\pgfqpoint{0.800000in}{0.528000in}}{\pgfqpoint{3.968000in}{3.696000in}}%
\pgfusepath{clip}%
\pgfsetbuttcap%
\pgfsetroundjoin%
\definecolor{currentfill}{rgb}{0.278826,0.175490,0.483397}%
\pgfsetfillcolor{currentfill}%
\pgfsetlinewidth{0.000000pt}%
\definecolor{currentstroke}{rgb}{0.000000,0.000000,0.000000}%
\pgfsetstrokecolor{currentstroke}%
\pgfsetdash{}{0pt}%
\pgfpathmoveto{\pgfqpoint{2.425218in}{0.528000in}}%
\pgfpathlineto{\pgfqpoint{2.159413in}{0.789333in}}%
\pgfpathlineto{\pgfqpoint{2.042505in}{0.908832in}}%
\pgfpathlineto{\pgfqpoint{1.942534in}{1.013333in}}%
\pgfpathlineto{\pgfqpoint{1.837489in}{1.125333in}}%
\pgfpathlineto{\pgfqpoint{1.721859in}{1.251681in}}%
\pgfpathlineto{\pgfqpoint{1.601302in}{1.386667in}}%
\pgfpathlineto{\pgfqpoint{1.503913in}{1.498667in}}%
\pgfpathlineto{\pgfqpoint{1.401212in}{1.619368in}}%
\pgfpathlineto{\pgfqpoint{1.280970in}{1.764665in}}%
\pgfpathlineto{\pgfqpoint{1.049055in}{2.058667in}}%
\pgfpathlineto{\pgfqpoint{0.960323in}{2.176572in}}%
\pgfpathlineto{\pgfqpoint{0.855929in}{2.320000in}}%
\pgfpathlineto{\pgfqpoint{0.817916in}{2.374021in}}%
\pgfpathlineto{\pgfqpoint{0.800000in}{2.398939in}}%
\pgfpathlineto{\pgfqpoint{0.800000in}{2.389611in}}%
\pgfpathlineto{\pgfqpoint{1.000404in}{2.114429in}}%
\pgfpathlineto{\pgfqpoint{1.100194in}{1.984000in}}%
\pgfpathlineto{\pgfqpoint{1.160727in}{1.906591in}}%
\pgfpathlineto{\pgfqpoint{1.280970in}{1.756887in}}%
\pgfpathlineto{\pgfqpoint{1.370931in}{1.648000in}}%
\pgfpathlineto{\pgfqpoint{1.481374in}{1.517512in}}%
\pgfpathlineto{\pgfqpoint{1.601616in}{1.379188in}}%
\pgfpathlineto{\pgfqpoint{1.728441in}{1.237333in}}%
\pgfpathlineto{\pgfqpoint{1.802020in}{1.156791in}}%
\pgfpathlineto{\pgfqpoint{1.936058in}{1.013333in}}%
\pgfpathlineto{\pgfqpoint{2.024113in}{0.921536in}}%
\pgfpathlineto{\pgfqpoint{2.082586in}{0.860808in}}%
\pgfpathlineto{\pgfqpoint{2.177039in}{0.765312in}}%
\pgfpathlineto{\pgfqpoint{2.227346in}{0.714667in}}%
\pgfpathlineto{\pgfqpoint{2.341113in}{0.602667in}}%
\pgfpathlineto{\pgfqpoint{2.418399in}{0.528000in}}%
\pgfpathmoveto{\pgfqpoint{4.768000in}{1.177547in}}%
\pgfpathlineto{\pgfqpoint{4.664174in}{1.312000in}}%
\pgfpathlineto{\pgfqpoint{4.567596in}{1.433602in}}%
\pgfpathlineto{\pgfqpoint{4.327111in}{1.723343in}}%
\pgfpathlineto{\pgfqpoint{4.219847in}{1.846755in}}%
\pgfpathlineto{\pgfqpoint{4.165016in}{1.909333in}}%
\pgfpathlineto{\pgfqpoint{4.064386in}{2.021333in}}%
\pgfpathlineto{\pgfqpoint{3.961716in}{2.133333in}}%
\pgfpathlineto{\pgfqpoint{3.886222in}{2.214089in}}%
\pgfpathlineto{\pgfqpoint{3.749292in}{2.357333in}}%
\pgfpathlineto{\pgfqpoint{3.639639in}{2.469333in}}%
\pgfpathlineto{\pgfqpoint{3.525495in}{2.583323in}}%
\pgfpathlineto{\pgfqpoint{3.428526in}{2.677678in}}%
\pgfpathlineto{\pgfqpoint{3.365172in}{2.739005in}}%
\pgfpathlineto{\pgfqpoint{3.285010in}{2.815027in}}%
\pgfpathlineto{\pgfqpoint{3.164768in}{2.926819in}}%
\pgfpathlineto{\pgfqpoint{3.044525in}{3.035957in}}%
\pgfpathlineto{\pgfqpoint{2.945320in}{3.123596in}}%
\pgfpathlineto{\pgfqpoint{2.882713in}{3.178667in}}%
\pgfpathlineto{\pgfqpoint{2.603636in}{3.413395in}}%
\pgfpathlineto{\pgfqpoint{2.477704in}{3.514667in}}%
\pgfpathlineto{\pgfqpoint{2.349715in}{3.614151in}}%
\pgfpathlineto{\pgfqpoint{2.282990in}{3.665354in}}%
\pgfpathlineto{\pgfqpoint{2.028796in}{3.850667in}}%
\pgfpathlineto{\pgfqpoint{1.921054in}{3.925333in}}%
\pgfpathlineto{\pgfqpoint{1.681778in}{4.081599in}}%
\pgfpathlineto{\pgfqpoint{1.601616in}{4.130875in}}%
\pgfpathlineto{\pgfqpoint{1.507522in}{4.186667in}}%
\pgfpathlineto{\pgfqpoint{1.441293in}{4.224000in}}%
\pgfpathlineto{\pgfqpoint{1.428907in}{4.224000in}}%
\pgfpathlineto{\pgfqpoint{1.494813in}{4.186667in}}%
\pgfpathlineto{\pgfqpoint{1.601616in}{4.123693in}}%
\pgfpathlineto{\pgfqpoint{1.721859in}{4.049266in}}%
\pgfpathlineto{\pgfqpoint{1.776728in}{4.013775in}}%
\pgfpathlineto{\pgfqpoint{1.842101in}{3.971379in}}%
\pgfpathlineto{\pgfqpoint{1.965578in}{3.888000in}}%
\pgfpathlineto{\pgfqpoint{2.225976in}{3.701333in}}%
\pgfpathlineto{\pgfqpoint{2.325293in}{3.626667in}}%
\pgfpathlineto{\pgfqpoint{2.608892in}{3.402667in}}%
\pgfpathlineto{\pgfqpoint{2.884202in}{3.170883in}}%
\pgfpathlineto{\pgfqpoint{3.004444in}{3.065349in}}%
\pgfpathlineto{\pgfqpoint{3.105488in}{2.974117in}}%
\pgfpathlineto{\pgfqpoint{3.168103in}{2.917333in}}%
\pgfpathlineto{\pgfqpoint{3.288454in}{2.805333in}}%
\pgfpathlineto{\pgfqpoint{3.406011in}{2.693333in}}%
\pgfpathlineto{\pgfqpoint{3.503384in}{2.598071in}}%
\pgfpathlineto{\pgfqpoint{3.565576in}{2.536945in}}%
\pgfpathlineto{\pgfqpoint{3.706438in}{2.394667in}}%
\pgfpathlineto{\pgfqpoint{3.814665in}{2.282667in}}%
\pgfpathlineto{\pgfqpoint{3.926303in}{2.164483in}}%
\pgfpathlineto{\pgfqpoint{4.034222in}{2.047188in}}%
\pgfpathlineto{\pgfqpoint{4.091842in}{1.984000in}}%
\pgfpathlineto{\pgfqpoint{4.206869in}{1.854646in}}%
\pgfpathlineto{\pgfqpoint{4.447354in}{1.573093in}}%
\pgfpathlineto{\pgfqpoint{4.569050in}{1.424000in}}%
\pgfpathlineto{\pgfqpoint{4.670286in}{1.295650in}}%
\pgfpathlineto{\pgfqpoint{4.715855in}{1.237333in}}%
\pgfpathlineto{\pgfqpoint{4.768000in}{1.169178in}}%
\pgfpathlineto{\pgfqpoint{4.768000in}{1.169178in}}%
\pgfusepath{fill}%
\end{pgfscope}%
\begin{pgfscope}%
\pgfpathrectangle{\pgfqpoint{0.800000in}{0.528000in}}{\pgfqpoint{3.968000in}{3.696000in}}%
\pgfusepath{clip}%
\pgfsetbuttcap%
\pgfsetroundjoin%
\definecolor{currentfill}{rgb}{0.278826,0.175490,0.483397}%
\pgfsetfillcolor{currentfill}%
\pgfsetlinewidth{0.000000pt}%
\definecolor{currentstroke}{rgb}{0.000000,0.000000,0.000000}%
\pgfsetstrokecolor{currentstroke}%
\pgfsetdash{}{0pt}%
\pgfpathmoveto{\pgfqpoint{2.418399in}{0.528000in}}%
\pgfpathlineto{\pgfqpoint{2.152881in}{0.789333in}}%
\pgfpathlineto{\pgfqpoint{2.042505in}{0.902026in}}%
\pgfpathlineto{\pgfqpoint{1.936058in}{1.013333in}}%
\pgfpathlineto{\pgfqpoint{1.831126in}{1.125333in}}%
\pgfpathlineto{\pgfqpoint{1.721859in}{1.244596in}}%
\pgfpathlineto{\pgfqpoint{1.595029in}{1.386667in}}%
\pgfpathlineto{\pgfqpoint{1.497583in}{1.498667in}}%
\pgfpathlineto{\pgfqpoint{1.401212in}{1.611769in}}%
\pgfpathlineto{\pgfqpoint{1.278427in}{1.760000in}}%
\pgfpathlineto{\pgfqpoint{1.188248in}{1.872000in}}%
\pgfpathlineto{\pgfqpoint{1.100194in}{1.984000in}}%
\pgfpathlineto{\pgfqpoint{1.000404in}{2.114429in}}%
\pgfpathlineto{\pgfqpoint{0.822849in}{2.357333in}}%
\pgfpathlineto{\pgfqpoint{0.800000in}{2.389611in}}%
\pgfpathlineto{\pgfqpoint{0.800000in}{2.380432in}}%
\pgfpathlineto{\pgfqpoint{0.979829in}{2.133333in}}%
\pgfpathlineto{\pgfqpoint{1.054921in}{2.034780in}}%
\pgfpathlineto{\pgfqpoint{1.093838in}{1.984000in}}%
\pgfpathlineto{\pgfqpoint{1.160727in}{1.898674in}}%
\pgfpathlineto{\pgfqpoint{1.280970in}{1.749244in}}%
\pgfpathlineto{\pgfqpoint{1.523433in}{1.461333in}}%
\pgfpathlineto{\pgfqpoint{1.621753in}{1.349333in}}%
\pgfpathlineto{\pgfqpoint{1.722020in}{1.237333in}}%
\pgfpathlineto{\pgfqpoint{1.832900in}{1.116763in}}%
\pgfpathlineto{\pgfqpoint{1.894406in}{1.050667in}}%
\pgfpathlineto{\pgfqpoint{1.965000in}{0.976000in}}%
\pgfpathlineto{\pgfqpoint{2.082586in}{0.854183in}}%
\pgfpathlineto{\pgfqpoint{2.183408in}{0.752000in}}%
\pgfpathlineto{\pgfqpoint{2.296215in}{0.640000in}}%
\pgfpathlineto{\pgfqpoint{2.411580in}{0.528000in}}%
\pgfpathmoveto{\pgfqpoint{4.768000in}{1.185917in}}%
\pgfpathlineto{\pgfqpoint{4.670503in}{1.312000in}}%
\pgfpathlineto{\pgfqpoint{4.567596in}{1.441397in}}%
\pgfpathlineto{\pgfqpoint{4.330858in}{1.726157in}}%
\pgfpathlineto{\pgfqpoint{4.287030in}{1.777161in}}%
\pgfpathlineto{\pgfqpoint{4.036782in}{2.058667in}}%
\pgfpathlineto{\pgfqpoint{3.948501in}{2.154010in}}%
\pgfpathlineto{\pgfqpoint{3.886222in}{2.220861in}}%
\pgfpathlineto{\pgfqpoint{3.755817in}{2.357333in}}%
\pgfpathlineto{\pgfqpoint{3.664982in}{2.449925in}}%
\pgfpathlineto{\pgfqpoint{3.605657in}{2.510168in}}%
\pgfpathlineto{\pgfqpoint{3.485414in}{2.629198in}}%
\pgfpathlineto{\pgfqpoint{3.341517in}{2.768000in}}%
\pgfpathlineto{\pgfqpoint{3.222374in}{2.880000in}}%
\pgfpathlineto{\pgfqpoint{3.100340in}{2.992000in}}%
\pgfpathlineto{\pgfqpoint{2.844121in}{3.218389in}}%
\pgfpathlineto{\pgfqpoint{2.714801in}{3.328000in}}%
\pgfpathlineto{\pgfqpoint{2.578814in}{3.440000in}}%
\pgfpathlineto{\pgfqpoint{2.469038in}{3.528039in}}%
\pgfpathlineto{\pgfqpoint{2.342280in}{3.626667in}}%
\pgfpathlineto{\pgfqpoint{2.122667in}{3.790385in}}%
\pgfpathlineto{\pgfqpoint{2.002424in}{3.875907in}}%
\pgfpathlineto{\pgfqpoint{1.930912in}{3.925333in}}%
\pgfpathlineto{\pgfqpoint{1.842101in}{3.985143in}}%
\pgfpathlineto{\pgfqpoint{1.785528in}{4.021972in}}%
\pgfpathlineto{\pgfqpoint{1.721859in}{4.063324in}}%
\pgfpathlineto{\pgfqpoint{1.601616in}{4.138057in}}%
\pgfpathlineto{\pgfqpoint{1.520231in}{4.186667in}}%
\pgfpathlineto{\pgfqpoint{1.455297in}{4.224000in}}%
\pgfpathlineto{\pgfqpoint{1.442437in}{4.224000in}}%
\pgfpathlineto{\pgfqpoint{1.659227in}{4.095671in}}%
\pgfpathlineto{\pgfqpoint{1.761939in}{4.030708in}}%
\pgfpathlineto{\pgfqpoint{2.002424in}{3.869153in}}%
\pgfpathlineto{\pgfqpoint{2.127290in}{3.780306in}}%
\pgfpathlineto{\pgfqpoint{2.184438in}{3.738667in}}%
\pgfpathlineto{\pgfqpoint{2.403232in}{3.573091in}}%
\pgfpathlineto{\pgfqpoint{2.524636in}{3.477333in}}%
\pgfpathlineto{\pgfqpoint{2.804040in}{3.246252in}}%
\pgfpathlineto{\pgfqpoint{2.925588in}{3.141333in}}%
\pgfpathlineto{\pgfqpoint{3.027500in}{3.050808in}}%
\pgfpathlineto{\pgfqpoint{3.093288in}{2.992000in}}%
\pgfpathlineto{\pgfqpoint{3.389144in}{2.715662in}}%
\pgfpathlineto{\pgfqpoint{3.451265in}{2.656000in}}%
\pgfpathlineto{\pgfqpoint{3.565576in}{2.543627in}}%
\pgfpathlineto{\pgfqpoint{3.685818in}{2.422477in}}%
\pgfpathlineto{\pgfqpoint{3.785328in}{2.320000in}}%
\pgfpathlineto{\pgfqpoint{3.891951in}{2.208000in}}%
\pgfpathlineto{\pgfqpoint{3.982226in}{2.110756in}}%
\pgfpathlineto{\pgfqpoint{4.037716in}{2.050442in}}%
\pgfpathlineto{\pgfqpoint{4.092827in}{1.989776in}}%
\pgfpathlineto{\pgfqpoint{4.131719in}{1.946667in}}%
\pgfpathlineto{\pgfqpoint{4.237888in}{1.826226in}}%
\pgfpathlineto{\pgfqpoint{4.295561in}{1.760000in}}%
\pgfpathlineto{\pgfqpoint{4.380342in}{1.660249in}}%
\pgfpathlineto{\pgfqpoint{4.422271in}{1.610667in}}%
\pgfpathlineto{\pgfqpoint{4.527515in}{1.483088in}}%
\pgfpathlineto{\pgfqpoint{4.727919in}{1.230004in}}%
\pgfpathlineto{\pgfqpoint{4.768000in}{1.177547in}}%
\pgfpathlineto{\pgfqpoint{4.768000in}{1.177547in}}%
\pgfusepath{fill}%
\end{pgfscope}%
\begin{pgfscope}%
\pgfpathrectangle{\pgfqpoint{0.800000in}{0.528000in}}{\pgfqpoint{3.968000in}{3.696000in}}%
\pgfusepath{clip}%
\pgfsetbuttcap%
\pgfsetroundjoin%
\definecolor{currentfill}{rgb}{0.278826,0.175490,0.483397}%
\pgfsetfillcolor{currentfill}%
\pgfsetlinewidth{0.000000pt}%
\definecolor{currentstroke}{rgb}{0.000000,0.000000,0.000000}%
\pgfsetstrokecolor{currentstroke}%
\pgfsetdash{}{0pt}%
\pgfpathmoveto{\pgfqpoint{2.411580in}{0.528000in}}%
\pgfpathlineto{\pgfqpoint{2.146350in}{0.789333in}}%
\pgfpathlineto{\pgfqpoint{2.036742in}{0.901333in}}%
\pgfpathlineto{\pgfqpoint{1.945142in}{0.997311in}}%
\pgfpathlineto{\pgfqpoint{1.882182in}{1.063679in}}%
\pgfpathlineto{\pgfqpoint{1.756042in}{1.200000in}}%
\pgfpathlineto{\pgfqpoint{1.681778in}{1.282008in}}%
\pgfpathlineto{\pgfqpoint{1.555967in}{1.424000in}}%
\pgfpathlineto{\pgfqpoint{1.459271in}{1.536000in}}%
\pgfpathlineto{\pgfqpoint{1.361131in}{1.652134in}}%
\pgfpathlineto{\pgfqpoint{1.240889in}{1.798426in}}%
\pgfpathlineto{\pgfqpoint{1.120646in}{1.949510in}}%
\pgfpathlineto{\pgfqpoint{0.920242in}{2.213424in}}%
\pgfpathlineto{\pgfqpoint{0.800000in}{2.380432in}}%
\pgfpathlineto{\pgfqpoint{0.800000in}{2.371253in}}%
\pgfpathlineto{\pgfqpoint{0.973451in}{2.133333in}}%
\pgfpathlineto{\pgfqpoint{1.051195in}{2.031309in}}%
\pgfpathlineto{\pgfqpoint{1.087481in}{1.984000in}}%
\pgfpathlineto{\pgfqpoint{1.186540in}{1.858710in}}%
\pgfpathlineto{\pgfqpoint{1.240889in}{1.790737in}}%
\pgfpathlineto{\pgfqpoint{1.484922in}{1.498667in}}%
\pgfpathlineto{\pgfqpoint{1.582484in}{1.386667in}}%
\pgfpathlineto{\pgfqpoint{1.683143in}{1.273395in}}%
\pgfpathlineto{\pgfqpoint{1.818402in}{1.125333in}}%
\pgfpathlineto{\pgfqpoint{1.904191in}{1.033834in}}%
\pgfpathlineto{\pgfqpoint{1.962343in}{0.972057in}}%
\pgfpathlineto{\pgfqpoint{2.103063in}{0.826667in}}%
\pgfpathlineto{\pgfqpoint{2.384529in}{0.547912in}}%
\pgfpathlineto{\pgfqpoint{2.404762in}{0.528000in}}%
\pgfpathmoveto{\pgfqpoint{4.768000in}{1.194287in}}%
\pgfpathlineto{\pgfqpoint{4.676832in}{1.312000in}}%
\pgfpathlineto{\pgfqpoint{4.587865in}{1.424000in}}%
\pgfpathlineto{\pgfqpoint{4.509918in}{1.519609in}}%
\pgfpathlineto{\pgfqpoint{4.465914in}{1.573333in}}%
\pgfpathlineto{\pgfqpoint{4.387345in}{1.666772in}}%
\pgfpathlineto{\pgfqpoint{4.334286in}{1.729350in}}%
\pgfpathlineto{\pgfqpoint{4.275838in}{1.797333in}}%
\pgfpathlineto{\pgfqpoint{4.166788in}{1.921462in}}%
\pgfpathlineto{\pgfqpoint{3.886222in}{2.227634in}}%
\pgfpathlineto{\pgfqpoint{3.762342in}{2.357333in}}%
\pgfpathlineto{\pgfqpoint{3.668387in}{2.453097in}}%
\pgfpathlineto{\pgfqpoint{3.605657in}{2.516696in}}%
\pgfpathlineto{\pgfqpoint{3.464590in}{2.656000in}}%
\pgfpathlineto{\pgfqpoint{3.188987in}{2.917333in}}%
\pgfpathlineto{\pgfqpoint{2.897335in}{3.178667in}}%
\pgfpathlineto{\pgfqpoint{2.804040in}{3.259064in}}%
\pgfpathlineto{\pgfqpoint{2.677696in}{3.365333in}}%
\pgfpathlineto{\pgfqpoint{2.540555in}{3.477333in}}%
\pgfpathlineto{\pgfqpoint{2.424370in}{3.569644in}}%
\pgfpathlineto{\pgfqpoint{2.301894in}{3.664000in}}%
\pgfpathlineto{\pgfqpoint{2.082586in}{3.825877in}}%
\pgfpathlineto{\pgfqpoint{1.962343in}{3.910469in}}%
\pgfpathlineto{\pgfqpoint{1.842101in}{3.992026in}}%
\pgfpathlineto{\pgfqpoint{1.773088in}{4.037333in}}%
\pgfpathlineto{\pgfqpoint{1.681778in}{4.095632in}}%
\pgfpathlineto{\pgfqpoint{1.594813in}{4.149333in}}%
\pgfpathlineto{\pgfqpoint{1.468156in}{4.224000in}}%
\pgfpathlineto{\pgfqpoint{1.455297in}{4.224000in}}%
\pgfpathlineto{\pgfqpoint{1.644306in}{4.112000in}}%
\pgfpathlineto{\pgfqpoint{1.762561in}{4.037333in}}%
\pgfpathlineto{\pgfqpoint{2.002424in}{3.875907in}}%
\pgfpathlineto{\pgfqpoint{2.122667in}{3.790385in}}%
\pgfpathlineto{\pgfqpoint{2.198642in}{3.734768in}}%
\pgfpathlineto{\pgfqpoint{2.243847in}{3.701333in}}%
\pgfpathlineto{\pgfqpoint{2.523475in}{3.484671in}}%
\pgfpathlineto{\pgfqpoint{2.643717in}{3.386971in}}%
\pgfpathlineto{\pgfqpoint{2.763960in}{3.286719in}}%
\pgfpathlineto{\pgfqpoint{2.890047in}{3.178667in}}%
\pgfpathlineto{\pgfqpoint{3.017292in}{3.066667in}}%
\pgfpathlineto{\pgfqpoint{3.302114in}{2.805333in}}%
\pgfpathlineto{\pgfqpoint{3.419418in}{2.693333in}}%
\pgfpathlineto{\pgfqpoint{3.719567in}{2.394667in}}%
\pgfpathlineto{\pgfqpoint{4.002598in}{2.096000in}}%
\pgfpathlineto{\pgfqpoint{4.104480in}{1.984000in}}%
\pgfpathlineto{\pgfqpoint{4.206869in}{1.869129in}}%
\pgfpathlineto{\pgfqpoint{4.459662in}{1.573333in}}%
\pgfpathlineto{\pgfqpoint{4.567596in}{1.441397in}}%
\pgfpathlineto{\pgfqpoint{4.768000in}{1.185917in}}%
\pgfpathlineto{\pgfqpoint{4.768000in}{1.185917in}}%
\pgfusepath{fill}%
\end{pgfscope}%
\begin{pgfscope}%
\pgfpathrectangle{\pgfqpoint{0.800000in}{0.528000in}}{\pgfqpoint{3.968000in}{3.696000in}}%
\pgfusepath{clip}%
\pgfsetbuttcap%
\pgfsetroundjoin%
\definecolor{currentfill}{rgb}{0.278826,0.175490,0.483397}%
\pgfsetfillcolor{currentfill}%
\pgfsetlinewidth{0.000000pt}%
\definecolor{currentstroke}{rgb}{0.000000,0.000000,0.000000}%
\pgfsetstrokecolor{currentstroke}%
\pgfsetdash{}{0pt}%
\pgfpathmoveto{\pgfqpoint{2.404762in}{0.528000in}}%
\pgfpathlineto{\pgfqpoint{2.139818in}{0.789333in}}%
\pgfpathlineto{\pgfqpoint{2.030326in}{0.901333in}}%
\pgfpathlineto{\pgfqpoint{1.941742in}{0.994144in}}%
\pgfpathlineto{\pgfqpoint{1.882182in}{1.056826in}}%
\pgfpathlineto{\pgfqpoint{1.749753in}{1.200000in}}%
\pgfpathlineto{\pgfqpoint{1.681778in}{1.274911in}}%
\pgfpathlineto{\pgfqpoint{1.549730in}{1.424000in}}%
\pgfpathlineto{\pgfqpoint{1.465747in}{1.521445in}}%
\pgfpathlineto{\pgfqpoint{1.421228in}{1.573333in}}%
\pgfpathlineto{\pgfqpoint{1.348445in}{1.659817in}}%
\pgfpathlineto{\pgfqpoint{1.235543in}{1.797333in}}%
\pgfpathlineto{\pgfqpoint{1.146033in}{1.909333in}}%
\pgfpathlineto{\pgfqpoint{1.067964in}{2.009595in}}%
\pgfpathlineto{\pgfqpoint{1.029979in}{2.058667in}}%
\pgfpathlineto{\pgfqpoint{0.945560in}{2.170667in}}%
\pgfpathlineto{\pgfqpoint{0.863387in}{2.282667in}}%
\pgfpathlineto{\pgfqpoint{0.800000in}{2.371253in}}%
\pgfpathlineto{\pgfqpoint{0.800000in}{2.362074in}}%
\pgfpathlineto{\pgfqpoint{0.948016in}{2.159203in}}%
\pgfpathlineto{\pgfqpoint{1.000404in}{2.089170in}}%
\pgfpathlineto{\pgfqpoint{1.229337in}{1.797333in}}%
\pgfpathlineto{\pgfqpoint{1.303813in}{1.706611in}}%
\pgfpathlineto{\pgfqpoint{1.361131in}{1.637216in}}%
\pgfpathlineto{\pgfqpoint{1.481374in}{1.495520in}}%
\pgfpathlineto{\pgfqpoint{1.609134in}{1.349333in}}%
\pgfpathlineto{\pgfqpoint{1.721859in}{1.223707in}}%
\pgfpathlineto{\pgfqpoint{2.002424in}{0.923590in}}%
\pgfpathlineto{\pgfqpoint{2.282990in}{0.639846in}}%
\pgfpathlineto{\pgfqpoint{2.381120in}{0.544737in}}%
\pgfpathlineto{\pgfqpoint{2.398090in}{0.528000in}}%
\pgfpathlineto{\pgfqpoint{2.403232in}{0.528000in}}%
\pgfpathmoveto{\pgfqpoint{4.768000in}{1.202575in}}%
\pgfpathlineto{\pgfqpoint{4.647758in}{1.356878in}}%
\pgfpathlineto{\pgfqpoint{4.527515in}{1.506206in}}%
\pgfpathlineto{\pgfqpoint{4.282141in}{1.797333in}}%
\pgfpathlineto{\pgfqpoint{4.166788in}{1.928498in}}%
\pgfpathlineto{\pgfqpoint{3.911066in}{2.208000in}}%
\pgfpathlineto{\pgfqpoint{3.824060in}{2.299432in}}%
\pgfpathlineto{\pgfqpoint{3.765980in}{2.360241in}}%
\pgfpathlineto{\pgfqpoint{3.652668in}{2.475789in}}%
\pgfpathlineto{\pgfqpoint{3.605657in}{2.523224in}}%
\pgfpathlineto{\pgfqpoint{3.471252in}{2.656000in}}%
\pgfpathlineto{\pgfqpoint{3.180098in}{2.931612in}}%
\pgfpathlineto{\pgfqpoint{3.114443in}{2.992000in}}%
\pgfpathlineto{\pgfqpoint{2.818125in}{3.253333in}}%
\pgfpathlineto{\pgfqpoint{2.563556in}{3.465214in}}%
\pgfpathlineto{\pgfqpoint{2.483394in}{3.529464in}}%
\pgfpathlineto{\pgfqpoint{2.359266in}{3.626667in}}%
\pgfpathlineto{\pgfqpoint{2.109195in}{3.813333in}}%
\pgfpathlineto{\pgfqpoint{2.002424in}{3.889378in}}%
\pgfpathlineto{\pgfqpoint{1.761939in}{4.051456in}}%
\pgfpathlineto{\pgfqpoint{1.521455in}{4.200290in}}%
\pgfpathlineto{\pgfqpoint{1.481016in}{4.224000in}}%
\pgfpathlineto{\pgfqpoint{1.468156in}{4.224000in}}%
\pgfpathlineto{\pgfqpoint{1.614297in}{4.137522in}}%
\pgfpathlineto{\pgfqpoint{1.721859in}{4.070353in}}%
\pgfpathlineto{\pgfqpoint{1.962343in}{3.910469in}}%
\pgfpathlineto{\pgfqpoint{2.025597in}{3.866416in}}%
\pgfpathlineto{\pgfqpoint{2.122667in}{3.797003in}}%
\pgfpathlineto{\pgfqpoint{2.162747in}{3.767852in}}%
\pgfpathlineto{\pgfqpoint{2.282990in}{3.678351in}}%
\pgfpathlineto{\pgfqpoint{2.379271in}{3.604348in}}%
\pgfpathlineto{\pgfqpoint{2.446841in}{3.552000in}}%
\pgfpathlineto{\pgfqpoint{2.586716in}{3.440000in}}%
\pgfpathlineto{\pgfqpoint{2.683798in}{3.360284in}}%
\pgfpathlineto{\pgfqpoint{2.810738in}{3.253333in}}%
\pgfpathlineto{\pgfqpoint{2.940068in}{3.141333in}}%
\pgfpathlineto{\pgfqpoint{3.229291in}{2.880000in}}%
\pgfpathlineto{\pgfqpoint{3.348304in}{2.768000in}}%
\pgfpathlineto{\pgfqpoint{3.615647in}{2.506667in}}%
\pgfpathlineto{\pgfqpoint{3.726125in}{2.394667in}}%
\pgfpathlineto{\pgfqpoint{3.834008in}{2.282667in}}%
\pgfpathlineto{\pgfqpoint{3.939681in}{2.170667in}}%
\pgfpathlineto{\pgfqpoint{4.026061in}{2.076920in}}%
\pgfpathlineto{\pgfqpoint{4.086626in}{2.010822in}}%
\pgfpathlineto{\pgfqpoint{4.340152in}{1.722667in}}%
\pgfpathlineto{\pgfqpoint{4.447354in}{1.595688in}}%
\pgfpathlineto{\pgfqpoint{4.676832in}{1.312000in}}%
\pgfpathlineto{\pgfqpoint{4.734941in}{1.237333in}}%
\pgfpathlineto{\pgfqpoint{4.768000in}{1.194287in}}%
\pgfpathlineto{\pgfqpoint{4.768000in}{1.200000in}}%
\pgfpathlineto{\pgfqpoint{4.768000in}{1.200000in}}%
\pgfusepath{fill}%
\end{pgfscope}%
\begin{pgfscope}%
\pgfpathrectangle{\pgfqpoint{0.800000in}{0.528000in}}{\pgfqpoint{3.968000in}{3.696000in}}%
\pgfusepath{clip}%
\pgfsetbuttcap%
\pgfsetroundjoin%
\definecolor{currentfill}{rgb}{0.278012,0.180367,0.486697}%
\pgfsetfillcolor{currentfill}%
\pgfsetlinewidth{0.000000pt}%
\definecolor{currentstroke}{rgb}{0.000000,0.000000,0.000000}%
\pgfsetstrokecolor{currentstroke}%
\pgfsetdash{}{0pt}%
\pgfpathmoveto{\pgfqpoint{2.398090in}{0.528000in}}%
\pgfpathlineto{\pgfqpoint{2.302669in}{0.620997in}}%
\pgfpathlineto{\pgfqpoint{2.242909in}{0.679433in}}%
\pgfpathlineto{\pgfqpoint{2.122667in}{0.800086in}}%
\pgfpathlineto{\pgfqpoint{1.842101in}{1.092899in}}%
\pgfpathlineto{\pgfqpoint{1.561535in}{1.403359in}}%
\pgfpathlineto{\pgfqpoint{1.441293in}{1.542316in}}%
\pgfpathlineto{\pgfqpoint{1.320971in}{1.685333in}}%
\pgfpathlineto{\pgfqpoint{1.080566in}{1.984720in}}%
\pgfpathlineto{\pgfqpoint{0.880162in}{2.250683in}}%
\pgfpathlineto{\pgfqpoint{0.800000in}{2.362074in}}%
\pgfpathlineto{\pgfqpoint{0.800000in}{2.353044in}}%
\pgfpathlineto{\pgfqpoint{0.905211in}{2.208000in}}%
\pgfpathlineto{\pgfqpoint{0.977324in}{2.111835in}}%
\pgfpathlineto{\pgfqpoint{1.017413in}{2.058667in}}%
\pgfpathlineto{\pgfqpoint{1.111218in}{1.937884in}}%
\pgfpathlineto{\pgfqpoint{1.163053in}{1.872000in}}%
\pgfpathlineto{\pgfqpoint{1.253460in}{1.760000in}}%
\pgfpathlineto{\pgfqpoint{1.361131in}{1.629815in}}%
\pgfpathlineto{\pgfqpoint{1.481374in}{1.488359in}}%
\pgfpathlineto{\pgfqpoint{1.608584in}{1.342843in}}%
\pgfpathlineto{\pgfqpoint{1.737176in}{1.200000in}}%
\pgfpathlineto{\pgfqpoint{1.842101in}{1.086084in}}%
\pgfpathlineto{\pgfqpoint{1.981578in}{0.938667in}}%
\pgfpathlineto{\pgfqpoint{2.276324in}{0.640000in}}%
\pgfpathlineto{\pgfqpoint{2.391462in}{0.528000in}}%
\pgfpathmoveto{\pgfqpoint{4.768000in}{1.210688in}}%
\pgfpathlineto{\pgfqpoint{4.659936in}{1.349333in}}%
\pgfpathlineto{\pgfqpoint{4.586134in}{1.441267in}}%
\pgfpathlineto{\pgfqpoint{4.534443in}{1.505119in}}%
\pgfpathlineto{\pgfqpoint{4.478416in}{1.573333in}}%
\pgfpathlineto{\pgfqpoint{4.376694in}{1.694184in}}%
\pgfpathlineto{\pgfqpoint{4.320628in}{1.760000in}}%
\pgfpathlineto{\pgfqpoint{4.206869in}{1.890351in}}%
\pgfpathlineto{\pgfqpoint{4.086626in}{2.024754in}}%
\pgfpathlineto{\pgfqpoint{3.952351in}{2.170667in}}%
\pgfpathlineto{\pgfqpoint{3.846141in}{2.283446in}}%
\pgfpathlineto{\pgfqpoint{3.738901in}{2.394667in}}%
\pgfpathlineto{\pgfqpoint{3.628653in}{2.506667in}}%
\pgfpathlineto{\pgfqpoint{3.516014in}{2.618667in}}%
\pgfpathlineto{\pgfqpoint{3.400852in}{2.730667in}}%
\pgfpathlineto{\pgfqpoint{3.303716in}{2.822757in}}%
\pgfpathlineto{\pgfqpoint{3.243125in}{2.880000in}}%
\pgfpathlineto{\pgfqpoint{3.121495in}{2.992000in}}%
\pgfpathlineto{\pgfqpoint{2.825512in}{3.253333in}}%
\pgfpathlineto{\pgfqpoint{2.556475in}{3.477333in}}%
\pgfpathlineto{\pgfqpoint{2.415544in}{3.589333in}}%
\pgfpathlineto{\pgfqpoint{2.319014in}{3.664000in}}%
\pgfpathlineto{\pgfqpoint{2.066364in}{3.850667in}}%
\pgfpathlineto{\pgfqpoint{1.842101in}{4.005644in}}%
\pgfpathlineto{\pgfqpoint{1.736595in}{4.074667in}}%
\pgfpathlineto{\pgfqpoint{1.641697in}{4.134638in}}%
\pgfpathlineto{\pgfqpoint{1.556486in}{4.186667in}}%
\pgfpathlineto{\pgfqpoint{1.493234in}{4.224000in}}%
\pgfpathlineto{\pgfqpoint{1.481016in}{4.224000in}}%
\pgfpathlineto{\pgfqpoint{1.481374in}{4.223796in}}%
\pgfpathlineto{\pgfqpoint{1.681778in}{4.102648in}}%
\pgfpathlineto{\pgfqpoint{1.922263in}{3.944730in}}%
\pgfpathlineto{\pgfqpoint{2.042505in}{3.861069in}}%
\pgfpathlineto{\pgfqpoint{2.109195in}{3.813333in}}%
\pgfpathlineto{\pgfqpoint{2.211203in}{3.738667in}}%
\pgfpathlineto{\pgfqpoint{2.323071in}{3.654406in}}%
\pgfpathlineto{\pgfqpoint{2.407405in}{3.589333in}}%
\pgfpathlineto{\pgfqpoint{2.548515in}{3.477333in}}%
\pgfpathlineto{\pgfqpoint{2.804040in}{3.265386in}}%
\pgfpathlineto{\pgfqpoint{2.904624in}{3.178667in}}%
\pgfpathlineto{\pgfqpoint{3.031581in}{3.066667in}}%
\pgfpathlineto{\pgfqpoint{3.300318in}{2.819592in}}%
\pgfpathlineto{\pgfqpoint{3.365172in}{2.758399in}}%
\pgfpathlineto{\pgfqpoint{3.471252in}{2.656000in}}%
\pgfpathlineto{\pgfqpoint{3.584837in}{2.544000in}}%
\pgfpathlineto{\pgfqpoint{3.695984in}{2.432000in}}%
\pgfpathlineto{\pgfqpoint{3.786213in}{2.338847in}}%
\pgfpathlineto{\pgfqpoint{3.846141in}{2.276704in}}%
\pgfpathlineto{\pgfqpoint{4.126707in}{1.973341in}}%
\pgfpathlineto{\pgfqpoint{4.261395in}{1.821211in}}%
\pgfpathlineto{\pgfqpoint{4.378210in}{1.685333in}}%
\pgfpathlineto{\pgfqpoint{4.461184in}{1.586216in}}%
\pgfpathlineto{\pgfqpoint{4.503013in}{1.536000in}}%
\pgfpathlineto{\pgfqpoint{4.582590in}{1.437967in}}%
\pgfpathlineto{\pgfqpoint{4.624021in}{1.386667in}}%
\pgfpathlineto{\pgfqpoint{4.727919in}{1.254518in}}%
\pgfpathlineto{\pgfqpoint{4.768000in}{1.202575in}}%
\pgfpathlineto{\pgfqpoint{4.768000in}{1.202575in}}%
\pgfusepath{fill}%
\end{pgfscope}%
\begin{pgfscope}%
\pgfpathrectangle{\pgfqpoint{0.800000in}{0.528000in}}{\pgfqpoint{3.968000in}{3.696000in}}%
\pgfusepath{clip}%
\pgfsetbuttcap%
\pgfsetroundjoin%
\definecolor{currentfill}{rgb}{0.278012,0.180367,0.486697}%
\pgfsetfillcolor{currentfill}%
\pgfsetlinewidth{0.000000pt}%
\definecolor{currentstroke}{rgb}{0.000000,0.000000,0.000000}%
\pgfsetstrokecolor{currentstroke}%
\pgfsetdash{}{0pt}%
\pgfpathmoveto{\pgfqpoint{2.391462in}{0.528000in}}%
\pgfpathlineto{\pgfqpoint{2.090078in}{0.826667in}}%
\pgfpathlineto{\pgfqpoint{1.805677in}{1.125333in}}%
\pgfpathlineto{\pgfqpoint{1.703259in}{1.237333in}}%
\pgfpathlineto{\pgfqpoint{1.601616in}{1.350697in}}%
\pgfpathlineto{\pgfqpoint{1.472498in}{1.498667in}}%
\pgfpathlineto{\pgfqpoint{1.377229in}{1.610667in}}%
\pgfpathlineto{\pgfqpoint{1.280970in}{1.726315in}}%
\pgfpathlineto{\pgfqpoint{1.160727in}{1.874923in}}%
\pgfpathlineto{\pgfqpoint{1.040485in}{2.028513in}}%
\pgfpathlineto{\pgfqpoint{0.932879in}{2.170667in}}%
\pgfpathlineto{\pgfqpoint{0.840081in}{2.297202in}}%
\pgfpathlineto{\pgfqpoint{0.800000in}{2.353044in}}%
\pgfpathlineto{\pgfqpoint{0.800000in}{2.344172in}}%
\pgfpathlineto{\pgfqpoint{0.898907in}{2.208000in}}%
\pgfpathlineto{\pgfqpoint{0.973617in}{2.108383in}}%
\pgfpathlineto{\pgfqpoint{1.011130in}{2.058667in}}%
\pgfpathlineto{\pgfqpoint{1.097905in}{1.946667in}}%
\pgfpathlineto{\pgfqpoint{1.175823in}{1.848728in}}%
\pgfpathlineto{\pgfqpoint{1.216924in}{1.797333in}}%
\pgfpathlineto{\pgfqpoint{1.296788in}{1.700067in}}%
\pgfpathlineto{\pgfqpoint{1.339746in}{1.648000in}}%
\pgfpathlineto{\pgfqpoint{1.407150in}{1.567803in}}%
\pgfpathlineto{\pgfqpoint{1.531020in}{1.424000in}}%
\pgfpathlineto{\pgfqpoint{1.641697in}{1.298774in}}%
\pgfpathlineto{\pgfqpoint{1.764989in}{1.162667in}}%
\pgfpathlineto{\pgfqpoint{1.869012in}{1.050667in}}%
\pgfpathlineto{\pgfqpoint{1.975199in}{0.938667in}}%
\pgfpathlineto{\pgfqpoint{2.047205in}{0.864000in}}%
\pgfpathlineto{\pgfqpoint{2.162747in}{0.746486in}}%
\pgfpathlineto{\pgfqpoint{2.307872in}{0.602667in}}%
\pgfpathlineto{\pgfqpoint{2.384833in}{0.528000in}}%
\pgfpathmoveto{\pgfqpoint{4.768000in}{1.218802in}}%
\pgfpathlineto{\pgfqpoint{4.666136in}{1.349333in}}%
\pgfpathlineto{\pgfqpoint{4.589677in}{1.444567in}}%
\pgfpathlineto{\pgfqpoint{4.546040in}{1.498667in}}%
\pgfpathlineto{\pgfqpoint{4.468154in}{1.592708in}}%
\pgfpathlineto{\pgfqpoint{4.422166in}{1.648000in}}%
\pgfpathlineto{\pgfqpoint{4.326010in}{1.761026in}}%
\pgfpathlineto{\pgfqpoint{4.196293in}{1.909333in}}%
\pgfpathlineto{\pgfqpoint{4.086626in}{2.031584in}}%
\pgfpathlineto{\pgfqpoint{3.958685in}{2.170667in}}%
\pgfpathlineto{\pgfqpoint{3.846141in}{2.290038in}}%
\pgfpathlineto{\pgfqpoint{3.745289in}{2.394667in}}%
\pgfpathlineto{\pgfqpoint{3.635156in}{2.506667in}}%
\pgfpathlineto{\pgfqpoint{3.522636in}{2.618667in}}%
\pgfpathlineto{\pgfqpoint{3.405253in}{2.732864in}}%
\pgfpathlineto{\pgfqpoint{3.307115in}{2.825923in}}%
\pgfpathlineto{\pgfqpoint{3.244929in}{2.884643in}}%
\pgfpathlineto{\pgfqpoint{3.124687in}{2.995422in}}%
\pgfpathlineto{\pgfqpoint{3.024477in}{3.085326in}}%
\pgfpathlineto{\pgfqpoint{2.961787in}{3.141333in}}%
\pgfpathlineto{\pgfqpoint{2.655570in}{3.402667in}}%
\pgfpathlineto{\pgfqpoint{2.557986in}{3.482521in}}%
\pgfpathlineto{\pgfqpoint{2.423683in}{3.589333in}}%
\pgfpathlineto{\pgfqpoint{2.323071in}{3.667342in}}%
\pgfpathlineto{\pgfqpoint{2.042505in}{3.874261in}}%
\pgfpathlineto{\pgfqpoint{1.915682in}{3.962667in}}%
\pgfpathlineto{\pgfqpoint{1.795852in}{4.043079in}}%
\pgfpathlineto{\pgfqpoint{1.681778in}{4.116561in}}%
\pgfpathlineto{\pgfqpoint{1.505434in}{4.224000in}}%
\pgfpathlineto{\pgfqpoint{1.493234in}{4.224000in}}%
\pgfpathlineto{\pgfqpoint{1.681778in}{4.109665in}}%
\pgfpathlineto{\pgfqpoint{1.922263in}{3.951461in}}%
\pgfpathlineto{\pgfqpoint{2.042505in}{3.867665in}}%
\pgfpathlineto{\pgfqpoint{2.118375in}{3.813333in}}%
\pgfpathlineto{\pgfqpoint{2.202828in}{3.751381in}}%
\pgfpathlineto{\pgfqpoint{2.298964in}{3.678879in}}%
\pgfpathlineto{\pgfqpoint{2.367600in}{3.626667in}}%
\pgfpathlineto{\pgfqpoint{2.647976in}{3.402667in}}%
\pgfpathlineto{\pgfqpoint{2.781733in}{3.290667in}}%
\pgfpathlineto{\pgfqpoint{2.911912in}{3.178667in}}%
\pgfpathlineto{\pgfqpoint{3.021022in}{3.082108in}}%
\pgfpathlineto{\pgfqpoint{3.084606in}{3.025440in}}%
\pgfpathlineto{\pgfqpoint{3.183520in}{2.934800in}}%
\pgfpathlineto{\pgfqpoint{3.244929in}{2.878322in}}%
\pgfpathlineto{\pgfqpoint{3.325091in}{2.802983in}}%
\pgfpathlineto{\pgfqpoint{3.445333in}{2.687717in}}%
\pgfpathlineto{\pgfqpoint{3.553834in}{2.581333in}}%
\pgfpathlineto{\pgfqpoint{3.665662in}{2.469333in}}%
\pgfpathlineto{\pgfqpoint{3.952351in}{2.170667in}}%
\pgfpathlineto{\pgfqpoint{4.055716in}{2.058667in}}%
\pgfpathlineto{\pgfqpoint{4.166788in}{1.935534in}}%
\pgfpathlineto{\pgfqpoint{4.294173in}{1.790680in}}%
\pgfpathlineto{\pgfqpoint{4.416005in}{1.648000in}}%
\pgfpathlineto{\pgfqpoint{4.517039in}{1.526242in}}%
\pgfpathlineto{\pgfqpoint{4.570315in}{1.461333in}}%
\pgfpathlineto{\pgfqpoint{4.768000in}{1.210688in}}%
\pgfpathlineto{\pgfqpoint{4.768000in}{1.210688in}}%
\pgfusepath{fill}%
\end{pgfscope}%
\begin{pgfscope}%
\pgfpathrectangle{\pgfqpoint{0.800000in}{0.528000in}}{\pgfqpoint{3.968000in}{3.696000in}}%
\pgfusepath{clip}%
\pgfsetbuttcap%
\pgfsetroundjoin%
\definecolor{currentfill}{rgb}{0.278012,0.180367,0.486697}%
\pgfsetfillcolor{currentfill}%
\pgfsetlinewidth{0.000000pt}%
\definecolor{currentstroke}{rgb}{0.000000,0.000000,0.000000}%
\pgfsetstrokecolor{currentstroke}%
\pgfsetdash{}{0pt}%
\pgfpathmoveto{\pgfqpoint{2.384833in}{0.528000in}}%
\pgfpathlineto{\pgfqpoint{2.083585in}{0.826667in}}%
\pgfpathlineto{\pgfqpoint{1.819394in}{1.104182in}}%
\pgfpathlineto{\pgfqpoint{1.761939in}{1.165988in}}%
\pgfpathlineto{\pgfqpoint{1.629890in}{1.312000in}}%
\pgfpathlineto{\pgfqpoint{1.521455in}{1.434963in}}%
\pgfpathlineto{\pgfqpoint{1.401212in}{1.574802in}}%
\pgfpathlineto{\pgfqpoint{1.267337in}{1.735365in}}%
\pgfpathlineto{\pgfqpoint{1.145720in}{1.885979in}}%
\pgfpathlineto{\pgfqpoint{1.037738in}{2.023892in}}%
\pgfpathlineto{\pgfqpoint{0.920242in}{2.179135in}}%
\pgfpathlineto{\pgfqpoint{0.800000in}{2.344172in}}%
\pgfpathlineto{\pgfqpoint{0.800000in}{2.335301in}}%
\pgfpathlineto{\pgfqpoint{0.892604in}{2.208000in}}%
\pgfpathlineto{\pgfqpoint{0.969910in}{2.104930in}}%
\pgfpathlineto{\pgfqpoint{1.015869in}{2.044262in}}%
\pgfpathlineto{\pgfqpoint{1.121928in}{1.908139in}}%
\pgfpathlineto{\pgfqpoint{1.241253in}{1.759661in}}%
\pgfpathlineto{\pgfqpoint{1.378086in}{1.594874in}}%
\pgfpathlineto{\pgfqpoint{1.492375in}{1.461333in}}%
\pgfpathlineto{\pgfqpoint{1.601616in}{1.336790in}}%
\pgfpathlineto{\pgfqpoint{1.740176in}{1.182938in}}%
\pgfpathlineto{\pgfqpoint{1.862744in}{1.050667in}}%
\pgfpathlineto{\pgfqpoint{1.968821in}{0.938667in}}%
\pgfpathlineto{\pgfqpoint{2.042505in}{0.862241in}}%
\pgfpathlineto{\pgfqpoint{2.188088in}{0.714667in}}%
\pgfpathlineto{\pgfqpoint{2.301324in}{0.602667in}}%
\pgfpathlineto{\pgfqpoint{2.378204in}{0.528000in}}%
\pgfpathmoveto{\pgfqpoint{4.768000in}{1.226915in}}%
\pgfpathlineto{\pgfqpoint{4.672336in}{1.349333in}}%
\pgfpathlineto{\pgfqpoint{4.593220in}{1.447868in}}%
\pgfpathlineto{\pgfqpoint{4.552220in}{1.498667in}}%
\pgfpathlineto{\pgfqpoint{4.471639in}{1.595954in}}%
\pgfpathlineto{\pgfqpoint{4.428326in}{1.648000in}}%
\pgfpathlineto{\pgfqpoint{4.327111in}{1.766841in}}%
\pgfpathlineto{\pgfqpoint{4.202539in}{1.909333in}}%
\pgfpathlineto{\pgfqpoint{4.086626in}{2.038415in}}%
\pgfpathlineto{\pgfqpoint{3.955913in}{2.180420in}}%
\pgfpathlineto{\pgfqpoint{3.823761in}{2.320000in}}%
\pgfpathlineto{\pgfqpoint{3.715261in}{2.432000in}}%
\pgfpathlineto{\pgfqpoint{3.604463in}{2.544000in}}%
\pgfpathlineto{\pgfqpoint{3.525495in}{2.622276in}}%
\pgfpathlineto{\pgfqpoint{3.405253in}{2.739184in}}%
\pgfpathlineto{\pgfqpoint{3.290692in}{2.847959in}}%
\pgfpathlineto{\pgfqpoint{3.244929in}{2.890924in}}%
\pgfpathlineto{\pgfqpoint{3.124687in}{3.001673in}}%
\pgfpathlineto{\pgfqpoint{3.027932in}{3.088545in}}%
\pgfpathlineto{\pgfqpoint{2.964364in}{3.145335in}}%
\pgfpathlineto{\pgfqpoint{2.883578in}{3.216000in}}%
\pgfpathlineto{\pgfqpoint{2.752534in}{3.328000in}}%
\pgfpathlineto{\pgfqpoint{2.617851in}{3.440000in}}%
\pgfpathlineto{\pgfqpoint{2.508418in}{3.528692in}}%
\pgfpathlineto{\pgfqpoint{2.384001in}{3.626667in}}%
\pgfpathlineto{\pgfqpoint{2.266241in}{3.716934in}}%
\pgfpathlineto{\pgfqpoint{2.136212in}{3.813333in}}%
\pgfpathlineto{\pgfqpoint{1.922263in}{3.964866in}}%
\pgfpathlineto{\pgfqpoint{1.814639in}{4.037333in}}%
\pgfpathlineto{\pgfqpoint{1.640583in}{4.149333in}}%
\pgfpathlineto{\pgfqpoint{1.517633in}{4.224000in}}%
\pgfpathlineto{\pgfqpoint{1.505434in}{4.224000in}}%
\pgfpathlineto{\pgfqpoint{1.660488in}{4.129503in}}%
\pgfpathlineto{\pgfqpoint{1.721859in}{4.091005in}}%
\pgfpathlineto{\pgfqpoint{1.842101in}{4.012352in}}%
\pgfpathlineto{\pgfqpoint{1.882182in}{3.985410in}}%
\pgfpathlineto{\pgfqpoint{2.002424in}{3.902548in}}%
\pgfpathlineto{\pgfqpoint{2.127373in}{3.813333in}}%
\pgfpathlineto{\pgfqpoint{2.390600in}{3.614900in}}%
\pgfpathlineto{\pgfqpoint{2.443313in}{3.573937in}}%
\pgfpathlineto{\pgfqpoint{2.523475in}{3.510286in}}%
\pgfpathlineto{\pgfqpoint{2.655570in}{3.402667in}}%
\pgfpathlineto{\pgfqpoint{2.789171in}{3.290667in}}%
\pgfpathlineto{\pgfqpoint{2.884202in}{3.209119in}}%
\pgfpathlineto{\pgfqpoint{3.004444in}{3.103613in}}%
\pgfpathlineto{\pgfqpoint{3.106033in}{3.011958in}}%
\pgfpathlineto{\pgfqpoint{3.169243in}{2.954667in}}%
\pgfpathlineto{\pgfqpoint{3.267215in}{2.863424in}}%
\pgfpathlineto{\pgfqpoint{3.329313in}{2.805333in}}%
\pgfpathlineto{\pgfqpoint{3.446207in}{2.693333in}}%
\pgfpathlineto{\pgfqpoint{3.543277in}{2.597897in}}%
\pgfpathlineto{\pgfqpoint{3.605657in}{2.536281in}}%
\pgfpathlineto{\pgfqpoint{3.745289in}{2.394667in}}%
\pgfpathlineto{\pgfqpoint{3.868493in}{2.266152in}}%
\pgfpathlineto{\pgfqpoint{3.926303in}{2.205346in}}%
\pgfpathlineto{\pgfqpoint{4.061941in}{2.058667in}}%
\pgfpathlineto{\pgfqpoint{4.166788in}{1.942570in}}%
\pgfpathlineto{\pgfqpoint{4.294548in}{1.797333in}}%
\pgfpathlineto{\pgfqpoint{4.407273in}{1.665667in}}%
\pgfpathlineto{\pgfqpoint{4.527515in}{1.521321in}}%
\pgfpathlineto{\pgfqpoint{4.753710in}{1.237333in}}%
\pgfpathlineto{\pgfqpoint{4.768000in}{1.218802in}}%
\pgfpathlineto{\pgfqpoint{4.768000in}{1.218802in}}%
\pgfusepath{fill}%
\end{pgfscope}%
\begin{pgfscope}%
\pgfpathrectangle{\pgfqpoint{0.800000in}{0.528000in}}{\pgfqpoint{3.968000in}{3.696000in}}%
\pgfusepath{clip}%
\pgfsetbuttcap%
\pgfsetroundjoin%
\definecolor{currentfill}{rgb}{0.278012,0.180367,0.486697}%
\pgfsetfillcolor{currentfill}%
\pgfsetlinewidth{0.000000pt}%
\definecolor{currentstroke}{rgb}{0.000000,0.000000,0.000000}%
\pgfsetstrokecolor{currentstroke}%
\pgfsetdash{}{0pt}%
\pgfpathmoveto{\pgfqpoint{2.378204in}{0.528000in}}%
\pgfpathlineto{\pgfqpoint{2.099124in}{0.804738in}}%
\pgfpathlineto{\pgfqpoint{2.040797in}{0.864000in}}%
\pgfpathlineto{\pgfqpoint{1.928140in}{0.981475in}}%
\pgfpathlineto{\pgfqpoint{1.882182in}{1.029952in}}%
\pgfpathlineto{\pgfqpoint{1.758749in}{1.162667in}}%
\pgfpathlineto{\pgfqpoint{1.641697in}{1.291850in}}%
\pgfpathlineto{\pgfqpoint{1.396328in}{1.573333in}}%
\pgfpathlineto{\pgfqpoint{1.302518in}{1.685333in}}%
\pgfpathlineto{\pgfqpoint{1.223894in}{1.781504in}}%
\pgfpathlineto{\pgfqpoint{1.180634in}{1.834667in}}%
\pgfpathlineto{\pgfqpoint{1.104108in}{1.931262in}}%
\pgfpathlineto{\pgfqpoint{1.062537in}{1.984000in}}%
\pgfpathlineto{\pgfqpoint{1.000404in}{2.064485in}}%
\pgfpathlineto{\pgfqpoint{0.892604in}{2.208000in}}%
\pgfpathlineto{\pgfqpoint{0.822860in}{2.303960in}}%
\pgfpathlineto{\pgfqpoint{0.800000in}{2.335301in}}%
\pgfpathlineto{\pgfqpoint{0.800000in}{2.326430in}}%
\pgfpathlineto{\pgfqpoint{0.880162in}{2.216322in}}%
\pgfpathlineto{\pgfqpoint{1.056346in}{1.984000in}}%
\pgfpathlineto{\pgfqpoint{1.120646in}{1.902046in}}%
\pgfpathlineto{\pgfqpoint{1.240889in}{1.752662in}}%
\pgfpathlineto{\pgfqpoint{1.361131in}{1.607696in}}%
\pgfpathlineto{\pgfqpoint{1.486174in}{1.461333in}}%
\pgfpathlineto{\pgfqpoint{1.601616in}{1.329854in}}%
\pgfpathlineto{\pgfqpoint{1.721859in}{1.196206in}}%
\pgfpathlineto{\pgfqpoint{1.856476in}{1.050667in}}%
\pgfpathlineto{\pgfqpoint{1.962443in}{0.938667in}}%
\pgfpathlineto{\pgfqpoint{2.076546in}{0.821041in}}%
\pgfpathlineto{\pgfqpoint{2.122667in}{0.774015in}}%
\pgfpathlineto{\pgfqpoint{2.371576in}{0.528000in}}%
\pgfpathmoveto{\pgfqpoint{4.768000in}{1.235028in}}%
\pgfpathlineto{\pgfqpoint{4.678536in}{1.349333in}}%
\pgfpathlineto{\pgfqpoint{4.588750in}{1.461333in}}%
\pgfpathlineto{\pgfqpoint{4.487434in}{1.584755in}}%
\pgfpathlineto{\pgfqpoint{4.241669in}{1.872000in}}%
\pgfpathlineto{\pgfqpoint{4.126707in}{2.000985in}}%
\pgfpathlineto{\pgfqpoint{3.846141in}{2.303224in}}%
\pgfpathlineto{\pgfqpoint{3.565576in}{2.589068in}}%
\pgfpathlineto{\pgfqpoint{3.459246in}{2.693333in}}%
\pgfpathlineto{\pgfqpoint{3.342591in}{2.805333in}}%
\pgfpathlineto{\pgfqpoint{3.059702in}{3.066667in}}%
\pgfpathlineto{\pgfqpoint{2.933494in}{3.178667in}}%
\pgfpathlineto{\pgfqpoint{2.683798in}{3.391863in}}%
\pgfpathlineto{\pgfqpoint{2.563556in}{3.490549in}}%
\pgfpathlineto{\pgfqpoint{2.439960in}{3.589333in}}%
\pgfpathlineto{\pgfqpoint{2.295174in}{3.701333in}}%
\pgfpathlineto{\pgfqpoint{2.082586in}{3.858711in}}%
\pgfpathlineto{\pgfqpoint{1.824720in}{4.037333in}}%
\pgfpathlineto{\pgfqpoint{1.621741in}{4.167922in}}%
\pgfpathlineto{\pgfqpoint{1.529424in}{4.224000in}}%
\pgfpathlineto{\pgfqpoint{1.517633in}{4.224000in}}%
\pgfpathlineto{\pgfqpoint{1.641697in}{4.148647in}}%
\pgfpathlineto{\pgfqpoint{1.882182in}{3.992129in}}%
\pgfpathlineto{\pgfqpoint{1.979253in}{3.925333in}}%
\pgfpathlineto{\pgfqpoint{2.202828in}{3.764336in}}%
\pgfpathlineto{\pgfqpoint{2.306556in}{3.685951in}}%
\pgfpathlineto{\pgfqpoint{2.363152in}{3.642845in}}%
\pgfpathlineto{\pgfqpoint{2.483394in}{3.548644in}}%
\pgfpathlineto{\pgfqpoint{2.603636in}{3.451658in}}%
\pgfpathlineto{\pgfqpoint{2.723879in}{3.352111in}}%
\pgfpathlineto{\pgfqpoint{2.844121in}{3.250047in}}%
\pgfpathlineto{\pgfqpoint{2.968889in}{3.141333in}}%
\pgfpathlineto{\pgfqpoint{3.256619in}{2.880000in}}%
\pgfpathlineto{\pgfqpoint{3.546629in}{2.601019in}}%
\pgfpathlineto{\pgfqpoint{3.605657in}{2.542809in}}%
\pgfpathlineto{\pgfqpoint{3.751677in}{2.394667in}}%
\pgfpathlineto{\pgfqpoint{4.034009in}{2.096000in}}%
\pgfpathlineto{\pgfqpoint{4.135832in}{1.984000in}}%
\pgfpathlineto{\pgfqpoint{4.246949in}{1.858936in}}%
\pgfpathlineto{\pgfqpoint{4.367192in}{1.720188in}}%
\pgfpathlineto{\pgfqpoint{4.612816in}{1.424000in}}%
\pgfpathlineto{\pgfqpoint{4.701777in}{1.312000in}}%
\pgfpathlineto{\pgfqpoint{4.768000in}{1.226915in}}%
\pgfpathlineto{\pgfqpoint{4.768000in}{1.226915in}}%
\pgfusepath{fill}%
\end{pgfscope}%
\begin{pgfscope}%
\pgfpathrectangle{\pgfqpoint{0.800000in}{0.528000in}}{\pgfqpoint{3.968000in}{3.696000in}}%
\pgfusepath{clip}%
\pgfsetbuttcap%
\pgfsetroundjoin%
\definecolor{currentfill}{rgb}{0.277134,0.185228,0.489898}%
\pgfsetfillcolor{currentfill}%
\pgfsetlinewidth{0.000000pt}%
\definecolor{currentstroke}{rgb}{0.000000,0.000000,0.000000}%
\pgfsetstrokecolor{currentstroke}%
\pgfsetdash{}{0pt}%
\pgfpathmoveto{\pgfqpoint{2.371576in}{0.528000in}}%
\pgfpathlineto{\pgfqpoint{2.107574in}{0.789333in}}%
\pgfpathlineto{\pgfqpoint{2.019474in}{0.879881in}}%
\pgfpathlineto{\pgfqpoint{1.962343in}{0.938770in}}%
\pgfpathlineto{\pgfqpoint{1.856476in}{1.050667in}}%
\pgfpathlineto{\pgfqpoint{1.752588in}{1.162667in}}%
\pgfpathlineto{\pgfqpoint{1.641697in}{1.284926in}}%
\pgfpathlineto{\pgfqpoint{1.390231in}{1.573333in}}%
\pgfpathlineto{\pgfqpoint{1.280970in}{1.703927in}}%
\pgfpathlineto{\pgfqpoint{1.174463in}{1.834667in}}%
\pgfpathlineto{\pgfqpoint{1.100553in}{1.927951in}}%
\pgfpathlineto{\pgfqpoint{1.056346in}{1.984000in}}%
\pgfpathlineto{\pgfqpoint{0.994157in}{2.064485in}}%
\pgfpathlineto{\pgfqpoint{0.880162in}{2.216322in}}%
\pgfpathlineto{\pgfqpoint{0.800000in}{2.326430in}}%
\pgfpathlineto{\pgfqpoint{0.800000in}{2.317637in}}%
\pgfpathlineto{\pgfqpoint{1.000404in}{2.048349in}}%
\pgfpathlineto{\pgfqpoint{1.108772in}{1.909333in}}%
\pgfpathlineto{\pgfqpoint{1.200808in}{1.794327in}}%
\pgfpathlineto{\pgfqpoint{1.447834in}{1.498667in}}%
\pgfpathlineto{\pgfqpoint{1.561535in}{1.368181in}}%
\pgfpathlineto{\pgfqpoint{1.681778in}{1.233532in}}%
\pgfpathlineto{\pgfqpoint{1.815389in}{1.088000in}}%
\pgfpathlineto{\pgfqpoint{1.907937in}{0.989344in}}%
\pgfpathlineto{\pgfqpoint{2.028230in}{0.864000in}}%
\pgfpathlineto{\pgfqpoint{2.323071in}{0.568611in}}%
\pgfpathlineto{\pgfqpoint{2.364947in}{0.528000in}}%
\pgfpathmoveto{\pgfqpoint{4.768000in}{1.242968in}}%
\pgfpathlineto{\pgfqpoint{4.647758in}{1.395738in}}%
\pgfpathlineto{\pgfqpoint{4.407273in}{1.687529in}}%
\pgfpathlineto{\pgfqpoint{4.138243in}{1.994745in}}%
\pgfpathlineto{\pgfqpoint{4.080619in}{2.058667in}}%
\pgfpathlineto{\pgfqpoint{3.966384in}{2.182491in}}%
\pgfpathlineto{\pgfqpoint{3.836389in}{2.320000in}}%
\pgfpathlineto{\pgfqpoint{3.725899in}{2.434205in}}%
\pgfpathlineto{\pgfqpoint{3.579768in}{2.581333in}}%
\pgfpathlineto{\pgfqpoint{3.465766in}{2.693333in}}%
\pgfpathlineto{\pgfqpoint{3.349230in}{2.805333in}}%
\pgfpathlineto{\pgfqpoint{3.066638in}{3.066667in}}%
\pgfpathlineto{\pgfqpoint{2.940566in}{3.178667in}}%
\pgfpathlineto{\pgfqpoint{2.678352in}{3.402667in}}%
\pgfpathlineto{\pgfqpoint{2.541438in}{3.514667in}}%
\pgfpathlineto{\pgfqpoint{2.303500in}{3.701333in}}%
\pgfpathlineto{\pgfqpoint{2.082586in}{3.865157in}}%
\pgfpathlineto{\pgfqpoint{1.998092in}{3.925333in}}%
\pgfpathlineto{\pgfqpoint{1.882182in}{4.005430in}}%
\pgfpathlineto{\pgfqpoint{1.641697in}{4.162312in}}%
\pgfpathlineto{\pgfqpoint{1.541028in}{4.224000in}}%
\pgfpathlineto{\pgfqpoint{1.529424in}{4.224000in}}%
\pgfpathlineto{\pgfqpoint{1.721859in}{4.104700in}}%
\pgfpathlineto{\pgfqpoint{1.824720in}{4.037333in}}%
\pgfpathlineto{\pgfqpoint{1.934948in}{3.962667in}}%
\pgfpathlineto{\pgfqpoint{2.042505in}{3.887453in}}%
\pgfpathlineto{\pgfqpoint{2.310352in}{3.689487in}}%
\pgfpathlineto{\pgfqpoint{2.363152in}{3.649207in}}%
\pgfpathlineto{\pgfqpoint{2.487105in}{3.552000in}}%
\pgfpathlineto{\pgfqpoint{2.625499in}{3.440000in}}%
\pgfpathlineto{\pgfqpoint{2.890715in}{3.216000in}}%
\pgfpathlineto{\pgfqpoint{3.017982in}{3.104000in}}%
\pgfpathlineto{\pgfqpoint{3.303117in}{2.842667in}}%
\pgfpathlineto{\pgfqpoint{3.588796in}{2.565628in}}%
\pgfpathlineto{\pgfqpoint{3.648097in}{2.506667in}}%
\pgfpathlineto{\pgfqpoint{3.725899in}{2.427700in}}%
\pgfpathlineto{\pgfqpoint{3.865715in}{2.282667in}}%
\pgfpathlineto{\pgfqpoint{3.968949in}{2.173056in}}%
\pgfpathlineto{\pgfqpoint{4.006465in}{2.132751in}}%
\pgfpathlineto{\pgfqpoint{4.287030in}{1.820141in}}%
\pgfpathlineto{\pgfqpoint{4.527865in}{1.536000in}}%
\pgfpathlineto{\pgfqpoint{4.768000in}{1.235028in}}%
\pgfpathlineto{\pgfqpoint{4.768000in}{1.237333in}}%
\pgfusepath{fill}%
\end{pgfscope}%
\begin{pgfscope}%
\pgfpathrectangle{\pgfqpoint{0.800000in}{0.528000in}}{\pgfqpoint{3.968000in}{3.696000in}}%
\pgfusepath{clip}%
\pgfsetbuttcap%
\pgfsetroundjoin%
\definecolor{currentfill}{rgb}{0.277134,0.185228,0.489898}%
\pgfsetfillcolor{currentfill}%
\pgfsetlinewidth{0.000000pt}%
\definecolor{currentstroke}{rgb}{0.000000,0.000000,0.000000}%
\pgfsetstrokecolor{currentstroke}%
\pgfsetdash{}{0pt}%
\pgfpathmoveto{\pgfqpoint{2.364947in}{0.528000in}}%
\pgfpathlineto{\pgfqpoint{2.101217in}{0.789333in}}%
\pgfpathlineto{\pgfqpoint{1.815389in}{1.088000in}}%
\pgfpathlineto{\pgfqpoint{1.545275in}{1.386667in}}%
\pgfpathlineto{\pgfqpoint{1.441293in}{1.506276in}}%
\pgfpathlineto{\pgfqpoint{1.321051in}{1.648164in}}%
\pgfpathlineto{\pgfqpoint{1.198369in}{1.797333in}}%
\pgfpathlineto{\pgfqpoint{0.960323in}{2.100833in}}%
\pgfpathlineto{\pgfqpoint{0.852621in}{2.245333in}}%
\pgfpathlineto{\pgfqpoint{0.800000in}{2.317637in}}%
\pgfpathlineto{\pgfqpoint{0.800000in}{2.309054in}}%
\pgfpathlineto{\pgfqpoint{1.000404in}{2.040369in}}%
\pgfpathlineto{\pgfqpoint{1.102670in}{1.909333in}}%
\pgfpathlineto{\pgfqpoint{1.200808in}{1.786871in}}%
\pgfpathlineto{\pgfqpoint{1.443186in}{1.496904in}}%
\pgfpathlineto{\pgfqpoint{1.572062in}{1.349333in}}%
\pgfpathlineto{\pgfqpoint{1.676661in}{1.232567in}}%
\pgfpathlineto{\pgfqpoint{1.721859in}{1.182757in}}%
\pgfpathlineto{\pgfqpoint{1.856702in}{1.037067in}}%
\pgfpathlineto{\pgfqpoint{1.985861in}{0.901333in}}%
\pgfpathlineto{\pgfqpoint{2.094860in}{0.789333in}}%
\pgfpathlineto{\pgfqpoint{2.206150in}{0.677333in}}%
\pgfpathlineto{\pgfqpoint{2.282990in}{0.601412in}}%
\pgfpathlineto{\pgfqpoint{2.358449in}{0.528000in}}%
\pgfpathlineto{\pgfqpoint{2.363152in}{0.528000in}}%
\pgfpathmoveto{\pgfqpoint{4.768000in}{1.250840in}}%
\pgfpathlineto{\pgfqpoint{4.661086in}{1.386667in}}%
\pgfpathlineto{\pgfqpoint{4.567596in}{1.502436in}}%
\pgfpathlineto{\pgfqpoint{4.319109in}{1.797333in}}%
\pgfpathlineto{\pgfqpoint{4.214476in}{1.916419in}}%
\pgfpathlineto{\pgfqpoint{4.154297in}{1.984000in}}%
\pgfpathlineto{\pgfqpoint{4.086626in}{2.058900in}}%
\pgfpathlineto{\pgfqpoint{3.948697in}{2.208000in}}%
\pgfpathlineto{\pgfqpoint{3.842702in}{2.320000in}}%
\pgfpathlineto{\pgfqpoint{3.725899in}{2.440607in}}%
\pgfpathlineto{\pgfqpoint{3.586172in}{2.581333in}}%
\pgfpathlineto{\pgfqpoint{3.472286in}{2.693333in}}%
\pgfpathlineto{\pgfqpoint{3.355869in}{2.805333in}}%
\pgfpathlineto{\pgfqpoint{3.073574in}{3.066667in}}%
\pgfpathlineto{\pgfqpoint{2.947637in}{3.178667in}}%
\pgfpathlineto{\pgfqpoint{2.666752in}{3.418545in}}%
\pgfpathlineto{\pgfqpoint{2.549195in}{3.514667in}}%
\pgfpathlineto{\pgfqpoint{2.282990in}{3.723301in}}%
\pgfpathlineto{\pgfqpoint{2.162729in}{3.813333in}}%
\pgfpathlineto{\pgfqpoint{1.899710in}{4.000000in}}%
\pgfpathlineto{\pgfqpoint{1.788557in}{4.074667in}}%
\pgfpathlineto{\pgfqpoint{1.601616in}{4.194095in}}%
\pgfpathlineto{\pgfqpoint{1.552632in}{4.224000in}}%
\pgfpathlineto{\pgfqpoint{1.541028in}{4.224000in}}%
\pgfpathlineto{\pgfqpoint{1.723436in}{4.110530in}}%
\pgfpathlineto{\pgfqpoint{1.842101in}{4.032476in}}%
\pgfpathlineto{\pgfqpoint{2.102618in}{3.850667in}}%
\pgfpathlineto{\pgfqpoint{2.204513in}{3.776000in}}%
\pgfpathlineto{\pgfqpoint{2.483394in}{3.561208in}}%
\pgfpathlineto{\pgfqpoint{2.603636in}{3.464202in}}%
\pgfpathlineto{\pgfqpoint{2.730752in}{3.358931in}}%
\pgfpathlineto{\pgfqpoint{2.854733in}{3.253333in}}%
\pgfpathlineto{\pgfqpoint{2.982940in}{3.141333in}}%
\pgfpathlineto{\pgfqpoint{3.270062in}{2.880000in}}%
\pgfpathlineto{\pgfqpoint{3.542040in}{2.618667in}}%
\pgfpathlineto{\pgfqpoint{3.836389in}{2.320000in}}%
\pgfpathlineto{\pgfqpoint{4.114488in}{2.021333in}}%
\pgfpathlineto{\pgfqpoint{4.214823in}{1.909333in}}%
\pgfpathlineto{\pgfqpoint{4.327111in}{1.781013in}}%
\pgfpathlineto{\pgfqpoint{4.567596in}{1.494972in}}%
\pgfpathlineto{\pgfqpoint{4.687838in}{1.345410in}}%
\pgfpathlineto{\pgfqpoint{4.768000in}{1.242968in}}%
\pgfpathlineto{\pgfqpoint{4.768000in}{1.242968in}}%
\pgfusepath{fill}%
\end{pgfscope}%
\begin{pgfscope}%
\pgfpathrectangle{\pgfqpoint{0.800000in}{0.528000in}}{\pgfqpoint{3.968000in}{3.696000in}}%
\pgfusepath{clip}%
\pgfsetbuttcap%
\pgfsetroundjoin%
\definecolor{currentfill}{rgb}{0.277134,0.185228,0.489898}%
\pgfsetfillcolor{currentfill}%
\pgfsetlinewidth{0.000000pt}%
\definecolor{currentstroke}{rgb}{0.000000,0.000000,0.000000}%
\pgfsetstrokecolor{currentstroke}%
\pgfsetdash{}{0pt}%
\pgfpathmoveto{\pgfqpoint{2.358449in}{0.528000in}}%
\pgfpathlineto{\pgfqpoint{2.242909in}{0.640856in}}%
\pgfpathlineto{\pgfqpoint{2.127414in}{0.756422in}}%
\pgfpathlineto{\pgfqpoint{2.082586in}{0.801811in}}%
\pgfpathlineto{\pgfqpoint{1.802020in}{1.095675in}}%
\pgfpathlineto{\pgfqpoint{1.521455in}{1.406834in}}%
\pgfpathlineto{\pgfqpoint{1.280970in}{1.689071in}}%
\pgfpathlineto{\pgfqpoint{1.160727in}{1.836403in}}%
\pgfpathlineto{\pgfqpoint{0.929630in}{2.133333in}}%
\pgfpathlineto{\pgfqpoint{0.860317in}{2.226849in}}%
\pgfpathlineto{\pgfqpoint{0.819156in}{2.282667in}}%
\pgfpathlineto{\pgfqpoint{0.800000in}{2.309054in}}%
\pgfpathlineto{\pgfqpoint{0.800000in}{2.300470in}}%
\pgfpathlineto{\pgfqpoint{0.980248in}{2.058667in}}%
\pgfpathlineto{\pgfqpoint{1.056054in}{1.961169in}}%
\pgfpathlineto{\pgfqpoint{1.096569in}{1.909333in}}%
\pgfpathlineto{\pgfqpoint{1.200808in}{1.779415in}}%
\pgfpathlineto{\pgfqpoint{1.441293in}{1.492107in}}%
\pgfpathlineto{\pgfqpoint{1.565916in}{1.349333in}}%
\pgfpathlineto{\pgfqpoint{1.842101in}{1.046056in}}%
\pgfpathlineto{\pgfqpoint{1.979614in}{0.901333in}}%
\pgfpathlineto{\pgfqpoint{2.088503in}{0.789333in}}%
\pgfpathlineto{\pgfqpoint{2.202828in}{0.674278in}}%
\pgfpathlineto{\pgfqpoint{2.313532in}{0.565333in}}%
\pgfpathlineto{\pgfqpoint{2.352001in}{0.528000in}}%
\pgfpathmoveto{\pgfqpoint{4.768000in}{1.258712in}}%
\pgfpathlineto{\pgfqpoint{4.667162in}{1.386667in}}%
\pgfpathlineto{\pgfqpoint{4.567596in}{1.509797in}}%
\pgfpathlineto{\pgfqpoint{4.325249in}{1.797333in}}%
\pgfpathlineto{\pgfqpoint{4.046545in}{2.109281in}}%
\pgfpathlineto{\pgfqpoint{3.765980in}{2.405958in}}%
\pgfpathlineto{\pgfqpoint{3.629961in}{2.544000in}}%
\pgfpathlineto{\pgfqpoint{3.325091in}{2.840843in}}%
\pgfpathlineto{\pgfqpoint{3.038923in}{3.104000in}}%
\pgfpathlineto{\pgfqpoint{2.912068in}{3.216000in}}%
\pgfpathlineto{\pgfqpoint{2.643717in}{3.443791in}}%
\pgfpathlineto{\pgfqpoint{2.510543in}{3.552000in}}%
\pgfpathlineto{\pgfqpoint{2.403232in}{3.636964in}}%
\pgfpathlineto{\pgfqpoint{2.271054in}{3.738667in}}%
\pgfpathlineto{\pgfqpoint{2.016377in}{3.925333in}}%
\pgfpathlineto{\pgfqpoint{1.823760in}{4.057583in}}%
\pgfpathlineto{\pgfqpoint{1.761939in}{4.098841in}}%
\pgfpathlineto{\pgfqpoint{1.641697in}{4.175959in}}%
\pgfpathlineto{\pgfqpoint{1.564110in}{4.224000in}}%
\pgfpathlineto{\pgfqpoint{1.552632in}{4.224000in}}%
\pgfpathlineto{\pgfqpoint{1.681778in}{4.143902in}}%
\pgfpathlineto{\pgfqpoint{1.922263in}{3.984557in}}%
\pgfpathlineto{\pgfqpoint{2.004512in}{3.927278in}}%
\pgfpathlineto{\pgfqpoint{2.059730in}{3.888000in}}%
\pgfpathlineto{\pgfqpoint{2.282990in}{3.723301in}}%
\pgfpathlineto{\pgfqpoint{2.363152in}{3.661933in}}%
\pgfpathlineto{\pgfqpoint{2.483394in}{3.567451in}}%
\pgfpathlineto{\pgfqpoint{2.603636in}{3.470474in}}%
\pgfpathlineto{\pgfqpoint{2.730470in}{3.365333in}}%
\pgfpathlineto{\pgfqpoint{2.861897in}{3.253333in}}%
\pgfpathlineto{\pgfqpoint{2.989966in}{3.141333in}}%
\pgfpathlineto{\pgfqpoint{3.260707in}{2.894696in}}%
\pgfpathlineto{\pgfqpoint{3.325091in}{2.834542in}}%
\pgfpathlineto{\pgfqpoint{3.433768in}{2.730667in}}%
\pgfpathlineto{\pgfqpoint{3.548483in}{2.618667in}}%
\pgfpathlineto{\pgfqpoint{3.842702in}{2.320000in}}%
\pgfpathlineto{\pgfqpoint{4.120678in}{2.021333in}}%
\pgfpathlineto{\pgfqpoint{4.383344in}{1.722667in}}%
\pgfpathlineto{\pgfqpoint{4.447354in}{1.647349in}}%
\pgfpathlineto{\pgfqpoint{4.580487in}{1.486659in}}%
\pgfpathlineto{\pgfqpoint{4.690858in}{1.349333in}}%
\pgfpathlineto{\pgfqpoint{4.768000in}{1.250840in}}%
\pgfpathlineto{\pgfqpoint{4.768000in}{1.250840in}}%
\pgfusepath{fill}%
\end{pgfscope}%
\begin{pgfscope}%
\pgfpathrectangle{\pgfqpoint{0.800000in}{0.528000in}}{\pgfqpoint{3.968000in}{3.696000in}}%
\pgfusepath{clip}%
\pgfsetbuttcap%
\pgfsetroundjoin%
\definecolor{currentfill}{rgb}{0.276194,0.190074,0.493001}%
\pgfsetfillcolor{currentfill}%
\pgfsetlinewidth{0.000000pt}%
\definecolor{currentstroke}{rgb}{0.000000,0.000000,0.000000}%
\pgfsetstrokecolor{currentstroke}%
\pgfsetdash{}{0pt}%
\pgfpathmoveto{\pgfqpoint{2.352001in}{0.528000in}}%
\pgfpathlineto{\pgfqpoint{2.237418in}{0.640000in}}%
\pgfpathlineto{\pgfqpoint{2.157619in}{0.719443in}}%
\pgfpathlineto{\pgfqpoint{2.015663in}{0.864000in}}%
\pgfpathlineto{\pgfqpoint{1.734104in}{1.162667in}}%
\pgfpathlineto{\pgfqpoint{1.632500in}{1.274667in}}%
\pgfpathlineto{\pgfqpoint{1.533052in}{1.386667in}}%
\pgfpathlineto{\pgfqpoint{1.413050in}{1.524973in}}%
\pgfpathlineto{\pgfqpoint{1.309126in}{1.648000in}}%
\pgfpathlineto{\pgfqpoint{1.200808in}{1.779415in}}%
\pgfpathlineto{\pgfqpoint{1.096569in}{1.909333in}}%
\pgfpathlineto{\pgfqpoint{1.022353in}{2.004444in}}%
\pgfpathlineto{\pgfqpoint{0.980248in}{2.058667in}}%
\pgfpathlineto{\pgfqpoint{0.880162in}{2.191223in}}%
\pgfpathlineto{\pgfqpoint{0.800000in}{2.300470in}}%
\pgfpathlineto{\pgfqpoint{0.800000in}{2.291886in}}%
\pgfpathlineto{\pgfqpoint{0.974127in}{2.058667in}}%
\pgfpathlineto{\pgfqpoint{1.061057in}{1.946667in}}%
\pgfpathlineto{\pgfqpoint{1.160727in}{1.821414in}}%
\pgfpathlineto{\pgfqpoint{1.401212in}{1.531758in}}%
\pgfpathlineto{\pgfqpoint{1.526940in}{1.386667in}}%
\pgfpathlineto{\pgfqpoint{1.802020in}{1.082413in}}%
\pgfpathlineto{\pgfqpoint{2.082586in}{0.788896in}}%
\pgfpathlineto{\pgfqpoint{2.231082in}{0.640000in}}%
\pgfpathlineto{\pgfqpoint{2.345552in}{0.528000in}}%
\pgfpathmoveto{\pgfqpoint{4.768000in}{1.266584in}}%
\pgfpathlineto{\pgfqpoint{4.673239in}{1.386667in}}%
\pgfpathlineto{\pgfqpoint{4.567596in}{1.517159in}}%
\pgfpathlineto{\pgfqpoint{4.327111in}{1.802142in}}%
\pgfpathlineto{\pgfqpoint{4.046545in}{2.115928in}}%
\pgfpathlineto{\pgfqpoint{3.765980in}{2.412370in}}%
\pgfpathlineto{\pgfqpoint{3.636328in}{2.544000in}}%
\pgfpathlineto{\pgfqpoint{3.561367in}{2.618667in}}%
\pgfpathlineto{\pgfqpoint{3.445333in}{2.732130in}}%
\pgfpathlineto{\pgfqpoint{3.325091in}{2.847039in}}%
\pgfpathlineto{\pgfqpoint{3.227277in}{2.938225in}}%
\pgfpathlineto{\pgfqpoint{3.164768in}{2.996250in}}%
\pgfpathlineto{\pgfqpoint{3.013881in}{3.132544in}}%
\pgfpathlineto{\pgfqpoint{2.919186in}{3.216000in}}%
\pgfpathlineto{\pgfqpoint{2.643717in}{3.449927in}}%
\pgfpathlineto{\pgfqpoint{2.518356in}{3.552000in}}%
\pgfpathlineto{\pgfqpoint{2.376528in}{3.664000in}}%
\pgfpathlineto{\pgfqpoint{2.279444in}{3.738667in}}%
\pgfpathlineto{\pgfqpoint{2.025437in}{3.925333in}}%
\pgfpathlineto{\pgfqpoint{1.802020in}{4.079088in}}%
\pgfpathlineto{\pgfqpoint{1.575173in}{4.224000in}}%
\pgfpathlineto{\pgfqpoint{1.564110in}{4.224000in}}%
\pgfpathlineto{\pgfqpoint{1.776950in}{4.088649in}}%
\pgfpathlineto{\pgfqpoint{1.842101in}{4.045680in}}%
\pgfpathlineto{\pgfqpoint{1.963410in}{3.962667in}}%
\pgfpathlineto{\pgfqpoint{2.082586in}{3.878049in}}%
\pgfpathlineto{\pgfqpoint{2.368543in}{3.664000in}}%
\pgfpathlineto{\pgfqpoint{2.510543in}{3.552000in}}%
\pgfpathlineto{\pgfqpoint{2.608913in}{3.472418in}}%
\pgfpathlineto{\pgfqpoint{2.737780in}{3.365333in}}%
\pgfpathlineto{\pgfqpoint{2.869062in}{3.253333in}}%
\pgfpathlineto{\pgfqpoint{2.996992in}{3.141333in}}%
\pgfpathlineto{\pgfqpoint{3.244929in}{2.916047in}}%
\pgfpathlineto{\pgfqpoint{3.343769in}{2.822731in}}%
\pgfpathlineto{\pgfqpoint{3.405253in}{2.764466in}}%
\pgfpathlineto{\pgfqpoint{3.525495in}{2.647678in}}%
\pgfpathlineto{\pgfqpoint{3.667086in}{2.506667in}}%
\pgfpathlineto{\pgfqpoint{3.954903in}{2.208000in}}%
\pgfpathlineto{\pgfqpoint{4.058734in}{2.096000in}}%
\pgfpathlineto{\pgfqpoint{4.166788in}{1.976946in}}%
\pgfpathlineto{\pgfqpoint{4.421213in}{1.685333in}}%
\pgfpathlineto{\pgfqpoint{4.527515in}{1.558468in}}%
\pgfpathlineto{\pgfqpoint{4.647758in}{1.410940in}}%
\pgfpathlineto{\pgfqpoint{4.755576in}{1.274667in}}%
\pgfpathlineto{\pgfqpoint{4.768000in}{1.258712in}}%
\pgfpathlineto{\pgfqpoint{4.768000in}{1.258712in}}%
\pgfusepath{fill}%
\end{pgfscope}%
\begin{pgfscope}%
\pgfpathrectangle{\pgfqpoint{0.800000in}{0.528000in}}{\pgfqpoint{3.968000in}{3.696000in}}%
\pgfusepath{clip}%
\pgfsetbuttcap%
\pgfsetroundjoin%
\definecolor{currentfill}{rgb}{0.276194,0.190074,0.493001}%
\pgfsetfillcolor{currentfill}%
\pgfsetlinewidth{0.000000pt}%
\definecolor{currentstroke}{rgb}{0.000000,0.000000,0.000000}%
\pgfsetstrokecolor{currentstroke}%
\pgfsetdash{}{0pt}%
\pgfpathmoveto{\pgfqpoint{2.345552in}{0.528000in}}%
\pgfpathlineto{\pgfqpoint{2.231082in}{0.640000in}}%
\pgfpathlineto{\pgfqpoint{2.119005in}{0.752000in}}%
\pgfpathlineto{\pgfqpoint{2.002424in}{0.871177in}}%
\pgfpathlineto{\pgfqpoint{1.866754in}{1.013333in}}%
\pgfpathlineto{\pgfqpoint{1.761939in}{1.125624in}}%
\pgfpathlineto{\pgfqpoint{1.626441in}{1.274667in}}%
\pgfpathlineto{\pgfqpoint{1.542728in}{1.369148in}}%
\pgfpathlineto{\pgfqpoint{1.481374in}{1.438842in}}%
\pgfpathlineto{\pgfqpoint{1.361131in}{1.578898in}}%
\pgfpathlineto{\pgfqpoint{1.240889in}{1.722893in}}%
\pgfpathlineto{\pgfqpoint{1.118346in}{1.874143in}}%
\pgfpathlineto{\pgfqpoint{1.000404in}{2.024409in}}%
\pgfpathlineto{\pgfqpoint{0.800000in}{2.291886in}}%
\pgfpathlineto{\pgfqpoint{0.800462in}{2.282667in}}%
\pgfpathlineto{\pgfqpoint{0.883144in}{2.170667in}}%
\pgfpathlineto{\pgfqpoint{0.968005in}{2.058667in}}%
\pgfpathlineto{\pgfqpoint{1.054990in}{1.946667in}}%
\pgfpathlineto{\pgfqpoint{1.160727in}{1.813944in}}%
\pgfpathlineto{\pgfqpoint{1.391611in}{1.536000in}}%
\pgfpathlineto{\pgfqpoint{1.488233in}{1.424000in}}%
\pgfpathlineto{\pgfqpoint{1.761939in}{1.119069in}}%
\pgfpathlineto{\pgfqpoint{2.042505in}{0.823480in}}%
\pgfpathlineto{\pgfqpoint{2.339103in}{0.528000in}}%
\pgfpathmoveto{\pgfqpoint{4.768000in}{1.274456in}}%
\pgfpathlineto{\pgfqpoint{4.679315in}{1.386667in}}%
\pgfpathlineto{\pgfqpoint{4.588750in}{1.498667in}}%
\pgfpathlineto{\pgfqpoint{4.487434in}{1.621137in}}%
\pgfpathlineto{\pgfqpoint{4.367192in}{1.762772in}}%
\pgfpathlineto{\pgfqpoint{4.086626in}{2.078876in}}%
\pgfpathlineto{\pgfqpoint{3.966384in}{2.208969in}}%
\pgfpathlineto{\pgfqpoint{3.679745in}{2.506667in}}%
\pgfpathlineto{\pgfqpoint{3.565576in}{2.620820in}}%
\pgfpathlineto{\pgfqpoint{3.445333in}{2.738313in}}%
\pgfpathlineto{\pgfqpoint{3.296619in}{2.880000in}}%
\pgfpathlineto{\pgfqpoint{3.176050in}{2.992000in}}%
\pgfpathlineto{\pgfqpoint{3.052646in}{3.104000in}}%
\pgfpathlineto{\pgfqpoint{2.796447in}{3.328000in}}%
\pgfpathlineto{\pgfqpoint{2.507219in}{3.567142in}}%
\pgfpathlineto{\pgfqpoint{2.384513in}{3.664000in}}%
\pgfpathlineto{\pgfqpoint{2.282990in}{3.742243in}}%
\pgfpathlineto{\pgfqpoint{2.002424in}{3.948079in}}%
\pgfpathlineto{\pgfqpoint{1.873781in}{4.037333in}}%
\pgfpathlineto{\pgfqpoint{1.761188in}{4.112700in}}%
\pgfpathlineto{\pgfqpoint{1.632214in}{4.195500in}}%
\pgfpathlineto{\pgfqpoint{1.586237in}{4.224000in}}%
\pgfpathlineto{\pgfqpoint{1.575173in}{4.224000in}}%
\pgfpathlineto{\pgfqpoint{1.761939in}{4.105526in}}%
\pgfpathlineto{\pgfqpoint{2.025437in}{3.925333in}}%
\pgfpathlineto{\pgfqpoint{2.129110in}{3.850667in}}%
\pgfpathlineto{\pgfqpoint{2.376528in}{3.664000in}}%
\pgfpathlineto{\pgfqpoint{2.483394in}{3.579936in}}%
\pgfpathlineto{\pgfqpoint{2.610398in}{3.477333in}}%
\pgfpathlineto{\pgfqpoint{2.745089in}{3.365333in}}%
\pgfpathlineto{\pgfqpoint{2.876227in}{3.253333in}}%
\pgfpathlineto{\pgfqpoint{2.964364in}{3.176397in}}%
\pgfpathlineto{\pgfqpoint{3.087366in}{3.066667in}}%
\pgfpathlineto{\pgfqpoint{3.209930in}{2.954667in}}%
\pgfpathlineto{\pgfqpoint{3.523483in}{2.656000in}}%
\pgfpathlineto{\pgfqpoint{3.636328in}{2.544000in}}%
\pgfpathlineto{\pgfqpoint{3.746824in}{2.432000in}}%
\pgfpathlineto{\pgfqpoint{3.855092in}{2.320000in}}%
\pgfpathlineto{\pgfqpoint{3.966384in}{2.202368in}}%
\pgfpathlineto{\pgfqpoint{4.246949in}{1.893662in}}%
\pgfpathlineto{\pgfqpoint{4.490224in}{1.610667in}}%
\pgfpathlineto{\pgfqpoint{4.732483in}{1.312000in}}%
\pgfpathlineto{\pgfqpoint{4.768000in}{1.266584in}}%
\pgfpathlineto{\pgfqpoint{4.768000in}{1.266584in}}%
\pgfusepath{fill}%
\end{pgfscope}%
\begin{pgfscope}%
\pgfpathrectangle{\pgfqpoint{0.800000in}{0.528000in}}{\pgfqpoint{3.968000in}{3.696000in}}%
\pgfusepath{clip}%
\pgfsetbuttcap%
\pgfsetroundjoin%
\definecolor{currentfill}{rgb}{0.276194,0.190074,0.493001}%
\pgfsetfillcolor{currentfill}%
\pgfsetlinewidth{0.000000pt}%
\definecolor{currentstroke}{rgb}{0.000000,0.000000,0.000000}%
\pgfsetstrokecolor{currentstroke}%
\pgfsetdash{}{0pt}%
\pgfpathmoveto{\pgfqpoint{2.339103in}{0.528000in}}%
\pgfpathlineto{\pgfqpoint{2.224746in}{0.640000in}}%
\pgfpathlineto{\pgfqpoint{1.931382in}{0.938667in}}%
\pgfpathlineto{\pgfqpoint{1.653972in}{1.237333in}}%
\pgfpathlineto{\pgfqpoint{1.539405in}{1.366054in}}%
\pgfpathlineto{\pgfqpoint{1.481374in}{1.431869in}}%
\pgfpathlineto{\pgfqpoint{1.354745in}{1.579282in}}%
\pgfpathlineto{\pgfqpoint{1.235107in}{1.722667in}}%
\pgfpathlineto{\pgfqpoint{1.144041in}{1.834667in}}%
\pgfpathlineto{\pgfqpoint{1.080566in}{1.914139in}}%
\pgfpathlineto{\pgfqpoint{0.960323in}{2.068702in}}%
\pgfpathlineto{\pgfqpoint{0.800000in}{2.283303in}}%
\pgfpathlineto{\pgfqpoint{0.800000in}{2.282667in}}%
\pgfpathlineto{\pgfqpoint{0.800000in}{2.274969in}}%
\pgfpathlineto{\pgfqpoint{0.905179in}{2.133333in}}%
\pgfpathlineto{\pgfqpoint{0.978197in}{2.037982in}}%
\pgfpathlineto{\pgfqpoint{1.019740in}{1.984000in}}%
\pgfpathlineto{\pgfqpoint{1.120646in}{1.856292in}}%
\pgfpathlineto{\pgfqpoint{1.357100in}{1.569578in}}%
\pgfpathlineto{\pgfqpoint{1.401212in}{1.517765in}}%
\pgfpathlineto{\pgfqpoint{1.521455in}{1.379191in}}%
\pgfpathlineto{\pgfqpoint{1.802020in}{1.069340in}}%
\pgfpathlineto{\pgfqpoint{2.082586in}{0.776280in}}%
\pgfpathlineto{\pgfqpoint{2.218410in}{0.640000in}}%
\pgfpathlineto{\pgfqpoint{2.332654in}{0.528000in}}%
\pgfpathmoveto{\pgfqpoint{4.768000in}{1.282106in}}%
\pgfpathlineto{\pgfqpoint{4.527515in}{1.580318in}}%
\pgfpathlineto{\pgfqpoint{4.278118in}{1.872000in}}%
\pgfpathlineto{\pgfqpoint{4.006465in}{2.172537in}}%
\pgfpathlineto{\pgfqpoint{3.722832in}{2.469333in}}%
\pgfpathlineto{\pgfqpoint{3.608712in}{2.584180in}}%
\pgfpathlineto{\pgfqpoint{3.565576in}{2.627032in}}%
\pgfpathlineto{\pgfqpoint{3.420924in}{2.768000in}}%
\pgfpathlineto{\pgfqpoint{3.303155in}{2.880000in}}%
\pgfpathlineto{\pgfqpoint{3.182706in}{2.992000in}}%
\pgfpathlineto{\pgfqpoint{3.059427in}{3.104000in}}%
\pgfpathlineto{\pgfqpoint{2.803707in}{3.328000in}}%
\pgfpathlineto{\pgfqpoint{2.523475in}{3.560214in}}%
\pgfpathlineto{\pgfqpoint{2.392498in}{3.664000in}}%
\pgfpathlineto{\pgfqpoint{2.282990in}{3.748437in}}%
\pgfpathlineto{\pgfqpoint{2.002424in}{3.954505in}}%
\pgfpathlineto{\pgfqpoint{1.878812in}{4.040472in}}%
\pgfpathlineto{\pgfqpoint{1.737492in}{4.134772in}}%
\pgfpathlineto{\pgfqpoint{1.641697in}{4.196185in}}%
\pgfpathlineto{\pgfqpoint{1.597301in}{4.224000in}}%
\pgfpathlineto{\pgfqpoint{1.586237in}{4.224000in}}%
\pgfpathlineto{\pgfqpoint{1.762252in}{4.112000in}}%
\pgfpathlineto{\pgfqpoint{2.002424in}{3.948079in}}%
\pgfpathlineto{\pgfqpoint{2.106616in}{3.873050in}}%
\pgfpathlineto{\pgfqpoint{2.162747in}{3.832258in}}%
\pgfpathlineto{\pgfqpoint{2.287669in}{3.738667in}}%
\pgfpathlineto{\pgfqpoint{2.572184in}{3.514667in}}%
\pgfpathlineto{\pgfqpoint{2.844121in}{3.287226in}}%
\pgfpathlineto{\pgfqpoint{2.945657in}{3.198576in}}%
\pgfpathlineto{\pgfqpoint{3.010853in}{3.141333in}}%
\pgfpathlineto{\pgfqpoint{3.135238in}{3.029333in}}%
\pgfpathlineto{\pgfqpoint{3.256734in}{2.917333in}}%
\pgfpathlineto{\pgfqpoint{3.529846in}{2.656000in}}%
\pgfpathlineto{\pgfqpoint{3.645737in}{2.540951in}}%
\pgfpathlineto{\pgfqpoint{3.789376in}{2.394667in}}%
\pgfpathlineto{\pgfqpoint{3.896821in}{2.282667in}}%
\pgfpathlineto{\pgfqpoint{4.006465in}{2.165948in}}%
\pgfpathlineto{\pgfqpoint{4.138930in}{2.021333in}}%
\pgfpathlineto{\pgfqpoint{4.239168in}{1.909333in}}%
\pgfpathlineto{\pgfqpoint{4.337267in}{1.797333in}}%
\pgfpathlineto{\pgfqpoint{4.433288in}{1.685333in}}%
\pgfpathlineto{\pgfqpoint{4.527515in}{1.573164in}}%
\pgfpathlineto{\pgfqpoint{4.654462in}{1.417755in}}%
\pgfpathlineto{\pgfqpoint{4.768000in}{1.274456in}}%
\pgfpathlineto{\pgfqpoint{4.768000in}{1.274667in}}%
\pgfusepath{fill}%
\end{pgfscope}%
\begin{pgfscope}%
\pgfpathrectangle{\pgfqpoint{0.800000in}{0.528000in}}{\pgfqpoint{3.968000in}{3.696000in}}%
\pgfusepath{clip}%
\pgfsetbuttcap%
\pgfsetroundjoin%
\definecolor{currentfill}{rgb}{0.276194,0.190074,0.493001}%
\pgfsetfillcolor{currentfill}%
\pgfsetlinewidth{0.000000pt}%
\definecolor{currentstroke}{rgb}{0.000000,0.000000,0.000000}%
\pgfsetstrokecolor{currentstroke}%
\pgfsetdash{}{0pt}%
\pgfpathmoveto{\pgfqpoint{2.332654in}{0.528000in}}%
\pgfpathlineto{\pgfqpoint{2.218410in}{0.640000in}}%
\pgfpathlineto{\pgfqpoint{1.923818in}{0.940116in}}%
\pgfpathlineto{\pgfqpoint{1.882182in}{0.983936in}}%
\pgfpathlineto{\pgfqpoint{1.601616in}{1.288846in}}%
\pgfpathlineto{\pgfqpoint{1.481374in}{1.424897in}}%
\pgfpathlineto{\pgfqpoint{1.353837in}{1.573333in}}%
\pgfpathlineto{\pgfqpoint{1.240889in}{1.708409in}}%
\pgfpathlineto{\pgfqpoint{1.040485in}{1.957418in}}%
\pgfpathlineto{\pgfqpoint{0.933457in}{2.096000in}}%
\pgfpathlineto{\pgfqpoint{0.840081in}{2.220450in}}%
\pgfpathlineto{\pgfqpoint{0.800000in}{2.274969in}}%
\pgfpathlineto{\pgfqpoint{0.800000in}{2.266655in}}%
\pgfpathlineto{\pgfqpoint{0.899126in}{2.133333in}}%
\pgfpathlineto{\pgfqpoint{0.974668in}{2.034695in}}%
\pgfpathlineto{\pgfqpoint{1.013707in}{1.984000in}}%
\pgfpathlineto{\pgfqpoint{1.102111in}{1.872000in}}%
\pgfpathlineto{\pgfqpoint{1.200808in}{1.749875in}}%
\pgfpathlineto{\pgfqpoint{1.321051in}{1.605038in}}%
\pgfpathlineto{\pgfqpoint{1.454034in}{1.449466in}}%
\pgfpathlineto{\pgfqpoint{1.574953in}{1.312000in}}%
\pgfpathlineto{\pgfqpoint{1.681778in}{1.193287in}}%
\pgfpathlineto{\pgfqpoint{1.962343in}{0.893502in}}%
\pgfpathlineto{\pgfqpoint{2.100325in}{0.752000in}}%
\pgfpathlineto{\pgfqpoint{2.326205in}{0.528000in}}%
\pgfpathmoveto{\pgfqpoint{4.768000in}{1.289751in}}%
\pgfpathlineto{\pgfqpoint{4.539256in}{1.573333in}}%
\pgfpathlineto{\pgfqpoint{4.437841in}{1.694194in}}%
\pgfpathlineto{\pgfqpoint{4.316802in}{1.834667in}}%
\pgfpathlineto{\pgfqpoint{4.046545in}{2.135809in}}%
\pgfpathlineto{\pgfqpoint{3.909053in}{2.282667in}}%
\pgfpathlineto{\pgfqpoint{3.617864in}{2.581333in}}%
\pgfpathlineto{\pgfqpoint{3.349200in}{2.842667in}}%
\pgfpathlineto{\pgfqpoint{3.229779in}{2.954667in}}%
\pgfpathlineto{\pgfqpoint{3.107584in}{3.066667in}}%
\pgfpathlineto{\pgfqpoint{2.827950in}{3.312938in}}%
\pgfpathlineto{\pgfqpoint{2.763960in}{3.367854in}}%
\pgfpathlineto{\pgfqpoint{2.483394in}{3.598449in}}%
\pgfpathlineto{\pgfqpoint{2.352427in}{3.701333in}}%
\pgfpathlineto{\pgfqpoint{2.242909in}{3.785024in}}%
\pgfpathlineto{\pgfqpoint{2.122667in}{3.874303in}}%
\pgfpathlineto{\pgfqpoint{2.023845in}{3.945286in}}%
\pgfpathlineto{\pgfqpoint{1.962343in}{3.989075in}}%
\pgfpathlineto{\pgfqpoint{1.886591in}{4.041441in}}%
\pgfpathlineto{\pgfqpoint{1.837898in}{4.074667in}}%
\pgfpathlineto{\pgfqpoint{1.714011in}{4.156643in}}%
\pgfpathlineto{\pgfqpoint{1.608065in}{4.224000in}}%
\pgfpathlineto{\pgfqpoint{1.597301in}{4.224000in}}%
\pgfpathlineto{\pgfqpoint{1.737492in}{4.134772in}}%
\pgfpathlineto{\pgfqpoint{1.842101in}{4.065307in}}%
\pgfpathlineto{\pgfqpoint{1.962343in}{3.982660in}}%
\pgfpathlineto{\pgfqpoint{2.082586in}{3.897164in}}%
\pgfpathlineto{\pgfqpoint{2.202828in}{3.808873in}}%
\pgfpathlineto{\pgfqpoint{2.323071in}{3.717770in}}%
\pgfpathlineto{\pgfqpoint{2.461276in}{3.609936in}}%
\pgfpathlineto{\pgfqpoint{2.579696in}{3.514667in}}%
\pgfpathlineto{\pgfqpoint{2.847227in}{3.290667in}}%
\pgfpathlineto{\pgfqpoint{3.141935in}{3.029333in}}%
\pgfpathlineto{\pgfqpoint{3.263311in}{2.917333in}}%
\pgfpathlineto{\pgfqpoint{3.536154in}{2.656000in}}%
\pgfpathlineto{\pgfqpoint{3.648975in}{2.544000in}}%
\pgfpathlineto{\pgfqpoint{3.765980in}{2.425193in}}%
\pgfpathlineto{\pgfqpoint{3.902937in}{2.282667in}}%
\pgfpathlineto{\pgfqpoint{4.019229in}{2.158777in}}%
\pgfpathlineto{\pgfqpoint{4.144964in}{2.021333in}}%
\pgfpathlineto{\pgfqpoint{4.227666in}{1.928705in}}%
\pgfpathlineto{\pgfqpoint{4.287030in}{1.861855in}}%
\pgfpathlineto{\pgfqpoint{4.533317in}{1.573333in}}%
\pgfpathlineto{\pgfqpoint{4.625168in}{1.461333in}}%
\pgfpathlineto{\pgfqpoint{4.727919in}{1.333055in}}%
\pgfpathlineto{\pgfqpoint{4.768000in}{1.282106in}}%
\pgfpathlineto{\pgfqpoint{4.768000in}{1.282106in}}%
\pgfusepath{fill}%
\end{pgfscope}%
\begin{pgfscope}%
\pgfpathrectangle{\pgfqpoint{0.800000in}{0.528000in}}{\pgfqpoint{3.968000in}{3.696000in}}%
\pgfusepath{clip}%
\pgfsetbuttcap%
\pgfsetroundjoin%
\definecolor{currentfill}{rgb}{0.275191,0.194905,0.496005}%
\pgfsetfillcolor{currentfill}%
\pgfsetlinewidth{0.000000pt}%
\definecolor{currentstroke}{rgb}{0.000000,0.000000,0.000000}%
\pgfsetstrokecolor{currentstroke}%
\pgfsetdash{}{0pt}%
\pgfpathmoveto{\pgfqpoint{2.326205in}{0.528000in}}%
\pgfpathlineto{\pgfqpoint{2.212074in}{0.640000in}}%
\pgfpathlineto{\pgfqpoint{1.954822in}{0.901333in}}%
\pgfpathlineto{\pgfqpoint{1.842101in}{1.019952in}}%
\pgfpathlineto{\pgfqpoint{1.561535in}{1.327078in}}%
\pgfpathlineto{\pgfqpoint{1.441293in}{1.464169in}}%
\pgfpathlineto{\pgfqpoint{1.192528in}{1.760000in}}%
\pgfpathlineto{\pgfqpoint{1.102111in}{1.872000in}}%
\pgfpathlineto{\pgfqpoint{1.000404in}{2.001079in}}%
\pgfpathlineto{\pgfqpoint{0.899126in}{2.133333in}}%
\pgfpathlineto{\pgfqpoint{0.815660in}{2.245333in}}%
\pgfpathlineto{\pgfqpoint{0.800000in}{2.266655in}}%
\pgfpathlineto{\pgfqpoint{0.800000in}{2.258341in}}%
\pgfpathlineto{\pgfqpoint{0.893073in}{2.133333in}}%
\pgfpathlineto{\pgfqpoint{0.978714in}{2.021333in}}%
\pgfpathlineto{\pgfqpoint{1.080566in}{1.891509in}}%
\pgfpathlineto{\pgfqpoint{1.186598in}{1.760000in}}%
\pgfpathlineto{\pgfqpoint{1.262584in}{1.668208in}}%
\pgfpathlineto{\pgfqpoint{1.321051in}{1.598017in}}%
\pgfpathlineto{\pgfqpoint{1.441293in}{1.457292in}}%
\pgfpathlineto{\pgfqpoint{1.568928in}{1.312000in}}%
\pgfpathlineto{\pgfqpoint{1.681778in}{1.186718in}}%
\pgfpathlineto{\pgfqpoint{1.962343in}{0.887163in}}%
\pgfpathlineto{\pgfqpoint{2.094098in}{0.752000in}}%
\pgfpathlineto{\pgfqpoint{2.319844in}{0.528000in}}%
\pgfpathlineto{\pgfqpoint{2.323071in}{0.528000in}}%
\pgfpathmoveto{\pgfqpoint{4.768000in}{1.297395in}}%
\pgfpathlineto{\pgfqpoint{4.567596in}{1.546317in}}%
\pgfpathlineto{\pgfqpoint{4.447354in}{1.689985in}}%
\pgfpathlineto{\pgfqpoint{4.322820in}{1.834667in}}%
\pgfpathlineto{\pgfqpoint{4.046545in}{2.142293in}}%
\pgfpathlineto{\pgfqpoint{3.915168in}{2.282667in}}%
\pgfpathlineto{\pgfqpoint{3.825839in}{2.375756in}}%
\pgfpathlineto{\pgfqpoint{3.765980in}{2.437875in}}%
\pgfpathlineto{\pgfqpoint{3.472376in}{2.730667in}}%
\pgfpathlineto{\pgfqpoint{3.355697in}{2.842667in}}%
\pgfpathlineto{\pgfqpoint{3.236395in}{2.954667in}}%
\pgfpathlineto{\pgfqpoint{3.114323in}{3.066667in}}%
\pgfpathlineto{\pgfqpoint{2.817808in}{3.328000in}}%
\pgfpathlineto{\pgfqpoint{2.723879in}{3.407698in}}%
\pgfpathlineto{\pgfqpoint{2.594719in}{3.514667in}}%
\pgfpathlineto{\pgfqpoint{2.311981in}{3.738667in}}%
\pgfpathlineto{\pgfqpoint{2.202828in}{3.821328in}}%
\pgfpathlineto{\pgfqpoint{2.082586in}{3.909750in}}%
\pgfpathlineto{\pgfqpoint{2.002424in}{3.967244in}}%
\pgfpathlineto{\pgfqpoint{1.882182in}{4.050962in}}%
\pgfpathlineto{\pgfqpoint{1.761939in}{4.131769in}}%
\pgfpathlineto{\pgfqpoint{1.641697in}{4.209489in}}%
\pgfpathlineto{\pgfqpoint{1.618636in}{4.224000in}}%
\pgfpathlineto{\pgfqpoint{1.608065in}{4.224000in}}%
\pgfpathlineto{\pgfqpoint{1.725239in}{4.149333in}}%
\pgfpathlineto{\pgfqpoint{1.853153in}{4.064372in}}%
\pgfpathlineto{\pgfqpoint{1.962343in}{3.989075in}}%
\pgfpathlineto{\pgfqpoint{2.082586in}{3.903457in}}%
\pgfpathlineto{\pgfqpoint{2.205252in}{3.813333in}}%
\pgfpathlineto{\pgfqpoint{2.352427in}{3.701333in}}%
\pgfpathlineto{\pgfqpoint{2.587207in}{3.514667in}}%
\pgfpathlineto{\pgfqpoint{2.854226in}{3.290667in}}%
\pgfpathlineto{\pgfqpoint{3.148633in}{3.029333in}}%
\pgfpathlineto{\pgfqpoint{3.269887in}{2.917333in}}%
\pgfpathlineto{\pgfqpoint{3.542463in}{2.656000in}}%
\pgfpathlineto{\pgfqpoint{3.655175in}{2.544000in}}%
\pgfpathlineto{\pgfqpoint{3.765980in}{2.431605in}}%
\pgfpathlineto{\pgfqpoint{3.909053in}{2.282667in}}%
\pgfpathlineto{\pgfqpoint{4.014216in}{2.170667in}}%
\pgfpathlineto{\pgfqpoint{4.126707in}{2.048186in}}%
\pgfpathlineto{\pgfqpoint{4.251231in}{1.909333in}}%
\pgfpathlineto{\pgfqpoint{4.527515in}{1.587467in}}%
\pgfpathlineto{\pgfqpoint{4.631158in}{1.461333in}}%
\pgfpathlineto{\pgfqpoint{4.727919in}{1.340685in}}%
\pgfpathlineto{\pgfqpoint{4.768000in}{1.289751in}}%
\pgfpathlineto{\pgfqpoint{4.768000in}{1.289751in}}%
\pgfusepath{fill}%
\end{pgfscope}%
\begin{pgfscope}%
\pgfpathrectangle{\pgfqpoint{0.800000in}{0.528000in}}{\pgfqpoint{3.968000in}{3.696000in}}%
\pgfusepath{clip}%
\pgfsetbuttcap%
\pgfsetroundjoin%
\definecolor{currentfill}{rgb}{0.275191,0.194905,0.496005}%
\pgfsetfillcolor{currentfill}%
\pgfsetlinewidth{0.000000pt}%
\definecolor{currentstroke}{rgb}{0.000000,0.000000,0.000000}%
\pgfsetstrokecolor{currentstroke}%
\pgfsetdash{}{0pt}%
\pgfpathmoveto{\pgfqpoint{2.319844in}{0.528000in}}%
\pgfpathlineto{\pgfqpoint{2.202828in}{0.642884in}}%
\pgfpathlineto{\pgfqpoint{2.057391in}{0.789333in}}%
\pgfpathlineto{\pgfqpoint{1.772539in}{1.088000in}}%
\pgfpathlineto{\pgfqpoint{1.502969in}{1.386667in}}%
\pgfpathlineto{\pgfqpoint{1.421694in}{1.480411in}}%
\pgfpathlineto{\pgfqpoint{1.361131in}{1.550715in}}%
\pgfpathlineto{\pgfqpoint{1.120646in}{1.841324in}}%
\pgfpathlineto{\pgfqpoint{1.000404in}{1.993332in}}%
\pgfpathlineto{\pgfqpoint{0.893073in}{2.133333in}}%
\pgfpathlineto{\pgfqpoint{0.800000in}{2.258341in}}%
\pgfpathlineto{\pgfqpoint{0.800000in}{2.250027in}}%
\pgfpathlineto{\pgfqpoint{0.887020in}{2.133333in}}%
\pgfpathlineto{\pgfqpoint{1.120646in}{1.833863in}}%
\pgfpathlineto{\pgfqpoint{1.242015in}{1.685333in}}%
\pgfpathlineto{\pgfqpoint{1.497011in}{1.386667in}}%
\pgfpathlineto{\pgfqpoint{1.601616in}{1.268721in}}%
\pgfpathlineto{\pgfqpoint{1.732006in}{1.125333in}}%
\pgfpathlineto{\pgfqpoint{1.842101in}{1.007055in}}%
\pgfpathlineto{\pgfqpoint{2.124792in}{0.714667in}}%
\pgfpathlineto{\pgfqpoint{2.313566in}{0.528000in}}%
\pgfpathmoveto{\pgfqpoint{4.768000in}{1.305040in}}%
\pgfpathlineto{\pgfqpoint{4.567596in}{1.553480in}}%
\pgfpathlineto{\pgfqpoint{4.447354in}{1.696922in}}%
\pgfpathlineto{\pgfqpoint{4.327111in}{1.836598in}}%
\pgfpathlineto{\pgfqpoint{4.046545in}{2.148778in}}%
\pgfpathlineto{\pgfqpoint{3.921284in}{2.282667in}}%
\pgfpathlineto{\pgfqpoint{3.630337in}{2.581333in}}%
\pgfpathlineto{\pgfqpoint{3.325091in}{2.877810in}}%
\pgfpathlineto{\pgfqpoint{3.038149in}{3.141333in}}%
\pgfpathlineto{\pgfqpoint{2.736987in}{3.402667in}}%
\pgfpathlineto{\pgfqpoint{2.643717in}{3.480534in}}%
\pgfpathlineto{\pgfqpoint{2.363152in}{3.705381in}}%
\pgfpathlineto{\pgfqpoint{2.242909in}{3.797393in}}%
\pgfpathlineto{\pgfqpoint{2.121167in}{3.888000in}}%
\pgfpathlineto{\pgfqpoint{2.002424in}{3.973517in}}%
\pgfpathlineto{\pgfqpoint{1.882182in}{4.057357in}}%
\pgfpathlineto{\pgfqpoint{1.761939in}{4.138290in}}%
\pgfpathlineto{\pgfqpoint{1.641697in}{4.216141in}}%
\pgfpathlineto{\pgfqpoint{1.629208in}{4.224000in}}%
\pgfpathlineto{\pgfqpoint{1.618636in}{4.224000in}}%
\pgfpathlineto{\pgfqpoint{1.727285in}{4.154387in}}%
\pgfpathlineto{\pgfqpoint{1.802020in}{4.105214in}}%
\pgfpathlineto{\pgfqpoint{1.901987in}{4.037333in}}%
\pgfpathlineto{\pgfqpoint{2.008844in}{3.962667in}}%
\pgfpathlineto{\pgfqpoint{2.262988in}{3.776000in}}%
\pgfpathlineto{\pgfqpoint{2.502396in}{3.589333in}}%
\pgfpathlineto{\pgfqpoint{2.603636in}{3.507393in}}%
\pgfpathlineto{\pgfqpoint{2.685293in}{3.440000in}}%
\pgfpathlineto{\pgfqpoint{2.817808in}{3.328000in}}%
\pgfpathlineto{\pgfqpoint{3.099091in}{3.080159in}}%
\pgfpathlineto{\pgfqpoint{3.164768in}{3.020716in}}%
\pgfpathlineto{\pgfqpoint{3.285010in}{2.909347in}}%
\pgfpathlineto{\pgfqpoint{3.433766in}{2.768000in}}%
\pgfpathlineto{\pgfqpoint{3.548772in}{2.656000in}}%
\pgfpathlineto{\pgfqpoint{3.661376in}{2.544000in}}%
\pgfpathlineto{\pgfqpoint{3.771701in}{2.432000in}}%
\pgfpathlineto{\pgfqpoint{3.886222in}{2.313161in}}%
\pgfpathlineto{\pgfqpoint{4.166788in}{2.010528in}}%
\pgfpathlineto{\pgfqpoint{4.290143in}{1.872000in}}%
\pgfpathlineto{\pgfqpoint{4.545195in}{1.573333in}}%
\pgfpathlineto{\pgfqpoint{4.647758in}{1.448248in}}%
\pgfpathlineto{\pgfqpoint{4.768000in}{1.297395in}}%
\pgfpathlineto{\pgfqpoint{4.768000in}{1.297395in}}%
\pgfusepath{fill}%
\end{pgfscope}%
\begin{pgfscope}%
\pgfpathrectangle{\pgfqpoint{0.800000in}{0.528000in}}{\pgfqpoint{3.968000in}{3.696000in}}%
\pgfusepath{clip}%
\pgfsetbuttcap%
\pgfsetroundjoin%
\definecolor{currentfill}{rgb}{0.275191,0.194905,0.496005}%
\pgfsetfillcolor{currentfill}%
\pgfsetlinewidth{0.000000pt}%
\definecolor{currentstroke}{rgb}{0.000000,0.000000,0.000000}%
\pgfsetstrokecolor{currentstroke}%
\pgfsetdash{}{0pt}%
\pgfpathmoveto{\pgfqpoint{2.313566in}{0.528000in}}%
\pgfpathlineto{\pgfqpoint{2.199491in}{0.640000in}}%
\pgfpathlineto{\pgfqpoint{2.082586in}{0.757354in}}%
\pgfpathlineto{\pgfqpoint{1.942647in}{0.901333in}}%
\pgfpathlineto{\pgfqpoint{1.836200in}{1.013333in}}%
\pgfpathlineto{\pgfqpoint{1.561535in}{1.313538in}}%
\pgfpathlineto{\pgfqpoint{1.431899in}{1.461333in}}%
\pgfpathlineto{\pgfqpoint{1.180668in}{1.760000in}}%
\pgfpathlineto{\pgfqpoint{1.080566in}{1.884011in}}%
\pgfpathlineto{\pgfqpoint{0.880162in}{2.142430in}}%
\pgfpathlineto{\pgfqpoint{0.800000in}{2.250027in}}%
\pgfpathlineto{\pgfqpoint{0.800000in}{2.241823in}}%
\pgfpathlineto{\pgfqpoint{0.920242in}{2.081776in}}%
\pgfpathlineto{\pgfqpoint{1.120646in}{1.826585in}}%
\pgfpathlineto{\pgfqpoint{1.240889in}{1.679611in}}%
\pgfpathlineto{\pgfqpoint{1.491052in}{1.386667in}}%
\pgfpathlineto{\pgfqpoint{1.601616in}{1.262131in}}%
\pgfpathlineto{\pgfqpoint{1.725967in}{1.125333in}}%
\pgfpathlineto{\pgfqpoint{1.842101in}{1.000687in}}%
\pgfpathlineto{\pgfqpoint{2.122667in}{0.710602in}}%
\pgfpathlineto{\pgfqpoint{2.307287in}{0.528000in}}%
\pgfpathmoveto{\pgfqpoint{4.768000in}{1.312665in}}%
\pgfpathlineto{\pgfqpoint{4.647758in}{1.462979in}}%
\pgfpathlineto{\pgfqpoint{4.399343in}{1.760000in}}%
\pgfpathlineto{\pgfqpoint{4.287030in}{1.888970in}}%
\pgfpathlineto{\pgfqpoint{4.032259in}{2.170667in}}%
\pgfpathlineto{\pgfqpoint{3.926303in}{2.283797in}}%
\pgfpathlineto{\pgfqpoint{3.636573in}{2.581333in}}%
\pgfpathlineto{\pgfqpoint{3.325091in}{2.883874in}}%
\pgfpathlineto{\pgfqpoint{3.204848in}{2.996033in}}%
\pgfpathlineto{\pgfqpoint{3.084606in}{3.105711in}}%
\pgfpathlineto{\pgfqpoint{2.788200in}{3.365333in}}%
\pgfpathlineto{\pgfqpoint{2.654810in}{3.477333in}}%
\pgfpathlineto{\pgfqpoint{2.563556in}{3.552274in}}%
\pgfpathlineto{\pgfqpoint{2.279318in}{3.776000in}}%
\pgfpathlineto{\pgfqpoint{2.026440in}{3.962667in}}%
\pgfpathlineto{\pgfqpoint{1.802020in}{4.118125in}}%
\pgfpathlineto{\pgfqpoint{1.697848in}{4.186667in}}%
\pgfpathlineto{\pgfqpoint{1.639779in}{4.224000in}}%
\pgfpathlineto{\pgfqpoint{1.629208in}{4.224000in}}%
\pgfpathlineto{\pgfqpoint{1.721859in}{4.164569in}}%
\pgfpathlineto{\pgfqpoint{1.761939in}{4.138290in}}%
\pgfpathlineto{\pgfqpoint{1.882182in}{4.057357in}}%
\pgfpathlineto{\pgfqpoint{1.964978in}{4.000000in}}%
\pgfpathlineto{\pgfqpoint{2.082586in}{3.916042in}}%
\pgfpathlineto{\pgfqpoint{2.202828in}{3.827503in}}%
\pgfpathlineto{\pgfqpoint{2.337333in}{3.725382in}}%
\pgfpathlineto{\pgfqpoint{2.463244in}{3.626667in}}%
\pgfpathlineto{\pgfqpoint{2.563556in}{3.546163in}}%
\pgfpathlineto{\pgfqpoint{2.868226in}{3.290667in}}%
\pgfpathlineto{\pgfqpoint{3.124687in}{3.063382in}}%
\pgfpathlineto{\pgfqpoint{3.204848in}{2.990001in}}%
\pgfpathlineto{\pgfqpoint{3.325091in}{2.877810in}}%
\pgfpathlineto{\pgfqpoint{3.445333in}{2.763044in}}%
\pgfpathlineto{\pgfqpoint{3.592839in}{2.618667in}}%
\pgfpathlineto{\pgfqpoint{3.704564in}{2.506667in}}%
\pgfpathlineto{\pgfqpoint{3.991481in}{2.208000in}}%
\pgfpathlineto{\pgfqpoint{4.263099in}{1.909333in}}%
\pgfpathlineto{\pgfqpoint{4.345769in}{1.814713in}}%
\pgfpathlineto{\pgfqpoint{4.407273in}{1.743849in}}%
\pgfpathlineto{\pgfqpoint{4.527515in}{1.601766in}}%
\pgfpathlineto{\pgfqpoint{4.647758in}{1.455636in}}%
\pgfpathlineto{\pgfqpoint{4.768000in}{1.305040in}}%
\pgfpathlineto{\pgfqpoint{4.768000in}{1.312000in}}%
\pgfpathlineto{\pgfqpoint{4.768000in}{1.312000in}}%
\pgfusepath{fill}%
\end{pgfscope}%
\begin{pgfscope}%
\pgfpathrectangle{\pgfqpoint{0.800000in}{0.528000in}}{\pgfqpoint{3.968000in}{3.696000in}}%
\pgfusepath{clip}%
\pgfsetbuttcap%
\pgfsetroundjoin%
\definecolor{currentfill}{rgb}{0.275191,0.194905,0.496005}%
\pgfsetfillcolor{currentfill}%
\pgfsetlinewidth{0.000000pt}%
\definecolor{currentstroke}{rgb}{0.000000,0.000000,0.000000}%
\pgfsetstrokecolor{currentstroke}%
\pgfsetdash{}{0pt}%
\pgfpathmoveto{\pgfqpoint{2.307287in}{0.528000in}}%
\pgfpathlineto{\pgfqpoint{2.193320in}{0.640000in}}%
\pgfpathlineto{\pgfqpoint{2.071672in}{0.762166in}}%
\pgfpathlineto{\pgfqpoint{1.936560in}{0.901333in}}%
\pgfpathlineto{\pgfqpoint{1.830213in}{1.013333in}}%
\pgfpathlineto{\pgfqpoint{1.725967in}{1.125333in}}%
\pgfpathlineto{\pgfqpoint{1.623966in}{1.237333in}}%
\pgfpathlineto{\pgfqpoint{1.521455in}{1.352064in}}%
\pgfpathlineto{\pgfqpoint{1.393765in}{1.498667in}}%
\pgfpathlineto{\pgfqpoint{1.280970in}{1.631612in}}%
\pgfpathlineto{\pgfqpoint{1.054526in}{1.909333in}}%
\pgfpathlineto{\pgfqpoint{0.960323in}{2.029599in}}%
\pgfpathlineto{\pgfqpoint{0.852971in}{2.170667in}}%
\pgfpathlineto{\pgfqpoint{0.800000in}{2.241823in}}%
\pgfpathlineto{\pgfqpoint{0.800000in}{2.233763in}}%
\pgfpathlineto{\pgfqpoint{0.903452in}{2.096000in}}%
\pgfpathlineto{\pgfqpoint{1.000404in}{1.970488in}}%
\pgfpathlineto{\pgfqpoint{1.120646in}{1.819307in}}%
\pgfpathlineto{\pgfqpoint{1.240889in}{1.672565in}}%
\pgfpathlineto{\pgfqpoint{1.501857in}{1.367587in}}%
\pgfpathlineto{\pgfqpoint{1.618026in}{1.237333in}}%
\pgfpathlineto{\pgfqpoint{1.721859in}{1.123288in}}%
\pgfpathlineto{\pgfqpoint{1.859415in}{0.976000in}}%
\pgfpathlineto{\pgfqpoint{2.156066in}{0.671110in}}%
\pgfpathlineto{\pgfqpoint{2.202828in}{0.624417in}}%
\pgfpathlineto{\pgfqpoint{2.301009in}{0.528000in}}%
\pgfpathmoveto{\pgfqpoint{4.768000in}{1.320095in}}%
\pgfpathlineto{\pgfqpoint{4.647758in}{1.470167in}}%
\pgfpathlineto{\pgfqpoint{4.397283in}{1.769305in}}%
\pgfpathlineto{\pgfqpoint{4.274968in}{1.909333in}}%
\pgfpathlineto{\pgfqpoint{4.166788in}{2.030343in}}%
\pgfpathlineto{\pgfqpoint{4.038273in}{2.170667in}}%
\pgfpathlineto{\pgfqpoint{3.926303in}{2.290096in}}%
\pgfpathlineto{\pgfqpoint{3.642809in}{2.581333in}}%
\pgfpathlineto{\pgfqpoint{3.330304in}{2.884856in}}%
\pgfpathlineto{\pgfqpoint{3.285010in}{2.927543in}}%
\pgfpathlineto{\pgfqpoint{3.134269in}{3.066667in}}%
\pgfpathlineto{\pgfqpoint{3.009771in}{3.178667in}}%
\pgfpathlineto{\pgfqpoint{2.924283in}{3.254069in}}%
\pgfpathlineto{\pgfqpoint{2.616837in}{3.514667in}}%
\pgfpathlineto{\pgfqpoint{2.523475in}{3.590722in}}%
\pgfpathlineto{\pgfqpoint{2.238160in}{3.813333in}}%
\pgfpathlineto{\pgfqpoint{2.122667in}{3.899246in}}%
\pgfpathlineto{\pgfqpoint{2.002424in}{3.986063in}}%
\pgfpathlineto{\pgfqpoint{1.875555in}{4.074667in}}%
\pgfpathlineto{\pgfqpoint{1.649982in}{4.224000in}}%
\pgfpathlineto{\pgfqpoint{1.639779in}{4.224000in}}%
\pgfpathlineto{\pgfqpoint{1.721859in}{4.171079in}}%
\pgfpathlineto{\pgfqpoint{1.811106in}{4.112000in}}%
\pgfpathlineto{\pgfqpoint{1.927191in}{4.032743in}}%
\pgfpathlineto{\pgfqpoint{2.042505in}{3.951194in}}%
\pgfpathlineto{\pgfqpoint{2.328021in}{3.738667in}}%
\pgfpathlineto{\pgfqpoint{2.470915in}{3.626667in}}%
\pgfpathlineto{\pgfqpoint{2.563892in}{3.552000in}}%
\pgfpathlineto{\pgfqpoint{2.844121in}{3.317517in}}%
\pgfpathlineto{\pgfqpoint{2.964364in}{3.212869in}}%
\pgfpathlineto{\pgfqpoint{3.064929in}{3.123005in}}%
\pgfpathlineto{\pgfqpoint{3.127716in}{3.066667in}}%
\pgfpathlineto{\pgfqpoint{3.445333in}{2.769198in}}%
\pgfpathlineto{\pgfqpoint{3.565576in}{2.651878in}}%
\pgfpathlineto{\pgfqpoint{3.856097in}{2.357333in}}%
\pgfpathlineto{\pgfqpoint{3.966384in}{2.241287in}}%
\pgfpathlineto{\pgfqpoint{4.246949in}{1.934279in}}%
\pgfpathlineto{\pgfqpoint{4.367210in}{1.797317in}}%
\pgfpathlineto{\pgfqpoint{4.494672in}{1.648000in}}%
\pgfpathlineto{\pgfqpoint{4.607677in}{1.511997in}}%
\pgfpathlineto{\pgfqpoint{4.768000in}{1.312665in}}%
\pgfpathlineto{\pgfqpoint{4.768000in}{1.312665in}}%
\pgfusepath{fill}%
\end{pgfscope}%
\begin{pgfscope}%
\pgfpathrectangle{\pgfqpoint{0.800000in}{0.528000in}}{\pgfqpoint{3.968000in}{3.696000in}}%
\pgfusepath{clip}%
\pgfsetbuttcap%
\pgfsetroundjoin%
\definecolor{currentfill}{rgb}{0.274128,0.199721,0.498911}%
\pgfsetfillcolor{currentfill}%
\pgfsetlinewidth{0.000000pt}%
\definecolor{currentstroke}{rgb}{0.000000,0.000000,0.000000}%
\pgfsetstrokecolor{currentstroke}%
\pgfsetdash{}{0pt}%
\pgfpathmoveto{\pgfqpoint{2.301009in}{0.528000in}}%
\pgfpathlineto{\pgfqpoint{2.187148in}{0.640000in}}%
\pgfpathlineto{\pgfqpoint{2.075600in}{0.752000in}}%
\pgfpathlineto{\pgfqpoint{1.962343in}{0.868147in}}%
\pgfpathlineto{\pgfqpoint{1.681778in}{1.167012in}}%
\pgfpathlineto{\pgfqpoint{1.551119in}{1.312000in}}%
\pgfpathlineto{\pgfqpoint{1.441293in}{1.436877in}}%
\pgfpathlineto{\pgfqpoint{1.321051in}{1.576953in}}%
\pgfpathlineto{\pgfqpoint{1.078242in}{1.872000in}}%
\pgfpathlineto{\pgfqpoint{0.840081in}{2.179849in}}%
\pgfpathlineto{\pgfqpoint{0.800000in}{2.233763in}}%
\pgfpathlineto{\pgfqpoint{0.800000in}{2.225702in}}%
\pgfpathlineto{\pgfqpoint{0.897517in}{2.096000in}}%
\pgfpathlineto{\pgfqpoint{1.000404in}{1.962961in}}%
\pgfpathlineto{\pgfqpoint{1.102400in}{1.834667in}}%
\pgfpathlineto{\pgfqpoint{1.179513in}{1.740164in}}%
\pgfpathlineto{\pgfqpoint{1.224443in}{1.685333in}}%
\pgfpathlineto{\pgfqpoint{1.321051in}{1.570020in}}%
\pgfpathlineto{\pgfqpoint{1.589185in}{1.263088in}}%
\pgfpathlineto{\pgfqpoint{1.645812in}{1.200000in}}%
\pgfpathlineto{\pgfqpoint{1.761939in}{1.073595in}}%
\pgfpathlineto{\pgfqpoint{2.042505in}{0.779486in}}%
\pgfpathlineto{\pgfqpoint{2.180977in}{0.640000in}}%
\pgfpathlineto{\pgfqpoint{2.294730in}{0.528000in}}%
\pgfpathmoveto{\pgfqpoint{4.768000in}{1.327525in}}%
\pgfpathlineto{\pgfqpoint{4.647758in}{1.477355in}}%
\pgfpathlineto{\pgfqpoint{4.407273in}{1.764507in}}%
\pgfpathlineto{\pgfqpoint{4.280902in}{1.909333in}}%
\pgfpathlineto{\pgfqpoint{4.166788in}{2.036859in}}%
\pgfpathlineto{\pgfqpoint{4.029782in}{2.186281in}}%
\pgfpathlineto{\pgfqpoint{3.903839in}{2.320000in}}%
\pgfpathlineto{\pgfqpoint{3.796085in}{2.432000in}}%
\pgfpathlineto{\pgfqpoint{3.491567in}{2.736398in}}%
\pgfpathlineto{\pgfqpoint{3.445333in}{2.781281in}}%
\pgfpathlineto{\pgfqpoint{3.302289in}{2.917333in}}%
\pgfpathlineto{\pgfqpoint{3.016445in}{3.178667in}}%
\pgfpathlineto{\pgfqpoint{2.889082in}{3.290667in}}%
\pgfpathlineto{\pgfqpoint{2.802383in}{3.365333in}}%
\pgfpathlineto{\pgfqpoint{2.669276in}{3.477333in}}%
\pgfpathlineto{\pgfqpoint{2.532572in}{3.589333in}}%
\pgfpathlineto{\pgfqpoint{2.439199in}{3.664000in}}%
\pgfpathlineto{\pgfqpoint{2.162747in}{3.875846in}}%
\pgfpathlineto{\pgfqpoint{2.042505in}{3.963735in}}%
\pgfpathlineto{\pgfqpoint{1.802020in}{4.130874in}}%
\pgfpathlineto{\pgfqpoint{1.681778in}{4.210082in}}%
\pgfpathlineto{\pgfqpoint{1.660104in}{4.224000in}}%
\pgfpathlineto{\pgfqpoint{1.649982in}{4.224000in}}%
\pgfpathlineto{\pgfqpoint{1.882182in}{4.070146in}}%
\pgfpathlineto{\pgfqpoint{2.137927in}{3.888000in}}%
\pgfpathlineto{\pgfqpoint{2.242909in}{3.809763in}}%
\pgfpathlineto{\pgfqpoint{2.525192in}{3.589333in}}%
\pgfpathlineto{\pgfqpoint{2.804040in}{3.357890in}}%
\pgfpathlineto{\pgfqpoint{2.925121in}{3.253333in}}%
\pgfpathlineto{\pgfqpoint{3.051594in}{3.141333in}}%
\pgfpathlineto{\pgfqpoint{3.335550in}{2.880000in}}%
\pgfpathlineto{\pgfqpoint{3.452825in}{2.768000in}}%
\pgfpathlineto{\pgfqpoint{3.567643in}{2.656000in}}%
\pgfpathlineto{\pgfqpoint{3.862125in}{2.357333in}}%
\pgfpathlineto{\pgfqpoint{3.968601in}{2.245333in}}%
\pgfpathlineto{\pgfqpoint{4.246949in}{1.940982in}}%
\pgfpathlineto{\pgfqpoint{4.373034in}{1.797333in}}%
\pgfpathlineto{\pgfqpoint{4.487434in}{1.663538in}}%
\pgfpathlineto{\pgfqpoint{4.715076in}{1.386667in}}%
\pgfpathlineto{\pgfqpoint{4.768000in}{1.320095in}}%
\pgfpathlineto{\pgfqpoint{4.768000in}{1.320095in}}%
\pgfusepath{fill}%
\end{pgfscope}%
\begin{pgfscope}%
\pgfpathrectangle{\pgfqpoint{0.800000in}{0.528000in}}{\pgfqpoint{3.968000in}{3.696000in}}%
\pgfusepath{clip}%
\pgfsetbuttcap%
\pgfsetroundjoin%
\definecolor{currentfill}{rgb}{0.274128,0.199721,0.498911}%
\pgfsetfillcolor{currentfill}%
\pgfsetlinewidth{0.000000pt}%
\definecolor{currentstroke}{rgb}{0.000000,0.000000,0.000000}%
\pgfsetstrokecolor{currentstroke}%
\pgfsetdash{}{0pt}%
\pgfpathmoveto{\pgfqpoint{2.294730in}{0.528000in}}%
\pgfpathlineto{\pgfqpoint{2.180977in}{0.640000in}}%
\pgfpathlineto{\pgfqpoint{2.069532in}{0.752000in}}%
\pgfpathlineto{\pgfqpoint{1.960281in}{0.864000in}}%
\pgfpathlineto{\pgfqpoint{1.668574in}{1.174965in}}%
\pgfpathlineto{\pgfqpoint{1.545243in}{1.312000in}}%
\pgfpathlineto{\pgfqpoint{1.441293in}{1.430072in}}%
\pgfpathlineto{\pgfqpoint{1.318241in}{1.573333in}}%
\pgfpathlineto{\pgfqpoint{1.072408in}{1.872000in}}%
\pgfpathlineto{\pgfqpoint{0.840081in}{2.171804in}}%
\pgfpathlineto{\pgfqpoint{0.800000in}{2.225702in}}%
\pgfpathlineto{\pgfqpoint{0.800000in}{2.217641in}}%
\pgfpathlineto{\pgfqpoint{0.891582in}{2.096000in}}%
\pgfpathlineto{\pgfqpoint{0.978071in}{1.984000in}}%
\pgfpathlineto{\pgfqpoint{1.066573in}{1.872000in}}%
\pgfpathlineto{\pgfqpoint{1.160727in}{1.755492in}}%
\pgfpathlineto{\pgfqpoint{1.408324in}{1.461333in}}%
\pgfpathlineto{\pgfqpoint{1.521455in}{1.332159in}}%
\pgfpathlineto{\pgfqpoint{1.641697in}{1.198003in}}%
\pgfpathlineto{\pgfqpoint{1.777352in}{1.050667in}}%
\pgfpathlineto{\pgfqpoint{1.882722in}{0.938667in}}%
\pgfpathlineto{\pgfqpoint{2.174806in}{0.640000in}}%
\pgfpathlineto{\pgfqpoint{2.288452in}{0.528000in}}%
\pgfpathmoveto{\pgfqpoint{4.768000in}{1.334954in}}%
\pgfpathlineto{\pgfqpoint{4.666624in}{1.461333in}}%
\pgfpathlineto{\pgfqpoint{4.567596in}{1.581896in}}%
\pgfpathlineto{\pgfqpoint{4.447354in}{1.724619in}}%
\pgfpathlineto{\pgfqpoint{4.319653in}{1.872000in}}%
\pgfpathlineto{\pgfqpoint{4.206869in}{1.998949in}}%
\pgfpathlineto{\pgfqpoint{4.084714in}{2.133333in}}%
\pgfpathlineto{\pgfqpoint{3.802181in}{2.432000in}}%
\pgfpathlineto{\pgfqpoint{3.503796in}{2.730667in}}%
\pgfpathlineto{\pgfqpoint{3.216575in}{3.002922in}}%
\pgfpathlineto{\pgfqpoint{3.156070in}{3.058565in}}%
\pgfpathlineto{\pgfqpoint{3.106278in}{3.104000in}}%
\pgfpathlineto{\pgfqpoint{2.981046in}{3.216000in}}%
\pgfpathlineto{\pgfqpoint{2.852778in}{3.328000in}}%
\pgfpathlineto{\pgfqpoint{2.603636in}{3.537499in}}%
\pgfpathlineto{\pgfqpoint{2.509480in}{3.613632in}}%
\pgfpathlineto{\pgfqpoint{2.443313in}{3.666783in}}%
\pgfpathlineto{\pgfqpoint{2.154634in}{3.888000in}}%
\pgfpathlineto{\pgfqpoint{1.909977in}{4.063224in}}%
\pgfpathlineto{\pgfqpoint{1.839469in}{4.112000in}}%
\pgfpathlineto{\pgfqpoint{1.670225in}{4.224000in}}%
\pgfpathlineto{\pgfqpoint{1.660104in}{4.224000in}}%
\pgfpathlineto{\pgfqpoint{1.842101in}{4.103838in}}%
\pgfpathlineto{\pgfqpoint{2.095412in}{3.925333in}}%
\pgfpathlineto{\pgfqpoint{2.228477in}{3.826776in}}%
\pgfpathlineto{\pgfqpoint{2.343695in}{3.738667in}}%
\pgfpathlineto{\pgfqpoint{2.443313in}{3.660757in}}%
\pgfpathlineto{\pgfqpoint{2.723879in}{3.431758in}}%
\pgfpathlineto{\pgfqpoint{2.845935in}{3.328000in}}%
\pgfpathlineto{\pgfqpoint{2.974331in}{3.216000in}}%
\pgfpathlineto{\pgfqpoint{3.262374in}{2.954667in}}%
\pgfpathlineto{\pgfqpoint{3.381247in}{2.842667in}}%
\pgfpathlineto{\pgfqpoint{3.497580in}{2.730667in}}%
\pgfpathlineto{\pgfqpoint{3.796085in}{2.432000in}}%
\pgfpathlineto{\pgfqpoint{3.903839in}{2.320000in}}%
\pgfpathlineto{\pgfqpoint{4.180807in}{2.021333in}}%
\pgfpathlineto{\pgfqpoint{4.287030in}{1.902399in}}%
\pgfpathlineto{\pgfqpoint{4.426683in}{1.741920in}}%
\pgfpathlineto{\pgfqpoint{4.537738in}{1.610667in}}%
\pgfpathlineto{\pgfqpoint{4.647758in}{1.477355in}}%
\pgfpathlineto{\pgfqpoint{4.750703in}{1.349333in}}%
\pgfpathlineto{\pgfqpoint{4.768000in}{1.327525in}}%
\pgfpathlineto{\pgfqpoint{4.768000in}{1.327525in}}%
\pgfusepath{fill}%
\end{pgfscope}%
\begin{pgfscope}%
\pgfpathrectangle{\pgfqpoint{0.800000in}{0.528000in}}{\pgfqpoint{3.968000in}{3.696000in}}%
\pgfusepath{clip}%
\pgfsetbuttcap%
\pgfsetroundjoin%
\definecolor{currentfill}{rgb}{0.274128,0.199721,0.498911}%
\pgfsetfillcolor{currentfill}%
\pgfsetlinewidth{0.000000pt}%
\definecolor{currentstroke}{rgb}{0.000000,0.000000,0.000000}%
\pgfsetstrokecolor{currentstroke}%
\pgfsetdash{}{0pt}%
\pgfpathmoveto{\pgfqpoint{2.288452in}{0.528000in}}%
\pgfpathlineto{\pgfqpoint{2.174806in}{0.640000in}}%
\pgfpathlineto{\pgfqpoint{2.063464in}{0.752000in}}%
\pgfpathlineto{\pgfqpoint{1.954313in}{0.864000in}}%
\pgfpathlineto{\pgfqpoint{1.673938in}{1.162667in}}%
\pgfpathlineto{\pgfqpoint{1.561535in}{1.287094in}}%
\pgfpathlineto{\pgfqpoint{1.438011in}{1.427057in}}%
\pgfpathlineto{\pgfqpoint{1.312442in}{1.573333in}}%
\pgfpathlineto{\pgfqpoint{1.066573in}{1.872000in}}%
\pgfpathlineto{\pgfqpoint{0.823332in}{2.186267in}}%
\pgfpathlineto{\pgfqpoint{0.800000in}{2.217641in}}%
\pgfpathlineto{\pgfqpoint{0.800000in}{2.209581in}}%
\pgfpathlineto{\pgfqpoint{0.885647in}{2.096000in}}%
\pgfpathlineto{\pgfqpoint{0.972186in}{1.984000in}}%
\pgfpathlineto{\pgfqpoint{1.212746in}{1.685333in}}%
\pgfpathlineto{\pgfqpoint{1.321051in}{1.556340in}}%
\pgfpathlineto{\pgfqpoint{1.566749in}{1.274667in}}%
\pgfpathlineto{\pgfqpoint{1.842101in}{0.975230in}}%
\pgfpathlineto{\pgfqpoint{2.131305in}{0.677333in}}%
\pgfpathlineto{\pgfqpoint{2.282990in}{0.528000in}}%
\pgfpathmoveto{\pgfqpoint{4.768000in}{1.342384in}}%
\pgfpathlineto{\pgfqpoint{4.672467in}{1.461333in}}%
\pgfpathlineto{\pgfqpoint{4.567596in}{1.588870in}}%
\pgfpathlineto{\pgfqpoint{4.447354in}{1.731379in}}%
\pgfpathlineto{\pgfqpoint{4.318526in}{1.879997in}}%
\pgfpathlineto{\pgfqpoint{4.192574in}{2.021333in}}%
\pgfpathlineto{\pgfqpoint{3.915830in}{2.320000in}}%
\pgfpathlineto{\pgfqpoint{3.806061in}{2.434227in}}%
\pgfpathlineto{\pgfqpoint{3.510012in}{2.730667in}}%
\pgfpathlineto{\pgfqpoint{3.204848in}{3.019982in}}%
\pgfpathlineto{\pgfqpoint{3.071495in}{3.141333in}}%
\pgfpathlineto{\pgfqpoint{2.772422in}{3.402667in}}%
\pgfpathlineto{\pgfqpoint{2.523475in}{3.608633in}}%
\pgfpathlineto{\pgfqpoint{2.426510in}{3.685682in}}%
\pgfpathlineto{\pgfqpoint{2.359369in}{3.738667in}}%
\pgfpathlineto{\pgfqpoint{2.082586in}{3.946991in}}%
\pgfpathlineto{\pgfqpoint{1.982352in}{4.018637in}}%
\pgfpathlineto{\pgfqpoint{1.922263in}{4.061210in}}%
\pgfpathlineto{\pgfqpoint{1.844848in}{4.114559in}}%
\pgfpathlineto{\pgfqpoint{1.802020in}{4.143622in}}%
\pgfpathlineto{\pgfqpoint{1.680347in}{4.224000in}}%
\pgfpathlineto{\pgfqpoint{1.670225in}{4.224000in}}%
\pgfpathlineto{\pgfqpoint{1.842101in}{4.110223in}}%
\pgfpathlineto{\pgfqpoint{2.103830in}{3.925333in}}%
\pgfpathlineto{\pgfqpoint{2.204821in}{3.850667in}}%
\pgfpathlineto{\pgfqpoint{2.351532in}{3.738667in}}%
\pgfpathlineto{\pgfqpoint{2.446811in}{3.664000in}}%
\pgfpathlineto{\pgfqpoint{2.723879in}{3.437773in}}%
\pgfpathlineto{\pgfqpoint{2.852778in}{3.328000in}}%
\pgfpathlineto{\pgfqpoint{2.981046in}{3.216000in}}%
\pgfpathlineto{\pgfqpoint{3.268811in}{2.954667in}}%
\pgfpathlineto{\pgfqpoint{3.387571in}{2.842667in}}%
\pgfpathlineto{\pgfqpoint{3.503796in}{2.730667in}}%
\pgfpathlineto{\pgfqpoint{3.806061in}{2.428010in}}%
\pgfpathlineto{\pgfqpoint{3.945259in}{2.282667in}}%
\pgfpathlineto{\pgfqpoint{4.050208in}{2.170667in}}%
\pgfpathlineto{\pgfqpoint{4.327111in}{1.863501in}}%
\pgfpathlineto{\pgfqpoint{4.574711in}{1.573333in}}%
\pgfpathlineto{\pgfqpoint{4.768000in}{1.334954in}}%
\pgfpathlineto{\pgfqpoint{4.768000in}{1.334954in}}%
\pgfusepath{fill}%
\end{pgfscope}%
\begin{pgfscope}%
\pgfpathrectangle{\pgfqpoint{0.800000in}{0.528000in}}{\pgfqpoint{3.968000in}{3.696000in}}%
\pgfusepath{clip}%
\pgfsetbuttcap%
\pgfsetroundjoin%
\definecolor{currentfill}{rgb}{0.273006,0.204520,0.501721}%
\pgfsetfillcolor{currentfill}%
\pgfsetlinewidth{0.000000pt}%
\definecolor{currentstroke}{rgb}{0.000000,0.000000,0.000000}%
\pgfsetstrokecolor{currentstroke}%
\pgfsetdash{}{0pt}%
\pgfpathmoveto{\pgfqpoint{2.282194in}{0.528000in}}%
\pgfpathlineto{\pgfqpoint{2.162747in}{0.645859in}}%
\pgfpathlineto{\pgfqpoint{1.876806in}{0.938667in}}%
\pgfpathlineto{\pgfqpoint{1.600241in}{1.237333in}}%
\pgfpathlineto{\pgfqpoint{1.321051in}{1.556340in}}%
\pgfpathlineto{\pgfqpoint{1.200808in}{1.699741in}}%
\pgfpathlineto{\pgfqpoint{1.080566in}{1.847222in}}%
\pgfpathlineto{\pgfqpoint{0.960323in}{1.999202in}}%
\pgfpathlineto{\pgfqpoint{0.800000in}{2.209581in}}%
\pgfpathlineto{\pgfqpoint{0.800000in}{2.208000in}}%
\pgfpathlineto{\pgfqpoint{0.800000in}{2.201711in}}%
\pgfpathlineto{\pgfqpoint{0.920242in}{2.043342in}}%
\pgfpathlineto{\pgfqpoint{1.025165in}{1.909333in}}%
\pgfpathlineto{\pgfqpoint{1.100395in}{1.815804in}}%
\pgfpathlineto{\pgfqpoint{1.145466in}{1.760000in}}%
\pgfpathlineto{\pgfqpoint{1.240889in}{1.644475in}}%
\pgfpathlineto{\pgfqpoint{1.494588in}{1.349333in}}%
\pgfpathlineto{\pgfqpoint{1.788770in}{1.025676in}}%
\pgfpathlineto{\pgfqpoint{1.906526in}{0.901333in}}%
\pgfpathlineto{\pgfqpoint{2.202828in}{0.599947in}}%
\pgfpathlineto{\pgfqpoint{2.276078in}{0.528000in}}%
\pgfpathmoveto{\pgfqpoint{4.768000in}{1.349801in}}%
\pgfpathlineto{\pgfqpoint{4.647758in}{1.498912in}}%
\pgfpathlineto{\pgfqpoint{4.396388in}{1.797333in}}%
\pgfpathlineto{\pgfqpoint{4.287030in}{1.922216in}}%
\pgfpathlineto{\pgfqpoint{4.153386in}{2.071150in}}%
\pgfpathlineto{\pgfqpoint{4.027238in}{2.208000in}}%
\pgfpathlineto{\pgfqpoint{3.921825in}{2.320000in}}%
\pgfpathlineto{\pgfqpoint{3.806061in}{2.440352in}}%
\pgfpathlineto{\pgfqpoint{3.516228in}{2.730667in}}%
\pgfpathlineto{\pgfqpoint{3.204848in}{3.025969in}}%
\pgfpathlineto{\pgfqpoint{3.078128in}{3.141333in}}%
\pgfpathlineto{\pgfqpoint{2.779352in}{3.402667in}}%
\pgfpathlineto{\pgfqpoint{2.523475in}{3.614603in}}%
\pgfpathlineto{\pgfqpoint{2.430007in}{3.688940in}}%
\pgfpathlineto{\pgfqpoint{2.363152in}{3.741737in}}%
\pgfpathlineto{\pgfqpoint{2.076761in}{3.957241in}}%
\pgfpathlineto{\pgfqpoint{2.008956in}{4.006084in}}%
\pgfpathlineto{\pgfqpoint{1.962343in}{4.039391in}}%
\pgfpathlineto{\pgfqpoint{1.721859in}{4.203223in}}%
\pgfpathlineto{\pgfqpoint{1.690113in}{4.224000in}}%
\pgfpathlineto{\pgfqpoint{1.680347in}{4.224000in}}%
\pgfpathlineto{\pgfqpoint{1.681778in}{4.223081in}}%
\pgfpathlineto{\pgfqpoint{1.774931in}{4.161435in}}%
\pgfpathlineto{\pgfqpoint{1.848662in}{4.112000in}}%
\pgfpathlineto{\pgfqpoint{2.082586in}{3.946991in}}%
\pgfpathlineto{\pgfqpoint{2.212832in}{3.850667in}}%
\pgfpathlineto{\pgfqpoint{2.339348in}{3.753828in}}%
\pgfpathlineto{\pgfqpoint{2.407116in}{3.701333in}}%
\pgfpathlineto{\pgfqpoint{2.684285in}{3.476880in}}%
\pgfpathlineto{\pgfqpoint{2.816203in}{3.365333in}}%
\pgfpathlineto{\pgfqpoint{2.945393in}{3.253333in}}%
\pgfpathlineto{\pgfqpoint{3.071495in}{3.141333in}}%
\pgfpathlineto{\pgfqpoint{3.354634in}{2.880000in}}%
\pgfpathlineto{\pgfqpoint{3.471580in}{2.768000in}}%
\pgfpathlineto{\pgfqpoint{3.586079in}{2.656000in}}%
\pgfpathlineto{\pgfqpoint{3.886222in}{2.351061in}}%
\pgfpathlineto{\pgfqpoint{4.021340in}{2.208000in}}%
\pgfpathlineto{\pgfqpoint{4.126707in}{2.093983in}}%
\pgfpathlineto{\pgfqpoint{4.259475in}{1.946667in}}%
\pgfpathlineto{\pgfqpoint{4.367192in}{1.824286in}}%
\pgfpathlineto{\pgfqpoint{4.487434in}{1.684385in}}%
\pgfpathlineto{\pgfqpoint{4.746793in}{1.369087in}}%
\pgfpathlineto{\pgfqpoint{4.768000in}{1.342384in}}%
\pgfpathlineto{\pgfqpoint{4.768000in}{1.349333in}}%
\pgfpathlineto{\pgfqpoint{4.768000in}{1.349333in}}%
\pgfusepath{fill}%
\end{pgfscope}%
\begin{pgfscope}%
\pgfpathrectangle{\pgfqpoint{0.800000in}{0.528000in}}{\pgfqpoint{3.968000in}{3.696000in}}%
\pgfusepath{clip}%
\pgfsetbuttcap%
\pgfsetroundjoin%
\definecolor{currentfill}{rgb}{0.273006,0.204520,0.501721}%
\pgfsetfillcolor{currentfill}%
\pgfsetlinewidth{0.000000pt}%
\definecolor{currentstroke}{rgb}{0.000000,0.000000,0.000000}%
\pgfsetstrokecolor{currentstroke}%
\pgfsetdash{}{0pt}%
\pgfpathmoveto{\pgfqpoint{2.276078in}{0.528000in}}%
\pgfpathlineto{\pgfqpoint{2.162470in}{0.640000in}}%
\pgfpathlineto{\pgfqpoint{1.870903in}{0.938667in}}%
\pgfpathlineto{\pgfqpoint{1.594446in}{1.237333in}}%
\pgfpathlineto{\pgfqpoint{1.321051in}{1.549501in}}%
\pgfpathlineto{\pgfqpoint{1.200808in}{1.692683in}}%
\pgfpathlineto{\pgfqpoint{1.080566in}{1.839931in}}%
\pgfpathlineto{\pgfqpoint{0.960323in}{1.991661in}}%
\pgfpathlineto{\pgfqpoint{0.800000in}{2.201711in}}%
\pgfpathlineto{\pgfqpoint{0.800000in}{2.193889in}}%
\pgfpathlineto{\pgfqpoint{0.902613in}{2.058667in}}%
\pgfpathlineto{\pgfqpoint{0.989811in}{1.946667in}}%
\pgfpathlineto{\pgfqpoint{1.080566in}{1.832694in}}%
\pgfpathlineto{\pgfqpoint{1.326729in}{1.536000in}}%
\pgfpathlineto{\pgfqpoint{1.441293in}{1.403384in}}%
\pgfpathlineto{\pgfqpoint{1.690530in}{1.125333in}}%
\pgfpathlineto{\pgfqpoint{1.802020in}{1.005294in}}%
\pgfpathlineto{\pgfqpoint{1.936409in}{0.864000in}}%
\pgfpathlineto{\pgfqpoint{2.231869in}{0.565333in}}%
\pgfpathlineto{\pgfqpoint{2.269961in}{0.528000in}}%
\pgfpathmoveto{\pgfqpoint{4.768000in}{1.357028in}}%
\pgfpathlineto{\pgfqpoint{4.647758in}{1.505910in}}%
\pgfpathlineto{\pgfqpoint{4.402227in}{1.797333in}}%
\pgfpathlineto{\pgfqpoint{4.287030in}{1.928764in}}%
\pgfpathlineto{\pgfqpoint{4.166788in}{2.062822in}}%
\pgfpathlineto{\pgfqpoint{4.033135in}{2.208000in}}%
\pgfpathlineto{\pgfqpoint{3.926303in}{2.321557in}}%
\pgfpathlineto{\pgfqpoint{3.783758in}{2.469333in}}%
\pgfpathlineto{\pgfqpoint{3.484083in}{2.768000in}}%
\pgfpathlineto{\pgfqpoint{3.365172in}{2.882024in}}%
\pgfpathlineto{\pgfqpoint{3.043142in}{3.178667in}}%
\pgfpathlineto{\pgfqpoint{2.916281in}{3.290667in}}%
\pgfpathlineto{\pgfqpoint{2.786282in}{3.402667in}}%
\pgfpathlineto{\pgfqpoint{2.515891in}{3.626667in}}%
\pgfpathlineto{\pgfqpoint{2.242909in}{3.840061in}}%
\pgfpathlineto{\pgfqpoint{2.099388in}{3.947017in}}%
\pgfpathlineto{\pgfqpoint{1.973861in}{4.037333in}}%
\pgfpathlineto{\pgfqpoint{1.866848in}{4.112000in}}%
\pgfpathlineto{\pgfqpoint{1.699821in}{4.224000in}}%
\pgfpathlineto{\pgfqpoint{1.690113in}{4.224000in}}%
\pgfpathlineto{\pgfqpoint{1.802973in}{4.149333in}}%
\pgfpathlineto{\pgfqpoint{2.042505in}{3.982146in}}%
\pgfpathlineto{\pgfqpoint{2.171049in}{3.888000in}}%
\pgfpathlineto{\pgfqpoint{2.443313in}{3.678688in}}%
\pgfpathlineto{\pgfqpoint{2.563556in}{3.582170in}}%
\pgfpathlineto{\pgfqpoint{2.866463in}{3.328000in}}%
\pgfpathlineto{\pgfqpoint{2.994477in}{3.216000in}}%
\pgfpathlineto{\pgfqpoint{3.244929in}{2.988905in}}%
\pgfpathlineto{\pgfqpoint{3.559758in}{2.687914in}}%
\pgfpathlineto{\pgfqpoint{3.605657in}{2.642715in}}%
\pgfpathlineto{\pgfqpoint{3.741160in}{2.506667in}}%
\pgfpathlineto{\pgfqpoint{3.850317in}{2.394667in}}%
\pgfpathlineto{\pgfqpoint{3.966384in}{2.272931in}}%
\pgfpathlineto{\pgfqpoint{4.246949in}{1.967273in}}%
\pgfpathlineto{\pgfqpoint{4.492396in}{1.685333in}}%
\pgfpathlineto{\pgfqpoint{4.738512in}{1.386667in}}%
\pgfpathlineto{\pgfqpoint{4.768000in}{1.349801in}}%
\pgfpathlineto{\pgfqpoint{4.768000in}{1.349801in}}%
\pgfusepath{fill}%
\end{pgfscope}%
\begin{pgfscope}%
\pgfpathrectangle{\pgfqpoint{0.800000in}{0.528000in}}{\pgfqpoint{3.968000in}{3.696000in}}%
\pgfusepath{clip}%
\pgfsetbuttcap%
\pgfsetroundjoin%
\definecolor{currentfill}{rgb}{0.273006,0.204520,0.501721}%
\pgfsetfillcolor{currentfill}%
\pgfsetlinewidth{0.000000pt}%
\definecolor{currentstroke}{rgb}{0.000000,0.000000,0.000000}%
\pgfsetstrokecolor{currentstroke}%
\pgfsetdash{}{0pt}%
\pgfpathmoveto{\pgfqpoint{2.269961in}{0.528000in}}%
\pgfpathlineto{\pgfqpoint{2.156455in}{0.640000in}}%
\pgfpathlineto{\pgfqpoint{1.865000in}{0.938667in}}%
\pgfpathlineto{\pgfqpoint{1.588651in}{1.237333in}}%
\pgfpathlineto{\pgfqpoint{1.321051in}{1.542661in}}%
\pgfpathlineto{\pgfqpoint{1.200808in}{1.685624in}}%
\pgfpathlineto{\pgfqpoint{1.074195in}{1.840601in}}%
\pgfpathlineto{\pgfqpoint{0.960323in}{1.984121in}}%
\pgfpathlineto{\pgfqpoint{0.800000in}{2.193889in}}%
\pgfpathlineto{\pgfqpoint{0.800000in}{2.186067in}}%
\pgfpathlineto{\pgfqpoint{0.896791in}{2.058667in}}%
\pgfpathlineto{\pgfqpoint{1.133894in}{1.760000in}}%
\pgfpathlineto{\pgfqpoint{1.240889in}{1.630749in}}%
\pgfpathlineto{\pgfqpoint{1.482900in}{1.349333in}}%
\pgfpathlineto{\pgfqpoint{1.761939in}{1.041862in}}%
\pgfpathlineto{\pgfqpoint{1.894656in}{0.901333in}}%
\pgfpathlineto{\pgfqpoint{2.195255in}{0.595613in}}%
\pgfpathlineto{\pgfqpoint{2.242909in}{0.548485in}}%
\pgfpathlineto{\pgfqpoint{2.263844in}{0.528000in}}%
\pgfpathmoveto{\pgfqpoint{4.768000in}{1.364255in}}%
\pgfpathlineto{\pgfqpoint{4.647758in}{1.512908in}}%
\pgfpathlineto{\pgfqpoint{4.407273in}{1.798227in}}%
\pgfpathlineto{\pgfqpoint{4.276938in}{1.946667in}}%
\pgfpathlineto{\pgfqpoint{4.004180in}{2.245333in}}%
\pgfpathlineto{\pgfqpoint{3.716384in}{2.544000in}}%
\pgfpathlineto{\pgfqpoint{3.585574in}{2.674706in}}%
\pgfpathlineto{\pgfqpoint{3.451575in}{2.805333in}}%
\pgfpathlineto{\pgfqpoint{3.173353in}{3.066667in}}%
\pgfpathlineto{\pgfqpoint{2.923081in}{3.290667in}}%
\pgfpathlineto{\pgfqpoint{2.793212in}{3.402667in}}%
\pgfpathlineto{\pgfqpoint{2.523322in}{3.626667in}}%
\pgfpathlineto{\pgfqpoint{2.236863in}{3.850667in}}%
\pgfpathlineto{\pgfqpoint{2.122667in}{3.935936in}}%
\pgfpathlineto{\pgfqpoint{2.002424in}{4.023153in}}%
\pgfpathlineto{\pgfqpoint{1.902279in}{4.093386in}}%
\pgfpathlineto{\pgfqpoint{1.842101in}{4.135200in}}%
\pgfpathlineto{\pgfqpoint{1.763624in}{4.188235in}}%
\pgfpathlineto{\pgfqpoint{1.721859in}{4.215930in}}%
\pgfpathlineto{\pgfqpoint{1.709529in}{4.224000in}}%
\pgfpathlineto{\pgfqpoint{1.699821in}{4.224000in}}%
\pgfpathlineto{\pgfqpoint{1.812143in}{4.149333in}}%
\pgfpathlineto{\pgfqpoint{2.042505in}{3.988284in}}%
\pgfpathlineto{\pgfqpoint{2.162747in}{3.900223in}}%
\pgfpathlineto{\pgfqpoint{2.305696in}{3.792184in}}%
\pgfpathlineto{\pgfqpoint{2.422184in}{3.701333in}}%
\pgfpathlineto{\pgfqpoint{2.683798in}{3.489034in}}%
\pgfpathlineto{\pgfqpoint{2.829974in}{3.365333in}}%
\pgfpathlineto{\pgfqpoint{2.958908in}{3.253333in}}%
\pgfpathlineto{\pgfqpoint{3.064058in}{3.159527in}}%
\pgfpathlineto{\pgfqpoint{3.126021in}{3.104000in}}%
\pgfpathlineto{\pgfqpoint{3.247980in}{2.992000in}}%
\pgfpathlineto{\pgfqpoint{3.565576in}{2.688386in}}%
\pgfpathlineto{\pgfqpoint{3.710341in}{2.544000in}}%
\pgfpathlineto{\pgfqpoint{4.002035in}{2.241208in}}%
\pgfpathlineto{\pgfqpoint{4.046545in}{2.193608in}}%
\pgfpathlineto{\pgfqpoint{4.327111in}{1.883389in}}%
\pgfpathlineto{\pgfqpoint{4.447354in}{1.744899in}}%
\pgfpathlineto{\pgfqpoint{4.567596in}{1.602817in}}%
\pgfpathlineto{\pgfqpoint{4.768000in}{1.357028in}}%
\pgfpathlineto{\pgfqpoint{4.768000in}{1.357028in}}%
\pgfusepath{fill}%
\end{pgfscope}%
\begin{pgfscope}%
\pgfpathrectangle{\pgfqpoint{0.800000in}{0.528000in}}{\pgfqpoint{3.968000in}{3.696000in}}%
\pgfusepath{clip}%
\pgfsetbuttcap%
\pgfsetroundjoin%
\definecolor{currentfill}{rgb}{0.273006,0.204520,0.501721}%
\pgfsetfillcolor{currentfill}%
\pgfsetlinewidth{0.000000pt}%
\definecolor{currentstroke}{rgb}{0.000000,0.000000,0.000000}%
\pgfsetstrokecolor{currentstroke}%
\pgfsetdash{}{0pt}%
\pgfpathmoveto{\pgfqpoint{2.263844in}{0.528000in}}%
\pgfpathlineto{\pgfqpoint{2.150440in}{0.640000in}}%
\pgfpathlineto{\pgfqpoint{1.859097in}{0.938667in}}%
\pgfpathlineto{\pgfqpoint{1.582856in}{1.237333in}}%
\pgfpathlineto{\pgfqpoint{1.320330in}{1.536671in}}%
\pgfpathlineto{\pgfqpoint{1.195336in}{1.685333in}}%
\pgfpathlineto{\pgfqpoint{1.103490in}{1.797333in}}%
\pgfpathlineto{\pgfqpoint{0.868143in}{2.096000in}}%
\pgfpathlineto{\pgfqpoint{0.800000in}{2.186067in}}%
\pgfpathlineto{\pgfqpoint{0.800000in}{2.178244in}}%
\pgfpathlineto{\pgfqpoint{0.890970in}{2.058667in}}%
\pgfpathlineto{\pgfqpoint{1.128108in}{1.760000in}}%
\pgfpathlineto{\pgfqpoint{1.240889in}{1.623886in}}%
\pgfpathlineto{\pgfqpoint{1.481374in}{1.344559in}}%
\pgfpathlineto{\pgfqpoint{1.761939in}{1.035623in}}%
\pgfpathlineto{\pgfqpoint{1.888720in}{0.901333in}}%
\pgfpathlineto{\pgfqpoint{2.181938in}{0.602667in}}%
\pgfpathlineto{\pgfqpoint{2.257727in}{0.528000in}}%
\pgfpathmoveto{\pgfqpoint{4.768000in}{1.371482in}}%
\pgfpathlineto{\pgfqpoint{4.665157in}{1.498667in}}%
\pgfpathlineto{\pgfqpoint{4.407273in}{1.804807in}}%
\pgfpathlineto{\pgfqpoint{4.282758in}{1.946667in}}%
\pgfpathlineto{\pgfqpoint{4.006465in}{2.249127in}}%
\pgfpathlineto{\pgfqpoint{3.722427in}{2.544000in}}%
\pgfpathlineto{\pgfqpoint{3.605657in}{2.660839in}}%
\pgfpathlineto{\pgfqpoint{3.457700in}{2.805333in}}%
\pgfpathlineto{\pgfqpoint{3.179731in}{3.066667in}}%
\pgfpathlineto{\pgfqpoint{2.924283in}{3.295437in}}%
\pgfpathlineto{\pgfqpoint{2.800141in}{3.402667in}}%
\pgfpathlineto{\pgfqpoint{2.523475in}{3.632385in}}%
\pgfpathlineto{\pgfqpoint{2.234604in}{3.858402in}}%
\pgfpathlineto{\pgfqpoint{2.094462in}{3.962667in}}%
\pgfpathlineto{\pgfqpoint{1.882182in}{4.113913in}}%
\pgfpathlineto{\pgfqpoint{1.719237in}{4.224000in}}%
\pgfpathlineto{\pgfqpoint{1.709529in}{4.224000in}}%
\pgfpathlineto{\pgfqpoint{1.802020in}{4.162429in}}%
\pgfpathlineto{\pgfqpoint{1.929630in}{4.074667in}}%
\pgfpathlineto{\pgfqpoint{2.187189in}{3.888000in}}%
\pgfpathlineto{\pgfqpoint{2.429718in}{3.701333in}}%
\pgfpathlineto{\pgfqpoint{2.683798in}{3.494906in}}%
\pgfpathlineto{\pgfqpoint{2.804040in}{3.393449in}}%
\pgfpathlineto{\pgfqpoint{2.943466in}{3.272799in}}%
\pgfpathlineto{\pgfqpoint{3.049672in}{3.178667in}}%
\pgfpathlineto{\pgfqpoint{3.349330in}{2.902578in}}%
\pgfpathlineto{\pgfqpoint{3.412672in}{2.842667in}}%
\pgfpathlineto{\pgfqpoint{3.720896in}{2.539340in}}%
\pgfpathlineto{\pgfqpoint{3.765980in}{2.493641in}}%
\pgfpathlineto{\pgfqpoint{4.046545in}{2.199937in}}%
\pgfpathlineto{\pgfqpoint{4.327111in}{1.889948in}}%
\pgfpathlineto{\pgfqpoint{4.447354in}{1.751659in}}%
\pgfpathlineto{\pgfqpoint{4.567596in}{1.609790in}}%
\pgfpathlineto{\pgfqpoint{4.768000in}{1.364255in}}%
\pgfpathlineto{\pgfqpoint{4.768000in}{1.364255in}}%
\pgfusepath{fill}%
\end{pgfscope}%
\begin{pgfscope}%
\pgfpathrectangle{\pgfqpoint{0.800000in}{0.528000in}}{\pgfqpoint{3.968000in}{3.696000in}}%
\pgfusepath{clip}%
\pgfsetbuttcap%
\pgfsetroundjoin%
\definecolor{currentfill}{rgb}{0.271828,0.209303,0.504434}%
\pgfsetfillcolor{currentfill}%
\pgfsetlinewidth{0.000000pt}%
\definecolor{currentstroke}{rgb}{0.000000,0.000000,0.000000}%
\pgfsetstrokecolor{currentstroke}%
\pgfsetdash{}{0pt}%
\pgfpathmoveto{\pgfqpoint{2.257727in}{0.528000in}}%
\pgfpathlineto{\pgfqpoint{2.144424in}{0.640000in}}%
\pgfpathlineto{\pgfqpoint{1.848050in}{0.944208in}}%
\pgfpathlineto{\pgfqpoint{1.802020in}{0.992836in}}%
\pgfpathlineto{\pgfqpoint{1.521455in}{1.299412in}}%
\pgfpathlineto{\pgfqpoint{1.280970in}{1.576258in}}%
\pgfpathlineto{\pgfqpoint{1.037473in}{1.872000in}}%
\pgfpathlineto{\pgfqpoint{0.800000in}{2.178244in}}%
\pgfpathlineto{\pgfqpoint{0.800000in}{2.170429in}}%
\pgfpathlineto{\pgfqpoint{1.031778in}{1.872000in}}%
\pgfpathlineto{\pgfqpoint{1.127693in}{1.753436in}}%
\pgfpathlineto{\pgfqpoint{1.246231in}{1.610667in}}%
\pgfpathlineto{\pgfqpoint{1.521455in}{1.292961in}}%
\pgfpathlineto{\pgfqpoint{1.641697in}{1.159559in}}%
\pgfpathlineto{\pgfqpoint{1.776964in}{1.013333in}}%
\pgfpathlineto{\pgfqpoint{2.064189in}{0.714667in}}%
\pgfpathlineto{\pgfqpoint{2.175889in}{0.602667in}}%
\pgfpathlineto{\pgfqpoint{2.251610in}{0.528000in}}%
\pgfpathmoveto{\pgfqpoint{4.768000in}{1.378709in}}%
\pgfpathlineto{\pgfqpoint{4.670890in}{1.498667in}}%
\pgfpathlineto{\pgfqpoint{4.407273in}{1.811388in}}%
\pgfpathlineto{\pgfqpoint{4.287030in}{1.948367in}}%
\pgfpathlineto{\pgfqpoint{4.006465in}{2.255299in}}%
\pgfpathlineto{\pgfqpoint{3.725899in}{2.546539in}}%
\pgfpathlineto{\pgfqpoint{3.578737in}{2.693333in}}%
\pgfpathlineto{\pgfqpoint{3.295846in}{2.964759in}}%
\pgfpathlineto{\pgfqpoint{3.235726in}{3.020761in}}%
\pgfpathlineto{\pgfqpoint{3.175244in}{3.076425in}}%
\pgfpathlineto{\pgfqpoint{3.124687in}{3.122727in}}%
\pgfpathlineto{\pgfqpoint{2.978770in}{3.253333in}}%
\pgfpathlineto{\pgfqpoint{2.718841in}{3.477333in}}%
\pgfpathlineto{\pgfqpoint{2.583613in}{3.589333in}}%
\pgfpathlineto{\pgfqpoint{2.483394in}{3.670475in}}%
\pgfpathlineto{\pgfqpoint{2.349603in}{3.776000in}}%
\pgfpathlineto{\pgfqpoint{2.102653in}{3.962667in}}%
\pgfpathlineto{\pgfqpoint{1.882182in}{4.120013in}}%
\pgfpathlineto{\pgfqpoint{1.728667in}{4.224000in}}%
\pgfpathlineto{\pgfqpoint{1.719237in}{4.224000in}}%
\pgfpathlineto{\pgfqpoint{1.802020in}{4.168653in}}%
\pgfpathlineto{\pgfqpoint{1.922263in}{4.085957in}}%
\pgfpathlineto{\pgfqpoint{2.043256in}{4.000000in}}%
\pgfpathlineto{\pgfqpoint{2.195259in}{3.888000in}}%
\pgfpathlineto{\pgfqpoint{2.403232in}{3.728172in}}%
\pgfpathlineto{\pgfqpoint{2.530532in}{3.626667in}}%
\pgfpathlineto{\pgfqpoint{2.667091in}{3.514667in}}%
\pgfpathlineto{\pgfqpoint{2.800141in}{3.402667in}}%
\pgfpathlineto{\pgfqpoint{2.886912in}{3.328000in}}%
\pgfpathlineto{\pgfqpoint{3.014344in}{3.216000in}}%
\pgfpathlineto{\pgfqpoint{3.138852in}{3.104000in}}%
\pgfpathlineto{\pgfqpoint{3.260587in}{2.992000in}}%
\pgfpathlineto{\pgfqpoint{3.565576in}{2.700363in}}%
\pgfpathlineto{\pgfqpoint{3.846141in}{2.417402in}}%
\pgfpathlineto{\pgfqpoint{4.126707in}{2.119430in}}%
\pgfpathlineto{\pgfqpoint{4.381279in}{1.834667in}}%
\pgfpathlineto{\pgfqpoint{4.487434in}{1.711495in}}%
\pgfpathlineto{\pgfqpoint{4.607677in}{1.568515in}}%
\pgfpathlineto{\pgfqpoint{4.768000in}{1.371482in}}%
\pgfpathlineto{\pgfqpoint{4.768000in}{1.371482in}}%
\pgfusepath{fill}%
\end{pgfscope}%
\begin{pgfscope}%
\pgfpathrectangle{\pgfqpoint{0.800000in}{0.528000in}}{\pgfqpoint{3.968000in}{3.696000in}}%
\pgfusepath{clip}%
\pgfsetbuttcap%
\pgfsetroundjoin%
\definecolor{currentfill}{rgb}{0.271828,0.209303,0.504434}%
\pgfsetfillcolor{currentfill}%
\pgfsetlinewidth{0.000000pt}%
\definecolor{currentstroke}{rgb}{0.000000,0.000000,0.000000}%
\pgfsetstrokecolor{currentstroke}%
\pgfsetdash{}{0pt}%
\pgfpathmoveto{\pgfqpoint{2.251610in}{0.528000in}}%
\pgfpathlineto{\pgfqpoint{2.130900in}{0.647669in}}%
\pgfpathlineto{\pgfqpoint{2.082586in}{0.696047in}}%
\pgfpathlineto{\pgfqpoint{1.954649in}{0.826667in}}%
\pgfpathlineto{\pgfqpoint{1.673078in}{1.125333in}}%
\pgfpathlineto{\pgfqpoint{1.561535in}{1.248148in}}%
\pgfpathlineto{\pgfqpoint{1.309520in}{1.536000in}}%
\pgfpathlineto{\pgfqpoint{1.200808in}{1.664992in}}%
\pgfpathlineto{\pgfqpoint{0.972493in}{1.946667in}}%
\pgfpathlineto{\pgfqpoint{0.900037in}{2.039846in}}%
\pgfpathlineto{\pgfqpoint{0.856562in}{2.096000in}}%
\pgfpathlineto{\pgfqpoint{0.800000in}{2.170429in}}%
\pgfpathlineto{\pgfqpoint{0.800000in}{2.162831in}}%
\pgfpathlineto{\pgfqpoint{1.026083in}{1.872000in}}%
\pgfpathlineto{\pgfqpoint{1.120646in}{1.755099in}}%
\pgfpathlineto{\pgfqpoint{1.240889in}{1.610172in}}%
\pgfpathlineto{\pgfqpoint{1.509187in}{1.300573in}}%
\pgfpathlineto{\pgfqpoint{1.565471in}{1.237333in}}%
\pgfpathlineto{\pgfqpoint{1.842101in}{0.937932in}}%
\pgfpathlineto{\pgfqpoint{2.132394in}{0.640000in}}%
\pgfpathlineto{\pgfqpoint{2.245493in}{0.528000in}}%
\pgfpathmoveto{\pgfqpoint{4.768000in}{1.385935in}}%
\pgfpathlineto{\pgfqpoint{4.676623in}{1.498667in}}%
\pgfpathlineto{\pgfqpoint{4.407273in}{1.817969in}}%
\pgfpathlineto{\pgfqpoint{4.287030in}{1.954757in}}%
\pgfpathlineto{\pgfqpoint{4.006465in}{2.261472in}}%
\pgfpathlineto{\pgfqpoint{3.725899in}{2.552508in}}%
\pgfpathlineto{\pgfqpoint{3.584761in}{2.693333in}}%
\pgfpathlineto{\pgfqpoint{3.285010in}{2.980993in}}%
\pgfpathlineto{\pgfqpoint{2.985341in}{3.253333in}}%
\pgfpathlineto{\pgfqpoint{2.706392in}{3.493621in}}%
\pgfpathlineto{\pgfqpoint{2.590771in}{3.589333in}}%
\pgfpathlineto{\pgfqpoint{2.452044in}{3.701333in}}%
\pgfpathlineto{\pgfqpoint{2.242909in}{3.863932in}}%
\pgfpathlineto{\pgfqpoint{2.110845in}{3.962667in}}%
\pgfpathlineto{\pgfqpoint{1.882182in}{4.126112in}}%
\pgfpathlineto{\pgfqpoint{1.737994in}{4.224000in}}%
\pgfpathlineto{\pgfqpoint{1.728667in}{4.224000in}}%
\pgfpathlineto{\pgfqpoint{1.842101in}{4.147668in}}%
\pgfpathlineto{\pgfqpoint{2.102653in}{3.962667in}}%
\pgfpathlineto{\pgfqpoint{2.203312in}{3.888000in}}%
\pgfpathlineto{\pgfqpoint{2.349603in}{3.776000in}}%
\pgfpathlineto{\pgfqpoint{2.583613in}{3.589333in}}%
\pgfpathlineto{\pgfqpoint{2.850449in}{3.365333in}}%
\pgfpathlineto{\pgfqpoint{3.145267in}{3.104000in}}%
\pgfpathlineto{\pgfqpoint{3.266890in}{2.992000in}}%
\pgfpathlineto{\pgfqpoint{3.540694in}{2.730667in}}%
\pgfpathlineto{\pgfqpoint{3.654064in}{2.618667in}}%
\pgfpathlineto{\pgfqpoint{3.765980in}{2.505871in}}%
\pgfpathlineto{\pgfqpoint{4.050724in}{2.208000in}}%
\pgfpathlineto{\pgfqpoint{4.327111in}{1.903066in}}%
\pgfpathlineto{\pgfqpoint{4.469039in}{1.739801in}}%
\pgfpathlineto{\pgfqpoint{4.578253in}{1.610667in}}%
\pgfpathlineto{\pgfqpoint{4.768000in}{1.378709in}}%
\pgfpathlineto{\pgfqpoint{4.768000in}{1.378709in}}%
\pgfusepath{fill}%
\end{pgfscope}%
\begin{pgfscope}%
\pgfpathrectangle{\pgfqpoint{0.800000in}{0.528000in}}{\pgfqpoint{3.968000in}{3.696000in}}%
\pgfusepath{clip}%
\pgfsetbuttcap%
\pgfsetroundjoin%
\definecolor{currentfill}{rgb}{0.271828,0.209303,0.504434}%
\pgfsetfillcolor{currentfill}%
\pgfsetlinewidth{0.000000pt}%
\definecolor{currentstroke}{rgb}{0.000000,0.000000,0.000000}%
\pgfsetstrokecolor{currentstroke}%
\pgfsetdash{}{0pt}%
\pgfpathmoveto{\pgfqpoint{2.245493in}{0.528000in}}%
\pgfpathlineto{\pgfqpoint{2.122667in}{0.649723in}}%
\pgfpathlineto{\pgfqpoint{1.985044in}{0.789333in}}%
\pgfpathlineto{\pgfqpoint{1.701720in}{1.088000in}}%
\pgfpathlineto{\pgfqpoint{1.599162in}{1.200000in}}%
\pgfpathlineto{\pgfqpoint{1.321051in}{1.515824in}}%
\pgfpathlineto{\pgfqpoint{1.200808in}{1.658117in}}%
\pgfpathlineto{\pgfqpoint{0.960323in}{1.954791in}}%
\pgfpathlineto{\pgfqpoint{0.840081in}{2.110002in}}%
\pgfpathlineto{\pgfqpoint{0.800000in}{2.162831in}}%
\pgfpathlineto{\pgfqpoint{0.800000in}{2.155233in}}%
\pgfpathlineto{\pgfqpoint{1.000404in}{1.896989in}}%
\pgfpathlineto{\pgfqpoint{1.240889in}{1.603483in}}%
\pgfpathlineto{\pgfqpoint{1.493065in}{1.312000in}}%
\pgfpathlineto{\pgfqpoint{1.765285in}{1.013333in}}%
\pgfpathlineto{\pgfqpoint{2.052290in}{0.714667in}}%
\pgfpathlineto{\pgfqpoint{2.163792in}{0.602667in}}%
\pgfpathlineto{\pgfqpoint{2.239465in}{0.528000in}}%
\pgfpathlineto{\pgfqpoint{2.242909in}{0.528000in}}%
\pgfpathmoveto{\pgfqpoint{4.768000in}{1.392990in}}%
\pgfpathlineto{\pgfqpoint{4.524444in}{1.688194in}}%
\pgfpathlineto{\pgfqpoint{4.398465in}{1.834667in}}%
\pgfpathlineto{\pgfqpoint{4.287030in}{1.961147in}}%
\pgfpathlineto{\pgfqpoint{4.006465in}{2.267644in}}%
\pgfpathlineto{\pgfqpoint{3.725899in}{2.558476in}}%
\pgfpathlineto{\pgfqpoint{3.590785in}{2.693333in}}%
\pgfpathlineto{\pgfqpoint{3.285010in}{2.986865in}}%
\pgfpathlineto{\pgfqpoint{2.991911in}{3.253333in}}%
\pgfpathlineto{\pgfqpoint{2.723879in}{3.484711in}}%
\pgfpathlineto{\pgfqpoint{2.597929in}{3.589333in}}%
\pgfpathlineto{\pgfqpoint{2.459346in}{3.701333in}}%
\pgfpathlineto{\pgfqpoint{2.218922in}{3.888000in}}%
\pgfpathlineto{\pgfqpoint{2.119036in}{3.962667in}}%
\pgfpathlineto{\pgfqpoint{1.857400in}{4.149333in}}%
\pgfpathlineto{\pgfqpoint{1.747321in}{4.224000in}}%
\pgfpathlineto{\pgfqpoint{1.737994in}{4.224000in}}%
\pgfpathlineto{\pgfqpoint{1.848571in}{4.149333in}}%
\pgfpathlineto{\pgfqpoint{2.094888in}{3.974125in}}%
\pgfpathlineto{\pgfqpoint{2.168207in}{3.920248in}}%
\pgfpathlineto{\pgfqpoint{2.309013in}{3.813333in}}%
\pgfpathlineto{\pgfqpoint{2.404955in}{3.738667in}}%
\pgfpathlineto{\pgfqpoint{2.544944in}{3.626667in}}%
\pgfpathlineto{\pgfqpoint{2.681221in}{3.514667in}}%
\pgfpathlineto{\pgfqpoint{2.769942in}{3.440000in}}%
\pgfpathlineto{\pgfqpoint{3.069148in}{3.178667in}}%
\pgfpathlineto{\pgfqpoint{3.352736in}{2.917333in}}%
\pgfpathlineto{\pgfqpoint{3.622516in}{2.656000in}}%
\pgfpathlineto{\pgfqpoint{3.734299in}{2.544000in}}%
\pgfpathlineto{\pgfqpoint{3.846141in}{2.429671in}}%
\pgfpathlineto{\pgfqpoint{4.126707in}{2.132129in}}%
\pgfpathlineto{\pgfqpoint{4.407273in}{1.817969in}}%
\pgfpathlineto{\pgfqpoint{4.647758in}{1.533901in}}%
\pgfpathlineto{\pgfqpoint{4.768000in}{1.385935in}}%
\pgfpathlineto{\pgfqpoint{4.768000in}{1.386667in}}%
\pgfpathlineto{\pgfqpoint{4.768000in}{1.386667in}}%
\pgfusepath{fill}%
\end{pgfscope}%
\begin{pgfscope}%
\pgfpathrectangle{\pgfqpoint{0.800000in}{0.528000in}}{\pgfqpoint{3.968000in}{3.696000in}}%
\pgfusepath{clip}%
\pgfsetbuttcap%
\pgfsetroundjoin%
\definecolor{currentfill}{rgb}{0.271828,0.209303,0.504434}%
\pgfsetfillcolor{currentfill}%
\pgfsetlinewidth{0.000000pt}%
\definecolor{currentstroke}{rgb}{0.000000,0.000000,0.000000}%
\pgfsetstrokecolor{currentstroke}%
\pgfsetdash{}{0pt}%
\pgfpathmoveto{\pgfqpoint{2.239465in}{0.528000in}}%
\pgfpathlineto{\pgfqpoint{2.122667in}{0.643710in}}%
\pgfpathlineto{\pgfqpoint{1.979159in}{0.789333in}}%
\pgfpathlineto{\pgfqpoint{1.695942in}{1.088000in}}%
\pgfpathlineto{\pgfqpoint{1.593476in}{1.200000in}}%
\pgfpathlineto{\pgfqpoint{1.321051in}{1.509156in}}%
\pgfpathlineto{\pgfqpoint{1.200808in}{1.651242in}}%
\pgfpathlineto{\pgfqpoint{0.960323in}{1.947459in}}%
\pgfpathlineto{\pgfqpoint{0.840081in}{2.102419in}}%
\pgfpathlineto{\pgfqpoint{0.800000in}{2.155233in}}%
\pgfpathlineto{\pgfqpoint{0.800000in}{2.147635in}}%
\pgfpathlineto{\pgfqpoint{1.000404in}{1.889867in}}%
\pgfpathlineto{\pgfqpoint{1.229209in}{1.610667in}}%
\pgfpathlineto{\pgfqpoint{1.324330in}{1.498667in}}%
\pgfpathlineto{\pgfqpoint{1.601616in}{1.184736in}}%
\pgfpathlineto{\pgfqpoint{1.724700in}{1.050667in}}%
\pgfpathlineto{\pgfqpoint{2.009690in}{0.752000in}}%
\pgfpathlineto{\pgfqpoint{2.122667in}{0.637749in}}%
\pgfpathlineto{\pgfqpoint{2.233502in}{0.528000in}}%
\pgfpathmoveto{\pgfqpoint{4.768000in}{1.400024in}}%
\pgfpathlineto{\pgfqpoint{4.527515in}{1.691206in}}%
\pgfpathlineto{\pgfqpoint{4.390880in}{1.849935in}}%
\pgfpathlineto{\pgfqpoint{4.272317in}{1.984000in}}%
\pgfpathlineto{\pgfqpoint{3.998124in}{2.282667in}}%
\pgfpathlineto{\pgfqpoint{3.886222in}{2.400227in}}%
\pgfpathlineto{\pgfqpoint{3.596809in}{2.693333in}}%
\pgfpathlineto{\pgfqpoint{3.285010in}{2.992721in}}%
\pgfpathlineto{\pgfqpoint{2.964364in}{3.283470in}}%
\pgfpathlineto{\pgfqpoint{2.827186in}{3.402667in}}%
\pgfpathlineto{\pgfqpoint{2.695018in}{3.514667in}}%
\pgfpathlineto{\pgfqpoint{2.443313in}{3.719936in}}%
\pgfpathlineto{\pgfqpoint{2.323071in}{3.814336in}}%
\pgfpathlineto{\pgfqpoint{2.042505in}{4.024538in}}%
\pgfpathlineto{\pgfqpoint{1.962343in}{4.082049in}}%
\pgfpathlineto{\pgfqpoint{1.842101in}{4.165978in}}%
\pgfpathlineto{\pgfqpoint{1.756648in}{4.224000in}}%
\pgfpathlineto{\pgfqpoint{1.747321in}{4.224000in}}%
\pgfpathlineto{\pgfqpoint{1.857400in}{4.149333in}}%
\pgfpathlineto{\pgfqpoint{2.082586in}{3.989397in}}%
\pgfpathlineto{\pgfqpoint{2.364830in}{3.776000in}}%
\pgfpathlineto{\pgfqpoint{2.505958in}{3.664000in}}%
\pgfpathlineto{\pgfqpoint{2.769984in}{3.445612in}}%
\pgfpathlineto{\pgfqpoint{2.820452in}{3.402667in}}%
\pgfpathlineto{\pgfqpoint{2.949556in}{3.290667in}}%
\pgfpathlineto{\pgfqpoint{3.075640in}{3.178667in}}%
\pgfpathlineto{\pgfqpoint{3.325091in}{2.949304in}}%
\pgfpathlineto{\pgfqpoint{3.476078in}{2.805333in}}%
\pgfpathlineto{\pgfqpoint{3.605657in}{2.678664in}}%
\pgfpathlineto{\pgfqpoint{3.740192in}{2.544000in}}%
\pgfpathlineto{\pgfqpoint{3.849738in}{2.432000in}}%
\pgfpathlineto{\pgfqpoint{4.131322in}{2.133333in}}%
\pgfpathlineto{\pgfqpoint{4.407273in}{1.824549in}}%
\pgfpathlineto{\pgfqpoint{4.651697in}{1.536000in}}%
\pgfpathlineto{\pgfqpoint{4.768000in}{1.392990in}}%
\pgfpathlineto{\pgfqpoint{4.768000in}{1.392990in}}%
\pgfusepath{fill}%
\end{pgfscope}%
\begin{pgfscope}%
\pgfpathrectangle{\pgfqpoint{0.800000in}{0.528000in}}{\pgfqpoint{3.968000in}{3.696000in}}%
\pgfusepath{clip}%
\pgfsetbuttcap%
\pgfsetroundjoin%
\definecolor{currentfill}{rgb}{0.270595,0.214069,0.507052}%
\pgfsetfillcolor{currentfill}%
\pgfsetlinewidth{0.000000pt}%
\definecolor{currentstroke}{rgb}{0.000000,0.000000,0.000000}%
\pgfsetstrokecolor{currentstroke}%
\pgfsetdash{}{0pt}%
\pgfpathmoveto{\pgfqpoint{2.233502in}{0.528000in}}%
\pgfpathlineto{\pgfqpoint{2.091393in}{0.669130in}}%
\pgfpathlineto{\pgfqpoint{1.973274in}{0.789333in}}%
\pgfpathlineto{\pgfqpoint{1.681778in}{1.097085in}}%
\pgfpathlineto{\pgfqpoint{1.401212in}{1.409899in}}%
\pgfpathlineto{\pgfqpoint{1.280970in}{1.549470in}}%
\pgfpathlineto{\pgfqpoint{1.160727in}{1.692656in}}%
\pgfpathlineto{\pgfqpoint{0.920242in}{1.991316in}}%
\pgfpathlineto{\pgfqpoint{0.800000in}{2.147635in}}%
\pgfpathlineto{\pgfqpoint{0.800000in}{2.140037in}}%
\pgfpathlineto{\pgfqpoint{0.979253in}{1.909333in}}%
\pgfpathlineto{\pgfqpoint{1.080566in}{1.783396in}}%
\pgfpathlineto{\pgfqpoint{1.200808in}{1.637758in}}%
\pgfpathlineto{\pgfqpoint{1.448641in}{1.349333in}}%
\pgfpathlineto{\pgfqpoint{1.561535in}{1.222740in}}%
\pgfpathlineto{\pgfqpoint{1.842101in}{0.919703in}}%
\pgfpathlineto{\pgfqpoint{2.122667in}{0.631870in}}%
\pgfpathlineto{\pgfqpoint{2.227538in}{0.528000in}}%
\pgfpathmoveto{\pgfqpoint{4.768000in}{1.407059in}}%
\pgfpathlineto{\pgfqpoint{4.538107in}{1.685333in}}%
\pgfpathlineto{\pgfqpoint{4.442238in}{1.797333in}}%
\pgfpathlineto{\pgfqpoint{4.166788in}{2.107074in}}%
\pgfpathlineto{\pgfqpoint{3.886222in}{2.406231in}}%
\pgfpathlineto{\pgfqpoint{3.602832in}{2.693333in}}%
\pgfpathlineto{\pgfqpoint{3.285010in}{2.998465in}}%
\pgfpathlineto{\pgfqpoint{2.962777in}{3.290667in}}%
\pgfpathlineto{\pgfqpoint{2.657127in}{3.552000in}}%
\pgfpathlineto{\pgfqpoint{2.403232in}{3.757468in}}%
\pgfpathlineto{\pgfqpoint{2.280442in}{3.853040in}}%
\pgfpathlineto{\pgfqpoint{2.134995in}{3.962667in}}%
\pgfpathlineto{\pgfqpoint{2.033079in}{4.037333in}}%
\pgfpathlineto{\pgfqpoint{1.802020in}{4.199473in}}%
\pgfpathlineto{\pgfqpoint{1.765822in}{4.224000in}}%
\pgfpathlineto{\pgfqpoint{1.756648in}{4.224000in}}%
\pgfpathlineto{\pgfqpoint{1.842101in}{4.165978in}}%
\pgfpathlineto{\pgfqpoint{1.972690in}{4.074667in}}%
\pgfpathlineto{\pgfqpoint{2.082586in}{3.995404in}}%
\pgfpathlineto{\pgfqpoint{2.372232in}{3.776000in}}%
\pgfpathlineto{\pgfqpoint{2.513212in}{3.664000in}}%
\pgfpathlineto{\pgfqpoint{2.763960in}{3.456611in}}%
\pgfpathlineto{\pgfqpoint{3.084606in}{3.176445in}}%
\pgfpathlineto{\pgfqpoint{3.205231in}{3.066667in}}%
\pgfpathlineto{\pgfqpoint{3.325628in}{2.954667in}}%
\pgfpathlineto{\pgfqpoint{3.445333in}{2.840884in}}%
\pgfpathlineto{\pgfqpoint{3.746085in}{2.544000in}}%
\pgfpathlineto{\pgfqpoint{3.855537in}{2.432000in}}%
\pgfpathlineto{\pgfqpoint{4.137016in}{2.133333in}}%
\pgfpathlineto{\pgfqpoint{4.407273in}{1.831130in}}%
\pgfpathlineto{\pgfqpoint{4.657324in}{1.536000in}}%
\pgfpathlineto{\pgfqpoint{4.768000in}{1.400024in}}%
\pgfpathlineto{\pgfqpoint{4.768000in}{1.400024in}}%
\pgfusepath{fill}%
\end{pgfscope}%
\begin{pgfscope}%
\pgfpathrectangle{\pgfqpoint{0.800000in}{0.528000in}}{\pgfqpoint{3.968000in}{3.696000in}}%
\pgfusepath{clip}%
\pgfsetbuttcap%
\pgfsetroundjoin%
\definecolor{currentfill}{rgb}{0.270595,0.214069,0.507052}%
\pgfsetfillcolor{currentfill}%
\pgfsetlinewidth{0.000000pt}%
\definecolor{currentstroke}{rgb}{0.000000,0.000000,0.000000}%
\pgfsetstrokecolor{currentstroke}%
\pgfsetdash{}{0pt}%
\pgfpathmoveto{\pgfqpoint{2.227538in}{0.528000in}}%
\pgfpathlineto{\pgfqpoint{2.114554in}{0.640000in}}%
\pgfpathlineto{\pgfqpoint{2.002424in}{0.753377in}}%
\pgfpathlineto{\pgfqpoint{1.718963in}{1.050667in}}%
\pgfpathlineto{\pgfqpoint{1.441293in}{1.357671in}}%
\pgfpathlineto{\pgfqpoint{1.192245in}{1.648000in}}%
\pgfpathlineto{\pgfqpoint{0.949661in}{1.946667in}}%
\pgfpathlineto{\pgfqpoint{0.862301in}{2.058667in}}%
\pgfpathlineto{\pgfqpoint{0.800000in}{2.140037in}}%
\pgfpathlineto{\pgfqpoint{0.800000in}{2.132464in}}%
\pgfpathlineto{\pgfqpoint{1.040485in}{1.825826in}}%
\pgfpathlineto{\pgfqpoint{1.281440in}{1.535562in}}%
\pgfpathlineto{\pgfqpoint{1.410199in}{1.386667in}}%
\pgfpathlineto{\pgfqpoint{1.681778in}{1.084648in}}%
\pgfpathlineto{\pgfqpoint{1.962343in}{0.788490in}}%
\pgfpathlineto{\pgfqpoint{2.221575in}{0.528000in}}%
\pgfpathmoveto{\pgfqpoint{4.768000in}{1.414094in}}%
\pgfpathlineto{\pgfqpoint{4.543717in}{1.685333in}}%
\pgfpathlineto{\pgfqpoint{4.447354in}{1.797991in}}%
\pgfpathlineto{\pgfqpoint{4.166788in}{2.113284in}}%
\pgfpathlineto{\pgfqpoint{3.886222in}{2.412236in}}%
\pgfpathlineto{\pgfqpoint{3.605657in}{2.696420in}}%
\pgfpathlineto{\pgfqpoint{3.285010in}{3.004210in}}%
\pgfpathlineto{\pgfqpoint{2.964364in}{3.294983in}}%
\pgfpathlineto{\pgfqpoint{2.664032in}{3.552000in}}%
\pgfpathlineto{\pgfqpoint{2.403232in}{3.763281in}}%
\pgfpathlineto{\pgfqpoint{2.282990in}{3.856876in}}%
\pgfpathlineto{\pgfqpoint{2.162747in}{3.947911in}}%
\pgfpathlineto{\pgfqpoint{2.082586in}{4.007253in}}%
\pgfpathlineto{\pgfqpoint{1.962343in}{4.094009in}}%
\pgfpathlineto{\pgfqpoint{1.829666in}{4.186667in}}%
\pgfpathlineto{\pgfqpoint{1.774797in}{4.224000in}}%
\pgfpathlineto{\pgfqpoint{1.765822in}{4.224000in}}%
\pgfpathlineto{\pgfqpoint{2.002424in}{4.059408in}}%
\pgfpathlineto{\pgfqpoint{2.283537in}{3.850667in}}%
\pgfpathlineto{\pgfqpoint{2.566472in}{3.626667in}}%
\pgfpathlineto{\pgfqpoint{2.701879in}{3.514667in}}%
\pgfpathlineto{\pgfqpoint{2.833919in}{3.402667in}}%
\pgfpathlineto{\pgfqpoint{2.964364in}{3.289274in}}%
\pgfpathlineto{\pgfqpoint{3.285010in}{2.998465in}}%
\pgfpathlineto{\pgfqpoint{3.565576in}{2.730028in}}%
\pgfpathlineto{\pgfqpoint{3.861336in}{2.432000in}}%
\pgfpathlineto{\pgfqpoint{4.142710in}{2.133333in}}%
\pgfpathlineto{\pgfqpoint{4.409860in}{1.834667in}}%
\pgfpathlineto{\pgfqpoint{4.673608in}{1.522745in}}%
\pgfpathlineto{\pgfqpoint{4.727919in}{1.456704in}}%
\pgfpathlineto{\pgfqpoint{4.768000in}{1.407059in}}%
\pgfpathlineto{\pgfqpoint{4.768000in}{1.407059in}}%
\pgfusepath{fill}%
\end{pgfscope}%
\begin{pgfscope}%
\pgfpathrectangle{\pgfqpoint{0.800000in}{0.528000in}}{\pgfqpoint{3.968000in}{3.696000in}}%
\pgfusepath{clip}%
\pgfsetbuttcap%
\pgfsetroundjoin%
\definecolor{currentfill}{rgb}{0.270595,0.214069,0.507052}%
\pgfsetfillcolor{currentfill}%
\pgfsetlinewidth{0.000000pt}%
\definecolor{currentstroke}{rgb}{0.000000,0.000000,0.000000}%
\pgfsetstrokecolor{currentstroke}%
\pgfsetdash{}{0pt}%
\pgfpathmoveto{\pgfqpoint{2.221575in}{0.528000in}}%
\pgfpathlineto{\pgfqpoint{2.108687in}{0.640000in}}%
\pgfpathlineto{\pgfqpoint{1.997967in}{0.752000in}}%
\pgfpathlineto{\pgfqpoint{1.713293in}{1.050667in}}%
\pgfpathlineto{\pgfqpoint{1.441293in}{1.351199in}}%
\pgfpathlineto{\pgfqpoint{1.175459in}{1.661722in}}%
\pgfpathlineto{\pgfqpoint{1.120646in}{1.727504in}}%
\pgfpathlineto{\pgfqpoint{0.880162in}{2.028178in}}%
\pgfpathlineto{\pgfqpoint{0.800000in}{2.132464in}}%
\pgfpathlineto{\pgfqpoint{0.800000in}{2.125079in}}%
\pgfpathlineto{\pgfqpoint{1.027756in}{1.834667in}}%
\pgfpathlineto{\pgfqpoint{1.120646in}{1.720658in}}%
\pgfpathlineto{\pgfqpoint{1.372001in}{1.424000in}}%
\pgfpathlineto{\pgfqpoint{1.481374in}{1.299631in}}%
\pgfpathlineto{\pgfqpoint{1.761939in}{0.992446in}}%
\pgfpathlineto{\pgfqpoint{2.042505in}{0.700780in}}%
\pgfpathlineto{\pgfqpoint{2.177767in}{0.565333in}}%
\pgfpathlineto{\pgfqpoint{2.215612in}{0.528000in}}%
\pgfpathmoveto{\pgfqpoint{4.768000in}{1.421129in}}%
\pgfpathlineto{\pgfqpoint{4.567596in}{1.663763in}}%
\pgfpathlineto{\pgfqpoint{4.322631in}{1.946667in}}%
\pgfpathlineto{\pgfqpoint{4.046545in}{2.249482in}}%
\pgfpathlineto{\pgfqpoint{3.763765in}{2.544000in}}%
\pgfpathlineto{\pgfqpoint{3.461536in}{2.842667in}}%
\pgfpathlineto{\pgfqpoint{3.325091in}{2.972431in}}%
\pgfpathlineto{\pgfqpoint{3.017824in}{3.253333in}}%
\pgfpathlineto{\pgfqpoint{2.884202in}{3.370748in}}%
\pgfpathlineto{\pgfqpoint{2.759893in}{3.477333in}}%
\pgfpathlineto{\pgfqpoint{2.483394in}{3.705366in}}%
\pgfpathlineto{\pgfqpoint{2.200737in}{3.925333in}}%
\pgfpathlineto{\pgfqpoint{2.049470in}{4.037333in}}%
\pgfpathlineto{\pgfqpoint{1.945380in}{4.112000in}}%
\pgfpathlineto{\pgfqpoint{1.783772in}{4.224000in}}%
\pgfpathlineto{\pgfqpoint{1.774797in}{4.224000in}}%
\pgfpathlineto{\pgfqpoint{2.002424in}{4.065397in}}%
\pgfpathlineto{\pgfqpoint{2.291041in}{3.850667in}}%
\pgfpathlineto{\pgfqpoint{2.573466in}{3.626667in}}%
\pgfpathlineto{\pgfqpoint{2.708740in}{3.514667in}}%
\pgfpathlineto{\pgfqpoint{2.844121in}{3.399690in}}%
\pgfpathlineto{\pgfqpoint{2.969252in}{3.290667in}}%
\pgfpathlineto{\pgfqpoint{3.271376in}{3.016634in}}%
\pgfpathlineto{\pgfqpoint{3.337841in}{2.954667in}}%
\pgfpathlineto{\pgfqpoint{3.455531in}{2.842667in}}%
\pgfpathlineto{\pgfqpoint{3.765980in}{2.535776in}}%
\pgfpathlineto{\pgfqpoint{4.046545in}{2.243397in}}%
\pgfpathlineto{\pgfqpoint{4.327111in}{1.935220in}}%
\pgfpathlineto{\pgfqpoint{4.450267in}{1.794620in}}%
\pgfpathlineto{\pgfqpoint{4.575296in}{1.648000in}}%
\pgfpathlineto{\pgfqpoint{4.687838in}{1.512623in}}%
\pgfpathlineto{\pgfqpoint{4.768000in}{1.414094in}}%
\pgfpathlineto{\pgfqpoint{4.768000in}{1.414094in}}%
\pgfusepath{fill}%
\end{pgfscope}%
\begin{pgfscope}%
\pgfpathrectangle{\pgfqpoint{0.800000in}{0.528000in}}{\pgfqpoint{3.968000in}{3.696000in}}%
\pgfusepath{clip}%
\pgfsetbuttcap%
\pgfsetroundjoin%
\definecolor{currentfill}{rgb}{0.270595,0.214069,0.507052}%
\pgfsetfillcolor{currentfill}%
\pgfsetlinewidth{0.000000pt}%
\definecolor{currentstroke}{rgb}{0.000000,0.000000,0.000000}%
\pgfsetstrokecolor{currentstroke}%
\pgfsetdash{}{0pt}%
\pgfpathmoveto{\pgfqpoint{2.215612in}{0.528000in}}%
\pgfpathlineto{\pgfqpoint{2.082586in}{0.660306in}}%
\pgfpathlineto{\pgfqpoint{1.955782in}{0.789333in}}%
\pgfpathlineto{\pgfqpoint{1.842101in}{0.907551in}}%
\pgfpathlineto{\pgfqpoint{1.561535in}{1.210165in}}%
\pgfpathlineto{\pgfqpoint{1.437329in}{1.349333in}}%
\pgfpathlineto{\pgfqpoint{1.160727in}{1.672334in}}%
\pgfpathlineto{\pgfqpoint{1.050561in}{1.806719in}}%
\pgfpathlineto{\pgfqpoint{0.989664in}{1.882004in}}%
\pgfpathlineto{\pgfqpoint{0.878747in}{2.022651in}}%
\pgfpathlineto{\pgfqpoint{0.800000in}{2.125079in}}%
\pgfpathlineto{\pgfqpoint{0.800000in}{2.117693in}}%
\pgfpathlineto{\pgfqpoint{1.000404in}{1.861661in}}%
\pgfpathlineto{\pgfqpoint{1.120646in}{1.713935in}}%
\pgfpathlineto{\pgfqpoint{1.366356in}{1.424000in}}%
\pgfpathlineto{\pgfqpoint{1.481374in}{1.293324in}}%
\pgfpathlineto{\pgfqpoint{1.748587in}{1.000897in}}%
\pgfpathlineto{\pgfqpoint{1.806845in}{0.938667in}}%
\pgfpathlineto{\pgfqpoint{2.109472in}{0.627710in}}%
\pgfpathlineto{\pgfqpoint{2.171837in}{0.565333in}}%
\pgfpathlineto{\pgfqpoint{2.209648in}{0.528000in}}%
\pgfpathmoveto{\pgfqpoint{4.768000in}{1.428056in}}%
\pgfpathlineto{\pgfqpoint{4.523245in}{1.722667in}}%
\pgfpathlineto{\pgfqpoint{4.246949in}{2.037467in}}%
\pgfpathlineto{\pgfqpoint{3.966384in}{2.340469in}}%
\pgfpathlineto{\pgfqpoint{3.685818in}{2.628401in}}%
\pgfpathlineto{\pgfqpoint{3.377319in}{2.928648in}}%
\pgfpathlineto{\pgfqpoint{3.310343in}{2.992000in}}%
\pgfpathlineto{\pgfqpoint{3.024217in}{3.253333in}}%
\pgfpathlineto{\pgfqpoint{2.884202in}{3.376411in}}%
\pgfpathlineto{\pgfqpoint{2.763960in}{3.479609in}}%
\pgfpathlineto{\pgfqpoint{2.467280in}{3.723658in}}%
\pgfpathlineto{\pgfqpoint{2.401840in}{3.776000in}}%
\pgfpathlineto{\pgfqpoint{2.257468in}{3.888000in}}%
\pgfpathlineto{\pgfqpoint{2.042505in}{4.048282in}}%
\pgfpathlineto{\pgfqpoint{1.792747in}{4.224000in}}%
\pgfpathlineto{\pgfqpoint{1.783772in}{4.224000in}}%
\pgfpathlineto{\pgfqpoint{2.002424in}{4.071386in}}%
\pgfpathlineto{\pgfqpoint{2.282990in}{3.862664in}}%
\pgfpathlineto{\pgfqpoint{2.403232in}{3.769094in}}%
\pgfpathlineto{\pgfqpoint{2.698695in}{3.528542in}}%
\pgfpathlineto{\pgfqpoint{2.763960in}{3.473897in}}%
\pgfpathlineto{\pgfqpoint{3.059646in}{3.216000in}}%
\pgfpathlineto{\pgfqpoint{3.183259in}{3.104000in}}%
\pgfpathlineto{\pgfqpoint{3.473150in}{2.831243in}}%
\pgfpathlineto{\pgfqpoint{3.538604in}{2.768000in}}%
\pgfpathlineto{\pgfqpoint{3.846141in}{2.459700in}}%
\pgfpathlineto{\pgfqpoint{4.126707in}{2.163162in}}%
\pgfpathlineto{\pgfqpoint{4.255862in}{2.021333in}}%
\pgfpathlineto{\pgfqpoint{4.527515in}{1.711046in}}%
\pgfpathlineto{\pgfqpoint{4.647758in}{1.568043in}}%
\pgfpathlineto{\pgfqpoint{4.768000in}{1.421129in}}%
\pgfpathlineto{\pgfqpoint{4.768000in}{1.424000in}}%
\pgfusepath{fill}%
\end{pgfscope}%
\begin{pgfscope}%
\pgfpathrectangle{\pgfqpoint{0.800000in}{0.528000in}}{\pgfqpoint{3.968000in}{3.696000in}}%
\pgfusepath{clip}%
\pgfsetbuttcap%
\pgfsetroundjoin%
\definecolor{currentfill}{rgb}{0.269308,0.218818,0.509577}%
\pgfsetfillcolor{currentfill}%
\pgfsetlinewidth{0.000000pt}%
\definecolor{currentstroke}{rgb}{0.000000,0.000000,0.000000}%
\pgfsetstrokecolor{currentstroke}%
\pgfsetdash{}{0pt}%
\pgfpathmoveto{\pgfqpoint{2.209648in}{0.528000in}}%
\pgfpathlineto{\pgfqpoint{2.082586in}{0.654418in}}%
\pgfpathlineto{\pgfqpoint{1.950039in}{0.789333in}}%
\pgfpathlineto{\pgfqpoint{1.842101in}{0.901474in}}%
\pgfpathlineto{\pgfqpoint{1.561535in}{1.203878in}}%
\pgfpathlineto{\pgfqpoint{1.431759in}{1.349333in}}%
\pgfpathlineto{\pgfqpoint{1.160727in}{1.665622in}}%
\pgfpathlineto{\pgfqpoint{1.040485in}{1.811981in}}%
\pgfpathlineto{\pgfqpoint{0.920242in}{1.962531in}}%
\pgfpathlineto{\pgfqpoint{0.800000in}{2.117693in}}%
\pgfpathlineto{\pgfqpoint{0.800000in}{2.110307in}}%
\pgfpathlineto{\pgfqpoint{1.000404in}{1.854726in}}%
\pgfpathlineto{\pgfqpoint{1.120646in}{1.707211in}}%
\pgfpathlineto{\pgfqpoint{1.361131in}{1.423529in}}%
\pgfpathlineto{\pgfqpoint{1.492392in}{1.274667in}}%
\pgfpathlineto{\pgfqpoint{1.601616in}{1.153563in}}%
\pgfpathlineto{\pgfqpoint{1.882182in}{0.853650in}}%
\pgfpathlineto{\pgfqpoint{2.017227in}{0.714667in}}%
\pgfpathlineto{\pgfqpoint{2.203685in}{0.528000in}}%
\pgfpathmoveto{\pgfqpoint{4.768000in}{1.434908in}}%
\pgfpathlineto{\pgfqpoint{4.527515in}{1.724233in}}%
\pgfpathlineto{\pgfqpoint{4.246949in}{2.043697in}}%
\pgfpathlineto{\pgfqpoint{3.966384in}{2.346492in}}%
\pgfpathlineto{\pgfqpoint{3.685818in}{2.634229in}}%
\pgfpathlineto{\pgfqpoint{3.395558in}{2.917333in}}%
\pgfpathlineto{\pgfqpoint{3.084606in}{3.205010in}}%
\pgfpathlineto{\pgfqpoint{2.946159in}{3.328000in}}%
\pgfpathlineto{\pgfqpoint{2.683798in}{3.552772in}}%
\pgfpathlineto{\pgfqpoint{2.403232in}{3.780618in}}%
\pgfpathlineto{\pgfqpoint{2.265024in}{3.888000in}}%
\pgfpathlineto{\pgfqpoint{2.014182in}{4.074667in}}%
\pgfpathlineto{\pgfqpoint{1.801721in}{4.224000in}}%
\pgfpathlineto{\pgfqpoint{1.792747in}{4.224000in}}%
\pgfpathlineto{\pgfqpoint{1.962343in}{4.105970in}}%
\pgfpathlineto{\pgfqpoint{2.242909in}{3.899135in}}%
\pgfpathlineto{\pgfqpoint{2.359016in}{3.809481in}}%
\pgfpathlineto{\pgfqpoint{2.403232in}{3.774907in}}%
\pgfpathlineto{\pgfqpoint{2.683798in}{3.547043in}}%
\pgfpathlineto{\pgfqpoint{2.993457in}{3.280432in}}%
\pgfpathlineto{\pgfqpoint{3.044525in}{3.235244in}}%
\pgfpathlineto{\pgfqpoint{3.189506in}{3.104000in}}%
\pgfpathlineto{\pgfqpoint{3.467541in}{2.842667in}}%
\pgfpathlineto{\pgfqpoint{3.765980in}{2.547648in}}%
\pgfpathlineto{\pgfqpoint{3.886222in}{2.424244in}}%
\pgfpathlineto{\pgfqpoint{4.021033in}{2.282667in}}%
\pgfpathlineto{\pgfqpoint{4.294971in}{1.984000in}}%
\pgfpathlineto{\pgfqpoint{4.567596in}{1.670387in}}%
\pgfpathlineto{\pgfqpoint{4.687838in}{1.526281in}}%
\pgfpathlineto{\pgfqpoint{4.768000in}{1.428056in}}%
\pgfpathlineto{\pgfqpoint{4.768000in}{1.428056in}}%
\pgfusepath{fill}%
\end{pgfscope}%
\begin{pgfscope}%
\pgfpathrectangle{\pgfqpoint{0.800000in}{0.528000in}}{\pgfqpoint{3.968000in}{3.696000in}}%
\pgfusepath{clip}%
\pgfsetbuttcap%
\pgfsetroundjoin%
\definecolor{currentfill}{rgb}{0.269308,0.218818,0.509577}%
\pgfsetfillcolor{currentfill}%
\pgfsetlinewidth{0.000000pt}%
\definecolor{currentstroke}{rgb}{0.000000,0.000000,0.000000}%
\pgfsetstrokecolor{currentstroke}%
\pgfsetdash{}{0pt}%
\pgfpathmoveto{\pgfqpoint{2.203685in}{0.528000in}}%
\pgfpathlineto{\pgfqpoint{2.082586in}{0.648531in}}%
\pgfpathlineto{\pgfqpoint{1.944297in}{0.789333in}}%
\pgfpathlineto{\pgfqpoint{1.836579in}{0.901333in}}%
\pgfpathlineto{\pgfqpoint{1.548147in}{1.212470in}}%
\pgfpathlineto{\pgfqpoint{1.426189in}{1.349333in}}%
\pgfpathlineto{\pgfqpoint{1.160727in}{1.658910in}}%
\pgfpathlineto{\pgfqpoint{1.040485in}{1.805059in}}%
\pgfpathlineto{\pgfqpoint{0.920242in}{1.955385in}}%
\pgfpathlineto{\pgfqpoint{0.800000in}{2.110307in}}%
\pgfpathlineto{\pgfqpoint{0.800000in}{2.102921in}}%
\pgfpathlineto{\pgfqpoint{0.989406in}{1.861756in}}%
\pgfpathlineto{\pgfqpoint{1.041136in}{1.797333in}}%
\pgfpathlineto{\pgfqpoint{1.290735in}{1.498667in}}%
\pgfpathlineto{\pgfqpoint{1.561535in}{1.191504in}}%
\pgfpathlineto{\pgfqpoint{1.842101in}{0.889591in}}%
\pgfpathlineto{\pgfqpoint{1.974875in}{0.752000in}}%
\pgfpathlineto{\pgfqpoint{2.085220in}{0.640000in}}%
\pgfpathlineto{\pgfqpoint{2.197847in}{0.528000in}}%
\pgfpathlineto{\pgfqpoint{2.202828in}{0.528000in}}%
\pgfpathmoveto{\pgfqpoint{4.768000in}{1.441761in}}%
\pgfpathlineto{\pgfqpoint{4.527515in}{1.730685in}}%
\pgfpathlineto{\pgfqpoint{4.272681in}{2.021333in}}%
\pgfpathlineto{\pgfqpoint{4.166788in}{2.138016in}}%
\pgfpathlineto{\pgfqpoint{4.032388in}{2.282667in}}%
\pgfpathlineto{\pgfqpoint{3.925049in}{2.395834in}}%
\pgfpathlineto{\pgfqpoint{3.781070in}{2.544000in}}%
\pgfpathlineto{\pgfqpoint{3.645737in}{2.679999in}}%
\pgfpathlineto{\pgfqpoint{3.518115in}{2.805333in}}%
\pgfpathlineto{\pgfqpoint{3.401630in}{2.917333in}}%
\pgfpathlineto{\pgfqpoint{3.084606in}{3.210714in}}%
\pgfpathlineto{\pgfqpoint{2.952629in}{3.328000in}}%
\pgfpathlineto{\pgfqpoint{2.666077in}{3.572827in}}%
\pgfpathlineto{\pgfqpoint{2.601441in}{3.626667in}}%
\pgfpathlineto{\pgfqpoint{2.463157in}{3.738667in}}%
\pgfpathlineto{\pgfqpoint{2.333178in}{3.841253in}}%
\pgfpathlineto{\pgfqpoint{2.223633in}{3.925333in}}%
\pgfpathlineto{\pgfqpoint{1.988333in}{4.098875in}}%
\pgfpathlineto{\pgfqpoint{1.917876in}{4.149333in}}%
\pgfpathlineto{\pgfqpoint{1.810380in}{4.224000in}}%
\pgfpathlineto{\pgfqpoint{1.801721in}{4.224000in}}%
\pgfpathlineto{\pgfqpoint{1.802836in}{4.223240in}}%
\pgfpathlineto{\pgfqpoint{1.922263in}{4.140285in}}%
\pgfpathlineto{\pgfqpoint{2.202828in}{3.935341in}}%
\pgfpathlineto{\pgfqpoint{2.318708in}{3.846603in}}%
\pgfpathlineto{\pgfqpoint{2.371229in}{3.805809in}}%
\pgfpathlineto{\pgfqpoint{2.502573in}{3.701333in}}%
\pgfpathlineto{\pgfqpoint{2.639790in}{3.589333in}}%
\pgfpathlineto{\pgfqpoint{2.729171in}{3.514667in}}%
\pgfpathlineto{\pgfqpoint{3.030611in}{3.253333in}}%
\pgfpathlineto{\pgfqpoint{3.316484in}{2.992000in}}%
\pgfpathlineto{\pgfqpoint{3.588567in}{2.730667in}}%
\pgfpathlineto{\pgfqpoint{3.886222in}{2.430248in}}%
\pgfpathlineto{\pgfqpoint{4.026710in}{2.282667in}}%
\pgfpathlineto{\pgfqpoint{4.300549in}{1.984000in}}%
\pgfpathlineto{\pgfqpoint{4.567596in}{1.677012in}}%
\pgfpathlineto{\pgfqpoint{4.687838in}{1.533110in}}%
\pgfpathlineto{\pgfqpoint{4.768000in}{1.434908in}}%
\pgfpathlineto{\pgfqpoint{4.768000in}{1.434908in}}%
\pgfusepath{fill}%
\end{pgfscope}%
\begin{pgfscope}%
\pgfpathrectangle{\pgfqpoint{0.800000in}{0.528000in}}{\pgfqpoint{3.968000in}{3.696000in}}%
\pgfusepath{clip}%
\pgfsetbuttcap%
\pgfsetroundjoin%
\definecolor{currentfill}{rgb}{0.269308,0.218818,0.509577}%
\pgfsetfillcolor{currentfill}%
\pgfsetlinewidth{0.000000pt}%
\definecolor{currentstroke}{rgb}{0.000000,0.000000,0.000000}%
\pgfsetstrokecolor{currentstroke}%
\pgfsetdash{}{0pt}%
\pgfpathmoveto{\pgfqpoint{2.197847in}{0.528000in}}%
\pgfpathlineto{\pgfqpoint{2.082586in}{0.642644in}}%
\pgfpathlineto{\pgfqpoint{1.938554in}{0.789333in}}%
\pgfpathlineto{\pgfqpoint{1.817507in}{0.915758in}}%
\pgfpathlineto{\pgfqpoint{1.760252in}{0.976000in}}%
\pgfpathlineto{\pgfqpoint{1.481374in}{1.280709in}}%
\pgfpathlineto{\pgfqpoint{1.355208in}{1.424000in}}%
\pgfpathlineto{\pgfqpoint{1.240889in}{1.557066in}}%
\pgfpathlineto{\pgfqpoint{1.010985in}{1.834667in}}%
\pgfpathlineto{\pgfqpoint{0.920242in}{1.948238in}}%
\pgfpathlineto{\pgfqpoint{0.800000in}{2.102921in}}%
\pgfpathlineto{\pgfqpoint{0.800000in}{2.095548in}}%
\pgfpathlineto{\pgfqpoint{0.920242in}{1.941238in}}%
\pgfpathlineto{\pgfqpoint{1.040485in}{1.791369in}}%
\pgfpathlineto{\pgfqpoint{1.285149in}{1.498667in}}%
\pgfpathlineto{\pgfqpoint{1.561535in}{1.185363in}}%
\pgfpathlineto{\pgfqpoint{1.842101in}{0.883652in}}%
\pgfpathlineto{\pgfqpoint{1.969101in}{0.752000in}}%
\pgfpathlineto{\pgfqpoint{2.082586in}{0.636827in}}%
\pgfpathlineto{\pgfqpoint{2.192029in}{0.528000in}}%
\pgfpathmoveto{\pgfqpoint{4.768000in}{1.448614in}}%
\pgfpathlineto{\pgfqpoint{4.539870in}{1.722667in}}%
\pgfpathlineto{\pgfqpoint{4.278287in}{2.021333in}}%
\pgfpathlineto{\pgfqpoint{4.166788in}{2.144084in}}%
\pgfpathlineto{\pgfqpoint{4.038065in}{2.282667in}}%
\pgfpathlineto{\pgfqpoint{3.926303in}{2.400412in}}%
\pgfpathlineto{\pgfqpoint{3.786820in}{2.544000in}}%
\pgfpathlineto{\pgfqpoint{3.661056in}{2.670269in}}%
\pgfpathlineto{\pgfqpoint{3.600381in}{2.730667in}}%
\pgfpathlineto{\pgfqpoint{3.485414in}{2.842800in}}%
\pgfpathlineto{\pgfqpoint{3.328673in}{2.992000in}}%
\pgfpathlineto{\pgfqpoint{3.204848in}{3.107049in}}%
\pgfpathlineto{\pgfqpoint{2.884202in}{3.393401in}}%
\pgfpathlineto{\pgfqpoint{2.742508in}{3.514667in}}%
\pgfpathlineto{\pgfqpoint{2.603636in}{3.630514in}}%
\pgfpathlineto{\pgfqpoint{2.470288in}{3.738667in}}%
\pgfpathlineto{\pgfqpoint{2.328395in}{3.850667in}}%
\pgfpathlineto{\pgfqpoint{2.122667in}{4.006964in}}%
\pgfpathlineto{\pgfqpoint{1.842101in}{4.208117in}}%
\pgfpathlineto{\pgfqpoint{1.819029in}{4.224000in}}%
\pgfpathlineto{\pgfqpoint{1.810380in}{4.224000in}}%
\pgfpathlineto{\pgfqpoint{1.922263in}{4.146256in}}%
\pgfpathlineto{\pgfqpoint{2.202828in}{3.941112in}}%
\pgfpathlineto{\pgfqpoint{2.282990in}{3.880027in}}%
\pgfpathlineto{\pgfqpoint{2.416239in}{3.776000in}}%
\pgfpathlineto{\pgfqpoint{2.691429in}{3.552000in}}%
\pgfpathlineto{\pgfqpoint{2.994978in}{3.290667in}}%
\pgfpathlineto{\pgfqpoint{3.244929in}{3.064420in}}%
\pgfpathlineto{\pgfqpoint{3.383386in}{2.934299in}}%
\pgfpathlineto{\pgfqpoint{3.445333in}{2.875590in}}%
\pgfpathlineto{\pgfqpoint{3.594474in}{2.730667in}}%
\pgfpathlineto{\pgfqpoint{3.886222in}{2.436158in}}%
\pgfpathlineto{\pgfqpoint{4.006465in}{2.310206in}}%
\pgfpathlineto{\pgfqpoint{4.136711in}{2.170667in}}%
\pgfpathlineto{\pgfqpoint{4.246949in}{2.049927in}}%
\pgfpathlineto{\pgfqpoint{4.372453in}{1.909333in}}%
\pgfpathlineto{\pgfqpoint{4.647758in}{1.588115in}}%
\pgfpathlineto{\pgfqpoint{4.768000in}{1.441761in}}%
\pgfpathlineto{\pgfqpoint{4.768000in}{1.441761in}}%
\pgfusepath{fill}%
\end{pgfscope}%
\begin{pgfscope}%
\pgfpathrectangle{\pgfqpoint{0.800000in}{0.528000in}}{\pgfqpoint{3.968000in}{3.696000in}}%
\pgfusepath{clip}%
\pgfsetbuttcap%
\pgfsetroundjoin%
\definecolor{currentfill}{rgb}{0.267968,0.223549,0.512008}%
\pgfsetfillcolor{currentfill}%
\pgfsetlinewidth{0.000000pt}%
\definecolor{currentstroke}{rgb}{0.000000,0.000000,0.000000}%
\pgfsetstrokecolor{currentstroke}%
\pgfsetdash{}{0pt}%
\pgfpathmoveto{\pgfqpoint{2.192029in}{0.528000in}}%
\pgfpathlineto{\pgfqpoint{2.079431in}{0.640000in}}%
\pgfpathlineto{\pgfqpoint{1.962343in}{0.758922in}}%
\pgfpathlineto{\pgfqpoint{1.825273in}{0.901333in}}%
\pgfpathlineto{\pgfqpoint{1.548299in}{1.200000in}}%
\pgfpathlineto{\pgfqpoint{1.441293in}{1.319568in}}%
\pgfpathlineto{\pgfqpoint{1.190042in}{1.610667in}}%
\pgfpathlineto{\pgfqpoint{0.945612in}{1.909333in}}%
\pgfpathlineto{\pgfqpoint{0.840081in}{2.043680in}}%
\pgfpathlineto{\pgfqpoint{0.800000in}{2.095548in}}%
\pgfpathlineto{\pgfqpoint{0.800000in}{2.088363in}}%
\pgfpathlineto{\pgfqpoint{0.920242in}{1.934279in}}%
\pgfpathlineto{\pgfqpoint{1.040485in}{1.784623in}}%
\pgfpathlineto{\pgfqpoint{1.280970in}{1.497065in}}%
\pgfpathlineto{\pgfqpoint{1.561535in}{1.179221in}}%
\pgfpathlineto{\pgfqpoint{1.842101in}{0.877712in}}%
\pgfpathlineto{\pgfqpoint{1.973369in}{0.741730in}}%
\pgfpathlineto{\pgfqpoint{2.110969in}{0.602667in}}%
\pgfpathlineto{\pgfqpoint{2.186212in}{0.528000in}}%
\pgfpathmoveto{\pgfqpoint{4.768000in}{1.455466in}}%
\pgfpathlineto{\pgfqpoint{4.545379in}{1.722667in}}%
\pgfpathlineto{\pgfqpoint{4.267641in}{2.039393in}}%
\pgfpathlineto{\pgfqpoint{4.147890in}{2.170667in}}%
\pgfpathlineto{\pgfqpoint{3.865449in}{2.469333in}}%
\pgfpathlineto{\pgfqpoint{3.725899in}{2.611511in}}%
\pgfpathlineto{\pgfqpoint{3.605657in}{2.731274in}}%
\pgfpathlineto{\pgfqpoint{3.452615in}{2.880000in}}%
\pgfpathlineto{\pgfqpoint{3.325091in}{3.000999in}}%
\pgfpathlineto{\pgfqpoint{3.004444in}{3.293607in}}%
\pgfpathlineto{\pgfqpoint{2.880036in}{3.402667in}}%
\pgfpathlineto{\pgfqpoint{2.603636in}{3.636121in}}%
\pgfpathlineto{\pgfqpoint{2.477420in}{3.738667in}}%
\pgfpathlineto{\pgfqpoint{2.335670in}{3.850667in}}%
\pgfpathlineto{\pgfqpoint{2.122667in}{4.012719in}}%
\pgfpathlineto{\pgfqpoint{1.842101in}{4.214070in}}%
\pgfpathlineto{\pgfqpoint{1.827677in}{4.224000in}}%
\pgfpathlineto{\pgfqpoint{1.819029in}{4.224000in}}%
\pgfpathlineto{\pgfqpoint{1.926246in}{4.149333in}}%
\pgfpathlineto{\pgfqpoint{2.181869in}{3.962667in}}%
\pgfpathlineto{\pgfqpoint{2.296627in}{3.875297in}}%
\pgfpathlineto{\pgfqpoint{2.423418in}{3.776000in}}%
\pgfpathlineto{\pgfqpoint{2.683798in}{3.564019in}}%
\pgfpathlineto{\pgfqpoint{2.830176in}{3.440000in}}%
\pgfpathlineto{\pgfqpoint{2.964364in}{3.323379in}}%
\pgfpathlineto{\pgfqpoint{3.085059in}{3.216000in}}%
\pgfpathlineto{\pgfqpoint{3.407641in}{2.917333in}}%
\pgfpathlineto{\pgfqpoint{3.725899in}{2.605674in}}%
\pgfpathlineto{\pgfqpoint{4.006465in}{2.316237in}}%
\pgfpathlineto{\pgfqpoint{4.142301in}{2.170667in}}%
\pgfpathlineto{\pgfqpoint{4.246949in}{2.056157in}}%
\pgfpathlineto{\pgfqpoint{4.377974in}{1.909333in}}%
\pgfpathlineto{\pgfqpoint{4.647758in}{1.594761in}}%
\pgfpathlineto{\pgfqpoint{4.768000in}{1.448614in}}%
\pgfpathlineto{\pgfqpoint{4.768000in}{1.448614in}}%
\pgfusepath{fill}%
\end{pgfscope}%
\begin{pgfscope}%
\pgfpathrectangle{\pgfqpoint{0.800000in}{0.528000in}}{\pgfqpoint{3.968000in}{3.696000in}}%
\pgfusepath{clip}%
\pgfsetbuttcap%
\pgfsetroundjoin%
\definecolor{currentfill}{rgb}{0.267968,0.223549,0.512008}%
\pgfsetfillcolor{currentfill}%
\pgfsetlinewidth{0.000000pt}%
\definecolor{currentstroke}{rgb}{0.000000,0.000000,0.000000}%
\pgfsetstrokecolor{currentstroke}%
\pgfsetdash{}{0pt}%
\pgfpathmoveto{\pgfqpoint{2.186212in}{0.528000in}}%
\pgfpathlineto{\pgfqpoint{2.073706in}{0.640000in}}%
\pgfpathlineto{\pgfqpoint{1.962343in}{0.753009in}}%
\pgfpathlineto{\pgfqpoint{1.819620in}{0.901333in}}%
\pgfpathlineto{\pgfqpoint{1.542745in}{1.200000in}}%
\pgfpathlineto{\pgfqpoint{1.441293in}{1.313250in}}%
\pgfpathlineto{\pgfqpoint{1.174221in}{1.623235in}}%
\pgfpathlineto{\pgfqpoint{1.120646in}{1.687041in}}%
\pgfpathlineto{\pgfqpoint{0.999818in}{1.834667in}}%
\pgfpathlineto{\pgfqpoint{0.800000in}{2.088363in}}%
\pgfpathlineto{\pgfqpoint{0.800000in}{2.081178in}}%
\pgfpathlineto{\pgfqpoint{0.911521in}{1.938543in}}%
\pgfpathlineto{\pgfqpoint{0.964303in}{1.872000in}}%
\pgfpathlineto{\pgfqpoint{1.210515in}{1.573333in}}%
\pgfpathlineto{\pgfqpoint{1.481374in}{1.262086in}}%
\pgfpathlineto{\pgfqpoint{1.624748in}{1.103787in}}%
\pgfpathlineto{\pgfqpoint{1.743468in}{0.976000in}}%
\pgfpathlineto{\pgfqpoint{1.865099in}{0.848089in}}%
\pgfpathlineto{\pgfqpoint{1.922263in}{0.788389in}}%
\pgfpathlineto{\pgfqpoint{2.067980in}{0.640000in}}%
\pgfpathlineto{\pgfqpoint{2.180394in}{0.528000in}}%
\pgfpathmoveto{\pgfqpoint{4.768000in}{1.462294in}}%
\pgfpathlineto{\pgfqpoint{4.518970in}{1.760000in}}%
\pgfpathlineto{\pgfqpoint{4.246949in}{2.068388in}}%
\pgfpathlineto{\pgfqpoint{4.118997in}{2.208000in}}%
\pgfpathlineto{\pgfqpoint{4.006465in}{2.328115in}}%
\pgfpathlineto{\pgfqpoint{3.724585in}{2.618667in}}%
\pgfpathlineto{\pgfqpoint{3.405253in}{2.930904in}}%
\pgfpathlineto{\pgfqpoint{3.084606in}{3.227575in}}%
\pgfpathlineto{\pgfqpoint{2.929339in}{3.365333in}}%
\pgfpathlineto{\pgfqpoint{2.804040in}{3.473698in}}%
\pgfpathlineto{\pgfqpoint{2.483394in}{3.739573in}}%
\pgfpathlineto{\pgfqpoint{2.363152in}{3.834889in}}%
\pgfpathlineto{\pgfqpoint{2.242909in}{3.927974in}}%
\pgfpathlineto{\pgfqpoint{1.962343in}{4.135343in}}%
\pgfpathlineto{\pgfqpoint{1.882182in}{4.192098in}}%
\pgfpathlineto{\pgfqpoint{1.836325in}{4.224000in}}%
\pgfpathlineto{\pgfqpoint{1.827677in}{4.224000in}}%
\pgfpathlineto{\pgfqpoint{1.934464in}{4.149333in}}%
\pgfpathlineto{\pgfqpoint{2.189533in}{3.962667in}}%
\pgfpathlineto{\pgfqpoint{2.287549in}{3.888000in}}%
\pgfpathlineto{\pgfqpoint{2.430597in}{3.776000in}}%
\pgfpathlineto{\pgfqpoint{2.683798in}{3.569642in}}%
\pgfpathlineto{\pgfqpoint{2.836764in}{3.440000in}}%
\pgfpathlineto{\pgfqpoint{2.965537in}{3.328000in}}%
\pgfpathlineto{\pgfqpoint{3.091250in}{3.216000in}}%
\pgfpathlineto{\pgfqpoint{3.405253in}{2.925258in}}%
\pgfpathlineto{\pgfqpoint{3.529947in}{2.805333in}}%
\pgfpathlineto{\pgfqpoint{3.837137in}{2.498280in}}%
\pgfpathlineto{\pgfqpoint{3.901552in}{2.432000in}}%
\pgfpathlineto{\pgfqpoint{4.023798in}{2.303855in}}%
\pgfpathlineto{\pgfqpoint{4.147890in}{2.170667in}}%
\pgfpathlineto{\pgfqpoint{4.267641in}{2.039393in}}%
\pgfpathlineto{\pgfqpoint{4.383495in}{1.909333in}}%
\pgfpathlineto{\pgfqpoint{4.647758in}{1.601408in}}%
\pgfpathlineto{\pgfqpoint{4.768000in}{1.455466in}}%
\pgfpathlineto{\pgfqpoint{4.768000in}{1.461333in}}%
\pgfpathlineto{\pgfqpoint{4.768000in}{1.461333in}}%
\pgfusepath{fill}%
\end{pgfscope}%
\begin{pgfscope}%
\pgfpathrectangle{\pgfqpoint{0.800000in}{0.528000in}}{\pgfqpoint{3.968000in}{3.696000in}}%
\pgfusepath{clip}%
\pgfsetbuttcap%
\pgfsetroundjoin%
\definecolor{currentfill}{rgb}{0.267968,0.223549,0.512008}%
\pgfsetfillcolor{currentfill}%
\pgfsetlinewidth{0.000000pt}%
\definecolor{currentstroke}{rgb}{0.000000,0.000000,0.000000}%
\pgfsetstrokecolor{currentstroke}%
\pgfsetdash{}{0pt}%
\pgfpathmoveto{\pgfqpoint{2.180394in}{0.528000in}}%
\pgfpathlineto{\pgfqpoint{2.042505in}{0.665663in}}%
\pgfpathlineto{\pgfqpoint{1.921348in}{0.789333in}}%
\pgfpathlineto{\pgfqpoint{1.624748in}{1.103787in}}%
\pgfpathlineto{\pgfqpoint{1.503575in}{1.237333in}}%
\pgfpathlineto{\pgfqpoint{1.401212in}{1.352397in}}%
\pgfpathlineto{\pgfqpoint{1.274137in}{1.498667in}}%
\pgfpathlineto{\pgfqpoint{1.024653in}{1.797333in}}%
\pgfpathlineto{\pgfqpoint{0.920242in}{1.927320in}}%
\pgfpathlineto{\pgfqpoint{0.800000in}{2.081178in}}%
\pgfpathlineto{\pgfqpoint{0.800000in}{2.073992in}}%
\pgfpathlineto{\pgfqpoint{0.899432in}{1.946667in}}%
\pgfpathlineto{\pgfqpoint{1.160727in}{1.625909in}}%
\pgfpathlineto{\pgfqpoint{1.401212in}{1.346146in}}%
\pgfpathlineto{\pgfqpoint{1.681778in}{1.036077in}}%
\pgfpathlineto{\pgfqpoint{1.808315in}{0.901333in}}%
\pgfpathlineto{\pgfqpoint{1.922263in}{0.782596in}}%
\pgfpathlineto{\pgfqpoint{2.062255in}{0.640000in}}%
\pgfpathlineto{\pgfqpoint{2.174577in}{0.528000in}}%
\pgfpathmoveto{\pgfqpoint{4.768000in}{1.468974in}}%
\pgfpathlineto{\pgfqpoint{4.524506in}{1.760000in}}%
\pgfpathlineto{\pgfqpoint{4.246949in}{2.074474in}}%
\pgfpathlineto{\pgfqpoint{4.111208in}{2.222436in}}%
\pgfpathlineto{\pgfqpoint{3.984310in}{2.357333in}}%
\pgfpathlineto{\pgfqpoint{3.846141in}{2.500950in}}%
\pgfpathlineto{\pgfqpoint{3.565576in}{2.781961in}}%
\pgfpathlineto{\pgfqpoint{3.266769in}{3.066667in}}%
\pgfpathlineto{\pgfqpoint{3.124687in}{3.196985in}}%
\pgfpathlineto{\pgfqpoint{2.804040in}{3.479303in}}%
\pgfpathlineto{\pgfqpoint{2.673659in}{3.589333in}}%
\pgfpathlineto{\pgfqpoint{2.537577in}{3.701333in}}%
\pgfpathlineto{\pgfqpoint{2.443313in}{3.777274in}}%
\pgfpathlineto{\pgfqpoint{2.155160in}{4.000000in}}%
\pgfpathlineto{\pgfqpoint{2.024996in}{4.095691in}}%
\pgfpathlineto{\pgfqpoint{1.957329in}{4.144663in}}%
\pgfpathlineto{\pgfqpoint{1.882182in}{4.197929in}}%
\pgfpathlineto{\pgfqpoint{1.844872in}{4.224000in}}%
\pgfpathlineto{\pgfqpoint{1.836325in}{4.224000in}}%
\pgfpathlineto{\pgfqpoint{1.922263in}{4.163843in}}%
\pgfpathlineto{\pgfqpoint{2.021498in}{4.092433in}}%
\pgfpathlineto{\pgfqpoint{2.088976in}{4.043285in}}%
\pgfpathlineto{\pgfqpoint{2.162747in}{3.988572in}}%
\pgfpathlineto{\pgfqpoint{2.443313in}{3.771606in}}%
\pgfpathlineto{\pgfqpoint{2.576484in}{3.664000in}}%
\pgfpathlineto{\pgfqpoint{2.711558in}{3.552000in}}%
\pgfpathlineto{\pgfqpoint{2.971835in}{3.328000in}}%
\pgfpathlineto{\pgfqpoint{3.111138in}{3.203380in}}%
\pgfpathlineto{\pgfqpoint{3.179661in}{3.141333in}}%
\pgfpathlineto{\pgfqpoint{3.485414in}{2.854125in}}%
\pgfpathlineto{\pgfqpoint{3.765980in}{2.576875in}}%
\pgfpathlineto{\pgfqpoint{3.907213in}{2.432000in}}%
\pgfpathlineto{\pgfqpoint{4.014135in}{2.320000in}}%
\pgfpathlineto{\pgfqpoint{4.126707in}{2.199686in}}%
\pgfpathlineto{\pgfqpoint{4.407273in}{1.888602in}}%
\pgfpathlineto{\pgfqpoint{4.647758in}{1.608054in}}%
\pgfpathlineto{\pgfqpoint{4.768000in}{1.462294in}}%
\pgfpathlineto{\pgfqpoint{4.768000in}{1.462294in}}%
\pgfusepath{fill}%
\end{pgfscope}%
\begin{pgfscope}%
\pgfpathrectangle{\pgfqpoint{0.800000in}{0.528000in}}{\pgfqpoint{3.968000in}{3.696000in}}%
\pgfusepath{clip}%
\pgfsetbuttcap%
\pgfsetroundjoin%
\definecolor{currentfill}{rgb}{0.267968,0.223549,0.512008}%
\pgfsetfillcolor{currentfill}%
\pgfsetlinewidth{0.000000pt}%
\definecolor{currentstroke}{rgb}{0.000000,0.000000,0.000000}%
\pgfsetstrokecolor{currentstroke}%
\pgfsetdash{}{0pt}%
\pgfpathmoveto{\pgfqpoint{2.174577in}{0.528000in}}%
\pgfpathlineto{\pgfqpoint{2.042505in}{0.659896in}}%
\pgfpathlineto{\pgfqpoint{1.915741in}{0.789333in}}%
\pgfpathlineto{\pgfqpoint{1.633751in}{1.088000in}}%
\pgfpathlineto{\pgfqpoint{1.521455in}{1.211278in}}%
\pgfpathlineto{\pgfqpoint{1.256497in}{1.513205in}}%
\pgfpathlineto{\pgfqpoint{1.200808in}{1.578270in}}%
\pgfpathlineto{\pgfqpoint{1.078909in}{1.724210in}}%
\pgfpathlineto{\pgfqpoint{0.958778in}{1.872000in}}%
\pgfpathlineto{\pgfqpoint{0.800000in}{2.073992in}}%
\pgfpathlineto{\pgfqpoint{0.800000in}{2.066807in}}%
\pgfpathlineto{\pgfqpoint{0.893926in}{1.946667in}}%
\pgfpathlineto{\pgfqpoint{1.136782in}{1.648000in}}%
\pgfpathlineto{\pgfqpoint{1.240889in}{1.524720in}}%
\pgfpathlineto{\pgfqpoint{1.505858in}{1.222805in}}%
\pgfpathlineto{\pgfqpoint{1.561535in}{1.160838in}}%
\pgfpathlineto{\pgfqpoint{1.697397in}{1.013333in}}%
\pgfpathlineto{\pgfqpoint{1.807669in}{0.896072in}}%
\pgfpathlineto{\pgfqpoint{1.946396in}{0.752000in}}%
\pgfpathlineto{\pgfqpoint{2.168760in}{0.528000in}}%
\pgfpathmoveto{\pgfqpoint{4.768000in}{1.475653in}}%
\pgfpathlineto{\pgfqpoint{4.527515in}{1.762876in}}%
\pgfpathlineto{\pgfqpoint{4.246949in}{2.080561in}}%
\pgfpathlineto{\pgfqpoint{4.126707in}{2.211719in}}%
\pgfpathlineto{\pgfqpoint{3.989912in}{2.357333in}}%
\pgfpathlineto{\pgfqpoint{3.847617in}{2.505292in}}%
\pgfpathlineto{\pgfqpoint{3.735963in}{2.618667in}}%
\pgfpathlineto{\pgfqpoint{3.445333in}{2.903955in}}%
\pgfpathlineto{\pgfqpoint{3.151011in}{3.178667in}}%
\pgfpathlineto{\pgfqpoint{3.004444in}{3.310310in}}%
\pgfpathlineto{\pgfqpoint{2.856192in}{3.440000in}}%
\pgfpathlineto{\pgfqpoint{2.723879in}{3.552903in}}%
\pgfpathlineto{\pgfqpoint{2.590157in}{3.664000in}}%
\pgfpathlineto{\pgfqpoint{2.443313in}{3.782851in}}%
\pgfpathlineto{\pgfqpoint{2.162224in}{4.000487in}}%
\pgfpathlineto{\pgfqpoint{2.010922in}{4.112000in}}%
\pgfpathlineto{\pgfqpoint{1.906457in}{4.186667in}}%
\pgfpathlineto{\pgfqpoint{1.853217in}{4.224000in}}%
\pgfpathlineto{\pgfqpoint{1.844872in}{4.224000in}}%
\pgfpathlineto{\pgfqpoint{1.962343in}{4.141191in}}%
\pgfpathlineto{\pgfqpoint{2.082586in}{4.053876in}}%
\pgfpathlineto{\pgfqpoint{2.181350in}{3.979994in}}%
\pgfpathlineto{\pgfqpoint{2.253725in}{3.925333in}}%
\pgfpathlineto{\pgfqpoint{2.523475in}{3.712795in}}%
\pgfpathlineto{\pgfqpoint{2.673659in}{3.589333in}}%
\pgfpathlineto{\pgfqpoint{2.935674in}{3.365333in}}%
\pgfpathlineto{\pgfqpoint{3.235627in}{3.095335in}}%
\pgfpathlineto{\pgfqpoint{3.285010in}{3.049723in}}%
\pgfpathlineto{\pgfqpoint{3.425404in}{2.917333in}}%
\pgfpathlineto{\pgfqpoint{3.728158in}{2.620771in}}%
\pgfpathlineto{\pgfqpoint{3.782286in}{2.566145in}}%
\pgfpathlineto{\pgfqpoint{3.912874in}{2.432000in}}%
\pgfpathlineto{\pgfqpoint{4.046545in}{2.291563in}}%
\pgfpathlineto{\pgfqpoint{4.294920in}{2.021333in}}%
\pgfpathlineto{\pgfqpoint{4.407273in}{1.894871in}}%
\pgfpathlineto{\pgfqpoint{4.662609in}{1.596833in}}%
\pgfpathlineto{\pgfqpoint{4.768000in}{1.468974in}}%
\pgfpathlineto{\pgfqpoint{4.768000in}{1.468974in}}%
\pgfusepath{fill}%
\end{pgfscope}%
\begin{pgfscope}%
\pgfpathrectangle{\pgfqpoint{0.800000in}{0.528000in}}{\pgfqpoint{3.968000in}{3.696000in}}%
\pgfusepath{clip}%
\pgfsetbuttcap%
\pgfsetroundjoin%
\definecolor{currentfill}{rgb}{0.266580,0.228262,0.514349}%
\pgfsetfillcolor{currentfill}%
\pgfsetlinewidth{0.000000pt}%
\definecolor{currentstroke}{rgb}{0.000000,0.000000,0.000000}%
\pgfsetstrokecolor{currentstroke}%
\pgfsetdash{}{0pt}%
\pgfpathmoveto{\pgfqpoint{2.168760in}{0.528000in}}%
\pgfpathlineto{\pgfqpoint{2.042505in}{0.654128in}}%
\pgfpathlineto{\pgfqpoint{1.910134in}{0.789333in}}%
\pgfpathlineto{\pgfqpoint{1.628242in}{1.088000in}}%
\pgfpathlineto{\pgfqpoint{1.521455in}{1.205127in}}%
\pgfpathlineto{\pgfqpoint{1.240889in}{1.524720in}}%
\pgfpathlineto{\pgfqpoint{1.120646in}{1.667328in}}%
\pgfpathlineto{\pgfqpoint{0.893926in}{1.946667in}}%
\pgfpathlineto{\pgfqpoint{0.800000in}{2.066807in}}%
\pgfpathlineto{\pgfqpoint{0.800738in}{2.058667in}}%
\pgfpathlineto{\pgfqpoint{1.040485in}{1.757697in}}%
\pgfpathlineto{\pgfqpoint{1.169097in}{1.602871in}}%
\pgfpathlineto{\pgfqpoint{1.289858in}{1.461333in}}%
\pgfpathlineto{\pgfqpoint{1.401212in}{1.333786in}}%
\pgfpathlineto{\pgfqpoint{1.681778in}{1.024126in}}%
\pgfpathlineto{\pgfqpoint{1.802020in}{0.896175in}}%
\pgfpathlineto{\pgfqpoint{1.940760in}{0.752000in}}%
\pgfpathlineto{\pgfqpoint{2.162942in}{0.528000in}}%
\pgfpathmoveto{\pgfqpoint{4.768000in}{1.482333in}}%
\pgfpathlineto{\pgfqpoint{4.527515in}{1.769174in}}%
\pgfpathlineto{\pgfqpoint{4.260551in}{2.071336in}}%
\pgfpathlineto{\pgfqpoint{4.204446in}{2.133333in}}%
\pgfpathlineto{\pgfqpoint{3.905622in}{2.451264in}}%
\pgfpathlineto{\pgfqpoint{3.778606in}{2.581333in}}%
\pgfpathlineto{\pgfqpoint{3.645737in}{2.714453in}}%
\pgfpathlineto{\pgfqpoint{3.358603in}{2.992000in}}%
\pgfpathlineto{\pgfqpoint{3.221461in}{3.119474in}}%
\pgfpathlineto{\pgfqpoint{3.157166in}{3.178667in}}%
\pgfpathlineto{\pgfqpoint{3.004444in}{3.315878in}}%
\pgfpathlineto{\pgfqpoint{2.862602in}{3.440000in}}%
\pgfpathlineto{\pgfqpoint{2.723879in}{3.558417in}}%
\pgfpathlineto{\pgfqpoint{2.596993in}{3.664000in}}%
\pgfpathlineto{\pgfqpoint{2.458842in}{3.776000in}}%
\pgfpathlineto{\pgfqpoint{2.363152in}{3.851903in}}%
\pgfpathlineto{\pgfqpoint{2.069918in}{4.074667in}}%
\pgfpathlineto{\pgfqpoint{1.962343in}{4.152810in}}%
\pgfpathlineto{\pgfqpoint{1.861561in}{4.224000in}}%
\pgfpathlineto{\pgfqpoint{1.853217in}{4.224000in}}%
\pgfpathlineto{\pgfqpoint{1.973041in}{4.139369in}}%
\pgfpathlineto{\pgfqpoint{2.112737in}{4.037333in}}%
\pgfpathlineto{\pgfqpoint{2.323071in}{3.877506in}}%
\pgfpathlineto{\pgfqpoint{2.618438in}{3.640454in}}%
\pgfpathlineto{\pgfqpoint{2.683798in}{3.586512in}}%
\pgfpathlineto{\pgfqpoint{2.812786in}{3.477333in}}%
\pgfpathlineto{\pgfqpoint{3.109822in}{3.216000in}}%
\pgfpathlineto{\pgfqpoint{3.392104in}{2.954667in}}%
\pgfpathlineto{\pgfqpoint{3.685818in}{2.668914in}}%
\pgfpathlineto{\pgfqpoint{3.809733in}{2.544000in}}%
\pgfpathlineto{\pgfqpoint{4.095399in}{2.245333in}}%
\pgfpathlineto{\pgfqpoint{4.206869in}{2.124609in}}%
\pgfpathlineto{\pgfqpoint{4.487434in}{1.809282in}}%
\pgfpathlineto{\pgfqpoint{4.727919in}{1.524564in}}%
\pgfpathlineto{\pgfqpoint{4.768000in}{1.475653in}}%
\pgfpathlineto{\pgfqpoint{4.768000in}{1.475653in}}%
\pgfusepath{fill}%
\end{pgfscope}%
\begin{pgfscope}%
\pgfpathrectangle{\pgfqpoint{0.800000in}{0.528000in}}{\pgfqpoint{3.968000in}{3.696000in}}%
\pgfusepath{clip}%
\pgfsetbuttcap%
\pgfsetroundjoin%
\definecolor{currentfill}{rgb}{0.266580,0.228262,0.514349}%
\pgfsetfillcolor{currentfill}%
\pgfsetlinewidth{0.000000pt}%
\definecolor{currentstroke}{rgb}{0.000000,0.000000,0.000000}%
\pgfsetstrokecolor{currentstroke}%
\pgfsetdash{}{0pt}%
\pgfpathmoveto{\pgfqpoint{2.162942in}{0.528000in}}%
\pgfpathlineto{\pgfqpoint{2.042505in}{0.648360in}}%
\pgfpathlineto{\pgfqpoint{1.904527in}{0.789333in}}%
\pgfpathlineto{\pgfqpoint{1.622732in}{1.088000in}}%
\pgfpathlineto{\pgfqpoint{1.520551in}{1.200000in}}%
\pgfpathlineto{\pgfqpoint{1.240889in}{1.518353in}}%
\pgfpathlineto{\pgfqpoint{1.120646in}{1.660771in}}%
\pgfpathlineto{\pgfqpoint{0.880162in}{1.957125in}}%
\pgfpathlineto{\pgfqpoint{0.800000in}{2.059622in}}%
\pgfpathlineto{\pgfqpoint{0.800000in}{2.052601in}}%
\pgfpathlineto{\pgfqpoint{1.040485in}{1.751119in}}%
\pgfpathlineto{\pgfqpoint{1.160727in}{1.606377in}}%
\pgfpathlineto{\pgfqpoint{1.298417in}{1.445082in}}%
\pgfpathlineto{\pgfqpoint{1.415026in}{1.312000in}}%
\pgfpathlineto{\pgfqpoint{1.686265in}{1.013333in}}%
\pgfpathlineto{\pgfqpoint{1.802020in}{0.890357in}}%
\pgfpathlineto{\pgfqpoint{1.935124in}{0.752000in}}%
\pgfpathlineto{\pgfqpoint{2.157259in}{0.528000in}}%
\pgfpathlineto{\pgfqpoint{2.162747in}{0.528000in}}%
\pgfpathmoveto{\pgfqpoint{4.768000in}{1.489013in}}%
\pgfpathlineto{\pgfqpoint{4.540807in}{1.760000in}}%
\pgfpathlineto{\pgfqpoint{4.443790in}{1.872000in}}%
\pgfpathlineto{\pgfqpoint{4.166788in}{2.180272in}}%
\pgfpathlineto{\pgfqpoint{3.886222in}{2.477083in}}%
\pgfpathlineto{\pgfqpoint{3.747310in}{2.618667in}}%
\pgfpathlineto{\pgfqpoint{3.443166in}{2.917333in}}%
\pgfpathlineto{\pgfqpoint{3.324896in}{3.029333in}}%
\pgfpathlineto{\pgfqpoint{3.204145in}{3.141333in}}%
\pgfpathlineto{\pgfqpoint{3.080786in}{3.253333in}}%
\pgfpathlineto{\pgfqpoint{2.772493in}{3.522615in}}%
\pgfpathlineto{\pgfqpoint{2.723879in}{3.563931in}}%
\pgfpathlineto{\pgfqpoint{2.602228in}{3.665312in}}%
\pgfpathlineto{\pgfqpoint{2.465810in}{3.776000in}}%
\pgfpathlineto{\pgfqpoint{2.323071in}{3.888832in}}%
\pgfpathlineto{\pgfqpoint{2.202828in}{3.981107in}}%
\pgfpathlineto{\pgfqpoint{2.102580in}{4.055957in}}%
\pgfpathlineto{\pgfqpoint{2.035491in}{4.105467in}}%
\pgfpathlineto{\pgfqpoint{1.962343in}{4.158531in}}%
\pgfpathlineto{\pgfqpoint{1.869906in}{4.224000in}}%
\pgfpathlineto{\pgfqpoint{1.861561in}{4.224000in}}%
\pgfpathlineto{\pgfqpoint{1.967170in}{4.149333in}}%
\pgfpathlineto{\pgfqpoint{2.099179in}{4.052789in}}%
\pgfpathlineto{\pgfqpoint{2.170349in}{4.000000in}}%
\pgfpathlineto{\pgfqpoint{2.443313in}{3.788427in}}%
\pgfpathlineto{\pgfqpoint{2.563556in}{3.691388in}}%
\pgfpathlineto{\pgfqpoint{2.687069in}{3.589333in}}%
\pgfpathlineto{\pgfqpoint{2.819234in}{3.477333in}}%
\pgfpathlineto{\pgfqpoint{3.116013in}{3.216000in}}%
\pgfpathlineto{\pgfqpoint{3.398057in}{2.954667in}}%
\pgfpathlineto{\pgfqpoint{3.676005in}{2.684193in}}%
\pgfpathlineto{\pgfqpoint{3.741637in}{2.618667in}}%
\pgfpathlineto{\pgfqpoint{4.030855in}{2.320000in}}%
\pgfpathlineto{\pgfqpoint{4.149896in}{2.192266in}}%
\pgfpathlineto{\pgfqpoint{4.206869in}{2.130686in}}%
\pgfpathlineto{\pgfqpoint{4.487434in}{1.815571in}}%
\pgfpathlineto{\pgfqpoint{4.727919in}{1.531232in}}%
\pgfpathlineto{\pgfqpoint{4.768000in}{1.482333in}}%
\pgfpathlineto{\pgfqpoint{4.768000in}{1.482333in}}%
\pgfusepath{fill}%
\end{pgfscope}%
\begin{pgfscope}%
\pgfpathrectangle{\pgfqpoint{0.800000in}{0.528000in}}{\pgfqpoint{3.968000in}{3.696000in}}%
\pgfusepath{clip}%
\pgfsetbuttcap%
\pgfsetroundjoin%
\definecolor{currentfill}{rgb}{0.266580,0.228262,0.514349}%
\pgfsetfillcolor{currentfill}%
\pgfsetlinewidth{0.000000pt}%
\definecolor{currentstroke}{rgb}{0.000000,0.000000,0.000000}%
\pgfsetstrokecolor{currentstroke}%
\pgfsetdash{}{0pt}%
\pgfpathmoveto{\pgfqpoint{2.157259in}{0.528000in}}%
\pgfpathlineto{\pgfqpoint{1.853206in}{0.837010in}}%
\pgfpathlineto{\pgfqpoint{1.791603in}{0.901333in}}%
\pgfpathlineto{\pgfqpoint{1.681778in}{1.018151in}}%
\pgfpathlineto{\pgfqpoint{1.401212in}{1.327607in}}%
\pgfpathlineto{\pgfqpoint{1.280970in}{1.465277in}}%
\pgfpathlineto{\pgfqpoint{1.033205in}{1.760000in}}%
\pgfpathlineto{\pgfqpoint{0.920242in}{1.899738in}}%
\pgfpathlineto{\pgfqpoint{0.824361in}{2.021333in}}%
\pgfpathlineto{\pgfqpoint{0.800000in}{2.052601in}}%
\pgfpathlineto{\pgfqpoint{0.800000in}{2.045606in}}%
\pgfpathlineto{\pgfqpoint{1.027813in}{1.760000in}}%
\pgfpathlineto{\pgfqpoint{1.280970in}{1.458976in}}%
\pgfpathlineto{\pgfqpoint{1.409556in}{1.312000in}}%
\pgfpathlineto{\pgfqpoint{1.681778in}{1.012201in}}%
\pgfpathlineto{\pgfqpoint{1.821613in}{0.864000in}}%
\pgfpathlineto{\pgfqpoint{1.962343in}{0.718281in}}%
\pgfpathlineto{\pgfqpoint{2.082586in}{0.596643in}}%
\pgfpathlineto{\pgfqpoint{2.151581in}{0.528000in}}%
\pgfpathmoveto{\pgfqpoint{4.768000in}{1.495692in}}%
\pgfpathlineto{\pgfqpoint{4.546218in}{1.760000in}}%
\pgfpathlineto{\pgfqpoint{4.447354in}{1.874155in}}%
\pgfpathlineto{\pgfqpoint{4.166788in}{2.186204in}}%
\pgfpathlineto{\pgfqpoint{3.886222in}{2.482827in}}%
\pgfpathlineto{\pgfqpoint{3.752983in}{2.618667in}}%
\pgfpathlineto{\pgfqpoint{3.445333in}{2.920843in}}%
\pgfpathlineto{\pgfqpoint{3.290832in}{3.066667in}}%
\pgfpathlineto{\pgfqpoint{3.044525in}{3.291364in}}%
\pgfpathlineto{\pgfqpoint{2.733600in}{3.561055in}}%
\pgfpathlineto{\pgfqpoint{2.683798in}{3.603089in}}%
\pgfpathlineto{\pgfqpoint{2.553004in}{3.711161in}}%
\pgfpathlineto{\pgfqpoint{2.426015in}{3.813333in}}%
\pgfpathlineto{\pgfqpoint{2.282990in}{3.925506in}}%
\pgfpathlineto{\pgfqpoint{2.002424in}{4.135364in}}%
\pgfpathlineto{\pgfqpoint{1.878250in}{4.224000in}}%
\pgfpathlineto{\pgfqpoint{1.869906in}{4.224000in}}%
\pgfpathlineto{\pgfqpoint{1.975114in}{4.149333in}}%
\pgfpathlineto{\pgfqpoint{2.082586in}{4.071115in}}%
\pgfpathlineto{\pgfqpoint{2.371779in}{3.850667in}}%
\pgfpathlineto{\pgfqpoint{2.648911in}{3.626667in}}%
\pgfpathlineto{\pgfqpoint{2.782013in}{3.514667in}}%
\pgfpathlineto{\pgfqpoint{2.924283in}{3.391973in}}%
\pgfpathlineto{\pgfqpoint{3.062122in}{3.269724in}}%
\pgfpathlineto{\pgfqpoint{3.124687in}{3.213757in}}%
\pgfpathlineto{\pgfqpoint{3.244929in}{3.103769in}}%
\pgfpathlineto{\pgfqpoint{3.365172in}{2.991451in}}%
\pgfpathlineto{\pgfqpoint{3.520694in}{2.842667in}}%
\pgfpathlineto{\pgfqpoint{3.645737in}{2.720148in}}%
\pgfpathlineto{\pgfqpoint{3.784250in}{2.581333in}}%
\pgfpathlineto{\pgfqpoint{3.926303in}{2.435610in}}%
\pgfpathlineto{\pgfqpoint{4.046545in}{2.309280in}}%
\pgfpathlineto{\pgfqpoint{4.327111in}{2.003766in}}%
\pgfpathlineto{\pgfqpoint{4.572801in}{1.722667in}}%
\pgfpathlineto{\pgfqpoint{4.768000in}{1.489013in}}%
\pgfpathlineto{\pgfqpoint{4.768000in}{1.489013in}}%
\pgfusepath{fill}%
\end{pgfscope}%
\begin{pgfscope}%
\pgfpathrectangle{\pgfqpoint{0.800000in}{0.528000in}}{\pgfqpoint{3.968000in}{3.696000in}}%
\pgfusepath{clip}%
\pgfsetbuttcap%
\pgfsetroundjoin%
\definecolor{currentfill}{rgb}{0.266580,0.228262,0.514349}%
\pgfsetfillcolor{currentfill}%
\pgfsetlinewidth{0.000000pt}%
\definecolor{currentstroke}{rgb}{0.000000,0.000000,0.000000}%
\pgfsetstrokecolor{currentstroke}%
\pgfsetdash{}{0pt}%
\pgfpathmoveto{\pgfqpoint{2.151581in}{0.528000in}}%
\pgfpathlineto{\pgfqpoint{1.842101in}{0.842553in}}%
\pgfpathlineto{\pgfqpoint{1.715640in}{0.976000in}}%
\pgfpathlineto{\pgfqpoint{1.601616in}{1.098983in}}%
\pgfpathlineto{\pgfqpoint{1.321051in}{1.412813in}}%
\pgfpathlineto{\pgfqpoint{1.200808in}{1.552640in}}%
\pgfpathlineto{\pgfqpoint{1.080566in}{1.695929in}}%
\pgfpathlineto{\pgfqpoint{0.840081in}{1.994209in}}%
\pgfpathlineto{\pgfqpoint{0.800000in}{2.045606in}}%
\pgfpathlineto{\pgfqpoint{0.800000in}{2.038611in}}%
\pgfpathlineto{\pgfqpoint{1.022421in}{1.760000in}}%
\pgfpathlineto{\pgfqpoint{1.280970in}{1.452768in}}%
\pgfpathlineto{\pgfqpoint{1.404086in}{1.312000in}}%
\pgfpathlineto{\pgfqpoint{1.681778in}{1.006357in}}%
\pgfpathlineto{\pgfqpoint{1.816063in}{0.864000in}}%
\pgfpathlineto{\pgfqpoint{1.942359in}{0.733386in}}%
\pgfpathlineto{\pgfqpoint{2.002424in}{0.671760in}}%
\pgfpathlineto{\pgfqpoint{2.145902in}{0.528000in}}%
\pgfpathmoveto{\pgfqpoint{4.768000in}{1.502281in}}%
\pgfpathlineto{\pgfqpoint{4.519518in}{1.797333in}}%
\pgfpathlineto{\pgfqpoint{4.246949in}{2.104707in}}%
\pgfpathlineto{\pgfqpoint{3.966384in}{2.405378in}}%
\pgfpathlineto{\pgfqpoint{3.659818in}{2.717551in}}%
\pgfpathlineto{\pgfqpoint{3.532241in}{2.842667in}}%
\pgfpathlineto{\pgfqpoint{3.256555in}{3.104000in}}%
\pgfpathlineto{\pgfqpoint{3.124687in}{3.224754in}}%
\pgfpathlineto{\pgfqpoint{2.804040in}{3.506949in}}%
\pgfpathlineto{\pgfqpoint{2.662121in}{3.626667in}}%
\pgfpathlineto{\pgfqpoint{2.523475in}{3.740711in}}%
\pgfpathlineto{\pgfqpoint{2.385897in}{3.850667in}}%
\pgfpathlineto{\pgfqpoint{2.242909in}{3.961910in}}%
\pgfpathlineto{\pgfqpoint{2.131696in}{4.045743in}}%
\pgfpathlineto{\pgfqpoint{2.082586in}{4.082439in}}%
\pgfpathlineto{\pgfqpoint{1.962343in}{4.169975in}}%
\pgfpathlineto{\pgfqpoint{1.886445in}{4.224000in}}%
\pgfpathlineto{\pgfqpoint{1.878250in}{4.224000in}}%
\pgfpathlineto{\pgfqpoint{1.962343in}{4.164253in}}%
\pgfpathlineto{\pgfqpoint{2.085480in}{4.074667in}}%
\pgfpathlineto{\pgfqpoint{2.363152in}{3.863024in}}%
\pgfpathlineto{\pgfqpoint{2.483394in}{3.767493in}}%
\pgfpathlineto{\pgfqpoint{2.610469in}{3.664000in}}%
\pgfpathlineto{\pgfqpoint{2.884202in}{3.432406in}}%
\pgfpathlineto{\pgfqpoint{3.004444in}{3.327013in}}%
\pgfpathlineto{\pgfqpoint{3.325091in}{3.034668in}}%
\pgfpathlineto{\pgfqpoint{3.605657in}{2.765393in}}%
\pgfpathlineto{\pgfqpoint{3.899278in}{2.469333in}}%
\pgfpathlineto{\pgfqpoint{4.008574in}{2.355369in}}%
\pgfpathlineto{\pgfqpoint{4.146620in}{2.208000in}}%
\pgfpathlineto{\pgfqpoint{4.416407in}{1.909333in}}%
\pgfpathlineto{\pgfqpoint{4.687838in}{1.592615in}}%
\pgfpathlineto{\pgfqpoint{4.768000in}{1.495692in}}%
\pgfpathlineto{\pgfqpoint{4.768000in}{1.498667in}}%
\pgfpathlineto{\pgfqpoint{4.768000in}{1.498667in}}%
\pgfusepath{fill}%
\end{pgfscope}%
\begin{pgfscope}%
\pgfpathrectangle{\pgfqpoint{0.800000in}{0.528000in}}{\pgfqpoint{3.968000in}{3.696000in}}%
\pgfusepath{clip}%
\pgfsetbuttcap%
\pgfsetroundjoin%
\definecolor{currentfill}{rgb}{0.265145,0.232956,0.516599}%
\pgfsetfillcolor{currentfill}%
\pgfsetlinewidth{0.000000pt}%
\definecolor{currentstroke}{rgb}{0.000000,0.000000,0.000000}%
\pgfsetstrokecolor{currentstroke}%
\pgfsetdash{}{0pt}%
\pgfpathmoveto{\pgfqpoint{2.145902in}{0.528000in}}%
\pgfpathlineto{\pgfqpoint{1.842101in}{0.836744in}}%
\pgfpathlineto{\pgfqpoint{1.710174in}{0.976000in}}%
\pgfpathlineto{\pgfqpoint{1.601616in}{1.092990in}}%
\pgfpathlineto{\pgfqpoint{1.321051in}{1.406614in}}%
\pgfpathlineto{\pgfqpoint{1.200808in}{1.546262in}}%
\pgfpathlineto{\pgfqpoint{1.080566in}{1.689361in}}%
\pgfpathlineto{\pgfqpoint{0.840081in}{1.987226in}}%
\pgfpathlineto{\pgfqpoint{0.800000in}{2.038611in}}%
\pgfpathlineto{\pgfqpoint{0.800000in}{2.031615in}}%
\pgfpathlineto{\pgfqpoint{1.000404in}{1.780315in}}%
\pgfpathlineto{\pgfqpoint{1.120646in}{1.634869in}}%
\pgfpathlineto{\pgfqpoint{1.240889in}{1.493018in}}%
\pgfpathlineto{\pgfqpoint{1.521455in}{1.174954in}}%
\pgfpathlineto{\pgfqpoint{1.641697in}{1.043622in}}%
\pgfpathlineto{\pgfqpoint{1.922263in}{0.747929in}}%
\pgfpathlineto{\pgfqpoint{2.140224in}{0.528000in}}%
\pgfpathmoveto{\pgfqpoint{4.768000in}{1.508796in}}%
\pgfpathlineto{\pgfqpoint{4.524955in}{1.797333in}}%
\pgfpathlineto{\pgfqpoint{4.246949in}{2.110656in}}%
\pgfpathlineto{\pgfqpoint{3.966384in}{2.411138in}}%
\pgfpathlineto{\pgfqpoint{3.685818in}{2.697341in}}%
\pgfpathlineto{\pgfqpoint{3.537957in}{2.842667in}}%
\pgfpathlineto{\pgfqpoint{3.244929in}{3.120268in}}%
\pgfpathlineto{\pgfqpoint{2.924283in}{3.408507in}}%
\pgfpathlineto{\pgfqpoint{2.801473in}{3.514667in}}%
\pgfpathlineto{\pgfqpoint{2.523475in}{3.746187in}}%
\pgfpathlineto{\pgfqpoint{2.392955in}{3.850667in}}%
\pgfpathlineto{\pgfqpoint{2.242909in}{3.967463in}}%
\pgfpathlineto{\pgfqpoint{2.122667in}{4.058299in}}%
\pgfpathlineto{\pgfqpoint{2.042505in}{4.117588in}}%
\pgfpathlineto{\pgfqpoint{1.922263in}{4.204329in}}%
\pgfpathlineto{\pgfqpoint{1.894507in}{4.224000in}}%
\pgfpathlineto{\pgfqpoint{1.886445in}{4.224000in}}%
\pgfpathlineto{\pgfqpoint{2.143159in}{4.037333in}}%
\pgfpathlineto{\pgfqpoint{2.242909in}{3.961910in}}%
\pgfpathlineto{\pgfqpoint{2.525986in}{3.738667in}}%
\pgfpathlineto{\pgfqpoint{2.662121in}{3.626667in}}%
\pgfpathlineto{\pgfqpoint{2.804040in}{3.506949in}}%
\pgfpathlineto{\pgfqpoint{2.924742in}{3.402667in}}%
\pgfpathlineto{\pgfqpoint{3.068165in}{3.275352in}}%
\pgfpathlineto{\pgfqpoint{3.134334in}{3.216000in}}%
\pgfpathlineto{\pgfqpoint{3.285010in}{3.077603in}}%
\pgfpathlineto{\pgfqpoint{3.570602in}{2.805333in}}%
\pgfpathlineto{\pgfqpoint{3.685818in}{2.691724in}}%
\pgfpathlineto{\pgfqpoint{3.832192in}{2.544000in}}%
\pgfpathlineto{\pgfqpoint{4.126707in}{2.235412in}}%
\pgfpathlineto{\pgfqpoint{4.407273in}{1.925825in}}%
\pgfpathlineto{\pgfqpoint{4.647758in}{1.647019in}}%
\pgfpathlineto{\pgfqpoint{4.768000in}{1.502281in}}%
\pgfpathlineto{\pgfqpoint{4.768000in}{1.502281in}}%
\pgfusepath{fill}%
\end{pgfscope}%
\begin{pgfscope}%
\pgfpathrectangle{\pgfqpoint{0.800000in}{0.528000in}}{\pgfqpoint{3.968000in}{3.696000in}}%
\pgfusepath{clip}%
\pgfsetbuttcap%
\pgfsetroundjoin%
\definecolor{currentfill}{rgb}{0.265145,0.232956,0.516599}%
\pgfsetfillcolor{currentfill}%
\pgfsetlinewidth{0.000000pt}%
\definecolor{currentstroke}{rgb}{0.000000,0.000000,0.000000}%
\pgfsetstrokecolor{currentstroke}%
\pgfsetdash{}{0pt}%
\pgfpathmoveto{\pgfqpoint{2.140224in}{0.528000in}}%
\pgfpathlineto{\pgfqpoint{1.842101in}{0.830934in}}%
\pgfpathlineto{\pgfqpoint{1.704708in}{0.976000in}}%
\pgfpathlineto{\pgfqpoint{1.582865in}{1.107867in}}%
\pgfpathlineto{\pgfqpoint{1.521455in}{1.174954in}}%
\pgfpathlineto{\pgfqpoint{1.386838in}{1.325389in}}%
\pgfpathlineto{\pgfqpoint{1.268210in}{1.461333in}}%
\pgfpathlineto{\pgfqpoint{1.000404in}{1.780315in}}%
\pgfpathlineto{\pgfqpoint{0.800000in}{2.031615in}}%
\pgfpathlineto{\pgfqpoint{0.800000in}{2.024620in}}%
\pgfpathlineto{\pgfqpoint{1.000404in}{1.773726in}}%
\pgfpathlineto{\pgfqpoint{1.120646in}{1.628471in}}%
\pgfpathlineto{\pgfqpoint{1.240889in}{1.486800in}}%
\pgfpathlineto{\pgfqpoint{1.493416in}{1.200000in}}%
\pgfpathlineto{\pgfqpoint{1.601616in}{1.081158in}}%
\pgfpathlineto{\pgfqpoint{1.882182in}{0.783572in}}%
\pgfpathlineto{\pgfqpoint{2.134545in}{0.528000in}}%
\pgfpathmoveto{\pgfqpoint{4.768000in}{1.515311in}}%
\pgfpathlineto{\pgfqpoint{4.527515in}{1.800590in}}%
\pgfpathlineto{\pgfqpoint{4.246949in}{2.116606in}}%
\pgfpathlineto{\pgfqpoint{3.966384in}{2.416898in}}%
\pgfpathlineto{\pgfqpoint{3.685818in}{2.702923in}}%
\pgfpathlineto{\pgfqpoint{3.543673in}{2.842667in}}%
\pgfpathlineto{\pgfqpoint{3.244929in}{3.125765in}}%
\pgfpathlineto{\pgfqpoint{2.924283in}{3.413945in}}%
\pgfpathlineto{\pgfqpoint{2.784757in}{3.534039in}}%
\pgfpathlineto{\pgfqpoint{2.719894in}{3.589333in}}%
\pgfpathlineto{\pgfqpoint{2.421703in}{3.833462in}}%
\pgfpathlineto{\pgfqpoint{2.304667in}{3.925333in}}%
\pgfpathlineto{\pgfqpoint{2.182657in}{4.018544in}}%
\pgfpathlineto{\pgfqpoint{2.116184in}{4.068628in}}%
\pgfpathlineto{\pgfqpoint{2.042505in}{4.123205in}}%
\pgfpathlineto{\pgfqpoint{1.913654in}{4.215982in}}%
\pgfpathlineto{\pgfqpoint{1.902568in}{4.224000in}}%
\pgfpathlineto{\pgfqpoint{1.894507in}{4.224000in}}%
\pgfpathlineto{\pgfqpoint{2.135012in}{4.048833in}}%
\pgfpathlineto{\pgfqpoint{2.214247in}{3.989364in}}%
\pgfpathlineto{\pgfqpoint{2.345450in}{3.888000in}}%
\pgfpathlineto{\pgfqpoint{2.603636in}{3.680629in}}%
\pgfpathlineto{\pgfqpoint{2.757575in}{3.552000in}}%
\pgfpathlineto{\pgfqpoint{2.884202in}{3.443423in}}%
\pgfpathlineto{\pgfqpoint{3.181359in}{3.178667in}}%
\pgfpathlineto{\pgfqpoint{3.325091in}{3.045694in}}%
\pgfpathlineto{\pgfqpoint{3.629459in}{2.752837in}}%
\pgfpathlineto{\pgfqpoint{3.689837in}{2.693333in}}%
\pgfpathlineto{\pgfqpoint{3.806061in}{2.576382in}}%
\pgfpathlineto{\pgfqpoint{3.946359in}{2.432000in}}%
\pgfpathlineto{\pgfqpoint{4.235649in}{2.122808in}}%
\pgfpathlineto{\pgfqpoint{4.294111in}{2.058667in}}%
\pgfpathlineto{\pgfqpoint{4.567596in}{1.747691in}}%
\pgfpathlineto{\pgfqpoint{4.687838in}{1.605603in}}%
\pgfpathlineto{\pgfqpoint{4.768000in}{1.508796in}}%
\pgfpathlineto{\pgfqpoint{4.768000in}{1.508796in}}%
\pgfusepath{fill}%
\end{pgfscope}%
\begin{pgfscope}%
\pgfpathrectangle{\pgfqpoint{0.800000in}{0.528000in}}{\pgfqpoint{3.968000in}{3.696000in}}%
\pgfusepath{clip}%
\pgfsetbuttcap%
\pgfsetroundjoin%
\definecolor{currentfill}{rgb}{0.265145,0.232956,0.516599}%
\pgfsetfillcolor{currentfill}%
\pgfsetlinewidth{0.000000pt}%
\definecolor{currentstroke}{rgb}{0.000000,0.000000,0.000000}%
\pgfsetstrokecolor{currentstroke}%
\pgfsetdash{}{0pt}%
\pgfpathmoveto{\pgfqpoint{2.134545in}{0.528000in}}%
\pgfpathlineto{\pgfqpoint{1.840654in}{0.826667in}}%
\pgfpathlineto{\pgfqpoint{1.721859in}{0.951857in}}%
\pgfpathlineto{\pgfqpoint{1.441293in}{1.258077in}}%
\pgfpathlineto{\pgfqpoint{1.321051in}{1.394217in}}%
\pgfpathlineto{\pgfqpoint{1.190785in}{1.545336in}}%
\pgfpathlineto{\pgfqpoint{1.073183in}{1.685333in}}%
\pgfpathlineto{\pgfqpoint{0.831846in}{1.984000in}}%
\pgfpathlineto{\pgfqpoint{0.800000in}{2.024620in}}%
\pgfpathlineto{\pgfqpoint{0.800000in}{2.017720in}}%
\pgfpathlineto{\pgfqpoint{1.040485in}{1.718334in}}%
\pgfpathlineto{\pgfqpoint{1.289772in}{1.424000in}}%
\pgfpathlineto{\pgfqpoint{1.401212in}{1.297057in}}%
\pgfpathlineto{\pgfqpoint{1.681778in}{0.988827in}}%
\pgfpathlineto{\pgfqpoint{1.802020in}{0.861326in}}%
\pgfpathlineto{\pgfqpoint{1.943666in}{0.714667in}}%
\pgfpathlineto{\pgfqpoint{2.128867in}{0.528000in}}%
\pgfpathmoveto{\pgfqpoint{4.768000in}{1.521826in}}%
\pgfpathlineto{\pgfqpoint{4.527515in}{1.806742in}}%
\pgfpathlineto{\pgfqpoint{4.246949in}{2.122556in}}%
\pgfpathlineto{\pgfqpoint{3.966384in}{2.422658in}}%
\pgfpathlineto{\pgfqpoint{3.685818in}{2.708505in}}%
\pgfpathlineto{\pgfqpoint{3.549389in}{2.842667in}}%
\pgfpathlineto{\pgfqpoint{3.244929in}{3.131263in}}%
\pgfpathlineto{\pgfqpoint{2.933440in}{3.411196in}}%
\pgfpathlineto{\pgfqpoint{2.884202in}{3.454284in}}%
\pgfpathlineto{\pgfqpoint{2.745890in}{3.572502in}}%
\pgfpathlineto{\pgfqpoint{2.681936in}{3.626667in}}%
\pgfpathlineto{\pgfqpoint{2.533950in}{3.748424in}}%
\pgfpathlineto{\pgfqpoint{2.483394in}{3.789543in}}%
\pgfpathlineto{\pgfqpoint{2.359660in}{3.888000in}}%
\pgfpathlineto{\pgfqpoint{2.082586in}{4.099311in}}%
\pgfpathlineto{\pgfqpoint{1.960174in}{4.188688in}}%
\pgfpathlineto{\pgfqpoint{1.910630in}{4.224000in}}%
\pgfpathlineto{\pgfqpoint{1.902568in}{4.224000in}}%
\pgfpathlineto{\pgfqpoint{2.122667in}{4.063931in}}%
\pgfpathlineto{\pgfqpoint{2.421703in}{3.833462in}}%
\pgfpathlineto{\pgfqpoint{2.539442in}{3.738667in}}%
\pgfpathlineto{\pgfqpoint{2.683798in}{3.619608in}}%
\pgfpathlineto{\pgfqpoint{2.992236in}{3.353961in}}%
\pgfpathlineto{\pgfqpoint{3.053785in}{3.299292in}}%
\pgfpathlineto{\pgfqpoint{3.124687in}{3.235705in}}%
\pgfpathlineto{\pgfqpoint{3.445333in}{2.937450in}}%
\pgfpathlineto{\pgfqpoint{3.748587in}{2.639799in}}%
\pgfpathlineto{\pgfqpoint{3.814820in}{2.573174in}}%
\pgfpathlineto{\pgfqpoint{3.951888in}{2.432000in}}%
\pgfpathlineto{\pgfqpoint{4.246949in}{2.116606in}}%
\pgfpathlineto{\pgfqpoint{4.510995in}{1.819279in}}%
\pgfpathlineto{\pgfqpoint{4.567596in}{1.754000in}}%
\pgfpathlineto{\pgfqpoint{4.689007in}{1.610667in}}%
\pgfpathlineto{\pgfqpoint{4.768000in}{1.515311in}}%
\pgfpathlineto{\pgfqpoint{4.768000in}{1.515311in}}%
\pgfusepath{fill}%
\end{pgfscope}%
\begin{pgfscope}%
\pgfpathrectangle{\pgfqpoint{0.800000in}{0.528000in}}{\pgfqpoint{3.968000in}{3.696000in}}%
\pgfusepath{clip}%
\pgfsetbuttcap%
\pgfsetroundjoin%
\definecolor{currentfill}{rgb}{0.265145,0.232956,0.516599}%
\pgfsetfillcolor{currentfill}%
\pgfsetlinewidth{0.000000pt}%
\definecolor{currentstroke}{rgb}{0.000000,0.000000,0.000000}%
\pgfsetstrokecolor{currentstroke}%
\pgfsetdash{}{0pt}%
\pgfpathmoveto{\pgfqpoint{2.128867in}{0.528000in}}%
\pgfpathlineto{\pgfqpoint{1.842101in}{0.819473in}}%
\pgfpathlineto{\pgfqpoint{1.721859in}{0.946022in}}%
\pgfpathlineto{\pgfqpoint{1.441293in}{1.252047in}}%
\pgfpathlineto{\pgfqpoint{1.321051in}{1.388018in}}%
\pgfpathlineto{\pgfqpoint{1.193423in}{1.536000in}}%
\pgfpathlineto{\pgfqpoint{1.080566in}{1.670038in}}%
\pgfpathlineto{\pgfqpoint{0.960323in}{1.816390in}}%
\pgfpathlineto{\pgfqpoint{0.800000in}{2.017720in}}%
\pgfpathlineto{\pgfqpoint{0.800000in}{2.010905in}}%
\pgfpathlineto{\pgfqpoint{1.040485in}{1.711916in}}%
\pgfpathlineto{\pgfqpoint{1.298909in}{1.407290in}}%
\pgfpathlineto{\pgfqpoint{1.415762in}{1.274667in}}%
\pgfpathlineto{\pgfqpoint{1.521455in}{1.157047in}}%
\pgfpathlineto{\pgfqpoint{1.802020in}{0.855634in}}%
\pgfpathlineto{\pgfqpoint{1.938132in}{0.714667in}}%
\pgfpathlineto{\pgfqpoint{2.123189in}{0.528000in}}%
\pgfpathmoveto{\pgfqpoint{4.768000in}{1.528342in}}%
\pgfpathlineto{\pgfqpoint{4.540961in}{1.797333in}}%
\pgfpathlineto{\pgfqpoint{4.426916in}{1.927630in}}%
\pgfpathlineto{\pgfqpoint{4.367192in}{1.995251in}}%
\pgfpathlineto{\pgfqpoint{4.086626in}{2.301630in}}%
\pgfpathlineto{\pgfqpoint{3.806061in}{2.593305in}}%
\pgfpathlineto{\pgfqpoint{3.669106in}{2.730667in}}%
\pgfpathlineto{\pgfqpoint{3.525495in}{2.871424in}}%
\pgfpathlineto{\pgfqpoint{3.239967in}{3.141333in}}%
\pgfpathlineto{\pgfqpoint{3.084606in}{3.282785in}}%
\pgfpathlineto{\pgfqpoint{2.763960in}{3.562934in}}%
\pgfpathlineto{\pgfqpoint{2.460571in}{3.813333in}}%
\pgfpathlineto{\pgfqpoint{2.202828in}{4.014681in}}%
\pgfpathlineto{\pgfqpoint{1.918691in}{4.224000in}}%
\pgfpathlineto{\pgfqpoint{1.910630in}{4.224000in}}%
\pgfpathlineto{\pgfqpoint{2.122667in}{4.069563in}}%
\pgfpathlineto{\pgfqpoint{2.406962in}{3.850667in}}%
\pgfpathlineto{\pgfqpoint{2.563556in}{3.724496in}}%
\pgfpathlineto{\pgfqpoint{2.857558in}{3.477333in}}%
\pgfpathlineto{\pgfqpoint{3.004444in}{3.348845in}}%
\pgfpathlineto{\pgfqpoint{3.325091in}{3.056720in}}%
\pgfpathlineto{\pgfqpoint{3.635331in}{2.758307in}}%
\pgfpathlineto{\pgfqpoint{3.701033in}{2.693333in}}%
\pgfpathlineto{\pgfqpoint{3.846141in}{2.546894in}}%
\pgfpathlineto{\pgfqpoint{4.126707in}{2.253010in}}%
\pgfpathlineto{\pgfqpoint{4.246949in}{2.122556in}}%
\pgfpathlineto{\pgfqpoint{4.527515in}{1.806742in}}%
\pgfpathlineto{\pgfqpoint{4.647758in}{1.666028in}}%
\pgfpathlineto{\pgfqpoint{4.768000in}{1.521826in}}%
\pgfpathlineto{\pgfqpoint{4.768000in}{1.521826in}}%
\pgfusepath{fill}%
\end{pgfscope}%
\begin{pgfscope}%
\pgfpathrectangle{\pgfqpoint{0.800000in}{0.528000in}}{\pgfqpoint{3.968000in}{3.696000in}}%
\pgfusepath{clip}%
\pgfsetbuttcap%
\pgfsetroundjoin%
\definecolor{currentfill}{rgb}{0.263663,0.237631,0.518762}%
\pgfsetfillcolor{currentfill}%
\pgfsetlinewidth{0.000000pt}%
\definecolor{currentstroke}{rgb}{0.000000,0.000000,0.000000}%
\pgfsetstrokecolor{currentstroke}%
\pgfsetdash{}{0pt}%
\pgfpathmoveto{\pgfqpoint{2.123189in}{0.528000in}}%
\pgfpathlineto{\pgfqpoint{1.842101in}{0.813788in}}%
\pgfpathlineto{\pgfqpoint{1.721859in}{0.940187in}}%
\pgfpathlineto{\pgfqpoint{1.441293in}{1.246018in}}%
\pgfpathlineto{\pgfqpoint{1.298909in}{1.407290in}}%
\pgfpathlineto{\pgfqpoint{1.188113in}{1.536000in}}%
\pgfpathlineto{\pgfqpoint{1.080566in}{1.663629in}}%
\pgfpathlineto{\pgfqpoint{0.960323in}{1.809790in}}%
\pgfpathlineto{\pgfqpoint{0.800000in}{2.010905in}}%
\pgfpathlineto{\pgfqpoint{0.800000in}{2.004090in}}%
\pgfpathlineto{\pgfqpoint{1.026312in}{1.722667in}}%
\pgfpathlineto{\pgfqpoint{1.120646in}{1.609309in}}%
\pgfpathlineto{\pgfqpoint{1.401212in}{1.284980in}}%
\pgfpathlineto{\pgfqpoint{1.648092in}{1.013333in}}%
\pgfpathlineto{\pgfqpoint{1.946922in}{0.700302in}}%
\pgfpathlineto{\pgfqpoint{2.005883in}{0.640000in}}%
\pgfpathlineto{\pgfqpoint{2.117631in}{0.528000in}}%
\pgfpathlineto{\pgfqpoint{2.122667in}{0.528000in}}%
\pgfpathmoveto{\pgfqpoint{4.768000in}{1.534857in}}%
\pgfpathlineto{\pgfqpoint{4.546277in}{1.797333in}}%
\pgfpathlineto{\pgfqpoint{4.447354in}{1.910917in}}%
\pgfpathlineto{\pgfqpoint{4.166788in}{2.221494in}}%
\pgfpathlineto{\pgfqpoint{3.886222in}{2.517063in}}%
\pgfpathlineto{\pgfqpoint{3.749495in}{2.656000in}}%
\pgfpathlineto{\pgfqpoint{3.605657in}{2.798848in}}%
\pgfpathlineto{\pgfqpoint{3.483656in}{2.917333in}}%
\pgfpathlineto{\pgfqpoint{3.164502in}{3.216000in}}%
\pgfpathlineto{\pgfqpoint{2.844121in}{3.499788in}}%
\pgfpathlineto{\pgfqpoint{2.694839in}{3.626667in}}%
\pgfpathlineto{\pgfqpoint{2.443313in}{3.832630in}}%
\pgfpathlineto{\pgfqpoint{2.307910in}{3.939455in}}%
\pgfpathlineto{\pgfqpoint{2.180239in}{4.037333in}}%
\pgfpathlineto{\pgfqpoint{2.080630in}{4.112000in}}%
\pgfpathlineto{\pgfqpoint{1.926605in}{4.224000in}}%
\pgfpathlineto{\pgfqpoint{1.918691in}{4.224000in}}%
\pgfpathlineto{\pgfqpoint{2.123355in}{4.074667in}}%
\pgfpathlineto{\pgfqpoint{2.413817in}{3.850667in}}%
\pgfpathlineto{\pgfqpoint{2.563556in}{3.729979in}}%
\pgfpathlineto{\pgfqpoint{2.863835in}{3.477333in}}%
\pgfpathlineto{\pgfqpoint{3.004444in}{3.354298in}}%
\pgfpathlineto{\pgfqpoint{3.325091in}{3.062233in}}%
\pgfpathlineto{\pgfqpoint{3.645737in}{2.753817in}}%
\pgfpathlineto{\pgfqpoint{3.780992in}{2.618667in}}%
\pgfpathlineto{\pgfqpoint{3.927047in}{2.469333in}}%
\pgfpathlineto{\pgfqpoint{4.058565in}{2.331196in}}%
\pgfpathlineto{\pgfqpoint{4.126707in}{2.258803in}}%
\pgfpathlineto{\pgfqpoint{4.246949in}{2.128505in}}%
\pgfpathlineto{\pgfqpoint{4.527515in}{1.812894in}}%
\pgfpathlineto{\pgfqpoint{4.647758in}{1.672356in}}%
\pgfpathlineto{\pgfqpoint{4.768000in}{1.528342in}}%
\pgfpathlineto{\pgfqpoint{4.768000in}{1.528342in}}%
\pgfusepath{fill}%
\end{pgfscope}%
\begin{pgfscope}%
\pgfpathrectangle{\pgfqpoint{0.800000in}{0.528000in}}{\pgfqpoint{3.968000in}{3.696000in}}%
\pgfusepath{clip}%
\pgfsetbuttcap%
\pgfsetroundjoin%
\definecolor{currentfill}{rgb}{0.263663,0.237631,0.518762}%
\pgfsetfillcolor{currentfill}%
\pgfsetlinewidth{0.000000pt}%
\definecolor{currentstroke}{rgb}{0.000000,0.000000,0.000000}%
\pgfsetstrokecolor{currentstroke}%
\pgfsetdash{}{0pt}%
\pgfpathmoveto{\pgfqpoint{2.117631in}{0.528000in}}%
\pgfpathlineto{\pgfqpoint{1.969130in}{0.677333in}}%
\pgfpathlineto{\pgfqpoint{1.842101in}{0.808104in}}%
\pgfpathlineto{\pgfqpoint{1.717888in}{0.938667in}}%
\pgfpathlineto{\pgfqpoint{1.441293in}{1.239988in}}%
\pgfpathlineto{\pgfqpoint{1.311615in}{1.386667in}}%
\pgfpathlineto{\pgfqpoint{1.200808in}{1.514881in}}%
\pgfpathlineto{\pgfqpoint{1.080566in}{1.657221in}}%
\pgfpathlineto{\pgfqpoint{0.960323in}{1.803190in}}%
\pgfpathlineto{\pgfqpoint{0.800000in}{2.004090in}}%
\pgfpathlineto{\pgfqpoint{0.800000in}{1.997275in}}%
\pgfpathlineto{\pgfqpoint{1.021014in}{1.722667in}}%
\pgfpathlineto{\pgfqpoint{1.120646in}{1.603061in}}%
\pgfpathlineto{\pgfqpoint{1.371941in}{1.312000in}}%
\pgfpathlineto{\pgfqpoint{1.481374in}{1.189516in}}%
\pgfpathlineto{\pgfqpoint{1.608125in}{1.050667in}}%
\pgfpathlineto{\pgfqpoint{1.890780in}{0.752000in}}%
\pgfpathlineto{\pgfqpoint{2.037364in}{0.602667in}}%
\pgfpathlineto{\pgfqpoint{2.112085in}{0.528000in}}%
\pgfpathmoveto{\pgfqpoint{4.768000in}{1.541243in}}%
\pgfpathlineto{\pgfqpoint{4.647486in}{1.685333in}}%
\pgfpathlineto{\pgfqpoint{4.367192in}{2.007204in}}%
\pgfpathlineto{\pgfqpoint{4.115225in}{2.282667in}}%
\pgfpathlineto{\pgfqpoint{4.006465in}{2.397903in}}%
\pgfpathlineto{\pgfqpoint{3.865424in}{2.544000in}}%
\pgfpathlineto{\pgfqpoint{3.565576in}{2.843582in}}%
\pgfpathlineto{\pgfqpoint{3.244929in}{3.147626in}}%
\pgfpathlineto{\pgfqpoint{3.087929in}{3.290667in}}%
\pgfpathlineto{\pgfqpoint{2.961978in}{3.402667in}}%
\pgfpathlineto{\pgfqpoint{2.656626in}{3.664000in}}%
\pgfpathlineto{\pgfqpoint{2.403232in}{3.869995in}}%
\pgfpathlineto{\pgfqpoint{2.272797in}{3.972160in}}%
\pgfpathlineto{\pgfqpoint{2.138046in}{4.074667in}}%
\pgfpathlineto{\pgfqpoint{2.037490in}{4.149333in}}%
\pgfpathlineto{\pgfqpoint{1.934402in}{4.224000in}}%
\pgfpathlineto{\pgfqpoint{1.926605in}{4.224000in}}%
\pgfpathlineto{\pgfqpoint{2.080630in}{4.112000in}}%
\pgfpathlineto{\pgfqpoint{2.180239in}{4.037333in}}%
\pgfpathlineto{\pgfqpoint{2.443313in}{3.832630in}}%
\pgfpathlineto{\pgfqpoint{2.752035in}{3.578226in}}%
\pgfpathlineto{\pgfqpoint{2.814843in}{3.524729in}}%
\pgfpathlineto{\pgfqpoint{2.884202in}{3.465146in}}%
\pgfpathlineto{\pgfqpoint{3.205348in}{3.178667in}}%
\pgfpathlineto{\pgfqpoint{3.365172in}{3.030091in}}%
\pgfpathlineto{\pgfqpoint{3.485414in}{2.915647in}}%
\pgfpathlineto{\pgfqpoint{3.637002in}{2.768000in}}%
\pgfpathlineto{\pgfqpoint{3.926303in}{2.475734in}}%
\pgfpathlineto{\pgfqpoint{4.046545in}{2.349951in}}%
\pgfpathlineto{\pgfqpoint{4.327111in}{2.045934in}}%
\pgfpathlineto{\pgfqpoint{4.448748in}{1.909333in}}%
\pgfpathlineto{\pgfqpoint{4.727919in}{1.583145in}}%
\pgfpathlineto{\pgfqpoint{4.768000in}{1.534857in}}%
\pgfpathlineto{\pgfqpoint{4.768000in}{1.536000in}}%
\pgfusepath{fill}%
\end{pgfscope}%
\begin{pgfscope}%
\pgfpathrectangle{\pgfqpoint{0.800000in}{0.528000in}}{\pgfqpoint{3.968000in}{3.696000in}}%
\pgfusepath{clip}%
\pgfsetbuttcap%
\pgfsetroundjoin%
\definecolor{currentfill}{rgb}{0.263663,0.237631,0.518762}%
\pgfsetfillcolor{currentfill}%
\pgfsetlinewidth{0.000000pt}%
\definecolor{currentstroke}{rgb}{0.000000,0.000000,0.000000}%
\pgfsetstrokecolor{currentstroke}%
\pgfsetdash{}{0pt}%
\pgfpathmoveto{\pgfqpoint{2.112085in}{0.528000in}}%
\pgfpathlineto{\pgfqpoint{1.962343in}{0.678579in}}%
\pgfpathlineto{\pgfqpoint{1.661020in}{0.993998in}}%
\pgfpathlineto{\pgfqpoint{1.601616in}{1.057716in}}%
\pgfpathlineto{\pgfqpoint{1.471924in}{1.200000in}}%
\pgfpathlineto{\pgfqpoint{1.361131in}{1.324227in}}%
\pgfpathlineto{\pgfqpoint{1.240889in}{1.461927in}}%
\pgfpathlineto{\pgfqpoint{1.114246in}{1.610667in}}%
\pgfpathlineto{\pgfqpoint{1.000404in}{1.747676in}}%
\pgfpathlineto{\pgfqpoint{0.899473in}{1.872000in}}%
\pgfpathlineto{\pgfqpoint{0.800000in}{1.997275in}}%
\pgfpathlineto{\pgfqpoint{0.800000in}{1.990460in}}%
\pgfpathlineto{\pgfqpoint{1.000404in}{1.741247in}}%
\pgfpathlineto{\pgfqpoint{1.240889in}{1.455838in}}%
\pgfpathlineto{\pgfqpoint{1.366594in}{1.312000in}}%
\pgfpathlineto{\pgfqpoint{1.481374in}{1.183630in}}%
\pgfpathlineto{\pgfqpoint{1.602714in}{1.050667in}}%
\pgfpathlineto{\pgfqpoint{1.914628in}{0.721778in}}%
\pgfpathlineto{\pgfqpoint{2.037034in}{0.597571in}}%
\pgfpathlineto{\pgfqpoint{2.082586in}{0.551837in}}%
\pgfpathlineto{\pgfqpoint{2.106539in}{0.528000in}}%
\pgfpathmoveto{\pgfqpoint{4.768000in}{1.547601in}}%
\pgfpathlineto{\pgfqpoint{4.647758in}{1.691201in}}%
\pgfpathlineto{\pgfqpoint{4.393204in}{1.984000in}}%
\pgfpathlineto{\pgfqpoint{4.287030in}{2.102122in}}%
\pgfpathlineto{\pgfqpoint{4.006465in}{2.403549in}}%
\pgfpathlineto{\pgfqpoint{3.870910in}{2.544000in}}%
\pgfpathlineto{\pgfqpoint{3.565576in}{2.849027in}}%
\pgfpathlineto{\pgfqpoint{3.244929in}{3.153012in}}%
\pgfpathlineto{\pgfqpoint{3.093877in}{3.290667in}}%
\pgfpathlineto{\pgfqpoint{2.964364in}{3.405951in}}%
\pgfpathlineto{\pgfqpoint{2.652774in}{3.672436in}}%
\pgfpathlineto{\pgfqpoint{2.603636in}{3.713326in}}%
\pgfpathlineto{\pgfqpoint{2.292064in}{3.962667in}}%
\pgfpathlineto{\pgfqpoint{2.082586in}{4.121603in}}%
\pgfpathlineto{\pgfqpoint{1.942199in}{4.224000in}}%
\pgfpathlineto{\pgfqpoint{1.934402in}{4.224000in}}%
\pgfpathlineto{\pgfqpoint{2.042505in}{4.145669in}}%
\pgfpathlineto{\pgfqpoint{2.332976in}{3.925333in}}%
\pgfpathlineto{\pgfqpoint{2.611632in}{3.701333in}}%
\pgfpathlineto{\pgfqpoint{2.763960in}{3.573752in}}%
\pgfpathlineto{\pgfqpoint{2.919344in}{3.440000in}}%
\pgfpathlineto{\pgfqpoint{3.046263in}{3.328000in}}%
\pgfpathlineto{\pgfqpoint{3.365172in}{3.035499in}}%
\pgfpathlineto{\pgfqpoint{3.489338in}{2.917333in}}%
\pgfpathlineto{\pgfqpoint{3.642659in}{2.768000in}}%
\pgfpathlineto{\pgfqpoint{3.926303in}{2.481363in}}%
\pgfpathlineto{\pgfqpoint{4.046545in}{2.355728in}}%
\pgfpathlineto{\pgfqpoint{4.327111in}{2.051901in}}%
\pgfpathlineto{\pgfqpoint{4.454023in}{1.909333in}}%
\pgfpathlineto{\pgfqpoint{4.727919in}{1.589493in}}%
\pgfpathlineto{\pgfqpoint{4.768000in}{1.541243in}}%
\pgfpathlineto{\pgfqpoint{4.768000in}{1.541243in}}%
\pgfusepath{fill}%
\end{pgfscope}%
\begin{pgfscope}%
\pgfpathrectangle{\pgfqpoint{0.800000in}{0.528000in}}{\pgfqpoint{3.968000in}{3.696000in}}%
\pgfusepath{clip}%
\pgfsetbuttcap%
\pgfsetroundjoin%
\definecolor{currentfill}{rgb}{0.262138,0.242286,0.520837}%
\pgfsetfillcolor{currentfill}%
\pgfsetlinewidth{0.000000pt}%
\definecolor{currentstroke}{rgb}{0.000000,0.000000,0.000000}%
\pgfsetstrokecolor{currentstroke}%
\pgfsetdash{}{0pt}%
\pgfpathmoveto{\pgfqpoint{2.106539in}{0.528000in}}%
\pgfpathlineto{\pgfqpoint{1.962343in}{0.673011in}}%
\pgfpathlineto{\pgfqpoint{1.842101in}{0.796734in}}%
\pgfpathlineto{\pgfqpoint{1.707149in}{0.938667in}}%
\pgfpathlineto{\pgfqpoint{1.433058in}{1.237333in}}%
\pgfpathlineto{\pgfqpoint{1.321051in}{1.363759in}}%
\pgfpathlineto{\pgfqpoint{1.200808in}{1.502426in}}%
\pgfpathlineto{\pgfqpoint{1.067354in}{1.660306in}}%
\pgfpathlineto{\pgfqpoint{0.954485in}{1.797333in}}%
\pgfpathlineto{\pgfqpoint{0.800000in}{1.990460in}}%
\pgfpathlineto{\pgfqpoint{0.800000in}{1.983654in}}%
\pgfpathlineto{\pgfqpoint{0.920242in}{1.832976in}}%
\pgfpathlineto{\pgfqpoint{1.166875in}{1.536000in}}%
\pgfpathlineto{\pgfqpoint{1.280970in}{1.403579in}}%
\pgfpathlineto{\pgfqpoint{1.401212in}{1.267038in}}%
\pgfpathlineto{\pgfqpoint{1.528932in}{1.125333in}}%
\pgfpathlineto{\pgfqpoint{1.641697in}{1.002889in}}%
\pgfpathlineto{\pgfqpoint{1.922263in}{0.708383in}}%
\pgfpathlineto{\pgfqpoint{2.100993in}{0.528000in}}%
\pgfpathmoveto{\pgfqpoint{4.768000in}{1.553960in}}%
\pgfpathlineto{\pgfqpoint{4.647758in}{1.697382in}}%
\pgfpathlineto{\pgfqpoint{4.398531in}{1.984000in}}%
\pgfpathlineto{\pgfqpoint{4.287030in}{2.107949in}}%
\pgfpathlineto{\pgfqpoint{4.006465in}{2.409194in}}%
\pgfpathlineto{\pgfqpoint{3.876395in}{2.544000in}}%
\pgfpathlineto{\pgfqpoint{3.565576in}{2.854472in}}%
\pgfpathlineto{\pgfqpoint{3.244929in}{3.158398in}}%
\pgfpathlineto{\pgfqpoint{3.099826in}{3.290667in}}%
\pgfpathlineto{\pgfqpoint{2.964364in}{3.411287in}}%
\pgfpathlineto{\pgfqpoint{2.643717in}{3.685530in}}%
\pgfpathlineto{\pgfqpoint{2.487696in}{3.813333in}}%
\pgfpathlineto{\pgfqpoint{2.394254in}{3.888000in}}%
\pgfpathlineto{\pgfqpoint{2.242909in}{4.006098in}}%
\pgfpathlineto{\pgfqpoint{1.949997in}{4.224000in}}%
\pgfpathlineto{\pgfqpoint{1.942199in}{4.224000in}}%
\pgfpathlineto{\pgfqpoint{2.045093in}{4.149333in}}%
\pgfpathlineto{\pgfqpoint{2.323071in}{3.938532in}}%
\pgfpathlineto{\pgfqpoint{2.460570in}{3.829407in}}%
\pgfpathlineto{\pgfqpoint{2.527133in}{3.776000in}}%
\pgfpathlineto{\pgfqpoint{2.820837in}{3.530312in}}%
\pgfpathlineto{\pgfqpoint{2.884202in}{3.476007in}}%
\pgfpathlineto{\pgfqpoint{3.204848in}{3.189883in}}%
\pgfpathlineto{\pgfqpoint{3.337737in}{3.066667in}}%
\pgfpathlineto{\pgfqpoint{3.648255in}{2.768000in}}%
\pgfpathlineto{\pgfqpoint{3.797615in}{2.618667in}}%
\pgfpathlineto{\pgfqpoint{3.926303in}{2.486993in}}%
\pgfpathlineto{\pgfqpoint{4.050409in}{2.357333in}}%
\pgfpathlineto{\pgfqpoint{4.327111in}{2.057869in}}%
\pgfpathlineto{\pgfqpoint{4.459299in}{1.909333in}}%
\pgfpathlineto{\pgfqpoint{4.727919in}{1.595842in}}%
\pgfpathlineto{\pgfqpoint{4.768000in}{1.547601in}}%
\pgfpathlineto{\pgfqpoint{4.768000in}{1.547601in}}%
\pgfusepath{fill}%
\end{pgfscope}%
\begin{pgfscope}%
\pgfpathrectangle{\pgfqpoint{0.800000in}{0.528000in}}{\pgfqpoint{3.968000in}{3.696000in}}%
\pgfusepath{clip}%
\pgfsetbuttcap%
\pgfsetroundjoin%
\definecolor{currentfill}{rgb}{0.262138,0.242286,0.520837}%
\pgfsetfillcolor{currentfill}%
\pgfsetlinewidth{0.000000pt}%
\definecolor{currentstroke}{rgb}{0.000000,0.000000,0.000000}%
\pgfsetstrokecolor{currentstroke}%
\pgfsetdash{}{0pt}%
\pgfpathmoveto{\pgfqpoint{2.100993in}{0.528000in}}%
\pgfpathlineto{\pgfqpoint{1.962343in}{0.667469in}}%
\pgfpathlineto{\pgfqpoint{1.842101in}{0.791050in}}%
\pgfpathlineto{\pgfqpoint{1.701780in}{0.938667in}}%
\pgfpathlineto{\pgfqpoint{1.427778in}{1.237333in}}%
\pgfpathlineto{\pgfqpoint{1.321051in}{1.357702in}}%
\pgfpathlineto{\pgfqpoint{1.190587in}{1.508187in}}%
\pgfpathlineto{\pgfqpoint{1.072397in}{1.648000in}}%
\pgfpathlineto{\pgfqpoint{0.960323in}{1.783729in}}%
\pgfpathlineto{\pgfqpoint{0.858997in}{1.909333in}}%
\pgfpathlineto{\pgfqpoint{0.800000in}{1.983654in}}%
\pgfpathlineto{\pgfqpoint{0.800000in}{1.977010in}}%
\pgfpathlineto{\pgfqpoint{0.920242in}{1.826526in}}%
\pgfpathlineto{\pgfqpoint{1.164801in}{1.532206in}}%
\pgfpathlineto{\pgfqpoint{1.290437in}{1.386667in}}%
\pgfpathlineto{\pgfqpoint{1.401212in}{1.261134in}}%
\pgfpathlineto{\pgfqpoint{1.535614in}{1.112144in}}%
\pgfpathlineto{\pgfqpoint{1.661447in}{0.976000in}}%
\pgfpathlineto{\pgfqpoint{1.962343in}{0.661927in}}%
\pgfpathlineto{\pgfqpoint{2.095447in}{0.528000in}}%
\pgfpathmoveto{\pgfqpoint{4.768000in}{1.560319in}}%
\pgfpathlineto{\pgfqpoint{4.647758in}{1.703562in}}%
\pgfpathlineto{\pgfqpoint{4.387152in}{2.002742in}}%
\pgfpathlineto{\pgfqpoint{4.269201in}{2.133333in}}%
\pgfpathlineto{\pgfqpoint{4.131375in}{2.282667in}}%
\pgfpathlineto{\pgfqpoint{3.845398in}{2.581333in}}%
\pgfpathlineto{\pgfqpoint{3.525495in}{2.898786in}}%
\pgfpathlineto{\pgfqpoint{3.388891in}{3.029333in}}%
\pgfpathlineto{\pgfqpoint{3.244929in}{3.163784in}}%
\pgfpathlineto{\pgfqpoint{3.105775in}{3.290667in}}%
\pgfpathlineto{\pgfqpoint{2.964364in}{3.416623in}}%
\pgfpathlineto{\pgfqpoint{2.643717in}{3.690917in}}%
\pgfpathlineto{\pgfqpoint{2.494318in}{3.813333in}}%
\pgfpathlineto{\pgfqpoint{2.401153in}{3.888000in}}%
\pgfpathlineto{\pgfqpoint{2.242909in}{4.011522in}}%
\pgfpathlineto{\pgfqpoint{1.957794in}{4.224000in}}%
\pgfpathlineto{\pgfqpoint{1.949997in}{4.224000in}}%
\pgfpathlineto{\pgfqpoint{2.052539in}{4.149333in}}%
\pgfpathlineto{\pgfqpoint{2.323071in}{3.943970in}}%
\pgfpathlineto{\pgfqpoint{2.443313in}{3.849013in}}%
\pgfpathlineto{\pgfqpoint{2.723879in}{3.618462in}}%
\pgfpathlineto{\pgfqpoint{2.845699in}{3.514667in}}%
\pgfpathlineto{\pgfqpoint{3.164768in}{3.231878in}}%
\pgfpathlineto{\pgfqpoint{3.303569in}{3.104000in}}%
\pgfpathlineto{\pgfqpoint{3.605657in}{2.815338in}}%
\pgfpathlineto{\pgfqpoint{3.886222in}{2.533928in}}%
\pgfpathlineto{\pgfqpoint{4.020300in}{2.394667in}}%
\pgfpathlineto{\pgfqpoint{4.160809in}{2.245333in}}%
\pgfpathlineto{\pgfqpoint{4.447354in}{1.928900in}}%
\pgfpathlineto{\pgfqpoint{4.567596in}{1.791109in}}%
\pgfpathlineto{\pgfqpoint{4.768000in}{1.553960in}}%
\pgfpathlineto{\pgfqpoint{4.768000in}{1.553960in}}%
\pgfusepath{fill}%
\end{pgfscope}%
\begin{pgfscope}%
\pgfpathrectangle{\pgfqpoint{0.800000in}{0.528000in}}{\pgfqpoint{3.968000in}{3.696000in}}%
\pgfusepath{clip}%
\pgfsetbuttcap%
\pgfsetroundjoin%
\definecolor{currentfill}{rgb}{0.262138,0.242286,0.520837}%
\pgfsetfillcolor{currentfill}%
\pgfsetlinewidth{0.000000pt}%
\definecolor{currentstroke}{rgb}{0.000000,0.000000,0.000000}%
\pgfsetstrokecolor{currentstroke}%
\pgfsetdash{}{0pt}%
\pgfpathmoveto{\pgfqpoint{2.095447in}{0.528000in}}%
\pgfpathlineto{\pgfqpoint{1.962343in}{0.661927in}}%
\pgfpathlineto{\pgfqpoint{1.838363in}{0.789333in}}%
\pgfpathlineto{\pgfqpoint{1.696410in}{0.938667in}}%
\pgfpathlineto{\pgfqpoint{1.422498in}{1.237333in}}%
\pgfpathlineto{\pgfqpoint{1.321051in}{1.351645in}}%
\pgfpathlineto{\pgfqpoint{1.193521in}{1.498667in}}%
\pgfpathlineto{\pgfqpoint{1.080566in}{1.631975in}}%
\pgfpathlineto{\pgfqpoint{0.853709in}{1.909333in}}%
\pgfpathlineto{\pgfqpoint{0.800000in}{1.977010in}}%
\pgfpathlineto{\pgfqpoint{0.800000in}{1.970366in}}%
\pgfpathlineto{\pgfqpoint{0.920242in}{1.820076in}}%
\pgfpathlineto{\pgfqpoint{1.160727in}{1.530869in}}%
\pgfpathlineto{\pgfqpoint{1.303429in}{1.365747in}}%
\pgfpathlineto{\pgfqpoint{1.417218in}{1.237333in}}%
\pgfpathlineto{\pgfqpoint{1.552458in}{1.088000in}}%
\pgfpathlineto{\pgfqpoint{1.681778in}{0.948529in}}%
\pgfpathlineto{\pgfqpoint{1.962343in}{0.656385in}}%
\pgfpathlineto{\pgfqpoint{2.089901in}{0.528000in}}%
\pgfpathmoveto{\pgfqpoint{4.768000in}{1.566677in}}%
\pgfpathlineto{\pgfqpoint{4.659210in}{1.696000in}}%
\pgfpathlineto{\pgfqpoint{4.604886in}{1.760000in}}%
\pgfpathlineto{\pgfqpoint{4.327111in}{2.075394in}}%
\pgfpathlineto{\pgfqpoint{4.046545in}{2.378376in}}%
\pgfpathlineto{\pgfqpoint{3.898096in}{2.532940in}}%
\pgfpathlineto{\pgfqpoint{3.765980in}{2.667184in}}%
\pgfpathlineto{\pgfqpoint{3.626954in}{2.805333in}}%
\pgfpathlineto{\pgfqpoint{3.325091in}{3.094727in}}%
\pgfpathlineto{\pgfqpoint{3.028359in}{3.365333in}}%
\pgfpathlineto{\pgfqpoint{2.884202in}{3.491998in}}%
\pgfpathlineto{\pgfqpoint{2.563556in}{3.762389in}}%
\pgfpathlineto{\pgfqpoint{2.264881in}{4.000000in}}%
\pgfpathlineto{\pgfqpoint{2.143305in}{4.092777in}}%
\pgfpathlineto{\pgfqpoint{2.016708in}{4.186667in}}%
\pgfpathlineto{\pgfqpoint{1.965487in}{4.224000in}}%
\pgfpathlineto{\pgfqpoint{1.957794in}{4.224000in}}%
\pgfpathlineto{\pgfqpoint{2.059985in}{4.149333in}}%
\pgfpathlineto{\pgfqpoint{2.306039in}{3.962667in}}%
\pgfpathlineto{\pgfqpoint{2.423685in}{3.869718in}}%
\pgfpathlineto{\pgfqpoint{2.494318in}{3.813333in}}%
\pgfpathlineto{\pgfqpoint{2.764701in}{3.589333in}}%
\pgfpathlineto{\pgfqpoint{2.924283in}{3.451770in}}%
\pgfpathlineto{\pgfqpoint{3.244929in}{3.163784in}}%
\pgfpathlineto{\pgfqpoint{3.565576in}{2.859916in}}%
\pgfpathlineto{\pgfqpoint{3.696978in}{2.730667in}}%
\pgfpathlineto{\pgfqpoint{4.006465in}{2.414839in}}%
\pgfpathlineto{\pgfqpoint{4.131375in}{2.282667in}}%
\pgfpathlineto{\pgfqpoint{4.269201in}{2.133333in}}%
\pgfpathlineto{\pgfqpoint{4.535148in}{1.834667in}}%
\pgfpathlineto{\pgfqpoint{4.647758in}{1.703562in}}%
\pgfpathlineto{\pgfqpoint{4.768000in}{1.560319in}}%
\pgfpathlineto{\pgfqpoint{4.768000in}{1.560319in}}%
\pgfusepath{fill}%
\end{pgfscope}%
\begin{pgfscope}%
\pgfpathrectangle{\pgfqpoint{0.800000in}{0.528000in}}{\pgfqpoint{3.968000in}{3.696000in}}%
\pgfusepath{clip}%
\pgfsetbuttcap%
\pgfsetroundjoin%
\definecolor{currentfill}{rgb}{0.262138,0.242286,0.520837}%
\pgfsetfillcolor{currentfill}%
\pgfsetlinewidth{0.000000pt}%
\definecolor{currentstroke}{rgb}{0.000000,0.000000,0.000000}%
\pgfsetstrokecolor{currentstroke}%
\pgfsetdash{}{0pt}%
\pgfpathmoveto{\pgfqpoint{2.089901in}{0.528000in}}%
\pgfpathlineto{\pgfqpoint{1.962343in}{0.656385in}}%
\pgfpathlineto{\pgfqpoint{1.833009in}{0.789333in}}%
\pgfpathlineto{\pgfqpoint{1.691040in}{0.938667in}}%
\pgfpathlineto{\pgfqpoint{1.401212in}{1.255230in}}%
\pgfpathlineto{\pgfqpoint{1.156354in}{1.536000in}}%
\pgfpathlineto{\pgfqpoint{0.896448in}{1.849837in}}%
\pgfpathlineto{\pgfqpoint{0.840081in}{1.919798in}}%
\pgfpathlineto{\pgfqpoint{0.800000in}{1.970366in}}%
\pgfpathlineto{\pgfqpoint{0.800000in}{1.963721in}}%
\pgfpathlineto{\pgfqpoint{0.920242in}{1.813626in}}%
\pgfpathlineto{\pgfqpoint{1.160727in}{1.524775in}}%
\pgfpathlineto{\pgfqpoint{1.280970in}{1.385409in}}%
\pgfpathlineto{\pgfqpoint{1.411938in}{1.237333in}}%
\pgfpathlineto{\pgfqpoint{1.521455in}{1.116107in}}%
\pgfpathlineto{\pgfqpoint{1.650761in}{0.976000in}}%
\pgfpathlineto{\pgfqpoint{1.936365in}{0.677333in}}%
\pgfpathlineto{\pgfqpoint{2.084355in}{0.528000in}}%
\pgfpathmoveto{\pgfqpoint{4.768000in}{1.573036in}}%
\pgfpathlineto{\pgfqpoint{4.673688in}{1.685333in}}%
\pgfpathlineto{\pgfqpoint{4.407273in}{1.991955in}}%
\pgfpathlineto{\pgfqpoint{4.279826in}{2.133333in}}%
\pgfpathlineto{\pgfqpoint{4.166788in}{2.256070in}}%
\pgfpathlineto{\pgfqpoint{4.036429in}{2.394667in}}%
\pgfpathlineto{\pgfqpoint{3.892700in}{2.544000in}}%
\pgfpathlineto{\pgfqpoint{3.765980in}{2.672666in}}%
\pgfpathlineto{\pgfqpoint{3.632507in}{2.805333in}}%
\pgfpathlineto{\pgfqpoint{3.320939in}{3.104000in}}%
\pgfpathlineto{\pgfqpoint{3.164768in}{3.247992in}}%
\pgfpathlineto{\pgfqpoint{2.844121in}{3.531973in}}%
\pgfpathlineto{\pgfqpoint{2.523475in}{3.800442in}}%
\pgfpathlineto{\pgfqpoint{2.223382in}{4.037333in}}%
\pgfpathlineto{\pgfqpoint{2.101256in}{4.129390in}}%
\pgfpathlineto{\pgfqpoint{2.042505in}{4.173243in}}%
\pgfpathlineto{\pgfqpoint{1.973037in}{4.224000in}}%
\pgfpathlineto{\pgfqpoint{1.965487in}{4.224000in}}%
\pgfpathlineto{\pgfqpoint{2.242909in}{4.016946in}}%
\pgfpathlineto{\pgfqpoint{2.376438in}{3.912958in}}%
\pgfpathlineto{\pgfqpoint{2.523475in}{3.795077in}}%
\pgfpathlineto{\pgfqpoint{2.661840in}{3.680881in}}%
\pgfpathlineto{\pgfqpoint{2.726889in}{3.626667in}}%
\pgfpathlineto{\pgfqpoint{2.884202in}{3.491998in}}%
\pgfpathlineto{\pgfqpoint{3.028359in}{3.365333in}}%
\pgfpathlineto{\pgfqpoint{3.315149in}{3.104000in}}%
\pgfpathlineto{\pgfqpoint{3.605657in}{2.826243in}}%
\pgfpathlineto{\pgfqpoint{3.898096in}{2.532940in}}%
\pgfpathlineto{\pgfqpoint{4.031053in}{2.394667in}}%
\pgfpathlineto{\pgfqpoint{4.166788in}{2.250393in}}%
\pgfpathlineto{\pgfqpoint{4.287030in}{2.119604in}}%
\pgfpathlineto{\pgfqpoint{4.418151in}{1.973867in}}%
\pgfpathlineto{\pgfqpoint{4.540373in}{1.834667in}}%
\pgfpathlineto{\pgfqpoint{4.647758in}{1.709743in}}%
\pgfpathlineto{\pgfqpoint{4.768000in}{1.566677in}}%
\pgfpathlineto{\pgfqpoint{4.768000in}{1.566677in}}%
\pgfusepath{fill}%
\end{pgfscope}%
\begin{pgfscope}%
\pgfpathrectangle{\pgfqpoint{0.800000in}{0.528000in}}{\pgfqpoint{3.968000in}{3.696000in}}%
\pgfusepath{clip}%
\pgfsetbuttcap%
\pgfsetroundjoin%
\definecolor{currentfill}{rgb}{0.260571,0.246922,0.522828}%
\pgfsetfillcolor{currentfill}%
\pgfsetlinewidth{0.000000pt}%
\definecolor{currentstroke}{rgb}{0.000000,0.000000,0.000000}%
\pgfsetstrokecolor{currentstroke}%
\pgfsetdash{}{0pt}%
\pgfpathmoveto{\pgfqpoint{2.084355in}{0.528000in}}%
\pgfpathlineto{\pgfqpoint{1.962343in}{0.650843in}}%
\pgfpathlineto{\pgfqpoint{1.827654in}{0.789333in}}%
\pgfpathlineto{\pgfqpoint{1.685671in}{0.938667in}}%
\pgfpathlineto{\pgfqpoint{1.401212in}{1.249326in}}%
\pgfpathlineto{\pgfqpoint{1.137833in}{1.552008in}}%
\pgfpathlineto{\pgfqpoint{1.080566in}{1.619461in}}%
\pgfpathlineto{\pgfqpoint{0.840081in}{1.913165in}}%
\pgfpathlineto{\pgfqpoint{0.800000in}{1.963721in}}%
\pgfpathlineto{\pgfqpoint{0.800000in}{1.957077in}}%
\pgfpathlineto{\pgfqpoint{0.897986in}{1.834667in}}%
\pgfpathlineto{\pgfqpoint{1.160727in}{1.518681in}}%
\pgfpathlineto{\pgfqpoint{1.280970in}{1.379479in}}%
\pgfpathlineto{\pgfqpoint{1.406658in}{1.237333in}}%
\pgfpathlineto{\pgfqpoint{1.521455in}{1.110357in}}%
\pgfpathlineto{\pgfqpoint{1.645418in}{0.976000in}}%
\pgfpathlineto{\pgfqpoint{1.930931in}{0.677333in}}%
\pgfpathlineto{\pgfqpoint{2.078895in}{0.528000in}}%
\pgfpathlineto{\pgfqpoint{2.082586in}{0.528000in}}%
\pgfpathmoveto{\pgfqpoint{4.768000in}{1.579252in}}%
\pgfpathlineto{\pgfqpoint{4.645425in}{1.724840in}}%
\pgfpathlineto{\pgfqpoint{4.518335in}{1.872000in}}%
\pgfpathlineto{\pgfqpoint{4.407273in}{1.997808in}}%
\pgfpathlineto{\pgfqpoint{4.274386in}{2.145111in}}%
\pgfpathlineto{\pgfqpoint{4.147268in}{2.282667in}}%
\pgfpathlineto{\pgfqpoint{3.846141in}{2.597083in}}%
\pgfpathlineto{\pgfqpoint{3.561712in}{2.880000in}}%
\pgfpathlineto{\pgfqpoint{3.405253in}{3.030098in}}%
\pgfpathlineto{\pgfqpoint{3.244929in}{3.179917in}}%
\pgfpathlineto{\pgfqpoint{3.084606in}{3.325802in}}%
\pgfpathlineto{\pgfqpoint{2.763960in}{3.605865in}}%
\pgfpathlineto{\pgfqpoint{2.605479in}{3.738667in}}%
\pgfpathlineto{\pgfqpoint{2.374306in}{3.925333in}}%
\pgfpathlineto{\pgfqpoint{2.132162in}{4.112000in}}%
\pgfpathlineto{\pgfqpoint{1.980586in}{4.224000in}}%
\pgfpathlineto{\pgfqpoint{1.973037in}{4.224000in}}%
\pgfpathlineto{\pgfqpoint{2.234002in}{4.029037in}}%
\pgfpathlineto{\pgfqpoint{2.282990in}{3.991446in}}%
\pgfpathlineto{\pgfqpoint{2.563556in}{3.767761in}}%
\pgfpathlineto{\pgfqpoint{2.688812in}{3.664000in}}%
\pgfpathlineto{\pgfqpoint{2.992279in}{3.402667in}}%
\pgfpathlineto{\pgfqpoint{3.124687in}{3.284341in}}%
\pgfpathlineto{\pgfqpoint{3.280836in}{3.141333in}}%
\pgfpathlineto{\pgfqpoint{3.405253in}{3.024698in}}%
\pgfpathlineto{\pgfqpoint{3.556101in}{2.880000in}}%
\pgfpathlineto{\pgfqpoint{3.846141in}{2.591586in}}%
\pgfpathlineto{\pgfqpoint{3.966384in}{2.467957in}}%
\pgfpathlineto{\pgfqpoint{4.246949in}{2.169343in}}%
\pgfpathlineto{\pgfqpoint{4.380992in}{2.021333in}}%
\pgfpathlineto{\pgfqpoint{4.487434in}{1.901330in}}%
\pgfpathlineto{\pgfqpoint{4.619790in}{1.748717in}}%
\pgfpathlineto{\pgfqpoint{4.736558in}{1.610667in}}%
\pgfpathlineto{\pgfqpoint{4.768000in}{1.573036in}}%
\pgfpathlineto{\pgfqpoint{4.768000in}{1.573333in}}%
\pgfusepath{fill}%
\end{pgfscope}%
\begin{pgfscope}%
\pgfpathrectangle{\pgfqpoint{0.800000in}{0.528000in}}{\pgfqpoint{3.968000in}{3.696000in}}%
\pgfusepath{clip}%
\pgfsetbuttcap%
\pgfsetroundjoin%
\definecolor{currentfill}{rgb}{0.260571,0.246922,0.522828}%
\pgfsetfillcolor{currentfill}%
\pgfsetlinewidth{0.000000pt}%
\definecolor{currentstroke}{rgb}{0.000000,0.000000,0.000000}%
\pgfsetstrokecolor{currentstroke}%
\pgfsetdash{}{0pt}%
\pgfpathmoveto{\pgfqpoint{2.078895in}{0.528000in}}%
\pgfpathlineto{\pgfqpoint{1.930931in}{0.677333in}}%
\pgfpathlineto{\pgfqpoint{1.641697in}{0.979987in}}%
\pgfpathlineto{\pgfqpoint{1.361131in}{1.288478in}}%
\pgfpathlineto{\pgfqpoint{1.090433in}{1.601476in}}%
\pgfpathlineto{\pgfqpoint{0.989468in}{1.722667in}}%
\pgfpathlineto{\pgfqpoint{0.880162in}{1.856719in}}%
\pgfpathlineto{\pgfqpoint{0.800000in}{1.957077in}}%
\pgfpathlineto{\pgfqpoint{0.800000in}{1.950433in}}%
\pgfpathlineto{\pgfqpoint{0.892765in}{1.834667in}}%
\pgfpathlineto{\pgfqpoint{1.160727in}{1.512587in}}%
\pgfpathlineto{\pgfqpoint{1.280970in}{1.373549in}}%
\pgfpathlineto{\pgfqpoint{1.402211in}{1.236403in}}%
\pgfpathlineto{\pgfqpoint{1.536661in}{1.088000in}}%
\pgfpathlineto{\pgfqpoint{1.675083in}{0.938667in}}%
\pgfpathlineto{\pgfqpoint{1.802020in}{0.804867in}}%
\pgfpathlineto{\pgfqpoint{1.925496in}{0.677333in}}%
\pgfpathlineto{\pgfqpoint{2.073476in}{0.528000in}}%
\pgfpathmoveto{\pgfqpoint{4.768000in}{1.585462in}}%
\pgfpathlineto{\pgfqpoint{4.647758in}{1.728157in}}%
\pgfpathlineto{\pgfqpoint{4.506318in}{1.891744in}}%
\pgfpathlineto{\pgfqpoint{4.391464in}{2.021333in}}%
\pgfpathlineto{\pgfqpoint{4.117636in}{2.320000in}}%
\pgfpathlineto{\pgfqpoint{3.975706in}{2.469333in}}%
\pgfpathlineto{\pgfqpoint{3.681405in}{2.768000in}}%
\pgfpathlineto{\pgfqpoint{3.365172in}{3.073220in}}%
\pgfpathlineto{\pgfqpoint{3.044525in}{3.366971in}}%
\pgfpathlineto{\pgfqpoint{2.919405in}{3.477333in}}%
\pgfpathlineto{\pgfqpoint{2.763960in}{3.611166in}}%
\pgfpathlineto{\pgfqpoint{2.611847in}{3.738667in}}%
\pgfpathlineto{\pgfqpoint{2.363152in}{3.939494in}}%
\pgfpathlineto{\pgfqpoint{2.237539in}{4.037333in}}%
\pgfpathlineto{\pgfqpoint{2.082586in}{4.154537in}}%
\pgfpathlineto{\pgfqpoint{1.988135in}{4.224000in}}%
\pgfpathlineto{\pgfqpoint{1.980586in}{4.224000in}}%
\pgfpathlineto{\pgfqpoint{2.230461in}{4.037333in}}%
\pgfpathlineto{\pgfqpoint{2.467947in}{3.850667in}}%
\pgfpathlineto{\pgfqpoint{2.723879in}{3.639800in}}%
\pgfpathlineto{\pgfqpoint{2.870302in}{3.514667in}}%
\pgfpathlineto{\pgfqpoint{3.124687in}{3.289705in}}%
\pgfpathlineto{\pgfqpoint{3.265527in}{3.160519in}}%
\pgfpathlineto{\pgfqpoint{3.326690in}{3.104000in}}%
\pgfpathlineto{\pgfqpoint{3.645737in}{2.797786in}}%
\pgfpathlineto{\pgfqpoint{3.934333in}{2.506667in}}%
\pgfpathlineto{\pgfqpoint{4.077159in}{2.357333in}}%
\pgfpathlineto{\pgfqpoint{4.367192in}{2.042582in}}%
\pgfpathlineto{\pgfqpoint{4.615297in}{1.760000in}}%
\pgfpathlineto{\pgfqpoint{4.727919in}{1.627188in}}%
\pgfpathlineto{\pgfqpoint{4.768000in}{1.579252in}}%
\pgfpathlineto{\pgfqpoint{4.768000in}{1.579252in}}%
\pgfusepath{fill}%
\end{pgfscope}%
\begin{pgfscope}%
\pgfpathrectangle{\pgfqpoint{0.800000in}{0.528000in}}{\pgfqpoint{3.968000in}{3.696000in}}%
\pgfusepath{clip}%
\pgfsetbuttcap%
\pgfsetroundjoin%
\definecolor{currentfill}{rgb}{0.260571,0.246922,0.522828}%
\pgfsetfillcolor{currentfill}%
\pgfsetlinewidth{0.000000pt}%
\definecolor{currentstroke}{rgb}{0.000000,0.000000,0.000000}%
\pgfsetstrokecolor{currentstroke}%
\pgfsetdash{}{0pt}%
\pgfpathmoveto{\pgfqpoint{2.073476in}{0.528000in}}%
\pgfpathlineto{\pgfqpoint{1.925496in}{0.677333in}}%
\pgfpathlineto{\pgfqpoint{1.640110in}{0.976000in}}%
\pgfpathlineto{\pgfqpoint{1.361131in}{1.282565in}}%
\pgfpathlineto{\pgfqpoint{1.109061in}{1.573333in}}%
\pgfpathlineto{\pgfqpoint{1.000404in}{1.703146in}}%
\pgfpathlineto{\pgfqpoint{0.800000in}{1.950433in}}%
\pgfpathlineto{\pgfqpoint{0.800000in}{1.943860in}}%
\pgfpathlineto{\pgfqpoint{0.920242in}{1.794349in}}%
\pgfpathlineto{\pgfqpoint{1.041000in}{1.648000in}}%
\pgfpathlineto{\pgfqpoint{1.321051in}{1.321987in}}%
\pgfpathlineto{\pgfqpoint{1.441293in}{1.187158in}}%
\pgfpathlineto{\pgfqpoint{1.721859in}{0.883449in}}%
\pgfpathlineto{\pgfqpoint{2.002424in}{0.593800in}}%
\pgfpathlineto{\pgfqpoint{2.068057in}{0.528000in}}%
\pgfpathmoveto{\pgfqpoint{4.768000in}{1.591671in}}%
\pgfpathlineto{\pgfqpoint{4.647758in}{1.734196in}}%
\pgfpathlineto{\pgfqpoint{4.527515in}{1.873478in}}%
\pgfpathlineto{\pgfqpoint{4.396699in}{2.021333in}}%
\pgfpathlineto{\pgfqpoint{4.096508in}{2.348129in}}%
\pgfpathlineto{\pgfqpoint{3.974031in}{2.476456in}}%
\pgfpathlineto{\pgfqpoint{3.908784in}{2.544000in}}%
\pgfpathlineto{\pgfqpoint{3.765980in}{2.689112in}}%
\pgfpathlineto{\pgfqpoint{3.645737in}{2.808637in}}%
\pgfpathlineto{\pgfqpoint{3.325091in}{3.116082in}}%
\pgfpathlineto{\pgfqpoint{3.170442in}{3.258619in}}%
\pgfpathlineto{\pgfqpoint{3.124687in}{3.300240in}}%
\pgfpathlineto{\pgfqpoint{2.964364in}{3.443236in}}%
\pgfpathlineto{\pgfqpoint{2.643717in}{3.717521in}}%
\pgfpathlineto{\pgfqpoint{2.497331in}{3.837685in}}%
\pgfpathlineto{\pgfqpoint{2.363152in}{3.944831in}}%
\pgfpathlineto{\pgfqpoint{2.235042in}{4.044661in}}%
\pgfpathlineto{\pgfqpoint{2.082586in}{4.159932in}}%
\pgfpathlineto{\pgfqpoint{1.995685in}{4.224000in}}%
\pgfpathlineto{\pgfqpoint{1.988135in}{4.224000in}}%
\pgfpathlineto{\pgfqpoint{2.202828in}{4.063896in}}%
\pgfpathlineto{\pgfqpoint{2.333676in}{3.962667in}}%
\pgfpathlineto{\pgfqpoint{2.483394in}{3.843603in}}%
\pgfpathlineto{\pgfqpoint{2.611847in}{3.738667in}}%
\pgfpathlineto{\pgfqpoint{2.763960in}{3.611166in}}%
\pgfpathlineto{\pgfqpoint{3.088046in}{3.328000in}}%
\pgfpathlineto{\pgfqpoint{3.244929in}{3.185195in}}%
\pgfpathlineto{\pgfqpoint{3.372125in}{3.066667in}}%
\pgfpathlineto{\pgfqpoint{3.525495in}{2.920471in}}%
\pgfpathlineto{\pgfqpoint{3.645737in}{2.803245in}}%
\pgfpathlineto{\pgfqpoint{3.939668in}{2.506667in}}%
\pgfpathlineto{\pgfqpoint{4.082509in}{2.357333in}}%
\pgfpathlineto{\pgfqpoint{4.367192in}{2.048427in}}%
\pgfpathlineto{\pgfqpoint{4.620472in}{1.760000in}}%
\pgfpathlineto{\pgfqpoint{4.727919in}{1.633388in}}%
\pgfpathlineto{\pgfqpoint{4.768000in}{1.585462in}}%
\pgfpathlineto{\pgfqpoint{4.768000in}{1.585462in}}%
\pgfusepath{fill}%
\end{pgfscope}%
\begin{pgfscope}%
\pgfpathrectangle{\pgfqpoint{0.800000in}{0.528000in}}{\pgfqpoint{3.968000in}{3.696000in}}%
\pgfusepath{clip}%
\pgfsetbuttcap%
\pgfsetroundjoin%
\definecolor{currentfill}{rgb}{0.260571,0.246922,0.522828}%
\pgfsetfillcolor{currentfill}%
\pgfsetlinewidth{0.000000pt}%
\definecolor{currentstroke}{rgb}{0.000000,0.000000,0.000000}%
\pgfsetstrokecolor{currentstroke}%
\pgfsetdash{}{0pt}%
\pgfpathmoveto{\pgfqpoint{2.068057in}{0.528000in}}%
\pgfpathlineto{\pgfqpoint{1.920110in}{0.677333in}}%
\pgfpathlineto{\pgfqpoint{1.775850in}{0.826667in}}%
\pgfpathlineto{\pgfqpoint{1.481374in}{1.142865in}}%
\pgfpathlineto{\pgfqpoint{1.218130in}{1.440134in}}%
\pgfpathlineto{\pgfqpoint{1.160727in}{1.506493in}}%
\pgfpathlineto{\pgfqpoint{0.917809in}{1.797333in}}%
\pgfpathlineto{\pgfqpoint{0.800000in}{1.943860in}}%
\pgfpathlineto{\pgfqpoint{0.800000in}{1.937379in}}%
\pgfpathlineto{\pgfqpoint{0.920242in}{1.788053in}}%
\pgfpathlineto{\pgfqpoint{1.040485in}{1.642481in}}%
\pgfpathlineto{\pgfqpoint{1.305123in}{1.334498in}}%
\pgfpathlineto{\pgfqpoint{1.361131in}{1.270825in}}%
\pgfpathlineto{\pgfqpoint{1.641697in}{0.963090in}}%
\pgfpathlineto{\pgfqpoint{1.922263in}{0.669695in}}%
\pgfpathlineto{\pgfqpoint{2.062637in}{0.528000in}}%
\pgfpathmoveto{\pgfqpoint{4.768000in}{1.597881in}}%
\pgfpathlineto{\pgfqpoint{4.647758in}{1.740236in}}%
\pgfpathlineto{\pgfqpoint{4.527515in}{1.879357in}}%
\pgfpathlineto{\pgfqpoint{4.401935in}{2.021333in}}%
\pgfpathlineto{\pgfqpoint{4.126707in}{2.321643in}}%
\pgfpathlineto{\pgfqpoint{3.986323in}{2.469333in}}%
\pgfpathlineto{\pgfqpoint{3.685818in}{2.774434in}}%
\pgfpathlineto{\pgfqpoint{3.365172in}{3.083819in}}%
\pgfpathlineto{\pgfqpoint{3.044525in}{3.377458in}}%
\pgfpathlineto{\pgfqpoint{2.884202in}{3.518527in}}%
\pgfpathlineto{\pgfqpoint{2.758175in}{3.626667in}}%
\pgfpathlineto{\pgfqpoint{2.603636in}{3.756010in}}%
\pgfpathlineto{\pgfqpoint{2.454986in}{3.877127in}}%
\pgfpathlineto{\pgfqpoint{2.323071in}{3.981652in}}%
\pgfpathlineto{\pgfqpoint{2.024970in}{4.207667in}}%
\pgfpathlineto{\pgfqpoint{2.002424in}{4.224000in}}%
\pgfpathlineto{\pgfqpoint{1.995685in}{4.224000in}}%
\pgfpathlineto{\pgfqpoint{2.202828in}{4.069312in}}%
\pgfpathlineto{\pgfqpoint{2.340464in}{3.962667in}}%
\pgfpathlineto{\pgfqpoint{2.483394in}{3.848961in}}%
\pgfpathlineto{\pgfqpoint{2.618214in}{3.738667in}}%
\pgfpathlineto{\pgfqpoint{2.763960in}{3.616466in}}%
\pgfpathlineto{\pgfqpoint{3.084606in}{3.336347in}}%
\pgfpathlineto{\pgfqpoint{3.377748in}{3.066667in}}%
\pgfpathlineto{\pgfqpoint{3.525495in}{2.925799in}}%
\pgfpathlineto{\pgfqpoint{3.649088in}{2.805333in}}%
\pgfpathlineto{\pgfqpoint{3.966384in}{2.484545in}}%
\pgfpathlineto{\pgfqpoint{4.246949in}{2.186453in}}%
\pgfpathlineto{\pgfqpoint{4.367192in}{2.054271in}}%
\pgfpathlineto{\pgfqpoint{4.647758in}{1.734196in}}%
\pgfpathlineto{\pgfqpoint{4.768000in}{1.591671in}}%
\pgfpathlineto{\pgfqpoint{4.768000in}{1.591671in}}%
\pgfusepath{fill}%
\end{pgfscope}%
\begin{pgfscope}%
\pgfpathrectangle{\pgfqpoint{0.800000in}{0.528000in}}{\pgfqpoint{3.968000in}{3.696000in}}%
\pgfusepath{clip}%
\pgfsetbuttcap%
\pgfsetroundjoin%
\definecolor{currentfill}{rgb}{0.258965,0.251537,0.524736}%
\pgfsetfillcolor{currentfill}%
\pgfsetlinewidth{0.000000pt}%
\definecolor{currentstroke}{rgb}{0.000000,0.000000,0.000000}%
\pgfsetstrokecolor{currentstroke}%
\pgfsetdash{}{0pt}%
\pgfpathmoveto{\pgfqpoint{2.062637in}{0.528000in}}%
\pgfpathlineto{\pgfqpoint{1.922263in}{0.669695in}}%
\pgfpathlineto{\pgfqpoint{1.641697in}{0.963090in}}%
\pgfpathlineto{\pgfqpoint{1.521455in}{1.093107in}}%
\pgfpathlineto{\pgfqpoint{1.240889in}{1.407612in}}%
\pgfpathlineto{\pgfqpoint{1.120646in}{1.547447in}}%
\pgfpathlineto{\pgfqpoint{0.880162in}{1.837338in}}%
\pgfpathlineto{\pgfqpoint{0.800000in}{1.937379in}}%
\pgfpathlineto{\pgfqpoint{0.800000in}{1.930897in}}%
\pgfpathlineto{\pgfqpoint{0.920242in}{1.781756in}}%
\pgfpathlineto{\pgfqpoint{1.040485in}{1.636360in}}%
\pgfpathlineto{\pgfqpoint{1.286608in}{1.349333in}}%
\pgfpathlineto{\pgfqpoint{1.561535in}{1.043799in}}%
\pgfpathlineto{\pgfqpoint{1.694420in}{0.901333in}}%
\pgfpathlineto{\pgfqpoint{1.836893in}{0.752000in}}%
\pgfpathlineto{\pgfqpoint{2.057218in}{0.528000in}}%
\pgfpathmoveto{\pgfqpoint{4.768000in}{1.604090in}}%
\pgfpathlineto{\pgfqpoint{4.647758in}{1.746275in}}%
\pgfpathlineto{\pgfqpoint{4.527515in}{1.885236in}}%
\pgfpathlineto{\pgfqpoint{4.407171in}{2.021333in}}%
\pgfpathlineto{\pgfqpoint{4.126707in}{2.327194in}}%
\pgfpathlineto{\pgfqpoint{3.991632in}{2.469333in}}%
\pgfpathlineto{\pgfqpoint{3.685818in}{2.779791in}}%
\pgfpathlineto{\pgfqpoint{3.365172in}{3.089119in}}%
\pgfpathlineto{\pgfqpoint{3.044525in}{3.382702in}}%
\pgfpathlineto{\pgfqpoint{2.894635in}{3.514667in}}%
\pgfpathlineto{\pgfqpoint{2.630948in}{3.738667in}}%
\pgfpathlineto{\pgfqpoint{2.354040in}{3.962667in}}%
\pgfpathlineto{\pgfqpoint{2.202828in}{4.080036in}}%
\pgfpathlineto{\pgfqpoint{2.010526in}{4.224000in}}%
\pgfpathlineto{\pgfqpoint{2.003209in}{4.224000in}}%
\pgfpathlineto{\pgfqpoint{2.282990in}{4.012907in}}%
\pgfpathlineto{\pgfqpoint{2.579353in}{3.776000in}}%
\pgfpathlineto{\pgfqpoint{2.723879in}{3.655679in}}%
\pgfpathlineto{\pgfqpoint{2.845517in}{3.552000in}}%
\pgfpathlineto{\pgfqpoint{3.004444in}{3.413081in}}%
\pgfpathlineto{\pgfqpoint{3.141057in}{3.290667in}}%
\pgfpathlineto{\pgfqpoint{3.285010in}{3.158686in}}%
\pgfpathlineto{\pgfqpoint{3.578245in}{2.880000in}}%
\pgfpathlineto{\pgfqpoint{3.727934in}{2.732562in}}%
\pgfpathlineto{\pgfqpoint{3.767209in}{2.693333in}}%
\pgfpathlineto{\pgfqpoint{4.057683in}{2.394667in}}%
\pgfpathlineto{\pgfqpoint{4.197846in}{2.245333in}}%
\pgfpathlineto{\pgfqpoint{4.487434in}{1.924988in}}%
\pgfpathlineto{\pgfqpoint{4.727919in}{1.645788in}}%
\pgfpathlineto{\pgfqpoint{4.768000in}{1.597881in}}%
\pgfpathlineto{\pgfqpoint{4.768000in}{1.597881in}}%
\pgfusepath{fill}%
\end{pgfscope}%
\begin{pgfscope}%
\pgfpathrectangle{\pgfqpoint{0.800000in}{0.528000in}}{\pgfqpoint{3.968000in}{3.696000in}}%
\pgfusepath{clip}%
\pgfsetbuttcap%
\pgfsetroundjoin%
\definecolor{currentfill}{rgb}{0.258965,0.251537,0.524736}%
\pgfsetfillcolor{currentfill}%
\pgfsetlinewidth{0.000000pt}%
\definecolor{currentstroke}{rgb}{0.000000,0.000000,0.000000}%
\pgfsetstrokecolor{currentstroke}%
\pgfsetdash{}{0pt}%
\pgfpathmoveto{\pgfqpoint{2.057218in}{0.528000in}}%
\pgfpathlineto{\pgfqpoint{1.922263in}{0.664260in}}%
\pgfpathlineto{\pgfqpoint{1.641697in}{0.957486in}}%
\pgfpathlineto{\pgfqpoint{1.520878in}{1.088000in}}%
\pgfpathlineto{\pgfqpoint{1.240889in}{1.401673in}}%
\pgfpathlineto{\pgfqpoint{1.120646in}{1.541344in}}%
\pgfpathlineto{\pgfqpoint{0.868005in}{1.845990in}}%
\pgfpathlineto{\pgfqpoint{0.800000in}{1.930897in}}%
\pgfpathlineto{\pgfqpoint{0.800000in}{1.924416in}}%
\pgfpathlineto{\pgfqpoint{0.920242in}{1.775460in}}%
\pgfpathlineto{\pgfqpoint{1.040485in}{1.630238in}}%
\pgfpathlineto{\pgfqpoint{1.281404in}{1.349333in}}%
\pgfpathlineto{\pgfqpoint{1.561535in}{1.038179in}}%
\pgfpathlineto{\pgfqpoint{1.689144in}{0.901333in}}%
\pgfpathlineto{\pgfqpoint{1.831631in}{0.752000in}}%
\pgfpathlineto{\pgfqpoint{2.051798in}{0.528000in}}%
\pgfpathmoveto{\pgfqpoint{4.768000in}{1.610300in}}%
\pgfpathlineto{\pgfqpoint{4.673041in}{1.722667in}}%
\pgfpathlineto{\pgfqpoint{4.567596in}{1.845203in}}%
\pgfpathlineto{\pgfqpoint{4.445618in}{1.984000in}}%
\pgfpathlineto{\pgfqpoint{4.166788in}{2.289978in}}%
\pgfpathlineto{\pgfqpoint{3.886222in}{2.583631in}}%
\pgfpathlineto{\pgfqpoint{3.725899in}{2.745356in}}%
\pgfpathlineto{\pgfqpoint{3.589207in}{2.880000in}}%
\pgfpathlineto{\pgfqpoint{3.285010in}{3.169257in}}%
\pgfpathlineto{\pgfqpoint{2.964364in}{3.458926in}}%
\pgfpathlineto{\pgfqpoint{2.643717in}{3.733359in}}%
\pgfpathlineto{\pgfqpoint{2.500774in}{3.850667in}}%
\pgfpathlineto{\pgfqpoint{2.242909in}{4.054562in}}%
\pgfpathlineto{\pgfqpoint{2.118429in}{4.149333in}}%
\pgfpathlineto{\pgfqpoint{2.017843in}{4.224000in}}%
\pgfpathlineto{\pgfqpoint{2.010526in}{4.224000in}}%
\pgfpathlineto{\pgfqpoint{2.282990in}{4.018229in}}%
\pgfpathlineto{\pgfqpoint{2.585757in}{3.776000in}}%
\pgfpathlineto{\pgfqpoint{2.723879in}{3.660973in}}%
\pgfpathlineto{\pgfqpoint{2.851546in}{3.552000in}}%
\pgfpathlineto{\pgfqpoint{3.004444in}{3.418318in}}%
\pgfpathlineto{\pgfqpoint{3.146860in}{3.290667in}}%
\pgfpathlineto{\pgfqpoint{3.285010in}{3.163971in}}%
\pgfpathlineto{\pgfqpoint{3.605657in}{2.858626in}}%
\pgfpathlineto{\pgfqpoint{3.886222in}{2.578174in}}%
\pgfpathlineto{\pgfqpoint{4.027387in}{2.432000in}}%
\pgfpathlineto{\pgfqpoint{4.168423in}{2.282667in}}%
\pgfpathlineto{\pgfqpoint{4.305966in}{2.133333in}}%
\pgfpathlineto{\pgfqpoint{4.440407in}{1.984000in}}%
\pgfpathlineto{\pgfqpoint{4.567596in}{1.839316in}}%
\pgfpathlineto{\pgfqpoint{4.768000in}{1.604090in}}%
\pgfpathlineto{\pgfqpoint{4.768000in}{1.604090in}}%
\pgfusepath{fill}%
\end{pgfscope}%
\begin{pgfscope}%
\pgfpathrectangle{\pgfqpoint{0.800000in}{0.528000in}}{\pgfqpoint{3.968000in}{3.696000in}}%
\pgfusepath{clip}%
\pgfsetbuttcap%
\pgfsetroundjoin%
\definecolor{currentfill}{rgb}{0.258965,0.251537,0.524736}%
\pgfsetfillcolor{currentfill}%
\pgfsetlinewidth{0.000000pt}%
\definecolor{currentstroke}{rgb}{0.000000,0.000000,0.000000}%
\pgfsetstrokecolor{currentstroke}%
\pgfsetdash{}{0pt}%
\pgfpathmoveto{\pgfqpoint{2.051798in}{0.528000in}}%
\pgfpathlineto{\pgfqpoint{1.922263in}{0.658824in}}%
\pgfpathlineto{\pgfqpoint{1.641697in}{0.951882in}}%
\pgfpathlineto{\pgfqpoint{1.515727in}{1.088000in}}%
\pgfpathlineto{\pgfqpoint{1.240889in}{1.395734in}}%
\pgfpathlineto{\pgfqpoint{1.120014in}{1.536000in}}%
\pgfpathlineto{\pgfqpoint{0.872055in}{1.834667in}}%
\pgfpathlineto{\pgfqpoint{0.800000in}{1.924416in}}%
\pgfpathlineto{\pgfqpoint{0.800000in}{1.917935in}}%
\pgfpathlineto{\pgfqpoint{0.897274in}{1.797333in}}%
\pgfpathlineto{\pgfqpoint{1.160727in}{1.482490in}}%
\pgfpathlineto{\pgfqpoint{1.423875in}{1.183776in}}%
\pgfpathlineto{\pgfqpoint{1.481374in}{1.119948in}}%
\pgfpathlineto{\pgfqpoint{1.761939in}{0.819073in}}%
\pgfpathlineto{\pgfqpoint{1.898857in}{0.677333in}}%
\pgfpathlineto{\pgfqpoint{2.046379in}{0.528000in}}%
\pgfpathmoveto{\pgfqpoint{4.768000in}{1.616375in}}%
\pgfpathlineto{\pgfqpoint{4.503365in}{1.924172in}}%
\pgfpathlineto{\pgfqpoint{4.447354in}{1.987824in}}%
\pgfpathlineto{\pgfqpoint{4.166788in}{2.295537in}}%
\pgfpathlineto{\pgfqpoint{3.886222in}{2.589024in}}%
\pgfpathlineto{\pgfqpoint{3.736196in}{2.740258in}}%
\pgfpathlineto{\pgfqpoint{3.670794in}{2.805333in}}%
\pgfpathlineto{\pgfqpoint{3.525495in}{2.947110in}}%
\pgfpathlineto{\pgfqpoint{3.222199in}{3.232161in}}%
\pgfpathlineto{\pgfqpoint{3.158467in}{3.290667in}}%
\pgfpathlineto{\pgfqpoint{2.844121in}{3.568835in}}%
\pgfpathlineto{\pgfqpoint{2.523475in}{3.837507in}}%
\pgfpathlineto{\pgfqpoint{2.367492in}{3.962667in}}%
\pgfpathlineto{\pgfqpoint{2.162747in}{4.121199in}}%
\pgfpathlineto{\pgfqpoint{2.025160in}{4.224000in}}%
\pgfpathlineto{\pgfqpoint{2.017843in}{4.224000in}}%
\pgfpathlineto{\pgfqpoint{2.265190in}{4.037333in}}%
\pgfpathlineto{\pgfqpoint{2.403232in}{3.929062in}}%
\pgfpathlineto{\pgfqpoint{2.726503in}{3.664000in}}%
\pgfpathlineto{\pgfqpoint{2.884202in}{3.528961in}}%
\pgfpathlineto{\pgfqpoint{3.204848in}{3.243119in}}%
\pgfpathlineto{\pgfqpoint{3.525495in}{2.941782in}}%
\pgfpathlineto{\pgfqpoint{3.665368in}{2.805333in}}%
\pgfpathlineto{\pgfqpoint{3.966384in}{2.501105in}}%
\pgfpathlineto{\pgfqpoint{4.246949in}{2.203534in}}%
\pgfpathlineto{\pgfqpoint{4.378762in}{2.058667in}}%
\pgfpathlineto{\pgfqpoint{4.647758in}{1.752315in}}%
\pgfpathlineto{\pgfqpoint{4.768000in}{1.610300in}}%
\pgfpathlineto{\pgfqpoint{4.768000in}{1.610667in}}%
\pgfusepath{fill}%
\end{pgfscope}%
\begin{pgfscope}%
\pgfpathrectangle{\pgfqpoint{0.800000in}{0.528000in}}{\pgfqpoint{3.968000in}{3.696000in}}%
\pgfusepath{clip}%
\pgfsetbuttcap%
\pgfsetroundjoin%
\definecolor{currentfill}{rgb}{0.257322,0.256130,0.526563}%
\pgfsetfillcolor{currentfill}%
\pgfsetlinewidth{0.000000pt}%
\definecolor{currentstroke}{rgb}{0.000000,0.000000,0.000000}%
\pgfsetstrokecolor{currentstroke}%
\pgfsetdash{}{0pt}%
\pgfpathmoveto{\pgfqpoint{2.046379in}{0.528000in}}%
\pgfpathlineto{\pgfqpoint{1.922263in}{0.653388in}}%
\pgfpathlineto{\pgfqpoint{1.641697in}{0.946278in}}%
\pgfpathlineto{\pgfqpoint{1.510575in}{1.088000in}}%
\pgfpathlineto{\pgfqpoint{1.240889in}{1.389795in}}%
\pgfpathlineto{\pgfqpoint{1.114932in}{1.536000in}}%
\pgfpathlineto{\pgfqpoint{0.866946in}{1.834667in}}%
\pgfpathlineto{\pgfqpoint{0.800000in}{1.917935in}}%
\pgfpathlineto{\pgfqpoint{0.800000in}{1.911453in}}%
\pgfpathlineto{\pgfqpoint{0.892141in}{1.797333in}}%
\pgfpathlineto{\pgfqpoint{1.160727in}{1.476534in}}%
\pgfpathlineto{\pgfqpoint{1.403786in}{1.200000in}}%
\pgfpathlineto{\pgfqpoint{1.681778in}{0.898023in}}%
\pgfpathlineto{\pgfqpoint{1.831147in}{0.741797in}}%
\pgfpathlineto{\pgfqpoint{1.893543in}{0.677333in}}%
\pgfpathlineto{\pgfqpoint{2.042505in}{0.528000in}}%
\pgfpathmoveto{\pgfqpoint{4.768000in}{1.622442in}}%
\pgfpathlineto{\pgfqpoint{4.521842in}{1.909333in}}%
\pgfpathlineto{\pgfqpoint{4.246949in}{2.214775in}}%
\pgfpathlineto{\pgfqpoint{4.113992in}{2.357333in}}%
\pgfpathlineto{\pgfqpoint{3.806061in}{2.675764in}}%
\pgfpathlineto{\pgfqpoint{3.523178in}{2.954667in}}%
\pgfpathlineto{\pgfqpoint{3.365172in}{3.104997in}}%
\pgfpathlineto{\pgfqpoint{3.204848in}{3.253656in}}%
\pgfpathlineto{\pgfqpoint{2.884202in}{3.539394in}}%
\pgfpathlineto{\pgfqpoint{2.738765in}{3.664000in}}%
\pgfpathlineto{\pgfqpoint{2.467567in}{3.888000in}}%
\pgfpathlineto{\pgfqpoint{2.323071in}{4.002911in}}%
\pgfpathlineto{\pgfqpoint{2.032477in}{4.224000in}}%
\pgfpathlineto{\pgfqpoint{2.025160in}{4.224000in}}%
\pgfpathlineto{\pgfqpoint{2.272063in}{4.037333in}}%
\pgfpathlineto{\pgfqpoint{2.367492in}{3.962667in}}%
\pgfpathlineto{\pgfqpoint{2.643717in}{3.738638in}}%
\pgfpathlineto{\pgfqpoint{2.957008in}{3.470482in}}%
\pgfpathlineto{\pgfqpoint{3.004444in}{3.428792in}}%
\pgfpathlineto{\pgfqpoint{3.164768in}{3.284951in}}%
\pgfpathlineto{\pgfqpoint{3.485414in}{2.985634in}}%
\pgfpathlineto{\pgfqpoint{3.639070in}{2.836456in}}%
\pgfpathlineto{\pgfqpoint{3.685818in}{2.790503in}}%
\pgfpathlineto{\pgfqpoint{3.820292in}{2.656000in}}%
\pgfpathlineto{\pgfqpoint{3.966384in}{2.506625in}}%
\pgfpathlineto{\pgfqpoint{4.254669in}{2.200810in}}%
\pgfpathlineto{\pgfqpoint{4.383909in}{2.058667in}}%
\pgfpathlineto{\pgfqpoint{4.647758in}{1.758354in}}%
\pgfpathlineto{\pgfqpoint{4.768000in}{1.616375in}}%
\pgfpathlineto{\pgfqpoint{4.768000in}{1.616375in}}%
\pgfusepath{fill}%
\end{pgfscope}%
\begin{pgfscope}%
\pgfpathrectangle{\pgfqpoint{0.800000in}{0.528000in}}{\pgfqpoint{3.968000in}{3.696000in}}%
\pgfusepath{clip}%
\pgfsetbuttcap%
\pgfsetroundjoin%
\definecolor{currentfill}{rgb}{0.257322,0.256130,0.526563}%
\pgfsetfillcolor{currentfill}%
\pgfsetlinewidth{0.000000pt}%
\definecolor{currentstroke}{rgb}{0.000000,0.000000,0.000000}%
\pgfsetstrokecolor{currentstroke}%
\pgfsetdash{}{0pt}%
\pgfpathmoveto{\pgfqpoint{2.040994in}{0.528000in}}%
\pgfpathlineto{\pgfqpoint{1.749488in}{0.826667in}}%
\pgfpathlineto{\pgfqpoint{1.608756in}{0.976000in}}%
\pgfpathlineto{\pgfqpoint{1.321051in}{1.292809in}}%
\pgfpathlineto{\pgfqpoint{1.200808in}{1.430044in}}%
\pgfpathlineto{\pgfqpoint{1.078103in}{1.573333in}}%
\pgfpathlineto{\pgfqpoint{0.831675in}{1.872000in}}%
\pgfpathlineto{\pgfqpoint{0.800000in}{1.911453in}}%
\pgfpathlineto{\pgfqpoint{0.800000in}{1.905076in}}%
\pgfpathlineto{\pgfqpoint{1.041498in}{1.610667in}}%
\pgfpathlineto{\pgfqpoint{1.321051in}{1.287017in}}%
\pgfpathlineto{\pgfqpoint{1.568925in}{1.013333in}}%
\pgfpathlineto{\pgfqpoint{1.708790in}{0.864000in}}%
\pgfpathlineto{\pgfqpoint{2.002424in}{0.561356in}}%
\pgfpathlineto{\pgfqpoint{2.035695in}{0.528000in}}%
\pgfpathmoveto{\pgfqpoint{4.768000in}{1.628509in}}%
\pgfpathlineto{\pgfqpoint{4.527004in}{1.909333in}}%
\pgfpathlineto{\pgfqpoint{4.246949in}{2.220349in}}%
\pgfpathlineto{\pgfqpoint{4.119225in}{2.357333in}}%
\pgfpathlineto{\pgfqpoint{3.806061in}{2.681142in}}%
\pgfpathlineto{\pgfqpoint{3.525495in}{2.957704in}}%
\pgfpathlineto{\pgfqpoint{3.365172in}{3.110193in}}%
\pgfpathlineto{\pgfqpoint{3.204848in}{3.258824in}}%
\pgfpathlineto{\pgfqpoint{2.884202in}{3.544610in}}%
\pgfpathlineto{\pgfqpoint{2.744896in}{3.664000in}}%
\pgfpathlineto{\pgfqpoint{2.474087in}{3.888000in}}%
\pgfpathlineto{\pgfqpoint{2.323071in}{4.008135in}}%
\pgfpathlineto{\pgfqpoint{2.039794in}{4.224000in}}%
\pgfpathlineto{\pgfqpoint{2.032477in}{4.224000in}}%
\pgfpathlineto{\pgfqpoint{2.242909in}{4.065192in}}%
\pgfpathlineto{\pgfqpoint{2.540151in}{3.828867in}}%
\pgfpathlineto{\pgfqpoint{2.604936in}{3.776000in}}%
\pgfpathlineto{\pgfqpoint{2.912622in}{3.514667in}}%
\pgfpathlineto{\pgfqpoint{3.044525in}{3.398434in}}%
\pgfpathlineto{\pgfqpoint{3.366232in}{3.104000in}}%
\pgfpathlineto{\pgfqpoint{3.525495in}{2.952438in}}%
\pgfpathlineto{\pgfqpoint{3.685818in}{2.795860in}}%
\pgfpathlineto{\pgfqpoint{3.825612in}{2.656000in}}%
\pgfpathlineto{\pgfqpoint{3.966384in}{2.512033in}}%
\pgfpathlineto{\pgfqpoint{4.246949in}{2.214775in}}%
\pgfpathlineto{\pgfqpoint{4.488992in}{1.946667in}}%
\pgfpathlineto{\pgfqpoint{4.768000in}{1.622442in}}%
\pgfpathlineto{\pgfqpoint{4.768000in}{1.622442in}}%
\pgfusepath{fill}%
\end{pgfscope}%
\begin{pgfscope}%
\pgfpathrectangle{\pgfqpoint{0.800000in}{0.528000in}}{\pgfqpoint{3.968000in}{3.696000in}}%
\pgfusepath{clip}%
\pgfsetbuttcap%
\pgfsetroundjoin%
\definecolor{currentfill}{rgb}{0.257322,0.256130,0.526563}%
\pgfsetfillcolor{currentfill}%
\pgfsetlinewidth{0.000000pt}%
\definecolor{currentstroke}{rgb}{0.000000,0.000000,0.000000}%
\pgfsetstrokecolor{currentstroke}%
\pgfsetdash{}{0pt}%
\pgfpathmoveto{\pgfqpoint{2.035695in}{0.528000in}}%
\pgfpathlineto{\pgfqpoint{1.721859in}{0.850209in}}%
\pgfpathlineto{\pgfqpoint{1.441293in}{1.152774in}}%
\pgfpathlineto{\pgfqpoint{1.321051in}{1.287017in}}%
\pgfpathlineto{\pgfqpoint{1.200808in}{1.424096in}}%
\pgfpathlineto{\pgfqpoint{1.073044in}{1.573333in}}%
\pgfpathlineto{\pgfqpoint{0.815351in}{1.886299in}}%
\pgfpathlineto{\pgfqpoint{0.800000in}{1.905076in}}%
\pgfpathlineto{\pgfqpoint{0.800000in}{1.898750in}}%
\pgfpathlineto{\pgfqpoint{1.040485in}{1.605861in}}%
\pgfpathlineto{\pgfqpoint{1.321051in}{1.281225in}}%
\pgfpathlineto{\pgfqpoint{1.577534in}{0.998431in}}%
\pgfpathlineto{\pgfqpoint{1.712249in}{0.855049in}}%
\pgfpathlineto{\pgfqpoint{1.774724in}{0.789333in}}%
\pgfpathlineto{\pgfqpoint{2.030397in}{0.528000in}}%
\pgfpathmoveto{\pgfqpoint{4.768000in}{1.634576in}}%
\pgfpathlineto{\pgfqpoint{4.527515in}{1.914514in}}%
\pgfpathlineto{\pgfqpoint{4.246949in}{2.225923in}}%
\pgfpathlineto{\pgfqpoint{4.124457in}{2.357333in}}%
\pgfpathlineto{\pgfqpoint{3.821632in}{2.670504in}}%
\pgfpathlineto{\pgfqpoint{3.762104in}{2.730667in}}%
\pgfpathlineto{\pgfqpoint{3.445333in}{3.039643in}}%
\pgfpathlineto{\pgfqpoint{3.124687in}{3.336868in}}%
\pgfpathlineto{\pgfqpoint{2.804040in}{3.618868in}}%
\pgfpathlineto{\pgfqpoint{2.480607in}{3.888000in}}%
\pgfpathlineto{\pgfqpoint{2.323071in}{4.013359in}}%
\pgfpathlineto{\pgfqpoint{2.046973in}{4.224000in}}%
\pgfpathlineto{\pgfqpoint{2.039794in}{4.224000in}}%
\pgfpathlineto{\pgfqpoint{2.242909in}{4.070507in}}%
\pgfpathlineto{\pgfqpoint{2.523475in}{3.848024in}}%
\pgfpathlineto{\pgfqpoint{2.832393in}{3.589333in}}%
\pgfpathlineto{\pgfqpoint{3.087352in}{3.365333in}}%
\pgfpathlineto{\pgfqpoint{3.411343in}{3.066667in}}%
\pgfpathlineto{\pgfqpoint{3.725899in}{2.761446in}}%
\pgfpathlineto{\pgfqpoint{4.028510in}{2.452534in}}%
\pgfpathlineto{\pgfqpoint{4.086626in}{2.391867in}}%
\pgfpathlineto{\pgfqpoint{4.381751in}{2.072227in}}%
\pgfpathlineto{\pgfqpoint{4.447354in}{1.999293in}}%
\pgfpathlineto{\pgfqpoint{4.567596in}{1.862866in}}%
\pgfpathlineto{\pgfqpoint{4.768000in}{1.628509in}}%
\pgfpathlineto{\pgfqpoint{4.768000in}{1.628509in}}%
\pgfusepath{fill}%
\end{pgfscope}%
\begin{pgfscope}%
\pgfpathrectangle{\pgfqpoint{0.800000in}{0.528000in}}{\pgfqpoint{3.968000in}{3.696000in}}%
\pgfusepath{clip}%
\pgfsetbuttcap%
\pgfsetroundjoin%
\definecolor{currentfill}{rgb}{0.257322,0.256130,0.526563}%
\pgfsetfillcolor{currentfill}%
\pgfsetlinewidth{0.000000pt}%
\definecolor{currentstroke}{rgb}{0.000000,0.000000,0.000000}%
\pgfsetstrokecolor{currentstroke}%
\pgfsetdash{}{0pt}%
\pgfpathmoveto{\pgfqpoint{2.030397in}{0.528000in}}%
\pgfpathlineto{\pgfqpoint{1.721859in}{0.844737in}}%
\pgfpathlineto{\pgfqpoint{1.441293in}{1.147130in}}%
\pgfpathlineto{\pgfqpoint{1.321051in}{1.281225in}}%
\pgfpathlineto{\pgfqpoint{1.175129in}{1.447919in}}%
\pgfpathlineto{\pgfqpoint{1.067985in}{1.573333in}}%
\pgfpathlineto{\pgfqpoint{0.800000in}{1.898750in}}%
\pgfpathlineto{\pgfqpoint{0.800000in}{1.892424in}}%
\pgfpathlineto{\pgfqpoint{1.040485in}{1.599878in}}%
\pgfpathlineto{\pgfqpoint{1.288747in}{1.312000in}}%
\pgfpathlineto{\pgfqpoint{1.422142in}{1.162667in}}%
\pgfpathlineto{\pgfqpoint{1.561535in}{1.010147in}}%
\pgfpathlineto{\pgfqpoint{1.709519in}{0.852506in}}%
\pgfpathlineto{\pgfqpoint{1.769488in}{0.789333in}}%
\pgfpathlineto{\pgfqpoint{2.025098in}{0.528000in}}%
\pgfpathmoveto{\pgfqpoint{4.768000in}{1.640643in}}%
\pgfpathlineto{\pgfqpoint{4.527515in}{1.920265in}}%
\pgfpathlineto{\pgfqpoint{4.246949in}{2.231496in}}%
\pgfpathlineto{\pgfqpoint{4.126707in}{2.360432in}}%
\pgfpathlineto{\pgfqpoint{3.824376in}{2.673060in}}%
\pgfpathlineto{\pgfqpoint{3.765980in}{2.732134in}}%
\pgfpathlineto{\pgfqpoint{3.445333in}{3.044852in}}%
\pgfpathlineto{\pgfqpoint{3.124687in}{3.342023in}}%
\pgfpathlineto{\pgfqpoint{2.800999in}{3.626667in}}%
\pgfpathlineto{\pgfqpoint{2.487028in}{3.888000in}}%
\pgfpathlineto{\pgfqpoint{2.251092in}{4.074667in}}%
\pgfpathlineto{\pgfqpoint{2.054072in}{4.224000in}}%
\pgfpathlineto{\pgfqpoint{2.046973in}{4.224000in}}%
\pgfpathlineto{\pgfqpoint{2.202828in}{4.106576in}}%
\pgfpathlineto{\pgfqpoint{2.340049in}{4.000000in}}%
\pgfpathlineto{\pgfqpoint{2.483394in}{3.885756in}}%
\pgfpathlineto{\pgfqpoint{2.794903in}{3.626667in}}%
\pgfpathlineto{\pgfqpoint{3.051390in}{3.402667in}}%
\pgfpathlineto{\pgfqpoint{3.365172in}{3.115388in}}%
\pgfpathlineto{\pgfqpoint{3.649140in}{2.842667in}}%
\pgfpathlineto{\pgfqpoint{3.799287in}{2.693333in}}%
\pgfpathlineto{\pgfqpoint{4.089172in}{2.394667in}}%
\pgfpathlineto{\pgfqpoint{4.384525in}{2.074811in}}%
\pgfpathlineto{\pgfqpoint{4.447354in}{2.005028in}}%
\pgfpathlineto{\pgfqpoint{4.567596in}{1.868753in}}%
\pgfpathlineto{\pgfqpoint{4.768000in}{1.634576in}}%
\pgfpathlineto{\pgfqpoint{4.768000in}{1.634576in}}%
\pgfusepath{fill}%
\end{pgfscope}%
\begin{pgfscope}%
\pgfpathrectangle{\pgfqpoint{0.800000in}{0.528000in}}{\pgfqpoint{3.968000in}{3.696000in}}%
\pgfusepath{clip}%
\pgfsetbuttcap%
\pgfsetroundjoin%
\definecolor{currentfill}{rgb}{0.255645,0.260703,0.528312}%
\pgfsetfillcolor{currentfill}%
\pgfsetlinewidth{0.000000pt}%
\definecolor{currentstroke}{rgb}{0.000000,0.000000,0.000000}%
\pgfsetstrokecolor{currentstroke}%
\pgfsetdash{}{0pt}%
\pgfpathmoveto{\pgfqpoint{2.025098in}{0.528000in}}%
\pgfpathlineto{\pgfqpoint{1.721859in}{0.839264in}}%
\pgfpathlineto{\pgfqpoint{1.441293in}{1.141486in}}%
\pgfpathlineto{\pgfqpoint{1.321051in}{1.275434in}}%
\pgfpathlineto{\pgfqpoint{1.190799in}{1.424000in}}%
\pgfpathlineto{\pgfqpoint{1.062927in}{1.573333in}}%
\pgfpathlineto{\pgfqpoint{0.800000in}{1.892424in}}%
\pgfpathlineto{\pgfqpoint{0.800000in}{1.886097in}}%
\pgfpathlineto{\pgfqpoint{1.026371in}{1.610667in}}%
\pgfpathlineto{\pgfqpoint{1.153454in}{1.461333in}}%
\pgfpathlineto{\pgfqpoint{1.441293in}{1.135842in}}%
\pgfpathlineto{\pgfqpoint{1.721859in}{0.833791in}}%
\pgfpathlineto{\pgfqpoint{1.842101in}{0.708577in}}%
\pgfpathlineto{\pgfqpoint{2.002424in}{0.545420in}}%
\pgfpathlineto{\pgfqpoint{2.019799in}{0.528000in}}%
\pgfpathmoveto{\pgfqpoint{4.768000in}{1.646711in}}%
\pgfpathlineto{\pgfqpoint{4.542168in}{1.909333in}}%
\pgfpathlineto{\pgfqpoint{4.407273in}{2.061250in}}%
\pgfpathlineto{\pgfqpoint{4.112092in}{2.381054in}}%
\pgfpathlineto{\pgfqpoint{4.046545in}{2.450291in}}%
\pgfpathlineto{\pgfqpoint{3.749968in}{2.753085in}}%
\pgfpathlineto{\pgfqpoint{3.685818in}{2.817048in}}%
\pgfpathlineto{\pgfqpoint{3.525495in}{2.973372in}}%
\pgfpathlineto{\pgfqpoint{3.376698in}{3.114736in}}%
\pgfpathlineto{\pgfqpoint{3.308546in}{3.178667in}}%
\pgfpathlineto{\pgfqpoint{3.004444in}{3.454687in}}%
\pgfpathlineto{\pgfqpoint{2.844121in}{3.594777in}}%
\pgfpathlineto{\pgfqpoint{2.539232in}{3.850667in}}%
\pgfpathlineto{\pgfqpoint{2.400488in}{3.962667in}}%
\pgfpathlineto{\pgfqpoint{2.111108in}{4.186667in}}%
\pgfpathlineto{\pgfqpoint{2.061170in}{4.224000in}}%
\pgfpathlineto{\pgfqpoint{2.054072in}{4.224000in}}%
\pgfpathlineto{\pgfqpoint{2.203534in}{4.111343in}}%
\pgfpathlineto{\pgfqpoint{2.363152in}{3.987030in}}%
\pgfpathlineto{\pgfqpoint{2.676487in}{3.731857in}}%
\pgfpathlineto{\pgfqpoint{2.723879in}{3.692169in}}%
\pgfpathlineto{\pgfqpoint{3.044525in}{3.413942in}}%
\pgfpathlineto{\pgfqpoint{3.365172in}{3.120584in}}%
\pgfpathlineto{\pgfqpoint{3.654472in}{2.842667in}}%
\pgfpathlineto{\pgfqpoint{3.806061in}{2.691898in}}%
\pgfpathlineto{\pgfqpoint{4.109333in}{2.378484in}}%
\pgfpathlineto{\pgfqpoint{4.166788in}{2.317769in}}%
\pgfpathlineto{\pgfqpoint{4.302849in}{2.170667in}}%
\pgfpathlineto{\pgfqpoint{4.586610in}{1.852377in}}%
\pgfpathlineto{\pgfqpoint{4.647758in}{1.782010in}}%
\pgfpathlineto{\pgfqpoint{4.768000in}{1.640643in}}%
\pgfpathlineto{\pgfqpoint{4.768000in}{1.640643in}}%
\pgfusepath{fill}%
\end{pgfscope}%
\begin{pgfscope}%
\pgfpathrectangle{\pgfqpoint{0.800000in}{0.528000in}}{\pgfqpoint{3.968000in}{3.696000in}}%
\pgfusepath{clip}%
\pgfsetbuttcap%
\pgfsetroundjoin%
\definecolor{currentfill}{rgb}{0.255645,0.260703,0.528312}%
\pgfsetfillcolor{currentfill}%
\pgfsetlinewidth{0.000000pt}%
\definecolor{currentstroke}{rgb}{0.000000,0.000000,0.000000}%
\pgfsetstrokecolor{currentstroke}%
\pgfsetdash{}{0pt}%
\pgfpathmoveto{\pgfqpoint{2.019799in}{0.528000in}}%
\pgfpathlineto{\pgfqpoint{1.721859in}{0.833791in}}%
\pgfpathlineto{\pgfqpoint{1.441293in}{1.135842in}}%
\pgfpathlineto{\pgfqpoint{1.295990in}{1.298009in}}%
\pgfpathlineto{\pgfqpoint{1.185753in}{1.424000in}}%
\pgfpathlineto{\pgfqpoint{1.057868in}{1.573333in}}%
\pgfpathlineto{\pgfqpoint{0.800000in}{1.886097in}}%
\pgfpathlineto{\pgfqpoint{0.800000in}{1.879771in}}%
\pgfpathlineto{\pgfqpoint{1.021335in}{1.610667in}}%
\pgfpathlineto{\pgfqpoint{1.148432in}{1.461333in}}%
\pgfpathlineto{\pgfqpoint{1.411936in}{1.162667in}}%
\pgfpathlineto{\pgfqpoint{1.548410in}{1.013333in}}%
\pgfpathlineto{\pgfqpoint{1.681778in}{0.870621in}}%
\pgfpathlineto{\pgfqpoint{1.802020in}{0.744660in}}%
\pgfpathlineto{\pgfqpoint{2.014501in}{0.528000in}}%
\pgfpathmoveto{\pgfqpoint{4.768000in}{1.652670in}}%
\pgfpathlineto{\pgfqpoint{4.644730in}{1.797333in}}%
\pgfpathlineto{\pgfqpoint{4.367192in}{2.111218in}}%
\pgfpathlineto{\pgfqpoint{4.086626in}{2.413641in}}%
\pgfpathlineto{\pgfqpoint{3.961601in}{2.544000in}}%
\pgfpathlineto{\pgfqpoint{3.645737in}{2.861743in}}%
\pgfpathlineto{\pgfqpoint{3.339496in}{3.154751in}}%
\pgfpathlineto{\pgfqpoint{3.273882in}{3.216000in}}%
\pgfpathlineto{\pgfqpoint{2.964364in}{3.495185in}}%
\pgfpathlineto{\pgfqpoint{2.804040in}{3.634327in}}%
\pgfpathlineto{\pgfqpoint{2.483394in}{3.901247in}}%
\pgfpathlineto{\pgfqpoint{2.359968in}{4.000000in}}%
\pgfpathlineto{\pgfqpoint{2.068268in}{4.224000in}}%
\pgfpathlineto{\pgfqpoint{2.061170in}{4.224000in}}%
\pgfpathlineto{\pgfqpoint{2.172329in}{4.140408in}}%
\pgfpathlineto{\pgfqpoint{2.323071in}{4.023807in}}%
\pgfpathlineto{\pgfqpoint{2.629883in}{3.776000in}}%
\pgfpathlineto{\pgfqpoint{2.893512in}{3.552000in}}%
\pgfpathlineto{\pgfqpoint{3.204848in}{3.274330in}}%
\pgfpathlineto{\pgfqpoint{3.525495in}{2.973372in}}%
\pgfpathlineto{\pgfqpoint{3.827120in}{2.675615in}}%
\pgfpathlineto{\pgfqpoint{3.886222in}{2.615986in}}%
\pgfpathlineto{\pgfqpoint{4.190372in}{2.298033in}}%
\pgfpathlineto{\pgfqpoint{4.316672in}{2.160943in}}%
\pgfpathlineto{\pgfqpoint{4.375892in}{2.096000in}}%
\pgfpathlineto{\pgfqpoint{4.647758in}{1.787915in}}%
\pgfpathlineto{\pgfqpoint{4.768000in}{1.646711in}}%
\pgfpathlineto{\pgfqpoint{4.768000in}{1.648000in}}%
\pgfusepath{fill}%
\end{pgfscope}%
\begin{pgfscope}%
\pgfpathrectangle{\pgfqpoint{0.800000in}{0.528000in}}{\pgfqpoint{3.968000in}{3.696000in}}%
\pgfusepath{clip}%
\pgfsetbuttcap%
\pgfsetroundjoin%
\definecolor{currentfill}{rgb}{0.255645,0.260703,0.528312}%
\pgfsetfillcolor{currentfill}%
\pgfsetlinewidth{0.000000pt}%
\definecolor{currentstroke}{rgb}{0.000000,0.000000,0.000000}%
\pgfsetstrokecolor{currentstroke}%
\pgfsetdash{}{0pt}%
\pgfpathmoveto{\pgfqpoint{2.014501in}{0.528000in}}%
\pgfpathlineto{\pgfqpoint{1.721859in}{0.828318in}}%
\pgfpathlineto{\pgfqpoint{1.441293in}{1.130199in}}%
\pgfpathlineto{\pgfqpoint{1.311653in}{1.274667in}}%
\pgfpathlineto{\pgfqpoint{1.180708in}{1.424000in}}%
\pgfpathlineto{\pgfqpoint{1.052809in}{1.573333in}}%
\pgfpathlineto{\pgfqpoint{0.800000in}{1.879771in}}%
\pgfpathlineto{\pgfqpoint{0.800670in}{1.872624in}}%
\pgfpathlineto{\pgfqpoint{0.840081in}{1.823919in}}%
\pgfpathlineto{\pgfqpoint{1.080566in}{1.534609in}}%
\pgfpathlineto{\pgfqpoint{1.208077in}{1.386667in}}%
\pgfpathlineto{\pgfqpoint{1.339853in}{1.237333in}}%
\pgfpathlineto{\pgfqpoint{1.474662in}{1.088000in}}%
\pgfpathlineto{\pgfqpoint{1.761939in}{0.780988in}}%
\pgfpathlineto{\pgfqpoint{2.009202in}{0.528000in}}%
\pgfpathmoveto{\pgfqpoint{4.768000in}{1.658602in}}%
\pgfpathlineto{\pgfqpoint{4.647758in}{1.799673in}}%
\pgfpathlineto{\pgfqpoint{4.367192in}{2.116815in}}%
\pgfpathlineto{\pgfqpoint{4.086626in}{2.419070in}}%
\pgfpathlineto{\pgfqpoint{3.966384in}{2.544466in}}%
\pgfpathlineto{\pgfqpoint{3.645737in}{2.866986in}}%
\pgfpathlineto{\pgfqpoint{3.342250in}{3.157316in}}%
\pgfpathlineto{\pgfqpoint{3.279489in}{3.216000in}}%
\pgfpathlineto{\pgfqpoint{2.964364in}{3.500314in}}%
\pgfpathlineto{\pgfqpoint{2.818908in}{3.626667in}}%
\pgfpathlineto{\pgfqpoint{2.683798in}{3.741366in}}%
\pgfpathlineto{\pgfqpoint{2.551850in}{3.850667in}}%
\pgfpathlineto{\pgfqpoint{2.403232in}{3.970803in}}%
\pgfpathlineto{\pgfqpoint{2.112167in}{4.196446in}}%
\pgfpathlineto{\pgfqpoint{2.075367in}{4.224000in}}%
\pgfpathlineto{\pgfqpoint{2.068268in}{4.224000in}}%
\pgfpathlineto{\pgfqpoint{2.167462in}{4.149333in}}%
\pgfpathlineto{\pgfqpoint{2.323071in}{4.029031in}}%
\pgfpathlineto{\pgfqpoint{2.636120in}{3.776000in}}%
\pgfpathlineto{\pgfqpoint{2.899392in}{3.552000in}}%
\pgfpathlineto{\pgfqpoint{3.198735in}{3.284972in}}%
\pgfpathlineto{\pgfqpoint{3.244929in}{3.242726in}}%
\pgfpathlineto{\pgfqpoint{3.565576in}{2.939890in}}%
\pgfpathlineto{\pgfqpoint{3.702961in}{2.805333in}}%
\pgfpathlineto{\pgfqpoint{3.846141in}{2.662023in}}%
\pgfpathlineto{\pgfqpoint{3.966384in}{2.539068in}}%
\pgfpathlineto{\pgfqpoint{4.263969in}{2.223853in}}%
\pgfpathlineto{\pgfqpoint{4.327111in}{2.155303in}}%
\pgfpathlineto{\pgfqpoint{4.452000in}{2.017005in}}%
\pgfpathlineto{\pgfqpoint{4.579907in}{1.872000in}}%
\pgfpathlineto{\pgfqpoint{4.768000in}{1.652670in}}%
\pgfpathlineto{\pgfqpoint{4.768000in}{1.652670in}}%
\pgfusepath{fill}%
\end{pgfscope}%
\begin{pgfscope}%
\pgfpathrectangle{\pgfqpoint{0.800000in}{0.528000in}}{\pgfqpoint{3.968000in}{3.696000in}}%
\pgfusepath{clip}%
\pgfsetbuttcap%
\pgfsetroundjoin%
\definecolor{currentfill}{rgb}{0.255645,0.260703,0.528312}%
\pgfsetfillcolor{currentfill}%
\pgfsetlinewidth{0.000000pt}%
\definecolor{currentstroke}{rgb}{0.000000,0.000000,0.000000}%
\pgfsetstrokecolor{currentstroke}%
\pgfsetdash{}{0pt}%
\pgfpathmoveto{\pgfqpoint{2.009202in}{0.528000in}}%
\pgfpathlineto{\pgfqpoint{1.718297in}{0.826667in}}%
\pgfpathlineto{\pgfqpoint{1.436832in}{1.129488in}}%
\pgfpathlineto{\pgfqpoint{1.306621in}{1.274667in}}%
\pgfpathlineto{\pgfqpoint{1.175663in}{1.424000in}}%
\pgfpathlineto{\pgfqpoint{1.047751in}{1.573333in}}%
\pgfpathlineto{\pgfqpoint{0.800000in}{1.873444in}}%
\pgfpathlineto{\pgfqpoint{0.800000in}{1.872000in}}%
\pgfpathlineto{\pgfqpoint{0.800000in}{1.867232in}}%
\pgfpathlineto{\pgfqpoint{1.050929in}{1.563605in}}%
\pgfpathlineto{\pgfqpoint{1.170618in}{1.424000in}}%
\pgfpathlineto{\pgfqpoint{1.441293in}{1.119045in}}%
\pgfpathlineto{\pgfqpoint{1.572810in}{0.976000in}}%
\pgfpathlineto{\pgfqpoint{1.882182in}{0.651404in}}%
\pgfpathlineto{\pgfqpoint{2.003904in}{0.528000in}}%
\pgfpathmoveto{\pgfqpoint{4.768000in}{1.664533in}}%
\pgfpathlineto{\pgfqpoint{4.647758in}{1.805449in}}%
\pgfpathlineto{\pgfqpoint{4.367192in}{2.122412in}}%
\pgfpathlineto{\pgfqpoint{4.086626in}{2.424499in}}%
\pgfpathlineto{\pgfqpoint{3.966384in}{2.549764in}}%
\pgfpathlineto{\pgfqpoint{3.675797in}{2.842667in}}%
\pgfpathlineto{\pgfqpoint{3.522413in}{2.992000in}}%
\pgfpathlineto{\pgfqpoint{3.203938in}{3.290667in}}%
\pgfpathlineto{\pgfqpoint{2.884202in}{3.575447in}}%
\pgfpathlineto{\pgfqpoint{2.737305in}{3.701333in}}%
\pgfpathlineto{\pgfqpoint{2.603551in}{3.813333in}}%
\pgfpathlineto{\pgfqpoint{2.443313in}{3.943855in}}%
\pgfpathlineto{\pgfqpoint{2.131988in}{4.186667in}}%
\pgfpathlineto{\pgfqpoint{2.082465in}{4.224000in}}%
\pgfpathlineto{\pgfqpoint{2.075367in}{4.224000in}}%
\pgfpathlineto{\pgfqpoint{2.174268in}{4.149333in}}%
\pgfpathlineto{\pgfqpoint{2.323071in}{4.034255in}}%
\pgfpathlineto{\pgfqpoint{2.603636in}{3.808094in}}%
\pgfpathlineto{\pgfqpoint{2.731295in}{3.701333in}}%
\pgfpathlineto{\pgfqpoint{3.032563in}{3.440000in}}%
\pgfpathlineto{\pgfqpoint{3.164768in}{3.321196in}}%
\pgfpathlineto{\pgfqpoint{3.485414in}{3.022270in}}%
\pgfpathlineto{\pgfqpoint{3.806061in}{2.707739in}}%
\pgfpathlineto{\pgfqpoint{3.930522in}{2.581333in}}%
\pgfpathlineto{\pgfqpoint{4.229607in}{2.266513in}}%
\pgfpathlineto{\pgfqpoint{4.287030in}{2.204693in}}%
\pgfpathlineto{\pgfqpoint{4.567596in}{1.891861in}}%
\pgfpathlineto{\pgfqpoint{4.768000in}{1.658602in}}%
\pgfpathlineto{\pgfqpoint{4.768000in}{1.658602in}}%
\pgfusepath{fill}%
\end{pgfscope}%
\begin{pgfscope}%
\pgfpathrectangle{\pgfqpoint{0.800000in}{0.528000in}}{\pgfqpoint{3.968000in}{3.696000in}}%
\pgfusepath{clip}%
\pgfsetbuttcap%
\pgfsetroundjoin%
\definecolor{currentfill}{rgb}{0.253935,0.265254,0.529983}%
\pgfsetfillcolor{currentfill}%
\pgfsetlinewidth{0.000000pt}%
\definecolor{currentstroke}{rgb}{0.000000,0.000000,0.000000}%
\pgfsetstrokecolor{currentstroke}%
\pgfsetdash{}{0pt}%
\pgfpathmoveto{\pgfqpoint{2.003904in}{0.528000in}}%
\pgfpathlineto{\pgfqpoint{1.721859in}{0.817560in}}%
\pgfpathlineto{\pgfqpoint{1.441293in}{1.119045in}}%
\pgfpathlineto{\pgfqpoint{1.200808in}{1.389191in}}%
\pgfpathlineto{\pgfqpoint{0.948864in}{1.685333in}}%
\pgfpathlineto{\pgfqpoint{0.840081in}{1.817749in}}%
\pgfpathlineto{\pgfqpoint{0.800000in}{1.867232in}}%
\pgfpathlineto{\pgfqpoint{0.800000in}{1.861053in}}%
\pgfpathlineto{\pgfqpoint{1.040485in}{1.570035in}}%
\pgfpathlineto{\pgfqpoint{1.165572in}{1.424000in}}%
\pgfpathlineto{\pgfqpoint{1.441293in}{1.113519in}}%
\pgfpathlineto{\pgfqpoint{1.567697in}{0.976000in}}%
\pgfpathlineto{\pgfqpoint{1.866354in}{0.662590in}}%
\pgfpathlineto{\pgfqpoint{1.924741in}{0.602667in}}%
\pgfpathlineto{\pgfqpoint{1.998688in}{0.528000in}}%
\pgfpathlineto{\pgfqpoint{2.002424in}{0.528000in}}%
\pgfpathmoveto{\pgfqpoint{4.768000in}{1.670464in}}%
\pgfpathlineto{\pgfqpoint{4.647758in}{1.811225in}}%
\pgfpathlineto{\pgfqpoint{4.396138in}{2.096000in}}%
\pgfpathlineto{\pgfqpoint{4.259658in}{2.245333in}}%
\pgfpathlineto{\pgfqpoint{4.126707in}{2.387614in}}%
\pgfpathlineto{\pgfqpoint{3.846141in}{2.677856in}}%
\pgfpathlineto{\pgfqpoint{3.546185in}{2.973938in}}%
\pgfpathlineto{\pgfqpoint{3.485414in}{3.032637in}}%
\pgfpathlineto{\pgfqpoint{3.164768in}{3.331453in}}%
\pgfpathlineto{\pgfqpoint{3.002060in}{3.477333in}}%
\pgfpathlineto{\pgfqpoint{2.683798in}{3.751532in}}%
\pgfpathlineto{\pgfqpoint{2.557026in}{3.856749in}}%
\pgfpathlineto{\pgfqpoint{2.403232in}{3.981075in}}%
\pgfpathlineto{\pgfqpoint{2.089361in}{4.224000in}}%
\pgfpathlineto{\pgfqpoint{2.082465in}{4.224000in}}%
\pgfpathlineto{\pgfqpoint{2.083081in}{4.223539in}}%
\pgfpathlineto{\pgfqpoint{2.242909in}{4.101852in}}%
\pgfpathlineto{\pgfqpoint{2.523475in}{3.879012in}}%
\pgfpathlineto{\pgfqpoint{2.648468in}{3.776000in}}%
\pgfpathlineto{\pgfqpoint{2.804040in}{3.644532in}}%
\pgfpathlineto{\pgfqpoint{2.964364in}{3.505442in}}%
\pgfpathlineto{\pgfqpoint{3.285094in}{3.216000in}}%
\pgfpathlineto{\pgfqpoint{3.605657in}{2.911413in}}%
\pgfpathlineto{\pgfqpoint{3.926303in}{2.590957in}}%
\pgfpathlineto{\pgfqpoint{4.046545in}{2.466556in}}%
\pgfpathlineto{\pgfqpoint{4.185125in}{2.320000in}}%
\pgfpathlineto{\pgfqpoint{4.327111in}{2.166482in}}%
\pgfpathlineto{\pgfqpoint{4.458168in}{2.021333in}}%
\pgfpathlineto{\pgfqpoint{4.589963in}{1.872000in}}%
\pgfpathlineto{\pgfqpoint{4.768000in}{1.664533in}}%
\pgfpathlineto{\pgfqpoint{4.768000in}{1.664533in}}%
\pgfusepath{fill}%
\end{pgfscope}%
\begin{pgfscope}%
\pgfpathrectangle{\pgfqpoint{0.800000in}{0.528000in}}{\pgfqpoint{3.968000in}{3.696000in}}%
\pgfusepath{clip}%
\pgfsetbuttcap%
\pgfsetroundjoin%
\definecolor{currentfill}{rgb}{0.253935,0.265254,0.529983}%
\pgfsetfillcolor{currentfill}%
\pgfsetlinewidth{0.000000pt}%
\definecolor{currentstroke}{rgb}{0.000000,0.000000,0.000000}%
\pgfsetstrokecolor{currentstroke}%
\pgfsetdash{}{0pt}%
\pgfpathmoveto{\pgfqpoint{1.998688in}{0.528000in}}%
\pgfpathlineto{\pgfqpoint{1.851717in}{0.677333in}}%
\pgfpathlineto{\pgfqpoint{1.721859in}{0.812198in}}%
\pgfpathlineto{\pgfqpoint{1.441293in}{1.113519in}}%
\pgfpathlineto{\pgfqpoint{1.197999in}{1.386667in}}%
\pgfpathlineto{\pgfqpoint{0.920242in}{1.713805in}}%
\pgfpathlineto{\pgfqpoint{0.800000in}{1.861053in}}%
\pgfpathlineto{\pgfqpoint{0.800000in}{1.854875in}}%
\pgfpathlineto{\pgfqpoint{1.040485in}{1.564184in}}%
\pgfpathlineto{\pgfqpoint{1.160727in}{1.423774in}}%
\pgfpathlineto{\pgfqpoint{1.441293in}{1.107993in}}%
\pgfpathlineto{\pgfqpoint{1.562584in}{0.976000in}}%
\pgfpathlineto{\pgfqpoint{1.863654in}{0.660075in}}%
\pgfpathlineto{\pgfqpoint{1.922263in}{0.599911in}}%
\pgfpathlineto{\pgfqpoint{1.993505in}{0.528000in}}%
\pgfpathmoveto{\pgfqpoint{4.768000in}{1.676395in}}%
\pgfpathlineto{\pgfqpoint{4.647758in}{1.817000in}}%
\pgfpathlineto{\pgfqpoint{4.401200in}{2.096000in}}%
\pgfpathlineto{\pgfqpoint{4.264706in}{2.245333in}}%
\pgfpathlineto{\pgfqpoint{4.125178in}{2.394667in}}%
\pgfpathlineto{\pgfqpoint{3.836037in}{2.693333in}}%
\pgfpathlineto{\pgfqpoint{3.685818in}{2.843286in}}%
\pgfpathlineto{\pgfqpoint{3.525495in}{2.999340in}}%
\pgfpathlineto{\pgfqpoint{3.204848in}{3.299991in}}%
\pgfpathlineto{\pgfqpoint{2.879979in}{3.589333in}}%
\pgfpathlineto{\pgfqpoint{2.723879in}{3.722883in}}%
\pgfpathlineto{\pgfqpoint{2.403232in}{3.986212in}}%
\pgfpathlineto{\pgfqpoint{2.096254in}{4.224000in}}%
\pgfpathlineto{\pgfqpoint{2.089361in}{4.224000in}}%
\pgfpathlineto{\pgfqpoint{2.242909in}{4.107062in}}%
\pgfpathlineto{\pgfqpoint{2.523475in}{3.884168in}}%
\pgfpathlineto{\pgfqpoint{2.654546in}{3.776000in}}%
\pgfpathlineto{\pgfqpoint{2.804040in}{3.649635in}}%
\pgfpathlineto{\pgfqpoint{2.964364in}{3.510571in}}%
\pgfpathlineto{\pgfqpoint{3.290571in}{3.216000in}}%
\pgfpathlineto{\pgfqpoint{3.605657in}{2.916649in}}%
\pgfpathlineto{\pgfqpoint{3.926303in}{2.596248in}}%
\pgfpathlineto{\pgfqpoint{4.069033in}{2.448387in}}%
\pgfpathlineto{\pgfqpoint{4.206869in}{2.302189in}}%
\pgfpathlineto{\pgfqpoint{4.335488in}{2.162864in}}%
\pgfpathlineto{\pgfqpoint{4.463183in}{2.021333in}}%
\pgfpathlineto{\pgfqpoint{4.594991in}{1.872000in}}%
\pgfpathlineto{\pgfqpoint{4.768000in}{1.670464in}}%
\pgfpathlineto{\pgfqpoint{4.768000in}{1.670464in}}%
\pgfusepath{fill}%
\end{pgfscope}%
\begin{pgfscope}%
\pgfpathrectangle{\pgfqpoint{0.800000in}{0.528000in}}{\pgfqpoint{3.968000in}{3.696000in}}%
\pgfusepath{clip}%
\pgfsetbuttcap%
\pgfsetroundjoin%
\definecolor{currentfill}{rgb}{0.253935,0.265254,0.529983}%
\pgfsetfillcolor{currentfill}%
\pgfsetlinewidth{0.000000pt}%
\definecolor{currentstroke}{rgb}{0.000000,0.000000,0.000000}%
\pgfsetstrokecolor{currentstroke}%
\pgfsetdash{}{0pt}%
\pgfpathmoveto{\pgfqpoint{1.993505in}{0.528000in}}%
\pgfpathlineto{\pgfqpoint{1.842101in}{0.681874in}}%
\pgfpathlineto{\pgfqpoint{1.693139in}{0.837249in}}%
\pgfpathlineto{\pgfqpoint{1.632405in}{0.901333in}}%
\pgfpathlineto{\pgfqpoint{1.493704in}{1.050667in}}%
\pgfpathlineto{\pgfqpoint{1.225702in}{1.349333in}}%
\pgfpathlineto{\pgfqpoint{1.096321in}{1.498667in}}%
\pgfpathlineto{\pgfqpoint{0.969966in}{1.648000in}}%
\pgfpathlineto{\pgfqpoint{0.846661in}{1.797333in}}%
\pgfpathlineto{\pgfqpoint{0.800000in}{1.854875in}}%
\pgfpathlineto{\pgfqpoint{0.800000in}{1.848696in}}%
\pgfpathlineto{\pgfqpoint{1.033449in}{1.566780in}}%
\pgfpathlineto{\pgfqpoint{1.091322in}{1.498667in}}%
\pgfpathlineto{\pgfqpoint{1.361131in}{1.191064in}}%
\pgfpathlineto{\pgfqpoint{1.488639in}{1.050667in}}%
\pgfpathlineto{\pgfqpoint{1.627353in}{0.901333in}}%
\pgfpathlineto{\pgfqpoint{1.765812in}{0.755608in}}%
\pgfpathlineto{\pgfqpoint{1.834189in}{0.684703in}}%
\pgfpathlineto{\pgfqpoint{1.962343in}{0.554136in}}%
\pgfpathlineto{\pgfqpoint{1.988322in}{0.528000in}}%
\pgfpathmoveto{\pgfqpoint{4.768000in}{1.682326in}}%
\pgfpathlineto{\pgfqpoint{4.647758in}{1.822776in}}%
\pgfpathlineto{\pgfqpoint{4.400857in}{2.101976in}}%
\pgfpathlineto{\pgfqpoint{4.269754in}{2.245333in}}%
\pgfpathlineto{\pgfqpoint{4.126707in}{2.398410in}}%
\pgfpathlineto{\pgfqpoint{3.841265in}{2.693333in}}%
\pgfpathlineto{\pgfqpoint{3.691660in}{2.842667in}}%
\pgfpathlineto{\pgfqpoint{3.405253in}{3.118961in}}%
\pgfpathlineto{\pgfqpoint{3.084606in}{3.413921in}}%
\pgfpathlineto{\pgfqpoint{2.759810in}{3.697469in}}%
\pgfpathlineto{\pgfqpoint{2.683798in}{3.761699in}}%
\pgfpathlineto{\pgfqpoint{2.523475in}{3.894356in}}%
\pgfpathlineto{\pgfqpoint{2.223966in}{4.131689in}}%
\pgfpathlineto{\pgfqpoint{2.152535in}{4.186667in}}%
\pgfpathlineto{\pgfqpoint{2.103146in}{4.224000in}}%
\pgfpathlineto{\pgfqpoint{2.096254in}{4.224000in}}%
\pgfpathlineto{\pgfqpoint{2.243253in}{4.112000in}}%
\pgfpathlineto{\pgfqpoint{2.525062in}{3.888000in}}%
\pgfpathlineto{\pgfqpoint{2.683798in}{3.756616in}}%
\pgfpathlineto{\pgfqpoint{2.844121in}{3.620323in}}%
\pgfpathlineto{\pgfqpoint{3.004444in}{3.480306in}}%
\pgfpathlineto{\pgfqpoint{3.325091in}{3.189017in}}%
\pgfpathlineto{\pgfqpoint{3.455063in}{3.066667in}}%
\pgfpathlineto{\pgfqpoint{3.765980in}{2.763717in}}%
\pgfpathlineto{\pgfqpoint{4.054106in}{2.469333in}}%
\pgfpathlineto{\pgfqpoint{4.195316in}{2.320000in}}%
\pgfpathlineto{\pgfqpoint{4.327111in}{2.177516in}}%
\pgfpathlineto{\pgfqpoint{4.447354in}{2.044664in}}%
\pgfpathlineto{\pgfqpoint{4.567596in}{1.909139in}}%
\pgfpathlineto{\pgfqpoint{4.768000in}{1.676395in}}%
\pgfpathlineto{\pgfqpoint{4.768000in}{1.676395in}}%
\pgfusepath{fill}%
\end{pgfscope}%
\begin{pgfscope}%
\pgfpathrectangle{\pgfqpoint{0.800000in}{0.528000in}}{\pgfqpoint{3.968000in}{3.696000in}}%
\pgfusepath{clip}%
\pgfsetbuttcap%
\pgfsetroundjoin%
\definecolor{currentfill}{rgb}{0.253935,0.265254,0.529983}%
\pgfsetfillcolor{currentfill}%
\pgfsetlinewidth{0.000000pt}%
\definecolor{currentstroke}{rgb}{0.000000,0.000000,0.000000}%
\pgfsetstrokecolor{currentstroke}%
\pgfsetdash{}{0pt}%
\pgfpathmoveto{\pgfqpoint{1.988322in}{0.528000in}}%
\pgfpathlineto{\pgfqpoint{1.841340in}{0.677333in}}%
\pgfpathlineto{\pgfqpoint{1.697899in}{0.826667in}}%
\pgfpathlineto{\pgfqpoint{1.401212in}{1.146628in}}%
\pgfpathlineto{\pgfqpoint{1.140169in}{1.442185in}}%
\pgfpathlineto{\pgfqpoint{1.080566in}{1.511238in}}%
\pgfpathlineto{\pgfqpoint{0.840081in}{1.799242in}}%
\pgfpathlineto{\pgfqpoint{0.800000in}{1.848696in}}%
\pgfpathlineto{\pgfqpoint{0.800000in}{1.842517in}}%
\pgfpathlineto{\pgfqpoint{1.022833in}{1.573333in}}%
\pgfpathlineto{\pgfqpoint{1.299575in}{1.254664in}}%
\pgfpathlineto{\pgfqpoint{1.361131in}{1.185523in}}%
\pgfpathlineto{\pgfqpoint{1.496669in}{1.036419in}}%
\pgfpathlineto{\pgfqpoint{1.641697in}{0.880692in}}%
\pgfpathlineto{\pgfqpoint{1.922263in}{0.589469in}}%
\pgfpathlineto{\pgfqpoint{1.983139in}{0.528000in}}%
\pgfpathmoveto{\pgfqpoint{4.768000in}{1.688193in}}%
\pgfpathlineto{\pgfqpoint{4.487434in}{2.011013in}}%
\pgfpathlineto{\pgfqpoint{4.367192in}{2.144563in}}%
\pgfpathlineto{\pgfqpoint{4.240246in}{2.282667in}}%
\pgfpathlineto{\pgfqpoint{4.099871in}{2.432000in}}%
\pgfpathlineto{\pgfqpoint{3.806061in}{2.734026in}}%
\pgfpathlineto{\pgfqpoint{3.485414in}{3.047982in}}%
\pgfpathlineto{\pgfqpoint{3.164768in}{3.346642in}}%
\pgfpathlineto{\pgfqpoint{3.004444in}{3.490381in}}%
\pgfpathlineto{\pgfqpoint{2.683798in}{3.766782in}}%
\pgfpathlineto{\pgfqpoint{2.537423in}{3.888000in}}%
\pgfpathlineto{\pgfqpoint{2.398810in}{4.000000in}}%
\pgfpathlineto{\pgfqpoint{2.242909in}{4.122487in}}%
\pgfpathlineto{\pgfqpoint{2.110039in}{4.224000in}}%
\pgfpathlineto{\pgfqpoint{2.103146in}{4.224000in}}%
\pgfpathlineto{\pgfqpoint{2.242909in}{4.117377in}}%
\pgfpathlineto{\pgfqpoint{2.392351in}{4.000000in}}%
\pgfpathlineto{\pgfqpoint{2.643717in}{3.795199in}}%
\pgfpathlineto{\pgfqpoint{2.964364in}{3.520712in}}%
\pgfpathlineto{\pgfqpoint{3.097128in}{3.402667in}}%
\pgfpathlineto{\pgfqpoint{3.405253in}{3.118961in}}%
\pgfpathlineto{\pgfqpoint{3.551643in}{2.979022in}}%
\pgfpathlineto{\pgfqpoint{3.615498in}{2.917333in}}%
\pgfpathlineto{\pgfqpoint{3.926303in}{2.606832in}}%
\pgfpathlineto{\pgfqpoint{4.059188in}{2.469333in}}%
\pgfpathlineto{\pgfqpoint{4.200411in}{2.320000in}}%
\pgfpathlineto{\pgfqpoint{4.327111in}{2.182990in}}%
\pgfpathlineto{\pgfqpoint{4.447354in}{2.050276in}}%
\pgfpathlineto{\pgfqpoint{4.572375in}{1.909333in}}%
\pgfpathlineto{\pgfqpoint{4.768000in}{1.682326in}}%
\pgfpathlineto{\pgfqpoint{4.768000in}{1.685333in}}%
\pgfpathlineto{\pgfqpoint{4.768000in}{1.685333in}}%
\pgfusepath{fill}%
\end{pgfscope}%
\begin{pgfscope}%
\pgfpathrectangle{\pgfqpoint{0.800000in}{0.528000in}}{\pgfqpoint{3.968000in}{3.696000in}}%
\pgfusepath{clip}%
\pgfsetbuttcap%
\pgfsetroundjoin%
\definecolor{currentfill}{rgb}{0.252194,0.269783,0.531579}%
\pgfsetfillcolor{currentfill}%
\pgfsetlinewidth{0.000000pt}%
\definecolor{currentstroke}{rgb}{0.000000,0.000000,0.000000}%
\pgfsetstrokecolor{currentstroke}%
\pgfsetdash{}{0pt}%
\pgfpathmoveto{\pgfqpoint{1.983139in}{0.528000in}}%
\pgfpathlineto{\pgfqpoint{1.836254in}{0.677333in}}%
\pgfpathlineto{\pgfqpoint{1.692800in}{0.826667in}}%
\pgfpathlineto{\pgfqpoint{1.401212in}{1.141095in}}%
\pgfpathlineto{\pgfqpoint{1.137409in}{1.439614in}}%
\pgfpathlineto{\pgfqpoint{1.080566in}{1.505396in}}%
\pgfpathlineto{\pgfqpoint{0.825967in}{1.810479in}}%
\pgfpathlineto{\pgfqpoint{0.800000in}{1.842517in}}%
\pgfpathlineto{\pgfqpoint{0.800779in}{1.835392in}}%
\pgfpathlineto{\pgfqpoint{0.840081in}{1.787141in}}%
\pgfpathlineto{\pgfqpoint{0.960323in}{1.641685in}}%
\pgfpathlineto{\pgfqpoint{1.224447in}{1.334019in}}%
\pgfpathlineto{\pgfqpoint{1.280970in}{1.269653in}}%
\pgfpathlineto{\pgfqpoint{1.561535in}{0.960932in}}%
\pgfpathlineto{\pgfqpoint{1.855553in}{0.652530in}}%
\pgfpathlineto{\pgfqpoint{1.922263in}{0.584247in}}%
\pgfpathlineto{\pgfqpoint{1.977956in}{0.528000in}}%
\pgfpathmoveto{\pgfqpoint{4.768000in}{1.693994in}}%
\pgfpathlineto{\pgfqpoint{4.503155in}{1.998643in}}%
\pgfpathlineto{\pgfqpoint{4.447354in}{2.061443in}}%
\pgfpathlineto{\pgfqpoint{4.314202in}{2.208000in}}%
\pgfpathlineto{\pgfqpoint{4.175512in}{2.357333in}}%
\pgfpathlineto{\pgfqpoint{4.046545in}{2.493176in}}%
\pgfpathlineto{\pgfqpoint{3.912265in}{2.631743in}}%
\pgfpathlineto{\pgfqpoint{3.765980in}{2.779284in}}%
\pgfpathlineto{\pgfqpoint{3.445333in}{3.091250in}}%
\pgfpathlineto{\pgfqpoint{3.124687in}{3.387979in}}%
\pgfpathlineto{\pgfqpoint{2.804040in}{3.669931in}}%
\pgfpathlineto{\pgfqpoint{2.678854in}{3.776000in}}%
\pgfpathlineto{\pgfqpoint{2.391450in}{4.010974in}}%
\pgfpathlineto{\pgfqpoint{2.242909in}{4.127597in}}%
\pgfpathlineto{\pgfqpoint{2.116931in}{4.224000in}}%
\pgfpathlineto{\pgfqpoint{2.110039in}{4.224000in}}%
\pgfpathlineto{\pgfqpoint{2.208147in}{4.149333in}}%
\pgfpathlineto{\pgfqpoint{2.491615in}{3.925333in}}%
\pgfpathlineto{\pgfqpoint{2.792096in}{3.675126in}}%
\pgfpathlineto{\pgfqpoint{2.924283in}{3.560878in}}%
\pgfpathlineto{\pgfqpoint{3.061088in}{3.440000in}}%
\pgfpathlineto{\pgfqpoint{3.365172in}{3.161747in}}%
\pgfpathlineto{\pgfqpoint{3.685818in}{2.853582in}}%
\pgfpathlineto{\pgfqpoint{3.966384in}{2.570959in}}%
\pgfpathlineto{\pgfqpoint{4.112339in}{2.418617in}}%
\pgfpathlineto{\pgfqpoint{4.170501in}{2.357333in}}%
\pgfpathlineto{\pgfqpoint{4.447354in}{2.055889in}}%
\pgfpathlineto{\pgfqpoint{4.577321in}{1.909333in}}%
\pgfpathlineto{\pgfqpoint{4.768000in}{1.688193in}}%
\pgfpathlineto{\pgfqpoint{4.768000in}{1.688193in}}%
\pgfusepath{fill}%
\end{pgfscope}%
\begin{pgfscope}%
\pgfpathrectangle{\pgfqpoint{0.800000in}{0.528000in}}{\pgfqpoint{3.968000in}{3.696000in}}%
\pgfusepath{clip}%
\pgfsetbuttcap%
\pgfsetroundjoin%
\definecolor{currentfill}{rgb}{0.252194,0.269783,0.531579}%
\pgfsetfillcolor{currentfill}%
\pgfsetlinewidth{0.000000pt}%
\definecolor{currentstroke}{rgb}{0.000000,0.000000,0.000000}%
\pgfsetstrokecolor{currentstroke}%
\pgfsetdash{}{0pt}%
\pgfpathmoveto{\pgfqpoint{1.977956in}{0.528000in}}%
\pgfpathlineto{\pgfqpoint{1.842101in}{0.666080in}}%
\pgfpathlineto{\pgfqpoint{1.721859in}{0.790750in}}%
\pgfpathlineto{\pgfqpoint{1.424753in}{1.109927in}}%
\pgfpathlineto{\pgfqpoint{1.361131in}{1.179982in}}%
\pgfpathlineto{\pgfqpoint{1.098976in}{1.478481in}}%
\pgfpathlineto{\pgfqpoint{1.040485in}{1.546631in}}%
\pgfpathlineto{\pgfqpoint{0.800000in}{1.836339in}}%
\pgfpathlineto{\pgfqpoint{0.800000in}{1.834667in}}%
\pgfpathlineto{\pgfqpoint{0.800000in}{1.830263in}}%
\pgfpathlineto{\pgfqpoint{0.920242in}{1.683788in}}%
\pgfpathlineto{\pgfqpoint{1.200808in}{1.354985in}}%
\pgfpathlineto{\pgfqpoint{1.321051in}{1.219129in}}%
\pgfpathlineto{\pgfqpoint{1.441293in}{1.085933in}}%
\pgfpathlineto{\pgfqpoint{1.721859in}{0.785466in}}%
\pgfpathlineto{\pgfqpoint{1.882182in}{0.619810in}}%
\pgfpathlineto{\pgfqpoint{1.972773in}{0.528000in}}%
\pgfpathmoveto{\pgfqpoint{4.768000in}{1.699795in}}%
\pgfpathlineto{\pgfqpoint{4.505862in}{2.001165in}}%
\pgfpathlineto{\pgfqpoint{4.447354in}{2.066940in}}%
\pgfpathlineto{\pgfqpoint{4.319226in}{2.208000in}}%
\pgfpathlineto{\pgfqpoint{4.180524in}{2.357333in}}%
\pgfpathlineto{\pgfqpoint{4.046545in}{2.498489in}}%
\pgfpathlineto{\pgfqpoint{3.926303in}{2.622628in}}%
\pgfpathlineto{\pgfqpoint{3.765980in}{2.784446in}}%
\pgfpathlineto{\pgfqpoint{3.445333in}{3.096358in}}%
\pgfpathlineto{\pgfqpoint{3.124687in}{3.393036in}}%
\pgfpathlineto{\pgfqpoint{2.804040in}{3.674937in}}%
\pgfpathlineto{\pgfqpoint{2.643717in}{3.810429in}}%
\pgfpathlineto{\pgfqpoint{2.504048in}{3.925333in}}%
\pgfpathlineto{\pgfqpoint{2.363152in}{4.038553in}}%
\pgfpathlineto{\pgfqpoint{2.122667in}{4.224000in}}%
\pgfpathlineto{\pgfqpoint{2.116931in}{4.224000in}}%
\pgfpathlineto{\pgfqpoint{2.214763in}{4.149333in}}%
\pgfpathlineto{\pgfqpoint{2.497832in}{3.925333in}}%
\pgfpathlineto{\pgfqpoint{2.767271in}{3.701333in}}%
\pgfpathlineto{\pgfqpoint{2.924283in}{3.565902in}}%
\pgfpathlineto{\pgfqpoint{3.084606in}{3.424022in}}%
\pgfpathlineto{\pgfqpoint{3.244929in}{3.278462in}}%
\pgfpathlineto{\pgfqpoint{3.405253in}{3.129165in}}%
\pgfpathlineto{\pgfqpoint{3.565576in}{2.976073in}}%
\pgfpathlineto{\pgfqpoint{3.851603in}{2.693333in}}%
\pgfpathlineto{\pgfqpoint{3.997614in}{2.544000in}}%
\pgfpathlineto{\pgfqpoint{4.287030in}{2.237559in}}%
\pgfpathlineto{\pgfqpoint{4.416194in}{2.096000in}}%
\pgfpathlineto{\pgfqpoint{4.549424in}{1.946667in}}%
\pgfpathlineto{\pgfqpoint{4.679666in}{1.797333in}}%
\pgfpathlineto{\pgfqpoint{4.768000in}{1.693994in}}%
\pgfpathlineto{\pgfqpoint{4.768000in}{1.693994in}}%
\pgfusepath{fill}%
\end{pgfscope}%
\begin{pgfscope}%
\pgfpathrectangle{\pgfqpoint{0.800000in}{0.528000in}}{\pgfqpoint{3.968000in}{3.696000in}}%
\pgfusepath{clip}%
\pgfsetbuttcap%
\pgfsetroundjoin%
\definecolor{currentfill}{rgb}{0.252194,0.269783,0.531579}%
\pgfsetfillcolor{currentfill}%
\pgfsetlinewidth{0.000000pt}%
\definecolor{currentstroke}{rgb}{0.000000,0.000000,0.000000}%
\pgfsetstrokecolor{currentstroke}%
\pgfsetdash{}{0pt}%
\pgfpathmoveto{\pgfqpoint{1.972773in}{0.528000in}}%
\pgfpathlineto{\pgfqpoint{1.842101in}{0.660845in}}%
\pgfpathlineto{\pgfqpoint{1.718168in}{0.789333in}}%
\pgfpathlineto{\pgfqpoint{1.422058in}{1.107417in}}%
\pgfpathlineto{\pgfqpoint{1.361131in}{1.174440in}}%
\pgfpathlineto{\pgfqpoint{1.080566in}{1.493818in}}%
\pgfpathlineto{\pgfqpoint{0.857394in}{1.760000in}}%
\pgfpathlineto{\pgfqpoint{0.800000in}{1.830263in}}%
\pgfpathlineto{\pgfqpoint{0.800000in}{1.824225in}}%
\pgfpathlineto{\pgfqpoint{0.920242in}{1.677911in}}%
\pgfpathlineto{\pgfqpoint{1.168221in}{1.386667in}}%
\pgfpathlineto{\pgfqpoint{1.299861in}{1.237333in}}%
\pgfpathlineto{\pgfqpoint{1.586049in}{0.924166in}}%
\pgfpathlineto{\pgfqpoint{1.646145in}{0.859857in}}%
\pgfpathlineto{\pgfqpoint{1.802020in}{0.696890in}}%
\pgfpathlineto{\pgfqpoint{1.945984in}{0.550095in}}%
\pgfpathlineto{\pgfqpoint{1.967590in}{0.528000in}}%
\pgfpathmoveto{\pgfqpoint{4.768000in}{1.705596in}}%
\pgfpathlineto{\pgfqpoint{4.526400in}{1.984000in}}%
\pgfpathlineto{\pgfqpoint{4.246949in}{2.291643in}}%
\pgfpathlineto{\pgfqpoint{4.101566in}{2.445916in}}%
\pgfpathlineto{\pgfqpoint{4.043793in}{2.506667in}}%
\pgfpathlineto{\pgfqpoint{3.725899in}{2.829439in}}%
\pgfpathlineto{\pgfqpoint{3.442658in}{3.104000in}}%
\pgfpathlineto{\pgfqpoint{3.283192in}{3.253333in}}%
\pgfpathlineto{\pgfqpoint{3.119604in}{3.402667in}}%
\pgfpathlineto{\pgfqpoint{2.964364in}{3.540837in}}%
\pgfpathlineto{\pgfqpoint{2.643717in}{3.815465in}}%
\pgfpathlineto{\pgfqpoint{2.510264in}{3.925333in}}%
\pgfpathlineto{\pgfqpoint{2.363152in}{4.043585in}}%
\pgfpathlineto{\pgfqpoint{2.130489in}{4.224000in}}%
\pgfpathlineto{\pgfqpoint{2.123791in}{4.224000in}}%
\pgfpathlineto{\pgfqpoint{2.411504in}{4.000000in}}%
\pgfpathlineto{\pgfqpoint{2.563556in}{3.876719in}}%
\pgfpathlineto{\pgfqpoint{2.884202in}{3.605823in}}%
\pgfpathlineto{\pgfqpoint{3.204848in}{3.320270in}}%
\pgfpathlineto{\pgfqpoint{3.365172in}{3.171938in}}%
\pgfpathlineto{\pgfqpoint{3.685818in}{2.863878in}}%
\pgfpathlineto{\pgfqpoint{3.968589in}{2.579279in}}%
\pgfpathlineto{\pgfqpoint{4.126707in}{2.414391in}}%
\pgfpathlineto{\pgfqpoint{4.271651in}{2.259658in}}%
\pgfpathlineto{\pgfqpoint{4.407273in}{2.111369in}}%
\pgfpathlineto{\pgfqpoint{4.668030in}{1.816216in}}%
\pgfpathlineto{\pgfqpoint{4.727919in}{1.746860in}}%
\pgfpathlineto{\pgfqpoint{4.768000in}{1.699795in}}%
\pgfpathlineto{\pgfqpoint{4.768000in}{1.699795in}}%
\pgfusepath{fill}%
\end{pgfscope}%
\begin{pgfscope}%
\pgfpathrectangle{\pgfqpoint{0.800000in}{0.528000in}}{\pgfqpoint{3.968000in}{3.696000in}}%
\pgfusepath{clip}%
\pgfsetbuttcap%
\pgfsetroundjoin%
\definecolor{currentfill}{rgb}{0.250425,0.274290,0.533103}%
\pgfsetfillcolor{currentfill}%
\pgfsetlinewidth{0.000000pt}%
\definecolor{currentstroke}{rgb}{0.000000,0.000000,0.000000}%
\pgfsetstrokecolor{currentstroke}%
\pgfsetdash{}{0pt}%
\pgfpathmoveto{\pgfqpoint{1.967590in}{0.528000in}}%
\pgfpathlineto{\pgfqpoint{1.842101in}{0.655610in}}%
\pgfpathlineto{\pgfqpoint{1.698545in}{0.804951in}}%
\pgfpathlineto{\pgfqpoint{1.641697in}{0.864563in}}%
\pgfpathlineto{\pgfqpoint{1.481374in}{1.036797in}}%
\pgfpathlineto{\pgfqpoint{1.219009in}{1.328953in}}%
\pgfpathlineto{\pgfqpoint{1.160727in}{1.395273in}}%
\pgfpathlineto{\pgfqpoint{1.039592in}{1.536000in}}%
\pgfpathlineto{\pgfqpoint{0.800000in}{1.824225in}}%
\pgfpathlineto{\pgfqpoint{0.800000in}{1.818188in}}%
\pgfpathlineto{\pgfqpoint{0.920242in}{1.672034in}}%
\pgfpathlineto{\pgfqpoint{1.174200in}{1.374117in}}%
\pgfpathlineto{\pgfqpoint{1.306900in}{1.224152in}}%
\pgfpathlineto{\pgfqpoint{1.361754in}{1.162667in}}%
\pgfpathlineto{\pgfqpoint{1.641697in}{0.859282in}}%
\pgfpathlineto{\pgfqpoint{1.802020in}{0.691648in}}%
\pgfpathlineto{\pgfqpoint{1.925474in}{0.565333in}}%
\pgfpathlineto{\pgfqpoint{1.962407in}{0.528000in}}%
\pgfpathmoveto{\pgfqpoint{4.768000in}{1.711397in}}%
\pgfpathlineto{\pgfqpoint{4.527515in}{1.988280in}}%
\pgfpathlineto{\pgfqpoint{4.246949in}{2.296991in}}%
\pgfpathlineto{\pgfqpoint{4.104225in}{2.448392in}}%
\pgfpathlineto{\pgfqpoint{4.046545in}{2.509067in}}%
\pgfpathlineto{\pgfqpoint{3.725899in}{2.834593in}}%
\pgfpathlineto{\pgfqpoint{3.445333in}{3.106527in}}%
\pgfpathlineto{\pgfqpoint{3.285010in}{3.256673in}}%
\pgfpathlineto{\pgfqpoint{3.124687in}{3.403141in}}%
\pgfpathlineto{\pgfqpoint{2.964364in}{3.545868in}}%
\pgfpathlineto{\pgfqpoint{2.643717in}{3.820446in}}%
\pgfpathlineto{\pgfqpoint{2.516481in}{3.925333in}}%
\pgfpathlineto{\pgfqpoint{2.363152in}{4.048617in}}%
\pgfpathlineto{\pgfqpoint{2.137188in}{4.224000in}}%
\pgfpathlineto{\pgfqpoint{2.130489in}{4.224000in}}%
\pgfpathlineto{\pgfqpoint{2.417792in}{4.000000in}}%
\pgfpathlineto{\pgfqpoint{2.563556in}{3.881782in}}%
\pgfpathlineto{\pgfqpoint{2.884202in}{3.610841in}}%
\pgfpathlineto{\pgfqpoint{3.204848in}{3.325339in}}%
\pgfpathlineto{\pgfqpoint{3.365172in}{3.177033in}}%
\pgfpathlineto{\pgfqpoint{3.685818in}{2.869026in}}%
\pgfpathlineto{\pgfqpoint{3.971642in}{2.581333in}}%
\pgfpathlineto{\pgfqpoint{4.115045in}{2.432000in}}%
\pgfpathlineto{\pgfqpoint{4.407273in}{2.116858in}}%
\pgfpathlineto{\pgfqpoint{4.670767in}{1.818766in}}%
\pgfpathlineto{\pgfqpoint{4.727919in}{1.752653in}}%
\pgfpathlineto{\pgfqpoint{4.768000in}{1.705596in}}%
\pgfpathlineto{\pgfqpoint{4.768000in}{1.705596in}}%
\pgfusepath{fill}%
\end{pgfscope}%
\begin{pgfscope}%
\pgfpathrectangle{\pgfqpoint{0.800000in}{0.528000in}}{\pgfqpoint{3.968000in}{3.696000in}}%
\pgfusepath{clip}%
\pgfsetbuttcap%
\pgfsetroundjoin%
\definecolor{currentfill}{rgb}{0.250425,0.274290,0.533103}%
\pgfsetfillcolor{currentfill}%
\pgfsetlinewidth{0.000000pt}%
\definecolor{currentstroke}{rgb}{0.000000,0.000000,0.000000}%
\pgfsetstrokecolor{currentstroke}%
\pgfsetdash{}{0pt}%
\pgfpathmoveto{\pgfqpoint{1.962407in}{0.528000in}}%
\pgfpathlineto{\pgfqpoint{1.842101in}{0.650375in}}%
\pgfpathlineto{\pgfqpoint{1.695864in}{0.802454in}}%
\pgfpathlineto{\pgfqpoint{1.637248in}{0.864000in}}%
\pgfpathlineto{\pgfqpoint{1.498019in}{1.013333in}}%
\pgfpathlineto{\pgfqpoint{1.361131in}{1.163358in}}%
\pgfpathlineto{\pgfqpoint{1.080566in}{1.482385in}}%
\pgfpathlineto{\pgfqpoint{0.840081in}{1.769056in}}%
\pgfpathlineto{\pgfqpoint{0.800000in}{1.818188in}}%
\pgfpathlineto{\pgfqpoint{0.800000in}{1.812150in}}%
\pgfpathlineto{\pgfqpoint{0.920242in}{1.666157in}}%
\pgfpathlineto{\pgfqpoint{1.160727in}{1.383932in}}%
\pgfpathlineto{\pgfqpoint{1.304220in}{1.221656in}}%
\pgfpathlineto{\pgfqpoint{1.361131in}{1.157916in}}%
\pgfpathlineto{\pgfqpoint{1.641697in}{0.854013in}}%
\pgfpathlineto{\pgfqpoint{1.802020in}{0.686407in}}%
\pgfpathlineto{\pgfqpoint{1.922263in}{0.563401in}}%
\pgfpathlineto{\pgfqpoint{1.957333in}{0.528000in}}%
\pgfpathlineto{\pgfqpoint{1.962343in}{0.528000in}}%
\pgfpathmoveto{\pgfqpoint{4.768000in}{1.717198in}}%
\pgfpathlineto{\pgfqpoint{4.527515in}{1.993791in}}%
\pgfpathlineto{\pgfqpoint{4.246949in}{2.302340in}}%
\pgfpathlineto{\pgfqpoint{4.125162in}{2.432000in}}%
\pgfpathlineto{\pgfqpoint{3.806061in}{2.759866in}}%
\pgfpathlineto{\pgfqpoint{3.525495in}{3.035080in}}%
\pgfpathlineto{\pgfqpoint{3.365172in}{3.187063in}}%
\pgfpathlineto{\pgfqpoint{3.204848in}{3.335338in}}%
\pgfpathlineto{\pgfqpoint{3.044525in}{3.479961in}}%
\pgfpathlineto{\pgfqpoint{2.723879in}{3.758136in}}%
\pgfpathlineto{\pgfqpoint{2.563556in}{3.891837in}}%
\pgfpathlineto{\pgfqpoint{2.430369in}{4.000000in}}%
\pgfpathlineto{\pgfqpoint{2.154182in}{4.216022in}}%
\pgfpathlineto{\pgfqpoint{2.143886in}{4.224000in}}%
\pgfpathlineto{\pgfqpoint{2.137188in}{4.224000in}}%
\pgfpathlineto{\pgfqpoint{2.403232in}{4.016704in}}%
\pgfpathlineto{\pgfqpoint{2.723879in}{3.753142in}}%
\pgfpathlineto{\pgfqpoint{3.044525in}{3.474967in}}%
\pgfpathlineto{\pgfqpoint{3.206117in}{3.329182in}}%
\pgfpathlineto{\pgfqpoint{3.248150in}{3.290667in}}%
\pgfpathlineto{\pgfqpoint{3.408513in}{3.141333in}}%
\pgfpathlineto{\pgfqpoint{3.565576in}{2.991458in}}%
\pgfpathlineto{\pgfqpoint{3.886222in}{2.673817in}}%
\pgfpathlineto{\pgfqpoint{4.166788in}{2.382622in}}%
\pgfpathlineto{\pgfqpoint{4.294829in}{2.245333in}}%
\pgfpathlineto{\pgfqpoint{4.431062in}{2.096000in}}%
\pgfpathlineto{\pgfqpoint{4.567596in}{1.942963in}}%
\pgfpathlineto{\pgfqpoint{4.694473in}{1.797333in}}%
\pgfpathlineto{\pgfqpoint{4.768000in}{1.711397in}}%
\pgfpathlineto{\pgfqpoint{4.768000in}{1.711397in}}%
\pgfusepath{fill}%
\end{pgfscope}%
\begin{pgfscope}%
\pgfpathrectangle{\pgfqpoint{0.800000in}{0.528000in}}{\pgfqpoint{3.968000in}{3.696000in}}%
\pgfusepath{clip}%
\pgfsetbuttcap%
\pgfsetroundjoin%
\definecolor{currentfill}{rgb}{0.250425,0.274290,0.533103}%
\pgfsetfillcolor{currentfill}%
\pgfsetlinewidth{0.000000pt}%
\definecolor{currentstroke}{rgb}{0.000000,0.000000,0.000000}%
\pgfsetstrokecolor{currentstroke}%
\pgfsetdash{}{0pt}%
\pgfpathmoveto{\pgfqpoint{1.957333in}{0.528000in}}%
\pgfpathlineto{\pgfqpoint{1.641697in}{0.854013in}}%
\pgfpathlineto{\pgfqpoint{1.361131in}{1.157916in}}%
\pgfpathlineto{\pgfqpoint{1.120646in}{1.430126in}}%
\pgfpathlineto{\pgfqpoint{0.998115in}{1.573333in}}%
\pgfpathlineto{\pgfqpoint{0.800000in}{1.812150in}}%
\pgfpathlineto{\pgfqpoint{0.800000in}{1.806112in}}%
\pgfpathlineto{\pgfqpoint{0.920242in}{1.660280in}}%
\pgfpathlineto{\pgfqpoint{1.160727in}{1.378352in}}%
\pgfpathlineto{\pgfqpoint{1.285011in}{1.237333in}}%
\pgfpathlineto{\pgfqpoint{1.578052in}{0.916718in}}%
\pgfpathlineto{\pgfqpoint{1.641697in}{0.848743in}}%
\pgfpathlineto{\pgfqpoint{1.785135in}{0.698939in}}%
\pgfpathlineto{\pgfqpoint{1.842101in}{0.639907in}}%
\pgfpathlineto{\pgfqpoint{1.952261in}{0.528000in}}%
\pgfpathmoveto{\pgfqpoint{4.768000in}{1.722992in}}%
\pgfpathlineto{\pgfqpoint{4.487434in}{2.044251in}}%
\pgfpathlineto{\pgfqpoint{4.367192in}{2.177313in}}%
\pgfpathlineto{\pgfqpoint{4.222029in}{2.334121in}}%
\pgfpathlineto{\pgfqpoint{4.155929in}{2.404781in}}%
\pgfpathlineto{\pgfqpoint{4.006465in}{2.561033in}}%
\pgfpathlineto{\pgfqpoint{3.862161in}{2.708255in}}%
\pgfpathlineto{\pgfqpoint{3.803096in}{2.768000in}}%
\pgfpathlineto{\pgfqpoint{3.649026in}{2.920396in}}%
\pgfpathlineto{\pgfqpoint{3.590110in}{2.977519in}}%
\pgfpathlineto{\pgfqpoint{3.525495in}{3.040105in}}%
\pgfpathlineto{\pgfqpoint{3.365172in}{3.192062in}}%
\pgfpathlineto{\pgfqpoint{3.204848in}{3.340312in}}%
\pgfpathlineto{\pgfqpoint{3.044525in}{3.484911in}}%
\pgfpathlineto{\pgfqpoint{2.723879in}{3.763130in}}%
\pgfpathlineto{\pgfqpoint{2.563556in}{3.896805in}}%
\pgfpathlineto{\pgfqpoint{2.436657in}{4.000000in}}%
\pgfpathlineto{\pgfqpoint{2.150584in}{4.224000in}}%
\pgfpathlineto{\pgfqpoint{2.143886in}{4.224000in}}%
\pgfpathlineto{\pgfqpoint{2.403232in}{4.021742in}}%
\pgfpathlineto{\pgfqpoint{2.723879in}{3.758136in}}%
\pgfpathlineto{\pgfqpoint{3.047469in}{3.477333in}}%
\pgfpathlineto{\pgfqpoint{3.204848in}{3.335338in}}%
\pgfpathlineto{\pgfqpoint{3.525495in}{3.035080in}}%
\pgfpathlineto{\pgfqpoint{3.821005in}{2.744587in}}%
\pgfpathlineto{\pgfqpoint{3.886222in}{2.678998in}}%
\pgfpathlineto{\pgfqpoint{4.166788in}{2.387956in}}%
\pgfpathlineto{\pgfqpoint{4.299772in}{2.245333in}}%
\pgfpathlineto{\pgfqpoint{4.436018in}{2.096000in}}%
\pgfpathlineto{\pgfqpoint{4.569265in}{1.946667in}}%
\pgfpathlineto{\pgfqpoint{4.699352in}{1.797333in}}%
\pgfpathlineto{\pgfqpoint{4.768000in}{1.717198in}}%
\pgfpathlineto{\pgfqpoint{4.768000in}{1.722667in}}%
\pgfpathlineto{\pgfqpoint{4.768000in}{1.722667in}}%
\pgfusepath{fill}%
\end{pgfscope}%
\begin{pgfscope}%
\pgfpathrectangle{\pgfqpoint{0.800000in}{0.528000in}}{\pgfqpoint{3.968000in}{3.696000in}}%
\pgfusepath{clip}%
\pgfsetbuttcap%
\pgfsetroundjoin%
\definecolor{currentfill}{rgb}{0.250425,0.274290,0.533103}%
\pgfsetfillcolor{currentfill}%
\pgfsetlinewidth{0.000000pt}%
\definecolor{currentstroke}{rgb}{0.000000,0.000000,0.000000}%
\pgfsetstrokecolor{currentstroke}%
\pgfsetdash{}{0pt}%
\pgfpathmoveto{\pgfqpoint{1.952261in}{0.528000in}}%
\pgfpathlineto{\pgfqpoint{1.641697in}{0.848743in}}%
\pgfpathlineto{\pgfqpoint{1.361131in}{1.152489in}}%
\pgfpathlineto{\pgfqpoint{1.120646in}{1.424418in}}%
\pgfpathlineto{\pgfqpoint{0.993263in}{1.573333in}}%
\pgfpathlineto{\pgfqpoint{0.800000in}{1.806112in}}%
\pgfpathlineto{\pgfqpoint{0.800000in}{1.800075in}}%
\pgfpathlineto{\pgfqpoint{0.894526in}{1.685333in}}%
\pgfpathlineto{\pgfqpoint{1.020097in}{1.536000in}}%
\pgfpathlineto{\pgfqpoint{1.148622in}{1.386667in}}%
\pgfpathlineto{\pgfqpoint{1.280970in}{1.236334in}}%
\pgfpathlineto{\pgfqpoint{1.575387in}{0.914235in}}%
\pgfpathlineto{\pgfqpoint{1.641697in}{0.843474in}}%
\pgfpathlineto{\pgfqpoint{1.782494in}{0.696479in}}%
\pgfpathlineto{\pgfqpoint{1.842101in}{0.634774in}}%
\pgfpathlineto{\pgfqpoint{1.947188in}{0.528000in}}%
\pgfpathmoveto{\pgfqpoint{4.768000in}{1.728669in}}%
\pgfpathlineto{\pgfqpoint{4.487434in}{2.049755in}}%
\pgfpathlineto{\pgfqpoint{4.367192in}{2.182683in}}%
\pgfpathlineto{\pgfqpoint{4.224674in}{2.336585in}}%
\pgfpathlineto{\pgfqpoint{4.166788in}{2.398546in}}%
\pgfpathlineto{\pgfqpoint{4.006465in}{2.566235in}}%
\pgfpathlineto{\pgfqpoint{3.864801in}{2.710714in}}%
\pgfpathlineto{\pgfqpoint{3.806061in}{2.770159in}}%
\pgfpathlineto{\pgfqpoint{3.645737in}{2.928714in}}%
\pgfpathlineto{\pgfqpoint{3.485414in}{3.083462in}}%
\pgfpathlineto{\pgfqpoint{3.325091in}{3.234463in}}%
\pgfpathlineto{\pgfqpoint{3.164768in}{3.381770in}}%
\pgfpathlineto{\pgfqpoint{3.004444in}{3.525440in}}%
\pgfpathlineto{\pgfqpoint{2.844121in}{3.665525in}}%
\pgfpathlineto{\pgfqpoint{2.523475in}{3.934638in}}%
\pgfpathlineto{\pgfqpoint{2.396310in}{4.037333in}}%
\pgfpathlineto{\pgfqpoint{2.157282in}{4.224000in}}%
\pgfpathlineto{\pgfqpoint{2.150584in}{4.224000in}}%
\pgfpathlineto{\pgfqpoint{2.389985in}{4.037333in}}%
\pgfpathlineto{\pgfqpoint{2.525932in}{3.927622in}}%
\pgfpathlineto{\pgfqpoint{2.603636in}{3.863717in}}%
\pgfpathlineto{\pgfqpoint{2.926181in}{3.589333in}}%
\pgfpathlineto{\pgfqpoint{3.084606in}{3.449100in}}%
\pgfpathlineto{\pgfqpoint{3.405253in}{3.154422in}}%
\pgfpathlineto{\pgfqpoint{3.536697in}{3.029333in}}%
\pgfpathlineto{\pgfqpoint{3.690269in}{2.880000in}}%
\pgfpathlineto{\pgfqpoint{4.006465in}{2.561033in}}%
\pgfpathlineto{\pgfqpoint{4.155929in}{2.404781in}}%
\pgfpathlineto{\pgfqpoint{4.287030in}{2.264494in}}%
\pgfpathlineto{\pgfqpoint{4.567596in}{1.954078in}}%
\pgfpathlineto{\pgfqpoint{4.768000in}{1.722992in}}%
\pgfpathlineto{\pgfqpoint{4.768000in}{1.722992in}}%
\pgfusepath{fill}%
\end{pgfscope}%
\begin{pgfscope}%
\pgfpathrectangle{\pgfqpoint{0.800000in}{0.528000in}}{\pgfqpoint{3.968000in}{3.696000in}}%
\pgfusepath{clip}%
\pgfsetbuttcap%
\pgfsetroundjoin%
\definecolor{currentfill}{rgb}{0.248629,0.278775,0.534556}%
\pgfsetfillcolor{currentfill}%
\pgfsetlinewidth{0.000000pt}%
\definecolor{currentstroke}{rgb}{0.000000,0.000000,0.000000}%
\pgfsetstrokecolor{currentstroke}%
\pgfsetdash{}{0pt}%
\pgfpathmoveto{\pgfqpoint{1.947188in}{0.528000in}}%
\pgfpathlineto{\pgfqpoint{1.641697in}{0.843474in}}%
\pgfpathlineto{\pgfqpoint{1.353551in}{1.155606in}}%
\pgfpathlineto{\pgfqpoint{1.298860in}{1.216664in}}%
\pgfpathlineto{\pgfqpoint{1.240889in}{1.281519in}}%
\pgfpathlineto{\pgfqpoint{0.960323in}{1.606566in}}%
\pgfpathlineto{\pgfqpoint{0.840081in}{1.751171in}}%
\pgfpathlineto{\pgfqpoint{0.800000in}{1.800075in}}%
\pgfpathlineto{\pgfqpoint{0.800000in}{1.794110in}}%
\pgfpathlineto{\pgfqpoint{1.062081in}{1.481449in}}%
\pgfpathlineto{\pgfqpoint{1.120646in}{1.413234in}}%
\pgfpathlineto{\pgfqpoint{1.242030in}{1.274667in}}%
\pgfpathlineto{\pgfqpoint{1.521455in}{0.966542in}}%
\pgfpathlineto{\pgfqpoint{1.681778in}{0.795941in}}%
\pgfpathlineto{\pgfqpoint{1.942116in}{0.528000in}}%
\pgfpathmoveto{\pgfqpoint{4.768000in}{1.734346in}}%
\pgfpathlineto{\pgfqpoint{4.517706in}{2.021333in}}%
\pgfpathlineto{\pgfqpoint{4.383052in}{2.170667in}}%
\pgfpathlineto{\pgfqpoint{4.245443in}{2.320000in}}%
\pgfpathlineto{\pgfqpoint{3.960583in}{2.618667in}}%
\pgfpathlineto{\pgfqpoint{3.813269in}{2.768000in}}%
\pgfpathlineto{\pgfqpoint{3.685818in}{2.894484in}}%
\pgfpathlineto{\pgfqpoint{3.525495in}{3.050153in}}%
\pgfpathlineto{\pgfqpoint{3.365172in}{3.202060in}}%
\pgfpathlineto{\pgfqpoint{3.204848in}{3.350261in}}%
\pgfpathlineto{\pgfqpoint{3.044525in}{3.494810in}}%
\pgfpathlineto{\pgfqpoint{2.720457in}{3.776000in}}%
\pgfpathlineto{\pgfqpoint{2.563556in}{3.906743in}}%
\pgfpathlineto{\pgfqpoint{2.402635in}{4.037333in}}%
\pgfpathlineto{\pgfqpoint{2.162747in}{4.224000in}}%
\pgfpathlineto{\pgfqpoint{2.157282in}{4.224000in}}%
\pgfpathlineto{\pgfqpoint{2.378281in}{4.051426in}}%
\pgfpathlineto{\pgfqpoint{2.443313in}{3.999705in}}%
\pgfpathlineto{\pgfqpoint{2.758644in}{3.738667in}}%
\pgfpathlineto{\pgfqpoint{2.888995in}{3.626667in}}%
\pgfpathlineto{\pgfqpoint{3.204848in}{3.345286in}}%
\pgfpathlineto{\pgfqpoint{3.525495in}{3.045129in}}%
\pgfpathlineto{\pgfqpoint{3.826295in}{2.749514in}}%
\pgfpathlineto{\pgfqpoint{3.886222in}{2.689361in}}%
\pgfpathlineto{\pgfqpoint{4.196561in}{2.366935in}}%
\pgfpathlineto{\pgfqpoint{4.327111in}{2.226399in}}%
\pgfpathlineto{\pgfqpoint{4.607677in}{1.914102in}}%
\pgfpathlineto{\pgfqpoint{4.768000in}{1.728669in}}%
\pgfpathlineto{\pgfqpoint{4.768000in}{1.728669in}}%
\pgfusepath{fill}%
\end{pgfscope}%
\begin{pgfscope}%
\pgfpathrectangle{\pgfqpoint{0.800000in}{0.528000in}}{\pgfqpoint{3.968000in}{3.696000in}}%
\pgfusepath{clip}%
\pgfsetbuttcap%
\pgfsetroundjoin%
\definecolor{currentfill}{rgb}{0.248629,0.278775,0.534556}%
\pgfsetfillcolor{currentfill}%
\pgfsetlinewidth{0.000000pt}%
\definecolor{currentstroke}{rgb}{0.000000,0.000000,0.000000}%
\pgfsetstrokecolor{currentstroke}%
\pgfsetdash{}{0pt}%
\pgfpathmoveto{\pgfqpoint{1.942116in}{0.528000in}}%
\pgfpathlineto{\pgfqpoint{1.641697in}{0.838205in}}%
\pgfpathlineto{\pgfqpoint{1.350914in}{1.153150in}}%
\pgfpathlineto{\pgfqpoint{1.280970in}{1.230892in}}%
\pgfpathlineto{\pgfqpoint{1.160727in}{1.367192in}}%
\pgfpathlineto{\pgfqpoint{1.040485in}{1.506329in}}%
\pgfpathlineto{\pgfqpoint{0.920242in}{1.648526in}}%
\pgfpathlineto{\pgfqpoint{0.800000in}{1.794110in}}%
\pgfpathlineto{\pgfqpoint{0.800000in}{1.788207in}}%
\pgfpathlineto{\pgfqpoint{1.042142in}{1.498667in}}%
\pgfpathlineto{\pgfqpoint{1.321051in}{1.180692in}}%
\pgfpathlineto{\pgfqpoint{1.601616in}{0.875449in}}%
\pgfpathlineto{\pgfqpoint{1.721859in}{0.748742in}}%
\pgfpathlineto{\pgfqpoint{1.937043in}{0.528000in}}%
\pgfpathmoveto{\pgfqpoint{4.768000in}{1.740022in}}%
\pgfpathlineto{\pgfqpoint{4.498511in}{2.048349in}}%
\pgfpathlineto{\pgfqpoint{4.367192in}{2.193423in}}%
\pgfpathlineto{\pgfqpoint{4.246949in}{2.323660in}}%
\pgfpathlineto{\pgfqpoint{3.958251in}{2.626242in}}%
\pgfpathlineto{\pgfqpoint{3.806061in}{2.780297in}}%
\pgfpathlineto{\pgfqpoint{3.645737in}{2.938800in}}%
\pgfpathlineto{\pgfqpoint{3.485414in}{3.093498in}}%
\pgfpathlineto{\pgfqpoint{3.340220in}{3.230092in}}%
\pgfpathlineto{\pgfqpoint{3.275186in}{3.290667in}}%
\pgfpathlineto{\pgfqpoint{2.964364in}{3.570670in}}%
\pgfpathlineto{\pgfqpoint{2.804040in}{3.709819in}}%
\pgfpathlineto{\pgfqpoint{2.483394in}{3.977196in}}%
\pgfpathlineto{\pgfqpoint{2.170462in}{4.224000in}}%
\pgfpathlineto{\pgfqpoint{2.163947in}{4.224000in}}%
\pgfpathlineto{\pgfqpoint{2.323071in}{4.100415in}}%
\pgfpathlineto{\pgfqpoint{2.631334in}{3.850667in}}%
\pgfpathlineto{\pgfqpoint{2.764523in}{3.738667in}}%
\pgfpathlineto{\pgfqpoint{3.084606in}{3.459012in}}%
\pgfpathlineto{\pgfqpoint{3.405253in}{3.164433in}}%
\pgfpathlineto{\pgfqpoint{3.565576in}{3.011591in}}%
\pgfpathlineto{\pgfqpoint{3.725899in}{2.854972in}}%
\pgfpathlineto{\pgfqpoint{3.867442in}{2.713173in}}%
\pgfpathlineto{\pgfqpoint{3.926303in}{2.653756in}}%
\pgfpathlineto{\pgfqpoint{4.086626in}{2.488112in}}%
\pgfpathlineto{\pgfqpoint{4.227318in}{2.339048in}}%
\pgfpathlineto{\pgfqpoint{4.287030in}{2.275205in}}%
\pgfpathlineto{\pgfqpoint{4.576451in}{1.954914in}}%
\pgfpathlineto{\pgfqpoint{4.616703in}{1.909333in}}%
\pgfpathlineto{\pgfqpoint{4.746046in}{1.760000in}}%
\pgfpathlineto{\pgfqpoint{4.768000in}{1.734346in}}%
\pgfpathlineto{\pgfqpoint{4.768000in}{1.734346in}}%
\pgfusepath{fill}%
\end{pgfscope}%
\begin{pgfscope}%
\pgfpathrectangle{\pgfqpoint{0.800000in}{0.528000in}}{\pgfqpoint{3.968000in}{3.696000in}}%
\pgfusepath{clip}%
\pgfsetbuttcap%
\pgfsetroundjoin%
\definecolor{currentfill}{rgb}{0.248629,0.278775,0.534556}%
\pgfsetfillcolor{currentfill}%
\pgfsetlinewidth{0.000000pt}%
\definecolor{currentstroke}{rgb}{0.000000,0.000000,0.000000}%
\pgfsetstrokecolor{currentstroke}%
\pgfsetdash{}{0pt}%
\pgfpathmoveto{\pgfqpoint{1.937043in}{0.528000in}}%
\pgfpathlineto{\pgfqpoint{1.641697in}{0.832935in}}%
\pgfpathlineto{\pgfqpoint{1.348277in}{1.150694in}}%
\pgfpathlineto{\pgfqpoint{1.280970in}{1.225449in}}%
\pgfpathlineto{\pgfqpoint{1.160727in}{1.361613in}}%
\pgfpathlineto{\pgfqpoint{1.040485in}{1.500604in}}%
\pgfpathlineto{\pgfqpoint{0.900518in}{1.666372in}}%
\pgfpathlineto{\pgfqpoint{0.800000in}{1.788207in}}%
\pgfpathlineto{\pgfqpoint{0.800000in}{1.782304in}}%
\pgfpathlineto{\pgfqpoint{1.040485in}{1.494960in}}%
\pgfpathlineto{\pgfqpoint{1.321051in}{1.175257in}}%
\pgfpathlineto{\pgfqpoint{1.601616in}{0.870173in}}%
\pgfpathlineto{\pgfqpoint{1.721859in}{0.743589in}}%
\pgfpathlineto{\pgfqpoint{1.931971in}{0.528000in}}%
\pgfpathmoveto{\pgfqpoint{4.768000in}{1.745699in}}%
\pgfpathlineto{\pgfqpoint{4.527515in}{2.021348in}}%
\pgfpathlineto{\pgfqpoint{4.367192in}{2.198793in}}%
\pgfpathlineto{\pgfqpoint{4.246949in}{2.328902in}}%
\pgfpathlineto{\pgfqpoint{3.966384in}{2.623025in}}%
\pgfpathlineto{\pgfqpoint{3.806061in}{2.785366in}}%
\pgfpathlineto{\pgfqpoint{3.645737in}{2.943844in}}%
\pgfpathlineto{\pgfqpoint{3.502145in}{3.082250in}}%
\pgfpathlineto{\pgfqpoint{3.440323in}{3.141333in}}%
\pgfpathlineto{\pgfqpoint{3.280594in}{3.290667in}}%
\pgfpathlineto{\pgfqpoint{2.964364in}{3.575607in}}%
\pgfpathlineto{\pgfqpoint{2.804040in}{3.714732in}}%
\pgfpathlineto{\pgfqpoint{2.483394in}{3.982152in}}%
\pgfpathlineto{\pgfqpoint{2.176976in}{4.224000in}}%
\pgfpathlineto{\pgfqpoint{2.170462in}{4.224000in}}%
\pgfpathlineto{\pgfqpoint{2.323071in}{4.105440in}}%
\pgfpathlineto{\pgfqpoint{2.637325in}{3.850667in}}%
\pgfpathlineto{\pgfqpoint{2.763960in}{3.744055in}}%
\pgfpathlineto{\pgfqpoint{2.912596in}{3.615781in}}%
\pgfpathlineto{\pgfqpoint{2.985549in}{3.552000in}}%
\pgfpathlineto{\pgfqpoint{3.124687in}{3.427950in}}%
\pgfpathlineto{\pgfqpoint{3.445333in}{3.131585in}}%
\pgfpathlineto{\pgfqpoint{3.605657in}{2.977829in}}%
\pgfpathlineto{\pgfqpoint{3.765980in}{2.820283in}}%
\pgfpathlineto{\pgfqpoint{3.908462in}{2.676715in}}%
\pgfpathlineto{\pgfqpoint{3.966384in}{2.617916in}}%
\pgfpathlineto{\pgfqpoint{4.126707in}{2.451292in}}%
\pgfpathlineto{\pgfqpoint{4.267218in}{2.301546in}}%
\pgfpathlineto{\pgfqpoint{4.327111in}{2.237124in}}%
\pgfpathlineto{\pgfqpoint{4.615246in}{1.916384in}}%
\pgfpathlineto{\pgfqpoint{4.681830in}{1.840263in}}%
\pgfpathlineto{\pgfqpoint{4.768000in}{1.740022in}}%
\pgfpathlineto{\pgfqpoint{4.768000in}{1.740022in}}%
\pgfusepath{fill}%
\end{pgfscope}%
\begin{pgfscope}%
\pgfpathrectangle{\pgfqpoint{0.800000in}{0.528000in}}{\pgfqpoint{3.968000in}{3.696000in}}%
\pgfusepath{clip}%
\pgfsetbuttcap%
\pgfsetroundjoin%
\definecolor{currentfill}{rgb}{0.248629,0.278775,0.534556}%
\pgfsetfillcolor{currentfill}%
\pgfsetlinewidth{0.000000pt}%
\definecolor{currentstroke}{rgb}{0.000000,0.000000,0.000000}%
\pgfsetstrokecolor{currentstroke}%
\pgfsetdash{}{0pt}%
\pgfpathmoveto{\pgfqpoint{1.931971in}{0.528000in}}%
\pgfpathlineto{\pgfqpoint{1.641697in}{0.827666in}}%
\pgfpathlineto{\pgfqpoint{1.345641in}{1.148238in}}%
\pgfpathlineto{\pgfqpoint{1.280970in}{1.220007in}}%
\pgfpathlineto{\pgfqpoint{1.160727in}{1.356033in}}%
\pgfpathlineto{\pgfqpoint{1.024905in}{1.513178in}}%
\pgfpathlineto{\pgfqpoint{0.911055in}{1.648000in}}%
\pgfpathlineto{\pgfqpoint{0.800000in}{1.782304in}}%
\pgfpathlineto{\pgfqpoint{0.800000in}{1.776401in}}%
\pgfpathlineto{\pgfqpoint{1.040485in}{1.489356in}}%
\pgfpathlineto{\pgfqpoint{1.321051in}{1.169822in}}%
\pgfpathlineto{\pgfqpoint{1.583537in}{0.884493in}}%
\pgfpathlineto{\pgfqpoint{1.641697in}{0.822480in}}%
\pgfpathlineto{\pgfqpoint{1.802020in}{0.655391in}}%
\pgfpathlineto{\pgfqpoint{1.926898in}{0.528000in}}%
\pgfpathmoveto{\pgfqpoint{4.768000in}{1.751375in}}%
\pgfpathlineto{\pgfqpoint{4.527515in}{2.026747in}}%
\pgfpathlineto{\pgfqpoint{4.397747in}{2.170667in}}%
\pgfpathlineto{\pgfqpoint{4.260132in}{2.320000in}}%
\pgfpathlineto{\pgfqpoint{3.966384in}{2.628120in}}%
\pgfpathlineto{\pgfqpoint{3.806061in}{2.790435in}}%
\pgfpathlineto{\pgfqpoint{3.645737in}{2.948887in}}%
\pgfpathlineto{\pgfqpoint{3.504781in}{3.084706in}}%
\pgfpathlineto{\pgfqpoint{3.445333in}{3.141603in}}%
\pgfpathlineto{\pgfqpoint{3.285010in}{3.291563in}}%
\pgfpathlineto{\pgfqpoint{2.964364in}{3.580545in}}%
\pgfpathlineto{\pgfqpoint{2.804040in}{3.719645in}}%
\pgfpathlineto{\pgfqpoint{2.483394in}{3.987109in}}%
\pgfpathlineto{\pgfqpoint{2.183491in}{4.224000in}}%
\pgfpathlineto{\pgfqpoint{2.176976in}{4.224000in}}%
\pgfpathlineto{\pgfqpoint{2.323071in}{4.110466in}}%
\pgfpathlineto{\pgfqpoint{2.622166in}{3.867926in}}%
\pgfpathlineto{\pgfqpoint{2.687872in}{3.813333in}}%
\pgfpathlineto{\pgfqpoint{2.844121in}{3.680283in}}%
\pgfpathlineto{\pgfqpoint{3.004444in}{3.540270in}}%
\pgfpathlineto{\pgfqpoint{3.164768in}{3.396674in}}%
\pgfpathlineto{\pgfqpoint{3.485414in}{3.098516in}}%
\pgfpathlineto{\pgfqpoint{3.806061in}{2.785366in}}%
\pgfpathlineto{\pgfqpoint{3.934118in}{2.656000in}}%
\pgfpathlineto{\pgfqpoint{4.082476in}{2.502801in}}%
\pgfpathlineto{\pgfqpoint{4.149958in}{2.432000in}}%
\pgfpathlineto{\pgfqpoint{4.289982in}{2.282667in}}%
\pgfpathlineto{\pgfqpoint{4.581748in}{1.959849in}}%
\pgfpathlineto{\pgfqpoint{4.647758in}{1.884932in}}%
\pgfpathlineto{\pgfqpoint{4.768000in}{1.745699in}}%
\pgfpathlineto{\pgfqpoint{4.768000in}{1.745699in}}%
\pgfusepath{fill}%
\end{pgfscope}%
\begin{pgfscope}%
\pgfpathrectangle{\pgfqpoint{0.800000in}{0.528000in}}{\pgfqpoint{3.968000in}{3.696000in}}%
\pgfusepath{clip}%
\pgfsetbuttcap%
\pgfsetroundjoin%
\definecolor{currentfill}{rgb}{0.246811,0.283237,0.535941}%
\pgfsetfillcolor{currentfill}%
\pgfsetlinewidth{0.000000pt}%
\definecolor{currentstroke}{rgb}{0.000000,0.000000,0.000000}%
\pgfsetstrokecolor{currentstroke}%
\pgfsetdash{}{0pt}%
\pgfpathmoveto{\pgfqpoint{1.926898in}{0.528000in}}%
\pgfpathlineto{\pgfqpoint{1.637734in}{0.826667in}}%
\pgfpathlineto{\pgfqpoint{1.343004in}{1.145782in}}%
\pgfpathlineto{\pgfqpoint{1.280970in}{1.214564in}}%
\pgfpathlineto{\pgfqpoint{1.160727in}{1.350453in}}%
\pgfpathlineto{\pgfqpoint{1.032514in}{1.498667in}}%
\pgfpathlineto{\pgfqpoint{0.906246in}{1.648000in}}%
\pgfpathlineto{\pgfqpoint{0.800000in}{1.776401in}}%
\pgfpathlineto{\pgfqpoint{0.800000in}{1.770498in}}%
\pgfpathlineto{\pgfqpoint{1.040485in}{1.483753in}}%
\pgfpathlineto{\pgfqpoint{1.289134in}{1.200000in}}%
\pgfpathlineto{\pgfqpoint{1.424363in}{1.050667in}}%
\pgfpathlineto{\pgfqpoint{1.562507in}{0.901333in}}%
\pgfpathlineto{\pgfqpoint{1.882182in}{0.568210in}}%
\pgfpathlineto{\pgfqpoint{1.922263in}{0.528000in}}%
\pgfpathmoveto{\pgfqpoint{4.768000in}{1.757052in}}%
\pgfpathlineto{\pgfqpoint{4.527515in}{2.032146in}}%
\pgfpathlineto{\pgfqpoint{4.376494in}{2.199336in}}%
\pgfpathlineto{\pgfqpoint{4.246949in}{2.339387in}}%
\pgfpathlineto{\pgfqpoint{3.966384in}{2.633215in}}%
\pgfpathlineto{\pgfqpoint{3.806061in}{2.795504in}}%
\pgfpathlineto{\pgfqpoint{3.664728in}{2.935022in}}%
\pgfpathlineto{\pgfqpoint{3.605657in}{2.992923in}}%
\pgfpathlineto{\pgfqpoint{3.445333in}{3.146522in}}%
\pgfpathlineto{\pgfqpoint{3.285010in}{3.296458in}}%
\pgfpathlineto{\pgfqpoint{2.959971in}{3.589333in}}%
\pgfpathlineto{\pgfqpoint{2.804040in}{3.724559in}}%
\pgfpathlineto{\pgfqpoint{2.483394in}{3.992065in}}%
\pgfpathlineto{\pgfqpoint{2.190006in}{4.224000in}}%
\pgfpathlineto{\pgfqpoint{2.183491in}{4.224000in}}%
\pgfpathlineto{\pgfqpoint{2.327400in}{4.112000in}}%
\pgfpathlineto{\pgfqpoint{2.604335in}{3.888000in}}%
\pgfpathlineto{\pgfqpoint{2.924283in}{3.615653in}}%
\pgfpathlineto{\pgfqpoint{3.084606in}{3.473880in}}%
\pgfpathlineto{\pgfqpoint{3.406067in}{3.178667in}}%
\pgfpathlineto{\pgfqpoint{3.565576in}{3.026683in}}%
\pgfpathlineto{\pgfqpoint{3.725899in}{2.870140in}}%
\pgfpathlineto{\pgfqpoint{3.886222in}{2.709764in}}%
\pgfpathlineto{\pgfqpoint{4.179128in}{2.406161in}}%
\pgfpathlineto{\pgfqpoint{4.246949in}{2.334145in}}%
\pgfpathlineto{\pgfqpoint{4.527515in}{2.026747in}}%
\pgfpathlineto{\pgfqpoint{4.768000in}{1.751375in}}%
\pgfpathlineto{\pgfqpoint{4.768000in}{1.751375in}}%
\pgfusepath{fill}%
\end{pgfscope}%
\begin{pgfscope}%
\pgfpathrectangle{\pgfqpoint{0.800000in}{0.528000in}}{\pgfqpoint{3.968000in}{3.696000in}}%
\pgfusepath{clip}%
\pgfsetbuttcap%
\pgfsetroundjoin%
\definecolor{currentfill}{rgb}{0.246811,0.283237,0.535941}%
\pgfsetfillcolor{currentfill}%
\pgfsetlinewidth{0.000000pt}%
\definecolor{currentstroke}{rgb}{0.000000,0.000000,0.000000}%
\pgfsetstrokecolor{currentstroke}%
\pgfsetdash{}{0pt}%
\pgfpathmoveto{\pgfqpoint{1.921835in}{0.528000in}}%
\pgfpathlineto{\pgfqpoint{1.775783in}{0.677333in}}%
\pgfpathlineto{\pgfqpoint{1.481374in}{0.988651in}}%
\pgfpathlineto{\pgfqpoint{1.340368in}{1.143326in}}%
\pgfpathlineto{\pgfqpoint{1.280970in}{1.209122in}}%
\pgfpathlineto{\pgfqpoint{1.140108in}{1.368539in}}%
\pgfpathlineto{\pgfqpoint{1.027717in}{1.498667in}}%
\pgfpathlineto{\pgfqpoint{0.901437in}{1.648000in}}%
\pgfpathlineto{\pgfqpoint{0.800000in}{1.770498in}}%
\pgfpathlineto{\pgfqpoint{0.800000in}{1.764595in}}%
\pgfpathlineto{\pgfqpoint{1.022920in}{1.498667in}}%
\pgfpathlineto{\pgfqpoint{1.152117in}{1.349333in}}%
\pgfpathlineto{\pgfqpoint{1.441293in}{1.026878in}}%
\pgfpathlineto{\pgfqpoint{1.721859in}{0.728129in}}%
\pgfpathlineto{\pgfqpoint{1.916869in}{0.528000in}}%
\pgfpathmoveto{\pgfqpoint{4.768000in}{1.762671in}}%
\pgfpathlineto{\pgfqpoint{4.625823in}{1.926235in}}%
\pgfpathlineto{\pgfqpoint{4.567596in}{1.992533in}}%
\pgfpathlineto{\pgfqpoint{4.409014in}{2.169045in}}%
\pgfpathlineto{\pgfqpoint{4.287030in}{2.301600in}}%
\pgfpathlineto{\pgfqpoint{4.006465in}{2.597133in}}%
\pgfpathlineto{\pgfqpoint{3.846141in}{2.760371in}}%
\pgfpathlineto{\pgfqpoint{3.549554in}{3.051743in}}%
\pgfpathlineto{\pgfqpoint{3.485414in}{3.113392in}}%
\pgfpathlineto{\pgfqpoint{3.325091in}{3.264215in}}%
\pgfpathlineto{\pgfqpoint{3.004444in}{3.555044in}}%
\pgfpathlineto{\pgfqpoint{2.683798in}{3.831450in}}%
\pgfpathlineto{\pgfqpoint{2.363152in}{4.093546in}}%
\pgfpathlineto{\pgfqpoint{2.196520in}{4.224000in}}%
\pgfpathlineto{\pgfqpoint{2.190006in}{4.224000in}}%
\pgfpathlineto{\pgfqpoint{2.323071in}{4.120358in}}%
\pgfpathlineto{\pgfqpoint{2.483394in}{3.992065in}}%
\pgfpathlineto{\pgfqpoint{2.610211in}{3.888000in}}%
\pgfpathlineto{\pgfqpoint{2.920878in}{3.623495in}}%
\pgfpathlineto{\pgfqpoint{2.964364in}{3.585482in}}%
\pgfpathlineto{\pgfqpoint{3.291266in}{3.290667in}}%
\pgfpathlineto{\pgfqpoint{3.606607in}{2.992000in}}%
\pgfpathlineto{\pgfqpoint{3.765980in}{2.835470in}}%
\pgfpathlineto{\pgfqpoint{3.926303in}{2.674152in}}%
\pgfpathlineto{\pgfqpoint{4.206869in}{2.382166in}}%
\pgfpathlineto{\pgfqpoint{4.349286in}{2.228655in}}%
\pgfpathlineto{\pgfqpoint{4.407273in}{2.165588in}}%
\pgfpathlineto{\pgfqpoint{4.537152in}{2.021333in}}%
\pgfpathlineto{\pgfqpoint{4.668677in}{1.872000in}}%
\pgfpathlineto{\pgfqpoint{4.768000in}{1.757052in}}%
\pgfpathlineto{\pgfqpoint{4.768000in}{1.760000in}}%
\pgfpathlineto{\pgfqpoint{4.768000in}{1.760000in}}%
\pgfusepath{fill}%
\end{pgfscope}%
\begin{pgfscope}%
\pgfpathrectangle{\pgfqpoint{0.800000in}{0.528000in}}{\pgfqpoint{3.968000in}{3.696000in}}%
\pgfusepath{clip}%
\pgfsetbuttcap%
\pgfsetroundjoin%
\definecolor{currentfill}{rgb}{0.246811,0.283237,0.535941}%
\pgfsetfillcolor{currentfill}%
\pgfsetlinewidth{0.000000pt}%
\definecolor{currentstroke}{rgb}{0.000000,0.000000,0.000000}%
\pgfsetstrokecolor{currentstroke}%
\pgfsetdash{}{0pt}%
\pgfpathmoveto{\pgfqpoint{1.916869in}{0.528000in}}%
\pgfpathlineto{\pgfqpoint{1.770804in}{0.677333in}}%
\pgfpathlineto{\pgfqpoint{1.481374in}{0.983354in}}%
\pgfpathlineto{\pgfqpoint{1.337731in}{1.140871in}}%
\pgfpathlineto{\pgfqpoint{1.280970in}{1.203680in}}%
\pgfpathlineto{\pgfqpoint{1.152117in}{1.349333in}}%
\pgfpathlineto{\pgfqpoint{1.022920in}{1.498667in}}%
\pgfpathlineto{\pgfqpoint{0.896627in}{1.648000in}}%
\pgfpathlineto{\pgfqpoint{0.800000in}{1.764595in}}%
\pgfpathlineto{\pgfqpoint{0.800000in}{1.758721in}}%
\pgfpathlineto{\pgfqpoint{0.933267in}{1.598535in}}%
\pgfpathlineto{\pgfqpoint{1.063822in}{1.445737in}}%
\pgfpathlineto{\pgfqpoint{1.120646in}{1.379851in}}%
\pgfpathlineto{\pgfqpoint{1.271736in}{1.208600in}}%
\pgfpathlineto{\pgfqpoint{1.401212in}{1.065353in}}%
\pgfpathlineto{\pgfqpoint{1.521455in}{0.934876in}}%
\pgfpathlineto{\pgfqpoint{1.681778in}{0.764853in}}%
\pgfpathlineto{\pgfqpoint{1.802020in}{0.639971in}}%
\pgfpathlineto{\pgfqpoint{1.911902in}{0.528000in}}%
\pgfpathmoveto{\pgfqpoint{4.768000in}{1.768229in}}%
\pgfpathlineto{\pgfqpoint{4.645776in}{1.909333in}}%
\pgfpathlineto{\pgfqpoint{4.354546in}{2.233555in}}%
\pgfpathlineto{\pgfqpoint{4.287030in}{2.306849in}}%
\pgfpathlineto{\pgfqpoint{4.006465in}{2.602235in}}%
\pgfpathlineto{\pgfqpoint{3.846141in}{2.765447in}}%
\pgfpathlineto{\pgfqpoint{3.552185in}{3.054194in}}%
\pgfpathlineto{\pgfqpoint{3.485414in}{3.118317in}}%
\pgfpathlineto{\pgfqpoint{3.325091in}{3.269115in}}%
\pgfpathlineto{\pgfqpoint{3.004444in}{3.559897in}}%
\pgfpathlineto{\pgfqpoint{2.683798in}{3.836345in}}%
\pgfpathlineto{\pgfqpoint{2.363152in}{4.098484in}}%
\pgfpathlineto{\pgfqpoint{2.202828in}{4.224000in}}%
\pgfpathlineto{\pgfqpoint{2.196520in}{4.224000in}}%
\pgfpathlineto{\pgfqpoint{2.292517in}{4.149333in}}%
\pgfpathlineto{\pgfqpoint{2.570990in}{3.925333in}}%
\pgfpathlineto{\pgfqpoint{2.880029in}{3.664000in}}%
\pgfpathlineto{\pgfqpoint{3.133161in}{3.440000in}}%
\pgfpathlineto{\pgfqpoint{3.445333in}{3.151441in}}%
\pgfpathlineto{\pgfqpoint{3.605657in}{2.997866in}}%
\pgfpathlineto{\pgfqpoint{3.899771in}{2.705954in}}%
\pgfpathlineto{\pgfqpoint{3.966384in}{2.638310in}}%
\pgfpathlineto{\pgfqpoint{4.126707in}{2.472125in}}%
\pgfpathlineto{\pgfqpoint{4.373389in}{2.208000in}}%
\pgfpathlineto{\pgfqpoint{4.508607in}{2.058667in}}%
\pgfpathlineto{\pgfqpoint{4.640930in}{1.909333in}}%
\pgfpathlineto{\pgfqpoint{4.768000in}{1.762671in}}%
\pgfpathlineto{\pgfqpoint{4.768000in}{1.762671in}}%
\pgfusepath{fill}%
\end{pgfscope}%
\begin{pgfscope}%
\pgfpathrectangle{\pgfqpoint{0.800000in}{0.528000in}}{\pgfqpoint{3.968000in}{3.696000in}}%
\pgfusepath{clip}%
\pgfsetbuttcap%
\pgfsetroundjoin%
\definecolor{currentfill}{rgb}{0.244972,0.287675,0.537260}%
\pgfsetfillcolor{currentfill}%
\pgfsetlinewidth{0.000000pt}%
\definecolor{currentstroke}{rgb}{0.000000,0.000000,0.000000}%
\pgfsetstrokecolor{currentstroke}%
\pgfsetdash{}{0pt}%
\pgfpathmoveto{\pgfqpoint{1.911902in}{0.528000in}}%
\pgfpathlineto{\pgfqpoint{1.761939in}{0.681349in}}%
\pgfpathlineto{\pgfqpoint{1.601616in}{0.849359in}}%
\pgfpathlineto{\pgfqpoint{1.481374in}{0.978057in}}%
\pgfpathlineto{\pgfqpoint{1.321051in}{1.153701in}}%
\pgfpathlineto{\pgfqpoint{1.200808in}{1.288496in}}%
\pgfpathlineto{\pgfqpoint{1.080566in}{1.425985in}}%
\pgfpathlineto{\pgfqpoint{0.829765in}{1.722667in}}%
\pgfpathlineto{\pgfqpoint{0.800000in}{1.758721in}}%
\pgfpathlineto{\pgfqpoint{0.800000in}{1.752947in}}%
\pgfpathlineto{\pgfqpoint{0.920242in}{1.608321in}}%
\pgfpathlineto{\pgfqpoint{1.061171in}{1.443268in}}%
\pgfpathlineto{\pgfqpoint{1.120646in}{1.374379in}}%
\pgfpathlineto{\pgfqpoint{1.243226in}{1.235156in}}%
\pgfpathlineto{\pgfqpoint{1.401212in}{1.060041in}}%
\pgfpathlineto{\pgfqpoint{1.521455in}{0.929690in}}%
\pgfpathlineto{\pgfqpoint{1.681778in}{0.759693in}}%
\pgfpathlineto{\pgfqpoint{1.802020in}{0.634929in}}%
\pgfpathlineto{\pgfqpoint{1.906936in}{0.528000in}}%
\pgfpathmoveto{\pgfqpoint{4.768000in}{1.773786in}}%
\pgfpathlineto{\pgfqpoint{4.647758in}{1.912537in}}%
\pgfpathlineto{\pgfqpoint{4.357177in}{2.236004in}}%
\pgfpathlineto{\pgfqpoint{4.287030in}{2.312099in}}%
\pgfpathlineto{\pgfqpoint{4.006465in}{2.607336in}}%
\pgfpathlineto{\pgfqpoint{3.878785in}{2.737594in}}%
\pgfpathlineto{\pgfqpoint{3.725899in}{2.890170in}}%
\pgfpathlineto{\pgfqpoint{3.565576in}{3.046479in}}%
\pgfpathlineto{\pgfqpoint{3.244929in}{3.348041in}}%
\pgfpathlineto{\pgfqpoint{2.924283in}{3.635219in}}%
\pgfpathlineto{\pgfqpoint{2.761011in}{3.776000in}}%
\pgfpathlineto{\pgfqpoint{2.603636in}{3.908113in}}%
\pgfpathlineto{\pgfqpoint{2.282990in}{4.166672in}}%
\pgfpathlineto{\pgfqpoint{2.209370in}{4.224000in}}%
\pgfpathlineto{\pgfqpoint{2.203029in}{4.224000in}}%
\pgfpathlineto{\pgfqpoint{2.363152in}{4.098484in}}%
\pgfpathlineto{\pgfqpoint{2.485777in}{4.000000in}}%
\pgfpathlineto{\pgfqpoint{2.643717in}{3.869896in}}%
\pgfpathlineto{\pgfqpoint{2.804040in}{3.734385in}}%
\pgfpathlineto{\pgfqpoint{2.964364in}{3.595247in}}%
\pgfpathlineto{\pgfqpoint{3.285010in}{3.306247in}}%
\pgfpathlineto{\pgfqpoint{3.605657in}{3.002809in}}%
\pgfpathlineto{\pgfqpoint{3.902360in}{2.708365in}}%
\pgfpathlineto{\pgfqpoint{3.966384in}{2.643404in}}%
\pgfpathlineto{\pgfqpoint{4.126707in}{2.477246in}}%
\pgfpathlineto{\pgfqpoint{4.409990in}{2.173198in}}%
\pgfpathlineto{\pgfqpoint{4.447354in}{2.132144in}}%
\pgfpathlineto{\pgfqpoint{4.592344in}{1.969718in}}%
\pgfpathlineto{\pgfqpoint{4.647758in}{1.907069in}}%
\pgfpathlineto{\pgfqpoint{4.768000in}{1.768229in}}%
\pgfpathlineto{\pgfqpoint{4.768000in}{1.768229in}}%
\pgfusepath{fill}%
\end{pgfscope}%
\begin{pgfscope}%
\pgfpathrectangle{\pgfqpoint{0.800000in}{0.528000in}}{\pgfqpoint{3.968000in}{3.696000in}}%
\pgfusepath{clip}%
\pgfsetbuttcap%
\pgfsetroundjoin%
\definecolor{currentfill}{rgb}{0.244972,0.287675,0.537260}%
\pgfsetfillcolor{currentfill}%
\pgfsetlinewidth{0.000000pt}%
\definecolor{currentstroke}{rgb}{0.000000,0.000000,0.000000}%
\pgfsetstrokecolor{currentstroke}%
\pgfsetdash{}{0pt}%
\pgfpathmoveto{\pgfqpoint{1.906936in}{0.528000in}}%
\pgfpathlineto{\pgfqpoint{1.760868in}{0.677333in}}%
\pgfpathlineto{\pgfqpoint{1.618176in}{0.826667in}}%
\pgfpathlineto{\pgfqpoint{1.321051in}{1.148376in}}%
\pgfpathlineto{\pgfqpoint{1.200808in}{1.283038in}}%
\pgfpathlineto{\pgfqpoint{1.061171in}{1.443268in}}%
\pgfpathlineto{\pgfqpoint{1.000404in}{1.513782in}}%
\pgfpathlineto{\pgfqpoint{0.880162in}{1.656198in}}%
\pgfpathlineto{\pgfqpoint{0.800000in}{1.752947in}}%
\pgfpathlineto{\pgfqpoint{0.800000in}{1.747173in}}%
\pgfpathlineto{\pgfqpoint{0.920242in}{1.602694in}}%
\pgfpathlineto{\pgfqpoint{1.040490in}{1.461333in}}%
\pgfpathlineto{\pgfqpoint{1.329822in}{1.133503in}}%
\pgfpathlineto{\pgfqpoint{1.384959in}{1.072861in}}%
\pgfpathlineto{\pgfqpoint{1.441293in}{1.011011in}}%
\pgfpathlineto{\pgfqpoint{1.601616in}{0.839013in}}%
\pgfpathlineto{\pgfqpoint{1.882182in}{0.548038in}}%
\pgfpathlineto{\pgfqpoint{1.901969in}{0.528000in}}%
\pgfpathmoveto{\pgfqpoint{4.768000in}{1.779344in}}%
\pgfpathlineto{\pgfqpoint{4.647758in}{1.917958in}}%
\pgfpathlineto{\pgfqpoint{4.359807in}{2.238454in}}%
\pgfpathlineto{\pgfqpoint{4.319166in}{2.282667in}}%
\pgfpathlineto{\pgfqpoint{4.179412in}{2.432000in}}%
\pgfpathlineto{\pgfqpoint{3.886222in}{2.735088in}}%
\pgfpathlineto{\pgfqpoint{3.725899in}{2.895132in}}%
\pgfpathlineto{\pgfqpoint{3.565576in}{3.051416in}}%
\pgfpathlineto{\pgfqpoint{3.244929in}{3.352930in}}%
\pgfpathlineto{\pgfqpoint{2.924283in}{3.640060in}}%
\pgfpathlineto{\pgfqpoint{2.763960in}{3.778361in}}%
\pgfpathlineto{\pgfqpoint{2.633712in}{3.888000in}}%
\pgfpathlineto{\pgfqpoint{2.483394in}{4.011672in}}%
\pgfpathlineto{\pgfqpoint{2.358559in}{4.112000in}}%
\pgfpathlineto{\pgfqpoint{2.215711in}{4.224000in}}%
\pgfpathlineto{\pgfqpoint{2.209370in}{4.224000in}}%
\pgfpathlineto{\pgfqpoint{2.363152in}{4.103422in}}%
\pgfpathlineto{\pgfqpoint{2.491750in}{4.000000in}}%
\pgfpathlineto{\pgfqpoint{2.643717in}{3.874785in}}%
\pgfpathlineto{\pgfqpoint{2.804762in}{3.738667in}}%
\pgfpathlineto{\pgfqpoint{2.964364in}{3.600094in}}%
\pgfpathlineto{\pgfqpoint{3.285010in}{3.311141in}}%
\pgfpathlineto{\pgfqpoint{3.605657in}{3.007751in}}%
\pgfpathlineto{\pgfqpoint{3.904949in}{2.710777in}}%
\pgfpathlineto{\pgfqpoint{3.966384in}{2.648499in}}%
\pgfpathlineto{\pgfqpoint{4.126707in}{2.482367in}}%
\pgfpathlineto{\pgfqpoint{4.412565in}{2.175596in}}%
\pgfpathlineto{\pgfqpoint{4.470967in}{2.111338in}}%
\pgfpathlineto{\pgfqpoint{4.607677in}{1.958077in}}%
\pgfpathlineto{\pgfqpoint{4.768000in}{1.773786in}}%
\pgfpathlineto{\pgfqpoint{4.768000in}{1.773786in}}%
\pgfusepath{fill}%
\end{pgfscope}%
\begin{pgfscope}%
\pgfpathrectangle{\pgfqpoint{0.800000in}{0.528000in}}{\pgfqpoint{3.968000in}{3.696000in}}%
\pgfusepath{clip}%
\pgfsetbuttcap%
\pgfsetroundjoin%
\definecolor{currentfill}{rgb}{0.244972,0.287675,0.537260}%
\pgfsetfillcolor{currentfill}%
\pgfsetlinewidth{0.000000pt}%
\definecolor{currentstroke}{rgb}{0.000000,0.000000,0.000000}%
\pgfsetstrokecolor{currentstroke}%
\pgfsetdash{}{0pt}%
\pgfpathmoveto{\pgfqpoint{1.901969in}{0.528000in}}%
\pgfpathlineto{\pgfqpoint{1.755991in}{0.677333in}}%
\pgfpathlineto{\pgfqpoint{1.613286in}{0.826667in}}%
\pgfpathlineto{\pgfqpoint{1.321051in}{1.143050in}}%
\pgfpathlineto{\pgfqpoint{1.200808in}{1.277581in}}%
\pgfpathlineto{\pgfqpoint{1.058521in}{1.440800in}}%
\pgfpathlineto{\pgfqpoint{1.000404in}{1.508171in}}%
\pgfpathlineto{\pgfqpoint{0.880162in}{1.650440in}}%
\pgfpathlineto{\pgfqpoint{0.800000in}{1.747173in}}%
\pgfpathlineto{\pgfqpoint{0.800000in}{1.741398in}}%
\pgfpathlineto{\pgfqpoint{0.920242in}{1.597067in}}%
\pgfpathlineto{\pgfqpoint{1.040485in}{1.455852in}}%
\pgfpathlineto{\pgfqpoint{1.321051in}{1.137725in}}%
\pgfpathlineto{\pgfqpoint{1.601616in}{0.833840in}}%
\pgfpathlineto{\pgfqpoint{1.882182in}{0.543009in}}%
\pgfpathlineto{\pgfqpoint{1.897003in}{0.528000in}}%
\pgfpathmoveto{\pgfqpoint{4.768000in}{1.784901in}}%
\pgfpathlineto{\pgfqpoint{4.647758in}{1.923379in}}%
\pgfpathlineto{\pgfqpoint{4.367192in}{2.235818in}}%
\pgfpathlineto{\pgfqpoint{4.086626in}{2.534499in}}%
\pgfpathlineto{\pgfqpoint{3.948445in}{2.676624in}}%
\pgfpathlineto{\pgfqpoint{3.886222in}{2.740074in}}%
\pgfpathlineto{\pgfqpoint{3.725899in}{2.900093in}}%
\pgfpathlineto{\pgfqpoint{3.565576in}{3.056353in}}%
\pgfpathlineto{\pgfqpoint{3.244929in}{3.357818in}}%
\pgfpathlineto{\pgfqpoint{2.924283in}{3.644901in}}%
\pgfpathlineto{\pgfqpoint{2.763960in}{3.783179in}}%
\pgfpathlineto{\pgfqpoint{2.603636in}{3.917880in}}%
\pgfpathlineto{\pgfqpoint{2.282990in}{4.176524in}}%
\pgfpathlineto{\pgfqpoint{2.222052in}{4.224000in}}%
\pgfpathlineto{\pgfqpoint{2.215711in}{4.224000in}}%
\pgfpathlineto{\pgfqpoint{2.363152in}{4.108360in}}%
\pgfpathlineto{\pgfqpoint{2.497722in}{4.000000in}}%
\pgfpathlineto{\pgfqpoint{2.643717in}{3.879674in}}%
\pgfpathlineto{\pgfqpoint{2.804040in}{3.744110in}}%
\pgfpathlineto{\pgfqpoint{2.964364in}{3.604941in}}%
\pgfpathlineto{\pgfqpoint{3.285010in}{3.316036in}}%
\pgfpathlineto{\pgfqpoint{3.605657in}{3.012694in}}%
\pgfpathlineto{\pgfqpoint{3.926303in}{2.694484in}}%
\pgfpathlineto{\pgfqpoint{4.060152in}{2.556674in}}%
\pgfpathlineto{\pgfqpoint{4.126707in}{2.487488in}}%
\pgfpathlineto{\pgfqpoint{4.415140in}{2.177995in}}%
\pgfpathlineto{\pgfqpoint{4.455855in}{2.133333in}}%
\pgfpathlineto{\pgfqpoint{4.589529in}{1.984000in}}%
\pgfpathlineto{\pgfqpoint{4.768000in}{1.779344in}}%
\pgfpathlineto{\pgfqpoint{4.768000in}{1.779344in}}%
\pgfusepath{fill}%
\end{pgfscope}%
\begin{pgfscope}%
\pgfpathrectangle{\pgfqpoint{0.800000in}{0.528000in}}{\pgfqpoint{3.968000in}{3.696000in}}%
\pgfusepath{clip}%
\pgfsetbuttcap%
\pgfsetroundjoin%
\definecolor{currentfill}{rgb}{0.244972,0.287675,0.537260}%
\pgfsetfillcolor{currentfill}%
\pgfsetlinewidth{0.000000pt}%
\definecolor{currentstroke}{rgb}{0.000000,0.000000,0.000000}%
\pgfsetstrokecolor{currentstroke}%
\pgfsetdash{}{0pt}%
\pgfpathmoveto{\pgfqpoint{1.897003in}{0.528000in}}%
\pgfpathlineto{\pgfqpoint{1.756289in}{0.672070in}}%
\pgfpathlineto{\pgfqpoint{1.681778in}{0.749423in}}%
\pgfpathlineto{\pgfqpoint{1.400613in}{1.050109in}}%
\pgfpathlineto{\pgfqpoint{1.361131in}{1.093439in}}%
\pgfpathlineto{\pgfqpoint{1.231774in}{1.237333in}}%
\pgfpathlineto{\pgfqpoint{1.100421in}{1.386667in}}%
\pgfpathlineto{\pgfqpoint{0.826296in}{1.709827in}}%
\pgfpathlineto{\pgfqpoint{0.800000in}{1.741398in}}%
\pgfpathlineto{\pgfqpoint{0.800000in}{1.735624in}}%
\pgfpathlineto{\pgfqpoint{0.920242in}{1.591440in}}%
\pgfpathlineto{\pgfqpoint{1.040485in}{1.450365in}}%
\pgfpathlineto{\pgfqpoint{1.321051in}{1.132399in}}%
\pgfpathlineto{\pgfqpoint{1.568369in}{0.864000in}}%
\pgfpathlineto{\pgfqpoint{1.710266in}{0.714667in}}%
\pgfpathlineto{\pgfqpoint{1.892036in}{0.528000in}}%
\pgfpathmoveto{\pgfqpoint{4.768000in}{1.790458in}}%
\pgfpathlineto{\pgfqpoint{4.647758in}{1.928800in}}%
\pgfpathlineto{\pgfqpoint{4.367192in}{2.241082in}}%
\pgfpathlineto{\pgfqpoint{4.086626in}{2.539614in}}%
\pgfpathlineto{\pgfqpoint{3.966384in}{2.663636in}}%
\pgfpathlineto{\pgfqpoint{3.645737in}{2.983638in}}%
\pgfpathlineto{\pgfqpoint{3.485414in}{3.138016in}}%
\pgfpathlineto{\pgfqpoint{3.343923in}{3.270874in}}%
\pgfpathlineto{\pgfqpoint{3.282649in}{3.328000in}}%
\pgfpathlineto{\pgfqpoint{3.118762in}{3.477333in}}%
\pgfpathlineto{\pgfqpoint{2.964364in}{3.614635in}}%
\pgfpathlineto{\pgfqpoint{2.804040in}{3.753757in}}%
\pgfpathlineto{\pgfqpoint{2.643717in}{3.889426in}}%
\pgfpathlineto{\pgfqpoint{2.509666in}{4.000000in}}%
\pgfpathlineto{\pgfqpoint{2.228393in}{4.224000in}}%
\pgfpathlineto{\pgfqpoint{2.222052in}{4.224000in}}%
\pgfpathlineto{\pgfqpoint{2.364748in}{4.112000in}}%
\pgfpathlineto{\pgfqpoint{2.523475in}{3.983866in}}%
\pgfpathlineto{\pgfqpoint{2.844121in}{3.714473in}}%
\pgfpathlineto{\pgfqpoint{3.164768in}{3.430921in}}%
\pgfpathlineto{\pgfqpoint{3.325091in}{3.283817in}}%
\pgfpathlineto{\pgfqpoint{3.645737in}{2.978689in}}%
\pgfpathlineto{\pgfqpoint{3.966384in}{2.658638in}}%
\pgfpathlineto{\pgfqpoint{4.254532in}{2.357333in}}%
\pgfpathlineto{\pgfqpoint{4.545769in}{2.038336in}}%
\pgfpathlineto{\pgfqpoint{4.607677in}{1.968905in}}%
\pgfpathlineto{\pgfqpoint{4.768000in}{1.784901in}}%
\pgfpathlineto{\pgfqpoint{4.768000in}{1.784901in}}%
\pgfusepath{fill}%
\end{pgfscope}%
\begin{pgfscope}%
\pgfpathrectangle{\pgfqpoint{0.800000in}{0.528000in}}{\pgfqpoint{3.968000in}{3.696000in}}%
\pgfusepath{clip}%
\pgfsetbuttcap%
\pgfsetroundjoin%
\definecolor{currentfill}{rgb}{0.243113,0.292092,0.538516}%
\pgfsetfillcolor{currentfill}%
\pgfsetlinewidth{0.000000pt}%
\definecolor{currentstroke}{rgb}{0.000000,0.000000,0.000000}%
\pgfsetstrokecolor{currentstroke}%
\pgfsetdash{}{0pt}%
\pgfpathmoveto{\pgfqpoint{1.892036in}{0.528000in}}%
\pgfpathlineto{\pgfqpoint{1.761939in}{0.661079in}}%
\pgfpathlineto{\pgfqpoint{1.464026in}{0.976000in}}%
\pgfpathlineto{\pgfqpoint{1.321051in}{1.132399in}}%
\pgfpathlineto{\pgfqpoint{1.161593in}{1.311193in}}%
\pgfpathlineto{\pgfqpoint{1.040485in}{1.450365in}}%
\pgfpathlineto{\pgfqpoint{0.920242in}{1.591440in}}%
\pgfpathlineto{\pgfqpoint{0.800000in}{1.735624in}}%
\pgfpathlineto{\pgfqpoint{0.800000in}{1.729850in}}%
\pgfpathlineto{\pgfqpoint{0.920242in}{1.585813in}}%
\pgfpathlineto{\pgfqpoint{1.040485in}{1.444878in}}%
\pgfpathlineto{\pgfqpoint{1.289009in}{1.162667in}}%
\pgfpathlineto{\pgfqpoint{1.424811in}{1.013333in}}%
\pgfpathlineto{\pgfqpoint{1.721859in}{0.697543in}}%
\pgfpathlineto{\pgfqpoint{1.882182in}{0.532950in}}%
\pgfpathlineto{\pgfqpoint{1.887070in}{0.528000in}}%
\pgfpathmoveto{\pgfqpoint{4.768000in}{1.796016in}}%
\pgfpathlineto{\pgfqpoint{4.647758in}{1.934221in}}%
\pgfpathlineto{\pgfqpoint{4.373502in}{2.239456in}}%
\pgfpathlineto{\pgfqpoint{4.246949in}{2.375720in}}%
\pgfpathlineto{\pgfqpoint{4.087312in}{2.544000in}}%
\pgfpathlineto{\pgfqpoint{3.966384in}{2.668635in}}%
\pgfpathlineto{\pgfqpoint{3.645737in}{2.988587in}}%
\pgfpathlineto{\pgfqpoint{3.505890in}{3.123072in}}%
\pgfpathlineto{\pgfqpoint{3.445333in}{3.180910in}}%
\pgfpathlineto{\pgfqpoint{3.285010in}{3.330670in}}%
\pgfpathlineto{\pgfqpoint{3.124181in}{3.477333in}}%
\pgfpathlineto{\pgfqpoint{2.964364in}{3.619482in}}%
\pgfpathlineto{\pgfqpoint{2.804040in}{3.758581in}}%
\pgfpathlineto{\pgfqpoint{2.643717in}{3.894227in}}%
\pgfpathlineto{\pgfqpoint{2.515639in}{4.000000in}}%
\pgfpathlineto{\pgfqpoint{2.234734in}{4.224000in}}%
\pgfpathlineto{\pgfqpoint{2.228393in}{4.224000in}}%
\pgfpathlineto{\pgfqpoint{2.363152in}{4.118122in}}%
\pgfpathlineto{\pgfqpoint{2.689925in}{3.850667in}}%
\pgfpathlineto{\pgfqpoint{2.844121in}{3.719303in}}%
\pgfpathlineto{\pgfqpoint{3.164768in}{3.435797in}}%
\pgfpathlineto{\pgfqpoint{3.325091in}{3.288717in}}%
\pgfpathlineto{\pgfqpoint{3.645737in}{2.983638in}}%
\pgfpathlineto{\pgfqpoint{3.966384in}{2.663636in}}%
\pgfpathlineto{\pgfqpoint{4.259340in}{2.357333in}}%
\pgfpathlineto{\pgfqpoint{4.548375in}{2.040764in}}%
\pgfpathlineto{\pgfqpoint{4.607677in}{1.974318in}}%
\pgfpathlineto{\pgfqpoint{4.768000in}{1.790458in}}%
\pgfpathlineto{\pgfqpoint{4.768000in}{1.790458in}}%
\pgfusepath{fill}%
\end{pgfscope}%
\begin{pgfscope}%
\pgfpathrectangle{\pgfqpoint{0.800000in}{0.528000in}}{\pgfqpoint{3.968000in}{3.696000in}}%
\pgfusepath{clip}%
\pgfsetbuttcap%
\pgfsetroundjoin%
\definecolor{currentfill}{rgb}{0.243113,0.292092,0.538516}%
\pgfsetfillcolor{currentfill}%
\pgfsetlinewidth{0.000000pt}%
\definecolor{currentstroke}{rgb}{0.000000,0.000000,0.000000}%
\pgfsetstrokecolor{currentstroke}%
\pgfsetdash{}{0pt}%
\pgfpathmoveto{\pgfqpoint{1.887070in}{0.528000in}}%
\pgfpathlineto{\pgfqpoint{1.761939in}{0.656030in}}%
\pgfpathlineto{\pgfqpoint{1.459222in}{0.976000in}}%
\pgfpathlineto{\pgfqpoint{1.321051in}{1.127073in}}%
\pgfpathlineto{\pgfqpoint{1.160727in}{1.306819in}}%
\pgfpathlineto{\pgfqpoint{1.040485in}{1.444878in}}%
\pgfpathlineto{\pgfqpoint{0.920242in}{1.585813in}}%
\pgfpathlineto{\pgfqpoint{0.800000in}{1.729850in}}%
\pgfpathlineto{\pgfqpoint{0.800000in}{1.724076in}}%
\pgfpathlineto{\pgfqpoint{0.905989in}{1.597390in}}%
\pgfpathlineto{\pgfqpoint{0.960323in}{1.532878in}}%
\pgfpathlineto{\pgfqpoint{1.101749in}{1.369065in}}%
\pgfpathlineto{\pgfqpoint{1.160727in}{1.301465in}}%
\pgfpathlineto{\pgfqpoint{1.301066in}{1.143948in}}%
\pgfpathlineto{\pgfqpoint{1.441293in}{0.990211in}}%
\pgfpathlineto{\pgfqpoint{1.561535in}{0.860971in}}%
\pgfpathlineto{\pgfqpoint{1.721859in}{0.692489in}}%
\pgfpathlineto{\pgfqpoint{1.882105in}{0.528000in}}%
\pgfpathlineto{\pgfqpoint{1.882182in}{0.528000in}}%
\pgfpathmoveto{\pgfqpoint{4.768000in}{1.801486in}}%
\pgfpathlineto{\pgfqpoint{4.626138in}{1.963862in}}%
\pgfpathlineto{\pgfqpoint{4.567596in}{2.030196in}}%
\pgfpathlineto{\pgfqpoint{4.277285in}{2.348256in}}%
\pgfpathlineto{\pgfqpoint{4.206869in}{2.423481in}}%
\pgfpathlineto{\pgfqpoint{3.917894in}{2.722834in}}%
\pgfpathlineto{\pgfqpoint{3.846141in}{2.795374in}}%
\pgfpathlineto{\pgfqpoint{3.685663in}{2.954522in}}%
\pgfpathlineto{\pgfqpoint{3.626743in}{3.011641in}}%
\pgfpathlineto{\pgfqpoint{3.565576in}{3.071081in}}%
\pgfpathlineto{\pgfqpoint{3.244929in}{3.372354in}}%
\pgfpathlineto{\pgfqpoint{3.084606in}{3.517670in}}%
\pgfpathlineto{\pgfqpoint{2.763960in}{3.797632in}}%
\pgfpathlineto{\pgfqpoint{2.603636in}{3.932399in}}%
\pgfpathlineto{\pgfqpoint{2.475741in}{4.037333in}}%
\pgfpathlineto{\pgfqpoint{2.323071in}{4.159623in}}%
\pgfpathlineto{\pgfqpoint{2.241074in}{4.224000in}}%
\pgfpathlineto{\pgfqpoint{2.234734in}{4.224000in}}%
\pgfpathlineto{\pgfqpoint{2.363152in}{4.122970in}}%
\pgfpathlineto{\pgfqpoint{2.683798in}{3.860636in}}%
\pgfpathlineto{\pgfqpoint{3.004444in}{3.584162in}}%
\pgfpathlineto{\pgfqpoint{3.328206in}{3.290667in}}%
\pgfpathlineto{\pgfqpoint{3.645737in}{2.988587in}}%
\pgfpathlineto{\pgfqpoint{3.966384in}{2.668635in}}%
\pgfpathlineto{\pgfqpoint{4.264149in}{2.357333in}}%
\pgfpathlineto{\pgfqpoint{4.567596in}{2.024898in}}%
\pgfpathlineto{\pgfqpoint{4.768000in}{1.796016in}}%
\pgfpathlineto{\pgfqpoint{4.768000in}{1.797333in}}%
\pgfusepath{fill}%
\end{pgfscope}%
\begin{pgfscope}%
\pgfpathrectangle{\pgfqpoint{0.800000in}{0.528000in}}{\pgfqpoint{3.968000in}{3.696000in}}%
\pgfusepath{clip}%
\pgfsetbuttcap%
\pgfsetroundjoin%
\definecolor{currentfill}{rgb}{0.243113,0.292092,0.538516}%
\pgfsetfillcolor{currentfill}%
\pgfsetlinewidth{0.000000pt}%
\definecolor{currentstroke}{rgb}{0.000000,0.000000,0.000000}%
\pgfsetstrokecolor{currentstroke}%
\pgfsetdash{}{0pt}%
\pgfpathmoveto{\pgfqpoint{1.882105in}{0.528000in}}%
\pgfpathlineto{\pgfqpoint{1.736482in}{0.677333in}}%
\pgfpathlineto{\pgfqpoint{1.593886in}{0.826667in}}%
\pgfpathlineto{\pgfqpoint{1.454419in}{0.976000in}}%
\pgfpathlineto{\pgfqpoint{1.317873in}{1.125333in}}%
\pgfpathlineto{\pgfqpoint{1.184409in}{1.274667in}}%
\pgfpathlineto{\pgfqpoint{1.053789in}{1.424000in}}%
\pgfpathlineto{\pgfqpoint{0.800000in}{1.724076in}}%
\pgfpathlineto{\pgfqpoint{0.800000in}{1.722667in}}%
\pgfpathlineto{\pgfqpoint{0.800000in}{1.718395in}}%
\pgfpathlineto{\pgfqpoint{1.049046in}{1.424000in}}%
\pgfpathlineto{\pgfqpoint{1.179678in}{1.274667in}}%
\pgfpathlineto{\pgfqpoint{1.481374in}{0.941663in}}%
\pgfpathlineto{\pgfqpoint{1.761939in}{0.645934in}}%
\pgfpathlineto{\pgfqpoint{1.877240in}{0.528000in}}%
\pgfpathmoveto{\pgfqpoint{4.768000in}{1.806929in}}%
\pgfpathlineto{\pgfqpoint{4.628735in}{1.966282in}}%
\pgfpathlineto{\pgfqpoint{4.567596in}{2.035494in}}%
\pgfpathlineto{\pgfqpoint{4.279878in}{2.350671in}}%
\pgfpathlineto{\pgfqpoint{4.223748in}{2.410389in}}%
\pgfpathlineto{\pgfqpoint{4.166788in}{2.470947in}}%
\pgfpathlineto{\pgfqpoint{3.846141in}{2.800354in}}%
\pgfpathlineto{\pgfqpoint{3.707571in}{2.937595in}}%
\pgfpathlineto{\pgfqpoint{3.645737in}{2.998366in}}%
\pgfpathlineto{\pgfqpoint{3.325091in}{3.303187in}}%
\pgfpathlineto{\pgfqpoint{3.004444in}{3.593786in}}%
\pgfpathlineto{\pgfqpoint{2.683798in}{3.870249in}}%
\pgfpathlineto{\pgfqpoint{2.523475in}{4.003291in}}%
\pgfpathlineto{\pgfqpoint{2.247298in}{4.224000in}}%
\pgfpathlineto{\pgfqpoint{2.241074in}{4.224000in}}%
\pgfpathlineto{\pgfqpoint{2.336048in}{4.149333in}}%
\pgfpathlineto{\pgfqpoint{2.483394in}{4.031134in}}%
\pgfpathlineto{\pgfqpoint{2.804040in}{3.763405in}}%
\pgfpathlineto{\pgfqpoint{2.964364in}{3.624329in}}%
\pgfpathlineto{\pgfqpoint{3.293094in}{3.328000in}}%
\pgfpathlineto{\pgfqpoint{3.608829in}{3.029333in}}%
\pgfpathlineto{\pgfqpoint{3.926303in}{2.714453in}}%
\pgfpathlineto{\pgfqpoint{4.086626in}{2.549732in}}%
\pgfpathlineto{\pgfqpoint{4.221195in}{2.408011in}}%
\pgfpathlineto{\pgfqpoint{4.287030in}{2.337989in}}%
\pgfpathlineto{\pgfqpoint{4.571873in}{2.025317in}}%
\pgfpathlineto{\pgfqpoint{4.626138in}{1.963862in}}%
\pgfpathlineto{\pgfqpoint{4.687838in}{1.893864in}}%
\pgfpathlineto{\pgfqpoint{4.768000in}{1.801486in}}%
\pgfpathlineto{\pgfqpoint{4.768000in}{1.801486in}}%
\pgfusepath{fill}%
\end{pgfscope}%
\begin{pgfscope}%
\pgfpathrectangle{\pgfqpoint{0.800000in}{0.528000in}}{\pgfqpoint{3.968000in}{3.696000in}}%
\pgfusepath{clip}%
\pgfsetbuttcap%
\pgfsetroundjoin%
\definecolor{currentfill}{rgb}{0.243113,0.292092,0.538516}%
\pgfsetfillcolor{currentfill}%
\pgfsetlinewidth{0.000000pt}%
\definecolor{currentstroke}{rgb}{0.000000,0.000000,0.000000}%
\pgfsetstrokecolor{currentstroke}%
\pgfsetdash{}{0pt}%
\pgfpathmoveto{\pgfqpoint{1.877240in}{0.528000in}}%
\pgfpathlineto{\pgfqpoint{1.731604in}{0.677333in}}%
\pgfpathlineto{\pgfqpoint{1.601616in}{0.813406in}}%
\pgfpathlineto{\pgfqpoint{1.313153in}{1.125333in}}%
\pgfpathlineto{\pgfqpoint{1.179678in}{1.274667in}}%
\pgfpathlineto{\pgfqpoint{1.049046in}{1.424000in}}%
\pgfpathlineto{\pgfqpoint{0.800000in}{1.718395in}}%
\pgfpathlineto{\pgfqpoint{0.800000in}{1.712744in}}%
\pgfpathlineto{\pgfqpoint{1.060190in}{1.405646in}}%
\pgfpathlineto{\pgfqpoint{1.200808in}{1.245441in}}%
\pgfpathlineto{\pgfqpoint{1.321051in}{1.111378in}}%
\pgfpathlineto{\pgfqpoint{1.479373in}{0.938667in}}%
\pgfpathlineto{\pgfqpoint{1.601616in}{0.808332in}}%
\pgfpathlineto{\pgfqpoint{1.762799in}{0.640000in}}%
\pgfpathlineto{\pgfqpoint{1.872376in}{0.528000in}}%
\pgfpathmoveto{\pgfqpoint{4.768000in}{1.812372in}}%
\pgfpathlineto{\pgfqpoint{4.631333in}{1.968701in}}%
\pgfpathlineto{\pgfqpoint{4.567596in}{2.040792in}}%
\pgfpathlineto{\pgfqpoint{4.282471in}{2.353086in}}%
\pgfpathlineto{\pgfqpoint{4.226301in}{2.412767in}}%
\pgfpathlineto{\pgfqpoint{4.166788in}{2.475977in}}%
\pgfpathlineto{\pgfqpoint{3.846141in}{2.805334in}}%
\pgfpathlineto{\pgfqpoint{3.525495in}{3.119207in}}%
\pgfpathlineto{\pgfqpoint{3.204848in}{3.418600in}}%
\pgfpathlineto{\pgfqpoint{3.044525in}{3.562996in}}%
\pgfpathlineto{\pgfqpoint{2.723879in}{3.841268in}}%
\pgfpathlineto{\pgfqpoint{2.563556in}{3.975137in}}%
\pgfpathlineto{\pgfqpoint{2.253474in}{4.224000in}}%
\pgfpathlineto{\pgfqpoint{2.247298in}{4.224000in}}%
\pgfpathlineto{\pgfqpoint{2.403232in}{4.100655in}}%
\pgfpathlineto{\pgfqpoint{2.723879in}{3.836456in}}%
\pgfpathlineto{\pgfqpoint{3.051505in}{3.552000in}}%
\pgfpathlineto{\pgfqpoint{3.365172in}{3.265871in}}%
\pgfpathlineto{\pgfqpoint{3.685818in}{2.959243in}}%
\pgfpathlineto{\pgfqpoint{3.988492in}{2.656000in}}%
\pgfpathlineto{\pgfqpoint{4.132725in}{2.506667in}}%
\pgfpathlineto{\pgfqpoint{4.445935in}{2.170667in}}%
\pgfpathlineto{\pgfqpoint{4.567596in}{2.035494in}}%
\pgfpathlineto{\pgfqpoint{4.768000in}{1.806929in}}%
\pgfpathlineto{\pgfqpoint{4.768000in}{1.806929in}}%
\pgfusepath{fill}%
\end{pgfscope}%
\begin{pgfscope}%
\pgfpathrectangle{\pgfqpoint{0.800000in}{0.528000in}}{\pgfqpoint{3.968000in}{3.696000in}}%
\pgfusepath{clip}%
\pgfsetbuttcap%
\pgfsetroundjoin%
\definecolor{currentfill}{rgb}{0.241237,0.296485,0.539709}%
\pgfsetfillcolor{currentfill}%
\pgfsetlinewidth{0.000000pt}%
\definecolor{currentstroke}{rgb}{0.000000,0.000000,0.000000}%
\pgfsetstrokecolor{currentstroke}%
\pgfsetdash{}{0pt}%
\pgfpathmoveto{\pgfqpoint{1.872376in}{0.528000in}}%
\pgfpathlineto{\pgfqpoint{1.721859in}{0.682379in}}%
\pgfpathlineto{\pgfqpoint{1.410465in}{1.013333in}}%
\pgfpathlineto{\pgfqpoint{1.274753in}{1.162667in}}%
\pgfpathlineto{\pgfqpoint{0.989233in}{1.488261in}}%
\pgfpathlineto{\pgfqpoint{0.920242in}{1.569023in}}%
\pgfpathlineto{\pgfqpoint{0.800000in}{1.712744in}}%
\pgfpathlineto{\pgfqpoint{0.800000in}{1.707093in}}%
\pgfpathlineto{\pgfqpoint{1.040485in}{1.422951in}}%
\pgfpathlineto{\pgfqpoint{1.200808in}{1.240094in}}%
\pgfpathlineto{\pgfqpoint{1.321051in}{1.106157in}}%
\pgfpathlineto{\pgfqpoint{1.459339in}{0.955476in}}%
\pgfpathlineto{\pgfqpoint{1.521455in}{0.888449in}}%
\pgfpathlineto{\pgfqpoint{1.681778in}{0.719056in}}%
\pgfpathlineto{\pgfqpoint{1.802020in}{0.594744in}}%
\pgfpathlineto{\pgfqpoint{1.867511in}{0.528000in}}%
\pgfpathmoveto{\pgfqpoint{4.768000in}{1.817815in}}%
\pgfpathlineto{\pgfqpoint{4.647758in}{1.955719in}}%
\pgfpathlineto{\pgfqpoint{4.367192in}{2.266967in}}%
\pgfpathlineto{\pgfqpoint{4.228853in}{2.415144in}}%
\pgfpathlineto{\pgfqpoint{4.166788in}{2.481007in}}%
\pgfpathlineto{\pgfqpoint{3.851012in}{2.805333in}}%
\pgfpathlineto{\pgfqpoint{3.725899in}{2.929630in}}%
\pgfpathlineto{\pgfqpoint{3.405253in}{3.237977in}}%
\pgfpathlineto{\pgfqpoint{3.084606in}{3.532001in}}%
\pgfpathlineto{\pgfqpoint{2.762490in}{3.813333in}}%
\pgfpathlineto{\pgfqpoint{2.443313in}{4.078089in}}%
\pgfpathlineto{\pgfqpoint{2.259650in}{4.224000in}}%
\pgfpathlineto{\pgfqpoint{2.253474in}{4.224000in}}%
\pgfpathlineto{\pgfqpoint{2.403232in}{4.105509in}}%
\pgfpathlineto{\pgfqpoint{2.723879in}{3.841268in}}%
\pgfpathlineto{\pgfqpoint{3.044525in}{3.562996in}}%
\pgfpathlineto{\pgfqpoint{3.204848in}{3.418600in}}%
\pgfpathlineto{\pgfqpoint{3.525495in}{3.119207in}}%
\pgfpathlineto{\pgfqpoint{3.685818in}{2.964107in}}%
\pgfpathlineto{\pgfqpoint{3.993375in}{2.656000in}}%
\pgfpathlineto{\pgfqpoint{4.137522in}{2.506667in}}%
\pgfpathlineto{\pgfqpoint{4.447354in}{2.174312in}}%
\pgfpathlineto{\pgfqpoint{4.567596in}{2.040792in}}%
\pgfpathlineto{\pgfqpoint{4.768000in}{1.812372in}}%
\pgfpathlineto{\pgfqpoint{4.768000in}{1.812372in}}%
\pgfusepath{fill}%
\end{pgfscope}%
\begin{pgfscope}%
\pgfpathrectangle{\pgfqpoint{0.800000in}{0.528000in}}{\pgfqpoint{3.968000in}{3.696000in}}%
\pgfusepath{clip}%
\pgfsetbuttcap%
\pgfsetroundjoin%
\definecolor{currentfill}{rgb}{0.241237,0.296485,0.539709}%
\pgfsetfillcolor{currentfill}%
\pgfsetlinewidth{0.000000pt}%
\definecolor{currentstroke}{rgb}{0.000000,0.000000,0.000000}%
\pgfsetstrokecolor{currentstroke}%
\pgfsetdash{}{0pt}%
\pgfpathmoveto{\pgfqpoint{1.867511in}{0.528000in}}%
\pgfpathlineto{\pgfqpoint{1.721850in}{0.677333in}}%
\pgfpathlineto{\pgfqpoint{1.405683in}{1.013333in}}%
\pgfpathlineto{\pgfqpoint{1.280970in}{1.150524in}}%
\pgfpathlineto{\pgfqpoint{1.160727in}{1.285403in}}%
\pgfpathlineto{\pgfqpoint{1.035838in}{1.428328in}}%
\pgfpathlineto{\pgfqpoint{0.898095in}{1.590038in}}%
\pgfpathlineto{\pgfqpoint{0.840081in}{1.658873in}}%
\pgfpathlineto{\pgfqpoint{0.800000in}{1.707093in}}%
\pgfpathlineto{\pgfqpoint{0.800000in}{1.701442in}}%
\pgfpathlineto{\pgfqpoint{1.040485in}{1.417575in}}%
\pgfpathlineto{\pgfqpoint{1.181608in}{1.256782in}}%
\pgfpathlineto{\pgfqpoint{1.240889in}{1.189916in}}%
\pgfpathlineto{\pgfqpoint{1.521455in}{0.883362in}}%
\pgfpathlineto{\pgfqpoint{1.681778in}{0.714008in}}%
\pgfpathlineto{\pgfqpoint{1.862646in}{0.528000in}}%
\pgfpathmoveto{\pgfqpoint{4.768000in}{1.823258in}}%
\pgfpathlineto{\pgfqpoint{4.647758in}{1.961031in}}%
\pgfpathlineto{\pgfqpoint{4.367192in}{2.272128in}}%
\pgfpathlineto{\pgfqpoint{4.231406in}{2.417522in}}%
\pgfpathlineto{\pgfqpoint{4.166788in}{2.486037in}}%
\pgfpathlineto{\pgfqpoint{3.855883in}{2.805333in}}%
\pgfpathlineto{\pgfqpoint{3.725899in}{2.934500in}}%
\pgfpathlineto{\pgfqpoint{3.405253in}{3.242800in}}%
\pgfpathlineto{\pgfqpoint{3.084606in}{3.536778in}}%
\pgfpathlineto{\pgfqpoint{2.763960in}{3.816839in}}%
\pgfpathlineto{\pgfqpoint{2.443313in}{4.082861in}}%
\pgfpathlineto{\pgfqpoint{2.265826in}{4.224000in}}%
\pgfpathlineto{\pgfqpoint{2.259650in}{4.224000in}}%
\pgfpathlineto{\pgfqpoint{2.403232in}{4.110362in}}%
\pgfpathlineto{\pgfqpoint{2.718443in}{3.850667in}}%
\pgfpathlineto{\pgfqpoint{2.849554in}{3.738667in}}%
\pgfpathlineto{\pgfqpoint{3.164768in}{3.459815in}}%
\pgfpathlineto{\pgfqpoint{3.485414in}{3.162250in}}%
\pgfpathlineto{\pgfqpoint{3.806061in}{2.850256in}}%
\pgfpathlineto{\pgfqpoint{4.106584in}{2.544000in}}%
\pgfpathlineto{\pgfqpoint{4.407273in}{2.223354in}}%
\pgfpathlineto{\pgfqpoint{4.527515in}{2.090878in}}%
\pgfpathlineto{\pgfqpoint{4.687838in}{1.910133in}}%
\pgfpathlineto{\pgfqpoint{4.768000in}{1.817815in}}%
\pgfpathlineto{\pgfqpoint{4.768000in}{1.817815in}}%
\pgfusepath{fill}%
\end{pgfscope}%
\begin{pgfscope}%
\pgfpathrectangle{\pgfqpoint{0.800000in}{0.528000in}}{\pgfqpoint{3.968000in}{3.696000in}}%
\pgfusepath{clip}%
\pgfsetbuttcap%
\pgfsetroundjoin%
\definecolor{currentfill}{rgb}{0.241237,0.296485,0.539709}%
\pgfsetfillcolor{currentfill}%
\pgfsetlinewidth{0.000000pt}%
\definecolor{currentstroke}{rgb}{0.000000,0.000000,0.000000}%
\pgfsetstrokecolor{currentstroke}%
\pgfsetdash{}{0pt}%
\pgfpathmoveto{\pgfqpoint{1.862646in}{0.528000in}}%
\pgfpathlineto{\pgfqpoint{1.717071in}{0.677333in}}%
\pgfpathlineto{\pgfqpoint{1.400907in}{1.013333in}}%
\pgfpathlineto{\pgfqpoint{1.099900in}{1.349333in}}%
\pgfpathlineto{\pgfqpoint{0.825165in}{1.671440in}}%
\pgfpathlineto{\pgfqpoint{0.800000in}{1.701442in}}%
\pgfpathlineto{\pgfqpoint{0.800000in}{1.695791in}}%
\pgfpathlineto{\pgfqpoint{1.040485in}{1.412199in}}%
\pgfpathlineto{\pgfqpoint{1.179036in}{1.254387in}}%
\pgfpathlineto{\pgfqpoint{1.240889in}{1.184682in}}%
\pgfpathlineto{\pgfqpoint{1.521455in}{0.878275in}}%
\pgfpathlineto{\pgfqpoint{1.676388in}{0.714667in}}%
\pgfpathlineto{\pgfqpoint{1.802020in}{0.584849in}}%
\pgfpathlineto{\pgfqpoint{1.857781in}{0.528000in}}%
\pgfpathmoveto{\pgfqpoint{4.768000in}{1.828702in}}%
\pgfpathlineto{\pgfqpoint{4.647758in}{1.966343in}}%
\pgfpathlineto{\pgfqpoint{4.367192in}{2.277288in}}%
\pgfpathlineto{\pgfqpoint{4.233959in}{2.419900in}}%
\pgfpathlineto{\pgfqpoint{4.166788in}{2.491067in}}%
\pgfpathlineto{\pgfqpoint{3.872894in}{2.792919in}}%
\pgfpathlineto{\pgfqpoint{3.806061in}{2.860020in}}%
\pgfpathlineto{\pgfqpoint{3.485414in}{3.171919in}}%
\pgfpathlineto{\pgfqpoint{3.325091in}{3.322433in}}%
\pgfpathlineto{\pgfqpoint{3.004444in}{3.612848in}}%
\pgfpathlineto{\pgfqpoint{2.683798in}{3.889447in}}%
\pgfpathlineto{\pgfqpoint{2.363152in}{4.152007in}}%
\pgfpathlineto{\pgfqpoint{2.272002in}{4.224000in}}%
\pgfpathlineto{\pgfqpoint{2.265826in}{4.224000in}}%
\pgfpathlineto{\pgfqpoint{2.407155in}{4.112000in}}%
\pgfpathlineto{\pgfqpoint{2.702794in}{3.868360in}}%
\pgfpathlineto{\pgfqpoint{2.768066in}{3.813333in}}%
\pgfpathlineto{\pgfqpoint{3.084606in}{3.536778in}}%
\pgfpathlineto{\pgfqpoint{3.405253in}{3.242800in}}%
\pgfpathlineto{\pgfqpoint{3.565576in}{3.090467in}}%
\pgfpathlineto{\pgfqpoint{3.725899in}{2.934500in}}%
\pgfpathlineto{\pgfqpoint{4.039417in}{2.618667in}}%
\pgfpathlineto{\pgfqpoint{4.327111in}{2.315479in}}%
\pgfpathlineto{\pgfqpoint{4.627454in}{1.984000in}}%
\pgfpathlineto{\pgfqpoint{4.768000in}{1.823258in}}%
\pgfpathlineto{\pgfqpoint{4.768000in}{1.823258in}}%
\pgfusepath{fill}%
\end{pgfscope}%
\begin{pgfscope}%
\pgfpathrectangle{\pgfqpoint{0.800000in}{0.528000in}}{\pgfqpoint{3.968000in}{3.696000in}}%
\pgfusepath{clip}%
\pgfsetbuttcap%
\pgfsetroundjoin%
\definecolor{currentfill}{rgb}{0.241237,0.296485,0.539709}%
\pgfsetfillcolor{currentfill}%
\pgfsetlinewidth{0.000000pt}%
\definecolor{currentstroke}{rgb}{0.000000,0.000000,0.000000}%
\pgfsetstrokecolor{currentstroke}%
\pgfsetdash{}{0pt}%
\pgfpathmoveto{\pgfqpoint{1.857781in}{0.528000in}}%
\pgfpathlineto{\pgfqpoint{1.712292in}{0.677333in}}%
\pgfpathlineto{\pgfqpoint{1.401212in}{1.007895in}}%
\pgfpathlineto{\pgfqpoint{1.280970in}{1.140069in}}%
\pgfpathlineto{\pgfqpoint{1.160727in}{1.274695in}}%
\pgfpathlineto{\pgfqpoint{1.000404in}{1.458578in}}%
\pgfpathlineto{\pgfqpoint{0.800000in}{1.695791in}}%
\pgfpathlineto{\pgfqpoint{0.800000in}{1.690140in}}%
\pgfpathlineto{\pgfqpoint{1.032325in}{1.416400in}}%
\pgfpathlineto{\pgfqpoint{1.090519in}{1.349333in}}%
\pgfpathlineto{\pgfqpoint{1.222517in}{1.200000in}}%
\pgfpathlineto{\pgfqpoint{1.361131in}{1.046491in}}%
\pgfpathlineto{\pgfqpoint{1.671630in}{0.714667in}}%
\pgfpathlineto{\pgfqpoint{1.802020in}{0.579901in}}%
\pgfpathlineto{\pgfqpoint{1.852917in}{0.528000in}}%
\pgfpathmoveto{\pgfqpoint{4.768000in}{1.834145in}}%
\pgfpathlineto{\pgfqpoint{4.647758in}{1.971655in}}%
\pgfpathlineto{\pgfqpoint{4.385423in}{2.262315in}}%
\pgfpathlineto{\pgfqpoint{4.327111in}{2.325677in}}%
\pgfpathlineto{\pgfqpoint{4.012777in}{2.656000in}}%
\pgfpathlineto{\pgfqpoint{3.886222in}{2.784635in}}%
\pgfpathlineto{\pgfqpoint{3.565576in}{3.100159in}}%
\pgfpathlineto{\pgfqpoint{3.244929in}{3.401153in}}%
\pgfpathlineto{\pgfqpoint{3.084606in}{3.546331in}}%
\pgfpathlineto{\pgfqpoint{2.763960in}{3.826303in}}%
\pgfpathlineto{\pgfqpoint{2.443313in}{4.092405in}}%
\pgfpathlineto{\pgfqpoint{2.278179in}{4.224000in}}%
\pgfpathlineto{\pgfqpoint{2.272002in}{4.224000in}}%
\pgfpathlineto{\pgfqpoint{2.403232in}{4.119924in}}%
\pgfpathlineto{\pgfqpoint{2.563556in}{3.989503in}}%
\pgfpathlineto{\pgfqpoint{2.723879in}{3.855614in}}%
\pgfpathlineto{\pgfqpoint{2.884202in}{3.718170in}}%
\pgfpathlineto{\pgfqpoint{3.044525in}{3.577309in}}%
\pgfpathlineto{\pgfqpoint{3.204848in}{3.432982in}}%
\pgfpathlineto{\pgfqpoint{3.525495in}{3.133729in}}%
\pgfpathlineto{\pgfqpoint{3.685818in}{2.978699in}}%
\pgfpathlineto{\pgfqpoint{4.007993in}{2.656000in}}%
\pgfpathlineto{\pgfqpoint{4.327688in}{2.320000in}}%
\pgfpathlineto{\pgfqpoint{4.632149in}{1.984000in}}%
\pgfpathlineto{\pgfqpoint{4.768000in}{1.828702in}}%
\pgfpathlineto{\pgfqpoint{4.768000in}{1.828702in}}%
\pgfusepath{fill}%
\end{pgfscope}%
\begin{pgfscope}%
\pgfpathrectangle{\pgfqpoint{0.800000in}{0.528000in}}{\pgfqpoint{3.968000in}{3.696000in}}%
\pgfusepath{clip}%
\pgfsetbuttcap%
\pgfsetroundjoin%
\definecolor{currentfill}{rgb}{0.239346,0.300855,0.540844}%
\pgfsetfillcolor{currentfill}%
\pgfsetlinewidth{0.000000pt}%
\definecolor{currentstroke}{rgb}{0.000000,0.000000,0.000000}%
\pgfsetstrokecolor{currentstroke}%
\pgfsetdash{}{0pt}%
\pgfpathmoveto{\pgfqpoint{1.852917in}{0.528000in}}%
\pgfpathlineto{\pgfqpoint{1.721859in}{0.662444in}}%
\pgfpathlineto{\pgfqpoint{1.565139in}{0.826667in}}%
\pgfpathlineto{\pgfqpoint{1.441293in}{0.959337in}}%
\pgfpathlineto{\pgfqpoint{1.289555in}{1.125333in}}%
\pgfpathlineto{\pgfqpoint{1.156112in}{1.274667in}}%
\pgfpathlineto{\pgfqpoint{1.025626in}{1.424000in}}%
\pgfpathlineto{\pgfqpoint{0.897949in}{1.573333in}}%
\pgfpathlineto{\pgfqpoint{0.800000in}{1.690140in}}%
\pgfpathlineto{\pgfqpoint{0.800000in}{1.684507in}}%
\pgfpathlineto{\pgfqpoint{1.080566in}{1.355356in}}%
\pgfpathlineto{\pgfqpoint{1.321051in}{1.085328in}}%
\pgfpathlineto{\pgfqpoint{1.481374in}{0.911044in}}%
\pgfpathlineto{\pgfqpoint{1.617426in}{0.766726in}}%
\pgfpathlineto{\pgfqpoint{1.681778in}{0.699109in}}%
\pgfpathlineto{\pgfqpoint{1.848052in}{0.528000in}}%
\pgfpathmoveto{\pgfqpoint{4.768000in}{1.839489in}}%
\pgfpathlineto{\pgfqpoint{4.480199in}{2.163927in}}%
\pgfpathlineto{\pgfqpoint{4.440213in}{2.208000in}}%
\pgfpathlineto{\pgfqpoint{4.302310in}{2.357333in}}%
\pgfpathlineto{\pgfqpoint{4.161505in}{2.506667in}}%
\pgfpathlineto{\pgfqpoint{4.017562in}{2.656000in}}%
\pgfpathlineto{\pgfqpoint{3.886222in}{2.789529in}}%
\pgfpathlineto{\pgfqpoint{3.566602in}{3.104000in}}%
\pgfpathlineto{\pgfqpoint{3.285010in}{3.369047in}}%
\pgfpathlineto{\pgfqpoint{2.957164in}{3.664000in}}%
\pgfpathlineto{\pgfqpoint{2.804040in}{3.796791in}}%
\pgfpathlineto{\pgfqpoint{2.483394in}{4.064688in}}%
\pgfpathlineto{\pgfqpoint{2.323071in}{4.193392in}}%
\pgfpathlineto{\pgfqpoint{2.282990in}{4.224000in}}%
\pgfpathlineto{\pgfqpoint{2.278179in}{4.224000in}}%
\pgfpathlineto{\pgfqpoint{2.403232in}{4.124690in}}%
\pgfpathlineto{\pgfqpoint{2.563556in}{3.994292in}}%
\pgfpathlineto{\pgfqpoint{2.708164in}{3.873362in}}%
\pgfpathlineto{\pgfqpoint{2.779149in}{3.813333in}}%
\pgfpathlineto{\pgfqpoint{3.084606in}{3.546331in}}%
\pgfpathlineto{\pgfqpoint{3.405253in}{3.252446in}}%
\pgfpathlineto{\pgfqpoint{3.565576in}{3.100159in}}%
\pgfpathlineto{\pgfqpoint{3.725899in}{2.944240in}}%
\pgfpathlineto{\pgfqpoint{4.047871in}{2.619901in}}%
\pgfpathlineto{\pgfqpoint{4.086626in}{2.579836in}}%
\pgfpathlineto{\pgfqpoint{4.246949in}{2.411380in}}%
\pgfpathlineto{\pgfqpoint{4.401324in}{2.245333in}}%
\pgfpathlineto{\pgfqpoint{4.536992in}{2.096000in}}%
\pgfpathlineto{\pgfqpoint{4.669752in}{1.946667in}}%
\pgfpathlineto{\pgfqpoint{4.768000in}{1.834145in}}%
\pgfpathlineto{\pgfqpoint{4.768000in}{1.834667in}}%
\pgfpathlineto{\pgfqpoint{4.768000in}{1.834667in}}%
\pgfusepath{fill}%
\end{pgfscope}%
\begin{pgfscope}%
\pgfpathrectangle{\pgfqpoint{0.800000in}{0.528000in}}{\pgfqpoint{3.968000in}{3.696000in}}%
\pgfusepath{clip}%
\pgfsetbuttcap%
\pgfsetroundjoin%
\definecolor{currentfill}{rgb}{0.239346,0.300855,0.540844}%
\pgfsetfillcolor{currentfill}%
\pgfsetlinewidth{0.000000pt}%
\definecolor{currentstroke}{rgb}{0.000000,0.000000,0.000000}%
\pgfsetstrokecolor{currentstroke}%
\pgfsetdash{}{0pt}%
\pgfpathmoveto{\pgfqpoint{1.848052in}{0.528000in}}%
\pgfpathlineto{\pgfqpoint{1.721859in}{0.657484in}}%
\pgfpathlineto{\pgfqpoint{1.560371in}{0.826667in}}%
\pgfpathlineto{\pgfqpoint{1.251256in}{1.162667in}}%
\pgfpathlineto{\pgfqpoint{0.960323in}{1.494450in}}%
\pgfpathlineto{\pgfqpoint{0.840081in}{1.636545in}}%
\pgfpathlineto{\pgfqpoint{0.800000in}{1.684507in}}%
\pgfpathlineto{\pgfqpoint{0.800000in}{1.678974in}}%
\pgfpathlineto{\pgfqpoint{1.048656in}{1.386667in}}%
\pgfpathlineto{\pgfqpoint{1.179918in}{1.237333in}}%
\pgfpathlineto{\pgfqpoint{1.321051in}{1.080208in}}%
\pgfpathlineto{\pgfqpoint{1.481374in}{0.905951in}}%
\pgfpathlineto{\pgfqpoint{1.614894in}{0.764367in}}%
\pgfpathlineto{\pgfqpoint{1.681778in}{0.694143in}}%
\pgfpathlineto{\pgfqpoint{1.843187in}{0.528000in}}%
\pgfpathmoveto{\pgfqpoint{4.768000in}{1.844822in}}%
\pgfpathlineto{\pgfqpoint{4.482769in}{2.166321in}}%
\pgfpathlineto{\pgfqpoint{4.431062in}{2.223175in}}%
\pgfpathlineto{\pgfqpoint{4.287030in}{2.378750in}}%
\pgfpathlineto{\pgfqpoint{4.130594in}{2.544000in}}%
\pgfpathlineto{\pgfqpoint{3.838128in}{2.842667in}}%
\pgfpathlineto{\pgfqpoint{3.525495in}{3.148126in}}%
\pgfpathlineto{\pgfqpoint{3.204848in}{3.447234in}}%
\pgfpathlineto{\pgfqpoint{2.876981in}{3.738667in}}%
\pgfpathlineto{\pgfqpoint{2.563556in}{4.003801in}}%
\pgfpathlineto{\pgfqpoint{2.430840in}{4.112000in}}%
\pgfpathlineto{\pgfqpoint{2.290340in}{4.224000in}}%
\pgfpathlineto{\pgfqpoint{2.284320in}{4.224000in}}%
\pgfpathlineto{\pgfqpoint{2.572303in}{3.991852in}}%
\pgfpathlineto{\pgfqpoint{2.723879in}{3.865067in}}%
\pgfpathlineto{\pgfqpoint{2.884202in}{3.727667in}}%
\pgfpathlineto{\pgfqpoint{3.044525in}{3.586852in}}%
\pgfpathlineto{\pgfqpoint{3.207614in}{3.440000in}}%
\pgfpathlineto{\pgfqpoint{3.527627in}{3.141333in}}%
\pgfpathlineto{\pgfqpoint{3.685818in}{2.988427in}}%
\pgfpathlineto{\pgfqpoint{4.006465in}{2.667392in}}%
\pgfpathlineto{\pgfqpoint{4.161505in}{2.506667in}}%
\pgfpathlineto{\pgfqpoint{4.447354in}{2.200181in}}%
\pgfpathlineto{\pgfqpoint{4.608483in}{2.021333in}}%
\pgfpathlineto{\pgfqpoint{4.739742in}{1.872000in}}%
\pgfpathlineto{\pgfqpoint{4.768000in}{1.839489in}}%
\pgfpathlineto{\pgfqpoint{4.768000in}{1.839489in}}%
\pgfusepath{fill}%
\end{pgfscope}%
\begin{pgfscope}%
\pgfpathrectangle{\pgfqpoint{0.800000in}{0.528000in}}{\pgfqpoint{3.968000in}{3.696000in}}%
\pgfusepath{clip}%
\pgfsetbuttcap%
\pgfsetroundjoin%
\definecolor{currentfill}{rgb}{0.239346,0.300855,0.540844}%
\pgfsetfillcolor{currentfill}%
\pgfsetlinewidth{0.000000pt}%
\definecolor{currentstroke}{rgb}{0.000000,0.000000,0.000000}%
\pgfsetstrokecolor{currentstroke}%
\pgfsetdash{}{0pt}%
\pgfpathmoveto{\pgfqpoint{1.843187in}{0.528000in}}%
\pgfpathlineto{\pgfqpoint{1.721859in}{0.652524in}}%
\pgfpathlineto{\pgfqpoint{1.561535in}{0.820446in}}%
\pgfpathlineto{\pgfqpoint{1.416495in}{0.976000in}}%
\pgfpathlineto{\pgfqpoint{1.120646in}{1.304330in}}%
\pgfpathlineto{\pgfqpoint{0.880162in}{1.583351in}}%
\pgfpathlineto{\pgfqpoint{0.800000in}{1.678974in}}%
\pgfpathlineto{\pgfqpoint{0.800000in}{1.673442in}}%
\pgfpathlineto{\pgfqpoint{1.058944in}{1.369472in}}%
\pgfpathlineto{\pgfqpoint{1.200808in}{1.208593in}}%
\pgfpathlineto{\pgfqpoint{1.478108in}{0.904375in}}%
\pgfpathlineto{\pgfqpoint{1.601616in}{0.773125in}}%
\pgfpathlineto{\pgfqpoint{1.838398in}{0.528000in}}%
\pgfpathlineto{\pgfqpoint{1.842101in}{0.528000in}}%
\pgfpathmoveto{\pgfqpoint{4.768000in}{1.850155in}}%
\pgfpathlineto{\pgfqpoint{4.487434in}{2.166444in}}%
\pgfpathlineto{\pgfqpoint{4.241771in}{2.432000in}}%
\pgfpathlineto{\pgfqpoint{4.099439in}{2.581333in}}%
\pgfpathlineto{\pgfqpoint{3.954070in}{2.730667in}}%
\pgfpathlineto{\pgfqpoint{3.645737in}{3.037089in}}%
\pgfpathlineto{\pgfqpoint{3.485414in}{3.191032in}}%
\pgfpathlineto{\pgfqpoint{3.164768in}{3.488346in}}%
\pgfpathlineto{\pgfqpoint{2.839264in}{3.776000in}}%
\pgfpathlineto{\pgfqpoint{2.523475in}{4.041481in}}%
\pgfpathlineto{\pgfqpoint{2.390310in}{4.149333in}}%
\pgfpathlineto{\pgfqpoint{2.296359in}{4.224000in}}%
\pgfpathlineto{\pgfqpoint{2.290340in}{4.224000in}}%
\pgfpathlineto{\pgfqpoint{2.568151in}{4.000000in}}%
\pgfpathlineto{\pgfqpoint{2.723879in}{3.869793in}}%
\pgfpathlineto{\pgfqpoint{2.884202in}{3.732416in}}%
\pgfpathlineto{\pgfqpoint{3.047048in}{3.589333in}}%
\pgfpathlineto{\pgfqpoint{3.212772in}{3.440000in}}%
\pgfpathlineto{\pgfqpoint{3.532596in}{3.141333in}}%
\pgfpathlineto{\pgfqpoint{3.687114in}{2.992000in}}%
\pgfpathlineto{\pgfqpoint{4.006465in}{2.672304in}}%
\pgfpathlineto{\pgfqpoint{4.166788in}{2.506157in}}%
\pgfpathlineto{\pgfqpoint{4.327111in}{2.335788in}}%
\pgfpathlineto{\pgfqpoint{4.478896in}{2.170667in}}%
\pgfpathlineto{\pgfqpoint{4.768000in}{1.844822in}}%
\pgfpathlineto{\pgfqpoint{4.768000in}{1.844822in}}%
\pgfusepath{fill}%
\end{pgfscope}%
\begin{pgfscope}%
\pgfpathrectangle{\pgfqpoint{0.800000in}{0.528000in}}{\pgfqpoint{3.968000in}{3.696000in}}%
\pgfusepath{clip}%
\pgfsetbuttcap%
\pgfsetroundjoin%
\definecolor{currentfill}{rgb}{0.237441,0.305202,0.541921}%
\pgfsetfillcolor{currentfill}%
\pgfsetlinewidth{0.000000pt}%
\definecolor{currentstroke}{rgb}{0.000000,0.000000,0.000000}%
\pgfsetstrokecolor{currentstroke}%
\pgfsetdash{}{0pt}%
\pgfpathmoveto{\pgfqpoint{1.838398in}{0.528000in}}%
\pgfpathlineto{\pgfqpoint{1.687838in}{0.682978in}}%
\pgfpathlineto{\pgfqpoint{1.621718in}{0.752000in}}%
\pgfpathlineto{\pgfqpoint{1.321051in}{1.075089in}}%
\pgfpathlineto{\pgfqpoint{1.058944in}{1.369472in}}%
\pgfpathlineto{\pgfqpoint{0.920242in}{1.530567in}}%
\pgfpathlineto{\pgfqpoint{0.800000in}{1.673442in}}%
\pgfpathlineto{\pgfqpoint{0.800000in}{1.667909in}}%
\pgfpathlineto{\pgfqpoint{1.040485in}{1.385348in}}%
\pgfpathlineto{\pgfqpoint{1.200808in}{1.203352in}}%
\pgfpathlineto{\pgfqpoint{1.443151in}{0.936936in}}%
\pgfpathlineto{\pgfqpoint{1.601616in}{0.768146in}}%
\pgfpathlineto{\pgfqpoint{1.833631in}{0.528000in}}%
\pgfpathmoveto{\pgfqpoint{4.768000in}{1.855489in}}%
\pgfpathlineto{\pgfqpoint{4.492926in}{2.165551in}}%
\pgfpathlineto{\pgfqpoint{4.367192in}{2.302700in}}%
\pgfpathlineto{\pgfqpoint{4.068141in}{2.618667in}}%
\pgfpathlineto{\pgfqpoint{3.765980in}{2.924066in}}%
\pgfpathlineto{\pgfqpoint{3.445333in}{3.233704in}}%
\pgfpathlineto{\pgfqpoint{3.285010in}{3.383207in}}%
\pgfpathlineto{\pgfqpoint{2.964364in}{3.671828in}}%
\pgfpathlineto{\pgfqpoint{2.801314in}{3.813333in}}%
\pgfpathlineto{\pgfqpoint{2.643717in}{3.946645in}}%
\pgfpathlineto{\pgfqpoint{2.323071in}{4.207656in}}%
\pgfpathlineto{\pgfqpoint{2.302379in}{4.224000in}}%
\pgfpathlineto{\pgfqpoint{2.296359in}{4.224000in}}%
\pgfpathlineto{\pgfqpoint{2.573838in}{4.000000in}}%
\pgfpathlineto{\pgfqpoint{2.723879in}{3.874519in}}%
\pgfpathlineto{\pgfqpoint{2.884202in}{3.737164in}}%
\pgfpathlineto{\pgfqpoint{3.044525in}{3.596269in}}%
\pgfpathlineto{\pgfqpoint{3.365172in}{3.304162in}}%
\pgfpathlineto{\pgfqpoint{3.525495in}{3.152880in}}%
\pgfpathlineto{\pgfqpoint{3.685818in}{2.998045in}}%
\pgfpathlineto{\pgfqpoint{3.843021in}{2.842667in}}%
\pgfpathlineto{\pgfqpoint{4.135317in}{2.544000in}}%
\pgfpathlineto{\pgfqpoint{4.276841in}{2.394667in}}%
\pgfpathlineto{\pgfqpoint{4.415398in}{2.245333in}}%
\pgfpathlineto{\pgfqpoint{4.550985in}{2.096000in}}%
\pgfpathlineto{\pgfqpoint{4.687838in}{1.942048in}}%
\pgfpathlineto{\pgfqpoint{4.768000in}{1.850155in}}%
\pgfpathlineto{\pgfqpoint{4.768000in}{1.850155in}}%
\pgfusepath{fill}%
\end{pgfscope}%
\begin{pgfscope}%
\pgfpathrectangle{\pgfqpoint{0.800000in}{0.528000in}}{\pgfqpoint{3.968000in}{3.696000in}}%
\pgfusepath{clip}%
\pgfsetbuttcap%
\pgfsetroundjoin%
\definecolor{currentfill}{rgb}{0.237441,0.305202,0.541921}%
\pgfsetfillcolor{currentfill}%
\pgfsetlinewidth{0.000000pt}%
\definecolor{currentstroke}{rgb}{0.000000,0.000000,0.000000}%
\pgfsetstrokecolor{currentstroke}%
\pgfsetdash{}{0pt}%
\pgfpathmoveto{\pgfqpoint{1.833631in}{0.528000in}}%
\pgfpathlineto{\pgfqpoint{1.681778in}{0.684211in}}%
\pgfpathlineto{\pgfqpoint{1.521455in}{0.853051in}}%
\pgfpathlineto{\pgfqpoint{1.372781in}{1.013333in}}%
\pgfpathlineto{\pgfqpoint{1.080566in}{1.339451in}}%
\pgfpathlineto{\pgfqpoint{0.840081in}{1.619970in}}%
\pgfpathlineto{\pgfqpoint{0.800000in}{1.667909in}}%
\pgfpathlineto{\pgfqpoint{0.800000in}{1.662376in}}%
\pgfpathlineto{\pgfqpoint{1.040485in}{1.380079in}}%
\pgfpathlineto{\pgfqpoint{1.199154in}{1.200000in}}%
\pgfpathlineto{\pgfqpoint{1.481374in}{0.890872in}}%
\pgfpathlineto{\pgfqpoint{1.641697in}{0.721085in}}%
\pgfpathlineto{\pgfqpoint{1.828864in}{0.528000in}}%
\pgfpathmoveto{\pgfqpoint{4.768000in}{1.860822in}}%
\pgfpathlineto{\pgfqpoint{4.522614in}{2.137899in}}%
\pgfpathlineto{\pgfqpoint{4.367192in}{2.307762in}}%
\pgfpathlineto{\pgfqpoint{4.072905in}{2.618667in}}%
\pgfpathlineto{\pgfqpoint{3.765980in}{2.928854in}}%
\pgfpathlineto{\pgfqpoint{3.445333in}{3.238447in}}%
\pgfpathlineto{\pgfqpoint{3.285010in}{3.387927in}}%
\pgfpathlineto{\pgfqpoint{2.964364in}{3.676504in}}%
\pgfpathlineto{\pgfqpoint{2.804040in}{3.815698in}}%
\pgfpathlineto{\pgfqpoint{2.643717in}{3.951360in}}%
\pgfpathlineto{\pgfqpoint{2.323071in}{4.212411in}}%
\pgfpathlineto{\pgfqpoint{2.308399in}{4.224000in}}%
\pgfpathlineto{\pgfqpoint{2.302379in}{4.224000in}}%
\pgfpathlineto{\pgfqpoint{2.579524in}{4.000000in}}%
\pgfpathlineto{\pgfqpoint{2.723879in}{3.879245in}}%
\pgfpathlineto{\pgfqpoint{2.887867in}{3.738667in}}%
\pgfpathlineto{\pgfqpoint{3.044525in}{3.600956in}}%
\pgfpathlineto{\pgfqpoint{3.365172in}{3.308893in}}%
\pgfpathlineto{\pgfqpoint{3.525495in}{3.157633in}}%
\pgfpathlineto{\pgfqpoint{3.685818in}{3.002821in}}%
\pgfpathlineto{\pgfqpoint{3.847877in}{2.842667in}}%
\pgfpathlineto{\pgfqpoint{4.000681in}{2.687946in}}%
\pgfpathlineto{\pgfqpoint{4.068141in}{2.618667in}}%
\pgfpathlineto{\pgfqpoint{4.367192in}{2.302700in}}%
\pgfpathlineto{\pgfqpoint{4.522058in}{2.133333in}}%
\pgfpathlineto{\pgfqpoint{4.768000in}{1.855489in}}%
\pgfpathlineto{\pgfqpoint{4.768000in}{1.855489in}}%
\pgfusepath{fill}%
\end{pgfscope}%
\begin{pgfscope}%
\pgfpathrectangle{\pgfqpoint{0.800000in}{0.528000in}}{\pgfqpoint{3.968000in}{3.696000in}}%
\pgfusepath{clip}%
\pgfsetbuttcap%
\pgfsetroundjoin%
\definecolor{currentfill}{rgb}{0.237441,0.305202,0.541921}%
\pgfsetfillcolor{currentfill}%
\pgfsetlinewidth{0.000000pt}%
\definecolor{currentstroke}{rgb}{0.000000,0.000000,0.000000}%
\pgfsetstrokecolor{currentstroke}%
\pgfsetdash{}{0pt}%
\pgfpathmoveto{\pgfqpoint{1.828864in}{0.528000in}}%
\pgfpathlineto{\pgfqpoint{1.681778in}{0.679245in}}%
\pgfpathlineto{\pgfqpoint{1.521455in}{0.848060in}}%
\pgfpathlineto{\pgfqpoint{1.368093in}{1.013333in}}%
\pgfpathlineto{\pgfqpoint{1.080566in}{1.334190in}}%
\pgfpathlineto{\pgfqpoint{0.840081in}{1.614445in}}%
\pgfpathlineto{\pgfqpoint{0.800000in}{1.662376in}}%
\pgfpathlineto{\pgfqpoint{0.800000in}{1.656843in}}%
\pgfpathlineto{\pgfqpoint{1.040485in}{1.374810in}}%
\pgfpathlineto{\pgfqpoint{1.194565in}{1.200000in}}%
\pgfpathlineto{\pgfqpoint{1.481374in}{0.885875in}}%
\pgfpathlineto{\pgfqpoint{1.643081in}{0.714667in}}%
\pgfpathlineto{\pgfqpoint{1.787517in}{0.565333in}}%
\pgfpathlineto{\pgfqpoint{1.824096in}{0.528000in}}%
\pgfpathmoveto{\pgfqpoint{4.768000in}{1.866156in}}%
\pgfpathlineto{\pgfqpoint{4.527515in}{2.137581in}}%
\pgfpathlineto{\pgfqpoint{4.367192in}{2.312824in}}%
\pgfpathlineto{\pgfqpoint{4.077669in}{2.618667in}}%
\pgfpathlineto{\pgfqpoint{3.765980in}{2.933642in}}%
\pgfpathlineto{\pgfqpoint{3.445333in}{3.243189in}}%
\pgfpathlineto{\pgfqpoint{3.285010in}{3.392647in}}%
\pgfpathlineto{\pgfqpoint{2.964364in}{3.681180in}}%
\pgfpathlineto{\pgfqpoint{2.804040in}{3.820352in}}%
\pgfpathlineto{\pgfqpoint{2.643717in}{3.956075in}}%
\pgfpathlineto{\pgfqpoint{2.319064in}{4.220268in}}%
\pgfpathlineto{\pgfqpoint{2.314418in}{4.224000in}}%
\pgfpathlineto{\pgfqpoint{2.308399in}{4.224000in}}%
\pgfpathlineto{\pgfqpoint{2.563556in}{4.017913in}}%
\pgfpathlineto{\pgfqpoint{2.893229in}{3.738667in}}%
\pgfpathlineto{\pgfqpoint{3.044525in}{3.605642in}}%
\pgfpathlineto{\pgfqpoint{3.365172in}{3.313625in}}%
\pgfpathlineto{\pgfqpoint{3.525495in}{3.162387in}}%
\pgfpathlineto{\pgfqpoint{3.685818in}{3.007597in}}%
\pgfpathlineto{\pgfqpoint{3.846141in}{2.849204in}}%
\pgfpathlineto{\pgfqpoint{4.000307in}{2.693333in}}%
\pgfpathlineto{\pgfqpoint{4.287030in}{2.393897in}}%
\pgfpathlineto{\pgfqpoint{4.447354in}{2.220630in}}%
\pgfpathlineto{\pgfqpoint{4.716356in}{1.920104in}}%
\pgfpathlineto{\pgfqpoint{4.768000in}{1.860822in}}%
\pgfpathlineto{\pgfqpoint{4.768000in}{1.860822in}}%
\pgfusepath{fill}%
\end{pgfscope}%
\begin{pgfscope}%
\pgfpathrectangle{\pgfqpoint{0.800000in}{0.528000in}}{\pgfqpoint{3.968000in}{3.696000in}}%
\pgfusepath{clip}%
\pgfsetbuttcap%
\pgfsetroundjoin%
\definecolor{currentfill}{rgb}{0.237441,0.305202,0.541921}%
\pgfsetfillcolor{currentfill}%
\pgfsetlinewidth{0.000000pt}%
\definecolor{currentstroke}{rgb}{0.000000,0.000000,0.000000}%
\pgfsetstrokecolor{currentstroke}%
\pgfsetdash{}{0pt}%
\pgfpathmoveto{\pgfqpoint{1.824096in}{0.528000in}}%
\pgfpathlineto{\pgfqpoint{1.678896in}{0.677333in}}%
\pgfpathlineto{\pgfqpoint{1.536889in}{0.826667in}}%
\pgfpathlineto{\pgfqpoint{1.240889in}{1.148327in}}%
\pgfpathlineto{\pgfqpoint{0.981255in}{1.443497in}}%
\pgfpathlineto{\pgfqpoint{0.920242in}{1.514375in}}%
\pgfpathlineto{\pgfqpoint{0.800000in}{1.656843in}}%
\pgfpathlineto{\pgfqpoint{0.800000in}{1.651311in}}%
\pgfpathlineto{\pgfqpoint{1.032326in}{1.379067in}}%
\pgfpathlineto{\pgfqpoint{1.090808in}{1.312000in}}%
\pgfpathlineto{\pgfqpoint{1.393115in}{0.976000in}}%
\pgfpathlineto{\pgfqpoint{1.681778in}{0.669460in}}%
\pgfpathlineto{\pgfqpoint{1.819329in}{0.528000in}}%
\pgfpathmoveto{\pgfqpoint{4.768000in}{1.871489in}}%
\pgfpathlineto{\pgfqpoint{4.527515in}{2.142669in}}%
\pgfpathlineto{\pgfqpoint{4.365232in}{2.320000in}}%
\pgfpathlineto{\pgfqpoint{4.225257in}{2.469333in}}%
\pgfpathlineto{\pgfqpoint{3.926303in}{2.778261in}}%
\pgfpathlineto{\pgfqpoint{3.765980in}{2.938430in}}%
\pgfpathlineto{\pgfqpoint{3.445333in}{3.247931in}}%
\pgfpathlineto{\pgfqpoint{3.285010in}{3.397367in}}%
\pgfpathlineto{\pgfqpoint{2.964364in}{3.685855in}}%
\pgfpathlineto{\pgfqpoint{2.804040in}{3.825006in}}%
\pgfpathlineto{\pgfqpoint{2.641465in}{3.962667in}}%
\pgfpathlineto{\pgfqpoint{2.483394in}{4.093021in}}%
\pgfpathlineto{\pgfqpoint{2.320438in}{4.224000in}}%
\pgfpathlineto{\pgfqpoint{2.314418in}{4.224000in}}%
\pgfpathlineto{\pgfqpoint{2.563556in}{4.022617in}}%
\pgfpathlineto{\pgfqpoint{2.884202in}{3.751185in}}%
\pgfpathlineto{\pgfqpoint{3.044525in}{3.610329in}}%
\pgfpathlineto{\pgfqpoint{3.365172in}{3.318356in}}%
\pgfpathlineto{\pgfqpoint{3.525495in}{3.167141in}}%
\pgfpathlineto{\pgfqpoint{3.685818in}{3.012373in}}%
\pgfpathlineto{\pgfqpoint{3.846141in}{2.854003in}}%
\pgfpathlineto{\pgfqpoint{4.006465in}{2.691952in}}%
\pgfpathlineto{\pgfqpoint{4.166788in}{2.525911in}}%
\pgfpathlineto{\pgfqpoint{4.463517in}{2.208000in}}%
\pgfpathlineto{\pgfqpoint{4.602553in}{2.053894in}}%
\pgfpathlineto{\pgfqpoint{4.664686in}{1.984000in}}%
\pgfpathlineto{\pgfqpoint{4.768000in}{1.866156in}}%
\pgfpathlineto{\pgfqpoint{4.768000in}{1.866156in}}%
\pgfusepath{fill}%
\end{pgfscope}%
\begin{pgfscope}%
\pgfpathrectangle{\pgfqpoint{0.800000in}{0.528000in}}{\pgfqpoint{3.968000in}{3.696000in}}%
\pgfusepath{clip}%
\pgfsetbuttcap%
\pgfsetroundjoin%
\definecolor{currentfill}{rgb}{0.235526,0.309527,0.542944}%
\pgfsetfillcolor{currentfill}%
\pgfsetlinewidth{0.000000pt}%
\definecolor{currentstroke}{rgb}{0.000000,0.000000,0.000000}%
\pgfsetstrokecolor{currentstroke}%
\pgfsetdash{}{0pt}%
\pgfpathmoveto{\pgfqpoint{1.819329in}{0.528000in}}%
\pgfpathlineto{\pgfqpoint{1.674211in}{0.677333in}}%
\pgfpathlineto{\pgfqpoint{1.527224in}{0.832041in}}%
\pgfpathlineto{\pgfqpoint{1.462314in}{0.901333in}}%
\pgfpathlineto{\pgfqpoint{1.321051in}{1.054611in}}%
\pgfpathlineto{\pgfqpoint{1.025597in}{1.386667in}}%
\pgfpathlineto{\pgfqpoint{0.800000in}{1.651311in}}%
\pgfpathlineto{\pgfqpoint{0.800000in}{1.645824in}}%
\pgfpathlineto{\pgfqpoint{0.956435in}{1.461333in}}%
\pgfpathlineto{\pgfqpoint{1.086188in}{1.312000in}}%
\pgfpathlineto{\pgfqpoint{1.388498in}{0.976000in}}%
\pgfpathlineto{\pgfqpoint{1.681778in}{0.664585in}}%
\pgfpathlineto{\pgfqpoint{1.814562in}{0.528000in}}%
\pgfpathmoveto{\pgfqpoint{4.768000in}{1.876727in}}%
\pgfpathlineto{\pgfqpoint{4.607565in}{2.058770in}}%
\pgfpathlineto{\pgfqpoint{4.447354in}{2.235854in}}%
\pgfpathlineto{\pgfqpoint{4.300247in}{2.394667in}}%
\pgfpathlineto{\pgfqpoint{4.006465in}{2.701622in}}%
\pgfpathlineto{\pgfqpoint{3.685818in}{3.021926in}}%
\pgfpathlineto{\pgfqpoint{3.523375in}{3.178667in}}%
\pgfpathlineto{\pgfqpoint{3.364976in}{3.328000in}}%
\pgfpathlineto{\pgfqpoint{3.036674in}{3.626667in}}%
\pgfpathlineto{\pgfqpoint{2.723879in}{3.897971in}}%
\pgfpathlineto{\pgfqpoint{2.403232in}{4.162580in}}%
\pgfpathlineto{\pgfqpoint{2.326374in}{4.224000in}}%
\pgfpathlineto{\pgfqpoint{2.320438in}{4.224000in}}%
\pgfpathlineto{\pgfqpoint{2.563556in}{4.027320in}}%
\pgfpathlineto{\pgfqpoint{2.884202in}{3.755850in}}%
\pgfpathlineto{\pgfqpoint{3.044525in}{3.615016in}}%
\pgfpathlineto{\pgfqpoint{3.365172in}{3.323087in}}%
\pgfpathlineto{\pgfqpoint{3.525495in}{3.171894in}}%
\pgfpathlineto{\pgfqpoint{3.685818in}{3.017150in}}%
\pgfpathlineto{\pgfqpoint{3.846141in}{2.858802in}}%
\pgfpathlineto{\pgfqpoint{4.009850in}{2.693333in}}%
\pgfpathlineto{\pgfqpoint{4.154205in}{2.544000in}}%
\pgfpathlineto{\pgfqpoint{4.295606in}{2.394667in}}%
\pgfpathlineto{\pgfqpoint{4.434010in}{2.245333in}}%
\pgfpathlineto{\pgfqpoint{4.569602in}{2.096000in}}%
\pgfpathlineto{\pgfqpoint{4.702199in}{1.946667in}}%
\pgfpathlineto{\pgfqpoint{4.768000in}{1.871489in}}%
\pgfpathlineto{\pgfqpoint{4.768000in}{1.872000in}}%
\pgfusepath{fill}%
\end{pgfscope}%
\begin{pgfscope}%
\pgfpathrectangle{\pgfqpoint{0.800000in}{0.528000in}}{\pgfqpoint{3.968000in}{3.696000in}}%
\pgfusepath{clip}%
\pgfsetbuttcap%
\pgfsetroundjoin%
\definecolor{currentfill}{rgb}{0.235526,0.309527,0.542944}%
\pgfsetfillcolor{currentfill}%
\pgfsetlinewidth{0.000000pt}%
\definecolor{currentstroke}{rgb}{0.000000,0.000000,0.000000}%
\pgfsetstrokecolor{currentstroke}%
\pgfsetdash{}{0pt}%
\pgfpathmoveto{\pgfqpoint{1.814562in}{0.528000in}}%
\pgfpathlineto{\pgfqpoint{1.669526in}{0.677333in}}%
\pgfpathlineto{\pgfqpoint{1.521455in}{0.833087in}}%
\pgfpathlineto{\pgfqpoint{1.361131in}{1.005734in}}%
\pgfpathlineto{\pgfqpoint{1.101858in}{1.294499in}}%
\pgfpathlineto{\pgfqpoint{1.040485in}{1.364273in}}%
\pgfpathlineto{\pgfqpoint{0.800000in}{1.645824in}}%
\pgfpathlineto{\pgfqpoint{0.800000in}{1.640404in}}%
\pgfpathlineto{\pgfqpoint{0.920242in}{1.498192in}}%
\pgfpathlineto{\pgfqpoint{1.081569in}{1.312000in}}%
\pgfpathlineto{\pgfqpoint{1.383881in}{0.976000in}}%
\pgfpathlineto{\pgfqpoint{1.681778in}{0.659710in}}%
\pgfpathlineto{\pgfqpoint{1.809795in}{0.528000in}}%
\pgfpathmoveto{\pgfqpoint{4.768000in}{1.881955in}}%
\pgfpathlineto{\pgfqpoint{4.612214in}{2.058667in}}%
\pgfpathlineto{\pgfqpoint{4.327111in}{2.370916in}}%
\pgfpathlineto{\pgfqpoint{4.163650in}{2.544000in}}%
\pgfpathlineto{\pgfqpoint{3.846141in}{2.868401in}}%
\pgfpathlineto{\pgfqpoint{3.528308in}{3.178667in}}%
\pgfpathlineto{\pgfqpoint{3.244929in}{3.443546in}}%
\pgfpathlineto{\pgfqpoint{3.083853in}{3.589333in}}%
\pgfpathlineto{\pgfqpoint{2.763960in}{3.868568in}}%
\pgfpathlineto{\pgfqpoint{2.603636in}{4.003515in}}%
\pgfpathlineto{\pgfqpoint{2.332245in}{4.224000in}}%
\pgfpathlineto{\pgfqpoint{2.326374in}{4.224000in}}%
\pgfpathlineto{\pgfqpoint{2.483394in}{4.097714in}}%
\pgfpathlineto{\pgfqpoint{2.804040in}{3.829660in}}%
\pgfpathlineto{\pgfqpoint{3.124687in}{3.548023in}}%
\pgfpathlineto{\pgfqpoint{3.445333in}{3.252673in}}%
\pgfpathlineto{\pgfqpoint{3.765980in}{2.943217in}}%
\pgfpathlineto{\pgfqpoint{3.926303in}{2.783072in}}%
\pgfpathlineto{\pgfqpoint{4.229938in}{2.469333in}}%
\pgfpathlineto{\pgfqpoint{4.369872in}{2.320000in}}%
\pgfpathlineto{\pgfqpoint{4.506741in}{2.170667in}}%
\pgfpathlineto{\pgfqpoint{4.647758in}{2.013559in}}%
\pgfpathlineto{\pgfqpoint{4.768000in}{1.876727in}}%
\pgfpathlineto{\pgfqpoint{4.768000in}{1.876727in}}%
\pgfusepath{fill}%
\end{pgfscope}%
\begin{pgfscope}%
\pgfpathrectangle{\pgfqpoint{0.800000in}{0.528000in}}{\pgfqpoint{3.968000in}{3.696000in}}%
\pgfusepath{clip}%
\pgfsetbuttcap%
\pgfsetroundjoin%
\definecolor{currentfill}{rgb}{0.235526,0.309527,0.542944}%
\pgfsetfillcolor{currentfill}%
\pgfsetlinewidth{0.000000pt}%
\definecolor{currentstroke}{rgb}{0.000000,0.000000,0.000000}%
\pgfsetstrokecolor{currentstroke}%
\pgfsetdash{}{0pt}%
\pgfpathmoveto{\pgfqpoint{1.809795in}{0.528000in}}%
\pgfpathlineto{\pgfqpoint{1.681778in}{0.659710in}}%
\pgfpathlineto{\pgfqpoint{1.521455in}{0.828096in}}%
\pgfpathlineto{\pgfqpoint{1.361131in}{1.000718in}}%
\pgfpathlineto{\pgfqpoint{1.081569in}{1.312000in}}%
\pgfpathlineto{\pgfqpoint{0.951892in}{1.461333in}}%
\pgfpathlineto{\pgfqpoint{0.824983in}{1.610667in}}%
\pgfpathlineto{\pgfqpoint{0.800000in}{1.640404in}}%
\pgfpathlineto{\pgfqpoint{0.800000in}{1.634985in}}%
\pgfpathlineto{\pgfqpoint{0.920242in}{1.492903in}}%
\pgfpathlineto{\pgfqpoint{1.077018in}{1.312000in}}%
\pgfpathlineto{\pgfqpoint{1.209581in}{1.162667in}}%
\pgfpathlineto{\pgfqpoint{1.518168in}{0.826667in}}%
\pgfpathlineto{\pgfqpoint{1.805028in}{0.528000in}}%
\pgfpathmoveto{\pgfqpoint{4.768000in}{1.887183in}}%
\pgfpathlineto{\pgfqpoint{4.616769in}{2.058667in}}%
\pgfpathlineto{\pgfqpoint{4.327111in}{2.375877in}}%
\pgfpathlineto{\pgfqpoint{4.166788in}{2.545625in}}%
\pgfpathlineto{\pgfqpoint{3.846141in}{2.873200in}}%
\pgfpathlineto{\pgfqpoint{3.533197in}{3.178667in}}%
\pgfpathlineto{\pgfqpoint{3.244929in}{3.448178in}}%
\pgfpathlineto{\pgfqpoint{3.084606in}{3.593284in}}%
\pgfpathlineto{\pgfqpoint{2.763960in}{3.873216in}}%
\pgfpathlineto{\pgfqpoint{2.603636in}{4.008142in}}%
\pgfpathlineto{\pgfqpoint{2.338116in}{4.224000in}}%
\pgfpathlineto{\pgfqpoint{2.332245in}{4.224000in}}%
\pgfpathlineto{\pgfqpoint{2.483394in}{4.102407in}}%
\pgfpathlineto{\pgfqpoint{2.804040in}{3.834313in}}%
\pgfpathlineto{\pgfqpoint{3.125472in}{3.552000in}}%
\pgfpathlineto{\pgfqpoint{3.449581in}{3.253333in}}%
\pgfpathlineto{\pgfqpoint{3.765980in}{2.948005in}}%
\pgfpathlineto{\pgfqpoint{3.926303in}{2.787882in}}%
\pgfpathlineto{\pgfqpoint{4.234620in}{2.469333in}}%
\pgfpathlineto{\pgfqpoint{4.374474in}{2.320000in}}%
\pgfpathlineto{\pgfqpoint{4.511354in}{2.170667in}}%
\pgfpathlineto{\pgfqpoint{4.647758in}{2.018767in}}%
\pgfpathlineto{\pgfqpoint{4.768000in}{1.881955in}}%
\pgfpathlineto{\pgfqpoint{4.768000in}{1.881955in}}%
\pgfusepath{fill}%
\end{pgfscope}%
\begin{pgfscope}%
\pgfpathrectangle{\pgfqpoint{0.800000in}{0.528000in}}{\pgfqpoint{3.968000in}{3.696000in}}%
\pgfusepath{clip}%
\pgfsetbuttcap%
\pgfsetroundjoin%
\definecolor{currentfill}{rgb}{0.235526,0.309527,0.542944}%
\pgfsetfillcolor{currentfill}%
\pgfsetlinewidth{0.000000pt}%
\definecolor{currentstroke}{rgb}{0.000000,0.000000,0.000000}%
\pgfsetstrokecolor{currentstroke}%
\pgfsetdash{}{0pt}%
\pgfpathmoveto{\pgfqpoint{1.805028in}{0.528000in}}%
\pgfpathlineto{\pgfqpoint{1.681778in}{0.654836in}}%
\pgfpathlineto{\pgfqpoint{1.518168in}{0.826667in}}%
\pgfpathlineto{\pgfqpoint{1.379265in}{0.976000in}}%
\pgfpathlineto{\pgfqpoint{1.240889in}{1.127796in}}%
\pgfpathlineto{\pgfqpoint{0.947350in}{1.461333in}}%
\pgfpathlineto{\pgfqpoint{0.820430in}{1.610667in}}%
\pgfpathlineto{\pgfqpoint{0.800000in}{1.634985in}}%
\pgfpathlineto{\pgfqpoint{0.800000in}{1.629565in}}%
\pgfpathlineto{\pgfqpoint{0.920242in}{1.487613in}}%
\pgfpathlineto{\pgfqpoint{1.072487in}{1.312000in}}%
\pgfpathlineto{\pgfqpoint{1.204973in}{1.162667in}}%
\pgfpathlineto{\pgfqpoint{1.513563in}{0.826667in}}%
\pgfpathlineto{\pgfqpoint{1.800295in}{0.528000in}}%
\pgfpathlineto{\pgfqpoint{1.802020in}{0.528000in}}%
\pgfpathmoveto{\pgfqpoint{4.768000in}{1.892411in}}%
\pgfpathlineto{\pgfqpoint{4.621324in}{2.058667in}}%
\pgfpathlineto{\pgfqpoint{4.327111in}{2.380837in}}%
\pgfpathlineto{\pgfqpoint{4.172971in}{2.544000in}}%
\pgfpathlineto{\pgfqpoint{4.046545in}{2.675034in}}%
\pgfpathlineto{\pgfqpoint{3.886222in}{2.837866in}}%
\pgfpathlineto{\pgfqpoint{3.730929in}{2.992000in}}%
\pgfpathlineto{\pgfqpoint{3.605657in}{3.113851in}}%
\pgfpathlineto{\pgfqpoint{3.285010in}{3.416008in}}%
\pgfpathlineto{\pgfqpoint{3.124687in}{3.561939in}}%
\pgfpathlineto{\pgfqpoint{2.795802in}{3.850667in}}%
\pgfpathlineto{\pgfqpoint{2.643717in}{3.979352in}}%
\pgfpathlineto{\pgfqpoint{2.483138in}{4.112000in}}%
\pgfpathlineto{\pgfqpoint{2.343987in}{4.224000in}}%
\pgfpathlineto{\pgfqpoint{2.338116in}{4.224000in}}%
\pgfpathlineto{\pgfqpoint{2.483394in}{4.107100in}}%
\pgfpathlineto{\pgfqpoint{2.804040in}{3.838967in}}%
\pgfpathlineto{\pgfqpoint{3.130592in}{3.552000in}}%
\pgfpathlineto{\pgfqpoint{3.454514in}{3.253333in}}%
\pgfpathlineto{\pgfqpoint{3.765980in}{2.952793in}}%
\pgfpathlineto{\pgfqpoint{3.926303in}{2.792693in}}%
\pgfpathlineto{\pgfqpoint{4.239301in}{2.469333in}}%
\pgfpathlineto{\pgfqpoint{4.379076in}{2.320000in}}%
\pgfpathlineto{\pgfqpoint{4.515967in}{2.170667in}}%
\pgfpathlineto{\pgfqpoint{4.650058in}{2.021333in}}%
\pgfpathlineto{\pgfqpoint{4.768000in}{1.887183in}}%
\pgfpathlineto{\pgfqpoint{4.768000in}{1.887183in}}%
\pgfusepath{fill}%
\end{pgfscope}%
\begin{pgfscope}%
\pgfpathrectangle{\pgfqpoint{0.800000in}{0.528000in}}{\pgfqpoint{3.968000in}{3.696000in}}%
\pgfusepath{clip}%
\pgfsetbuttcap%
\pgfsetroundjoin%
\definecolor{currentfill}{rgb}{0.233603,0.313828,0.543914}%
\pgfsetfillcolor{currentfill}%
\pgfsetlinewidth{0.000000pt}%
\definecolor{currentstroke}{rgb}{0.000000,0.000000,0.000000}%
\pgfsetstrokecolor{currentstroke}%
\pgfsetdash{}{0pt}%
\pgfpathmoveto{\pgfqpoint{1.800295in}{0.528000in}}%
\pgfpathlineto{\pgfqpoint{1.478518in}{0.864000in}}%
\pgfpathlineto{\pgfqpoint{1.171622in}{1.200000in}}%
\pgfpathlineto{\pgfqpoint{1.039743in}{1.349333in}}%
\pgfpathlineto{\pgfqpoint{0.800000in}{1.629565in}}%
\pgfpathlineto{\pgfqpoint{0.800000in}{1.624146in}}%
\pgfpathlineto{\pgfqpoint{0.920242in}{1.482323in}}%
\pgfpathlineto{\pgfqpoint{1.067956in}{1.312000in}}%
\pgfpathlineto{\pgfqpoint{1.200808in}{1.162182in}}%
\pgfpathlineto{\pgfqpoint{1.508957in}{0.826667in}}%
\pgfpathlineto{\pgfqpoint{1.795622in}{0.528000in}}%
\pgfpathmoveto{\pgfqpoint{4.768000in}{1.897639in}}%
\pgfpathlineto{\pgfqpoint{4.647758in}{2.034138in}}%
\pgfpathlineto{\pgfqpoint{4.491240in}{2.208000in}}%
\pgfpathlineto{\pgfqpoint{4.353631in}{2.357333in}}%
\pgfpathlineto{\pgfqpoint{4.046545in}{2.679862in}}%
\pgfpathlineto{\pgfqpoint{3.886222in}{2.842671in}}%
\pgfpathlineto{\pgfqpoint{3.725899in}{3.001638in}}%
\pgfpathlineto{\pgfqpoint{3.565576in}{3.157056in}}%
\pgfpathlineto{\pgfqpoint{3.405253in}{3.308974in}}%
\pgfpathlineto{\pgfqpoint{3.244929in}{3.457441in}}%
\pgfpathlineto{\pgfqpoint{3.084606in}{3.602505in}}%
\pgfpathlineto{\pgfqpoint{2.757503in}{3.888000in}}%
\pgfpathlineto{\pgfqpoint{2.603636in}{4.017396in}}%
\pgfpathlineto{\pgfqpoint{2.349858in}{4.224000in}}%
\pgfpathlineto{\pgfqpoint{2.343987in}{4.224000in}}%
\pgfpathlineto{\pgfqpoint{2.483394in}{4.111792in}}%
\pgfpathlineto{\pgfqpoint{2.804040in}{3.843621in}}%
\pgfpathlineto{\pgfqpoint{3.124687in}{3.561939in}}%
\pgfpathlineto{\pgfqpoint{3.285010in}{3.416008in}}%
\pgfpathlineto{\pgfqpoint{3.445333in}{3.266661in}}%
\pgfpathlineto{\pgfqpoint{3.605657in}{3.113851in}}%
\pgfpathlineto{\pgfqpoint{3.918536in}{2.805333in}}%
\pgfpathlineto{\pgfqpoint{4.046545in}{2.675034in}}%
\pgfpathlineto{\pgfqpoint{4.208593in}{2.506667in}}%
\pgfpathlineto{\pgfqpoint{4.520581in}{2.170667in}}%
\pgfpathlineto{\pgfqpoint{4.654594in}{2.021333in}}%
\pgfpathlineto{\pgfqpoint{4.768000in}{1.892411in}}%
\pgfpathlineto{\pgfqpoint{4.768000in}{1.892411in}}%
\pgfusepath{fill}%
\end{pgfscope}%
\begin{pgfscope}%
\pgfpathrectangle{\pgfqpoint{0.800000in}{0.528000in}}{\pgfqpoint{3.968000in}{3.696000in}}%
\pgfusepath{clip}%
\pgfsetbuttcap%
\pgfsetroundjoin%
\definecolor{currentfill}{rgb}{0.233603,0.313828,0.543914}%
\pgfsetfillcolor{currentfill}%
\pgfsetlinewidth{0.000000pt}%
\definecolor{currentstroke}{rgb}{0.000000,0.000000,0.000000}%
\pgfsetstrokecolor{currentstroke}%
\pgfsetdash{}{0pt}%
\pgfpathmoveto{\pgfqpoint{1.795622in}{0.528000in}}%
\pgfpathlineto{\pgfqpoint{1.481374in}{0.856042in}}%
\pgfpathlineto{\pgfqpoint{1.335779in}{1.013333in}}%
\pgfpathlineto{\pgfqpoint{1.200373in}{1.162667in}}%
\pgfpathlineto{\pgfqpoint{0.906267in}{1.498667in}}%
\pgfpathlineto{\pgfqpoint{0.800000in}{1.624146in}}%
\pgfpathlineto{\pgfqpoint{0.800000in}{1.618726in}}%
\pgfpathlineto{\pgfqpoint{0.920242in}{1.477034in}}%
\pgfpathlineto{\pgfqpoint{1.053214in}{1.323857in}}%
\pgfpathlineto{\pgfqpoint{1.120646in}{1.247075in}}%
\pgfpathlineto{\pgfqpoint{1.263201in}{1.088000in}}%
\pgfpathlineto{\pgfqpoint{1.561535in}{0.766052in}}%
\pgfpathlineto{\pgfqpoint{1.718119in}{0.602667in}}%
\pgfpathlineto{\pgfqpoint{1.790948in}{0.528000in}}%
\pgfpathmoveto{\pgfqpoint{4.768000in}{1.902867in}}%
\pgfpathlineto{\pgfqpoint{4.647758in}{2.039245in}}%
\pgfpathlineto{\pgfqpoint{4.495785in}{2.208000in}}%
\pgfpathlineto{\pgfqpoint{4.358252in}{2.357333in}}%
\pgfpathlineto{\pgfqpoint{4.046545in}{2.684691in}}%
\pgfpathlineto{\pgfqpoint{3.890925in}{2.842667in}}%
\pgfpathlineto{\pgfqpoint{3.605657in}{3.123212in}}%
\pgfpathlineto{\pgfqpoint{3.285010in}{3.425283in}}%
\pgfpathlineto{\pgfqpoint{3.124687in}{3.571171in}}%
\pgfpathlineto{\pgfqpoint{2.804040in}{3.852890in}}%
\pgfpathlineto{\pgfqpoint{2.643717in}{3.988617in}}%
\pgfpathlineto{\pgfqpoint{2.483394in}{4.121018in}}%
\pgfpathlineto{\pgfqpoint{2.355728in}{4.224000in}}%
\pgfpathlineto{\pgfqpoint{2.349858in}{4.224000in}}%
\pgfpathlineto{\pgfqpoint{2.488785in}{4.112000in}}%
\pgfpathlineto{\pgfqpoint{2.804040in}{3.848275in}}%
\pgfpathlineto{\pgfqpoint{3.124687in}{3.566555in}}%
\pgfpathlineto{\pgfqpoint{3.285010in}{3.420645in}}%
\pgfpathlineto{\pgfqpoint{3.445333in}{3.271320in}}%
\pgfpathlineto{\pgfqpoint{3.605657in}{3.118532in}}%
\pgfpathlineto{\pgfqpoint{3.923308in}{2.805333in}}%
\pgfpathlineto{\pgfqpoint{4.046545in}{2.679862in}}%
\pgfpathlineto{\pgfqpoint{4.206869in}{2.513333in}}%
\pgfpathlineto{\pgfqpoint{4.353631in}{2.357333in}}%
\pgfpathlineto{\pgfqpoint{4.491240in}{2.208000in}}%
\pgfpathlineto{\pgfqpoint{4.625879in}{2.058667in}}%
\pgfpathlineto{\pgfqpoint{4.757787in}{1.909333in}}%
\pgfpathlineto{\pgfqpoint{4.768000in}{1.897639in}}%
\pgfpathlineto{\pgfqpoint{4.768000in}{1.897639in}}%
\pgfusepath{fill}%
\end{pgfscope}%
\begin{pgfscope}%
\pgfpathrectangle{\pgfqpoint{0.800000in}{0.528000in}}{\pgfqpoint{3.968000in}{3.696000in}}%
\pgfusepath{clip}%
\pgfsetbuttcap%
\pgfsetroundjoin%
\definecolor{currentfill}{rgb}{0.233603,0.313828,0.543914}%
\pgfsetfillcolor{currentfill}%
\pgfsetlinewidth{0.000000pt}%
\definecolor{currentstroke}{rgb}{0.000000,0.000000,0.000000}%
\pgfsetstrokecolor{currentstroke}%
\pgfsetdash{}{0pt}%
\pgfpathmoveto{\pgfqpoint{1.790948in}{0.528000in}}%
\pgfpathlineto{\pgfqpoint{1.481374in}{0.851137in}}%
\pgfpathlineto{\pgfqpoint{1.331182in}{1.013333in}}%
\pgfpathlineto{\pgfqpoint{1.195853in}{1.162667in}}%
\pgfpathlineto{\pgfqpoint{0.901744in}{1.498667in}}%
\pgfpathlineto{\pgfqpoint{0.800000in}{1.618726in}}%
\pgfpathlineto{\pgfqpoint{0.800000in}{1.613307in}}%
\pgfpathlineto{\pgfqpoint{0.920242in}{1.471744in}}%
\pgfpathlineto{\pgfqpoint{1.050700in}{1.321515in}}%
\pgfpathlineto{\pgfqpoint{1.120646in}{1.241923in}}%
\pgfpathlineto{\pgfqpoint{1.250599in}{1.097045in}}%
\pgfpathlineto{\pgfqpoint{1.321051in}{1.019379in}}%
\pgfpathlineto{\pgfqpoint{1.464760in}{0.864000in}}%
\pgfpathlineto{\pgfqpoint{1.605735in}{0.714667in}}%
\pgfpathlineto{\pgfqpoint{1.786275in}{0.528000in}}%
\pgfpathmoveto{\pgfqpoint{4.768000in}{1.908095in}}%
\pgfpathlineto{\pgfqpoint{4.647758in}{2.044352in}}%
\pgfpathlineto{\pgfqpoint{4.500329in}{2.208000in}}%
\pgfpathlineto{\pgfqpoint{4.362874in}{2.357333in}}%
\pgfpathlineto{\pgfqpoint{4.042824in}{2.693333in}}%
\pgfpathlineto{\pgfqpoint{3.886222in}{2.852110in}}%
\pgfpathlineto{\pgfqpoint{3.725899in}{3.011033in}}%
\pgfpathlineto{\pgfqpoint{3.565576in}{3.166407in}}%
\pgfpathlineto{\pgfqpoint{3.405253in}{3.318281in}}%
\pgfpathlineto{\pgfqpoint{3.244929in}{3.466705in}}%
\pgfpathlineto{\pgfqpoint{3.084606in}{3.611725in}}%
\pgfpathlineto{\pgfqpoint{2.763960in}{3.891744in}}%
\pgfpathlineto{\pgfqpoint{2.603636in}{4.026650in}}%
\pgfpathlineto{\pgfqpoint{2.361599in}{4.224000in}}%
\pgfpathlineto{\pgfqpoint{2.355728in}{4.224000in}}%
\pgfpathlineto{\pgfqpoint{2.483394in}{4.121018in}}%
\pgfpathlineto{\pgfqpoint{2.806627in}{3.850667in}}%
\pgfpathlineto{\pgfqpoint{3.124687in}{3.571171in}}%
\pgfpathlineto{\pgfqpoint{3.285010in}{3.425283in}}%
\pgfpathlineto{\pgfqpoint{3.445333in}{3.275979in}}%
\pgfpathlineto{\pgfqpoint{3.605657in}{3.123212in}}%
\pgfpathlineto{\pgfqpoint{3.928046in}{2.805333in}}%
\pgfpathlineto{\pgfqpoint{4.246949in}{2.475970in}}%
\pgfpathlineto{\pgfqpoint{4.529763in}{2.170667in}}%
\pgfpathlineto{\pgfqpoint{4.768000in}{1.902867in}}%
\pgfpathlineto{\pgfqpoint{4.768000in}{1.902867in}}%
\pgfusepath{fill}%
\end{pgfscope}%
\begin{pgfscope}%
\pgfpathrectangle{\pgfqpoint{0.800000in}{0.528000in}}{\pgfqpoint{3.968000in}{3.696000in}}%
\pgfusepath{clip}%
\pgfsetbuttcap%
\pgfsetroundjoin%
\definecolor{currentfill}{rgb}{0.233603,0.313828,0.543914}%
\pgfsetfillcolor{currentfill}%
\pgfsetlinewidth{0.000000pt}%
\definecolor{currentstroke}{rgb}{0.000000,0.000000,0.000000}%
\pgfsetstrokecolor{currentstroke}%
\pgfsetdash{}{0pt}%
\pgfpathmoveto{\pgfqpoint{1.786275in}{0.528000in}}%
\pgfpathlineto{\pgfqpoint{1.481374in}{0.846232in}}%
\pgfpathlineto{\pgfqpoint{1.326585in}{1.013333in}}%
\pgfpathlineto{\pgfqpoint{1.191332in}{1.162667in}}%
\pgfpathlineto{\pgfqpoint{0.897221in}{1.498667in}}%
\pgfpathlineto{\pgfqpoint{0.800000in}{1.613307in}}%
\pgfpathlineto{\pgfqpoint{0.800000in}{1.607943in}}%
\pgfpathlineto{\pgfqpoint{0.956793in}{1.424000in}}%
\pgfpathlineto{\pgfqpoint{1.240889in}{1.102574in}}%
\pgfpathlineto{\pgfqpoint{1.530277in}{0.789333in}}%
\pgfpathlineto{\pgfqpoint{1.781601in}{0.528000in}}%
\pgfpathmoveto{\pgfqpoint{4.768000in}{1.913246in}}%
\pgfpathlineto{\pgfqpoint{4.470773in}{2.245333in}}%
\pgfpathlineto{\pgfqpoint{4.166788in}{2.569854in}}%
\pgfpathlineto{\pgfqpoint{3.863051in}{2.880000in}}%
\pgfpathlineto{\pgfqpoint{3.557643in}{3.178667in}}%
\pgfpathlineto{\pgfqpoint{3.238363in}{3.477333in}}%
\pgfpathlineto{\pgfqpoint{3.073025in}{3.626667in}}%
\pgfpathlineto{\pgfqpoint{2.924283in}{3.757979in}}%
\pgfpathlineto{\pgfqpoint{2.763960in}{3.896313in}}%
\pgfpathlineto{\pgfqpoint{2.603636in}{4.031277in}}%
\pgfpathlineto{\pgfqpoint{2.367366in}{4.224000in}}%
\pgfpathlineto{\pgfqpoint{2.361599in}{4.224000in}}%
\pgfpathlineto{\pgfqpoint{2.483394in}{4.125629in}}%
\pgfpathlineto{\pgfqpoint{2.811947in}{3.850667in}}%
\pgfpathlineto{\pgfqpoint{3.124687in}{3.575787in}}%
\pgfpathlineto{\pgfqpoint{3.285010in}{3.429920in}}%
\pgfpathlineto{\pgfqpoint{3.445333in}{3.280638in}}%
\pgfpathlineto{\pgfqpoint{3.605657in}{3.127893in}}%
\pgfpathlineto{\pgfqpoint{3.926303in}{2.811818in}}%
\pgfpathlineto{\pgfqpoint{4.079106in}{2.656000in}}%
\pgfpathlineto{\pgfqpoint{4.367192in}{2.352693in}}%
\pgfpathlineto{\pgfqpoint{4.668203in}{2.021333in}}%
\pgfpathlineto{\pgfqpoint{4.768000in}{1.908095in}}%
\pgfpathlineto{\pgfqpoint{4.768000in}{1.909333in}}%
\pgfpathlineto{\pgfqpoint{4.768000in}{1.909333in}}%
\pgfusepath{fill}%
\end{pgfscope}%
\begin{pgfscope}%
\pgfpathrectangle{\pgfqpoint{0.800000in}{0.528000in}}{\pgfqpoint{3.968000in}{3.696000in}}%
\pgfusepath{clip}%
\pgfsetbuttcap%
\pgfsetroundjoin%
\definecolor{currentfill}{rgb}{0.231674,0.318106,0.544834}%
\pgfsetfillcolor{currentfill}%
\pgfsetlinewidth{0.000000pt}%
\definecolor{currentstroke}{rgb}{0.000000,0.000000,0.000000}%
\pgfsetstrokecolor{currentstroke}%
\pgfsetdash{}{0pt}%
\pgfpathmoveto{\pgfqpoint{1.781601in}{0.528000in}}%
\pgfpathlineto{\pgfqpoint{1.481374in}{0.841328in}}%
\pgfpathlineto{\pgfqpoint{1.321051in}{1.014357in}}%
\pgfpathlineto{\pgfqpoint{1.160727in}{1.191793in}}%
\pgfpathlineto{\pgfqpoint{1.021693in}{1.349333in}}%
\pgfpathlineto{\pgfqpoint{0.800000in}{1.607943in}}%
\pgfpathlineto{\pgfqpoint{0.800000in}{1.602633in}}%
\pgfpathlineto{\pgfqpoint{0.952317in}{1.424000in}}%
\pgfpathlineto{\pgfqpoint{1.240889in}{1.097539in}}%
\pgfpathlineto{\pgfqpoint{1.525652in}{0.789333in}}%
\pgfpathlineto{\pgfqpoint{1.776928in}{0.528000in}}%
\pgfpathmoveto{\pgfqpoint{4.768000in}{1.918373in}}%
\pgfpathlineto{\pgfqpoint{4.475337in}{2.245333in}}%
\pgfpathlineto{\pgfqpoint{4.166788in}{2.574700in}}%
\pgfpathlineto{\pgfqpoint{3.867770in}{2.880000in}}%
\pgfpathlineto{\pgfqpoint{3.562533in}{3.178667in}}%
\pgfpathlineto{\pgfqpoint{3.243435in}{3.477333in}}%
\pgfpathlineto{\pgfqpoint{3.078193in}{3.626667in}}%
\pgfpathlineto{\pgfqpoint{2.924283in}{3.762569in}}%
\pgfpathlineto{\pgfqpoint{2.763960in}{3.900881in}}%
\pgfpathlineto{\pgfqpoint{2.601912in}{4.037333in}}%
\pgfpathlineto{\pgfqpoint{2.443313in}{4.167427in}}%
\pgfpathlineto{\pgfqpoint{2.373096in}{4.224000in}}%
\pgfpathlineto{\pgfqpoint{2.367366in}{4.224000in}}%
\pgfpathlineto{\pgfqpoint{2.666482in}{3.978796in}}%
\pgfpathlineto{\pgfqpoint{2.804040in}{3.862037in}}%
\pgfpathlineto{\pgfqpoint{2.964364in}{3.722881in}}%
\pgfpathlineto{\pgfqpoint{3.124687in}{3.580403in}}%
\pgfpathlineto{\pgfqpoint{3.285010in}{3.434557in}}%
\pgfpathlineto{\pgfqpoint{3.445333in}{3.285297in}}%
\pgfpathlineto{\pgfqpoint{3.605657in}{3.132574in}}%
\pgfpathlineto{\pgfqpoint{3.926303in}{2.816543in}}%
\pgfpathlineto{\pgfqpoint{4.086626in}{2.653083in}}%
\pgfpathlineto{\pgfqpoint{4.246949in}{2.485685in}}%
\pgfpathlineto{\pgfqpoint{4.538814in}{2.170667in}}%
\pgfpathlineto{\pgfqpoint{4.768000in}{1.913246in}}%
\pgfpathlineto{\pgfqpoint{4.768000in}{1.913246in}}%
\pgfusepath{fill}%
\end{pgfscope}%
\begin{pgfscope}%
\pgfpathrectangle{\pgfqpoint{0.800000in}{0.528000in}}{\pgfqpoint{3.968000in}{3.696000in}}%
\pgfusepath{clip}%
\pgfsetbuttcap%
\pgfsetroundjoin%
\definecolor{currentfill}{rgb}{0.231674,0.318106,0.544834}%
\pgfsetfillcolor{currentfill}%
\pgfsetlinewidth{0.000000pt}%
\definecolor{currentstroke}{rgb}{0.000000,0.000000,0.000000}%
\pgfsetstrokecolor{currentstroke}%
\pgfsetdash{}{0pt}%
\pgfpathmoveto{\pgfqpoint{1.776928in}{0.528000in}}%
\pgfpathlineto{\pgfqpoint{1.481374in}{0.836423in}}%
\pgfpathlineto{\pgfqpoint{1.317460in}{1.013333in}}%
\pgfpathlineto{\pgfqpoint{1.182291in}{1.162667in}}%
\pgfpathlineto{\pgfqpoint{0.888175in}{1.498667in}}%
\pgfpathlineto{\pgfqpoint{0.800000in}{1.602633in}}%
\pgfpathlineto{\pgfqpoint{0.800000in}{1.597322in}}%
\pgfpathlineto{\pgfqpoint{0.947842in}{1.424000in}}%
\pgfpathlineto{\pgfqpoint{1.243119in}{1.090077in}}%
\pgfpathlineto{\pgfqpoint{1.280970in}{1.048328in}}%
\pgfpathlineto{\pgfqpoint{1.441293in}{0.874397in}}%
\pgfpathlineto{\pgfqpoint{1.735828in}{0.565333in}}%
\pgfpathlineto{\pgfqpoint{1.772254in}{0.528000in}}%
\pgfpathmoveto{\pgfqpoint{4.768000in}{1.923500in}}%
\pgfpathlineto{\pgfqpoint{4.479900in}{2.245333in}}%
\pgfpathlineto{\pgfqpoint{4.165073in}{2.581333in}}%
\pgfpathlineto{\pgfqpoint{3.846141in}{2.906317in}}%
\pgfpathlineto{\pgfqpoint{3.683366in}{3.066667in}}%
\pgfpathlineto{\pgfqpoint{3.365172in}{3.369578in}}%
\pgfpathlineto{\pgfqpoint{3.204848in}{3.517114in}}%
\pgfpathlineto{\pgfqpoint{3.041446in}{3.664000in}}%
\pgfpathlineto{\pgfqpoint{2.884202in}{3.802039in}}%
\pgfpathlineto{\pgfqpoint{2.723879in}{3.939510in}}%
\pgfpathlineto{\pgfqpoint{2.562396in}{4.074667in}}%
\pgfpathlineto{\pgfqpoint{2.378825in}{4.224000in}}%
\pgfpathlineto{\pgfqpoint{2.373096in}{4.224000in}}%
\pgfpathlineto{\pgfqpoint{2.646662in}{4.000000in}}%
\pgfpathlineto{\pgfqpoint{2.804040in}{3.866611in}}%
\pgfpathlineto{\pgfqpoint{2.964364in}{3.727475in}}%
\pgfpathlineto{\pgfqpoint{3.124687in}{3.585018in}}%
\pgfpathlineto{\pgfqpoint{3.285010in}{3.439194in}}%
\pgfpathlineto{\pgfqpoint{3.445333in}{3.289955in}}%
\pgfpathlineto{\pgfqpoint{3.605657in}{3.137254in}}%
\pgfpathlineto{\pgfqpoint{3.926303in}{2.821268in}}%
\pgfpathlineto{\pgfqpoint{4.088450in}{2.656000in}}%
\pgfpathlineto{\pgfqpoint{4.231646in}{2.506667in}}%
\pgfpathlineto{\pgfqpoint{4.527515in}{2.188121in}}%
\pgfpathlineto{\pgfqpoint{4.768000in}{1.918373in}}%
\pgfpathlineto{\pgfqpoint{4.768000in}{1.918373in}}%
\pgfusepath{fill}%
\end{pgfscope}%
\begin{pgfscope}%
\pgfpathrectangle{\pgfqpoint{0.800000in}{0.528000in}}{\pgfqpoint{3.968000in}{3.696000in}}%
\pgfusepath{clip}%
\pgfsetbuttcap%
\pgfsetroundjoin%
\definecolor{currentfill}{rgb}{0.231674,0.318106,0.544834}%
\pgfsetfillcolor{currentfill}%
\pgfsetlinewidth{0.000000pt}%
\definecolor{currentstroke}{rgb}{0.000000,0.000000,0.000000}%
\pgfsetstrokecolor{currentstroke}%
\pgfsetdash{}{0pt}%
\pgfpathmoveto{\pgfqpoint{1.772254in}{0.528000in}}%
\pgfpathlineto{\pgfqpoint{1.481374in}{0.831519in}}%
\pgfpathlineto{\pgfqpoint{1.321051in}{1.004480in}}%
\pgfpathlineto{\pgfqpoint{1.177771in}{1.162667in}}%
\pgfpathlineto{\pgfqpoint{0.883652in}{1.498667in}}%
\pgfpathlineto{\pgfqpoint{0.800000in}{1.597322in}}%
\pgfpathlineto{\pgfqpoint{0.800000in}{1.592011in}}%
\pgfpathlineto{\pgfqpoint{0.943367in}{1.424000in}}%
\pgfpathlineto{\pgfqpoint{1.240889in}{1.087479in}}%
\pgfpathlineto{\pgfqpoint{1.551841in}{0.752000in}}%
\pgfpathlineto{\pgfqpoint{1.694951in}{0.602667in}}%
\pgfpathlineto{\pgfqpoint{1.767581in}{0.528000in}}%
\pgfpathmoveto{\pgfqpoint{4.768000in}{1.928626in}}%
\pgfpathlineto{\pgfqpoint{4.484464in}{2.245333in}}%
\pgfpathlineto{\pgfqpoint{4.169666in}{2.581333in}}%
\pgfpathlineto{\pgfqpoint{4.046545in}{2.708554in}}%
\pgfpathlineto{\pgfqpoint{3.886222in}{2.870987in}}%
\pgfpathlineto{\pgfqpoint{3.725899in}{3.029813in}}%
\pgfpathlineto{\pgfqpoint{3.405253in}{3.336741in}}%
\pgfpathlineto{\pgfqpoint{3.244929in}{3.485096in}}%
\pgfpathlineto{\pgfqpoint{3.084606in}{3.630107in}}%
\pgfpathlineto{\pgfqpoint{2.919400in}{3.776000in}}%
\pgfpathlineto{\pgfqpoint{2.603636in}{4.045024in}}%
\pgfpathlineto{\pgfqpoint{2.443313in}{4.176639in}}%
\pgfpathlineto{\pgfqpoint{2.384554in}{4.224000in}}%
\pgfpathlineto{\pgfqpoint{2.378825in}{4.224000in}}%
\pgfpathlineto{\pgfqpoint{2.652089in}{4.000000in}}%
\pgfpathlineto{\pgfqpoint{2.804040in}{3.871184in}}%
\pgfpathlineto{\pgfqpoint{2.964364in}{3.732070in}}%
\pgfpathlineto{\pgfqpoint{3.125015in}{3.589333in}}%
\pgfpathlineto{\pgfqpoint{3.289094in}{3.440000in}}%
\pgfpathlineto{\pgfqpoint{3.449447in}{3.290667in}}%
\pgfpathlineto{\pgfqpoint{3.606270in}{3.141333in}}%
\pgfpathlineto{\pgfqpoint{3.926303in}{2.825993in}}%
\pgfpathlineto{\pgfqpoint{4.086626in}{2.662630in}}%
\pgfpathlineto{\pgfqpoint{4.376557in}{2.357333in}}%
\pgfpathlineto{\pgfqpoint{4.681811in}{2.021333in}}%
\pgfpathlineto{\pgfqpoint{4.768000in}{1.923500in}}%
\pgfpathlineto{\pgfqpoint{4.768000in}{1.923500in}}%
\pgfusepath{fill}%
\end{pgfscope}%
\begin{pgfscope}%
\pgfpathrectangle{\pgfqpoint{0.800000in}{0.528000in}}{\pgfqpoint{3.968000in}{3.696000in}}%
\pgfusepath{clip}%
\pgfsetbuttcap%
\pgfsetroundjoin%
\definecolor{currentfill}{rgb}{0.229739,0.322361,0.545706}%
\pgfsetfillcolor{currentfill}%
\pgfsetlinewidth{0.000000pt}%
\definecolor{currentstroke}{rgb}{0.000000,0.000000,0.000000}%
\pgfsetstrokecolor{currentstroke}%
\pgfsetdash{}{0pt}%
\pgfpathmoveto{\pgfqpoint{1.767581in}{0.528000in}}%
\pgfpathlineto{\pgfqpoint{1.480992in}{0.827022in}}%
\pgfpathlineto{\pgfqpoint{1.321051in}{0.999552in}}%
\pgfpathlineto{\pgfqpoint{1.173250in}{1.162667in}}%
\pgfpathlineto{\pgfqpoint{0.879148in}{1.498667in}}%
\pgfpathlineto{\pgfqpoint{0.800000in}{1.592011in}}%
\pgfpathlineto{\pgfqpoint{0.800000in}{1.586701in}}%
\pgfpathlineto{\pgfqpoint{0.920242in}{1.445611in}}%
\pgfpathlineto{\pgfqpoint{1.069196in}{1.274667in}}%
\pgfpathlineto{\pgfqpoint{1.361131in}{0.951031in}}%
\pgfpathlineto{\pgfqpoint{1.654295in}{0.640000in}}%
\pgfpathlineto{\pgfqpoint{1.762907in}{0.528000in}}%
\pgfpathmoveto{\pgfqpoint{4.768000in}{1.933753in}}%
\pgfpathlineto{\pgfqpoint{4.497941in}{2.235547in}}%
\pgfpathlineto{\pgfqpoint{4.367192in}{2.377155in}}%
\pgfpathlineto{\pgfqpoint{4.065991in}{2.693333in}}%
\pgfpathlineto{\pgfqpoint{3.919119in}{2.842667in}}%
\pgfpathlineto{\pgfqpoint{3.605657in}{3.151123in}}%
\pgfpathlineto{\pgfqpoint{3.285010in}{3.452880in}}%
\pgfpathlineto{\pgfqpoint{2.964364in}{3.741214in}}%
\pgfpathlineto{\pgfqpoint{2.804040in}{3.880331in}}%
\pgfpathlineto{\pgfqpoint{2.643717in}{4.016129in}}%
\pgfpathlineto{\pgfqpoint{2.482596in}{4.149333in}}%
\pgfpathlineto{\pgfqpoint{2.390284in}{4.224000in}}%
\pgfpathlineto{\pgfqpoint{2.384554in}{4.224000in}}%
\pgfpathlineto{\pgfqpoint{2.657516in}{4.000000in}}%
\pgfpathlineto{\pgfqpoint{2.804040in}{3.875758in}}%
\pgfpathlineto{\pgfqpoint{2.964364in}{3.736664in}}%
\pgfpathlineto{\pgfqpoint{3.130050in}{3.589333in}}%
\pgfpathlineto{\pgfqpoint{3.294038in}{3.440000in}}%
\pgfpathlineto{\pgfqpoint{3.454302in}{3.290667in}}%
\pgfpathlineto{\pgfqpoint{3.611039in}{3.141333in}}%
\pgfpathlineto{\pgfqpoint{3.926303in}{2.830718in}}%
\pgfpathlineto{\pgfqpoint{4.086626in}{2.667377in}}%
\pgfpathlineto{\pgfqpoint{4.381091in}{2.357333in}}%
\pgfpathlineto{\pgfqpoint{4.687838in}{2.019653in}}%
\pgfpathlineto{\pgfqpoint{4.768000in}{1.928626in}}%
\pgfpathlineto{\pgfqpoint{4.768000in}{1.928626in}}%
\pgfusepath{fill}%
\end{pgfscope}%
\begin{pgfscope}%
\pgfpathrectangle{\pgfqpoint{0.800000in}{0.528000in}}{\pgfqpoint{3.968000in}{3.696000in}}%
\pgfusepath{clip}%
\pgfsetbuttcap%
\pgfsetroundjoin%
\definecolor{currentfill}{rgb}{0.229739,0.322361,0.545706}%
\pgfsetfillcolor{currentfill}%
\pgfsetlinewidth{0.000000pt}%
\definecolor{currentstroke}{rgb}{0.000000,0.000000,0.000000}%
\pgfsetstrokecolor{currentstroke}%
\pgfsetdash{}{0pt}%
\pgfpathmoveto{\pgfqpoint{1.762907in}{0.528000in}}%
\pgfpathlineto{\pgfqpoint{1.476807in}{0.826667in}}%
\pgfpathlineto{\pgfqpoint{1.338161in}{0.976000in}}%
\pgfpathlineto{\pgfqpoint{1.040485in}{1.307250in}}%
\pgfpathlineto{\pgfqpoint{0.906730in}{1.461333in}}%
\pgfpathlineto{\pgfqpoint{0.800000in}{1.586701in}}%
\pgfpathlineto{\pgfqpoint{0.800000in}{1.581390in}}%
\pgfpathlineto{\pgfqpoint{0.920242in}{1.440425in}}%
\pgfpathlineto{\pgfqpoint{1.064731in}{1.274667in}}%
\pgfpathlineto{\pgfqpoint{1.361131in}{0.946108in}}%
\pgfpathlineto{\pgfqpoint{1.649681in}{0.640000in}}%
\pgfpathlineto{\pgfqpoint{1.758305in}{0.528000in}}%
\pgfpathlineto{\pgfqpoint{1.761939in}{0.528000in}}%
\pgfpathmoveto{\pgfqpoint{4.768000in}{1.938880in}}%
\pgfpathlineto{\pgfqpoint{4.527515in}{2.208086in}}%
\pgfpathlineto{\pgfqpoint{4.367192in}{2.382031in}}%
\pgfpathlineto{\pgfqpoint{4.070610in}{2.693333in}}%
\pgfpathlineto{\pgfqpoint{3.923818in}{2.842667in}}%
\pgfpathlineto{\pgfqpoint{3.605657in}{3.155722in}}%
\pgfpathlineto{\pgfqpoint{3.285010in}{3.457437in}}%
\pgfpathlineto{\pgfqpoint{2.964364in}{3.745731in}}%
\pgfpathlineto{\pgfqpoint{2.800422in}{3.888000in}}%
\pgfpathlineto{\pgfqpoint{2.483394in}{4.153227in}}%
\pgfpathlineto{\pgfqpoint{2.396013in}{4.224000in}}%
\pgfpathlineto{\pgfqpoint{2.390284in}{4.224000in}}%
\pgfpathlineto{\pgfqpoint{2.652751in}{4.008415in}}%
\pgfpathlineto{\pgfqpoint{2.723879in}{3.948636in}}%
\pgfpathlineto{\pgfqpoint{3.051677in}{3.664000in}}%
\pgfpathlineto{\pgfqpoint{3.217511in}{3.514667in}}%
\pgfpathlineto{\pgfqpoint{3.379522in}{3.365333in}}%
\pgfpathlineto{\pgfqpoint{3.537912in}{3.216000in}}%
\pgfpathlineto{\pgfqpoint{3.846141in}{2.915745in}}%
\pgfpathlineto{\pgfqpoint{4.166788in}{2.589095in}}%
\pgfpathlineto{\pgfqpoint{4.315889in}{2.432000in}}%
\pgfpathlineto{\pgfqpoint{4.607677in}{2.114398in}}%
\pgfpathlineto{\pgfqpoint{4.768000in}{1.933753in}}%
\pgfpathlineto{\pgfqpoint{4.768000in}{1.933753in}}%
\pgfusepath{fill}%
\end{pgfscope}%
\begin{pgfscope}%
\pgfpathrectangle{\pgfqpoint{0.800000in}{0.528000in}}{\pgfqpoint{3.968000in}{3.696000in}}%
\pgfusepath{clip}%
\pgfsetbuttcap%
\pgfsetroundjoin%
\definecolor{currentfill}{rgb}{0.229739,0.322361,0.545706}%
\pgfsetfillcolor{currentfill}%
\pgfsetlinewidth{0.000000pt}%
\definecolor{currentstroke}{rgb}{0.000000,0.000000,0.000000}%
\pgfsetstrokecolor{currentstroke}%
\pgfsetdash{}{0pt}%
\pgfpathmoveto{\pgfqpoint{1.758305in}{0.528000in}}%
\pgfpathlineto{\pgfqpoint{1.601616in}{0.690115in}}%
\pgfpathlineto{\pgfqpoint{1.299421in}{1.013333in}}%
\pgfpathlineto{\pgfqpoint{1.160727in}{1.166555in}}%
\pgfpathlineto{\pgfqpoint{0.999155in}{1.349333in}}%
\pgfpathlineto{\pgfqpoint{0.870270in}{1.498667in}}%
\pgfpathlineto{\pgfqpoint{0.800000in}{1.581390in}}%
\pgfpathlineto{\pgfqpoint{0.800000in}{1.576079in}}%
\pgfpathlineto{\pgfqpoint{0.920242in}{1.435239in}}%
\pgfpathlineto{\pgfqpoint{1.060267in}{1.274667in}}%
\pgfpathlineto{\pgfqpoint{1.363458in}{0.938667in}}%
\pgfpathlineto{\pgfqpoint{1.681778in}{0.601929in}}%
\pgfpathlineto{\pgfqpoint{1.753722in}{0.528000in}}%
\pgfpathmoveto{\pgfqpoint{4.768000in}{1.944007in}}%
\pgfpathlineto{\pgfqpoint{4.527515in}{2.212986in}}%
\pgfpathlineto{\pgfqpoint{4.367192in}{2.386906in}}%
\pgfpathlineto{\pgfqpoint{4.075229in}{2.693333in}}%
\pgfpathlineto{\pgfqpoint{3.926303in}{2.844854in}}%
\pgfpathlineto{\pgfqpoint{3.605657in}{3.160322in}}%
\pgfpathlineto{\pgfqpoint{3.285010in}{3.461995in}}%
\pgfpathlineto{\pgfqpoint{2.964364in}{3.750247in}}%
\pgfpathlineto{\pgfqpoint{2.804040in}{3.889453in}}%
\pgfpathlineto{\pgfqpoint{2.483394in}{4.157759in}}%
\pgfpathlineto{\pgfqpoint{2.401743in}{4.224000in}}%
\pgfpathlineto{\pgfqpoint{2.396013in}{4.224000in}}%
\pgfpathlineto{\pgfqpoint{2.643717in}{4.020681in}}%
\pgfpathlineto{\pgfqpoint{2.972388in}{3.738667in}}%
\pgfpathlineto{\pgfqpoint{3.124687in}{3.603239in}}%
\pgfpathlineto{\pgfqpoint{3.285010in}{3.457437in}}%
\pgfpathlineto{\pgfqpoint{3.445333in}{3.308281in}}%
\pgfpathlineto{\pgfqpoint{3.605657in}{3.155722in}}%
\pgfpathlineto{\pgfqpoint{3.765980in}{2.999715in}}%
\pgfpathlineto{\pgfqpoint{4.070610in}{2.693333in}}%
\pgfpathlineto{\pgfqpoint{4.214494in}{2.544000in}}%
\pgfpathlineto{\pgfqpoint{4.367192in}{2.382031in}}%
\pgfpathlineto{\pgfqpoint{4.527594in}{2.208000in}}%
\pgfpathlineto{\pgfqpoint{4.768000in}{1.938880in}}%
\pgfpathlineto{\pgfqpoint{4.768000in}{1.938880in}}%
\pgfusepath{fill}%
\end{pgfscope}%
\begin{pgfscope}%
\pgfpathrectangle{\pgfqpoint{0.800000in}{0.528000in}}{\pgfqpoint{3.968000in}{3.696000in}}%
\pgfusepath{clip}%
\pgfsetbuttcap%
\pgfsetroundjoin%
\definecolor{currentfill}{rgb}{0.229739,0.322361,0.545706}%
\pgfsetfillcolor{currentfill}%
\pgfsetlinewidth{0.000000pt}%
\definecolor{currentstroke}{rgb}{0.000000,0.000000,0.000000}%
\pgfsetstrokecolor{currentstroke}%
\pgfsetdash{}{0pt}%
\pgfpathmoveto{\pgfqpoint{1.753722in}{0.528000in}}%
\pgfpathlineto{\pgfqpoint{1.601616in}{0.685317in}}%
\pgfpathlineto{\pgfqpoint{1.294911in}{1.013333in}}%
\pgfpathlineto{\pgfqpoint{1.154617in}{1.168358in}}%
\pgfpathlineto{\pgfqpoint{1.000404in}{1.342829in}}%
\pgfpathlineto{\pgfqpoint{0.865832in}{1.498667in}}%
\pgfpathlineto{\pgfqpoint{0.800000in}{1.576079in}}%
\pgfpathlineto{\pgfqpoint{0.800000in}{1.570819in}}%
\pgfpathlineto{\pgfqpoint{1.088774in}{1.237333in}}%
\pgfpathlineto{\pgfqpoint{1.222526in}{1.088000in}}%
\pgfpathlineto{\pgfqpoint{1.361131in}{0.936307in}}%
\pgfpathlineto{\pgfqpoint{1.676520in}{0.602667in}}%
\pgfpathlineto{\pgfqpoint{1.749139in}{0.528000in}}%
\pgfpathmoveto{\pgfqpoint{4.768000in}{1.949086in}}%
\pgfpathlineto{\pgfqpoint{4.604144in}{2.133333in}}%
\pgfpathlineto{\pgfqpoint{4.294399in}{2.469333in}}%
\pgfpathlineto{\pgfqpoint{4.152066in}{2.618667in}}%
\pgfpathlineto{\pgfqpoint{4.006465in}{2.768439in}}%
\pgfpathlineto{\pgfqpoint{3.846141in}{2.929666in}}%
\pgfpathlineto{\pgfqpoint{3.525495in}{3.241607in}}%
\pgfpathlineto{\pgfqpoint{3.365172in}{3.392417in}}%
\pgfpathlineto{\pgfqpoint{3.204848in}{3.539848in}}%
\pgfpathlineto{\pgfqpoint{3.044525in}{3.683948in}}%
\pgfpathlineto{\pgfqpoint{2.884202in}{3.824761in}}%
\pgfpathlineto{\pgfqpoint{2.723475in}{3.962667in}}%
\pgfpathlineto{\pgfqpoint{2.563556in}{4.096440in}}%
\pgfpathlineto{\pgfqpoint{2.407372in}{4.224000in}}%
\pgfpathlineto{\pgfqpoint{2.401743in}{4.224000in}}%
\pgfpathlineto{\pgfqpoint{2.402537in}{4.223352in}}%
\pgfpathlineto{\pgfqpoint{2.447892in}{4.186667in}}%
\pgfpathlineto{\pgfqpoint{2.603636in}{4.058666in}}%
\pgfpathlineto{\pgfqpoint{2.934970in}{3.776000in}}%
\pgfpathlineto{\pgfqpoint{3.084606in}{3.643702in}}%
\pgfpathlineto{\pgfqpoint{3.405253in}{3.350460in}}%
\pgfpathlineto{\pgfqpoint{3.725899in}{3.043658in}}%
\pgfpathlineto{\pgfqpoint{3.891267in}{2.880000in}}%
\pgfpathlineto{\pgfqpoint{4.206869in}{2.556764in}}%
\pgfpathlineto{\pgfqpoint{4.497952in}{2.245333in}}%
\pgfpathlineto{\pgfqpoint{4.633153in}{2.096000in}}%
\pgfpathlineto{\pgfqpoint{4.768000in}{1.944007in}}%
\pgfpathlineto{\pgfqpoint{4.768000in}{1.946667in}}%
\pgfpathlineto{\pgfqpoint{4.768000in}{1.946667in}}%
\pgfusepath{fill}%
\end{pgfscope}%
\begin{pgfscope}%
\pgfpathrectangle{\pgfqpoint{0.800000in}{0.528000in}}{\pgfqpoint{3.968000in}{3.696000in}}%
\pgfusepath{clip}%
\pgfsetbuttcap%
\pgfsetroundjoin%
\definecolor{currentfill}{rgb}{0.227802,0.326594,0.546532}%
\pgfsetfillcolor{currentfill}%
\pgfsetlinewidth{0.000000pt}%
\definecolor{currentstroke}{rgb}{0.000000,0.000000,0.000000}%
\pgfsetstrokecolor{currentstroke}%
\pgfsetdash{}{0pt}%
\pgfpathmoveto{\pgfqpoint{1.749139in}{0.528000in}}%
\pgfpathlineto{\pgfqpoint{1.601616in}{0.680519in}}%
\pgfpathlineto{\pgfqpoint{1.290401in}{1.013333in}}%
\pgfpathlineto{\pgfqpoint{1.155272in}{1.162667in}}%
\pgfpathlineto{\pgfqpoint{1.022977in}{1.312000in}}%
\pgfpathlineto{\pgfqpoint{0.800000in}{1.570819in}}%
\pgfpathlineto{\pgfqpoint{0.800000in}{1.565613in}}%
\pgfpathlineto{\pgfqpoint{1.084291in}{1.237333in}}%
\pgfpathlineto{\pgfqpoint{1.218054in}{1.088000in}}%
\pgfpathlineto{\pgfqpoint{1.361131in}{0.931474in}}%
\pgfpathlineto{\pgfqpoint{1.671975in}{0.602667in}}%
\pgfpathlineto{\pgfqpoint{1.744555in}{0.528000in}}%
\pgfpathmoveto{\pgfqpoint{4.768000in}{1.954116in}}%
\pgfpathlineto{\pgfqpoint{4.607677in}{2.134394in}}%
\pgfpathlineto{\pgfqpoint{4.298903in}{2.469333in}}%
\pgfpathlineto{\pgfqpoint{4.156646in}{2.618667in}}%
\pgfpathlineto{\pgfqpoint{4.006465in}{2.773092in}}%
\pgfpathlineto{\pgfqpoint{3.846141in}{2.934297in}}%
\pgfpathlineto{\pgfqpoint{3.525495in}{3.246196in}}%
\pgfpathlineto{\pgfqpoint{3.365172in}{3.396984in}}%
\pgfpathlineto{\pgfqpoint{3.204848in}{3.544395in}}%
\pgfpathlineto{\pgfqpoint{3.044525in}{3.688475in}}%
\pgfpathlineto{\pgfqpoint{2.884202in}{3.829267in}}%
\pgfpathlineto{\pgfqpoint{2.723879in}{3.966817in}}%
\pgfpathlineto{\pgfqpoint{2.563556in}{4.100982in}}%
\pgfpathlineto{\pgfqpoint{2.412967in}{4.224000in}}%
\pgfpathlineto{\pgfqpoint{2.407372in}{4.224000in}}%
\pgfpathlineto{\pgfqpoint{2.563556in}{4.096440in}}%
\pgfpathlineto{\pgfqpoint{2.723879in}{3.962326in}}%
\pgfpathlineto{\pgfqpoint{3.044525in}{3.683948in}}%
\pgfpathlineto{\pgfqpoint{3.365172in}{3.392417in}}%
\pgfpathlineto{\pgfqpoint{3.685818in}{3.087373in}}%
\pgfpathlineto{\pgfqpoint{3.846141in}{2.929666in}}%
\pgfpathlineto{\pgfqpoint{4.152066in}{2.618667in}}%
\pgfpathlineto{\pgfqpoint{4.447354in}{2.305315in}}%
\pgfpathlineto{\pgfqpoint{4.737226in}{1.984000in}}%
\pgfpathlineto{\pgfqpoint{4.768000in}{1.949086in}}%
\pgfpathlineto{\pgfqpoint{4.768000in}{1.949086in}}%
\pgfusepath{fill}%
\end{pgfscope}%
\begin{pgfscope}%
\pgfpathrectangle{\pgfqpoint{0.800000in}{0.528000in}}{\pgfqpoint{3.968000in}{3.696000in}}%
\pgfusepath{clip}%
\pgfsetbuttcap%
\pgfsetroundjoin%
\definecolor{currentfill}{rgb}{0.227802,0.326594,0.546532}%
\pgfsetfillcolor{currentfill}%
\pgfsetlinewidth{0.000000pt}%
\definecolor{currentstroke}{rgb}{0.000000,0.000000,0.000000}%
\pgfsetstrokecolor{currentstroke}%
\pgfsetdash{}{0pt}%
\pgfpathmoveto{\pgfqpoint{1.744555in}{0.528000in}}%
\pgfpathlineto{\pgfqpoint{1.600102in}{0.677333in}}%
\pgfpathlineto{\pgfqpoint{1.285891in}{1.013333in}}%
\pgfpathlineto{\pgfqpoint{1.150836in}{1.162667in}}%
\pgfpathlineto{\pgfqpoint{1.018531in}{1.312000in}}%
\pgfpathlineto{\pgfqpoint{0.800000in}{1.565613in}}%
\pgfpathlineto{\pgfqpoint{0.800000in}{1.560407in}}%
\pgfpathlineto{\pgfqpoint{1.079822in}{1.237333in}}%
\pgfpathlineto{\pgfqpoint{1.361131in}{0.926641in}}%
\pgfpathlineto{\pgfqpoint{1.667430in}{0.602667in}}%
\pgfpathlineto{\pgfqpoint{1.739972in}{0.528000in}}%
\pgfpathmoveto{\pgfqpoint{4.768000in}{1.959145in}}%
\pgfpathlineto{\pgfqpoint{4.613055in}{2.133333in}}%
\pgfpathlineto{\pgfqpoint{4.477181in}{2.282667in}}%
\pgfpathlineto{\pgfqpoint{4.161226in}{2.618667in}}%
\pgfpathlineto{\pgfqpoint{4.006465in}{2.777744in}}%
\pgfpathlineto{\pgfqpoint{3.846141in}{2.938929in}}%
\pgfpathlineto{\pgfqpoint{3.522810in}{3.253333in}}%
\pgfpathlineto{\pgfqpoint{3.201478in}{3.552000in}}%
\pgfpathlineto{\pgfqpoint{2.884202in}{3.833773in}}%
\pgfpathlineto{\pgfqpoint{2.723879in}{3.971303in}}%
\pgfpathlineto{\pgfqpoint{2.563556in}{4.105525in}}%
\pgfpathlineto{\pgfqpoint{2.418561in}{4.224000in}}%
\pgfpathlineto{\pgfqpoint{2.412967in}{4.224000in}}%
\pgfpathlineto{\pgfqpoint{2.563556in}{4.100982in}}%
\pgfpathlineto{\pgfqpoint{2.728764in}{3.962667in}}%
\pgfpathlineto{\pgfqpoint{3.044525in}{3.688475in}}%
\pgfpathlineto{\pgfqpoint{3.365172in}{3.396984in}}%
\pgfpathlineto{\pgfqpoint{3.685818in}{3.091983in}}%
\pgfpathlineto{\pgfqpoint{3.846141in}{2.934297in}}%
\pgfpathlineto{\pgfqpoint{4.156646in}{2.618667in}}%
\pgfpathlineto{\pgfqpoint{4.447354in}{2.310202in}}%
\pgfpathlineto{\pgfqpoint{4.741659in}{1.984000in}}%
\pgfpathlineto{\pgfqpoint{4.768000in}{1.954116in}}%
\pgfpathlineto{\pgfqpoint{4.768000in}{1.954116in}}%
\pgfusepath{fill}%
\end{pgfscope}%
\begin{pgfscope}%
\pgfpathrectangle{\pgfqpoint{0.800000in}{0.528000in}}{\pgfqpoint{3.968000in}{3.696000in}}%
\pgfusepath{clip}%
\pgfsetbuttcap%
\pgfsetroundjoin%
\definecolor{currentfill}{rgb}{0.227802,0.326594,0.546532}%
\pgfsetfillcolor{currentfill}%
\pgfsetlinewidth{0.000000pt}%
\definecolor{currentstroke}{rgb}{0.000000,0.000000,0.000000}%
\pgfsetstrokecolor{currentstroke}%
\pgfsetdash{}{0pt}%
\pgfpathmoveto{\pgfqpoint{1.739972in}{0.528000in}}%
\pgfpathlineto{\pgfqpoint{1.595594in}{0.677333in}}%
\pgfpathlineto{\pgfqpoint{1.280970in}{1.013784in}}%
\pgfpathlineto{\pgfqpoint{1.120646in}{1.191457in}}%
\pgfpathlineto{\pgfqpoint{0.981441in}{1.349333in}}%
\pgfpathlineto{\pgfqpoint{0.852516in}{1.498667in}}%
\pgfpathlineto{\pgfqpoint{0.800000in}{1.560407in}}%
\pgfpathlineto{\pgfqpoint{0.800000in}{1.555200in}}%
\pgfpathlineto{\pgfqpoint{1.059788in}{1.255313in}}%
\pgfpathlineto{\pgfqpoint{1.120646in}{1.186498in}}%
\pgfpathlineto{\pgfqpoint{1.280970in}{1.008931in}}%
\pgfpathlineto{\pgfqpoint{1.441293in}{0.835638in}}%
\pgfpathlineto{\pgfqpoint{1.591087in}{0.677333in}}%
\pgfpathlineto{\pgfqpoint{1.735389in}{0.528000in}}%
\pgfpathmoveto{\pgfqpoint{4.768000in}{1.964174in}}%
\pgfpathlineto{\pgfqpoint{4.617478in}{2.133333in}}%
\pgfpathlineto{\pgfqpoint{4.481677in}{2.282667in}}%
\pgfpathlineto{\pgfqpoint{4.165806in}{2.618667in}}%
\pgfpathlineto{\pgfqpoint{4.020603in}{2.768000in}}%
\pgfpathlineto{\pgfqpoint{3.872458in}{2.917333in}}%
\pgfpathlineto{\pgfqpoint{3.565576in}{3.217134in}}%
\pgfpathlineto{\pgfqpoint{3.405253in}{3.368694in}}%
\pgfpathlineto{\pgfqpoint{3.082170in}{3.664000in}}%
\pgfpathlineto{\pgfqpoint{2.763960in}{3.941713in}}%
\pgfpathlineto{\pgfqpoint{2.443313in}{4.208499in}}%
\pgfpathlineto{\pgfqpoint{2.424156in}{4.224000in}}%
\pgfpathlineto{\pgfqpoint{2.418561in}{4.224000in}}%
\pgfpathlineto{\pgfqpoint{2.563556in}{4.105525in}}%
\pgfpathlineto{\pgfqpoint{2.723879in}{3.971303in}}%
\pgfpathlineto{\pgfqpoint{3.044525in}{3.693001in}}%
\pgfpathlineto{\pgfqpoint{3.365172in}{3.401552in}}%
\pgfpathlineto{\pgfqpoint{3.685818in}{3.096592in}}%
\pgfpathlineto{\pgfqpoint{3.846141in}{2.938929in}}%
\pgfpathlineto{\pgfqpoint{4.161226in}{2.618667in}}%
\pgfpathlineto{\pgfqpoint{4.447354in}{2.315090in}}%
\pgfpathlineto{\pgfqpoint{4.746092in}{1.984000in}}%
\pgfpathlineto{\pgfqpoint{4.768000in}{1.959145in}}%
\pgfpathlineto{\pgfqpoint{4.768000in}{1.959145in}}%
\pgfusepath{fill}%
\end{pgfscope}%
\begin{pgfscope}%
\pgfpathrectangle{\pgfqpoint{0.800000in}{0.528000in}}{\pgfqpoint{3.968000in}{3.696000in}}%
\pgfusepath{clip}%
\pgfsetbuttcap%
\pgfsetroundjoin%
\definecolor{currentfill}{rgb}{0.227802,0.326594,0.546532}%
\pgfsetfillcolor{currentfill}%
\pgfsetlinewidth{0.000000pt}%
\definecolor{currentstroke}{rgb}{0.000000,0.000000,0.000000}%
\pgfsetstrokecolor{currentstroke}%
\pgfsetdash{}{0pt}%
\pgfpathmoveto{\pgfqpoint{1.735389in}{0.528000in}}%
\pgfpathlineto{\pgfqpoint{1.591087in}{0.677333in}}%
\pgfpathlineto{\pgfqpoint{1.276948in}{1.013333in}}%
\pgfpathlineto{\pgfqpoint{1.141963in}{1.162667in}}%
\pgfpathlineto{\pgfqpoint{1.000404in}{1.322535in}}%
\pgfpathlineto{\pgfqpoint{0.848077in}{1.498667in}}%
\pgfpathlineto{\pgfqpoint{0.800000in}{1.555200in}}%
\pgfpathlineto{\pgfqpoint{0.800000in}{1.549994in}}%
\pgfpathlineto{\pgfqpoint{1.040485in}{1.271836in}}%
\pgfpathlineto{\pgfqpoint{1.341104in}{0.938667in}}%
\pgfpathlineto{\pgfqpoint{1.481374in}{0.788114in}}%
\pgfpathlineto{\pgfqpoint{1.641697in}{0.619903in}}%
\pgfpathlineto{\pgfqpoint{1.730806in}{0.528000in}}%
\pgfpathmoveto{\pgfqpoint{4.768000in}{1.969204in}}%
\pgfpathlineto{\pgfqpoint{4.621901in}{2.133333in}}%
\pgfpathlineto{\pgfqpoint{4.486173in}{2.282667in}}%
\pgfpathlineto{\pgfqpoint{4.166788in}{2.622340in}}%
\pgfpathlineto{\pgfqpoint{4.006465in}{2.787050in}}%
\pgfpathlineto{\pgfqpoint{3.839615in}{2.954667in}}%
\pgfpathlineto{\pgfqpoint{3.525495in}{3.259849in}}%
\pgfpathlineto{\pgfqpoint{3.204848in}{3.557934in}}%
\pgfpathlineto{\pgfqpoint{2.875099in}{3.850667in}}%
\pgfpathlineto{\pgfqpoint{2.563556in}{4.114565in}}%
\pgfpathlineto{\pgfqpoint{2.429750in}{4.224000in}}%
\pgfpathlineto{\pgfqpoint{2.424156in}{4.224000in}}%
\pgfpathlineto{\pgfqpoint{2.563556in}{4.110067in}}%
\pgfpathlineto{\pgfqpoint{2.723879in}{3.975789in}}%
\pgfpathlineto{\pgfqpoint{3.044525in}{3.697527in}}%
\pgfpathlineto{\pgfqpoint{3.368817in}{3.402667in}}%
\pgfpathlineto{\pgfqpoint{3.685818in}{3.101202in}}%
\pgfpathlineto{\pgfqpoint{3.846141in}{2.943560in}}%
\pgfpathlineto{\pgfqpoint{4.166788in}{2.617647in}}%
\pgfpathlineto{\pgfqpoint{4.327111in}{2.448951in}}%
\pgfpathlineto{\pgfqpoint{4.617478in}{2.133333in}}%
\pgfpathlineto{\pgfqpoint{4.768000in}{1.964174in}}%
\pgfpathlineto{\pgfqpoint{4.768000in}{1.964174in}}%
\pgfusepath{fill}%
\end{pgfscope}%
\begin{pgfscope}%
\pgfpathrectangle{\pgfqpoint{0.800000in}{0.528000in}}{\pgfqpoint{3.968000in}{3.696000in}}%
\pgfusepath{clip}%
\pgfsetbuttcap%
\pgfsetroundjoin%
\definecolor{currentfill}{rgb}{0.225863,0.330805,0.547314}%
\pgfsetfillcolor{currentfill}%
\pgfsetlinewidth{0.000000pt}%
\definecolor{currentstroke}{rgb}{0.000000,0.000000,0.000000}%
\pgfsetstrokecolor{currentstroke}%
\pgfsetdash{}{0pt}%
\pgfpathmoveto{\pgfqpoint{1.730806in}{0.528000in}}%
\pgfpathlineto{\pgfqpoint{1.586580in}{0.677333in}}%
\pgfpathlineto{\pgfqpoint{1.280970in}{1.004086in}}%
\pgfpathlineto{\pgfqpoint{1.005191in}{1.312000in}}%
\pgfpathlineto{\pgfqpoint{0.800000in}{1.549994in}}%
\pgfpathlineto{\pgfqpoint{0.800000in}{1.544788in}}%
\pgfpathlineto{\pgfqpoint{1.040485in}{1.266864in}}%
\pgfpathlineto{\pgfqpoint{1.336641in}{0.938667in}}%
\pgfpathlineto{\pgfqpoint{1.481374in}{0.783385in}}%
\pgfpathlineto{\pgfqpoint{1.641697in}{0.615196in}}%
\pgfpathlineto{\pgfqpoint{1.726222in}{0.528000in}}%
\pgfpathmoveto{\pgfqpoint{4.768000in}{1.974233in}}%
\pgfpathlineto{\pgfqpoint{4.626323in}{2.133333in}}%
\pgfpathlineto{\pgfqpoint{4.487434in}{2.286123in}}%
\pgfpathlineto{\pgfqpoint{4.174812in}{2.618667in}}%
\pgfpathlineto{\pgfqpoint{3.881753in}{2.917333in}}%
\pgfpathlineto{\pgfqpoint{3.565576in}{3.226165in}}%
\pgfpathlineto{\pgfqpoint{3.405253in}{3.377684in}}%
\pgfpathlineto{\pgfqpoint{3.084606in}{3.670775in}}%
\pgfpathlineto{\pgfqpoint{2.757185in}{3.956356in}}%
\pgfpathlineto{\pgfqpoint{2.683798in}{4.018626in}}%
\pgfpathlineto{\pgfqpoint{2.435345in}{4.224000in}}%
\pgfpathlineto{\pgfqpoint{2.429750in}{4.224000in}}%
\pgfpathlineto{\pgfqpoint{2.566649in}{4.112000in}}%
\pgfpathlineto{\pgfqpoint{2.723879in}{3.980274in}}%
\pgfpathlineto{\pgfqpoint{3.045321in}{3.701333in}}%
\pgfpathlineto{\pgfqpoint{3.373638in}{3.402667in}}%
\pgfpathlineto{\pgfqpoint{3.687648in}{3.104000in}}%
\pgfpathlineto{\pgfqpoint{3.846141in}{2.948191in}}%
\pgfpathlineto{\pgfqpoint{4.170318in}{2.618667in}}%
\pgfpathlineto{\pgfqpoint{4.312415in}{2.469333in}}%
\pgfpathlineto{\pgfqpoint{4.451763in}{2.320000in}}%
\pgfpathlineto{\pgfqpoint{4.754958in}{1.984000in}}%
\pgfpathlineto{\pgfqpoint{4.768000in}{1.969204in}}%
\pgfpathlineto{\pgfqpoint{4.768000in}{1.969204in}}%
\pgfusepath{fill}%
\end{pgfscope}%
\begin{pgfscope}%
\pgfpathrectangle{\pgfqpoint{0.800000in}{0.528000in}}{\pgfqpoint{3.968000in}{3.696000in}}%
\pgfusepath{clip}%
\pgfsetbuttcap%
\pgfsetroundjoin%
\definecolor{currentfill}{rgb}{0.225863,0.330805,0.547314}%
\pgfsetfillcolor{currentfill}%
\pgfsetlinewidth{0.000000pt}%
\definecolor{currentstroke}{rgb}{0.000000,0.000000,0.000000}%
\pgfsetstrokecolor{currentstroke}%
\pgfsetdash{}{0pt}%
\pgfpathmoveto{\pgfqpoint{1.726222in}{0.528000in}}%
\pgfpathlineto{\pgfqpoint{1.601616in}{0.656900in}}%
\pgfpathlineto{\pgfqpoint{1.440676in}{0.826667in}}%
\pgfpathlineto{\pgfqpoint{1.133091in}{1.162667in}}%
\pgfpathlineto{\pgfqpoint{1.000404in}{1.312388in}}%
\pgfpathlineto{\pgfqpoint{0.839216in}{1.498667in}}%
\pgfpathlineto{\pgfqpoint{0.800000in}{1.544788in}}%
\pgfpathlineto{\pgfqpoint{0.800000in}{1.539582in}}%
\pgfpathlineto{\pgfqpoint{1.040485in}{1.261892in}}%
\pgfpathlineto{\pgfqpoint{1.332179in}{0.938667in}}%
\pgfpathlineto{\pgfqpoint{1.481374in}{0.778656in}}%
\pgfpathlineto{\pgfqpoint{1.645705in}{0.606400in}}%
\pgfpathlineto{\pgfqpoint{1.721643in}{0.528000in}}%
\pgfpathlineto{\pgfqpoint{1.721859in}{0.528000in}}%
\pgfpathmoveto{\pgfqpoint{4.768000in}{1.979262in}}%
\pgfpathlineto{\pgfqpoint{4.630746in}{2.133333in}}%
\pgfpathlineto{\pgfqpoint{4.487434in}{2.290927in}}%
\pgfpathlineto{\pgfqpoint{4.179305in}{2.618667in}}%
\pgfpathlineto{\pgfqpoint{3.886222in}{2.917508in}}%
\pgfpathlineto{\pgfqpoint{3.565576in}{3.230680in}}%
\pgfpathlineto{\pgfqpoint{3.405253in}{3.382179in}}%
\pgfpathlineto{\pgfqpoint{3.084606in}{3.675230in}}%
\pgfpathlineto{\pgfqpoint{2.755163in}{3.962667in}}%
\pgfpathlineto{\pgfqpoint{2.443313in}{4.222079in}}%
\pgfpathlineto{\pgfqpoint{2.440939in}{4.224000in}}%
\pgfpathlineto{\pgfqpoint{2.435345in}{4.224000in}}%
\pgfpathlineto{\pgfqpoint{2.572033in}{4.112000in}}%
\pgfpathlineto{\pgfqpoint{2.723879in}{3.984760in}}%
\pgfpathlineto{\pgfqpoint{3.050319in}{3.701333in}}%
\pgfpathlineto{\pgfqpoint{3.365172in}{3.415042in}}%
\pgfpathlineto{\pgfqpoint{3.525495in}{3.264359in}}%
\pgfpathlineto{\pgfqpoint{3.692304in}{3.104000in}}%
\pgfpathlineto{\pgfqpoint{3.846141in}{2.952822in}}%
\pgfpathlineto{\pgfqpoint{4.170989in}{2.622579in}}%
\pgfpathlineto{\pgfqpoint{4.246949in}{2.543303in}}%
\pgfpathlineto{\pgfqpoint{4.558807in}{2.208000in}}%
\pgfpathlineto{\pgfqpoint{4.768000in}{1.974233in}}%
\pgfpathlineto{\pgfqpoint{4.768000in}{1.974233in}}%
\pgfusepath{fill}%
\end{pgfscope}%
\begin{pgfscope}%
\pgfpathrectangle{\pgfqpoint{0.800000in}{0.528000in}}{\pgfqpoint{3.968000in}{3.696000in}}%
\pgfusepath{clip}%
\pgfsetbuttcap%
\pgfsetroundjoin%
\definecolor{currentfill}{rgb}{0.225863,0.330805,0.547314}%
\pgfsetfillcolor{currentfill}%
\pgfsetlinewidth{0.000000pt}%
\definecolor{currentstroke}{rgb}{0.000000,0.000000,0.000000}%
\pgfsetstrokecolor{currentstroke}%
\pgfsetdash{}{0pt}%
\pgfpathmoveto{\pgfqpoint{1.721643in}{0.528000in}}%
\pgfpathlineto{\pgfqpoint{1.577566in}{0.677333in}}%
\pgfpathlineto{\pgfqpoint{1.280970in}{0.994397in}}%
\pgfpathlineto{\pgfqpoint{0.996373in}{1.312000in}}%
\pgfpathlineto{\pgfqpoint{0.866867in}{1.461333in}}%
\pgfpathlineto{\pgfqpoint{0.800000in}{1.539582in}}%
\pgfpathlineto{\pgfqpoint{0.800000in}{1.534407in}}%
\pgfpathlineto{\pgfqpoint{0.960323in}{1.348181in}}%
\pgfpathlineto{\pgfqpoint{1.124219in}{1.162667in}}%
\pgfpathlineto{\pgfqpoint{1.431808in}{0.826667in}}%
\pgfpathlineto{\pgfqpoint{1.717146in}{0.528000in}}%
\pgfpathmoveto{\pgfqpoint{4.768000in}{1.984286in}}%
\pgfpathlineto{\pgfqpoint{4.465055in}{2.320000in}}%
\pgfpathlineto{\pgfqpoint{4.317371in}{2.478405in}}%
\pgfpathlineto{\pgfqpoint{4.166788in}{2.636363in}}%
\pgfpathlineto{\pgfqpoint{4.002200in}{2.805333in}}%
\pgfpathlineto{\pgfqpoint{3.685818in}{3.119374in}}%
\pgfpathlineto{\pgfqpoint{3.525495in}{3.273380in}}%
\pgfpathlineto{\pgfqpoint{3.204848in}{3.571345in}}%
\pgfpathlineto{\pgfqpoint{2.884202in}{3.856208in}}%
\pgfpathlineto{\pgfqpoint{2.563556in}{4.127962in}}%
\pgfpathlineto{\pgfqpoint{2.446460in}{4.224000in}}%
\pgfpathlineto{\pgfqpoint{2.440939in}{4.224000in}}%
\pgfpathlineto{\pgfqpoint{2.563556in}{4.123497in}}%
\pgfpathlineto{\pgfqpoint{2.885479in}{3.850667in}}%
\pgfpathlineto{\pgfqpoint{3.055318in}{3.701333in}}%
\pgfpathlineto{\pgfqpoint{3.365172in}{3.419532in}}%
\pgfpathlineto{\pgfqpoint{3.525495in}{3.268870in}}%
\pgfpathlineto{\pgfqpoint{3.685818in}{3.114843in}}%
\pgfpathlineto{\pgfqpoint{3.997611in}{2.805333in}}%
\pgfpathlineto{\pgfqpoint{4.143322in}{2.656000in}}%
\pgfpathlineto{\pgfqpoint{4.287030in}{2.505801in}}%
\pgfpathlineto{\pgfqpoint{4.447354in}{2.334373in}}%
\pgfpathlineto{\pgfqpoint{4.597083in}{2.170667in}}%
\pgfpathlineto{\pgfqpoint{4.768000in}{1.979262in}}%
\pgfpathlineto{\pgfqpoint{4.768000in}{1.984000in}}%
\pgfpathlineto{\pgfqpoint{4.768000in}{1.984000in}}%
\pgfusepath{fill}%
\end{pgfscope}%
\begin{pgfscope}%
\pgfpathrectangle{\pgfqpoint{0.800000in}{0.528000in}}{\pgfqpoint{3.968000in}{3.696000in}}%
\pgfusepath{clip}%
\pgfsetbuttcap%
\pgfsetroundjoin%
\definecolor{currentfill}{rgb}{0.223925,0.334994,0.548053}%
\pgfsetfillcolor{currentfill}%
\pgfsetlinewidth{0.000000pt}%
\definecolor{currentstroke}{rgb}{0.000000,0.000000,0.000000}%
\pgfsetstrokecolor{currentstroke}%
\pgfsetdash{}{0pt}%
\pgfpathmoveto{\pgfqpoint{1.717146in}{0.528000in}}%
\pgfpathlineto{\pgfqpoint{1.561535in}{0.689395in}}%
\pgfpathlineto{\pgfqpoint{1.396910in}{0.864000in}}%
\pgfpathlineto{\pgfqpoint{1.259244in}{1.013333in}}%
\pgfpathlineto{\pgfqpoint{1.120646in}{1.166660in}}%
\pgfpathlineto{\pgfqpoint{0.959318in}{1.349333in}}%
\pgfpathlineto{\pgfqpoint{0.800000in}{1.534407in}}%
\pgfpathlineto{\pgfqpoint{0.800000in}{1.529301in}}%
\pgfpathlineto{\pgfqpoint{0.954971in}{1.349333in}}%
\pgfpathlineto{\pgfqpoint{1.086527in}{1.200000in}}%
\pgfpathlineto{\pgfqpoint{1.220835in}{1.050667in}}%
\pgfpathlineto{\pgfqpoint{1.521455in}{0.726822in}}%
\pgfpathlineto{\pgfqpoint{1.678938in}{0.562688in}}%
\pgfpathlineto{\pgfqpoint{1.712650in}{0.528000in}}%
\pgfpathmoveto{\pgfqpoint{4.768000in}{1.989222in}}%
\pgfpathlineto{\pgfqpoint{4.469485in}{2.320000in}}%
\pgfpathlineto{\pgfqpoint{4.327111in}{2.472796in}}%
\pgfpathlineto{\pgfqpoint{4.166788in}{2.641038in}}%
\pgfpathlineto{\pgfqpoint{4.006465in}{2.805656in}}%
\pgfpathlineto{\pgfqpoint{3.685818in}{3.123905in}}%
\pgfpathlineto{\pgfqpoint{3.525495in}{3.277891in}}%
\pgfpathlineto{\pgfqpoint{3.204848in}{3.575815in}}%
\pgfpathlineto{\pgfqpoint{2.884202in}{3.860639in}}%
\pgfpathlineto{\pgfqpoint{2.563556in}{4.132428in}}%
\pgfpathlineto{\pgfqpoint{2.451926in}{4.224000in}}%
\pgfpathlineto{\pgfqpoint{2.446460in}{4.224000in}}%
\pgfpathlineto{\pgfqpoint{2.763960in}{3.959676in}}%
\pgfpathlineto{\pgfqpoint{2.924283in}{3.821305in}}%
\pgfpathlineto{\pgfqpoint{3.244929in}{3.534823in}}%
\pgfpathlineto{\pgfqpoint{3.405253in}{3.386675in}}%
\pgfpathlineto{\pgfqpoint{3.565576in}{3.235196in}}%
\pgfpathlineto{\pgfqpoint{3.890955in}{2.917333in}}%
\pgfpathlineto{\pgfqpoint{4.206869in}{2.594638in}}%
\pgfpathlineto{\pgfqpoint{4.360955in}{2.432000in}}%
\pgfpathlineto{\pgfqpoint{4.499434in}{2.282667in}}%
\pgfpathlineto{\pgfqpoint{4.768000in}{1.984286in}}%
\pgfpathlineto{\pgfqpoint{4.768000in}{1.984286in}}%
\pgfusepath{fill}%
\end{pgfscope}%
\begin{pgfscope}%
\pgfpathrectangle{\pgfqpoint{0.800000in}{0.528000in}}{\pgfqpoint{3.968000in}{3.696000in}}%
\pgfusepath{clip}%
\pgfsetbuttcap%
\pgfsetroundjoin%
\definecolor{currentfill}{rgb}{0.223925,0.334994,0.548053}%
\pgfsetfillcolor{currentfill}%
\pgfsetlinewidth{0.000000pt}%
\definecolor{currentstroke}{rgb}{0.000000,0.000000,0.000000}%
\pgfsetstrokecolor{currentstroke}%
\pgfsetdash{}{0pt}%
\pgfpathmoveto{\pgfqpoint{1.712650in}{0.528000in}}%
\pgfpathlineto{\pgfqpoint{1.561535in}{0.684677in}}%
\pgfpathlineto{\pgfqpoint{1.401212in}{0.854640in}}%
\pgfpathlineto{\pgfqpoint{1.119798in}{1.162667in}}%
\pgfpathlineto{\pgfqpoint{0.987644in}{1.312000in}}%
\pgfpathlineto{\pgfqpoint{0.858118in}{1.461333in}}%
\pgfpathlineto{\pgfqpoint{0.800000in}{1.529301in}}%
\pgfpathlineto{\pgfqpoint{0.800000in}{1.524195in}}%
\pgfpathlineto{\pgfqpoint{0.950623in}{1.349333in}}%
\pgfpathlineto{\pgfqpoint{1.089635in}{1.191552in}}%
\pgfpathlineto{\pgfqpoint{1.240889in}{1.023749in}}%
\pgfpathlineto{\pgfqpoint{1.388078in}{0.864000in}}%
\pgfpathlineto{\pgfqpoint{1.681778in}{0.555096in}}%
\pgfpathlineto{\pgfqpoint{1.708153in}{0.528000in}}%
\pgfpathmoveto{\pgfqpoint{4.768000in}{1.994157in}}%
\pgfpathlineto{\pgfqpoint{4.473916in}{2.320000in}}%
\pgfpathlineto{\pgfqpoint{4.327111in}{2.477492in}}%
\pgfpathlineto{\pgfqpoint{4.166788in}{2.645712in}}%
\pgfpathlineto{\pgfqpoint{4.006465in}{2.810228in}}%
\pgfpathlineto{\pgfqpoint{3.685818in}{3.128436in}}%
\pgfpathlineto{\pgfqpoint{3.525495in}{3.282401in}}%
\pgfpathlineto{\pgfqpoint{3.204848in}{3.580285in}}%
\pgfpathlineto{\pgfqpoint{2.884202in}{3.865069in}}%
\pgfpathlineto{\pgfqpoint{2.563556in}{4.136894in}}%
\pgfpathlineto{\pgfqpoint{2.457392in}{4.224000in}}%
\pgfpathlineto{\pgfqpoint{2.451926in}{4.224000in}}%
\pgfpathlineto{\pgfqpoint{2.763960in}{3.964142in}}%
\pgfpathlineto{\pgfqpoint{2.924283in}{3.825740in}}%
\pgfpathlineto{\pgfqpoint{3.244929in}{3.539298in}}%
\pgfpathlineto{\pgfqpoint{3.405253in}{3.391170in}}%
\pgfpathlineto{\pgfqpoint{3.565576in}{3.239712in}}%
\pgfpathlineto{\pgfqpoint{3.886222in}{2.926621in}}%
\pgfpathlineto{\pgfqpoint{4.046545in}{2.764842in}}%
\pgfpathlineto{\pgfqpoint{4.206869in}{2.599319in}}%
\pgfpathlineto{\pgfqpoint{4.367192in}{2.430130in}}%
\pgfpathlineto{\pgfqpoint{4.673067in}{2.096000in}}%
\pgfpathlineto{\pgfqpoint{4.768000in}{1.989222in}}%
\pgfpathlineto{\pgfqpoint{4.768000in}{1.989222in}}%
\pgfusepath{fill}%
\end{pgfscope}%
\begin{pgfscope}%
\pgfpathrectangle{\pgfqpoint{0.800000in}{0.528000in}}{\pgfqpoint{3.968000in}{3.696000in}}%
\pgfusepath{clip}%
\pgfsetbuttcap%
\pgfsetroundjoin%
\definecolor{currentfill}{rgb}{0.223925,0.334994,0.548053}%
\pgfsetfillcolor{currentfill}%
\pgfsetlinewidth{0.000000pt}%
\definecolor{currentstroke}{rgb}{0.000000,0.000000,0.000000}%
\pgfsetstrokecolor{currentstroke}%
\pgfsetdash{}{0pt}%
\pgfpathmoveto{\pgfqpoint{1.708153in}{0.528000in}}%
\pgfpathlineto{\pgfqpoint{1.561535in}{0.679959in}}%
\pgfpathlineto{\pgfqpoint{1.401212in}{0.849900in}}%
\pgfpathlineto{\pgfqpoint{1.115444in}{1.162667in}}%
\pgfpathlineto{\pgfqpoint{0.983279in}{1.312000in}}%
\pgfpathlineto{\pgfqpoint{0.853743in}{1.461333in}}%
\pgfpathlineto{\pgfqpoint{0.800000in}{1.524195in}}%
\pgfpathlineto{\pgfqpoint{0.800000in}{1.519090in}}%
\pgfpathlineto{\pgfqpoint{0.946276in}{1.349333in}}%
\pgfpathlineto{\pgfqpoint{1.080566in}{1.196829in}}%
\pgfpathlineto{\pgfqpoint{1.245967in}{1.013333in}}%
\pgfpathlineto{\pgfqpoint{1.561535in}{0.675278in}}%
\pgfpathlineto{\pgfqpoint{1.703657in}{0.528000in}}%
\pgfpathmoveto{\pgfqpoint{4.768000in}{1.999093in}}%
\pgfpathlineto{\pgfqpoint{4.478346in}{2.320000in}}%
\pgfpathlineto{\pgfqpoint{4.327111in}{2.482189in}}%
\pgfpathlineto{\pgfqpoint{4.161370in}{2.656000in}}%
\pgfpathlineto{\pgfqpoint{3.846141in}{2.975612in}}%
\pgfpathlineto{\pgfqpoint{3.521534in}{3.290667in}}%
\pgfpathlineto{\pgfqpoint{3.199799in}{3.589333in}}%
\pgfpathlineto{\pgfqpoint{3.033327in}{3.738667in}}%
\pgfpathlineto{\pgfqpoint{2.884202in}{3.869500in}}%
\pgfpathlineto{\pgfqpoint{2.553893in}{4.149333in}}%
\pgfpathlineto{\pgfqpoint{2.462857in}{4.224000in}}%
\pgfpathlineto{\pgfqpoint{2.457392in}{4.224000in}}%
\pgfpathlineto{\pgfqpoint{2.746173in}{3.983433in}}%
\pgfpathlineto{\pgfqpoint{2.814420in}{3.925333in}}%
\pgfpathlineto{\pgfqpoint{3.124687in}{3.652695in}}%
\pgfpathlineto{\pgfqpoint{3.445333in}{3.358120in}}%
\pgfpathlineto{\pgfqpoint{3.605657in}{3.205842in}}%
\pgfpathlineto{\pgfqpoint{3.926303in}{2.891079in}}%
\pgfpathlineto{\pgfqpoint{4.086626in}{2.728462in}}%
\pgfpathlineto{\pgfqpoint{4.407273in}{2.391961in}}%
\pgfpathlineto{\pgfqpoint{4.710780in}{2.058667in}}%
\pgfpathlineto{\pgfqpoint{4.768000in}{1.994157in}}%
\pgfpathlineto{\pgfqpoint{4.768000in}{1.994157in}}%
\pgfusepath{fill}%
\end{pgfscope}%
\begin{pgfscope}%
\pgfpathrectangle{\pgfqpoint{0.800000in}{0.528000in}}{\pgfqpoint{3.968000in}{3.696000in}}%
\pgfusepath{clip}%
\pgfsetbuttcap%
\pgfsetroundjoin%
\definecolor{currentfill}{rgb}{0.223925,0.334994,0.548053}%
\pgfsetfillcolor{currentfill}%
\pgfsetlinewidth{0.000000pt}%
\definecolor{currentstroke}{rgb}{0.000000,0.000000,0.000000}%
\pgfsetstrokecolor{currentstroke}%
\pgfsetdash{}{0pt}%
\pgfpathmoveto{\pgfqpoint{1.703657in}{0.528000in}}%
\pgfpathlineto{\pgfqpoint{1.559574in}{0.677333in}}%
\pgfpathlineto{\pgfqpoint{1.418508in}{0.826667in}}%
\pgfpathlineto{\pgfqpoint{1.120646in}{1.151982in}}%
\pgfpathlineto{\pgfqpoint{0.863191in}{1.445526in}}%
\pgfpathlineto{\pgfqpoint{0.800000in}{1.519090in}}%
\pgfpathlineto{\pgfqpoint{0.800000in}{1.513984in}}%
\pgfpathlineto{\pgfqpoint{0.941929in}{1.349333in}}%
\pgfpathlineto{\pgfqpoint{1.080566in}{1.191955in}}%
\pgfpathlineto{\pgfqpoint{1.241541in}{1.013333in}}%
\pgfpathlineto{\pgfqpoint{1.555150in}{0.677333in}}%
\pgfpathlineto{\pgfqpoint{1.699160in}{0.528000in}}%
\pgfpathmoveto{\pgfqpoint{4.768000in}{2.004028in}}%
\pgfpathlineto{\pgfqpoint{4.482777in}{2.320000in}}%
\pgfpathlineto{\pgfqpoint{4.343630in}{2.469333in}}%
\pgfpathlineto{\pgfqpoint{4.201772in}{2.618667in}}%
\pgfpathlineto{\pgfqpoint{3.886222in}{2.940292in}}%
\pgfpathlineto{\pgfqpoint{3.561958in}{3.256703in}}%
\pgfpathlineto{\pgfqpoint{3.405253in}{3.404623in}}%
\pgfpathlineto{\pgfqpoint{3.244929in}{3.552711in}}%
\pgfpathlineto{\pgfqpoint{2.924283in}{3.839047in}}%
\pgfpathlineto{\pgfqpoint{2.763960in}{3.977389in}}%
\pgfpathlineto{\pgfqpoint{2.468323in}{4.224000in}}%
\pgfpathlineto{\pgfqpoint{2.462857in}{4.224000in}}%
\pgfpathlineto{\pgfqpoint{2.732196in}{4.000000in}}%
\pgfpathlineto{\pgfqpoint{3.044525in}{3.728743in}}%
\pgfpathlineto{\pgfqpoint{3.204848in}{3.584755in}}%
\pgfpathlineto{\pgfqpoint{3.525495in}{3.286912in}}%
\pgfpathlineto{\pgfqpoint{3.685818in}{3.132967in}}%
\pgfpathlineto{\pgfqpoint{4.006465in}{2.814800in}}%
\pgfpathlineto{\pgfqpoint{4.166788in}{2.650387in}}%
\pgfpathlineto{\pgfqpoint{4.478346in}{2.320000in}}%
\pgfpathlineto{\pgfqpoint{4.768000in}{1.999093in}}%
\pgfpathlineto{\pgfqpoint{4.768000in}{1.999093in}}%
\pgfusepath{fill}%
\end{pgfscope}%
\begin{pgfscope}%
\pgfpathrectangle{\pgfqpoint{0.800000in}{0.528000in}}{\pgfqpoint{3.968000in}{3.696000in}}%
\pgfusepath{clip}%
\pgfsetbuttcap%
\pgfsetroundjoin%
\definecolor{currentfill}{rgb}{0.221989,0.339161,0.548752}%
\pgfsetfillcolor{currentfill}%
\pgfsetlinewidth{0.000000pt}%
\definecolor{currentstroke}{rgb}{0.000000,0.000000,0.000000}%
\pgfsetstrokecolor{currentstroke}%
\pgfsetdash{}{0pt}%
\pgfpathmoveto{\pgfqpoint{1.699160in}{0.528000in}}%
\pgfpathlineto{\pgfqpoint{1.555150in}{0.677333in}}%
\pgfpathlineto{\pgfqpoint{1.401212in}{0.840419in}}%
\pgfpathlineto{\pgfqpoint{1.106734in}{1.162667in}}%
\pgfpathlineto{\pgfqpoint{0.974549in}{1.312000in}}%
\pgfpathlineto{\pgfqpoint{0.840081in}{1.467059in}}%
\pgfpathlineto{\pgfqpoint{0.800000in}{1.513984in}}%
\pgfpathlineto{\pgfqpoint{0.800000in}{1.508878in}}%
\pgfpathlineto{\pgfqpoint{0.937581in}{1.349333in}}%
\pgfpathlineto{\pgfqpoint{1.080566in}{1.187081in}}%
\pgfpathlineto{\pgfqpoint{1.240889in}{1.009272in}}%
\pgfpathlineto{\pgfqpoint{1.550727in}{0.677333in}}%
\pgfpathlineto{\pgfqpoint{1.694663in}{0.528000in}}%
\pgfpathmoveto{\pgfqpoint{4.768000in}{2.008964in}}%
\pgfpathlineto{\pgfqpoint{4.485812in}{2.321511in}}%
\pgfpathlineto{\pgfqpoint{4.327111in}{2.491582in}}%
\pgfpathlineto{\pgfqpoint{4.166788in}{2.659671in}}%
\pgfpathlineto{\pgfqpoint{3.842321in}{2.988441in}}%
\pgfpathlineto{\pgfqpoint{3.763054in}{3.066667in}}%
\pgfpathlineto{\pgfqpoint{3.445333in}{3.371516in}}%
\pgfpathlineto{\pgfqpoint{3.285010in}{3.520399in}}%
\pgfpathlineto{\pgfqpoint{3.124687in}{3.666042in}}%
\pgfpathlineto{\pgfqpoint{2.958736in}{3.813333in}}%
\pgfpathlineto{\pgfqpoint{2.643717in}{4.083473in}}%
\pgfpathlineto{\pgfqpoint{2.473789in}{4.224000in}}%
\pgfpathlineto{\pgfqpoint{2.468323in}{4.224000in}}%
\pgfpathlineto{\pgfqpoint{2.737386in}{4.000000in}}%
\pgfpathlineto{\pgfqpoint{3.044525in}{3.733193in}}%
\pgfpathlineto{\pgfqpoint{3.204848in}{3.589225in}}%
\pgfpathlineto{\pgfqpoint{3.526276in}{3.290667in}}%
\pgfpathlineto{\pgfqpoint{3.685818in}{3.137498in}}%
\pgfpathlineto{\pgfqpoint{4.006465in}{2.819373in}}%
\pgfpathlineto{\pgfqpoint{4.166788in}{2.655062in}}%
\pgfpathlineto{\pgfqpoint{4.487434in}{2.314949in}}%
\pgfpathlineto{\pgfqpoint{4.652767in}{2.133333in}}%
\pgfpathlineto{\pgfqpoint{4.768000in}{2.004028in}}%
\pgfpathlineto{\pgfqpoint{4.768000in}{2.004028in}}%
\pgfusepath{fill}%
\end{pgfscope}%
\begin{pgfscope}%
\pgfpathrectangle{\pgfqpoint{0.800000in}{0.528000in}}{\pgfqpoint{3.968000in}{3.696000in}}%
\pgfusepath{clip}%
\pgfsetbuttcap%
\pgfsetroundjoin%
\definecolor{currentfill}{rgb}{0.221989,0.339161,0.548752}%
\pgfsetfillcolor{currentfill}%
\pgfsetlinewidth{0.000000pt}%
\definecolor{currentstroke}{rgb}{0.000000,0.000000,0.000000}%
\pgfsetstrokecolor{currentstroke}%
\pgfsetdash{}{0pt}%
\pgfpathmoveto{\pgfqpoint{1.694663in}{0.528000in}}%
\pgfpathlineto{\pgfqpoint{1.550727in}{0.677333in}}%
\pgfpathlineto{\pgfqpoint{1.401212in}{0.835678in}}%
\pgfpathlineto{\pgfqpoint{1.102379in}{1.162667in}}%
\pgfpathlineto{\pgfqpoint{0.960323in}{1.323260in}}%
\pgfpathlineto{\pgfqpoint{0.808715in}{1.498667in}}%
\pgfpathlineto{\pgfqpoint{0.800000in}{1.508878in}}%
\pgfpathlineto{\pgfqpoint{0.800000in}{1.503773in}}%
\pgfpathlineto{\pgfqpoint{0.933234in}{1.349333in}}%
\pgfpathlineto{\pgfqpoint{1.064731in}{1.200000in}}%
\pgfpathlineto{\pgfqpoint{1.361131in}{0.873980in}}%
\pgfpathlineto{\pgfqpoint{1.510764in}{0.714667in}}%
\pgfpathlineto{\pgfqpoint{1.653941in}{0.565333in}}%
\pgfpathlineto{\pgfqpoint{1.690167in}{0.528000in}}%
\pgfpathmoveto{\pgfqpoint{4.768000in}{2.013899in}}%
\pgfpathlineto{\pgfqpoint{4.516402in}{2.293018in}}%
\pgfpathlineto{\pgfqpoint{4.367192in}{2.453674in}}%
\pgfpathlineto{\pgfqpoint{4.065920in}{2.768000in}}%
\pgfpathlineto{\pgfqpoint{3.765980in}{3.068314in}}%
\pgfpathlineto{\pgfqpoint{3.445333in}{3.375941in}}%
\pgfpathlineto{\pgfqpoint{3.285010in}{3.524805in}}%
\pgfpathlineto{\pgfqpoint{3.124687in}{3.670428in}}%
\pgfpathlineto{\pgfqpoint{2.963805in}{3.813333in}}%
\pgfpathlineto{\pgfqpoint{2.643717in}{4.087875in}}%
\pgfpathlineto{\pgfqpoint{2.479255in}{4.224000in}}%
\pgfpathlineto{\pgfqpoint{2.473789in}{4.224000in}}%
\pgfpathlineto{\pgfqpoint{2.742575in}{4.000000in}}%
\pgfpathlineto{\pgfqpoint{3.044525in}{3.737643in}}%
\pgfpathlineto{\pgfqpoint{3.209562in}{3.589333in}}%
\pgfpathlineto{\pgfqpoint{3.530941in}{3.290667in}}%
\pgfpathlineto{\pgfqpoint{3.686522in}{3.141333in}}%
\pgfpathlineto{\pgfqpoint{4.006465in}{2.823945in}}%
\pgfpathlineto{\pgfqpoint{4.170327in}{2.656000in}}%
\pgfpathlineto{\pgfqpoint{4.487434in}{2.319754in}}%
\pgfpathlineto{\pgfqpoint{4.647758in}{2.143730in}}%
\pgfpathlineto{\pgfqpoint{4.768000in}{2.008964in}}%
\pgfpathlineto{\pgfqpoint{4.768000in}{2.008964in}}%
\pgfusepath{fill}%
\end{pgfscope}%
\begin{pgfscope}%
\pgfpathrectangle{\pgfqpoint{0.800000in}{0.528000in}}{\pgfqpoint{3.968000in}{3.696000in}}%
\pgfusepath{clip}%
\pgfsetbuttcap%
\pgfsetroundjoin%
\definecolor{currentfill}{rgb}{0.221989,0.339161,0.548752}%
\pgfsetfillcolor{currentfill}%
\pgfsetlinewidth{0.000000pt}%
\definecolor{currentstroke}{rgb}{0.000000,0.000000,0.000000}%
\pgfsetstrokecolor{currentstroke}%
\pgfsetdash{}{0pt}%
\pgfpathmoveto{\pgfqpoint{1.690167in}{0.528000in}}%
\pgfpathlineto{\pgfqpoint{1.546304in}{0.677333in}}%
\pgfpathlineto{\pgfqpoint{1.401212in}{0.830938in}}%
\pgfpathlineto{\pgfqpoint{1.098024in}{1.162667in}}%
\pgfpathlineto{\pgfqpoint{0.960323in}{1.318276in}}%
\pgfpathlineto{\pgfqpoint{0.804357in}{1.498667in}}%
\pgfpathlineto{\pgfqpoint{0.800000in}{1.503773in}}%
\pgfpathlineto{\pgfqpoint{0.800000in}{1.498667in}}%
\pgfpathlineto{\pgfqpoint{1.093669in}{1.162667in}}%
\pgfpathlineto{\pgfqpoint{1.401212in}{0.826205in}}%
\pgfpathlineto{\pgfqpoint{1.685670in}{0.528000in}}%
\pgfpathmoveto{\pgfqpoint{4.768000in}{2.018835in}}%
\pgfpathlineto{\pgfqpoint{4.527515in}{2.285671in}}%
\pgfpathlineto{\pgfqpoint{4.367192in}{2.458376in}}%
\pgfpathlineto{\pgfqpoint{4.070403in}{2.768000in}}%
\pgfpathlineto{\pgfqpoint{3.765980in}{3.072779in}}%
\pgfpathlineto{\pgfqpoint{3.445333in}{3.380366in}}%
\pgfpathlineto{\pgfqpoint{3.285010in}{3.529210in}}%
\pgfpathlineto{\pgfqpoint{3.124687in}{3.674814in}}%
\pgfpathlineto{\pgfqpoint{2.964364in}{3.817219in}}%
\pgfpathlineto{\pgfqpoint{2.643717in}{4.092276in}}%
\pgfpathlineto{\pgfqpoint{2.483394in}{4.224000in}}%
\pgfpathlineto{\pgfqpoint{2.479255in}{4.224000in}}%
\pgfpathlineto{\pgfqpoint{2.723879in}{4.020302in}}%
\pgfpathlineto{\pgfqpoint{2.884202in}{3.882791in}}%
\pgfpathlineto{\pgfqpoint{3.214393in}{3.589333in}}%
\pgfpathlineto{\pgfqpoint{3.535605in}{3.290667in}}%
\pgfpathlineto{\pgfqpoint{3.691108in}{3.141333in}}%
\pgfpathlineto{\pgfqpoint{4.006465in}{2.828518in}}%
\pgfpathlineto{\pgfqpoint{4.174755in}{2.656000in}}%
\pgfpathlineto{\pgfqpoint{4.491561in}{2.320000in}}%
\pgfpathlineto{\pgfqpoint{4.627793in}{2.170667in}}%
\pgfpathlineto{\pgfqpoint{4.768000in}{2.013899in}}%
\pgfpathlineto{\pgfqpoint{4.768000in}{2.013899in}}%
\pgfusepath{fill}%
\end{pgfscope}%
\begin{pgfscope}%
\pgfpathrectangle{\pgfqpoint{0.800000in}{0.528000in}}{\pgfqpoint{3.968000in}{3.696000in}}%
\pgfusepath{clip}%
\pgfsetbuttcap%
\pgfsetroundjoin%
\definecolor{currentfill}{rgb}{0.221989,0.339161,0.548752}%
\pgfsetfillcolor{currentfill}%
\pgfsetlinewidth{0.000000pt}%
\definecolor{currentstroke}{rgb}{0.000000,0.000000,0.000000}%
\pgfsetstrokecolor{currentstroke}%
\pgfsetdash{}{0pt}%
\pgfpathmoveto{\pgfqpoint{1.685670in}{0.528000in}}%
\pgfpathlineto{\pgfqpoint{1.551209in}{0.667715in}}%
\pgfpathlineto{\pgfqpoint{1.471003in}{0.752000in}}%
\pgfpathlineto{\pgfqpoint{1.160341in}{1.088360in}}%
\pgfpathlineto{\pgfqpoint{0.994278in}{1.274667in}}%
\pgfpathlineto{\pgfqpoint{0.800000in}{1.498667in}}%
\pgfpathlineto{\pgfqpoint{0.800000in}{1.498667in}}%
\pgfpathlineto{\pgfqpoint{0.800000in}{1.493658in}}%
\pgfpathlineto{\pgfqpoint{1.089314in}{1.162667in}}%
\pgfpathlineto{\pgfqpoint{1.396429in}{0.826667in}}%
\pgfpathlineto{\pgfqpoint{1.537457in}{0.677333in}}%
\pgfpathlineto{\pgfqpoint{1.681778in}{0.528000in}}%
\pgfpathmoveto{\pgfqpoint{4.768000in}{2.023726in}}%
\pgfpathlineto{\pgfqpoint{4.465761in}{2.357333in}}%
\pgfpathlineto{\pgfqpoint{4.319189in}{2.514046in}}%
\pgfpathlineto{\pgfqpoint{4.166788in}{2.673452in}}%
\pgfpathlineto{\pgfqpoint{3.849372in}{2.995009in}}%
\pgfpathlineto{\pgfqpoint{3.776752in}{3.066667in}}%
\pgfpathlineto{\pgfqpoint{3.645737in}{3.194135in}}%
\pgfpathlineto{\pgfqpoint{3.325091in}{3.496715in}}%
\pgfpathlineto{\pgfqpoint{3.004444in}{3.786287in}}%
\pgfpathlineto{\pgfqpoint{2.844121in}{3.926301in}}%
\pgfpathlineto{\pgfqpoint{2.683798in}{4.062998in}}%
\pgfpathlineto{\pgfqpoint{2.523475in}{4.196544in}}%
\pgfpathlineto{\pgfqpoint{2.490034in}{4.224000in}}%
\pgfpathlineto{\pgfqpoint{2.484691in}{4.224000in}}%
\pgfpathlineto{\pgfqpoint{2.643717in}{4.092276in}}%
\pgfpathlineto{\pgfqpoint{2.804040in}{3.956363in}}%
\pgfpathlineto{\pgfqpoint{2.968779in}{3.813333in}}%
\pgfpathlineto{\pgfqpoint{3.136703in}{3.664000in}}%
\pgfpathlineto{\pgfqpoint{3.445333in}{3.380366in}}%
\pgfpathlineto{\pgfqpoint{3.772205in}{3.066667in}}%
\pgfpathlineto{\pgfqpoint{3.926303in}{2.913889in}}%
\pgfpathlineto{\pgfqpoint{4.086626in}{2.751415in}}%
\pgfpathlineto{\pgfqpoint{4.391887in}{2.432000in}}%
\pgfpathlineto{\pgfqpoint{4.687838in}{2.108792in}}%
\pgfpathlineto{\pgfqpoint{4.768000in}{2.018835in}}%
\pgfpathlineto{\pgfqpoint{4.768000in}{2.021333in}}%
\pgfpathlineto{\pgfqpoint{4.768000in}{2.021333in}}%
\pgfusepath{fill}%
\end{pgfscope}%
\begin{pgfscope}%
\pgfpathrectangle{\pgfqpoint{0.800000in}{0.528000in}}{\pgfqpoint{3.968000in}{3.696000in}}%
\pgfusepath{clip}%
\pgfsetbuttcap%
\pgfsetroundjoin%
\definecolor{currentfill}{rgb}{0.220057,0.343307,0.549413}%
\pgfsetfillcolor{currentfill}%
\pgfsetlinewidth{0.000000pt}%
\definecolor{currentstroke}{rgb}{0.000000,0.000000,0.000000}%
\pgfsetstrokecolor{currentstroke}%
\pgfsetdash{}{0pt}%
\pgfpathmoveto{\pgfqpoint{1.681185in}{0.528000in}}%
\pgfpathlineto{\pgfqpoint{1.361131in}{0.864487in}}%
\pgfpathlineto{\pgfqpoint{1.056056in}{1.200000in}}%
\pgfpathlineto{\pgfqpoint{0.800000in}{1.493658in}}%
\pgfpathlineto{\pgfqpoint{0.800000in}{1.488649in}}%
\pgfpathlineto{\pgfqpoint{1.084960in}{1.162667in}}%
\pgfpathlineto{\pgfqpoint{1.392076in}{0.826667in}}%
\pgfpathlineto{\pgfqpoint{1.533034in}{0.677333in}}%
\pgfpathlineto{\pgfqpoint{1.676772in}{0.528000in}}%
\pgfpathmoveto{\pgfqpoint{4.768000in}{2.028571in}}%
\pgfpathlineto{\pgfqpoint{4.470128in}{2.357333in}}%
\pgfpathlineto{\pgfqpoint{4.327111in}{2.510303in}}%
\pgfpathlineto{\pgfqpoint{4.166788in}{2.678045in}}%
\pgfpathlineto{\pgfqpoint{3.856962in}{2.992000in}}%
\pgfpathlineto{\pgfqpoint{3.565576in}{3.275459in}}%
\pgfpathlineto{\pgfqpoint{3.405253in}{3.426723in}}%
\pgfpathlineto{\pgfqpoint{3.244929in}{3.574715in}}%
\pgfpathlineto{\pgfqpoint{2.924283in}{3.861050in}}%
\pgfpathlineto{\pgfqpoint{2.603636in}{4.134557in}}%
\pgfpathlineto{\pgfqpoint{2.495377in}{4.224000in}}%
\pgfpathlineto{\pgfqpoint{2.490034in}{4.224000in}}%
\pgfpathlineto{\pgfqpoint{2.643717in}{4.096677in}}%
\pgfpathlineto{\pgfqpoint{2.804040in}{3.960784in}}%
\pgfpathlineto{\pgfqpoint{2.964364in}{3.821586in}}%
\pgfpathlineto{\pgfqpoint{3.124687in}{3.679200in}}%
\pgfpathlineto{\pgfqpoint{3.445333in}{3.384791in}}%
\pgfpathlineto{\pgfqpoint{3.765980in}{3.077243in}}%
\pgfpathlineto{\pgfqpoint{3.927399in}{2.917333in}}%
\pgfpathlineto{\pgfqpoint{4.086626in}{2.755998in}}%
\pgfpathlineto{\pgfqpoint{4.396289in}{2.432000in}}%
\pgfpathlineto{\pgfqpoint{4.687838in}{2.113626in}}%
\pgfpathlineto{\pgfqpoint{4.768000in}{2.023726in}}%
\pgfpathlineto{\pgfqpoint{4.768000in}{2.023726in}}%
\pgfusepath{fill}%
\end{pgfscope}%
\begin{pgfscope}%
\pgfpathrectangle{\pgfqpoint{0.800000in}{0.528000in}}{\pgfqpoint{3.968000in}{3.696000in}}%
\pgfusepath{clip}%
\pgfsetbuttcap%
\pgfsetroundjoin%
\definecolor{currentfill}{rgb}{0.220057,0.343307,0.549413}%
\pgfsetfillcolor{currentfill}%
\pgfsetlinewidth{0.000000pt}%
\definecolor{currentstroke}{rgb}{0.000000,0.000000,0.000000}%
\pgfsetstrokecolor{currentstroke}%
\pgfsetdash{}{0pt}%
\pgfpathmoveto{\pgfqpoint{1.676772in}{0.528000in}}%
\pgfpathlineto{\pgfqpoint{1.357241in}{0.864000in}}%
\pgfpathlineto{\pgfqpoint{1.051719in}{1.200000in}}%
\pgfpathlineto{\pgfqpoint{0.800000in}{1.488649in}}%
\pgfpathlineto{\pgfqpoint{0.800000in}{1.483639in}}%
\pgfpathlineto{\pgfqpoint{1.080605in}{1.162667in}}%
\pgfpathlineto{\pgfqpoint{1.387724in}{0.826667in}}%
\pgfpathlineto{\pgfqpoint{1.528610in}{0.677333in}}%
\pgfpathlineto{\pgfqpoint{1.672359in}{0.528000in}}%
\pgfpathmoveto{\pgfqpoint{4.768000in}{2.033416in}}%
\pgfpathlineto{\pgfqpoint{4.474495in}{2.357333in}}%
\pgfpathlineto{\pgfqpoint{4.327111in}{2.514918in}}%
\pgfpathlineto{\pgfqpoint{4.161302in}{2.688223in}}%
\pgfpathlineto{\pgfqpoint{4.083852in}{2.768000in}}%
\pgfpathlineto{\pgfqpoint{3.765980in}{3.086172in}}%
\pgfpathlineto{\pgfqpoint{3.445333in}{3.393641in}}%
\pgfpathlineto{\pgfqpoint{3.285010in}{3.542427in}}%
\pgfpathlineto{\pgfqpoint{3.124687in}{3.687973in}}%
\pgfpathlineto{\pgfqpoint{2.964364in}{3.830320in}}%
\pgfpathlineto{\pgfqpoint{2.635917in}{4.112000in}}%
\pgfpathlineto{\pgfqpoint{2.500720in}{4.224000in}}%
\pgfpathlineto{\pgfqpoint{2.495377in}{4.224000in}}%
\pgfpathlineto{\pgfqpoint{2.643717in}{4.101078in}}%
\pgfpathlineto{\pgfqpoint{2.806943in}{3.962667in}}%
\pgfpathlineto{\pgfqpoint{2.978705in}{3.813333in}}%
\pgfpathlineto{\pgfqpoint{3.124687in}{3.683587in}}%
\pgfpathlineto{\pgfqpoint{3.445333in}{3.389216in}}%
\pgfpathlineto{\pgfqpoint{3.765980in}{3.081708in}}%
\pgfpathlineto{\pgfqpoint{3.931871in}{2.917333in}}%
\pgfpathlineto{\pgfqpoint{4.086626in}{2.760581in}}%
\pgfpathlineto{\pgfqpoint{4.403728in}{2.428698in}}%
\pgfpathlineto{\pgfqpoint{4.470128in}{2.357333in}}%
\pgfpathlineto{\pgfqpoint{4.607677in}{2.207382in}}%
\pgfpathlineto{\pgfqpoint{4.768000in}{2.028571in}}%
\pgfpathlineto{\pgfqpoint{4.768000in}{2.028571in}}%
\pgfusepath{fill}%
\end{pgfscope}%
\begin{pgfscope}%
\pgfpathrectangle{\pgfqpoint{0.800000in}{0.528000in}}{\pgfqpoint{3.968000in}{3.696000in}}%
\pgfusepath{clip}%
\pgfsetbuttcap%
\pgfsetroundjoin%
\definecolor{currentfill}{rgb}{0.220057,0.343307,0.549413}%
\pgfsetfillcolor{currentfill}%
\pgfsetlinewidth{0.000000pt}%
\definecolor{currentstroke}{rgb}{0.000000,0.000000,0.000000}%
\pgfsetstrokecolor{currentstroke}%
\pgfsetdash{}{0pt}%
\pgfpathmoveto{\pgfqpoint{1.672359in}{0.528000in}}%
\pgfpathlineto{\pgfqpoint{1.356757in}{0.859926in}}%
\pgfpathlineto{\pgfqpoint{1.280970in}{0.941735in}}%
\pgfpathlineto{\pgfqpoint{0.981369in}{1.274667in}}%
\pgfpathlineto{\pgfqpoint{0.800000in}{1.483639in}}%
\pgfpathlineto{\pgfqpoint{0.800000in}{1.478630in}}%
\pgfpathlineto{\pgfqpoint{1.060208in}{1.181038in}}%
\pgfpathlineto{\pgfqpoint{1.120646in}{1.113259in}}%
\pgfpathlineto{\pgfqpoint{1.280970in}{0.937007in}}%
\pgfpathlineto{\pgfqpoint{1.595703in}{0.602667in}}%
\pgfpathlineto{\pgfqpoint{1.667945in}{0.528000in}}%
\pgfpathmoveto{\pgfqpoint{4.768000in}{2.038261in}}%
\pgfpathlineto{\pgfqpoint{4.478862in}{2.357333in}}%
\pgfpathlineto{\pgfqpoint{4.327111in}{2.519533in}}%
\pgfpathlineto{\pgfqpoint{4.160882in}{2.693333in}}%
\pgfpathlineto{\pgfqpoint{4.006465in}{2.851232in}}%
\pgfpathlineto{\pgfqpoint{3.680552in}{3.173762in}}%
\pgfpathlineto{\pgfqpoint{3.598010in}{3.253333in}}%
\pgfpathlineto{\pgfqpoint{3.279370in}{3.552000in}}%
\pgfpathlineto{\pgfqpoint{3.114670in}{3.701333in}}%
\pgfpathlineto{\pgfqpoint{2.804040in}{3.973859in}}%
\pgfpathlineto{\pgfqpoint{2.641182in}{4.112000in}}%
\pgfpathlineto{\pgfqpoint{2.506063in}{4.224000in}}%
\pgfpathlineto{\pgfqpoint{2.500720in}{4.224000in}}%
\pgfpathlineto{\pgfqpoint{2.643717in}{4.105480in}}%
\pgfpathlineto{\pgfqpoint{2.807830in}{3.966197in}}%
\pgfpathlineto{\pgfqpoint{2.884202in}{3.900307in}}%
\pgfpathlineto{\pgfqpoint{3.044525in}{3.759544in}}%
\pgfpathlineto{\pgfqpoint{3.365172in}{3.468442in}}%
\pgfpathlineto{\pgfqpoint{3.525495in}{3.318019in}}%
\pgfpathlineto{\pgfqpoint{3.685818in}{3.164290in}}%
\pgfpathlineto{\pgfqpoint{4.010481in}{2.842667in}}%
\pgfpathlineto{\pgfqpoint{4.166788in}{2.682638in}}%
\pgfpathlineto{\pgfqpoint{4.334902in}{2.506667in}}%
\pgfpathlineto{\pgfqpoint{4.487434in}{2.343352in}}%
\pgfpathlineto{\pgfqpoint{4.768000in}{2.033416in}}%
\pgfpathlineto{\pgfqpoint{4.768000in}{2.033416in}}%
\pgfusepath{fill}%
\end{pgfscope}%
\begin{pgfscope}%
\pgfpathrectangle{\pgfqpoint{0.800000in}{0.528000in}}{\pgfqpoint{3.968000in}{3.696000in}}%
\pgfusepath{clip}%
\pgfsetbuttcap%
\pgfsetroundjoin%
\definecolor{currentfill}{rgb}{0.220057,0.343307,0.549413}%
\pgfsetfillcolor{currentfill}%
\pgfsetlinewidth{0.000000pt}%
\definecolor{currentstroke}{rgb}{0.000000,0.000000,0.000000}%
\pgfsetstrokecolor{currentstroke}%
\pgfsetdash{}{0pt}%
\pgfpathmoveto{\pgfqpoint{1.667945in}{0.528000in}}%
\pgfpathlineto{\pgfqpoint{1.361131in}{0.850491in}}%
\pgfpathlineto{\pgfqpoint{1.076327in}{1.162667in}}%
\pgfpathlineto{\pgfqpoint{0.800000in}{1.478630in}}%
\pgfpathlineto{\pgfqpoint{0.800000in}{1.473621in}}%
\pgfpathlineto{\pgfqpoint{1.057855in}{1.178846in}}%
\pgfpathlineto{\pgfqpoint{1.120646in}{1.108479in}}%
\pgfpathlineto{\pgfqpoint{1.280970in}{0.932333in}}%
\pgfpathlineto{\pgfqpoint{1.591325in}{0.602667in}}%
\pgfpathlineto{\pgfqpoint{1.663532in}{0.528000in}}%
\pgfpathmoveto{\pgfqpoint{4.768000in}{2.043106in}}%
\pgfpathlineto{\pgfqpoint{4.483229in}{2.357333in}}%
\pgfpathlineto{\pgfqpoint{4.343616in}{2.506667in}}%
\pgfpathlineto{\pgfqpoint{4.046545in}{2.815082in}}%
\pgfpathlineto{\pgfqpoint{3.886222in}{2.976372in}}%
\pgfpathlineto{\pgfqpoint{3.718621in}{3.141333in}}%
\pgfpathlineto{\pgfqpoint{3.405235in}{3.440000in}}%
\pgfpathlineto{\pgfqpoint{3.243375in}{3.589333in}}%
\pgfpathlineto{\pgfqpoint{2.916530in}{3.880778in}}%
\pgfpathlineto{\pgfqpoint{2.844121in}{3.943712in}}%
\pgfpathlineto{\pgfqpoint{2.683798in}{4.080525in}}%
\pgfpathlineto{\pgfqpoint{2.523475in}{4.214091in}}%
\pgfpathlineto{\pgfqpoint{2.511406in}{4.224000in}}%
\pgfpathlineto{\pgfqpoint{2.506063in}{4.224000in}}%
\pgfpathlineto{\pgfqpoint{2.643717in}{4.109881in}}%
\pgfpathlineto{\pgfqpoint{2.804040in}{3.973859in}}%
\pgfpathlineto{\pgfqpoint{3.124687in}{3.692359in}}%
\pgfpathlineto{\pgfqpoint{3.445333in}{3.398066in}}%
\pgfpathlineto{\pgfqpoint{3.765980in}{3.090637in}}%
\pgfpathlineto{\pgfqpoint{3.926303in}{2.931886in}}%
\pgfpathlineto{\pgfqpoint{4.232737in}{2.618667in}}%
\pgfpathlineto{\pgfqpoint{4.374438in}{2.469333in}}%
\pgfpathlineto{\pgfqpoint{4.527515in}{2.304569in}}%
\pgfpathlineto{\pgfqpoint{4.768000in}{2.038261in}}%
\pgfpathlineto{\pgfqpoint{4.768000in}{2.038261in}}%
\pgfusepath{fill}%
\end{pgfscope}%
\begin{pgfscope}%
\pgfpathrectangle{\pgfqpoint{0.800000in}{0.528000in}}{\pgfqpoint{3.968000in}{3.696000in}}%
\pgfusepath{clip}%
\pgfsetbuttcap%
\pgfsetroundjoin%
\definecolor{currentfill}{rgb}{0.218130,0.347432,0.550038}%
\pgfsetfillcolor{currentfill}%
\pgfsetlinewidth{0.000000pt}%
\definecolor{currentstroke}{rgb}{0.000000,0.000000,0.000000}%
\pgfsetstrokecolor{currentstroke}%
\pgfsetdash{}{0pt}%
\pgfpathmoveto{\pgfqpoint{1.663532in}{0.528000in}}%
\pgfpathlineto{\pgfqpoint{1.361131in}{0.845829in}}%
\pgfpathlineto{\pgfqpoint{1.072051in}{1.162667in}}%
\pgfpathlineto{\pgfqpoint{0.800000in}{1.473621in}}%
\pgfpathlineto{\pgfqpoint{0.800000in}{1.468612in}}%
\pgfpathlineto{\pgfqpoint{1.040485in}{1.193244in}}%
\pgfpathlineto{\pgfqpoint{1.339901in}{0.864000in}}%
\pgfpathlineto{\pgfqpoint{1.641697in}{0.545944in}}%
\pgfpathlineto{\pgfqpoint{1.659119in}{0.528000in}}%
\pgfpathmoveto{\pgfqpoint{4.768000in}{2.047951in}}%
\pgfpathlineto{\pgfqpoint{4.487434in}{2.357505in}}%
\pgfpathlineto{\pgfqpoint{4.327111in}{2.528762in}}%
\pgfpathlineto{\pgfqpoint{4.166788in}{2.696366in}}%
\pgfpathlineto{\pgfqpoint{4.006465in}{2.860221in}}%
\pgfpathlineto{\pgfqpoint{3.684772in}{3.178667in}}%
\pgfpathlineto{\pgfqpoint{3.365172in}{3.481616in}}%
\pgfpathlineto{\pgfqpoint{3.204848in}{3.628755in}}%
\pgfpathlineto{\pgfqpoint{3.040770in}{3.776000in}}%
\pgfpathlineto{\pgfqpoint{2.723879in}{4.050951in}}%
\pgfpathlineto{\pgfqpoint{2.516749in}{4.224000in}}%
\pgfpathlineto{\pgfqpoint{2.511406in}{4.224000in}}%
\pgfpathlineto{\pgfqpoint{2.646388in}{4.112000in}}%
\pgfpathlineto{\pgfqpoint{2.812646in}{3.970682in}}%
\pgfpathlineto{\pgfqpoint{2.884202in}{3.909023in}}%
\pgfpathlineto{\pgfqpoint{3.044525in}{3.768297in}}%
\pgfpathlineto{\pgfqpoint{3.365172in}{3.477272in}}%
\pgfpathlineto{\pgfqpoint{3.525495in}{3.326889in}}%
\pgfpathlineto{\pgfqpoint{3.685818in}{3.173200in}}%
\pgfpathlineto{\pgfqpoint{4.006465in}{2.855726in}}%
\pgfpathlineto{\pgfqpoint{4.166788in}{2.691825in}}%
\pgfpathlineto{\pgfqpoint{4.327111in}{2.524147in}}%
\pgfpathlineto{\pgfqpoint{4.620019in}{2.208000in}}%
\pgfpathlineto{\pgfqpoint{4.768000in}{2.043106in}}%
\pgfpathlineto{\pgfqpoint{4.768000in}{2.043106in}}%
\pgfusepath{fill}%
\end{pgfscope}%
\begin{pgfscope}%
\pgfpathrectangle{\pgfqpoint{0.800000in}{0.528000in}}{\pgfqpoint{3.968000in}{3.696000in}}%
\pgfusepath{clip}%
\pgfsetbuttcap%
\pgfsetroundjoin%
\definecolor{currentfill}{rgb}{0.218130,0.347432,0.550038}%
\pgfsetfillcolor{currentfill}%
\pgfsetlinewidth{0.000000pt}%
\definecolor{currentstroke}{rgb}{0.000000,0.000000,0.000000}%
\pgfsetstrokecolor{currentstroke}%
\pgfsetdash{}{0pt}%
\pgfpathmoveto{\pgfqpoint{1.659119in}{0.528000in}}%
\pgfpathlineto{\pgfqpoint{1.361131in}{0.841166in}}%
\pgfpathlineto{\pgfqpoint{1.067774in}{1.162667in}}%
\pgfpathlineto{\pgfqpoint{0.800000in}{1.468612in}}%
\pgfpathlineto{\pgfqpoint{0.801091in}{1.462349in}}%
\pgfpathlineto{\pgfqpoint{0.840081in}{1.417054in}}%
\pgfpathlineto{\pgfqpoint{1.000404in}{1.233578in}}%
\pgfpathlineto{\pgfqpoint{1.164211in}{1.050667in}}%
\pgfpathlineto{\pgfqpoint{1.300975in}{0.901333in}}%
\pgfpathlineto{\pgfqpoint{1.441293in}{0.750954in}}%
\pgfpathlineto{\pgfqpoint{1.601616in}{0.582868in}}%
\pgfpathlineto{\pgfqpoint{1.654706in}{0.528000in}}%
\pgfpathmoveto{\pgfqpoint{4.768000in}{2.052797in}}%
\pgfpathlineto{\pgfqpoint{4.491881in}{2.357333in}}%
\pgfpathlineto{\pgfqpoint{4.352330in}{2.506667in}}%
\pgfpathlineto{\pgfqpoint{4.046545in}{2.824082in}}%
\pgfpathlineto{\pgfqpoint{3.879509in}{2.992000in}}%
\pgfpathlineto{\pgfqpoint{3.565576in}{3.297542in}}%
\pgfpathlineto{\pgfqpoint{3.405253in}{3.448679in}}%
\pgfpathlineto{\pgfqpoint{3.244929in}{3.596598in}}%
\pgfpathlineto{\pgfqpoint{2.918383in}{3.888000in}}%
\pgfpathlineto{\pgfqpoint{2.763960in}{4.021194in}}%
\pgfpathlineto{\pgfqpoint{2.603636in}{4.156421in}}%
\pgfpathlineto{\pgfqpoint{2.522092in}{4.224000in}}%
\pgfpathlineto{\pgfqpoint{2.516749in}{4.224000in}}%
\pgfpathlineto{\pgfqpoint{2.651539in}{4.112000in}}%
\pgfpathlineto{\pgfqpoint{2.804040in}{3.982555in}}%
\pgfpathlineto{\pgfqpoint{3.124687in}{3.701131in}}%
\pgfpathlineto{\pgfqpoint{3.449783in}{3.402667in}}%
\pgfpathlineto{\pgfqpoint{3.765980in}{3.099566in}}%
\pgfpathlineto{\pgfqpoint{3.926303in}{2.940855in}}%
\pgfpathlineto{\pgfqpoint{4.246949in}{2.613037in}}%
\pgfpathlineto{\pgfqpoint{4.407273in}{2.443586in}}%
\pgfpathlineto{\pgfqpoint{4.556263in}{2.282667in}}%
\pgfpathlineto{\pgfqpoint{4.691773in}{2.133333in}}%
\pgfpathlineto{\pgfqpoint{4.768000in}{2.047951in}}%
\pgfpathlineto{\pgfqpoint{4.768000in}{2.047951in}}%
\pgfusepath{fill}%
\end{pgfscope}%
\begin{pgfscope}%
\pgfpathrectangle{\pgfqpoint{0.800000in}{0.528000in}}{\pgfqpoint{3.968000in}{3.696000in}}%
\pgfusepath{clip}%
\pgfsetbuttcap%
\pgfsetroundjoin%
\definecolor{currentfill}{rgb}{0.218130,0.347432,0.550038}%
\pgfsetfillcolor{currentfill}%
\pgfsetlinewidth{0.000000pt}%
\definecolor{currentstroke}{rgb}{0.000000,0.000000,0.000000}%
\pgfsetstrokecolor{currentstroke}%
\pgfsetdash{}{0pt}%
\pgfpathmoveto{\pgfqpoint{1.654706in}{0.528000in}}%
\pgfpathlineto{\pgfqpoint{1.361131in}{0.836504in}}%
\pgfpathlineto{\pgfqpoint{1.063498in}{1.162667in}}%
\pgfpathlineto{\pgfqpoint{0.800000in}{1.463603in}}%
\pgfpathlineto{\pgfqpoint{0.800000in}{1.461333in}}%
\pgfpathlineto{\pgfqpoint{0.800000in}{1.458645in}}%
\pgfpathlineto{\pgfqpoint{1.092628in}{1.125333in}}%
\pgfpathlineto{\pgfqpoint{1.240889in}{0.961895in}}%
\pgfpathlineto{\pgfqpoint{1.542397in}{0.640000in}}%
\pgfpathlineto{\pgfqpoint{1.650293in}{0.528000in}}%
\pgfpathmoveto{\pgfqpoint{4.768000in}{2.057642in}}%
\pgfpathlineto{\pgfqpoint{4.510539in}{2.341521in}}%
\pgfpathlineto{\pgfqpoint{4.436245in}{2.421653in}}%
\pgfpathlineto{\pgfqpoint{4.356686in}{2.506667in}}%
\pgfpathlineto{\pgfqpoint{4.046545in}{2.828581in}}%
\pgfpathlineto{\pgfqpoint{3.884019in}{2.992000in}}%
\pgfpathlineto{\pgfqpoint{3.565576in}{3.301909in}}%
\pgfpathlineto{\pgfqpoint{3.405253in}{3.453027in}}%
\pgfpathlineto{\pgfqpoint{3.244929in}{3.600927in}}%
\pgfpathlineto{\pgfqpoint{2.923391in}{3.888000in}}%
\pgfpathlineto{\pgfqpoint{2.750101in}{4.037333in}}%
\pgfpathlineto{\pgfqpoint{2.603636in}{4.160746in}}%
\pgfpathlineto{\pgfqpoint{2.527348in}{4.224000in}}%
\pgfpathlineto{\pgfqpoint{2.522092in}{4.224000in}}%
\pgfpathlineto{\pgfqpoint{2.643717in}{4.122903in}}%
\pgfpathlineto{\pgfqpoint{2.964364in}{3.847788in}}%
\pgfpathlineto{\pgfqpoint{3.129263in}{3.701333in}}%
\pgfpathlineto{\pgfqpoint{3.454416in}{3.402667in}}%
\pgfpathlineto{\pgfqpoint{3.766594in}{3.103428in}}%
\pgfpathlineto{\pgfqpoint{3.926303in}{2.945340in}}%
\pgfpathlineto{\pgfqpoint{4.246949in}{2.617641in}}%
\pgfpathlineto{\pgfqpoint{4.407273in}{2.448212in}}%
\pgfpathlineto{\pgfqpoint{4.567596in}{2.275023in}}%
\pgfpathlineto{\pgfqpoint{4.768000in}{2.052797in}}%
\pgfpathlineto{\pgfqpoint{4.768000in}{2.052797in}}%
\pgfusepath{fill}%
\end{pgfscope}%
\begin{pgfscope}%
\pgfpathrectangle{\pgfqpoint{0.800000in}{0.528000in}}{\pgfqpoint{3.968000in}{3.696000in}}%
\pgfusepath{clip}%
\pgfsetbuttcap%
\pgfsetroundjoin%
\definecolor{currentfill}{rgb}{0.216210,0.351535,0.550627}%
\pgfsetfillcolor{currentfill}%
\pgfsetlinewidth{0.000000pt}%
\definecolor{currentstroke}{rgb}{0.000000,0.000000,0.000000}%
\pgfsetstrokecolor{currentstroke}%
\pgfsetdash{}{0pt}%
\pgfpathmoveto{\pgfqpoint{1.650293in}{0.528000in}}%
\pgfpathlineto{\pgfqpoint{1.361131in}{0.831841in}}%
\pgfpathlineto{\pgfqpoint{1.059221in}{1.162667in}}%
\pgfpathlineto{\pgfqpoint{0.800000in}{1.458645in}}%
\pgfpathlineto{\pgfqpoint{0.800000in}{1.453729in}}%
\pgfpathlineto{\pgfqpoint{1.088335in}{1.125333in}}%
\pgfpathlineto{\pgfqpoint{1.223691in}{0.976000in}}%
\pgfpathlineto{\pgfqpoint{1.364821in}{0.823230in}}%
\pgfpathlineto{\pgfqpoint{1.521455in}{0.657346in}}%
\pgfpathlineto{\pgfqpoint{1.645880in}{0.528000in}}%
\pgfpathmoveto{\pgfqpoint{4.768000in}{2.062418in}}%
\pgfpathlineto{\pgfqpoint{4.603310in}{2.245333in}}%
\pgfpathlineto{\pgfqpoint{4.287030in}{2.584780in}}%
\pgfpathlineto{\pgfqpoint{4.126707in}{2.751185in}}%
\pgfpathlineto{\pgfqpoint{3.809599in}{3.069963in}}%
\pgfpathlineto{\pgfqpoint{3.736755in}{3.141333in}}%
\pgfpathlineto{\pgfqpoint{3.445333in}{3.419903in}}%
\pgfpathlineto{\pgfqpoint{3.124687in}{3.714079in}}%
\pgfpathlineto{\pgfqpoint{2.798900in}{4.000000in}}%
\pgfpathlineto{\pgfqpoint{2.532573in}{4.224000in}}%
\pgfpathlineto{\pgfqpoint{2.527348in}{4.224000in}}%
\pgfpathlineto{\pgfqpoint{2.683798in}{4.093527in}}%
\pgfpathlineto{\pgfqpoint{3.008399in}{3.813333in}}%
\pgfpathlineto{\pgfqpoint{3.325091in}{3.527377in}}%
\pgfpathlineto{\pgfqpoint{3.485414in}{3.377873in}}%
\pgfpathlineto{\pgfqpoint{3.655202in}{3.216000in}}%
\pgfpathlineto{\pgfqpoint{3.966384in}{2.909625in}}%
\pgfpathlineto{\pgfqpoint{4.126707in}{2.746675in}}%
\pgfpathlineto{\pgfqpoint{4.287030in}{2.580230in}}%
\pgfpathlineto{\pgfqpoint{4.447354in}{2.409921in}}%
\pgfpathlineto{\pgfqpoint{4.733775in}{2.096000in}}%
\pgfpathlineto{\pgfqpoint{4.768000in}{2.057642in}}%
\pgfpathlineto{\pgfqpoint{4.768000in}{2.058667in}}%
\pgfusepath{fill}%
\end{pgfscope}%
\begin{pgfscope}%
\pgfpathrectangle{\pgfqpoint{0.800000in}{0.528000in}}{\pgfqpoint{3.968000in}{3.696000in}}%
\pgfusepath{clip}%
\pgfsetbuttcap%
\pgfsetroundjoin%
\definecolor{currentfill}{rgb}{0.216210,0.351535,0.550627}%
\pgfsetfillcolor{currentfill}%
\pgfsetlinewidth{0.000000pt}%
\definecolor{currentstroke}{rgb}{0.000000,0.000000,0.000000}%
\pgfsetstrokecolor{currentstroke}%
\pgfsetdash{}{0pt}%
\pgfpathmoveto{\pgfqpoint{1.645880in}{0.528000in}}%
\pgfpathlineto{\pgfqpoint{1.361131in}{0.827179in}}%
\pgfpathlineto{\pgfqpoint{1.054945in}{1.162667in}}%
\pgfpathlineto{\pgfqpoint{0.800000in}{1.453729in}}%
\pgfpathlineto{\pgfqpoint{0.800000in}{1.448813in}}%
\pgfpathlineto{\pgfqpoint{1.084042in}{1.125333in}}%
\pgfpathlineto{\pgfqpoint{1.219407in}{0.976000in}}%
\pgfpathlineto{\pgfqpoint{1.361131in}{0.822588in}}%
\pgfpathlineto{\pgfqpoint{1.527969in}{0.646068in}}%
\pgfpathlineto{\pgfqpoint{1.605367in}{0.565333in}}%
\pgfpathlineto{\pgfqpoint{1.641697in}{0.528000in}}%
\pgfpathmoveto{\pgfqpoint{4.768000in}{2.067176in}}%
\pgfpathlineto{\pgfqpoint{4.607625in}{2.245333in}}%
\pgfpathlineto{\pgfqpoint{4.291025in}{2.585054in}}%
\pgfpathlineto{\pgfqpoint{4.223160in}{2.656000in}}%
\pgfpathlineto{\pgfqpoint{3.926303in}{2.958725in}}%
\pgfpathlineto{\pgfqpoint{3.605657in}{3.272364in}}%
\pgfpathlineto{\pgfqpoint{3.285010in}{3.572918in}}%
\pgfpathlineto{\pgfqpoint{3.124687in}{3.718394in}}%
\pgfpathlineto{\pgfqpoint{2.803979in}{4.000000in}}%
\pgfpathlineto{\pgfqpoint{2.537799in}{4.224000in}}%
\pgfpathlineto{\pgfqpoint{2.532573in}{4.224000in}}%
\pgfpathlineto{\pgfqpoint{2.683798in}{4.097861in}}%
\pgfpathlineto{\pgfqpoint{3.013261in}{3.813333in}}%
\pgfpathlineto{\pgfqpoint{3.325091in}{3.531715in}}%
\pgfpathlineto{\pgfqpoint{3.485414in}{3.382230in}}%
\pgfpathlineto{\pgfqpoint{3.645737in}{3.229506in}}%
\pgfpathlineto{\pgfqpoint{3.812997in}{3.066667in}}%
\pgfpathlineto{\pgfqpoint{4.126707in}{2.751185in}}%
\pgfpathlineto{\pgfqpoint{4.290311in}{2.581333in}}%
\pgfpathlineto{\pgfqpoint{4.438575in}{2.423823in}}%
\pgfpathlineto{\pgfqpoint{4.500457in}{2.357333in}}%
\pgfpathlineto{\pgfqpoint{4.768000in}{2.062418in}}%
\pgfpathlineto{\pgfqpoint{4.768000in}{2.062418in}}%
\pgfusepath{fill}%
\end{pgfscope}%
\begin{pgfscope}%
\pgfpathrectangle{\pgfqpoint{0.800000in}{0.528000in}}{\pgfqpoint{3.968000in}{3.696000in}}%
\pgfusepath{clip}%
\pgfsetbuttcap%
\pgfsetroundjoin%
\definecolor{currentfill}{rgb}{0.216210,0.351535,0.550627}%
\pgfsetfillcolor{currentfill}%
\pgfsetlinewidth{0.000000pt}%
\definecolor{currentstroke}{rgb}{0.000000,0.000000,0.000000}%
\pgfsetstrokecolor{currentstroke}%
\pgfsetdash{}{0pt}%
\pgfpathmoveto{\pgfqpoint{1.641471in}{0.528000in}}%
\pgfpathlineto{\pgfqpoint{1.498082in}{0.677333in}}%
\pgfpathlineto{\pgfqpoint{1.357326in}{0.826667in}}%
\pgfpathlineto{\pgfqpoint{1.050668in}{1.162667in}}%
\pgfpathlineto{\pgfqpoint{0.800000in}{1.448813in}}%
\pgfpathlineto{\pgfqpoint{0.800000in}{1.443897in}}%
\pgfpathlineto{\pgfqpoint{1.080566in}{1.124438in}}%
\pgfpathlineto{\pgfqpoint{1.249337in}{0.938667in}}%
\pgfpathlineto{\pgfqpoint{1.401212in}{0.775226in}}%
\pgfpathlineto{\pgfqpoint{1.565060in}{0.602667in}}%
\pgfpathlineto{\pgfqpoint{1.637138in}{0.528000in}}%
\pgfpathmoveto{\pgfqpoint{4.768000in}{2.071934in}}%
\pgfpathlineto{\pgfqpoint{4.607677in}{2.249930in}}%
\pgfpathlineto{\pgfqpoint{4.298936in}{2.581333in}}%
\pgfpathlineto{\pgfqpoint{4.006465in}{2.882650in}}%
\pgfpathlineto{\pgfqpoint{3.846141in}{3.042781in}}%
\pgfpathlineto{\pgfqpoint{3.525495in}{3.353078in}}%
\pgfpathlineto{\pgfqpoint{3.365172in}{3.503331in}}%
\pgfpathlineto{\pgfqpoint{3.204848in}{3.650376in}}%
\pgfpathlineto{\pgfqpoint{3.044525in}{3.794255in}}%
\pgfpathlineto{\pgfqpoint{2.721489in}{4.074667in}}%
\pgfpathlineto{\pgfqpoint{2.543024in}{4.224000in}}%
\pgfpathlineto{\pgfqpoint{2.537799in}{4.224000in}}%
\pgfpathlineto{\pgfqpoint{2.683798in}{4.102195in}}%
\pgfpathlineto{\pgfqpoint{3.011056in}{3.819491in}}%
\pgfpathlineto{\pgfqpoint{3.084606in}{3.754269in}}%
\pgfpathlineto{\pgfqpoint{3.244929in}{3.609584in}}%
\pgfpathlineto{\pgfqpoint{3.405253in}{3.461722in}}%
\pgfpathlineto{\pgfqpoint{3.725899in}{3.156302in}}%
\pgfpathlineto{\pgfqpoint{4.046545in}{2.837581in}}%
\pgfpathlineto{\pgfqpoint{4.367192in}{2.504767in}}%
\pgfpathlineto{\pgfqpoint{4.675242in}{2.170667in}}%
\pgfpathlineto{\pgfqpoint{4.768000in}{2.067176in}}%
\pgfpathlineto{\pgfqpoint{4.768000in}{2.067176in}}%
\pgfusepath{fill}%
\end{pgfscope}%
\begin{pgfscope}%
\pgfpathrectangle{\pgfqpoint{0.800000in}{0.528000in}}{\pgfqpoint{3.968000in}{3.696000in}}%
\pgfusepath{clip}%
\pgfsetbuttcap%
\pgfsetroundjoin%
\definecolor{currentfill}{rgb}{0.216210,0.351535,0.550627}%
\pgfsetfillcolor{currentfill}%
\pgfsetlinewidth{0.000000pt}%
\definecolor{currentstroke}{rgb}{0.000000,0.000000,0.000000}%
\pgfsetstrokecolor{currentstroke}%
\pgfsetdash{}{0pt}%
\pgfpathmoveto{\pgfqpoint{1.637138in}{0.528000in}}%
\pgfpathlineto{\pgfqpoint{1.481374in}{0.690336in}}%
\pgfpathlineto{\pgfqpoint{1.318276in}{0.864000in}}%
\pgfpathlineto{\pgfqpoint{1.181061in}{1.013333in}}%
\pgfpathlineto{\pgfqpoint{1.040485in}{1.169285in}}%
\pgfpathlineto{\pgfqpoint{0.880162in}{1.351180in}}%
\pgfpathlineto{\pgfqpoint{0.800000in}{1.443897in}}%
\pgfpathlineto{\pgfqpoint{0.800000in}{1.438981in}}%
\pgfpathlineto{\pgfqpoint{1.075545in}{1.125333in}}%
\pgfpathlineto{\pgfqpoint{1.361131in}{0.813424in}}%
\pgfpathlineto{\pgfqpoint{1.524957in}{0.640000in}}%
\pgfpathlineto{\pgfqpoint{1.632806in}{0.528000in}}%
\pgfpathmoveto{\pgfqpoint{4.768000in}{2.076692in}}%
\pgfpathlineto{\pgfqpoint{4.616102in}{2.245333in}}%
\pgfpathlineto{\pgfqpoint{4.478756in}{2.394667in}}%
\pgfpathlineto{\pgfqpoint{4.159790in}{2.730667in}}%
\pgfpathlineto{\pgfqpoint{4.006465in}{2.887069in}}%
\pgfpathlineto{\pgfqpoint{3.846141in}{3.047181in}}%
\pgfpathlineto{\pgfqpoint{3.517146in}{3.365333in}}%
\pgfpathlineto{\pgfqpoint{3.357614in}{3.514667in}}%
\pgfpathlineto{\pgfqpoint{3.194577in}{3.664000in}}%
\pgfpathlineto{\pgfqpoint{3.027845in}{3.813333in}}%
\pgfpathlineto{\pgfqpoint{2.723879in}{4.076946in}}%
\pgfpathlineto{\pgfqpoint{2.548250in}{4.224000in}}%
\pgfpathlineto{\pgfqpoint{2.543024in}{4.224000in}}%
\pgfpathlineto{\pgfqpoint{2.683798in}{4.106529in}}%
\pgfpathlineto{\pgfqpoint{3.004444in}{3.829735in}}%
\pgfpathlineto{\pgfqpoint{3.325091in}{3.540391in}}%
\pgfpathlineto{\pgfqpoint{3.485414in}{3.390944in}}%
\pgfpathlineto{\pgfqpoint{3.645737in}{3.238258in}}%
\pgfpathlineto{\pgfqpoint{3.806061in}{3.082291in}}%
\pgfpathlineto{\pgfqpoint{4.122692in}{2.764260in}}%
\pgfpathlineto{\pgfqpoint{4.191542in}{2.693333in}}%
\pgfpathlineto{\pgfqpoint{4.334404in}{2.544000in}}%
\pgfpathlineto{\pgfqpoint{4.647758in}{2.205847in}}%
\pgfpathlineto{\pgfqpoint{4.768000in}{2.071934in}}%
\pgfpathlineto{\pgfqpoint{4.768000in}{2.071934in}}%
\pgfusepath{fill}%
\end{pgfscope}%
\begin{pgfscope}%
\pgfpathrectangle{\pgfqpoint{0.800000in}{0.528000in}}{\pgfqpoint{3.968000in}{3.696000in}}%
\pgfusepath{clip}%
\pgfsetbuttcap%
\pgfsetroundjoin%
\definecolor{currentfill}{rgb}{0.214298,0.355619,0.551184}%
\pgfsetfillcolor{currentfill}%
\pgfsetlinewidth{0.000000pt}%
\definecolor{currentstroke}{rgb}{0.000000,0.000000,0.000000}%
\pgfsetstrokecolor{currentstroke}%
\pgfsetdash{}{0pt}%
\pgfpathmoveto{\pgfqpoint{1.632806in}{0.528000in}}%
\pgfpathlineto{\pgfqpoint{1.481374in}{0.685770in}}%
\pgfpathlineto{\pgfqpoint{1.314019in}{0.864000in}}%
\pgfpathlineto{\pgfqpoint{1.176794in}{1.013333in}}%
\pgfpathlineto{\pgfqpoint{1.040485in}{1.164494in}}%
\pgfpathlineto{\pgfqpoint{0.877547in}{1.349333in}}%
\pgfpathlineto{\pgfqpoint{0.800000in}{1.438981in}}%
\pgfpathlineto{\pgfqpoint{0.800000in}{1.434065in}}%
\pgfpathlineto{\pgfqpoint{1.040485in}{1.159755in}}%
\pgfpathlineto{\pgfqpoint{1.206556in}{0.976000in}}%
\pgfpathlineto{\pgfqpoint{1.352301in}{0.818441in}}%
\pgfpathlineto{\pgfqpoint{1.414459in}{0.752000in}}%
\pgfpathlineto{\pgfqpoint{1.628473in}{0.528000in}}%
\pgfpathmoveto{\pgfqpoint{4.768000in}{2.081450in}}%
\pgfpathlineto{\pgfqpoint{4.620340in}{2.245333in}}%
\pgfpathlineto{\pgfqpoint{4.483061in}{2.394667in}}%
\pgfpathlineto{\pgfqpoint{4.164172in}{2.730667in}}%
\pgfpathlineto{\pgfqpoint{4.006465in}{2.891489in}}%
\pgfpathlineto{\pgfqpoint{3.846141in}{3.051581in}}%
\pgfpathlineto{\pgfqpoint{3.521759in}{3.365333in}}%
\pgfpathlineto{\pgfqpoint{3.362307in}{3.514667in}}%
\pgfpathlineto{\pgfqpoint{3.199352in}{3.664000in}}%
\pgfpathlineto{\pgfqpoint{3.032706in}{3.813333in}}%
\pgfpathlineto{\pgfqpoint{2.723879in}{4.081214in}}%
\pgfpathlineto{\pgfqpoint{2.553476in}{4.224000in}}%
\pgfpathlineto{\pgfqpoint{2.548250in}{4.224000in}}%
\pgfpathlineto{\pgfqpoint{2.683798in}{4.110863in}}%
\pgfpathlineto{\pgfqpoint{3.004444in}{3.834035in}}%
\pgfpathlineto{\pgfqpoint{3.325091in}{3.544729in}}%
\pgfpathlineto{\pgfqpoint{3.485414in}{3.395301in}}%
\pgfpathlineto{\pgfqpoint{3.645737in}{3.242634in}}%
\pgfpathlineto{\pgfqpoint{3.806061in}{3.086686in}}%
\pgfpathlineto{\pgfqpoint{4.126707in}{2.764715in}}%
\pgfpathlineto{\pgfqpoint{4.287030in}{2.598371in}}%
\pgfpathlineto{\pgfqpoint{4.581993in}{2.282667in}}%
\pgfpathlineto{\pgfqpoint{4.768000in}{2.076692in}}%
\pgfpathlineto{\pgfqpoint{4.768000in}{2.076692in}}%
\pgfusepath{fill}%
\end{pgfscope}%
\begin{pgfscope}%
\pgfpathrectangle{\pgfqpoint{0.800000in}{0.528000in}}{\pgfqpoint{3.968000in}{3.696000in}}%
\pgfusepath{clip}%
\pgfsetbuttcap%
\pgfsetroundjoin%
\definecolor{currentfill}{rgb}{0.214298,0.355619,0.551184}%
\pgfsetfillcolor{currentfill}%
\pgfsetlinewidth{0.000000pt}%
\definecolor{currentstroke}{rgb}{0.000000,0.000000,0.000000}%
\pgfsetstrokecolor{currentstroke}%
\pgfsetdash{}{0pt}%
\pgfpathmoveto{\pgfqpoint{1.628473in}{0.528000in}}%
\pgfpathlineto{\pgfqpoint{1.481374in}{0.681203in}}%
\pgfpathlineto{\pgfqpoint{1.321051in}{0.851837in}}%
\pgfpathlineto{\pgfqpoint{1.172527in}{1.013333in}}%
\pgfpathlineto{\pgfqpoint{1.025158in}{1.176943in}}%
\pgfpathlineto{\pgfqpoint{0.873353in}{1.349333in}}%
\pgfpathlineto{\pgfqpoint{0.800000in}{1.434065in}}%
\pgfpathlineto{\pgfqpoint{0.800000in}{1.429149in}}%
\pgfpathlineto{\pgfqpoint{1.040485in}{1.155048in}}%
\pgfpathlineto{\pgfqpoint{1.202272in}{0.976000in}}%
\pgfpathlineto{\pgfqpoint{1.340230in}{0.826667in}}%
\pgfpathlineto{\pgfqpoint{1.624140in}{0.528000in}}%
\pgfpathmoveto{\pgfqpoint{4.768000in}{2.086208in}}%
\pgfpathlineto{\pgfqpoint{4.624578in}{2.245333in}}%
\pgfpathlineto{\pgfqpoint{4.487366in}{2.394667in}}%
\pgfpathlineto{\pgfqpoint{4.166788in}{2.732456in}}%
\pgfpathlineto{\pgfqpoint{4.006465in}{2.895909in}}%
\pgfpathlineto{\pgfqpoint{3.835307in}{3.066667in}}%
\pgfpathlineto{\pgfqpoint{3.525495in}{3.366150in}}%
\pgfpathlineto{\pgfqpoint{3.365172in}{3.516332in}}%
\pgfpathlineto{\pgfqpoint{3.204128in}{3.664000in}}%
\pgfpathlineto{\pgfqpoint{3.037567in}{3.813333in}}%
\pgfpathlineto{\pgfqpoint{2.723879in}{4.085483in}}%
\pgfpathlineto{\pgfqpoint{2.558701in}{4.224000in}}%
\pgfpathlineto{\pgfqpoint{2.553476in}{4.224000in}}%
\pgfpathlineto{\pgfqpoint{2.687517in}{4.112000in}}%
\pgfpathlineto{\pgfqpoint{3.004444in}{3.838336in}}%
\pgfpathlineto{\pgfqpoint{3.325091in}{3.549067in}}%
\pgfpathlineto{\pgfqpoint{3.485414in}{3.399658in}}%
\pgfpathlineto{\pgfqpoint{3.645737in}{3.247011in}}%
\pgfpathlineto{\pgfqpoint{3.806061in}{3.091081in}}%
\pgfpathlineto{\pgfqpoint{4.127880in}{2.768000in}}%
\pgfpathlineto{\pgfqpoint{4.287030in}{2.602901in}}%
\pgfpathlineto{\pgfqpoint{4.586248in}{2.282667in}}%
\pgfpathlineto{\pgfqpoint{4.768000in}{2.081450in}}%
\pgfpathlineto{\pgfqpoint{4.768000in}{2.081450in}}%
\pgfusepath{fill}%
\end{pgfscope}%
\begin{pgfscope}%
\pgfpathrectangle{\pgfqpoint{0.800000in}{0.528000in}}{\pgfqpoint{3.968000in}{3.696000in}}%
\pgfusepath{clip}%
\pgfsetbuttcap%
\pgfsetroundjoin%
\definecolor{currentfill}{rgb}{0.214298,0.355619,0.551184}%
\pgfsetfillcolor{currentfill}%
\pgfsetlinewidth{0.000000pt}%
\definecolor{currentstroke}{rgb}{0.000000,0.000000,0.000000}%
\pgfsetstrokecolor{currentstroke}%
\pgfsetdash{}{0pt}%
\pgfpathmoveto{\pgfqpoint{1.624140in}{0.528000in}}%
\pgfpathlineto{\pgfqpoint{1.475772in}{0.682551in}}%
\pgfpathlineto{\pgfqpoint{1.321051in}{0.847250in}}%
\pgfpathlineto{\pgfqpoint{1.160727in}{1.021613in}}%
\pgfpathlineto{\pgfqpoint{0.869159in}{1.349333in}}%
\pgfpathlineto{\pgfqpoint{0.800000in}{1.429149in}}%
\pgfpathlineto{\pgfqpoint{0.801020in}{1.423050in}}%
\pgfpathlineto{\pgfqpoint{0.963138in}{1.237333in}}%
\pgfpathlineto{\pgfqpoint{1.266692in}{0.901333in}}%
\pgfpathlineto{\pgfqpoint{1.561535in}{0.588359in}}%
\pgfpathlineto{\pgfqpoint{1.619808in}{0.528000in}}%
\pgfpathmoveto{\pgfqpoint{4.768000in}{2.090966in}}%
\pgfpathlineto{\pgfqpoint{4.628816in}{2.245333in}}%
\pgfpathlineto{\pgfqpoint{4.487434in}{2.399151in}}%
\pgfpathlineto{\pgfqpoint{4.166788in}{2.736895in}}%
\pgfpathlineto{\pgfqpoint{4.006465in}{2.900329in}}%
\pgfpathlineto{\pgfqpoint{3.839768in}{3.066667in}}%
\pgfpathlineto{\pgfqpoint{3.525495in}{3.370441in}}%
\pgfpathlineto{\pgfqpoint{3.365172in}{3.520605in}}%
\pgfpathlineto{\pgfqpoint{3.204848in}{3.667613in}}%
\pgfpathlineto{\pgfqpoint{3.042429in}{3.813333in}}%
\pgfpathlineto{\pgfqpoint{2.723879in}{4.089752in}}%
\pgfpathlineto{\pgfqpoint{2.563556in}{4.224000in}}%
\pgfpathlineto{\pgfqpoint{2.558701in}{4.224000in}}%
\pgfpathlineto{\pgfqpoint{2.692558in}{4.112000in}}%
\pgfpathlineto{\pgfqpoint{3.004444in}{3.842637in}}%
\pgfpathlineto{\pgfqpoint{3.326588in}{3.552000in}}%
\pgfpathlineto{\pgfqpoint{3.486821in}{3.402667in}}%
\pgfpathlineto{\pgfqpoint{3.645737in}{3.251387in}}%
\pgfpathlineto{\pgfqpoint{3.806061in}{3.095477in}}%
\pgfpathlineto{\pgfqpoint{4.132200in}{2.768000in}}%
\pgfpathlineto{\pgfqpoint{4.287030in}{2.607432in}}%
\pgfpathlineto{\pgfqpoint{4.590502in}{2.282667in}}%
\pgfpathlineto{\pgfqpoint{4.768000in}{2.086208in}}%
\pgfpathlineto{\pgfqpoint{4.768000in}{2.086208in}}%
\pgfusepath{fill}%
\end{pgfscope}%
\begin{pgfscope}%
\pgfpathrectangle{\pgfqpoint{0.800000in}{0.528000in}}{\pgfqpoint{3.968000in}{3.696000in}}%
\pgfusepath{clip}%
\pgfsetbuttcap%
\pgfsetroundjoin%
\definecolor{currentfill}{rgb}{0.214298,0.355619,0.551184}%
\pgfsetfillcolor{currentfill}%
\pgfsetlinewidth{0.000000pt}%
\definecolor{currentstroke}{rgb}{0.000000,0.000000,0.000000}%
\pgfsetstrokecolor{currentstroke}%
\pgfsetdash{}{0pt}%
\pgfpathmoveto{\pgfqpoint{1.619808in}{0.528000in}}%
\pgfpathlineto{\pgfqpoint{1.476459in}{0.677333in}}%
\pgfpathlineto{\pgfqpoint{1.321051in}{0.842663in}}%
\pgfpathlineto{\pgfqpoint{1.160727in}{1.016923in}}%
\pgfpathlineto{\pgfqpoint{0.864965in}{1.349333in}}%
\pgfpathlineto{\pgfqpoint{0.800000in}{1.424233in}}%
\pgfpathlineto{\pgfqpoint{0.800000in}{1.424000in}}%
\pgfpathlineto{\pgfqpoint{0.800000in}{1.419402in}}%
\pgfpathlineto{\pgfqpoint{0.960323in}{1.235745in}}%
\pgfpathlineto{\pgfqpoint{1.262451in}{0.901333in}}%
\pgfpathlineto{\pgfqpoint{1.561535in}{0.583880in}}%
\pgfpathlineto{\pgfqpoint{1.615475in}{0.528000in}}%
\pgfpathmoveto{\pgfqpoint{4.768000in}{2.095724in}}%
\pgfpathlineto{\pgfqpoint{4.633054in}{2.245333in}}%
\pgfpathlineto{\pgfqpoint{4.487434in}{2.403707in}}%
\pgfpathlineto{\pgfqpoint{4.177128in}{2.730667in}}%
\pgfpathlineto{\pgfqpoint{3.881940in}{3.029333in}}%
\pgfpathlineto{\pgfqpoint{3.725899in}{3.182552in}}%
\pgfpathlineto{\pgfqpoint{3.565576in}{3.336698in}}%
\pgfpathlineto{\pgfqpoint{3.244929in}{3.635413in}}%
\pgfpathlineto{\pgfqpoint{3.084606in}{3.780064in}}%
\pgfpathlineto{\pgfqpoint{2.919960in}{3.925333in}}%
\pgfpathlineto{\pgfqpoint{2.603636in}{4.195203in}}%
\pgfpathlineto{\pgfqpoint{2.569032in}{4.224000in}}%
\pgfpathlineto{\pgfqpoint{2.563919in}{4.224000in}}%
\pgfpathlineto{\pgfqpoint{2.884202in}{3.952156in}}%
\pgfpathlineto{\pgfqpoint{3.208824in}{3.664000in}}%
\pgfpathlineto{\pgfqpoint{3.371568in}{3.514667in}}%
\pgfpathlineto{\pgfqpoint{3.686935in}{3.216000in}}%
\pgfpathlineto{\pgfqpoint{3.846141in}{3.060382in}}%
\pgfpathlineto{\pgfqpoint{4.006465in}{2.900329in}}%
\pgfpathlineto{\pgfqpoint{4.316186in}{2.581333in}}%
\pgfpathlineto{\pgfqpoint{4.456828in}{2.432000in}}%
\pgfpathlineto{\pgfqpoint{4.768000in}{2.090966in}}%
\pgfpathlineto{\pgfqpoint{4.768000in}{2.090966in}}%
\pgfusepath{fill}%
\end{pgfscope}%
\begin{pgfscope}%
\pgfpathrectangle{\pgfqpoint{0.800000in}{0.528000in}}{\pgfqpoint{3.968000in}{3.696000in}}%
\pgfusepath{clip}%
\pgfsetbuttcap%
\pgfsetroundjoin%
\definecolor{currentfill}{rgb}{0.212395,0.359683,0.551710}%
\pgfsetfillcolor{currentfill}%
\pgfsetlinewidth{0.000000pt}%
\definecolor{currentstroke}{rgb}{0.000000,0.000000,0.000000}%
\pgfsetstrokecolor{currentstroke}%
\pgfsetdash{}{0pt}%
\pgfpathmoveto{\pgfqpoint{1.615475in}{0.528000in}}%
\pgfpathlineto{\pgfqpoint{1.472194in}{0.677333in}}%
\pgfpathlineto{\pgfqpoint{1.321051in}{0.838076in}}%
\pgfpathlineto{\pgfqpoint{1.159744in}{1.013333in}}%
\pgfpathlineto{\pgfqpoint{0.860771in}{1.349333in}}%
\pgfpathlineto{\pgfqpoint{0.800000in}{1.419402in}}%
\pgfpathlineto{\pgfqpoint{0.800000in}{1.414576in}}%
\pgfpathlineto{\pgfqpoint{0.960323in}{1.231028in}}%
\pgfpathlineto{\pgfqpoint{1.258211in}{0.901333in}}%
\pgfpathlineto{\pgfqpoint{1.561535in}{0.579401in}}%
\pgfpathlineto{\pgfqpoint{1.611142in}{0.528000in}}%
\pgfpathmoveto{\pgfqpoint{4.768000in}{2.100403in}}%
\pgfpathlineto{\pgfqpoint{4.603266in}{2.282667in}}%
\pgfpathlineto{\pgfqpoint{4.287030in}{2.620983in}}%
\pgfpathlineto{\pgfqpoint{3.961212in}{2.954667in}}%
\pgfpathlineto{\pgfqpoint{3.806061in}{3.108587in}}%
\pgfpathlineto{\pgfqpoint{3.485414in}{3.416853in}}%
\pgfpathlineto{\pgfqpoint{3.325091in}{3.566187in}}%
\pgfpathlineto{\pgfqpoint{3.004444in}{3.855461in}}%
\pgfpathlineto{\pgfqpoint{2.683798in}{4.132201in}}%
\pgfpathlineto{\pgfqpoint{2.574145in}{4.224000in}}%
\pgfpathlineto{\pgfqpoint{2.569032in}{4.224000in}}%
\pgfpathlineto{\pgfqpoint{2.884202in}{3.956443in}}%
\pgfpathlineto{\pgfqpoint{3.213506in}{3.664000in}}%
\pgfpathlineto{\pgfqpoint{3.376170in}{3.514667in}}%
\pgfpathlineto{\pgfqpoint{3.691387in}{3.216000in}}%
\pgfpathlineto{\pgfqpoint{3.846141in}{3.064782in}}%
\pgfpathlineto{\pgfqpoint{4.006465in}{2.904748in}}%
\pgfpathlineto{\pgfqpoint{4.327111in}{2.574379in}}%
\pgfpathlineto{\pgfqpoint{4.495824in}{2.394667in}}%
\pgfpathlineto{\pgfqpoint{4.647758in}{2.229173in}}%
\pgfpathlineto{\pgfqpoint{4.768000in}{2.095724in}}%
\pgfpathlineto{\pgfqpoint{4.768000in}{2.096000in}}%
\pgfusepath{fill}%
\end{pgfscope}%
\begin{pgfscope}%
\pgfpathrectangle{\pgfqpoint{0.800000in}{0.528000in}}{\pgfqpoint{3.968000in}{3.696000in}}%
\pgfusepath{clip}%
\pgfsetbuttcap%
\pgfsetroundjoin%
\definecolor{currentfill}{rgb}{0.212395,0.359683,0.551710}%
\pgfsetfillcolor{currentfill}%
\pgfsetlinewidth{0.000000pt}%
\definecolor{currentstroke}{rgb}{0.000000,0.000000,0.000000}%
\pgfsetstrokecolor{currentstroke}%
\pgfsetdash{}{0pt}%
\pgfpathmoveto{\pgfqpoint{1.611142in}{0.528000in}}%
\pgfpathlineto{\pgfqpoint{1.467930in}{0.677333in}}%
\pgfpathlineto{\pgfqpoint{1.321051in}{0.833489in}}%
\pgfpathlineto{\pgfqpoint{1.155552in}{1.013333in}}%
\pgfpathlineto{\pgfqpoint{0.856577in}{1.349333in}}%
\pgfpathlineto{\pgfqpoint{0.800000in}{1.414576in}}%
\pgfpathlineto{\pgfqpoint{0.800000in}{1.409749in}}%
\pgfpathlineto{\pgfqpoint{0.950583in}{1.237333in}}%
\pgfpathlineto{\pgfqpoint{1.240889in}{0.915515in}}%
\pgfpathlineto{\pgfqpoint{1.401212in}{0.743340in}}%
\pgfpathlineto{\pgfqpoint{1.606810in}{0.528000in}}%
\pgfpathmoveto{\pgfqpoint{4.768000in}{2.105077in}}%
\pgfpathlineto{\pgfqpoint{4.607520in}{2.282667in}}%
\pgfpathlineto{\pgfqpoint{4.287030in}{2.625437in}}%
\pgfpathlineto{\pgfqpoint{3.965620in}{2.954667in}}%
\pgfpathlineto{\pgfqpoint{3.806061in}{3.112910in}}%
\pgfpathlineto{\pgfqpoint{3.485414in}{3.421139in}}%
\pgfpathlineto{\pgfqpoint{3.325091in}{3.570455in}}%
\pgfpathlineto{\pgfqpoint{3.004444in}{3.859693in}}%
\pgfpathlineto{\pgfqpoint{2.676484in}{4.142520in}}%
\pgfpathlineto{\pgfqpoint{2.603636in}{4.203713in}}%
\pgfpathlineto{\pgfqpoint{2.579258in}{4.224000in}}%
\pgfpathlineto{\pgfqpoint{2.574145in}{4.224000in}}%
\pgfpathlineto{\pgfqpoint{2.884202in}{3.960730in}}%
\pgfpathlineto{\pgfqpoint{3.218188in}{3.664000in}}%
\pgfpathlineto{\pgfqpoint{3.380772in}{3.514667in}}%
\pgfpathlineto{\pgfqpoint{3.695838in}{3.216000in}}%
\pgfpathlineto{\pgfqpoint{3.848645in}{3.066667in}}%
\pgfpathlineto{\pgfqpoint{4.006465in}{2.909168in}}%
\pgfpathlineto{\pgfqpoint{4.327111in}{2.578914in}}%
\pgfpathlineto{\pgfqpoint{4.487434in}{2.408263in}}%
\pgfpathlineto{\pgfqpoint{4.637292in}{2.245333in}}%
\pgfpathlineto{\pgfqpoint{4.768000in}{2.100403in}}%
\pgfpathlineto{\pgfqpoint{4.768000in}{2.100403in}}%
\pgfusepath{fill}%
\end{pgfscope}%
\begin{pgfscope}%
\pgfpathrectangle{\pgfqpoint{0.800000in}{0.528000in}}{\pgfqpoint{3.968000in}{3.696000in}}%
\pgfusepath{clip}%
\pgfsetbuttcap%
\pgfsetroundjoin%
\definecolor{currentfill}{rgb}{0.212395,0.359683,0.551710}%
\pgfsetfillcolor{currentfill}%
\pgfsetlinewidth{0.000000pt}%
\definecolor{currentstroke}{rgb}{0.000000,0.000000,0.000000}%
\pgfsetstrokecolor{currentstroke}%
\pgfsetdash{}{0pt}%
\pgfpathmoveto{\pgfqpoint{1.606810in}{0.528000in}}%
\pgfpathlineto{\pgfqpoint{1.463665in}{0.677333in}}%
\pgfpathlineto{\pgfqpoint{1.321051in}{0.828902in}}%
\pgfpathlineto{\pgfqpoint{1.151360in}{1.013333in}}%
\pgfpathlineto{\pgfqpoint{0.852383in}{1.349333in}}%
\pgfpathlineto{\pgfqpoint{0.800000in}{1.409749in}}%
\pgfpathlineto{\pgfqpoint{0.800000in}{1.404923in}}%
\pgfpathlineto{\pgfqpoint{0.946414in}{1.237333in}}%
\pgfpathlineto{\pgfqpoint{1.249729in}{0.901333in}}%
\pgfpathlineto{\pgfqpoint{1.566460in}{0.565333in}}%
\pgfpathlineto{\pgfqpoint{1.602477in}{0.528000in}}%
\pgfpathmoveto{\pgfqpoint{4.768000in}{2.109751in}}%
\pgfpathlineto{\pgfqpoint{4.607677in}{2.287070in}}%
\pgfpathlineto{\pgfqpoint{4.297746in}{2.618667in}}%
\pgfpathlineto{\pgfqpoint{4.153799in}{2.768000in}}%
\pgfpathlineto{\pgfqpoint{3.846141in}{3.077798in}}%
\pgfpathlineto{\pgfqpoint{3.525495in}{3.387605in}}%
\pgfpathlineto{\pgfqpoint{3.365172in}{3.537695in}}%
\pgfpathlineto{\pgfqpoint{3.204848in}{3.684631in}}%
\pgfpathlineto{\pgfqpoint{3.044525in}{3.828452in}}%
\pgfpathlineto{\pgfqpoint{2.717768in}{4.112000in}}%
\pgfpathlineto{\pgfqpoint{2.584371in}{4.224000in}}%
\pgfpathlineto{\pgfqpoint{2.579258in}{4.224000in}}%
\pgfpathlineto{\pgfqpoint{2.886861in}{3.962667in}}%
\pgfpathlineto{\pgfqpoint{3.204848in}{3.680376in}}%
\pgfpathlineto{\pgfqpoint{3.525495in}{3.383314in}}%
\pgfpathlineto{\pgfqpoint{3.685818in}{3.230010in}}%
\pgfpathlineto{\pgfqpoint{4.006465in}{2.913588in}}%
\pgfpathlineto{\pgfqpoint{4.329087in}{2.581333in}}%
\pgfpathlineto{\pgfqpoint{4.477795in}{2.423021in}}%
\pgfpathlineto{\pgfqpoint{4.538846in}{2.357333in}}%
\pgfpathlineto{\pgfqpoint{4.687838in}{2.194251in}}%
\pgfpathlineto{\pgfqpoint{4.768000in}{2.105077in}}%
\pgfpathlineto{\pgfqpoint{4.768000in}{2.105077in}}%
\pgfusepath{fill}%
\end{pgfscope}%
\begin{pgfscope}%
\pgfpathrectangle{\pgfqpoint{0.800000in}{0.528000in}}{\pgfqpoint{3.968000in}{3.696000in}}%
\pgfusepath{clip}%
\pgfsetbuttcap%
\pgfsetroundjoin%
\definecolor{currentfill}{rgb}{0.210503,0.363727,0.552206}%
\pgfsetfillcolor{currentfill}%
\pgfsetlinewidth{0.000000pt}%
\definecolor{currentstroke}{rgb}{0.000000,0.000000,0.000000}%
\pgfsetstrokecolor{currentstroke}%
\pgfsetdash{}{0pt}%
\pgfpathmoveto{\pgfqpoint{1.602477in}{0.528000in}}%
\pgfpathlineto{\pgfqpoint{1.481374in}{0.654206in}}%
\pgfpathlineto{\pgfqpoint{1.318898in}{0.826667in}}%
\pgfpathlineto{\pgfqpoint{1.181207in}{0.976000in}}%
\pgfpathlineto{\pgfqpoint{1.040485in}{1.131516in}}%
\pgfpathlineto{\pgfqpoint{0.800000in}{1.404923in}}%
\pgfpathlineto{\pgfqpoint{0.800000in}{1.400096in}}%
\pgfpathlineto{\pgfqpoint{0.942245in}{1.237333in}}%
\pgfpathlineto{\pgfqpoint{1.245488in}{0.901333in}}%
\pgfpathlineto{\pgfqpoint{1.562145in}{0.565333in}}%
\pgfpathlineto{\pgfqpoint{1.598207in}{0.528000in}}%
\pgfpathlineto{\pgfqpoint{1.601616in}{0.528000in}}%
\pgfpathmoveto{\pgfqpoint{4.768000in}{2.114425in}}%
\pgfpathlineto{\pgfqpoint{4.615883in}{2.282667in}}%
\pgfpathlineto{\pgfqpoint{4.478054in}{2.432000in}}%
\pgfpathlineto{\pgfqpoint{4.327111in}{2.592332in}}%
\pgfpathlineto{\pgfqpoint{4.006465in}{2.922343in}}%
\pgfpathlineto{\pgfqpoint{3.846141in}{3.082126in}}%
\pgfpathlineto{\pgfqpoint{3.525495in}{3.391896in}}%
\pgfpathlineto{\pgfqpoint{3.354320in}{3.552000in}}%
\pgfpathlineto{\pgfqpoint{3.044525in}{3.832689in}}%
\pgfpathlineto{\pgfqpoint{2.722810in}{4.112000in}}%
\pgfpathlineto{\pgfqpoint{2.589484in}{4.224000in}}%
\pgfpathlineto{\pgfqpoint{2.584371in}{4.224000in}}%
\pgfpathlineto{\pgfqpoint{2.848639in}{4.000000in}}%
\pgfpathlineto{\pgfqpoint{3.019435in}{3.850667in}}%
\pgfpathlineto{\pgfqpoint{3.325091in}{3.574723in}}%
\pgfpathlineto{\pgfqpoint{3.645737in}{3.272943in}}%
\pgfpathlineto{\pgfqpoint{3.969961in}{2.954667in}}%
\pgfpathlineto{\pgfqpoint{4.126707in}{2.795810in}}%
\pgfpathlineto{\pgfqpoint{4.438952in}{2.469333in}}%
\pgfpathlineto{\pgfqpoint{4.577455in}{2.320000in}}%
\pgfpathlineto{\pgfqpoint{4.768000in}{2.109751in}}%
\pgfpathlineto{\pgfqpoint{4.768000in}{2.109751in}}%
\pgfusepath{fill}%
\end{pgfscope}%
\begin{pgfscope}%
\pgfpathrectangle{\pgfqpoint{0.800000in}{0.528000in}}{\pgfqpoint{3.968000in}{3.696000in}}%
\pgfusepath{clip}%
\pgfsetbuttcap%
\pgfsetroundjoin%
\definecolor{currentfill}{rgb}{0.210503,0.363727,0.552206}%
\pgfsetfillcolor{currentfill}%
\pgfsetlinewidth{0.000000pt}%
\definecolor{currentstroke}{rgb}{0.000000,0.000000,0.000000}%
\pgfsetstrokecolor{currentstroke}%
\pgfsetdash{}{0pt}%
\pgfpathmoveto{\pgfqpoint{1.598207in}{0.528000in}}%
\pgfpathlineto{\pgfqpoint{1.279977in}{0.864000in}}%
\pgfpathlineto{\pgfqpoint{0.975292in}{1.200000in}}%
\pgfpathlineto{\pgfqpoint{0.800000in}{1.400096in}}%
\pgfpathlineto{\pgfqpoint{0.800000in}{1.395270in}}%
\pgfpathlineto{\pgfqpoint{0.938077in}{1.237333in}}%
\pgfpathlineto{\pgfqpoint{1.241247in}{0.901333in}}%
\pgfpathlineto{\pgfqpoint{1.561535in}{0.561551in}}%
\pgfpathlineto{\pgfqpoint{1.593951in}{0.528000in}}%
\pgfpathmoveto{\pgfqpoint{4.768000in}{2.119099in}}%
\pgfpathlineto{\pgfqpoint{4.620062in}{2.282667in}}%
\pgfpathlineto{\pgfqpoint{4.482299in}{2.432000in}}%
\pgfpathlineto{\pgfqpoint{4.327111in}{2.596791in}}%
\pgfpathlineto{\pgfqpoint{4.011252in}{2.921793in}}%
\pgfpathlineto{\pgfqpoint{3.941312in}{2.992000in}}%
\pgfpathlineto{\pgfqpoint{3.636236in}{3.290667in}}%
\pgfpathlineto{\pgfqpoint{3.318486in}{3.589333in}}%
\pgfpathlineto{\pgfqpoint{2.995763in}{3.879913in}}%
\pgfpathlineto{\pgfqpoint{2.913144in}{3.952291in}}%
\pgfpathlineto{\pgfqpoint{2.844121in}{4.012337in}}%
\pgfpathlineto{\pgfqpoint{2.683708in}{4.149333in}}%
\pgfpathlineto{\pgfqpoint{2.594597in}{4.224000in}}%
\pgfpathlineto{\pgfqpoint{2.589484in}{4.224000in}}%
\pgfpathlineto{\pgfqpoint{2.853510in}{4.000000in}}%
\pgfpathlineto{\pgfqpoint{3.004444in}{3.868157in}}%
\pgfpathlineto{\pgfqpoint{3.164768in}{3.725127in}}%
\pgfpathlineto{\pgfqpoint{3.325091in}{3.578992in}}%
\pgfpathlineto{\pgfqpoint{3.645737in}{3.277248in}}%
\pgfpathlineto{\pgfqpoint{3.974288in}{2.954667in}}%
\pgfpathlineto{\pgfqpoint{4.126707in}{2.800245in}}%
\pgfpathlineto{\pgfqpoint{4.447354in}{2.464913in}}%
\pgfpathlineto{\pgfqpoint{4.615883in}{2.282667in}}%
\pgfpathlineto{\pgfqpoint{4.751054in}{2.133333in}}%
\pgfpathlineto{\pgfqpoint{4.768000in}{2.114425in}}%
\pgfpathlineto{\pgfqpoint{4.768000in}{2.114425in}}%
\pgfusepath{fill}%
\end{pgfscope}%
\begin{pgfscope}%
\pgfpathrectangle{\pgfqpoint{0.800000in}{0.528000in}}{\pgfqpoint{3.968000in}{3.696000in}}%
\pgfusepath{clip}%
\pgfsetbuttcap%
\pgfsetroundjoin%
\definecolor{currentfill}{rgb}{0.210503,0.363727,0.552206}%
\pgfsetfillcolor{currentfill}%
\pgfsetlinewidth{0.000000pt}%
\definecolor{currentstroke}{rgb}{0.000000,0.000000,0.000000}%
\pgfsetstrokecolor{currentstroke}%
\pgfsetdash{}{0pt}%
\pgfpathmoveto{\pgfqpoint{1.593951in}{0.528000in}}%
\pgfpathlineto{\pgfqpoint{1.275795in}{0.864000in}}%
\pgfpathlineto{\pgfqpoint{0.971108in}{1.200000in}}%
\pgfpathlineto{\pgfqpoint{0.800000in}{1.395270in}}%
\pgfpathlineto{\pgfqpoint{0.800000in}{1.390444in}}%
\pgfpathlineto{\pgfqpoint{0.933908in}{1.237333in}}%
\pgfpathlineto{\pgfqpoint{1.240889in}{0.897196in}}%
\pgfpathlineto{\pgfqpoint{1.557411in}{0.561492in}}%
\pgfpathlineto{\pgfqpoint{1.589696in}{0.528000in}}%
\pgfpathmoveto{\pgfqpoint{4.768000in}{2.123773in}}%
\pgfpathlineto{\pgfqpoint{4.624242in}{2.282667in}}%
\pgfpathlineto{\pgfqpoint{4.480645in}{2.438324in}}%
\pgfpathlineto{\pgfqpoint{4.327111in}{2.601250in}}%
\pgfpathlineto{\pgfqpoint{4.020052in}{2.917333in}}%
\pgfpathlineto{\pgfqpoint{3.718096in}{3.216000in}}%
\pgfpathlineto{\pgfqpoint{3.403784in}{3.514667in}}%
\pgfpathlineto{\pgfqpoint{3.241596in}{3.664000in}}%
\pgfpathlineto{\pgfqpoint{2.915505in}{3.954490in}}%
\pgfpathlineto{\pgfqpoint{2.844121in}{4.016552in}}%
\pgfpathlineto{\pgfqpoint{2.683798in}{4.153455in}}%
\pgfpathlineto{\pgfqpoint{2.599710in}{4.224000in}}%
\pgfpathlineto{\pgfqpoint{2.594597in}{4.224000in}}%
\pgfpathlineto{\pgfqpoint{2.858381in}{4.000000in}}%
\pgfpathlineto{\pgfqpoint{3.004444in}{3.872389in}}%
\pgfpathlineto{\pgfqpoint{3.164768in}{3.729377in}}%
\pgfpathlineto{\pgfqpoint{3.325091in}{3.583260in}}%
\pgfpathlineto{\pgfqpoint{3.645737in}{3.281553in}}%
\pgfpathlineto{\pgfqpoint{3.972681in}{2.960532in}}%
\pgfpathlineto{\pgfqpoint{4.052693in}{2.880000in}}%
\pgfpathlineto{\pgfqpoint{4.206869in}{2.722169in}}%
\pgfpathlineto{\pgfqpoint{4.377178in}{2.544000in}}%
\pgfpathlineto{\pgfqpoint{4.527515in}{2.383288in}}%
\pgfpathlineto{\pgfqpoint{4.768000in}{2.119099in}}%
\pgfpathlineto{\pgfqpoint{4.768000in}{2.119099in}}%
\pgfusepath{fill}%
\end{pgfscope}%
\begin{pgfscope}%
\pgfpathrectangle{\pgfqpoint{0.800000in}{0.528000in}}{\pgfqpoint{3.968000in}{3.696000in}}%
\pgfusepath{clip}%
\pgfsetbuttcap%
\pgfsetroundjoin%
\definecolor{currentfill}{rgb}{0.210503,0.363727,0.552206}%
\pgfsetfillcolor{currentfill}%
\pgfsetlinewidth{0.000000pt}%
\definecolor{currentstroke}{rgb}{0.000000,0.000000,0.000000}%
\pgfsetstrokecolor{currentstroke}%
\pgfsetdash{}{0pt}%
\pgfpathmoveto{\pgfqpoint{1.589696in}{0.528000in}}%
\pgfpathlineto{\pgfqpoint{1.271613in}{0.864000in}}%
\pgfpathlineto{\pgfqpoint{0.960323in}{1.207440in}}%
\pgfpathlineto{\pgfqpoint{0.800000in}{1.390444in}}%
\pgfpathlineto{\pgfqpoint{0.800000in}{1.385636in}}%
\pgfpathlineto{\pgfqpoint{1.096575in}{1.050667in}}%
\pgfpathlineto{\pgfqpoint{1.407047in}{0.714667in}}%
\pgfpathlineto{\pgfqpoint{1.585441in}{0.528000in}}%
\pgfpathmoveto{\pgfqpoint{4.768000in}{2.128447in}}%
\pgfpathlineto{\pgfqpoint{4.628422in}{2.282667in}}%
\pgfpathlineto{\pgfqpoint{4.487434in}{2.435539in}}%
\pgfpathlineto{\pgfqpoint{4.314755in}{2.618667in}}%
\pgfpathlineto{\pgfqpoint{4.166788in}{2.772337in}}%
\pgfpathlineto{\pgfqpoint{3.837110in}{3.104000in}}%
\pgfpathlineto{\pgfqpoint{3.683977in}{3.253333in}}%
\pgfpathlineto{\pgfqpoint{3.365172in}{3.554742in}}%
\pgfpathlineto{\pgfqpoint{3.038575in}{3.850667in}}%
\pgfpathlineto{\pgfqpoint{2.868122in}{4.000000in}}%
\pgfpathlineto{\pgfqpoint{2.604798in}{4.224000in}}%
\pgfpathlineto{\pgfqpoint{2.599710in}{4.224000in}}%
\pgfpathlineto{\pgfqpoint{2.863251in}{4.000000in}}%
\pgfpathlineto{\pgfqpoint{3.004444in}{3.876622in}}%
\pgfpathlineto{\pgfqpoint{3.164768in}{3.733627in}}%
\pgfpathlineto{\pgfqpoint{3.325091in}{3.587528in}}%
\pgfpathlineto{\pgfqpoint{3.645737in}{3.285858in}}%
\pgfpathlineto{\pgfqpoint{3.966384in}{2.971283in}}%
\pgfpathlineto{\pgfqpoint{4.274814in}{2.656000in}}%
\pgfpathlineto{\pgfqpoint{4.567596in}{2.344431in}}%
\pgfpathlineto{\pgfqpoint{4.727919in}{2.168454in}}%
\pgfpathlineto{\pgfqpoint{4.768000in}{2.123773in}}%
\pgfpathlineto{\pgfqpoint{4.768000in}{2.123773in}}%
\pgfusepath{fill}%
\end{pgfscope}%
\begin{pgfscope}%
\pgfpathrectangle{\pgfqpoint{0.800000in}{0.528000in}}{\pgfqpoint{3.968000in}{3.696000in}}%
\pgfusepath{clip}%
\pgfsetbuttcap%
\pgfsetroundjoin%
\definecolor{currentfill}{rgb}{0.208623,0.367752,0.552675}%
\pgfsetfillcolor{currentfill}%
\pgfsetlinewidth{0.000000pt}%
\definecolor{currentstroke}{rgb}{0.000000,0.000000,0.000000}%
\pgfsetstrokecolor{currentstroke}%
\pgfsetdash{}{0pt}%
\pgfpathmoveto{\pgfqpoint{1.585441in}{0.528000in}}%
\pgfpathlineto{\pgfqpoint{1.273755in}{0.857280in}}%
\pgfpathlineto{\pgfqpoint{1.198536in}{0.938667in}}%
\pgfpathlineto{\pgfqpoint{0.896878in}{1.274667in}}%
\pgfpathlineto{\pgfqpoint{0.800000in}{1.385636in}}%
\pgfpathlineto{\pgfqpoint{0.800000in}{1.380896in}}%
\pgfpathlineto{\pgfqpoint{1.092400in}{1.050667in}}%
\pgfpathlineto{\pgfqpoint{1.402799in}{0.714667in}}%
\pgfpathlineto{\pgfqpoint{1.581186in}{0.528000in}}%
\pgfpathmoveto{\pgfqpoint{4.768000in}{2.133121in}}%
\pgfpathlineto{\pgfqpoint{4.632601in}{2.282667in}}%
\pgfpathlineto{\pgfqpoint{4.487434in}{2.440019in}}%
\pgfpathlineto{\pgfqpoint{4.319007in}{2.618667in}}%
\pgfpathlineto{\pgfqpoint{4.166788in}{2.776703in}}%
\pgfpathlineto{\pgfqpoint{3.841507in}{3.104000in}}%
\pgfpathlineto{\pgfqpoint{3.685818in}{3.255827in}}%
\pgfpathlineto{\pgfqpoint{3.365172in}{3.558947in}}%
\pgfpathlineto{\pgfqpoint{3.043360in}{3.850667in}}%
\pgfpathlineto{\pgfqpoint{2.872993in}{4.000000in}}%
\pgfpathlineto{\pgfqpoint{2.609804in}{4.224000in}}%
\pgfpathlineto{\pgfqpoint{2.604798in}{4.224000in}}%
\pgfpathlineto{\pgfqpoint{2.781394in}{4.074667in}}%
\pgfpathlineto{\pgfqpoint{3.084606in}{3.809751in}}%
\pgfpathlineto{\pgfqpoint{3.408327in}{3.514667in}}%
\pgfpathlineto{\pgfqpoint{3.567059in}{3.365333in}}%
\pgfpathlineto{\pgfqpoint{3.725899in}{3.212751in}}%
\pgfpathlineto{\pgfqpoint{3.886222in}{3.055488in}}%
\pgfpathlineto{\pgfqpoint{4.046545in}{2.894936in}}%
\pgfpathlineto{\pgfqpoint{4.210572in}{2.727218in}}%
\pgfpathlineto{\pgfqpoint{4.367192in}{2.563494in}}%
\pgfpathlineto{\pgfqpoint{4.662457in}{2.245333in}}%
\pgfpathlineto{\pgfqpoint{4.768000in}{2.128447in}}%
\pgfpathlineto{\pgfqpoint{4.768000in}{2.128447in}}%
\pgfusepath{fill}%
\end{pgfscope}%
\begin{pgfscope}%
\pgfpathrectangle{\pgfqpoint{0.800000in}{0.528000in}}{\pgfqpoint{3.968000in}{3.696000in}}%
\pgfusepath{clip}%
\pgfsetbuttcap%
\pgfsetroundjoin%
\definecolor{currentfill}{rgb}{0.208623,0.367752,0.552675}%
\pgfsetfillcolor{currentfill}%
\pgfsetlinewidth{0.000000pt}%
\definecolor{currentstroke}{rgb}{0.000000,0.000000,0.000000}%
\pgfsetstrokecolor{currentstroke}%
\pgfsetdash{}{0pt}%
\pgfpathmoveto{\pgfqpoint{1.581186in}{0.528000in}}%
\pgfpathlineto{\pgfqpoint{1.280970in}{0.844873in}}%
\pgfpathlineto{\pgfqpoint{0.991827in}{1.162667in}}%
\pgfpathlineto{\pgfqpoint{0.860017in}{1.312000in}}%
\pgfpathlineto{\pgfqpoint{0.800000in}{1.380896in}}%
\pgfpathlineto{\pgfqpoint{0.800000in}{1.376156in}}%
\pgfpathlineto{\pgfqpoint{1.088224in}{1.050667in}}%
\pgfpathlineto{\pgfqpoint{1.401212in}{0.711895in}}%
\pgfpathlineto{\pgfqpoint{1.576931in}{0.528000in}}%
\pgfpathmoveto{\pgfqpoint{4.768000in}{2.137718in}}%
\pgfpathlineto{\pgfqpoint{4.464220in}{2.469333in}}%
\pgfpathlineto{\pgfqpoint{4.166788in}{2.781069in}}%
\pgfpathlineto{\pgfqpoint{3.841645in}{3.108188in}}%
\pgfpathlineto{\pgfqpoint{3.685818in}{3.260068in}}%
\pgfpathlineto{\pgfqpoint{3.365172in}{3.563152in}}%
\pgfpathlineto{\pgfqpoint{3.044525in}{3.853821in}}%
\pgfpathlineto{\pgfqpoint{2.877863in}{4.000000in}}%
\pgfpathlineto{\pgfqpoint{2.614809in}{4.224000in}}%
\pgfpathlineto{\pgfqpoint{2.609804in}{4.224000in}}%
\pgfpathlineto{\pgfqpoint{2.774578in}{4.084557in}}%
\pgfpathlineto{\pgfqpoint{2.857918in}{4.012851in}}%
\pgfpathlineto{\pgfqpoint{2.924283in}{3.955413in}}%
\pgfpathlineto{\pgfqpoint{3.085331in}{3.813333in}}%
\pgfpathlineto{\pgfqpoint{3.412842in}{3.514667in}}%
\pgfpathlineto{\pgfqpoint{3.571500in}{3.365333in}}%
\pgfpathlineto{\pgfqpoint{3.726978in}{3.216000in}}%
\pgfpathlineto{\pgfqpoint{3.886222in}{3.059821in}}%
\pgfpathlineto{\pgfqpoint{4.046545in}{2.899287in}}%
\pgfpathlineto{\pgfqpoint{4.211467in}{2.730667in}}%
\pgfpathlineto{\pgfqpoint{4.367192in}{2.567958in}}%
\pgfpathlineto{\pgfqpoint{4.666621in}{2.245333in}}%
\pgfpathlineto{\pgfqpoint{4.768000in}{2.133121in}}%
\pgfpathlineto{\pgfqpoint{4.768000in}{2.133333in}}%
\pgfusepath{fill}%
\end{pgfscope}%
\begin{pgfscope}%
\pgfpathrectangle{\pgfqpoint{0.800000in}{0.528000in}}{\pgfqpoint{3.968000in}{3.696000in}}%
\pgfusepath{clip}%
\pgfsetbuttcap%
\pgfsetroundjoin%
\definecolor{currentfill}{rgb}{0.208623,0.367752,0.552675}%
\pgfsetfillcolor{currentfill}%
\pgfsetlinewidth{0.000000pt}%
\definecolor{currentstroke}{rgb}{0.000000,0.000000,0.000000}%
\pgfsetstrokecolor{currentstroke}%
\pgfsetdash{}{0pt}%
\pgfpathmoveto{\pgfqpoint{1.576931in}{0.528000in}}%
\pgfpathlineto{\pgfqpoint{1.280970in}{0.840359in}}%
\pgfpathlineto{\pgfqpoint{0.987699in}{1.162667in}}%
\pgfpathlineto{\pgfqpoint{0.840081in}{1.330081in}}%
\pgfpathlineto{\pgfqpoint{0.800000in}{1.376156in}}%
\pgfpathlineto{\pgfqpoint{0.800000in}{1.371416in}}%
\pgfpathlineto{\pgfqpoint{1.084049in}{1.050667in}}%
\pgfpathlineto{\pgfqpoint{1.401212in}{0.707472in}}%
\pgfpathlineto{\pgfqpoint{1.572676in}{0.528000in}}%
\pgfpathmoveto{\pgfqpoint{4.768000in}{2.142311in}}%
\pgfpathlineto{\pgfqpoint{4.468407in}{2.469333in}}%
\pgfpathlineto{\pgfqpoint{4.166788in}{2.785435in}}%
\pgfpathlineto{\pgfqpoint{3.846141in}{3.108029in}}%
\pgfpathlineto{\pgfqpoint{3.685818in}{3.264308in}}%
\pgfpathlineto{\pgfqpoint{3.365172in}{3.567357in}}%
\pgfpathlineto{\pgfqpoint{3.044525in}{3.857990in}}%
\pgfpathlineto{\pgfqpoint{2.882734in}{4.000000in}}%
\pgfpathlineto{\pgfqpoint{2.619814in}{4.224000in}}%
\pgfpathlineto{\pgfqpoint{2.614809in}{4.224000in}}%
\pgfpathlineto{\pgfqpoint{2.763960in}{4.097995in}}%
\pgfpathlineto{\pgfqpoint{3.090001in}{3.813333in}}%
\pgfpathlineto{\pgfqpoint{3.417357in}{3.514667in}}%
\pgfpathlineto{\pgfqpoint{3.575941in}{3.365333in}}%
\pgfpathlineto{\pgfqpoint{3.731348in}{3.216000in}}%
\pgfpathlineto{\pgfqpoint{3.886222in}{3.064154in}}%
\pgfpathlineto{\pgfqpoint{4.046545in}{2.903639in}}%
\pgfpathlineto{\pgfqpoint{4.215693in}{2.730667in}}%
\pgfpathlineto{\pgfqpoint{4.367192in}{2.572422in}}%
\pgfpathlineto{\pgfqpoint{4.670785in}{2.245333in}}%
\pgfpathlineto{\pgfqpoint{4.768000in}{2.137718in}}%
\pgfpathlineto{\pgfqpoint{4.768000in}{2.137718in}}%
\pgfusepath{fill}%
\end{pgfscope}%
\begin{pgfscope}%
\pgfpathrectangle{\pgfqpoint{0.800000in}{0.528000in}}{\pgfqpoint{3.968000in}{3.696000in}}%
\pgfusepath{clip}%
\pgfsetbuttcap%
\pgfsetroundjoin%
\definecolor{currentfill}{rgb}{0.208623,0.367752,0.552675}%
\pgfsetfillcolor{currentfill}%
\pgfsetlinewidth{0.000000pt}%
\definecolor{currentstroke}{rgb}{0.000000,0.000000,0.000000}%
\pgfsetstrokecolor{currentstroke}%
\pgfsetdash{}{0pt}%
\pgfpathmoveto{\pgfqpoint{1.572676in}{0.528000in}}%
\pgfpathlineto{\pgfqpoint{1.280970in}{0.835845in}}%
\pgfpathlineto{\pgfqpoint{0.983571in}{1.162667in}}%
\pgfpathlineto{\pgfqpoint{0.840081in}{1.325346in}}%
\pgfpathlineto{\pgfqpoint{0.800000in}{1.371416in}}%
\pgfpathlineto{\pgfqpoint{0.800000in}{1.366676in}}%
\pgfpathlineto{\pgfqpoint{1.080566in}{1.049914in}}%
\pgfpathlineto{\pgfqpoint{1.390251in}{0.714667in}}%
\pgfpathlineto{\pgfqpoint{1.568421in}{0.528000in}}%
\pgfpathmoveto{\pgfqpoint{4.768000in}{2.146904in}}%
\pgfpathlineto{\pgfqpoint{4.472594in}{2.469333in}}%
\pgfpathlineto{\pgfqpoint{4.158790in}{2.797884in}}%
\pgfpathlineto{\pgfqpoint{4.078452in}{2.880000in}}%
\pgfpathlineto{\pgfqpoint{3.926303in}{3.032951in}}%
\pgfpathlineto{\pgfqpoint{3.765980in}{3.190819in}}%
\pgfpathlineto{\pgfqpoint{3.605657in}{3.345483in}}%
\pgfpathlineto{\pgfqpoint{3.445333in}{3.496984in}}%
\pgfpathlineto{\pgfqpoint{3.285010in}{3.645363in}}%
\pgfpathlineto{\pgfqpoint{2.964364in}{3.932907in}}%
\pgfpathlineto{\pgfqpoint{2.801054in}{4.074667in}}%
\pgfpathlineto{\pgfqpoint{2.624820in}{4.224000in}}%
\pgfpathlineto{\pgfqpoint{2.619814in}{4.224000in}}%
\pgfpathlineto{\pgfqpoint{2.763960in}{4.102201in}}%
\pgfpathlineto{\pgfqpoint{3.094672in}{3.813333in}}%
\pgfpathlineto{\pgfqpoint{3.413505in}{3.522353in}}%
\pgfpathlineto{\pgfqpoint{3.501534in}{3.440000in}}%
\pgfpathlineto{\pgfqpoint{3.812242in}{3.141333in}}%
\pgfpathlineto{\pgfqpoint{3.966384in}{2.988653in}}%
\pgfpathlineto{\pgfqpoint{4.291804in}{2.656000in}}%
\pgfpathlineto{\pgfqpoint{4.447354in}{2.491832in}}%
\pgfpathlineto{\pgfqpoint{4.742489in}{2.170667in}}%
\pgfpathlineto{\pgfqpoint{4.768000in}{2.142311in}}%
\pgfpathlineto{\pgfqpoint{4.768000in}{2.142311in}}%
\pgfusepath{fill}%
\end{pgfscope}%
\begin{pgfscope}%
\pgfpathrectangle{\pgfqpoint{0.800000in}{0.528000in}}{\pgfqpoint{3.968000in}{3.696000in}}%
\pgfusepath{clip}%
\pgfsetbuttcap%
\pgfsetroundjoin%
\definecolor{currentfill}{rgb}{0.206756,0.371758,0.553117}%
\pgfsetfillcolor{currentfill}%
\pgfsetlinewidth{0.000000pt}%
\definecolor{currentstroke}{rgb}{0.000000,0.000000,0.000000}%
\pgfsetstrokecolor{currentstroke}%
\pgfsetdash{}{0pt}%
\pgfpathmoveto{\pgfqpoint{1.568421in}{0.528000in}}%
\pgfpathlineto{\pgfqpoint{1.280970in}{0.831331in}}%
\pgfpathlineto{\pgfqpoint{0.979443in}{1.162667in}}%
\pgfpathlineto{\pgfqpoint{0.840081in}{1.320612in}}%
\pgfpathlineto{\pgfqpoint{0.800000in}{1.366676in}}%
\pgfpathlineto{\pgfqpoint{0.800000in}{1.361936in}}%
\pgfpathlineto{\pgfqpoint{1.075782in}{1.050667in}}%
\pgfpathlineto{\pgfqpoint{1.212076in}{0.901333in}}%
\pgfpathlineto{\pgfqpoint{1.528226in}{0.565333in}}%
\pgfpathlineto{\pgfqpoint{1.564166in}{0.528000in}}%
\pgfpathmoveto{\pgfqpoint{4.768000in}{2.151496in}}%
\pgfpathlineto{\pgfqpoint{4.476780in}{2.469333in}}%
\pgfpathlineto{\pgfqpoint{4.155895in}{2.805333in}}%
\pgfpathlineto{\pgfqpoint{3.846141in}{3.116547in}}%
\pgfpathlineto{\pgfqpoint{3.685818in}{3.272789in}}%
\pgfpathlineto{\pgfqpoint{3.357803in}{3.582470in}}%
\pgfpathlineto{\pgfqpoint{3.277125in}{3.656656in}}%
\pgfpathlineto{\pgfqpoint{3.204848in}{3.722580in}}%
\pgfpathlineto{\pgfqpoint{3.044525in}{3.866330in}}%
\pgfpathlineto{\pgfqpoint{2.884202in}{4.007053in}}%
\pgfpathlineto{\pgfqpoint{2.629825in}{4.224000in}}%
\pgfpathlineto{\pgfqpoint{2.624820in}{4.224000in}}%
\pgfpathlineto{\pgfqpoint{2.763960in}{4.106406in}}%
\pgfpathlineto{\pgfqpoint{3.099342in}{3.813333in}}%
\pgfpathlineto{\pgfqpoint{3.405253in}{3.534370in}}%
\pgfpathlineto{\pgfqpoint{3.725899in}{3.229784in}}%
\pgfpathlineto{\pgfqpoint{3.892329in}{3.066667in}}%
\pgfpathlineto{\pgfqpoint{4.046545in}{2.912343in}}%
\pgfpathlineto{\pgfqpoint{4.206869in}{2.748534in}}%
\pgfpathlineto{\pgfqpoint{4.367208in}{2.581333in}}%
\pgfpathlineto{\pgfqpoint{4.679112in}{2.245333in}}%
\pgfpathlineto{\pgfqpoint{4.768000in}{2.146904in}}%
\pgfpathlineto{\pgfqpoint{4.768000in}{2.146904in}}%
\pgfusepath{fill}%
\end{pgfscope}%
\begin{pgfscope}%
\pgfpathrectangle{\pgfqpoint{0.800000in}{0.528000in}}{\pgfqpoint{3.968000in}{3.696000in}}%
\pgfusepath{clip}%
\pgfsetbuttcap%
\pgfsetroundjoin%
\definecolor{currentfill}{rgb}{0.206756,0.371758,0.553117}%
\pgfsetfillcolor{currentfill}%
\pgfsetlinewidth{0.000000pt}%
\definecolor{currentstroke}{rgb}{0.000000,0.000000,0.000000}%
\pgfsetstrokecolor{currentstroke}%
\pgfsetdash{}{0pt}%
\pgfpathmoveto{\pgfqpoint{1.564166in}{0.528000in}}%
\pgfpathlineto{\pgfqpoint{1.280970in}{0.826817in}}%
\pgfpathlineto{\pgfqpoint{0.975315in}{1.162667in}}%
\pgfpathlineto{\pgfqpoint{0.840081in}{1.315878in}}%
\pgfpathlineto{\pgfqpoint{0.800000in}{1.361936in}}%
\pgfpathlineto{\pgfqpoint{0.800000in}{1.357196in}}%
\pgfpathlineto{\pgfqpoint{1.057553in}{1.066565in}}%
\pgfpathlineto{\pgfqpoint{1.130918in}{0.985568in}}%
\pgfpathlineto{\pgfqpoint{1.207910in}{0.901333in}}%
\pgfpathlineto{\pgfqpoint{1.523987in}{0.565333in}}%
\pgfpathlineto{\pgfqpoint{1.561535in}{0.528000in}}%
\pgfpathmoveto{\pgfqpoint{4.768000in}{2.156089in}}%
\pgfpathlineto{\pgfqpoint{4.480967in}{2.469333in}}%
\pgfpathlineto{\pgfqpoint{4.160154in}{2.805333in}}%
\pgfpathlineto{\pgfqpoint{3.846141in}{3.120805in}}%
\pgfpathlineto{\pgfqpoint{3.678565in}{3.283911in}}%
\pgfpathlineto{\pgfqpoint{3.593705in}{3.365333in}}%
\pgfpathlineto{\pgfqpoint{3.273800in}{3.664000in}}%
\pgfpathlineto{\pgfqpoint{3.108682in}{3.813333in}}%
\pgfpathlineto{\pgfqpoint{2.964364in}{3.941229in}}%
\pgfpathlineto{\pgfqpoint{2.804040in}{4.080436in}}%
\pgfpathlineto{\pgfqpoint{2.634831in}{4.224000in}}%
\pgfpathlineto{\pgfqpoint{2.629825in}{4.224000in}}%
\pgfpathlineto{\pgfqpoint{2.763960in}{4.110611in}}%
\pgfpathlineto{\pgfqpoint{3.084606in}{3.830678in}}%
\pgfpathlineto{\pgfqpoint{3.244929in}{3.686166in}}%
\pgfpathlineto{\pgfqpoint{3.405253in}{3.538579in}}%
\pgfpathlineto{\pgfqpoint{3.725899in}{3.234029in}}%
\pgfpathlineto{\pgfqpoint{3.896630in}{3.066667in}}%
\pgfpathlineto{\pgfqpoint{4.046545in}{2.916695in}}%
\pgfpathlineto{\pgfqpoint{4.206869in}{2.752905in}}%
\pgfpathlineto{\pgfqpoint{4.371369in}{2.581333in}}%
\pgfpathlineto{\pgfqpoint{4.687838in}{2.240312in}}%
\pgfpathlineto{\pgfqpoint{4.768000in}{2.151496in}}%
\pgfpathlineto{\pgfqpoint{4.768000in}{2.151496in}}%
\pgfusepath{fill}%
\end{pgfscope}%
\begin{pgfscope}%
\pgfpathrectangle{\pgfqpoint{0.800000in}{0.528000in}}{\pgfqpoint{3.968000in}{3.696000in}}%
\pgfusepath{clip}%
\pgfsetbuttcap%
\pgfsetroundjoin%
\definecolor{currentfill}{rgb}{0.206756,0.371758,0.553117}%
\pgfsetfillcolor{currentfill}%
\pgfsetlinewidth{0.000000pt}%
\definecolor{currentstroke}{rgb}{0.000000,0.000000,0.000000}%
\pgfsetstrokecolor{currentstroke}%
\pgfsetdash{}{0pt}%
\pgfpathmoveto{\pgfqpoint{1.559939in}{0.528000in}}%
\pgfpathlineto{\pgfqpoint{1.240889in}{0.865563in}}%
\pgfpathlineto{\pgfqpoint{0.938023in}{1.200000in}}%
\pgfpathlineto{\pgfqpoint{0.800000in}{1.357196in}}%
\pgfpathlineto{\pgfqpoint{0.800000in}{1.352457in}}%
\pgfpathlineto{\pgfqpoint{1.055308in}{1.064474in}}%
\pgfpathlineto{\pgfqpoint{1.135368in}{0.976000in}}%
\pgfpathlineto{\pgfqpoint{1.280970in}{0.817938in}}%
\pgfpathlineto{\pgfqpoint{1.448422in}{0.640000in}}%
\pgfpathlineto{\pgfqpoint{1.555759in}{0.528000in}}%
\pgfpathmoveto{\pgfqpoint{4.768000in}{2.160682in}}%
\pgfpathlineto{\pgfqpoint{4.485154in}{2.469333in}}%
\pgfpathlineto{\pgfqpoint{4.164413in}{2.805333in}}%
\pgfpathlineto{\pgfqpoint{3.846141in}{3.125064in}}%
\pgfpathlineto{\pgfqpoint{3.676057in}{3.290667in}}%
\pgfpathlineto{\pgfqpoint{3.519445in}{3.440000in}}%
\pgfpathlineto{\pgfqpoint{3.359587in}{3.589333in}}%
\pgfpathlineto{\pgfqpoint{3.029465in}{3.888000in}}%
\pgfpathlineto{\pgfqpoint{2.871884in}{4.025860in}}%
\pgfpathlineto{\pgfqpoint{2.788589in}{4.097608in}}%
\pgfpathlineto{\pgfqpoint{2.723879in}{4.153059in}}%
\pgfpathlineto{\pgfqpoint{2.639836in}{4.224000in}}%
\pgfpathlineto{\pgfqpoint{2.634831in}{4.224000in}}%
\pgfpathlineto{\pgfqpoint{2.767199in}{4.112000in}}%
\pgfpathlineto{\pgfqpoint{3.084606in}{3.834852in}}%
\pgfpathlineto{\pgfqpoint{3.244929in}{3.690357in}}%
\pgfpathlineto{\pgfqpoint{3.405253in}{3.542789in}}%
\pgfpathlineto{\pgfqpoint{3.725899in}{3.238274in}}%
\pgfpathlineto{\pgfqpoint{3.893750in}{3.073679in}}%
\pgfpathlineto{\pgfqpoint{3.975895in}{2.992000in}}%
\pgfpathlineto{\pgfqpoint{4.287030in}{2.674128in}}%
\pgfpathlineto{\pgfqpoint{4.584874in}{2.357333in}}%
\pgfpathlineto{\pgfqpoint{4.727919in}{2.200606in}}%
\pgfpathlineto{\pgfqpoint{4.768000in}{2.156089in}}%
\pgfpathlineto{\pgfqpoint{4.768000in}{2.156089in}}%
\pgfusepath{fill}%
\end{pgfscope}%
\begin{pgfscope}%
\pgfpathrectangle{\pgfqpoint{0.800000in}{0.528000in}}{\pgfqpoint{3.968000in}{3.696000in}}%
\pgfusepath{clip}%
\pgfsetbuttcap%
\pgfsetroundjoin%
\definecolor{currentfill}{rgb}{0.206756,0.371758,0.553117}%
\pgfsetfillcolor{currentfill}%
\pgfsetlinewidth{0.000000pt}%
\definecolor{currentstroke}{rgb}{0.000000,0.000000,0.000000}%
\pgfsetstrokecolor{currentstroke}%
\pgfsetdash{}{0pt}%
\pgfpathmoveto{\pgfqpoint{0.800000in}{1.347745in}}%
\pgfpathlineto{\pgfqpoint{0.962932in}{1.162667in}}%
\pgfpathlineto{\pgfqpoint{1.097281in}{1.013333in}}%
\pgfpathlineto{\pgfqpoint{1.240889in}{0.856650in}}%
\pgfpathlineto{\pgfqpoint{1.551579in}{0.528000in}}%
\pgfpathlineto{\pgfqpoint{1.555759in}{0.528000in}}%
\pgfpathlineto{\pgfqpoint{1.238200in}{0.864000in}}%
\pgfpathlineto{\pgfqpoint{0.933910in}{1.200000in}}%
\pgfpathlineto{\pgfqpoint{0.800000in}{1.352457in}}%
\pgfpathlineto{\pgfqpoint{0.800000in}{1.349333in}}%
\pgfpathmoveto{\pgfqpoint{4.768000in}{2.165275in}}%
\pgfpathlineto{\pgfqpoint{4.487434in}{2.471338in}}%
\pgfpathlineto{\pgfqpoint{4.166788in}{2.807234in}}%
\pgfpathlineto{\pgfqpoint{3.833916in}{3.141333in}}%
\pgfpathlineto{\pgfqpoint{3.680462in}{3.290667in}}%
\pgfpathlineto{\pgfqpoint{3.523922in}{3.440000in}}%
\pgfpathlineto{\pgfqpoint{3.364140in}{3.589333in}}%
\pgfpathlineto{\pgfqpoint{3.034176in}{3.888000in}}%
\pgfpathlineto{\pgfqpoint{2.863626in}{4.037333in}}%
\pgfpathlineto{\pgfqpoint{2.723879in}{4.157194in}}%
\pgfpathlineto{\pgfqpoint{2.643717in}{4.224000in}}%
\pgfpathlineto{\pgfqpoint{2.639836in}{4.224000in}}%
\pgfpathlineto{\pgfqpoint{2.772036in}{4.112000in}}%
\pgfpathlineto{\pgfqpoint{3.084606in}{3.839026in}}%
\pgfpathlineto{\pgfqpoint{3.244929in}{3.694549in}}%
\pgfpathlineto{\pgfqpoint{3.405253in}{3.546998in}}%
\pgfpathlineto{\pgfqpoint{3.725899in}{3.242519in}}%
\pgfpathlineto{\pgfqpoint{3.886222in}{3.085510in}}%
\pgfpathlineto{\pgfqpoint{4.206869in}{2.761647in}}%
\pgfpathlineto{\pgfqpoint{4.379692in}{2.581333in}}%
\pgfpathlineto{\pgfqpoint{4.691538in}{2.245333in}}%
\pgfpathlineto{\pgfqpoint{4.768000in}{2.160682in}}%
\pgfpathlineto{\pgfqpoint{4.768000in}{2.160682in}}%
\pgfusepath{fill}%
\end{pgfscope}%
\begin{pgfscope}%
\pgfpathrectangle{\pgfqpoint{0.800000in}{0.528000in}}{\pgfqpoint{3.968000in}{3.696000in}}%
\pgfusepath{clip}%
\pgfsetbuttcap%
\pgfsetroundjoin%
\definecolor{currentfill}{rgb}{0.204903,0.375746,0.553533}%
\pgfsetfillcolor{currentfill}%
\pgfsetlinewidth{0.000000pt}%
\definecolor{currentstroke}{rgb}{0.000000,0.000000,0.000000}%
\pgfsetstrokecolor{currentstroke}%
\pgfsetdash{}{0pt}%
\pgfpathmoveto{\pgfqpoint{0.800000in}{1.343089in}}%
\pgfpathlineto{\pgfqpoint{0.960323in}{1.160990in}}%
\pgfpathlineto{\pgfqpoint{1.127099in}{0.976000in}}%
\pgfpathlineto{\pgfqpoint{1.280970in}{0.809062in}}%
\pgfpathlineto{\pgfqpoint{1.441293in}{0.638675in}}%
\pgfpathlineto{\pgfqpoint{1.547399in}{0.528000in}}%
\pgfpathlineto{\pgfqpoint{1.551579in}{0.528000in}}%
\pgfpathlineto{\pgfqpoint{1.234090in}{0.864000in}}%
\pgfpathlineto{\pgfqpoint{0.925516in}{1.204912in}}%
\pgfpathlineto{\pgfqpoint{0.863940in}{1.274667in}}%
\pgfpathlineto{\pgfqpoint{0.800000in}{1.347745in}}%
\pgfpathmoveto{\pgfqpoint{4.768000in}{2.169868in}}%
\pgfpathlineto{\pgfqpoint{4.493421in}{2.469333in}}%
\pgfpathlineto{\pgfqpoint{4.352570in}{2.618667in}}%
\pgfpathlineto{\pgfqpoint{4.206869in}{2.770350in}}%
\pgfpathlineto{\pgfqpoint{4.046545in}{2.933831in}}%
\pgfpathlineto{\pgfqpoint{3.723496in}{3.253333in}}%
\pgfpathlineto{\pgfqpoint{3.565576in}{3.404755in}}%
\pgfpathlineto{\pgfqpoint{3.244929in}{3.702907in}}%
\pgfpathlineto{\pgfqpoint{3.080906in}{3.850667in}}%
\pgfpathlineto{\pgfqpoint{2.911401in}{4.000000in}}%
\pgfpathlineto{\pgfqpoint{2.763960in}{4.127191in}}%
\pgfpathlineto{\pgfqpoint{2.649720in}{4.224000in}}%
\pgfpathlineto{\pgfqpoint{2.644818in}{4.224000in}}%
\pgfpathlineto{\pgfqpoint{2.964364in}{3.949551in}}%
\pgfpathlineto{\pgfqpoint{3.285010in}{3.662147in}}%
\pgfpathlineto{\pgfqpoint{3.445333in}{3.513839in}}%
\pgfpathlineto{\pgfqpoint{3.765980in}{3.207817in}}%
\pgfpathlineto{\pgfqpoint{3.926303in}{3.050022in}}%
\pgfpathlineto{\pgfqpoint{4.095465in}{2.880000in}}%
\pgfpathlineto{\pgfqpoint{4.419159in}{2.544000in}}%
\pgfpathlineto{\pgfqpoint{4.729502in}{2.208000in}}%
\pgfpathlineto{\pgfqpoint{4.768000in}{2.165275in}}%
\pgfpathlineto{\pgfqpoint{4.768000in}{2.165275in}}%
\pgfusepath{fill}%
\end{pgfscope}%
\begin{pgfscope}%
\pgfpathrectangle{\pgfqpoint{0.800000in}{0.528000in}}{\pgfqpoint{3.968000in}{3.696000in}}%
\pgfusepath{clip}%
\pgfsetbuttcap%
\pgfsetroundjoin%
\definecolor{currentfill}{rgb}{0.204903,0.375746,0.553533}%
\pgfsetfillcolor{currentfill}%
\pgfsetlinewidth{0.000000pt}%
\definecolor{currentstroke}{rgb}{0.000000,0.000000,0.000000}%
\pgfsetstrokecolor{currentstroke}%
\pgfsetdash{}{0pt}%
\pgfpathmoveto{\pgfqpoint{0.800000in}{1.338432in}}%
\pgfpathlineto{\pgfqpoint{0.960323in}{1.156435in}}%
\pgfpathlineto{\pgfqpoint{1.122964in}{0.976000in}}%
\pgfpathlineto{\pgfqpoint{1.260478in}{0.826667in}}%
\pgfpathlineto{\pgfqpoint{1.401212in}{0.676521in}}%
\pgfpathlineto{\pgfqpoint{1.543218in}{0.528000in}}%
\pgfpathlineto{\pgfqpoint{1.547399in}{0.528000in}}%
\pgfpathlineto{\pgfqpoint{1.229980in}{0.864000in}}%
\pgfpathlineto{\pgfqpoint{0.920242in}{1.206142in}}%
\pgfpathlineto{\pgfqpoint{0.800000in}{1.343089in}}%
\pgfpathmoveto{\pgfqpoint{4.768000in}{2.174396in}}%
\pgfpathlineto{\pgfqpoint{4.601429in}{2.357333in}}%
\pgfpathlineto{\pgfqpoint{4.285412in}{2.693333in}}%
\pgfpathlineto{\pgfqpoint{3.955635in}{3.029333in}}%
\pgfpathlineto{\pgfqpoint{3.804538in}{3.178667in}}%
\pgfpathlineto{\pgfqpoint{3.645737in}{3.332460in}}%
\pgfpathlineto{\pgfqpoint{3.485414in}{3.484596in}}%
\pgfpathlineto{\pgfqpoint{3.164768in}{3.779660in}}%
\pgfpathlineto{\pgfqpoint{3.001362in}{3.925333in}}%
\pgfpathlineto{\pgfqpoint{2.683798in}{4.199414in}}%
\pgfpathlineto{\pgfqpoint{2.654623in}{4.224000in}}%
\pgfpathlineto{\pgfqpoint{2.649720in}{4.224000in}}%
\pgfpathlineto{\pgfqpoint{2.964364in}{3.953712in}}%
\pgfpathlineto{\pgfqpoint{3.287525in}{3.664000in}}%
\pgfpathlineto{\pgfqpoint{3.448894in}{3.514667in}}%
\pgfpathlineto{\pgfqpoint{3.765980in}{3.212066in}}%
\pgfpathlineto{\pgfqpoint{3.926303in}{3.054290in}}%
\pgfpathlineto{\pgfqpoint{4.099681in}{2.880000in}}%
\pgfpathlineto{\pgfqpoint{4.407273in}{2.560996in}}%
\pgfpathlineto{\pgfqpoint{4.567596in}{2.389541in}}%
\pgfpathlineto{\pgfqpoint{4.768000in}{2.169868in}}%
\pgfpathlineto{\pgfqpoint{4.768000in}{2.170667in}}%
\pgfusepath{fill}%
\end{pgfscope}%
\begin{pgfscope}%
\pgfpathrectangle{\pgfqpoint{0.800000in}{0.528000in}}{\pgfqpoint{3.968000in}{3.696000in}}%
\pgfusepath{clip}%
\pgfsetbuttcap%
\pgfsetroundjoin%
\definecolor{currentfill}{rgb}{0.204903,0.375746,0.553533}%
\pgfsetfillcolor{currentfill}%
\pgfsetlinewidth{0.000000pt}%
\definecolor{currentstroke}{rgb}{0.000000,0.000000,0.000000}%
\pgfsetstrokecolor{currentstroke}%
\pgfsetdash{}{0pt}%
\pgfpathmoveto{\pgfqpoint{0.800000in}{1.333775in}}%
\pgfpathlineto{\pgfqpoint{0.950715in}{1.162667in}}%
\pgfpathlineto{\pgfqpoint{1.084924in}{1.013333in}}%
\pgfpathlineto{\pgfqpoint{1.230687in}{0.854498in}}%
\pgfpathlineto{\pgfqpoint{1.305149in}{0.774522in}}%
\pgfpathlineto{\pgfqpoint{1.379984in}{0.694893in}}%
\pgfpathlineto{\pgfqpoint{1.441293in}{0.629983in}}%
\pgfpathlineto{\pgfqpoint{1.539038in}{0.528000in}}%
\pgfpathlineto{\pgfqpoint{1.543218in}{0.528000in}}%
\pgfpathlineto{\pgfqpoint{1.232879in}{0.856539in}}%
\pgfpathlineto{\pgfqpoint{1.157094in}{0.938667in}}%
\pgfpathlineto{\pgfqpoint{1.021627in}{1.088000in}}%
\pgfpathlineto{\pgfqpoint{0.880162in}{1.246911in}}%
\pgfpathlineto{\pgfqpoint{0.800000in}{1.338432in}}%
\pgfpathmoveto{\pgfqpoint{4.768000in}{2.178911in}}%
\pgfpathlineto{\pgfqpoint{4.605568in}{2.357333in}}%
\pgfpathlineto{\pgfqpoint{4.287030in}{2.695986in}}%
\pgfpathlineto{\pgfqpoint{3.959918in}{3.029333in}}%
\pgfpathlineto{\pgfqpoint{3.806061in}{3.181388in}}%
\pgfpathlineto{\pgfqpoint{3.645737in}{3.336629in}}%
\pgfpathlineto{\pgfqpoint{3.485414in}{3.488748in}}%
\pgfpathlineto{\pgfqpoint{3.164768in}{3.783777in}}%
\pgfpathlineto{\pgfqpoint{3.004444in}{3.926763in}}%
\pgfpathlineto{\pgfqpoint{2.683798in}{4.203545in}}%
\pgfpathlineto{\pgfqpoint{2.659525in}{4.224000in}}%
\pgfpathlineto{\pgfqpoint{2.654623in}{4.224000in}}%
\pgfpathlineto{\pgfqpoint{2.964364in}{3.957873in}}%
\pgfpathlineto{\pgfqpoint{3.292029in}{3.664000in}}%
\pgfpathlineto{\pgfqpoint{3.453325in}{3.514667in}}%
\pgfpathlineto{\pgfqpoint{3.773004in}{3.209457in}}%
\pgfpathlineto{\pgfqpoint{3.926303in}{3.058557in}}%
\pgfpathlineto{\pgfqpoint{4.086626in}{2.897555in}}%
\pgfpathlineto{\pgfqpoint{4.249458in}{2.730667in}}%
\pgfpathlineto{\pgfqpoint{4.567596in}{2.394030in}}%
\pgfpathlineto{\pgfqpoint{4.768000in}{2.174396in}}%
\pgfpathlineto{\pgfqpoint{4.768000in}{2.174396in}}%
\pgfusepath{fill}%
\end{pgfscope}%
\begin{pgfscope}%
\pgfpathrectangle{\pgfqpoint{0.800000in}{0.528000in}}{\pgfqpoint{3.968000in}{3.696000in}}%
\pgfusepath{clip}%
\pgfsetbuttcap%
\pgfsetroundjoin%
\definecolor{currentfill}{rgb}{0.203063,0.379716,0.553925}%
\pgfsetfillcolor{currentfill}%
\pgfsetlinewidth{0.000000pt}%
\definecolor{currentstroke}{rgb}{0.000000,0.000000,0.000000}%
\pgfsetstrokecolor{currentstroke}%
\pgfsetdash{}{0pt}%
\pgfpathmoveto{\pgfqpoint{0.800000in}{1.329119in}}%
\pgfpathlineto{\pgfqpoint{0.946657in}{1.162667in}}%
\pgfpathlineto{\pgfqpoint{1.082118in}{1.011887in}}%
\pgfpathlineto{\pgfqpoint{1.252227in}{0.826667in}}%
\pgfpathlineto{\pgfqpoint{1.401212in}{0.667819in}}%
\pgfpathlineto{\pgfqpoint{1.534858in}{0.528000in}}%
\pgfpathlineto{\pgfqpoint{1.539038in}{0.528000in}}%
\pgfpathlineto{\pgfqpoint{1.240889in}{0.843320in}}%
\pgfpathlineto{\pgfqpoint{0.950715in}{1.162667in}}%
\pgfpathlineto{\pgfqpoint{0.800000in}{1.333775in}}%
\pgfpathmoveto{\pgfqpoint{4.768000in}{2.183425in}}%
\pgfpathlineto{\pgfqpoint{4.607677in}{2.359500in}}%
\pgfpathlineto{\pgfqpoint{4.287030in}{2.700295in}}%
\pgfpathlineto{\pgfqpoint{3.964202in}{3.029333in}}%
\pgfpathlineto{\pgfqpoint{3.806061in}{3.185575in}}%
\pgfpathlineto{\pgfqpoint{3.645737in}{3.340798in}}%
\pgfpathlineto{\pgfqpoint{3.485414in}{3.492900in}}%
\pgfpathlineto{\pgfqpoint{3.164768in}{3.787895in}}%
\pgfpathlineto{\pgfqpoint{3.004444in}{3.930864in}}%
\pgfpathlineto{\pgfqpoint{2.674526in}{4.215364in}}%
\pgfpathlineto{\pgfqpoint{2.664427in}{4.224000in}}%
\pgfpathlineto{\pgfqpoint{2.659525in}{4.224000in}}%
\pgfpathlineto{\pgfqpoint{2.964364in}{3.962034in}}%
\pgfpathlineto{\pgfqpoint{3.296533in}{3.664000in}}%
\pgfpathlineto{\pgfqpoint{3.457755in}{3.514667in}}%
\pgfpathlineto{\pgfqpoint{3.770590in}{3.216000in}}%
\pgfpathlineto{\pgfqpoint{3.926303in}{3.062825in}}%
\pgfpathlineto{\pgfqpoint{4.086626in}{2.901841in}}%
\pgfpathlineto{\pgfqpoint{4.253611in}{2.730667in}}%
\pgfpathlineto{\pgfqpoint{4.571100in}{2.394667in}}%
\pgfpathlineto{\pgfqpoint{4.768000in}{2.178911in}}%
\pgfpathlineto{\pgfqpoint{4.768000in}{2.178911in}}%
\pgfusepath{fill}%
\end{pgfscope}%
\begin{pgfscope}%
\pgfpathrectangle{\pgfqpoint{0.800000in}{0.528000in}}{\pgfqpoint{3.968000in}{3.696000in}}%
\pgfusepath{clip}%
\pgfsetbuttcap%
\pgfsetroundjoin%
\definecolor{currentfill}{rgb}{0.203063,0.379716,0.553925}%
\pgfsetfillcolor{currentfill}%
\pgfsetlinewidth{0.000000pt}%
\definecolor{currentstroke}{rgb}{0.000000,0.000000,0.000000}%
\pgfsetstrokecolor{currentstroke}%
\pgfsetdash{}{0pt}%
\pgfpathmoveto{\pgfqpoint{0.800000in}{1.324462in}}%
\pgfpathlineto{\pgfqpoint{0.942600in}{1.162667in}}%
\pgfpathlineto{\pgfqpoint{1.080566in}{1.009129in}}%
\pgfpathlineto{\pgfqpoint{1.248101in}{0.826667in}}%
\pgfpathlineto{\pgfqpoint{1.401212in}{0.663468in}}%
\pgfpathlineto{\pgfqpoint{1.530678in}{0.528000in}}%
\pgfpathlineto{\pgfqpoint{1.534858in}{0.528000in}}%
\pgfpathlineto{\pgfqpoint{1.240889in}{0.838877in}}%
\pgfpathlineto{\pgfqpoint{0.946657in}{1.162667in}}%
\pgfpathlineto{\pgfqpoint{0.800000in}{1.329119in}}%
\pgfpathmoveto{\pgfqpoint{4.768000in}{2.187940in}}%
\pgfpathlineto{\pgfqpoint{4.607677in}{2.363920in}}%
\pgfpathlineto{\pgfqpoint{4.297850in}{2.693333in}}%
\pgfpathlineto{\pgfqpoint{4.153139in}{2.842667in}}%
\pgfpathlineto{\pgfqpoint{3.996381in}{3.001393in}}%
\pgfpathlineto{\pgfqpoint{3.846141in}{3.150468in}}%
\pgfpathlineto{\pgfqpoint{3.685818in}{3.306460in}}%
\pgfpathlineto{\pgfqpoint{3.525495in}{3.459322in}}%
\pgfpathlineto{\pgfqpoint{3.365172in}{3.609091in}}%
\pgfpathlineto{\pgfqpoint{3.044525in}{3.899507in}}%
\pgfpathlineto{\pgfqpoint{2.884202in}{4.040227in}}%
\pgfpathlineto{\pgfqpoint{2.713498in}{4.186667in}}%
\pgfpathlineto{\pgfqpoint{2.669329in}{4.224000in}}%
\pgfpathlineto{\pgfqpoint{2.664427in}{4.224000in}}%
\pgfpathlineto{\pgfqpoint{2.964364in}{3.966140in}}%
\pgfpathlineto{\pgfqpoint{3.136472in}{3.813333in}}%
\pgfpathlineto{\pgfqpoint{3.301038in}{3.664000in}}%
\pgfpathlineto{\pgfqpoint{3.462186in}{3.514667in}}%
\pgfpathlineto{\pgfqpoint{3.774880in}{3.216000in}}%
\pgfpathlineto{\pgfqpoint{3.932870in}{3.060550in}}%
\pgfpathlineto{\pgfqpoint{4.086626in}{2.906127in}}%
\pgfpathlineto{\pgfqpoint{4.257763in}{2.730667in}}%
\pgfpathlineto{\pgfqpoint{4.575184in}{2.394667in}}%
\pgfpathlineto{\pgfqpoint{4.768000in}{2.183425in}}%
\pgfpathlineto{\pgfqpoint{4.768000in}{2.183425in}}%
\pgfusepath{fill}%
\end{pgfscope}%
\begin{pgfscope}%
\pgfpathrectangle{\pgfqpoint{0.800000in}{0.528000in}}{\pgfqpoint{3.968000in}{3.696000in}}%
\pgfusepath{clip}%
\pgfsetbuttcap%
\pgfsetroundjoin%
\definecolor{currentfill}{rgb}{0.203063,0.379716,0.553925}%
\pgfsetfillcolor{currentfill}%
\pgfsetlinewidth{0.000000pt}%
\definecolor{currentstroke}{rgb}{0.000000,0.000000,0.000000}%
\pgfsetstrokecolor{currentstroke}%
\pgfsetdash{}{0pt}%
\pgfpathmoveto{\pgfqpoint{0.800000in}{1.319806in}}%
\pgfpathlineto{\pgfqpoint{0.938542in}{1.162667in}}%
\pgfpathlineto{\pgfqpoint{1.080566in}{1.004666in}}%
\pgfpathlineto{\pgfqpoint{1.243975in}{0.826667in}}%
\pgfpathlineto{\pgfqpoint{1.401212in}{0.659117in}}%
\pgfpathlineto{\pgfqpoint{1.526497in}{0.528000in}}%
\pgfpathlineto{\pgfqpoint{1.530678in}{0.528000in}}%
\pgfpathlineto{\pgfqpoint{1.240889in}{0.834434in}}%
\pgfpathlineto{\pgfqpoint{0.942600in}{1.162667in}}%
\pgfpathlineto{\pgfqpoint{0.800000in}{1.324462in}}%
\pgfpathmoveto{\pgfqpoint{4.768000in}{2.192454in}}%
\pgfpathlineto{\pgfqpoint{4.613092in}{2.362377in}}%
\pgfpathlineto{\pgfqpoint{4.548746in}{2.432000in}}%
\pgfpathlineto{\pgfqpoint{4.407273in}{2.582944in}}%
\pgfpathlineto{\pgfqpoint{4.084024in}{2.917333in}}%
\pgfpathlineto{\pgfqpoint{3.926303in}{3.075486in}}%
\pgfpathlineto{\pgfqpoint{3.597513in}{3.395082in}}%
\pgfpathlineto{\pgfqpoint{3.510784in}{3.477333in}}%
\pgfpathlineto{\pgfqpoint{3.350610in}{3.626667in}}%
\pgfpathlineto{\pgfqpoint{3.187065in}{3.776000in}}%
\pgfpathlineto{\pgfqpoint{2.884202in}{4.044316in}}%
\pgfpathlineto{\pgfqpoint{2.718378in}{4.186667in}}%
\pgfpathlineto{\pgfqpoint{2.674231in}{4.224000in}}%
\pgfpathlineto{\pgfqpoint{2.669329in}{4.224000in}}%
\pgfpathlineto{\pgfqpoint{2.947769in}{3.984543in}}%
\pgfpathlineto{\pgfqpoint{3.015340in}{3.925333in}}%
\pgfpathlineto{\pgfqpoint{3.182504in}{3.776000in}}%
\pgfpathlineto{\pgfqpoint{3.485414in}{3.497052in}}%
\pgfpathlineto{\pgfqpoint{3.645737in}{3.344968in}}%
\pgfpathlineto{\pgfqpoint{3.968448in}{3.029333in}}%
\pgfpathlineto{\pgfqpoint{4.126707in}{2.869665in}}%
\pgfpathlineto{\pgfqpoint{4.297850in}{2.693333in}}%
\pgfpathlineto{\pgfqpoint{4.613740in}{2.357333in}}%
\pgfpathlineto{\pgfqpoint{4.768000in}{2.187940in}}%
\pgfpathlineto{\pgfqpoint{4.768000in}{2.187940in}}%
\pgfusepath{fill}%
\end{pgfscope}%
\begin{pgfscope}%
\pgfpathrectangle{\pgfqpoint{0.800000in}{0.528000in}}{\pgfqpoint{3.968000in}{3.696000in}}%
\pgfusepath{clip}%
\pgfsetbuttcap%
\pgfsetroundjoin%
\definecolor{currentfill}{rgb}{0.203063,0.379716,0.553925}%
\pgfsetfillcolor{currentfill}%
\pgfsetlinewidth{0.000000pt}%
\definecolor{currentstroke}{rgb}{0.000000,0.000000,0.000000}%
\pgfsetstrokecolor{currentstroke}%
\pgfsetdash{}{0pt}%
\pgfpathmoveto{\pgfqpoint{0.800000in}{1.315149in}}%
\pgfpathlineto{\pgfqpoint{0.934485in}{1.162667in}}%
\pgfpathlineto{\pgfqpoint{1.080566in}{1.000203in}}%
\pgfpathlineto{\pgfqpoint{1.240889in}{0.825566in}}%
\pgfpathlineto{\pgfqpoint{1.415238in}{0.640000in}}%
\pgfpathlineto{\pgfqpoint{1.522317in}{0.528000in}}%
\pgfpathlineto{\pgfqpoint{1.526497in}{0.528000in}}%
\pgfpathlineto{\pgfqpoint{1.240889in}{0.829991in}}%
\pgfpathlineto{\pgfqpoint{0.938542in}{1.162667in}}%
\pgfpathlineto{\pgfqpoint{0.800000in}{1.319806in}}%
\pgfpathmoveto{\pgfqpoint{4.768000in}{2.196969in}}%
\pgfpathlineto{\pgfqpoint{4.621875in}{2.357333in}}%
\pgfpathlineto{\pgfqpoint{4.327111in}{2.671447in}}%
\pgfpathlineto{\pgfqpoint{4.006465in}{2.999723in}}%
\pgfpathlineto{\pgfqpoint{3.846141in}{3.158851in}}%
\pgfpathlineto{\pgfqpoint{3.672093in}{3.328000in}}%
\pgfpathlineto{\pgfqpoint{3.355096in}{3.626667in}}%
\pgfpathlineto{\pgfqpoint{3.191627in}{3.776000in}}%
\pgfpathlineto{\pgfqpoint{2.884202in}{4.048404in}}%
\pgfpathlineto{\pgfqpoint{2.723258in}{4.186667in}}%
\pgfpathlineto{\pgfqpoint{2.679133in}{4.224000in}}%
\pgfpathlineto{\pgfqpoint{2.674231in}{4.224000in}}%
\pgfpathlineto{\pgfqpoint{2.950053in}{3.986671in}}%
\pgfpathlineto{\pgfqpoint{3.019978in}{3.925333in}}%
\pgfpathlineto{\pgfqpoint{3.187065in}{3.776000in}}%
\pgfpathlineto{\pgfqpoint{3.485414in}{3.501204in}}%
\pgfpathlineto{\pgfqpoint{3.645737in}{3.349137in}}%
\pgfpathlineto{\pgfqpoint{3.972656in}{3.029333in}}%
\pgfpathlineto{\pgfqpoint{4.126707in}{2.873955in}}%
\pgfpathlineto{\pgfqpoint{4.301987in}{2.693333in}}%
\pgfpathlineto{\pgfqpoint{4.617807in}{2.357333in}}%
\pgfpathlineto{\pgfqpoint{4.768000in}{2.192454in}}%
\pgfpathlineto{\pgfqpoint{4.768000in}{2.192454in}}%
\pgfusepath{fill}%
\end{pgfscope}%
\begin{pgfscope}%
\pgfpathrectangle{\pgfqpoint{0.800000in}{0.528000in}}{\pgfqpoint{3.968000in}{3.696000in}}%
\pgfusepath{clip}%
\pgfsetbuttcap%
\pgfsetroundjoin%
\definecolor{currentfill}{rgb}{0.201239,0.383670,0.554294}%
\pgfsetfillcolor{currentfill}%
\pgfsetlinewidth{0.000000pt}%
\definecolor{currentstroke}{rgb}{0.000000,0.000000,0.000000}%
\pgfsetstrokecolor{currentstroke}%
\pgfsetdash{}{0pt}%
\pgfpathmoveto{\pgfqpoint{0.800000in}{1.310519in}}%
\pgfpathlineto{\pgfqpoint{0.963694in}{1.125333in}}%
\pgfpathlineto{\pgfqpoint{1.270569in}{0.789333in}}%
\pgfpathlineto{\pgfqpoint{1.411105in}{0.640000in}}%
\pgfpathlineto{\pgfqpoint{1.518194in}{0.528000in}}%
\pgfpathlineto{\pgfqpoint{1.522317in}{0.528000in}}%
\pgfpathlineto{\pgfqpoint{1.521455in}{0.528895in}}%
\pgfpathlineto{\pgfqpoint{1.361131in}{0.697147in}}%
\pgfpathlineto{\pgfqpoint{1.200808in}{0.868882in}}%
\pgfpathlineto{\pgfqpoint{1.034844in}{1.050667in}}%
\pgfpathlineto{\pgfqpoint{0.901339in}{1.200000in}}%
\pgfpathlineto{\pgfqpoint{0.800000in}{1.315149in}}%
\pgfpathlineto{\pgfqpoint{0.800000in}{1.312000in}}%
\pgfpathmoveto{\pgfqpoint{4.768000in}{2.201483in}}%
\pgfpathlineto{\pgfqpoint{4.625943in}{2.357333in}}%
\pgfpathlineto{\pgfqpoint{4.318146in}{2.684983in}}%
\pgfpathlineto{\pgfqpoint{4.238325in}{2.768000in}}%
\pgfpathlineto{\pgfqpoint{3.915761in}{3.094181in}}%
\pgfpathlineto{\pgfqpoint{3.830208in}{3.178667in}}%
\pgfpathlineto{\pgfqpoint{3.519610in}{3.477333in}}%
\pgfpathlineto{\pgfqpoint{3.359581in}{3.626667in}}%
\pgfpathlineto{\pgfqpoint{3.196188in}{3.776000in}}%
\pgfpathlineto{\pgfqpoint{2.884202in}{4.052492in}}%
\pgfpathlineto{\pgfqpoint{2.723879in}{4.190221in}}%
\pgfpathlineto{\pgfqpoint{2.683798in}{4.224000in}}%
\pgfpathlineto{\pgfqpoint{2.679133in}{4.224000in}}%
\pgfpathlineto{\pgfqpoint{2.939728in}{4.000000in}}%
\pgfpathlineto{\pgfqpoint{3.244929in}{3.727665in}}%
\pgfpathlineto{\pgfqpoint{3.565576in}{3.429719in}}%
\pgfpathlineto{\pgfqpoint{3.725899in}{3.276113in}}%
\pgfpathlineto{\pgfqpoint{4.051277in}{2.954667in}}%
\pgfpathlineto{\pgfqpoint{4.206869in}{2.796149in}}%
\pgfpathlineto{\pgfqpoint{4.522445in}{2.464610in}}%
\pgfpathlineto{\pgfqpoint{4.587433in}{2.394667in}}%
\pgfpathlineto{\pgfqpoint{4.768000in}{2.196969in}}%
\pgfpathlineto{\pgfqpoint{4.768000in}{2.196969in}}%
\pgfusepath{fill}%
\end{pgfscope}%
\begin{pgfscope}%
\pgfpathrectangle{\pgfqpoint{0.800000in}{0.528000in}}{\pgfqpoint{3.968000in}{3.696000in}}%
\pgfusepath{clip}%
\pgfsetbuttcap%
\pgfsetroundjoin%
\definecolor{currentfill}{rgb}{0.201239,0.383670,0.554294}%
\pgfsetfillcolor{currentfill}%
\pgfsetlinewidth{0.000000pt}%
\definecolor{currentstroke}{rgb}{0.000000,0.000000,0.000000}%
\pgfsetstrokecolor{currentstroke}%
\pgfsetdash{}{0pt}%
\pgfpathmoveto{\pgfqpoint{0.800000in}{1.305943in}}%
\pgfpathlineto{\pgfqpoint{0.960323in}{1.124562in}}%
\pgfpathlineto{\pgfqpoint{1.266499in}{0.789333in}}%
\pgfpathlineto{\pgfqpoint{1.406973in}{0.640000in}}%
\pgfpathlineto{\pgfqpoint{1.514086in}{0.528000in}}%
\pgfpathlineto{\pgfqpoint{1.518194in}{0.528000in}}%
\pgfpathlineto{\pgfqpoint{1.361131in}{0.692791in}}%
\pgfpathlineto{\pgfqpoint{1.200808in}{0.864433in}}%
\pgfpathlineto{\pgfqpoint{1.030810in}{1.050667in}}%
\pgfpathlineto{\pgfqpoint{0.880162in}{1.219352in}}%
\pgfpathlineto{\pgfqpoint{0.800000in}{1.310519in}}%
\pgfpathmoveto{\pgfqpoint{4.768000in}{2.205998in}}%
\pgfpathlineto{\pgfqpoint{4.630011in}{2.357333in}}%
\pgfpathlineto{\pgfqpoint{4.314396in}{2.693333in}}%
\pgfpathlineto{\pgfqpoint{4.166788in}{2.845839in}}%
\pgfpathlineto{\pgfqpoint{4.006465in}{3.008141in}}%
\pgfpathlineto{\pgfqpoint{3.834482in}{3.178667in}}%
\pgfpathlineto{\pgfqpoint{3.524023in}{3.477333in}}%
\pgfpathlineto{\pgfqpoint{3.364067in}{3.626667in}}%
\pgfpathlineto{\pgfqpoint{3.200749in}{3.776000in}}%
\pgfpathlineto{\pgfqpoint{2.884202in}{4.056580in}}%
\pgfpathlineto{\pgfqpoint{2.723879in}{4.194292in}}%
\pgfpathlineto{\pgfqpoint{2.688834in}{4.224000in}}%
\pgfpathlineto{\pgfqpoint{2.684031in}{4.224000in}}%
\pgfpathlineto{\pgfqpoint{3.004444in}{3.947267in}}%
\pgfpathlineto{\pgfqpoint{3.325091in}{3.658459in}}%
\pgfpathlineto{\pgfqpoint{3.485414in}{3.509508in}}%
\pgfpathlineto{\pgfqpoint{3.645737in}{3.357475in}}%
\pgfpathlineto{\pgfqpoint{3.973931in}{3.036363in}}%
\pgfpathlineto{\pgfqpoint{4.055453in}{2.954667in}}%
\pgfpathlineto{\pgfqpoint{4.206869in}{2.800448in}}%
\pgfpathlineto{\pgfqpoint{4.527515in}{2.463659in}}%
\pgfpathlineto{\pgfqpoint{4.694363in}{2.282667in}}%
\pgfpathlineto{\pgfqpoint{4.768000in}{2.201483in}}%
\pgfpathlineto{\pgfqpoint{4.768000in}{2.201483in}}%
\pgfusepath{fill}%
\end{pgfscope}%
\begin{pgfscope}%
\pgfpathrectangle{\pgfqpoint{0.800000in}{0.528000in}}{\pgfqpoint{3.968000in}{3.696000in}}%
\pgfusepath{clip}%
\pgfsetbuttcap%
\pgfsetroundjoin%
\definecolor{currentfill}{rgb}{0.201239,0.383670,0.554294}%
\pgfsetfillcolor{currentfill}%
\pgfsetlinewidth{0.000000pt}%
\definecolor{currentstroke}{rgb}{0.000000,0.000000,0.000000}%
\pgfsetstrokecolor{currentstroke}%
\pgfsetdash{}{0pt}%
\pgfpathmoveto{\pgfqpoint{0.800000in}{1.301367in}}%
\pgfpathlineto{\pgfqpoint{0.960323in}{1.120084in}}%
\pgfpathlineto{\pgfqpoint{1.262428in}{0.789333in}}%
\pgfpathlineto{\pgfqpoint{1.402840in}{0.640000in}}%
\pgfpathlineto{\pgfqpoint{1.509978in}{0.528000in}}%
\pgfpathlineto{\pgfqpoint{1.514086in}{0.528000in}}%
\pgfpathlineto{\pgfqpoint{1.361131in}{0.688436in}}%
\pgfpathlineto{\pgfqpoint{1.197162in}{0.864000in}}%
\pgfpathlineto{\pgfqpoint{1.060556in}{1.013333in}}%
\pgfpathlineto{\pgfqpoint{0.920242in}{1.169553in}}%
\pgfpathlineto{\pgfqpoint{0.800000in}{1.305943in}}%
\pgfpathmoveto{\pgfqpoint{4.768000in}{2.210470in}}%
\pgfpathlineto{\pgfqpoint{4.599683in}{2.394667in}}%
\pgfpathlineto{\pgfqpoint{4.447354in}{2.557844in}}%
\pgfpathlineto{\pgfqpoint{4.282678in}{2.730667in}}%
\pgfpathlineto{\pgfqpoint{3.952068in}{3.066667in}}%
\pgfpathlineto{\pgfqpoint{3.645737in}{3.365807in}}%
\pgfpathlineto{\pgfqpoint{3.485414in}{3.517763in}}%
\pgfpathlineto{\pgfqpoint{3.325091in}{3.666687in}}%
\pgfpathlineto{\pgfqpoint{3.163954in}{3.813333in}}%
\pgfpathlineto{\pgfqpoint{2.844121in}{4.095367in}}%
\pgfpathlineto{\pgfqpoint{2.693637in}{4.224000in}}%
\pgfpathlineto{\pgfqpoint{2.688834in}{4.224000in}}%
\pgfpathlineto{\pgfqpoint{3.004444in}{3.951368in}}%
\pgfpathlineto{\pgfqpoint{3.325091in}{3.662594in}}%
\pgfpathlineto{\pgfqpoint{3.485414in}{3.513660in}}%
\pgfpathlineto{\pgfqpoint{3.645737in}{3.361645in}}%
\pgfpathlineto{\pgfqpoint{3.976093in}{3.038377in}}%
\pgfpathlineto{\pgfqpoint{4.059628in}{2.954667in}}%
\pgfpathlineto{\pgfqpoint{4.206869in}{2.804748in}}%
\pgfpathlineto{\pgfqpoint{4.527515in}{2.468068in}}%
\pgfpathlineto{\pgfqpoint{4.698401in}{2.282667in}}%
\pgfpathlineto{\pgfqpoint{4.768000in}{2.205998in}}%
\pgfpathlineto{\pgfqpoint{4.768000in}{2.208000in}}%
\pgfpathlineto{\pgfqpoint{4.768000in}{2.208000in}}%
\pgfusepath{fill}%
\end{pgfscope}%
\begin{pgfscope}%
\pgfpathrectangle{\pgfqpoint{0.800000in}{0.528000in}}{\pgfqpoint{3.968000in}{3.696000in}}%
\pgfusepath{clip}%
\pgfsetbuttcap%
\pgfsetroundjoin%
\definecolor{currentfill}{rgb}{0.201239,0.383670,0.554294}%
\pgfsetfillcolor{currentfill}%
\pgfsetlinewidth{0.000000pt}%
\definecolor{currentstroke}{rgb}{0.000000,0.000000,0.000000}%
\pgfsetstrokecolor{currentstroke}%
\pgfsetdash{}{0pt}%
\pgfpathmoveto{\pgfqpoint{0.800000in}{1.296791in}}%
\pgfpathlineto{\pgfqpoint{0.960323in}{1.115606in}}%
\pgfpathlineto{\pgfqpoint{1.258358in}{0.789333in}}%
\pgfpathlineto{\pgfqpoint{1.401212in}{0.637406in}}%
\pgfpathlineto{\pgfqpoint{1.505870in}{0.528000in}}%
\pgfpathlineto{\pgfqpoint{1.509978in}{0.528000in}}%
\pgfpathlineto{\pgfqpoint{1.361131in}{0.684080in}}%
\pgfpathlineto{\pgfqpoint{1.193121in}{0.864000in}}%
\pgfpathlineto{\pgfqpoint{1.040485in}{1.031015in}}%
\pgfpathlineto{\pgfqpoint{0.800000in}{1.301367in}}%
\pgfpathmoveto{\pgfqpoint{4.768000in}{2.214909in}}%
\pgfpathlineto{\pgfqpoint{4.603766in}{2.394667in}}%
\pgfpathlineto{\pgfqpoint{4.447354in}{2.562172in}}%
\pgfpathlineto{\pgfqpoint{4.286830in}{2.730667in}}%
\pgfpathlineto{\pgfqpoint{3.956292in}{3.066667in}}%
\pgfpathlineto{\pgfqpoint{3.645737in}{3.369911in}}%
\pgfpathlineto{\pgfqpoint{3.485414in}{3.521851in}}%
\pgfpathlineto{\pgfqpoint{3.325091in}{3.670758in}}%
\pgfpathlineto{\pgfqpoint{3.164768in}{3.816668in}}%
\pgfpathlineto{\pgfqpoint{2.829556in}{4.112000in}}%
\pgfpathlineto{\pgfqpoint{2.698440in}{4.224000in}}%
\pgfpathlineto{\pgfqpoint{2.693637in}{4.224000in}}%
\pgfpathlineto{\pgfqpoint{3.004444in}{3.955469in}}%
\pgfpathlineto{\pgfqpoint{3.328008in}{3.664000in}}%
\pgfpathlineto{\pgfqpoint{3.488709in}{3.514667in}}%
\pgfpathlineto{\pgfqpoint{3.646231in}{3.365333in}}%
\pgfpathlineto{\pgfqpoint{3.966384in}{3.052418in}}%
\pgfpathlineto{\pgfqpoint{4.287030in}{2.726150in}}%
\pgfpathlineto{\pgfqpoint{4.460388in}{2.544000in}}%
\pgfpathlineto{\pgfqpoint{4.607677in}{2.386015in}}%
\pgfpathlineto{\pgfqpoint{4.768000in}{2.210470in}}%
\pgfpathlineto{\pgfqpoint{4.768000in}{2.210470in}}%
\pgfusepath{fill}%
\end{pgfscope}%
\begin{pgfscope}%
\pgfpathrectangle{\pgfqpoint{0.800000in}{0.528000in}}{\pgfqpoint{3.968000in}{3.696000in}}%
\pgfusepath{clip}%
\pgfsetbuttcap%
\pgfsetroundjoin%
\definecolor{currentfill}{rgb}{0.199430,0.387607,0.554642}%
\pgfsetfillcolor{currentfill}%
\pgfsetlinewidth{0.000000pt}%
\definecolor{currentstroke}{rgb}{0.000000,0.000000,0.000000}%
\pgfsetstrokecolor{currentstroke}%
\pgfsetdash{}{0pt}%
\pgfpathmoveto{\pgfqpoint{0.800000in}{1.292215in}}%
\pgfpathlineto{\pgfqpoint{0.947621in}{1.125333in}}%
\pgfpathlineto{\pgfqpoint{1.254287in}{0.789333in}}%
\pgfpathlineto{\pgfqpoint{1.401212in}{0.633125in}}%
\pgfpathlineto{\pgfqpoint{1.501762in}{0.528000in}}%
\pgfpathlineto{\pgfqpoint{1.505870in}{0.528000in}}%
\pgfpathlineto{\pgfqpoint{1.361131in}{0.679724in}}%
\pgfpathlineto{\pgfqpoint{1.189081in}{0.864000in}}%
\pgfpathlineto{\pgfqpoint{1.040485in}{1.026547in}}%
\pgfpathlineto{\pgfqpoint{0.800000in}{1.296791in}}%
\pgfpathmoveto{\pgfqpoint{4.768000in}{2.219348in}}%
\pgfpathlineto{\pgfqpoint{4.607677in}{2.394851in}}%
\pgfpathlineto{\pgfqpoint{4.433335in}{2.581333in}}%
\pgfpathlineto{\pgfqpoint{4.287030in}{2.734702in}}%
\pgfpathlineto{\pgfqpoint{3.960516in}{3.066667in}}%
\pgfpathlineto{\pgfqpoint{3.645737in}{3.374016in}}%
\pgfpathlineto{\pgfqpoint{3.485414in}{3.525939in}}%
\pgfpathlineto{\pgfqpoint{3.325091in}{3.674829in}}%
\pgfpathlineto{\pgfqpoint{3.164768in}{3.820722in}}%
\pgfpathlineto{\pgfqpoint{2.834296in}{4.112000in}}%
\pgfpathlineto{\pgfqpoint{2.703243in}{4.224000in}}%
\pgfpathlineto{\pgfqpoint{2.698440in}{4.224000in}}%
\pgfpathlineto{\pgfqpoint{3.004444in}{3.959570in}}%
\pgfpathlineto{\pgfqpoint{3.332429in}{3.664000in}}%
\pgfpathlineto{\pgfqpoint{3.493059in}{3.514667in}}%
\pgfpathlineto{\pgfqpoint{3.650512in}{3.365333in}}%
\pgfpathlineto{\pgfqpoint{3.966384in}{3.056622in}}%
\pgfpathlineto{\pgfqpoint{4.287030in}{2.730459in}}%
\pgfpathlineto{\pgfqpoint{4.456404in}{2.552430in}}%
\pgfpathlineto{\pgfqpoint{4.534443in}{2.469333in}}%
\pgfpathlineto{\pgfqpoint{4.687838in}{2.303110in}}%
\pgfpathlineto{\pgfqpoint{4.768000in}{2.214909in}}%
\pgfpathlineto{\pgfqpoint{4.768000in}{2.214909in}}%
\pgfusepath{fill}%
\end{pgfscope}%
\begin{pgfscope}%
\pgfpathrectangle{\pgfqpoint{0.800000in}{0.528000in}}{\pgfqpoint{3.968000in}{3.696000in}}%
\pgfusepath{clip}%
\pgfsetbuttcap%
\pgfsetroundjoin%
\definecolor{currentfill}{rgb}{0.199430,0.387607,0.554642}%
\pgfsetfillcolor{currentfill}%
\pgfsetlinewidth{0.000000pt}%
\definecolor{currentstroke}{rgb}{0.000000,0.000000,0.000000}%
\pgfsetstrokecolor{currentstroke}%
\pgfsetdash{}{0pt}%
\pgfpathmoveto{\pgfqpoint{0.800000in}{1.287639in}}%
\pgfpathlineto{\pgfqpoint{0.943617in}{1.125333in}}%
\pgfpathlineto{\pgfqpoint{1.250217in}{0.789333in}}%
\pgfpathlineto{\pgfqpoint{1.401212in}{0.628845in}}%
\pgfpathlineto{\pgfqpoint{1.497654in}{0.528000in}}%
\pgfpathlineto{\pgfqpoint{1.501762in}{0.528000in}}%
\pgfpathlineto{\pgfqpoint{1.346082in}{0.691351in}}%
\pgfpathlineto{\pgfqpoint{1.192392in}{0.856161in}}%
\pgfpathlineto{\pgfqpoint{1.116375in}{0.938667in}}%
\pgfpathlineto{\pgfqpoint{0.815387in}{1.274667in}}%
\pgfpathlineto{\pgfqpoint{0.800000in}{1.292215in}}%
\pgfpathmoveto{\pgfqpoint{4.768000in}{2.223786in}}%
\pgfpathlineto{\pgfqpoint{4.607677in}{2.399197in}}%
\pgfpathlineto{\pgfqpoint{4.437425in}{2.581333in}}%
\pgfpathlineto{\pgfqpoint{4.287030in}{2.738942in}}%
\pgfpathlineto{\pgfqpoint{3.964740in}{3.066667in}}%
\pgfpathlineto{\pgfqpoint{3.645737in}{3.378120in}}%
\pgfpathlineto{\pgfqpoint{3.485414in}{3.530027in}}%
\pgfpathlineto{\pgfqpoint{3.325091in}{3.678900in}}%
\pgfpathlineto{\pgfqpoint{3.164768in}{3.824777in}}%
\pgfpathlineto{\pgfqpoint{2.839037in}{4.112000in}}%
\pgfpathlineto{\pgfqpoint{2.708046in}{4.224000in}}%
\pgfpathlineto{\pgfqpoint{2.703243in}{4.224000in}}%
\pgfpathlineto{\pgfqpoint{3.005563in}{3.962667in}}%
\pgfpathlineto{\pgfqpoint{3.336849in}{3.664000in}}%
\pgfpathlineto{\pgfqpoint{3.497409in}{3.514667in}}%
\pgfpathlineto{\pgfqpoint{3.654793in}{3.365333in}}%
\pgfpathlineto{\pgfqpoint{3.966384in}{3.060826in}}%
\pgfpathlineto{\pgfqpoint{4.290914in}{2.730667in}}%
\pgfpathlineto{\pgfqpoint{4.447354in}{2.566500in}}%
\pgfpathlineto{\pgfqpoint{4.744450in}{2.245333in}}%
\pgfpathlineto{\pgfqpoint{4.768000in}{2.219348in}}%
\pgfpathlineto{\pgfqpoint{4.768000in}{2.219348in}}%
\pgfusepath{fill}%
\end{pgfscope}%
\begin{pgfscope}%
\pgfpathrectangle{\pgfqpoint{0.800000in}{0.528000in}}{\pgfqpoint{3.968000in}{3.696000in}}%
\pgfusepath{clip}%
\pgfsetbuttcap%
\pgfsetroundjoin%
\definecolor{currentfill}{rgb}{0.199430,0.387607,0.554642}%
\pgfsetfillcolor{currentfill}%
\pgfsetlinewidth{0.000000pt}%
\definecolor{currentstroke}{rgb}{0.000000,0.000000,0.000000}%
\pgfsetstrokecolor{currentstroke}%
\pgfsetdash{}{0pt}%
\pgfpathmoveto{\pgfqpoint{0.800000in}{1.283063in}}%
\pgfpathlineto{\pgfqpoint{0.939613in}{1.125333in}}%
\pgfpathlineto{\pgfqpoint{1.246146in}{0.789333in}}%
\pgfpathlineto{\pgfqpoint{1.401212in}{0.624564in}}%
\pgfpathlineto{\pgfqpoint{1.493546in}{0.528000in}}%
\pgfpathlineto{\pgfqpoint{1.497654in}{0.528000in}}%
\pgfpathlineto{\pgfqpoint{1.355259in}{0.677333in}}%
\pgfpathlineto{\pgfqpoint{1.200808in}{0.842553in}}%
\pgfpathlineto{\pgfqpoint{0.910309in}{1.162667in}}%
\pgfpathlineto{\pgfqpoint{0.800000in}{1.287639in}}%
\pgfpathmoveto{\pgfqpoint{4.768000in}{2.228225in}}%
\pgfpathlineto{\pgfqpoint{4.607677in}{2.403544in}}%
\pgfpathlineto{\pgfqpoint{4.441515in}{2.581333in}}%
\pgfpathlineto{\pgfqpoint{4.287030in}{2.743182in}}%
\pgfpathlineto{\pgfqpoint{3.966384in}{3.069195in}}%
\pgfpathlineto{\pgfqpoint{3.645737in}{3.382225in}}%
\pgfpathlineto{\pgfqpoint{3.485414in}{3.534115in}}%
\pgfpathlineto{\pgfqpoint{3.325091in}{3.682971in}}%
\pgfpathlineto{\pgfqpoint{3.164768in}{3.828832in}}%
\pgfpathlineto{\pgfqpoint{2.843777in}{4.112000in}}%
\pgfpathlineto{\pgfqpoint{2.712849in}{4.224000in}}%
\pgfpathlineto{\pgfqpoint{2.708046in}{4.224000in}}%
\pgfpathlineto{\pgfqpoint{3.004444in}{3.967694in}}%
\pgfpathlineto{\pgfqpoint{3.325091in}{3.678900in}}%
\pgfpathlineto{\pgfqpoint{3.493564in}{3.522258in}}%
\pgfpathlineto{\pgfqpoint{3.580803in}{3.440000in}}%
\pgfpathlineto{\pgfqpoint{3.736589in}{3.290667in}}%
\pgfpathlineto{\pgfqpoint{4.046545in}{2.984721in}}%
\pgfpathlineto{\pgfqpoint{4.367192in}{2.655372in}}%
\pgfpathlineto{\pgfqpoint{4.542532in}{2.469333in}}%
\pgfpathlineto{\pgfqpoint{4.687838in}{2.311968in}}%
\pgfpathlineto{\pgfqpoint{4.768000in}{2.223786in}}%
\pgfpathlineto{\pgfqpoint{4.768000in}{2.223786in}}%
\pgfusepath{fill}%
\end{pgfscope}%
\begin{pgfscope}%
\pgfpathrectangle{\pgfqpoint{0.800000in}{0.528000in}}{\pgfqpoint{3.968000in}{3.696000in}}%
\pgfusepath{clip}%
\pgfsetbuttcap%
\pgfsetroundjoin%
\definecolor{currentfill}{rgb}{0.199430,0.387607,0.554642}%
\pgfsetfillcolor{currentfill}%
\pgfsetlinewidth{0.000000pt}%
\definecolor{currentstroke}{rgb}{0.000000,0.000000,0.000000}%
\pgfsetstrokecolor{currentstroke}%
\pgfsetdash{}{0pt}%
\pgfpathmoveto{\pgfqpoint{0.800000in}{1.278487in}}%
\pgfpathlineto{\pgfqpoint{0.935609in}{1.125333in}}%
\pgfpathlineto{\pgfqpoint{1.249920in}{0.780922in}}%
\pgfpathlineto{\pgfqpoint{1.417993in}{0.602667in}}%
\pgfpathlineto{\pgfqpoint{1.489438in}{0.528000in}}%
\pgfpathlineto{\pgfqpoint{1.493546in}{0.528000in}}%
\pgfpathlineto{\pgfqpoint{1.351213in}{0.677333in}}%
\pgfpathlineto{\pgfqpoint{1.200808in}{0.838179in}}%
\pgfpathlineto{\pgfqpoint{0.906320in}{1.162667in}}%
\pgfpathlineto{\pgfqpoint{0.800000in}{1.283063in}}%
\pgfpathmoveto{\pgfqpoint{4.768000in}{2.232664in}}%
\pgfpathlineto{\pgfqpoint{4.614196in}{2.400739in}}%
\pgfpathlineto{\pgfqpoint{4.539793in}{2.480770in}}%
\pgfpathlineto{\pgfqpoint{4.465026in}{2.560461in}}%
\pgfpathlineto{\pgfqpoint{4.407273in}{2.621800in}}%
\pgfpathlineto{\pgfqpoint{4.230981in}{2.805333in}}%
\pgfpathlineto{\pgfqpoint{4.084679in}{2.954667in}}%
\pgfpathlineto{\pgfqpoint{3.755247in}{3.280670in}}%
\pgfpathlineto{\pgfqpoint{3.667636in}{3.365333in}}%
\pgfpathlineto{\pgfqpoint{3.350111in}{3.664000in}}%
\pgfpathlineto{\pgfqpoint{3.186438in}{3.813333in}}%
\pgfpathlineto{\pgfqpoint{3.032133in}{3.951123in}}%
\pgfpathlineto{\pgfqpoint{2.964364in}{4.011033in}}%
\pgfpathlineto{\pgfqpoint{2.804040in}{4.150267in}}%
\pgfpathlineto{\pgfqpoint{2.717652in}{4.224000in}}%
\pgfpathlineto{\pgfqpoint{2.712849in}{4.224000in}}%
\pgfpathlineto{\pgfqpoint{2.972324in}{4.000000in}}%
\pgfpathlineto{\pgfqpoint{3.140468in}{3.850667in}}%
\pgfpathlineto{\pgfqpoint{3.445333in}{3.571612in}}%
\pgfpathlineto{\pgfqpoint{3.779299in}{3.253333in}}%
\pgfpathlineto{\pgfqpoint{4.086626in}{2.948482in}}%
\pgfpathlineto{\pgfqpoint{4.263053in}{2.768000in}}%
\pgfpathlineto{\pgfqpoint{4.581298in}{2.432000in}}%
\pgfpathlineto{\pgfqpoint{4.727919in}{2.272421in}}%
\pgfpathlineto{\pgfqpoint{4.768000in}{2.228225in}}%
\pgfpathlineto{\pgfqpoint{4.768000in}{2.228225in}}%
\pgfusepath{fill}%
\end{pgfscope}%
\begin{pgfscope}%
\pgfpathrectangle{\pgfqpoint{0.800000in}{0.528000in}}{\pgfqpoint{3.968000in}{3.696000in}}%
\pgfusepath{clip}%
\pgfsetbuttcap%
\pgfsetroundjoin%
\definecolor{currentfill}{rgb}{0.197636,0.391528,0.554969}%
\pgfsetfillcolor{currentfill}%
\pgfsetlinewidth{0.000000pt}%
\definecolor{currentstroke}{rgb}{0.000000,0.000000,0.000000}%
\pgfsetstrokecolor{currentstroke}%
\pgfsetdash{}{0pt}%
\pgfpathmoveto{\pgfqpoint{0.800000in}{1.273924in}}%
\pgfpathlineto{\pgfqpoint{1.100333in}{0.938667in}}%
\pgfpathlineto{\pgfqpoint{1.240889in}{0.786288in}}%
\pgfpathlineto{\pgfqpoint{1.413916in}{0.602667in}}%
\pgfpathlineto{\pgfqpoint{1.485330in}{0.528000in}}%
\pgfpathlineto{\pgfqpoint{1.489438in}{0.528000in}}%
\pgfpathlineto{\pgfqpoint{1.347166in}{0.677333in}}%
\pgfpathlineto{\pgfqpoint{1.200808in}{0.833804in}}%
\pgfpathlineto{\pgfqpoint{0.902331in}{1.162667in}}%
\pgfpathlineto{\pgfqpoint{0.800000in}{1.278487in}}%
\pgfpathlineto{\pgfqpoint{0.800000in}{1.274667in}}%
\pgfpathmoveto{\pgfqpoint{4.768000in}{2.237103in}}%
\pgfpathlineto{\pgfqpoint{4.623904in}{2.394667in}}%
\pgfpathlineto{\pgfqpoint{4.327111in}{2.709999in}}%
\pgfpathlineto{\pgfqpoint{4.162303in}{2.880000in}}%
\pgfpathlineto{\pgfqpoint{3.835751in}{3.206322in}}%
\pgfpathlineto{\pgfqpoint{3.749332in}{3.290667in}}%
\pgfpathlineto{\pgfqpoint{3.593749in}{3.440000in}}%
\pgfpathlineto{\pgfqpoint{3.273157in}{3.738667in}}%
\pgfpathlineto{\pgfqpoint{2.964364in}{4.015067in}}%
\pgfpathlineto{\pgfqpoint{2.804040in}{4.154285in}}%
\pgfpathlineto{\pgfqpoint{2.722455in}{4.224000in}}%
\pgfpathlineto{\pgfqpoint{2.717652in}{4.224000in}}%
\pgfpathlineto{\pgfqpoint{2.976913in}{4.000000in}}%
\pgfpathlineto{\pgfqpoint{3.144980in}{3.850667in}}%
\pgfpathlineto{\pgfqpoint{3.445333in}{3.575696in}}%
\pgfpathlineto{\pgfqpoint{3.765980in}{3.270412in}}%
\pgfpathlineto{\pgfqpoint{3.935514in}{3.104000in}}%
\pgfpathlineto{\pgfqpoint{4.246949in}{2.788885in}}%
\pgfpathlineto{\pgfqpoint{4.410245in}{2.618667in}}%
\pgfpathlineto{\pgfqpoint{4.567596in}{2.451108in}}%
\pgfpathlineto{\pgfqpoint{4.768000in}{2.232664in}}%
\pgfpathlineto{\pgfqpoint{4.768000in}{2.232664in}}%
\pgfusepath{fill}%
\end{pgfscope}%
\begin{pgfscope}%
\pgfpathrectangle{\pgfqpoint{0.800000in}{0.528000in}}{\pgfqpoint{3.968000in}{3.696000in}}%
\pgfusepath{clip}%
\pgfsetbuttcap%
\pgfsetroundjoin%
\definecolor{currentfill}{rgb}{0.197636,0.391528,0.554969}%
\pgfsetfillcolor{currentfill}%
\pgfsetlinewidth{0.000000pt}%
\definecolor{currentstroke}{rgb}{0.000000,0.000000,0.000000}%
\pgfsetstrokecolor{currentstroke}%
\pgfsetdash{}{0pt}%
\pgfpathmoveto{\pgfqpoint{0.800000in}{1.269426in}}%
\pgfpathlineto{\pgfqpoint{1.096323in}{0.938667in}}%
\pgfpathlineto{\pgfqpoint{1.240889in}{0.781989in}}%
\pgfpathlineto{\pgfqpoint{1.409838in}{0.602667in}}%
\pgfpathlineto{\pgfqpoint{1.481374in}{0.528000in}}%
\pgfpathlineto{\pgfqpoint{1.485330in}{0.528000in}}%
\pgfpathlineto{\pgfqpoint{1.343119in}{0.677333in}}%
\pgfpathlineto{\pgfqpoint{1.200808in}{0.829429in}}%
\pgfpathlineto{\pgfqpoint{0.898341in}{1.162667in}}%
\pgfpathlineto{\pgfqpoint{0.800000in}{1.273924in}}%
\pgfpathmoveto{\pgfqpoint{4.768000in}{2.241542in}}%
\pgfpathlineto{\pgfqpoint{4.627918in}{2.394667in}}%
\pgfpathlineto{\pgfqpoint{4.327111in}{2.714243in}}%
\pgfpathlineto{\pgfqpoint{4.166447in}{2.880000in}}%
\pgfpathlineto{\pgfqpoint{3.837926in}{3.208348in}}%
\pgfpathlineto{\pgfqpoint{3.753579in}{3.290667in}}%
\pgfpathlineto{\pgfqpoint{3.598064in}{3.440000in}}%
\pgfpathlineto{\pgfqpoint{3.277614in}{3.738667in}}%
\pgfpathlineto{\pgfqpoint{2.964364in}{4.019101in}}%
\pgfpathlineto{\pgfqpoint{2.804040in}{4.158303in}}%
\pgfpathlineto{\pgfqpoint{2.727192in}{4.224000in}}%
\pgfpathlineto{\pgfqpoint{2.722455in}{4.224000in}}%
\pgfpathlineto{\pgfqpoint{2.723195in}{4.223364in}}%
\pgfpathlineto{\pgfqpoint{2.766262in}{4.186667in}}%
\pgfpathlineto{\pgfqpoint{3.084606in}{3.908742in}}%
\pgfpathlineto{\pgfqpoint{3.244929in}{3.764399in}}%
\pgfpathlineto{\pgfqpoint{3.565576in}{3.466744in}}%
\pgfpathlineto{\pgfqpoint{3.894267in}{3.148827in}}%
\pgfpathlineto{\pgfqpoint{3.977220in}{3.066667in}}%
\pgfpathlineto{\pgfqpoint{4.287030in}{2.751662in}}%
\pgfpathlineto{\pgfqpoint{4.449655in}{2.581333in}}%
\pgfpathlineto{\pgfqpoint{4.768000in}{2.237103in}}%
\pgfpathlineto{\pgfqpoint{4.768000in}{2.237103in}}%
\pgfusepath{fill}%
\end{pgfscope}%
\begin{pgfscope}%
\pgfpathrectangle{\pgfqpoint{0.800000in}{0.528000in}}{\pgfqpoint{3.968000in}{3.696000in}}%
\pgfusepath{clip}%
\pgfsetbuttcap%
\pgfsetroundjoin%
\definecolor{currentfill}{rgb}{0.197636,0.391528,0.554969}%
\pgfsetfillcolor{currentfill}%
\pgfsetlinewidth{0.000000pt}%
\definecolor{currentstroke}{rgb}{0.000000,0.000000,0.000000}%
\pgfsetstrokecolor{currentstroke}%
\pgfsetdash{}{0pt}%
\pgfpathmoveto{\pgfqpoint{0.800000in}{1.264927in}}%
\pgfpathlineto{\pgfqpoint{1.092312in}{0.938667in}}%
\pgfpathlineto{\pgfqpoint{1.240889in}{0.777690in}}%
\pgfpathlineto{\pgfqpoint{1.405761in}{0.602667in}}%
\pgfpathlineto{\pgfqpoint{1.477187in}{0.528000in}}%
\pgfpathlineto{\pgfqpoint{1.481225in}{0.528000in}}%
\pgfpathlineto{\pgfqpoint{1.160727in}{0.868457in}}%
\pgfpathlineto{\pgfqpoint{0.994604in}{1.050667in}}%
\pgfpathlineto{\pgfqpoint{0.800000in}{1.269426in}}%
\pgfpathmoveto{\pgfqpoint{4.768000in}{2.245970in}}%
\pgfpathlineto{\pgfqpoint{4.457696in}{2.581333in}}%
\pgfpathlineto{\pgfqpoint{4.315401in}{2.730667in}}%
\pgfpathlineto{\pgfqpoint{4.166788in}{2.883818in}}%
\pgfpathlineto{\pgfqpoint{3.834438in}{3.216000in}}%
\pgfpathlineto{\pgfqpoint{3.523507in}{3.514667in}}%
\pgfpathlineto{\pgfqpoint{3.199919in}{3.813333in}}%
\pgfpathlineto{\pgfqpoint{3.032979in}{3.962667in}}%
\pgfpathlineto{\pgfqpoint{2.731899in}{4.224000in}}%
\pgfpathlineto{\pgfqpoint{2.727192in}{4.224000in}}%
\pgfpathlineto{\pgfqpoint{2.900751in}{4.074667in}}%
\pgfpathlineto{\pgfqpoint{3.044525in}{3.948409in}}%
\pgfpathlineto{\pgfqpoint{3.204848in}{3.804822in}}%
\pgfpathlineto{\pgfqpoint{3.365172in}{3.658266in}}%
\pgfpathlineto{\pgfqpoint{3.685818in}{3.356101in}}%
\pgfpathlineto{\pgfqpoint{4.018722in}{3.029333in}}%
\pgfpathlineto{\pgfqpoint{4.337622in}{2.703124in}}%
\pgfpathlineto{\pgfqpoint{4.418317in}{2.618667in}}%
\pgfpathlineto{\pgfqpoint{4.567596in}{2.459791in}}%
\pgfpathlineto{\pgfqpoint{4.768000in}{2.241542in}}%
\pgfpathlineto{\pgfqpoint{4.768000in}{2.245333in}}%
\pgfpathlineto{\pgfqpoint{4.768000in}{2.245333in}}%
\pgfusepath{fill}%
\end{pgfscope}%
\begin{pgfscope}%
\pgfpathrectangle{\pgfqpoint{0.800000in}{0.528000in}}{\pgfqpoint{3.968000in}{3.696000in}}%
\pgfusepath{clip}%
\pgfsetbuttcap%
\pgfsetroundjoin%
\definecolor{currentfill}{rgb}{0.195860,0.395433,0.555276}%
\pgfsetfillcolor{currentfill}%
\pgfsetlinewidth{0.000000pt}%
\definecolor{currentstroke}{rgb}{0.000000,0.000000,0.000000}%
\pgfsetstrokecolor{currentstroke}%
\pgfsetdash{}{0pt}%
\pgfpathmoveto{\pgfqpoint{0.800000in}{1.260429in}}%
\pgfpathlineto{\pgfqpoint{1.088302in}{0.938667in}}%
\pgfpathlineto{\pgfqpoint{1.240889in}{0.773391in}}%
\pgfpathlineto{\pgfqpoint{1.401684in}{0.602667in}}%
\pgfpathlineto{\pgfqpoint{1.473149in}{0.528000in}}%
\pgfpathlineto{\pgfqpoint{1.477187in}{0.528000in}}%
\pgfpathlineto{\pgfqpoint{1.160727in}{0.864078in}}%
\pgfpathlineto{\pgfqpoint{0.990638in}{1.050667in}}%
\pgfpathlineto{\pgfqpoint{0.800000in}{1.264927in}}%
\pgfpathmoveto{\pgfqpoint{4.768000in}{2.250335in}}%
\pgfpathlineto{\pgfqpoint{4.461717in}{2.581333in}}%
\pgfpathlineto{\pgfqpoint{4.319482in}{2.730667in}}%
\pgfpathlineto{\pgfqpoint{4.166788in}{2.887978in}}%
\pgfpathlineto{\pgfqpoint{3.838653in}{3.216000in}}%
\pgfpathlineto{\pgfqpoint{3.525495in}{3.516854in}}%
\pgfpathlineto{\pgfqpoint{3.204412in}{3.813333in}}%
\pgfpathlineto{\pgfqpoint{3.037548in}{3.962667in}}%
\pgfpathlineto{\pgfqpoint{2.736607in}{4.224000in}}%
\pgfpathlineto{\pgfqpoint{2.731899in}{4.224000in}}%
\pgfpathlineto{\pgfqpoint{2.905379in}{4.074667in}}%
\pgfpathlineto{\pgfqpoint{3.044525in}{3.952452in}}%
\pgfpathlineto{\pgfqpoint{3.204848in}{3.808880in}}%
\pgfpathlineto{\pgfqpoint{3.365172in}{3.662341in}}%
\pgfpathlineto{\pgfqpoint{3.685818in}{3.360210in}}%
\pgfpathlineto{\pgfqpoint{4.006465in}{3.045759in}}%
\pgfpathlineto{\pgfqpoint{4.170525in}{2.880000in}}%
\pgfpathlineto{\pgfqpoint{4.327111in}{2.718488in}}%
\pgfpathlineto{\pgfqpoint{4.631932in}{2.394667in}}%
\pgfpathlineto{\pgfqpoint{4.768000in}{2.245970in}}%
\pgfpathlineto{\pgfqpoint{4.768000in}{2.245970in}}%
\pgfusepath{fill}%
\end{pgfscope}%
\begin{pgfscope}%
\pgfpathrectangle{\pgfqpoint{0.800000in}{0.528000in}}{\pgfqpoint{3.968000in}{3.696000in}}%
\pgfusepath{clip}%
\pgfsetbuttcap%
\pgfsetroundjoin%
\definecolor{currentfill}{rgb}{0.195860,0.395433,0.555276}%
\pgfsetfillcolor{currentfill}%
\pgfsetlinewidth{0.000000pt}%
\definecolor{currentstroke}{rgb}{0.000000,0.000000,0.000000}%
\pgfsetstrokecolor{currentstroke}%
\pgfsetdash{}{0pt}%
\pgfpathmoveto{\pgfqpoint{0.800000in}{1.255931in}}%
\pgfpathlineto{\pgfqpoint{1.084291in}{0.938667in}}%
\pgfpathlineto{\pgfqpoint{1.240889in}{0.769092in}}%
\pgfpathlineto{\pgfqpoint{1.401212in}{0.598942in}}%
\pgfpathlineto{\pgfqpoint{1.469110in}{0.528000in}}%
\pgfpathlineto{\pgfqpoint{1.473149in}{0.528000in}}%
\pgfpathlineto{\pgfqpoint{1.156825in}{0.864000in}}%
\pgfpathlineto{\pgfqpoint{1.000404in}{1.035438in}}%
\pgfpathlineto{\pgfqpoint{0.800000in}{1.260429in}}%
\pgfpathmoveto{\pgfqpoint{4.768000in}{2.254701in}}%
\pgfpathlineto{\pgfqpoint{4.465738in}{2.581333in}}%
\pgfpathlineto{\pgfqpoint{4.323563in}{2.730667in}}%
\pgfpathlineto{\pgfqpoint{4.166788in}{2.892138in}}%
\pgfpathlineto{\pgfqpoint{3.842867in}{3.216000in}}%
\pgfpathlineto{\pgfqpoint{3.525495in}{3.520884in}}%
\pgfpathlineto{\pgfqpoint{3.204848in}{3.816943in}}%
\pgfpathlineto{\pgfqpoint{3.042117in}{3.962667in}}%
\pgfpathlineto{\pgfqpoint{2.741315in}{4.224000in}}%
\pgfpathlineto{\pgfqpoint{2.736607in}{4.224000in}}%
\pgfpathlineto{\pgfqpoint{2.896573in}{4.086190in}}%
\pgfpathlineto{\pgfqpoint{2.979260in}{4.013875in}}%
\pgfpathlineto{\pgfqpoint{3.044525in}{3.956494in}}%
\pgfpathlineto{\pgfqpoint{3.220938in}{3.798346in}}%
\pgfpathlineto{\pgfqpoint{3.367745in}{3.664000in}}%
\pgfpathlineto{\pgfqpoint{3.685818in}{3.364318in}}%
\pgfpathlineto{\pgfqpoint{4.006465in}{3.049902in}}%
\pgfpathlineto{\pgfqpoint{4.174597in}{2.880000in}}%
\pgfpathlineto{\pgfqpoint{4.327111in}{2.722732in}}%
\pgfpathlineto{\pgfqpoint{4.641360in}{2.388707in}}%
\pgfpathlineto{\pgfqpoint{4.715189in}{2.308143in}}%
\pgfpathlineto{\pgfqpoint{4.768000in}{2.250335in}}%
\pgfpathlineto{\pgfqpoint{4.768000in}{2.250335in}}%
\pgfusepath{fill}%
\end{pgfscope}%
\begin{pgfscope}%
\pgfpathrectangle{\pgfqpoint{0.800000in}{0.528000in}}{\pgfqpoint{3.968000in}{3.696000in}}%
\pgfusepath{clip}%
\pgfsetbuttcap%
\pgfsetroundjoin%
\definecolor{currentfill}{rgb}{0.195860,0.395433,0.555276}%
\pgfsetfillcolor{currentfill}%
\pgfsetlinewidth{0.000000pt}%
\definecolor{currentstroke}{rgb}{0.000000,0.000000,0.000000}%
\pgfsetstrokecolor{currentstroke}%
\pgfsetdash{}{0pt}%
\pgfpathmoveto{\pgfqpoint{0.800000in}{1.251433in}}%
\pgfpathlineto{\pgfqpoint{1.080566in}{0.938360in}}%
\pgfpathlineto{\pgfqpoint{1.252843in}{0.752000in}}%
\pgfpathlineto{\pgfqpoint{1.465072in}{0.528000in}}%
\pgfpathlineto{\pgfqpoint{1.469110in}{0.528000in}}%
\pgfpathlineto{\pgfqpoint{1.152852in}{0.864000in}}%
\pgfpathlineto{\pgfqpoint{1.000404in}{1.031039in}}%
\pgfpathlineto{\pgfqpoint{0.800000in}{1.255931in}}%
\pgfpathmoveto{\pgfqpoint{4.768000in}{2.259066in}}%
\pgfpathlineto{\pgfqpoint{4.469758in}{2.581333in}}%
\pgfpathlineto{\pgfqpoint{4.327111in}{2.731213in}}%
\pgfpathlineto{\pgfqpoint{4.155920in}{2.907210in}}%
\pgfpathlineto{\pgfqpoint{4.072380in}{2.992000in}}%
\pgfpathlineto{\pgfqpoint{3.922877in}{3.141333in}}%
\pgfpathlineto{\pgfqpoint{3.605657in}{3.449047in}}%
\pgfpathlineto{\pgfqpoint{3.285010in}{3.748041in}}%
\pgfpathlineto{\pgfqpoint{2.961970in}{4.037333in}}%
\pgfpathlineto{\pgfqpoint{2.789700in}{4.186667in}}%
\pgfpathlineto{\pgfqpoint{2.746023in}{4.224000in}}%
\pgfpathlineto{\pgfqpoint{2.741315in}{4.224000in}}%
\pgfpathlineto{\pgfqpoint{2.884202in}{4.101142in}}%
\pgfpathlineto{\pgfqpoint{3.208831in}{3.813333in}}%
\pgfpathlineto{\pgfqpoint{3.372084in}{3.664000in}}%
\pgfpathlineto{\pgfqpoint{3.688985in}{3.365333in}}%
\pgfpathlineto{\pgfqpoint{4.006465in}{3.054045in}}%
\pgfpathlineto{\pgfqpoint{4.178670in}{2.880000in}}%
\pgfpathlineto{\pgfqpoint{4.327111in}{2.726977in}}%
\pgfpathlineto{\pgfqpoint{4.647758in}{2.386215in}}%
\pgfpathlineto{\pgfqpoint{4.768000in}{2.254701in}}%
\pgfpathlineto{\pgfqpoint{4.768000in}{2.254701in}}%
\pgfusepath{fill}%
\end{pgfscope}%
\begin{pgfscope}%
\pgfpathrectangle{\pgfqpoint{0.800000in}{0.528000in}}{\pgfqpoint{3.968000in}{3.696000in}}%
\pgfusepath{clip}%
\pgfsetbuttcap%
\pgfsetroundjoin%
\definecolor{currentfill}{rgb}{0.195860,0.395433,0.555276}%
\pgfsetfillcolor{currentfill}%
\pgfsetlinewidth{0.000000pt}%
\definecolor{currentstroke}{rgb}{0.000000,0.000000,0.000000}%
\pgfsetstrokecolor{currentstroke}%
\pgfsetdash{}{0pt}%
\pgfpathmoveto{\pgfqpoint{0.800000in}{1.246934in}}%
\pgfpathlineto{\pgfqpoint{1.080566in}{0.934043in}}%
\pgfpathlineto{\pgfqpoint{1.248826in}{0.752000in}}%
\pgfpathlineto{\pgfqpoint{1.461034in}{0.528000in}}%
\pgfpathlineto{\pgfqpoint{1.465072in}{0.528000in}}%
\pgfpathlineto{\pgfqpoint{1.148880in}{0.864000in}}%
\pgfpathlineto{\pgfqpoint{1.000404in}{1.026640in}}%
\pgfpathlineto{\pgfqpoint{0.800000in}{1.251433in}}%
\pgfpathmoveto{\pgfqpoint{4.768000in}{2.263432in}}%
\pgfpathlineto{\pgfqpoint{4.473779in}{2.581333in}}%
\pgfpathlineto{\pgfqpoint{4.327111in}{2.735390in}}%
\pgfpathlineto{\pgfqpoint{4.150207in}{2.917333in}}%
\pgfpathlineto{\pgfqpoint{4.002121in}{3.066667in}}%
\pgfpathlineto{\pgfqpoint{3.846141in}{3.220969in}}%
\pgfpathlineto{\pgfqpoint{3.525495in}{3.528944in}}%
\pgfpathlineto{\pgfqpoint{3.204848in}{3.824938in}}%
\pgfpathlineto{\pgfqpoint{3.044525in}{3.968531in}}%
\pgfpathlineto{\pgfqpoint{2.750731in}{4.224000in}}%
\pgfpathlineto{\pgfqpoint{2.746023in}{4.224000in}}%
\pgfpathlineto{\pgfqpoint{2.884202in}{4.105168in}}%
\pgfpathlineto{\pgfqpoint{3.213241in}{3.813333in}}%
\pgfpathlineto{\pgfqpoint{3.376424in}{3.664000in}}%
\pgfpathlineto{\pgfqpoint{3.693190in}{3.365333in}}%
\pgfpathlineto{\pgfqpoint{4.006465in}{3.058188in}}%
\pgfpathlineto{\pgfqpoint{4.182742in}{2.880000in}}%
\pgfpathlineto{\pgfqpoint{4.327635in}{2.730667in}}%
\pgfpathlineto{\pgfqpoint{4.647758in}{2.390567in}}%
\pgfpathlineto{\pgfqpoint{4.768000in}{2.259066in}}%
\pgfpathlineto{\pgfqpoint{4.768000in}{2.259066in}}%
\pgfusepath{fill}%
\end{pgfscope}%
\begin{pgfscope}%
\pgfpathrectangle{\pgfqpoint{0.800000in}{0.528000in}}{\pgfqpoint{3.968000in}{3.696000in}}%
\pgfusepath{clip}%
\pgfsetbuttcap%
\pgfsetroundjoin%
\definecolor{currentfill}{rgb}{0.194100,0.399323,0.555565}%
\pgfsetfillcolor{currentfill}%
\pgfsetlinewidth{0.000000pt}%
\definecolor{currentstroke}{rgb}{0.000000,0.000000,0.000000}%
\pgfsetstrokecolor{currentstroke}%
\pgfsetdash{}{0pt}%
\pgfpathmoveto{\pgfqpoint{0.800000in}{1.242436in}}%
\pgfpathlineto{\pgfqpoint{1.072398in}{0.938667in}}%
\pgfpathlineto{\pgfqpoint{1.361131in}{0.628452in}}%
\pgfpathlineto{\pgfqpoint{1.456996in}{0.528000in}}%
\pgfpathlineto{\pgfqpoint{1.461034in}{0.528000in}}%
\pgfpathlineto{\pgfqpoint{1.160727in}{0.846844in}}%
\pgfpathlineto{\pgfqpoint{1.000404in}{1.022242in}}%
\pgfpathlineto{\pgfqpoint{0.800000in}{1.246934in}}%
\pgfpathmoveto{\pgfqpoint{4.768000in}{2.267797in}}%
\pgfpathlineto{\pgfqpoint{4.477800in}{2.581333in}}%
\pgfpathlineto{\pgfqpoint{4.327111in}{2.739568in}}%
\pgfpathlineto{\pgfqpoint{4.154294in}{2.917333in}}%
\pgfpathlineto{\pgfqpoint{4.003530in}{3.069400in}}%
\pgfpathlineto{\pgfqpoint{3.846141in}{3.225031in}}%
\pgfpathlineto{\pgfqpoint{3.515254in}{3.542461in}}%
\pgfpathlineto{\pgfqpoint{3.425351in}{3.626667in}}%
\pgfpathlineto{\pgfqpoint{3.263135in}{3.776000in}}%
\pgfpathlineto{\pgfqpoint{3.124687in}{3.901087in}}%
\pgfpathlineto{\pgfqpoint{2.964364in}{4.043217in}}%
\pgfpathlineto{\pgfqpoint{2.799075in}{4.186667in}}%
\pgfpathlineto{\pgfqpoint{2.755439in}{4.224000in}}%
\pgfpathlineto{\pgfqpoint{2.750731in}{4.224000in}}%
\pgfpathlineto{\pgfqpoint{2.884202in}{4.109195in}}%
\pgfpathlineto{\pgfqpoint{3.217651in}{3.813333in}}%
\pgfpathlineto{\pgfqpoint{3.380764in}{3.664000in}}%
\pgfpathlineto{\pgfqpoint{3.697395in}{3.365333in}}%
\pgfpathlineto{\pgfqpoint{4.006465in}{3.062331in}}%
\pgfpathlineto{\pgfqpoint{4.177190in}{2.889689in}}%
\pgfpathlineto{\pgfqpoint{4.259547in}{2.805333in}}%
\pgfpathlineto{\pgfqpoint{4.407273in}{2.651575in}}%
\pgfpathlineto{\pgfqpoint{4.721693in}{2.314201in}}%
\pgfpathlineto{\pgfqpoint{4.768000in}{2.263432in}}%
\pgfpathlineto{\pgfqpoint{4.768000in}{2.263432in}}%
\pgfusepath{fill}%
\end{pgfscope}%
\begin{pgfscope}%
\pgfpathrectangle{\pgfqpoint{0.800000in}{0.528000in}}{\pgfqpoint{3.968000in}{3.696000in}}%
\pgfusepath{clip}%
\pgfsetbuttcap%
\pgfsetroundjoin%
\definecolor{currentfill}{rgb}{0.194100,0.399323,0.555565}%
\pgfsetfillcolor{currentfill}%
\pgfsetlinewidth{0.000000pt}%
\definecolor{currentstroke}{rgb}{0.000000,0.000000,0.000000}%
\pgfsetstrokecolor{currentstroke}%
\pgfsetdash{}{0pt}%
\pgfpathmoveto{\pgfqpoint{0.800000in}{1.237938in}}%
\pgfpathlineto{\pgfqpoint{1.040485in}{0.969318in}}%
\pgfpathlineto{\pgfqpoint{1.206037in}{0.789333in}}%
\pgfpathlineto{\pgfqpoint{1.361131in}{0.624235in}}%
\pgfpathlineto{\pgfqpoint{1.452958in}{0.528000in}}%
\pgfpathlineto{\pgfqpoint{1.456996in}{0.528000in}}%
\pgfpathlineto{\pgfqpoint{1.160727in}{0.842536in}}%
\pgfpathlineto{\pgfqpoint{1.000404in}{1.017843in}}%
\pgfpathlineto{\pgfqpoint{0.800000in}{1.242436in}}%
\pgfpathmoveto{\pgfqpoint{4.768000in}{2.272163in}}%
\pgfpathlineto{\pgfqpoint{4.481820in}{2.581333in}}%
\pgfpathlineto{\pgfqpoint{4.327111in}{2.743745in}}%
\pgfpathlineto{\pgfqpoint{4.158382in}{2.917333in}}%
\pgfpathlineto{\pgfqpoint{4.006465in}{3.070556in}}%
\pgfpathlineto{\pgfqpoint{3.846141in}{3.229094in}}%
\pgfpathlineto{\pgfqpoint{3.517423in}{3.544481in}}%
\pgfpathlineto{\pgfqpoint{3.429674in}{3.626667in}}%
\pgfpathlineto{\pgfqpoint{3.267527in}{3.776000in}}%
\pgfpathlineto{\pgfqpoint{3.102018in}{3.925333in}}%
\pgfpathlineto{\pgfqpoint{2.803763in}{4.186667in}}%
\pgfpathlineto{\pgfqpoint{2.760147in}{4.224000in}}%
\pgfpathlineto{\pgfqpoint{2.755439in}{4.224000in}}%
\pgfpathlineto{\pgfqpoint{2.885584in}{4.112000in}}%
\pgfpathlineto{\pgfqpoint{3.222061in}{3.813333in}}%
\pgfpathlineto{\pgfqpoint{3.385104in}{3.664000in}}%
\pgfpathlineto{\pgfqpoint{3.701601in}{3.365333in}}%
\pgfpathlineto{\pgfqpoint{4.006465in}{3.066474in}}%
\pgfpathlineto{\pgfqpoint{4.166788in}{2.904619in}}%
\pgfpathlineto{\pgfqpoint{4.487434in}{2.571119in}}%
\pgfpathlineto{\pgfqpoint{4.651933in}{2.394667in}}%
\pgfpathlineto{\pgfqpoint{4.768000in}{2.267797in}}%
\pgfpathlineto{\pgfqpoint{4.768000in}{2.267797in}}%
\pgfusepath{fill}%
\end{pgfscope}%
\begin{pgfscope}%
\pgfpathrectangle{\pgfqpoint{0.800000in}{0.528000in}}{\pgfqpoint{3.968000in}{3.696000in}}%
\pgfusepath{clip}%
\pgfsetbuttcap%
\pgfsetroundjoin%
\definecolor{currentfill}{rgb}{0.194100,0.399323,0.555565}%
\pgfsetfillcolor{currentfill}%
\pgfsetlinewidth{0.000000pt}%
\definecolor{currentstroke}{rgb}{0.000000,0.000000,0.000000}%
\pgfsetstrokecolor{currentstroke}%
\pgfsetdash{}{0pt}%
\pgfpathmoveto{\pgfqpoint{0.800000in}{1.233505in}}%
\pgfpathlineto{\pgfqpoint{0.962873in}{1.050667in}}%
\pgfpathlineto{\pgfqpoint{1.280970in}{0.704912in}}%
\pgfpathlineto{\pgfqpoint{1.448919in}{0.528000in}}%
\pgfpathlineto{\pgfqpoint{1.452958in}{0.528000in}}%
\pgfpathlineto{\pgfqpoint{1.160727in}{0.838228in}}%
\pgfpathlineto{\pgfqpoint{1.000404in}{1.013444in}}%
\pgfpathlineto{\pgfqpoint{0.800000in}{1.237938in}}%
\pgfpathlineto{\pgfqpoint{0.800000in}{1.237333in}}%
\pgfpathmoveto{\pgfqpoint{4.768000in}{2.276529in}}%
\pgfpathlineto{\pgfqpoint{4.485841in}{2.581333in}}%
\pgfpathlineto{\pgfqpoint{4.327111in}{2.747923in}}%
\pgfpathlineto{\pgfqpoint{4.162470in}{2.917333in}}%
\pgfpathlineto{\pgfqpoint{4.006465in}{3.074635in}}%
\pgfpathlineto{\pgfqpoint{3.835496in}{3.243418in}}%
\pgfpathlineto{\pgfqpoint{3.748687in}{3.328000in}}%
\pgfpathlineto{\pgfqpoint{3.433996in}{3.626667in}}%
\pgfpathlineto{\pgfqpoint{3.271920in}{3.776000in}}%
\pgfpathlineto{\pgfqpoint{3.106482in}{3.925333in}}%
\pgfpathlineto{\pgfqpoint{2.804040in}{4.190390in}}%
\pgfpathlineto{\pgfqpoint{2.763960in}{4.224000in}}%
\pgfpathlineto{\pgfqpoint{2.760147in}{4.224000in}}%
\pgfpathlineto{\pgfqpoint{2.890142in}{4.112000in}}%
\pgfpathlineto{\pgfqpoint{3.204848in}{3.832933in}}%
\pgfpathlineto{\pgfqpoint{3.525495in}{3.537003in}}%
\pgfpathlineto{\pgfqpoint{3.685818in}{3.384564in}}%
\pgfpathlineto{\pgfqpoint{3.859489in}{3.216000in}}%
\pgfpathlineto{\pgfqpoint{4.010353in}{3.066667in}}%
\pgfpathlineto{\pgfqpoint{4.166788in}{2.908779in}}%
\pgfpathlineto{\pgfqpoint{4.487434in}{2.575382in}}%
\pgfpathlineto{\pgfqpoint{4.655881in}{2.394667in}}%
\pgfpathlineto{\pgfqpoint{4.768000in}{2.272163in}}%
\pgfpathlineto{\pgfqpoint{4.768000in}{2.272163in}}%
\pgfusepath{fill}%
\end{pgfscope}%
\begin{pgfscope}%
\pgfpathrectangle{\pgfqpoint{0.800000in}{0.528000in}}{\pgfqpoint{3.968000in}{3.696000in}}%
\pgfusepath{clip}%
\pgfsetbuttcap%
\pgfsetroundjoin%
\definecolor{currentfill}{rgb}{0.194100,0.399323,0.555565}%
\pgfsetfillcolor{currentfill}%
\pgfsetlinewidth{0.000000pt}%
\definecolor{currentstroke}{rgb}{0.000000,0.000000,0.000000}%
\pgfsetstrokecolor{currentstroke}%
\pgfsetdash{}{0pt}%
\pgfpathmoveto{\pgfqpoint{0.800000in}{1.229082in}}%
\pgfpathlineto{\pgfqpoint{0.960323in}{1.049119in}}%
\pgfpathlineto{\pgfqpoint{1.267853in}{0.714667in}}%
\pgfpathlineto{\pgfqpoint{1.409173in}{0.565333in}}%
\pgfpathlineto{\pgfqpoint{1.444881in}{0.528000in}}%
\pgfpathlineto{\pgfqpoint{1.448919in}{0.528000in}}%
\pgfpathlineto{\pgfqpoint{1.160727in}{0.833920in}}%
\pgfpathlineto{\pgfqpoint{0.996587in}{1.013333in}}%
\pgfpathlineto{\pgfqpoint{0.800000in}{1.233505in}}%
\pgfpathmoveto{\pgfqpoint{4.768000in}{2.280894in}}%
\pgfpathlineto{\pgfqpoint{4.487434in}{2.583866in}}%
\pgfpathlineto{\pgfqpoint{4.311784in}{2.768000in}}%
\pgfpathlineto{\pgfqpoint{4.006465in}{3.078714in}}%
\pgfpathlineto{\pgfqpoint{3.829652in}{3.253333in}}%
\pgfpathlineto{\pgfqpoint{3.675373in}{3.402667in}}%
\pgfpathlineto{\pgfqpoint{3.518112in}{3.552000in}}%
\pgfpathlineto{\pgfqpoint{3.194060in}{3.850667in}}%
\pgfpathlineto{\pgfqpoint{2.884202in}{4.125099in}}%
\pgfpathlineto{\pgfqpoint{2.769454in}{4.224000in}}%
\pgfpathlineto{\pgfqpoint{2.764838in}{4.224000in}}%
\pgfpathlineto{\pgfqpoint{2.937515in}{4.074667in}}%
\pgfpathlineto{\pgfqpoint{3.106482in}{3.925333in}}%
\pgfpathlineto{\pgfqpoint{3.244929in}{3.800588in}}%
\pgfpathlineto{\pgfqpoint{3.405253in}{3.653382in}}%
\pgfpathlineto{\pgfqpoint{3.565576in}{3.503209in}}%
\pgfpathlineto{\pgfqpoint{3.901590in}{3.178667in}}%
\pgfpathlineto{\pgfqpoint{4.206869in}{2.871986in}}%
\pgfpathlineto{\pgfqpoint{4.527515in}{2.536986in}}%
\pgfpathlineto{\pgfqpoint{4.694195in}{2.357333in}}%
\pgfpathlineto{\pgfqpoint{4.768000in}{2.276529in}}%
\pgfpathlineto{\pgfqpoint{4.768000in}{2.276529in}}%
\pgfusepath{fill}%
\end{pgfscope}%
\begin{pgfscope}%
\pgfpathrectangle{\pgfqpoint{0.800000in}{0.528000in}}{\pgfqpoint{3.968000in}{3.696000in}}%
\pgfusepath{clip}%
\pgfsetbuttcap%
\pgfsetroundjoin%
\definecolor{currentfill}{rgb}{0.192357,0.403199,0.555836}%
\pgfsetfillcolor{currentfill}%
\pgfsetlinewidth{0.000000pt}%
\definecolor{currentstroke}{rgb}{0.000000,0.000000,0.000000}%
\pgfsetstrokecolor{currentstroke}%
\pgfsetdash{}{0pt}%
\pgfpathmoveto{\pgfqpoint{0.800000in}{1.224659in}}%
\pgfpathlineto{\pgfqpoint{0.960323in}{1.044788in}}%
\pgfpathlineto{\pgfqpoint{1.263889in}{0.714667in}}%
\pgfpathlineto{\pgfqpoint{1.405150in}{0.565333in}}%
\pgfpathlineto{\pgfqpoint{1.441293in}{0.528000in}}%
\pgfpathlineto{\pgfqpoint{1.444881in}{0.528000in}}%
\pgfpathlineto{\pgfqpoint{1.160727in}{0.829612in}}%
\pgfpathlineto{\pgfqpoint{0.992672in}{1.013333in}}%
\pgfpathlineto{\pgfqpoint{0.800000in}{1.229082in}}%
\pgfpathmoveto{\pgfqpoint{4.768000in}{2.285218in}}%
\pgfpathlineto{\pgfqpoint{4.598622in}{2.469333in}}%
\pgfpathlineto{\pgfqpoint{4.447354in}{2.630421in}}%
\pgfpathlineto{\pgfqpoint{4.126707in}{2.961890in}}%
\pgfpathlineto{\pgfqpoint{3.795527in}{3.290667in}}%
\pgfpathlineto{\pgfqpoint{3.482619in}{3.589333in}}%
\pgfpathlineto{\pgfqpoint{3.321497in}{3.738667in}}%
\pgfpathlineto{\pgfqpoint{3.157059in}{3.888000in}}%
\pgfpathlineto{\pgfqpoint{2.844121in}{4.163774in}}%
\pgfpathlineto{\pgfqpoint{2.774071in}{4.224000in}}%
\pgfpathlineto{\pgfqpoint{2.769454in}{4.224000in}}%
\pgfpathlineto{\pgfqpoint{2.942054in}{4.074667in}}%
\pgfpathlineto{\pgfqpoint{3.097405in}{3.937255in}}%
\pgfpathlineto{\pgfqpoint{3.179074in}{3.863993in}}%
\pgfpathlineto{\pgfqpoint{3.244929in}{3.804590in}}%
\pgfpathlineto{\pgfqpoint{3.405253in}{3.657400in}}%
\pgfpathlineto{\pgfqpoint{3.565576in}{3.507243in}}%
\pgfpathlineto{\pgfqpoint{3.896175in}{3.187938in}}%
\pgfpathlineto{\pgfqpoint{3.981083in}{3.104000in}}%
\pgfpathlineto{\pgfqpoint{4.129795in}{2.954667in}}%
\pgfpathlineto{\pgfqpoint{4.287030in}{2.793652in}}%
\pgfpathlineto{\pgfqpoint{4.600645in}{2.462783in}}%
\pgfpathlineto{\pgfqpoint{4.663776in}{2.394667in}}%
\pgfpathlineto{\pgfqpoint{4.768000in}{2.280894in}}%
\pgfpathlineto{\pgfqpoint{4.768000in}{2.282667in}}%
\pgfusepath{fill}%
\end{pgfscope}%
\begin{pgfscope}%
\pgfpathrectangle{\pgfqpoint{0.800000in}{0.528000in}}{\pgfqpoint{3.968000in}{3.696000in}}%
\pgfusepath{clip}%
\pgfsetbuttcap%
\pgfsetroundjoin%
\definecolor{currentfill}{rgb}{0.192357,0.403199,0.555836}%
\pgfsetfillcolor{currentfill}%
\pgfsetlinewidth{0.000000pt}%
\definecolor{currentstroke}{rgb}{0.000000,0.000000,0.000000}%
\pgfsetstrokecolor{currentstroke}%
\pgfsetdash{}{0pt}%
\pgfpathmoveto{\pgfqpoint{0.800000in}{1.220236in}}%
\pgfpathlineto{\pgfqpoint{0.960323in}{1.040456in}}%
\pgfpathlineto{\pgfqpoint{1.259924in}{0.714667in}}%
\pgfpathlineto{\pgfqpoint{1.401212in}{0.565245in}}%
\pgfpathlineto{\pgfqpoint{1.436880in}{0.528000in}}%
\pgfpathlineto{\pgfqpoint{1.440851in}{0.528000in}}%
\pgfpathlineto{\pgfqpoint{1.280970in}{0.696461in}}%
\pgfpathlineto{\pgfqpoint{1.120646in}{0.868773in}}%
\pgfpathlineto{\pgfqpoint{0.821795in}{1.200000in}}%
\pgfpathlineto{\pgfqpoint{0.800000in}{1.224659in}}%
\pgfpathmoveto{\pgfqpoint{4.768000in}{2.289512in}}%
\pgfpathlineto{\pgfqpoint{4.602599in}{2.469333in}}%
\pgfpathlineto{\pgfqpoint{4.447354in}{2.634612in}}%
\pgfpathlineto{\pgfqpoint{4.126707in}{2.965982in}}%
\pgfpathlineto{\pgfqpoint{3.799700in}{3.290667in}}%
\pgfpathlineto{\pgfqpoint{3.485414in}{3.590725in}}%
\pgfpathlineto{\pgfqpoint{3.325091in}{3.739372in}}%
\pgfpathlineto{\pgfqpoint{3.161505in}{3.888000in}}%
\pgfpathlineto{\pgfqpoint{2.833488in}{4.176762in}}%
\pgfpathlineto{\pgfqpoint{2.778687in}{4.224000in}}%
\pgfpathlineto{\pgfqpoint{2.774071in}{4.224000in}}%
\pgfpathlineto{\pgfqpoint{2.946594in}{4.074667in}}%
\pgfpathlineto{\pgfqpoint{3.099574in}{3.939275in}}%
\pgfpathlineto{\pgfqpoint{3.181237in}{3.866007in}}%
\pgfpathlineto{\pgfqpoint{3.244929in}{3.808591in}}%
\pgfpathlineto{\pgfqpoint{3.405253in}{3.661417in}}%
\pgfpathlineto{\pgfqpoint{3.565576in}{3.511277in}}%
\pgfpathlineto{\pgfqpoint{3.898282in}{3.189900in}}%
\pgfpathlineto{\pgfqpoint{3.985178in}{3.104000in}}%
\pgfpathlineto{\pgfqpoint{4.133828in}{2.954667in}}%
\pgfpathlineto{\pgfqpoint{4.287030in}{2.797825in}}%
\pgfpathlineto{\pgfqpoint{4.607677in}{2.459596in}}%
\pgfpathlineto{\pgfqpoint{4.768000in}{2.285218in}}%
\pgfpathlineto{\pgfqpoint{4.768000in}{2.285218in}}%
\pgfusepath{fill}%
\end{pgfscope}%
\begin{pgfscope}%
\pgfpathrectangle{\pgfqpoint{0.800000in}{0.528000in}}{\pgfqpoint{3.968000in}{3.696000in}}%
\pgfusepath{clip}%
\pgfsetbuttcap%
\pgfsetroundjoin%
\definecolor{currentfill}{rgb}{0.192357,0.403199,0.555836}%
\pgfsetfillcolor{currentfill}%
\pgfsetlinewidth{0.000000pt}%
\definecolor{currentstroke}{rgb}{0.000000,0.000000,0.000000}%
\pgfsetstrokecolor{currentstroke}%
\pgfsetdash{}{0pt}%
\pgfpathmoveto{\pgfqpoint{0.800000in}{1.215813in}}%
\pgfpathlineto{\pgfqpoint{0.947225in}{1.050667in}}%
\pgfpathlineto{\pgfqpoint{1.082846in}{0.901333in}}%
\pgfpathlineto{\pgfqpoint{1.401212in}{0.561099in}}%
\pgfpathlineto{\pgfqpoint{1.432909in}{0.528000in}}%
\pgfpathlineto{\pgfqpoint{1.436880in}{0.528000in}}%
\pgfpathlineto{\pgfqpoint{1.280970in}{0.692235in}}%
\pgfpathlineto{\pgfqpoint{1.120646in}{0.864460in}}%
\pgfpathlineto{\pgfqpoint{0.817886in}{1.200000in}}%
\pgfpathlineto{\pgfqpoint{0.800000in}{1.220236in}}%
\pgfpathmoveto{\pgfqpoint{4.768000in}{2.293807in}}%
\pgfpathlineto{\pgfqpoint{4.599389in}{2.477053in}}%
\pgfpathlineto{\pgfqpoint{4.431009in}{2.656000in}}%
\pgfpathlineto{\pgfqpoint{4.115338in}{2.981410in}}%
\pgfpathlineto{\pgfqpoint{4.030748in}{3.066667in}}%
\pgfpathlineto{\pgfqpoint{3.725899in}{3.366219in}}%
\pgfpathlineto{\pgfqpoint{3.565576in}{3.519274in}}%
\pgfpathlineto{\pgfqpoint{3.405253in}{3.669371in}}%
\pgfpathlineto{\pgfqpoint{3.082511in}{3.962667in}}%
\pgfpathlineto{\pgfqpoint{2.783304in}{4.224000in}}%
\pgfpathlineto{\pgfqpoint{2.778687in}{4.224000in}}%
\pgfpathlineto{\pgfqpoint{2.924283in}{4.098147in}}%
\pgfpathlineto{\pgfqpoint{3.084606in}{3.956818in}}%
\pgfpathlineto{\pgfqpoint{3.406775in}{3.664000in}}%
\pgfpathlineto{\pgfqpoint{3.566245in}{3.514667in}}%
\pgfpathlineto{\pgfqpoint{3.886222in}{3.206015in}}%
\pgfpathlineto{\pgfqpoint{4.211175in}{2.880000in}}%
\pgfpathlineto{\pgfqpoint{4.532859in}{2.544000in}}%
\pgfpathlineto{\pgfqpoint{4.687838in}{2.377115in}}%
\pgfpathlineto{\pgfqpoint{4.768000in}{2.289512in}}%
\pgfpathlineto{\pgfqpoint{4.768000in}{2.289512in}}%
\pgfusepath{fill}%
\end{pgfscope}%
\begin{pgfscope}%
\pgfpathrectangle{\pgfqpoint{0.800000in}{0.528000in}}{\pgfqpoint{3.968000in}{3.696000in}}%
\pgfusepath{clip}%
\pgfsetbuttcap%
\pgfsetroundjoin%
\definecolor{currentfill}{rgb}{0.190631,0.407061,0.556089}%
\pgfsetfillcolor{currentfill}%
\pgfsetlinewidth{0.000000pt}%
\definecolor{currentstroke}{rgb}{0.000000,0.000000,0.000000}%
\pgfsetstrokecolor{currentstroke}%
\pgfsetdash{}{0pt}%
\pgfpathmoveto{\pgfqpoint{0.800000in}{1.211390in}}%
\pgfpathlineto{\pgfqpoint{0.943324in}{1.050667in}}%
\pgfpathlineto{\pgfqpoint{1.080566in}{0.899533in}}%
\pgfpathlineto{\pgfqpoint{1.401212in}{0.556952in}}%
\pgfpathlineto{\pgfqpoint{1.428938in}{0.528000in}}%
\pgfpathlineto{\pgfqpoint{1.432909in}{0.528000in}}%
\pgfpathlineto{\pgfqpoint{1.280970in}{0.688009in}}%
\pgfpathlineto{\pgfqpoint{1.117156in}{0.864000in}}%
\pgfpathlineto{\pgfqpoint{0.813976in}{1.200000in}}%
\pgfpathlineto{\pgfqpoint{0.800000in}{1.215813in}}%
\pgfpathmoveto{\pgfqpoint{4.768000in}{2.298102in}}%
\pgfpathlineto{\pgfqpoint{4.607677in}{2.472376in}}%
\pgfpathlineto{\pgfqpoint{4.434992in}{2.656000in}}%
\pgfpathlineto{\pgfqpoint{4.109059in}{2.992000in}}%
\pgfpathlineto{\pgfqpoint{3.959925in}{3.141333in}}%
\pgfpathlineto{\pgfqpoint{3.645737in}{3.447100in}}%
\pgfpathlineto{\pgfqpoint{3.485414in}{3.598655in}}%
\pgfpathlineto{\pgfqpoint{3.325091in}{3.747271in}}%
\pgfpathlineto{\pgfqpoint{3.164768in}{3.892981in}}%
\pgfpathlineto{\pgfqpoint{2.831350in}{4.186667in}}%
\pgfpathlineto{\pgfqpoint{2.787920in}{4.224000in}}%
\pgfpathlineto{\pgfqpoint{2.783304in}{4.224000in}}%
\pgfpathlineto{\pgfqpoint{2.924283in}{4.102117in}}%
\pgfpathlineto{\pgfqpoint{3.084606in}{3.960804in}}%
\pgfpathlineto{\pgfqpoint{3.411037in}{3.664000in}}%
\pgfpathlineto{\pgfqpoint{3.570441in}{3.514667in}}%
\pgfpathlineto{\pgfqpoint{3.886222in}{3.210081in}}%
\pgfpathlineto{\pgfqpoint{4.215179in}{2.880000in}}%
\pgfpathlineto{\pgfqpoint{4.536798in}{2.544000in}}%
\pgfpathlineto{\pgfqpoint{4.687838in}{2.381401in}}%
\pgfpathlineto{\pgfqpoint{4.768000in}{2.293807in}}%
\pgfpathlineto{\pgfqpoint{4.768000in}{2.293807in}}%
\pgfusepath{fill}%
\end{pgfscope}%
\begin{pgfscope}%
\pgfpathrectangle{\pgfqpoint{0.800000in}{0.528000in}}{\pgfqpoint{3.968000in}{3.696000in}}%
\pgfusepath{clip}%
\pgfsetbuttcap%
\pgfsetroundjoin%
\definecolor{currentfill}{rgb}{0.190631,0.407061,0.556089}%
\pgfsetfillcolor{currentfill}%
\pgfsetlinewidth{0.000000pt}%
\definecolor{currentstroke}{rgb}{0.000000,0.000000,0.000000}%
\pgfsetstrokecolor{currentstroke}%
\pgfsetdash{}{0pt}%
\pgfpathmoveto{\pgfqpoint{0.800000in}{1.206967in}}%
\pgfpathlineto{\pgfqpoint{0.939422in}{1.050667in}}%
\pgfpathlineto{\pgfqpoint{1.080566in}{0.895284in}}%
\pgfpathlineto{\pgfqpoint{1.394927in}{0.559479in}}%
\pgfpathlineto{\pgfqpoint{1.424968in}{0.528000in}}%
\pgfpathlineto{\pgfqpoint{1.428938in}{0.528000in}}%
\pgfpathlineto{\pgfqpoint{1.280970in}{0.683784in}}%
\pgfpathlineto{\pgfqpoint{1.113249in}{0.864000in}}%
\pgfpathlineto{\pgfqpoint{0.800000in}{1.211390in}}%
\pgfpathmoveto{\pgfqpoint{4.768000in}{2.302396in}}%
\pgfpathlineto{\pgfqpoint{4.607677in}{2.476584in}}%
\pgfpathlineto{\pgfqpoint{4.438975in}{2.656000in}}%
\pgfpathlineto{\pgfqpoint{4.113108in}{2.992000in}}%
\pgfpathlineto{\pgfqpoint{3.964035in}{3.141333in}}%
\pgfpathlineto{\pgfqpoint{3.645737in}{3.451081in}}%
\pgfpathlineto{\pgfqpoint{3.485414in}{3.602621in}}%
\pgfpathlineto{\pgfqpoint{3.325091in}{3.751221in}}%
\pgfpathlineto{\pgfqpoint{3.164768in}{3.896915in}}%
\pgfpathlineto{\pgfqpoint{2.835947in}{4.186667in}}%
\pgfpathlineto{\pgfqpoint{2.792537in}{4.224000in}}%
\pgfpathlineto{\pgfqpoint{2.787920in}{4.224000in}}%
\pgfpathlineto{\pgfqpoint{2.924283in}{4.106087in}}%
\pgfpathlineto{\pgfqpoint{3.086949in}{3.962667in}}%
\pgfpathlineto{\pgfqpoint{3.415299in}{3.664000in}}%
\pgfpathlineto{\pgfqpoint{3.574638in}{3.514667in}}%
\pgfpathlineto{\pgfqpoint{3.886222in}{3.214148in}}%
\pgfpathlineto{\pgfqpoint{4.219183in}{2.880000in}}%
\pgfpathlineto{\pgfqpoint{4.540737in}{2.544000in}}%
\pgfpathlineto{\pgfqpoint{4.687838in}{2.385686in}}%
\pgfpathlineto{\pgfqpoint{4.768000in}{2.298102in}}%
\pgfpathlineto{\pgfqpoint{4.768000in}{2.298102in}}%
\pgfusepath{fill}%
\end{pgfscope}%
\begin{pgfscope}%
\pgfpathrectangle{\pgfqpoint{0.800000in}{0.528000in}}{\pgfqpoint{3.968000in}{3.696000in}}%
\pgfusepath{clip}%
\pgfsetbuttcap%
\pgfsetroundjoin%
\definecolor{currentfill}{rgb}{0.190631,0.407061,0.556089}%
\pgfsetfillcolor{currentfill}%
\pgfsetlinewidth{0.000000pt}%
\definecolor{currentstroke}{rgb}{0.000000,0.000000,0.000000}%
\pgfsetstrokecolor{currentstroke}%
\pgfsetdash{}{0pt}%
\pgfpathmoveto{\pgfqpoint{0.800000in}{1.202544in}}%
\pgfpathlineto{\pgfqpoint{0.935521in}{1.050667in}}%
\pgfpathlineto{\pgfqpoint{1.080566in}{0.891036in}}%
\pgfpathlineto{\pgfqpoint{1.385303in}{0.565333in}}%
\pgfpathlineto{\pgfqpoint{1.420997in}{0.528000in}}%
\pgfpathlineto{\pgfqpoint{1.424968in}{0.528000in}}%
\pgfpathlineto{\pgfqpoint{1.280970in}{0.679558in}}%
\pgfpathlineto{\pgfqpoint{1.109341in}{0.864000in}}%
\pgfpathlineto{\pgfqpoint{0.800000in}{1.206967in}}%
\pgfpathmoveto{\pgfqpoint{4.768000in}{2.306691in}}%
\pgfpathlineto{\pgfqpoint{4.607677in}{2.480792in}}%
\pgfpathlineto{\pgfqpoint{4.442957in}{2.656000in}}%
\pgfpathlineto{\pgfqpoint{4.117157in}{2.992000in}}%
\pgfpathlineto{\pgfqpoint{3.966384in}{3.143052in}}%
\pgfpathlineto{\pgfqpoint{3.645737in}{3.455062in}}%
\pgfpathlineto{\pgfqpoint{3.485414in}{3.606586in}}%
\pgfpathlineto{\pgfqpoint{3.325091in}{3.755170in}}%
\pgfpathlineto{\pgfqpoint{3.164768in}{3.900850in}}%
\pgfpathlineto{\pgfqpoint{2.840544in}{4.186667in}}%
\pgfpathlineto{\pgfqpoint{2.797153in}{4.224000in}}%
\pgfpathlineto{\pgfqpoint{2.792537in}{4.224000in}}%
\pgfpathlineto{\pgfqpoint{2.924283in}{4.110056in}}%
\pgfpathlineto{\pgfqpoint{3.091349in}{3.962667in}}%
\pgfpathlineto{\pgfqpoint{3.419561in}{3.664000in}}%
\pgfpathlineto{\pgfqpoint{3.578834in}{3.514667in}}%
\pgfpathlineto{\pgfqpoint{3.888439in}{3.216000in}}%
\pgfpathlineto{\pgfqpoint{4.223187in}{2.880000in}}%
\pgfpathlineto{\pgfqpoint{4.536611in}{2.552473in}}%
\pgfpathlineto{\pgfqpoint{4.614415in}{2.469333in}}%
\pgfpathlineto{\pgfqpoint{4.768000in}{2.302396in}}%
\pgfpathlineto{\pgfqpoint{4.768000in}{2.302396in}}%
\pgfusepath{fill}%
\end{pgfscope}%
\begin{pgfscope}%
\pgfpathrectangle{\pgfqpoint{0.800000in}{0.528000in}}{\pgfqpoint{3.968000in}{3.696000in}}%
\pgfusepath{clip}%
\pgfsetbuttcap%
\pgfsetroundjoin%
\definecolor{currentfill}{rgb}{0.190631,0.407061,0.556089}%
\pgfsetfillcolor{currentfill}%
\pgfsetlinewidth{0.000000pt}%
\definecolor{currentstroke}{rgb}{0.000000,0.000000,0.000000}%
\pgfsetstrokecolor{currentstroke}%
\pgfsetdash{}{0pt}%
\pgfpathmoveto{\pgfqpoint{0.800000in}{1.198151in}}%
\pgfpathlineto{\pgfqpoint{1.101526in}{0.864000in}}%
\pgfpathlineto{\pgfqpoint{1.417026in}{0.528000in}}%
\pgfpathlineto{\pgfqpoint{1.420997in}{0.528000in}}%
\pgfpathlineto{\pgfqpoint{1.266084in}{0.691198in}}%
\pgfpathlineto{\pgfqpoint{1.105434in}{0.864000in}}%
\pgfpathlineto{\pgfqpoint{0.800000in}{1.202544in}}%
\pgfpathlineto{\pgfqpoint{0.800000in}{1.200000in}}%
\pgfpathmoveto{\pgfqpoint{4.768000in}{2.310986in}}%
\pgfpathlineto{\pgfqpoint{4.615428in}{2.476553in}}%
\pgfpathlineto{\pgfqpoint{4.540787in}{2.556362in}}%
\pgfpathlineto{\pgfqpoint{4.465791in}{2.635840in}}%
\pgfpathlineto{\pgfqpoint{4.375639in}{2.730667in}}%
\pgfpathlineto{\pgfqpoint{4.084217in}{3.029333in}}%
\pgfpathlineto{\pgfqpoint{3.754374in}{3.354523in}}%
\pgfpathlineto{\pgfqpoint{3.665655in}{3.440000in}}%
\pgfpathlineto{\pgfqpoint{3.347336in}{3.738667in}}%
\pgfpathlineto{\pgfqpoint{3.044525in}{4.012146in}}%
\pgfpathlineto{\pgfqpoint{2.801770in}{4.224000in}}%
\pgfpathlineto{\pgfqpoint{2.797153in}{4.224000in}}%
\pgfpathlineto{\pgfqpoint{2.926565in}{4.112000in}}%
\pgfpathlineto{\pgfqpoint{3.095750in}{3.962667in}}%
\pgfpathlineto{\pgfqpoint{3.414459in}{3.672575in}}%
\pgfpathlineto{\pgfqpoint{3.503813in}{3.589333in}}%
\pgfpathlineto{\pgfqpoint{3.816214in}{3.290667in}}%
\pgfpathlineto{\pgfqpoint{3.968114in}{3.141333in}}%
\pgfpathlineto{\pgfqpoint{4.299752in}{2.805333in}}%
\pgfpathlineto{\pgfqpoint{4.618326in}{2.469333in}}%
\pgfpathlineto{\pgfqpoint{4.768000in}{2.306691in}}%
\pgfpathlineto{\pgfqpoint{4.768000in}{2.306691in}}%
\pgfusepath{fill}%
\end{pgfscope}%
\begin{pgfscope}%
\pgfpathrectangle{\pgfqpoint{0.800000in}{0.528000in}}{\pgfqpoint{3.968000in}{3.696000in}}%
\pgfusepath{clip}%
\pgfsetbuttcap%
\pgfsetroundjoin%
\definecolor{currentfill}{rgb}{0.188923,0.410910,0.556326}%
\pgfsetfillcolor{currentfill}%
\pgfsetlinewidth{0.000000pt}%
\definecolor{currentstroke}{rgb}{0.000000,0.000000,0.000000}%
\pgfsetstrokecolor{currentstroke}%
\pgfsetdash{}{0pt}%
\pgfpathmoveto{\pgfqpoint{0.800000in}{1.193801in}}%
\pgfpathlineto{\pgfqpoint{1.097619in}{0.864000in}}%
\pgfpathlineto{\pgfqpoint{1.413056in}{0.528000in}}%
\pgfpathlineto{\pgfqpoint{1.417026in}{0.528000in}}%
\pgfpathlineto{\pgfqpoint{1.275203in}{0.677333in}}%
\pgfpathlineto{\pgfqpoint{1.120646in}{0.843235in}}%
\pgfpathlineto{\pgfqpoint{0.831484in}{1.162667in}}%
\pgfpathlineto{\pgfqpoint{0.800000in}{1.198151in}}%
\pgfpathmoveto{\pgfqpoint{4.768000in}{2.315281in}}%
\pgfpathlineto{\pgfqpoint{4.626149in}{2.469333in}}%
\pgfpathlineto{\pgfqpoint{4.316825in}{2.795753in}}%
\pgfpathlineto{\pgfqpoint{4.235198in}{2.880000in}}%
\pgfpathlineto{\pgfqpoint{4.086626in}{3.030972in}}%
\pgfpathlineto{\pgfqpoint{3.756493in}{3.356497in}}%
\pgfpathlineto{\pgfqpoint{3.669819in}{3.440000in}}%
\pgfpathlineto{\pgfqpoint{3.351632in}{3.738667in}}%
\pgfpathlineto{\pgfqpoint{3.044525in}{4.016068in}}%
\pgfpathlineto{\pgfqpoint{2.806341in}{4.224000in}}%
\pgfpathlineto{\pgfqpoint{2.801770in}{4.224000in}}%
\pgfpathlineto{\pgfqpoint{2.931038in}{4.112000in}}%
\pgfpathlineto{\pgfqpoint{3.100150in}{3.962667in}}%
\pgfpathlineto{\pgfqpoint{3.405253in}{3.685201in}}%
\pgfpathlineto{\pgfqpoint{3.725899in}{3.382175in}}%
\pgfpathlineto{\pgfqpoint{3.896579in}{3.216000in}}%
\pgfpathlineto{\pgfqpoint{4.206869in}{2.904910in}}%
\pgfpathlineto{\pgfqpoint{4.375639in}{2.730667in}}%
\pgfpathlineto{\pgfqpoint{4.527515in}{2.570694in}}%
\pgfpathlineto{\pgfqpoint{4.768000in}{2.310986in}}%
\pgfpathlineto{\pgfqpoint{4.768000in}{2.310986in}}%
\pgfusepath{fill}%
\end{pgfscope}%
\begin{pgfscope}%
\pgfpathrectangle{\pgfqpoint{0.800000in}{0.528000in}}{\pgfqpoint{3.968000in}{3.696000in}}%
\pgfusepath{clip}%
\pgfsetbuttcap%
\pgfsetroundjoin%
\definecolor{currentfill}{rgb}{0.188923,0.410910,0.556326}%
\pgfsetfillcolor{currentfill}%
\pgfsetlinewidth{0.000000pt}%
\definecolor{currentstroke}{rgb}{0.000000,0.000000,0.000000}%
\pgfsetstrokecolor{currentstroke}%
\pgfsetdash{}{0pt}%
\pgfpathmoveto{\pgfqpoint{0.800000in}{1.189451in}}%
\pgfpathlineto{\pgfqpoint{1.093711in}{0.864000in}}%
\pgfpathlineto{\pgfqpoint{1.409085in}{0.528000in}}%
\pgfpathlineto{\pgfqpoint{1.413056in}{0.528000in}}%
\pgfpathlineto{\pgfqpoint{1.271290in}{0.677333in}}%
\pgfpathlineto{\pgfqpoint{1.120646in}{0.838991in}}%
\pgfpathlineto{\pgfqpoint{0.827624in}{1.162667in}}%
\pgfpathlineto{\pgfqpoint{0.800000in}{1.193801in}}%
\pgfpathmoveto{\pgfqpoint{4.768000in}{2.319575in}}%
\pgfpathlineto{\pgfqpoint{4.630060in}{2.469333in}}%
\pgfpathlineto{\pgfqpoint{4.311675in}{2.805333in}}%
\pgfpathlineto{\pgfqpoint{3.992828in}{3.128631in}}%
\pgfpathlineto{\pgfqpoint{3.904720in}{3.216000in}}%
\pgfpathlineto{\pgfqpoint{3.751610in}{3.365333in}}%
\pgfpathlineto{\pgfqpoint{3.595618in}{3.514667in}}%
\pgfpathlineto{\pgfqpoint{3.274437in}{3.813333in}}%
\pgfpathlineto{\pgfqpoint{3.108950in}{3.962667in}}%
\pgfpathlineto{\pgfqpoint{2.939983in}{4.112000in}}%
\pgfpathlineto{\pgfqpoint{2.810870in}{4.224000in}}%
\pgfpathlineto{\pgfqpoint{2.806341in}{4.224000in}}%
\pgfpathlineto{\pgfqpoint{2.978106in}{4.074667in}}%
\pgfpathlineto{\pgfqpoint{3.285010in}{3.799753in}}%
\pgfpathlineto{\pgfqpoint{3.605657in}{3.501175in}}%
\pgfpathlineto{\pgfqpoint{3.765980in}{3.347454in}}%
\pgfpathlineto{\pgfqpoint{3.938507in}{3.178667in}}%
\pgfpathlineto{\pgfqpoint{4.259761in}{2.854600in}}%
\pgfpathlineto{\pgfqpoint{4.343719in}{2.768000in}}%
\pgfpathlineto{\pgfqpoint{4.487434in}{2.617427in}}%
\pgfpathlineto{\pgfqpoint{4.768000in}{2.315281in}}%
\pgfpathlineto{\pgfqpoint{4.768000in}{2.315281in}}%
\pgfusepath{fill}%
\end{pgfscope}%
\begin{pgfscope}%
\pgfpathrectangle{\pgfqpoint{0.800000in}{0.528000in}}{\pgfqpoint{3.968000in}{3.696000in}}%
\pgfusepath{clip}%
\pgfsetbuttcap%
\pgfsetroundjoin%
\definecolor{currentfill}{rgb}{0.188923,0.410910,0.556326}%
\pgfsetfillcolor{currentfill}%
\pgfsetlinewidth{0.000000pt}%
\definecolor{currentstroke}{rgb}{0.000000,0.000000,0.000000}%
\pgfsetstrokecolor{currentstroke}%
\pgfsetdash{}{0pt}%
\pgfpathmoveto{\pgfqpoint{0.800000in}{1.185101in}}%
\pgfpathlineto{\pgfqpoint{1.089804in}{0.864000in}}%
\pgfpathlineto{\pgfqpoint{1.405114in}{0.528000in}}%
\pgfpathlineto{\pgfqpoint{1.409085in}{0.528000in}}%
\pgfpathlineto{\pgfqpoint{1.267376in}{0.677333in}}%
\pgfpathlineto{\pgfqpoint{1.120646in}{0.834748in}}%
\pgfpathlineto{\pgfqpoint{0.823764in}{1.162667in}}%
\pgfpathlineto{\pgfqpoint{0.800000in}{1.189451in}}%
\pgfpathmoveto{\pgfqpoint{4.768000in}{2.323808in}}%
\pgfpathlineto{\pgfqpoint{4.447354in}{2.667948in}}%
\pgfpathlineto{\pgfqpoint{4.126707in}{2.998612in}}%
\pgfpathlineto{\pgfqpoint{3.946617in}{3.178667in}}%
\pgfpathlineto{\pgfqpoint{3.794269in}{3.328000in}}%
\pgfpathlineto{\pgfqpoint{3.639073in}{3.477333in}}%
\pgfpathlineto{\pgfqpoint{3.480897in}{3.626667in}}%
\pgfpathlineto{\pgfqpoint{3.155024in}{3.925333in}}%
\pgfpathlineto{\pgfqpoint{2.987015in}{4.074667in}}%
\pgfpathlineto{\pgfqpoint{2.844121in}{4.199242in}}%
\pgfpathlineto{\pgfqpoint{2.815398in}{4.224000in}}%
\pgfpathlineto{\pgfqpoint{2.810870in}{4.224000in}}%
\pgfpathlineto{\pgfqpoint{2.982560in}{4.074667in}}%
\pgfpathlineto{\pgfqpoint{3.285010in}{3.803698in}}%
\pgfpathlineto{\pgfqpoint{3.605657in}{3.505152in}}%
\pgfpathlineto{\pgfqpoint{3.765980in}{3.351447in}}%
\pgfpathlineto{\pgfqpoint{3.942562in}{3.178667in}}%
\pgfpathlineto{\pgfqpoint{4.246949in}{2.872058in}}%
\pgfpathlineto{\pgfqpoint{4.567596in}{2.536358in}}%
\pgfpathlineto{\pgfqpoint{4.768000in}{2.319575in}}%
\pgfpathlineto{\pgfqpoint{4.768000in}{2.320000in}}%
\pgfusepath{fill}%
\end{pgfscope}%
\begin{pgfscope}%
\pgfpathrectangle{\pgfqpoint{0.800000in}{0.528000in}}{\pgfqpoint{3.968000in}{3.696000in}}%
\pgfusepath{clip}%
\pgfsetbuttcap%
\pgfsetroundjoin%
\definecolor{currentfill}{rgb}{0.188923,0.410910,0.556326}%
\pgfsetfillcolor{currentfill}%
\pgfsetlinewidth{0.000000pt}%
\definecolor{currentstroke}{rgb}{0.000000,0.000000,0.000000}%
\pgfsetstrokecolor{currentstroke}%
\pgfsetdash{}{0pt}%
\pgfpathmoveto{\pgfqpoint{0.800000in}{1.180750in}}%
\pgfpathlineto{\pgfqpoint{1.085897in}{0.864000in}}%
\pgfpathlineto{\pgfqpoint{1.401212in}{0.528000in}}%
\pgfpathlineto{\pgfqpoint{1.405114in}{0.528000in}}%
\pgfpathlineto{\pgfqpoint{1.263463in}{0.677333in}}%
\pgfpathlineto{\pgfqpoint{1.120646in}{0.830504in}}%
\pgfpathlineto{\pgfqpoint{0.819904in}{1.162667in}}%
\pgfpathlineto{\pgfqpoint{0.800000in}{1.185101in}}%
\pgfpathmoveto{\pgfqpoint{4.768000in}{2.328034in}}%
\pgfpathlineto{\pgfqpoint{4.462615in}{2.656000in}}%
\pgfpathlineto{\pgfqpoint{4.319624in}{2.805333in}}%
\pgfpathlineto{\pgfqpoint{3.988314in}{3.141333in}}%
\pgfpathlineto{\pgfqpoint{3.836719in}{3.290667in}}%
\pgfpathlineto{\pgfqpoint{3.682310in}{3.440000in}}%
\pgfpathlineto{\pgfqpoint{3.364519in}{3.738667in}}%
\pgfpathlineto{\pgfqpoint{3.033784in}{4.037333in}}%
\pgfpathlineto{\pgfqpoint{2.863160in}{4.186667in}}%
\pgfpathlineto{\pgfqpoint{2.819927in}{4.224000in}}%
\pgfpathlineto{\pgfqpoint{2.815398in}{4.224000in}}%
\pgfpathlineto{\pgfqpoint{2.987015in}{4.074667in}}%
\pgfpathlineto{\pgfqpoint{3.285010in}{3.807644in}}%
\pgfpathlineto{\pgfqpoint{3.605657in}{3.509129in}}%
\pgfpathlineto{\pgfqpoint{3.765980in}{3.355440in}}%
\pgfpathlineto{\pgfqpoint{3.926303in}{3.198751in}}%
\pgfpathlineto{\pgfqpoint{4.096244in}{3.029333in}}%
\pgfpathlineto{\pgfqpoint{4.246949in}{2.876162in}}%
\pgfpathlineto{\pgfqpoint{4.567596in}{2.540562in}}%
\pgfpathlineto{\pgfqpoint{4.768000in}{2.323808in}}%
\pgfpathlineto{\pgfqpoint{4.768000in}{2.323808in}}%
\pgfusepath{fill}%
\end{pgfscope}%
\begin{pgfscope}%
\pgfpathrectangle{\pgfqpoint{0.800000in}{0.528000in}}{\pgfqpoint{3.968000in}{3.696000in}}%
\pgfusepath{clip}%
\pgfsetbuttcap%
\pgfsetroundjoin%
\definecolor{currentfill}{rgb}{0.187231,0.414746,0.556547}%
\pgfsetfillcolor{currentfill}%
\pgfsetlinewidth{0.000000pt}%
\definecolor{currentstroke}{rgb}{0.000000,0.000000,0.000000}%
\pgfsetstrokecolor{currentstroke}%
\pgfsetdash{}{0pt}%
\pgfpathmoveto{\pgfqpoint{0.800000in}{1.176400in}}%
\pgfpathlineto{\pgfqpoint{1.081989in}{0.864000in}}%
\pgfpathlineto{\pgfqpoint{1.397239in}{0.528000in}}%
\pgfpathlineto{\pgfqpoint{1.401144in}{0.528000in}}%
\pgfpathlineto{\pgfqpoint{1.080566in}{0.869796in}}%
\pgfpathlineto{\pgfqpoint{0.916082in}{1.050667in}}%
\pgfpathlineto{\pgfqpoint{0.800000in}{1.180750in}}%
\pgfpathmoveto{\pgfqpoint{4.768000in}{2.332260in}}%
\pgfpathlineto{\pgfqpoint{4.466533in}{2.656000in}}%
\pgfpathlineto{\pgfqpoint{4.323599in}{2.805333in}}%
\pgfpathlineto{\pgfqpoint{3.992354in}{3.141333in}}%
\pgfpathlineto{\pgfqpoint{3.840820in}{3.290667in}}%
\pgfpathlineto{\pgfqpoint{3.685818in}{3.440619in}}%
\pgfpathlineto{\pgfqpoint{3.365172in}{3.741970in}}%
\pgfpathlineto{\pgfqpoint{3.038220in}{4.037333in}}%
\pgfpathlineto{\pgfqpoint{2.867670in}{4.186667in}}%
\pgfpathlineto{\pgfqpoint{2.824455in}{4.224000in}}%
\pgfpathlineto{\pgfqpoint{2.819927in}{4.224000in}}%
\pgfpathlineto{\pgfqpoint{2.964364in}{4.098489in}}%
\pgfpathlineto{\pgfqpoint{3.124687in}{3.956471in}}%
\pgfpathlineto{\pgfqpoint{3.285010in}{3.811590in}}%
\pgfpathlineto{\pgfqpoint{3.605657in}{3.513106in}}%
\pgfpathlineto{\pgfqpoint{3.765980in}{3.359433in}}%
\pgfpathlineto{\pgfqpoint{3.926303in}{3.202760in}}%
\pgfpathlineto{\pgfqpoint{4.100239in}{3.029333in}}%
\pgfpathlineto{\pgfqpoint{4.247205in}{2.880000in}}%
\pgfpathlineto{\pgfqpoint{4.568302in}{2.544000in}}%
\pgfpathlineto{\pgfqpoint{4.768000in}{2.328034in}}%
\pgfpathlineto{\pgfqpoint{4.768000in}{2.328034in}}%
\pgfusepath{fill}%
\end{pgfscope}%
\begin{pgfscope}%
\pgfpathrectangle{\pgfqpoint{0.800000in}{0.528000in}}{\pgfqpoint{3.968000in}{3.696000in}}%
\pgfusepath{clip}%
\pgfsetbuttcap%
\pgfsetroundjoin%
\definecolor{currentfill}{rgb}{0.187231,0.414746,0.556547}%
\pgfsetfillcolor{currentfill}%
\pgfsetlinewidth{0.000000pt}%
\definecolor{currentstroke}{rgb}{0.000000,0.000000,0.000000}%
\pgfsetstrokecolor{currentstroke}%
\pgfsetdash{}{0pt}%
\pgfpathmoveto{\pgfqpoint{0.800000in}{1.172050in}}%
\pgfpathlineto{\pgfqpoint{1.080566in}{0.861342in}}%
\pgfpathlineto{\pgfqpoint{1.393334in}{0.528000in}}%
\pgfpathlineto{\pgfqpoint{1.397239in}{0.528000in}}%
\pgfpathlineto{\pgfqpoint{1.080566in}{0.865548in}}%
\pgfpathlineto{\pgfqpoint{0.912244in}{1.050667in}}%
\pgfpathlineto{\pgfqpoint{0.800000in}{1.176400in}}%
\pgfpathmoveto{\pgfqpoint{4.768000in}{2.336487in}}%
\pgfpathlineto{\pgfqpoint{4.470450in}{2.656000in}}%
\pgfpathlineto{\pgfqpoint{4.327111in}{2.805804in}}%
\pgfpathlineto{\pgfqpoint{3.996394in}{3.141333in}}%
\pgfpathlineto{\pgfqpoint{3.844921in}{3.290667in}}%
\pgfpathlineto{\pgfqpoint{3.685818in}{3.444544in}}%
\pgfpathlineto{\pgfqpoint{3.365172in}{3.745865in}}%
\pgfpathlineto{\pgfqpoint{3.042656in}{4.037333in}}%
\pgfpathlineto{\pgfqpoint{2.872180in}{4.186667in}}%
\pgfpathlineto{\pgfqpoint{2.828984in}{4.224000in}}%
\pgfpathlineto{\pgfqpoint{2.824455in}{4.224000in}}%
\pgfpathlineto{\pgfqpoint{2.964364in}{4.102404in}}%
\pgfpathlineto{\pgfqpoint{3.124687in}{3.960401in}}%
\pgfpathlineto{\pgfqpoint{3.287384in}{3.813333in}}%
\pgfpathlineto{\pgfqpoint{3.608162in}{3.514667in}}%
\pgfpathlineto{\pgfqpoint{3.765980in}{3.363427in}}%
\pgfpathlineto{\pgfqpoint{3.926303in}{3.206769in}}%
\pgfpathlineto{\pgfqpoint{4.104235in}{3.029333in}}%
\pgfpathlineto{\pgfqpoint{4.251142in}{2.880000in}}%
\pgfpathlineto{\pgfqpoint{4.572177in}{2.544000in}}%
\pgfpathlineto{\pgfqpoint{4.768000in}{2.332260in}}%
\pgfpathlineto{\pgfqpoint{4.768000in}{2.332260in}}%
\pgfusepath{fill}%
\end{pgfscope}%
\begin{pgfscope}%
\pgfpathrectangle{\pgfqpoint{0.800000in}{0.528000in}}{\pgfqpoint{3.968000in}{3.696000in}}%
\pgfusepath{clip}%
\pgfsetbuttcap%
\pgfsetroundjoin%
\definecolor{currentfill}{rgb}{0.187231,0.414746,0.556547}%
\pgfsetfillcolor{currentfill}%
\pgfsetlinewidth{0.000000pt}%
\definecolor{currentstroke}{rgb}{0.000000,0.000000,0.000000}%
\pgfsetstrokecolor{currentstroke}%
\pgfsetdash{}{0pt}%
\pgfpathmoveto{\pgfqpoint{0.800000in}{1.167699in}}%
\pgfpathlineto{\pgfqpoint{1.074278in}{0.864000in}}%
\pgfpathlineto{\pgfqpoint{1.212819in}{0.714667in}}%
\pgfpathlineto{\pgfqpoint{1.389428in}{0.528000in}}%
\pgfpathlineto{\pgfqpoint{1.393334in}{0.528000in}}%
\pgfpathlineto{\pgfqpoint{1.078122in}{0.864000in}}%
\pgfpathlineto{\pgfqpoint{0.920242in}{1.037510in}}%
\pgfpathlineto{\pgfqpoint{0.800000in}{1.172050in}}%
\pgfpathmoveto{\pgfqpoint{4.768000in}{2.340713in}}%
\pgfpathlineto{\pgfqpoint{4.474367in}{2.656000in}}%
\pgfpathlineto{\pgfqpoint{4.327111in}{2.809854in}}%
\pgfpathlineto{\pgfqpoint{4.000434in}{3.141333in}}%
\pgfpathlineto{\pgfqpoint{3.846141in}{3.293435in}}%
\pgfpathlineto{\pgfqpoint{3.685818in}{3.448470in}}%
\pgfpathlineto{\pgfqpoint{3.365172in}{3.749761in}}%
\pgfpathlineto{\pgfqpoint{3.044525in}{4.039570in}}%
\pgfpathlineto{\pgfqpoint{2.876689in}{4.186667in}}%
\pgfpathlineto{\pgfqpoint{2.833513in}{4.224000in}}%
\pgfpathlineto{\pgfqpoint{2.828984in}{4.224000in}}%
\pgfpathlineto{\pgfqpoint{2.964364in}{4.106319in}}%
\pgfpathlineto{\pgfqpoint{3.126516in}{3.962667in}}%
\pgfpathlineto{\pgfqpoint{3.291636in}{3.813333in}}%
\pgfpathlineto{\pgfqpoint{3.612285in}{3.514667in}}%
\pgfpathlineto{\pgfqpoint{3.768102in}{3.365333in}}%
\pgfpathlineto{\pgfqpoint{3.926303in}{3.210778in}}%
\pgfpathlineto{\pgfqpoint{4.097776in}{3.039718in}}%
\pgfpathlineto{\pgfqpoint{4.181973in}{2.954667in}}%
\pgfpathlineto{\pgfqpoint{4.331576in}{2.801175in}}%
\pgfpathlineto{\pgfqpoint{4.505796in}{2.618667in}}%
\pgfpathlineto{\pgfqpoint{4.768000in}{2.336487in}}%
\pgfpathlineto{\pgfqpoint{4.768000in}{2.336487in}}%
\pgfusepath{fill}%
\end{pgfscope}%
\begin{pgfscope}%
\pgfpathrectangle{\pgfqpoint{0.800000in}{0.528000in}}{\pgfqpoint{3.968000in}{3.696000in}}%
\pgfusepath{clip}%
\pgfsetbuttcap%
\pgfsetroundjoin%
\definecolor{currentfill}{rgb}{0.187231,0.414746,0.556547}%
\pgfsetfillcolor{currentfill}%
\pgfsetlinewidth{0.000000pt}%
\definecolor{currentstroke}{rgb}{0.000000,0.000000,0.000000}%
\pgfsetstrokecolor{currentstroke}%
\pgfsetdash{}{0pt}%
\pgfpathmoveto{\pgfqpoint{0.800000in}{1.163349in}}%
\pgfpathlineto{\pgfqpoint{1.070433in}{0.864000in}}%
\pgfpathlineto{\pgfqpoint{1.208920in}{0.714667in}}%
\pgfpathlineto{\pgfqpoint{1.385523in}{0.528000in}}%
\pgfpathlineto{\pgfqpoint{1.389428in}{0.528000in}}%
\pgfpathlineto{\pgfqpoint{1.074278in}{0.864000in}}%
\pgfpathlineto{\pgfqpoint{0.920242in}{1.033244in}}%
\pgfpathlineto{\pgfqpoint{0.800000in}{1.167699in}}%
\pgfpathmoveto{\pgfqpoint{4.768000in}{2.344939in}}%
\pgfpathlineto{\pgfqpoint{4.478284in}{2.656000in}}%
\pgfpathlineto{\pgfqpoint{4.327111in}{2.813904in}}%
\pgfpathlineto{\pgfqpoint{4.004474in}{3.141333in}}%
\pgfpathlineto{\pgfqpoint{3.846141in}{3.297377in}}%
\pgfpathlineto{\pgfqpoint{3.685818in}{3.452396in}}%
\pgfpathlineto{\pgfqpoint{3.353017in}{3.764678in}}%
\pgfpathlineto{\pgfqpoint{3.259205in}{3.850667in}}%
\pgfpathlineto{\pgfqpoint{2.964364in}{4.114117in}}%
\pgfpathlineto{\pgfqpoint{2.838041in}{4.224000in}}%
\pgfpathlineto{\pgfqpoint{2.833513in}{4.224000in}}%
\pgfpathlineto{\pgfqpoint{2.964364in}{4.110234in}}%
\pgfpathlineto{\pgfqpoint{3.130836in}{3.962667in}}%
\pgfpathlineto{\pgfqpoint{3.295888in}{3.813333in}}%
\pgfpathlineto{\pgfqpoint{3.616409in}{3.514667in}}%
\pgfpathlineto{\pgfqpoint{3.772163in}{3.365333in}}%
\pgfpathlineto{\pgfqpoint{3.926303in}{3.214787in}}%
\pgfpathlineto{\pgfqpoint{4.086626in}{3.055124in}}%
\pgfpathlineto{\pgfqpoint{4.407273in}{2.726474in}}%
\pgfpathlineto{\pgfqpoint{4.579927in}{2.544000in}}%
\pgfpathlineto{\pgfqpoint{4.768000in}{2.340713in}}%
\pgfpathlineto{\pgfqpoint{4.768000in}{2.340713in}}%
\pgfusepath{fill}%
\end{pgfscope}%
\begin{pgfscope}%
\pgfpathrectangle{\pgfqpoint{0.800000in}{0.528000in}}{\pgfqpoint{3.968000in}{3.696000in}}%
\pgfusepath{clip}%
\pgfsetbuttcap%
\pgfsetroundjoin%
\definecolor{currentfill}{rgb}{0.185556,0.418570,0.556753}%
\pgfsetfillcolor{currentfill}%
\pgfsetlinewidth{0.000000pt}%
\definecolor{currentstroke}{rgb}{0.000000,0.000000,0.000000}%
\pgfsetstrokecolor{currentstroke}%
\pgfsetdash{}{0pt}%
\pgfpathmoveto{\pgfqpoint{0.800000in}{1.159058in}}%
\pgfpathlineto{\pgfqpoint{1.100988in}{0.826667in}}%
\pgfpathlineto{\pgfqpoint{1.240889in}{0.676383in}}%
\pgfpathlineto{\pgfqpoint{1.381617in}{0.528000in}}%
\pgfpathlineto{\pgfqpoint{1.385523in}{0.528000in}}%
\pgfpathlineto{\pgfqpoint{1.070433in}{0.864000in}}%
\pgfpathlineto{\pgfqpoint{0.920242in}{1.028978in}}%
\pgfpathlineto{\pgfqpoint{0.800000in}{1.163349in}}%
\pgfpathlineto{\pgfqpoint{0.800000in}{1.162667in}}%
\pgfpathmoveto{\pgfqpoint{4.768000in}{2.349165in}}%
\pgfpathlineto{\pgfqpoint{4.482201in}{2.656000in}}%
\pgfpathlineto{\pgfqpoint{4.327111in}{2.817953in}}%
\pgfpathlineto{\pgfqpoint{4.006465in}{3.143341in}}%
\pgfpathlineto{\pgfqpoint{3.846141in}{3.301318in}}%
\pgfpathlineto{\pgfqpoint{3.674608in}{3.466892in}}%
\pgfpathlineto{\pgfqpoint{3.585291in}{3.552000in}}%
\pgfpathlineto{\pgfqpoint{3.285010in}{3.831054in}}%
\pgfpathlineto{\pgfqpoint{3.124687in}{3.975925in}}%
\pgfpathlineto{\pgfqpoint{2.964364in}{4.117975in}}%
\pgfpathlineto{\pgfqpoint{2.842570in}{4.224000in}}%
\pgfpathlineto{\pgfqpoint{2.838041in}{4.224000in}}%
\pgfpathlineto{\pgfqpoint{2.966773in}{4.112000in}}%
\pgfpathlineto{\pgfqpoint{3.135157in}{3.962667in}}%
\pgfpathlineto{\pgfqpoint{3.300141in}{3.813333in}}%
\pgfpathlineto{\pgfqpoint{3.620532in}{3.514667in}}%
\pgfpathlineto{\pgfqpoint{3.776225in}{3.365333in}}%
\pgfpathlineto{\pgfqpoint{3.929094in}{3.216000in}}%
\pgfpathlineto{\pgfqpoint{4.086626in}{3.059149in}}%
\pgfpathlineto{\pgfqpoint{4.407273in}{2.730595in}}%
\pgfpathlineto{\pgfqpoint{4.583803in}{2.544000in}}%
\pgfpathlineto{\pgfqpoint{4.768000in}{2.344939in}}%
\pgfpathlineto{\pgfqpoint{4.768000in}{2.344939in}}%
\pgfusepath{fill}%
\end{pgfscope}%
\begin{pgfscope}%
\pgfpathrectangle{\pgfqpoint{0.800000in}{0.528000in}}{\pgfqpoint{3.968000in}{3.696000in}}%
\pgfusepath{clip}%
\pgfsetbuttcap%
\pgfsetroundjoin%
\definecolor{currentfill}{rgb}{0.185556,0.418570,0.556753}%
\pgfsetfillcolor{currentfill}%
\pgfsetlinewidth{0.000000pt}%
\definecolor{currentstroke}{rgb}{0.000000,0.000000,0.000000}%
\pgfsetstrokecolor{currentstroke}%
\pgfsetdash{}{0pt}%
\pgfpathmoveto{\pgfqpoint{0.800000in}{1.154778in}}%
\pgfpathlineto{\pgfqpoint{1.097130in}{0.826667in}}%
\pgfpathlineto{\pgfqpoint{1.240889in}{0.672284in}}%
\pgfpathlineto{\pgfqpoint{1.377712in}{0.528000in}}%
\pgfpathlineto{\pgfqpoint{1.381617in}{0.528000in}}%
\pgfpathlineto{\pgfqpoint{1.066589in}{0.864000in}}%
\pgfpathlineto{\pgfqpoint{0.920242in}{1.024712in}}%
\pgfpathlineto{\pgfqpoint{0.800000in}{1.159058in}}%
\pgfpathmoveto{\pgfqpoint{4.768000in}{2.353391in}}%
\pgfpathlineto{\pgfqpoint{4.476046in}{2.666608in}}%
\pgfpathlineto{\pgfqpoint{4.307093in}{2.842667in}}%
\pgfpathlineto{\pgfqpoint{4.161105in}{2.992000in}}%
\pgfpathlineto{\pgfqpoint{3.846141in}{3.305260in}}%
\pgfpathlineto{\pgfqpoint{3.667942in}{3.477333in}}%
\pgfpathlineto{\pgfqpoint{3.349346in}{3.776000in}}%
\pgfpathlineto{\pgfqpoint{3.185321in}{3.925333in}}%
\pgfpathlineto{\pgfqpoint{2.884202in}{4.191803in}}%
\pgfpathlineto{\pgfqpoint{2.847042in}{4.224000in}}%
\pgfpathlineto{\pgfqpoint{2.842570in}{4.224000in}}%
\pgfpathlineto{\pgfqpoint{2.971163in}{4.112000in}}%
\pgfpathlineto{\pgfqpoint{3.139477in}{3.962667in}}%
\pgfpathlineto{\pgfqpoint{3.304393in}{3.813333in}}%
\pgfpathlineto{\pgfqpoint{3.624655in}{3.514667in}}%
\pgfpathlineto{\pgfqpoint{3.780287in}{3.365333in}}%
\pgfpathlineto{\pgfqpoint{3.933095in}{3.216000in}}%
\pgfpathlineto{\pgfqpoint{4.086626in}{3.063174in}}%
\pgfpathlineto{\pgfqpoint{4.411087in}{2.730667in}}%
\pgfpathlineto{\pgfqpoint{4.567596in}{2.565445in}}%
\pgfpathlineto{\pgfqpoint{4.727919in}{2.392905in}}%
\pgfpathlineto{\pgfqpoint{4.768000in}{2.349165in}}%
\pgfpathlineto{\pgfqpoint{4.768000in}{2.349165in}}%
\pgfusepath{fill}%
\end{pgfscope}%
\begin{pgfscope}%
\pgfpathrectangle{\pgfqpoint{0.800000in}{0.528000in}}{\pgfqpoint{3.968000in}{3.696000in}}%
\pgfusepath{clip}%
\pgfsetbuttcap%
\pgfsetroundjoin%
\definecolor{currentfill}{rgb}{0.185556,0.418570,0.556753}%
\pgfsetfillcolor{currentfill}%
\pgfsetlinewidth{0.000000pt}%
\definecolor{currentstroke}{rgb}{0.000000,0.000000,0.000000}%
\pgfsetstrokecolor{currentstroke}%
\pgfsetdash{}{0pt}%
\pgfpathmoveto{\pgfqpoint{0.800000in}{1.150498in}}%
\pgfpathlineto{\pgfqpoint{1.093272in}{0.826667in}}%
\pgfpathlineto{\pgfqpoint{1.240889in}{0.668185in}}%
\pgfpathlineto{\pgfqpoint{1.373806in}{0.528000in}}%
\pgfpathlineto{\pgfqpoint{1.377712in}{0.528000in}}%
\pgfpathlineto{\pgfqpoint{1.062745in}{0.864000in}}%
\pgfpathlineto{\pgfqpoint{0.920242in}{1.020445in}}%
\pgfpathlineto{\pgfqpoint{0.800000in}{1.154778in}}%
\pgfpathmoveto{\pgfqpoint{4.768000in}{2.357613in}}%
\pgfpathlineto{\pgfqpoint{4.447354in}{2.700834in}}%
\pgfpathlineto{\pgfqpoint{4.274767in}{2.880000in}}%
\pgfpathlineto{\pgfqpoint{4.126707in}{3.030824in}}%
\pgfpathlineto{\pgfqpoint{3.788410in}{3.365333in}}%
\pgfpathlineto{\pgfqpoint{3.474444in}{3.664000in}}%
\pgfpathlineto{\pgfqpoint{3.312898in}{3.813333in}}%
\pgfpathlineto{\pgfqpoint{3.004444in}{4.090448in}}%
\pgfpathlineto{\pgfqpoint{2.851486in}{4.224000in}}%
\pgfpathlineto{\pgfqpoint{2.847042in}{4.224000in}}%
\pgfpathlineto{\pgfqpoint{3.174803in}{3.934681in}}%
\pgfpathlineto{\pgfqpoint{3.267743in}{3.850667in}}%
\pgfpathlineto{\pgfqpoint{3.430150in}{3.701333in}}%
\pgfpathlineto{\pgfqpoint{3.735914in}{3.411995in}}%
\pgfpathlineto{\pgfqpoint{3.822795in}{3.328000in}}%
\pgfpathlineto{\pgfqpoint{3.974858in}{3.178667in}}%
\pgfpathlineto{\pgfqpoint{4.126707in}{3.026817in}}%
\pgfpathlineto{\pgfqpoint{4.450627in}{2.693333in}}%
\pgfpathlineto{\pgfqpoint{4.607677in}{2.526764in}}%
\pgfpathlineto{\pgfqpoint{4.768000in}{2.353391in}}%
\pgfpathlineto{\pgfqpoint{4.768000in}{2.357333in}}%
\pgfpathlineto{\pgfqpoint{4.768000in}{2.357333in}}%
\pgfusepath{fill}%
\end{pgfscope}%
\begin{pgfscope}%
\pgfpathrectangle{\pgfqpoint{0.800000in}{0.528000in}}{\pgfqpoint{3.968000in}{3.696000in}}%
\pgfusepath{clip}%
\pgfsetbuttcap%
\pgfsetroundjoin%
\definecolor{currentfill}{rgb}{0.183898,0.422383,0.556944}%
\pgfsetfillcolor{currentfill}%
\pgfsetlinewidth{0.000000pt}%
\definecolor{currentstroke}{rgb}{0.000000,0.000000,0.000000}%
\pgfsetstrokecolor{currentstroke}%
\pgfsetdash{}{0pt}%
\pgfpathmoveto{\pgfqpoint{0.800000in}{1.146218in}}%
\pgfpathlineto{\pgfqpoint{1.089414in}{0.826667in}}%
\pgfpathlineto{\pgfqpoint{1.240889in}{0.664086in}}%
\pgfpathlineto{\pgfqpoint{1.369901in}{0.528000in}}%
\pgfpathlineto{\pgfqpoint{1.373806in}{0.528000in}}%
\pgfpathlineto{\pgfqpoint{1.080566in}{0.840437in}}%
\pgfpathlineto{\pgfqpoint{0.920242in}{1.016179in}}%
\pgfpathlineto{\pgfqpoint{0.800000in}{1.150498in}}%
\pgfpathmoveto{\pgfqpoint{4.768000in}{2.361772in}}%
\pgfpathlineto{\pgfqpoint{4.447354in}{2.704896in}}%
\pgfpathlineto{\pgfqpoint{4.278705in}{2.880000in}}%
\pgfpathlineto{\pgfqpoint{4.126707in}{3.034792in}}%
\pgfpathlineto{\pgfqpoint{3.792471in}{3.365333in}}%
\pgfpathlineto{\pgfqpoint{3.478631in}{3.664000in}}%
\pgfpathlineto{\pgfqpoint{3.317151in}{3.813333in}}%
\pgfpathlineto{\pgfqpoint{2.984333in}{4.112000in}}%
\pgfpathlineto{\pgfqpoint{2.855930in}{4.224000in}}%
\pgfpathlineto{\pgfqpoint{2.851486in}{4.224000in}}%
\pgfpathlineto{\pgfqpoint{3.164768in}{3.947726in}}%
\pgfpathlineto{\pgfqpoint{3.485414in}{3.653764in}}%
\pgfpathlineto{\pgfqpoint{3.816607in}{3.337823in}}%
\pgfpathlineto{\pgfqpoint{3.903184in}{3.253333in}}%
\pgfpathlineto{\pgfqpoint{4.053840in}{3.104000in}}%
\pgfpathlineto{\pgfqpoint{4.206869in}{2.949511in}}%
\pgfpathlineto{\pgfqpoint{4.527515in}{2.616332in}}%
\pgfpathlineto{\pgfqpoint{4.699540in}{2.432000in}}%
\pgfpathlineto{\pgfqpoint{4.768000in}{2.357613in}}%
\pgfpathlineto{\pgfqpoint{4.768000in}{2.357613in}}%
\pgfusepath{fill}%
\end{pgfscope}%
\begin{pgfscope}%
\pgfpathrectangle{\pgfqpoint{0.800000in}{0.528000in}}{\pgfqpoint{3.968000in}{3.696000in}}%
\pgfusepath{clip}%
\pgfsetbuttcap%
\pgfsetroundjoin%
\definecolor{currentfill}{rgb}{0.183898,0.422383,0.556944}%
\pgfsetfillcolor{currentfill}%
\pgfsetlinewidth{0.000000pt}%
\definecolor{currentstroke}{rgb}{0.000000,0.000000,0.000000}%
\pgfsetstrokecolor{currentstroke}%
\pgfsetdash{}{0pt}%
\pgfpathmoveto{\pgfqpoint{0.800000in}{1.141938in}}%
\pgfpathlineto{\pgfqpoint{1.085556in}{0.826667in}}%
\pgfpathlineto{\pgfqpoint{1.240889in}{0.659987in}}%
\pgfpathlineto{\pgfqpoint{1.365995in}{0.528000in}}%
\pgfpathlineto{\pgfqpoint{1.369901in}{0.528000in}}%
\pgfpathlineto{\pgfqpoint{1.080566in}{0.836256in}}%
\pgfpathlineto{\pgfqpoint{0.918980in}{1.013333in}}%
\pgfpathlineto{\pgfqpoint{0.800000in}{1.146218in}}%
\pgfpathmoveto{\pgfqpoint{4.768000in}{2.365932in}}%
\pgfpathlineto{\pgfqpoint{4.455185in}{2.700628in}}%
\pgfpathlineto{\pgfqpoint{4.367192in}{2.792617in}}%
\pgfpathlineto{\pgfqpoint{4.206869in}{2.957542in}}%
\pgfpathlineto{\pgfqpoint{3.873162in}{3.290667in}}%
\pgfpathlineto{\pgfqpoint{3.719201in}{3.440000in}}%
\pgfpathlineto{\pgfqpoint{3.402505in}{3.738667in}}%
\pgfpathlineto{\pgfqpoint{3.239494in}{3.888000in}}%
\pgfpathlineto{\pgfqpoint{3.073174in}{4.037333in}}%
\pgfpathlineto{\pgfqpoint{2.860374in}{4.224000in}}%
\pgfpathlineto{\pgfqpoint{2.855930in}{4.224000in}}%
\pgfpathlineto{\pgfqpoint{3.164768in}{3.951602in}}%
\pgfpathlineto{\pgfqpoint{3.485414in}{3.657670in}}%
\pgfpathlineto{\pgfqpoint{3.818660in}{3.339736in}}%
\pgfpathlineto{\pgfqpoint{3.907201in}{3.253333in}}%
\pgfpathlineto{\pgfqpoint{4.057798in}{3.104000in}}%
\pgfpathlineto{\pgfqpoint{4.206869in}{2.953548in}}%
\pgfpathlineto{\pgfqpoint{4.529187in}{2.618667in}}%
\pgfpathlineto{\pgfqpoint{4.687838in}{2.448820in}}%
\pgfpathlineto{\pgfqpoint{4.768000in}{2.361772in}}%
\pgfpathlineto{\pgfqpoint{4.768000in}{2.361772in}}%
\pgfusepath{fill}%
\end{pgfscope}%
\begin{pgfscope}%
\pgfpathrectangle{\pgfqpoint{0.800000in}{0.528000in}}{\pgfqpoint{3.968000in}{3.696000in}}%
\pgfusepath{clip}%
\pgfsetbuttcap%
\pgfsetroundjoin%
\definecolor{currentfill}{rgb}{0.183898,0.422383,0.556944}%
\pgfsetfillcolor{currentfill}%
\pgfsetlinewidth{0.000000pt}%
\definecolor{currentstroke}{rgb}{0.000000,0.000000,0.000000}%
\pgfsetstrokecolor{currentstroke}%
\pgfsetdash{}{0pt}%
\pgfpathmoveto{\pgfqpoint{0.800000in}{1.137658in}}%
\pgfpathlineto{\pgfqpoint{1.088677in}{0.819111in}}%
\pgfpathlineto{\pgfqpoint{1.255864in}{0.640000in}}%
\pgfpathlineto{\pgfqpoint{1.362090in}{0.528000in}}%
\pgfpathlineto{\pgfqpoint{1.365995in}{0.528000in}}%
\pgfpathlineto{\pgfqpoint{1.080566in}{0.832075in}}%
\pgfpathlineto{\pgfqpoint{0.915190in}{1.013333in}}%
\pgfpathlineto{\pgfqpoint{0.800000in}{1.141938in}}%
\pgfpathmoveto{\pgfqpoint{4.768000in}{2.370092in}}%
\pgfpathlineto{\pgfqpoint{4.466096in}{2.693333in}}%
\pgfpathlineto{\pgfqpoint{4.152735in}{3.016244in}}%
\pgfpathlineto{\pgfqpoint{4.065714in}{3.104000in}}%
\pgfpathlineto{\pgfqpoint{3.762033in}{3.402667in}}%
\pgfpathlineto{\pgfqpoint{3.445333in}{3.702824in}}%
\pgfpathlineto{\pgfqpoint{3.284820in}{3.850667in}}%
\pgfpathlineto{\pgfqpoint{2.950574in}{4.149333in}}%
\pgfpathlineto{\pgfqpoint{2.864818in}{4.224000in}}%
\pgfpathlineto{\pgfqpoint{2.860374in}{4.224000in}}%
\pgfpathlineto{\pgfqpoint{3.164768in}{3.955479in}}%
\pgfpathlineto{\pgfqpoint{3.485414in}{3.661577in}}%
\pgfpathlineto{\pgfqpoint{3.806061in}{3.356097in}}%
\pgfpathlineto{\pgfqpoint{4.136041in}{3.029333in}}%
\pgfpathlineto{\pgfqpoint{4.462229in}{2.693333in}}%
\pgfpathlineto{\pgfqpoint{4.607677in}{2.539192in}}%
\pgfpathlineto{\pgfqpoint{4.768000in}{2.365932in}}%
\pgfpathlineto{\pgfqpoint{4.768000in}{2.365932in}}%
\pgfusepath{fill}%
\end{pgfscope}%
\begin{pgfscope}%
\pgfpathrectangle{\pgfqpoint{0.800000in}{0.528000in}}{\pgfqpoint{3.968000in}{3.696000in}}%
\pgfusepath{clip}%
\pgfsetbuttcap%
\pgfsetroundjoin%
\definecolor{currentfill}{rgb}{0.183898,0.422383,0.556944}%
\pgfsetfillcolor{currentfill}%
\pgfsetlinewidth{0.000000pt}%
\definecolor{currentstroke}{rgb}{0.000000,0.000000,0.000000}%
\pgfsetstrokecolor{currentstroke}%
\pgfsetdash{}{0pt}%
\pgfpathmoveto{\pgfqpoint{0.800000in}{1.133378in}}%
\pgfpathlineto{\pgfqpoint{1.080566in}{0.823759in}}%
\pgfpathlineto{\pgfqpoint{1.252001in}{0.640000in}}%
\pgfpathlineto{\pgfqpoint{1.358232in}{0.528000in}}%
\pgfpathlineto{\pgfqpoint{1.362090in}{0.528000in}}%
\pgfpathlineto{\pgfqpoint{1.361131in}{0.529003in}}%
\pgfpathlineto{\pgfqpoint{1.185770in}{0.714667in}}%
\pgfpathlineto{\pgfqpoint{1.040485in}{0.871494in}}%
\pgfpathlineto{\pgfqpoint{0.865360in}{1.064454in}}%
\pgfpathlineto{\pgfqpoint{0.800000in}{1.137658in}}%
\pgfpathmoveto{\pgfqpoint{4.768000in}{2.374252in}}%
\pgfpathlineto{\pgfqpoint{4.469964in}{2.693333in}}%
\pgfpathlineto{\pgfqpoint{4.154794in}{3.018162in}}%
\pgfpathlineto{\pgfqpoint{4.069672in}{3.104000in}}%
\pgfpathlineto{\pgfqpoint{3.765980in}{3.402790in}}%
\pgfpathlineto{\pgfqpoint{3.445333in}{3.706670in}}%
\pgfpathlineto{\pgfqpoint{3.285010in}{3.854327in}}%
\pgfpathlineto{\pgfqpoint{2.954981in}{4.149333in}}%
\pgfpathlineto{\pgfqpoint{2.869262in}{4.224000in}}%
\pgfpathlineto{\pgfqpoint{2.864818in}{4.224000in}}%
\pgfpathlineto{\pgfqpoint{3.164768in}{3.959355in}}%
\pgfpathlineto{\pgfqpoint{3.486977in}{3.664000in}}%
\pgfpathlineto{\pgfqpoint{3.806061in}{3.360034in}}%
\pgfpathlineto{\pgfqpoint{4.139970in}{3.029333in}}%
\pgfpathlineto{\pgfqpoint{4.466096in}{2.693333in}}%
\pgfpathlineto{\pgfqpoint{4.607677in}{2.543334in}}%
\pgfpathlineto{\pgfqpoint{4.768000in}{2.370092in}}%
\pgfpathlineto{\pgfqpoint{4.768000in}{2.370092in}}%
\pgfusepath{fill}%
\end{pgfscope}%
\begin{pgfscope}%
\pgfpathrectangle{\pgfqpoint{0.800000in}{0.528000in}}{\pgfqpoint{3.968000in}{3.696000in}}%
\pgfusepath{clip}%
\pgfsetbuttcap%
\pgfsetroundjoin%
\definecolor{currentfill}{rgb}{0.182256,0.426184,0.557120}%
\pgfsetfillcolor{currentfill}%
\pgfsetlinewidth{0.000000pt}%
\definecolor{currentstroke}{rgb}{0.000000,0.000000,0.000000}%
\pgfsetstrokecolor{currentstroke}%
\pgfsetdash{}{0pt}%
\pgfpathmoveto{\pgfqpoint{0.800000in}{1.129098in}}%
\pgfpathlineto{\pgfqpoint{1.080566in}{0.819643in}}%
\pgfpathlineto{\pgfqpoint{1.248137in}{0.640000in}}%
\pgfpathlineto{\pgfqpoint{1.354390in}{0.528000in}}%
\pgfpathlineto{\pgfqpoint{1.358232in}{0.528000in}}%
\pgfpathlineto{\pgfqpoint{1.200808in}{0.694479in}}%
\pgfpathlineto{\pgfqpoint{1.040485in}{0.867309in}}%
\pgfpathlineto{\pgfqpoint{0.873960in}{1.050667in}}%
\pgfpathlineto{\pgfqpoint{0.800000in}{1.133378in}}%
\pgfpathmoveto{\pgfqpoint{4.768000in}{2.378411in}}%
\pgfpathlineto{\pgfqpoint{4.473831in}{2.693333in}}%
\pgfpathlineto{\pgfqpoint{4.147829in}{3.029333in}}%
\pgfpathlineto{\pgfqpoint{3.998781in}{3.178667in}}%
\pgfpathlineto{\pgfqpoint{3.685818in}{3.483709in}}%
\pgfpathlineto{\pgfqpoint{3.365172in}{3.784691in}}%
\pgfpathlineto{\pgfqpoint{3.204848in}{3.930924in}}%
\pgfpathlineto{\pgfqpoint{2.873706in}{4.224000in}}%
\pgfpathlineto{\pgfqpoint{2.869262in}{4.224000in}}%
\pgfpathlineto{\pgfqpoint{3.165386in}{3.962667in}}%
\pgfpathlineto{\pgfqpoint{3.491091in}{3.664000in}}%
\pgfpathlineto{\pgfqpoint{3.806061in}{3.363972in}}%
\pgfpathlineto{\pgfqpoint{4.143900in}{3.029333in}}%
\pgfpathlineto{\pgfqpoint{4.459257in}{2.704421in}}%
\pgfpathlineto{\pgfqpoint{4.540707in}{2.618667in}}%
\pgfpathlineto{\pgfqpoint{4.687838in}{2.461273in}}%
\pgfpathlineto{\pgfqpoint{4.768000in}{2.374252in}}%
\pgfpathlineto{\pgfqpoint{4.768000in}{2.374252in}}%
\pgfusepath{fill}%
\end{pgfscope}%
\begin{pgfscope}%
\pgfpathrectangle{\pgfqpoint{0.800000in}{0.528000in}}{\pgfqpoint{3.968000in}{3.696000in}}%
\pgfusepath{clip}%
\pgfsetbuttcap%
\pgfsetroundjoin%
\definecolor{currentfill}{rgb}{0.182256,0.426184,0.557120}%
\pgfsetfillcolor{currentfill}%
\pgfsetlinewidth{0.000000pt}%
\definecolor{currentstroke}{rgb}{0.000000,0.000000,0.000000}%
\pgfsetstrokecolor{currentstroke}%
\pgfsetdash{}{0pt}%
\pgfpathmoveto{\pgfqpoint{0.800000in}{1.124826in}}%
\pgfpathlineto{\pgfqpoint{1.104810in}{0.789333in}}%
\pgfpathlineto{\pgfqpoint{1.244273in}{0.640000in}}%
\pgfpathlineto{\pgfqpoint{1.350547in}{0.528000in}}%
\pgfpathlineto{\pgfqpoint{1.354390in}{0.528000in}}%
\pgfpathlineto{\pgfqpoint{1.200808in}{0.690376in}}%
\pgfpathlineto{\pgfqpoint{1.035045in}{0.869067in}}%
\pgfpathlineto{\pgfqpoint{0.870183in}{1.050667in}}%
\pgfpathlineto{\pgfqpoint{0.800000in}{1.129098in}}%
\pgfpathlineto{\pgfqpoint{0.800000in}{1.125333in}}%
\pgfpathmoveto{\pgfqpoint{4.768000in}{2.382571in}}%
\pgfpathlineto{\pgfqpoint{4.477699in}{2.693333in}}%
\pgfpathlineto{\pgfqpoint{4.151758in}{3.029333in}}%
\pgfpathlineto{\pgfqpoint{4.002768in}{3.178667in}}%
\pgfpathlineto{\pgfqpoint{3.671357in}{3.501197in}}%
\pgfpathlineto{\pgfqpoint{3.578749in}{3.589333in}}%
\pgfpathlineto{\pgfqpoint{3.419136in}{3.738667in}}%
\pgfpathlineto{\pgfqpoint{3.124687in}{4.006807in}}%
\pgfpathlineto{\pgfqpoint{2.963797in}{4.149333in}}%
\pgfpathlineto{\pgfqpoint{2.878149in}{4.224000in}}%
\pgfpathlineto{\pgfqpoint{2.873706in}{4.224000in}}%
\pgfpathlineto{\pgfqpoint{3.169629in}{3.962667in}}%
\pgfpathlineto{\pgfqpoint{3.495206in}{3.664000in}}%
\pgfpathlineto{\pgfqpoint{3.808673in}{3.365333in}}%
\pgfpathlineto{\pgfqpoint{4.126707in}{3.050668in}}%
\pgfpathlineto{\pgfqpoint{4.294333in}{2.880000in}}%
\pgfpathlineto{\pgfqpoint{4.447354in}{2.721144in}}%
\pgfpathlineto{\pgfqpoint{4.759899in}{2.387121in}}%
\pgfpathlineto{\pgfqpoint{4.768000in}{2.378411in}}%
\pgfpathlineto{\pgfqpoint{4.768000in}{2.378411in}}%
\pgfusepath{fill}%
\end{pgfscope}%
\begin{pgfscope}%
\pgfpathrectangle{\pgfqpoint{0.800000in}{0.528000in}}{\pgfqpoint{3.968000in}{3.696000in}}%
\pgfusepath{clip}%
\pgfsetbuttcap%
\pgfsetroundjoin%
\definecolor{currentfill}{rgb}{0.182256,0.426184,0.557120}%
\pgfsetfillcolor{currentfill}%
\pgfsetlinewidth{0.000000pt}%
\definecolor{currentstroke}{rgb}{0.000000,0.000000,0.000000}%
\pgfsetstrokecolor{currentstroke}%
\pgfsetdash{}{0pt}%
\pgfpathmoveto{\pgfqpoint{0.800000in}{1.120615in}}%
\pgfpathlineto{\pgfqpoint{1.101000in}{0.789333in}}%
\pgfpathlineto{\pgfqpoint{1.240889in}{0.639499in}}%
\pgfpathlineto{\pgfqpoint{1.346705in}{0.528000in}}%
\pgfpathlineto{\pgfqpoint{1.350547in}{0.528000in}}%
\pgfpathlineto{\pgfqpoint{1.200808in}{0.686272in}}%
\pgfpathlineto{\pgfqpoint{1.035909in}{0.864000in}}%
\pgfpathlineto{\pgfqpoint{0.880162in}{1.035361in}}%
\pgfpathlineto{\pgfqpoint{0.800000in}{1.124826in}}%
\pgfpathmoveto{\pgfqpoint{4.768000in}{2.386731in}}%
\pgfpathlineto{\pgfqpoint{4.481566in}{2.693333in}}%
\pgfpathlineto{\pgfqpoint{4.155688in}{3.029333in}}%
\pgfpathlineto{\pgfqpoint{4.006465in}{3.178950in}}%
\pgfpathlineto{\pgfqpoint{3.673423in}{3.503121in}}%
\pgfpathlineto{\pgfqpoint{3.582832in}{3.589333in}}%
\pgfpathlineto{\pgfqpoint{3.423282in}{3.738667in}}%
\pgfpathlineto{\pgfqpoint{3.124687in}{4.010624in}}%
\pgfpathlineto{\pgfqpoint{2.964364in}{4.152647in}}%
\pgfpathlineto{\pgfqpoint{2.882593in}{4.224000in}}%
\pgfpathlineto{\pgfqpoint{2.878149in}{4.224000in}}%
\pgfpathlineto{\pgfqpoint{3.164768in}{3.970865in}}%
\pgfpathlineto{\pgfqpoint{3.338179in}{3.813333in}}%
\pgfpathlineto{\pgfqpoint{3.499320in}{3.664000in}}%
\pgfpathlineto{\pgfqpoint{3.812667in}{3.365333in}}%
\pgfpathlineto{\pgfqpoint{4.126707in}{3.054637in}}%
\pgfpathlineto{\pgfqpoint{4.298207in}{2.880000in}}%
\pgfpathlineto{\pgfqpoint{4.447354in}{2.725206in}}%
\pgfpathlineto{\pgfqpoint{4.768000in}{2.382571in}}%
\pgfpathlineto{\pgfqpoint{4.768000in}{2.382571in}}%
\pgfusepath{fill}%
\end{pgfscope}%
\begin{pgfscope}%
\pgfpathrectangle{\pgfqpoint{0.800000in}{0.528000in}}{\pgfqpoint{3.968000in}{3.696000in}}%
\pgfusepath{clip}%
\pgfsetbuttcap%
\pgfsetroundjoin%
\definecolor{currentfill}{rgb}{0.182256,0.426184,0.557120}%
\pgfsetfillcolor{currentfill}%
\pgfsetlinewidth{0.000000pt}%
\definecolor{currentstroke}{rgb}{0.000000,0.000000,0.000000}%
\pgfsetstrokecolor{currentstroke}%
\pgfsetdash{}{0pt}%
\pgfpathmoveto{\pgfqpoint{0.800000in}{1.116403in}}%
\pgfpathlineto{\pgfqpoint{1.097190in}{0.789333in}}%
\pgfpathlineto{\pgfqpoint{1.240889in}{0.635462in}}%
\pgfpathlineto{\pgfqpoint{1.342863in}{0.528000in}}%
\pgfpathlineto{\pgfqpoint{1.346705in}{0.528000in}}%
\pgfpathlineto{\pgfqpoint{1.200808in}{0.682169in}}%
\pgfpathlineto{\pgfqpoint{1.032126in}{0.864000in}}%
\pgfpathlineto{\pgfqpoint{0.880162in}{1.031158in}}%
\pgfpathlineto{\pgfqpoint{0.800000in}{1.120615in}}%
\pgfpathmoveto{\pgfqpoint{4.768000in}{2.390890in}}%
\pgfpathlineto{\pgfqpoint{4.485433in}{2.693333in}}%
\pgfpathlineto{\pgfqpoint{4.159617in}{3.029333in}}%
\pgfpathlineto{\pgfqpoint{4.006465in}{3.182849in}}%
\pgfpathlineto{\pgfqpoint{3.665544in}{3.514667in}}%
\pgfpathlineto{\pgfqpoint{3.507549in}{3.664000in}}%
\pgfpathlineto{\pgfqpoint{3.193712in}{3.952294in}}%
\pgfpathlineto{\pgfqpoint{3.099038in}{4.037333in}}%
\pgfpathlineto{\pgfqpoint{2.886985in}{4.224000in}}%
\pgfpathlineto{\pgfqpoint{2.882593in}{4.224000in}}%
\pgfpathlineto{\pgfqpoint{2.883421in}{4.223273in}}%
\pgfpathlineto{\pgfqpoint{2.925490in}{4.186667in}}%
\pgfpathlineto{\pgfqpoint{3.094761in}{4.037333in}}%
\pgfpathlineto{\pgfqpoint{3.260640in}{3.888000in}}%
\pgfpathlineto{\pgfqpoint{3.582832in}{3.589333in}}%
\pgfpathlineto{\pgfqpoint{3.893201in}{3.290667in}}%
\pgfpathlineto{\pgfqpoint{4.046545in}{3.139027in}}%
\pgfpathlineto{\pgfqpoint{4.374357in}{2.805333in}}%
\pgfpathlineto{\pgfqpoint{4.691845in}{2.469333in}}%
\pgfpathlineto{\pgfqpoint{4.768000in}{2.386731in}}%
\pgfpathlineto{\pgfqpoint{4.768000in}{2.386731in}}%
\pgfusepath{fill}%
\end{pgfscope}%
\begin{pgfscope}%
\pgfpathrectangle{\pgfqpoint{0.800000in}{0.528000in}}{\pgfqpoint{3.968000in}{3.696000in}}%
\pgfusepath{clip}%
\pgfsetbuttcap%
\pgfsetroundjoin%
\definecolor{currentfill}{rgb}{0.180629,0.429975,0.557282}%
\pgfsetfillcolor{currentfill}%
\pgfsetlinewidth{0.000000pt}%
\definecolor{currentstroke}{rgb}{0.000000,0.000000,0.000000}%
\pgfsetstrokecolor{currentstroke}%
\pgfsetdash{}{0pt}%
\pgfpathmoveto{\pgfqpoint{0.800000in}{1.112191in}}%
\pgfpathlineto{\pgfqpoint{1.093381in}{0.789333in}}%
\pgfpathlineto{\pgfqpoint{1.240889in}{0.631426in}}%
\pgfpathlineto{\pgfqpoint{1.339020in}{0.528000in}}%
\pgfpathlineto{\pgfqpoint{1.342863in}{0.528000in}}%
\pgfpathlineto{\pgfqpoint{1.200808in}{0.678066in}}%
\pgfpathlineto{\pgfqpoint{1.028343in}{0.864000in}}%
\pgfpathlineto{\pgfqpoint{0.880162in}{1.026955in}}%
\pgfpathlineto{\pgfqpoint{0.800000in}{1.116403in}}%
\pgfpathmoveto{\pgfqpoint{4.768000in}{2.395044in}}%
\pgfpathlineto{\pgfqpoint{4.447354in}{2.737290in}}%
\pgfpathlineto{\pgfqpoint{4.273489in}{2.917333in}}%
\pgfpathlineto{\pgfqpoint{4.126584in}{3.066667in}}%
\pgfpathlineto{\pgfqpoint{3.786148in}{3.402667in}}%
\pgfpathlineto{\pgfqpoint{3.471714in}{3.701333in}}%
\pgfpathlineto{\pgfqpoint{3.145062in}{4.000000in}}%
\pgfpathlineto{\pgfqpoint{2.891347in}{4.224000in}}%
\pgfpathlineto{\pgfqpoint{2.886985in}{4.224000in}}%
\pgfpathlineto{\pgfqpoint{3.057061in}{4.074667in}}%
\pgfpathlineto{\pgfqpoint{3.375998in}{3.786084in}}%
\pgfpathlineto{\pgfqpoint{3.467584in}{3.701333in}}%
\pgfpathlineto{\pgfqpoint{3.626321in}{3.552000in}}%
\pgfpathlineto{\pgfqpoint{3.782140in}{3.402667in}}%
\pgfpathlineto{\pgfqpoint{4.086626in}{3.102876in}}%
\pgfpathlineto{\pgfqpoint{4.414104in}{2.768000in}}%
\pgfpathlineto{\pgfqpoint{4.567596in}{2.606434in}}%
\pgfpathlineto{\pgfqpoint{4.768000in}{2.390890in}}%
\pgfpathlineto{\pgfqpoint{4.768000in}{2.394667in}}%
\pgfpathlineto{\pgfqpoint{4.768000in}{2.394667in}}%
\pgfusepath{fill}%
\end{pgfscope}%
\begin{pgfscope}%
\pgfpathrectangle{\pgfqpoint{0.800000in}{0.528000in}}{\pgfqpoint{3.968000in}{3.696000in}}%
\pgfusepath{clip}%
\pgfsetbuttcap%
\pgfsetroundjoin%
\definecolor{currentfill}{rgb}{0.180629,0.429975,0.557282}%
\pgfsetfillcolor{currentfill}%
\pgfsetlinewidth{0.000000pt}%
\definecolor{currentstroke}{rgb}{0.000000,0.000000,0.000000}%
\pgfsetstrokecolor{currentstroke}%
\pgfsetdash{}{0pt}%
\pgfpathmoveto{\pgfqpoint{0.800000in}{1.107979in}}%
\pgfpathlineto{\pgfqpoint{1.089571in}{0.789333in}}%
\pgfpathlineto{\pgfqpoint{1.240889in}{0.627389in}}%
\pgfpathlineto{\pgfqpoint{1.335178in}{0.528000in}}%
\pgfpathlineto{\pgfqpoint{1.339020in}{0.528000in}}%
\pgfpathlineto{\pgfqpoint{1.176399in}{0.700069in}}%
\pgfpathlineto{\pgfqpoint{1.024560in}{0.864000in}}%
\pgfpathlineto{\pgfqpoint{0.880162in}{1.022752in}}%
\pgfpathlineto{\pgfqpoint{0.800000in}{1.112191in}}%
\pgfpathmoveto{\pgfqpoint{4.768000in}{2.399139in}}%
\pgfpathlineto{\pgfqpoint{4.447354in}{2.741291in}}%
\pgfpathlineto{\pgfqpoint{4.277376in}{2.917333in}}%
\pgfpathlineto{\pgfqpoint{4.126707in}{3.070455in}}%
\pgfpathlineto{\pgfqpoint{3.790156in}{3.402667in}}%
\pgfpathlineto{\pgfqpoint{3.475844in}{3.701333in}}%
\pgfpathlineto{\pgfqpoint{3.149322in}{4.000000in}}%
\pgfpathlineto{\pgfqpoint{2.895710in}{4.224000in}}%
\pgfpathlineto{\pgfqpoint{2.891347in}{4.224000in}}%
\pgfpathlineto{\pgfqpoint{3.061355in}{4.074667in}}%
\pgfpathlineto{\pgfqpoint{3.365172in}{3.800046in}}%
\pgfpathlineto{\pgfqpoint{3.685818in}{3.499184in}}%
\pgfpathlineto{\pgfqpoint{3.862973in}{3.328000in}}%
\pgfpathlineto{\pgfqpoint{4.014592in}{3.178667in}}%
\pgfpathlineto{\pgfqpoint{4.346013in}{2.842667in}}%
\pgfpathlineto{\pgfqpoint{4.664749in}{2.506667in}}%
\pgfpathlineto{\pgfqpoint{4.768000in}{2.395044in}}%
\pgfpathlineto{\pgfqpoint{4.768000in}{2.395044in}}%
\pgfusepath{fill}%
\end{pgfscope}%
\begin{pgfscope}%
\pgfpathrectangle{\pgfqpoint{0.800000in}{0.528000in}}{\pgfqpoint{3.968000in}{3.696000in}}%
\pgfusepath{clip}%
\pgfsetbuttcap%
\pgfsetroundjoin%
\definecolor{currentfill}{rgb}{0.180629,0.429975,0.557282}%
\pgfsetfillcolor{currentfill}%
\pgfsetlinewidth{0.000000pt}%
\definecolor{currentstroke}{rgb}{0.000000,0.000000,0.000000}%
\pgfsetstrokecolor{currentstroke}%
\pgfsetdash{}{0pt}%
\pgfpathmoveto{\pgfqpoint{0.800000in}{1.103767in}}%
\pgfpathlineto{\pgfqpoint{1.085762in}{0.789333in}}%
\pgfpathlineto{\pgfqpoint{1.240889in}{0.623353in}}%
\pgfpathlineto{\pgfqpoint{1.331336in}{0.528000in}}%
\pgfpathlineto{\pgfqpoint{1.335178in}{0.528000in}}%
\pgfpathlineto{\pgfqpoint{1.193907in}{0.677333in}}%
\pgfpathlineto{\pgfqpoint{1.040485in}{0.842536in}}%
\pgfpathlineto{\pgfqpoint{0.800000in}{1.107979in}}%
\pgfpathmoveto{\pgfqpoint{4.768000in}{2.403235in}}%
\pgfpathlineto{\pgfqpoint{4.454695in}{2.737504in}}%
\pgfpathlineto{\pgfqpoint{4.367192in}{2.828742in}}%
\pgfpathlineto{\pgfqpoint{4.046545in}{3.154671in}}%
\pgfpathlineto{\pgfqpoint{3.716701in}{3.477333in}}%
\pgfpathlineto{\pgfqpoint{3.559622in}{3.626667in}}%
\pgfpathlineto{\pgfqpoint{3.236386in}{3.925333in}}%
\pgfpathlineto{\pgfqpoint{3.069941in}{4.074667in}}%
\pgfpathlineto{\pgfqpoint{2.900072in}{4.224000in}}%
\pgfpathlineto{\pgfqpoint{2.895710in}{4.224000in}}%
\pgfpathlineto{\pgfqpoint{3.065648in}{4.074667in}}%
\pgfpathlineto{\pgfqpoint{3.365172in}{3.803885in}}%
\pgfpathlineto{\pgfqpoint{3.685818in}{3.503052in}}%
\pgfpathlineto{\pgfqpoint{3.866952in}{3.328000in}}%
\pgfpathlineto{\pgfqpoint{4.018514in}{3.178667in}}%
\pgfpathlineto{\pgfqpoint{4.327111in}{2.866189in}}%
\pgfpathlineto{\pgfqpoint{4.493076in}{2.693333in}}%
\pgfpathlineto{\pgfqpoint{4.647758in}{2.529007in}}%
\pgfpathlineto{\pgfqpoint{4.768000in}{2.399139in}}%
\pgfpathlineto{\pgfqpoint{4.768000in}{2.399139in}}%
\pgfusepath{fill}%
\end{pgfscope}%
\begin{pgfscope}%
\pgfpathrectangle{\pgfqpoint{0.800000in}{0.528000in}}{\pgfqpoint{3.968000in}{3.696000in}}%
\pgfusepath{clip}%
\pgfsetbuttcap%
\pgfsetroundjoin%
\definecolor{currentfill}{rgb}{0.179019,0.433756,0.557430}%
\pgfsetfillcolor{currentfill}%
\pgfsetlinewidth{0.000000pt}%
\definecolor{currentstroke}{rgb}{0.000000,0.000000,0.000000}%
\pgfsetstrokecolor{currentstroke}%
\pgfsetdash{}{0pt}%
\pgfpathmoveto{\pgfqpoint{0.800000in}{1.099555in}}%
\pgfpathlineto{\pgfqpoint{1.090686in}{0.779907in}}%
\pgfpathlineto{\pgfqpoint{1.256626in}{0.602667in}}%
\pgfpathlineto{\pgfqpoint{1.327493in}{0.528000in}}%
\pgfpathlineto{\pgfqpoint{1.331336in}{0.528000in}}%
\pgfpathlineto{\pgfqpoint{1.190118in}{0.677333in}}%
\pgfpathlineto{\pgfqpoint{1.040485in}{0.838416in}}%
\pgfpathlineto{\pgfqpoint{0.800000in}{1.103767in}}%
\pgfpathmoveto{\pgfqpoint{4.768000in}{2.407330in}}%
\pgfpathlineto{\pgfqpoint{4.465133in}{2.730667in}}%
\pgfpathlineto{\pgfqpoint{4.166788in}{3.037847in}}%
\pgfpathlineto{\pgfqpoint{3.988743in}{3.216000in}}%
\pgfpathlineto{\pgfqpoint{3.836627in}{3.365333in}}%
\pgfpathlineto{\pgfqpoint{3.681756in}{3.514667in}}%
\pgfpathlineto{\pgfqpoint{3.524007in}{3.664000in}}%
\pgfpathlineto{\pgfqpoint{3.199330in}{3.962667in}}%
\pgfpathlineto{\pgfqpoint{3.032112in}{4.112000in}}%
\pgfpathlineto{\pgfqpoint{2.904434in}{4.224000in}}%
\pgfpathlineto{\pgfqpoint{2.900072in}{4.224000in}}%
\pgfpathlineto{\pgfqpoint{3.069941in}{4.074667in}}%
\pgfpathlineto{\pgfqpoint{3.365172in}{3.807724in}}%
\pgfpathlineto{\pgfqpoint{3.685818in}{3.506921in}}%
\pgfpathlineto{\pgfqpoint{3.858742in}{3.339737in}}%
\pgfpathlineto{\pgfqpoint{3.947016in}{3.253333in}}%
\pgfpathlineto{\pgfqpoint{4.246949in}{2.952472in}}%
\pgfpathlineto{\pgfqpoint{4.425601in}{2.768000in}}%
\pgfpathlineto{\pgfqpoint{4.567596in}{2.618657in}}%
\pgfpathlineto{\pgfqpoint{4.768000in}{2.403235in}}%
\pgfpathlineto{\pgfqpoint{4.768000in}{2.403235in}}%
\pgfusepath{fill}%
\end{pgfscope}%
\begin{pgfscope}%
\pgfpathrectangle{\pgfqpoint{0.800000in}{0.528000in}}{\pgfqpoint{3.968000in}{3.696000in}}%
\pgfusepath{clip}%
\pgfsetbuttcap%
\pgfsetroundjoin%
\definecolor{currentfill}{rgb}{0.179019,0.433756,0.557430}%
\pgfsetfillcolor{currentfill}%
\pgfsetlinewidth{0.000000pt}%
\definecolor{currentstroke}{rgb}{0.000000,0.000000,0.000000}%
\pgfsetstrokecolor{currentstroke}%
\pgfsetdash{}{0pt}%
\pgfpathmoveto{\pgfqpoint{0.800000in}{1.095344in}}%
\pgfpathlineto{\pgfqpoint{1.080566in}{0.786755in}}%
\pgfpathlineto{\pgfqpoint{1.252811in}{0.602667in}}%
\pgfpathlineto{\pgfqpoint{1.323651in}{0.528000in}}%
\pgfpathlineto{\pgfqpoint{1.327493in}{0.528000in}}%
\pgfpathlineto{\pgfqpoint{1.186330in}{0.677333in}}%
\pgfpathlineto{\pgfqpoint{1.040485in}{0.834296in}}%
\pgfpathlineto{\pgfqpoint{0.800000in}{1.099555in}}%
\pgfpathmoveto{\pgfqpoint{4.768000in}{2.411425in}}%
\pgfpathlineto{\pgfqpoint{4.468951in}{2.730667in}}%
\pgfpathlineto{\pgfqpoint{4.166788in}{3.041761in}}%
\pgfpathlineto{\pgfqpoint{3.992679in}{3.216000in}}%
\pgfpathlineto{\pgfqpoint{3.840620in}{3.365333in}}%
\pgfpathlineto{\pgfqpoint{3.685428in}{3.515030in}}%
\pgfpathlineto{\pgfqpoint{3.525495in}{3.666424in}}%
\pgfpathlineto{\pgfqpoint{3.203573in}{3.962667in}}%
\pgfpathlineto{\pgfqpoint{3.036422in}{4.112000in}}%
\pgfpathlineto{\pgfqpoint{2.908797in}{4.224000in}}%
\pgfpathlineto{\pgfqpoint{2.904434in}{4.224000in}}%
\pgfpathlineto{\pgfqpoint{3.074234in}{4.074667in}}%
\pgfpathlineto{\pgfqpoint{3.365172in}{3.811563in}}%
\pgfpathlineto{\pgfqpoint{3.685818in}{3.510789in}}%
\pgfpathlineto{\pgfqpoint{3.846141in}{3.356080in}}%
\pgfpathlineto{\pgfqpoint{4.175196in}{3.029333in}}%
\pgfpathlineto{\pgfqpoint{4.327111in}{2.874167in}}%
\pgfpathlineto{\pgfqpoint{4.500687in}{2.693333in}}%
\pgfpathlineto{\pgfqpoint{4.647758in}{2.537173in}}%
\pgfpathlineto{\pgfqpoint{4.768000in}{2.407330in}}%
\pgfpathlineto{\pgfqpoint{4.768000in}{2.407330in}}%
\pgfusepath{fill}%
\end{pgfscope}%
\begin{pgfscope}%
\pgfpathrectangle{\pgfqpoint{0.800000in}{0.528000in}}{\pgfqpoint{3.968000in}{3.696000in}}%
\pgfusepath{clip}%
\pgfsetbuttcap%
\pgfsetroundjoin%
\definecolor{currentfill}{rgb}{0.179019,0.433756,0.557430}%
\pgfsetfillcolor{currentfill}%
\pgfsetlinewidth{0.000000pt}%
\definecolor{currentstroke}{rgb}{0.000000,0.000000,0.000000}%
\pgfsetstrokecolor{currentstroke}%
\pgfsetdash{}{0pt}%
\pgfpathmoveto{\pgfqpoint{0.800000in}{1.091132in}}%
\pgfpathlineto{\pgfqpoint{1.080566in}{0.782702in}}%
\pgfpathlineto{\pgfqpoint{1.248995in}{0.602667in}}%
\pgfpathlineto{\pgfqpoint{1.321051in}{0.528000in}}%
\pgfpathlineto{\pgfqpoint{1.323651in}{0.528000in}}%
\pgfpathlineto{\pgfqpoint{1.182541in}{0.677333in}}%
\pgfpathlineto{\pgfqpoint{1.040485in}{0.830175in}}%
\pgfpathlineto{\pgfqpoint{0.800000in}{1.095344in}}%
\pgfpathmoveto{\pgfqpoint{4.768000in}{2.415521in}}%
\pgfpathlineto{\pgfqpoint{4.472770in}{2.730667in}}%
\pgfpathlineto{\pgfqpoint{4.145982in}{3.066667in}}%
\pgfpathlineto{\pgfqpoint{3.996614in}{3.216000in}}%
\pgfpathlineto{\pgfqpoint{3.844613in}{3.365333in}}%
\pgfpathlineto{\pgfqpoint{3.685818in}{3.518471in}}%
\pgfpathlineto{\pgfqpoint{3.525495in}{3.670223in}}%
\pgfpathlineto{\pgfqpoint{3.204848in}{3.965303in}}%
\pgfpathlineto{\pgfqpoint{3.040732in}{4.112000in}}%
\pgfpathlineto{\pgfqpoint{2.913159in}{4.224000in}}%
\pgfpathlineto{\pgfqpoint{2.908797in}{4.224000in}}%
\pgfpathlineto{\pgfqpoint{3.061001in}{4.090013in}}%
\pgfpathlineto{\pgfqpoint{3.142906in}{4.016971in}}%
\pgfpathlineto{\pgfqpoint{3.204848in}{3.961517in}}%
\pgfpathlineto{\pgfqpoint{3.367384in}{3.813333in}}%
\pgfpathlineto{\pgfqpoint{3.685818in}{3.514658in}}%
\pgfpathlineto{\pgfqpoint{3.846141in}{3.359964in}}%
\pgfpathlineto{\pgfqpoint{4.179061in}{3.029333in}}%
\pgfpathlineto{\pgfqpoint{4.327111in}{2.878155in}}%
\pgfpathlineto{\pgfqpoint{4.504492in}{2.693333in}}%
\pgfpathlineto{\pgfqpoint{4.647758in}{2.541255in}}%
\pgfpathlineto{\pgfqpoint{4.768000in}{2.411425in}}%
\pgfpathlineto{\pgfqpoint{4.768000in}{2.411425in}}%
\pgfusepath{fill}%
\end{pgfscope}%
\begin{pgfscope}%
\pgfpathrectangle{\pgfqpoint{0.800000in}{0.528000in}}{\pgfqpoint{3.968000in}{3.696000in}}%
\pgfusepath{clip}%
\pgfsetbuttcap%
\pgfsetroundjoin%
\definecolor{currentfill}{rgb}{0.179019,0.433756,0.557430}%
\pgfsetfillcolor{currentfill}%
\pgfsetlinewidth{0.000000pt}%
\definecolor{currentstroke}{rgb}{0.000000,0.000000,0.000000}%
\pgfsetstrokecolor{currentstroke}%
\pgfsetdash{}{0pt}%
\pgfpathmoveto{\pgfqpoint{0.800000in}{1.086937in}}%
\pgfpathlineto{\pgfqpoint{0.967733in}{0.901333in}}%
\pgfpathlineto{\pgfqpoint{1.280970in}{0.564847in}}%
\pgfpathlineto{\pgfqpoint{1.316047in}{0.528000in}}%
\pgfpathlineto{\pgfqpoint{1.319828in}{0.528000in}}%
\pgfpathlineto{\pgfqpoint{1.000404in}{0.869714in}}%
\pgfpathlineto{\pgfqpoint{0.800000in}{1.091132in}}%
\pgfpathlineto{\pgfqpoint{0.800000in}{1.088000in}}%
\pgfpathmoveto{\pgfqpoint{4.768000in}{2.419616in}}%
\pgfpathlineto{\pgfqpoint{4.476589in}{2.730667in}}%
\pgfpathlineto{\pgfqpoint{4.149862in}{3.066667in}}%
\pgfpathlineto{\pgfqpoint{4.000550in}{3.216000in}}%
\pgfpathlineto{\pgfqpoint{3.846141in}{3.367696in}}%
\pgfpathlineto{\pgfqpoint{3.685818in}{3.522284in}}%
\pgfpathlineto{\pgfqpoint{3.525495in}{3.674021in}}%
\pgfpathlineto{\pgfqpoint{3.204848in}{3.969073in}}%
\pgfpathlineto{\pgfqpoint{3.035100in}{4.120779in}}%
\pgfpathlineto{\pgfqpoint{2.917521in}{4.224000in}}%
\pgfpathlineto{\pgfqpoint{2.913159in}{4.224000in}}%
\pgfpathlineto{\pgfqpoint{3.044525in}{4.108648in}}%
\pgfpathlineto{\pgfqpoint{3.371489in}{3.813333in}}%
\pgfpathlineto{\pgfqpoint{3.689794in}{3.514667in}}%
\pgfpathlineto{\pgfqpoint{3.846141in}{3.363847in}}%
\pgfpathlineto{\pgfqpoint{4.182926in}{3.029333in}}%
\pgfpathlineto{\pgfqpoint{4.348129in}{2.860423in}}%
\pgfpathlineto{\pgfqpoint{4.508298in}{2.693333in}}%
\pgfpathlineto{\pgfqpoint{4.648988in}{2.544000in}}%
\pgfpathlineto{\pgfqpoint{4.768000in}{2.415521in}}%
\pgfpathlineto{\pgfqpoint{4.768000in}{2.415521in}}%
\pgfusepath{fill}%
\end{pgfscope}%
\begin{pgfscope}%
\pgfpathrectangle{\pgfqpoint{0.800000in}{0.528000in}}{\pgfqpoint{3.968000in}{3.696000in}}%
\pgfusepath{clip}%
\pgfsetbuttcap%
\pgfsetroundjoin%
\definecolor{currentfill}{rgb}{0.177423,0.437527,0.557565}%
\pgfsetfillcolor{currentfill}%
\pgfsetlinewidth{0.000000pt}%
\definecolor{currentstroke}{rgb}{0.000000,0.000000,0.000000}%
\pgfsetstrokecolor{currentstroke}%
\pgfsetdash{}{0pt}%
\pgfpathmoveto{\pgfqpoint{0.800000in}{1.082791in}}%
\pgfpathlineto{\pgfqpoint{0.963963in}{0.901333in}}%
\pgfpathlineto{\pgfqpoint{1.280970in}{0.560875in}}%
\pgfpathlineto{\pgfqpoint{1.312266in}{0.528000in}}%
\pgfpathlineto{\pgfqpoint{1.316047in}{0.528000in}}%
\pgfpathlineto{\pgfqpoint{1.000404in}{0.865590in}}%
\pgfpathlineto{\pgfqpoint{0.800000in}{1.086937in}}%
\pgfpathmoveto{\pgfqpoint{4.768000in}{2.423711in}}%
\pgfpathlineto{\pgfqpoint{4.480408in}{2.730667in}}%
\pgfpathlineto{\pgfqpoint{4.153741in}{3.066667in}}%
\pgfpathlineto{\pgfqpoint{4.004485in}{3.216000in}}%
\pgfpathlineto{\pgfqpoint{3.846141in}{3.371524in}}%
\pgfpathlineto{\pgfqpoint{3.685818in}{3.526097in}}%
\pgfpathlineto{\pgfqpoint{3.512816in}{3.689524in}}%
\pgfpathlineto{\pgfqpoint{3.420112in}{3.776000in}}%
\pgfpathlineto{\pgfqpoint{3.257298in}{3.925333in}}%
\pgfpathlineto{\pgfqpoint{2.964364in}{4.186865in}}%
\pgfpathlineto{\pgfqpoint{2.921883in}{4.224000in}}%
\pgfpathlineto{\pgfqpoint{2.917521in}{4.224000in}}%
\pgfpathlineto{\pgfqpoint{3.045033in}{4.112000in}}%
\pgfpathlineto{\pgfqpoint{3.375595in}{3.813333in}}%
\pgfpathlineto{\pgfqpoint{3.693779in}{3.514667in}}%
\pgfpathlineto{\pgfqpoint{3.848566in}{3.365333in}}%
\pgfpathlineto{\pgfqpoint{4.186792in}{3.029333in}}%
\pgfpathlineto{\pgfqpoint{4.332971in}{2.880000in}}%
\pgfpathlineto{\pgfqpoint{4.487434in}{2.719294in}}%
\pgfpathlineto{\pgfqpoint{4.768000in}{2.419616in}}%
\pgfpathlineto{\pgfqpoint{4.768000in}{2.419616in}}%
\pgfusepath{fill}%
\end{pgfscope}%
\begin{pgfscope}%
\pgfpathrectangle{\pgfqpoint{0.800000in}{0.528000in}}{\pgfqpoint{3.968000in}{3.696000in}}%
\pgfusepath{clip}%
\pgfsetbuttcap%
\pgfsetroundjoin%
\definecolor{currentfill}{rgb}{0.177423,0.437527,0.557565}%
\pgfsetfillcolor{currentfill}%
\pgfsetlinewidth{0.000000pt}%
\definecolor{currentstroke}{rgb}{0.000000,0.000000,0.000000}%
\pgfsetstrokecolor{currentstroke}%
\pgfsetdash{}{0pt}%
\pgfpathmoveto{\pgfqpoint{0.800000in}{1.078645in}}%
\pgfpathlineto{\pgfqpoint{0.960323in}{0.901193in}}%
\pgfpathlineto{\pgfqpoint{1.280970in}{0.556903in}}%
\pgfpathlineto{\pgfqpoint{1.308485in}{0.528000in}}%
\pgfpathlineto{\pgfqpoint{1.312266in}{0.528000in}}%
\pgfpathlineto{\pgfqpoint{0.998116in}{0.864000in}}%
\pgfpathlineto{\pgfqpoint{0.800000in}{1.082791in}}%
\pgfpathmoveto{\pgfqpoint{4.768000in}{2.427807in}}%
\pgfpathlineto{\pgfqpoint{4.484227in}{2.730667in}}%
\pgfpathlineto{\pgfqpoint{4.157620in}{3.066667in}}%
\pgfpathlineto{\pgfqpoint{4.006465in}{3.217910in}}%
\pgfpathlineto{\pgfqpoint{3.846141in}{3.375351in}}%
\pgfpathlineto{\pgfqpoint{3.662645in}{3.552000in}}%
\pgfpathlineto{\pgfqpoint{3.343217in}{3.850667in}}%
\pgfpathlineto{\pgfqpoint{3.178886in}{4.000000in}}%
\pgfpathlineto{\pgfqpoint{3.044525in}{4.119962in}}%
\pgfpathlineto{\pgfqpoint{2.926210in}{4.224000in}}%
\pgfpathlineto{\pgfqpoint{2.921883in}{4.224000in}}%
\pgfpathlineto{\pgfqpoint{3.049267in}{4.112000in}}%
\pgfpathlineto{\pgfqpoint{3.379700in}{3.813333in}}%
\pgfpathlineto{\pgfqpoint{3.697764in}{3.514667in}}%
\pgfpathlineto{\pgfqpoint{3.852493in}{3.365333in}}%
\pgfpathlineto{\pgfqpoint{4.179139in}{3.040838in}}%
\pgfpathlineto{\pgfqpoint{4.264024in}{2.954667in}}%
\pgfpathlineto{\pgfqpoint{4.408943in}{2.805333in}}%
\pgfpathlineto{\pgfqpoint{4.727919in}{2.467177in}}%
\pgfpathlineto{\pgfqpoint{4.768000in}{2.423711in}}%
\pgfpathlineto{\pgfqpoint{4.768000in}{2.423711in}}%
\pgfusepath{fill}%
\end{pgfscope}%
\begin{pgfscope}%
\pgfpathrectangle{\pgfqpoint{0.800000in}{0.528000in}}{\pgfqpoint{3.968000in}{3.696000in}}%
\pgfusepath{clip}%
\pgfsetbuttcap%
\pgfsetroundjoin%
\definecolor{currentfill}{rgb}{0.177423,0.437527,0.557565}%
\pgfsetfillcolor{currentfill}%
\pgfsetlinewidth{0.000000pt}%
\definecolor{currentstroke}{rgb}{0.000000,0.000000,0.000000}%
\pgfsetstrokecolor{currentstroke}%
\pgfsetdash{}{0pt}%
\pgfpathmoveto{\pgfqpoint{0.800000in}{1.074499in}}%
\pgfpathlineto{\pgfqpoint{0.960323in}{0.897128in}}%
\pgfpathlineto{\pgfqpoint{1.280970in}{0.552931in}}%
\pgfpathlineto{\pgfqpoint{1.304704in}{0.528000in}}%
\pgfpathlineto{\pgfqpoint{1.308485in}{0.528000in}}%
\pgfpathlineto{\pgfqpoint{0.994392in}{0.864000in}}%
\pgfpathlineto{\pgfqpoint{0.800000in}{1.078645in}}%
\pgfpathmoveto{\pgfqpoint{4.768000in}{2.431902in}}%
\pgfpathlineto{\pgfqpoint{4.487434in}{2.731298in}}%
\pgfpathlineto{\pgfqpoint{4.161499in}{3.066667in}}%
\pgfpathlineto{\pgfqpoint{4.006465in}{3.221752in}}%
\pgfpathlineto{\pgfqpoint{3.821953in}{3.402667in}}%
\pgfpathlineto{\pgfqpoint{3.666645in}{3.552000in}}%
\pgfpathlineto{\pgfqpoint{3.347338in}{3.850667in}}%
\pgfpathlineto{\pgfqpoint{3.183071in}{4.000000in}}%
\pgfpathlineto{\pgfqpoint{3.015536in}{4.149333in}}%
\pgfpathlineto{\pgfqpoint{2.930494in}{4.224000in}}%
\pgfpathlineto{\pgfqpoint{2.926210in}{4.224000in}}%
\pgfpathlineto{\pgfqpoint{3.261451in}{3.925333in}}%
\pgfpathlineto{\pgfqpoint{3.583899in}{3.626667in}}%
\pgfpathlineto{\pgfqpoint{3.740677in}{3.477333in}}%
\pgfpathlineto{\pgfqpoint{3.894662in}{3.328000in}}%
\pgfpathlineto{\pgfqpoint{4.219522in}{3.003786in}}%
\pgfpathlineto{\pgfqpoint{4.304303in}{2.917333in}}%
\pgfpathlineto{\pgfqpoint{4.625323in}{2.581333in}}%
\pgfpathlineto{\pgfqpoint{4.768000in}{2.427807in}}%
\pgfpathlineto{\pgfqpoint{4.768000in}{2.427807in}}%
\pgfusepath{fill}%
\end{pgfscope}%
\begin{pgfscope}%
\pgfpathrectangle{\pgfqpoint{0.800000in}{0.528000in}}{\pgfqpoint{3.968000in}{3.696000in}}%
\pgfusepath{clip}%
\pgfsetbuttcap%
\pgfsetroundjoin%
\definecolor{currentfill}{rgb}{0.177423,0.437527,0.557565}%
\pgfsetfillcolor{currentfill}%
\pgfsetlinewidth{0.000000pt}%
\definecolor{currentstroke}{rgb}{0.000000,0.000000,0.000000}%
\pgfsetstrokecolor{currentstroke}%
\pgfsetdash{}{0pt}%
\pgfpathmoveto{\pgfqpoint{0.800000in}{1.070353in}}%
\pgfpathlineto{\pgfqpoint{0.960323in}{0.893063in}}%
\pgfpathlineto{\pgfqpoint{1.265436in}{0.565333in}}%
\pgfpathlineto{\pgfqpoint{1.300922in}{0.528000in}}%
\pgfpathlineto{\pgfqpoint{1.304704in}{0.528000in}}%
\pgfpathlineto{\pgfqpoint{0.990668in}{0.864000in}}%
\pgfpathlineto{\pgfqpoint{0.800000in}{1.074499in}}%
\pgfpathmoveto{\pgfqpoint{4.768000in}{2.435936in}}%
\pgfpathlineto{\pgfqpoint{4.597819in}{2.618667in}}%
\pgfpathlineto{\pgfqpoint{4.447354in}{2.777158in}}%
\pgfpathlineto{\pgfqpoint{4.126707in}{3.105623in}}%
\pgfpathlineto{\pgfqpoint{3.966384in}{3.265219in}}%
\pgfpathlineto{\pgfqpoint{3.787342in}{3.440000in}}%
\pgfpathlineto{\pgfqpoint{3.631391in}{3.589333in}}%
\pgfpathlineto{\pgfqpoint{3.310706in}{3.888000in}}%
\pgfpathlineto{\pgfqpoint{3.145700in}{4.037333in}}%
\pgfpathlineto{\pgfqpoint{2.977390in}{4.186667in}}%
\pgfpathlineto{\pgfqpoint{2.934778in}{4.224000in}}%
\pgfpathlineto{\pgfqpoint{2.930494in}{4.224000in}}%
\pgfpathlineto{\pgfqpoint{3.255070in}{3.934779in}}%
\pgfpathlineto{\pgfqpoint{3.347338in}{3.850667in}}%
\pgfpathlineto{\pgfqpoint{3.508483in}{3.701333in}}%
\pgfpathlineto{\pgfqpoint{3.821953in}{3.402667in}}%
\pgfpathlineto{\pgfqpoint{3.974529in}{3.253333in}}%
\pgfpathlineto{\pgfqpoint{4.126707in}{3.101737in}}%
\pgfpathlineto{\pgfqpoint{4.308128in}{2.917333in}}%
\pgfpathlineto{\pgfqpoint{4.619032in}{2.591910in}}%
\pgfpathlineto{\pgfqpoint{4.698772in}{2.506667in}}%
\pgfpathlineto{\pgfqpoint{4.768000in}{2.431902in}}%
\pgfpathlineto{\pgfqpoint{4.768000in}{2.432000in}}%
\pgfusepath{fill}%
\end{pgfscope}%
\begin{pgfscope}%
\pgfpathrectangle{\pgfqpoint{0.800000in}{0.528000in}}{\pgfqpoint{3.968000in}{3.696000in}}%
\pgfusepath{clip}%
\pgfsetbuttcap%
\pgfsetroundjoin%
\definecolor{currentfill}{rgb}{0.175841,0.441290,0.557685}%
\pgfsetfillcolor{currentfill}%
\pgfsetlinewidth{0.000000pt}%
\definecolor{currentstroke}{rgb}{0.000000,0.000000,0.000000}%
\pgfsetstrokecolor{currentstroke}%
\pgfsetdash{}{0pt}%
\pgfpathmoveto{\pgfqpoint{0.800000in}{1.066207in}}%
\pgfpathlineto{\pgfqpoint{0.960323in}{0.888997in}}%
\pgfpathlineto{\pgfqpoint{1.261668in}{0.565333in}}%
\pgfpathlineto{\pgfqpoint{1.297141in}{0.528000in}}%
\pgfpathlineto{\pgfqpoint{1.300922in}{0.528000in}}%
\pgfpathlineto{\pgfqpoint{0.986944in}{0.864000in}}%
\pgfpathlineto{\pgfqpoint{0.800000in}{1.070353in}}%
\pgfpathmoveto{\pgfqpoint{4.768000in}{2.439969in}}%
\pgfpathlineto{\pgfqpoint{4.601598in}{2.618667in}}%
\pgfpathlineto{\pgfqpoint{4.447354in}{2.781099in}}%
\pgfpathlineto{\pgfqpoint{4.126707in}{3.109476in}}%
\pgfpathlineto{\pgfqpoint{3.944433in}{3.290667in}}%
\pgfpathlineto{\pgfqpoint{3.635406in}{3.589333in}}%
\pgfpathlineto{\pgfqpoint{3.314843in}{3.888000in}}%
\pgfpathlineto{\pgfqpoint{3.149901in}{4.037333in}}%
\pgfpathlineto{\pgfqpoint{2.981657in}{4.186667in}}%
\pgfpathlineto{\pgfqpoint{2.939061in}{4.224000in}}%
\pgfpathlineto{\pgfqpoint{2.934778in}{4.224000in}}%
\pgfpathlineto{\pgfqpoint{3.257107in}{3.936676in}}%
\pgfpathlineto{\pgfqpoint{3.351459in}{3.850667in}}%
\pgfpathlineto{\pgfqpoint{3.512543in}{3.701333in}}%
\pgfpathlineto{\pgfqpoint{3.825894in}{3.402667in}}%
\pgfpathlineto{\pgfqpoint{3.978414in}{3.253333in}}%
\pgfpathlineto{\pgfqpoint{4.148227in}{3.083955in}}%
\pgfpathlineto{\pgfqpoint{4.287030in}{2.942913in}}%
\pgfpathlineto{\pgfqpoint{4.456117in}{2.768000in}}%
\pgfpathlineto{\pgfqpoint{4.768000in}{2.435936in}}%
\pgfpathlineto{\pgfqpoint{4.768000in}{2.435936in}}%
\pgfusepath{fill}%
\end{pgfscope}%
\begin{pgfscope}%
\pgfpathrectangle{\pgfqpoint{0.800000in}{0.528000in}}{\pgfqpoint{3.968000in}{3.696000in}}%
\pgfusepath{clip}%
\pgfsetbuttcap%
\pgfsetroundjoin%
\definecolor{currentfill}{rgb}{0.175841,0.441290,0.557685}%
\pgfsetfillcolor{currentfill}%
\pgfsetlinewidth{0.000000pt}%
\definecolor{currentstroke}{rgb}{0.000000,0.000000,0.000000}%
\pgfsetstrokecolor{currentstroke}%
\pgfsetdash{}{0pt}%
\pgfpathmoveto{\pgfqpoint{0.800000in}{1.062062in}}%
\pgfpathlineto{\pgfqpoint{0.960323in}{0.884932in}}%
\pgfpathlineto{\pgfqpoint{1.257900in}{0.565333in}}%
\pgfpathlineto{\pgfqpoint{1.293360in}{0.528000in}}%
\pgfpathlineto{\pgfqpoint{1.297141in}{0.528000in}}%
\pgfpathlineto{\pgfqpoint{0.983221in}{0.864000in}}%
\pgfpathlineto{\pgfqpoint{0.800000in}{1.066207in}}%
\pgfpathmoveto{\pgfqpoint{4.768000in}{2.444002in}}%
\pgfpathlineto{\pgfqpoint{4.588945in}{2.636114in}}%
\pgfpathlineto{\pgfqpoint{4.427866in}{2.805333in}}%
\pgfpathlineto{\pgfqpoint{4.283215in}{2.954667in}}%
\pgfpathlineto{\pgfqpoint{3.948332in}{3.290667in}}%
\pgfpathlineto{\pgfqpoint{3.639420in}{3.589333in}}%
\pgfpathlineto{\pgfqpoint{3.318980in}{3.888000in}}%
\pgfpathlineto{\pgfqpoint{3.154102in}{4.037333in}}%
\pgfpathlineto{\pgfqpoint{2.985923in}{4.186667in}}%
\pgfpathlineto{\pgfqpoint{2.943345in}{4.224000in}}%
\pgfpathlineto{\pgfqpoint{2.939061in}{4.224000in}}%
\pgfpathlineto{\pgfqpoint{3.244929in}{3.951665in}}%
\pgfpathlineto{\pgfqpoint{3.580778in}{3.640827in}}%
\pgfpathlineto{\pgfqpoint{3.674644in}{3.552000in}}%
\pgfpathlineto{\pgfqpoint{3.829836in}{3.402667in}}%
\pgfpathlineto{\pgfqpoint{3.982299in}{3.253333in}}%
\pgfpathlineto{\pgfqpoint{4.132151in}{3.104000in}}%
\pgfpathlineto{\pgfqpoint{4.287030in}{2.946839in}}%
\pgfpathlineto{\pgfqpoint{4.459889in}{2.768000in}}%
\pgfpathlineto{\pgfqpoint{4.768000in}{2.439969in}}%
\pgfpathlineto{\pgfqpoint{4.768000in}{2.439969in}}%
\pgfusepath{fill}%
\end{pgfscope}%
\begin{pgfscope}%
\pgfpathrectangle{\pgfqpoint{0.800000in}{0.528000in}}{\pgfqpoint{3.968000in}{3.696000in}}%
\pgfusepath{clip}%
\pgfsetbuttcap%
\pgfsetroundjoin%
\definecolor{currentfill}{rgb}{0.175841,0.441290,0.557685}%
\pgfsetfillcolor{currentfill}%
\pgfsetlinewidth{0.000000pt}%
\definecolor{currentstroke}{rgb}{0.000000,0.000000,0.000000}%
\pgfsetstrokecolor{currentstroke}%
\pgfsetdash{}{0pt}%
\pgfpathmoveto{\pgfqpoint{0.800000in}{1.057916in}}%
\pgfpathlineto{\pgfqpoint{0.941640in}{0.901333in}}%
\pgfpathlineto{\pgfqpoint{1.080566in}{0.750305in}}%
\pgfpathlineto{\pgfqpoint{1.289579in}{0.528000in}}%
\pgfpathlineto{\pgfqpoint{1.293360in}{0.528000in}}%
\pgfpathlineto{\pgfqpoint{0.979497in}{0.864000in}}%
\pgfpathlineto{\pgfqpoint{0.800000in}{1.062062in}}%
\pgfpathmoveto{\pgfqpoint{4.768000in}{2.448035in}}%
\pgfpathlineto{\pgfqpoint{4.607677in}{2.620215in}}%
\pgfpathlineto{\pgfqpoint{4.431650in}{2.805333in}}%
\pgfpathlineto{\pgfqpoint{4.287030in}{2.954690in}}%
\pgfpathlineto{\pgfqpoint{3.952231in}{3.290667in}}%
\pgfpathlineto{\pgfqpoint{3.643435in}{3.589333in}}%
\pgfpathlineto{\pgfqpoint{3.323117in}{3.888000in}}%
\pgfpathlineto{\pgfqpoint{3.158303in}{4.037333in}}%
\pgfpathlineto{\pgfqpoint{2.990190in}{4.186667in}}%
\pgfpathlineto{\pgfqpoint{2.947629in}{4.224000in}}%
\pgfpathlineto{\pgfqpoint{2.943345in}{4.224000in}}%
\pgfpathlineto{\pgfqpoint{3.244929in}{3.955438in}}%
\pgfpathlineto{\pgfqpoint{3.565576in}{3.659164in}}%
\pgfpathlineto{\pgfqpoint{3.886222in}{3.351587in}}%
\pgfpathlineto{\pgfqpoint{4.061402in}{3.178667in}}%
\pgfpathlineto{\pgfqpoint{4.209935in}{3.029333in}}%
\pgfpathlineto{\pgfqpoint{4.534819in}{2.693333in}}%
\pgfpathlineto{\pgfqpoint{4.768000in}{2.444002in}}%
\pgfpathlineto{\pgfqpoint{4.768000in}{2.444002in}}%
\pgfusepath{fill}%
\end{pgfscope}%
\begin{pgfscope}%
\pgfpathrectangle{\pgfqpoint{0.800000in}{0.528000in}}{\pgfqpoint{3.968000in}{3.696000in}}%
\pgfusepath{clip}%
\pgfsetbuttcap%
\pgfsetroundjoin%
\definecolor{currentfill}{rgb}{0.175841,0.441290,0.557685}%
\pgfsetfillcolor{currentfill}%
\pgfsetlinewidth{0.000000pt}%
\definecolor{currentstroke}{rgb}{0.000000,0.000000,0.000000}%
\pgfsetstrokecolor{currentstroke}%
\pgfsetdash{}{0pt}%
\pgfpathmoveto{\pgfqpoint{0.800000in}{1.053770in}}%
\pgfpathlineto{\pgfqpoint{0.937929in}{0.901333in}}%
\pgfpathlineto{\pgfqpoint{1.080566in}{0.746313in}}%
\pgfpathlineto{\pgfqpoint{1.285798in}{0.528000in}}%
\pgfpathlineto{\pgfqpoint{1.289579in}{0.528000in}}%
\pgfpathlineto{\pgfqpoint{0.987206in}{0.851707in}}%
\pgfpathlineto{\pgfqpoint{0.907638in}{0.938667in}}%
\pgfpathlineto{\pgfqpoint{0.800000in}{1.057916in}}%
\pgfpathmoveto{\pgfqpoint{4.768000in}{2.452068in}}%
\pgfpathlineto{\pgfqpoint{4.607677in}{2.624171in}}%
\pgfpathlineto{\pgfqpoint{4.435435in}{2.805333in}}%
\pgfpathlineto{\pgfqpoint{4.287030in}{2.958558in}}%
\pgfpathlineto{\pgfqpoint{3.956130in}{3.290667in}}%
\pgfpathlineto{\pgfqpoint{3.645737in}{3.590935in}}%
\pgfpathlineto{\pgfqpoint{3.325091in}{3.889948in}}%
\pgfpathlineto{\pgfqpoint{3.162503in}{4.037333in}}%
\pgfpathlineto{\pgfqpoint{2.994457in}{4.186667in}}%
\pgfpathlineto{\pgfqpoint{2.951912in}{4.224000in}}%
\pgfpathlineto{\pgfqpoint{2.947629in}{4.224000in}}%
\pgfpathlineto{\pgfqpoint{3.244929in}{3.959212in}}%
\pgfpathlineto{\pgfqpoint{3.565576in}{3.662966in}}%
\pgfpathlineto{\pgfqpoint{3.886222in}{3.355418in}}%
\pgfpathlineto{\pgfqpoint{4.065260in}{3.178667in}}%
\pgfpathlineto{\pgfqpoint{4.213738in}{3.029333in}}%
\pgfpathlineto{\pgfqpoint{4.538564in}{2.693333in}}%
\pgfpathlineto{\pgfqpoint{4.768000in}{2.448035in}}%
\pgfpathlineto{\pgfqpoint{4.768000in}{2.448035in}}%
\pgfusepath{fill}%
\end{pgfscope}%
\begin{pgfscope}%
\pgfpathrectangle{\pgfqpoint{0.800000in}{0.528000in}}{\pgfqpoint{3.968000in}{3.696000in}}%
\pgfusepath{clip}%
\pgfsetbuttcap%
\pgfsetroundjoin%
\definecolor{currentfill}{rgb}{0.174274,0.445044,0.557792}%
\pgfsetfillcolor{currentfill}%
\pgfsetlinewidth{0.000000pt}%
\definecolor{currentstroke}{rgb}{0.000000,0.000000,0.000000}%
\pgfsetstrokecolor{currentstroke}%
\pgfsetdash{}{0pt}%
\pgfpathmoveto{\pgfqpoint{0.800000in}{1.049640in}}%
\pgfpathlineto{\pgfqpoint{1.106313in}{0.714667in}}%
\pgfpathlineto{\pgfqpoint{1.246596in}{0.565333in}}%
\pgfpathlineto{\pgfqpoint{1.282016in}{0.528000in}}%
\pgfpathlineto{\pgfqpoint{1.285798in}{0.528000in}}%
\pgfpathlineto{\pgfqpoint{0.985211in}{0.849848in}}%
\pgfpathlineto{\pgfqpoint{0.903940in}{0.938667in}}%
\pgfpathlineto{\pgfqpoint{0.800000in}{1.053770in}}%
\pgfpathlineto{\pgfqpoint{0.800000in}{1.050667in}}%
\pgfpathmoveto{\pgfqpoint{4.768000in}{2.456101in}}%
\pgfpathlineto{\pgfqpoint{4.607677in}{2.628128in}}%
\pgfpathlineto{\pgfqpoint{4.439220in}{2.805333in}}%
\pgfpathlineto{\pgfqpoint{4.287030in}{2.962426in}}%
\pgfpathlineto{\pgfqpoint{3.960030in}{3.290667in}}%
\pgfpathlineto{\pgfqpoint{3.645737in}{3.594690in}}%
\pgfpathlineto{\pgfqpoint{3.325091in}{3.893675in}}%
\pgfpathlineto{\pgfqpoint{3.164768in}{4.039045in}}%
\pgfpathlineto{\pgfqpoint{2.998724in}{4.186667in}}%
\pgfpathlineto{\pgfqpoint{2.956196in}{4.224000in}}%
\pgfpathlineto{\pgfqpoint{2.951912in}{4.224000in}}%
\pgfpathlineto{\pgfqpoint{3.245275in}{3.962667in}}%
\pgfpathlineto{\pgfqpoint{3.568471in}{3.664000in}}%
\pgfpathlineto{\pgfqpoint{3.886222in}{3.359249in}}%
\pgfpathlineto{\pgfqpoint{4.069118in}{3.178667in}}%
\pgfpathlineto{\pgfqpoint{4.217542in}{3.029333in}}%
\pgfpathlineto{\pgfqpoint{4.542310in}{2.693333in}}%
\pgfpathlineto{\pgfqpoint{4.768000in}{2.452068in}}%
\pgfpathlineto{\pgfqpoint{4.768000in}{2.452068in}}%
\pgfusepath{fill}%
\end{pgfscope}%
\begin{pgfscope}%
\pgfpathrectangle{\pgfqpoint{0.800000in}{0.528000in}}{\pgfqpoint{3.968000in}{3.696000in}}%
\pgfusepath{clip}%
\pgfsetbuttcap%
\pgfsetroundjoin%
\definecolor{currentfill}{rgb}{0.174274,0.445044,0.557792}%
\pgfsetfillcolor{currentfill}%
\pgfsetlinewidth{0.000000pt}%
\definecolor{currentstroke}{rgb}{0.000000,0.000000,0.000000}%
\pgfsetstrokecolor{currentstroke}%
\pgfsetdash{}{0pt}%
\pgfpathmoveto{\pgfqpoint{0.800000in}{1.045558in}}%
\pgfpathlineto{\pgfqpoint{1.102597in}{0.714667in}}%
\pgfpathlineto{\pgfqpoint{1.257551in}{0.549813in}}%
\pgfpathlineto{\pgfqpoint{1.278278in}{0.528000in}}%
\pgfpathlineto{\pgfqpoint{1.282016in}{0.528000in}}%
\pgfpathlineto{\pgfqpoint{1.280970in}{0.529100in}}%
\pgfpathlineto{\pgfqpoint{1.106313in}{0.714667in}}%
\pgfpathlineto{\pgfqpoint{0.800000in}{1.049640in}}%
\pgfpathmoveto{\pgfqpoint{4.768000in}{2.460134in}}%
\pgfpathlineto{\pgfqpoint{4.607677in}{2.632085in}}%
\pgfpathlineto{\pgfqpoint{4.443004in}{2.805333in}}%
\pgfpathlineto{\pgfqpoint{4.287030in}{2.966294in}}%
\pgfpathlineto{\pgfqpoint{3.963929in}{3.290667in}}%
\pgfpathlineto{\pgfqpoint{3.630620in}{3.612586in}}%
\pgfpathlineto{\pgfqpoint{3.536708in}{3.701333in}}%
\pgfpathlineto{\pgfqpoint{3.375998in}{3.850667in}}%
\pgfpathlineto{\pgfqpoint{3.084606in}{4.114419in}}%
\pgfpathlineto{\pgfqpoint{2.960480in}{4.224000in}}%
\pgfpathlineto{\pgfqpoint{2.956196in}{4.224000in}}%
\pgfpathlineto{\pgfqpoint{3.249372in}{3.962667in}}%
\pgfpathlineto{\pgfqpoint{3.572447in}{3.664000in}}%
\pgfpathlineto{\pgfqpoint{3.886222in}{3.363080in}}%
\pgfpathlineto{\pgfqpoint{4.046545in}{3.205016in}}%
\pgfpathlineto{\pgfqpoint{4.221345in}{3.029333in}}%
\pgfpathlineto{\pgfqpoint{4.546055in}{2.693333in}}%
\pgfpathlineto{\pgfqpoint{4.768000in}{2.456101in}}%
\pgfpathlineto{\pgfqpoint{4.768000in}{2.456101in}}%
\pgfusepath{fill}%
\end{pgfscope}%
\begin{pgfscope}%
\pgfpathrectangle{\pgfqpoint{0.800000in}{0.528000in}}{\pgfqpoint{3.968000in}{3.696000in}}%
\pgfusepath{clip}%
\pgfsetbuttcap%
\pgfsetroundjoin%
\definecolor{currentfill}{rgb}{0.174274,0.445044,0.557792}%
\pgfsetfillcolor{currentfill}%
\pgfsetlinewidth{0.000000pt}%
\definecolor{currentstroke}{rgb}{0.000000,0.000000,0.000000}%
\pgfsetstrokecolor{currentstroke}%
\pgfsetdash{}{0pt}%
\pgfpathmoveto{\pgfqpoint{0.800000in}{1.041476in}}%
\pgfpathlineto{\pgfqpoint{1.098880in}{0.714667in}}%
\pgfpathlineto{\pgfqpoint{1.240889in}{0.563432in}}%
\pgfpathlineto{\pgfqpoint{1.274556in}{0.528000in}}%
\pgfpathlineto{\pgfqpoint{1.278278in}{0.528000in}}%
\pgfpathlineto{\pgfqpoint{1.120646in}{0.695299in}}%
\pgfpathlineto{\pgfqpoint{0.960323in}{0.868671in}}%
\pgfpathlineto{\pgfqpoint{0.800000in}{1.045558in}}%
\pgfpathmoveto{\pgfqpoint{4.768000in}{2.464167in}}%
\pgfpathlineto{\pgfqpoint{4.607677in}{2.636041in}}%
\pgfpathlineto{\pgfqpoint{4.441956in}{2.810361in}}%
\pgfpathlineto{\pgfqpoint{4.265632in}{2.992000in}}%
\pgfpathlineto{\pgfqpoint{3.966384in}{3.292068in}}%
\pgfpathlineto{\pgfqpoint{3.632632in}{3.614459in}}%
\pgfpathlineto{\pgfqpoint{3.540699in}{3.701333in}}%
\pgfpathlineto{\pgfqpoint{3.380049in}{3.850667in}}%
\pgfpathlineto{\pgfqpoint{3.084606in}{4.118125in}}%
\pgfpathlineto{\pgfqpoint{2.964364in}{4.224000in}}%
\pgfpathlineto{\pgfqpoint{2.960480in}{4.224000in}}%
\pgfpathlineto{\pgfqpoint{3.244929in}{3.970421in}}%
\pgfpathlineto{\pgfqpoint{3.416456in}{3.813333in}}%
\pgfpathlineto{\pgfqpoint{3.576424in}{3.664000in}}%
\pgfpathlineto{\pgfqpoint{3.887813in}{3.365333in}}%
\pgfpathlineto{\pgfqpoint{4.046545in}{3.208862in}}%
\pgfpathlineto{\pgfqpoint{4.225149in}{3.029333in}}%
\pgfpathlineto{\pgfqpoint{4.549801in}{2.693333in}}%
\pgfpathlineto{\pgfqpoint{4.768000in}{2.460134in}}%
\pgfpathlineto{\pgfqpoint{4.768000in}{2.460134in}}%
\pgfusepath{fill}%
\end{pgfscope}%
\begin{pgfscope}%
\pgfpathrectangle{\pgfqpoint{0.800000in}{0.528000in}}{\pgfqpoint{3.968000in}{3.696000in}}%
\pgfusepath{clip}%
\pgfsetbuttcap%
\pgfsetroundjoin%
\definecolor{currentfill}{rgb}{0.172719,0.448791,0.557885}%
\pgfsetfillcolor{currentfill}%
\pgfsetlinewidth{0.000000pt}%
\definecolor{currentstroke}{rgb}{0.000000,0.000000,0.000000}%
\pgfsetstrokecolor{currentstroke}%
\pgfsetdash{}{0pt}%
\pgfpathmoveto{\pgfqpoint{0.800000in}{1.037394in}}%
\pgfpathlineto{\pgfqpoint{1.095164in}{0.714667in}}%
\pgfpathlineto{\pgfqpoint{1.240889in}{0.559515in}}%
\pgfpathlineto{\pgfqpoint{1.270834in}{0.528000in}}%
\pgfpathlineto{\pgfqpoint{1.274556in}{0.528000in}}%
\pgfpathlineto{\pgfqpoint{1.120646in}{0.691311in}}%
\pgfpathlineto{\pgfqpoint{0.960323in}{0.864606in}}%
\pgfpathlineto{\pgfqpoint{0.800000in}{1.041476in}}%
\pgfpathmoveto{\pgfqpoint{4.768000in}{2.468200in}}%
\pgfpathlineto{\pgfqpoint{4.627731in}{2.618667in}}%
\pgfpathlineto{\pgfqpoint{4.476892in}{2.777819in}}%
\pgfpathlineto{\pgfqpoint{4.305938in}{2.954667in}}%
\pgfpathlineto{\pgfqpoint{4.006465in}{3.256288in}}%
\pgfpathlineto{\pgfqpoint{3.663212in}{3.589333in}}%
\pgfpathlineto{\pgfqpoint{3.504822in}{3.738667in}}%
\pgfpathlineto{\pgfqpoint{3.343480in}{3.888000in}}%
\pgfpathlineto{\pgfqpoint{3.044525in}{4.157402in}}%
\pgfpathlineto{\pgfqpoint{2.968964in}{4.224000in}}%
\pgfpathlineto{\pgfqpoint{2.964756in}{4.224000in}}%
\pgfpathlineto{\pgfqpoint{3.133305in}{4.074667in}}%
\pgfpathlineto{\pgfqpoint{3.298587in}{3.925333in}}%
\pgfpathlineto{\pgfqpoint{3.460747in}{3.776000in}}%
\pgfpathlineto{\pgfqpoint{3.619922in}{3.626667in}}%
\pgfpathlineto{\pgfqpoint{3.929823in}{3.328000in}}%
\pgfpathlineto{\pgfqpoint{4.265632in}{2.992000in}}%
\pgfpathlineto{\pgfqpoint{4.588848in}{2.656000in}}%
\pgfpathlineto{\pgfqpoint{4.728679in}{2.506667in}}%
\pgfpathlineto{\pgfqpoint{4.768000in}{2.464167in}}%
\pgfpathlineto{\pgfqpoint{4.768000in}{2.464167in}}%
\pgfusepath{fill}%
\end{pgfscope}%
\begin{pgfscope}%
\pgfpathrectangle{\pgfqpoint{0.800000in}{0.528000in}}{\pgfqpoint{3.968000in}{3.696000in}}%
\pgfusepath{clip}%
\pgfsetbuttcap%
\pgfsetroundjoin%
\definecolor{currentfill}{rgb}{0.172719,0.448791,0.557885}%
\pgfsetfillcolor{currentfill}%
\pgfsetlinewidth{0.000000pt}%
\definecolor{currentstroke}{rgb}{0.000000,0.000000,0.000000}%
\pgfsetstrokecolor{currentstroke}%
\pgfsetdash{}{0pt}%
\pgfpathmoveto{\pgfqpoint{0.800000in}{1.033313in}}%
\pgfpathlineto{\pgfqpoint{1.091447in}{0.714667in}}%
\pgfpathlineto{\pgfqpoint{1.240889in}{0.555598in}}%
\pgfpathlineto{\pgfqpoint{1.267112in}{0.528000in}}%
\pgfpathlineto{\pgfqpoint{1.270834in}{0.528000in}}%
\pgfpathlineto{\pgfqpoint{1.120646in}{0.687324in}}%
\pgfpathlineto{\pgfqpoint{0.957203in}{0.864000in}}%
\pgfpathlineto{\pgfqpoint{0.800000in}{1.037394in}}%
\pgfpathmoveto{\pgfqpoint{4.768000in}{2.472189in}}%
\pgfpathlineto{\pgfqpoint{4.447354in}{2.812520in}}%
\pgfpathlineto{\pgfqpoint{4.273213in}{2.992000in}}%
\pgfpathlineto{\pgfqpoint{3.951469in}{3.314108in}}%
\pgfpathlineto{\pgfqpoint{3.861120in}{3.402667in}}%
\pgfpathlineto{\pgfqpoint{3.548682in}{3.701333in}}%
\pgfpathlineto{\pgfqpoint{3.388150in}{3.850667in}}%
\pgfpathlineto{\pgfqpoint{3.071437in}{4.137067in}}%
\pgfpathlineto{\pgfqpoint{2.973172in}{4.224000in}}%
\pgfpathlineto{\pgfqpoint{2.968964in}{4.224000in}}%
\pgfpathlineto{\pgfqpoint{3.137449in}{4.074667in}}%
\pgfpathlineto{\pgfqpoint{3.302669in}{3.925333in}}%
\pgfpathlineto{\pgfqpoint{3.464768in}{3.776000in}}%
\pgfpathlineto{\pgfqpoint{3.623884in}{3.626667in}}%
\pgfpathlineto{\pgfqpoint{3.933672in}{3.328000in}}%
\pgfpathlineto{\pgfqpoint{4.269423in}{2.992000in}}%
\pgfpathlineto{\pgfqpoint{4.567596in}{2.682459in}}%
\pgfpathlineto{\pgfqpoint{4.732361in}{2.506667in}}%
\pgfpathlineto{\pgfqpoint{4.768000in}{2.468200in}}%
\pgfpathlineto{\pgfqpoint{4.768000in}{2.469333in}}%
\pgfpathlineto{\pgfqpoint{4.768000in}{2.469333in}}%
\pgfusepath{fill}%
\end{pgfscope}%
\begin{pgfscope}%
\pgfpathrectangle{\pgfqpoint{0.800000in}{0.528000in}}{\pgfqpoint{3.968000in}{3.696000in}}%
\pgfusepath{clip}%
\pgfsetbuttcap%
\pgfsetroundjoin%
\definecolor{currentfill}{rgb}{0.172719,0.448791,0.557885}%
\pgfsetfillcolor{currentfill}%
\pgfsetlinewidth{0.000000pt}%
\definecolor{currentstroke}{rgb}{0.000000,0.000000,0.000000}%
\pgfsetstrokecolor{currentstroke}%
\pgfsetdash{}{0pt}%
\pgfpathmoveto{\pgfqpoint{0.800000in}{1.029231in}}%
\pgfpathlineto{\pgfqpoint{1.087731in}{0.714667in}}%
\pgfpathlineto{\pgfqpoint{1.240889in}{0.551681in}}%
\pgfpathlineto{\pgfqpoint{1.263390in}{0.528000in}}%
\pgfpathlineto{\pgfqpoint{1.267112in}{0.528000in}}%
\pgfpathlineto{\pgfqpoint{1.120646in}{0.683336in}}%
\pgfpathlineto{\pgfqpoint{0.953537in}{0.864000in}}%
\pgfpathlineto{\pgfqpoint{0.800000in}{1.033313in}}%
\pgfpathmoveto{\pgfqpoint{4.768000in}{2.476161in}}%
\pgfpathlineto{\pgfqpoint{4.447354in}{2.816403in}}%
\pgfpathlineto{\pgfqpoint{4.277003in}{2.992000in}}%
\pgfpathlineto{\pgfqpoint{3.953459in}{3.315961in}}%
\pgfpathlineto{\pgfqpoint{3.864998in}{3.402667in}}%
\pgfpathlineto{\pgfqpoint{3.552673in}{3.701333in}}%
\pgfpathlineto{\pgfqpoint{3.392201in}{3.850667in}}%
\pgfpathlineto{\pgfqpoint{3.061975in}{4.149333in}}%
\pgfpathlineto{\pgfqpoint{2.977380in}{4.224000in}}%
\pgfpathlineto{\pgfqpoint{2.973172in}{4.224000in}}%
\pgfpathlineto{\pgfqpoint{3.141592in}{4.074667in}}%
\pgfpathlineto{\pgfqpoint{3.306750in}{3.925333in}}%
\pgfpathlineto{\pgfqpoint{3.468789in}{3.776000in}}%
\pgfpathlineto{\pgfqpoint{3.627846in}{3.626667in}}%
\pgfpathlineto{\pgfqpoint{3.937522in}{3.328000in}}%
\pgfpathlineto{\pgfqpoint{4.260585in}{3.004701in}}%
\pgfpathlineto{\pgfqpoint{4.346069in}{2.917333in}}%
\pgfpathlineto{\pgfqpoint{4.666451in}{2.581333in}}%
\pgfpathlineto{\pgfqpoint{4.768000in}{2.472189in}}%
\pgfpathlineto{\pgfqpoint{4.768000in}{2.472189in}}%
\pgfusepath{fill}%
\end{pgfscope}%
\begin{pgfscope}%
\pgfpathrectangle{\pgfqpoint{0.800000in}{0.528000in}}{\pgfqpoint{3.968000in}{3.696000in}}%
\pgfusepath{clip}%
\pgfsetbuttcap%
\pgfsetroundjoin%
\definecolor{currentfill}{rgb}{0.172719,0.448791,0.557885}%
\pgfsetfillcolor{currentfill}%
\pgfsetlinewidth{0.000000pt}%
\definecolor{currentstroke}{rgb}{0.000000,0.000000,0.000000}%
\pgfsetstrokecolor{currentstroke}%
\pgfsetdash{}{0pt}%
\pgfpathmoveto{\pgfqpoint{0.800000in}{1.025149in}}%
\pgfpathlineto{\pgfqpoint{1.084014in}{0.714667in}}%
\pgfpathlineto{\pgfqpoint{1.240889in}{0.547764in}}%
\pgfpathlineto{\pgfqpoint{1.259668in}{0.528000in}}%
\pgfpathlineto{\pgfqpoint{1.263390in}{0.528000in}}%
\pgfpathlineto{\pgfqpoint{1.120646in}{0.679348in}}%
\pgfpathlineto{\pgfqpoint{0.949871in}{0.864000in}}%
\pgfpathlineto{\pgfqpoint{0.800000in}{1.029231in}}%
\pgfpathmoveto{\pgfqpoint{4.768000in}{2.480134in}}%
\pgfpathlineto{\pgfqpoint{4.447354in}{2.820287in}}%
\pgfpathlineto{\pgfqpoint{4.280793in}{2.992000in}}%
\pgfpathlineto{\pgfqpoint{3.945222in}{3.328000in}}%
\pgfpathlineto{\pgfqpoint{3.791860in}{3.477333in}}%
\pgfpathlineto{\pgfqpoint{3.635770in}{3.626667in}}%
\pgfpathlineto{\pgfqpoint{3.476831in}{3.776000in}}%
\pgfpathlineto{\pgfqpoint{3.149880in}{4.074667in}}%
\pgfpathlineto{\pgfqpoint{2.981587in}{4.224000in}}%
\pgfpathlineto{\pgfqpoint{2.977380in}{4.224000in}}%
\pgfpathlineto{\pgfqpoint{3.145736in}{4.074667in}}%
\pgfpathlineto{\pgfqpoint{3.310831in}{3.925333in}}%
\pgfpathlineto{\pgfqpoint{3.472810in}{3.776000in}}%
\pgfpathlineto{\pgfqpoint{3.631808in}{3.626667in}}%
\pgfpathlineto{\pgfqpoint{3.941372in}{3.328000in}}%
\pgfpathlineto{\pgfqpoint{4.262553in}{3.006534in}}%
\pgfpathlineto{\pgfqpoint{4.349833in}{2.917333in}}%
\pgfpathlineto{\pgfqpoint{4.670158in}{2.581333in}}%
\pgfpathlineto{\pgfqpoint{4.768000in}{2.476161in}}%
\pgfpathlineto{\pgfqpoint{4.768000in}{2.476161in}}%
\pgfusepath{fill}%
\end{pgfscope}%
\begin{pgfscope}%
\pgfpathrectangle{\pgfqpoint{0.800000in}{0.528000in}}{\pgfqpoint{3.968000in}{3.696000in}}%
\pgfusepath{clip}%
\pgfsetbuttcap%
\pgfsetroundjoin%
\definecolor{currentfill}{rgb}{0.171176,0.452530,0.557965}%
\pgfsetfillcolor{currentfill}%
\pgfsetlinewidth{0.000000pt}%
\definecolor{currentstroke}{rgb}{0.000000,0.000000,0.000000}%
\pgfsetstrokecolor{currentstroke}%
\pgfsetdash{}{0pt}%
\pgfpathmoveto{\pgfqpoint{0.800000in}{1.021067in}}%
\pgfpathlineto{\pgfqpoint{1.080566in}{0.714383in}}%
\pgfpathlineto{\pgfqpoint{1.255946in}{0.528000in}}%
\pgfpathlineto{\pgfqpoint{1.259668in}{0.528000in}}%
\pgfpathlineto{\pgfqpoint{1.106706in}{0.690318in}}%
\pgfpathlineto{\pgfqpoint{0.946204in}{0.864000in}}%
\pgfpathlineto{\pgfqpoint{0.800000in}{1.025149in}}%
\pgfpathmoveto{\pgfqpoint{4.768000in}{2.484106in}}%
\pgfpathlineto{\pgfqpoint{4.456835in}{2.814165in}}%
\pgfpathlineto{\pgfqpoint{4.367192in}{2.907211in}}%
\pgfpathlineto{\pgfqpoint{4.046545in}{3.231708in}}%
\pgfpathlineto{\pgfqpoint{3.718097in}{3.552000in}}%
\pgfpathlineto{\pgfqpoint{3.400303in}{3.850667in}}%
\pgfpathlineto{\pgfqpoint{3.070326in}{4.149333in}}%
\pgfpathlineto{\pgfqpoint{2.985795in}{4.224000in}}%
\pgfpathlineto{\pgfqpoint{2.981587in}{4.224000in}}%
\pgfpathlineto{\pgfqpoint{3.149880in}{4.074667in}}%
\pgfpathlineto{\pgfqpoint{3.314912in}{3.925333in}}%
\pgfpathlineto{\pgfqpoint{3.476831in}{3.776000in}}%
\pgfpathlineto{\pgfqpoint{3.635770in}{3.626667in}}%
\pgfpathlineto{\pgfqpoint{3.945222in}{3.328000in}}%
\pgfpathlineto{\pgfqpoint{4.246949in}{3.026506in}}%
\pgfpathlineto{\pgfqpoint{4.575308in}{2.686150in}}%
\pgfpathlineto{\pgfqpoint{4.743407in}{2.506667in}}%
\pgfpathlineto{\pgfqpoint{4.768000in}{2.480134in}}%
\pgfpathlineto{\pgfqpoint{4.768000in}{2.480134in}}%
\pgfusepath{fill}%
\end{pgfscope}%
\begin{pgfscope}%
\pgfpathrectangle{\pgfqpoint{0.800000in}{0.528000in}}{\pgfqpoint{3.968000in}{3.696000in}}%
\pgfusepath{clip}%
\pgfsetbuttcap%
\pgfsetroundjoin%
\definecolor{currentfill}{rgb}{0.171176,0.452530,0.557965}%
\pgfsetfillcolor{currentfill}%
\pgfsetlinewidth{0.000000pt}%
\definecolor{currentstroke}{rgb}{0.000000,0.000000,0.000000}%
\pgfsetstrokecolor{currentstroke}%
\pgfsetdash{}{0pt}%
\pgfpathmoveto{\pgfqpoint{0.800000in}{1.016985in}}%
\pgfpathlineto{\pgfqpoint{1.080566in}{0.710451in}}%
\pgfpathlineto{\pgfqpoint{1.252224in}{0.528000in}}%
\pgfpathlineto{\pgfqpoint{1.255946in}{0.528000in}}%
\pgfpathlineto{\pgfqpoint{1.115158in}{0.677333in}}%
\pgfpathlineto{\pgfqpoint{0.960323in}{0.844578in}}%
\pgfpathlineto{\pgfqpoint{0.800000in}{1.021067in}}%
\pgfpathmoveto{\pgfqpoint{4.768000in}{2.488078in}}%
\pgfpathlineto{\pgfqpoint{4.469149in}{2.805333in}}%
\pgfpathlineto{\pgfqpoint{4.153227in}{3.128702in}}%
\pgfpathlineto{\pgfqpoint{4.066140in}{3.216000in}}%
\pgfpathlineto{\pgfqpoint{3.914858in}{3.365333in}}%
\pgfpathlineto{\pgfqpoint{3.604261in}{3.664000in}}%
\pgfpathlineto{\pgfqpoint{3.282145in}{3.962667in}}%
\pgfpathlineto{\pgfqpoint{3.116437in}{4.112000in}}%
\pgfpathlineto{\pgfqpoint{2.990003in}{4.224000in}}%
\pgfpathlineto{\pgfqpoint{2.985795in}{4.224000in}}%
\pgfpathlineto{\pgfqpoint{3.154023in}{4.074667in}}%
\pgfpathlineto{\pgfqpoint{3.301715in}{3.940893in}}%
\pgfpathlineto{\pgfqpoint{3.400303in}{3.850667in}}%
\pgfpathlineto{\pgfqpoint{3.685818in}{3.582845in}}%
\pgfpathlineto{\pgfqpoint{4.024738in}{3.253333in}}%
\pgfpathlineto{\pgfqpoint{4.327111in}{2.948450in}}%
\pgfpathlineto{\pgfqpoint{4.501157in}{2.768000in}}%
\pgfpathlineto{\pgfqpoint{4.768000in}{2.484106in}}%
\pgfpathlineto{\pgfqpoint{4.768000in}{2.484106in}}%
\pgfusepath{fill}%
\end{pgfscope}%
\begin{pgfscope}%
\pgfpathrectangle{\pgfqpoint{0.800000in}{0.528000in}}{\pgfqpoint{3.968000in}{3.696000in}}%
\pgfusepath{clip}%
\pgfsetbuttcap%
\pgfsetroundjoin%
\definecolor{currentfill}{rgb}{0.171176,0.452530,0.557965}%
\pgfsetfillcolor{currentfill}%
\pgfsetlinewidth{0.000000pt}%
\definecolor{currentstroke}{rgb}{0.000000,0.000000,0.000000}%
\pgfsetstrokecolor{currentstroke}%
\pgfsetdash{}{0pt}%
\pgfpathmoveto{\pgfqpoint{0.800000in}{1.012910in}}%
\pgfpathlineto{\pgfqpoint{0.969423in}{0.826667in}}%
\pgfpathlineto{\pgfqpoint{1.120646in}{0.663604in}}%
\pgfpathlineto{\pgfqpoint{1.248502in}{0.528000in}}%
\pgfpathlineto{\pgfqpoint{1.252224in}{0.528000in}}%
\pgfpathlineto{\pgfqpoint{1.111486in}{0.677333in}}%
\pgfpathlineto{\pgfqpoint{0.960323in}{0.840574in}}%
\pgfpathlineto{\pgfqpoint{0.800000in}{1.016985in}}%
\pgfpathlineto{\pgfqpoint{0.800000in}{1.013333in}}%
\pgfpathmoveto{\pgfqpoint{4.768000in}{2.492051in}}%
\pgfpathlineto{\pgfqpoint{4.472875in}{2.805333in}}%
\pgfpathlineto{\pgfqpoint{4.144609in}{3.141333in}}%
\pgfpathlineto{\pgfqpoint{3.994658in}{3.290667in}}%
\pgfpathlineto{\pgfqpoint{3.842126in}{3.440000in}}%
\pgfpathlineto{\pgfqpoint{3.525495in}{3.741766in}}%
\pgfpathlineto{\pgfqpoint{3.365172in}{3.890447in}}%
\pgfpathlineto{\pgfqpoint{3.036549in}{4.186667in}}%
\pgfpathlineto{\pgfqpoint{2.994211in}{4.224000in}}%
\pgfpathlineto{\pgfqpoint{2.990003in}{4.224000in}}%
\pgfpathlineto{\pgfqpoint{3.140987in}{4.089849in}}%
\pgfpathlineto{\pgfqpoint{3.241019in}{4.000000in}}%
\pgfpathlineto{\pgfqpoint{3.525495in}{3.738065in}}%
\pgfpathlineto{\pgfqpoint{3.685818in}{3.586603in}}%
\pgfpathlineto{\pgfqpoint{4.028561in}{3.253333in}}%
\pgfpathlineto{\pgfqpoint{4.327111in}{2.952322in}}%
\pgfpathlineto{\pgfqpoint{4.504870in}{2.768000in}}%
\pgfpathlineto{\pgfqpoint{4.768000in}{2.488078in}}%
\pgfpathlineto{\pgfqpoint{4.768000in}{2.488078in}}%
\pgfusepath{fill}%
\end{pgfscope}%
\begin{pgfscope}%
\pgfpathrectangle{\pgfqpoint{0.800000in}{0.528000in}}{\pgfqpoint{3.968000in}{3.696000in}}%
\pgfusepath{clip}%
\pgfsetbuttcap%
\pgfsetroundjoin%
\definecolor{currentfill}{rgb}{0.171176,0.452530,0.557965}%
\pgfsetfillcolor{currentfill}%
\pgfsetlinewidth{0.000000pt}%
\definecolor{currentstroke}{rgb}{0.000000,0.000000,0.000000}%
\pgfsetstrokecolor{currentstroke}%
\pgfsetdash{}{0pt}%
\pgfpathmoveto{\pgfqpoint{0.800000in}{1.008890in}}%
\pgfpathlineto{\pgfqpoint{0.965744in}{0.826667in}}%
\pgfpathlineto{\pgfqpoint{1.120646in}{0.659675in}}%
\pgfpathlineto{\pgfqpoint{1.244780in}{0.528000in}}%
\pgfpathlineto{\pgfqpoint{1.248502in}{0.528000in}}%
\pgfpathlineto{\pgfqpoint{1.107815in}{0.677333in}}%
\pgfpathlineto{\pgfqpoint{0.960323in}{0.836570in}}%
\pgfpathlineto{\pgfqpoint{0.800000in}{1.012910in}}%
\pgfpathmoveto{\pgfqpoint{4.768000in}{2.496023in}}%
\pgfpathlineto{\pgfqpoint{4.476600in}{2.805333in}}%
\pgfpathlineto{\pgfqpoint{4.148391in}{3.141333in}}%
\pgfpathlineto{\pgfqpoint{3.998494in}{3.290667in}}%
\pgfpathlineto{\pgfqpoint{3.846017in}{3.440000in}}%
\pgfpathlineto{\pgfqpoint{3.525495in}{3.745458in}}%
\pgfpathlineto{\pgfqpoint{3.365172in}{3.894126in}}%
\pgfpathlineto{\pgfqpoint{3.040741in}{4.186667in}}%
\pgfpathlineto{\pgfqpoint{2.998419in}{4.224000in}}%
\pgfpathlineto{\pgfqpoint{2.994211in}{4.224000in}}%
\pgfpathlineto{\pgfqpoint{3.143004in}{4.091728in}}%
\pgfpathlineto{\pgfqpoint{3.245127in}{4.000000in}}%
\pgfpathlineto{\pgfqpoint{3.408352in}{3.850667in}}%
\pgfpathlineto{\pgfqpoint{3.728096in}{3.549954in}}%
\pgfpathlineto{\pgfqpoint{3.886222in}{3.397100in}}%
\pgfpathlineto{\pgfqpoint{4.218649in}{3.066667in}}%
\pgfpathlineto{\pgfqpoint{4.367192in}{2.914962in}}%
\pgfpathlineto{\pgfqpoint{4.544149in}{2.730667in}}%
\pgfpathlineto{\pgfqpoint{4.687838in}{2.578283in}}%
\pgfpathlineto{\pgfqpoint{4.768000in}{2.492051in}}%
\pgfpathlineto{\pgfqpoint{4.768000in}{2.492051in}}%
\pgfusepath{fill}%
\end{pgfscope}%
\begin{pgfscope}%
\pgfpathrectangle{\pgfqpoint{0.800000in}{0.528000in}}{\pgfqpoint{3.968000in}{3.696000in}}%
\pgfusepath{clip}%
\pgfsetbuttcap%
\pgfsetroundjoin%
\definecolor{currentfill}{rgb}{0.169646,0.456262,0.558030}%
\pgfsetfillcolor{currentfill}%
\pgfsetlinewidth{0.000000pt}%
\definecolor{currentstroke}{rgb}{0.000000,0.000000,0.000000}%
\pgfsetstrokecolor{currentstroke}%
\pgfsetdash{}{0pt}%
\pgfpathmoveto{\pgfqpoint{0.800000in}{1.004870in}}%
\pgfpathlineto{\pgfqpoint{0.972477in}{0.815346in}}%
\pgfpathlineto{\pgfqpoint{1.135413in}{0.640000in}}%
\pgfpathlineto{\pgfqpoint{1.241058in}{0.528000in}}%
\pgfpathlineto{\pgfqpoint{1.244780in}{0.528000in}}%
\pgfpathlineto{\pgfqpoint{1.104143in}{0.677333in}}%
\pgfpathlineto{\pgfqpoint{0.960323in}{0.832566in}}%
\pgfpathlineto{\pgfqpoint{0.800000in}{1.008890in}}%
\pgfpathmoveto{\pgfqpoint{4.768000in}{2.499996in}}%
\pgfpathlineto{\pgfqpoint{4.480325in}{2.805333in}}%
\pgfpathlineto{\pgfqpoint{4.152174in}{3.141333in}}%
\pgfpathlineto{\pgfqpoint{4.002331in}{3.290667in}}%
\pgfpathlineto{\pgfqpoint{3.846141in}{3.443601in}}%
\pgfpathlineto{\pgfqpoint{3.525495in}{3.749151in}}%
\pgfpathlineto{\pgfqpoint{3.350180in}{3.911369in}}%
\pgfpathlineto{\pgfqpoint{3.253212in}{4.000000in}}%
\pgfpathlineto{\pgfqpoint{3.124687in}{4.115719in}}%
\pgfpathlineto{\pgfqpoint{3.002626in}{4.224000in}}%
\pgfpathlineto{\pgfqpoint{2.998419in}{4.224000in}}%
\pgfpathlineto{\pgfqpoint{3.124755in}{4.112000in}}%
\pgfpathlineto{\pgfqpoint{3.452654in}{3.813333in}}%
\pgfpathlineto{\pgfqpoint{3.768730in}{3.514667in}}%
\pgfpathlineto{\pgfqpoint{3.926303in}{3.361696in}}%
\pgfpathlineto{\pgfqpoint{4.111153in}{3.178667in}}%
\pgfpathlineto{\pgfqpoint{4.407273in}{2.877422in}}%
\pgfpathlineto{\pgfqpoint{4.727919in}{2.539233in}}%
\pgfpathlineto{\pgfqpoint{4.768000in}{2.496023in}}%
\pgfpathlineto{\pgfqpoint{4.768000in}{2.496023in}}%
\pgfusepath{fill}%
\end{pgfscope}%
\begin{pgfscope}%
\pgfpathrectangle{\pgfqpoint{0.800000in}{0.528000in}}{\pgfqpoint{3.968000in}{3.696000in}}%
\pgfusepath{clip}%
\pgfsetbuttcap%
\pgfsetroundjoin%
\definecolor{currentfill}{rgb}{0.169646,0.456262,0.558030}%
\pgfsetfillcolor{currentfill}%
\pgfsetlinewidth{0.000000pt}%
\definecolor{currentstroke}{rgb}{0.000000,0.000000,0.000000}%
\pgfsetstrokecolor{currentstroke}%
\pgfsetdash{}{0pt}%
\pgfpathmoveto{\pgfqpoint{0.800000in}{1.000850in}}%
\pgfpathlineto{\pgfqpoint{0.960323in}{0.824590in}}%
\pgfpathlineto{\pgfqpoint{1.131729in}{0.640000in}}%
\pgfpathlineto{\pgfqpoint{1.237391in}{0.528000in}}%
\pgfpathlineto{\pgfqpoint{1.241058in}{0.528000in}}%
\pgfpathlineto{\pgfqpoint{1.240889in}{0.528178in}}%
\pgfpathlineto{\pgfqpoint{0.920242in}{0.872341in}}%
\pgfpathlineto{\pgfqpoint{0.800000in}{1.004870in}}%
\pgfpathmoveto{\pgfqpoint{4.768000in}{2.503968in}}%
\pgfpathlineto{\pgfqpoint{4.484050in}{2.805333in}}%
\pgfpathlineto{\pgfqpoint{4.155957in}{3.141333in}}%
\pgfpathlineto{\pgfqpoint{4.001106in}{3.295658in}}%
\pgfpathlineto{\pgfqpoint{3.830240in}{3.462522in}}%
\pgfpathlineto{\pgfqpoint{3.737572in}{3.552000in}}%
\pgfpathlineto{\pgfqpoint{3.580364in}{3.701333in}}%
\pgfpathlineto{\pgfqpoint{3.285010in}{3.974787in}}%
\pgfpathlineto{\pgfqpoint{3.124687in}{4.119378in}}%
\pgfpathlineto{\pgfqpoint{3.006793in}{4.224000in}}%
\pgfpathlineto{\pgfqpoint{3.002626in}{4.224000in}}%
\pgfpathlineto{\pgfqpoint{3.128843in}{4.112000in}}%
\pgfpathlineto{\pgfqpoint{3.456623in}{3.813333in}}%
\pgfpathlineto{\pgfqpoint{3.772586in}{3.514667in}}%
\pgfpathlineto{\pgfqpoint{3.929127in}{3.362703in}}%
\pgfpathlineto{\pgfqpoint{4.086626in}{3.206977in}}%
\pgfpathlineto{\pgfqpoint{4.262929in}{3.029333in}}%
\pgfpathlineto{\pgfqpoint{4.586950in}{2.693333in}}%
\pgfpathlineto{\pgfqpoint{4.727919in}{2.543201in}}%
\pgfpathlineto{\pgfqpoint{4.768000in}{2.499996in}}%
\pgfpathlineto{\pgfqpoint{4.768000in}{2.499996in}}%
\pgfusepath{fill}%
\end{pgfscope}%
\begin{pgfscope}%
\pgfpathrectangle{\pgfqpoint{0.800000in}{0.528000in}}{\pgfqpoint{3.968000in}{3.696000in}}%
\pgfusepath{clip}%
\pgfsetbuttcap%
\pgfsetroundjoin%
\definecolor{currentfill}{rgb}{0.169646,0.456262,0.558030}%
\pgfsetfillcolor{currentfill}%
\pgfsetlinewidth{0.000000pt}%
\definecolor{currentstroke}{rgb}{0.000000,0.000000,0.000000}%
\pgfsetstrokecolor{currentstroke}%
\pgfsetdash{}{0pt}%
\pgfpathmoveto{\pgfqpoint{0.800000in}{0.996830in}}%
\pgfpathlineto{\pgfqpoint{0.960323in}{0.820646in}}%
\pgfpathlineto{\pgfqpoint{1.128045in}{0.640000in}}%
\pgfpathlineto{\pgfqpoint{1.233726in}{0.528000in}}%
\pgfpathlineto{\pgfqpoint{1.237391in}{0.528000in}}%
\pgfpathlineto{\pgfqpoint{0.920242in}{0.868333in}}%
\pgfpathlineto{\pgfqpoint{0.800000in}{1.000850in}}%
\pgfpathmoveto{\pgfqpoint{4.768000in}{2.507921in}}%
\pgfpathlineto{\pgfqpoint{4.594325in}{2.693333in}}%
\pgfpathlineto{\pgfqpoint{4.270416in}{3.029333in}}%
\pgfpathlineto{\pgfqpoint{3.966384in}{3.333607in}}%
\pgfpathlineto{\pgfqpoint{3.780297in}{3.514667in}}%
\pgfpathlineto{\pgfqpoint{3.464560in}{3.813333in}}%
\pgfpathlineto{\pgfqpoint{3.302330in}{3.962667in}}%
\pgfpathlineto{\pgfqpoint{3.137018in}{4.112000in}}%
\pgfpathlineto{\pgfqpoint{3.010927in}{4.224000in}}%
\pgfpathlineto{\pgfqpoint{3.006793in}{4.224000in}}%
\pgfpathlineto{\pgfqpoint{3.339159in}{3.925333in}}%
\pgfpathlineto{\pgfqpoint{3.659317in}{3.626667in}}%
\pgfpathlineto{\pgfqpoint{3.815142in}{3.477333in}}%
\pgfpathlineto{\pgfqpoint{4.126707in}{3.170701in}}%
\pgfpathlineto{\pgfqpoint{4.303276in}{2.992000in}}%
\pgfpathlineto{\pgfqpoint{4.625888in}{2.656000in}}%
\pgfpathlineto{\pgfqpoint{4.768000in}{2.503968in}}%
\pgfpathlineto{\pgfqpoint{4.768000in}{2.506667in}}%
\pgfusepath{fill}%
\end{pgfscope}%
\begin{pgfscope}%
\pgfpathrectangle{\pgfqpoint{0.800000in}{0.528000in}}{\pgfqpoint{3.968000in}{3.696000in}}%
\pgfusepath{clip}%
\pgfsetbuttcap%
\pgfsetroundjoin%
\definecolor{currentfill}{rgb}{0.169646,0.456262,0.558030}%
\pgfsetfillcolor{currentfill}%
\pgfsetlinewidth{0.000000pt}%
\definecolor{currentstroke}{rgb}{0.000000,0.000000,0.000000}%
\pgfsetstrokecolor{currentstroke}%
\pgfsetdash{}{0pt}%
\pgfpathmoveto{\pgfqpoint{0.800000in}{0.992810in}}%
\pgfpathlineto{\pgfqpoint{0.960323in}{0.816702in}}%
\pgfpathlineto{\pgfqpoint{1.124360in}{0.640000in}}%
\pgfpathlineto{\pgfqpoint{1.230062in}{0.528000in}}%
\pgfpathlineto{\pgfqpoint{1.233726in}{0.528000in}}%
\pgfpathlineto{\pgfqpoint{0.920242in}{0.864325in}}%
\pgfpathlineto{\pgfqpoint{0.800000in}{0.996830in}}%
\pgfpathmoveto{\pgfqpoint{4.768000in}{2.511835in}}%
\pgfpathlineto{\pgfqpoint{4.598012in}{2.693333in}}%
\pgfpathlineto{\pgfqpoint{4.274160in}{3.029333in}}%
\pgfpathlineto{\pgfqpoint{3.966384in}{3.337337in}}%
\pgfpathlineto{\pgfqpoint{3.784153in}{3.514667in}}%
\pgfpathlineto{\pgfqpoint{3.468528in}{3.813333in}}%
\pgfpathlineto{\pgfqpoint{3.306357in}{3.962667in}}%
\pgfpathlineto{\pgfqpoint{3.141106in}{4.112000in}}%
\pgfpathlineto{\pgfqpoint{3.015062in}{4.224000in}}%
\pgfpathlineto{\pgfqpoint{3.010927in}{4.224000in}}%
\pgfpathlineto{\pgfqpoint{3.343171in}{3.925333in}}%
\pgfpathlineto{\pgfqpoint{3.663214in}{3.626667in}}%
\pgfpathlineto{\pgfqpoint{3.818984in}{3.477333in}}%
\pgfpathlineto{\pgfqpoint{4.126707in}{3.174499in}}%
\pgfpathlineto{\pgfqpoint{4.307007in}{2.992000in}}%
\pgfpathlineto{\pgfqpoint{4.629563in}{2.656000in}}%
\pgfpathlineto{\pgfqpoint{4.768000in}{2.507921in}}%
\pgfpathlineto{\pgfqpoint{4.768000in}{2.507921in}}%
\pgfusepath{fill}%
\end{pgfscope}%
\begin{pgfscope}%
\pgfpathrectangle{\pgfqpoint{0.800000in}{0.528000in}}{\pgfqpoint{3.968000in}{3.696000in}}%
\pgfusepath{clip}%
\pgfsetbuttcap%
\pgfsetroundjoin%
\definecolor{currentfill}{rgb}{0.168126,0.459988,0.558082}%
\pgfsetfillcolor{currentfill}%
\pgfsetlinewidth{0.000000pt}%
\definecolor{currentstroke}{rgb}{0.000000,0.000000,0.000000}%
\pgfsetstrokecolor{currentstroke}%
\pgfsetdash{}{0pt}%
\pgfpathmoveto{\pgfqpoint{0.800000in}{0.988790in}}%
\pgfpathlineto{\pgfqpoint{0.960323in}{0.812758in}}%
\pgfpathlineto{\pgfqpoint{1.120884in}{0.639778in}}%
\pgfpathlineto{\pgfqpoint{1.226397in}{0.528000in}}%
\pgfpathlineto{\pgfqpoint{1.230062in}{0.528000in}}%
\pgfpathlineto{\pgfqpoint{0.916925in}{0.864000in}}%
\pgfpathlineto{\pgfqpoint{0.800000in}{0.992810in}}%
\pgfpathmoveto{\pgfqpoint{4.768000in}{2.515749in}}%
\pgfpathlineto{\pgfqpoint{4.601699in}{2.693333in}}%
\pgfpathlineto{\pgfqpoint{4.277903in}{3.029333in}}%
\pgfpathlineto{\pgfqpoint{3.941653in}{3.365333in}}%
\pgfpathlineto{\pgfqpoint{3.788009in}{3.514667in}}%
\pgfpathlineto{\pgfqpoint{3.472496in}{3.813333in}}%
\pgfpathlineto{\pgfqpoint{3.310385in}{3.962667in}}%
\pgfpathlineto{\pgfqpoint{3.145194in}{4.112000in}}%
\pgfpathlineto{\pgfqpoint{3.019197in}{4.224000in}}%
\pgfpathlineto{\pgfqpoint{3.015062in}{4.224000in}}%
\pgfpathlineto{\pgfqpoint{3.325091in}{3.945570in}}%
\pgfpathlineto{\pgfqpoint{3.508605in}{3.776000in}}%
\pgfpathlineto{\pgfqpoint{3.667112in}{3.626667in}}%
\pgfpathlineto{\pgfqpoint{3.822827in}{3.477333in}}%
\pgfpathlineto{\pgfqpoint{4.126707in}{3.178297in}}%
\pgfpathlineto{\pgfqpoint{4.310738in}{2.992000in}}%
\pgfpathlineto{\pgfqpoint{4.621200in}{2.668596in}}%
\pgfpathlineto{\pgfqpoint{4.703281in}{2.581333in}}%
\pgfpathlineto{\pgfqpoint{4.768000in}{2.511835in}}%
\pgfpathlineto{\pgfqpoint{4.768000in}{2.511835in}}%
\pgfusepath{fill}%
\end{pgfscope}%
\begin{pgfscope}%
\pgfpathrectangle{\pgfqpoint{0.800000in}{0.528000in}}{\pgfqpoint{3.968000in}{3.696000in}}%
\pgfusepath{clip}%
\pgfsetbuttcap%
\pgfsetroundjoin%
\definecolor{currentfill}{rgb}{0.168126,0.459988,0.558082}%
\pgfsetfillcolor{currentfill}%
\pgfsetlinewidth{0.000000pt}%
\definecolor{currentstroke}{rgb}{0.000000,0.000000,0.000000}%
\pgfsetstrokecolor{currentstroke}%
\pgfsetdash{}{0pt}%
\pgfpathmoveto{\pgfqpoint{0.800000in}{0.984770in}}%
\pgfpathlineto{\pgfqpoint{0.960323in}{0.808814in}}%
\pgfpathlineto{\pgfqpoint{1.120646in}{0.636160in}}%
\pgfpathlineto{\pgfqpoint{1.222733in}{0.528000in}}%
\pgfpathlineto{\pgfqpoint{1.226397in}{0.528000in}}%
\pgfpathlineto{\pgfqpoint{0.913314in}{0.864000in}}%
\pgfpathlineto{\pgfqpoint{0.800000in}{0.988790in}}%
\pgfpathmoveto{\pgfqpoint{4.768000in}{2.519662in}}%
\pgfpathlineto{\pgfqpoint{4.605387in}{2.693333in}}%
\pgfpathlineto{\pgfqpoint{4.281647in}{3.029333in}}%
\pgfpathlineto{\pgfqpoint{3.945455in}{3.365333in}}%
\pgfpathlineto{\pgfqpoint{3.791864in}{3.514667in}}%
\pgfpathlineto{\pgfqpoint{3.476465in}{3.813333in}}%
\pgfpathlineto{\pgfqpoint{3.314412in}{3.962667in}}%
\pgfpathlineto{\pgfqpoint{3.149282in}{4.112000in}}%
\pgfpathlineto{\pgfqpoint{3.023331in}{4.224000in}}%
\pgfpathlineto{\pgfqpoint{3.019197in}{4.224000in}}%
\pgfpathlineto{\pgfqpoint{3.325091in}{3.949246in}}%
\pgfpathlineto{\pgfqpoint{3.485414in}{3.801325in}}%
\pgfpathlineto{\pgfqpoint{3.671009in}{3.626667in}}%
\pgfpathlineto{\pgfqpoint{3.826669in}{3.477333in}}%
\pgfpathlineto{\pgfqpoint{4.130079in}{3.178667in}}%
\pgfpathlineto{\pgfqpoint{4.287030in}{3.020039in}}%
\pgfpathlineto{\pgfqpoint{4.607677in}{2.687014in}}%
\pgfpathlineto{\pgfqpoint{4.768000in}{2.515749in}}%
\pgfpathlineto{\pgfqpoint{4.768000in}{2.515749in}}%
\pgfusepath{fill}%
\end{pgfscope}%
\begin{pgfscope}%
\pgfpathrectangle{\pgfqpoint{0.800000in}{0.528000in}}{\pgfqpoint{3.968000in}{3.696000in}}%
\pgfusepath{clip}%
\pgfsetbuttcap%
\pgfsetroundjoin%
\definecolor{currentfill}{rgb}{0.168126,0.459988,0.558082}%
\pgfsetfillcolor{currentfill}%
\pgfsetlinewidth{0.000000pt}%
\definecolor{currentstroke}{rgb}{0.000000,0.000000,0.000000}%
\pgfsetstrokecolor{currentstroke}%
\pgfsetdash{}{0pt}%
\pgfpathmoveto{\pgfqpoint{0.800000in}{0.980750in}}%
\pgfpathlineto{\pgfqpoint{0.949602in}{0.816681in}}%
\pgfpathlineto{\pgfqpoint{1.042224in}{0.716287in}}%
\pgfpathlineto{\pgfqpoint{1.120646in}{0.632289in}}%
\pgfpathlineto{\pgfqpoint{1.219068in}{0.528000in}}%
\pgfpathlineto{\pgfqpoint{1.222733in}{0.528000in}}%
\pgfpathlineto{\pgfqpoint{0.909704in}{0.864000in}}%
\pgfpathlineto{\pgfqpoint{0.800000in}{0.984770in}}%
\pgfpathmoveto{\pgfqpoint{4.768000in}{2.523576in}}%
\pgfpathlineto{\pgfqpoint{4.607677in}{2.694789in}}%
\pgfpathlineto{\pgfqpoint{4.285391in}{3.029333in}}%
\pgfpathlineto{\pgfqpoint{3.949257in}{3.365333in}}%
\pgfpathlineto{\pgfqpoint{3.795720in}{3.514667in}}%
\pgfpathlineto{\pgfqpoint{3.480433in}{3.813333in}}%
\pgfpathlineto{\pgfqpoint{3.318439in}{3.962667in}}%
\pgfpathlineto{\pgfqpoint{3.153370in}{4.112000in}}%
\pgfpathlineto{\pgfqpoint{3.027466in}{4.224000in}}%
\pgfpathlineto{\pgfqpoint{3.023331in}{4.224000in}}%
\pgfpathlineto{\pgfqpoint{3.325091in}{3.952921in}}%
\pgfpathlineto{\pgfqpoint{3.485414in}{3.805014in}}%
\pgfpathlineto{\pgfqpoint{3.660371in}{3.640297in}}%
\pgfpathlineto{\pgfqpoint{3.753049in}{3.552000in}}%
\pgfpathlineto{\pgfqpoint{3.907304in}{3.402667in}}%
\pgfpathlineto{\pgfqpoint{4.208074in}{3.104000in}}%
\pgfpathlineto{\pgfqpoint{4.534462in}{2.768000in}}%
\pgfpathlineto{\pgfqpoint{4.687838in}{2.605670in}}%
\pgfpathlineto{\pgfqpoint{4.768000in}{2.519662in}}%
\pgfpathlineto{\pgfqpoint{4.768000in}{2.519662in}}%
\pgfusepath{fill}%
\end{pgfscope}%
\begin{pgfscope}%
\pgfpathrectangle{\pgfqpoint{0.800000in}{0.528000in}}{\pgfqpoint{3.968000in}{3.696000in}}%
\pgfusepath{clip}%
\pgfsetbuttcap%
\pgfsetroundjoin%
\definecolor{currentfill}{rgb}{0.166617,0.463708,0.558119}%
\pgfsetfillcolor{currentfill}%
\pgfsetlinewidth{0.000000pt}%
\definecolor{currentstroke}{rgb}{0.000000,0.000000,0.000000}%
\pgfsetstrokecolor{currentstroke}%
\pgfsetdash{}{0pt}%
\pgfpathmoveto{\pgfqpoint{0.800000in}{0.976730in}}%
\pgfpathlineto{\pgfqpoint{0.936680in}{0.826667in}}%
\pgfpathlineto{\pgfqpoint{1.215403in}{0.528000in}}%
\pgfpathlineto{\pgfqpoint{1.219068in}{0.528000in}}%
\pgfpathlineto{\pgfqpoint{0.906093in}{0.864000in}}%
\pgfpathlineto{\pgfqpoint{0.800000in}{0.980750in}}%
\pgfpathmoveto{\pgfqpoint{4.768000in}{2.527490in}}%
\pgfpathlineto{\pgfqpoint{4.607677in}{2.698631in}}%
\pgfpathlineto{\pgfqpoint{4.287030in}{3.031446in}}%
\pgfpathlineto{\pgfqpoint{3.953058in}{3.365333in}}%
\pgfpathlineto{\pgfqpoint{3.799576in}{3.514667in}}%
\pgfpathlineto{\pgfqpoint{3.484402in}{3.813333in}}%
\pgfpathlineto{\pgfqpoint{3.322467in}{3.962667in}}%
\pgfpathlineto{\pgfqpoint{3.157458in}{4.112000in}}%
\pgfpathlineto{\pgfqpoint{3.031600in}{4.224000in}}%
\pgfpathlineto{\pgfqpoint{3.027466in}{4.224000in}}%
\pgfpathlineto{\pgfqpoint{3.325091in}{3.956596in}}%
\pgfpathlineto{\pgfqpoint{3.485414in}{3.808703in}}%
\pgfpathlineto{\pgfqpoint{3.645737in}{3.658081in}}%
\pgfpathlineto{\pgfqpoint{3.987232in}{3.328000in}}%
\pgfpathlineto{\pgfqpoint{4.137552in}{3.178667in}}%
\pgfpathlineto{\pgfqpoint{4.287030in}{3.027664in}}%
\pgfpathlineto{\pgfqpoint{4.609053in}{2.693333in}}%
\pgfpathlineto{\pgfqpoint{4.768000in}{2.523576in}}%
\pgfpathlineto{\pgfqpoint{4.768000in}{2.523576in}}%
\pgfusepath{fill}%
\end{pgfscope}%
\begin{pgfscope}%
\pgfpathrectangle{\pgfqpoint{0.800000in}{0.528000in}}{\pgfqpoint{3.968000in}{3.696000in}}%
\pgfusepath{clip}%
\pgfsetbuttcap%
\pgfsetroundjoin%
\definecolor{currentfill}{rgb}{0.166617,0.463708,0.558119}%
\pgfsetfillcolor{currentfill}%
\pgfsetlinewidth{0.000000pt}%
\definecolor{currentstroke}{rgb}{0.000000,0.000000,0.000000}%
\pgfsetstrokecolor{currentstroke}%
\pgfsetdash{}{0pt}%
\pgfpathmoveto{\pgfqpoint{0.800000in}{0.972760in}}%
\pgfpathlineto{\pgfqpoint{1.120646in}{0.624547in}}%
\pgfpathlineto{\pgfqpoint{1.211739in}{0.528000in}}%
\pgfpathlineto{\pgfqpoint{1.215403in}{0.528000in}}%
\pgfpathlineto{\pgfqpoint{0.902482in}{0.864000in}}%
\pgfpathlineto{\pgfqpoint{0.800000in}{0.976730in}}%
\pgfpathlineto{\pgfqpoint{0.800000in}{0.976000in}}%
\pgfpathmoveto{\pgfqpoint{4.768000in}{2.531403in}}%
\pgfpathlineto{\pgfqpoint{4.607677in}{2.702473in}}%
\pgfpathlineto{\pgfqpoint{4.287030in}{3.035204in}}%
\pgfpathlineto{\pgfqpoint{3.956860in}{3.365333in}}%
\pgfpathlineto{\pgfqpoint{3.803432in}{3.514667in}}%
\pgfpathlineto{\pgfqpoint{3.485414in}{3.816043in}}%
\pgfpathlineto{\pgfqpoint{3.325091in}{3.963929in}}%
\pgfpathlineto{\pgfqpoint{3.161546in}{4.112000in}}%
\pgfpathlineto{\pgfqpoint{3.035735in}{4.224000in}}%
\pgfpathlineto{\pgfqpoint{3.031600in}{4.224000in}}%
\pgfpathlineto{\pgfqpoint{3.325091in}{3.960272in}}%
\pgfpathlineto{\pgfqpoint{3.485414in}{3.812392in}}%
\pgfpathlineto{\pgfqpoint{3.645737in}{3.661784in}}%
\pgfpathlineto{\pgfqpoint{3.991021in}{3.328000in}}%
\pgfpathlineto{\pgfqpoint{4.141288in}{3.178667in}}%
\pgfpathlineto{\pgfqpoint{4.289102in}{3.029333in}}%
\pgfpathlineto{\pgfqpoint{4.612684in}{2.693333in}}%
\pgfpathlineto{\pgfqpoint{4.768000in}{2.527490in}}%
\pgfpathlineto{\pgfqpoint{4.768000in}{2.527490in}}%
\pgfusepath{fill}%
\end{pgfscope}%
\begin{pgfscope}%
\pgfpathrectangle{\pgfqpoint{0.800000in}{0.528000in}}{\pgfqpoint{3.968000in}{3.696000in}}%
\pgfusepath{clip}%
\pgfsetbuttcap%
\pgfsetroundjoin%
\definecolor{currentfill}{rgb}{0.166617,0.463708,0.558119}%
\pgfsetfillcolor{currentfill}%
\pgfsetlinewidth{0.000000pt}%
\definecolor{currentstroke}{rgb}{0.000000,0.000000,0.000000}%
\pgfsetstrokecolor{currentstroke}%
\pgfsetdash{}{0pt}%
\pgfpathmoveto{\pgfqpoint{0.800000in}{0.968800in}}%
\pgfpathlineto{\pgfqpoint{1.111038in}{0.631050in}}%
\pgfpathlineto{\pgfqpoint{1.200808in}{0.535661in}}%
\pgfpathlineto{\pgfqpoint{1.208074in}{0.528000in}}%
\pgfpathlineto{\pgfqpoint{1.211739in}{0.528000in}}%
\pgfpathlineto{\pgfqpoint{0.898872in}{0.864000in}}%
\pgfpathlineto{\pgfqpoint{0.800000in}{0.972760in}}%
\pgfpathmoveto{\pgfqpoint{4.768000in}{2.535317in}}%
\pgfpathlineto{\pgfqpoint{4.607677in}{2.706315in}}%
\pgfpathlineto{\pgfqpoint{4.287030in}{3.038962in}}%
\pgfpathlineto{\pgfqpoint{3.960662in}{3.365333in}}%
\pgfpathlineto{\pgfqpoint{3.806061in}{3.515833in}}%
\pgfpathlineto{\pgfqpoint{3.485414in}{3.819682in}}%
\pgfpathlineto{\pgfqpoint{3.325091in}{3.967555in}}%
\pgfpathlineto{\pgfqpoint{3.164768in}{4.112765in}}%
\pgfpathlineto{\pgfqpoint{3.039870in}{4.224000in}}%
\pgfpathlineto{\pgfqpoint{3.035735in}{4.224000in}}%
\pgfpathlineto{\pgfqpoint{3.326470in}{3.962667in}}%
\pgfpathlineto{\pgfqpoint{3.488322in}{3.813333in}}%
\pgfpathlineto{\pgfqpoint{3.647282in}{3.664000in}}%
\pgfpathlineto{\pgfqpoint{3.980898in}{3.341519in}}%
\pgfpathlineto{\pgfqpoint{4.070230in}{3.253333in}}%
\pgfpathlineto{\pgfqpoint{4.219206in}{3.104000in}}%
\pgfpathlineto{\pgfqpoint{4.545428in}{2.768000in}}%
\pgfpathlineto{\pgfqpoint{4.687838in}{2.617388in}}%
\pgfpathlineto{\pgfqpoint{4.768000in}{2.531403in}}%
\pgfpathlineto{\pgfqpoint{4.768000in}{2.531403in}}%
\pgfusepath{fill}%
\end{pgfscope}%
\begin{pgfscope}%
\pgfpathrectangle{\pgfqpoint{0.800000in}{0.528000in}}{\pgfqpoint{3.968000in}{3.696000in}}%
\pgfusepath{clip}%
\pgfsetbuttcap%
\pgfsetroundjoin%
\definecolor{currentfill}{rgb}{0.166617,0.463708,0.558119}%
\pgfsetfillcolor{currentfill}%
\pgfsetlinewidth{0.000000pt}%
\definecolor{currentstroke}{rgb}{0.000000,0.000000,0.000000}%
\pgfsetstrokecolor{currentstroke}%
\pgfsetdash{}{0pt}%
\pgfpathmoveto{\pgfqpoint{0.800000in}{0.964840in}}%
\pgfpathlineto{\pgfqpoint{1.098909in}{0.640000in}}%
\pgfpathlineto{\pgfqpoint{1.204410in}{0.528000in}}%
\pgfpathlineto{\pgfqpoint{1.208074in}{0.528000in}}%
\pgfpathlineto{\pgfqpoint{0.906838in}{0.851515in}}%
\pgfpathlineto{\pgfqpoint{0.827295in}{0.938667in}}%
\pgfpathlineto{\pgfqpoint{0.800000in}{0.968800in}}%
\pgfpathmoveto{\pgfqpoint{4.768000in}{2.539230in}}%
\pgfpathlineto{\pgfqpoint{4.607677in}{2.710156in}}%
\pgfpathlineto{\pgfqpoint{4.287030in}{3.042721in}}%
\pgfpathlineto{\pgfqpoint{3.964463in}{3.365333in}}%
\pgfpathlineto{\pgfqpoint{3.806061in}{3.519497in}}%
\pgfpathlineto{\pgfqpoint{3.485414in}{3.823320in}}%
\pgfpathlineto{\pgfqpoint{3.325091in}{3.971180in}}%
\pgfpathlineto{\pgfqpoint{3.164768in}{4.116377in}}%
\pgfpathlineto{\pgfqpoint{3.044004in}{4.224000in}}%
\pgfpathlineto{\pgfqpoint{3.039870in}{4.224000in}}%
\pgfpathlineto{\pgfqpoint{3.330431in}{3.962667in}}%
\pgfpathlineto{\pgfqpoint{3.492225in}{3.813333in}}%
\pgfpathlineto{\pgfqpoint{3.651130in}{3.664000in}}%
\pgfpathlineto{\pgfqpoint{3.982833in}{3.343321in}}%
\pgfpathlineto{\pgfqpoint{4.073992in}{3.253333in}}%
\pgfpathlineto{\pgfqpoint{4.222917in}{3.104000in}}%
\pgfpathlineto{\pgfqpoint{4.549084in}{2.768000in}}%
\pgfpathlineto{\pgfqpoint{4.690264in}{2.618667in}}%
\pgfpathlineto{\pgfqpoint{4.768000in}{2.535317in}}%
\pgfpathlineto{\pgfqpoint{4.768000in}{2.535317in}}%
\pgfusepath{fill}%
\end{pgfscope}%
\begin{pgfscope}%
\pgfpathrectangle{\pgfqpoint{0.800000in}{0.528000in}}{\pgfqpoint{3.968000in}{3.696000in}}%
\pgfusepath{clip}%
\pgfsetbuttcap%
\pgfsetroundjoin%
\definecolor{currentfill}{rgb}{0.165117,0.467423,0.558141}%
\pgfsetfillcolor{currentfill}%
\pgfsetlinewidth{0.000000pt}%
\definecolor{currentstroke}{rgb}{0.000000,0.000000,0.000000}%
\pgfsetstrokecolor{currentstroke}%
\pgfsetdash{}{0pt}%
\pgfpathmoveto{\pgfqpoint{0.800000in}{0.960881in}}%
\pgfpathlineto{\pgfqpoint{1.095281in}{0.640000in}}%
\pgfpathlineto{\pgfqpoint{1.200808in}{0.528000in}}%
\pgfpathlineto{\pgfqpoint{1.204410in}{0.528000in}}%
\pgfpathlineto{\pgfqpoint{0.904901in}{0.849710in}}%
\pgfpathlineto{\pgfqpoint{0.823708in}{0.938667in}}%
\pgfpathlineto{\pgfqpoint{0.800000in}{0.964840in}}%
\pgfpathmoveto{\pgfqpoint{4.768000in}{2.543144in}}%
\pgfpathlineto{\pgfqpoint{4.607677in}{2.713998in}}%
\pgfpathlineto{\pgfqpoint{4.287030in}{3.046479in}}%
\pgfpathlineto{\pgfqpoint{3.966384in}{3.367154in}}%
\pgfpathlineto{\pgfqpoint{3.790737in}{3.537727in}}%
\pgfpathlineto{\pgfqpoint{3.698088in}{3.626667in}}%
\pgfpathlineto{\pgfqpoint{3.392144in}{3.913124in}}%
\pgfpathlineto{\pgfqpoint{3.297465in}{4.000000in}}%
\pgfpathlineto{\pgfqpoint{3.048077in}{4.224000in}}%
\pgfpathlineto{\pgfqpoint{3.044004in}{4.224000in}}%
\pgfpathlineto{\pgfqpoint{3.044525in}{4.223540in}}%
\pgfpathlineto{\pgfqpoint{3.211116in}{4.074667in}}%
\pgfpathlineto{\pgfqpoint{3.375103in}{3.925333in}}%
\pgfpathlineto{\pgfqpoint{3.694254in}{3.626667in}}%
\pgfpathlineto{\pgfqpoint{4.023701in}{3.306722in}}%
\pgfpathlineto{\pgfqpoint{4.115203in}{3.216000in}}%
\pgfpathlineto{\pgfqpoint{4.263467in}{3.066667in}}%
\pgfpathlineto{\pgfqpoint{4.588227in}{2.730667in}}%
\pgfpathlineto{\pgfqpoint{4.768000in}{2.539230in}}%
\pgfpathlineto{\pgfqpoint{4.768000in}{2.539230in}}%
\pgfusepath{fill}%
\end{pgfscope}%
\begin{pgfscope}%
\pgfpathrectangle{\pgfqpoint{0.800000in}{0.528000in}}{\pgfqpoint{3.968000in}{3.696000in}}%
\pgfusepath{clip}%
\pgfsetbuttcap%
\pgfsetroundjoin%
\definecolor{currentfill}{rgb}{0.165117,0.467423,0.558141}%
\pgfsetfillcolor{currentfill}%
\pgfsetlinewidth{0.000000pt}%
\definecolor{currentstroke}{rgb}{0.000000,0.000000,0.000000}%
\pgfsetstrokecolor{currentstroke}%
\pgfsetdash{}{0pt}%
\pgfpathmoveto{\pgfqpoint{0.800000in}{0.956921in}}%
\pgfpathlineto{\pgfqpoint{1.091653in}{0.640000in}}%
\pgfpathlineto{\pgfqpoint{1.197137in}{0.528000in}}%
\pgfpathlineto{\pgfqpoint{1.200746in}{0.528000in}}%
\pgfpathlineto{\pgfqpoint{1.200283in}{0.528489in}}%
\pgfpathlineto{\pgfqpoint{1.025641in}{0.714667in}}%
\pgfpathlineto{\pgfqpoint{0.880162in}{0.872623in}}%
\pgfpathlineto{\pgfqpoint{0.800000in}{0.960881in}}%
\pgfpathmoveto{\pgfqpoint{4.768000in}{2.547013in}}%
\pgfpathlineto{\pgfqpoint{4.447354in}{2.885589in}}%
\pgfpathlineto{\pgfqpoint{4.122702in}{3.216000in}}%
\pgfpathlineto{\pgfqpoint{3.792685in}{3.539541in}}%
\pgfpathlineto{\pgfqpoint{3.701922in}{3.626667in}}%
\pgfpathlineto{\pgfqpoint{3.382994in}{3.925333in}}%
\pgfpathlineto{\pgfqpoint{3.084606in}{4.195198in}}%
\pgfpathlineto{\pgfqpoint{3.052141in}{4.224000in}}%
\pgfpathlineto{\pgfqpoint{3.048077in}{4.224000in}}%
\pgfpathlineto{\pgfqpoint{3.215120in}{4.074667in}}%
\pgfpathlineto{\pgfqpoint{3.379049in}{3.925333in}}%
\pgfpathlineto{\pgfqpoint{3.698088in}{3.626667in}}%
\pgfpathlineto{\pgfqpoint{4.006465in}{3.327714in}}%
\pgfpathlineto{\pgfqpoint{4.340376in}{2.992000in}}%
\pgfpathlineto{\pgfqpoint{4.487434in}{2.840161in}}%
\pgfpathlineto{\pgfqpoint{4.662410in}{2.656000in}}%
\pgfpathlineto{\pgfqpoint{4.768000in}{2.543144in}}%
\pgfpathlineto{\pgfqpoint{4.768000in}{2.544000in}}%
\pgfpathlineto{\pgfqpoint{4.768000in}{2.544000in}}%
\pgfusepath{fill}%
\end{pgfscope}%
\begin{pgfscope}%
\pgfpathrectangle{\pgfqpoint{0.800000in}{0.528000in}}{\pgfqpoint{3.968000in}{3.696000in}}%
\pgfusepath{clip}%
\pgfsetbuttcap%
\pgfsetroundjoin%
\definecolor{currentfill}{rgb}{0.165117,0.467423,0.558141}%
\pgfsetfillcolor{currentfill}%
\pgfsetlinewidth{0.000000pt}%
\definecolor{currentstroke}{rgb}{0.000000,0.000000,0.000000}%
\pgfsetstrokecolor{currentstroke}%
\pgfsetdash{}{0pt}%
\pgfpathmoveto{\pgfqpoint{0.800000in}{0.952961in}}%
\pgfpathlineto{\pgfqpoint{1.088025in}{0.640000in}}%
\pgfpathlineto{\pgfqpoint{1.193528in}{0.528000in}}%
\pgfpathlineto{\pgfqpoint{1.197137in}{0.528000in}}%
\pgfpathlineto{\pgfqpoint{1.040485in}{0.694812in}}%
\pgfpathlineto{\pgfqpoint{0.880162in}{0.868671in}}%
\pgfpathlineto{\pgfqpoint{0.800000in}{0.956921in}}%
\pgfpathmoveto{\pgfqpoint{4.768000in}{2.550870in}}%
\pgfpathlineto{\pgfqpoint{4.447354in}{2.889362in}}%
\pgfpathlineto{\pgfqpoint{4.126451in}{3.216000in}}%
\pgfpathlineto{\pgfqpoint{3.783716in}{3.552000in}}%
\pgfpathlineto{\pgfqpoint{3.467721in}{3.850667in}}%
\pgfpathlineto{\pgfqpoint{3.152580in}{4.137981in}}%
\pgfpathlineto{\pgfqpoint{3.056205in}{4.224000in}}%
\pgfpathlineto{\pgfqpoint{3.052141in}{4.224000in}}%
\pgfpathlineto{\pgfqpoint{3.219124in}{4.074667in}}%
\pgfpathlineto{\pgfqpoint{3.382994in}{3.925333in}}%
\pgfpathlineto{\pgfqpoint{3.701922in}{3.626667in}}%
\pgfpathlineto{\pgfqpoint{4.009908in}{3.328000in}}%
\pgfpathlineto{\pgfqpoint{4.344049in}{2.992000in}}%
\pgfpathlineto{\pgfqpoint{4.499163in}{2.831742in}}%
\pgfpathlineto{\pgfqpoint{4.666029in}{2.656000in}}%
\pgfpathlineto{\pgfqpoint{4.768000in}{2.547013in}}%
\pgfpathlineto{\pgfqpoint{4.768000in}{2.547013in}}%
\pgfusepath{fill}%
\end{pgfscope}%
\begin{pgfscope}%
\pgfpathrectangle{\pgfqpoint{0.800000in}{0.528000in}}{\pgfqpoint{3.968000in}{3.696000in}}%
\pgfusepath{clip}%
\pgfsetbuttcap%
\pgfsetroundjoin%
\definecolor{currentfill}{rgb}{0.165117,0.467423,0.558141}%
\pgfsetfillcolor{currentfill}%
\pgfsetlinewidth{0.000000pt}%
\definecolor{currentstroke}{rgb}{0.000000,0.000000,0.000000}%
\pgfsetstrokecolor{currentstroke}%
\pgfsetdash{}{0pt}%
\pgfpathmoveto{\pgfqpoint{0.800000in}{0.949001in}}%
\pgfpathlineto{\pgfqpoint{1.084397in}{0.640000in}}%
\pgfpathlineto{\pgfqpoint{1.189919in}{0.528000in}}%
\pgfpathlineto{\pgfqpoint{1.193528in}{0.528000in}}%
\pgfpathlineto{\pgfqpoint{1.040485in}{0.690934in}}%
\pgfpathlineto{\pgfqpoint{0.880162in}{0.864719in}}%
\pgfpathlineto{\pgfqpoint{0.800000in}{0.952961in}}%
\pgfpathmoveto{\pgfqpoint{4.768000in}{2.554726in}}%
\pgfpathlineto{\pgfqpoint{4.447354in}{2.893134in}}%
\pgfpathlineto{\pgfqpoint{4.126707in}{3.219440in}}%
\pgfpathlineto{\pgfqpoint{3.787524in}{3.552000in}}%
\pgfpathlineto{\pgfqpoint{3.471639in}{3.850667in}}%
\pgfpathlineto{\pgfqpoint{3.144098in}{4.149333in}}%
\pgfpathlineto{\pgfqpoint{3.060269in}{4.224000in}}%
\pgfpathlineto{\pgfqpoint{3.056205in}{4.224000in}}%
\pgfpathlineto{\pgfqpoint{3.223128in}{4.074667in}}%
\pgfpathlineto{\pgfqpoint{3.386940in}{3.925333in}}%
\pgfpathlineto{\pgfqpoint{3.705757in}{3.626667in}}%
\pgfpathlineto{\pgfqpoint{4.013637in}{3.328000in}}%
\pgfpathlineto{\pgfqpoint{4.347722in}{2.992000in}}%
\pgfpathlineto{\pgfqpoint{4.492312in}{2.842667in}}%
\pgfpathlineto{\pgfqpoint{4.647758in}{2.679261in}}%
\pgfpathlineto{\pgfqpoint{4.768000in}{2.550870in}}%
\pgfpathlineto{\pgfqpoint{4.768000in}{2.550870in}}%
\pgfusepath{fill}%
\end{pgfscope}%
\begin{pgfscope}%
\pgfpathrectangle{\pgfqpoint{0.800000in}{0.528000in}}{\pgfqpoint{3.968000in}{3.696000in}}%
\pgfusepath{clip}%
\pgfsetbuttcap%
\pgfsetroundjoin%
\definecolor{currentfill}{rgb}{0.163625,0.471133,0.558148}%
\pgfsetfillcolor{currentfill}%
\pgfsetlinewidth{0.000000pt}%
\definecolor{currentstroke}{rgb}{0.000000,0.000000,0.000000}%
\pgfsetstrokecolor{currentstroke}%
\pgfsetdash{}{0pt}%
\pgfpathmoveto{\pgfqpoint{0.800000in}{0.945042in}}%
\pgfpathlineto{\pgfqpoint{1.080769in}{0.640000in}}%
\pgfpathlineto{\pgfqpoint{1.186310in}{0.528000in}}%
\pgfpathlineto{\pgfqpoint{1.189919in}{0.528000in}}%
\pgfpathlineto{\pgfqpoint{1.040485in}{0.687055in}}%
\pgfpathlineto{\pgfqpoint{0.877252in}{0.864000in}}%
\pgfpathlineto{\pgfqpoint{0.800000in}{0.949001in}}%
\pgfpathmoveto{\pgfqpoint{4.768000in}{2.558583in}}%
\pgfpathlineto{\pgfqpoint{4.447354in}{2.896907in}}%
\pgfpathlineto{\pgfqpoint{4.126707in}{3.223132in}}%
\pgfpathlineto{\pgfqpoint{3.791331in}{3.552000in}}%
\pgfpathlineto{\pgfqpoint{3.475556in}{3.850667in}}%
\pgfpathlineto{\pgfqpoint{3.148132in}{4.149333in}}%
\pgfpathlineto{\pgfqpoint{3.064333in}{4.224000in}}%
\pgfpathlineto{\pgfqpoint{3.060269in}{4.224000in}}%
\pgfpathlineto{\pgfqpoint{3.227132in}{4.074667in}}%
\pgfpathlineto{\pgfqpoint{3.390886in}{3.925333in}}%
\pgfpathlineto{\pgfqpoint{3.697766in}{3.637795in}}%
\pgfpathlineto{\pgfqpoint{3.787524in}{3.552000in}}%
\pgfpathlineto{\pgfqpoint{4.092707in}{3.253333in}}%
\pgfpathlineto{\pgfqpoint{4.423950in}{2.917333in}}%
\pgfpathlineto{\pgfqpoint{4.743198in}{2.581333in}}%
\pgfpathlineto{\pgfqpoint{4.768000in}{2.554726in}}%
\pgfpathlineto{\pgfqpoint{4.768000in}{2.554726in}}%
\pgfusepath{fill}%
\end{pgfscope}%
\begin{pgfscope}%
\pgfpathrectangle{\pgfqpoint{0.800000in}{0.528000in}}{\pgfqpoint{3.968000in}{3.696000in}}%
\pgfusepath{clip}%
\pgfsetbuttcap%
\pgfsetroundjoin%
\definecolor{currentfill}{rgb}{0.163625,0.471133,0.558148}%
\pgfsetfillcolor{currentfill}%
\pgfsetlinewidth{0.000000pt}%
\definecolor{currentstroke}{rgb}{0.000000,0.000000,0.000000}%
\pgfsetstrokecolor{currentstroke}%
\pgfsetdash{}{0pt}%
\pgfpathmoveto{\pgfqpoint{0.800000in}{0.941082in}}%
\pgfpathlineto{\pgfqpoint{1.080566in}{0.636395in}}%
\pgfpathlineto{\pgfqpoint{1.182701in}{0.528000in}}%
\pgfpathlineto{\pgfqpoint{1.186310in}{0.528000in}}%
\pgfpathlineto{\pgfqpoint{1.040485in}{0.683176in}}%
\pgfpathlineto{\pgfqpoint{0.873696in}{0.864000in}}%
\pgfpathlineto{\pgfqpoint{0.800000in}{0.945042in}}%
\pgfpathmoveto{\pgfqpoint{4.768000in}{2.562439in}}%
\pgfpathlineto{\pgfqpoint{4.467286in}{2.880000in}}%
\pgfpathlineto{\pgfqpoint{4.322272in}{3.029333in}}%
\pgfpathlineto{\pgfqpoint{4.166788in}{3.186705in}}%
\pgfpathlineto{\pgfqpoint{3.833827in}{3.514667in}}%
\pgfpathlineto{\pgfqpoint{3.678066in}{3.664000in}}%
\pgfpathlineto{\pgfqpoint{3.358152in}{3.962667in}}%
\pgfpathlineto{\pgfqpoint{3.068397in}{4.224000in}}%
\pgfpathlineto{\pgfqpoint{3.064333in}{4.224000in}}%
\pgfpathlineto{\pgfqpoint{3.231136in}{4.074667in}}%
\pgfpathlineto{\pgfqpoint{3.394832in}{3.925333in}}%
\pgfpathlineto{\pgfqpoint{3.699693in}{3.639590in}}%
\pgfpathlineto{\pgfqpoint{3.791331in}{3.552000in}}%
\pgfpathlineto{\pgfqpoint{4.096410in}{3.253333in}}%
\pgfpathlineto{\pgfqpoint{4.427599in}{2.917333in}}%
\pgfpathlineto{\pgfqpoint{4.746793in}{2.581333in}}%
\pgfpathlineto{\pgfqpoint{4.768000in}{2.558583in}}%
\pgfpathlineto{\pgfqpoint{4.768000in}{2.558583in}}%
\pgfusepath{fill}%
\end{pgfscope}%
\begin{pgfscope}%
\pgfpathrectangle{\pgfqpoint{0.800000in}{0.528000in}}{\pgfqpoint{3.968000in}{3.696000in}}%
\pgfusepath{clip}%
\pgfsetbuttcap%
\pgfsetroundjoin%
\definecolor{currentfill}{rgb}{0.163625,0.471133,0.558148}%
\pgfsetfillcolor{currentfill}%
\pgfsetlinewidth{0.000000pt}%
\definecolor{currentstroke}{rgb}{0.000000,0.000000,0.000000}%
\pgfsetstrokecolor{currentstroke}%
\pgfsetdash{}{0pt}%
\pgfpathmoveto{\pgfqpoint{0.800000in}{0.937145in}}%
\pgfpathlineto{\pgfqpoint{0.969467in}{0.752000in}}%
\pgfpathlineto{\pgfqpoint{1.120646in}{0.589893in}}%
\pgfpathlineto{\pgfqpoint{1.179092in}{0.528000in}}%
\pgfpathlineto{\pgfqpoint{1.182701in}{0.528000in}}%
\pgfpathlineto{\pgfqpoint{1.040485in}{0.679298in}}%
\pgfpathlineto{\pgfqpoint{0.870139in}{0.864000in}}%
\pgfpathlineto{\pgfqpoint{0.800000in}{0.941082in}}%
\pgfpathlineto{\pgfqpoint{0.800000in}{0.938667in}}%
\pgfpathmoveto{\pgfqpoint{4.768000in}{2.566296in}}%
\pgfpathlineto{\pgfqpoint{4.470922in}{2.880000in}}%
\pgfpathlineto{\pgfqpoint{4.313788in}{3.041743in}}%
\pgfpathlineto{\pgfqpoint{4.141219in}{3.216000in}}%
\pgfpathlineto{\pgfqpoint{3.990687in}{3.365333in}}%
\pgfpathlineto{\pgfqpoint{3.681914in}{3.664000in}}%
\pgfpathlineto{\pgfqpoint{3.362112in}{3.962667in}}%
\pgfpathlineto{\pgfqpoint{3.072460in}{4.224000in}}%
\pgfpathlineto{\pgfqpoint{3.068397in}{4.224000in}}%
\pgfpathlineto{\pgfqpoint{3.235140in}{4.074667in}}%
\pgfpathlineto{\pgfqpoint{3.398777in}{3.925333in}}%
\pgfpathlineto{\pgfqpoint{3.685818in}{3.656636in}}%
\pgfpathlineto{\pgfqpoint{4.024824in}{3.328000in}}%
\pgfpathlineto{\pgfqpoint{4.327111in}{3.024394in}}%
\pgfpathlineto{\pgfqpoint{4.503184in}{2.842667in}}%
\pgfpathlineto{\pgfqpoint{4.647758in}{2.690797in}}%
\pgfpathlineto{\pgfqpoint{4.768000in}{2.562439in}}%
\pgfpathlineto{\pgfqpoint{4.768000in}{2.562439in}}%
\pgfusepath{fill}%
\end{pgfscope}%
\begin{pgfscope}%
\pgfpathrectangle{\pgfqpoint{0.800000in}{0.528000in}}{\pgfqpoint{3.968000in}{3.696000in}}%
\pgfusepath{clip}%
\pgfsetbuttcap%
\pgfsetroundjoin%
\definecolor{currentfill}{rgb}{0.162142,0.474838,0.558140}%
\pgfsetfillcolor{currentfill}%
\pgfsetlinewidth{0.000000pt}%
\definecolor{currentstroke}{rgb}{0.000000,0.000000,0.000000}%
\pgfsetstrokecolor{currentstroke}%
\pgfsetdash{}{0pt}%
\pgfpathmoveto{\pgfqpoint{0.800000in}{0.933243in}}%
\pgfpathlineto{\pgfqpoint{0.965875in}{0.752000in}}%
\pgfpathlineto{\pgfqpoint{1.120646in}{0.586078in}}%
\pgfpathlineto{\pgfqpoint{1.175483in}{0.528000in}}%
\pgfpathlineto{\pgfqpoint{1.179092in}{0.528000in}}%
\pgfpathlineto{\pgfqpoint{1.027266in}{0.689646in}}%
\pgfpathlineto{\pgfqpoint{0.866583in}{0.864000in}}%
\pgfpathlineto{\pgfqpoint{0.800000in}{0.937145in}}%
\pgfpathmoveto{\pgfqpoint{4.768000in}{2.570153in}}%
\pgfpathlineto{\pgfqpoint{4.474558in}{2.880000in}}%
\pgfpathlineto{\pgfqpoint{4.327111in}{3.031882in}}%
\pgfpathlineto{\pgfqpoint{4.144910in}{3.216000in}}%
\pgfpathlineto{\pgfqpoint{3.994429in}{3.365333in}}%
\pgfpathlineto{\pgfqpoint{3.665808in}{3.682694in}}%
\pgfpathlineto{\pgfqpoint{3.565576in}{3.777521in}}%
\pgfpathlineto{\pgfqpoint{3.243148in}{4.074667in}}%
\pgfpathlineto{\pgfqpoint{3.076524in}{4.224000in}}%
\pgfpathlineto{\pgfqpoint{3.072460in}{4.224000in}}%
\pgfpathlineto{\pgfqpoint{3.239144in}{4.074667in}}%
\pgfpathlineto{\pgfqpoint{3.383707in}{3.942598in}}%
\pgfpathlineto{\pgfqpoint{3.485414in}{3.848787in}}%
\pgfpathlineto{\pgfqpoint{3.806061in}{3.545152in}}%
\pgfpathlineto{\pgfqpoint{3.990687in}{3.365333in}}%
\pgfpathlineto{\pgfqpoint{4.313788in}{3.041743in}}%
\pgfpathlineto{\pgfqpoint{4.470922in}{2.880000in}}%
\pgfpathlineto{\pgfqpoint{4.613637in}{2.730667in}}%
\pgfpathlineto{\pgfqpoint{4.768000in}{2.566296in}}%
\pgfpathlineto{\pgfqpoint{4.768000in}{2.566296in}}%
\pgfusepath{fill}%
\end{pgfscope}%
\begin{pgfscope}%
\pgfpathrectangle{\pgfqpoint{0.800000in}{0.528000in}}{\pgfqpoint{3.968000in}{3.696000in}}%
\pgfusepath{clip}%
\pgfsetbuttcap%
\pgfsetroundjoin%
\definecolor{currentfill}{rgb}{0.162142,0.474838,0.558140}%
\pgfsetfillcolor{currentfill}%
\pgfsetlinewidth{0.000000pt}%
\definecolor{currentstroke}{rgb}{0.000000,0.000000,0.000000}%
\pgfsetstrokecolor{currentstroke}%
\pgfsetdash{}{0pt}%
\pgfpathmoveto{\pgfqpoint{0.800000in}{0.929342in}}%
\pgfpathlineto{\pgfqpoint{0.974501in}{0.738794in}}%
\pgfpathlineto{\pgfqpoint{1.136607in}{0.565333in}}%
\pgfpathlineto{\pgfqpoint{1.171875in}{0.528000in}}%
\pgfpathlineto{\pgfqpoint{1.175483in}{0.528000in}}%
\pgfpathlineto{\pgfqpoint{1.035165in}{0.677333in}}%
\pgfpathlineto{\pgfqpoint{0.880162in}{0.845240in}}%
\pgfpathlineto{\pgfqpoint{0.800000in}{0.933243in}}%
\pgfpathmoveto{\pgfqpoint{4.768000in}{2.574009in}}%
\pgfpathlineto{\pgfqpoint{4.478195in}{2.880000in}}%
\pgfpathlineto{\pgfqpoint{4.327111in}{3.035591in}}%
\pgfpathlineto{\pgfqpoint{4.148601in}{3.216000in}}%
\pgfpathlineto{\pgfqpoint{3.998170in}{3.365333in}}%
\pgfpathlineto{\pgfqpoint{3.667737in}{3.684492in}}%
\pgfpathlineto{\pgfqpoint{3.568293in}{3.778531in}}%
\pgfpathlineto{\pgfqpoint{3.485414in}{3.855991in}}%
\pgfpathlineto{\pgfqpoint{3.164266in}{4.149333in}}%
\pgfpathlineto{\pgfqpoint{3.080588in}{4.224000in}}%
\pgfpathlineto{\pgfqpoint{3.076524in}{4.224000in}}%
\pgfpathlineto{\pgfqpoint{3.223579in}{4.092113in}}%
\pgfpathlineto{\pgfqpoint{3.325285in}{4.000000in}}%
\pgfpathlineto{\pgfqpoint{3.646411in}{3.701333in}}%
\pgfpathlineto{\pgfqpoint{3.980694in}{3.378662in}}%
\pgfpathlineto{\pgfqpoint{4.069979in}{3.290667in}}%
\pgfpathlineto{\pgfqpoint{4.385546in}{2.971762in}}%
\pgfpathlineto{\pgfqpoint{4.474558in}{2.880000in}}%
\pgfpathlineto{\pgfqpoint{4.617226in}{2.730667in}}%
\pgfpathlineto{\pgfqpoint{4.768000in}{2.570153in}}%
\pgfpathlineto{\pgfqpoint{4.768000in}{2.570153in}}%
\pgfusepath{fill}%
\end{pgfscope}%
\begin{pgfscope}%
\pgfpathrectangle{\pgfqpoint{0.800000in}{0.528000in}}{\pgfqpoint{3.968000in}{3.696000in}}%
\pgfusepath{clip}%
\pgfsetbuttcap%
\pgfsetroundjoin%
\definecolor{currentfill}{rgb}{0.162142,0.474838,0.558140}%
\pgfsetfillcolor{currentfill}%
\pgfsetlinewidth{0.000000pt}%
\definecolor{currentstroke}{rgb}{0.000000,0.000000,0.000000}%
\pgfsetstrokecolor{currentstroke}%
\pgfsetdash{}{0pt}%
\pgfpathmoveto{\pgfqpoint{0.800000in}{0.925441in}}%
\pgfpathlineto{\pgfqpoint{0.960323in}{0.750260in}}%
\pgfpathlineto{\pgfqpoint{1.133010in}{0.565333in}}%
\pgfpathlineto{\pgfqpoint{1.168266in}{0.528000in}}%
\pgfpathlineto{\pgfqpoint{1.171875in}{0.528000in}}%
\pgfpathlineto{\pgfqpoint{1.031604in}{0.677333in}}%
\pgfpathlineto{\pgfqpoint{0.880162in}{0.841346in}}%
\pgfpathlineto{\pgfqpoint{0.800000in}{0.929342in}}%
\pgfpathmoveto{\pgfqpoint{4.768000in}{2.577866in}}%
\pgfpathlineto{\pgfqpoint{4.481831in}{2.880000in}}%
\pgfpathlineto{\pgfqpoint{4.327111in}{3.039300in}}%
\pgfpathlineto{\pgfqpoint{4.152292in}{3.216000in}}%
\pgfpathlineto{\pgfqpoint{4.001912in}{3.365333in}}%
\pgfpathlineto{\pgfqpoint{3.669667in}{3.686289in}}%
\pgfpathlineto{\pgfqpoint{3.574847in}{3.776000in}}%
\pgfpathlineto{\pgfqpoint{3.285010in}{4.043855in}}%
\pgfpathlineto{\pgfqpoint{3.084606in}{4.224000in}}%
\pgfpathlineto{\pgfqpoint{3.080588in}{4.224000in}}%
\pgfpathlineto{\pgfqpoint{3.225537in}{4.093937in}}%
\pgfpathlineto{\pgfqpoint{3.325091in}{4.003753in}}%
\pgfpathlineto{\pgfqpoint{3.650211in}{3.701333in}}%
\pgfpathlineto{\pgfqpoint{3.982603in}{3.380441in}}%
\pgfpathlineto{\pgfqpoint{4.073695in}{3.290667in}}%
\pgfpathlineto{\pgfqpoint{4.369722in}{2.992000in}}%
\pgfpathlineto{\pgfqpoint{4.691309in}{2.656000in}}%
\pgfpathlineto{\pgfqpoint{4.768000in}{2.574009in}}%
\pgfpathlineto{\pgfqpoint{4.768000in}{2.574009in}}%
\pgfusepath{fill}%
\end{pgfscope}%
\begin{pgfscope}%
\pgfpathrectangle{\pgfqpoint{0.800000in}{0.528000in}}{\pgfqpoint{3.968000in}{3.696000in}}%
\pgfusepath{clip}%
\pgfsetbuttcap%
\pgfsetroundjoin%
\definecolor{currentfill}{rgb}{0.162142,0.474838,0.558140}%
\pgfsetfillcolor{currentfill}%
\pgfsetlinewidth{0.000000pt}%
\definecolor{currentstroke}{rgb}{0.000000,0.000000,0.000000}%
\pgfsetstrokecolor{currentstroke}%
\pgfsetdash{}{0pt}%
\pgfpathmoveto{\pgfqpoint{0.800000in}{0.921539in}}%
\pgfpathlineto{\pgfqpoint{0.960323in}{0.746430in}}%
\pgfpathlineto{\pgfqpoint{1.129413in}{0.565333in}}%
\pgfpathlineto{\pgfqpoint{1.164657in}{0.528000in}}%
\pgfpathlineto{\pgfqpoint{1.168266in}{0.528000in}}%
\pgfpathlineto{\pgfqpoint{1.028042in}{0.677333in}}%
\pgfpathlineto{\pgfqpoint{0.880162in}{0.837452in}}%
\pgfpathlineto{\pgfqpoint{0.800000in}{0.925441in}}%
\pgfpathmoveto{\pgfqpoint{4.768000in}{2.581717in}}%
\pgfpathlineto{\pgfqpoint{4.592565in}{2.768000in}}%
\pgfpathlineto{\pgfqpoint{4.267131in}{3.104000in}}%
\pgfpathlineto{\pgfqpoint{4.118632in}{3.253333in}}%
\pgfpathlineto{\pgfqpoint{3.966384in}{3.403920in}}%
\pgfpathlineto{\pgfqpoint{3.775239in}{3.589333in}}%
\pgfpathlineto{\pgfqpoint{3.471974in}{3.875481in}}%
\pgfpathlineto{\pgfqpoint{3.377743in}{3.962667in}}%
\pgfpathlineto{\pgfqpoint{3.213695in}{4.112000in}}%
\pgfpathlineto{\pgfqpoint{3.088647in}{4.224000in}}%
\pgfpathlineto{\pgfqpoint{3.084651in}{4.224000in}}%
\pgfpathlineto{\pgfqpoint{3.414409in}{3.925333in}}%
\pgfpathlineto{\pgfqpoint{3.574847in}{3.776000in}}%
\pgfpathlineto{\pgfqpoint{3.732491in}{3.626667in}}%
\pgfpathlineto{\pgfqpoint{3.914468in}{3.451024in}}%
\pgfpathlineto{\pgfqpoint{4.062362in}{3.305399in}}%
\pgfpathlineto{\pgfqpoint{4.152292in}{3.216000in}}%
\pgfpathlineto{\pgfqpoint{4.447354in}{2.915769in}}%
\pgfpathlineto{\pgfqpoint{4.624402in}{2.730667in}}%
\pgfpathlineto{\pgfqpoint{4.768000in}{2.577866in}}%
\pgfpathlineto{\pgfqpoint{4.768000in}{2.581333in}}%
\pgfpathlineto{\pgfqpoint{4.768000in}{2.581333in}}%
\pgfusepath{fill}%
\end{pgfscope}%
\begin{pgfscope}%
\pgfpathrectangle{\pgfqpoint{0.800000in}{0.528000in}}{\pgfqpoint{3.968000in}{3.696000in}}%
\pgfusepath{clip}%
\pgfsetbuttcap%
\pgfsetroundjoin%
\definecolor{currentfill}{rgb}{0.160665,0.478540,0.558115}%
\pgfsetfillcolor{currentfill}%
\pgfsetlinewidth{0.000000pt}%
\definecolor{currentstroke}{rgb}{0.000000,0.000000,0.000000}%
\pgfsetstrokecolor{currentstroke}%
\pgfsetdash{}{0pt}%
\pgfpathmoveto{\pgfqpoint{0.800000in}{0.917638in}}%
\pgfpathlineto{\pgfqpoint{0.960323in}{0.742600in}}%
\pgfpathlineto{\pgfqpoint{1.125816in}{0.565333in}}%
\pgfpathlineto{\pgfqpoint{1.161048in}{0.528000in}}%
\pgfpathlineto{\pgfqpoint{1.164657in}{0.528000in}}%
\pgfpathlineto{\pgfqpoint{1.024481in}{0.677333in}}%
\pgfpathlineto{\pgfqpoint{0.880162in}{0.833559in}}%
\pgfpathlineto{\pgfqpoint{0.800000in}{0.921539in}}%
\pgfpathmoveto{\pgfqpoint{4.768000in}{2.585518in}}%
\pgfpathlineto{\pgfqpoint{4.596165in}{2.768000in}}%
\pgfpathlineto{\pgfqpoint{4.270785in}{3.104000in}}%
\pgfpathlineto{\pgfqpoint{4.122335in}{3.253333in}}%
\pgfpathlineto{\pgfqpoint{3.966384in}{3.407548in}}%
\pgfpathlineto{\pgfqpoint{3.778999in}{3.589333in}}%
\pgfpathlineto{\pgfqpoint{3.462522in}{3.888000in}}%
\pgfpathlineto{\pgfqpoint{3.300019in}{4.037333in}}%
\pgfpathlineto{\pgfqpoint{3.092643in}{4.224000in}}%
\pgfpathlineto{\pgfqpoint{3.088647in}{4.224000in}}%
\pgfpathlineto{\pgfqpoint{3.418290in}{3.925333in}}%
\pgfpathlineto{\pgfqpoint{3.578674in}{3.776000in}}%
\pgfpathlineto{\pgfqpoint{3.736264in}{3.626667in}}%
\pgfpathlineto{\pgfqpoint{3.891176in}{3.477333in}}%
\pgfpathlineto{\pgfqpoint{4.046545in}{3.324960in}}%
\pgfpathlineto{\pgfqpoint{4.376956in}{2.992000in}}%
\pgfpathlineto{\pgfqpoint{4.698438in}{2.656000in}}%
\pgfpathlineto{\pgfqpoint{4.768000in}{2.581717in}}%
\pgfpathlineto{\pgfqpoint{4.768000in}{2.581717in}}%
\pgfusepath{fill}%
\end{pgfscope}%
\begin{pgfscope}%
\pgfpathrectangle{\pgfqpoint{0.800000in}{0.528000in}}{\pgfqpoint{3.968000in}{3.696000in}}%
\pgfusepath{clip}%
\pgfsetbuttcap%
\pgfsetroundjoin%
\definecolor{currentfill}{rgb}{0.160665,0.478540,0.558115}%
\pgfsetfillcolor{currentfill}%
\pgfsetlinewidth{0.000000pt}%
\definecolor{currentstroke}{rgb}{0.000000,0.000000,0.000000}%
\pgfsetstrokecolor{currentstroke}%
\pgfsetdash{}{0pt}%
\pgfpathmoveto{\pgfqpoint{0.800000in}{0.913737in}}%
\pgfpathlineto{\pgfqpoint{0.960323in}{0.738770in}}%
\pgfpathlineto{\pgfqpoint{1.122219in}{0.565333in}}%
\pgfpathlineto{\pgfqpoint{1.157488in}{0.528000in}}%
\pgfpathlineto{\pgfqpoint{1.161048in}{0.528000in}}%
\pgfpathlineto{\pgfqpoint{1.160727in}{0.528338in}}%
\pgfpathlineto{\pgfqpoint{0.840081in}{0.873555in}}%
\pgfpathlineto{\pgfqpoint{0.800000in}{0.917638in}}%
\pgfpathmoveto{\pgfqpoint{4.768000in}{2.589319in}}%
\pgfpathlineto{\pgfqpoint{4.599765in}{2.768000in}}%
\pgfpathlineto{\pgfqpoint{4.274439in}{3.104000in}}%
\pgfpathlineto{\pgfqpoint{4.116391in}{3.262942in}}%
\pgfpathlineto{\pgfqpoint{3.936916in}{3.440000in}}%
\pgfpathlineto{\pgfqpoint{3.625955in}{3.738667in}}%
\pgfpathlineto{\pgfqpoint{3.303942in}{4.037333in}}%
\pgfpathlineto{\pgfqpoint{3.096638in}{4.224000in}}%
\pgfpathlineto{\pgfqpoint{3.092643in}{4.224000in}}%
\pgfpathlineto{\pgfqpoint{3.422171in}{3.925333in}}%
\pgfpathlineto{\pgfqpoint{3.582501in}{3.776000in}}%
\pgfpathlineto{\pgfqpoint{3.740038in}{3.626667in}}%
\pgfpathlineto{\pgfqpoint{3.894897in}{3.477333in}}%
\pgfpathlineto{\pgfqpoint{4.047188in}{3.328000in}}%
\pgfpathlineto{\pgfqpoint{4.380574in}{2.992000in}}%
\pgfpathlineto{\pgfqpoint{4.702003in}{2.656000in}}%
\pgfpathlineto{\pgfqpoint{4.768000in}{2.585518in}}%
\pgfpathlineto{\pgfqpoint{4.768000in}{2.585518in}}%
\pgfusepath{fill}%
\end{pgfscope}%
\begin{pgfscope}%
\pgfpathrectangle{\pgfqpoint{0.800000in}{0.528000in}}{\pgfqpoint{3.968000in}{3.696000in}}%
\pgfusepath{clip}%
\pgfsetbuttcap%
\pgfsetroundjoin%
\definecolor{currentfill}{rgb}{0.160665,0.478540,0.558115}%
\pgfsetfillcolor{currentfill}%
\pgfsetlinewidth{0.000000pt}%
\definecolor{currentstroke}{rgb}{0.000000,0.000000,0.000000}%
\pgfsetstrokecolor{currentstroke}%
\pgfsetdash{}{0pt}%
\pgfpathmoveto{\pgfqpoint{0.800000in}{0.909835in}}%
\pgfpathlineto{\pgfqpoint{0.960323in}{0.734940in}}%
\pgfpathlineto{\pgfqpoint{1.120646in}{0.563217in}}%
\pgfpathlineto{\pgfqpoint{1.153933in}{0.528000in}}%
\pgfpathlineto{\pgfqpoint{1.157488in}{0.528000in}}%
\pgfpathlineto{\pgfqpoint{0.840081in}{0.869657in}}%
\pgfpathlineto{\pgfqpoint{0.800000in}{0.913737in}}%
\pgfpathmoveto{\pgfqpoint{4.768000in}{2.593120in}}%
\pgfpathlineto{\pgfqpoint{4.603365in}{2.768000in}}%
\pgfpathlineto{\pgfqpoint{4.278093in}{3.104000in}}%
\pgfpathlineto{\pgfqpoint{4.126707in}{3.256318in}}%
\pgfpathlineto{\pgfqpoint{3.940625in}{3.440000in}}%
\pgfpathlineto{\pgfqpoint{3.629769in}{3.738667in}}%
\pgfpathlineto{\pgfqpoint{3.307866in}{4.037333in}}%
\pgfpathlineto{\pgfqpoint{3.100634in}{4.224000in}}%
\pgfpathlineto{\pgfqpoint{3.096638in}{4.224000in}}%
\pgfpathlineto{\pgfqpoint{3.426053in}{3.925333in}}%
\pgfpathlineto{\pgfqpoint{3.586327in}{3.776000in}}%
\pgfpathlineto{\pgfqpoint{3.743811in}{3.626667in}}%
\pgfpathlineto{\pgfqpoint{3.898619in}{3.477333in}}%
\pgfpathlineto{\pgfqpoint{4.050860in}{3.328000in}}%
\pgfpathlineto{\pgfqpoint{4.384191in}{2.992000in}}%
\pgfpathlineto{\pgfqpoint{4.705567in}{2.656000in}}%
\pgfpathlineto{\pgfqpoint{4.768000in}{2.589319in}}%
\pgfpathlineto{\pgfqpoint{4.768000in}{2.589319in}}%
\pgfusepath{fill}%
\end{pgfscope}%
\begin{pgfscope}%
\pgfpathrectangle{\pgfqpoint{0.800000in}{0.528000in}}{\pgfqpoint{3.968000in}{3.696000in}}%
\pgfusepath{clip}%
\pgfsetbuttcap%
\pgfsetroundjoin%
\definecolor{currentfill}{rgb}{0.160665,0.478540,0.558115}%
\pgfsetfillcolor{currentfill}%
\pgfsetlinewidth{0.000000pt}%
\definecolor{currentstroke}{rgb}{0.000000,0.000000,0.000000}%
\pgfsetstrokecolor{currentstroke}%
\pgfsetdash{}{0pt}%
\pgfpathmoveto{\pgfqpoint{0.800000in}{0.905934in}}%
\pgfpathlineto{\pgfqpoint{0.960323in}{0.731110in}}%
\pgfpathlineto{\pgfqpoint{1.120646in}{0.559456in}}%
\pgfpathlineto{\pgfqpoint{1.150378in}{0.528000in}}%
\pgfpathlineto{\pgfqpoint{1.153933in}{0.528000in}}%
\pgfpathlineto{\pgfqpoint{0.840081in}{0.865760in}}%
\pgfpathlineto{\pgfqpoint{0.800000in}{0.909835in}}%
\pgfpathmoveto{\pgfqpoint{4.768000in}{2.596922in}}%
\pgfpathlineto{\pgfqpoint{4.606965in}{2.768000in}}%
\pgfpathlineto{\pgfqpoint{4.281746in}{3.104000in}}%
\pgfpathlineto{\pgfqpoint{4.126707in}{3.259959in}}%
\pgfpathlineto{\pgfqpoint{3.944334in}{3.440000in}}%
\pgfpathlineto{\pgfqpoint{3.633582in}{3.738667in}}%
\pgfpathlineto{\pgfqpoint{3.311789in}{4.037333in}}%
\pgfpathlineto{\pgfqpoint{3.104629in}{4.224000in}}%
\pgfpathlineto{\pgfqpoint{3.100634in}{4.224000in}}%
\pgfpathlineto{\pgfqpoint{3.417459in}{3.936703in}}%
\pgfpathlineto{\pgfqpoint{3.510400in}{3.850667in}}%
\pgfpathlineto{\pgfqpoint{3.825290in}{3.552000in}}%
\pgfpathlineto{\pgfqpoint{3.978750in}{3.402667in}}%
\pgfpathlineto{\pgfqpoint{4.287030in}{3.094935in}}%
\pgfpathlineto{\pgfqpoint{4.460237in}{2.917333in}}%
\pgfpathlineto{\pgfqpoint{4.607677in}{2.763465in}}%
\pgfpathlineto{\pgfqpoint{4.768000in}{2.593120in}}%
\pgfpathlineto{\pgfqpoint{4.768000in}{2.593120in}}%
\pgfusepath{fill}%
\end{pgfscope}%
\begin{pgfscope}%
\pgfpathrectangle{\pgfqpoint{0.800000in}{0.528000in}}{\pgfqpoint{3.968000in}{3.696000in}}%
\pgfusepath{clip}%
\pgfsetbuttcap%
\pgfsetroundjoin%
\definecolor{currentfill}{rgb}{0.159194,0.482237,0.558073}%
\pgfsetfillcolor{currentfill}%
\pgfsetlinewidth{0.000000pt}%
\definecolor{currentstroke}{rgb}{0.000000,0.000000,0.000000}%
\pgfsetstrokecolor{currentstroke}%
\pgfsetdash{}{0pt}%
\pgfpathmoveto{\pgfqpoint{0.800000in}{0.902033in}}%
\pgfpathlineto{\pgfqpoint{0.948125in}{0.740638in}}%
\pgfpathlineto{\pgfqpoint{1.041460in}{0.640000in}}%
\pgfpathlineto{\pgfqpoint{1.146823in}{0.528000in}}%
\pgfpathlineto{\pgfqpoint{1.150378in}{0.528000in}}%
\pgfpathlineto{\pgfqpoint{0.827356in}{0.875853in}}%
\pgfpathlineto{\pgfqpoint{0.800000in}{0.905934in}}%
\pgfpathmoveto{\pgfqpoint{4.768000in}{2.600723in}}%
\pgfpathlineto{\pgfqpoint{4.607677in}{2.770995in}}%
\pgfpathlineto{\pgfqpoint{4.285400in}{3.104000in}}%
\pgfpathlineto{\pgfqpoint{4.126707in}{3.263600in}}%
\pgfpathlineto{\pgfqpoint{3.948043in}{3.440000in}}%
\pgfpathlineto{\pgfqpoint{3.637395in}{3.738667in}}%
\pgfpathlineto{\pgfqpoint{3.315713in}{4.037333in}}%
\pgfpathlineto{\pgfqpoint{3.108625in}{4.224000in}}%
\pgfpathlineto{\pgfqpoint{3.104629in}{4.224000in}}%
\pgfpathlineto{\pgfqpoint{3.419379in}{3.938492in}}%
\pgfpathlineto{\pgfqpoint{3.514254in}{3.850667in}}%
\pgfpathlineto{\pgfqpoint{3.829038in}{3.552000in}}%
\pgfpathlineto{\pgfqpoint{3.982446in}{3.402667in}}%
\pgfpathlineto{\pgfqpoint{4.287030in}{3.098641in}}%
\pgfpathlineto{\pgfqpoint{4.463830in}{2.917333in}}%
\pgfpathlineto{\pgfqpoint{4.607677in}{2.767252in}}%
\pgfpathlineto{\pgfqpoint{4.768000in}{2.596922in}}%
\pgfpathlineto{\pgfqpoint{4.768000in}{2.596922in}}%
\pgfusepath{fill}%
\end{pgfscope}%
\begin{pgfscope}%
\pgfpathrectangle{\pgfqpoint{0.800000in}{0.528000in}}{\pgfqpoint{3.968000in}{3.696000in}}%
\pgfusepath{clip}%
\pgfsetbuttcap%
\pgfsetroundjoin%
\definecolor{currentfill}{rgb}{0.159194,0.482237,0.558073}%
\pgfsetfillcolor{currentfill}%
\pgfsetlinewidth{0.000000pt}%
\definecolor{currentstroke}{rgb}{0.000000,0.000000,0.000000}%
\pgfsetstrokecolor{currentstroke}%
\pgfsetdash{}{0pt}%
\pgfpathmoveto{\pgfqpoint{0.800000in}{0.898178in}}%
\pgfpathlineto{\pgfqpoint{0.968465in}{0.714667in}}%
\pgfpathlineto{\pgfqpoint{1.143268in}{0.528000in}}%
\pgfpathlineto{\pgfqpoint{1.146823in}{0.528000in}}%
\pgfpathlineto{\pgfqpoint{0.834655in}{0.864000in}}%
\pgfpathlineto{\pgfqpoint{0.800000in}{0.902033in}}%
\pgfpathlineto{\pgfqpoint{0.800000in}{0.901333in}}%
\pgfpathmoveto{\pgfqpoint{4.768000in}{2.604524in}}%
\pgfpathlineto{\pgfqpoint{4.607677in}{2.774729in}}%
\pgfpathlineto{\pgfqpoint{4.287030in}{3.106024in}}%
\pgfpathlineto{\pgfqpoint{4.103167in}{3.290667in}}%
\pgfpathlineto{\pgfqpoint{3.951752in}{3.440000in}}%
\pgfpathlineto{\pgfqpoint{3.641208in}{3.738667in}}%
\pgfpathlineto{\pgfqpoint{3.319636in}{4.037333in}}%
\pgfpathlineto{\pgfqpoint{3.112621in}{4.224000in}}%
\pgfpathlineto{\pgfqpoint{3.108625in}{4.224000in}}%
\pgfpathlineto{\pgfqpoint{3.405253in}{3.955279in}}%
\pgfpathlineto{\pgfqpoint{3.725899in}{3.654620in}}%
\pgfpathlineto{\pgfqpoint{3.909784in}{3.477333in}}%
\pgfpathlineto{\pgfqpoint{4.061874in}{3.328000in}}%
\pgfpathlineto{\pgfqpoint{4.367192in}{3.020585in}}%
\pgfpathlineto{\pgfqpoint{4.539246in}{2.842667in}}%
\pgfpathlineto{\pgfqpoint{4.768000in}{2.600723in}}%
\pgfpathlineto{\pgfqpoint{4.768000in}{2.600723in}}%
\pgfusepath{fill}%
\end{pgfscope}%
\begin{pgfscope}%
\pgfpathrectangle{\pgfqpoint{0.800000in}{0.528000in}}{\pgfqpoint{3.968000in}{3.696000in}}%
\pgfusepath{clip}%
\pgfsetbuttcap%
\pgfsetroundjoin%
\definecolor{currentfill}{rgb}{0.159194,0.482237,0.558073}%
\pgfsetfillcolor{currentfill}%
\pgfsetlinewidth{0.000000pt}%
\definecolor{currentstroke}{rgb}{0.000000,0.000000,0.000000}%
\pgfsetstrokecolor{currentstroke}%
\pgfsetdash{}{0pt}%
\pgfpathmoveto{\pgfqpoint{0.800000in}{0.894333in}}%
\pgfpathlineto{\pgfqpoint{0.964915in}{0.714667in}}%
\pgfpathlineto{\pgfqpoint{1.139713in}{0.528000in}}%
\pgfpathlineto{\pgfqpoint{1.143268in}{0.528000in}}%
\pgfpathlineto{\pgfqpoint{0.831150in}{0.864000in}}%
\pgfpathlineto{\pgfqpoint{0.800000in}{0.898178in}}%
\pgfpathmoveto{\pgfqpoint{4.768000in}{2.608325in}}%
\pgfpathlineto{\pgfqpoint{4.607677in}{2.778462in}}%
\pgfpathlineto{\pgfqpoint{4.287030in}{3.109678in}}%
\pgfpathlineto{\pgfqpoint{4.106826in}{3.290667in}}%
\pgfpathlineto{\pgfqpoint{3.955461in}{3.440000in}}%
\pgfpathlineto{\pgfqpoint{3.625343in}{3.757004in}}%
\pgfpathlineto{\pgfqpoint{3.525495in}{3.850961in}}%
\pgfpathlineto{\pgfqpoint{3.199972in}{4.149333in}}%
\pgfpathlineto{\pgfqpoint{3.116616in}{4.224000in}}%
\pgfpathlineto{\pgfqpoint{3.112621in}{4.224000in}}%
\pgfpathlineto{\pgfqpoint{3.405253in}{3.958861in}}%
\pgfpathlineto{\pgfqpoint{3.725899in}{3.658228in}}%
\pgfpathlineto{\pgfqpoint{3.913506in}{3.477333in}}%
\pgfpathlineto{\pgfqpoint{4.065545in}{3.328000in}}%
\pgfpathlineto{\pgfqpoint{4.367192in}{3.024297in}}%
\pgfpathlineto{\pgfqpoint{4.542816in}{2.842667in}}%
\pgfpathlineto{\pgfqpoint{4.768000in}{2.604524in}}%
\pgfpathlineto{\pgfqpoint{4.768000in}{2.604524in}}%
\pgfusepath{fill}%
\end{pgfscope}%
\begin{pgfscope}%
\pgfpathrectangle{\pgfqpoint{0.800000in}{0.528000in}}{\pgfqpoint{3.968000in}{3.696000in}}%
\pgfusepath{clip}%
\pgfsetbuttcap%
\pgfsetroundjoin%
\definecolor{currentfill}{rgb}{0.159194,0.482237,0.558073}%
\pgfsetfillcolor{currentfill}%
\pgfsetlinewidth{0.000000pt}%
\definecolor{currentstroke}{rgb}{0.000000,0.000000,0.000000}%
\pgfsetstrokecolor{currentstroke}%
\pgfsetdash{}{0pt}%
\pgfpathmoveto{\pgfqpoint{0.800000in}{0.890488in}}%
\pgfpathlineto{\pgfqpoint{0.961365in}{0.714667in}}%
\pgfpathlineto{\pgfqpoint{1.136158in}{0.528000in}}%
\pgfpathlineto{\pgfqpoint{1.139713in}{0.528000in}}%
\pgfpathlineto{\pgfqpoint{0.827646in}{0.864000in}}%
\pgfpathlineto{\pgfqpoint{0.800000in}{0.894333in}}%
\pgfpathmoveto{\pgfqpoint{4.768000in}{2.612126in}}%
\pgfpathlineto{\pgfqpoint{4.607677in}{2.782196in}}%
\pgfpathlineto{\pgfqpoint{4.287030in}{3.113333in}}%
\pgfpathlineto{\pgfqpoint{4.110485in}{3.290667in}}%
\pgfpathlineto{\pgfqpoint{3.959170in}{3.440000in}}%
\pgfpathlineto{\pgfqpoint{3.627251in}{3.758780in}}%
\pgfpathlineto{\pgfqpoint{3.525495in}{3.854505in}}%
\pgfpathlineto{\pgfqpoint{3.203939in}{4.149333in}}%
\pgfpathlineto{\pgfqpoint{3.120612in}{4.224000in}}%
\pgfpathlineto{\pgfqpoint{3.116616in}{4.224000in}}%
\pgfpathlineto{\pgfqpoint{3.405253in}{3.962444in}}%
\pgfpathlineto{\pgfqpoint{3.725899in}{3.661836in}}%
\pgfpathlineto{\pgfqpoint{3.901965in}{3.491997in}}%
\pgfpathlineto{\pgfqpoint{3.993536in}{3.402667in}}%
\pgfpathlineto{\pgfqpoint{4.292622in}{3.104000in}}%
\pgfpathlineto{\pgfqpoint{4.447354in}{2.945547in}}%
\pgfpathlineto{\pgfqpoint{4.617615in}{2.768000in}}%
\pgfpathlineto{\pgfqpoint{4.768000in}{2.608325in}}%
\pgfpathlineto{\pgfqpoint{4.768000in}{2.608325in}}%
\pgfusepath{fill}%
\end{pgfscope}%
\begin{pgfscope}%
\pgfpathrectangle{\pgfqpoint{0.800000in}{0.528000in}}{\pgfqpoint{3.968000in}{3.696000in}}%
\pgfusepath{clip}%
\pgfsetbuttcap%
\pgfsetroundjoin%
\definecolor{currentfill}{rgb}{0.157729,0.485932,0.558013}%
\pgfsetfillcolor{currentfill}%
\pgfsetlinewidth{0.000000pt}%
\definecolor{currentstroke}{rgb}{0.000000,0.000000,0.000000}%
\pgfsetstrokecolor{currentstroke}%
\pgfsetdash{}{0pt}%
\pgfpathmoveto{\pgfqpoint{0.800000in}{0.886644in}}%
\pgfpathlineto{\pgfqpoint{0.960323in}{0.711999in}}%
\pgfpathlineto{\pgfqpoint{1.132603in}{0.528000in}}%
\pgfpathlineto{\pgfqpoint{1.136158in}{0.528000in}}%
\pgfpathlineto{\pgfqpoint{0.824142in}{0.864000in}}%
\pgfpathlineto{\pgfqpoint{0.800000in}{0.890488in}}%
\pgfpathmoveto{\pgfqpoint{4.768000in}{2.615927in}}%
\pgfpathlineto{\pgfqpoint{4.607677in}{2.785929in}}%
\pgfpathlineto{\pgfqpoint{4.287030in}{3.116987in}}%
\pgfpathlineto{\pgfqpoint{4.114144in}{3.290667in}}%
\pgfpathlineto{\pgfqpoint{3.962879in}{3.440000in}}%
\pgfpathlineto{\pgfqpoint{3.629158in}{3.760557in}}%
\pgfpathlineto{\pgfqpoint{3.533395in}{3.850667in}}%
\pgfpathlineto{\pgfqpoint{3.372083in}{4.000000in}}%
\pgfpathlineto{\pgfqpoint{3.206318in}{4.150702in}}%
\pgfpathlineto{\pgfqpoint{3.124607in}{4.224000in}}%
\pgfpathlineto{\pgfqpoint{3.120612in}{4.224000in}}%
\pgfpathlineto{\pgfqpoint{3.408847in}{3.962667in}}%
\pgfpathlineto{\pgfqpoint{3.727391in}{3.664000in}}%
\pgfpathlineto{\pgfqpoint{3.886222in}{3.511124in}}%
\pgfpathlineto{\pgfqpoint{4.222370in}{3.178667in}}%
\pgfpathlineto{\pgfqpoint{4.392364in}{3.005887in}}%
\pgfpathlineto{\pgfqpoint{4.549955in}{2.842667in}}%
\pgfpathlineto{\pgfqpoint{4.768000in}{2.612126in}}%
\pgfpathlineto{\pgfqpoint{4.768000in}{2.612126in}}%
\pgfusepath{fill}%
\end{pgfscope}%
\begin{pgfscope}%
\pgfpathrectangle{\pgfqpoint{0.800000in}{0.528000in}}{\pgfqpoint{3.968000in}{3.696000in}}%
\pgfusepath{clip}%
\pgfsetbuttcap%
\pgfsetroundjoin%
\definecolor{currentfill}{rgb}{0.157729,0.485932,0.558013}%
\pgfsetfillcolor{currentfill}%
\pgfsetlinewidth{0.000000pt}%
\definecolor{currentstroke}{rgb}{0.000000,0.000000,0.000000}%
\pgfsetstrokecolor{currentstroke}%
\pgfsetdash{}{0pt}%
\pgfpathmoveto{\pgfqpoint{0.800000in}{0.882799in}}%
\pgfpathlineto{\pgfqpoint{0.960323in}{0.708224in}}%
\pgfpathlineto{\pgfqpoint{1.129048in}{0.528000in}}%
\pgfpathlineto{\pgfqpoint{1.132603in}{0.528000in}}%
\pgfpathlineto{\pgfqpoint{0.820638in}{0.864000in}}%
\pgfpathlineto{\pgfqpoint{0.800000in}{0.886644in}}%
\pgfpathmoveto{\pgfqpoint{4.768000in}{2.619714in}}%
\pgfpathlineto{\pgfqpoint{4.592742in}{2.805333in}}%
\pgfpathlineto{\pgfqpoint{4.447354in}{2.956677in}}%
\pgfpathlineto{\pgfqpoint{4.266589in}{3.141333in}}%
\pgfpathlineto{\pgfqpoint{3.946801in}{3.459093in}}%
\pgfpathlineto{\pgfqpoint{3.846141in}{3.557125in}}%
\pgfpathlineto{\pgfqpoint{3.656292in}{3.738667in}}%
\pgfpathlineto{\pgfqpoint{3.491237in}{3.893424in}}%
\pgfpathlineto{\pgfqpoint{3.405253in}{3.973051in}}%
\pgfpathlineto{\pgfqpoint{3.244929in}{4.119384in}}%
\pgfpathlineto{\pgfqpoint{3.128538in}{4.224000in}}%
\pgfpathlineto{\pgfqpoint{3.124607in}{4.224000in}}%
\pgfpathlineto{\pgfqpoint{3.124687in}{4.223929in}}%
\pgfpathlineto{\pgfqpoint{3.453095in}{3.925333in}}%
\pgfpathlineto{\pgfqpoint{3.612996in}{3.776000in}}%
\pgfpathlineto{\pgfqpoint{3.944914in}{3.457335in}}%
\pgfpathlineto{\pgfqpoint{4.038821in}{3.365333in}}%
\pgfpathlineto{\pgfqpoint{4.188859in}{3.216000in}}%
\pgfpathlineto{\pgfqpoint{4.503314in}{2.894791in}}%
\pgfpathlineto{\pgfqpoint{4.589184in}{2.805333in}}%
\pgfpathlineto{\pgfqpoint{4.768000in}{2.615927in}}%
\pgfpathlineto{\pgfqpoint{4.768000in}{2.618667in}}%
\pgfpathlineto{\pgfqpoint{4.768000in}{2.618667in}}%
\pgfusepath{fill}%
\end{pgfscope}%
\begin{pgfscope}%
\pgfpathrectangle{\pgfqpoint{0.800000in}{0.528000in}}{\pgfqpoint{3.968000in}{3.696000in}}%
\pgfusepath{clip}%
\pgfsetbuttcap%
\pgfsetroundjoin%
\definecolor{currentfill}{rgb}{0.157729,0.485932,0.558013}%
\pgfsetfillcolor{currentfill}%
\pgfsetlinewidth{0.000000pt}%
\definecolor{currentstroke}{rgb}{0.000000,0.000000,0.000000}%
\pgfsetstrokecolor{currentstroke}%
\pgfsetdash{}{0pt}%
\pgfpathmoveto{\pgfqpoint{0.800000in}{0.878954in}}%
\pgfpathlineto{\pgfqpoint{0.960323in}{0.704449in}}%
\pgfpathlineto{\pgfqpoint{1.125493in}{0.528000in}}%
\pgfpathlineto{\pgfqpoint{1.129048in}{0.528000in}}%
\pgfpathlineto{\pgfqpoint{0.817134in}{0.864000in}}%
\pgfpathlineto{\pgfqpoint{0.800000in}{0.882799in}}%
\pgfpathmoveto{\pgfqpoint{4.768000in}{2.623461in}}%
\pgfpathlineto{\pgfqpoint{4.596300in}{2.805333in}}%
\pgfpathlineto{\pgfqpoint{4.447354in}{2.960345in}}%
\pgfpathlineto{\pgfqpoint{4.270199in}{3.141333in}}%
\pgfpathlineto{\pgfqpoint{3.948688in}{3.460851in}}%
\pgfpathlineto{\pgfqpoint{3.850692in}{3.556238in}}%
\pgfpathlineto{\pgfqpoint{3.765980in}{3.637801in}}%
\pgfpathlineto{\pgfqpoint{3.580840in}{3.813333in}}%
\pgfpathlineto{\pgfqpoint{3.285010in}{4.086569in}}%
\pgfpathlineto{\pgfqpoint{3.132468in}{4.224000in}}%
\pgfpathlineto{\pgfqpoint{3.128538in}{4.224000in}}%
\pgfpathlineto{\pgfqpoint{3.456914in}{3.925333in}}%
\pgfpathlineto{\pgfqpoint{3.616762in}{3.776000in}}%
\pgfpathlineto{\pgfqpoint{3.928361in}{3.477333in}}%
\pgfpathlineto{\pgfqpoint{4.266589in}{3.141333in}}%
\pgfpathlineto{\pgfqpoint{4.413041in}{2.992000in}}%
\pgfpathlineto{\pgfqpoint{4.567596in}{2.831694in}}%
\pgfpathlineto{\pgfqpoint{4.733993in}{2.656000in}}%
\pgfpathlineto{\pgfqpoint{4.768000in}{2.619714in}}%
\pgfpathlineto{\pgfqpoint{4.768000in}{2.619714in}}%
\pgfusepath{fill}%
\end{pgfscope}%
\begin{pgfscope}%
\pgfpathrectangle{\pgfqpoint{0.800000in}{0.528000in}}{\pgfqpoint{3.968000in}{3.696000in}}%
\pgfusepath{clip}%
\pgfsetbuttcap%
\pgfsetroundjoin%
\definecolor{currentfill}{rgb}{0.156270,0.489624,0.557936}%
\pgfsetfillcolor{currentfill}%
\pgfsetlinewidth{0.000000pt}%
\definecolor{currentstroke}{rgb}{0.000000,0.000000,0.000000}%
\pgfsetstrokecolor{currentstroke}%
\pgfsetdash{}{0pt}%
\pgfpathmoveto{\pgfqpoint{0.800000in}{0.875110in}}%
\pgfpathlineto{\pgfqpoint{0.960323in}{0.700673in}}%
\pgfpathlineto{\pgfqpoint{1.121938in}{0.528000in}}%
\pgfpathlineto{\pgfqpoint{1.125493in}{0.528000in}}%
\pgfpathlineto{\pgfqpoint{0.813630in}{0.864000in}}%
\pgfpathlineto{\pgfqpoint{0.800000in}{0.878954in}}%
\pgfpathmoveto{\pgfqpoint{4.768000in}{2.627208in}}%
\pgfpathlineto{\pgfqpoint{4.599858in}{2.805333in}}%
\pgfpathlineto{\pgfqpoint{4.447354in}{2.964013in}}%
\pgfpathlineto{\pgfqpoint{4.273809in}{3.141333in}}%
\pgfpathlineto{\pgfqpoint{3.950575in}{3.462609in}}%
\pgfpathlineto{\pgfqpoint{3.858817in}{3.552000in}}%
\pgfpathlineto{\pgfqpoint{3.544772in}{3.850667in}}%
\pgfpathlineto{\pgfqpoint{3.383621in}{4.000000in}}%
\pgfpathlineto{\pgfqpoint{3.219559in}{4.149333in}}%
\pgfpathlineto{\pgfqpoint{3.136397in}{4.224000in}}%
\pgfpathlineto{\pgfqpoint{3.132468in}{4.224000in}}%
\pgfpathlineto{\pgfqpoint{3.460733in}{3.925333in}}%
\pgfpathlineto{\pgfqpoint{3.620528in}{3.776000in}}%
\pgfpathlineto{\pgfqpoint{3.932025in}{3.477333in}}%
\pgfpathlineto{\pgfqpoint{4.270199in}{3.141333in}}%
\pgfpathlineto{\pgfqpoint{4.416604in}{2.992000in}}%
\pgfpathlineto{\pgfqpoint{4.567596in}{2.835424in}}%
\pgfpathlineto{\pgfqpoint{4.737505in}{2.656000in}}%
\pgfpathlineto{\pgfqpoint{4.768000in}{2.623461in}}%
\pgfpathlineto{\pgfqpoint{4.768000in}{2.623461in}}%
\pgfusepath{fill}%
\end{pgfscope}%
\begin{pgfscope}%
\pgfpathrectangle{\pgfqpoint{0.800000in}{0.528000in}}{\pgfqpoint{3.968000in}{3.696000in}}%
\pgfusepath{clip}%
\pgfsetbuttcap%
\pgfsetroundjoin%
\definecolor{currentfill}{rgb}{0.156270,0.489624,0.557936}%
\pgfsetfillcolor{currentfill}%
\pgfsetlinewidth{0.000000pt}%
\definecolor{currentstroke}{rgb}{0.000000,0.000000,0.000000}%
\pgfsetstrokecolor{currentstroke}%
\pgfsetdash{}{0pt}%
\pgfpathmoveto{\pgfqpoint{0.800000in}{0.871265in}}%
\pgfpathlineto{\pgfqpoint{0.960323in}{0.696898in}}%
\pgfpathlineto{\pgfqpoint{1.120646in}{0.528000in}}%
\pgfpathlineto{\pgfqpoint{1.121938in}{0.528000in}}%
\pgfpathlineto{\pgfqpoint{1.121338in}{0.528644in}}%
\pgfpathlineto{\pgfqpoint{1.040485in}{0.614651in}}%
\pgfpathlineto{\pgfqpoint{0.800000in}{0.875110in}}%
\pgfpathmoveto{\pgfqpoint{4.768000in}{2.630956in}}%
\pgfpathlineto{\pgfqpoint{4.603416in}{2.805333in}}%
\pgfpathlineto{\pgfqpoint{4.447354in}{2.967681in}}%
\pgfpathlineto{\pgfqpoint{4.277420in}{3.141333in}}%
\pgfpathlineto{\pgfqpoint{3.939354in}{3.477333in}}%
\pgfpathlineto{\pgfqpoint{3.785017in}{3.626667in}}%
\pgfpathlineto{\pgfqpoint{3.616856in}{3.786431in}}%
\pgfpathlineto{\pgfqpoint{3.525495in}{3.872225in}}%
\pgfpathlineto{\pgfqpoint{3.181989in}{4.186667in}}%
\pgfpathlineto{\pgfqpoint{3.140327in}{4.224000in}}%
\pgfpathlineto{\pgfqpoint{3.136397in}{4.224000in}}%
\pgfpathlineto{\pgfqpoint{3.464552in}{3.925333in}}%
\pgfpathlineto{\pgfqpoint{3.624294in}{3.776000in}}%
\pgfpathlineto{\pgfqpoint{3.935689in}{3.477333in}}%
\pgfpathlineto{\pgfqpoint{4.273809in}{3.141333in}}%
\pgfpathlineto{\pgfqpoint{4.420167in}{2.992000in}}%
\pgfpathlineto{\pgfqpoint{4.567596in}{2.839154in}}%
\pgfpathlineto{\pgfqpoint{4.741017in}{2.656000in}}%
\pgfpathlineto{\pgfqpoint{4.768000in}{2.627208in}}%
\pgfpathlineto{\pgfqpoint{4.768000in}{2.627208in}}%
\pgfusepath{fill}%
\end{pgfscope}%
\begin{pgfscope}%
\pgfpathrectangle{\pgfqpoint{0.800000in}{0.528000in}}{\pgfqpoint{3.968000in}{3.696000in}}%
\pgfusepath{clip}%
\pgfsetbuttcap%
\pgfsetroundjoin%
\definecolor{currentfill}{rgb}{0.156270,0.489624,0.557936}%
\pgfsetfillcolor{currentfill}%
\pgfsetlinewidth{0.000000pt}%
\definecolor{currentstroke}{rgb}{0.000000,0.000000,0.000000}%
\pgfsetstrokecolor{currentstroke}%
\pgfsetdash{}{0pt}%
\pgfpathmoveto{\pgfqpoint{0.800000in}{0.867420in}}%
\pgfpathlineto{\pgfqpoint{0.960323in}{0.693123in}}%
\pgfpathlineto{\pgfqpoint{1.114914in}{0.528000in}}%
\pgfpathlineto{\pgfqpoint{1.118417in}{0.528000in}}%
\pgfpathlineto{\pgfqpoint{0.800000in}{0.871265in}}%
\pgfpathmoveto{\pgfqpoint{4.768000in}{2.634703in}}%
\pgfpathlineto{\pgfqpoint{4.601371in}{2.811207in}}%
\pgfpathlineto{\pgfqpoint{4.427293in}{2.992000in}}%
\pgfpathlineto{\pgfqpoint{4.126707in}{3.296293in}}%
\pgfpathlineto{\pgfqpoint{3.943018in}{3.477333in}}%
\pgfpathlineto{\pgfqpoint{3.788731in}{3.626667in}}%
\pgfpathlineto{\pgfqpoint{3.618738in}{3.788185in}}%
\pgfpathlineto{\pgfqpoint{3.525495in}{3.875769in}}%
\pgfpathlineto{\pgfqpoint{3.185905in}{4.186667in}}%
\pgfpathlineto{\pgfqpoint{3.144256in}{4.224000in}}%
\pgfpathlineto{\pgfqpoint{3.140327in}{4.224000in}}%
\pgfpathlineto{\pgfqpoint{3.468371in}{3.925333in}}%
\pgfpathlineto{\pgfqpoint{3.628060in}{3.776000in}}%
\pgfpathlineto{\pgfqpoint{3.939354in}{3.477333in}}%
\pgfpathlineto{\pgfqpoint{4.262713in}{3.156017in}}%
\pgfpathlineto{\pgfqpoint{4.350860in}{3.066667in}}%
\pgfpathlineto{\pgfqpoint{4.674237in}{2.730667in}}%
\pgfpathlineto{\pgfqpoint{4.768000in}{2.630956in}}%
\pgfpathlineto{\pgfqpoint{4.768000in}{2.630956in}}%
\pgfusepath{fill}%
\end{pgfscope}%
\begin{pgfscope}%
\pgfpathrectangle{\pgfqpoint{0.800000in}{0.528000in}}{\pgfqpoint{3.968000in}{3.696000in}}%
\pgfusepath{clip}%
\pgfsetbuttcap%
\pgfsetroundjoin%
\definecolor{currentfill}{rgb}{0.156270,0.489624,0.557936}%
\pgfsetfillcolor{currentfill}%
\pgfsetlinewidth{0.000000pt}%
\definecolor{currentstroke}{rgb}{0.000000,0.000000,0.000000}%
\pgfsetstrokecolor{currentstroke}%
\pgfsetdash{}{0pt}%
\pgfpathmoveto{\pgfqpoint{0.800000in}{0.863582in}}%
\pgfpathlineto{\pgfqpoint{1.111412in}{0.528000in}}%
\pgfpathlineto{\pgfqpoint{1.114914in}{0.528000in}}%
\pgfpathlineto{\pgfqpoint{0.800000in}{0.867420in}}%
\pgfpathlineto{\pgfqpoint{0.800000in}{0.864000in}}%
\pgfpathmoveto{\pgfqpoint{4.768000in}{2.638450in}}%
\pgfpathlineto{\pgfqpoint{4.607677in}{2.808287in}}%
\pgfpathlineto{\pgfqpoint{4.430856in}{2.992000in}}%
\pgfpathlineto{\pgfqpoint{4.098406in}{3.328000in}}%
\pgfpathlineto{\pgfqpoint{3.792446in}{3.626667in}}%
\pgfpathlineto{\pgfqpoint{3.635592in}{3.776000in}}%
\pgfpathlineto{\pgfqpoint{3.313582in}{4.074667in}}%
\pgfpathlineto{\pgfqpoint{3.148186in}{4.224000in}}%
\pgfpathlineto{\pgfqpoint{3.144256in}{4.224000in}}%
\pgfpathlineto{\pgfqpoint{3.445333in}{3.950212in}}%
\pgfpathlineto{\pgfqpoint{3.631826in}{3.776000in}}%
\pgfpathlineto{\pgfqpoint{3.943018in}{3.477333in}}%
\pgfpathlineto{\pgfqpoint{4.264581in}{3.157756in}}%
\pgfpathlineto{\pgfqpoint{4.354447in}{3.066667in}}%
\pgfpathlineto{\pgfqpoint{4.647758in}{2.762396in}}%
\pgfpathlineto{\pgfqpoint{4.768000in}{2.634703in}}%
\pgfpathlineto{\pgfqpoint{4.768000in}{2.634703in}}%
\pgfusepath{fill}%
\end{pgfscope}%
\begin{pgfscope}%
\pgfpathrectangle{\pgfqpoint{0.800000in}{0.528000in}}{\pgfqpoint{3.968000in}{3.696000in}}%
\pgfusepath{clip}%
\pgfsetbuttcap%
\pgfsetroundjoin%
\definecolor{currentfill}{rgb}{0.154815,0.493313,0.557840}%
\pgfsetfillcolor{currentfill}%
\pgfsetlinewidth{0.000000pt}%
\definecolor{currentstroke}{rgb}{0.000000,0.000000,0.000000}%
\pgfsetstrokecolor{currentstroke}%
\pgfsetdash{}{0pt}%
\pgfpathmoveto{\pgfqpoint{0.800000in}{0.859792in}}%
\pgfpathlineto{\pgfqpoint{1.107909in}{0.528000in}}%
\pgfpathlineto{\pgfqpoint{1.111412in}{0.528000in}}%
\pgfpathlineto{\pgfqpoint{0.800000in}{0.863582in}}%
\pgfpathmoveto{\pgfqpoint{4.768000in}{2.642198in}}%
\pgfpathlineto{\pgfqpoint{4.607677in}{2.811969in}}%
\pgfpathlineto{\pgfqpoint{4.434419in}{2.992000in}}%
\pgfpathlineto{\pgfqpoint{4.102021in}{3.328000in}}%
\pgfpathlineto{\pgfqpoint{3.796160in}{3.626667in}}%
\pgfpathlineto{\pgfqpoint{3.639358in}{3.776000in}}%
\pgfpathlineto{\pgfqpoint{3.317455in}{4.074667in}}%
\pgfpathlineto{\pgfqpoint{3.152115in}{4.224000in}}%
\pgfpathlineto{\pgfqpoint{3.148186in}{4.224000in}}%
\pgfpathlineto{\pgfqpoint{3.445333in}{3.953750in}}%
\pgfpathlineto{\pgfqpoint{3.635592in}{3.776000in}}%
\pgfpathlineto{\pgfqpoint{3.946682in}{3.477333in}}%
\pgfpathlineto{\pgfqpoint{4.266449in}{3.159496in}}%
\pgfpathlineto{\pgfqpoint{4.362373in}{3.062178in}}%
\pgfpathlineto{\pgfqpoint{4.447354in}{2.975017in}}%
\pgfpathlineto{\pgfqpoint{4.768000in}{2.638450in}}%
\pgfpathlineto{\pgfqpoint{4.768000in}{2.638450in}}%
\pgfusepath{fill}%
\end{pgfscope}%
\begin{pgfscope}%
\pgfpathrectangle{\pgfqpoint{0.800000in}{0.528000in}}{\pgfqpoint{3.968000in}{3.696000in}}%
\pgfusepath{clip}%
\pgfsetbuttcap%
\pgfsetroundjoin%
\definecolor{currentfill}{rgb}{0.154815,0.493313,0.557840}%
\pgfsetfillcolor{currentfill}%
\pgfsetlinewidth{0.000000pt}%
\definecolor{currentstroke}{rgb}{0.000000,0.000000,0.000000}%
\pgfsetstrokecolor{currentstroke}%
\pgfsetdash{}{0pt}%
\pgfpathmoveto{\pgfqpoint{0.800000in}{0.856002in}}%
\pgfpathlineto{\pgfqpoint{1.104406in}{0.528000in}}%
\pgfpathlineto{\pgfqpoint{1.107909in}{0.528000in}}%
\pgfpathlineto{\pgfqpoint{0.800000in}{0.859792in}}%
\pgfpathmoveto{\pgfqpoint{4.768000in}{2.645945in}}%
\pgfpathlineto{\pgfqpoint{4.607677in}{2.815650in}}%
\pgfpathlineto{\pgfqpoint{4.437982in}{2.992000in}}%
\pgfpathlineto{\pgfqpoint{4.105637in}{3.328000in}}%
\pgfpathlineto{\pgfqpoint{3.799875in}{3.626667in}}%
\pgfpathlineto{\pgfqpoint{3.643124in}{3.776000in}}%
\pgfpathlineto{\pgfqpoint{3.321329in}{4.074667in}}%
\pgfpathlineto{\pgfqpoint{3.156045in}{4.224000in}}%
\pgfpathlineto{\pgfqpoint{3.152115in}{4.224000in}}%
\pgfpathlineto{\pgfqpoint{3.445333in}{3.957288in}}%
\pgfpathlineto{\pgfqpoint{3.622504in}{3.791692in}}%
\pgfpathlineto{\pgfqpoint{3.721929in}{3.697636in}}%
\pgfpathlineto{\pgfqpoint{3.806061in}{3.617162in}}%
\pgfpathlineto{\pgfqpoint{3.988497in}{3.440000in}}%
\pgfpathlineto{\pgfqpoint{4.306718in}{3.122338in}}%
\pgfpathlineto{\pgfqpoint{4.398090in}{3.029333in}}%
\pgfpathlineto{\pgfqpoint{4.542573in}{2.880000in}}%
\pgfpathlineto{\pgfqpoint{4.687838in}{2.727495in}}%
\pgfpathlineto{\pgfqpoint{4.768000in}{2.642198in}}%
\pgfpathlineto{\pgfqpoint{4.768000in}{2.642198in}}%
\pgfusepath{fill}%
\end{pgfscope}%
\begin{pgfscope}%
\pgfpathrectangle{\pgfqpoint{0.800000in}{0.528000in}}{\pgfqpoint{3.968000in}{3.696000in}}%
\pgfusepath{clip}%
\pgfsetbuttcap%
\pgfsetroundjoin%
\definecolor{currentfill}{rgb}{0.154815,0.493313,0.557840}%
\pgfsetfillcolor{currentfill}%
\pgfsetlinewidth{0.000000pt}%
\definecolor{currentstroke}{rgb}{0.000000,0.000000,0.000000}%
\pgfsetstrokecolor{currentstroke}%
\pgfsetdash{}{0pt}%
\pgfpathmoveto{\pgfqpoint{0.800000in}{0.852213in}}%
\pgfpathlineto{\pgfqpoint{1.100904in}{0.528000in}}%
\pgfpathlineto{\pgfqpoint{1.104406in}{0.528000in}}%
\pgfpathlineto{\pgfqpoint{0.800000in}{0.856002in}}%
\pgfpathmoveto{\pgfqpoint{4.768000in}{2.649692in}}%
\pgfpathlineto{\pgfqpoint{4.607677in}{2.819332in}}%
\pgfpathlineto{\pgfqpoint{4.441545in}{2.992000in}}%
\pgfpathlineto{\pgfqpoint{4.109252in}{3.328000in}}%
\pgfpathlineto{\pgfqpoint{3.803589in}{3.626667in}}%
\pgfpathlineto{\pgfqpoint{3.645737in}{3.777073in}}%
\pgfpathlineto{\pgfqpoint{3.320348in}{4.079084in}}%
\pgfpathlineto{\pgfqpoint{3.159974in}{4.224000in}}%
\pgfpathlineto{\pgfqpoint{3.156045in}{4.224000in}}%
\pgfpathlineto{\pgfqpoint{3.445333in}{3.960825in}}%
\pgfpathlineto{\pgfqpoint{3.624386in}{3.793446in}}%
\pgfpathlineto{\pgfqpoint{3.725899in}{3.697463in}}%
\pgfpathlineto{\pgfqpoint{3.915716in}{3.514667in}}%
\pgfpathlineto{\pgfqpoint{4.217768in}{3.216000in}}%
\pgfpathlineto{\pgfqpoint{4.546101in}{2.880000in}}%
\pgfpathlineto{\pgfqpoint{4.692210in}{2.726595in}}%
\pgfpathlineto{\pgfqpoint{4.768000in}{2.645945in}}%
\pgfpathlineto{\pgfqpoint{4.768000in}{2.645945in}}%
\pgfusepath{fill}%
\end{pgfscope}%
\begin{pgfscope}%
\pgfpathrectangle{\pgfqpoint{0.800000in}{0.528000in}}{\pgfqpoint{3.968000in}{3.696000in}}%
\pgfusepath{clip}%
\pgfsetbuttcap%
\pgfsetroundjoin%
\definecolor{currentfill}{rgb}{0.154815,0.493313,0.557840}%
\pgfsetfillcolor{currentfill}%
\pgfsetlinewidth{0.000000pt}%
\definecolor{currentstroke}{rgb}{0.000000,0.000000,0.000000}%
\pgfsetstrokecolor{currentstroke}%
\pgfsetdash{}{0pt}%
\pgfpathmoveto{\pgfqpoint{0.800000in}{0.848423in}}%
\pgfpathlineto{\pgfqpoint{1.097401in}{0.528000in}}%
\pgfpathlineto{\pgfqpoint{1.100904in}{0.528000in}}%
\pgfpathlineto{\pgfqpoint{0.800000in}{0.852213in}}%
\pgfpathmoveto{\pgfqpoint{4.768000in}{2.653440in}}%
\pgfpathlineto{\pgfqpoint{4.607677in}{2.823013in}}%
\pgfpathlineto{\pgfqpoint{4.445108in}{2.992000in}}%
\pgfpathlineto{\pgfqpoint{4.112868in}{3.328000in}}%
\pgfpathlineto{\pgfqpoint{3.806061in}{3.627844in}}%
\pgfpathlineto{\pgfqpoint{3.628151in}{3.796953in}}%
\pgfpathlineto{\pgfqpoint{3.525495in}{3.893417in}}%
\pgfpathlineto{\pgfqpoint{3.187861in}{4.202489in}}%
\pgfpathlineto{\pgfqpoint{3.163904in}{4.224000in}}%
\pgfpathlineto{\pgfqpoint{3.159974in}{4.224000in}}%
\pgfpathlineto{\pgfqpoint{3.447142in}{3.962667in}}%
\pgfpathlineto{\pgfqpoint{3.607268in}{3.813333in}}%
\pgfpathlineto{\pgfqpoint{3.926303in}{3.507947in}}%
\pgfpathlineto{\pgfqpoint{4.109252in}{3.328000in}}%
\pgfpathlineto{\pgfqpoint{4.272053in}{3.164716in}}%
\pgfpathlineto{\pgfqpoint{4.368769in}{3.066667in}}%
\pgfpathlineto{\pgfqpoint{4.691851in}{2.730667in}}%
\pgfpathlineto{\pgfqpoint{4.768000in}{2.649692in}}%
\pgfpathlineto{\pgfqpoint{4.768000in}{2.649692in}}%
\pgfusepath{fill}%
\end{pgfscope}%
\begin{pgfscope}%
\pgfpathrectangle{\pgfqpoint{0.800000in}{0.528000in}}{\pgfqpoint{3.968000in}{3.696000in}}%
\pgfusepath{clip}%
\pgfsetbuttcap%
\pgfsetroundjoin%
\definecolor{currentfill}{rgb}{0.153364,0.497000,0.557724}%
\pgfsetfillcolor{currentfill}%
\pgfsetlinewidth{0.000000pt}%
\definecolor{currentstroke}{rgb}{0.000000,0.000000,0.000000}%
\pgfsetstrokecolor{currentstroke}%
\pgfsetdash{}{0pt}%
\pgfpathmoveto{\pgfqpoint{0.800000in}{0.844633in}}%
\pgfpathlineto{\pgfqpoint{1.093899in}{0.528000in}}%
\pgfpathlineto{\pgfqpoint{1.097401in}{0.528000in}}%
\pgfpathlineto{\pgfqpoint{0.800000in}{0.848423in}}%
\pgfpathmoveto{\pgfqpoint{4.768000in}{2.657171in}}%
\pgfpathlineto{\pgfqpoint{4.447354in}{2.993338in}}%
\pgfpathlineto{\pgfqpoint{4.116484in}{3.328000in}}%
\pgfpathlineto{\pgfqpoint{3.806061in}{3.631363in}}%
\pgfpathlineto{\pgfqpoint{3.614708in}{3.813333in}}%
\pgfpathlineto{\pgfqpoint{3.445333in}{3.971323in}}%
\pgfpathlineto{\pgfqpoint{3.267971in}{4.133462in}}%
\pgfpathlineto{\pgfqpoint{3.167784in}{4.224000in}}%
\pgfpathlineto{\pgfqpoint{3.163904in}{4.224000in}}%
\pgfpathlineto{\pgfqpoint{3.164768in}{4.223227in}}%
\pgfpathlineto{\pgfqpoint{3.329012in}{4.074667in}}%
\pgfpathlineto{\pgfqpoint{3.491192in}{3.925333in}}%
\pgfpathlineto{\pgfqpoint{3.807285in}{3.626667in}}%
\pgfpathlineto{\pgfqpoint{3.983235in}{3.455696in}}%
\pgfpathlineto{\pgfqpoint{4.075216in}{3.365333in}}%
\pgfpathlineto{\pgfqpoint{4.235445in}{3.205284in}}%
\pgfpathlineto{\pgfqpoint{4.327111in}{3.112715in}}%
\pgfpathlineto{\pgfqpoint{4.487434in}{2.948286in}}%
\pgfpathlineto{\pgfqpoint{4.768000in}{2.653440in}}%
\pgfpathlineto{\pgfqpoint{4.768000in}{2.656000in}}%
\pgfpathlineto{\pgfqpoint{4.768000in}{2.656000in}}%
\pgfusepath{fill}%
\end{pgfscope}%
\begin{pgfscope}%
\pgfpathrectangle{\pgfqpoint{0.800000in}{0.528000in}}{\pgfqpoint{3.968000in}{3.696000in}}%
\pgfusepath{clip}%
\pgfsetbuttcap%
\pgfsetroundjoin%
\definecolor{currentfill}{rgb}{0.153364,0.497000,0.557724}%
\pgfsetfillcolor{currentfill}%
\pgfsetlinewidth{0.000000pt}%
\definecolor{currentstroke}{rgb}{0.000000,0.000000,0.000000}%
\pgfsetstrokecolor{currentstroke}%
\pgfsetdash{}{0pt}%
\pgfpathmoveto{\pgfqpoint{0.800000in}{0.840844in}}%
\pgfpathlineto{\pgfqpoint{1.090396in}{0.528000in}}%
\pgfpathlineto{\pgfqpoint{1.093899in}{0.528000in}}%
\pgfpathlineto{\pgfqpoint{0.800000in}{0.844633in}}%
\pgfpathmoveto{\pgfqpoint{4.768000in}{2.660866in}}%
\pgfpathlineto{\pgfqpoint{4.447354in}{2.996955in}}%
\pgfpathlineto{\pgfqpoint{4.120099in}{3.328000in}}%
\pgfpathlineto{\pgfqpoint{3.806061in}{3.634882in}}%
\pgfpathlineto{\pgfqpoint{3.618428in}{3.813333in}}%
\pgfpathlineto{\pgfqpoint{3.451834in}{3.968722in}}%
\pgfpathlineto{\pgfqpoint{3.365172in}{4.048573in}}%
\pgfpathlineto{\pgfqpoint{3.204848in}{4.194174in}}%
\pgfpathlineto{\pgfqpoint{3.171649in}{4.224000in}}%
\pgfpathlineto{\pgfqpoint{3.167784in}{4.224000in}}%
\pgfpathlineto{\pgfqpoint{3.332823in}{4.074667in}}%
\pgfpathlineto{\pgfqpoint{3.494951in}{3.925333in}}%
\pgfpathlineto{\pgfqpoint{3.810942in}{3.626667in}}%
\pgfpathlineto{\pgfqpoint{3.985095in}{3.457429in}}%
\pgfpathlineto{\pgfqpoint{4.086626in}{3.357634in}}%
\pgfpathlineto{\pgfqpoint{4.412313in}{3.029333in}}%
\pgfpathlineto{\pgfqpoint{4.734021in}{2.693333in}}%
\pgfpathlineto{\pgfqpoint{4.768000in}{2.657171in}}%
\pgfpathlineto{\pgfqpoint{4.768000in}{2.657171in}}%
\pgfusepath{fill}%
\end{pgfscope}%
\begin{pgfscope}%
\pgfpathrectangle{\pgfqpoint{0.800000in}{0.528000in}}{\pgfqpoint{3.968000in}{3.696000in}}%
\pgfusepath{clip}%
\pgfsetbuttcap%
\pgfsetroundjoin%
\definecolor{currentfill}{rgb}{0.153364,0.497000,0.557724}%
\pgfsetfillcolor{currentfill}%
\pgfsetlinewidth{0.000000pt}%
\definecolor{currentstroke}{rgb}{0.000000,0.000000,0.000000}%
\pgfsetstrokecolor{currentstroke}%
\pgfsetdash{}{0pt}%
\pgfpathmoveto{\pgfqpoint{0.800000in}{0.837054in}}%
\pgfpathlineto{\pgfqpoint{1.086894in}{0.528000in}}%
\pgfpathlineto{\pgfqpoint{1.090396in}{0.528000in}}%
\pgfpathlineto{\pgfqpoint{0.800000in}{0.840844in}}%
\pgfpathmoveto{\pgfqpoint{4.768000in}{2.664561in}}%
\pgfpathlineto{\pgfqpoint{4.447354in}{3.000573in}}%
\pgfpathlineto{\pgfqpoint{4.123715in}{3.328000in}}%
\pgfpathlineto{\pgfqpoint{3.779363in}{3.664000in}}%
\pgfpathlineto{\pgfqpoint{3.462229in}{3.962667in}}%
\pgfpathlineto{\pgfqpoint{3.175515in}{4.224000in}}%
\pgfpathlineto{\pgfqpoint{3.171649in}{4.224000in}}%
\pgfpathlineto{\pgfqpoint{3.336634in}{4.074667in}}%
\pgfpathlineto{\pgfqpoint{3.498709in}{3.925333in}}%
\pgfpathlineto{\pgfqpoint{3.814599in}{3.626667in}}%
\pgfpathlineto{\pgfqpoint{3.986956in}{3.459162in}}%
\pgfpathlineto{\pgfqpoint{4.086626in}{3.361223in}}%
\pgfpathlineto{\pgfqpoint{4.415835in}{3.029333in}}%
\pgfpathlineto{\pgfqpoint{4.737493in}{2.693333in}}%
\pgfpathlineto{\pgfqpoint{4.768000in}{2.660866in}}%
\pgfpathlineto{\pgfqpoint{4.768000in}{2.660866in}}%
\pgfusepath{fill}%
\end{pgfscope}%
\begin{pgfscope}%
\pgfpathrectangle{\pgfqpoint{0.800000in}{0.528000in}}{\pgfqpoint{3.968000in}{3.696000in}}%
\pgfusepath{clip}%
\pgfsetbuttcap%
\pgfsetroundjoin%
\definecolor{currentfill}{rgb}{0.151918,0.500685,0.557587}%
\pgfsetfillcolor{currentfill}%
\pgfsetlinewidth{0.000000pt}%
\definecolor{currentstroke}{rgb}{0.000000,0.000000,0.000000}%
\pgfsetstrokecolor{currentstroke}%
\pgfsetdash{}{0pt}%
\pgfpathmoveto{\pgfqpoint{0.800000in}{0.833265in}}%
\pgfpathlineto{\pgfqpoint{1.083391in}{0.528000in}}%
\pgfpathlineto{\pgfqpoint{1.086894in}{0.528000in}}%
\pgfpathlineto{\pgfqpoint{0.800000in}{0.837054in}}%
\pgfpathmoveto{\pgfqpoint{4.768000in}{2.668256in}}%
\pgfpathlineto{\pgfqpoint{4.447354in}{3.004191in}}%
\pgfpathlineto{\pgfqpoint{4.126707in}{3.328611in}}%
\pgfpathlineto{\pgfqpoint{3.783032in}{3.664000in}}%
\pgfpathlineto{\pgfqpoint{3.466000in}{3.962667in}}%
\pgfpathlineto{\pgfqpoint{3.179380in}{4.224000in}}%
\pgfpathlineto{\pgfqpoint{3.175515in}{4.224000in}}%
\pgfpathlineto{\pgfqpoint{3.340446in}{4.074667in}}%
\pgfpathlineto{\pgfqpoint{3.513916in}{3.914548in}}%
\pgfpathlineto{\pgfqpoint{3.605657in}{3.828864in}}%
\pgfpathlineto{\pgfqpoint{3.779363in}{3.664000in}}%
\pgfpathlineto{\pgfqpoint{3.933981in}{3.514667in}}%
\pgfpathlineto{\pgfqpoint{4.105643in}{3.345713in}}%
\pgfpathlineto{\pgfqpoint{4.206869in}{3.244947in}}%
\pgfpathlineto{\pgfqpoint{4.382902in}{3.066667in}}%
\pgfpathlineto{\pgfqpoint{4.705783in}{2.730667in}}%
\pgfpathlineto{\pgfqpoint{4.768000in}{2.664561in}}%
\pgfpathlineto{\pgfqpoint{4.768000in}{2.664561in}}%
\pgfusepath{fill}%
\end{pgfscope}%
\begin{pgfscope}%
\pgfpathrectangle{\pgfqpoint{0.800000in}{0.528000in}}{\pgfqpoint{3.968000in}{3.696000in}}%
\pgfusepath{clip}%
\pgfsetbuttcap%
\pgfsetroundjoin%
\definecolor{currentfill}{rgb}{0.151918,0.500685,0.557587}%
\pgfsetfillcolor{currentfill}%
\pgfsetlinewidth{0.000000pt}%
\definecolor{currentstroke}{rgb}{0.000000,0.000000,0.000000}%
\pgfsetstrokecolor{currentstroke}%
\pgfsetdash{}{0pt}%
\pgfpathmoveto{\pgfqpoint{0.800000in}{0.829475in}}%
\pgfpathlineto{\pgfqpoint{1.079898in}{0.528000in}}%
\pgfpathlineto{\pgfqpoint{1.083391in}{0.528000in}}%
\pgfpathlineto{\pgfqpoint{0.800000in}{0.833265in}}%
\pgfpathmoveto{\pgfqpoint{4.768000in}{2.671951in}}%
\pgfpathlineto{\pgfqpoint{4.447354in}{3.007809in}}%
\pgfpathlineto{\pgfqpoint{4.126707in}{3.332154in}}%
\pgfpathlineto{\pgfqpoint{3.786702in}{3.664000in}}%
\pgfpathlineto{\pgfqpoint{3.469772in}{3.962667in}}%
\pgfpathlineto{\pgfqpoint{3.183246in}{4.224000in}}%
\pgfpathlineto{\pgfqpoint{3.179380in}{4.224000in}}%
\pgfpathlineto{\pgfqpoint{3.344257in}{4.074667in}}%
\pgfpathlineto{\pgfqpoint{3.515806in}{3.916309in}}%
\pgfpathlineto{\pgfqpoint{3.605657in}{3.832367in}}%
\pgfpathlineto{\pgfqpoint{3.783032in}{3.664000in}}%
\pgfpathlineto{\pgfqpoint{3.937602in}{3.514667in}}%
\pgfpathlineto{\pgfqpoint{4.107496in}{3.347440in}}%
\pgfpathlineto{\pgfqpoint{4.206869in}{3.248545in}}%
\pgfpathlineto{\pgfqpoint{4.386435in}{3.066667in}}%
\pgfpathlineto{\pgfqpoint{4.709266in}{2.730667in}}%
\pgfpathlineto{\pgfqpoint{4.768000in}{2.668256in}}%
\pgfpathlineto{\pgfqpoint{4.768000in}{2.668256in}}%
\pgfusepath{fill}%
\end{pgfscope}%
\begin{pgfscope}%
\pgfpathrectangle{\pgfqpoint{0.800000in}{0.528000in}}{\pgfqpoint{3.968000in}{3.696000in}}%
\pgfusepath{clip}%
\pgfsetbuttcap%
\pgfsetroundjoin%
\definecolor{currentfill}{rgb}{0.151918,0.500685,0.557587}%
\pgfsetfillcolor{currentfill}%
\pgfsetlinewidth{0.000000pt}%
\definecolor{currentstroke}{rgb}{0.000000,0.000000,0.000000}%
\pgfsetstrokecolor{currentstroke}%
\pgfsetdash{}{0pt}%
\pgfpathmoveto{\pgfqpoint{0.800000in}{0.825699in}}%
\pgfpathlineto{\pgfqpoint{1.076447in}{0.528000in}}%
\pgfpathlineto{\pgfqpoint{1.079898in}{0.528000in}}%
\pgfpathlineto{\pgfqpoint{0.920242in}{0.698842in}}%
\pgfpathlineto{\pgfqpoint{0.800000in}{0.829475in}}%
\pgfpathlineto{\pgfqpoint{0.800000in}{0.826667in}}%
\pgfpathmoveto{\pgfqpoint{4.768000in}{2.675646in}}%
\pgfpathlineto{\pgfqpoint{4.447354in}{3.011426in}}%
\pgfpathlineto{\pgfqpoint{4.126707in}{3.335698in}}%
\pgfpathlineto{\pgfqpoint{3.790372in}{3.664000in}}%
\pgfpathlineto{\pgfqpoint{3.473543in}{3.962667in}}%
\pgfpathlineto{\pgfqpoint{3.187112in}{4.224000in}}%
\pgfpathlineto{\pgfqpoint{3.183246in}{4.224000in}}%
\pgfpathlineto{\pgfqpoint{3.348068in}{4.074667in}}%
\pgfpathlineto{\pgfqpoint{3.517696in}{3.918069in}}%
\pgfpathlineto{\pgfqpoint{3.605657in}{3.835871in}}%
\pgfpathlineto{\pgfqpoint{3.786702in}{3.664000in}}%
\pgfpathlineto{\pgfqpoint{3.941223in}{3.514667in}}%
\pgfpathlineto{\pgfqpoint{4.109350in}{3.349166in}}%
\pgfpathlineto{\pgfqpoint{4.206869in}{3.252143in}}%
\pgfpathlineto{\pgfqpoint{4.389968in}{3.066667in}}%
\pgfpathlineto{\pgfqpoint{4.712749in}{2.730667in}}%
\pgfpathlineto{\pgfqpoint{4.768000in}{2.671951in}}%
\pgfpathlineto{\pgfqpoint{4.768000in}{2.671951in}}%
\pgfusepath{fill}%
\end{pgfscope}%
\begin{pgfscope}%
\pgfpathrectangle{\pgfqpoint{0.800000in}{0.528000in}}{\pgfqpoint{3.968000in}{3.696000in}}%
\pgfusepath{clip}%
\pgfsetbuttcap%
\pgfsetroundjoin%
\definecolor{currentfill}{rgb}{0.151918,0.500685,0.557587}%
\pgfsetfillcolor{currentfill}%
\pgfsetlinewidth{0.000000pt}%
\definecolor{currentstroke}{rgb}{0.000000,0.000000,0.000000}%
\pgfsetstrokecolor{currentstroke}%
\pgfsetdash{}{0pt}%
\pgfpathmoveto{\pgfqpoint{0.800000in}{0.821963in}}%
\pgfpathlineto{\pgfqpoint{1.072995in}{0.528000in}}%
\pgfpathlineto{\pgfqpoint{1.076447in}{0.528000in}}%
\pgfpathlineto{\pgfqpoint{0.920242in}{0.695116in}}%
\pgfpathlineto{\pgfqpoint{0.800000in}{0.825699in}}%
\pgfpathmoveto{\pgfqpoint{4.768000in}{2.679341in}}%
\pgfpathlineto{\pgfqpoint{4.469714in}{2.992000in}}%
\pgfpathlineto{\pgfqpoint{4.323796in}{3.141333in}}%
\pgfpathlineto{\pgfqpoint{4.151825in}{3.314062in}}%
\pgfpathlineto{\pgfqpoint{4.062643in}{3.402667in}}%
\pgfpathlineto{\pgfqpoint{3.910096in}{3.552000in}}%
\pgfpathlineto{\pgfqpoint{3.755035in}{3.701333in}}%
\pgfpathlineto{\pgfqpoint{3.581422in}{3.865427in}}%
\pgfpathlineto{\pgfqpoint{3.485414in}{3.955164in}}%
\pgfpathlineto{\pgfqpoint{3.314790in}{4.112000in}}%
\pgfpathlineto{\pgfqpoint{3.190977in}{4.224000in}}%
\pgfpathlineto{\pgfqpoint{3.187112in}{4.224000in}}%
\pgfpathlineto{\pgfqpoint{3.351880in}{4.074667in}}%
\pgfpathlineto{\pgfqpoint{3.519586in}{3.919830in}}%
\pgfpathlineto{\pgfqpoint{3.605657in}{3.839374in}}%
\pgfpathlineto{\pgfqpoint{3.790372in}{3.664000in}}%
\pgfpathlineto{\pgfqpoint{3.944844in}{3.514667in}}%
\pgfpathlineto{\pgfqpoint{4.111204in}{3.350893in}}%
\pgfpathlineto{\pgfqpoint{4.209237in}{3.253333in}}%
\pgfpathlineto{\pgfqpoint{4.380874in}{3.079411in}}%
\pgfpathlineto{\pgfqpoint{4.466203in}{2.992000in}}%
\pgfpathlineto{\pgfqpoint{4.768000in}{2.675646in}}%
\pgfpathlineto{\pgfqpoint{4.768000in}{2.675646in}}%
\pgfusepath{fill}%
\end{pgfscope}%
\begin{pgfscope}%
\pgfpathrectangle{\pgfqpoint{0.800000in}{0.528000in}}{\pgfqpoint{3.968000in}{3.696000in}}%
\pgfusepath{clip}%
\pgfsetbuttcap%
\pgfsetroundjoin%
\definecolor{currentfill}{rgb}{0.150476,0.504369,0.557430}%
\pgfsetfillcolor{currentfill}%
\pgfsetlinewidth{0.000000pt}%
\definecolor{currentstroke}{rgb}{0.000000,0.000000,0.000000}%
\pgfsetstrokecolor{currentstroke}%
\pgfsetdash{}{0pt}%
\pgfpathmoveto{\pgfqpoint{0.800000in}{0.818227in}}%
\pgfpathlineto{\pgfqpoint{1.069543in}{0.528000in}}%
\pgfpathlineto{\pgfqpoint{1.072995in}{0.528000in}}%
\pgfpathlineto{\pgfqpoint{0.920242in}{0.691391in}}%
\pgfpathlineto{\pgfqpoint{0.800000in}{0.821963in}}%
\pgfpathmoveto{\pgfqpoint{4.768000in}{2.683036in}}%
\pgfpathlineto{\pgfqpoint{4.473224in}{2.992000in}}%
\pgfpathlineto{\pgfqpoint{4.327111in}{3.141575in}}%
\pgfpathlineto{\pgfqpoint{4.141566in}{3.328000in}}%
\pgfpathlineto{\pgfqpoint{3.990286in}{3.477333in}}%
\pgfpathlineto{\pgfqpoint{3.821455in}{3.641005in}}%
\pgfpathlineto{\pgfqpoint{3.719560in}{3.738667in}}%
\pgfpathlineto{\pgfqpoint{3.561261in}{3.888000in}}%
\pgfpathlineto{\pgfqpoint{3.382452in}{4.053429in}}%
\pgfpathlineto{\pgfqpoint{3.281304in}{4.145882in}}%
\pgfpathlineto{\pgfqpoint{3.194843in}{4.224000in}}%
\pgfpathlineto{\pgfqpoint{3.190977in}{4.224000in}}%
\pgfpathlineto{\pgfqpoint{3.355691in}{4.074667in}}%
\pgfpathlineto{\pgfqpoint{3.521476in}{3.921590in}}%
\pgfpathlineto{\pgfqpoint{3.605657in}{3.842877in}}%
\pgfpathlineto{\pgfqpoint{3.794042in}{3.664000in}}%
\pgfpathlineto{\pgfqpoint{3.948464in}{3.514667in}}%
\pgfpathlineto{\pgfqpoint{4.113058in}{3.352620in}}%
\pgfpathlineto{\pgfqpoint{4.212775in}{3.253333in}}%
\pgfpathlineto{\pgfqpoint{4.382711in}{3.081122in}}%
\pgfpathlineto{\pgfqpoint{4.469714in}{2.992000in}}%
\pgfpathlineto{\pgfqpoint{4.768000in}{2.679341in}}%
\pgfpathlineto{\pgfqpoint{4.768000in}{2.679341in}}%
\pgfusepath{fill}%
\end{pgfscope}%
\begin{pgfscope}%
\pgfpathrectangle{\pgfqpoint{0.800000in}{0.528000in}}{\pgfqpoint{3.968000in}{3.696000in}}%
\pgfusepath{clip}%
\pgfsetbuttcap%
\pgfsetroundjoin%
\definecolor{currentfill}{rgb}{0.150476,0.504369,0.557430}%
\pgfsetfillcolor{currentfill}%
\pgfsetlinewidth{0.000000pt}%
\definecolor{currentstroke}{rgb}{0.000000,0.000000,0.000000}%
\pgfsetstrokecolor{currentstroke}%
\pgfsetdash{}{0pt}%
\pgfpathmoveto{\pgfqpoint{0.800000in}{0.814491in}}%
\pgfpathlineto{\pgfqpoint{1.066091in}{0.528000in}}%
\pgfpathlineto{\pgfqpoint{1.069543in}{0.528000in}}%
\pgfpathlineto{\pgfqpoint{0.920242in}{0.687665in}}%
\pgfpathlineto{\pgfqpoint{0.800000in}{0.818227in}}%
\pgfpathmoveto{\pgfqpoint{4.768000in}{2.686731in}}%
\pgfpathlineto{\pgfqpoint{4.476734in}{2.992000in}}%
\pgfpathlineto{\pgfqpoint{4.327111in}{3.145134in}}%
\pgfpathlineto{\pgfqpoint{4.145128in}{3.328000in}}%
\pgfpathlineto{\pgfqpoint{3.993895in}{3.477333in}}%
\pgfpathlineto{\pgfqpoint{3.823302in}{3.642726in}}%
\pgfpathlineto{\pgfqpoint{3.723254in}{3.738667in}}%
\pgfpathlineto{\pgfqpoint{3.565007in}{3.888000in}}%
\pgfpathlineto{\pgfqpoint{3.404007in}{4.037333in}}%
\pgfpathlineto{\pgfqpoint{3.198708in}{4.224000in}}%
\pgfpathlineto{\pgfqpoint{3.194843in}{4.224000in}}%
\pgfpathlineto{\pgfqpoint{3.359503in}{4.074667in}}%
\pgfpathlineto{\pgfqpoint{3.525495in}{3.921394in}}%
\pgfpathlineto{\pgfqpoint{3.685818in}{3.770717in}}%
\pgfpathlineto{\pgfqpoint{3.875215in}{3.589333in}}%
\pgfpathlineto{\pgfqpoint{4.179013in}{3.290667in}}%
\pgfpathlineto{\pgfqpoint{4.329947in}{3.138692in}}%
\pgfpathlineto{\pgfqpoint{4.509346in}{2.954667in}}%
\pgfpathlineto{\pgfqpoint{4.768000in}{2.683036in}}%
\pgfpathlineto{\pgfqpoint{4.768000in}{2.683036in}}%
\pgfusepath{fill}%
\end{pgfscope}%
\begin{pgfscope}%
\pgfpathrectangle{\pgfqpoint{0.800000in}{0.528000in}}{\pgfqpoint{3.968000in}{3.696000in}}%
\pgfusepath{clip}%
\pgfsetbuttcap%
\pgfsetroundjoin%
\definecolor{currentfill}{rgb}{0.150476,0.504369,0.557430}%
\pgfsetfillcolor{currentfill}%
\pgfsetlinewidth{0.000000pt}%
\definecolor{currentstroke}{rgb}{0.000000,0.000000,0.000000}%
\pgfsetstrokecolor{currentstroke}%
\pgfsetdash{}{0pt}%
\pgfpathmoveto{\pgfqpoint{0.800000in}{0.810755in}}%
\pgfpathlineto{\pgfqpoint{1.062640in}{0.528000in}}%
\pgfpathlineto{\pgfqpoint{1.066091in}{0.528000in}}%
\pgfpathlineto{\pgfqpoint{0.920242in}{0.683939in}}%
\pgfpathlineto{\pgfqpoint{0.800000in}{0.814491in}}%
\pgfpathmoveto{\pgfqpoint{4.768000in}{2.690426in}}%
\pgfpathlineto{\pgfqpoint{4.480245in}{2.992000in}}%
\pgfpathlineto{\pgfqpoint{4.327111in}{3.148694in}}%
\pgfpathlineto{\pgfqpoint{4.148689in}{3.328000in}}%
\pgfpathlineto{\pgfqpoint{3.997504in}{3.477333in}}%
\pgfpathlineto{\pgfqpoint{3.825149in}{3.644446in}}%
\pgfpathlineto{\pgfqpoint{3.725899in}{3.739652in}}%
\pgfpathlineto{\pgfqpoint{3.565576in}{3.890930in}}%
\pgfpathlineto{\pgfqpoint{3.386198in}{4.056919in}}%
\pgfpathlineto{\pgfqpoint{3.285010in}{4.149521in}}%
\pgfpathlineto{\pgfqpoint{3.202574in}{4.224000in}}%
\pgfpathlineto{\pgfqpoint{3.198708in}{4.224000in}}%
\pgfpathlineto{\pgfqpoint{3.365172in}{4.072968in}}%
\pgfpathlineto{\pgfqpoint{3.685818in}{3.774226in}}%
\pgfpathlineto{\pgfqpoint{3.878860in}{3.589333in}}%
\pgfpathlineto{\pgfqpoint{4.182563in}{3.290667in}}%
\pgfpathlineto{\pgfqpoint{4.330853in}{3.141333in}}%
\pgfpathlineto{\pgfqpoint{4.487434in}{2.980963in}}%
\pgfpathlineto{\pgfqpoint{4.655940in}{2.805333in}}%
\pgfpathlineto{\pgfqpoint{4.768000in}{2.686731in}}%
\pgfpathlineto{\pgfqpoint{4.768000in}{2.686731in}}%
\pgfusepath{fill}%
\end{pgfscope}%
\begin{pgfscope}%
\pgfpathrectangle{\pgfqpoint{0.800000in}{0.528000in}}{\pgfqpoint{3.968000in}{3.696000in}}%
\pgfusepath{clip}%
\pgfsetbuttcap%
\pgfsetroundjoin%
\definecolor{currentfill}{rgb}{0.150476,0.504369,0.557430}%
\pgfsetfillcolor{currentfill}%
\pgfsetlinewidth{0.000000pt}%
\definecolor{currentstroke}{rgb}{0.000000,0.000000,0.000000}%
\pgfsetstrokecolor{currentstroke}%
\pgfsetdash{}{0pt}%
\pgfpathmoveto{\pgfqpoint{0.800000in}{0.807019in}}%
\pgfpathlineto{\pgfqpoint{1.059188in}{0.528000in}}%
\pgfpathlineto{\pgfqpoint{1.062640in}{0.528000in}}%
\pgfpathlineto{\pgfqpoint{0.920242in}{0.680214in}}%
\pgfpathlineto{\pgfqpoint{0.800000in}{0.810755in}}%
\pgfpathmoveto{\pgfqpoint{4.768000in}{2.694110in}}%
\pgfpathlineto{\pgfqpoint{4.591612in}{2.880000in}}%
\pgfpathlineto{\pgfqpoint{4.264048in}{3.216000in}}%
\pgfpathlineto{\pgfqpoint{3.962947in}{3.514667in}}%
\pgfpathlineto{\pgfqpoint{3.787519in}{3.684062in}}%
\pgfpathlineto{\pgfqpoint{3.685818in}{3.781176in}}%
\pgfpathlineto{\pgfqpoint{3.492292in}{3.962667in}}%
\pgfpathlineto{\pgfqpoint{3.325091in}{4.116491in}}%
\pgfpathlineto{\pgfqpoint{3.206414in}{4.224000in}}%
\pgfpathlineto{\pgfqpoint{3.202574in}{4.224000in}}%
\pgfpathlineto{\pgfqpoint{3.365172in}{4.076430in}}%
\pgfpathlineto{\pgfqpoint{3.528727in}{3.925333in}}%
\pgfpathlineto{\pgfqpoint{3.687623in}{3.776000in}}%
\pgfpathlineto{\pgfqpoint{3.864544in}{3.606474in}}%
\pgfpathlineto{\pgfqpoint{3.966384in}{3.507784in}}%
\pgfpathlineto{\pgfqpoint{4.148689in}{3.328000in}}%
\pgfpathlineto{\pgfqpoint{4.447354in}{3.025897in}}%
\pgfpathlineto{\pgfqpoint{4.623831in}{2.842667in}}%
\pgfpathlineto{\pgfqpoint{4.768000in}{2.690426in}}%
\pgfpathlineto{\pgfqpoint{4.768000in}{2.693333in}}%
\pgfpathlineto{\pgfqpoint{4.768000in}{2.693333in}}%
\pgfusepath{fill}%
\end{pgfscope}%
\begin{pgfscope}%
\pgfpathrectangle{\pgfqpoint{0.800000in}{0.528000in}}{\pgfqpoint{3.968000in}{3.696000in}}%
\pgfusepath{clip}%
\pgfsetbuttcap%
\pgfsetroundjoin%
\definecolor{currentfill}{rgb}{0.149039,0.508051,0.557250}%
\pgfsetfillcolor{currentfill}%
\pgfsetlinewidth{0.000000pt}%
\definecolor{currentstroke}{rgb}{0.000000,0.000000,0.000000}%
\pgfsetstrokecolor{currentstroke}%
\pgfsetdash{}{0pt}%
\pgfpathmoveto{\pgfqpoint{0.800000in}{0.803282in}}%
\pgfpathlineto{\pgfqpoint{1.055736in}{0.528000in}}%
\pgfpathlineto{\pgfqpoint{1.059188in}{0.528000in}}%
\pgfpathlineto{\pgfqpoint{0.914585in}{0.682603in}}%
\pgfpathlineto{\pgfqpoint{0.800000in}{0.807019in}}%
\pgfpathmoveto{\pgfqpoint{4.768000in}{2.697754in}}%
\pgfpathlineto{\pgfqpoint{4.595089in}{2.880000in}}%
\pgfpathlineto{\pgfqpoint{4.267575in}{3.216000in}}%
\pgfpathlineto{\pgfqpoint{3.966384in}{3.514844in}}%
\pgfpathlineto{\pgfqpoint{3.773332in}{3.701333in}}%
\pgfpathlineto{\pgfqpoint{3.605657in}{3.860267in}}%
\pgfpathlineto{\pgfqpoint{3.285010in}{4.156388in}}%
\pgfpathlineto{\pgfqpoint{3.210218in}{4.224000in}}%
\pgfpathlineto{\pgfqpoint{3.206414in}{4.224000in}}%
\pgfpathlineto{\pgfqpoint{3.532427in}{3.925333in}}%
\pgfpathlineto{\pgfqpoint{3.691274in}{3.776000in}}%
\pgfpathlineto{\pgfqpoint{3.866389in}{3.608193in}}%
\pgfpathlineto{\pgfqpoint{3.966384in}{3.511315in}}%
\pgfpathlineto{\pgfqpoint{4.152251in}{3.328000in}}%
\pgfpathlineto{\pgfqpoint{4.449188in}{3.027625in}}%
\pgfpathlineto{\pgfqpoint{4.627297in}{2.842667in}}%
\pgfpathlineto{\pgfqpoint{4.768000in}{2.694110in}}%
\pgfpathlineto{\pgfqpoint{4.768000in}{2.694110in}}%
\pgfusepath{fill}%
\end{pgfscope}%
\begin{pgfscope}%
\pgfpathrectangle{\pgfqpoint{0.800000in}{0.528000in}}{\pgfqpoint{3.968000in}{3.696000in}}%
\pgfusepath{clip}%
\pgfsetbuttcap%
\pgfsetroundjoin%
\definecolor{currentfill}{rgb}{0.149039,0.508051,0.557250}%
\pgfsetfillcolor{currentfill}%
\pgfsetlinewidth{0.000000pt}%
\definecolor{currentstroke}{rgb}{0.000000,0.000000,0.000000}%
\pgfsetstrokecolor{currentstroke}%
\pgfsetdash{}{0pt}%
\pgfpathmoveto{\pgfqpoint{0.800000in}{0.799546in}}%
\pgfpathlineto{\pgfqpoint{1.052285in}{0.528000in}}%
\pgfpathlineto{\pgfqpoint{1.055736in}{0.528000in}}%
\pgfpathlineto{\pgfqpoint{0.889647in}{0.705831in}}%
\pgfpathlineto{\pgfqpoint{0.800000in}{0.803282in}}%
\pgfpathmoveto{\pgfqpoint{4.768000in}{2.701398in}}%
\pgfpathlineto{\pgfqpoint{4.598566in}{2.880000in}}%
\pgfpathlineto{\pgfqpoint{4.271102in}{3.216000in}}%
\pgfpathlineto{\pgfqpoint{3.966384in}{3.518329in}}%
\pgfpathlineto{\pgfqpoint{3.776958in}{3.701333in}}%
\pgfpathlineto{\pgfqpoint{3.605657in}{3.863724in}}%
\pgfpathlineto{\pgfqpoint{3.255377in}{4.186667in}}%
\pgfpathlineto{\pgfqpoint{3.214022in}{4.224000in}}%
\pgfpathlineto{\pgfqpoint{3.210218in}{4.224000in}}%
\pgfpathlineto{\pgfqpoint{3.536127in}{3.925333in}}%
\pgfpathlineto{\pgfqpoint{3.694924in}{3.776000in}}%
\pgfpathlineto{\pgfqpoint{3.868233in}{3.609911in}}%
\pgfpathlineto{\pgfqpoint{3.970204in}{3.511108in}}%
\pgfpathlineto{\pgfqpoint{4.155812in}{3.328000in}}%
\pgfpathlineto{\pgfqpoint{4.450998in}{3.029333in}}%
\pgfpathlineto{\pgfqpoint{4.607677in}{2.866854in}}%
\pgfpathlineto{\pgfqpoint{4.768000in}{2.697754in}}%
\pgfpathlineto{\pgfqpoint{4.768000in}{2.697754in}}%
\pgfusepath{fill}%
\end{pgfscope}%
\begin{pgfscope}%
\pgfpathrectangle{\pgfqpoint{0.800000in}{0.528000in}}{\pgfqpoint{3.968000in}{3.696000in}}%
\pgfusepath{clip}%
\pgfsetbuttcap%
\pgfsetroundjoin%
\definecolor{currentfill}{rgb}{0.149039,0.508051,0.557250}%
\pgfsetfillcolor{currentfill}%
\pgfsetlinewidth{0.000000pt}%
\definecolor{currentstroke}{rgb}{0.000000,0.000000,0.000000}%
\pgfsetstrokecolor{currentstroke}%
\pgfsetdash{}{0pt}%
\pgfpathmoveto{\pgfqpoint{0.800000in}{0.795810in}}%
\pgfpathlineto{\pgfqpoint{1.048833in}{0.528000in}}%
\pgfpathlineto{\pgfqpoint{1.052285in}{0.528000in}}%
\pgfpathlineto{\pgfqpoint{0.912652in}{0.677333in}}%
\pgfpathlineto{\pgfqpoint{0.800000in}{0.799546in}}%
\pgfpathmoveto{\pgfqpoint{4.768000in}{2.705043in}}%
\pgfpathlineto{\pgfqpoint{4.602042in}{2.880000in}}%
\pgfpathlineto{\pgfqpoint{4.274628in}{3.216000in}}%
\pgfpathlineto{\pgfqpoint{3.935387in}{3.552000in}}%
\pgfpathlineto{\pgfqpoint{3.623210in}{3.850667in}}%
\pgfpathlineto{\pgfqpoint{3.463162in}{4.000000in}}%
\pgfpathlineto{\pgfqpoint{3.285010in}{4.163256in}}%
\pgfpathlineto{\pgfqpoint{3.217825in}{4.224000in}}%
\pgfpathlineto{\pgfqpoint{3.214022in}{4.224000in}}%
\pgfpathlineto{\pgfqpoint{3.539827in}{3.925333in}}%
\pgfpathlineto{\pgfqpoint{3.698574in}{3.776000in}}%
\pgfpathlineto{\pgfqpoint{3.870078in}{3.611629in}}%
\pgfpathlineto{\pgfqpoint{3.970132in}{3.514667in}}%
\pgfpathlineto{\pgfqpoint{4.126707in}{3.360504in}}%
\pgfpathlineto{\pgfqpoint{4.308056in}{3.178667in}}%
\pgfpathlineto{\pgfqpoint{4.454469in}{3.029333in}}%
\pgfpathlineto{\pgfqpoint{4.607677in}{2.870485in}}%
\pgfpathlineto{\pgfqpoint{4.768000in}{2.701398in}}%
\pgfpathlineto{\pgfqpoint{4.768000in}{2.701398in}}%
\pgfusepath{fill}%
\end{pgfscope}%
\begin{pgfscope}%
\pgfpathrectangle{\pgfqpoint{0.800000in}{0.528000in}}{\pgfqpoint{3.968000in}{3.696000in}}%
\pgfusepath{clip}%
\pgfsetbuttcap%
\pgfsetroundjoin%
\definecolor{currentfill}{rgb}{0.149039,0.508051,0.557250}%
\pgfsetfillcolor{currentfill}%
\pgfsetlinewidth{0.000000pt}%
\definecolor{currentstroke}{rgb}{0.000000,0.000000,0.000000}%
\pgfsetstrokecolor{currentstroke}%
\pgfsetdash{}{0pt}%
\pgfpathmoveto{\pgfqpoint{0.800000in}{0.792074in}}%
\pgfpathlineto{\pgfqpoint{1.045381in}{0.528000in}}%
\pgfpathlineto{\pgfqpoint{1.048833in}{0.528000in}}%
\pgfpathlineto{\pgfqpoint{0.909244in}{0.677333in}}%
\pgfpathlineto{\pgfqpoint{0.800000in}{0.795810in}}%
\pgfpathmoveto{\pgfqpoint{4.768000in}{2.708687in}}%
\pgfpathlineto{\pgfqpoint{4.605519in}{2.880000in}}%
\pgfpathlineto{\pgfqpoint{4.278155in}{3.216000in}}%
\pgfpathlineto{\pgfqpoint{3.938965in}{3.552000in}}%
\pgfpathlineto{\pgfqpoint{3.626885in}{3.850667in}}%
\pgfpathlineto{\pgfqpoint{3.466887in}{4.000000in}}%
\pgfpathlineto{\pgfqpoint{3.294383in}{4.158063in}}%
\pgfpathlineto{\pgfqpoint{3.221629in}{4.224000in}}%
\pgfpathlineto{\pgfqpoint{3.217825in}{4.224000in}}%
\pgfpathlineto{\pgfqpoint{3.543527in}{3.925333in}}%
\pgfpathlineto{\pgfqpoint{3.702225in}{3.776000in}}%
\pgfpathlineto{\pgfqpoint{3.871923in}{3.613347in}}%
\pgfpathlineto{\pgfqpoint{3.970106in}{3.518134in}}%
\pgfpathlineto{\pgfqpoint{4.049864in}{3.440000in}}%
\pgfpathlineto{\pgfqpoint{4.222680in}{3.268061in}}%
\pgfpathlineto{\pgfqpoint{4.311571in}{3.178667in}}%
\pgfpathlineto{\pgfqpoint{4.457939in}{3.029333in}}%
\pgfpathlineto{\pgfqpoint{4.607677in}{2.874115in}}%
\pgfpathlineto{\pgfqpoint{4.768000in}{2.705043in}}%
\pgfpathlineto{\pgfqpoint{4.768000in}{2.705043in}}%
\pgfusepath{fill}%
\end{pgfscope}%
\begin{pgfscope}%
\pgfpathrectangle{\pgfqpoint{0.800000in}{0.528000in}}{\pgfqpoint{3.968000in}{3.696000in}}%
\pgfusepath{clip}%
\pgfsetbuttcap%
\pgfsetroundjoin%
\definecolor{currentfill}{rgb}{0.147607,0.511733,0.557049}%
\pgfsetfillcolor{currentfill}%
\pgfsetlinewidth{0.000000pt}%
\definecolor{currentstroke}{rgb}{0.000000,0.000000,0.000000}%
\pgfsetstrokecolor{currentstroke}%
\pgfsetdash{}{0pt}%
\pgfpathmoveto{\pgfqpoint{0.800000in}{0.788352in}}%
\pgfpathlineto{\pgfqpoint{1.041930in}{0.528000in}}%
\pgfpathlineto{\pgfqpoint{1.045381in}{0.528000in}}%
\pgfpathlineto{\pgfqpoint{0.905836in}{0.677333in}}%
\pgfpathlineto{\pgfqpoint{0.800000in}{0.792074in}}%
\pgfpathlineto{\pgfqpoint{0.800000in}{0.789333in}}%
\pgfpathmoveto{\pgfqpoint{4.768000in}{2.712331in}}%
\pgfpathlineto{\pgfqpoint{4.607677in}{2.881359in}}%
\pgfpathlineto{\pgfqpoint{4.281682in}{3.216000in}}%
\pgfpathlineto{\pgfqpoint{3.942543in}{3.552000in}}%
\pgfpathlineto{\pgfqpoint{3.630560in}{3.850667in}}%
\pgfpathlineto{\pgfqpoint{3.470612in}{4.000000in}}%
\pgfpathlineto{\pgfqpoint{3.296237in}{4.159791in}}%
\pgfpathlineto{\pgfqpoint{3.225433in}{4.224000in}}%
\pgfpathlineto{\pgfqpoint{3.221629in}{4.224000in}}%
\pgfpathlineto{\pgfqpoint{3.547227in}{3.925333in}}%
\pgfpathlineto{\pgfqpoint{3.705875in}{3.776000in}}%
\pgfpathlineto{\pgfqpoint{3.873767in}{3.615065in}}%
\pgfpathlineto{\pgfqpoint{3.966384in}{3.525298in}}%
\pgfpathlineto{\pgfqpoint{4.287030in}{3.207050in}}%
\pgfpathlineto{\pgfqpoint{4.607677in}{2.877746in}}%
\pgfpathlineto{\pgfqpoint{4.768000in}{2.708687in}}%
\pgfpathlineto{\pgfqpoint{4.768000in}{2.708687in}}%
\pgfusepath{fill}%
\end{pgfscope}%
\begin{pgfscope}%
\pgfpathrectangle{\pgfqpoint{0.800000in}{0.528000in}}{\pgfqpoint{3.968000in}{3.696000in}}%
\pgfusepath{clip}%
\pgfsetbuttcap%
\pgfsetroundjoin%
\definecolor{currentfill}{rgb}{0.147607,0.511733,0.557049}%
\pgfsetfillcolor{currentfill}%
\pgfsetlinewidth{0.000000pt}%
\definecolor{currentstroke}{rgb}{0.000000,0.000000,0.000000}%
\pgfsetstrokecolor{currentstroke}%
\pgfsetdash{}{0pt}%
\pgfpathmoveto{\pgfqpoint{0.800000in}{0.784668in}}%
\pgfpathlineto{\pgfqpoint{1.038507in}{0.528000in}}%
\pgfpathlineto{\pgfqpoint{1.041930in}{0.528000in}}%
\pgfpathlineto{\pgfqpoint{0.902427in}{0.677333in}}%
\pgfpathlineto{\pgfqpoint{0.800000in}{0.788352in}}%
\pgfpathmoveto{\pgfqpoint{4.768000in}{2.715975in}}%
\pgfpathlineto{\pgfqpoint{4.607677in}{2.884940in}}%
\pgfpathlineto{\pgfqpoint{4.285208in}{3.216000in}}%
\pgfpathlineto{\pgfqpoint{3.946122in}{3.552000in}}%
\pgfpathlineto{\pgfqpoint{3.634236in}{3.850667in}}%
\pgfpathlineto{\pgfqpoint{3.474338in}{4.000000in}}%
\pgfpathlineto{\pgfqpoint{3.298091in}{4.161518in}}%
\pgfpathlineto{\pgfqpoint{3.229237in}{4.224000in}}%
\pgfpathlineto{\pgfqpoint{3.225433in}{4.224000in}}%
\pgfpathlineto{\pgfqpoint{3.550927in}{3.925333in}}%
\pgfpathlineto{\pgfqpoint{3.709525in}{3.776000in}}%
\pgfpathlineto{\pgfqpoint{3.875612in}{3.616784in}}%
\pgfpathlineto{\pgfqpoint{3.966384in}{3.528783in}}%
\pgfpathlineto{\pgfqpoint{4.287030in}{3.210607in}}%
\pgfpathlineto{\pgfqpoint{4.619523in}{2.868966in}}%
\pgfpathlineto{\pgfqpoint{4.768000in}{2.712331in}}%
\pgfpathlineto{\pgfqpoint{4.768000in}{2.712331in}}%
\pgfusepath{fill}%
\end{pgfscope}%
\begin{pgfscope}%
\pgfpathrectangle{\pgfqpoint{0.800000in}{0.528000in}}{\pgfqpoint{3.968000in}{3.696000in}}%
\pgfusepath{clip}%
\pgfsetbuttcap%
\pgfsetroundjoin%
\definecolor{currentfill}{rgb}{0.147607,0.511733,0.557049}%
\pgfsetfillcolor{currentfill}%
\pgfsetlinewidth{0.000000pt}%
\definecolor{currentstroke}{rgb}{0.000000,0.000000,0.000000}%
\pgfsetstrokecolor{currentstroke}%
\pgfsetdash{}{0pt}%
\pgfpathmoveto{\pgfqpoint{0.800000in}{0.780984in}}%
\pgfpathlineto{\pgfqpoint{1.035104in}{0.528000in}}%
\pgfpathlineto{\pgfqpoint{1.038507in}{0.528000in}}%
\pgfpathlineto{\pgfqpoint{0.871776in}{0.706856in}}%
\pgfpathlineto{\pgfqpoint{0.800000in}{0.784668in}}%
\pgfpathmoveto{\pgfqpoint{4.768000in}{2.719619in}}%
\pgfpathlineto{\pgfqpoint{4.607677in}{2.888522in}}%
\pgfpathlineto{\pgfqpoint{4.287030in}{3.217696in}}%
\pgfpathlineto{\pgfqpoint{3.949700in}{3.552000in}}%
\pgfpathlineto{\pgfqpoint{3.637911in}{3.850667in}}%
\pgfpathlineto{\pgfqpoint{3.478063in}{4.000000in}}%
\pgfpathlineto{\pgfqpoint{3.299946in}{4.163245in}}%
\pgfpathlineto{\pgfqpoint{3.233040in}{4.224000in}}%
\pgfpathlineto{\pgfqpoint{3.229237in}{4.224000in}}%
\pgfpathlineto{\pgfqpoint{3.525495in}{3.952509in}}%
\pgfpathlineto{\pgfqpoint{3.713176in}{3.776000in}}%
\pgfpathlineto{\pgfqpoint{3.886222in}{3.610136in}}%
\pgfpathlineto{\pgfqpoint{4.060493in}{3.440000in}}%
\pgfpathlineto{\pgfqpoint{4.228151in}{3.273157in}}%
\pgfpathlineto{\pgfqpoint{4.327111in}{3.173610in}}%
\pgfpathlineto{\pgfqpoint{4.648085in}{2.842667in}}%
\pgfpathlineto{\pgfqpoint{4.768000in}{2.715975in}}%
\pgfpathlineto{\pgfqpoint{4.768000in}{2.715975in}}%
\pgfusepath{fill}%
\end{pgfscope}%
\begin{pgfscope}%
\pgfpathrectangle{\pgfqpoint{0.800000in}{0.528000in}}{\pgfqpoint{3.968000in}{3.696000in}}%
\pgfusepath{clip}%
\pgfsetbuttcap%
\pgfsetroundjoin%
\definecolor{currentfill}{rgb}{0.146180,0.515413,0.556823}%
\pgfsetfillcolor{currentfill}%
\pgfsetlinewidth{0.000000pt}%
\definecolor{currentstroke}{rgb}{0.000000,0.000000,0.000000}%
\pgfsetstrokecolor{currentstroke}%
\pgfsetdash{}{0pt}%
\pgfpathmoveto{\pgfqpoint{0.800000in}{0.777299in}}%
\pgfpathlineto{\pgfqpoint{1.031702in}{0.528000in}}%
\pgfpathlineto{\pgfqpoint{1.035104in}{0.528000in}}%
\pgfpathlineto{\pgfqpoint{0.869961in}{0.705165in}}%
\pgfpathlineto{\pgfqpoint{0.800000in}{0.780984in}}%
\pgfpathmoveto{\pgfqpoint{4.768000in}{2.723263in}}%
\pgfpathlineto{\pgfqpoint{4.607677in}{2.892104in}}%
\pgfpathlineto{\pgfqpoint{4.287030in}{3.221205in}}%
\pgfpathlineto{\pgfqpoint{3.953278in}{3.552000in}}%
\pgfpathlineto{\pgfqpoint{3.641586in}{3.850667in}}%
\pgfpathlineto{\pgfqpoint{3.481789in}{4.000000in}}%
\pgfpathlineto{\pgfqpoint{3.301800in}{4.164972in}}%
\pgfpathlineto{\pgfqpoint{3.236844in}{4.224000in}}%
\pgfpathlineto{\pgfqpoint{3.233040in}{4.224000in}}%
\pgfpathlineto{\pgfqpoint{3.525495in}{3.955961in}}%
\pgfpathlineto{\pgfqpoint{3.716826in}{3.776000in}}%
\pgfpathlineto{\pgfqpoint{3.886222in}{3.613615in}}%
\pgfpathlineto{\pgfqpoint{4.064036in}{3.440000in}}%
\pgfpathlineto{\pgfqpoint{4.229975in}{3.274856in}}%
\pgfpathlineto{\pgfqpoint{4.327111in}{3.177169in}}%
\pgfpathlineto{\pgfqpoint{4.651501in}{2.842667in}}%
\pgfpathlineto{\pgfqpoint{4.768000in}{2.719619in}}%
\pgfpathlineto{\pgfqpoint{4.768000in}{2.719619in}}%
\pgfusepath{fill}%
\end{pgfscope}%
\begin{pgfscope}%
\pgfpathrectangle{\pgfqpoint{0.800000in}{0.528000in}}{\pgfqpoint{3.968000in}{3.696000in}}%
\pgfusepath{clip}%
\pgfsetbuttcap%
\pgfsetroundjoin%
\definecolor{currentfill}{rgb}{0.146180,0.515413,0.556823}%
\pgfsetfillcolor{currentfill}%
\pgfsetlinewidth{0.000000pt}%
\definecolor{currentstroke}{rgb}{0.000000,0.000000,0.000000}%
\pgfsetstrokecolor{currentstroke}%
\pgfsetdash{}{0pt}%
\pgfpathmoveto{\pgfqpoint{0.800000in}{0.773615in}}%
\pgfpathlineto{\pgfqpoint{1.028300in}{0.528000in}}%
\pgfpathlineto{\pgfqpoint{1.031702in}{0.528000in}}%
\pgfpathlineto{\pgfqpoint{0.868146in}{0.703474in}}%
\pgfpathlineto{\pgfqpoint{0.800000in}{0.777299in}}%
\pgfpathmoveto{\pgfqpoint{4.768000in}{2.726907in}}%
\pgfpathlineto{\pgfqpoint{4.607677in}{2.895686in}}%
\pgfpathlineto{\pgfqpoint{4.287030in}{3.224715in}}%
\pgfpathlineto{\pgfqpoint{3.956856in}{3.552000in}}%
\pgfpathlineto{\pgfqpoint{3.645261in}{3.850667in}}%
\pgfpathlineto{\pgfqpoint{3.471321in}{4.013127in}}%
\pgfpathlineto{\pgfqpoint{3.303654in}{4.166699in}}%
\pgfpathlineto{\pgfqpoint{3.240648in}{4.224000in}}%
\pgfpathlineto{\pgfqpoint{3.236844in}{4.224000in}}%
\pgfpathlineto{\pgfqpoint{3.525495in}{3.959412in}}%
\pgfpathlineto{\pgfqpoint{3.703202in}{3.792192in}}%
\pgfpathlineto{\pgfqpoint{3.806061in}{3.694297in}}%
\pgfpathlineto{\pgfqpoint{3.991530in}{3.514667in}}%
\pgfpathlineto{\pgfqpoint{4.143031in}{3.365333in}}%
\pgfpathlineto{\pgfqpoint{4.308827in}{3.198969in}}%
\pgfpathlineto{\pgfqpoint{4.407273in}{3.099124in}}%
\pgfpathlineto{\pgfqpoint{4.583464in}{2.917333in}}%
\pgfpathlineto{\pgfqpoint{4.768000in}{2.723263in}}%
\pgfpathlineto{\pgfqpoint{4.768000in}{2.723263in}}%
\pgfusepath{fill}%
\end{pgfscope}%
\begin{pgfscope}%
\pgfpathrectangle{\pgfqpoint{0.800000in}{0.528000in}}{\pgfqpoint{3.968000in}{3.696000in}}%
\pgfusepath{clip}%
\pgfsetbuttcap%
\pgfsetroundjoin%
\definecolor{currentfill}{rgb}{0.146180,0.515413,0.556823}%
\pgfsetfillcolor{currentfill}%
\pgfsetlinewidth{0.000000pt}%
\definecolor{currentstroke}{rgb}{0.000000,0.000000,0.000000}%
\pgfsetstrokecolor{currentstroke}%
\pgfsetdash{}{0pt}%
\pgfpathmoveto{\pgfqpoint{0.800000in}{0.769931in}}%
\pgfpathlineto{\pgfqpoint{1.024897in}{0.528000in}}%
\pgfpathlineto{\pgfqpoint{1.028300in}{0.528000in}}%
\pgfpathlineto{\pgfqpoint{0.866330in}{0.701784in}}%
\pgfpathlineto{\pgfqpoint{0.800000in}{0.773615in}}%
\pgfpathmoveto{\pgfqpoint{4.768000in}{2.730552in}}%
\pgfpathlineto{\pgfqpoint{4.607677in}{2.899267in}}%
\pgfpathlineto{\pgfqpoint{4.287030in}{3.228224in}}%
\pgfpathlineto{\pgfqpoint{3.960434in}{3.552000in}}%
\pgfpathlineto{\pgfqpoint{3.645737in}{3.853639in}}%
\pgfpathlineto{\pgfqpoint{3.485414in}{4.003496in}}%
\pgfpathlineto{\pgfqpoint{3.305508in}{4.168427in}}%
\pgfpathlineto{\pgfqpoint{3.244452in}{4.224000in}}%
\pgfpathlineto{\pgfqpoint{3.240648in}{4.224000in}}%
\pgfpathlineto{\pgfqpoint{3.525704in}{3.962667in}}%
\pgfpathlineto{\pgfqpoint{3.705033in}{3.793898in}}%
\pgfpathlineto{\pgfqpoint{3.806061in}{3.697770in}}%
\pgfpathlineto{\pgfqpoint{3.995097in}{3.514667in}}%
\pgfpathlineto{\pgfqpoint{4.166788in}{3.345213in}}%
\pgfpathlineto{\pgfqpoint{4.487434in}{3.020405in}}%
\pgfpathlineto{\pgfqpoint{4.768000in}{2.726907in}}%
\pgfpathlineto{\pgfqpoint{4.768000in}{2.726907in}}%
\pgfusepath{fill}%
\end{pgfscope}%
\begin{pgfscope}%
\pgfpathrectangle{\pgfqpoint{0.800000in}{0.528000in}}{\pgfqpoint{3.968000in}{3.696000in}}%
\pgfusepath{clip}%
\pgfsetbuttcap%
\pgfsetroundjoin%
\definecolor{currentfill}{rgb}{0.146180,0.515413,0.556823}%
\pgfsetfillcolor{currentfill}%
\pgfsetlinewidth{0.000000pt}%
\definecolor{currentstroke}{rgb}{0.000000,0.000000,0.000000}%
\pgfsetstrokecolor{currentstroke}%
\pgfsetdash{}{0pt}%
\pgfpathmoveto{\pgfqpoint{0.800000in}{0.766247in}}%
\pgfpathlineto{\pgfqpoint{1.021495in}{0.528000in}}%
\pgfpathlineto{\pgfqpoint{1.024897in}{0.528000in}}%
\pgfpathlineto{\pgfqpoint{0.864515in}{0.700093in}}%
\pgfpathlineto{\pgfqpoint{0.800000in}{0.769931in}}%
\pgfpathmoveto{\pgfqpoint{4.768000in}{2.734148in}}%
\pgfpathlineto{\pgfqpoint{4.447354in}{3.068743in}}%
\pgfpathlineto{\pgfqpoint{4.265570in}{3.253333in}}%
\pgfpathlineto{\pgfqpoint{3.945454in}{3.569838in}}%
\pgfpathlineto{\pgfqpoint{3.846141in}{3.666175in}}%
\pgfpathlineto{\pgfqpoint{3.652506in}{3.850667in}}%
\pgfpathlineto{\pgfqpoint{3.365172in}{4.117632in}}%
\pgfpathlineto{\pgfqpoint{3.248203in}{4.224000in}}%
\pgfpathlineto{\pgfqpoint{3.244452in}{4.224000in}}%
\pgfpathlineto{\pgfqpoint{3.244929in}{4.223569in}}%
\pgfpathlineto{\pgfqpoint{3.408309in}{4.074667in}}%
\pgfpathlineto{\pgfqpoint{3.587351in}{3.908283in}}%
\pgfpathlineto{\pgfqpoint{3.688399in}{3.813333in}}%
\pgfpathlineto{\pgfqpoint{4.036740in}{3.477333in}}%
\pgfpathlineto{\pgfqpoint{4.206869in}{3.308717in}}%
\pgfpathlineto{\pgfqpoint{4.372817in}{3.141333in}}%
\pgfpathlineto{\pgfqpoint{4.541613in}{2.967798in}}%
\pgfpathlineto{\pgfqpoint{4.626109in}{2.880000in}}%
\pgfpathlineto{\pgfqpoint{4.768000in}{2.730552in}}%
\pgfpathlineto{\pgfqpoint{4.768000in}{2.730667in}}%
\pgfusepath{fill}%
\end{pgfscope}%
\begin{pgfscope}%
\pgfpathrectangle{\pgfqpoint{0.800000in}{0.528000in}}{\pgfqpoint{3.968000in}{3.696000in}}%
\pgfusepath{clip}%
\pgfsetbuttcap%
\pgfsetroundjoin%
\definecolor{currentfill}{rgb}{0.144759,0.519093,0.556572}%
\pgfsetfillcolor{currentfill}%
\pgfsetlinewidth{0.000000pt}%
\definecolor{currentstroke}{rgb}{0.000000,0.000000,0.000000}%
\pgfsetstrokecolor{currentstroke}%
\pgfsetdash{}{0pt}%
\pgfpathmoveto{\pgfqpoint{0.800000in}{0.762563in}}%
\pgfpathlineto{\pgfqpoint{1.018093in}{0.528000in}}%
\pgfpathlineto{\pgfqpoint{1.021495in}{0.528000in}}%
\pgfpathlineto{\pgfqpoint{0.862700in}{0.698402in}}%
\pgfpathlineto{\pgfqpoint{0.800000in}{0.766247in}}%
\pgfpathmoveto{\pgfqpoint{4.768000in}{2.737742in}}%
\pgfpathlineto{\pgfqpoint{4.447354in}{3.072265in}}%
\pgfpathlineto{\pgfqpoint{4.269057in}{3.253333in}}%
\pgfpathlineto{\pgfqpoint{3.947271in}{3.571531in}}%
\pgfpathlineto{\pgfqpoint{3.846141in}{3.669605in}}%
\pgfpathlineto{\pgfqpoint{3.656125in}{3.850667in}}%
\pgfpathlineto{\pgfqpoint{3.349734in}{4.134954in}}%
\pgfpathlineto{\pgfqpoint{3.251947in}{4.224000in}}%
\pgfpathlineto{\pgfqpoint{3.248203in}{4.224000in}}%
\pgfpathlineto{\pgfqpoint{3.412002in}{4.074667in}}%
\pgfpathlineto{\pgfqpoint{3.589189in}{3.909995in}}%
\pgfpathlineto{\pgfqpoint{3.692006in}{3.813333in}}%
\pgfpathlineto{\pgfqpoint{4.023716in}{3.493402in}}%
\pgfpathlineto{\pgfqpoint{4.126707in}{3.392039in}}%
\pgfpathlineto{\pgfqpoint{4.449377in}{3.066667in}}%
\pgfpathlineto{\pgfqpoint{4.768000in}{2.734148in}}%
\pgfpathlineto{\pgfqpoint{4.768000in}{2.734148in}}%
\pgfusepath{fill}%
\end{pgfscope}%
\begin{pgfscope}%
\pgfpathrectangle{\pgfqpoint{0.800000in}{0.528000in}}{\pgfqpoint{3.968000in}{3.696000in}}%
\pgfusepath{clip}%
\pgfsetbuttcap%
\pgfsetroundjoin%
\definecolor{currentfill}{rgb}{0.144759,0.519093,0.556572}%
\pgfsetfillcolor{currentfill}%
\pgfsetlinewidth{0.000000pt}%
\definecolor{currentstroke}{rgb}{0.000000,0.000000,0.000000}%
\pgfsetstrokecolor{currentstroke}%
\pgfsetdash{}{0pt}%
\pgfpathmoveto{\pgfqpoint{0.800000in}{0.758879in}}%
\pgfpathlineto{\pgfqpoint{1.014691in}{0.528000in}}%
\pgfpathlineto{\pgfqpoint{1.018093in}{0.528000in}}%
\pgfpathlineto{\pgfqpoint{0.860885in}{0.696711in}}%
\pgfpathlineto{\pgfqpoint{0.800000in}{0.762563in}}%
\pgfpathmoveto{\pgfqpoint{4.768000in}{2.741337in}}%
\pgfpathlineto{\pgfqpoint{4.447354in}{3.075786in}}%
\pgfpathlineto{\pgfqpoint{4.272543in}{3.253333in}}%
\pgfpathlineto{\pgfqpoint{3.949089in}{3.573224in}}%
\pgfpathlineto{\pgfqpoint{3.846141in}{3.673036in}}%
\pgfpathlineto{\pgfqpoint{3.659744in}{3.850667in}}%
\pgfpathlineto{\pgfqpoint{3.337895in}{4.149333in}}%
\pgfpathlineto{\pgfqpoint{3.255691in}{4.224000in}}%
\pgfpathlineto{\pgfqpoint{3.251947in}{4.224000in}}%
\pgfpathlineto{\pgfqpoint{3.415695in}{4.074667in}}%
\pgfpathlineto{\pgfqpoint{3.591027in}{3.911707in}}%
\pgfpathlineto{\pgfqpoint{3.685818in}{3.822617in}}%
\pgfpathlineto{\pgfqpoint{4.006465in}{3.514013in}}%
\pgfpathlineto{\pgfqpoint{4.342974in}{3.178667in}}%
\pgfpathlineto{\pgfqpoint{4.668580in}{2.842667in}}%
\pgfpathlineto{\pgfqpoint{4.768000in}{2.737742in}}%
\pgfpathlineto{\pgfqpoint{4.768000in}{2.737742in}}%
\pgfusepath{fill}%
\end{pgfscope}%
\begin{pgfscope}%
\pgfpathrectangle{\pgfqpoint{0.800000in}{0.528000in}}{\pgfqpoint{3.968000in}{3.696000in}}%
\pgfusepath{clip}%
\pgfsetbuttcap%
\pgfsetroundjoin%
\definecolor{currentfill}{rgb}{0.144759,0.519093,0.556572}%
\pgfsetfillcolor{currentfill}%
\pgfsetlinewidth{0.000000pt}%
\definecolor{currentstroke}{rgb}{0.000000,0.000000,0.000000}%
\pgfsetstrokecolor{currentstroke}%
\pgfsetdash{}{0pt}%
\pgfpathmoveto{\pgfqpoint{0.800000in}{0.755195in}}%
\pgfpathlineto{\pgfqpoint{1.011288in}{0.528000in}}%
\pgfpathlineto{\pgfqpoint{1.014691in}{0.528000in}}%
\pgfpathlineto{\pgfqpoint{0.859069in}{0.695020in}}%
\pgfpathlineto{\pgfqpoint{0.800000in}{0.758879in}}%
\pgfpathmoveto{\pgfqpoint{4.768000in}{2.744932in}}%
\pgfpathlineto{\pgfqpoint{4.447354in}{3.079308in}}%
\pgfpathlineto{\pgfqpoint{4.276029in}{3.253333in}}%
\pgfpathlineto{\pgfqpoint{3.936261in}{3.589333in}}%
\pgfpathlineto{\pgfqpoint{3.623735in}{3.888000in}}%
\pgfpathlineto{\pgfqpoint{3.445333in}{4.054171in}}%
\pgfpathlineto{\pgfqpoint{3.259434in}{4.224000in}}%
\pgfpathlineto{\pgfqpoint{3.255691in}{4.224000in}}%
\pgfpathlineto{\pgfqpoint{3.419388in}{4.074667in}}%
\pgfpathlineto{\pgfqpoint{3.592865in}{3.913418in}}%
\pgfpathlineto{\pgfqpoint{3.685818in}{3.826036in}}%
\pgfpathlineto{\pgfqpoint{4.009319in}{3.514667in}}%
\pgfpathlineto{\pgfqpoint{4.346438in}{3.178667in}}%
\pgfpathlineto{\pgfqpoint{4.671996in}{2.842667in}}%
\pgfpathlineto{\pgfqpoint{4.768000in}{2.741337in}}%
\pgfpathlineto{\pgfqpoint{4.768000in}{2.741337in}}%
\pgfusepath{fill}%
\end{pgfscope}%
\begin{pgfscope}%
\pgfpathrectangle{\pgfqpoint{0.800000in}{0.528000in}}{\pgfqpoint{3.968000in}{3.696000in}}%
\pgfusepath{clip}%
\pgfsetbuttcap%
\pgfsetroundjoin%
\definecolor{currentfill}{rgb}{0.144759,0.519093,0.556572}%
\pgfsetfillcolor{currentfill}%
\pgfsetlinewidth{0.000000pt}%
\definecolor{currentstroke}{rgb}{0.000000,0.000000,0.000000}%
\pgfsetstrokecolor{currentstroke}%
\pgfsetdash{}{0pt}%
\pgfpathmoveto{\pgfqpoint{0.800000in}{0.751517in}}%
\pgfpathlineto{\pgfqpoint{0.972856in}{0.565333in}}%
\pgfpathlineto{\pgfqpoint{1.007886in}{0.528000in}}%
\pgfpathlineto{\pgfqpoint{1.011288in}{0.528000in}}%
\pgfpathlineto{\pgfqpoint{0.857254in}{0.693330in}}%
\pgfpathlineto{\pgfqpoint{0.800000in}{0.755195in}}%
\pgfpathlineto{\pgfqpoint{0.800000in}{0.752000in}}%
\pgfpathmoveto{\pgfqpoint{4.768000in}{2.748526in}}%
\pgfpathlineto{\pgfqpoint{4.447354in}{3.082829in}}%
\pgfpathlineto{\pgfqpoint{4.279515in}{3.253333in}}%
\pgfpathlineto{\pgfqpoint{3.939798in}{3.589333in}}%
\pgfpathlineto{\pgfqpoint{3.627367in}{3.888000in}}%
\pgfpathlineto{\pgfqpoint{3.445333in}{4.057573in}}%
\pgfpathlineto{\pgfqpoint{3.263178in}{4.224000in}}%
\pgfpathlineto{\pgfqpoint{3.259434in}{4.224000in}}%
\pgfpathlineto{\pgfqpoint{3.423081in}{4.074667in}}%
\pgfpathlineto{\pgfqpoint{3.594703in}{3.915130in}}%
\pgfpathlineto{\pgfqpoint{3.685818in}{3.829454in}}%
\pgfpathlineto{\pgfqpoint{4.012833in}{3.514667in}}%
\pgfpathlineto{\pgfqpoint{4.349902in}{3.178667in}}%
\pgfpathlineto{\pgfqpoint{4.675411in}{2.842667in}}%
\pgfpathlineto{\pgfqpoint{4.768000in}{2.744932in}}%
\pgfpathlineto{\pgfqpoint{4.768000in}{2.744932in}}%
\pgfusepath{fill}%
\end{pgfscope}%
\begin{pgfscope}%
\pgfpathrectangle{\pgfqpoint{0.800000in}{0.528000in}}{\pgfqpoint{3.968000in}{3.696000in}}%
\pgfusepath{clip}%
\pgfsetbuttcap%
\pgfsetroundjoin%
\definecolor{currentfill}{rgb}{0.143343,0.522773,0.556295}%
\pgfsetfillcolor{currentfill}%
\pgfsetlinewidth{0.000000pt}%
\definecolor{currentstroke}{rgb}{0.000000,0.000000,0.000000}%
\pgfsetstrokecolor{currentstroke}%
\pgfsetdash{}{0pt}%
\pgfpathmoveto{\pgfqpoint{0.800000in}{0.747884in}}%
\pgfpathlineto{\pgfqpoint{0.969464in}{0.565333in}}%
\pgfpathlineto{\pgfqpoint{1.004484in}{0.528000in}}%
\pgfpathlineto{\pgfqpoint{1.007886in}{0.528000in}}%
\pgfpathlineto{\pgfqpoint{0.855439in}{0.691639in}}%
\pgfpathlineto{\pgfqpoint{0.800000in}{0.751517in}}%
\pgfpathmoveto{\pgfqpoint{4.768000in}{2.752121in}}%
\pgfpathlineto{\pgfqpoint{4.447354in}{3.086351in}}%
\pgfpathlineto{\pgfqpoint{4.283002in}{3.253333in}}%
\pgfpathlineto{\pgfqpoint{3.943334in}{3.589333in}}%
\pgfpathlineto{\pgfqpoint{3.630998in}{3.888000in}}%
\pgfpathlineto{\pgfqpoint{3.457997in}{4.049129in}}%
\pgfpathlineto{\pgfqpoint{3.365172in}{4.134610in}}%
\pgfpathlineto{\pgfqpoint{3.266922in}{4.224000in}}%
\pgfpathlineto{\pgfqpoint{3.263178in}{4.224000in}}%
\pgfpathlineto{\pgfqpoint{3.426774in}{4.074667in}}%
\pgfpathlineto{\pgfqpoint{3.605657in}{3.908405in}}%
\pgfpathlineto{\pgfqpoint{3.784854in}{3.738667in}}%
\pgfpathlineto{\pgfqpoint{3.952723in}{3.576609in}}%
\pgfpathlineto{\pgfqpoint{4.046545in}{3.485056in}}%
\pgfpathlineto{\pgfqpoint{4.206869in}{3.326231in}}%
\pgfpathlineto{\pgfqpoint{4.390082in}{3.141333in}}%
\pgfpathlineto{\pgfqpoint{4.550641in}{2.976207in}}%
\pgfpathlineto{\pgfqpoint{4.647758in}{2.875275in}}%
\pgfpathlineto{\pgfqpoint{4.768000in}{2.748526in}}%
\pgfpathlineto{\pgfqpoint{4.768000in}{2.748526in}}%
\pgfusepath{fill}%
\end{pgfscope}%
\begin{pgfscope}%
\pgfpathrectangle{\pgfqpoint{0.800000in}{0.528000in}}{\pgfqpoint{3.968000in}{3.696000in}}%
\pgfusepath{clip}%
\pgfsetbuttcap%
\pgfsetroundjoin%
\definecolor{currentfill}{rgb}{0.143343,0.522773,0.556295}%
\pgfsetfillcolor{currentfill}%
\pgfsetlinewidth{0.000000pt}%
\definecolor{currentstroke}{rgb}{0.000000,0.000000,0.000000}%
\pgfsetstrokecolor{currentstroke}%
\pgfsetdash{}{0pt}%
\pgfpathmoveto{\pgfqpoint{0.800000in}{0.744250in}}%
\pgfpathlineto{\pgfqpoint{0.966073in}{0.565333in}}%
\pgfpathlineto{\pgfqpoint{1.001081in}{0.528000in}}%
\pgfpathlineto{\pgfqpoint{1.004484in}{0.528000in}}%
\pgfpathlineto{\pgfqpoint{0.853624in}{0.689948in}}%
\pgfpathlineto{\pgfqpoint{0.800000in}{0.747884in}}%
\pgfpathmoveto{\pgfqpoint{4.768000in}{2.755716in}}%
\pgfpathlineto{\pgfqpoint{4.459137in}{3.077642in}}%
\pgfpathlineto{\pgfqpoint{4.360294in}{3.178667in}}%
\pgfpathlineto{\pgfqpoint{4.023374in}{3.514667in}}%
\pgfpathlineto{\pgfqpoint{3.846141in}{3.686758in}}%
\pgfpathlineto{\pgfqpoint{3.674220in}{3.850667in}}%
\pgfpathlineto{\pgfqpoint{3.352767in}{4.149333in}}%
\pgfpathlineto{\pgfqpoint{3.270666in}{4.224000in}}%
\pgfpathlineto{\pgfqpoint{3.266922in}{4.224000in}}%
\pgfpathlineto{\pgfqpoint{3.445333in}{4.060974in}}%
\pgfpathlineto{\pgfqpoint{3.788437in}{3.738667in}}%
\pgfpathlineto{\pgfqpoint{3.966384in}{3.566917in}}%
\pgfpathlineto{\pgfqpoint{4.133530in}{3.402667in}}%
\pgfpathlineto{\pgfqpoint{4.466533in}{3.066667in}}%
\pgfpathlineto{\pgfqpoint{4.768000in}{2.752121in}}%
\pgfpathlineto{\pgfqpoint{4.768000in}{2.752121in}}%
\pgfusepath{fill}%
\end{pgfscope}%
\begin{pgfscope}%
\pgfpathrectangle{\pgfqpoint{0.800000in}{0.528000in}}{\pgfqpoint{3.968000in}{3.696000in}}%
\pgfusepath{clip}%
\pgfsetbuttcap%
\pgfsetroundjoin%
\definecolor{currentfill}{rgb}{0.143343,0.522773,0.556295}%
\pgfsetfillcolor{currentfill}%
\pgfsetlinewidth{0.000000pt}%
\definecolor{currentstroke}{rgb}{0.000000,0.000000,0.000000}%
\pgfsetstrokecolor{currentstroke}%
\pgfsetdash{}{0pt}%
\pgfpathmoveto{\pgfqpoint{0.800000in}{0.740617in}}%
\pgfpathlineto{\pgfqpoint{0.962681in}{0.565333in}}%
\pgfpathlineto{\pgfqpoint{0.997718in}{0.528000in}}%
\pgfpathlineto{\pgfqpoint{1.001081in}{0.528000in}}%
\pgfpathlineto{\pgfqpoint{1.000404in}{0.528720in}}%
\pgfpathlineto{\pgfqpoint{0.800000in}{0.744250in}}%
\pgfpathmoveto{\pgfqpoint{4.768000in}{2.759310in}}%
\pgfpathlineto{\pgfqpoint{4.460925in}{3.079308in}}%
\pgfpathlineto{\pgfqpoint{4.363758in}{3.178667in}}%
\pgfpathlineto{\pgfqpoint{4.026888in}{3.514667in}}%
\pgfpathlineto{\pgfqpoint{3.846141in}{3.690188in}}%
\pgfpathlineto{\pgfqpoint{3.677839in}{3.850667in}}%
\pgfpathlineto{\pgfqpoint{3.356486in}{4.149333in}}%
\pgfpathlineto{\pgfqpoint{3.274410in}{4.224000in}}%
\pgfpathlineto{\pgfqpoint{3.270666in}{4.224000in}}%
\pgfpathlineto{\pgfqpoint{3.445333in}{4.064375in}}%
\pgfpathlineto{\pgfqpoint{3.792020in}{3.738667in}}%
\pgfpathlineto{\pgfqpoint{3.966384in}{3.570357in}}%
\pgfpathlineto{\pgfqpoint{4.137010in}{3.402667in}}%
\pgfpathlineto{\pgfqpoint{4.469964in}{3.066667in}}%
\pgfpathlineto{\pgfqpoint{4.768000in}{2.755716in}}%
\pgfpathlineto{\pgfqpoint{4.768000in}{2.755716in}}%
\pgfusepath{fill}%
\end{pgfscope}%
\begin{pgfscope}%
\pgfpathrectangle{\pgfqpoint{0.800000in}{0.528000in}}{\pgfqpoint{3.968000in}{3.696000in}}%
\pgfusepath{clip}%
\pgfsetbuttcap%
\pgfsetroundjoin%
\definecolor{currentfill}{rgb}{0.143343,0.522773,0.556295}%
\pgfsetfillcolor{currentfill}%
\pgfsetlinewidth{0.000000pt}%
\definecolor{currentstroke}{rgb}{0.000000,0.000000,0.000000}%
\pgfsetstrokecolor{currentstroke}%
\pgfsetdash{}{0pt}%
\pgfpathmoveto{\pgfqpoint{0.800000in}{0.736983in}}%
\pgfpathlineto{\pgfqpoint{0.960323in}{0.564245in}}%
\pgfpathlineto{\pgfqpoint{0.994363in}{0.528000in}}%
\pgfpathlineto{\pgfqpoint{0.997718in}{0.528000in}}%
\pgfpathlineto{\pgfqpoint{0.800000in}{0.740617in}}%
\pgfpathmoveto{\pgfqpoint{4.768000in}{2.762905in}}%
\pgfpathlineto{\pgfqpoint{4.476827in}{3.066667in}}%
\pgfpathlineto{\pgfqpoint{4.143970in}{3.402667in}}%
\pgfpathlineto{\pgfqpoint{3.966384in}{3.577235in}}%
\pgfpathlineto{\pgfqpoint{3.641891in}{3.888000in}}%
\pgfpathlineto{\pgfqpoint{3.463463in}{4.054220in}}%
\pgfpathlineto{\pgfqpoint{3.360204in}{4.149333in}}%
\pgfpathlineto{\pgfqpoint{3.278154in}{4.224000in}}%
\pgfpathlineto{\pgfqpoint{3.274410in}{4.224000in}}%
\pgfpathlineto{\pgfqpoint{3.445333in}{4.067776in}}%
\pgfpathlineto{\pgfqpoint{3.795603in}{3.738667in}}%
\pgfpathlineto{\pgfqpoint{3.966384in}{3.573796in}}%
\pgfpathlineto{\pgfqpoint{4.140490in}{3.402667in}}%
\pgfpathlineto{\pgfqpoint{4.473395in}{3.066667in}}%
\pgfpathlineto{\pgfqpoint{4.768000in}{2.759310in}}%
\pgfpathlineto{\pgfqpoint{4.768000in}{2.759310in}}%
\pgfusepath{fill}%
\end{pgfscope}%
\begin{pgfscope}%
\pgfpathrectangle{\pgfqpoint{0.800000in}{0.528000in}}{\pgfqpoint{3.968000in}{3.696000in}}%
\pgfusepath{clip}%
\pgfsetbuttcap%
\pgfsetroundjoin%
\definecolor{currentfill}{rgb}{0.141935,0.526453,0.555991}%
\pgfsetfillcolor{currentfill}%
\pgfsetlinewidth{0.000000pt}%
\definecolor{currentstroke}{rgb}{0.000000,0.000000,0.000000}%
\pgfsetstrokecolor{currentstroke}%
\pgfsetdash{}{0pt}%
\pgfpathmoveto{\pgfqpoint{0.800000in}{0.733350in}}%
\pgfpathlineto{\pgfqpoint{0.960323in}{0.560673in}}%
\pgfpathlineto{\pgfqpoint{0.991009in}{0.528000in}}%
\pgfpathlineto{\pgfqpoint{0.994363in}{0.528000in}}%
\pgfpathlineto{\pgfqpoint{0.800000in}{0.736983in}}%
\pgfpathmoveto{\pgfqpoint{4.768000in}{2.766500in}}%
\pgfpathlineto{\pgfqpoint{4.480258in}{3.066667in}}%
\pgfpathlineto{\pgfqpoint{4.147450in}{3.402667in}}%
\pgfpathlineto{\pgfqpoint{3.995773in}{3.552000in}}%
\pgfpathlineto{\pgfqpoint{3.824030in}{3.718071in}}%
\pgfpathlineto{\pgfqpoint{3.724469in}{3.813333in}}%
\pgfpathlineto{\pgfqpoint{3.404668in}{4.112000in}}%
\pgfpathlineto{\pgfqpoint{3.281897in}{4.224000in}}%
\pgfpathlineto{\pgfqpoint{3.278154in}{4.224000in}}%
\pgfpathlineto{\pgfqpoint{3.445333in}{4.071178in}}%
\pgfpathlineto{\pgfqpoint{3.782697in}{3.754238in}}%
\pgfpathlineto{\pgfqpoint{3.886222in}{3.654986in}}%
\pgfpathlineto{\pgfqpoint{4.218954in}{3.328000in}}%
\pgfpathlineto{\pgfqpoint{4.367548in}{3.178335in}}%
\pgfpathlineto{\pgfqpoint{4.549219in}{2.992000in}}%
\pgfpathlineto{\pgfqpoint{4.768000in}{2.762905in}}%
\pgfpathlineto{\pgfqpoint{4.768000in}{2.762905in}}%
\pgfusepath{fill}%
\end{pgfscope}%
\begin{pgfscope}%
\pgfpathrectangle{\pgfqpoint{0.800000in}{0.528000in}}{\pgfqpoint{3.968000in}{3.696000in}}%
\pgfusepath{clip}%
\pgfsetbuttcap%
\pgfsetroundjoin%
\definecolor{currentfill}{rgb}{0.141935,0.526453,0.555991}%
\pgfsetfillcolor{currentfill}%
\pgfsetlinewidth{0.000000pt}%
\definecolor{currentstroke}{rgb}{0.000000,0.000000,0.000000}%
\pgfsetstrokecolor{currentstroke}%
\pgfsetdash{}{0pt}%
\pgfpathmoveto{\pgfqpoint{0.800000in}{0.729716in}}%
\pgfpathlineto{\pgfqpoint{0.960323in}{0.557102in}}%
\pgfpathlineto{\pgfqpoint{0.987655in}{0.528000in}}%
\pgfpathlineto{\pgfqpoint{0.991009in}{0.528000in}}%
\pgfpathlineto{\pgfqpoint{0.800000in}{0.733350in}}%
\pgfpathmoveto{\pgfqpoint{4.768000in}{2.770066in}}%
\pgfpathlineto{\pgfqpoint{4.447313in}{3.104000in}}%
\pgfpathlineto{\pgfqpoint{4.113243in}{3.440000in}}%
\pgfpathlineto{\pgfqpoint{3.786305in}{3.757598in}}%
\pgfpathlineto{\pgfqpoint{3.685818in}{3.853351in}}%
\pgfpathlineto{\pgfqpoint{3.365172in}{4.151559in}}%
\pgfpathlineto{\pgfqpoint{3.285010in}{4.224000in}}%
\pgfpathlineto{\pgfqpoint{3.281897in}{4.224000in}}%
\pgfpathlineto{\pgfqpoint{3.445333in}{4.074579in}}%
\pgfpathlineto{\pgfqpoint{3.784501in}{3.755918in}}%
\pgfpathlineto{\pgfqpoint{3.886222in}{3.658419in}}%
\pgfpathlineto{\pgfqpoint{4.222412in}{3.328000in}}%
\pgfpathlineto{\pgfqpoint{4.370636in}{3.178667in}}%
\pgfpathlineto{\pgfqpoint{4.527515in}{3.017983in}}%
\pgfpathlineto{\pgfqpoint{4.695792in}{2.842667in}}%
\pgfpathlineto{\pgfqpoint{4.768000in}{2.766500in}}%
\pgfpathlineto{\pgfqpoint{4.768000in}{2.768000in}}%
\pgfpathlineto{\pgfqpoint{4.768000in}{2.768000in}}%
\pgfusepath{fill}%
\end{pgfscope}%
\begin{pgfscope}%
\pgfpathrectangle{\pgfqpoint{0.800000in}{0.528000in}}{\pgfqpoint{3.968000in}{3.696000in}}%
\pgfusepath{clip}%
\pgfsetbuttcap%
\pgfsetroundjoin%
\definecolor{currentfill}{rgb}{0.141935,0.526453,0.555991}%
\pgfsetfillcolor{currentfill}%
\pgfsetlinewidth{0.000000pt}%
\definecolor{currentstroke}{rgb}{0.000000,0.000000,0.000000}%
\pgfsetstrokecolor{currentstroke}%
\pgfsetdash{}{0pt}%
\pgfpathmoveto{\pgfqpoint{0.800000in}{0.726083in}}%
\pgfpathlineto{\pgfqpoint{0.960323in}{0.553530in}}%
\pgfpathlineto{\pgfqpoint{0.984301in}{0.528000in}}%
\pgfpathlineto{\pgfqpoint{0.987655in}{0.528000in}}%
\pgfpathlineto{\pgfqpoint{0.800000in}{0.729716in}}%
\pgfpathmoveto{\pgfqpoint{4.768000in}{2.773613in}}%
\pgfpathlineto{\pgfqpoint{4.447354in}{3.107434in}}%
\pgfpathlineto{\pgfqpoint{4.116734in}{3.440000in}}%
\pgfpathlineto{\pgfqpoint{3.788109in}{3.759279in}}%
\pgfpathlineto{\pgfqpoint{3.685818in}{3.856726in}}%
\pgfpathlineto{\pgfqpoint{3.365172in}{4.154912in}}%
\pgfpathlineto{\pgfqpoint{3.289317in}{4.224000in}}%
\pgfpathlineto{\pgfqpoint{3.285631in}{4.224000in}}%
\pgfpathlineto{\pgfqpoint{3.467107in}{4.057614in}}%
\pgfpathlineto{\pgfqpoint{3.569510in}{3.962667in}}%
\pgfpathlineto{\pgfqpoint{3.904640in}{3.643822in}}%
\pgfpathlineto{\pgfqpoint{4.006465in}{3.545002in}}%
\pgfpathlineto{\pgfqpoint{4.188472in}{3.365333in}}%
\pgfpathlineto{\pgfqpoint{4.519930in}{3.029333in}}%
\pgfpathlineto{\pgfqpoint{4.768000in}{2.770066in}}%
\pgfpathlineto{\pgfqpoint{4.768000in}{2.770066in}}%
\pgfusepath{fill}%
\end{pgfscope}%
\begin{pgfscope}%
\pgfpathrectangle{\pgfqpoint{0.800000in}{0.528000in}}{\pgfqpoint{3.968000in}{3.696000in}}%
\pgfusepath{clip}%
\pgfsetbuttcap%
\pgfsetroundjoin%
\definecolor{currentfill}{rgb}{0.140536,0.530132,0.555659}%
\pgfsetfillcolor{currentfill}%
\pgfsetlinewidth{0.000000pt}%
\definecolor{currentstroke}{rgb}{0.000000,0.000000,0.000000}%
\pgfsetstrokecolor{currentstroke}%
\pgfsetdash{}{0pt}%
\pgfpathmoveto{\pgfqpoint{0.800000in}{0.722449in}}%
\pgfpathlineto{\pgfqpoint{0.960323in}{0.549959in}}%
\pgfpathlineto{\pgfqpoint{0.980946in}{0.528000in}}%
\pgfpathlineto{\pgfqpoint{0.984301in}{0.528000in}}%
\pgfpathlineto{\pgfqpoint{0.800000in}{0.726083in}}%
\pgfpathmoveto{\pgfqpoint{4.768000in}{2.777159in}}%
\pgfpathlineto{\pgfqpoint{4.447354in}{3.110909in}}%
\pgfpathlineto{\pgfqpoint{4.120225in}{3.440000in}}%
\pgfpathlineto{\pgfqpoint{3.774358in}{3.776000in}}%
\pgfpathlineto{\pgfqpoint{3.456149in}{4.074667in}}%
\pgfpathlineto{\pgfqpoint{3.293003in}{4.224000in}}%
\pgfpathlineto{\pgfqpoint{3.289317in}{4.224000in}}%
\pgfpathlineto{\pgfqpoint{3.468929in}{4.059312in}}%
\pgfpathlineto{\pgfqpoint{3.565576in}{3.969713in}}%
\pgfpathlineto{\pgfqpoint{3.731597in}{3.813333in}}%
\pgfpathlineto{\pgfqpoint{4.078912in}{3.477333in}}%
\pgfpathlineto{\pgfqpoint{4.377465in}{3.178667in}}%
\pgfpathlineto{\pgfqpoint{4.527515in}{3.025038in}}%
\pgfpathlineto{\pgfqpoint{4.702526in}{2.842667in}}%
\pgfpathlineto{\pgfqpoint{4.768000in}{2.773613in}}%
\pgfpathlineto{\pgfqpoint{4.768000in}{2.773613in}}%
\pgfusepath{fill}%
\end{pgfscope}%
\begin{pgfscope}%
\pgfpathrectangle{\pgfqpoint{0.800000in}{0.528000in}}{\pgfqpoint{3.968000in}{3.696000in}}%
\pgfusepath{clip}%
\pgfsetbuttcap%
\pgfsetroundjoin%
\definecolor{currentfill}{rgb}{0.140536,0.530132,0.555659}%
\pgfsetfillcolor{currentfill}%
\pgfsetlinewidth{0.000000pt}%
\definecolor{currentstroke}{rgb}{0.000000,0.000000,0.000000}%
\pgfsetstrokecolor{currentstroke}%
\pgfsetdash{}{0pt}%
\pgfpathmoveto{\pgfqpoint{0.800000in}{0.718816in}}%
\pgfpathlineto{\pgfqpoint{0.960323in}{0.546387in}}%
\pgfpathlineto{\pgfqpoint{0.977592in}{0.528000in}}%
\pgfpathlineto{\pgfqpoint{0.980946in}{0.528000in}}%
\pgfpathlineto{\pgfqpoint{0.800000in}{0.722449in}}%
\pgfpathmoveto{\pgfqpoint{4.768000in}{2.780706in}}%
\pgfpathlineto{\pgfqpoint{4.447354in}{3.114384in}}%
\pgfpathlineto{\pgfqpoint{4.123717in}{3.440000in}}%
\pgfpathlineto{\pgfqpoint{3.777900in}{3.776000in}}%
\pgfpathlineto{\pgfqpoint{3.459785in}{4.074667in}}%
\pgfpathlineto{\pgfqpoint{3.296689in}{4.224000in}}%
\pgfpathlineto{\pgfqpoint{3.293003in}{4.224000in}}%
\pgfpathlineto{\pgfqpoint{3.470751in}{4.061009in}}%
\pgfpathlineto{\pgfqpoint{3.565576in}{3.973079in}}%
\pgfpathlineto{\pgfqpoint{3.735150in}{3.813333in}}%
\pgfpathlineto{\pgfqpoint{4.046545in}{3.512618in}}%
\pgfpathlineto{\pgfqpoint{4.232785in}{3.328000in}}%
\pgfpathlineto{\pgfqpoint{4.407273in}{3.151825in}}%
\pgfpathlineto{\pgfqpoint{4.727919in}{2.819490in}}%
\pgfpathlineto{\pgfqpoint{4.768000in}{2.777159in}}%
\pgfpathlineto{\pgfqpoint{4.768000in}{2.777159in}}%
\pgfusepath{fill}%
\end{pgfscope}%
\begin{pgfscope}%
\pgfpathrectangle{\pgfqpoint{0.800000in}{0.528000in}}{\pgfqpoint{3.968000in}{3.696000in}}%
\pgfusepath{clip}%
\pgfsetbuttcap%
\pgfsetroundjoin%
\definecolor{currentfill}{rgb}{0.140536,0.530132,0.555659}%
\pgfsetfillcolor{currentfill}%
\pgfsetlinewidth{0.000000pt}%
\definecolor{currentstroke}{rgb}{0.000000,0.000000,0.000000}%
\pgfsetstrokecolor{currentstroke}%
\pgfsetdash{}{0pt}%
\pgfpathmoveto{\pgfqpoint{0.800000in}{0.715182in}}%
\pgfpathlineto{\pgfqpoint{0.960323in}{0.542816in}}%
\pgfpathlineto{\pgfqpoint{0.974238in}{0.528000in}}%
\pgfpathlineto{\pgfqpoint{0.977592in}{0.528000in}}%
\pgfpathlineto{\pgfqpoint{0.800000in}{0.718816in}}%
\pgfpathmoveto{\pgfqpoint{4.768000in}{2.784252in}}%
\pgfpathlineto{\pgfqpoint{4.447354in}{3.117860in}}%
\pgfpathlineto{\pgfqpoint{4.126707in}{3.440489in}}%
\pgfpathlineto{\pgfqpoint{3.781441in}{3.776000in}}%
\pgfpathlineto{\pgfqpoint{3.463422in}{4.074667in}}%
\pgfpathlineto{\pgfqpoint{3.300375in}{4.224000in}}%
\pgfpathlineto{\pgfqpoint{3.296689in}{4.224000in}}%
\pgfpathlineto{\pgfqpoint{3.472573in}{4.062706in}}%
\pgfpathlineto{\pgfqpoint{3.565576in}{3.976446in}}%
\pgfpathlineto{\pgfqpoint{3.738703in}{3.813333in}}%
\pgfpathlineto{\pgfqpoint{4.073276in}{3.489768in}}%
\pgfpathlineto{\pgfqpoint{4.236243in}{3.328000in}}%
\pgfpathlineto{\pgfqpoint{4.407273in}{3.155297in}}%
\pgfpathlineto{\pgfqpoint{4.727919in}{2.823034in}}%
\pgfpathlineto{\pgfqpoint{4.768000in}{2.780706in}}%
\pgfpathlineto{\pgfqpoint{4.768000in}{2.780706in}}%
\pgfusepath{fill}%
\end{pgfscope}%
\begin{pgfscope}%
\pgfpathrectangle{\pgfqpoint{0.800000in}{0.528000in}}{\pgfqpoint{3.968000in}{3.696000in}}%
\pgfusepath{clip}%
\pgfsetbuttcap%
\pgfsetroundjoin%
\definecolor{currentfill}{rgb}{0.140536,0.530132,0.555659}%
\pgfsetfillcolor{currentfill}%
\pgfsetlinewidth{0.000000pt}%
\definecolor{currentstroke}{rgb}{0.000000,0.000000,0.000000}%
\pgfsetstrokecolor{currentstroke}%
\pgfsetdash{}{0pt}%
\pgfpathmoveto{\pgfqpoint{0.800000in}{0.711591in}}%
\pgfpathlineto{\pgfqpoint{0.970883in}{0.528000in}}%
\pgfpathlineto{\pgfqpoint{0.974238in}{0.528000in}}%
\pgfpathlineto{\pgfqpoint{0.800000in}{0.715182in}}%
\pgfpathlineto{\pgfqpoint{0.800000in}{0.714667in}}%
\pgfpathmoveto{\pgfqpoint{4.768000in}{2.787799in}}%
\pgfpathlineto{\pgfqpoint{4.447354in}{3.121335in}}%
\pgfpathlineto{\pgfqpoint{4.126707in}{3.443895in}}%
\pgfpathlineto{\pgfqpoint{3.784983in}{3.776000in}}%
\pgfpathlineto{\pgfqpoint{3.467058in}{4.074667in}}%
\pgfpathlineto{\pgfqpoint{3.304060in}{4.224000in}}%
\pgfpathlineto{\pgfqpoint{3.300375in}{4.224000in}}%
\pgfpathlineto{\pgfqpoint{3.485414in}{4.054340in}}%
\pgfpathlineto{\pgfqpoint{3.663408in}{3.888000in}}%
\pgfpathlineto{\pgfqpoint{3.975034in}{3.589333in}}%
\pgfpathlineto{\pgfqpoint{4.134913in}{3.432357in}}%
\pgfpathlineto{\pgfqpoint{4.313983in}{3.253333in}}%
\pgfpathlineto{\pgfqpoint{4.625393in}{2.933835in}}%
\pgfpathlineto{\pgfqpoint{4.719760in}{2.835067in}}%
\pgfpathlineto{\pgfqpoint{4.768000in}{2.784252in}}%
\pgfpathlineto{\pgfqpoint{4.768000in}{2.784252in}}%
\pgfusepath{fill}%
\end{pgfscope}%
\begin{pgfscope}%
\pgfpathrectangle{\pgfqpoint{0.800000in}{0.528000in}}{\pgfqpoint{3.968000in}{3.696000in}}%
\pgfusepath{clip}%
\pgfsetbuttcap%
\pgfsetroundjoin%
\definecolor{currentfill}{rgb}{0.139147,0.533812,0.555298}%
\pgfsetfillcolor{currentfill}%
\pgfsetlinewidth{0.000000pt}%
\definecolor{currentstroke}{rgb}{0.000000,0.000000,0.000000}%
\pgfsetstrokecolor{currentstroke}%
\pgfsetdash{}{0pt}%
\pgfpathmoveto{\pgfqpoint{0.800000in}{0.708007in}}%
\pgfpathlineto{\pgfqpoint{0.967529in}{0.528000in}}%
\pgfpathlineto{\pgfqpoint{0.970883in}{0.528000in}}%
\pgfpathlineto{\pgfqpoint{0.800000in}{0.711591in}}%
\pgfpathmoveto{\pgfqpoint{4.768000in}{2.791345in}}%
\pgfpathlineto{\pgfqpoint{4.447354in}{3.124810in}}%
\pgfpathlineto{\pgfqpoint{4.126707in}{3.447302in}}%
\pgfpathlineto{\pgfqpoint{3.788524in}{3.776000in}}%
\pgfpathlineto{\pgfqpoint{3.470694in}{4.074667in}}%
\pgfpathlineto{\pgfqpoint{3.307746in}{4.224000in}}%
\pgfpathlineto{\pgfqpoint{3.304060in}{4.224000in}}%
\pgfpathlineto{\pgfqpoint{3.485414in}{4.057701in}}%
\pgfpathlineto{\pgfqpoint{3.666985in}{3.888000in}}%
\pgfpathlineto{\pgfqpoint{3.978519in}{3.589333in}}%
\pgfpathlineto{\pgfqpoint{4.130641in}{3.440000in}}%
\pgfpathlineto{\pgfqpoint{4.317419in}{3.253333in}}%
\pgfpathlineto{\pgfqpoint{4.627172in}{2.935492in}}%
\pgfpathlineto{\pgfqpoint{4.727919in}{2.830120in}}%
\pgfpathlineto{\pgfqpoint{4.768000in}{2.787799in}}%
\pgfpathlineto{\pgfqpoint{4.768000in}{2.787799in}}%
\pgfusepath{fill}%
\end{pgfscope}%
\begin{pgfscope}%
\pgfpathrectangle{\pgfqpoint{0.800000in}{0.528000in}}{\pgfqpoint{3.968000in}{3.696000in}}%
\pgfusepath{clip}%
\pgfsetbuttcap%
\pgfsetroundjoin%
\definecolor{currentfill}{rgb}{0.139147,0.533812,0.555298}%
\pgfsetfillcolor{currentfill}%
\pgfsetlinewidth{0.000000pt}%
\definecolor{currentstroke}{rgb}{0.000000,0.000000,0.000000}%
\pgfsetstrokecolor{currentstroke}%
\pgfsetdash{}{0pt}%
\pgfpathmoveto{\pgfqpoint{0.800000in}{0.704422in}}%
\pgfpathlineto{\pgfqpoint{0.964175in}{0.528000in}}%
\pgfpathlineto{\pgfqpoint{0.967529in}{0.528000in}}%
\pgfpathlineto{\pgfqpoint{0.800000in}{0.708007in}}%
\pgfpathmoveto{\pgfqpoint{4.768000in}{2.794892in}}%
\pgfpathlineto{\pgfqpoint{4.459699in}{3.115499in}}%
\pgfpathlineto{\pgfqpoint{4.361189in}{3.216000in}}%
\pgfpathlineto{\pgfqpoint{4.023718in}{3.552000in}}%
\pgfpathlineto{\pgfqpoint{3.846141in}{3.724197in}}%
\pgfpathlineto{\pgfqpoint{3.514626in}{4.037333in}}%
\pgfpathlineto{\pgfqpoint{3.325091in}{4.211586in}}%
\pgfpathlineto{\pgfqpoint{3.311432in}{4.224000in}}%
\pgfpathlineto{\pgfqpoint{3.307746in}{4.224000in}}%
\pgfpathlineto{\pgfqpoint{3.485414in}{4.061062in}}%
\pgfpathlineto{\pgfqpoint{3.670561in}{3.888000in}}%
\pgfpathlineto{\pgfqpoint{3.982004in}{3.589333in}}%
\pgfpathlineto{\pgfqpoint{4.134082in}{3.440000in}}%
\pgfpathlineto{\pgfqpoint{4.287030in}{3.287426in}}%
\pgfpathlineto{\pgfqpoint{4.467671in}{3.104000in}}%
\pgfpathlineto{\pgfqpoint{4.768000in}{2.791345in}}%
\pgfpathlineto{\pgfqpoint{4.768000in}{2.791345in}}%
\pgfusepath{fill}%
\end{pgfscope}%
\begin{pgfscope}%
\pgfpathrectangle{\pgfqpoint{0.800000in}{0.528000in}}{\pgfqpoint{3.968000in}{3.696000in}}%
\pgfusepath{clip}%
\pgfsetbuttcap%
\pgfsetroundjoin%
\definecolor{currentfill}{rgb}{0.139147,0.533812,0.555298}%
\pgfsetfillcolor{currentfill}%
\pgfsetlinewidth{0.000000pt}%
\definecolor{currentstroke}{rgb}{0.000000,0.000000,0.000000}%
\pgfsetstrokecolor{currentstroke}%
\pgfsetdash{}{0pt}%
\pgfpathmoveto{\pgfqpoint{0.800000in}{0.700838in}}%
\pgfpathlineto{\pgfqpoint{0.960821in}{0.528000in}}%
\pgfpathlineto{\pgfqpoint{0.964175in}{0.528000in}}%
\pgfpathlineto{\pgfqpoint{0.800000in}{0.704422in}}%
\pgfpathmoveto{\pgfqpoint{4.768000in}{2.798438in}}%
\pgfpathlineto{\pgfqpoint{4.461466in}{3.117145in}}%
\pgfpathlineto{\pgfqpoint{4.364614in}{3.216000in}}%
\pgfpathlineto{\pgfqpoint{4.027192in}{3.552000in}}%
\pgfpathlineto{\pgfqpoint{3.846141in}{3.727583in}}%
\pgfpathlineto{\pgfqpoint{3.518250in}{4.037333in}}%
\pgfpathlineto{\pgfqpoint{3.340336in}{4.200866in}}%
\pgfpathlineto{\pgfqpoint{3.315118in}{4.224000in}}%
\pgfpathlineto{\pgfqpoint{3.311432in}{4.224000in}}%
\pgfpathlineto{\pgfqpoint{3.485414in}{4.064423in}}%
\pgfpathlineto{\pgfqpoint{3.674138in}{3.888000in}}%
\pgfpathlineto{\pgfqpoint{3.985489in}{3.589333in}}%
\pgfpathlineto{\pgfqpoint{4.166788in}{3.410997in}}%
\pgfpathlineto{\pgfqpoint{4.487434in}{3.087218in}}%
\pgfpathlineto{\pgfqpoint{4.768000in}{2.794892in}}%
\pgfpathlineto{\pgfqpoint{4.768000in}{2.794892in}}%
\pgfusepath{fill}%
\end{pgfscope}%
\begin{pgfscope}%
\pgfpathrectangle{\pgfqpoint{0.800000in}{0.528000in}}{\pgfqpoint{3.968000in}{3.696000in}}%
\pgfusepath{clip}%
\pgfsetbuttcap%
\pgfsetroundjoin%
\definecolor{currentfill}{rgb}{0.139147,0.533812,0.555298}%
\pgfsetfillcolor{currentfill}%
\pgfsetlinewidth{0.000000pt}%
\definecolor{currentstroke}{rgb}{0.000000,0.000000,0.000000}%
\pgfsetstrokecolor{currentstroke}%
\pgfsetdash{}{0pt}%
\pgfpathmoveto{\pgfqpoint{0.800000in}{0.697254in}}%
\pgfpathlineto{\pgfqpoint{0.957506in}{0.528000in}}%
\pgfpathlineto{\pgfqpoint{0.960821in}{0.528000in}}%
\pgfpathlineto{\pgfqpoint{0.960323in}{0.528530in}}%
\pgfpathlineto{\pgfqpoint{0.800000in}{0.700838in}}%
\pgfpathmoveto{\pgfqpoint{4.768000in}{2.801984in}}%
\pgfpathlineto{\pgfqpoint{4.477850in}{3.104000in}}%
\pgfpathlineto{\pgfqpoint{4.144403in}{3.440000in}}%
\pgfpathlineto{\pgfqpoint{3.966384in}{3.614738in}}%
\pgfpathlineto{\pgfqpoint{3.641683in}{3.925333in}}%
\pgfpathlineto{\pgfqpoint{3.318803in}{4.224000in}}%
\pgfpathlineto{\pgfqpoint{3.315118in}{4.224000in}}%
\pgfpathlineto{\pgfqpoint{3.485414in}{4.067784in}}%
\pgfpathlineto{\pgfqpoint{3.677714in}{3.888000in}}%
\pgfpathlineto{\pgfqpoint{3.988973in}{3.589333in}}%
\pgfpathlineto{\pgfqpoint{4.166788in}{3.414406in}}%
\pgfpathlineto{\pgfqpoint{4.487434in}{3.090696in}}%
\pgfpathlineto{\pgfqpoint{4.768000in}{2.798438in}}%
\pgfpathlineto{\pgfqpoint{4.768000in}{2.798438in}}%
\pgfusepath{fill}%
\end{pgfscope}%
\begin{pgfscope}%
\pgfpathrectangle{\pgfqpoint{0.800000in}{0.528000in}}{\pgfqpoint{3.968000in}{3.696000in}}%
\pgfusepath{clip}%
\pgfsetbuttcap%
\pgfsetroundjoin%
\definecolor{currentfill}{rgb}{0.137770,0.537492,0.554906}%
\pgfsetfillcolor{currentfill}%
\pgfsetlinewidth{0.000000pt}%
\definecolor{currentstroke}{rgb}{0.000000,0.000000,0.000000}%
\pgfsetstrokecolor{currentstroke}%
\pgfsetdash{}{0pt}%
\pgfpathmoveto{\pgfqpoint{0.800000in}{0.693669in}}%
\pgfpathlineto{\pgfqpoint{0.954199in}{0.528000in}}%
\pgfpathlineto{\pgfqpoint{0.957506in}{0.528000in}}%
\pgfpathlineto{\pgfqpoint{0.800000in}{0.697254in}}%
\pgfpathmoveto{\pgfqpoint{4.768000in}{2.805528in}}%
\pgfpathlineto{\pgfqpoint{4.589816in}{2.992000in}}%
\pgfpathlineto{\pgfqpoint{4.260253in}{3.328000in}}%
\pgfpathlineto{\pgfqpoint{4.086626in}{3.500471in}}%
\pgfpathlineto{\pgfqpoint{3.919101in}{3.664000in}}%
\pgfpathlineto{\pgfqpoint{3.605513in}{3.962667in}}%
\pgfpathlineto{\pgfqpoint{3.424789in}{4.130197in}}%
\pgfpathlineto{\pgfqpoint{3.322489in}{4.224000in}}%
\pgfpathlineto{\pgfqpoint{3.318803in}{4.224000in}}%
\pgfpathlineto{\pgfqpoint{3.485414in}{4.071145in}}%
\pgfpathlineto{\pgfqpoint{3.681291in}{3.888000in}}%
\pgfpathlineto{\pgfqpoint{3.992458in}{3.589333in}}%
\pgfpathlineto{\pgfqpoint{4.166788in}{3.417816in}}%
\pgfpathlineto{\pgfqpoint{4.487434in}{3.094174in}}%
\pgfpathlineto{\pgfqpoint{4.768000in}{2.801984in}}%
\pgfpathlineto{\pgfqpoint{4.768000in}{2.805333in}}%
\pgfpathlineto{\pgfqpoint{4.768000in}{2.805333in}}%
\pgfusepath{fill}%
\end{pgfscope}%
\begin{pgfscope}%
\pgfpathrectangle{\pgfqpoint{0.800000in}{0.528000in}}{\pgfqpoint{3.968000in}{3.696000in}}%
\pgfusepath{clip}%
\pgfsetbuttcap%
\pgfsetroundjoin%
\definecolor{currentfill}{rgb}{0.137770,0.537492,0.554906}%
\pgfsetfillcolor{currentfill}%
\pgfsetlinewidth{0.000000pt}%
\definecolor{currentstroke}{rgb}{0.000000,0.000000,0.000000}%
\pgfsetstrokecolor{currentstroke}%
\pgfsetdash{}{0pt}%
\pgfpathmoveto{\pgfqpoint{0.800000in}{0.690085in}}%
\pgfpathlineto{\pgfqpoint{0.950891in}{0.528000in}}%
\pgfpathlineto{\pgfqpoint{0.954199in}{0.528000in}}%
\pgfpathlineto{\pgfqpoint{0.800000in}{0.693669in}}%
\pgfpathmoveto{\pgfqpoint{4.768000in}{2.809028in}}%
\pgfpathlineto{\pgfqpoint{4.593177in}{2.992000in}}%
\pgfpathlineto{\pgfqpoint{4.263661in}{3.328000in}}%
\pgfpathlineto{\pgfqpoint{4.086626in}{3.503875in}}%
\pgfpathlineto{\pgfqpoint{3.922608in}{3.664000in}}%
\pgfpathlineto{\pgfqpoint{3.605657in}{3.965861in}}%
\pgfpathlineto{\pgfqpoint{3.407833in}{4.149333in}}%
\pgfpathlineto{\pgfqpoint{3.325091in}{4.224000in}}%
\pgfpathlineto{\pgfqpoint{3.322489in}{4.224000in}}%
\pgfpathlineto{\pgfqpoint{3.485414in}{4.074506in}}%
\pgfpathlineto{\pgfqpoint{3.645737in}{3.924896in}}%
\pgfpathlineto{\pgfqpoint{3.841649in}{3.738667in}}%
\pgfpathlineto{\pgfqpoint{4.147844in}{3.440000in}}%
\pgfpathlineto{\pgfqpoint{4.327111in}{3.260789in}}%
\pgfpathlineto{\pgfqpoint{4.487434in}{3.097653in}}%
\pgfpathlineto{\pgfqpoint{4.768000in}{2.805528in}}%
\pgfpathlineto{\pgfqpoint{4.768000in}{2.805528in}}%
\pgfusepath{fill}%
\end{pgfscope}%
\begin{pgfscope}%
\pgfpathrectangle{\pgfqpoint{0.800000in}{0.528000in}}{\pgfqpoint{3.968000in}{3.696000in}}%
\pgfusepath{clip}%
\pgfsetbuttcap%
\pgfsetroundjoin%
\definecolor{currentfill}{rgb}{0.137770,0.537492,0.554906}%
\pgfsetfillcolor{currentfill}%
\pgfsetlinewidth{0.000000pt}%
\definecolor{currentstroke}{rgb}{0.000000,0.000000,0.000000}%
\pgfsetstrokecolor{currentstroke}%
\pgfsetdash{}{0pt}%
\pgfpathmoveto{\pgfqpoint{0.800000in}{0.686501in}}%
\pgfpathlineto{\pgfqpoint{0.947583in}{0.528000in}}%
\pgfpathlineto{\pgfqpoint{0.950891in}{0.528000in}}%
\pgfpathlineto{\pgfqpoint{0.800000in}{0.690085in}}%
\pgfpathmoveto{\pgfqpoint{4.768000in}{2.812527in}}%
\pgfpathlineto{\pgfqpoint{4.596539in}{2.992000in}}%
\pgfpathlineto{\pgfqpoint{4.267070in}{3.328000in}}%
\pgfpathlineto{\pgfqpoint{4.086626in}{3.507279in}}%
\pgfpathlineto{\pgfqpoint{3.926115in}{3.664000in}}%
\pgfpathlineto{\pgfqpoint{3.572705in}{4.000000in}}%
\pgfpathlineto{\pgfqpoint{3.405253in}{4.155018in}}%
\pgfpathlineto{\pgfqpoint{3.329788in}{4.224000in}}%
\pgfpathlineto{\pgfqpoint{3.326158in}{4.224000in}}%
\pgfpathlineto{\pgfqpoint{3.648814in}{3.925333in}}%
\pgfpathlineto{\pgfqpoint{3.812101in}{3.770373in}}%
\pgfpathlineto{\pgfqpoint{3.999427in}{3.589333in}}%
\pgfpathlineto{\pgfqpoint{4.166788in}{3.424635in}}%
\pgfpathlineto{\pgfqpoint{4.487434in}{3.101131in}}%
\pgfpathlineto{\pgfqpoint{4.768000in}{2.809028in}}%
\pgfpathlineto{\pgfqpoint{4.768000in}{2.809028in}}%
\pgfusepath{fill}%
\end{pgfscope}%
\begin{pgfscope}%
\pgfpathrectangle{\pgfqpoint{0.800000in}{0.528000in}}{\pgfqpoint{3.968000in}{3.696000in}}%
\pgfusepath{clip}%
\pgfsetbuttcap%
\pgfsetroundjoin%
\definecolor{currentfill}{rgb}{0.136408,0.541173,0.554483}%
\pgfsetfillcolor{currentfill}%
\pgfsetlinewidth{0.000000pt}%
\definecolor{currentstroke}{rgb}{0.000000,0.000000,0.000000}%
\pgfsetstrokecolor{currentstroke}%
\pgfsetdash{}{0pt}%
\pgfpathmoveto{\pgfqpoint{0.800000in}{0.682916in}}%
\pgfpathlineto{\pgfqpoint{0.944276in}{0.528000in}}%
\pgfpathlineto{\pgfqpoint{0.947583in}{0.528000in}}%
\pgfpathlineto{\pgfqpoint{0.800000in}{0.686501in}}%
\pgfpathmoveto{\pgfqpoint{4.768000in}{2.816027in}}%
\pgfpathlineto{\pgfqpoint{4.599900in}{2.992000in}}%
\pgfpathlineto{\pgfqpoint{4.270478in}{3.328000in}}%
\pgfpathlineto{\pgfqpoint{4.086626in}{3.510683in}}%
\pgfpathlineto{\pgfqpoint{3.926303in}{3.667170in}}%
\pgfpathlineto{\pgfqpoint{3.576263in}{4.000000in}}%
\pgfpathlineto{\pgfqpoint{3.405253in}{4.158331in}}%
\pgfpathlineto{\pgfqpoint{3.333417in}{4.224000in}}%
\pgfpathlineto{\pgfqpoint{3.329788in}{4.224000in}}%
\pgfpathlineto{\pgfqpoint{3.652349in}{3.925333in}}%
\pgfpathlineto{\pgfqpoint{3.809719in}{3.776000in}}%
\pgfpathlineto{\pgfqpoint{3.966384in}{3.624924in}}%
\pgfpathlineto{\pgfqpoint{4.154725in}{3.440000in}}%
\pgfpathlineto{\pgfqpoint{4.327111in}{3.267631in}}%
\pgfpathlineto{\pgfqpoint{4.488020in}{3.104000in}}%
\pgfpathlineto{\pgfqpoint{4.768000in}{2.812527in}}%
\pgfpathlineto{\pgfqpoint{4.768000in}{2.812527in}}%
\pgfusepath{fill}%
\end{pgfscope}%
\begin{pgfscope}%
\pgfpathrectangle{\pgfqpoint{0.800000in}{0.528000in}}{\pgfqpoint{3.968000in}{3.696000in}}%
\pgfusepath{clip}%
\pgfsetbuttcap%
\pgfsetroundjoin%
\definecolor{currentfill}{rgb}{0.136408,0.541173,0.554483}%
\pgfsetfillcolor{currentfill}%
\pgfsetlinewidth{0.000000pt}%
\definecolor{currentstroke}{rgb}{0.000000,0.000000,0.000000}%
\pgfsetstrokecolor{currentstroke}%
\pgfsetdash{}{0pt}%
\pgfpathmoveto{\pgfqpoint{0.800000in}{0.679332in}}%
\pgfpathlineto{\pgfqpoint{0.940968in}{0.528000in}}%
\pgfpathlineto{\pgfqpoint{0.944276in}{0.528000in}}%
\pgfpathlineto{\pgfqpoint{0.800000in}{0.682916in}}%
\pgfpathmoveto{\pgfqpoint{4.768000in}{2.819527in}}%
\pgfpathlineto{\pgfqpoint{4.603262in}{2.992000in}}%
\pgfpathlineto{\pgfqpoint{4.273886in}{3.328000in}}%
\pgfpathlineto{\pgfqpoint{4.123894in}{3.477333in}}%
\pgfpathlineto{\pgfqpoint{3.966384in}{3.631650in}}%
\pgfpathlineto{\pgfqpoint{3.777614in}{3.813333in}}%
\pgfpathlineto{\pgfqpoint{3.605657in}{3.975842in}}%
\pgfpathlineto{\pgfqpoint{3.418649in}{4.149333in}}%
\pgfpathlineto{\pgfqpoint{3.337047in}{4.224000in}}%
\pgfpathlineto{\pgfqpoint{3.333417in}{4.224000in}}%
\pgfpathlineto{\pgfqpoint{3.655884in}{3.925333in}}%
\pgfpathlineto{\pgfqpoint{3.846141in}{3.744442in}}%
\pgfpathlineto{\pgfqpoint{4.166788in}{3.431455in}}%
\pgfpathlineto{\pgfqpoint{4.491365in}{3.104000in}}%
\pgfpathlineto{\pgfqpoint{4.768000in}{2.816027in}}%
\pgfpathlineto{\pgfqpoint{4.768000in}{2.816027in}}%
\pgfusepath{fill}%
\end{pgfscope}%
\begin{pgfscope}%
\pgfpathrectangle{\pgfqpoint{0.800000in}{0.528000in}}{\pgfqpoint{3.968000in}{3.696000in}}%
\pgfusepath{clip}%
\pgfsetbuttcap%
\pgfsetroundjoin%
\definecolor{currentfill}{rgb}{0.136408,0.541173,0.554483}%
\pgfsetfillcolor{currentfill}%
\pgfsetlinewidth{0.000000pt}%
\definecolor{currentstroke}{rgb}{0.000000,0.000000,0.000000}%
\pgfsetstrokecolor{currentstroke}%
\pgfsetdash{}{0pt}%
\pgfpathmoveto{\pgfqpoint{0.800000in}{0.675769in}}%
\pgfpathlineto{\pgfqpoint{0.937661in}{0.528000in}}%
\pgfpathlineto{\pgfqpoint{0.940968in}{0.528000in}}%
\pgfpathlineto{\pgfqpoint{0.800000in}{0.679332in}}%
\pgfpathlineto{\pgfqpoint{0.800000in}{0.677333in}}%
\pgfpathmoveto{\pgfqpoint{4.768000in}{2.823026in}}%
\pgfpathlineto{\pgfqpoint{4.606623in}{2.992000in}}%
\pgfpathlineto{\pgfqpoint{4.277294in}{3.328000in}}%
\pgfpathlineto{\pgfqpoint{4.126707in}{3.477955in}}%
\pgfpathlineto{\pgfqpoint{3.936487in}{3.664000in}}%
\pgfpathlineto{\pgfqpoint{3.765980in}{3.827765in}}%
\pgfpathlineto{\pgfqpoint{3.583379in}{4.000000in}}%
\pgfpathlineto{\pgfqpoint{3.405253in}{4.164958in}}%
\pgfpathlineto{\pgfqpoint{3.340676in}{4.224000in}}%
\pgfpathlineto{\pgfqpoint{3.337047in}{4.224000in}}%
\pgfpathlineto{\pgfqpoint{3.659418in}{3.925333in}}%
\pgfpathlineto{\pgfqpoint{3.846141in}{3.747786in}}%
\pgfpathlineto{\pgfqpoint{4.166788in}{3.434865in}}%
\pgfpathlineto{\pgfqpoint{4.494710in}{3.104000in}}%
\pgfpathlineto{\pgfqpoint{4.768000in}{2.819527in}}%
\pgfpathlineto{\pgfqpoint{4.768000in}{2.819527in}}%
\pgfusepath{fill}%
\end{pgfscope}%
\begin{pgfscope}%
\pgfpathrectangle{\pgfqpoint{0.800000in}{0.528000in}}{\pgfqpoint{3.968000in}{3.696000in}}%
\pgfusepath{clip}%
\pgfsetbuttcap%
\pgfsetroundjoin%
\definecolor{currentfill}{rgb}{0.136408,0.541173,0.554483}%
\pgfsetfillcolor{currentfill}%
\pgfsetlinewidth{0.000000pt}%
\definecolor{currentstroke}{rgb}{0.000000,0.000000,0.000000}%
\pgfsetstrokecolor{currentstroke}%
\pgfsetdash{}{0pt}%
\pgfpathmoveto{\pgfqpoint{0.800000in}{0.672233in}}%
\pgfpathlineto{\pgfqpoint{0.934353in}{0.528000in}}%
\pgfpathlineto{\pgfqpoint{0.937661in}{0.528000in}}%
\pgfpathlineto{\pgfqpoint{0.800000in}{0.675769in}}%
\pgfpathmoveto{\pgfqpoint{4.768000in}{2.826526in}}%
\pgfpathlineto{\pgfqpoint{4.607677in}{2.994363in}}%
\pgfpathlineto{\pgfqpoint{4.280702in}{3.328000in}}%
\pgfpathlineto{\pgfqpoint{4.092871in}{3.514667in}}%
\pgfpathlineto{\pgfqpoint{3.926303in}{3.677218in}}%
\pgfpathlineto{\pgfqpoint{3.586938in}{4.000000in}}%
\pgfpathlineto{\pgfqpoint{3.405253in}{4.168271in}}%
\pgfpathlineto{\pgfqpoint{3.344306in}{4.224000in}}%
\pgfpathlineto{\pgfqpoint{3.340676in}{4.224000in}}%
\pgfpathlineto{\pgfqpoint{3.662953in}{3.925333in}}%
\pgfpathlineto{\pgfqpoint{3.846141in}{3.751130in}}%
\pgfpathlineto{\pgfqpoint{4.166788in}{3.438274in}}%
\pgfpathlineto{\pgfqpoint{4.498056in}{3.104000in}}%
\pgfpathlineto{\pgfqpoint{4.768000in}{2.823026in}}%
\pgfpathlineto{\pgfqpoint{4.768000in}{2.823026in}}%
\pgfusepath{fill}%
\end{pgfscope}%
\begin{pgfscope}%
\pgfpathrectangle{\pgfqpoint{0.800000in}{0.528000in}}{\pgfqpoint{3.968000in}{3.696000in}}%
\pgfusepath{clip}%
\pgfsetbuttcap%
\pgfsetroundjoin%
\definecolor{currentfill}{rgb}{0.135066,0.544853,0.554029}%
\pgfsetfillcolor{currentfill}%
\pgfsetlinewidth{0.000000pt}%
\definecolor{currentstroke}{rgb}{0.000000,0.000000,0.000000}%
\pgfsetstrokecolor{currentstroke}%
\pgfsetdash{}{0pt}%
\pgfpathmoveto{\pgfqpoint{0.800000in}{0.668696in}}%
\pgfpathlineto{\pgfqpoint{0.931045in}{0.528000in}}%
\pgfpathlineto{\pgfqpoint{0.934353in}{0.528000in}}%
\pgfpathlineto{\pgfqpoint{0.800000in}{0.672233in}}%
\pgfpathmoveto{\pgfqpoint{4.768000in}{2.830025in}}%
\pgfpathlineto{\pgfqpoint{4.607677in}{2.997805in}}%
\pgfpathlineto{\pgfqpoint{4.284110in}{3.328000in}}%
\pgfpathlineto{\pgfqpoint{4.096283in}{3.514667in}}%
\pgfpathlineto{\pgfqpoint{3.926303in}{3.680568in}}%
\pgfpathlineto{\pgfqpoint{3.590496in}{4.000000in}}%
\pgfpathlineto{\pgfqpoint{3.405253in}{4.171584in}}%
\pgfpathlineto{\pgfqpoint{3.347935in}{4.224000in}}%
\pgfpathlineto{\pgfqpoint{3.344306in}{4.224000in}}%
\pgfpathlineto{\pgfqpoint{3.666488in}{3.925333in}}%
\pgfpathlineto{\pgfqpoint{3.846141in}{3.754474in}}%
\pgfpathlineto{\pgfqpoint{4.168463in}{3.440000in}}%
\pgfpathlineto{\pgfqpoint{4.501401in}{3.104000in}}%
\pgfpathlineto{\pgfqpoint{4.768000in}{2.826526in}}%
\pgfpathlineto{\pgfqpoint{4.768000in}{2.826526in}}%
\pgfusepath{fill}%
\end{pgfscope}%
\begin{pgfscope}%
\pgfpathrectangle{\pgfqpoint{0.800000in}{0.528000in}}{\pgfqpoint{3.968000in}{3.696000in}}%
\pgfusepath{clip}%
\pgfsetbuttcap%
\pgfsetroundjoin%
\definecolor{currentfill}{rgb}{0.135066,0.544853,0.554029}%
\pgfsetfillcolor{currentfill}%
\pgfsetlinewidth{0.000000pt}%
\definecolor{currentstroke}{rgb}{0.000000,0.000000,0.000000}%
\pgfsetstrokecolor{currentstroke}%
\pgfsetdash{}{0pt}%
\pgfpathmoveto{\pgfqpoint{0.800000in}{0.665160in}}%
\pgfpathlineto{\pgfqpoint{0.927738in}{0.528000in}}%
\pgfpathlineto{\pgfqpoint{0.931045in}{0.528000in}}%
\pgfpathlineto{\pgfqpoint{0.800000in}{0.668696in}}%
\pgfpathmoveto{\pgfqpoint{4.768000in}{2.833525in}}%
\pgfpathlineto{\pgfqpoint{4.607677in}{3.001247in}}%
\pgfpathlineto{\pgfqpoint{4.287030in}{3.328483in}}%
\pgfpathlineto{\pgfqpoint{4.099696in}{3.514667in}}%
\pgfpathlineto{\pgfqpoint{3.926303in}{3.683917in}}%
\pgfpathlineto{\pgfqpoint{3.594054in}{4.000000in}}%
\pgfpathlineto{\pgfqpoint{3.405253in}{4.174898in}}%
\pgfpathlineto{\pgfqpoint{3.351565in}{4.224000in}}%
\pgfpathlineto{\pgfqpoint{3.347935in}{4.224000in}}%
\pgfpathlineto{\pgfqpoint{3.670023in}{3.925333in}}%
\pgfpathlineto{\pgfqpoint{3.846141in}{3.757818in}}%
\pgfpathlineto{\pgfqpoint{4.171854in}{3.440000in}}%
\pgfpathlineto{\pgfqpoint{4.504746in}{3.104000in}}%
\pgfpathlineto{\pgfqpoint{4.768000in}{2.830025in}}%
\pgfpathlineto{\pgfqpoint{4.768000in}{2.830025in}}%
\pgfusepath{fill}%
\end{pgfscope}%
\begin{pgfscope}%
\pgfpathrectangle{\pgfqpoint{0.800000in}{0.528000in}}{\pgfqpoint{3.968000in}{3.696000in}}%
\pgfusepath{clip}%
\pgfsetbuttcap%
\pgfsetroundjoin%
\definecolor{currentfill}{rgb}{0.135066,0.544853,0.554029}%
\pgfsetfillcolor{currentfill}%
\pgfsetlinewidth{0.000000pt}%
\definecolor{currentstroke}{rgb}{0.000000,0.000000,0.000000}%
\pgfsetstrokecolor{currentstroke}%
\pgfsetdash{}{0pt}%
\pgfpathmoveto{\pgfqpoint{0.800000in}{0.661623in}}%
\pgfpathlineto{\pgfqpoint{0.924430in}{0.528000in}}%
\pgfpathlineto{\pgfqpoint{0.927738in}{0.528000in}}%
\pgfpathlineto{\pgfqpoint{0.800000in}{0.665160in}}%
\pgfpathmoveto{\pgfqpoint{4.768000in}{2.837025in}}%
\pgfpathlineto{\pgfqpoint{4.607677in}{3.004689in}}%
\pgfpathlineto{\pgfqpoint{4.287030in}{3.331858in}}%
\pgfpathlineto{\pgfqpoint{4.103109in}{3.514667in}}%
\pgfpathlineto{\pgfqpoint{3.926303in}{3.687266in}}%
\pgfpathlineto{\pgfqpoint{3.597612in}{4.000000in}}%
\pgfpathlineto{\pgfqpoint{3.405253in}{4.178211in}}%
\pgfpathlineto{\pgfqpoint{3.355194in}{4.224000in}}%
\pgfpathlineto{\pgfqpoint{3.351565in}{4.224000in}}%
\pgfpathlineto{\pgfqpoint{3.673558in}{3.925333in}}%
\pgfpathlineto{\pgfqpoint{3.846141in}{3.761161in}}%
\pgfpathlineto{\pgfqpoint{4.175246in}{3.440000in}}%
\pgfpathlineto{\pgfqpoint{4.508091in}{3.104000in}}%
\pgfpathlineto{\pgfqpoint{4.768000in}{2.833525in}}%
\pgfpathlineto{\pgfqpoint{4.768000in}{2.833525in}}%
\pgfusepath{fill}%
\end{pgfscope}%
\begin{pgfscope}%
\pgfpathrectangle{\pgfqpoint{0.800000in}{0.528000in}}{\pgfqpoint{3.968000in}{3.696000in}}%
\pgfusepath{clip}%
\pgfsetbuttcap%
\pgfsetroundjoin%
\definecolor{currentfill}{rgb}{0.135066,0.544853,0.554029}%
\pgfsetfillcolor{currentfill}%
\pgfsetlinewidth{0.000000pt}%
\definecolor{currentstroke}{rgb}{0.000000,0.000000,0.000000}%
\pgfsetstrokecolor{currentstroke}%
\pgfsetdash{}{0pt}%
\pgfpathmoveto{\pgfqpoint{0.800000in}{0.658087in}}%
\pgfpathlineto{\pgfqpoint{0.921123in}{0.528000in}}%
\pgfpathlineto{\pgfqpoint{0.924430in}{0.528000in}}%
\pgfpathlineto{\pgfqpoint{0.800000in}{0.661623in}}%
\pgfpathmoveto{\pgfqpoint{4.768000in}{2.840524in}}%
\pgfpathlineto{\pgfqpoint{4.607677in}{3.008131in}}%
\pgfpathlineto{\pgfqpoint{4.287030in}{3.335233in}}%
\pgfpathlineto{\pgfqpoint{4.106522in}{3.514667in}}%
\pgfpathlineto{\pgfqpoint{3.926303in}{3.690616in}}%
\pgfpathlineto{\pgfqpoint{3.601170in}{4.000000in}}%
\pgfpathlineto{\pgfqpoint{3.405253in}{4.181524in}}%
\pgfpathlineto{\pgfqpoint{3.358824in}{4.224000in}}%
\pgfpathlineto{\pgfqpoint{3.355194in}{4.224000in}}%
\pgfpathlineto{\pgfqpoint{3.645737in}{3.954870in}}%
\pgfpathlineto{\pgfqpoint{3.834146in}{3.776000in}}%
\pgfpathlineto{\pgfqpoint{4.006465in}{3.609398in}}%
\pgfpathlineto{\pgfqpoint{4.338725in}{3.279849in}}%
\pgfpathlineto{\pgfqpoint{4.511436in}{3.104000in}}%
\pgfpathlineto{\pgfqpoint{4.768000in}{2.837025in}}%
\pgfpathlineto{\pgfqpoint{4.768000in}{2.837025in}}%
\pgfusepath{fill}%
\end{pgfscope}%
\begin{pgfscope}%
\pgfpathrectangle{\pgfqpoint{0.800000in}{0.528000in}}{\pgfqpoint{3.968000in}{3.696000in}}%
\pgfusepath{clip}%
\pgfsetbuttcap%
\pgfsetroundjoin%
\definecolor{currentfill}{rgb}{0.133743,0.548535,0.553541}%
\pgfsetfillcolor{currentfill}%
\pgfsetlinewidth{0.000000pt}%
\definecolor{currentstroke}{rgb}{0.000000,0.000000,0.000000}%
\pgfsetstrokecolor{currentstroke}%
\pgfsetdash{}{0pt}%
\pgfpathmoveto{\pgfqpoint{0.800000in}{0.654551in}}%
\pgfpathlineto{\pgfqpoint{0.917848in}{0.528000in}}%
\pgfpathlineto{\pgfqpoint{0.921123in}{0.528000in}}%
\pgfpathlineto{\pgfqpoint{0.920242in}{0.528939in}}%
\pgfpathlineto{\pgfqpoint{0.800000in}{0.658087in}}%
\pgfpathmoveto{\pgfqpoint{4.768000in}{2.844006in}}%
\pgfpathlineto{\pgfqpoint{4.445205in}{3.178667in}}%
\pgfpathlineto{\pgfqpoint{4.260340in}{3.365333in}}%
\pgfpathlineto{\pgfqpoint{3.957224in}{3.664000in}}%
\pgfpathlineto{\pgfqpoint{3.765980in}{3.847794in}}%
\pgfpathlineto{\pgfqpoint{3.604728in}{4.000000in}}%
\pgfpathlineto{\pgfqpoint{3.443886in}{4.149333in}}%
\pgfpathlineto{\pgfqpoint{3.362453in}{4.224000in}}%
\pgfpathlineto{\pgfqpoint{3.358824in}{4.224000in}}%
\pgfpathlineto{\pgfqpoint{3.645737in}{3.958200in}}%
\pgfpathlineto{\pgfqpoint{3.837635in}{3.776000in}}%
\pgfpathlineto{\pgfqpoint{4.006465in}{3.612753in}}%
\pgfpathlineto{\pgfqpoint{4.331356in}{3.290667in}}%
\pgfpathlineto{\pgfqpoint{4.514781in}{3.104000in}}%
\pgfpathlineto{\pgfqpoint{4.768000in}{2.840524in}}%
\pgfpathlineto{\pgfqpoint{4.768000in}{2.842667in}}%
\pgfpathlineto{\pgfqpoint{4.768000in}{2.842667in}}%
\pgfusepath{fill}%
\end{pgfscope}%
\begin{pgfscope}%
\pgfpathrectangle{\pgfqpoint{0.800000in}{0.528000in}}{\pgfqpoint{3.968000in}{3.696000in}}%
\pgfusepath{clip}%
\pgfsetbuttcap%
\pgfsetroundjoin%
\definecolor{currentfill}{rgb}{0.133743,0.548535,0.553541}%
\pgfsetfillcolor{currentfill}%
\pgfsetlinewidth{0.000000pt}%
\definecolor{currentstroke}{rgb}{0.000000,0.000000,0.000000}%
\pgfsetstrokecolor{currentstroke}%
\pgfsetdash{}{0pt}%
\pgfpathmoveto{\pgfqpoint{0.800000in}{0.651014in}}%
\pgfpathlineto{\pgfqpoint{0.914586in}{0.528000in}}%
\pgfpathlineto{\pgfqpoint{0.917848in}{0.528000in}}%
\pgfpathlineto{\pgfqpoint{0.800000in}{0.654551in}}%
\pgfpathmoveto{\pgfqpoint{4.768000in}{2.847460in}}%
\pgfpathlineto{\pgfqpoint{4.447354in}{3.179891in}}%
\pgfpathlineto{\pgfqpoint{4.263711in}{3.365333in}}%
\pgfpathlineto{\pgfqpoint{3.960680in}{3.664000in}}%
\pgfpathlineto{\pgfqpoint{3.787055in}{3.831036in}}%
\pgfpathlineto{\pgfqpoint{3.606949in}{4.001204in}}%
\pgfpathlineto{\pgfqpoint{3.525495in}{4.077164in}}%
\pgfpathlineto{\pgfqpoint{3.365172in}{4.224000in}}%
\pgfpathlineto{\pgfqpoint{3.362453in}{4.224000in}}%
\pgfpathlineto{\pgfqpoint{3.645737in}{3.961530in}}%
\pgfpathlineto{\pgfqpoint{3.841125in}{3.776000in}}%
\pgfpathlineto{\pgfqpoint{4.006465in}{3.616108in}}%
\pgfpathlineto{\pgfqpoint{4.334705in}{3.290667in}}%
\pgfpathlineto{\pgfqpoint{4.518127in}{3.104000in}}%
\pgfpathlineto{\pgfqpoint{4.768000in}{2.844006in}}%
\pgfpathlineto{\pgfqpoint{4.768000in}{2.844006in}}%
\pgfusepath{fill}%
\end{pgfscope}%
\begin{pgfscope}%
\pgfpathrectangle{\pgfqpoint{0.800000in}{0.528000in}}{\pgfqpoint{3.968000in}{3.696000in}}%
\pgfusepath{clip}%
\pgfsetbuttcap%
\pgfsetroundjoin%
\definecolor{currentfill}{rgb}{0.133743,0.548535,0.553541}%
\pgfsetfillcolor{currentfill}%
\pgfsetlinewidth{0.000000pt}%
\definecolor{currentstroke}{rgb}{0.000000,0.000000,0.000000}%
\pgfsetstrokecolor{currentstroke}%
\pgfsetdash{}{0pt}%
\pgfpathmoveto{\pgfqpoint{0.800000in}{0.647478in}}%
\pgfpathlineto{\pgfqpoint{0.911324in}{0.528000in}}%
\pgfpathlineto{\pgfqpoint{0.914586in}{0.528000in}}%
\pgfpathlineto{\pgfqpoint{0.800000in}{0.651014in}}%
\pgfpathmoveto{\pgfqpoint{4.768000in}{2.850914in}}%
\pgfpathlineto{\pgfqpoint{4.447354in}{3.183278in}}%
\pgfpathlineto{\pgfqpoint{4.267081in}{3.365333in}}%
\pgfpathlineto{\pgfqpoint{3.964136in}{3.664000in}}%
\pgfpathlineto{\pgfqpoint{3.769924in}{3.850667in}}%
\pgfpathlineto{\pgfqpoint{3.605657in}{4.005714in}}%
\pgfpathlineto{\pgfqpoint{3.410412in}{4.186667in}}%
\pgfpathlineto{\pgfqpoint{3.369644in}{4.224000in}}%
\pgfpathlineto{\pgfqpoint{3.366069in}{4.224000in}}%
\pgfpathlineto{\pgfqpoint{3.706528in}{3.907291in}}%
\pgfpathlineto{\pgfqpoint{3.806061in}{3.812913in}}%
\pgfpathlineto{\pgfqpoint{3.999068in}{3.626667in}}%
\pgfpathlineto{\pgfqpoint{4.166788in}{3.461860in}}%
\pgfpathlineto{\pgfqpoint{4.487434in}{3.138932in}}%
\pgfpathlineto{\pgfqpoint{4.665705in}{2.954667in}}%
\pgfpathlineto{\pgfqpoint{4.768000in}{2.847460in}}%
\pgfpathlineto{\pgfqpoint{4.768000in}{2.847460in}}%
\pgfusepath{fill}%
\end{pgfscope}%
\begin{pgfscope}%
\pgfpathrectangle{\pgfqpoint{0.800000in}{0.528000in}}{\pgfqpoint{3.968000in}{3.696000in}}%
\pgfusepath{clip}%
\pgfsetbuttcap%
\pgfsetroundjoin%
\definecolor{currentfill}{rgb}{0.133743,0.548535,0.553541}%
\pgfsetfillcolor{currentfill}%
\pgfsetlinewidth{0.000000pt}%
\definecolor{currentstroke}{rgb}{0.000000,0.000000,0.000000}%
\pgfsetstrokecolor{currentstroke}%
\pgfsetdash{}{0pt}%
\pgfpathmoveto{\pgfqpoint{0.800000in}{0.643941in}}%
\pgfpathlineto{\pgfqpoint{0.908062in}{0.528000in}}%
\pgfpathlineto{\pgfqpoint{0.911324in}{0.528000in}}%
\pgfpathlineto{\pgfqpoint{0.800000in}{0.647478in}}%
\pgfpathmoveto{\pgfqpoint{4.768000in}{2.854368in}}%
\pgfpathlineto{\pgfqpoint{4.447354in}{3.186664in}}%
\pgfpathlineto{\pgfqpoint{4.270451in}{3.365333in}}%
\pgfpathlineto{\pgfqpoint{3.929038in}{3.701333in}}%
\pgfpathlineto{\pgfqpoint{3.615259in}{4.000000in}}%
\pgfpathlineto{\pgfqpoint{3.373219in}{4.224000in}}%
\pgfpathlineto{\pgfqpoint{3.369644in}{4.224000in}}%
\pgfpathlineto{\pgfqpoint{3.708293in}{3.908934in}}%
\pgfpathlineto{\pgfqpoint{3.809076in}{3.813333in}}%
\pgfpathlineto{\pgfqpoint{4.002513in}{3.626667in}}%
\pgfpathlineto{\pgfqpoint{4.166788in}{3.465227in}}%
\pgfpathlineto{\pgfqpoint{4.498318in}{3.131196in}}%
\pgfpathlineto{\pgfqpoint{4.669009in}{2.954667in}}%
\pgfpathlineto{\pgfqpoint{4.768000in}{2.850914in}}%
\pgfpathlineto{\pgfqpoint{4.768000in}{2.850914in}}%
\pgfusepath{fill}%
\end{pgfscope}%
\begin{pgfscope}%
\pgfpathrectangle{\pgfqpoint{0.800000in}{0.528000in}}{\pgfqpoint{3.968000in}{3.696000in}}%
\pgfusepath{clip}%
\pgfsetbuttcap%
\pgfsetroundjoin%
\definecolor{currentfill}{rgb}{0.132444,0.552216,0.553018}%
\pgfsetfillcolor{currentfill}%
\pgfsetlinewidth{0.000000pt}%
\definecolor{currentstroke}{rgb}{0.000000,0.000000,0.000000}%
\pgfsetstrokecolor{currentstroke}%
\pgfsetdash{}{0pt}%
\pgfpathmoveto{\pgfqpoint{0.800000in}{0.640405in}}%
\pgfpathlineto{\pgfqpoint{0.904800in}{0.528000in}}%
\pgfpathlineto{\pgfqpoint{0.908062in}{0.528000in}}%
\pgfpathlineto{\pgfqpoint{0.800000in}{0.643941in}}%
\pgfpathmoveto{\pgfqpoint{4.768000in}{2.857822in}}%
\pgfpathlineto{\pgfqpoint{4.447354in}{3.190050in}}%
\pgfpathlineto{\pgfqpoint{4.273822in}{3.365333in}}%
\pgfpathlineto{\pgfqpoint{3.932455in}{3.701333in}}%
\pgfpathlineto{\pgfqpoint{3.618764in}{4.000000in}}%
\pgfpathlineto{\pgfqpoint{3.376793in}{4.224000in}}%
\pgfpathlineto{\pgfqpoint{3.373219in}{4.224000in}}%
\pgfpathlineto{\pgfqpoint{3.710058in}{3.910578in}}%
\pgfpathlineto{\pgfqpoint{3.812526in}{3.813333in}}%
\pgfpathlineto{\pgfqpoint{4.006465in}{3.626173in}}%
\pgfpathlineto{\pgfqpoint{4.344754in}{3.290667in}}%
\pgfpathlineto{\pgfqpoint{4.528153in}{3.104000in}}%
\pgfpathlineto{\pgfqpoint{4.768000in}{2.854368in}}%
\pgfpathlineto{\pgfqpoint{4.768000in}{2.854368in}}%
\pgfusepath{fill}%
\end{pgfscope}%
\begin{pgfscope}%
\pgfpathrectangle{\pgfqpoint{0.800000in}{0.528000in}}{\pgfqpoint{3.968000in}{3.696000in}}%
\pgfusepath{clip}%
\pgfsetbuttcap%
\pgfsetroundjoin%
\definecolor{currentfill}{rgb}{0.132444,0.552216,0.553018}%
\pgfsetfillcolor{currentfill}%
\pgfsetlinewidth{0.000000pt}%
\definecolor{currentstroke}{rgb}{0.000000,0.000000,0.000000}%
\pgfsetstrokecolor{currentstroke}%
\pgfsetdash{}{0pt}%
\pgfpathmoveto{\pgfqpoint{0.800000in}{0.636910in}}%
\pgfpathlineto{\pgfqpoint{0.901537in}{0.528000in}}%
\pgfpathlineto{\pgfqpoint{0.904800in}{0.528000in}}%
\pgfpathlineto{\pgfqpoint{0.800000in}{0.640405in}}%
\pgfpathlineto{\pgfqpoint{0.800000in}{0.640000in}}%
\pgfpathmoveto{\pgfqpoint{4.768000in}{2.861275in}}%
\pgfpathlineto{\pgfqpoint{4.447354in}{3.193437in}}%
\pgfpathlineto{\pgfqpoint{4.277192in}{3.365333in}}%
\pgfpathlineto{\pgfqpoint{3.935872in}{3.701333in}}%
\pgfpathlineto{\pgfqpoint{3.622270in}{4.000000in}}%
\pgfpathlineto{\pgfqpoint{3.380368in}{4.224000in}}%
\pgfpathlineto{\pgfqpoint{3.376793in}{4.224000in}}%
\pgfpathlineto{\pgfqpoint{3.698120in}{3.925333in}}%
\pgfpathlineto{\pgfqpoint{4.027406in}{3.608840in}}%
\pgfpathlineto{\pgfqpoint{4.126707in}{3.511590in}}%
\pgfpathlineto{\pgfqpoint{4.458511in}{3.178667in}}%
\pgfpathlineto{\pgfqpoint{4.768000in}{2.857822in}}%
\pgfpathlineto{\pgfqpoint{4.768000in}{2.857822in}}%
\pgfusepath{fill}%
\end{pgfscope}%
\begin{pgfscope}%
\pgfpathrectangle{\pgfqpoint{0.800000in}{0.528000in}}{\pgfqpoint{3.968000in}{3.696000in}}%
\pgfusepath{clip}%
\pgfsetbuttcap%
\pgfsetroundjoin%
\definecolor{currentfill}{rgb}{0.132444,0.552216,0.553018}%
\pgfsetfillcolor{currentfill}%
\pgfsetlinewidth{0.000000pt}%
\definecolor{currentstroke}{rgb}{0.000000,0.000000,0.000000}%
\pgfsetstrokecolor{currentstroke}%
\pgfsetdash{}{0pt}%
\pgfpathmoveto{\pgfqpoint{0.800000in}{0.633420in}}%
\pgfpathlineto{\pgfqpoint{0.898275in}{0.528000in}}%
\pgfpathlineto{\pgfqpoint{0.901537in}{0.528000in}}%
\pgfpathlineto{\pgfqpoint{0.800000in}{0.636910in}}%
\pgfpathmoveto{\pgfqpoint{4.768000in}{2.864729in}}%
\pgfpathlineto{\pgfqpoint{4.447354in}{3.196823in}}%
\pgfpathlineto{\pgfqpoint{4.280563in}{3.365333in}}%
\pgfpathlineto{\pgfqpoint{3.939290in}{3.701333in}}%
\pgfpathlineto{\pgfqpoint{3.625776in}{4.000000in}}%
\pgfpathlineto{\pgfqpoint{3.383943in}{4.224000in}}%
\pgfpathlineto{\pgfqpoint{3.380368in}{4.224000in}}%
\pgfpathlineto{\pgfqpoint{3.701603in}{3.925333in}}%
\pgfpathlineto{\pgfqpoint{4.029154in}{3.610468in}}%
\pgfpathlineto{\pgfqpoint{4.126994in}{3.514667in}}%
\pgfpathlineto{\pgfqpoint{4.461830in}{3.178667in}}%
\pgfpathlineto{\pgfqpoint{4.768000in}{2.861275in}}%
\pgfpathlineto{\pgfqpoint{4.768000in}{2.861275in}}%
\pgfusepath{fill}%
\end{pgfscope}%
\begin{pgfscope}%
\pgfpathrectangle{\pgfqpoint{0.800000in}{0.528000in}}{\pgfqpoint{3.968000in}{3.696000in}}%
\pgfusepath{clip}%
\pgfsetbuttcap%
\pgfsetroundjoin%
\definecolor{currentfill}{rgb}{0.131172,0.555899,0.552459}%
\pgfsetfillcolor{currentfill}%
\pgfsetlinewidth{0.000000pt}%
\definecolor{currentstroke}{rgb}{0.000000,0.000000,0.000000}%
\pgfsetstrokecolor{currentstroke}%
\pgfsetdash{}{0pt}%
\pgfpathmoveto{\pgfqpoint{0.800000in}{0.629930in}}%
\pgfpathlineto{\pgfqpoint{0.895013in}{0.528000in}}%
\pgfpathlineto{\pgfqpoint{0.898275in}{0.528000in}}%
\pgfpathlineto{\pgfqpoint{0.800000in}{0.633420in}}%
\pgfpathmoveto{\pgfqpoint{4.768000in}{2.868183in}}%
\pgfpathlineto{\pgfqpoint{4.447354in}{3.200209in}}%
\pgfpathlineto{\pgfqpoint{4.283933in}{3.365333in}}%
\pgfpathlineto{\pgfqpoint{3.942707in}{3.701333in}}%
\pgfpathlineto{\pgfqpoint{3.629281in}{4.000000in}}%
\pgfpathlineto{\pgfqpoint{3.387518in}{4.224000in}}%
\pgfpathlineto{\pgfqpoint{3.383943in}{4.224000in}}%
\pgfpathlineto{\pgfqpoint{3.705086in}{3.925333in}}%
\pgfpathlineto{\pgfqpoint{4.030902in}{3.612096in}}%
\pgfpathlineto{\pgfqpoint{4.130359in}{3.514667in}}%
\pgfpathlineto{\pgfqpoint{4.465148in}{3.178667in}}%
\pgfpathlineto{\pgfqpoint{4.768000in}{2.864729in}}%
\pgfpathlineto{\pgfqpoint{4.768000in}{2.864729in}}%
\pgfusepath{fill}%
\end{pgfscope}%
\begin{pgfscope}%
\pgfpathrectangle{\pgfqpoint{0.800000in}{0.528000in}}{\pgfqpoint{3.968000in}{3.696000in}}%
\pgfusepath{clip}%
\pgfsetbuttcap%
\pgfsetroundjoin%
\definecolor{currentfill}{rgb}{0.131172,0.555899,0.552459}%
\pgfsetfillcolor{currentfill}%
\pgfsetlinewidth{0.000000pt}%
\definecolor{currentstroke}{rgb}{0.000000,0.000000,0.000000}%
\pgfsetstrokecolor{currentstroke}%
\pgfsetdash{}{0pt}%
\pgfpathmoveto{\pgfqpoint{0.800000in}{0.626440in}}%
\pgfpathlineto{\pgfqpoint{0.891751in}{0.528000in}}%
\pgfpathlineto{\pgfqpoint{0.895013in}{0.528000in}}%
\pgfpathlineto{\pgfqpoint{0.800000in}{0.629930in}}%
\pgfpathmoveto{\pgfqpoint{4.768000in}{2.871637in}}%
\pgfpathlineto{\pgfqpoint{4.447354in}{3.203595in}}%
\pgfpathlineto{\pgfqpoint{4.287030in}{3.365603in}}%
\pgfpathlineto{\pgfqpoint{3.946124in}{3.701333in}}%
\pgfpathlineto{\pgfqpoint{3.619186in}{4.012602in}}%
\pgfpathlineto{\pgfqpoint{3.519068in}{4.106014in}}%
\pgfpathlineto{\pgfqpoint{3.431791in}{4.186667in}}%
\pgfpathlineto{\pgfqpoint{3.391093in}{4.224000in}}%
\pgfpathlineto{\pgfqpoint{3.387518in}{4.224000in}}%
\pgfpathlineto{\pgfqpoint{3.708570in}{3.925333in}}%
\pgfpathlineto{\pgfqpoint{4.032650in}{3.613724in}}%
\pgfpathlineto{\pgfqpoint{4.133723in}{3.514667in}}%
\pgfpathlineto{\pgfqpoint{4.468467in}{3.178667in}}%
\pgfpathlineto{\pgfqpoint{4.768000in}{2.868183in}}%
\pgfpathlineto{\pgfqpoint{4.768000in}{2.868183in}}%
\pgfusepath{fill}%
\end{pgfscope}%
\begin{pgfscope}%
\pgfpathrectangle{\pgfqpoint{0.800000in}{0.528000in}}{\pgfqpoint{3.968000in}{3.696000in}}%
\pgfusepath{clip}%
\pgfsetbuttcap%
\pgfsetroundjoin%
\definecolor{currentfill}{rgb}{0.131172,0.555899,0.552459}%
\pgfsetfillcolor{currentfill}%
\pgfsetlinewidth{0.000000pt}%
\definecolor{currentstroke}{rgb}{0.000000,0.000000,0.000000}%
\pgfsetstrokecolor{currentstroke}%
\pgfsetdash{}{0pt}%
\pgfpathmoveto{\pgfqpoint{0.800000in}{0.622951in}}%
\pgfpathlineto{\pgfqpoint{0.888489in}{0.528000in}}%
\pgfpathlineto{\pgfqpoint{0.891751in}{0.528000in}}%
\pgfpathlineto{\pgfqpoint{0.800000in}{0.626440in}}%
\pgfpathmoveto{\pgfqpoint{4.768000in}{2.875091in}}%
\pgfpathlineto{\pgfqpoint{4.475105in}{3.178667in}}%
\pgfpathlineto{\pgfqpoint{4.290623in}{3.365333in}}%
\pgfpathlineto{\pgfqpoint{3.988015in}{3.664000in}}%
\pgfpathlineto{\pgfqpoint{3.806061in}{3.839313in}}%
\pgfpathlineto{\pgfqpoint{3.636293in}{4.000000in}}%
\pgfpathlineto{\pgfqpoint{3.394668in}{4.224000in}}%
\pgfpathlineto{\pgfqpoint{3.391093in}{4.224000in}}%
\pgfpathlineto{\pgfqpoint{3.712053in}{3.925333in}}%
\pgfpathlineto{\pgfqpoint{4.022945in}{3.626667in}}%
\pgfpathlineto{\pgfqpoint{4.361502in}{3.290667in}}%
\pgfpathlineto{\pgfqpoint{4.527515in}{3.121616in}}%
\pgfpathlineto{\pgfqpoint{4.696880in}{2.946245in}}%
\pgfpathlineto{\pgfqpoint{4.768000in}{2.871637in}}%
\pgfpathlineto{\pgfqpoint{4.768000in}{2.871637in}}%
\pgfusepath{fill}%
\end{pgfscope}%
\begin{pgfscope}%
\pgfpathrectangle{\pgfqpoint{0.800000in}{0.528000in}}{\pgfqpoint{3.968000in}{3.696000in}}%
\pgfusepath{clip}%
\pgfsetbuttcap%
\pgfsetroundjoin%
\definecolor{currentfill}{rgb}{0.131172,0.555899,0.552459}%
\pgfsetfillcolor{currentfill}%
\pgfsetlinewidth{0.000000pt}%
\definecolor{currentstroke}{rgb}{0.000000,0.000000,0.000000}%
\pgfsetstrokecolor{currentstroke}%
\pgfsetdash{}{0pt}%
\pgfpathmoveto{\pgfqpoint{0.800000in}{0.619461in}}%
\pgfpathlineto{\pgfqpoint{0.885227in}{0.528000in}}%
\pgfpathlineto{\pgfqpoint{0.888489in}{0.528000in}}%
\pgfpathlineto{\pgfqpoint{0.800000in}{0.622951in}}%
\pgfpathmoveto{\pgfqpoint{4.768000in}{2.878545in}}%
\pgfpathlineto{\pgfqpoint{4.478424in}{3.178667in}}%
\pgfpathlineto{\pgfqpoint{4.293946in}{3.365333in}}%
\pgfpathlineto{\pgfqpoint{3.991422in}{3.664000in}}%
\pgfpathlineto{\pgfqpoint{3.806061in}{3.842612in}}%
\pgfpathlineto{\pgfqpoint{3.639798in}{4.000000in}}%
\pgfpathlineto{\pgfqpoint{3.398243in}{4.224000in}}%
\pgfpathlineto{\pgfqpoint{3.394668in}{4.224000in}}%
\pgfpathlineto{\pgfqpoint{3.685818in}{3.953413in}}%
\pgfpathlineto{\pgfqpoint{3.872149in}{3.776000in}}%
\pgfpathlineto{\pgfqpoint{4.192002in}{3.463485in}}%
\pgfpathlineto{\pgfqpoint{4.290623in}{3.365333in}}%
\pgfpathlineto{\pgfqpoint{4.620268in}{3.029333in}}%
\pgfpathlineto{\pgfqpoint{4.768000in}{2.875091in}}%
\pgfpathlineto{\pgfqpoint{4.768000in}{2.875091in}}%
\pgfusepath{fill}%
\end{pgfscope}%
\begin{pgfscope}%
\pgfpathrectangle{\pgfqpoint{0.800000in}{0.528000in}}{\pgfqpoint{3.968000in}{3.696000in}}%
\pgfusepath{clip}%
\pgfsetbuttcap%
\pgfsetroundjoin%
\definecolor{currentfill}{rgb}{0.129933,0.559582,0.551864}%
\pgfsetfillcolor{currentfill}%
\pgfsetlinewidth{0.000000pt}%
\definecolor{currentstroke}{rgb}{0.000000,0.000000,0.000000}%
\pgfsetstrokecolor{currentstroke}%
\pgfsetdash{}{0pt}%
\pgfpathmoveto{\pgfqpoint{0.800000in}{0.615971in}}%
\pgfpathlineto{\pgfqpoint{0.881964in}{0.528000in}}%
\pgfpathlineto{\pgfqpoint{0.885227in}{0.528000in}}%
\pgfpathlineto{\pgfqpoint{0.800000in}{0.619461in}}%
\pgfpathmoveto{\pgfqpoint{4.768000in}{2.881973in}}%
\pgfpathlineto{\pgfqpoint{4.590748in}{3.066667in}}%
\pgfpathlineto{\pgfqpoint{4.407273in}{3.254477in}}%
\pgfpathlineto{\pgfqpoint{4.071293in}{3.589333in}}%
\pgfpathlineto{\pgfqpoint{3.886222in}{3.769085in}}%
\pgfpathlineto{\pgfqpoint{3.563466in}{4.074667in}}%
\pgfpathlineto{\pgfqpoint{3.401818in}{4.224000in}}%
\pgfpathlineto{\pgfqpoint{3.398243in}{4.224000in}}%
\pgfpathlineto{\pgfqpoint{3.685818in}{3.956705in}}%
\pgfpathlineto{\pgfqpoint{3.875588in}{3.776000in}}%
\pgfpathlineto{\pgfqpoint{4.181558in}{3.477333in}}%
\pgfpathlineto{\pgfqpoint{4.514900in}{3.141333in}}%
\pgfpathlineto{\pgfqpoint{4.687838in}{2.962501in}}%
\pgfpathlineto{\pgfqpoint{4.768000in}{2.878545in}}%
\pgfpathlineto{\pgfqpoint{4.768000in}{2.880000in}}%
\pgfpathlineto{\pgfqpoint{4.768000in}{2.880000in}}%
\pgfusepath{fill}%
\end{pgfscope}%
\begin{pgfscope}%
\pgfpathrectangle{\pgfqpoint{0.800000in}{0.528000in}}{\pgfqpoint{3.968000in}{3.696000in}}%
\pgfusepath{clip}%
\pgfsetbuttcap%
\pgfsetroundjoin%
\definecolor{currentfill}{rgb}{0.129933,0.559582,0.551864}%
\pgfsetfillcolor{currentfill}%
\pgfsetlinewidth{0.000000pt}%
\definecolor{currentstroke}{rgb}{0.000000,0.000000,0.000000}%
\pgfsetstrokecolor{currentstroke}%
\pgfsetdash{}{0pt}%
\pgfpathmoveto{\pgfqpoint{0.800000in}{0.612481in}}%
\pgfpathlineto{\pgfqpoint{0.880162in}{0.528000in}}%
\pgfpathlineto{\pgfqpoint{0.881964in}{0.528000in}}%
\pgfpathlineto{\pgfqpoint{0.881130in}{0.528902in}}%
\pgfpathlineto{\pgfqpoint{0.800000in}{0.615971in}}%
\pgfpathmoveto{\pgfqpoint{4.768000in}{2.885382in}}%
\pgfpathlineto{\pgfqpoint{4.594037in}{3.066667in}}%
\pgfpathlineto{\pgfqpoint{4.411693in}{3.253333in}}%
\pgfpathlineto{\pgfqpoint{4.112683in}{3.552000in}}%
\pgfpathlineto{\pgfqpoint{3.926303in}{3.733748in}}%
\pgfpathlineto{\pgfqpoint{3.765337in}{3.888000in}}%
\pgfpathlineto{\pgfqpoint{3.566973in}{4.074667in}}%
\pgfpathlineto{\pgfqpoint{3.405253in}{4.224000in}}%
\pgfpathlineto{\pgfqpoint{3.401818in}{4.224000in}}%
\pgfpathlineto{\pgfqpoint{3.685818in}{3.959996in}}%
\pgfpathlineto{\pgfqpoint{3.879027in}{3.776000in}}%
\pgfpathlineto{\pgfqpoint{4.184912in}{3.477333in}}%
\pgfpathlineto{\pgfqpoint{4.518208in}{3.141333in}}%
\pgfpathlineto{\pgfqpoint{4.687838in}{2.965905in}}%
\pgfpathlineto{\pgfqpoint{4.768000in}{2.881973in}}%
\pgfpathlineto{\pgfqpoint{4.768000in}{2.881973in}}%
\pgfusepath{fill}%
\end{pgfscope}%
\begin{pgfscope}%
\pgfpathrectangle{\pgfqpoint{0.800000in}{0.528000in}}{\pgfqpoint{3.968000in}{3.696000in}}%
\pgfusepath{clip}%
\pgfsetbuttcap%
\pgfsetroundjoin%
\definecolor{currentfill}{rgb}{0.129933,0.559582,0.551864}%
\pgfsetfillcolor{currentfill}%
\pgfsetlinewidth{0.000000pt}%
\definecolor{currentstroke}{rgb}{0.000000,0.000000,0.000000}%
\pgfsetstrokecolor{currentstroke}%
\pgfsetdash{}{0pt}%
\pgfpathmoveto{\pgfqpoint{0.800000in}{0.608992in}}%
\pgfpathlineto{\pgfqpoint{0.875504in}{0.528000in}}%
\pgfpathlineto{\pgfqpoint{0.878722in}{0.528000in}}%
\pgfpathlineto{\pgfqpoint{0.800000in}{0.612481in}}%
\pgfpathmoveto{\pgfqpoint{4.768000in}{2.888792in}}%
\pgfpathlineto{\pgfqpoint{4.597326in}{3.066667in}}%
\pgfpathlineto{\pgfqpoint{4.414986in}{3.253333in}}%
\pgfpathlineto{\pgfqpoint{4.116058in}{3.552000in}}%
\pgfpathlineto{\pgfqpoint{3.926303in}{3.737055in}}%
\pgfpathlineto{\pgfqpoint{3.765980in}{3.890653in}}%
\pgfpathlineto{\pgfqpoint{3.565576in}{4.079213in}}%
\pgfpathlineto{\pgfqpoint{3.408912in}{4.224000in}}%
\pgfpathlineto{\pgfqpoint{3.405390in}{4.224000in}}%
\pgfpathlineto{\pgfqpoint{3.730635in}{3.920922in}}%
\pgfpathlineto{\pgfqpoint{3.921205in}{3.738667in}}%
\pgfpathlineto{\pgfqpoint{4.086626in}{3.577622in}}%
\pgfpathlineto{\pgfqpoint{4.263289in}{3.402667in}}%
\pgfpathlineto{\pgfqpoint{4.594037in}{3.066667in}}%
\pgfpathlineto{\pgfqpoint{4.768000in}{2.885382in}}%
\pgfpathlineto{\pgfqpoint{4.768000in}{2.885382in}}%
\pgfusepath{fill}%
\end{pgfscope}%
\begin{pgfscope}%
\pgfpathrectangle{\pgfqpoint{0.800000in}{0.528000in}}{\pgfqpoint{3.968000in}{3.696000in}}%
\pgfusepath{clip}%
\pgfsetbuttcap%
\pgfsetroundjoin%
\definecolor{currentfill}{rgb}{0.129933,0.559582,0.551864}%
\pgfsetfillcolor{currentfill}%
\pgfsetlinewidth{0.000000pt}%
\definecolor{currentstroke}{rgb}{0.000000,0.000000,0.000000}%
\pgfsetstrokecolor{currentstroke}%
\pgfsetdash{}{0pt}%
\pgfpathmoveto{\pgfqpoint{0.800000in}{0.605502in}}%
\pgfpathlineto{\pgfqpoint{0.872286in}{0.528000in}}%
\pgfpathlineto{\pgfqpoint{0.875504in}{0.528000in}}%
\pgfpathlineto{\pgfqpoint{0.800000in}{0.608992in}}%
\pgfpathmoveto{\pgfqpoint{4.768000in}{2.892201in}}%
\pgfpathlineto{\pgfqpoint{4.600615in}{3.066667in}}%
\pgfpathlineto{\pgfqpoint{4.418279in}{3.253333in}}%
\pgfpathlineto{\pgfqpoint{4.119433in}{3.552000in}}%
\pgfpathlineto{\pgfqpoint{3.928036in}{3.738667in}}%
\pgfpathlineto{\pgfqpoint{3.765980in}{3.893909in}}%
\pgfpathlineto{\pgfqpoint{3.573927in}{4.074667in}}%
\pgfpathlineto{\pgfqpoint{3.412434in}{4.224000in}}%
\pgfpathlineto{\pgfqpoint{3.408912in}{4.224000in}}%
\pgfpathlineto{\pgfqpoint{3.729417in}{3.925333in}}%
\pgfpathlineto{\pgfqpoint{3.926303in}{3.737055in}}%
\pgfpathlineto{\pgfqpoint{4.266622in}{3.402667in}}%
\pgfpathlineto{\pgfqpoint{4.597326in}{3.066667in}}%
\pgfpathlineto{\pgfqpoint{4.768000in}{2.888792in}}%
\pgfpathlineto{\pgfqpoint{4.768000in}{2.888792in}}%
\pgfusepath{fill}%
\end{pgfscope}%
\begin{pgfscope}%
\pgfpathrectangle{\pgfqpoint{0.800000in}{0.528000in}}{\pgfqpoint{3.968000in}{3.696000in}}%
\pgfusepath{clip}%
\pgfsetbuttcap%
\pgfsetroundjoin%
\definecolor{currentfill}{rgb}{0.128729,0.563265,0.551229}%
\pgfsetfillcolor{currentfill}%
\pgfsetlinewidth{0.000000pt}%
\definecolor{currentstroke}{rgb}{0.000000,0.000000,0.000000}%
\pgfsetstrokecolor{currentstroke}%
\pgfsetdash{}{0pt}%
\pgfpathmoveto{\pgfqpoint{0.800000in}{0.602021in}}%
\pgfpathlineto{\pgfqpoint{0.869068in}{0.528000in}}%
\pgfpathlineto{\pgfqpoint{0.872286in}{0.528000in}}%
\pgfpathlineto{\pgfqpoint{0.800000in}{0.605502in}}%
\pgfpathlineto{\pgfqpoint{0.800000in}{0.602667in}}%
\pgfpathmoveto{\pgfqpoint{4.768000in}{2.895611in}}%
\pgfpathlineto{\pgfqpoint{4.603904in}{3.066667in}}%
\pgfpathlineto{\pgfqpoint{4.421572in}{3.253333in}}%
\pgfpathlineto{\pgfqpoint{4.122808in}{3.552000in}}%
\pgfpathlineto{\pgfqpoint{3.931416in}{3.738667in}}%
\pgfpathlineto{\pgfqpoint{3.765980in}{3.897166in}}%
\pgfpathlineto{\pgfqpoint{3.577404in}{4.074667in}}%
\pgfpathlineto{\pgfqpoint{3.415956in}{4.224000in}}%
\pgfpathlineto{\pgfqpoint{3.412434in}{4.224000in}}%
\pgfpathlineto{\pgfqpoint{3.732850in}{3.925333in}}%
\pgfpathlineto{\pgfqpoint{3.926303in}{3.740342in}}%
\pgfpathlineto{\pgfqpoint{4.086626in}{3.584259in}}%
\pgfpathlineto{\pgfqpoint{4.278141in}{3.394387in}}%
\pgfpathlineto{\pgfqpoint{4.374544in}{3.297514in}}%
\pgfpathlineto{\pgfqpoint{4.455026in}{3.216000in}}%
\pgfpathlineto{\pgfqpoint{4.636662in}{3.029333in}}%
\pgfpathlineto{\pgfqpoint{4.768000in}{2.892201in}}%
\pgfpathlineto{\pgfqpoint{4.768000in}{2.892201in}}%
\pgfusepath{fill}%
\end{pgfscope}%
\begin{pgfscope}%
\pgfpathrectangle{\pgfqpoint{0.800000in}{0.528000in}}{\pgfqpoint{3.968000in}{3.696000in}}%
\pgfusepath{clip}%
\pgfsetbuttcap%
\pgfsetroundjoin%
\definecolor{currentfill}{rgb}{0.128729,0.563265,0.551229}%
\pgfsetfillcolor{currentfill}%
\pgfsetlinewidth{0.000000pt}%
\definecolor{currentstroke}{rgb}{0.000000,0.000000,0.000000}%
\pgfsetstrokecolor{currentstroke}%
\pgfsetdash{}{0pt}%
\pgfpathmoveto{\pgfqpoint{0.800000in}{0.598576in}}%
\pgfpathlineto{\pgfqpoint{0.865850in}{0.528000in}}%
\pgfpathlineto{\pgfqpoint{0.869068in}{0.528000in}}%
\pgfpathlineto{\pgfqpoint{0.800000in}{0.602021in}}%
\pgfpathmoveto{\pgfqpoint{4.768000in}{2.899020in}}%
\pgfpathlineto{\pgfqpoint{4.602716in}{3.071287in}}%
\pgfpathlineto{\pgfqpoint{4.416393in}{3.261828in}}%
\pgfpathlineto{\pgfqpoint{4.320213in}{3.358908in}}%
\pgfpathlineto{\pgfqpoint{4.239222in}{3.440000in}}%
\pgfpathlineto{\pgfqpoint{3.896079in}{3.776000in}}%
\pgfpathlineto{\pgfqpoint{3.725899in}{3.938428in}}%
\pgfpathlineto{\pgfqpoint{3.533081in}{4.119066in}}%
\pgfpathlineto{\pgfqpoint{3.445333in}{4.200251in}}%
\pgfpathlineto{\pgfqpoint{3.419478in}{4.224000in}}%
\pgfpathlineto{\pgfqpoint{3.415956in}{4.224000in}}%
\pgfpathlineto{\pgfqpoint{3.736283in}{3.925333in}}%
\pgfpathlineto{\pgfqpoint{3.926303in}{3.743609in}}%
\pgfpathlineto{\pgfqpoint{4.086626in}{3.587577in}}%
\pgfpathlineto{\pgfqpoint{4.279876in}{3.396003in}}%
\pgfpathlineto{\pgfqpoint{4.367192in}{3.308362in}}%
\pgfpathlineto{\pgfqpoint{4.687838in}{2.979519in}}%
\pgfpathlineto{\pgfqpoint{4.768000in}{2.895611in}}%
\pgfpathlineto{\pgfqpoint{4.768000in}{2.895611in}}%
\pgfusepath{fill}%
\end{pgfscope}%
\begin{pgfscope}%
\pgfpathrectangle{\pgfqpoint{0.800000in}{0.528000in}}{\pgfqpoint{3.968000in}{3.696000in}}%
\pgfusepath{clip}%
\pgfsetbuttcap%
\pgfsetroundjoin%
\definecolor{currentfill}{rgb}{0.128729,0.563265,0.551229}%
\pgfsetfillcolor{currentfill}%
\pgfsetlinewidth{0.000000pt}%
\definecolor{currentstroke}{rgb}{0.000000,0.000000,0.000000}%
\pgfsetstrokecolor{currentstroke}%
\pgfsetdash{}{0pt}%
\pgfpathmoveto{\pgfqpoint{0.800000in}{0.595132in}}%
\pgfpathlineto{\pgfqpoint{0.862632in}{0.528000in}}%
\pgfpathlineto{\pgfqpoint{0.865850in}{0.528000in}}%
\pgfpathlineto{\pgfqpoint{0.800000in}{0.598576in}}%
\pgfpathmoveto{\pgfqpoint{4.768000in}{2.902429in}}%
\pgfpathlineto{\pgfqpoint{4.607677in}{3.069527in}}%
\pgfpathlineto{\pgfqpoint{4.418100in}{3.263418in}}%
\pgfpathlineto{\pgfqpoint{4.317209in}{3.365333in}}%
\pgfpathlineto{\pgfqpoint{3.976732in}{3.701333in}}%
\pgfpathlineto{\pgfqpoint{3.806061in}{3.865522in}}%
\pgfpathlineto{\pgfqpoint{3.624292in}{4.037333in}}%
\pgfpathlineto{\pgfqpoint{3.445333in}{4.203486in}}%
\pgfpathlineto{\pgfqpoint{3.423000in}{4.224000in}}%
\pgfpathlineto{\pgfqpoint{3.419478in}{4.224000in}}%
\pgfpathlineto{\pgfqpoint{3.739716in}{3.925333in}}%
\pgfpathlineto{\pgfqpoint{3.926303in}{3.746875in}}%
\pgfpathlineto{\pgfqpoint{4.116766in}{3.561260in}}%
\pgfpathlineto{\pgfqpoint{4.281612in}{3.397620in}}%
\pgfpathlineto{\pgfqpoint{4.367192in}{3.311700in}}%
\pgfpathlineto{\pgfqpoint{4.687838in}{2.982923in}}%
\pgfpathlineto{\pgfqpoint{4.768000in}{2.899020in}}%
\pgfpathlineto{\pgfqpoint{4.768000in}{2.899020in}}%
\pgfusepath{fill}%
\end{pgfscope}%
\begin{pgfscope}%
\pgfpathrectangle{\pgfqpoint{0.800000in}{0.528000in}}{\pgfqpoint{3.968000in}{3.696000in}}%
\pgfusepath{clip}%
\pgfsetbuttcap%
\pgfsetroundjoin%
\definecolor{currentfill}{rgb}{0.127568,0.566949,0.550556}%
\pgfsetfillcolor{currentfill}%
\pgfsetlinewidth{0.000000pt}%
\definecolor{currentstroke}{rgb}{0.000000,0.000000,0.000000}%
\pgfsetstrokecolor{currentstroke}%
\pgfsetdash{}{0pt}%
\pgfpathmoveto{\pgfqpoint{0.800000in}{0.591688in}}%
\pgfpathlineto{\pgfqpoint{0.859414in}{0.528000in}}%
\pgfpathlineto{\pgfqpoint{0.862632in}{0.528000in}}%
\pgfpathlineto{\pgfqpoint{0.800000in}{0.595132in}}%
\pgfpathmoveto{\pgfqpoint{4.768000in}{2.905839in}}%
\pgfpathlineto{\pgfqpoint{4.607677in}{3.072882in}}%
\pgfpathlineto{\pgfqpoint{4.419807in}{3.265008in}}%
\pgfpathlineto{\pgfqpoint{4.320532in}{3.365333in}}%
\pgfpathlineto{\pgfqpoint{3.980101in}{3.701333in}}%
\pgfpathlineto{\pgfqpoint{3.806061in}{3.868781in}}%
\pgfpathlineto{\pgfqpoint{3.627758in}{4.037333in}}%
\pgfpathlineto{\pgfqpoint{3.445333in}{4.206721in}}%
\pgfpathlineto{\pgfqpoint{3.426522in}{4.224000in}}%
\pgfpathlineto{\pgfqpoint{3.423000in}{4.224000in}}%
\pgfpathlineto{\pgfqpoint{3.743148in}{3.925333in}}%
\pgfpathlineto{\pgfqpoint{3.926303in}{3.750142in}}%
\pgfpathlineto{\pgfqpoint{4.091536in}{3.589333in}}%
\pgfpathlineto{\pgfqpoint{4.279956in}{3.402667in}}%
\pgfpathlineto{\pgfqpoint{4.574243in}{3.104000in}}%
\pgfpathlineto{\pgfqpoint{4.753796in}{2.917333in}}%
\pgfpathlineto{\pgfqpoint{4.768000in}{2.902429in}}%
\pgfpathlineto{\pgfqpoint{4.768000in}{2.902429in}}%
\pgfusepath{fill}%
\end{pgfscope}%
\begin{pgfscope}%
\pgfpathrectangle{\pgfqpoint{0.800000in}{0.528000in}}{\pgfqpoint{3.968000in}{3.696000in}}%
\pgfusepath{clip}%
\pgfsetbuttcap%
\pgfsetroundjoin%
\definecolor{currentfill}{rgb}{0.127568,0.566949,0.550556}%
\pgfsetfillcolor{currentfill}%
\pgfsetlinewidth{0.000000pt}%
\definecolor{currentstroke}{rgb}{0.000000,0.000000,0.000000}%
\pgfsetstrokecolor{currentstroke}%
\pgfsetdash{}{0pt}%
\pgfpathmoveto{\pgfqpoint{0.800000in}{0.588243in}}%
\pgfpathlineto{\pgfqpoint{0.856196in}{0.528000in}}%
\pgfpathlineto{\pgfqpoint{0.859414in}{0.528000in}}%
\pgfpathlineto{\pgfqpoint{0.800000in}{0.591688in}}%
\pgfpathmoveto{\pgfqpoint{4.768000in}{2.909248in}}%
\pgfpathlineto{\pgfqpoint{4.607677in}{3.076237in}}%
\pgfpathlineto{\pgfqpoint{4.421514in}{3.266599in}}%
\pgfpathlineto{\pgfqpoint{4.323856in}{3.365333in}}%
\pgfpathlineto{\pgfqpoint{3.983470in}{3.701333in}}%
\pgfpathlineto{\pgfqpoint{3.806061in}{3.872040in}}%
\pgfpathlineto{\pgfqpoint{3.631223in}{4.037333in}}%
\pgfpathlineto{\pgfqpoint{3.445333in}{4.209956in}}%
\pgfpathlineto{\pgfqpoint{3.430044in}{4.224000in}}%
\pgfpathlineto{\pgfqpoint{3.426522in}{4.224000in}}%
\pgfpathlineto{\pgfqpoint{3.746581in}{3.925333in}}%
\pgfpathlineto{\pgfqpoint{3.926303in}{3.753409in}}%
\pgfpathlineto{\pgfqpoint{4.094874in}{3.589333in}}%
\pgfpathlineto{\pgfqpoint{4.283289in}{3.402667in}}%
\pgfpathlineto{\pgfqpoint{4.577497in}{3.104000in}}%
\pgfpathlineto{\pgfqpoint{4.757045in}{2.917333in}}%
\pgfpathlineto{\pgfqpoint{4.768000in}{2.905839in}}%
\pgfpathlineto{\pgfqpoint{4.768000in}{2.905839in}}%
\pgfusepath{fill}%
\end{pgfscope}%
\begin{pgfscope}%
\pgfpathrectangle{\pgfqpoint{0.800000in}{0.528000in}}{\pgfqpoint{3.968000in}{3.696000in}}%
\pgfusepath{clip}%
\pgfsetbuttcap%
\pgfsetroundjoin%
\definecolor{currentfill}{rgb}{0.127568,0.566949,0.550556}%
\pgfsetfillcolor{currentfill}%
\pgfsetlinewidth{0.000000pt}%
\definecolor{currentstroke}{rgb}{0.000000,0.000000,0.000000}%
\pgfsetstrokecolor{currentstroke}%
\pgfsetdash{}{0pt}%
\pgfpathmoveto{\pgfqpoint{0.800000in}{0.584799in}}%
\pgfpathlineto{\pgfqpoint{0.852978in}{0.528000in}}%
\pgfpathlineto{\pgfqpoint{0.856196in}{0.528000in}}%
\pgfpathlineto{\pgfqpoint{0.800000in}{0.588243in}}%
\pgfpathmoveto{\pgfqpoint{4.768000in}{2.912658in}}%
\pgfpathlineto{\pgfqpoint{4.607677in}{3.079591in}}%
\pgfpathlineto{\pgfqpoint{4.423221in}{3.268189in}}%
\pgfpathlineto{\pgfqpoint{4.327111in}{3.365401in}}%
\pgfpathlineto{\pgfqpoint{3.986839in}{3.701333in}}%
\pgfpathlineto{\pgfqpoint{3.806061in}{3.875299in}}%
\pgfpathlineto{\pgfqpoint{3.634689in}{4.037333in}}%
\pgfpathlineto{\pgfqpoint{3.445333in}{4.213191in}}%
\pgfpathlineto{\pgfqpoint{3.433566in}{4.224000in}}%
\pgfpathlineto{\pgfqpoint{3.430044in}{4.224000in}}%
\pgfpathlineto{\pgfqpoint{3.750014in}{3.925333in}}%
\pgfpathlineto{\pgfqpoint{3.926303in}{3.756675in}}%
\pgfpathlineto{\pgfqpoint{4.098211in}{3.589333in}}%
\pgfpathlineto{\pgfqpoint{4.287030in}{3.402259in}}%
\pgfpathlineto{\pgfqpoint{4.616930in}{3.066667in}}%
\pgfpathlineto{\pgfqpoint{4.768000in}{2.909248in}}%
\pgfpathlineto{\pgfqpoint{4.768000in}{2.909248in}}%
\pgfusepath{fill}%
\end{pgfscope}%
\begin{pgfscope}%
\pgfpathrectangle{\pgfqpoint{0.800000in}{0.528000in}}{\pgfqpoint{3.968000in}{3.696000in}}%
\pgfusepath{clip}%
\pgfsetbuttcap%
\pgfsetroundjoin%
\definecolor{currentfill}{rgb}{0.127568,0.566949,0.550556}%
\pgfsetfillcolor{currentfill}%
\pgfsetlinewidth{0.000000pt}%
\definecolor{currentstroke}{rgb}{0.000000,0.000000,0.000000}%
\pgfsetstrokecolor{currentstroke}%
\pgfsetdash{}{0pt}%
\pgfpathmoveto{\pgfqpoint{0.800000in}{0.581355in}}%
\pgfpathlineto{\pgfqpoint{0.849760in}{0.528000in}}%
\pgfpathlineto{\pgfqpoint{0.852978in}{0.528000in}}%
\pgfpathlineto{\pgfqpoint{0.800000in}{0.584799in}}%
\pgfpathmoveto{\pgfqpoint{4.768000in}{2.916067in}}%
\pgfpathlineto{\pgfqpoint{4.607677in}{3.082946in}}%
\pgfpathlineto{\pgfqpoint{4.441330in}{3.253333in}}%
\pgfpathlineto{\pgfqpoint{4.104887in}{3.589333in}}%
\pgfpathlineto{\pgfqpoint{3.781137in}{3.902118in}}%
\pgfpathlineto{\pgfqpoint{3.677886in}{4.000000in}}%
\pgfpathlineto{\pgfqpoint{3.485414in}{4.179485in}}%
\pgfpathlineto{\pgfqpoint{3.437088in}{4.224000in}}%
\pgfpathlineto{\pgfqpoint{3.433566in}{4.224000in}}%
\pgfpathlineto{\pgfqpoint{3.753447in}{3.925333in}}%
\pgfpathlineto{\pgfqpoint{3.926303in}{3.759942in}}%
\pgfpathlineto{\pgfqpoint{4.101549in}{3.589333in}}%
\pgfpathlineto{\pgfqpoint{4.287030in}{3.405555in}}%
\pgfpathlineto{\pgfqpoint{4.447354in}{3.243874in}}%
\pgfpathlineto{\pgfqpoint{4.620174in}{3.066667in}}%
\pgfpathlineto{\pgfqpoint{4.768000in}{2.912658in}}%
\pgfpathlineto{\pgfqpoint{4.768000in}{2.912658in}}%
\pgfusepath{fill}%
\end{pgfscope}%
\begin{pgfscope}%
\pgfpathrectangle{\pgfqpoint{0.800000in}{0.528000in}}{\pgfqpoint{3.968000in}{3.696000in}}%
\pgfusepath{clip}%
\pgfsetbuttcap%
\pgfsetroundjoin%
\definecolor{currentfill}{rgb}{0.126453,0.570633,0.549841}%
\pgfsetfillcolor{currentfill}%
\pgfsetlinewidth{0.000000pt}%
\definecolor{currentstroke}{rgb}{0.000000,0.000000,0.000000}%
\pgfsetstrokecolor{currentstroke}%
\pgfsetdash{}{0pt}%
\pgfpathmoveto{\pgfqpoint{0.800000in}{0.577910in}}%
\pgfpathlineto{\pgfqpoint{0.846542in}{0.528000in}}%
\pgfpathlineto{\pgfqpoint{0.849760in}{0.528000in}}%
\pgfpathlineto{\pgfqpoint{0.800000in}{0.581355in}}%
\pgfpathmoveto{\pgfqpoint{4.768000in}{2.919449in}}%
\pgfpathlineto{\pgfqpoint{4.444623in}{3.253333in}}%
\pgfpathlineto{\pgfqpoint{4.108225in}{3.589333in}}%
\pgfpathlineto{\pgfqpoint{3.782856in}{3.903719in}}%
\pgfpathlineto{\pgfqpoint{3.681341in}{4.000000in}}%
\pgfpathlineto{\pgfqpoint{3.485414in}{4.182723in}}%
\pgfpathlineto{\pgfqpoint{3.440610in}{4.224000in}}%
\pgfpathlineto{\pgfqpoint{3.437088in}{4.224000in}}%
\pgfpathlineto{\pgfqpoint{3.741435in}{3.939805in}}%
\pgfpathlineto{\pgfqpoint{3.846141in}{3.840260in}}%
\pgfpathlineto{\pgfqpoint{4.028578in}{3.664000in}}%
\pgfpathlineto{\pgfqpoint{4.206869in}{3.488681in}}%
\pgfpathlineto{\pgfqpoint{4.367574in}{3.328000in}}%
\pgfpathlineto{\pgfqpoint{4.695362in}{2.992000in}}%
\pgfpathlineto{\pgfqpoint{4.768000in}{2.916067in}}%
\pgfpathlineto{\pgfqpoint{4.768000in}{2.917333in}}%
\pgfpathlineto{\pgfqpoint{4.768000in}{2.917333in}}%
\pgfusepath{fill}%
\end{pgfscope}%
\begin{pgfscope}%
\pgfpathrectangle{\pgfqpoint{0.800000in}{0.528000in}}{\pgfqpoint{3.968000in}{3.696000in}}%
\pgfusepath{clip}%
\pgfsetbuttcap%
\pgfsetroundjoin%
\definecolor{currentfill}{rgb}{0.126453,0.570633,0.549841}%
\pgfsetfillcolor{currentfill}%
\pgfsetlinewidth{0.000000pt}%
\definecolor{currentstroke}{rgb}{0.000000,0.000000,0.000000}%
\pgfsetstrokecolor{currentstroke}%
\pgfsetdash{}{0pt}%
\pgfpathmoveto{\pgfqpoint{0.800000in}{0.574466in}}%
\pgfpathlineto{\pgfqpoint{0.843324in}{0.528000in}}%
\pgfpathlineto{\pgfqpoint{0.846542in}{0.528000in}}%
\pgfpathlineto{\pgfqpoint{0.800000in}{0.577910in}}%
\pgfpathmoveto{\pgfqpoint{4.768000in}{2.922815in}}%
\pgfpathlineto{\pgfqpoint{4.447354in}{3.253898in}}%
\pgfpathlineto{\pgfqpoint{4.111563in}{3.589333in}}%
\pgfpathlineto{\pgfqpoint{3.784576in}{3.905321in}}%
\pgfpathlineto{\pgfqpoint{3.684796in}{4.000000in}}%
\pgfpathlineto{\pgfqpoint{3.484648in}{4.186667in}}%
\pgfpathlineto{\pgfqpoint{3.444131in}{4.224000in}}%
\pgfpathlineto{\pgfqpoint{3.440610in}{4.224000in}}%
\pgfpathlineto{\pgfqpoint{3.743157in}{3.941408in}}%
\pgfpathlineto{\pgfqpoint{3.846141in}{3.843522in}}%
\pgfpathlineto{\pgfqpoint{4.031937in}{3.664000in}}%
\pgfpathlineto{\pgfqpoint{4.206869in}{3.491966in}}%
\pgfpathlineto{\pgfqpoint{4.370841in}{3.328000in}}%
\pgfpathlineto{\pgfqpoint{4.698586in}{2.992000in}}%
\pgfpathlineto{\pgfqpoint{4.768000in}{2.919449in}}%
\pgfpathlineto{\pgfqpoint{4.768000in}{2.919449in}}%
\pgfusepath{fill}%
\end{pgfscope}%
\begin{pgfscope}%
\pgfpathrectangle{\pgfqpoint{0.800000in}{0.528000in}}{\pgfqpoint{3.968000in}{3.696000in}}%
\pgfusepath{clip}%
\pgfsetbuttcap%
\pgfsetroundjoin%
\definecolor{currentfill}{rgb}{0.126453,0.570633,0.549841}%
\pgfsetfillcolor{currentfill}%
\pgfsetlinewidth{0.000000pt}%
\definecolor{currentstroke}{rgb}{0.000000,0.000000,0.000000}%
\pgfsetstrokecolor{currentstroke}%
\pgfsetdash{}{0pt}%
\pgfpathmoveto{\pgfqpoint{0.800000in}{0.571021in}}%
\pgfpathlineto{\pgfqpoint{0.840106in}{0.528000in}}%
\pgfpathlineto{\pgfqpoint{0.843324in}{0.528000in}}%
\pgfpathlineto{\pgfqpoint{0.800000in}{0.574466in}}%
\pgfpathmoveto{\pgfqpoint{4.768000in}{2.926181in}}%
\pgfpathlineto{\pgfqpoint{4.447354in}{3.257199in}}%
\pgfpathlineto{\pgfqpoint{4.114901in}{3.589333in}}%
\pgfpathlineto{\pgfqpoint{3.767161in}{3.925333in}}%
\pgfpathlineto{\pgfqpoint{3.605657in}{4.077469in}}%
\pgfpathlineto{\pgfqpoint{3.447619in}{4.224000in}}%
\pgfpathlineto{\pgfqpoint{3.444131in}{4.224000in}}%
\pgfpathlineto{\pgfqpoint{3.445333in}{4.222896in}}%
\pgfpathlineto{\pgfqpoint{3.645737in}{4.036723in}}%
\pgfpathlineto{\pgfqpoint{3.846141in}{3.846783in}}%
\pgfpathlineto{\pgfqpoint{4.035295in}{3.664000in}}%
\pgfpathlineto{\pgfqpoint{4.206869in}{3.495252in}}%
\pgfpathlineto{\pgfqpoint{4.389926in}{3.311842in}}%
\pgfpathlineto{\pgfqpoint{4.487434in}{3.213084in}}%
\pgfpathlineto{\pgfqpoint{4.768000in}{2.922815in}}%
\pgfpathlineto{\pgfqpoint{4.768000in}{2.922815in}}%
\pgfusepath{fill}%
\end{pgfscope}%
\begin{pgfscope}%
\pgfpathrectangle{\pgfqpoint{0.800000in}{0.528000in}}{\pgfqpoint{3.968000in}{3.696000in}}%
\pgfusepath{clip}%
\pgfsetbuttcap%
\pgfsetroundjoin%
\definecolor{currentfill}{rgb}{0.126453,0.570633,0.549841}%
\pgfsetfillcolor{currentfill}%
\pgfsetlinewidth{0.000000pt}%
\definecolor{currentstroke}{rgb}{0.000000,0.000000,0.000000}%
\pgfsetstrokecolor{currentstroke}%
\pgfsetdash{}{0pt}%
\pgfpathmoveto{\pgfqpoint{0.800000in}{0.567577in}}%
\pgfpathlineto{\pgfqpoint{0.836930in}{0.528000in}}%
\pgfpathlineto{\pgfqpoint{0.840106in}{0.528000in}}%
\pgfpathlineto{\pgfqpoint{0.840081in}{0.528027in}}%
\pgfpathlineto{\pgfqpoint{0.800000in}{0.571021in}}%
\pgfpathmoveto{\pgfqpoint{4.768000in}{2.929547in}}%
\pgfpathlineto{\pgfqpoint{4.447354in}{3.260501in}}%
\pgfpathlineto{\pgfqpoint{4.118239in}{3.589333in}}%
\pgfpathlineto{\pgfqpoint{3.770545in}{3.925333in}}%
\pgfpathlineto{\pgfqpoint{3.605657in}{4.080675in}}%
\pgfpathlineto{\pgfqpoint{3.451090in}{4.224000in}}%
\pgfpathlineto{\pgfqpoint{3.447619in}{4.224000in}}%
\pgfpathlineto{\pgfqpoint{3.647122in}{4.038623in}}%
\pgfpathlineto{\pgfqpoint{3.727765in}{3.962667in}}%
\pgfpathlineto{\pgfqpoint{4.076849in}{3.626667in}}%
\pgfpathlineto{\pgfqpoint{4.246949in}{3.458706in}}%
\pgfpathlineto{\pgfqpoint{4.430050in}{3.274549in}}%
\pgfpathlineto{\pgfqpoint{4.527515in}{3.175452in}}%
\pgfpathlineto{\pgfqpoint{4.768000in}{2.926181in}}%
\pgfpathlineto{\pgfqpoint{4.768000in}{2.926181in}}%
\pgfusepath{fill}%
\end{pgfscope}%
\begin{pgfscope}%
\pgfpathrectangle{\pgfqpoint{0.800000in}{0.528000in}}{\pgfqpoint{3.968000in}{3.696000in}}%
\pgfusepath{clip}%
\pgfsetbuttcap%
\pgfsetroundjoin%
\definecolor{currentfill}{rgb}{0.125394,0.574318,0.549086}%
\pgfsetfillcolor{currentfill}%
\pgfsetlinewidth{0.000000pt}%
\definecolor{currentstroke}{rgb}{0.000000,0.000000,0.000000}%
\pgfsetstrokecolor{currentstroke}%
\pgfsetdash{}{0pt}%
\pgfpathmoveto{\pgfqpoint{0.800000in}{0.564148in}}%
\pgfpathlineto{\pgfqpoint{0.833755in}{0.528000in}}%
\pgfpathlineto{\pgfqpoint{0.836930in}{0.528000in}}%
\pgfpathlineto{\pgfqpoint{0.800000in}{0.567577in}}%
\pgfpathlineto{\pgfqpoint{0.800000in}{0.565333in}}%
\pgfpathmoveto{\pgfqpoint{4.768000in}{2.932913in}}%
\pgfpathlineto{\pgfqpoint{4.447354in}{3.263803in}}%
\pgfpathlineto{\pgfqpoint{4.121577in}{3.589333in}}%
\pgfpathlineto{\pgfqpoint{3.773929in}{3.925333in}}%
\pgfpathlineto{\pgfqpoint{3.605657in}{4.083881in}}%
\pgfpathlineto{\pgfqpoint{3.454560in}{4.224000in}}%
\pgfpathlineto{\pgfqpoint{3.451090in}{4.224000in}}%
\pgfpathlineto{\pgfqpoint{3.645737in}{4.043148in}}%
\pgfpathlineto{\pgfqpoint{3.966384in}{3.737531in}}%
\pgfpathlineto{\pgfqpoint{4.161280in}{3.546869in}}%
\pgfpathlineto{\pgfqpoint{4.246949in}{3.461994in}}%
\pgfpathlineto{\pgfqpoint{4.431757in}{3.276139in}}%
\pgfpathlineto{\pgfqpoint{4.527644in}{3.178667in}}%
\pgfpathlineto{\pgfqpoint{4.768000in}{2.929547in}}%
\pgfpathlineto{\pgfqpoint{4.768000in}{2.929547in}}%
\pgfusepath{fill}%
\end{pgfscope}%
\begin{pgfscope}%
\pgfpathrectangle{\pgfqpoint{0.800000in}{0.528000in}}{\pgfqpoint{3.968000in}{3.696000in}}%
\pgfusepath{clip}%
\pgfsetbuttcap%
\pgfsetroundjoin%
\definecolor{currentfill}{rgb}{0.125394,0.574318,0.549086}%
\pgfsetfillcolor{currentfill}%
\pgfsetlinewidth{0.000000pt}%
\definecolor{currentstroke}{rgb}{0.000000,0.000000,0.000000}%
\pgfsetstrokecolor{currentstroke}%
\pgfsetdash{}{0pt}%
\pgfpathmoveto{\pgfqpoint{0.800000in}{0.560748in}}%
\pgfpathlineto{\pgfqpoint{0.830580in}{0.528000in}}%
\pgfpathlineto{\pgfqpoint{0.833755in}{0.528000in}}%
\pgfpathlineto{\pgfqpoint{0.800000in}{0.564148in}}%
\pgfpathmoveto{\pgfqpoint{4.768000in}{2.936279in}}%
\pgfpathlineto{\pgfqpoint{4.447354in}{3.267105in}}%
\pgfpathlineto{\pgfqpoint{4.124915in}{3.589333in}}%
\pgfpathlineto{\pgfqpoint{3.777313in}{3.925333in}}%
\pgfpathlineto{\pgfqpoint{3.605657in}{4.087087in}}%
\pgfpathlineto{\pgfqpoint{3.458031in}{4.224000in}}%
\pgfpathlineto{\pgfqpoint{3.454560in}{4.224000in}}%
\pgfpathlineto{\pgfqpoint{3.645737in}{4.046357in}}%
\pgfpathlineto{\pgfqpoint{3.968558in}{3.738667in}}%
\pgfpathlineto{\pgfqpoint{4.159467in}{3.552000in}}%
\pgfpathlineto{\pgfqpoint{4.457653in}{3.253333in}}%
\pgfpathlineto{\pgfqpoint{4.639638in}{3.066667in}}%
\pgfpathlineto{\pgfqpoint{4.768000in}{2.932913in}}%
\pgfpathlineto{\pgfqpoint{4.768000in}{2.932913in}}%
\pgfusepath{fill}%
\end{pgfscope}%
\begin{pgfscope}%
\pgfpathrectangle{\pgfqpoint{0.800000in}{0.528000in}}{\pgfqpoint{3.968000in}{3.696000in}}%
\pgfusepath{clip}%
\pgfsetbuttcap%
\pgfsetroundjoin%
\definecolor{currentfill}{rgb}{0.125394,0.574318,0.549086}%
\pgfsetfillcolor{currentfill}%
\pgfsetlinewidth{0.000000pt}%
\definecolor{currentstroke}{rgb}{0.000000,0.000000,0.000000}%
\pgfsetstrokecolor{currentstroke}%
\pgfsetdash{}{0pt}%
\pgfpathmoveto{\pgfqpoint{0.800000in}{0.557348in}}%
\pgfpathlineto{\pgfqpoint{0.827405in}{0.528000in}}%
\pgfpathlineto{\pgfqpoint{0.830580in}{0.528000in}}%
\pgfpathlineto{\pgfqpoint{0.800000in}{0.560748in}}%
\pgfpathmoveto{\pgfqpoint{4.768000in}{2.939645in}}%
\pgfpathlineto{\pgfqpoint{4.447354in}{3.270407in}}%
\pgfpathlineto{\pgfqpoint{4.126707in}{3.590833in}}%
\pgfpathlineto{\pgfqpoint{3.780697in}{3.925333in}}%
\pgfpathlineto{\pgfqpoint{3.605657in}{4.090293in}}%
\pgfpathlineto{\pgfqpoint{3.461501in}{4.224000in}}%
\pgfpathlineto{\pgfqpoint{3.458031in}{4.224000in}}%
\pgfpathlineto{\pgfqpoint{3.645737in}{4.049565in}}%
\pgfpathlineto{\pgfqpoint{3.971890in}{3.738667in}}%
\pgfpathlineto{\pgfqpoint{4.164722in}{3.550076in}}%
\pgfpathlineto{\pgfqpoint{4.246949in}{3.468571in}}%
\pgfpathlineto{\pgfqpoint{4.435171in}{3.279319in}}%
\pgfpathlineto{\pgfqpoint{4.534102in}{3.178667in}}%
\pgfpathlineto{\pgfqpoint{4.768000in}{2.936279in}}%
\pgfpathlineto{\pgfqpoint{4.768000in}{2.936279in}}%
\pgfusepath{fill}%
\end{pgfscope}%
\begin{pgfscope}%
\pgfpathrectangle{\pgfqpoint{0.800000in}{0.528000in}}{\pgfqpoint{3.968000in}{3.696000in}}%
\pgfusepath{clip}%
\pgfsetbuttcap%
\pgfsetroundjoin%
\definecolor{currentfill}{rgb}{0.125394,0.574318,0.549086}%
\pgfsetfillcolor{currentfill}%
\pgfsetlinewidth{0.000000pt}%
\definecolor{currentstroke}{rgb}{0.000000,0.000000,0.000000}%
\pgfsetstrokecolor{currentstroke}%
\pgfsetdash{}{0pt}%
\pgfpathmoveto{\pgfqpoint{0.800000in}{0.553948in}}%
\pgfpathlineto{\pgfqpoint{0.824230in}{0.528000in}}%
\pgfpathlineto{\pgfqpoint{0.827405in}{0.528000in}}%
\pgfpathlineto{\pgfqpoint{0.800000in}{0.557348in}}%
\pgfpathmoveto{\pgfqpoint{4.768000in}{2.943011in}}%
\pgfpathlineto{\pgfqpoint{4.447354in}{3.273708in}}%
\pgfpathlineto{\pgfqpoint{4.126707in}{3.594073in}}%
\pgfpathlineto{\pgfqpoint{3.784081in}{3.925333in}}%
\pgfpathlineto{\pgfqpoint{3.605657in}{4.093499in}}%
\pgfpathlineto{\pgfqpoint{3.464972in}{4.224000in}}%
\pgfpathlineto{\pgfqpoint{3.461501in}{4.224000in}}%
\pgfpathlineto{\pgfqpoint{3.645737in}{4.052774in}}%
\pgfpathlineto{\pgfqpoint{3.975222in}{3.738667in}}%
\pgfpathlineto{\pgfqpoint{4.166788in}{3.551343in}}%
\pgfpathlineto{\pgfqpoint{4.353398in}{3.365333in}}%
\pgfpathlineto{\pgfqpoint{4.527515in}{3.188721in}}%
\pgfpathlineto{\pgfqpoint{4.687838in}{3.023363in}}%
\pgfpathlineto{\pgfqpoint{4.768000in}{2.939645in}}%
\pgfpathlineto{\pgfqpoint{4.768000in}{2.939645in}}%
\pgfusepath{fill}%
\end{pgfscope}%
\begin{pgfscope}%
\pgfpathrectangle{\pgfqpoint{0.800000in}{0.528000in}}{\pgfqpoint{3.968000in}{3.696000in}}%
\pgfusepath{clip}%
\pgfsetbuttcap%
\pgfsetroundjoin%
\definecolor{currentfill}{rgb}{0.124395,0.578002,0.548287}%
\pgfsetfillcolor{currentfill}%
\pgfsetlinewidth{0.000000pt}%
\definecolor{currentstroke}{rgb}{0.000000,0.000000,0.000000}%
\pgfsetstrokecolor{currentstroke}%
\pgfsetdash{}{0pt}%
\pgfpathmoveto{\pgfqpoint{0.800000in}{0.550548in}}%
\pgfpathlineto{\pgfqpoint{0.821055in}{0.528000in}}%
\pgfpathlineto{\pgfqpoint{0.824230in}{0.528000in}}%
\pgfpathlineto{\pgfqpoint{0.800000in}{0.553948in}}%
\pgfpathmoveto{\pgfqpoint{4.768000in}{2.946377in}}%
\pgfpathlineto{\pgfqpoint{4.447354in}{3.277010in}}%
\pgfpathlineto{\pgfqpoint{4.126707in}{3.597313in}}%
\pgfpathlineto{\pgfqpoint{3.787465in}{3.925333in}}%
\pgfpathlineto{\pgfqpoint{3.605657in}{4.096706in}}%
\pgfpathlineto{\pgfqpoint{3.468442in}{4.224000in}}%
\pgfpathlineto{\pgfqpoint{3.464972in}{4.224000in}}%
\pgfpathlineto{\pgfqpoint{3.645737in}{4.055983in}}%
\pgfpathlineto{\pgfqpoint{3.978555in}{3.738667in}}%
\pgfpathlineto{\pgfqpoint{4.169413in}{3.552000in}}%
\pgfpathlineto{\pgfqpoint{4.356675in}{3.365333in}}%
\pgfpathlineto{\pgfqpoint{4.527515in}{3.192028in}}%
\pgfpathlineto{\pgfqpoint{4.687838in}{3.026723in}}%
\pgfpathlineto{\pgfqpoint{4.768000in}{2.943011in}}%
\pgfpathlineto{\pgfqpoint{4.768000in}{2.943011in}}%
\pgfusepath{fill}%
\end{pgfscope}%
\begin{pgfscope}%
\pgfpathrectangle{\pgfqpoint{0.800000in}{0.528000in}}{\pgfqpoint{3.968000in}{3.696000in}}%
\pgfusepath{clip}%
\pgfsetbuttcap%
\pgfsetroundjoin%
\definecolor{currentfill}{rgb}{0.124395,0.578002,0.548287}%
\pgfsetfillcolor{currentfill}%
\pgfsetlinewidth{0.000000pt}%
\definecolor{currentstroke}{rgb}{0.000000,0.000000,0.000000}%
\pgfsetstrokecolor{currentstroke}%
\pgfsetdash{}{0pt}%
\pgfpathmoveto{\pgfqpoint{0.800000in}{0.547148in}}%
\pgfpathlineto{\pgfqpoint{0.817880in}{0.528000in}}%
\pgfpathlineto{\pgfqpoint{0.821055in}{0.528000in}}%
\pgfpathlineto{\pgfqpoint{0.800000in}{0.550548in}}%
\pgfpathmoveto{\pgfqpoint{4.768000in}{2.949743in}}%
\pgfpathlineto{\pgfqpoint{4.447354in}{3.280312in}}%
\pgfpathlineto{\pgfqpoint{4.126707in}{3.600553in}}%
\pgfpathlineto{\pgfqpoint{3.790849in}{3.925333in}}%
\pgfpathlineto{\pgfqpoint{3.605657in}{4.099912in}}%
\pgfpathlineto{\pgfqpoint{3.471913in}{4.224000in}}%
\pgfpathlineto{\pgfqpoint{3.468442in}{4.224000in}}%
\pgfpathlineto{\pgfqpoint{3.645737in}{4.059192in}}%
\pgfpathlineto{\pgfqpoint{3.981887in}{3.738667in}}%
\pgfpathlineto{\pgfqpoint{4.166788in}{3.557837in}}%
\pgfpathlineto{\pgfqpoint{4.344060in}{3.381121in}}%
\pgfpathlineto{\pgfqpoint{4.447354in}{3.277010in}}%
\pgfpathlineto{\pgfqpoint{4.768000in}{2.946377in}}%
\pgfpathlineto{\pgfqpoint{4.768000in}{2.946377in}}%
\pgfusepath{fill}%
\end{pgfscope}%
\begin{pgfscope}%
\pgfpathrectangle{\pgfqpoint{0.800000in}{0.528000in}}{\pgfqpoint{3.968000in}{3.696000in}}%
\pgfusepath{clip}%
\pgfsetbuttcap%
\pgfsetroundjoin%
\definecolor{currentfill}{rgb}{0.124395,0.578002,0.548287}%
\pgfsetfillcolor{currentfill}%
\pgfsetlinewidth{0.000000pt}%
\definecolor{currentstroke}{rgb}{0.000000,0.000000,0.000000}%
\pgfsetstrokecolor{currentstroke}%
\pgfsetdash{}{0pt}%
\pgfpathmoveto{\pgfqpoint{0.800000in}{0.543748in}}%
\pgfpathlineto{\pgfqpoint{0.814705in}{0.528000in}}%
\pgfpathlineto{\pgfqpoint{0.817880in}{0.528000in}}%
\pgfpathlineto{\pgfqpoint{0.800000in}{0.547148in}}%
\pgfpathmoveto{\pgfqpoint{4.768000in}{2.953109in}}%
\pgfpathlineto{\pgfqpoint{4.462812in}{3.267732in}}%
\pgfpathlineto{\pgfqpoint{4.366507in}{3.365333in}}%
\pgfpathlineto{\pgfqpoint{4.026976in}{3.701333in}}%
\pgfpathlineto{\pgfqpoint{3.846141in}{3.875826in}}%
\pgfpathlineto{\pgfqpoint{3.660767in}{4.051333in}}%
\pgfpathlineto{\pgfqpoint{3.556044in}{4.149333in}}%
\pgfpathlineto{\pgfqpoint{3.475383in}{4.224000in}}%
\pgfpathlineto{\pgfqpoint{3.471913in}{4.224000in}}%
\pgfpathlineto{\pgfqpoint{3.645737in}{4.062400in}}%
\pgfpathlineto{\pgfqpoint{3.985219in}{3.738667in}}%
\pgfpathlineto{\pgfqpoint{4.166788in}{3.561079in}}%
\pgfpathlineto{\pgfqpoint{4.345752in}{3.382696in}}%
\pgfpathlineto{\pgfqpoint{4.447354in}{3.280312in}}%
\pgfpathlineto{\pgfqpoint{4.768000in}{2.949743in}}%
\pgfpathlineto{\pgfqpoint{4.768000in}{2.949743in}}%
\pgfusepath{fill}%
\end{pgfscope}%
\begin{pgfscope}%
\pgfpathrectangle{\pgfqpoint{0.800000in}{0.528000in}}{\pgfqpoint{3.968000in}{3.696000in}}%
\pgfusepath{clip}%
\pgfsetbuttcap%
\pgfsetroundjoin%
\definecolor{currentfill}{rgb}{0.123463,0.581687,0.547445}%
\pgfsetfillcolor{currentfill}%
\pgfsetlinewidth{0.000000pt}%
\definecolor{currentstroke}{rgb}{0.000000,0.000000,0.000000}%
\pgfsetstrokecolor{currentstroke}%
\pgfsetdash{}{0pt}%
\pgfpathmoveto{\pgfqpoint{0.800000in}{0.540348in}}%
\pgfpathlineto{\pgfqpoint{0.811530in}{0.528000in}}%
\pgfpathlineto{\pgfqpoint{0.814705in}{0.528000in}}%
\pgfpathlineto{\pgfqpoint{0.800000in}{0.543748in}}%
\pgfpathmoveto{\pgfqpoint{4.768000in}{2.956453in}}%
\pgfpathlineto{\pgfqpoint{4.589825in}{3.141333in}}%
\pgfpathlineto{\pgfqpoint{4.406785in}{3.328000in}}%
\pgfpathlineto{\pgfqpoint{4.220253in}{3.514667in}}%
\pgfpathlineto{\pgfqpoint{3.900567in}{3.826695in}}%
\pgfpathlineto{\pgfqpoint{3.797617in}{3.925333in}}%
\pgfpathlineto{\pgfqpoint{3.605657in}{4.106324in}}%
\pgfpathlineto{\pgfqpoint{3.478854in}{4.224000in}}%
\pgfpathlineto{\pgfqpoint{3.475383in}{4.224000in}}%
\pgfpathlineto{\pgfqpoint{3.645737in}{4.065609in}}%
\pgfpathlineto{\pgfqpoint{3.988551in}{3.738667in}}%
\pgfpathlineto{\pgfqpoint{4.166788in}{3.564322in}}%
\pgfpathlineto{\pgfqpoint{4.347443in}{3.384272in}}%
\pgfpathlineto{\pgfqpoint{4.447354in}{3.283614in}}%
\pgfpathlineto{\pgfqpoint{4.768000in}{2.953109in}}%
\pgfpathlineto{\pgfqpoint{4.768000in}{2.954667in}}%
\pgfusepath{fill}%
\end{pgfscope}%
\begin{pgfscope}%
\pgfpathrectangle{\pgfqpoint{0.800000in}{0.528000in}}{\pgfqpoint{3.968000in}{3.696000in}}%
\pgfusepath{clip}%
\pgfsetbuttcap%
\pgfsetroundjoin%
\definecolor{currentfill}{rgb}{0.123463,0.581687,0.547445}%
\pgfsetfillcolor{currentfill}%
\pgfsetlinewidth{0.000000pt}%
\definecolor{currentstroke}{rgb}{0.000000,0.000000,0.000000}%
\pgfsetstrokecolor{currentstroke}%
\pgfsetdash{}{0pt}%
\pgfpathmoveto{\pgfqpoint{0.800000in}{0.536947in}}%
\pgfpathlineto{\pgfqpoint{0.808355in}{0.528000in}}%
\pgfpathlineto{\pgfqpoint{0.811530in}{0.528000in}}%
\pgfpathlineto{\pgfqpoint{0.800000in}{0.540348in}}%
\pgfpathmoveto{\pgfqpoint{4.768000in}{2.959777in}}%
\pgfpathlineto{\pgfqpoint{4.593044in}{3.141333in}}%
\pgfpathlineto{\pgfqpoint{4.407273in}{3.330772in}}%
\pgfpathlineto{\pgfqpoint{4.206869in}{3.531187in}}%
\pgfpathlineto{\pgfqpoint{4.033619in}{3.701333in}}%
\pgfpathlineto{\pgfqpoint{3.846141in}{3.882269in}}%
\pgfpathlineto{\pgfqpoint{3.664178in}{4.054510in}}%
\pgfpathlineto{\pgfqpoint{3.562941in}{4.149333in}}%
\pgfpathlineto{\pgfqpoint{3.482324in}{4.224000in}}%
\pgfpathlineto{\pgfqpoint{3.478854in}{4.224000in}}%
\pgfpathlineto{\pgfqpoint{3.645737in}{4.068818in}}%
\pgfpathlineto{\pgfqpoint{3.991883in}{3.738667in}}%
\pgfpathlineto{\pgfqpoint{4.166788in}{3.567564in}}%
\pgfpathlineto{\pgfqpoint{4.349135in}{3.385847in}}%
\pgfpathlineto{\pgfqpoint{4.447354in}{3.286916in}}%
\pgfpathlineto{\pgfqpoint{4.768000in}{2.956453in}}%
\pgfpathlineto{\pgfqpoint{4.768000in}{2.956453in}}%
\pgfusepath{fill}%
\end{pgfscope}%
\begin{pgfscope}%
\pgfpathrectangle{\pgfqpoint{0.800000in}{0.528000in}}{\pgfqpoint{3.968000in}{3.696000in}}%
\pgfusepath{clip}%
\pgfsetbuttcap%
\pgfsetroundjoin%
\definecolor{currentfill}{rgb}{0.123463,0.581687,0.547445}%
\pgfsetfillcolor{currentfill}%
\pgfsetlinewidth{0.000000pt}%
\definecolor{currentstroke}{rgb}{0.000000,0.000000,0.000000}%
\pgfsetstrokecolor{currentstroke}%
\pgfsetdash{}{0pt}%
\pgfpathmoveto{\pgfqpoint{0.800000in}{0.533547in}}%
\pgfpathlineto{\pgfqpoint{0.805180in}{0.528000in}}%
\pgfpathlineto{\pgfqpoint{0.808355in}{0.528000in}}%
\pgfpathlineto{\pgfqpoint{0.800000in}{0.536947in}}%
\pgfpathmoveto{\pgfqpoint{4.768000in}{2.963100in}}%
\pgfpathlineto{\pgfqpoint{4.596263in}{3.141333in}}%
\pgfpathlineto{\pgfqpoint{4.413238in}{3.328000in}}%
\pgfpathlineto{\pgfqpoint{4.113303in}{3.626667in}}%
\pgfpathlineto{\pgfqpoint{3.926303in}{3.808541in}}%
\pgfpathlineto{\pgfqpoint{3.765105in}{3.962667in}}%
\pgfpathlineto{\pgfqpoint{3.565576in}{4.150080in}}%
\pgfpathlineto{\pgfqpoint{3.485414in}{4.224000in}}%
\pgfpathlineto{\pgfqpoint{3.482324in}{4.224000in}}%
\pgfpathlineto{\pgfqpoint{3.645737in}{4.072027in}}%
\pgfpathlineto{\pgfqpoint{3.995215in}{3.738667in}}%
\pgfpathlineto{\pgfqpoint{4.166788in}{3.570807in}}%
\pgfpathlineto{\pgfqpoint{4.350826in}{3.387423in}}%
\pgfpathlineto{\pgfqpoint{4.447354in}{3.290217in}}%
\pgfpathlineto{\pgfqpoint{4.768000in}{2.959777in}}%
\pgfpathlineto{\pgfqpoint{4.768000in}{2.959777in}}%
\pgfusepath{fill}%
\end{pgfscope}%
\begin{pgfscope}%
\pgfpathrectangle{\pgfqpoint{0.800000in}{0.528000in}}{\pgfqpoint{3.968000in}{3.696000in}}%
\pgfusepath{clip}%
\pgfsetbuttcap%
\pgfsetroundjoin%
\definecolor{currentfill}{rgb}{0.123463,0.581687,0.547445}%
\pgfsetfillcolor{currentfill}%
\pgfsetlinewidth{0.000000pt}%
\definecolor{currentstroke}{rgb}{0.000000,0.000000,0.000000}%
\pgfsetstrokecolor{currentstroke}%
\pgfsetdash{}{0pt}%
\pgfpathmoveto{\pgfqpoint{0.800000in}{0.530147in}}%
\pgfpathlineto{\pgfqpoint{0.802005in}{0.528000in}}%
\pgfpathlineto{\pgfqpoint{0.805180in}{0.528000in}}%
\pgfpathlineto{\pgfqpoint{0.800000in}{0.533547in}}%
\pgfpathmoveto{\pgfqpoint{4.768000in}{2.966424in}}%
\pgfpathlineto{\pgfqpoint{4.599482in}{3.141333in}}%
\pgfpathlineto{\pgfqpoint{4.416461in}{3.328000in}}%
\pgfpathlineto{\pgfqpoint{4.116605in}{3.626667in}}%
\pgfpathlineto{\pgfqpoint{3.924677in}{3.813333in}}%
\pgfpathlineto{\pgfqpoint{3.569777in}{4.149333in}}%
\pgfpathlineto{\pgfqpoint{3.489210in}{4.224000in}}%
\pgfpathlineto{\pgfqpoint{3.485789in}{4.224000in}}%
\pgfpathlineto{\pgfqpoint{3.843514in}{3.888000in}}%
\pgfpathlineto{\pgfqpoint{4.006465in}{3.730986in}}%
\pgfpathlineto{\pgfqpoint{4.189103in}{3.552000in}}%
\pgfpathlineto{\pgfqpoint{4.367192in}{3.374421in}}%
\pgfpathlineto{\pgfqpoint{4.527515in}{3.211871in}}%
\pgfpathlineto{\pgfqpoint{4.704505in}{3.029333in}}%
\pgfpathlineto{\pgfqpoint{4.768000in}{2.963100in}}%
\pgfpathlineto{\pgfqpoint{4.768000in}{2.963100in}}%
\pgfusepath{fill}%
\end{pgfscope}%
\begin{pgfscope}%
\pgfpathrectangle{\pgfqpoint{0.800000in}{0.528000in}}{\pgfqpoint{3.968000in}{3.696000in}}%
\pgfusepath{clip}%
\pgfsetbuttcap%
\pgfsetroundjoin%
\definecolor{currentfill}{rgb}{0.122606,0.585371,0.546557}%
\pgfsetfillcolor{currentfill}%
\pgfsetlinewidth{0.000000pt}%
\definecolor{currentstroke}{rgb}{0.000000,0.000000,0.000000}%
\pgfsetstrokecolor{currentstroke}%
\pgfsetdash{}{0pt}%
\pgfpathmoveto{\pgfqpoint{0.802005in}{0.528000in}}%
\pgfpathlineto{\pgfqpoint{0.800000in}{0.530147in}}%
\pgfpathlineto{\pgfqpoint{0.800000in}{0.528000in}}%
\pgfpathmoveto{\pgfqpoint{4.768000in}{2.969748in}}%
\pgfpathlineto{\pgfqpoint{4.602701in}{3.141333in}}%
\pgfpathlineto{\pgfqpoint{4.419684in}{3.328000in}}%
\pgfpathlineto{\pgfqpoint{4.086626in}{3.659298in}}%
\pgfpathlineto{\pgfqpoint{3.765980in}{3.968203in}}%
\pgfpathlineto{\pgfqpoint{3.565576in}{4.156411in}}%
\pgfpathlineto{\pgfqpoint{3.492630in}{4.224000in}}%
\pgfpathlineto{\pgfqpoint{3.489210in}{4.224000in}}%
\pgfpathlineto{\pgfqpoint{3.807745in}{3.925333in}}%
\pgfpathlineto{\pgfqpoint{4.001879in}{3.738667in}}%
\pgfpathlineto{\pgfqpoint{4.166788in}{3.577292in}}%
\pgfpathlineto{\pgfqpoint{4.354209in}{3.390574in}}%
\pgfpathlineto{\pgfqpoint{4.453344in}{3.290667in}}%
\pgfpathlineto{\pgfqpoint{4.768000in}{2.966424in}}%
\pgfpathlineto{\pgfqpoint{4.768000in}{2.966424in}}%
\pgfusepath{fill}%
\end{pgfscope}%
\begin{pgfscope}%
\pgfpathrectangle{\pgfqpoint{0.800000in}{0.528000in}}{\pgfqpoint{3.968000in}{3.696000in}}%
\pgfusepath{clip}%
\pgfsetbuttcap%
\pgfsetroundjoin%
\definecolor{currentfill}{rgb}{0.122606,0.585371,0.546557}%
\pgfsetfillcolor{currentfill}%
\pgfsetlinewidth{0.000000pt}%
\definecolor{currentstroke}{rgb}{0.000000,0.000000,0.000000}%
\pgfsetstrokecolor{currentstroke}%
\pgfsetdash{}{0pt}%
\pgfpathmoveto{\pgfqpoint{4.768000in}{2.973072in}}%
\pgfpathlineto{\pgfqpoint{4.589016in}{3.158715in}}%
\pgfpathlineto{\pgfqpoint{4.407273in}{3.343806in}}%
\pgfpathlineto{\pgfqpoint{4.222123in}{3.528875in}}%
\pgfpathlineto{\pgfqpoint{4.123208in}{3.626667in}}%
\pgfpathlineto{\pgfqpoint{3.931311in}{3.813333in}}%
\pgfpathlineto{\pgfqpoint{3.765980in}{3.971381in}}%
\pgfpathlineto{\pgfqpoint{3.565576in}{4.159576in}}%
\pgfpathlineto{\pgfqpoint{3.496051in}{4.224000in}}%
\pgfpathlineto{\pgfqpoint{3.492630in}{4.224000in}}%
\pgfpathlineto{\pgfqpoint{3.811081in}{3.925333in}}%
\pgfpathlineto{\pgfqpoint{4.006465in}{3.737451in}}%
\pgfpathlineto{\pgfqpoint{4.206869in}{3.540923in}}%
\pgfpathlineto{\pgfqpoint{4.554284in}{3.191066in}}%
\pgfpathlineto{\pgfqpoint{4.710886in}{3.029333in}}%
\pgfpathlineto{\pgfqpoint{4.768000in}{2.969748in}}%
\pgfpathlineto{\pgfqpoint{4.768000in}{2.969748in}}%
\pgfusepath{fill}%
\end{pgfscope}%
\begin{pgfscope}%
\pgfpathrectangle{\pgfqpoint{0.800000in}{0.528000in}}{\pgfqpoint{3.968000in}{3.696000in}}%
\pgfusepath{clip}%
\pgfsetbuttcap%
\pgfsetroundjoin%
\definecolor{currentfill}{rgb}{0.122606,0.585371,0.546557}%
\pgfsetfillcolor{currentfill}%
\pgfsetlinewidth{0.000000pt}%
\definecolor{currentstroke}{rgb}{0.000000,0.000000,0.000000}%
\pgfsetstrokecolor{currentstroke}%
\pgfsetdash{}{0pt}%
\pgfpathmoveto{\pgfqpoint{4.768000in}{2.976395in}}%
\pgfpathlineto{\pgfqpoint{4.607677in}{3.142820in}}%
\pgfpathlineto{\pgfqpoint{4.407273in}{3.347064in}}%
\pgfpathlineto{\pgfqpoint{4.223801in}{3.530438in}}%
\pgfpathlineto{\pgfqpoint{4.122760in}{3.630343in}}%
\pgfpathlineto{\pgfqpoint{3.926303in}{3.821351in}}%
\pgfpathlineto{\pgfqpoint{3.579974in}{4.149333in}}%
\pgfpathlineto{\pgfqpoint{3.499471in}{4.224000in}}%
\pgfpathlineto{\pgfqpoint{3.496051in}{4.224000in}}%
\pgfpathlineto{\pgfqpoint{3.814418in}{3.925333in}}%
\pgfpathlineto{\pgfqpoint{4.008514in}{3.738667in}}%
\pgfpathlineto{\pgfqpoint{4.198949in}{3.552000in}}%
\pgfpathlineto{\pgfqpoint{4.367192in}{3.384188in}}%
\pgfpathlineto{\pgfqpoint{4.533111in}{3.216000in}}%
\pgfpathlineto{\pgfqpoint{4.714077in}{3.029333in}}%
\pgfpathlineto{\pgfqpoint{4.768000in}{2.973072in}}%
\pgfpathlineto{\pgfqpoint{4.768000in}{2.973072in}}%
\pgfusepath{fill}%
\end{pgfscope}%
\begin{pgfscope}%
\pgfpathrectangle{\pgfqpoint{0.800000in}{0.528000in}}{\pgfqpoint{3.968000in}{3.696000in}}%
\pgfusepath{clip}%
\pgfsetbuttcap%
\pgfsetroundjoin%
\definecolor{currentfill}{rgb}{0.122606,0.585371,0.546557}%
\pgfsetfillcolor{currentfill}%
\pgfsetlinewidth{0.000000pt}%
\definecolor{currentstroke}{rgb}{0.000000,0.000000,0.000000}%
\pgfsetstrokecolor{currentstroke}%
\pgfsetdash{}{0pt}%
\pgfpathmoveto{\pgfqpoint{4.768000in}{2.979719in}}%
\pgfpathlineto{\pgfqpoint{4.607677in}{3.146092in}}%
\pgfpathlineto{\pgfqpoint{4.418702in}{3.338646in}}%
\pgfpathlineto{\pgfqpoint{4.318035in}{3.440000in}}%
\pgfpathlineto{\pgfqpoint{4.126707in}{3.629676in}}%
\pgfpathlineto{\pgfqpoint{3.926303in}{3.824539in}}%
\pgfpathlineto{\pgfqpoint{3.583373in}{4.149333in}}%
\pgfpathlineto{\pgfqpoint{3.502892in}{4.224000in}}%
\pgfpathlineto{\pgfqpoint{3.499471in}{4.224000in}}%
\pgfpathlineto{\pgfqpoint{3.817754in}{3.925333in}}%
\pgfpathlineto{\pgfqpoint{4.006465in}{3.743851in}}%
\pgfpathlineto{\pgfqpoint{4.166788in}{3.587020in}}%
\pgfpathlineto{\pgfqpoint{4.367192in}{3.387444in}}%
\pgfpathlineto{\pgfqpoint{4.536306in}{3.216000in}}%
\pgfpathlineto{\pgfqpoint{4.717268in}{3.029333in}}%
\pgfpathlineto{\pgfqpoint{4.768000in}{2.976395in}}%
\pgfpathlineto{\pgfqpoint{4.768000in}{2.976395in}}%
\pgfusepath{fill}%
\end{pgfscope}%
\begin{pgfscope}%
\pgfpathrectangle{\pgfqpoint{0.800000in}{0.528000in}}{\pgfqpoint{3.968000in}{3.696000in}}%
\pgfusepath{clip}%
\pgfsetbuttcap%
\pgfsetroundjoin%
\definecolor{currentfill}{rgb}{0.121831,0.589055,0.545623}%
\pgfsetfillcolor{currentfill}%
\pgfsetlinewidth{0.000000pt}%
\definecolor{currentstroke}{rgb}{0.000000,0.000000,0.000000}%
\pgfsetstrokecolor{currentstroke}%
\pgfsetdash{}{0pt}%
\pgfpathmoveto{\pgfqpoint{4.768000in}{2.983043in}}%
\pgfpathlineto{\pgfqpoint{4.607677in}{3.149364in}}%
\pgfpathlineto{\pgfqpoint{4.420370in}{3.340200in}}%
\pgfpathlineto{\pgfqpoint{4.321287in}{3.440000in}}%
\pgfpathlineto{\pgfqpoint{4.133025in}{3.626667in}}%
\pgfpathlineto{\pgfqpoint{3.815299in}{3.933938in}}%
\pgfpathlineto{\pgfqpoint{3.725899in}{4.018841in}}%
\pgfpathlineto{\pgfqpoint{3.546622in}{4.186667in}}%
\pgfpathlineto{\pgfqpoint{3.506312in}{4.224000in}}%
\pgfpathlineto{\pgfqpoint{3.502892in}{4.224000in}}%
\pgfpathlineto{\pgfqpoint{3.821091in}{3.925333in}}%
\pgfpathlineto{\pgfqpoint{4.006465in}{3.747045in}}%
\pgfpathlineto{\pgfqpoint{4.183194in}{3.574052in}}%
\pgfpathlineto{\pgfqpoint{4.367192in}{3.390700in}}%
\pgfpathlineto{\pgfqpoint{4.539501in}{3.216000in}}%
\pgfpathlineto{\pgfqpoint{4.720459in}{3.029333in}}%
\pgfpathlineto{\pgfqpoint{4.768000in}{2.979719in}}%
\pgfpathlineto{\pgfqpoint{4.768000in}{2.979719in}}%
\pgfusepath{fill}%
\end{pgfscope}%
\begin{pgfscope}%
\pgfpathrectangle{\pgfqpoint{0.800000in}{0.528000in}}{\pgfqpoint{3.968000in}{3.696000in}}%
\pgfusepath{clip}%
\pgfsetbuttcap%
\pgfsetroundjoin%
\definecolor{currentfill}{rgb}{0.121831,0.589055,0.545623}%
\pgfsetfillcolor{currentfill}%
\pgfsetlinewidth{0.000000pt}%
\definecolor{currentstroke}{rgb}{0.000000,0.000000,0.000000}%
\pgfsetstrokecolor{currentstroke}%
\pgfsetdash{}{0pt}%
\pgfpathmoveto{\pgfqpoint{4.768000in}{2.986367in}}%
\pgfpathlineto{\pgfqpoint{4.607677in}{3.152636in}}%
\pgfpathlineto{\pgfqpoint{4.422039in}{3.341754in}}%
\pgfpathlineto{\pgfqpoint{4.324540in}{3.440000in}}%
\pgfpathlineto{\pgfqpoint{4.136282in}{3.626667in}}%
\pgfpathlineto{\pgfqpoint{3.816977in}{3.935501in}}%
\pgfpathlineto{\pgfqpoint{3.725899in}{4.022016in}}%
\pgfpathlineto{\pgfqpoint{3.550032in}{4.186667in}}%
\pgfpathlineto{\pgfqpoint{3.509733in}{4.224000in}}%
\pgfpathlineto{\pgfqpoint{3.506312in}{4.224000in}}%
\pgfpathlineto{\pgfqpoint{3.824427in}{3.925333in}}%
\pgfpathlineto{\pgfqpoint{4.006465in}{3.750238in}}%
\pgfpathlineto{\pgfqpoint{4.170965in}{3.589333in}}%
\pgfpathlineto{\pgfqpoint{4.358516in}{3.402667in}}%
\pgfpathlineto{\pgfqpoint{4.687838in}{3.066575in}}%
\pgfpathlineto{\pgfqpoint{4.768000in}{2.983043in}}%
\pgfpathlineto{\pgfqpoint{4.768000in}{2.983043in}}%
\pgfusepath{fill}%
\end{pgfscope}%
\begin{pgfscope}%
\pgfpathrectangle{\pgfqpoint{0.800000in}{0.528000in}}{\pgfqpoint{3.968000in}{3.696000in}}%
\pgfusepath{clip}%
\pgfsetbuttcap%
\pgfsetroundjoin%
\definecolor{currentfill}{rgb}{0.121831,0.589055,0.545623}%
\pgfsetfillcolor{currentfill}%
\pgfsetlinewidth{0.000000pt}%
\definecolor{currentstroke}{rgb}{0.000000,0.000000,0.000000}%
\pgfsetstrokecolor{currentstroke}%
\pgfsetdash{}{0pt}%
\pgfpathmoveto{\pgfqpoint{4.768000in}{2.989690in}}%
\pgfpathlineto{\pgfqpoint{4.607677in}{3.155907in}}%
\pgfpathlineto{\pgfqpoint{4.423707in}{3.343308in}}%
\pgfpathlineto{\pgfqpoint{4.327111in}{3.440673in}}%
\pgfpathlineto{\pgfqpoint{4.126707in}{3.639278in}}%
\pgfpathlineto{\pgfqpoint{3.947841in}{3.813333in}}%
\pgfpathlineto{\pgfqpoint{3.765980in}{3.987271in}}%
\pgfpathlineto{\pgfqpoint{3.579494in}{4.162297in}}%
\pgfpathlineto{\pgfqpoint{3.513153in}{4.224000in}}%
\pgfpathlineto{\pgfqpoint{3.509733in}{4.224000in}}%
\pgfpathlineto{\pgfqpoint{3.827764in}{3.925333in}}%
\pgfpathlineto{\pgfqpoint{4.006465in}{3.753431in}}%
\pgfpathlineto{\pgfqpoint{4.174211in}{3.589333in}}%
\pgfpathlineto{\pgfqpoint{4.364358in}{3.400027in}}%
\pgfpathlineto{\pgfqpoint{4.447354in}{3.316312in}}%
\pgfpathlineto{\pgfqpoint{4.632381in}{3.127011in}}%
\pgfpathlineto{\pgfqpoint{4.727919in}{3.028210in}}%
\pgfpathlineto{\pgfqpoint{4.768000in}{2.986367in}}%
\pgfpathlineto{\pgfqpoint{4.768000in}{2.986367in}}%
\pgfusepath{fill}%
\end{pgfscope}%
\begin{pgfscope}%
\pgfpathrectangle{\pgfqpoint{0.800000in}{0.528000in}}{\pgfqpoint{3.968000in}{3.696000in}}%
\pgfusepath{clip}%
\pgfsetbuttcap%
\pgfsetroundjoin%
\definecolor{currentfill}{rgb}{0.121831,0.589055,0.545623}%
\pgfsetfillcolor{currentfill}%
\pgfsetlinewidth{0.000000pt}%
\definecolor{currentstroke}{rgb}{0.000000,0.000000,0.000000}%
\pgfsetstrokecolor{currentstroke}%
\pgfsetdash{}{0pt}%
\pgfpathmoveto{\pgfqpoint{4.768000in}{2.993001in}}%
\pgfpathlineto{\pgfqpoint{4.442247in}{3.328000in}}%
\pgfpathlineto{\pgfqpoint{4.104746in}{3.664000in}}%
\pgfpathlineto{\pgfqpoint{3.780668in}{3.976348in}}%
\pgfpathlineto{\pgfqpoint{3.676738in}{4.074667in}}%
\pgfpathlineto{\pgfqpoint{3.516574in}{4.224000in}}%
\pgfpathlineto{\pgfqpoint{3.513153in}{4.224000in}}%
\pgfpathlineto{\pgfqpoint{3.831100in}{3.925333in}}%
\pgfpathlineto{\pgfqpoint{4.006465in}{3.756624in}}%
\pgfpathlineto{\pgfqpoint{4.177458in}{3.589333in}}%
\pgfpathlineto{\pgfqpoint{4.367192in}{3.400467in}}%
\pgfpathlineto{\pgfqpoint{4.549086in}{3.216000in}}%
\pgfpathlineto{\pgfqpoint{4.730003in}{3.029333in}}%
\pgfpathlineto{\pgfqpoint{4.768000in}{2.989690in}}%
\pgfpathlineto{\pgfqpoint{4.768000in}{2.992000in}}%
\pgfpathlineto{\pgfqpoint{4.768000in}{2.992000in}}%
\pgfusepath{fill}%
\end{pgfscope}%
\begin{pgfscope}%
\pgfpathrectangle{\pgfqpoint{0.800000in}{0.528000in}}{\pgfqpoint{3.968000in}{3.696000in}}%
\pgfusepath{clip}%
\pgfsetbuttcap%
\pgfsetroundjoin%
\definecolor{currentfill}{rgb}{0.121148,0.592739,0.544641}%
\pgfsetfillcolor{currentfill}%
\pgfsetlinewidth{0.000000pt}%
\definecolor{currentstroke}{rgb}{0.000000,0.000000,0.000000}%
\pgfsetstrokecolor{currentstroke}%
\pgfsetdash{}{0pt}%
\pgfpathmoveto{\pgfqpoint{4.768000in}{2.996284in}}%
\pgfpathlineto{\pgfqpoint{4.445471in}{3.328000in}}%
\pgfpathlineto{\pgfqpoint{4.108012in}{3.664000in}}%
\pgfpathlineto{\pgfqpoint{3.765980in}{3.993626in}}%
\pgfpathlineto{\pgfqpoint{3.582873in}{4.165445in}}%
\pgfpathlineto{\pgfqpoint{3.519995in}{4.224000in}}%
\pgfpathlineto{\pgfqpoint{3.516574in}{4.224000in}}%
\pgfpathlineto{\pgfqpoint{3.834437in}{3.925333in}}%
\pgfpathlineto{\pgfqpoint{4.006465in}{3.759817in}}%
\pgfpathlineto{\pgfqpoint{4.180705in}{3.589333in}}%
\pgfpathlineto{\pgfqpoint{4.368230in}{3.402667in}}%
\pgfpathlineto{\pgfqpoint{4.552280in}{3.216000in}}%
\pgfpathlineto{\pgfqpoint{4.727919in}{3.034784in}}%
\pgfpathlineto{\pgfqpoint{4.768000in}{2.993001in}}%
\pgfpathlineto{\pgfqpoint{4.768000in}{2.993001in}}%
\pgfusepath{fill}%
\end{pgfscope}%
\begin{pgfscope}%
\pgfpathrectangle{\pgfqpoint{0.800000in}{0.528000in}}{\pgfqpoint{3.968000in}{3.696000in}}%
\pgfusepath{clip}%
\pgfsetbuttcap%
\pgfsetroundjoin%
\definecolor{currentfill}{rgb}{0.121148,0.592739,0.544641}%
\pgfsetfillcolor{currentfill}%
\pgfsetlinewidth{0.000000pt}%
\definecolor{currentstroke}{rgb}{0.000000,0.000000,0.000000}%
\pgfsetstrokecolor{currentstroke}%
\pgfsetdash{}{0pt}%
\pgfpathmoveto{\pgfqpoint{4.768000in}{2.999567in}}%
\pgfpathlineto{\pgfqpoint{4.447354in}{3.329340in}}%
\pgfpathlineto{\pgfqpoint{4.111278in}{3.664000in}}%
\pgfpathlineto{\pgfqpoint{3.765980in}{3.996804in}}%
\pgfpathlineto{\pgfqpoint{3.584563in}{4.167019in}}%
\pgfpathlineto{\pgfqpoint{3.523415in}{4.224000in}}%
\pgfpathlineto{\pgfqpoint{3.519995in}{4.224000in}}%
\pgfpathlineto{\pgfqpoint{3.837773in}{3.925333in}}%
\pgfpathlineto{\pgfqpoint{4.006465in}{3.763010in}}%
\pgfpathlineto{\pgfqpoint{4.183952in}{3.589333in}}%
\pgfpathlineto{\pgfqpoint{4.371428in}{3.402667in}}%
\pgfpathlineto{\pgfqpoint{4.555475in}{3.216000in}}%
\pgfpathlineto{\pgfqpoint{4.727919in}{3.038064in}}%
\pgfpathlineto{\pgfqpoint{4.768000in}{2.996284in}}%
\pgfpathlineto{\pgfqpoint{4.768000in}{2.996284in}}%
\pgfusepath{fill}%
\end{pgfscope}%
\begin{pgfscope}%
\pgfpathrectangle{\pgfqpoint{0.800000in}{0.528000in}}{\pgfqpoint{3.968000in}{3.696000in}}%
\pgfusepath{clip}%
\pgfsetbuttcap%
\pgfsetroundjoin%
\definecolor{currentfill}{rgb}{0.121148,0.592739,0.544641}%
\pgfsetfillcolor{currentfill}%
\pgfsetlinewidth{0.000000pt}%
\definecolor{currentstroke}{rgb}{0.000000,0.000000,0.000000}%
\pgfsetstrokecolor{currentstroke}%
\pgfsetdash{}{0pt}%
\pgfpathmoveto{\pgfqpoint{4.768000in}{3.002849in}}%
\pgfpathlineto{\pgfqpoint{4.447354in}{3.332561in}}%
\pgfpathlineto{\pgfqpoint{4.114545in}{3.664000in}}%
\pgfpathlineto{\pgfqpoint{3.765961in}{4.000000in}}%
\pgfpathlineto{\pgfqpoint{3.525495in}{4.224000in}}%
\pgfpathlineto{\pgfqpoint{3.523415in}{4.224000in}}%
\pgfpathlineto{\pgfqpoint{3.524453in}{4.223029in}}%
\pgfpathlineto{\pgfqpoint{3.605657in}{4.147573in}}%
\pgfpathlineto{\pgfqpoint{3.957759in}{3.813333in}}%
\pgfpathlineto{\pgfqpoint{4.274328in}{3.502835in}}%
\pgfpathlineto{\pgfqpoint{4.374627in}{3.402667in}}%
\pgfpathlineto{\pgfqpoint{4.558670in}{3.216000in}}%
\pgfpathlineto{\pgfqpoint{4.727919in}{3.041343in}}%
\pgfpathlineto{\pgfqpoint{4.768000in}{2.999567in}}%
\pgfpathlineto{\pgfqpoint{4.768000in}{2.999567in}}%
\pgfusepath{fill}%
\end{pgfscope}%
\begin{pgfscope}%
\pgfpathrectangle{\pgfqpoint{0.800000in}{0.528000in}}{\pgfqpoint{3.968000in}{3.696000in}}%
\pgfusepath{clip}%
\pgfsetbuttcap%
\pgfsetroundjoin%
\definecolor{currentfill}{rgb}{0.120565,0.596422,0.543611}%
\pgfsetfillcolor{currentfill}%
\pgfsetlinewidth{0.000000pt}%
\definecolor{currentstroke}{rgb}{0.000000,0.000000,0.000000}%
\pgfsetstrokecolor{currentstroke}%
\pgfsetdash{}{0pt}%
\pgfpathmoveto{\pgfqpoint{4.768000in}{3.006132in}}%
\pgfpathlineto{\pgfqpoint{4.447354in}{3.335782in}}%
\pgfpathlineto{\pgfqpoint{4.117811in}{3.664000in}}%
\pgfpathlineto{\pgfqpoint{3.765980in}{4.003122in}}%
\pgfpathlineto{\pgfqpoint{3.530189in}{4.224000in}}%
\pgfpathlineto{\pgfqpoint{3.526817in}{4.224000in}}%
\pgfpathlineto{\pgfqpoint{3.726484in}{4.037333in}}%
\pgfpathlineto{\pgfqpoint{4.076387in}{3.701333in}}%
\pgfpathlineto{\pgfqpoint{4.246949in}{3.533405in}}%
\pgfpathlineto{\pgfqpoint{4.567596in}{3.210136in}}%
\pgfpathlineto{\pgfqpoint{4.768000in}{3.002849in}}%
\pgfpathlineto{\pgfqpoint{4.768000in}{3.002849in}}%
\pgfusepath{fill}%
\end{pgfscope}%
\begin{pgfscope}%
\pgfpathrectangle{\pgfqpoint{0.800000in}{0.528000in}}{\pgfqpoint{3.968000in}{3.696000in}}%
\pgfusepath{clip}%
\pgfsetbuttcap%
\pgfsetroundjoin%
\definecolor{currentfill}{rgb}{0.120565,0.596422,0.543611}%
\pgfsetfillcolor{currentfill}%
\pgfsetlinewidth{0.000000pt}%
\definecolor{currentstroke}{rgb}{0.000000,0.000000,0.000000}%
\pgfsetstrokecolor{currentstroke}%
\pgfsetdash{}{0pt}%
\pgfpathmoveto{\pgfqpoint{4.768000in}{3.009414in}}%
\pgfpathlineto{\pgfqpoint{4.447354in}{3.339004in}}%
\pgfpathlineto{\pgfqpoint{4.121077in}{3.664000in}}%
\pgfpathlineto{\pgfqpoint{3.769293in}{4.003086in}}%
\pgfpathlineto{\pgfqpoint{3.685818in}{4.081915in}}%
\pgfpathlineto{\pgfqpoint{3.533561in}{4.224000in}}%
\pgfpathlineto{\pgfqpoint{3.530189in}{4.224000in}}%
\pgfpathlineto{\pgfqpoint{3.729805in}{4.037333in}}%
\pgfpathlineto{\pgfqpoint{4.079663in}{3.701333in}}%
\pgfpathlineto{\pgfqpoint{4.246949in}{3.536614in}}%
\pgfpathlineto{\pgfqpoint{4.567596in}{3.213405in}}%
\pgfpathlineto{\pgfqpoint{4.768000in}{3.006132in}}%
\pgfpathlineto{\pgfqpoint{4.768000in}{3.006132in}}%
\pgfusepath{fill}%
\end{pgfscope}%
\begin{pgfscope}%
\pgfpathrectangle{\pgfqpoint{0.800000in}{0.528000in}}{\pgfqpoint{3.968000in}{3.696000in}}%
\pgfusepath{clip}%
\pgfsetbuttcap%
\pgfsetroundjoin%
\definecolor{currentfill}{rgb}{0.120565,0.596422,0.543611}%
\pgfsetfillcolor{currentfill}%
\pgfsetlinewidth{0.000000pt}%
\definecolor{currentstroke}{rgb}{0.000000,0.000000,0.000000}%
\pgfsetstrokecolor{currentstroke}%
\pgfsetdash{}{0pt}%
\pgfpathmoveto{\pgfqpoint{4.768000in}{3.012697in}}%
\pgfpathlineto{\pgfqpoint{4.447354in}{3.342225in}}%
\pgfpathlineto{\pgfqpoint{4.124343in}{3.664000in}}%
\pgfpathlineto{\pgfqpoint{3.770954in}{4.004633in}}%
\pgfpathlineto{\pgfqpoint{3.685818in}{4.085050in}}%
\pgfpathlineto{\pgfqpoint{3.536933in}{4.224000in}}%
\pgfpathlineto{\pgfqpoint{3.533561in}{4.224000in}}%
\pgfpathlineto{\pgfqpoint{3.725899in}{4.044162in}}%
\pgfpathlineto{\pgfqpoint{4.046545in}{3.736833in}}%
\pgfpathlineto{\pgfqpoint{4.384224in}{3.402667in}}%
\pgfpathlineto{\pgfqpoint{4.568246in}{3.216000in}}%
\pgfpathlineto{\pgfqpoint{4.768000in}{3.009414in}}%
\pgfpathlineto{\pgfqpoint{4.768000in}{3.009414in}}%
\pgfusepath{fill}%
\end{pgfscope}%
\begin{pgfscope}%
\pgfpathrectangle{\pgfqpoint{0.800000in}{0.528000in}}{\pgfqpoint{3.968000in}{3.696000in}}%
\pgfusepath{clip}%
\pgfsetbuttcap%
\pgfsetroundjoin%
\definecolor{currentfill}{rgb}{0.120565,0.596422,0.543611}%
\pgfsetfillcolor{currentfill}%
\pgfsetlinewidth{0.000000pt}%
\definecolor{currentstroke}{rgb}{0.000000,0.000000,0.000000}%
\pgfsetstrokecolor{currentstroke}%
\pgfsetdash{}{0pt}%
\pgfpathmoveto{\pgfqpoint{4.768000in}{3.015979in}}%
\pgfpathlineto{\pgfqpoint{4.447354in}{3.345447in}}%
\pgfpathlineto{\pgfqpoint{4.126707in}{3.664874in}}%
\pgfpathlineto{\pgfqpoint{3.772615in}{4.006180in}}%
\pgfpathlineto{\pgfqpoint{3.685818in}{4.088185in}}%
\pgfpathlineto{\pgfqpoint{3.540305in}{4.224000in}}%
\pgfpathlineto{\pgfqpoint{3.536933in}{4.224000in}}%
\pgfpathlineto{\pgfqpoint{3.725899in}{4.047299in}}%
\pgfpathlineto{\pgfqpoint{4.047927in}{3.738667in}}%
\pgfpathlineto{\pgfqpoint{4.387423in}{3.402667in}}%
\pgfpathlineto{\pgfqpoint{4.571398in}{3.216000in}}%
\pgfpathlineto{\pgfqpoint{4.768000in}{3.012697in}}%
\pgfpathlineto{\pgfqpoint{4.768000in}{3.012697in}}%
\pgfusepath{fill}%
\end{pgfscope}%
\begin{pgfscope}%
\pgfpathrectangle{\pgfqpoint{0.800000in}{0.528000in}}{\pgfqpoint{3.968000in}{3.696000in}}%
\pgfusepath{clip}%
\pgfsetbuttcap%
\pgfsetroundjoin%
\definecolor{currentfill}{rgb}{0.120092,0.600104,0.542530}%
\pgfsetfillcolor{currentfill}%
\pgfsetlinewidth{0.000000pt}%
\definecolor{currentstroke}{rgb}{0.000000,0.000000,0.000000}%
\pgfsetstrokecolor{currentstroke}%
\pgfsetdash{}{0pt}%
\pgfpathmoveto{\pgfqpoint{4.768000in}{3.019262in}}%
\pgfpathlineto{\pgfqpoint{4.447354in}{3.348668in}}%
\pgfpathlineto{\pgfqpoint{4.126707in}{3.668036in}}%
\pgfpathlineto{\pgfqpoint{3.782513in}{4.000000in}}%
\pgfpathlineto{\pgfqpoint{3.543677in}{4.224000in}}%
\pgfpathlineto{\pgfqpoint{3.540305in}{4.224000in}}%
\pgfpathlineto{\pgfqpoint{3.725899in}{4.050437in}}%
\pgfpathlineto{\pgfqpoint{4.051168in}{3.738667in}}%
\pgfpathlineto{\pgfqpoint{4.390622in}{3.402667in}}%
\pgfpathlineto{\pgfqpoint{4.571219in}{3.219375in}}%
\pgfpathlineto{\pgfqpoint{4.647758in}{3.140793in}}%
\pgfpathlineto{\pgfqpoint{4.768000in}{3.015979in}}%
\pgfpathlineto{\pgfqpoint{4.768000in}{3.015979in}}%
\pgfusepath{fill}%
\end{pgfscope}%
\begin{pgfscope}%
\pgfpathrectangle{\pgfqpoint{0.800000in}{0.528000in}}{\pgfqpoint{3.968000in}{3.696000in}}%
\pgfusepath{clip}%
\pgfsetbuttcap%
\pgfsetroundjoin%
\definecolor{currentfill}{rgb}{0.120092,0.600104,0.542530}%
\pgfsetfillcolor{currentfill}%
\pgfsetlinewidth{0.000000pt}%
\definecolor{currentstroke}{rgb}{0.000000,0.000000,0.000000}%
\pgfsetstrokecolor{currentstroke}%
\pgfsetdash{}{0pt}%
\pgfpathmoveto{\pgfqpoint{4.768000in}{3.022544in}}%
\pgfpathlineto{\pgfqpoint{4.447354in}{3.351889in}}%
\pgfpathlineto{\pgfqpoint{4.126707in}{3.671199in}}%
\pgfpathlineto{\pgfqpoint{3.785823in}{4.000000in}}%
\pgfpathlineto{\pgfqpoint{3.547049in}{4.224000in}}%
\pgfpathlineto{\pgfqpoint{3.543677in}{4.224000in}}%
\pgfpathlineto{\pgfqpoint{3.725899in}{4.053575in}}%
\pgfpathlineto{\pgfqpoint{4.054409in}{3.738667in}}%
\pgfpathlineto{\pgfqpoint{4.393821in}{3.402667in}}%
\pgfpathlineto{\pgfqpoint{4.567596in}{3.226353in}}%
\pgfpathlineto{\pgfqpoint{4.768000in}{3.019262in}}%
\pgfpathlineto{\pgfqpoint{4.768000in}{3.019262in}}%
\pgfusepath{fill}%
\end{pgfscope}%
\begin{pgfscope}%
\pgfpathrectangle{\pgfqpoint{0.800000in}{0.528000in}}{\pgfqpoint{3.968000in}{3.696000in}}%
\pgfusepath{clip}%
\pgfsetbuttcap%
\pgfsetroundjoin%
\definecolor{currentfill}{rgb}{0.120092,0.600104,0.542530}%
\pgfsetfillcolor{currentfill}%
\pgfsetlinewidth{0.000000pt}%
\definecolor{currentstroke}{rgb}{0.000000,0.000000,0.000000}%
\pgfsetstrokecolor{currentstroke}%
\pgfsetdash{}{0pt}%
\pgfpathmoveto{\pgfqpoint{4.768000in}{3.025827in}}%
\pgfpathlineto{\pgfqpoint{4.447354in}{3.355111in}}%
\pgfpathlineto{\pgfqpoint{4.126707in}{3.674361in}}%
\pgfpathlineto{\pgfqpoint{3.789133in}{4.000000in}}%
\pgfpathlineto{\pgfqpoint{3.550421in}{4.224000in}}%
\pgfpathlineto{\pgfqpoint{3.547049in}{4.224000in}}%
\pgfpathlineto{\pgfqpoint{3.725899in}{4.056712in}}%
\pgfpathlineto{\pgfqpoint{4.057651in}{3.738667in}}%
\pgfpathlineto{\pgfqpoint{4.397019in}{3.402667in}}%
\pgfpathlineto{\pgfqpoint{4.567596in}{3.229582in}}%
\pgfpathlineto{\pgfqpoint{4.768000in}{3.022544in}}%
\pgfpathlineto{\pgfqpoint{4.768000in}{3.022544in}}%
\pgfusepath{fill}%
\end{pgfscope}%
\begin{pgfscope}%
\pgfpathrectangle{\pgfqpoint{0.800000in}{0.528000in}}{\pgfqpoint{3.968000in}{3.696000in}}%
\pgfusepath{clip}%
\pgfsetbuttcap%
\pgfsetroundjoin%
\definecolor{currentfill}{rgb}{0.120092,0.600104,0.542530}%
\pgfsetfillcolor{currentfill}%
\pgfsetlinewidth{0.000000pt}%
\definecolor{currentstroke}{rgb}{0.000000,0.000000,0.000000}%
\pgfsetstrokecolor{currentstroke}%
\pgfsetdash{}{0pt}%
\pgfpathmoveto{\pgfqpoint{4.768000in}{3.029109in}}%
\pgfpathlineto{\pgfqpoint{4.477297in}{3.328000in}}%
\pgfpathlineto{\pgfqpoint{4.287030in}{3.519176in}}%
\pgfpathlineto{\pgfqpoint{3.948556in}{3.850667in}}%
\pgfpathlineto{\pgfqpoint{3.765980in}{4.025103in}}%
\pgfpathlineto{\pgfqpoint{3.553793in}{4.224000in}}%
\pgfpathlineto{\pgfqpoint{3.550421in}{4.224000in}}%
\pgfpathlineto{\pgfqpoint{3.725899in}{4.059850in}}%
\pgfpathlineto{\pgfqpoint{4.060892in}{3.738667in}}%
\pgfpathlineto{\pgfqpoint{4.400218in}{3.402667in}}%
\pgfpathlineto{\pgfqpoint{4.567596in}{3.232812in}}%
\pgfpathlineto{\pgfqpoint{4.768000in}{3.025827in}}%
\pgfpathlineto{\pgfqpoint{4.768000in}{3.025827in}}%
\pgfusepath{fill}%
\end{pgfscope}%
\begin{pgfscope}%
\pgfpathrectangle{\pgfqpoint{0.800000in}{0.528000in}}{\pgfqpoint{3.968000in}{3.696000in}}%
\pgfusepath{clip}%
\pgfsetbuttcap%
\pgfsetroundjoin%
\definecolor{currentfill}{rgb}{0.119738,0.603785,0.541400}%
\pgfsetfillcolor{currentfill}%
\pgfsetlinewidth{0.000000pt}%
\definecolor{currentstroke}{rgb}{0.000000,0.000000,0.000000}%
\pgfsetstrokecolor{currentstroke}%
\pgfsetdash{}{0pt}%
\pgfpathmoveto{\pgfqpoint{4.768000in}{3.032354in}}%
\pgfpathlineto{\pgfqpoint{4.567596in}{3.239270in}}%
\pgfpathlineto{\pgfqpoint{4.246949in}{3.562158in}}%
\pgfpathlineto{\pgfqpoint{3.913026in}{3.888000in}}%
\pgfpathlineto{\pgfqpoint{3.725899in}{4.066125in}}%
\pgfpathlineto{\pgfqpoint{3.557165in}{4.224000in}}%
\pgfpathlineto{\pgfqpoint{3.553793in}{4.224000in}}%
\pgfpathlineto{\pgfqpoint{3.725899in}{4.062988in}}%
\pgfpathlineto{\pgfqpoint{4.064133in}{3.738667in}}%
\pgfpathlineto{\pgfqpoint{4.407273in}{3.398787in}}%
\pgfpathlineto{\pgfqpoint{4.607677in}{3.194967in}}%
\pgfpathlineto{\pgfqpoint{4.768000in}{3.029109in}}%
\pgfpathlineto{\pgfqpoint{4.768000in}{3.029333in}}%
\pgfusepath{fill}%
\end{pgfscope}%
\begin{pgfscope}%
\pgfpathrectangle{\pgfqpoint{0.800000in}{0.528000in}}{\pgfqpoint{3.968000in}{3.696000in}}%
\pgfusepath{clip}%
\pgfsetbuttcap%
\pgfsetroundjoin%
\definecolor{currentfill}{rgb}{0.119738,0.603785,0.541400}%
\pgfsetfillcolor{currentfill}%
\pgfsetlinewidth{0.000000pt}%
\definecolor{currentstroke}{rgb}{0.000000,0.000000,0.000000}%
\pgfsetstrokecolor{currentstroke}%
\pgfsetdash{}{0pt}%
\pgfpathmoveto{\pgfqpoint{4.768000in}{3.035597in}}%
\pgfpathlineto{\pgfqpoint{4.581072in}{3.228552in}}%
\pgfpathlineto{\pgfqpoint{4.483657in}{3.328000in}}%
\pgfpathlineto{\pgfqpoint{4.297925in}{3.514667in}}%
\pgfpathlineto{\pgfqpoint{4.126707in}{3.683849in}}%
\pgfpathlineto{\pgfqpoint{3.799064in}{4.000000in}}%
\pgfpathlineto{\pgfqpoint{3.560537in}{4.224000in}}%
\pgfpathlineto{\pgfqpoint{3.557165in}{4.224000in}}%
\pgfpathlineto{\pgfqpoint{3.725899in}{4.066125in}}%
\pgfpathlineto{\pgfqpoint{4.067374in}{3.738667in}}%
\pgfpathlineto{\pgfqpoint{4.407273in}{3.402006in}}%
\pgfpathlineto{\pgfqpoint{4.607677in}{3.198199in}}%
\pgfpathlineto{\pgfqpoint{4.768000in}{3.032354in}}%
\pgfpathlineto{\pgfqpoint{4.768000in}{3.032354in}}%
\pgfusepath{fill}%
\end{pgfscope}%
\begin{pgfscope}%
\pgfpathrectangle{\pgfqpoint{0.800000in}{0.528000in}}{\pgfqpoint{3.968000in}{3.696000in}}%
\pgfusepath{clip}%
\pgfsetbuttcap%
\pgfsetroundjoin%
\definecolor{currentfill}{rgb}{0.119738,0.603785,0.541400}%
\pgfsetfillcolor{currentfill}%
\pgfsetlinewidth{0.000000pt}%
\definecolor{currentstroke}{rgb}{0.000000,0.000000,0.000000}%
\pgfsetstrokecolor{currentstroke}%
\pgfsetdash{}{0pt}%
\pgfpathmoveto{\pgfqpoint{4.768000in}{3.038839in}}%
\pgfpathlineto{\pgfqpoint{4.582714in}{3.230082in}}%
\pgfpathlineto{\pgfqpoint{4.480019in}{3.334907in}}%
\pgfpathlineto{\pgfqpoint{4.287030in}{3.528694in}}%
\pgfpathlineto{\pgfqpoint{3.958368in}{3.850667in}}%
\pgfpathlineto{\pgfqpoint{3.763009in}{4.037333in}}%
\pgfpathlineto{\pgfqpoint{3.563909in}{4.224000in}}%
\pgfpathlineto{\pgfqpoint{3.560537in}{4.224000in}}%
\pgfpathlineto{\pgfqpoint{3.725899in}{4.069263in}}%
\pgfpathlineto{\pgfqpoint{4.070615in}{3.738667in}}%
\pgfpathlineto{\pgfqpoint{4.407273in}{3.405194in}}%
\pgfpathlineto{\pgfqpoint{4.727919in}{3.077290in}}%
\pgfpathlineto{\pgfqpoint{4.768000in}{3.035597in}}%
\pgfpathlineto{\pgfqpoint{4.768000in}{3.035597in}}%
\pgfusepath{fill}%
\end{pgfscope}%
\begin{pgfscope}%
\pgfpathrectangle{\pgfqpoint{0.800000in}{0.528000in}}{\pgfqpoint{3.968000in}{3.696000in}}%
\pgfusepath{clip}%
\pgfsetbuttcap%
\pgfsetroundjoin%
\definecolor{currentfill}{rgb}{0.119512,0.607464,0.540218}%
\pgfsetfillcolor{currentfill}%
\pgfsetlinewidth{0.000000pt}%
\definecolor{currentstroke}{rgb}{0.000000,0.000000,0.000000}%
\pgfsetstrokecolor{currentstroke}%
\pgfsetdash{}{0pt}%
\pgfpathmoveto{\pgfqpoint{4.768000in}{3.042081in}}%
\pgfpathlineto{\pgfqpoint{4.584356in}{3.231611in}}%
\pgfpathlineto{\pgfqpoint{4.487434in}{3.330587in}}%
\pgfpathlineto{\pgfqpoint{4.287030in}{3.531867in}}%
\pgfpathlineto{\pgfqpoint{3.961638in}{3.850667in}}%
\pgfpathlineto{\pgfqpoint{3.765980in}{4.037660in}}%
\pgfpathlineto{\pgfqpoint{3.565576in}{4.224000in}}%
\pgfpathlineto{\pgfqpoint{3.563909in}{4.224000in}}%
\pgfpathlineto{\pgfqpoint{3.725899in}{4.072401in}}%
\pgfpathlineto{\pgfqpoint{4.080060in}{3.732550in}}%
\pgfpathlineto{\pgfqpoint{4.166788in}{3.647657in}}%
\pgfpathlineto{\pgfqpoint{4.487434in}{3.327395in}}%
\pgfpathlineto{\pgfqpoint{4.768000in}{3.038839in}}%
\pgfpathlineto{\pgfqpoint{4.768000in}{3.038839in}}%
\pgfusepath{fill}%
\end{pgfscope}%
\begin{pgfscope}%
\pgfpathrectangle{\pgfqpoint{0.800000in}{0.528000in}}{\pgfqpoint{3.968000in}{3.696000in}}%
\pgfusepath{clip}%
\pgfsetbuttcap%
\pgfsetroundjoin%
\definecolor{currentfill}{rgb}{0.119512,0.607464,0.540218}%
\pgfsetfillcolor{currentfill}%
\pgfsetlinewidth{0.000000pt}%
\definecolor{currentstroke}{rgb}{0.000000,0.000000,0.000000}%
\pgfsetstrokecolor{currentstroke}%
\pgfsetdash{}{0pt}%
\pgfpathmoveto{\pgfqpoint{4.768000in}{3.045323in}}%
\pgfpathlineto{\pgfqpoint{4.585998in}{3.233141in}}%
\pgfpathlineto{\pgfqpoint{4.487434in}{3.333773in}}%
\pgfpathlineto{\pgfqpoint{4.287030in}{3.535039in}}%
\pgfpathlineto{\pgfqpoint{3.964909in}{3.850667in}}%
\pgfpathlineto{\pgfqpoint{3.765980in}{4.040763in}}%
\pgfpathlineto{\pgfqpoint{3.570582in}{4.224000in}}%
\pgfpathlineto{\pgfqpoint{3.567257in}{4.224000in}}%
\pgfpathlineto{\pgfqpoint{3.926303in}{3.884701in}}%
\pgfpathlineto{\pgfqpoint{4.126707in}{3.690174in}}%
\pgfpathlineto{\pgfqpoint{4.453102in}{3.365333in}}%
\pgfpathlineto{\pgfqpoint{4.636104in}{3.178667in}}%
\pgfpathlineto{\pgfqpoint{4.768000in}{3.042081in}}%
\pgfpathlineto{\pgfqpoint{4.768000in}{3.042081in}}%
\pgfusepath{fill}%
\end{pgfscope}%
\begin{pgfscope}%
\pgfpathrectangle{\pgfqpoint{0.800000in}{0.528000in}}{\pgfqpoint{3.968000in}{3.696000in}}%
\pgfusepath{clip}%
\pgfsetbuttcap%
\pgfsetroundjoin%
\definecolor{currentfill}{rgb}{0.119512,0.607464,0.540218}%
\pgfsetfillcolor{currentfill}%
\pgfsetlinewidth{0.000000pt}%
\definecolor{currentstroke}{rgb}{0.000000,0.000000,0.000000}%
\pgfsetstrokecolor{currentstroke}%
\pgfsetdash{}{0pt}%
\pgfpathmoveto{\pgfqpoint{4.768000in}{3.048566in}}%
\pgfpathlineto{\pgfqpoint{4.587641in}{3.234671in}}%
\pgfpathlineto{\pgfqpoint{4.487434in}{3.336958in}}%
\pgfpathlineto{\pgfqpoint{4.287030in}{3.538212in}}%
\pgfpathlineto{\pgfqpoint{3.966384in}{3.852377in}}%
\pgfpathlineto{\pgfqpoint{3.765980in}{4.043867in}}%
\pgfpathlineto{\pgfqpoint{3.573906in}{4.224000in}}%
\pgfpathlineto{\pgfqpoint{3.570582in}{4.224000in}}%
\pgfpathlineto{\pgfqpoint{3.926303in}{3.887851in}}%
\pgfpathlineto{\pgfqpoint{4.126707in}{3.693336in}}%
\pgfpathlineto{\pgfqpoint{4.456250in}{3.365333in}}%
\pgfpathlineto{\pgfqpoint{4.639247in}{3.178667in}}%
\pgfpathlineto{\pgfqpoint{4.768000in}{3.045323in}}%
\pgfpathlineto{\pgfqpoint{4.768000in}{3.045323in}}%
\pgfusepath{fill}%
\end{pgfscope}%
\begin{pgfscope}%
\pgfpathrectangle{\pgfqpoint{0.800000in}{0.528000in}}{\pgfqpoint{3.968000in}{3.696000in}}%
\pgfusepath{clip}%
\pgfsetbuttcap%
\pgfsetroundjoin%
\definecolor{currentfill}{rgb}{0.119512,0.607464,0.540218}%
\pgfsetfillcolor{currentfill}%
\pgfsetlinewidth{0.000000pt}%
\definecolor{currentstroke}{rgb}{0.000000,0.000000,0.000000}%
\pgfsetstrokecolor{currentstroke}%
\pgfsetdash{}{0pt}%
\pgfpathmoveto{\pgfqpoint{4.768000in}{3.051808in}}%
\pgfpathlineto{\pgfqpoint{4.589283in}{3.236200in}}%
\pgfpathlineto{\pgfqpoint{4.487434in}{3.340143in}}%
\pgfpathlineto{\pgfqpoint{4.287030in}{3.541384in}}%
\pgfpathlineto{\pgfqpoint{3.966384in}{3.855492in}}%
\pgfpathlineto{\pgfqpoint{3.765980in}{4.046970in}}%
\pgfpathlineto{\pgfqpoint{3.577231in}{4.224000in}}%
\pgfpathlineto{\pgfqpoint{3.573906in}{4.224000in}}%
\pgfpathlineto{\pgfqpoint{3.926303in}{3.890966in}}%
\pgfpathlineto{\pgfqpoint{4.246949in}{3.578009in}}%
\pgfpathlineto{\pgfqpoint{4.422407in}{3.402667in}}%
\pgfpathlineto{\pgfqpoint{4.607677in}{3.214358in}}%
\pgfpathlineto{\pgfqpoint{4.768000in}{3.048566in}}%
\pgfpathlineto{\pgfqpoint{4.768000in}{3.048566in}}%
\pgfusepath{fill}%
\end{pgfscope}%
\begin{pgfscope}%
\pgfpathrectangle{\pgfqpoint{0.800000in}{0.528000in}}{\pgfqpoint{3.968000in}{3.696000in}}%
\pgfusepath{clip}%
\pgfsetbuttcap%
\pgfsetroundjoin%
\definecolor{currentfill}{rgb}{0.119423,0.611141,0.538982}%
\pgfsetfillcolor{currentfill}%
\pgfsetlinewidth{0.000000pt}%
\definecolor{currentstroke}{rgb}{0.000000,0.000000,0.000000}%
\pgfsetstrokecolor{currentstroke}%
\pgfsetdash{}{0pt}%
\pgfpathmoveto{\pgfqpoint{4.768000in}{3.055050in}}%
\pgfpathlineto{\pgfqpoint{4.590925in}{3.237730in}}%
\pgfpathlineto{\pgfqpoint{4.487434in}{3.343328in}}%
\pgfpathlineto{\pgfqpoint{4.317030in}{3.514667in}}%
\pgfpathlineto{\pgfqpoint{4.126707in}{3.702806in}}%
\pgfpathlineto{\pgfqpoint{3.926303in}{3.897192in}}%
\pgfpathlineto{\pgfqpoint{3.725899in}{4.087932in}}%
\pgfpathlineto{\pgfqpoint{3.580556in}{4.224000in}}%
\pgfpathlineto{\pgfqpoint{3.577231in}{4.224000in}}%
\pgfpathlineto{\pgfqpoint{3.893720in}{3.925333in}}%
\pgfpathlineto{\pgfqpoint{4.090718in}{3.734856in}}%
\pgfpathlineto{\pgfqpoint{4.287030in}{3.541384in}}%
\pgfpathlineto{\pgfqpoint{4.487434in}{3.340143in}}%
\pgfpathlineto{\pgfqpoint{4.665511in}{3.157869in}}%
\pgfpathlineto{\pgfqpoint{4.768000in}{3.051808in}}%
\pgfpathlineto{\pgfqpoint{4.768000in}{3.051808in}}%
\pgfusepath{fill}%
\end{pgfscope}%
\begin{pgfscope}%
\pgfpathrectangle{\pgfqpoint{0.800000in}{0.528000in}}{\pgfqpoint{3.968000in}{3.696000in}}%
\pgfusepath{clip}%
\pgfsetbuttcap%
\pgfsetroundjoin%
\definecolor{currentfill}{rgb}{0.119423,0.611141,0.538982}%
\pgfsetfillcolor{currentfill}%
\pgfsetlinewidth{0.000000pt}%
\definecolor{currentstroke}{rgb}{0.000000,0.000000,0.000000}%
\pgfsetstrokecolor{currentstroke}%
\pgfsetdash{}{0pt}%
\pgfpathmoveto{\pgfqpoint{4.768000in}{3.058293in}}%
\pgfpathlineto{\pgfqpoint{4.592567in}{3.239259in}}%
\pgfpathlineto{\pgfqpoint{4.487434in}{3.346513in}}%
\pgfpathlineto{\pgfqpoint{4.320214in}{3.514667in}}%
\pgfpathlineto{\pgfqpoint{4.126707in}{3.705932in}}%
\pgfpathlineto{\pgfqpoint{3.926303in}{3.900305in}}%
\pgfpathlineto{\pgfqpoint{3.725899in}{4.091033in}}%
\pgfpathlineto{\pgfqpoint{3.583881in}{4.224000in}}%
\pgfpathlineto{\pgfqpoint{3.580556in}{4.224000in}}%
\pgfpathlineto{\pgfqpoint{3.896965in}{3.925333in}}%
\pgfpathlineto{\pgfqpoint{4.090016in}{3.738667in}}%
\pgfpathlineto{\pgfqpoint{4.283142in}{3.548379in}}%
\pgfpathlineto{\pgfqpoint{4.367192in}{3.464520in}}%
\pgfpathlineto{\pgfqpoint{4.687838in}{3.138249in}}%
\pgfpathlineto{\pgfqpoint{4.768000in}{3.055050in}}%
\pgfpathlineto{\pgfqpoint{4.768000in}{3.055050in}}%
\pgfusepath{fill}%
\end{pgfscope}%
\begin{pgfscope}%
\pgfpathrectangle{\pgfqpoint{0.800000in}{0.528000in}}{\pgfqpoint{3.968000in}{3.696000in}}%
\pgfusepath{clip}%
\pgfsetbuttcap%
\pgfsetroundjoin%
\definecolor{currentfill}{rgb}{0.119423,0.611141,0.538982}%
\pgfsetfillcolor{currentfill}%
\pgfsetlinewidth{0.000000pt}%
\definecolor{currentstroke}{rgb}{0.000000,0.000000,0.000000}%
\pgfsetstrokecolor{currentstroke}%
\pgfsetdash{}{0pt}%
\pgfpathmoveto{\pgfqpoint{4.768000in}{3.061535in}}%
\pgfpathlineto{\pgfqpoint{4.594209in}{3.240789in}}%
\pgfpathlineto{\pgfqpoint{4.487434in}{3.349699in}}%
\pgfpathlineto{\pgfqpoint{4.323398in}{3.514667in}}%
\pgfpathlineto{\pgfqpoint{4.130726in}{3.705076in}}%
\pgfpathlineto{\pgfqpoint{4.046545in}{3.787241in}}%
\pgfpathlineto{\pgfqpoint{3.706917in}{4.112000in}}%
\pgfpathlineto{\pgfqpoint{3.587206in}{4.224000in}}%
\pgfpathlineto{\pgfqpoint{3.583881in}{4.224000in}}%
\pgfpathlineto{\pgfqpoint{3.900210in}{3.925333in}}%
\pgfpathlineto{\pgfqpoint{4.093214in}{3.738667in}}%
\pgfpathlineto{\pgfqpoint{4.287030in}{3.547729in}}%
\pgfpathlineto{\pgfqpoint{4.487434in}{3.346513in}}%
\pgfpathlineto{\pgfqpoint{4.668788in}{3.160922in}}%
\pgfpathlineto{\pgfqpoint{4.768000in}{3.058293in}}%
\pgfpathlineto{\pgfqpoint{4.768000in}{3.058293in}}%
\pgfusepath{fill}%
\end{pgfscope}%
\begin{pgfscope}%
\pgfpathrectangle{\pgfqpoint{0.800000in}{0.528000in}}{\pgfqpoint{3.968000in}{3.696000in}}%
\pgfusepath{clip}%
\pgfsetbuttcap%
\pgfsetroundjoin%
\definecolor{currentfill}{rgb}{0.119423,0.611141,0.538982}%
\pgfsetfillcolor{currentfill}%
\pgfsetlinewidth{0.000000pt}%
\definecolor{currentstroke}{rgb}{0.000000,0.000000,0.000000}%
\pgfsetstrokecolor{currentstroke}%
\pgfsetdash{}{0pt}%
\pgfpathmoveto{\pgfqpoint{4.768000in}{3.064777in}}%
\pgfpathlineto{\pgfqpoint{4.595852in}{3.242319in}}%
\pgfpathlineto{\pgfqpoint{4.511951in}{3.328000in}}%
\pgfpathlineto{\pgfqpoint{4.326582in}{3.514667in}}%
\pgfpathlineto{\pgfqpoint{4.137774in}{3.701333in}}%
\pgfpathlineto{\pgfqpoint{3.828544in}{4.000000in}}%
\pgfpathlineto{\pgfqpoint{3.645737in}{4.172505in}}%
\pgfpathlineto{\pgfqpoint{3.590530in}{4.224000in}}%
\pgfpathlineto{\pgfqpoint{3.587206in}{4.224000in}}%
\pgfpathlineto{\pgfqpoint{3.903456in}{3.925333in}}%
\pgfpathlineto{\pgfqpoint{4.086626in}{3.748223in}}%
\pgfpathlineto{\pgfqpoint{4.407273in}{3.430635in}}%
\pgfpathlineto{\pgfqpoint{4.727919in}{3.103207in}}%
\pgfpathlineto{\pgfqpoint{4.768000in}{3.061535in}}%
\pgfpathlineto{\pgfqpoint{4.768000in}{3.061535in}}%
\pgfusepath{fill}%
\end{pgfscope}%
\begin{pgfscope}%
\pgfpathrectangle{\pgfqpoint{0.800000in}{0.528000in}}{\pgfqpoint{3.968000in}{3.696000in}}%
\pgfusepath{clip}%
\pgfsetbuttcap%
\pgfsetroundjoin%
\definecolor{currentfill}{rgb}{0.119483,0.614817,0.537692}%
\pgfsetfillcolor{currentfill}%
\pgfsetlinewidth{0.000000pt}%
\definecolor{currentstroke}{rgb}{0.000000,0.000000,0.000000}%
\pgfsetstrokecolor{currentstroke}%
\pgfsetdash{}{0pt}%
\pgfpathmoveto{\pgfqpoint{4.768000in}{3.068003in}}%
\pgfpathlineto{\pgfqpoint{4.441345in}{3.402667in}}%
\pgfpathlineto{\pgfqpoint{4.094863in}{3.746339in}}%
\pgfpathlineto{\pgfqpoint{3.996761in}{3.841628in}}%
\pgfpathlineto{\pgfqpoint{3.898189in}{3.936480in}}%
\pgfpathlineto{\pgfqpoint{3.792522in}{4.037333in}}%
\pgfpathlineto{\pgfqpoint{3.605657in}{4.213019in}}%
\pgfpathlineto{\pgfqpoint{3.593855in}{4.224000in}}%
\pgfpathlineto{\pgfqpoint{3.590530in}{4.224000in}}%
\pgfpathlineto{\pgfqpoint{3.906701in}{3.925333in}}%
\pgfpathlineto{\pgfqpoint{4.086626in}{3.751345in}}%
\pgfpathlineto{\pgfqpoint{4.407273in}{3.433815in}}%
\pgfpathlineto{\pgfqpoint{4.730249in}{3.104000in}}%
\pgfpathlineto{\pgfqpoint{4.768000in}{3.064777in}}%
\pgfpathlineto{\pgfqpoint{4.768000in}{3.066667in}}%
\pgfpathlineto{\pgfqpoint{4.768000in}{3.066667in}}%
\pgfusepath{fill}%
\end{pgfscope}%
\begin{pgfscope}%
\pgfpathrectangle{\pgfqpoint{0.800000in}{0.528000in}}{\pgfqpoint{3.968000in}{3.696000in}}%
\pgfusepath{clip}%
\pgfsetbuttcap%
\pgfsetroundjoin%
\definecolor{currentfill}{rgb}{0.119483,0.614817,0.537692}%
\pgfsetfillcolor{currentfill}%
\pgfsetlinewidth{0.000000pt}%
\definecolor{currentstroke}{rgb}{0.000000,0.000000,0.000000}%
\pgfsetstrokecolor{currentstroke}%
\pgfsetdash{}{0pt}%
\pgfpathmoveto{\pgfqpoint{4.768000in}{3.071206in}}%
\pgfpathlineto{\pgfqpoint{4.444501in}{3.402667in}}%
\pgfpathlineto{\pgfqpoint{4.096491in}{3.747855in}}%
\pgfpathlineto{\pgfqpoint{3.990737in}{3.850667in}}%
\pgfpathlineto{\pgfqpoint{3.806061in}{4.027599in}}%
\pgfpathlineto{\pgfqpoint{3.597180in}{4.224000in}}%
\pgfpathlineto{\pgfqpoint{3.593855in}{4.224000in}}%
\pgfpathlineto{\pgfqpoint{3.909946in}{3.925333in}}%
\pgfpathlineto{\pgfqpoint{4.086626in}{3.754468in}}%
\pgfpathlineto{\pgfqpoint{4.407273in}{3.436995in}}%
\pgfpathlineto{\pgfqpoint{4.733334in}{3.104000in}}%
\pgfpathlineto{\pgfqpoint{4.768000in}{3.068003in}}%
\pgfpathlineto{\pgfqpoint{4.768000in}{3.068003in}}%
\pgfusepath{fill}%
\end{pgfscope}%
\begin{pgfscope}%
\pgfpathrectangle{\pgfqpoint{0.800000in}{0.528000in}}{\pgfqpoint{3.968000in}{3.696000in}}%
\pgfusepath{clip}%
\pgfsetbuttcap%
\pgfsetroundjoin%
\definecolor{currentfill}{rgb}{0.119483,0.614817,0.537692}%
\pgfsetfillcolor{currentfill}%
\pgfsetlinewidth{0.000000pt}%
\definecolor{currentstroke}{rgb}{0.000000,0.000000,0.000000}%
\pgfsetstrokecolor{currentstroke}%
\pgfsetdash{}{0pt}%
\pgfpathmoveto{\pgfqpoint{4.768000in}{3.074409in}}%
\pgfpathlineto{\pgfqpoint{4.447354in}{3.402970in}}%
\pgfpathlineto{\pgfqpoint{4.098119in}{3.749372in}}%
\pgfpathlineto{\pgfqpoint{3.993963in}{3.850667in}}%
\pgfpathlineto{\pgfqpoint{3.799072in}{4.037333in}}%
\pgfpathlineto{\pgfqpoint{3.600505in}{4.224000in}}%
\pgfpathlineto{\pgfqpoint{3.597180in}{4.224000in}}%
\pgfpathlineto{\pgfqpoint{3.913192in}{3.925333in}}%
\pgfpathlineto{\pgfqpoint{4.086626in}{3.757591in}}%
\pgfpathlineto{\pgfqpoint{4.407445in}{3.440000in}}%
\pgfpathlineto{\pgfqpoint{4.736418in}{3.104000in}}%
\pgfpathlineto{\pgfqpoint{4.768000in}{3.071206in}}%
\pgfpathlineto{\pgfqpoint{4.768000in}{3.071206in}}%
\pgfusepath{fill}%
\end{pgfscope}%
\begin{pgfscope}%
\pgfpathrectangle{\pgfqpoint{0.800000in}{0.528000in}}{\pgfqpoint{3.968000in}{3.696000in}}%
\pgfusepath{clip}%
\pgfsetbuttcap%
\pgfsetroundjoin%
\definecolor{currentfill}{rgb}{0.119483,0.614817,0.537692}%
\pgfsetfillcolor{currentfill}%
\pgfsetlinewidth{0.000000pt}%
\definecolor{currentstroke}{rgb}{0.000000,0.000000,0.000000}%
\pgfsetstrokecolor{currentstroke}%
\pgfsetdash{}{0pt}%
\pgfpathmoveto{\pgfqpoint{4.768000in}{3.077612in}}%
\pgfpathlineto{\pgfqpoint{4.447354in}{3.406115in}}%
\pgfpathlineto{\pgfqpoint{4.099747in}{3.750888in}}%
\pgfpathlineto{\pgfqpoint{3.997189in}{3.850667in}}%
\pgfpathlineto{\pgfqpoint{3.802346in}{4.037333in}}%
\pgfpathlineto{\pgfqpoint{3.603830in}{4.224000in}}%
\pgfpathlineto{\pgfqpoint{3.600505in}{4.224000in}}%
\pgfpathlineto{\pgfqpoint{3.916437in}{3.925333in}}%
\pgfpathlineto{\pgfqpoint{4.086626in}{3.760713in}}%
\pgfpathlineto{\pgfqpoint{4.410569in}{3.440000in}}%
\pgfpathlineto{\pgfqpoint{4.739503in}{3.104000in}}%
\pgfpathlineto{\pgfqpoint{4.768000in}{3.074409in}}%
\pgfpathlineto{\pgfqpoint{4.768000in}{3.074409in}}%
\pgfusepath{fill}%
\end{pgfscope}%
\begin{pgfscope}%
\pgfpathrectangle{\pgfqpoint{0.800000in}{0.528000in}}{\pgfqpoint{3.968000in}{3.696000in}}%
\pgfusepath{clip}%
\pgfsetbuttcap%
\pgfsetroundjoin%
\definecolor{currentfill}{rgb}{0.119699,0.618490,0.536347}%
\pgfsetfillcolor{currentfill}%
\pgfsetlinewidth{0.000000pt}%
\definecolor{currentstroke}{rgb}{0.000000,0.000000,0.000000}%
\pgfsetstrokecolor{currentstroke}%
\pgfsetdash{}{0pt}%
\pgfpathmoveto{\pgfqpoint{4.768000in}{3.080815in}}%
\pgfpathlineto{\pgfqpoint{4.447354in}{3.409260in}}%
\pgfpathlineto{\pgfqpoint{4.101375in}{3.752404in}}%
\pgfpathlineto{\pgfqpoint{4.000415in}{3.850667in}}%
\pgfpathlineto{\pgfqpoint{3.805621in}{4.037333in}}%
\pgfpathlineto{\pgfqpoint{3.607134in}{4.224000in}}%
\pgfpathlineto{\pgfqpoint{3.603830in}{4.224000in}}%
\pgfpathlineto{\pgfqpoint{3.604739in}{4.223145in}}%
\pgfpathlineto{\pgfqpoint{3.685818in}{4.147336in}}%
\pgfpathlineto{\pgfqpoint{3.886222in}{3.957404in}}%
\pgfpathlineto{\pgfqpoint{4.226378in}{3.626667in}}%
\pgfpathlineto{\pgfqpoint{4.413693in}{3.440000in}}%
\pgfpathlineto{\pgfqpoint{4.754532in}{3.091455in}}%
\pgfpathlineto{\pgfqpoint{4.768000in}{3.077612in}}%
\pgfpathlineto{\pgfqpoint{4.768000in}{3.077612in}}%
\pgfusepath{fill}%
\end{pgfscope}%
\begin{pgfscope}%
\pgfpathrectangle{\pgfqpoint{0.800000in}{0.528000in}}{\pgfqpoint{3.968000in}{3.696000in}}%
\pgfusepath{clip}%
\pgfsetbuttcap%
\pgfsetroundjoin%
\definecolor{currentfill}{rgb}{0.119699,0.618490,0.536347}%
\pgfsetfillcolor{currentfill}%
\pgfsetlinewidth{0.000000pt}%
\definecolor{currentstroke}{rgb}{0.000000,0.000000,0.000000}%
\pgfsetstrokecolor{currentstroke}%
\pgfsetdash{}{0pt}%
\pgfpathmoveto{\pgfqpoint{4.768000in}{3.084019in}}%
\pgfpathlineto{\pgfqpoint{4.447354in}{3.412404in}}%
\pgfpathlineto{\pgfqpoint{4.118794in}{3.738667in}}%
\pgfpathlineto{\pgfqpoint{3.765980in}{4.077965in}}%
\pgfpathlineto{\pgfqpoint{3.610413in}{4.224000in}}%
\pgfpathlineto{\pgfqpoint{3.607134in}{4.224000in}}%
\pgfpathlineto{\pgfqpoint{3.966384in}{3.883531in}}%
\pgfpathlineto{\pgfqpoint{4.166788in}{3.688509in}}%
\pgfpathlineto{\pgfqpoint{4.367192in}{3.489791in}}%
\pgfpathlineto{\pgfqpoint{4.687838in}{3.163869in}}%
\pgfpathlineto{\pgfqpoint{4.768000in}{3.080815in}}%
\pgfpathlineto{\pgfqpoint{4.768000in}{3.080815in}}%
\pgfusepath{fill}%
\end{pgfscope}%
\begin{pgfscope}%
\pgfpathrectangle{\pgfqpoint{0.800000in}{0.528000in}}{\pgfqpoint{3.968000in}{3.696000in}}%
\pgfusepath{clip}%
\pgfsetbuttcap%
\pgfsetroundjoin%
\definecolor{currentfill}{rgb}{0.119699,0.618490,0.536347}%
\pgfsetfillcolor{currentfill}%
\pgfsetlinewidth{0.000000pt}%
\definecolor{currentstroke}{rgb}{0.000000,0.000000,0.000000}%
\pgfsetstrokecolor{currentstroke}%
\pgfsetdash{}{0pt}%
\pgfpathmoveto{\pgfqpoint{4.768000in}{3.087222in}}%
\pgfpathlineto{\pgfqpoint{4.447354in}{3.415549in}}%
\pgfpathlineto{\pgfqpoint{4.121992in}{3.738667in}}%
\pgfpathlineto{\pgfqpoint{3.765980in}{4.081032in}}%
\pgfpathlineto{\pgfqpoint{3.613692in}{4.224000in}}%
\pgfpathlineto{\pgfqpoint{3.610413in}{4.224000in}}%
\pgfpathlineto{\pgfqpoint{3.966384in}{3.886646in}}%
\pgfpathlineto{\pgfqpoint{4.166788in}{3.691637in}}%
\pgfpathlineto{\pgfqpoint{4.367192in}{3.492931in}}%
\pgfpathlineto{\pgfqpoint{4.687838in}{3.167067in}}%
\pgfpathlineto{\pgfqpoint{4.768000in}{3.084019in}}%
\pgfpathlineto{\pgfqpoint{4.768000in}{3.084019in}}%
\pgfusepath{fill}%
\end{pgfscope}%
\begin{pgfscope}%
\pgfpathrectangle{\pgfqpoint{0.800000in}{0.528000in}}{\pgfqpoint{3.968000in}{3.696000in}}%
\pgfusepath{clip}%
\pgfsetbuttcap%
\pgfsetroundjoin%
\definecolor{currentfill}{rgb}{0.120081,0.622161,0.534946}%
\pgfsetfillcolor{currentfill}%
\pgfsetlinewidth{0.000000pt}%
\definecolor{currentstroke}{rgb}{0.000000,0.000000,0.000000}%
\pgfsetstrokecolor{currentstroke}%
\pgfsetdash{}{0pt}%
\pgfpathmoveto{\pgfqpoint{4.768000in}{3.090425in}}%
\pgfpathlineto{\pgfqpoint{4.447354in}{3.418694in}}%
\pgfpathlineto{\pgfqpoint{4.125189in}{3.738667in}}%
\pgfpathlineto{\pgfqpoint{3.775943in}{4.074667in}}%
\pgfpathlineto{\pgfqpoint{3.616971in}{4.224000in}}%
\pgfpathlineto{\pgfqpoint{3.613692in}{4.224000in}}%
\pgfpathlineto{\pgfqpoint{3.966384in}{3.889741in}}%
\pgfpathlineto{\pgfqpoint{4.143734in}{3.717193in}}%
\pgfpathlineto{\pgfqpoint{4.246949in}{3.615733in}}%
\pgfpathlineto{\pgfqpoint{4.570508in}{3.290667in}}%
\pgfpathlineto{\pgfqpoint{4.768000in}{3.087222in}}%
\pgfpathlineto{\pgfqpoint{4.768000in}{3.087222in}}%
\pgfusepath{fill}%
\end{pgfscope}%
\begin{pgfscope}%
\pgfpathrectangle{\pgfqpoint{0.800000in}{0.528000in}}{\pgfqpoint{3.968000in}{3.696000in}}%
\pgfusepath{clip}%
\pgfsetbuttcap%
\pgfsetroundjoin%
\definecolor{currentfill}{rgb}{0.120081,0.622161,0.534946}%
\pgfsetfillcolor{currentfill}%
\pgfsetlinewidth{0.000000pt}%
\definecolor{currentstroke}{rgb}{0.000000,0.000000,0.000000}%
\pgfsetstrokecolor{currentstroke}%
\pgfsetdash{}{0pt}%
\pgfpathmoveto{\pgfqpoint{4.768000in}{3.093628in}}%
\pgfpathlineto{\pgfqpoint{4.447354in}{3.421839in}}%
\pgfpathlineto{\pgfqpoint{4.126707in}{3.740289in}}%
\pgfpathlineto{\pgfqpoint{3.779183in}{4.074667in}}%
\pgfpathlineto{\pgfqpoint{3.620250in}{4.224000in}}%
\pgfpathlineto{\pgfqpoint{3.616971in}{4.224000in}}%
\pgfpathlineto{\pgfqpoint{3.966384in}{3.892820in}}%
\pgfpathlineto{\pgfqpoint{4.145360in}{3.718707in}}%
\pgfpathlineto{\pgfqpoint{4.246949in}{3.618865in}}%
\pgfpathlineto{\pgfqpoint{4.573596in}{3.290667in}}%
\pgfpathlineto{\pgfqpoint{4.768000in}{3.090425in}}%
\pgfpathlineto{\pgfqpoint{4.768000in}{3.090425in}}%
\pgfusepath{fill}%
\end{pgfscope}%
\begin{pgfscope}%
\pgfpathrectangle{\pgfqpoint{0.800000in}{0.528000in}}{\pgfqpoint{3.968000in}{3.696000in}}%
\pgfusepath{clip}%
\pgfsetbuttcap%
\pgfsetroundjoin%
\definecolor{currentfill}{rgb}{0.120081,0.622161,0.534946}%
\pgfsetfillcolor{currentfill}%
\pgfsetlinewidth{0.000000pt}%
\definecolor{currentstroke}{rgb}{0.000000,0.000000,0.000000}%
\pgfsetstrokecolor{currentstroke}%
\pgfsetdash{}{0pt}%
\pgfpathmoveto{\pgfqpoint{4.768000in}{3.096831in}}%
\pgfpathlineto{\pgfqpoint{4.447354in}{3.424984in}}%
\pgfpathlineto{\pgfqpoint{4.126707in}{3.743378in}}%
\pgfpathlineto{\pgfqpoint{3.782423in}{4.074667in}}%
\pgfpathlineto{\pgfqpoint{3.623529in}{4.224000in}}%
\pgfpathlineto{\pgfqpoint{3.620250in}{4.224000in}}%
\pgfpathlineto{\pgfqpoint{3.966384in}{3.895900in}}%
\pgfpathlineto{\pgfqpoint{4.146986in}{3.720222in}}%
\pgfpathlineto{\pgfqpoint{4.246949in}{3.621998in}}%
\pgfpathlineto{\pgfqpoint{4.591380in}{3.275487in}}%
\pgfpathlineto{\pgfqpoint{4.687838in}{3.176661in}}%
\pgfpathlineto{\pgfqpoint{4.768000in}{3.093628in}}%
\pgfpathlineto{\pgfqpoint{4.768000in}{3.093628in}}%
\pgfusepath{fill}%
\end{pgfscope}%
\begin{pgfscope}%
\pgfpathrectangle{\pgfqpoint{0.800000in}{0.528000in}}{\pgfqpoint{3.968000in}{3.696000in}}%
\pgfusepath{clip}%
\pgfsetbuttcap%
\pgfsetroundjoin%
\definecolor{currentfill}{rgb}{0.120081,0.622161,0.534946}%
\pgfsetfillcolor{currentfill}%
\pgfsetlinewidth{0.000000pt}%
\definecolor{currentstroke}{rgb}{0.000000,0.000000,0.000000}%
\pgfsetstrokecolor{currentstroke}%
\pgfsetdash{}{0pt}%
\pgfpathmoveto{\pgfqpoint{4.768000in}{3.100034in}}%
\pgfpathlineto{\pgfqpoint{4.447354in}{3.428129in}}%
\pgfpathlineto{\pgfqpoint{4.126707in}{3.746467in}}%
\pgfpathlineto{\pgfqpoint{3.785662in}{4.074667in}}%
\pgfpathlineto{\pgfqpoint{3.626808in}{4.224000in}}%
\pgfpathlineto{\pgfqpoint{3.623529in}{4.224000in}}%
\pgfpathlineto{\pgfqpoint{3.952388in}{3.912297in}}%
\pgfpathlineto{\pgfqpoint{4.054917in}{3.813333in}}%
\pgfpathlineto{\pgfqpoint{4.401025in}{3.471514in}}%
\pgfpathlineto{\pgfqpoint{4.487434in}{3.384505in}}%
\pgfpathlineto{\pgfqpoint{4.669297in}{3.198730in}}%
\pgfpathlineto{\pgfqpoint{4.768000in}{3.096831in}}%
\pgfpathlineto{\pgfqpoint{4.768000in}{3.096831in}}%
\pgfusepath{fill}%
\end{pgfscope}%
\begin{pgfscope}%
\pgfpathrectangle{\pgfqpoint{0.800000in}{0.528000in}}{\pgfqpoint{3.968000in}{3.696000in}}%
\pgfusepath{clip}%
\pgfsetbuttcap%
\pgfsetroundjoin%
\definecolor{currentfill}{rgb}{0.120638,0.625828,0.533488}%
\pgfsetfillcolor{currentfill}%
\pgfsetlinewidth{0.000000pt}%
\definecolor{currentstroke}{rgb}{0.000000,0.000000,0.000000}%
\pgfsetstrokecolor{currentstroke}%
\pgfsetdash{}{0pt}%
\pgfpathmoveto{\pgfqpoint{4.768000in}{3.103237in}}%
\pgfpathlineto{\pgfqpoint{4.447354in}{3.431273in}}%
\pgfpathlineto{\pgfqpoint{4.126707in}{3.749555in}}%
\pgfpathlineto{\pgfqpoint{3.788902in}{4.074667in}}%
\pgfpathlineto{\pgfqpoint{3.630087in}{4.224000in}}%
\pgfpathlineto{\pgfqpoint{3.626808in}{4.224000in}}%
\pgfpathlineto{\pgfqpoint{3.954024in}{3.913820in}}%
\pgfpathlineto{\pgfqpoint{4.058091in}{3.813333in}}%
\pgfpathlineto{\pgfqpoint{4.407273in}{3.468455in}}%
\pgfpathlineto{\pgfqpoint{4.607677in}{3.265318in}}%
\pgfpathlineto{\pgfqpoint{4.768000in}{3.100034in}}%
\pgfpathlineto{\pgfqpoint{4.768000in}{3.100034in}}%
\pgfusepath{fill}%
\end{pgfscope}%
\begin{pgfscope}%
\pgfpathrectangle{\pgfqpoint{0.800000in}{0.528000in}}{\pgfqpoint{3.968000in}{3.696000in}}%
\pgfusepath{clip}%
\pgfsetbuttcap%
\pgfsetroundjoin%
\definecolor{currentfill}{rgb}{0.120638,0.625828,0.533488}%
\pgfsetfillcolor{currentfill}%
\pgfsetlinewidth{0.000000pt}%
\definecolor{currentstroke}{rgb}{0.000000,0.000000,0.000000}%
\pgfsetstrokecolor{currentstroke}%
\pgfsetdash{}{0pt}%
\pgfpathmoveto{\pgfqpoint{4.768000in}{3.106411in}}%
\pgfpathlineto{\pgfqpoint{4.567596in}{3.312552in}}%
\pgfpathlineto{\pgfqpoint{4.246949in}{3.634436in}}%
\pgfpathlineto{\pgfqpoint{4.046545in}{3.830720in}}%
\pgfpathlineto{\pgfqpoint{3.846141in}{4.023379in}}%
\pgfpathlineto{\pgfqpoint{3.645737in}{4.212453in}}%
\pgfpathlineto{\pgfqpoint{3.633366in}{4.224000in}}%
\pgfpathlineto{\pgfqpoint{3.630087in}{4.224000in}}%
\pgfpathlineto{\pgfqpoint{3.955659in}{3.915343in}}%
\pgfpathlineto{\pgfqpoint{4.061264in}{3.813333in}}%
\pgfpathlineto{\pgfqpoint{4.407273in}{3.471597in}}%
\pgfpathlineto{\pgfqpoint{4.607677in}{3.268473in}}%
\pgfpathlineto{\pgfqpoint{4.768000in}{3.103237in}}%
\pgfpathlineto{\pgfqpoint{4.768000in}{3.104000in}}%
\pgfusepath{fill}%
\end{pgfscope}%
\begin{pgfscope}%
\pgfpathrectangle{\pgfqpoint{0.800000in}{0.528000in}}{\pgfqpoint{3.968000in}{3.696000in}}%
\pgfusepath{clip}%
\pgfsetbuttcap%
\pgfsetroundjoin%
\definecolor{currentfill}{rgb}{0.120638,0.625828,0.533488}%
\pgfsetfillcolor{currentfill}%
\pgfsetlinewidth{0.000000pt}%
\definecolor{currentstroke}{rgb}{0.000000,0.000000,0.000000}%
\pgfsetstrokecolor{currentstroke}%
\pgfsetdash{}{0pt}%
\pgfpathmoveto{\pgfqpoint{4.768000in}{3.109576in}}%
\pgfpathlineto{\pgfqpoint{4.567596in}{3.315704in}}%
\pgfpathlineto{\pgfqpoint{4.246949in}{3.637532in}}%
\pgfpathlineto{\pgfqpoint{4.046545in}{3.833804in}}%
\pgfpathlineto{\pgfqpoint{3.846141in}{4.026451in}}%
\pgfpathlineto{\pgfqpoint{3.645737in}{4.215513in}}%
\pgfpathlineto{\pgfqpoint{3.636645in}{4.224000in}}%
\pgfpathlineto{\pgfqpoint{3.633366in}{4.224000in}}%
\pgfpathlineto{\pgfqpoint{3.957294in}{3.916867in}}%
\pgfpathlineto{\pgfqpoint{4.046545in}{3.830720in}}%
\pgfpathlineto{\pgfqpoint{4.246949in}{3.634436in}}%
\pgfpathlineto{\pgfqpoint{4.567596in}{3.312552in}}%
\pgfpathlineto{\pgfqpoint{4.734303in}{3.141333in}}%
\pgfpathlineto{\pgfqpoint{4.768000in}{3.106411in}}%
\pgfpathlineto{\pgfqpoint{4.768000in}{3.106411in}}%
\pgfusepath{fill}%
\end{pgfscope}%
\begin{pgfscope}%
\pgfpathrectangle{\pgfqpoint{0.800000in}{0.528000in}}{\pgfqpoint{3.968000in}{3.696000in}}%
\pgfusepath{clip}%
\pgfsetbuttcap%
\pgfsetroundjoin%
\definecolor{currentfill}{rgb}{0.120638,0.625828,0.533488}%
\pgfsetfillcolor{currentfill}%
\pgfsetlinewidth{0.000000pt}%
\definecolor{currentstroke}{rgb}{0.000000,0.000000,0.000000}%
\pgfsetstrokecolor{currentstroke}%
\pgfsetdash{}{0pt}%
\pgfpathmoveto{\pgfqpoint{4.768000in}{3.112740in}}%
\pgfpathlineto{\pgfqpoint{4.567596in}{3.318857in}}%
\pgfpathlineto{\pgfqpoint{4.246949in}{3.640628in}}%
\pgfpathlineto{\pgfqpoint{4.046545in}{3.836888in}}%
\pgfpathlineto{\pgfqpoint{3.846141in}{4.029523in}}%
\pgfpathlineto{\pgfqpoint{3.662730in}{4.202495in}}%
\pgfpathlineto{\pgfqpoint{3.639923in}{4.224000in}}%
\pgfpathlineto{\pgfqpoint{3.636645in}{4.224000in}}%
\pgfpathlineto{\pgfqpoint{3.951788in}{3.925333in}}%
\pgfpathlineto{\pgfqpoint{4.287030in}{3.597839in}}%
\pgfpathlineto{\pgfqpoint{4.607677in}{3.274783in}}%
\pgfpathlineto{\pgfqpoint{4.768000in}{3.109576in}}%
\pgfpathlineto{\pgfqpoint{4.768000in}{3.109576in}}%
\pgfusepath{fill}%
\end{pgfscope}%
\begin{pgfscope}%
\pgfpathrectangle{\pgfqpoint{0.800000in}{0.528000in}}{\pgfqpoint{3.968000in}{3.696000in}}%
\pgfusepath{clip}%
\pgfsetbuttcap%
\pgfsetroundjoin%
\definecolor{currentfill}{rgb}{0.121380,0.629492,0.531973}%
\pgfsetfillcolor{currentfill}%
\pgfsetlinewidth{0.000000pt}%
\definecolor{currentstroke}{rgb}{0.000000,0.000000,0.000000}%
\pgfsetstrokecolor{currentstroke}%
\pgfsetdash{}{0pt}%
\pgfpathmoveto{\pgfqpoint{4.768000in}{3.115905in}}%
\pgfpathlineto{\pgfqpoint{4.567596in}{3.322009in}}%
\pgfpathlineto{\pgfqpoint{4.246949in}{3.643724in}}%
\pgfpathlineto{\pgfqpoint{4.046545in}{3.839972in}}%
\pgfpathlineto{\pgfqpoint{3.880312in}{4.000000in}}%
\pgfpathlineto{\pgfqpoint{3.683090in}{4.186667in}}%
\pgfpathlineto{\pgfqpoint{3.643202in}{4.224000in}}%
\pgfpathlineto{\pgfqpoint{3.639923in}{4.224000in}}%
\pgfpathlineto{\pgfqpoint{3.954989in}{3.925333in}}%
\pgfpathlineto{\pgfqpoint{4.287030in}{3.600937in}}%
\pgfpathlineto{\pgfqpoint{4.607677in}{3.277937in}}%
\pgfpathlineto{\pgfqpoint{4.768000in}{3.112740in}}%
\pgfpathlineto{\pgfqpoint{4.768000in}{3.112740in}}%
\pgfusepath{fill}%
\end{pgfscope}%
\begin{pgfscope}%
\pgfpathrectangle{\pgfqpoint{0.800000in}{0.528000in}}{\pgfqpoint{3.968000in}{3.696000in}}%
\pgfusepath{clip}%
\pgfsetbuttcap%
\pgfsetroundjoin%
\definecolor{currentfill}{rgb}{0.121380,0.629492,0.531973}%
\pgfsetfillcolor{currentfill}%
\pgfsetlinewidth{0.000000pt}%
\definecolor{currentstroke}{rgb}{0.000000,0.000000,0.000000}%
\pgfsetstrokecolor{currentstroke}%
\pgfsetdash{}{0pt}%
\pgfpathmoveto{\pgfqpoint{4.768000in}{3.119070in}}%
\pgfpathlineto{\pgfqpoint{4.567596in}{3.325161in}}%
\pgfpathlineto{\pgfqpoint{4.246949in}{3.646819in}}%
\pgfpathlineto{\pgfqpoint{4.046545in}{3.843056in}}%
\pgfpathlineto{\pgfqpoint{3.883533in}{4.000000in}}%
\pgfpathlineto{\pgfqpoint{3.685818in}{4.187168in}}%
\pgfpathlineto{\pgfqpoint{3.645737in}{4.224000in}}%
\pgfpathlineto{\pgfqpoint{3.643202in}{4.224000in}}%
\pgfpathlineto{\pgfqpoint{3.958191in}{3.925333in}}%
\pgfpathlineto{\pgfqpoint{4.275193in}{3.615640in}}%
\pgfpathlineto{\pgfqpoint{4.376736in}{3.514667in}}%
\pgfpathlineto{\pgfqpoint{4.717049in}{3.168542in}}%
\pgfpathlineto{\pgfqpoint{4.768000in}{3.115905in}}%
\pgfpathlineto{\pgfqpoint{4.768000in}{3.115905in}}%
\pgfusepath{fill}%
\end{pgfscope}%
\begin{pgfscope}%
\pgfpathrectangle{\pgfqpoint{0.800000in}{0.528000in}}{\pgfqpoint{3.968000in}{3.696000in}}%
\pgfusepath{clip}%
\pgfsetbuttcap%
\pgfsetroundjoin%
\definecolor{currentfill}{rgb}{0.121380,0.629492,0.531973}%
\pgfsetfillcolor{currentfill}%
\pgfsetlinewidth{0.000000pt}%
\definecolor{currentstroke}{rgb}{0.000000,0.000000,0.000000}%
\pgfsetstrokecolor{currentstroke}%
\pgfsetdash{}{0pt}%
\pgfpathmoveto{\pgfqpoint{4.768000in}{3.122235in}}%
\pgfpathlineto{\pgfqpoint{4.571110in}{3.324727in}}%
\pgfpathlineto{\pgfqpoint{4.407273in}{3.490297in}}%
\pgfpathlineto{\pgfqpoint{4.080305in}{3.813333in}}%
\pgfpathlineto{\pgfqpoint{3.886222in}{4.000501in}}%
\pgfpathlineto{\pgfqpoint{3.685818in}{4.190195in}}%
\pgfpathlineto{\pgfqpoint{3.649706in}{4.224000in}}%
\pgfpathlineto{\pgfqpoint{3.646471in}{4.224000in}}%
\pgfpathlineto{\pgfqpoint{4.006465in}{3.881867in}}%
\pgfpathlineto{\pgfqpoint{4.342457in}{3.552000in}}%
\pgfpathlineto{\pgfqpoint{4.534425in}{3.358897in}}%
\pgfpathlineto{\pgfqpoint{4.727919in}{3.160593in}}%
\pgfpathlineto{\pgfqpoint{4.768000in}{3.119070in}}%
\pgfpathlineto{\pgfqpoint{4.768000in}{3.119070in}}%
\pgfusepath{fill}%
\end{pgfscope}%
\begin{pgfscope}%
\pgfpathrectangle{\pgfqpoint{0.800000in}{0.528000in}}{\pgfqpoint{3.968000in}{3.696000in}}%
\pgfusepath{clip}%
\pgfsetbuttcap%
\pgfsetroundjoin%
\definecolor{currentfill}{rgb}{0.121380,0.629492,0.531973}%
\pgfsetfillcolor{currentfill}%
\pgfsetlinewidth{0.000000pt}%
\definecolor{currentstroke}{rgb}{0.000000,0.000000,0.000000}%
\pgfsetstrokecolor{currentstroke}%
\pgfsetdash{}{0pt}%
\pgfpathmoveto{\pgfqpoint{4.768000in}{3.125399in}}%
\pgfpathlineto{\pgfqpoint{4.606443in}{3.291816in}}%
\pgfpathlineto{\pgfqpoint{4.407273in}{3.493403in}}%
\pgfpathlineto{\pgfqpoint{4.083479in}{3.813333in}}%
\pgfpathlineto{\pgfqpoint{3.886222in}{4.003540in}}%
\pgfpathlineto{\pgfqpoint{3.685818in}{4.193223in}}%
\pgfpathlineto{\pgfqpoint{3.652940in}{4.224000in}}%
\pgfpathlineto{\pgfqpoint{3.649706in}{4.224000in}}%
\pgfpathlineto{\pgfqpoint{4.006465in}{3.884948in}}%
\pgfpathlineto{\pgfqpoint{4.345567in}{3.552000in}}%
\pgfpathlineto{\pgfqpoint{4.531156in}{3.365333in}}%
\pgfpathlineto{\pgfqpoint{4.713479in}{3.178667in}}%
\pgfpathlineto{\pgfqpoint{4.768000in}{3.122235in}}%
\pgfpathlineto{\pgfqpoint{4.768000in}{3.122235in}}%
\pgfusepath{fill}%
\end{pgfscope}%
\begin{pgfscope}%
\pgfpathrectangle{\pgfqpoint{0.800000in}{0.528000in}}{\pgfqpoint{3.968000in}{3.696000in}}%
\pgfusepath{clip}%
\pgfsetbuttcap%
\pgfsetroundjoin%
\definecolor{currentfill}{rgb}{0.122312,0.633153,0.530398}%
\pgfsetfillcolor{currentfill}%
\pgfsetlinewidth{0.000000pt}%
\definecolor{currentstroke}{rgb}{0.000000,0.000000,0.000000}%
\pgfsetstrokecolor{currentstroke}%
\pgfsetdash{}{0pt}%
\pgfpathmoveto{\pgfqpoint{4.768000in}{3.128564in}}%
\pgfpathlineto{\pgfqpoint{4.607677in}{3.293675in}}%
\pgfpathlineto{\pgfqpoint{4.407273in}{3.496508in}}%
\pgfpathlineto{\pgfqpoint{4.086626in}{3.813359in}}%
\pgfpathlineto{\pgfqpoint{3.886222in}{4.006580in}}%
\pgfpathlineto{\pgfqpoint{3.685818in}{4.196251in}}%
\pgfpathlineto{\pgfqpoint{3.656174in}{4.224000in}}%
\pgfpathlineto{\pgfqpoint{3.652940in}{4.224000in}}%
\pgfpathlineto{\pgfqpoint{4.006495in}{3.888000in}}%
\pgfpathlineto{\pgfqpoint{4.348677in}{3.552000in}}%
\pgfpathlineto{\pgfqpoint{4.534222in}{3.365333in}}%
\pgfpathlineto{\pgfqpoint{4.721905in}{3.173065in}}%
\pgfpathlineto{\pgfqpoint{4.768000in}{3.125399in}}%
\pgfpathlineto{\pgfqpoint{4.768000in}{3.125399in}}%
\pgfusepath{fill}%
\end{pgfscope}%
\begin{pgfscope}%
\pgfpathrectangle{\pgfqpoint{0.800000in}{0.528000in}}{\pgfqpoint{3.968000in}{3.696000in}}%
\pgfusepath{clip}%
\pgfsetbuttcap%
\pgfsetroundjoin%
\definecolor{currentfill}{rgb}{0.122312,0.633153,0.530398}%
\pgfsetfillcolor{currentfill}%
\pgfsetlinewidth{0.000000pt}%
\definecolor{currentstroke}{rgb}{0.000000,0.000000,0.000000}%
\pgfsetstrokecolor{currentstroke}%
\pgfsetdash{}{0pt}%
\pgfpathmoveto{\pgfqpoint{4.768000in}{3.131729in}}%
\pgfpathlineto{\pgfqpoint{4.577072in}{3.328000in}}%
\pgfpathlineto{\pgfqpoint{4.246949in}{3.659203in}}%
\pgfpathlineto{\pgfqpoint{4.086626in}{3.816409in}}%
\pgfpathlineto{\pgfqpoint{3.886222in}{4.009619in}}%
\pgfpathlineto{\pgfqpoint{3.685818in}{4.199278in}}%
\pgfpathlineto{\pgfqpoint{3.659409in}{4.224000in}}%
\pgfpathlineto{\pgfqpoint{3.656174in}{4.224000in}}%
\pgfpathlineto{\pgfqpoint{4.006465in}{3.891076in}}%
\pgfpathlineto{\pgfqpoint{4.327111in}{3.576602in}}%
\pgfpathlineto{\pgfqpoint{4.647758in}{3.252653in}}%
\pgfpathlineto{\pgfqpoint{4.768000in}{3.128564in}}%
\pgfpathlineto{\pgfqpoint{4.768000in}{3.128564in}}%
\pgfusepath{fill}%
\end{pgfscope}%
\begin{pgfscope}%
\pgfpathrectangle{\pgfqpoint{0.800000in}{0.528000in}}{\pgfqpoint{3.968000in}{3.696000in}}%
\pgfusepath{clip}%
\pgfsetbuttcap%
\pgfsetroundjoin%
\definecolor{currentfill}{rgb}{0.122312,0.633153,0.530398}%
\pgfsetfillcolor{currentfill}%
\pgfsetlinewidth{0.000000pt}%
\definecolor{currentstroke}{rgb}{0.000000,0.000000,0.000000}%
\pgfsetstrokecolor{currentstroke}%
\pgfsetdash{}{0pt}%
\pgfpathmoveto{\pgfqpoint{4.768000in}{3.134894in}}%
\pgfpathlineto{\pgfqpoint{4.580130in}{3.328000in}}%
\pgfpathlineto{\pgfqpoint{4.245226in}{3.664000in}}%
\pgfpathlineto{\pgfqpoint{3.899457in}{4.000000in}}%
\pgfpathlineto{\pgfqpoint{3.725899in}{4.164658in}}%
\pgfpathlineto{\pgfqpoint{3.662643in}{4.224000in}}%
\pgfpathlineto{\pgfqpoint{3.659409in}{4.224000in}}%
\pgfpathlineto{\pgfqpoint{4.006465in}{3.894122in}}%
\pgfpathlineto{\pgfqpoint{4.327111in}{3.579703in}}%
\pgfpathlineto{\pgfqpoint{4.673386in}{3.229462in}}%
\pgfpathlineto{\pgfqpoint{4.768000in}{3.131729in}}%
\pgfpathlineto{\pgfqpoint{4.768000in}{3.131729in}}%
\pgfusepath{fill}%
\end{pgfscope}%
\begin{pgfscope}%
\pgfpathrectangle{\pgfqpoint{0.800000in}{0.528000in}}{\pgfqpoint{3.968000in}{3.696000in}}%
\pgfusepath{clip}%
\pgfsetbuttcap%
\pgfsetroundjoin%
\definecolor{currentfill}{rgb}{0.123444,0.636809,0.528763}%
\pgfsetfillcolor{currentfill}%
\pgfsetlinewidth{0.000000pt}%
\definecolor{currentstroke}{rgb}{0.000000,0.000000,0.000000}%
\pgfsetstrokecolor{currentstroke}%
\pgfsetdash{}{0pt}%
\pgfpathmoveto{\pgfqpoint{4.768000in}{3.138058in}}%
\pgfpathlineto{\pgfqpoint{4.583187in}{3.328000in}}%
\pgfpathlineto{\pgfqpoint{4.246949in}{3.665378in}}%
\pgfpathlineto{\pgfqpoint{3.902634in}{4.000000in}}%
\pgfpathlineto{\pgfqpoint{3.725899in}{4.167688in}}%
\pgfpathlineto{\pgfqpoint{3.665877in}{4.224000in}}%
\pgfpathlineto{\pgfqpoint{3.662643in}{4.224000in}}%
\pgfpathlineto{\pgfqpoint{4.006465in}{3.897168in}}%
\pgfpathlineto{\pgfqpoint{4.327111in}{3.582804in}}%
\pgfpathlineto{\pgfqpoint{4.653183in}{3.253333in}}%
\pgfpathlineto{\pgfqpoint{4.768000in}{3.134894in}}%
\pgfpathlineto{\pgfqpoint{4.768000in}{3.134894in}}%
\pgfusepath{fill}%
\end{pgfscope}%
\begin{pgfscope}%
\pgfpathrectangle{\pgfqpoint{0.800000in}{0.528000in}}{\pgfqpoint{3.968000in}{3.696000in}}%
\pgfusepath{clip}%
\pgfsetbuttcap%
\pgfsetroundjoin%
\definecolor{currentfill}{rgb}{0.123444,0.636809,0.528763}%
\pgfsetfillcolor{currentfill}%
\pgfsetlinewidth{0.000000pt}%
\definecolor{currentstroke}{rgb}{0.000000,0.000000,0.000000}%
\pgfsetstrokecolor{currentstroke}%
\pgfsetdash{}{0pt}%
\pgfpathmoveto{\pgfqpoint{4.768000in}{3.141223in}}%
\pgfpathlineto{\pgfqpoint{4.586244in}{3.328000in}}%
\pgfpathlineto{\pgfqpoint{4.246949in}{3.668439in}}%
\pgfpathlineto{\pgfqpoint{3.905812in}{4.000000in}}%
\pgfpathlineto{\pgfqpoint{3.725899in}{4.170718in}}%
\pgfpathlineto{\pgfqpoint{3.669112in}{4.224000in}}%
\pgfpathlineto{\pgfqpoint{3.665877in}{4.224000in}}%
\pgfpathlineto{\pgfqpoint{4.006465in}{3.900214in}}%
\pgfpathlineto{\pgfqpoint{4.344600in}{3.568290in}}%
\pgfpathlineto{\pgfqpoint{4.447354in}{3.465563in}}%
\pgfpathlineto{\pgfqpoint{4.647758in}{3.262021in}}%
\pgfpathlineto{\pgfqpoint{4.768000in}{3.138058in}}%
\pgfpathlineto{\pgfqpoint{4.768000in}{3.138058in}}%
\pgfusepath{fill}%
\end{pgfscope}%
\begin{pgfscope}%
\pgfpathrectangle{\pgfqpoint{0.800000in}{0.528000in}}{\pgfqpoint{3.968000in}{3.696000in}}%
\pgfusepath{clip}%
\pgfsetbuttcap%
\pgfsetroundjoin%
\definecolor{currentfill}{rgb}{0.123444,0.636809,0.528763}%
\pgfsetfillcolor{currentfill}%
\pgfsetlinewidth{0.000000pt}%
\definecolor{currentstroke}{rgb}{0.000000,0.000000,0.000000}%
\pgfsetstrokecolor{currentstroke}%
\pgfsetdash{}{0pt}%
\pgfpathmoveto{\pgfqpoint{4.768000in}{3.144352in}}%
\pgfpathlineto{\pgfqpoint{4.567596in}{3.350116in}}%
\pgfpathlineto{\pgfqpoint{4.386487in}{3.532640in}}%
\pgfpathlineto{\pgfqpoint{4.287030in}{3.631859in}}%
\pgfpathlineto{\pgfqpoint{3.937222in}{3.972837in}}%
\pgfpathlineto{\pgfqpoint{3.830684in}{4.074667in}}%
\pgfpathlineto{\pgfqpoint{3.672346in}{4.224000in}}%
\pgfpathlineto{\pgfqpoint{3.669112in}{4.224000in}}%
\pgfpathlineto{\pgfqpoint{3.983572in}{3.925333in}}%
\pgfpathlineto{\pgfqpoint{4.175575in}{3.738667in}}%
\pgfpathlineto{\pgfqpoint{4.367192in}{3.549041in}}%
\pgfpathlineto{\pgfqpoint{4.695593in}{3.216000in}}%
\pgfpathlineto{\pgfqpoint{4.768000in}{3.141223in}}%
\pgfpathlineto{\pgfqpoint{4.768000in}{3.141333in}}%
\pgfusepath{fill}%
\end{pgfscope}%
\begin{pgfscope}%
\pgfpathrectangle{\pgfqpoint{0.800000in}{0.528000in}}{\pgfqpoint{3.968000in}{3.696000in}}%
\pgfusepath{clip}%
\pgfsetbuttcap%
\pgfsetroundjoin%
\definecolor{currentfill}{rgb}{0.123444,0.636809,0.528763}%
\pgfsetfillcolor{currentfill}%
\pgfsetlinewidth{0.000000pt}%
\definecolor{currentstroke}{rgb}{0.000000,0.000000,0.000000}%
\pgfsetstrokecolor{currentstroke}%
\pgfsetdash{}{0pt}%
\pgfpathmoveto{\pgfqpoint{4.768000in}{3.147479in}}%
\pgfpathlineto{\pgfqpoint{4.567596in}{3.353231in}}%
\pgfpathlineto{\pgfqpoint{4.370405in}{3.552000in}}%
\pgfpathlineto{\pgfqpoint{4.206869in}{3.714051in}}%
\pgfpathlineto{\pgfqpoint{4.006465in}{3.909352in}}%
\pgfpathlineto{\pgfqpoint{3.806061in}{4.101079in}}%
\pgfpathlineto{\pgfqpoint{3.675580in}{4.224000in}}%
\pgfpathlineto{\pgfqpoint{3.672346in}{4.224000in}}%
\pgfpathlineto{\pgfqpoint{3.986731in}{3.925333in}}%
\pgfpathlineto{\pgfqpoint{4.172861in}{3.744324in}}%
\pgfpathlineto{\pgfqpoint{4.270207in}{3.648330in}}%
\pgfpathlineto{\pgfqpoint{4.367335in}{3.552000in}}%
\pgfpathlineto{\pgfqpoint{4.698624in}{3.216000in}}%
\pgfpathlineto{\pgfqpoint{4.768000in}{3.144352in}}%
\pgfpathlineto{\pgfqpoint{4.768000in}{3.144352in}}%
\pgfusepath{fill}%
\end{pgfscope}%
\begin{pgfscope}%
\pgfpathrectangle{\pgfqpoint{0.800000in}{0.528000in}}{\pgfqpoint{3.968000in}{3.696000in}}%
\pgfusepath{clip}%
\pgfsetbuttcap%
\pgfsetroundjoin%
\definecolor{currentfill}{rgb}{0.124780,0.640461,0.527068}%
\pgfsetfillcolor{currentfill}%
\pgfsetlinewidth{0.000000pt}%
\definecolor{currentstroke}{rgb}{0.000000,0.000000,0.000000}%
\pgfsetstrokecolor{currentstroke}%
\pgfsetdash{}{0pt}%
\pgfpathmoveto{\pgfqpoint{4.768000in}{3.150606in}}%
\pgfpathlineto{\pgfqpoint{4.567596in}{3.356346in}}%
\pgfpathlineto{\pgfqpoint{4.373474in}{3.552000in}}%
\pgfpathlineto{\pgfqpoint{4.206869in}{3.717109in}}%
\pgfpathlineto{\pgfqpoint{4.006465in}{3.912398in}}%
\pgfpathlineto{\pgfqpoint{3.806061in}{4.104113in}}%
\pgfpathlineto{\pgfqpoint{3.678815in}{4.224000in}}%
\pgfpathlineto{\pgfqpoint{3.675580in}{4.224000in}}%
\pgfpathlineto{\pgfqpoint{3.989891in}{3.925333in}}%
\pgfpathlineto{\pgfqpoint{4.166788in}{3.753399in}}%
\pgfpathlineto{\pgfqpoint{4.487434in}{3.434484in}}%
\pgfpathlineto{\pgfqpoint{4.687838in}{3.230231in}}%
\pgfpathlineto{\pgfqpoint{4.768000in}{3.147479in}}%
\pgfpathlineto{\pgfqpoint{4.768000in}{3.147479in}}%
\pgfusepath{fill}%
\end{pgfscope}%
\begin{pgfscope}%
\pgfpathrectangle{\pgfqpoint{0.800000in}{0.528000in}}{\pgfqpoint{3.968000in}{3.696000in}}%
\pgfusepath{clip}%
\pgfsetbuttcap%
\pgfsetroundjoin%
\definecolor{currentfill}{rgb}{0.124780,0.640461,0.527068}%
\pgfsetfillcolor{currentfill}%
\pgfsetlinewidth{0.000000pt}%
\definecolor{currentstroke}{rgb}{0.000000,0.000000,0.000000}%
\pgfsetstrokecolor{currentstroke}%
\pgfsetdash{}{0pt}%
\pgfpathmoveto{\pgfqpoint{4.768000in}{3.153734in}}%
\pgfpathlineto{\pgfqpoint{4.567596in}{3.359462in}}%
\pgfpathlineto{\pgfqpoint{4.376544in}{3.552000in}}%
\pgfpathlineto{\pgfqpoint{4.206869in}{3.720167in}}%
\pgfpathlineto{\pgfqpoint{4.006465in}{3.915444in}}%
\pgfpathlineto{\pgfqpoint{3.806061in}{4.107148in}}%
\pgfpathlineto{\pgfqpoint{3.682049in}{4.224000in}}%
\pgfpathlineto{\pgfqpoint{3.678815in}{4.224000in}}%
\pgfpathlineto{\pgfqpoint{3.993050in}{3.925333in}}%
\pgfpathlineto{\pgfqpoint{4.166788in}{3.756455in}}%
\pgfpathlineto{\pgfqpoint{4.487434in}{3.437594in}}%
\pgfpathlineto{\pgfqpoint{4.687838in}{3.233353in}}%
\pgfpathlineto{\pgfqpoint{4.768000in}{3.150606in}}%
\pgfpathlineto{\pgfqpoint{4.768000in}{3.150606in}}%
\pgfusepath{fill}%
\end{pgfscope}%
\begin{pgfscope}%
\pgfpathrectangle{\pgfqpoint{0.800000in}{0.528000in}}{\pgfqpoint{3.968000in}{3.696000in}}%
\pgfusepath{clip}%
\pgfsetbuttcap%
\pgfsetroundjoin%
\definecolor{currentfill}{rgb}{0.124780,0.640461,0.527068}%
\pgfsetfillcolor{currentfill}%
\pgfsetlinewidth{0.000000pt}%
\definecolor{currentstroke}{rgb}{0.000000,0.000000,0.000000}%
\pgfsetstrokecolor{currentstroke}%
\pgfsetdash{}{0pt}%
\pgfpathmoveto{\pgfqpoint{4.768000in}{3.156861in}}%
\pgfpathlineto{\pgfqpoint{4.567596in}{3.362577in}}%
\pgfpathlineto{\pgfqpoint{4.379614in}{3.552000in}}%
\pgfpathlineto{\pgfqpoint{4.206869in}{3.723224in}}%
\pgfpathlineto{\pgfqpoint{4.006465in}{3.918490in}}%
\pgfpathlineto{\pgfqpoint{3.824844in}{4.092162in}}%
\pgfpathlineto{\pgfqpoint{3.725049in}{4.186667in}}%
\pgfpathlineto{\pgfqpoint{3.685283in}{4.224000in}}%
\pgfpathlineto{\pgfqpoint{3.682049in}{4.224000in}}%
\pgfpathlineto{\pgfqpoint{3.996209in}{3.925333in}}%
\pgfpathlineto{\pgfqpoint{4.166788in}{3.759510in}}%
\pgfpathlineto{\pgfqpoint{4.496223in}{3.431814in}}%
\pgfpathlineto{\pgfqpoint{4.687838in}{3.236476in}}%
\pgfpathlineto{\pgfqpoint{4.768000in}{3.153734in}}%
\pgfpathlineto{\pgfqpoint{4.768000in}{3.153734in}}%
\pgfusepath{fill}%
\end{pgfscope}%
\begin{pgfscope}%
\pgfpathrectangle{\pgfqpoint{0.800000in}{0.528000in}}{\pgfqpoint{3.968000in}{3.696000in}}%
\pgfusepath{clip}%
\pgfsetbuttcap%
\pgfsetroundjoin%
\definecolor{currentfill}{rgb}{0.124780,0.640461,0.527068}%
\pgfsetfillcolor{currentfill}%
\pgfsetlinewidth{0.000000pt}%
\definecolor{currentstroke}{rgb}{0.000000,0.000000,0.000000}%
\pgfsetstrokecolor{currentstroke}%
\pgfsetdash{}{0pt}%
\pgfpathmoveto{\pgfqpoint{4.768000in}{3.159988in}}%
\pgfpathlineto{\pgfqpoint{4.567596in}{3.365688in}}%
\pgfpathlineto{\pgfqpoint{4.232202in}{3.701333in}}%
\pgfpathlineto{\pgfqpoint{4.041144in}{3.888000in}}%
\pgfpathlineto{\pgfqpoint{3.688481in}{4.224000in}}%
\pgfpathlineto{\pgfqpoint{3.685283in}{4.224000in}}%
\pgfpathlineto{\pgfqpoint{3.685818in}{4.223499in}}%
\pgfpathlineto{\pgfqpoint{4.046545in}{3.879724in}}%
\pgfpathlineto{\pgfqpoint{4.246949in}{3.683739in}}%
\pgfpathlineto{\pgfqpoint{4.447354in}{3.484130in}}%
\pgfpathlineto{\pgfqpoint{4.638059in}{3.290667in}}%
\pgfpathlineto{\pgfqpoint{4.768000in}{3.156861in}}%
\pgfpathlineto{\pgfqpoint{4.768000in}{3.156861in}}%
\pgfusepath{fill}%
\end{pgfscope}%
\begin{pgfscope}%
\pgfpathrectangle{\pgfqpoint{0.800000in}{0.528000in}}{\pgfqpoint{3.968000in}{3.696000in}}%
\pgfusepath{clip}%
\pgfsetbuttcap%
\pgfsetroundjoin%
\definecolor{currentfill}{rgb}{0.126326,0.644107,0.525311}%
\pgfsetfillcolor{currentfill}%
\pgfsetlinewidth{0.000000pt}%
\definecolor{currentstroke}{rgb}{0.000000,0.000000,0.000000}%
\pgfsetstrokecolor{currentstroke}%
\pgfsetdash{}{0pt}%
\pgfpathmoveto{\pgfqpoint{4.768000in}{3.163116in}}%
\pgfpathlineto{\pgfqpoint{4.570971in}{3.365333in}}%
\pgfpathlineto{\pgfqpoint{4.246949in}{3.689859in}}%
\pgfpathlineto{\pgfqpoint{3.926303in}{4.001657in}}%
\pgfpathlineto{\pgfqpoint{3.725899in}{4.191868in}}%
\pgfpathlineto{\pgfqpoint{3.691672in}{4.224000in}}%
\pgfpathlineto{\pgfqpoint{3.688481in}{4.224000in}}%
\pgfpathlineto{\pgfqpoint{4.046545in}{3.882773in}}%
\pgfpathlineto{\pgfqpoint{4.246949in}{3.686799in}}%
\pgfpathlineto{\pgfqpoint{4.447354in}{3.487202in}}%
\pgfpathlineto{\pgfqpoint{4.641107in}{3.290667in}}%
\pgfpathlineto{\pgfqpoint{4.768000in}{3.159988in}}%
\pgfpathlineto{\pgfqpoint{4.768000in}{3.159988in}}%
\pgfusepath{fill}%
\end{pgfscope}%
\begin{pgfscope}%
\pgfpathrectangle{\pgfqpoint{0.800000in}{0.528000in}}{\pgfqpoint{3.968000in}{3.696000in}}%
\pgfusepath{clip}%
\pgfsetbuttcap%
\pgfsetroundjoin%
\definecolor{currentfill}{rgb}{0.126326,0.644107,0.525311}%
\pgfsetfillcolor{currentfill}%
\pgfsetlinewidth{0.000000pt}%
\definecolor{currentstroke}{rgb}{0.000000,0.000000,0.000000}%
\pgfsetstrokecolor{currentstroke}%
\pgfsetdash{}{0pt}%
\pgfpathmoveto{\pgfqpoint{4.768000in}{3.166243in}}%
\pgfpathlineto{\pgfqpoint{4.573998in}{3.365333in}}%
\pgfpathlineto{\pgfqpoint{4.246949in}{3.692919in}}%
\pgfpathlineto{\pgfqpoint{3.926303in}{4.004664in}}%
\pgfpathlineto{\pgfqpoint{3.725899in}{4.194864in}}%
\pgfpathlineto{\pgfqpoint{3.694863in}{4.224000in}}%
\pgfpathlineto{\pgfqpoint{3.691672in}{4.224000in}}%
\pgfpathlineto{\pgfqpoint{4.046545in}{3.885821in}}%
\pgfpathlineto{\pgfqpoint{4.246949in}{3.689859in}}%
\pgfpathlineto{\pgfqpoint{4.447354in}{3.490274in}}%
\pgfpathlineto{\pgfqpoint{4.644156in}{3.290667in}}%
\pgfpathlineto{\pgfqpoint{4.768000in}{3.163116in}}%
\pgfpathlineto{\pgfqpoint{4.768000in}{3.163116in}}%
\pgfusepath{fill}%
\end{pgfscope}%
\begin{pgfscope}%
\pgfpathrectangle{\pgfqpoint{0.800000in}{0.528000in}}{\pgfqpoint{3.968000in}{3.696000in}}%
\pgfusepath{clip}%
\pgfsetbuttcap%
\pgfsetroundjoin%
\definecolor{currentfill}{rgb}{0.126326,0.644107,0.525311}%
\pgfsetfillcolor{currentfill}%
\pgfsetlinewidth{0.000000pt}%
\definecolor{currentstroke}{rgb}{0.000000,0.000000,0.000000}%
\pgfsetstrokecolor{currentstroke}%
\pgfsetdash{}{0pt}%
\pgfpathmoveto{\pgfqpoint{4.768000in}{3.169371in}}%
\pgfpathlineto{\pgfqpoint{4.577025in}{3.365333in}}%
\pgfpathlineto{\pgfqpoint{4.244142in}{3.698718in}}%
\pgfpathlineto{\pgfqpoint{4.165548in}{3.776000in}}%
\pgfpathlineto{\pgfqpoint{3.806061in}{4.122203in}}%
\pgfpathlineto{\pgfqpoint{3.698054in}{4.224000in}}%
\pgfpathlineto{\pgfqpoint{3.694863in}{4.224000in}}%
\pgfpathlineto{\pgfqpoint{4.047432in}{3.888000in}}%
\pgfpathlineto{\pgfqpoint{4.246949in}{3.692919in}}%
\pgfpathlineto{\pgfqpoint{4.447354in}{3.493346in}}%
\pgfpathlineto{\pgfqpoint{4.647758in}{3.290101in}}%
\pgfpathlineto{\pgfqpoint{4.768000in}{3.166243in}}%
\pgfpathlineto{\pgfqpoint{4.768000in}{3.166243in}}%
\pgfusepath{fill}%
\end{pgfscope}%
\begin{pgfscope}%
\pgfpathrectangle{\pgfqpoint{0.800000in}{0.528000in}}{\pgfqpoint{3.968000in}{3.696000in}}%
\pgfusepath{clip}%
\pgfsetbuttcap%
\pgfsetroundjoin%
\definecolor{currentfill}{rgb}{0.128087,0.647749,0.523491}%
\pgfsetfillcolor{currentfill}%
\pgfsetlinewidth{0.000000pt}%
\definecolor{currentstroke}{rgb}{0.000000,0.000000,0.000000}%
\pgfsetstrokecolor{currentstroke}%
\pgfsetdash{}{0pt}%
\pgfpathmoveto{\pgfqpoint{4.768000in}{3.172498in}}%
\pgfpathlineto{\pgfqpoint{4.580052in}{3.365333in}}%
\pgfpathlineto{\pgfqpoint{4.244622in}{3.701333in}}%
\pgfpathlineto{\pgfqpoint{4.046545in}{3.894887in}}%
\pgfpathlineto{\pgfqpoint{3.701245in}{4.224000in}}%
\pgfpathlineto{\pgfqpoint{3.698054in}{4.224000in}}%
\pgfpathlineto{\pgfqpoint{4.046545in}{3.891873in}}%
\pgfpathlineto{\pgfqpoint{4.241517in}{3.701333in}}%
\pgfpathlineto{\pgfqpoint{4.567596in}{3.374924in}}%
\pgfpathlineto{\pgfqpoint{4.768000in}{3.169371in}}%
\pgfpathlineto{\pgfqpoint{4.768000in}{3.169371in}}%
\pgfusepath{fill}%
\end{pgfscope}%
\begin{pgfscope}%
\pgfpathrectangle{\pgfqpoint{0.800000in}{0.528000in}}{\pgfqpoint{3.968000in}{3.696000in}}%
\pgfusepath{clip}%
\pgfsetbuttcap%
\pgfsetroundjoin%
\definecolor{currentfill}{rgb}{0.128087,0.647749,0.523491}%
\pgfsetfillcolor{currentfill}%
\pgfsetlinewidth{0.000000pt}%
\definecolor{currentstroke}{rgb}{0.000000,0.000000,0.000000}%
\pgfsetstrokecolor{currentstroke}%
\pgfsetdash{}{0pt}%
\pgfpathmoveto{\pgfqpoint{4.768000in}{3.175625in}}%
\pgfpathlineto{\pgfqpoint{4.583079in}{3.365333in}}%
\pgfpathlineto{\pgfqpoint{4.246949in}{3.702091in}}%
\pgfpathlineto{\pgfqpoint{4.046545in}{3.897901in}}%
\pgfpathlineto{\pgfqpoint{3.704436in}{4.224000in}}%
\pgfpathlineto{\pgfqpoint{3.701245in}{4.224000in}}%
\pgfpathlineto{\pgfqpoint{4.046545in}{3.894887in}}%
\pgfpathlineto{\pgfqpoint{4.244622in}{3.701333in}}%
\pgfpathlineto{\pgfqpoint{4.567596in}{3.378003in}}%
\pgfpathlineto{\pgfqpoint{4.768000in}{3.172498in}}%
\pgfpathlineto{\pgfqpoint{4.768000in}{3.172498in}}%
\pgfusepath{fill}%
\end{pgfscope}%
\begin{pgfscope}%
\pgfpathrectangle{\pgfqpoint{0.800000in}{0.528000in}}{\pgfqpoint{3.968000in}{3.696000in}}%
\pgfusepath{clip}%
\pgfsetbuttcap%
\pgfsetroundjoin%
\definecolor{currentfill}{rgb}{0.128087,0.647749,0.523491}%
\pgfsetfillcolor{currentfill}%
\pgfsetlinewidth{0.000000pt}%
\definecolor{currentstroke}{rgb}{0.000000,0.000000,0.000000}%
\pgfsetstrokecolor{currentstroke}%
\pgfsetdash{}{0pt}%
\pgfpathmoveto{\pgfqpoint{4.768000in}{3.178752in}}%
\pgfpathlineto{\pgfqpoint{4.423301in}{3.529596in}}%
\pgfpathlineto{\pgfqpoint{4.312566in}{3.640214in}}%
\pgfpathlineto{\pgfqpoint{4.126707in}{3.823022in}}%
\pgfpathlineto{\pgfqpoint{3.776456in}{4.159091in}}%
\pgfpathlineto{\pgfqpoint{3.707627in}{4.224000in}}%
\pgfpathlineto{\pgfqpoint{3.704436in}{4.224000in}}%
\pgfpathlineto{\pgfqpoint{4.046545in}{3.897901in}}%
\pgfpathlineto{\pgfqpoint{4.247717in}{3.701333in}}%
\pgfpathlineto{\pgfqpoint{4.435290in}{3.514667in}}%
\pgfpathlineto{\pgfqpoint{4.607677in}{3.340295in}}%
\pgfpathlineto{\pgfqpoint{4.768000in}{3.175625in}}%
\pgfpathlineto{\pgfqpoint{4.768000in}{3.178667in}}%
\pgfpathlineto{\pgfqpoint{4.768000in}{3.178667in}}%
\pgfusepath{fill}%
\end{pgfscope}%
\begin{pgfscope}%
\pgfpathrectangle{\pgfqpoint{0.800000in}{0.528000in}}{\pgfqpoint{3.968000in}{3.696000in}}%
\pgfusepath{clip}%
\pgfsetbuttcap%
\pgfsetroundjoin%
\definecolor{currentfill}{rgb}{0.128087,0.647749,0.523491}%
\pgfsetfillcolor{currentfill}%
\pgfsetlinewidth{0.000000pt}%
\definecolor{currentstroke}{rgb}{0.000000,0.000000,0.000000}%
\pgfsetstrokecolor{currentstroke}%
\pgfsetdash{}{0pt}%
\pgfpathmoveto{\pgfqpoint{4.768000in}{3.181842in}}%
\pgfpathlineto{\pgfqpoint{4.424880in}{3.531067in}}%
\pgfpathlineto{\pgfqpoint{4.327111in}{3.628826in}}%
\pgfpathlineto{\pgfqpoint{4.126707in}{3.826040in}}%
\pgfpathlineto{\pgfqpoint{3.778045in}{4.160572in}}%
\pgfpathlineto{\pgfqpoint{3.710818in}{4.224000in}}%
\pgfpathlineto{\pgfqpoint{3.707627in}{4.224000in}}%
\pgfpathlineto{\pgfqpoint{4.046545in}{3.900914in}}%
\pgfpathlineto{\pgfqpoint{4.246949in}{3.705116in}}%
\pgfpathlineto{\pgfqpoint{4.567596in}{3.384161in}}%
\pgfpathlineto{\pgfqpoint{4.768000in}{3.178752in}}%
\pgfpathlineto{\pgfqpoint{4.768000in}{3.178752in}}%
\pgfusepath{fill}%
\end{pgfscope}%
\begin{pgfscope}%
\pgfpathrectangle{\pgfqpoint{0.800000in}{0.528000in}}{\pgfqpoint{3.968000in}{3.696000in}}%
\pgfusepath{clip}%
\pgfsetbuttcap%
\pgfsetroundjoin%
\definecolor{currentfill}{rgb}{0.130067,0.651384,0.521608}%
\pgfsetfillcolor{currentfill}%
\pgfsetlinewidth{0.000000pt}%
\definecolor{currentstroke}{rgb}{0.000000,0.000000,0.000000}%
\pgfsetstrokecolor{currentstroke}%
\pgfsetdash{}{0pt}%
\pgfpathmoveto{\pgfqpoint{4.768000in}{3.184933in}}%
\pgfpathlineto{\pgfqpoint{4.426459in}{3.532537in}}%
\pgfpathlineto{\pgfqpoint{4.327111in}{3.631855in}}%
\pgfpathlineto{\pgfqpoint{4.126707in}{3.829058in}}%
\pgfpathlineto{\pgfqpoint{3.779634in}{4.162052in}}%
\pgfpathlineto{\pgfqpoint{3.714009in}{4.224000in}}%
\pgfpathlineto{\pgfqpoint{3.710818in}{4.224000in}}%
\pgfpathlineto{\pgfqpoint{4.046545in}{3.903928in}}%
\pgfpathlineto{\pgfqpoint{4.246949in}{3.708141in}}%
\pgfpathlineto{\pgfqpoint{4.578780in}{3.375751in}}%
\pgfpathlineto{\pgfqpoint{4.687838in}{3.264447in}}%
\pgfpathlineto{\pgfqpoint{4.768000in}{3.181842in}}%
\pgfpathlineto{\pgfqpoint{4.768000in}{3.181842in}}%
\pgfusepath{fill}%
\end{pgfscope}%
\begin{pgfscope}%
\pgfpathrectangle{\pgfqpoint{0.800000in}{0.528000in}}{\pgfqpoint{3.968000in}{3.696000in}}%
\pgfusepath{clip}%
\pgfsetbuttcap%
\pgfsetroundjoin%
\definecolor{currentfill}{rgb}{0.130067,0.651384,0.521608}%
\pgfsetfillcolor{currentfill}%
\pgfsetlinewidth{0.000000pt}%
\definecolor{currentstroke}{rgb}{0.000000,0.000000,0.000000}%
\pgfsetstrokecolor{currentstroke}%
\pgfsetdash{}{0pt}%
\pgfpathmoveto{\pgfqpoint{4.768000in}{3.188024in}}%
\pgfpathlineto{\pgfqpoint{4.428037in}{3.534008in}}%
\pgfpathlineto{\pgfqpoint{4.327111in}{3.634885in}}%
\pgfpathlineto{\pgfqpoint{4.126707in}{3.832077in}}%
\pgfpathlineto{\pgfqpoint{3.781224in}{4.163532in}}%
\pgfpathlineto{\pgfqpoint{3.717199in}{4.224000in}}%
\pgfpathlineto{\pgfqpoint{3.714009in}{4.224000in}}%
\pgfpathlineto{\pgfqpoint{4.036805in}{3.916261in}}%
\pgfpathlineto{\pgfqpoint{4.142811in}{3.813333in}}%
\pgfpathlineto{\pgfqpoint{4.487434in}{3.471438in}}%
\pgfpathlineto{\pgfqpoint{4.768000in}{3.184933in}}%
\pgfpathlineto{\pgfqpoint{4.768000in}{3.184933in}}%
\pgfusepath{fill}%
\end{pgfscope}%
\begin{pgfscope}%
\pgfpathrectangle{\pgfqpoint{0.800000in}{0.528000in}}{\pgfqpoint{3.968000in}{3.696000in}}%
\pgfusepath{clip}%
\pgfsetbuttcap%
\pgfsetroundjoin%
\definecolor{currentfill}{rgb}{0.130067,0.651384,0.521608}%
\pgfsetfillcolor{currentfill}%
\pgfsetlinewidth{0.000000pt}%
\definecolor{currentstroke}{rgb}{0.000000,0.000000,0.000000}%
\pgfsetstrokecolor{currentstroke}%
\pgfsetdash{}{0pt}%
\pgfpathmoveto{\pgfqpoint{4.768000in}{3.191115in}}%
\pgfpathlineto{\pgfqpoint{4.429616in}{3.535478in}}%
\pgfpathlineto{\pgfqpoint{4.327111in}{3.637915in}}%
\pgfpathlineto{\pgfqpoint{4.126707in}{3.835095in}}%
\pgfpathlineto{\pgfqpoint{3.782813in}{4.165013in}}%
\pgfpathlineto{\pgfqpoint{3.720390in}{4.224000in}}%
\pgfpathlineto{\pgfqpoint{3.717199in}{4.224000in}}%
\pgfpathlineto{\pgfqpoint{4.038401in}{3.917747in}}%
\pgfpathlineto{\pgfqpoint{4.126707in}{3.832077in}}%
\pgfpathlineto{\pgfqpoint{4.327111in}{3.634885in}}%
\pgfpathlineto{\pgfqpoint{4.647758in}{3.311694in}}%
\pgfpathlineto{\pgfqpoint{4.768000in}{3.188024in}}%
\pgfpathlineto{\pgfqpoint{4.768000in}{3.188024in}}%
\pgfusepath{fill}%
\end{pgfscope}%
\begin{pgfscope}%
\pgfpathrectangle{\pgfqpoint{0.800000in}{0.528000in}}{\pgfqpoint{3.968000in}{3.696000in}}%
\pgfusepath{clip}%
\pgfsetbuttcap%
\pgfsetroundjoin%
\definecolor{currentfill}{rgb}{0.130067,0.651384,0.521608}%
\pgfsetfillcolor{currentfill}%
\pgfsetlinewidth{0.000000pt}%
\definecolor{currentstroke}{rgb}{0.000000,0.000000,0.000000}%
\pgfsetstrokecolor{currentstroke}%
\pgfsetdash{}{0pt}%
\pgfpathmoveto{\pgfqpoint{4.768000in}{3.194206in}}%
\pgfpathlineto{\pgfqpoint{4.431195in}{3.536949in}}%
\pgfpathlineto{\pgfqpoint{4.327111in}{3.640944in}}%
\pgfpathlineto{\pgfqpoint{4.126707in}{3.838113in}}%
\pgfpathlineto{\pgfqpoint{3.784402in}{4.166493in}}%
\pgfpathlineto{\pgfqpoint{3.723581in}{4.224000in}}%
\pgfpathlineto{\pgfqpoint{3.720390in}{4.224000in}}%
\pgfpathlineto{\pgfqpoint{4.039997in}{3.919234in}}%
\pgfpathlineto{\pgfqpoint{4.126707in}{3.835095in}}%
\pgfpathlineto{\pgfqpoint{4.327111in}{3.637915in}}%
\pgfpathlineto{\pgfqpoint{4.647758in}{3.314777in}}%
\pgfpathlineto{\pgfqpoint{4.768000in}{3.191115in}}%
\pgfpathlineto{\pgfqpoint{4.768000in}{3.191115in}}%
\pgfusepath{fill}%
\end{pgfscope}%
\begin{pgfscope}%
\pgfpathrectangle{\pgfqpoint{0.800000in}{0.528000in}}{\pgfqpoint{3.968000in}{3.696000in}}%
\pgfusepath{clip}%
\pgfsetbuttcap%
\pgfsetroundjoin%
\definecolor{currentfill}{rgb}{0.132268,0.655014,0.519661}%
\pgfsetfillcolor{currentfill}%
\pgfsetlinewidth{0.000000pt}%
\definecolor{currentstroke}{rgb}{0.000000,0.000000,0.000000}%
\pgfsetstrokecolor{currentstroke}%
\pgfsetdash{}{0pt}%
\pgfpathmoveto{\pgfqpoint{4.768000in}{3.197297in}}%
\pgfpathlineto{\pgfqpoint{4.432773in}{3.538419in}}%
\pgfpathlineto{\pgfqpoint{4.327111in}{3.643974in}}%
\pgfpathlineto{\pgfqpoint{4.126707in}{3.841131in}}%
\pgfpathlineto{\pgfqpoint{3.785991in}{4.167973in}}%
\pgfpathlineto{\pgfqpoint{3.725899in}{4.224000in}}%
\pgfpathlineto{\pgfqpoint{3.723581in}{4.224000in}}%
\pgfpathlineto{\pgfqpoint{3.724729in}{4.222911in}}%
\pgfpathlineto{\pgfqpoint{3.806061in}{4.146204in}}%
\pgfpathlineto{\pgfqpoint{4.166788in}{3.798965in}}%
\pgfpathlineto{\pgfqpoint{4.367192in}{3.601080in}}%
\pgfpathlineto{\pgfqpoint{4.687838in}{3.276791in}}%
\pgfpathlineto{\pgfqpoint{4.768000in}{3.194206in}}%
\pgfpathlineto{\pgfqpoint{4.768000in}{3.194206in}}%
\pgfusepath{fill}%
\end{pgfscope}%
\begin{pgfscope}%
\pgfpathrectangle{\pgfqpoint{0.800000in}{0.528000in}}{\pgfqpoint{3.968000in}{3.696000in}}%
\pgfusepath{clip}%
\pgfsetbuttcap%
\pgfsetroundjoin%
\definecolor{currentfill}{rgb}{0.132268,0.655014,0.519661}%
\pgfsetfillcolor{currentfill}%
\pgfsetlinewidth{0.000000pt}%
\definecolor{currentstroke}{rgb}{0.000000,0.000000,0.000000}%
\pgfsetstrokecolor{currentstroke}%
\pgfsetdash{}{0pt}%
\pgfpathmoveto{\pgfqpoint{4.768000in}{3.200387in}}%
\pgfpathlineto{\pgfqpoint{4.447354in}{3.526993in}}%
\pgfpathlineto{\pgfqpoint{4.103595in}{3.866473in}}%
\pgfpathlineto{\pgfqpoint{4.004452in}{3.962667in}}%
\pgfpathlineto{\pgfqpoint{3.806061in}{4.152172in}}%
\pgfpathlineto{\pgfqpoint{3.729909in}{4.224000in}}%
\pgfpathlineto{\pgfqpoint{3.726761in}{4.224000in}}%
\pgfpathlineto{\pgfqpoint{3.926303in}{4.034733in}}%
\pgfpathlineto{\pgfqpoint{4.269170in}{3.701333in}}%
\pgfpathlineto{\pgfqpoint{4.456599in}{3.514667in}}%
\pgfpathlineto{\pgfqpoint{4.647758in}{3.320945in}}%
\pgfpathlineto{\pgfqpoint{4.768000in}{3.197297in}}%
\pgfpathlineto{\pgfqpoint{4.768000in}{3.197297in}}%
\pgfusepath{fill}%
\end{pgfscope}%
\begin{pgfscope}%
\pgfpathrectangle{\pgfqpoint{0.800000in}{0.528000in}}{\pgfqpoint{3.968000in}{3.696000in}}%
\pgfusepath{clip}%
\pgfsetbuttcap%
\pgfsetroundjoin%
\definecolor{currentfill}{rgb}{0.132268,0.655014,0.519661}%
\pgfsetfillcolor{currentfill}%
\pgfsetlinewidth{0.000000pt}%
\definecolor{currentstroke}{rgb}{0.000000,0.000000,0.000000}%
\pgfsetstrokecolor{currentstroke}%
\pgfsetdash{}{0pt}%
\pgfpathmoveto{\pgfqpoint{4.768000in}{3.203478in}}%
\pgfpathlineto{\pgfqpoint{4.447354in}{3.530029in}}%
\pgfpathlineto{\pgfqpoint{4.105171in}{3.867940in}}%
\pgfpathlineto{\pgfqpoint{4.006465in}{3.963727in}}%
\pgfpathlineto{\pgfqpoint{3.806061in}{4.155139in}}%
\pgfpathlineto{\pgfqpoint{3.733058in}{4.224000in}}%
\pgfpathlineto{\pgfqpoint{3.729909in}{4.224000in}}%
\pgfpathlineto{\pgfqpoint{3.926723in}{4.037333in}}%
\pgfpathlineto{\pgfqpoint{4.287030in}{3.686719in}}%
\pgfpathlineto{\pgfqpoint{4.607677in}{3.364945in}}%
\pgfpathlineto{\pgfqpoint{4.768000in}{3.200387in}}%
\pgfpathlineto{\pgfqpoint{4.768000in}{3.200387in}}%
\pgfusepath{fill}%
\end{pgfscope}%
\begin{pgfscope}%
\pgfpathrectangle{\pgfqpoint{0.800000in}{0.528000in}}{\pgfqpoint{3.968000in}{3.696000in}}%
\pgfusepath{clip}%
\pgfsetbuttcap%
\pgfsetroundjoin%
\definecolor{currentfill}{rgb}{0.132268,0.655014,0.519661}%
\pgfsetfillcolor{currentfill}%
\pgfsetlinewidth{0.000000pt}%
\definecolor{currentstroke}{rgb}{0.000000,0.000000,0.000000}%
\pgfsetstrokecolor{currentstroke}%
\pgfsetdash{}{0pt}%
\pgfpathmoveto{\pgfqpoint{4.768000in}{3.206569in}}%
\pgfpathlineto{\pgfqpoint{4.447354in}{3.533066in}}%
\pgfpathlineto{\pgfqpoint{4.106746in}{3.869408in}}%
\pgfpathlineto{\pgfqpoint{4.006465in}{3.966705in}}%
\pgfpathlineto{\pgfqpoint{3.806061in}{4.158105in}}%
\pgfpathlineto{\pgfqpoint{3.736206in}{4.224000in}}%
\pgfpathlineto{\pgfqpoint{3.733058in}{4.224000in}}%
\pgfpathlineto{\pgfqpoint{3.929826in}{4.037333in}}%
\pgfpathlineto{\pgfqpoint{4.287030in}{3.689747in}}%
\pgfpathlineto{\pgfqpoint{4.637823in}{3.337254in}}%
\pgfpathlineto{\pgfqpoint{4.768000in}{3.203478in}}%
\pgfpathlineto{\pgfqpoint{4.768000in}{3.203478in}}%
\pgfusepath{fill}%
\end{pgfscope}%
\begin{pgfscope}%
\pgfpathrectangle{\pgfqpoint{0.800000in}{0.528000in}}{\pgfqpoint{3.968000in}{3.696000in}}%
\pgfusepath{clip}%
\pgfsetbuttcap%
\pgfsetroundjoin%
\definecolor{currentfill}{rgb}{0.134692,0.658636,0.517649}%
\pgfsetfillcolor{currentfill}%
\pgfsetlinewidth{0.000000pt}%
\definecolor{currentstroke}{rgb}{0.000000,0.000000,0.000000}%
\pgfsetstrokecolor{currentstroke}%
\pgfsetdash{}{0pt}%
\pgfpathmoveto{\pgfqpoint{4.768000in}{3.209660in}}%
\pgfpathlineto{\pgfqpoint{4.447354in}{3.536103in}}%
\pgfpathlineto{\pgfqpoint{4.108322in}{3.870875in}}%
\pgfpathlineto{\pgfqpoint{4.006465in}{3.969682in}}%
\pgfpathlineto{\pgfqpoint{3.806061in}{4.161072in}}%
\pgfpathlineto{\pgfqpoint{3.739355in}{4.224000in}}%
\pgfpathlineto{\pgfqpoint{3.736206in}{4.224000in}}%
\pgfpathlineto{\pgfqpoint{3.932930in}{4.037333in}}%
\pgfpathlineto{\pgfqpoint{4.287030in}{3.692774in}}%
\pgfpathlineto{\pgfqpoint{4.613278in}{3.365333in}}%
\pgfpathlineto{\pgfqpoint{4.768000in}{3.206569in}}%
\pgfpathlineto{\pgfqpoint{4.768000in}{3.206569in}}%
\pgfusepath{fill}%
\end{pgfscope}%
\begin{pgfscope}%
\pgfpathrectangle{\pgfqpoint{0.800000in}{0.528000in}}{\pgfqpoint{3.968000in}{3.696000in}}%
\pgfusepath{clip}%
\pgfsetbuttcap%
\pgfsetroundjoin%
\definecolor{currentfill}{rgb}{0.134692,0.658636,0.517649}%
\pgfsetfillcolor{currentfill}%
\pgfsetlinewidth{0.000000pt}%
\definecolor{currentstroke}{rgb}{0.000000,0.000000,0.000000}%
\pgfsetstrokecolor{currentstroke}%
\pgfsetdash{}{0pt}%
\pgfpathmoveto{\pgfqpoint{4.768000in}{3.212751in}}%
\pgfpathlineto{\pgfqpoint{4.447354in}{3.539139in}}%
\pgfpathlineto{\pgfqpoint{4.109897in}{3.872342in}}%
\pgfpathlineto{\pgfqpoint{4.006465in}{3.972660in}}%
\pgfpathlineto{\pgfqpoint{3.806061in}{4.164038in}}%
\pgfpathlineto{\pgfqpoint{3.742504in}{4.224000in}}%
\pgfpathlineto{\pgfqpoint{3.739355in}{4.224000in}}%
\pgfpathlineto{\pgfqpoint{3.936034in}{4.037333in}}%
\pgfpathlineto{\pgfqpoint{4.287030in}{3.695801in}}%
\pgfpathlineto{\pgfqpoint{4.616267in}{3.365333in}}%
\pgfpathlineto{\pgfqpoint{4.768000in}{3.209660in}}%
\pgfpathlineto{\pgfqpoint{4.768000in}{3.209660in}}%
\pgfusepath{fill}%
\end{pgfscope}%
\begin{pgfscope}%
\pgfpathrectangle{\pgfqpoint{0.800000in}{0.528000in}}{\pgfqpoint{3.968000in}{3.696000in}}%
\pgfusepath{clip}%
\pgfsetbuttcap%
\pgfsetroundjoin%
\definecolor{currentfill}{rgb}{0.134692,0.658636,0.517649}%
\pgfsetfillcolor{currentfill}%
\pgfsetlinewidth{0.000000pt}%
\definecolor{currentstroke}{rgb}{0.000000,0.000000,0.000000}%
\pgfsetstrokecolor{currentstroke}%
\pgfsetdash{}{0pt}%
\pgfpathmoveto{\pgfqpoint{4.768000in}{3.215841in}}%
\pgfpathlineto{\pgfqpoint{4.447354in}{3.542176in}}%
\pgfpathlineto{\pgfqpoint{4.111473in}{3.873810in}}%
\pgfpathlineto{\pgfqpoint{4.006465in}{3.975637in}}%
\pgfpathlineto{\pgfqpoint{3.806061in}{4.167005in}}%
\pgfpathlineto{\pgfqpoint{3.745652in}{4.224000in}}%
\pgfpathlineto{\pgfqpoint{3.742504in}{4.224000in}}%
\pgfpathlineto{\pgfqpoint{3.926303in}{4.049628in}}%
\pgfpathlineto{\pgfqpoint{4.126707in}{3.856160in}}%
\pgfpathlineto{\pgfqpoint{4.322191in}{3.664000in}}%
\pgfpathlineto{\pgfqpoint{4.487434in}{3.498856in}}%
\pgfpathlineto{\pgfqpoint{4.768000in}{3.212751in}}%
\pgfpathlineto{\pgfqpoint{4.768000in}{3.212751in}}%
\pgfusepath{fill}%
\end{pgfscope}%
\begin{pgfscope}%
\pgfpathrectangle{\pgfqpoint{0.800000in}{0.528000in}}{\pgfqpoint{3.968000in}{3.696000in}}%
\pgfusepath{clip}%
\pgfsetbuttcap%
\pgfsetroundjoin%
\definecolor{currentfill}{rgb}{0.137339,0.662252,0.515571}%
\pgfsetfillcolor{currentfill}%
\pgfsetlinewidth{0.000000pt}%
\definecolor{currentstroke}{rgb}{0.000000,0.000000,0.000000}%
\pgfsetstrokecolor{currentstroke}%
\pgfsetdash{}{0pt}%
\pgfpathmoveto{\pgfqpoint{4.768000in}{3.218898in}}%
\pgfpathlineto{\pgfqpoint{4.440579in}{3.552000in}}%
\pgfpathlineto{\pgfqpoint{4.246949in}{3.744376in}}%
\pgfpathlineto{\pgfqpoint{4.046545in}{3.939926in}}%
\pgfpathlineto{\pgfqpoint{3.846141in}{4.131977in}}%
\pgfpathlineto{\pgfqpoint{3.748801in}{4.224000in}}%
\pgfpathlineto{\pgfqpoint{3.745652in}{4.224000in}}%
\pgfpathlineto{\pgfqpoint{3.926303in}{4.052601in}}%
\pgfpathlineto{\pgfqpoint{4.126707in}{3.859144in}}%
\pgfpathlineto{\pgfqpoint{4.327111in}{3.662152in}}%
\pgfpathlineto{\pgfqpoint{4.658822in}{3.328000in}}%
\pgfpathlineto{\pgfqpoint{4.768000in}{3.215841in}}%
\pgfpathlineto{\pgfqpoint{4.768000in}{3.216000in}}%
\pgfusepath{fill}%
\end{pgfscope}%
\begin{pgfscope}%
\pgfpathrectangle{\pgfqpoint{0.800000in}{0.528000in}}{\pgfqpoint{3.968000in}{3.696000in}}%
\pgfusepath{clip}%
\pgfsetbuttcap%
\pgfsetroundjoin%
\definecolor{currentfill}{rgb}{0.137339,0.662252,0.515571}%
\pgfsetfillcolor{currentfill}%
\pgfsetlinewidth{0.000000pt}%
\definecolor{currentstroke}{rgb}{0.000000,0.000000,0.000000}%
\pgfsetstrokecolor{currentstroke}%
\pgfsetdash{}{0pt}%
\pgfpathmoveto{\pgfqpoint{4.768000in}{3.221954in}}%
\pgfpathlineto{\pgfqpoint{4.443610in}{3.552000in}}%
\pgfpathlineto{\pgfqpoint{4.246949in}{3.747367in}}%
\pgfpathlineto{\pgfqpoint{4.046545in}{3.942906in}}%
\pgfpathlineto{\pgfqpoint{3.846141in}{4.134946in}}%
\pgfpathlineto{\pgfqpoint{3.751950in}{4.224000in}}%
\pgfpathlineto{\pgfqpoint{3.748801in}{4.224000in}}%
\pgfpathlineto{\pgfqpoint{3.926303in}{4.055574in}}%
\pgfpathlineto{\pgfqpoint{4.126707in}{3.862128in}}%
\pgfpathlineto{\pgfqpoint{4.328288in}{3.664000in}}%
\pgfpathlineto{\pgfqpoint{4.661803in}{3.328000in}}%
\pgfpathlineto{\pgfqpoint{4.768000in}{3.218898in}}%
\pgfpathlineto{\pgfqpoint{4.768000in}{3.218898in}}%
\pgfusepath{fill}%
\end{pgfscope}%
\begin{pgfscope}%
\pgfpathrectangle{\pgfqpoint{0.800000in}{0.528000in}}{\pgfqpoint{3.968000in}{3.696000in}}%
\pgfusepath{clip}%
\pgfsetbuttcap%
\pgfsetroundjoin%
\definecolor{currentfill}{rgb}{0.137339,0.662252,0.515571}%
\pgfsetfillcolor{currentfill}%
\pgfsetlinewidth{0.000000pt}%
\definecolor{currentstroke}{rgb}{0.000000,0.000000,0.000000}%
\pgfsetstrokecolor{currentstroke}%
\pgfsetdash{}{0pt}%
\pgfpathmoveto{\pgfqpoint{4.768000in}{3.225009in}}%
\pgfpathlineto{\pgfqpoint{4.437116in}{3.561536in}}%
\pgfpathlineto{\pgfqpoint{4.246949in}{3.750358in}}%
\pgfpathlineto{\pgfqpoint{4.046545in}{3.945885in}}%
\pgfpathlineto{\pgfqpoint{3.846141in}{4.137915in}}%
\pgfpathlineto{\pgfqpoint{3.755098in}{4.224000in}}%
\pgfpathlineto{\pgfqpoint{3.751950in}{4.224000in}}%
\pgfpathlineto{\pgfqpoint{3.926303in}{4.058547in}}%
\pgfpathlineto{\pgfqpoint{4.126707in}{3.865112in}}%
\pgfpathlineto{\pgfqpoint{4.327111in}{3.668164in}}%
\pgfpathlineto{\pgfqpoint{4.517820in}{3.477333in}}%
\pgfpathlineto{\pgfqpoint{4.768000in}{3.221954in}}%
\pgfpathlineto{\pgfqpoint{4.768000in}{3.221954in}}%
\pgfusepath{fill}%
\end{pgfscope}%
\begin{pgfscope}%
\pgfpathrectangle{\pgfqpoint{0.800000in}{0.528000in}}{\pgfqpoint{3.968000in}{3.696000in}}%
\pgfusepath{clip}%
\pgfsetbuttcap%
\pgfsetroundjoin%
\definecolor{currentfill}{rgb}{0.137339,0.662252,0.515571}%
\pgfsetfillcolor{currentfill}%
\pgfsetlinewidth{0.000000pt}%
\definecolor{currentstroke}{rgb}{0.000000,0.000000,0.000000}%
\pgfsetstrokecolor{currentstroke}%
\pgfsetdash{}{0pt}%
\pgfpathmoveto{\pgfqpoint{4.768000in}{3.228064in}}%
\pgfpathlineto{\pgfqpoint{4.447354in}{3.554296in}}%
\pgfpathlineto{\pgfqpoint{4.246949in}{3.753349in}}%
\pgfpathlineto{\pgfqpoint{4.046545in}{3.948865in}}%
\pgfpathlineto{\pgfqpoint{3.846141in}{4.140883in}}%
\pgfpathlineto{\pgfqpoint{3.758247in}{4.224000in}}%
\pgfpathlineto{\pgfqpoint{3.755098in}{4.224000in}}%
\pgfpathlineto{\pgfqpoint{3.926303in}{4.061520in}}%
\pgfpathlineto{\pgfqpoint{4.126707in}{3.868096in}}%
\pgfpathlineto{\pgfqpoint{4.327111in}{3.671159in}}%
\pgfpathlineto{\pgfqpoint{4.520834in}{3.477333in}}%
\pgfpathlineto{\pgfqpoint{4.768000in}{3.225009in}}%
\pgfpathlineto{\pgfqpoint{4.768000in}{3.225009in}}%
\pgfusepath{fill}%
\end{pgfscope}%
\begin{pgfscope}%
\pgfpathrectangle{\pgfqpoint{0.800000in}{0.528000in}}{\pgfqpoint{3.968000in}{3.696000in}}%
\pgfusepath{clip}%
\pgfsetbuttcap%
\pgfsetroundjoin%
\definecolor{currentfill}{rgb}{0.140210,0.665859,0.513427}%
\pgfsetfillcolor{currentfill}%
\pgfsetlinewidth{0.000000pt}%
\definecolor{currentstroke}{rgb}{0.000000,0.000000,0.000000}%
\pgfsetstrokecolor{currentstroke}%
\pgfsetdash{}{0pt}%
\pgfpathmoveto{\pgfqpoint{4.768000in}{3.231119in}}%
\pgfpathlineto{\pgfqpoint{4.447354in}{3.557298in}}%
\pgfpathlineto{\pgfqpoint{4.246949in}{3.756340in}}%
\pgfpathlineto{\pgfqpoint{4.046545in}{3.951845in}}%
\pgfpathlineto{\pgfqpoint{3.846141in}{4.143852in}}%
\pgfpathlineto{\pgfqpoint{3.761396in}{4.224000in}}%
\pgfpathlineto{\pgfqpoint{3.758247in}{4.224000in}}%
\pgfpathlineto{\pgfqpoint{3.926303in}{4.064493in}}%
\pgfpathlineto{\pgfqpoint{4.126707in}{3.871081in}}%
\pgfpathlineto{\pgfqpoint{4.327111in}{3.674154in}}%
\pgfpathlineto{\pgfqpoint{4.523848in}{3.477333in}}%
\pgfpathlineto{\pgfqpoint{4.768000in}{3.228064in}}%
\pgfpathlineto{\pgfqpoint{4.768000in}{3.228064in}}%
\pgfusepath{fill}%
\end{pgfscope}%
\begin{pgfscope}%
\pgfpathrectangle{\pgfqpoint{0.800000in}{0.528000in}}{\pgfqpoint{3.968000in}{3.696000in}}%
\pgfusepath{clip}%
\pgfsetbuttcap%
\pgfsetroundjoin%
\definecolor{currentfill}{rgb}{0.140210,0.665859,0.513427}%
\pgfsetfillcolor{currentfill}%
\pgfsetlinewidth{0.000000pt}%
\definecolor{currentstroke}{rgb}{0.000000,0.000000,0.000000}%
\pgfsetstrokecolor{currentstroke}%
\pgfsetdash{}{0pt}%
\pgfpathmoveto{\pgfqpoint{4.768000in}{3.234174in}}%
\pgfpathlineto{\pgfqpoint{4.447354in}{3.560300in}}%
\pgfpathlineto{\pgfqpoint{4.246949in}{3.759331in}}%
\pgfpathlineto{\pgfqpoint{4.046545in}{3.954825in}}%
\pgfpathlineto{\pgfqpoint{3.844798in}{4.148082in}}%
\pgfpathlineto{\pgfqpoint{3.764544in}{4.224000in}}%
\pgfpathlineto{\pgfqpoint{3.761396in}{4.224000in}}%
\pgfpathlineto{\pgfqpoint{3.926303in}{4.067466in}}%
\pgfpathlineto{\pgfqpoint{4.126707in}{3.874065in}}%
\pgfpathlineto{\pgfqpoint{4.327111in}{3.677150in}}%
\pgfpathlineto{\pgfqpoint{4.527515in}{3.476674in}}%
\pgfpathlineto{\pgfqpoint{4.768000in}{3.231119in}}%
\pgfpathlineto{\pgfqpoint{4.768000in}{3.231119in}}%
\pgfusepath{fill}%
\end{pgfscope}%
\begin{pgfscope}%
\pgfpathrectangle{\pgfqpoint{0.800000in}{0.528000in}}{\pgfqpoint{3.968000in}{3.696000in}}%
\pgfusepath{clip}%
\pgfsetbuttcap%
\pgfsetroundjoin%
\definecolor{currentfill}{rgb}{0.140210,0.665859,0.513427}%
\pgfsetfillcolor{currentfill}%
\pgfsetlinewidth{0.000000pt}%
\definecolor{currentstroke}{rgb}{0.000000,0.000000,0.000000}%
\pgfsetstrokecolor{currentstroke}%
\pgfsetdash{}{0pt}%
\pgfpathmoveto{\pgfqpoint{4.768000in}{3.237229in}}%
\pgfpathlineto{\pgfqpoint{4.447354in}{3.563302in}}%
\pgfpathlineto{\pgfqpoint{4.246949in}{3.762322in}}%
\pgfpathlineto{\pgfqpoint{4.046545in}{3.957804in}}%
\pgfpathlineto{\pgfqpoint{3.846141in}{4.149784in}}%
\pgfpathlineto{\pgfqpoint{3.767670in}{4.224000in}}%
\pgfpathlineto{\pgfqpoint{3.764544in}{4.224000in}}%
\pgfpathlineto{\pgfqpoint{3.926303in}{4.070440in}}%
\pgfpathlineto{\pgfqpoint{4.126707in}{3.877049in}}%
\pgfpathlineto{\pgfqpoint{4.327111in}{3.680145in}}%
\pgfpathlineto{\pgfqpoint{4.529846in}{3.477333in}}%
\pgfpathlineto{\pgfqpoint{4.768000in}{3.234174in}}%
\pgfpathlineto{\pgfqpoint{4.768000in}{3.234174in}}%
\pgfusepath{fill}%
\end{pgfscope}%
\begin{pgfscope}%
\pgfpathrectangle{\pgfqpoint{0.800000in}{0.528000in}}{\pgfqpoint{3.968000in}{3.696000in}}%
\pgfusepath{clip}%
\pgfsetbuttcap%
\pgfsetroundjoin%
\definecolor{currentfill}{rgb}{0.140210,0.665859,0.513427}%
\pgfsetfillcolor{currentfill}%
\pgfsetlinewidth{0.000000pt}%
\definecolor{currentstroke}{rgb}{0.000000,0.000000,0.000000}%
\pgfsetstrokecolor{currentstroke}%
\pgfsetdash{}{0pt}%
\pgfpathmoveto{\pgfqpoint{4.768000in}{3.240284in}}%
\pgfpathlineto{\pgfqpoint{4.447354in}{3.566304in}}%
\pgfpathlineto{\pgfqpoint{4.246949in}{3.765313in}}%
\pgfpathlineto{\pgfqpoint{4.044596in}{3.962667in}}%
\pgfpathlineto{\pgfqpoint{3.770778in}{4.224000in}}%
\pgfpathlineto{\pgfqpoint{3.767670in}{4.224000in}}%
\pgfpathlineto{\pgfqpoint{3.966384in}{4.035016in}}%
\pgfpathlineto{\pgfqpoint{4.166788in}{3.840937in}}%
\pgfpathlineto{\pgfqpoint{4.510941in}{3.499229in}}%
\pgfpathlineto{\pgfqpoint{4.607677in}{3.401499in}}%
\pgfpathlineto{\pgfqpoint{4.768000in}{3.237229in}}%
\pgfpathlineto{\pgfqpoint{4.768000in}{3.237229in}}%
\pgfusepath{fill}%
\end{pgfscope}%
\begin{pgfscope}%
\pgfpathrectangle{\pgfqpoint{0.800000in}{0.528000in}}{\pgfqpoint{3.968000in}{3.696000in}}%
\pgfusepath{clip}%
\pgfsetbuttcap%
\pgfsetroundjoin%
\definecolor{currentfill}{rgb}{0.143303,0.669459,0.511215}%
\pgfsetfillcolor{currentfill}%
\pgfsetlinewidth{0.000000pt}%
\definecolor{currentstroke}{rgb}{0.000000,0.000000,0.000000}%
\pgfsetstrokecolor{currentstroke}%
\pgfsetdash{}{0pt}%
\pgfpathmoveto{\pgfqpoint{4.768000in}{3.243340in}}%
\pgfpathlineto{\pgfqpoint{4.447354in}{3.569306in}}%
\pgfpathlineto{\pgfqpoint{4.246949in}{3.768304in}}%
\pgfpathlineto{\pgfqpoint{4.046545in}{3.963752in}}%
\pgfpathlineto{\pgfqpoint{3.773885in}{4.224000in}}%
\pgfpathlineto{\pgfqpoint{3.770778in}{4.224000in}}%
\pgfpathlineto{\pgfqpoint{3.967061in}{4.037333in}}%
\pgfpathlineto{\pgfqpoint{4.166788in}{3.843923in}}%
\pgfpathlineto{\pgfqpoint{4.512500in}{3.500681in}}%
\pgfpathlineto{\pgfqpoint{4.609502in}{3.402667in}}%
\pgfpathlineto{\pgfqpoint{4.768000in}{3.240284in}}%
\pgfpathlineto{\pgfqpoint{4.768000in}{3.240284in}}%
\pgfusepath{fill}%
\end{pgfscope}%
\begin{pgfscope}%
\pgfpathrectangle{\pgfqpoint{0.800000in}{0.528000in}}{\pgfqpoint{3.968000in}{3.696000in}}%
\pgfusepath{clip}%
\pgfsetbuttcap%
\pgfsetroundjoin%
\definecolor{currentfill}{rgb}{0.143303,0.669459,0.511215}%
\pgfsetfillcolor{currentfill}%
\pgfsetlinewidth{0.000000pt}%
\definecolor{currentstroke}{rgb}{0.000000,0.000000,0.000000}%
\pgfsetstrokecolor{currentstroke}%
\pgfsetdash{}{0pt}%
\pgfpathmoveto{\pgfqpoint{4.768000in}{3.246395in}}%
\pgfpathlineto{\pgfqpoint{4.447354in}{3.572308in}}%
\pgfpathlineto{\pgfqpoint{4.246949in}{3.771294in}}%
\pgfpathlineto{\pgfqpoint{4.050714in}{3.962667in}}%
\pgfpathlineto{\pgfqpoint{3.776993in}{4.224000in}}%
\pgfpathlineto{\pgfqpoint{3.773885in}{4.224000in}}%
\pgfpathlineto{\pgfqpoint{3.970125in}{4.037333in}}%
\pgfpathlineto{\pgfqpoint{4.166788in}{3.846910in}}%
\pgfpathlineto{\pgfqpoint{4.514059in}{3.502133in}}%
\pgfpathlineto{\pgfqpoint{4.612462in}{3.402667in}}%
\pgfpathlineto{\pgfqpoint{4.768000in}{3.243340in}}%
\pgfpathlineto{\pgfqpoint{4.768000in}{3.243340in}}%
\pgfusepath{fill}%
\end{pgfscope}%
\begin{pgfscope}%
\pgfpathrectangle{\pgfqpoint{0.800000in}{0.528000in}}{\pgfqpoint{3.968000in}{3.696000in}}%
\pgfusepath{clip}%
\pgfsetbuttcap%
\pgfsetroundjoin%
\definecolor{currentfill}{rgb}{0.143303,0.669459,0.511215}%
\pgfsetfillcolor{currentfill}%
\pgfsetlinewidth{0.000000pt}%
\definecolor{currentstroke}{rgb}{0.000000,0.000000,0.000000}%
\pgfsetstrokecolor{currentstroke}%
\pgfsetdash{}{0pt}%
\pgfpathmoveto{\pgfqpoint{4.768000in}{3.249450in}}%
\pgfpathlineto{\pgfqpoint{4.447354in}{3.575310in}}%
\pgfpathlineto{\pgfqpoint{4.245205in}{3.776000in}}%
\pgfpathlineto{\pgfqpoint{3.886222in}{4.123426in}}%
\pgfpathlineto{\pgfqpoint{3.780100in}{4.224000in}}%
\pgfpathlineto{\pgfqpoint{3.776993in}{4.224000in}}%
\pgfpathlineto{\pgfqpoint{3.973189in}{4.037333in}}%
\pgfpathlineto{\pgfqpoint{4.166788in}{3.849896in}}%
\pgfpathlineto{\pgfqpoint{4.515617in}{3.503584in}}%
\pgfpathlineto{\pgfqpoint{4.615422in}{3.402667in}}%
\pgfpathlineto{\pgfqpoint{4.768000in}{3.246395in}}%
\pgfpathlineto{\pgfqpoint{4.768000in}{3.246395in}}%
\pgfusepath{fill}%
\end{pgfscope}%
\begin{pgfscope}%
\pgfpathrectangle{\pgfqpoint{0.800000in}{0.528000in}}{\pgfqpoint{3.968000in}{3.696000in}}%
\pgfusepath{clip}%
\pgfsetbuttcap%
\pgfsetroundjoin%
\definecolor{currentfill}{rgb}{0.146616,0.673050,0.508936}%
\pgfsetfillcolor{currentfill}%
\pgfsetlinewidth{0.000000pt}%
\definecolor{currentstroke}{rgb}{0.000000,0.000000,0.000000}%
\pgfsetstrokecolor{currentstroke}%
\pgfsetdash{}{0pt}%
\pgfpathmoveto{\pgfqpoint{4.768000in}{3.252505in}}%
\pgfpathlineto{\pgfqpoint{4.447354in}{3.578313in}}%
\pgfpathlineto{\pgfqpoint{4.246949in}{3.777262in}}%
\pgfpathlineto{\pgfqpoint{3.886222in}{4.126364in}}%
\pgfpathlineto{\pgfqpoint{3.783208in}{4.224000in}}%
\pgfpathlineto{\pgfqpoint{3.780100in}{4.224000in}}%
\pgfpathlineto{\pgfqpoint{3.976252in}{4.037333in}}%
\pgfpathlineto{\pgfqpoint{4.169030in}{3.850667in}}%
\pgfpathlineto{\pgfqpoint{4.517176in}{3.505036in}}%
\pgfpathlineto{\pgfqpoint{4.618382in}{3.402667in}}%
\pgfpathlineto{\pgfqpoint{4.768000in}{3.249450in}}%
\pgfpathlineto{\pgfqpoint{4.768000in}{3.249450in}}%
\pgfusepath{fill}%
\end{pgfscope}%
\begin{pgfscope}%
\pgfpathrectangle{\pgfqpoint{0.800000in}{0.528000in}}{\pgfqpoint{3.968000in}{3.696000in}}%
\pgfusepath{clip}%
\pgfsetbuttcap%
\pgfsetroundjoin%
\definecolor{currentfill}{rgb}{0.146616,0.673050,0.508936}%
\pgfsetfillcolor{currentfill}%
\pgfsetlinewidth{0.000000pt}%
\definecolor{currentstroke}{rgb}{0.000000,0.000000,0.000000}%
\pgfsetstrokecolor{currentstroke}%
\pgfsetdash{}{0pt}%
\pgfpathmoveto{\pgfqpoint{4.768000in}{3.255535in}}%
\pgfpathlineto{\pgfqpoint{4.567596in}{3.460229in}}%
\pgfpathlineto{\pgfqpoint{4.229552in}{3.797129in}}%
\pgfpathlineto{\pgfqpoint{4.126707in}{3.897828in}}%
\pgfpathlineto{\pgfqpoint{3.926303in}{4.091067in}}%
\pgfpathlineto{\pgfqpoint{3.786315in}{4.224000in}}%
\pgfpathlineto{\pgfqpoint{3.783208in}{4.224000in}}%
\pgfpathlineto{\pgfqpoint{3.972915in}{4.043417in}}%
\pgfpathlineto{\pgfqpoint{4.086626in}{3.933803in}}%
\pgfpathlineto{\pgfqpoint{4.407273in}{3.618390in}}%
\pgfpathlineto{\pgfqpoint{4.607677in}{3.416568in}}%
\pgfpathlineto{\pgfqpoint{4.768000in}{3.252505in}}%
\pgfpathlineto{\pgfqpoint{4.768000in}{3.253333in}}%
\pgfusepath{fill}%
\end{pgfscope}%
\begin{pgfscope}%
\pgfpathrectangle{\pgfqpoint{0.800000in}{0.528000in}}{\pgfqpoint{3.968000in}{3.696000in}}%
\pgfusepath{clip}%
\pgfsetbuttcap%
\pgfsetroundjoin%
\definecolor{currentfill}{rgb}{0.146616,0.673050,0.508936}%
\pgfsetfillcolor{currentfill}%
\pgfsetlinewidth{0.000000pt}%
\definecolor{currentstroke}{rgb}{0.000000,0.000000,0.000000}%
\pgfsetstrokecolor{currentstroke}%
\pgfsetdash{}{0pt}%
\pgfpathmoveto{\pgfqpoint{4.768000in}{3.258555in}}%
\pgfpathlineto{\pgfqpoint{4.567596in}{3.463238in}}%
\pgfpathlineto{\pgfqpoint{4.231106in}{3.798576in}}%
\pgfpathlineto{\pgfqpoint{4.126707in}{3.900779in}}%
\pgfpathlineto{\pgfqpoint{3.926303in}{4.094007in}}%
\pgfpathlineto{\pgfqpoint{3.789423in}{4.224000in}}%
\pgfpathlineto{\pgfqpoint{3.786315in}{4.224000in}}%
\pgfpathlineto{\pgfqpoint{3.974463in}{4.044858in}}%
\pgfpathlineto{\pgfqpoint{4.086626in}{3.936752in}}%
\pgfpathlineto{\pgfqpoint{4.407273in}{3.621390in}}%
\pgfpathlineto{\pgfqpoint{4.607677in}{3.419580in}}%
\pgfpathlineto{\pgfqpoint{4.768000in}{3.255535in}}%
\pgfpathlineto{\pgfqpoint{4.768000in}{3.255535in}}%
\pgfusepath{fill}%
\end{pgfscope}%
\begin{pgfscope}%
\pgfpathrectangle{\pgfqpoint{0.800000in}{0.528000in}}{\pgfqpoint{3.968000in}{3.696000in}}%
\pgfusepath{clip}%
\pgfsetbuttcap%
\pgfsetroundjoin%
\definecolor{currentfill}{rgb}{0.146616,0.673050,0.508936}%
\pgfsetfillcolor{currentfill}%
\pgfsetlinewidth{0.000000pt}%
\definecolor{currentstroke}{rgb}{0.000000,0.000000,0.000000}%
\pgfsetstrokecolor{currentstroke}%
\pgfsetdash{}{0pt}%
\pgfpathmoveto{\pgfqpoint{4.768000in}{3.261575in}}%
\pgfpathlineto{\pgfqpoint{4.567596in}{3.466247in}}%
\pgfpathlineto{\pgfqpoint{4.232660in}{3.800023in}}%
\pgfpathlineto{\pgfqpoint{4.126707in}{3.903729in}}%
\pgfpathlineto{\pgfqpoint{3.926303in}{4.096947in}}%
\pgfpathlineto{\pgfqpoint{3.792530in}{4.224000in}}%
\pgfpathlineto{\pgfqpoint{3.789423in}{4.224000in}}%
\pgfpathlineto{\pgfqpoint{3.966384in}{4.055637in}}%
\pgfpathlineto{\pgfqpoint{4.309004in}{3.721801in}}%
\pgfpathlineto{\pgfqpoint{4.407273in}{3.624390in}}%
\pgfpathlineto{\pgfqpoint{4.607677in}{3.422591in}}%
\pgfpathlineto{\pgfqpoint{4.768000in}{3.258555in}}%
\pgfpathlineto{\pgfqpoint{4.768000in}{3.258555in}}%
\pgfusepath{fill}%
\end{pgfscope}%
\begin{pgfscope}%
\pgfpathrectangle{\pgfqpoint{0.800000in}{0.528000in}}{\pgfqpoint{3.968000in}{3.696000in}}%
\pgfusepath{clip}%
\pgfsetbuttcap%
\pgfsetroundjoin%
\definecolor{currentfill}{rgb}{0.150148,0.676631,0.506589}%
\pgfsetfillcolor{currentfill}%
\pgfsetlinewidth{0.000000pt}%
\definecolor{currentstroke}{rgb}{0.000000,0.000000,0.000000}%
\pgfsetstrokecolor{currentstroke}%
\pgfsetdash{}{0pt}%
\pgfpathmoveto{\pgfqpoint{4.768000in}{3.264596in}}%
\pgfpathlineto{\pgfqpoint{4.567596in}{3.469256in}}%
\pgfpathlineto{\pgfqpoint{4.234214in}{3.801471in}}%
\pgfpathlineto{\pgfqpoint{4.126707in}{3.906680in}}%
\pgfpathlineto{\pgfqpoint{3.926303in}{4.099887in}}%
\pgfpathlineto{\pgfqpoint{3.795638in}{4.224000in}}%
\pgfpathlineto{\pgfqpoint{3.792530in}{4.224000in}}%
\pgfpathlineto{\pgfqpoint{3.966384in}{4.058580in}}%
\pgfpathlineto{\pgfqpoint{4.310555in}{3.723245in}}%
\pgfpathlineto{\pgfqpoint{4.417866in}{3.616799in}}%
\pgfpathlineto{\pgfqpoint{4.607677in}{3.425602in}}%
\pgfpathlineto{\pgfqpoint{4.768000in}{3.261575in}}%
\pgfpathlineto{\pgfqpoint{4.768000in}{3.261575in}}%
\pgfusepath{fill}%
\end{pgfscope}%
\begin{pgfscope}%
\pgfpathrectangle{\pgfqpoint{0.800000in}{0.528000in}}{\pgfqpoint{3.968000in}{3.696000in}}%
\pgfusepath{clip}%
\pgfsetbuttcap%
\pgfsetroundjoin%
\definecolor{currentfill}{rgb}{0.150148,0.676631,0.506589}%
\pgfsetfillcolor{currentfill}%
\pgfsetlinewidth{0.000000pt}%
\definecolor{currentstroke}{rgb}{0.000000,0.000000,0.000000}%
\pgfsetstrokecolor{currentstroke}%
\pgfsetdash{}{0pt}%
\pgfpathmoveto{\pgfqpoint{4.768000in}{3.267616in}}%
\pgfpathlineto{\pgfqpoint{4.564973in}{3.474890in}}%
\pgfpathlineto{\pgfqpoint{4.487434in}{3.553101in}}%
\pgfpathlineto{\pgfqpoint{4.287030in}{3.752576in}}%
\pgfpathlineto{\pgfqpoint{3.941139in}{4.088486in}}%
\pgfpathlineto{\pgfqpoint{3.838199in}{4.186667in}}%
\pgfpathlineto{\pgfqpoint{3.798745in}{4.224000in}}%
\pgfpathlineto{\pgfqpoint{3.795638in}{4.224000in}}%
\pgfpathlineto{\pgfqpoint{3.966384in}{4.061522in}}%
\pgfpathlineto{\pgfqpoint{4.312105in}{3.724689in}}%
\pgfpathlineto{\pgfqpoint{4.410960in}{3.626667in}}%
\pgfpathlineto{\pgfqpoint{4.607677in}{3.428613in}}%
\pgfpathlineto{\pgfqpoint{4.768000in}{3.264596in}}%
\pgfpathlineto{\pgfqpoint{4.768000in}{3.264596in}}%
\pgfusepath{fill}%
\end{pgfscope}%
\begin{pgfscope}%
\pgfpathrectangle{\pgfqpoint{0.800000in}{0.528000in}}{\pgfqpoint{3.968000in}{3.696000in}}%
\pgfusepath{clip}%
\pgfsetbuttcap%
\pgfsetroundjoin%
\definecolor{currentfill}{rgb}{0.150148,0.676631,0.506589}%
\pgfsetfillcolor{currentfill}%
\pgfsetlinewidth{0.000000pt}%
\definecolor{currentstroke}{rgb}{0.000000,0.000000,0.000000}%
\pgfsetstrokecolor{currentstroke}%
\pgfsetdash{}{0pt}%
\pgfpathmoveto{\pgfqpoint{4.768000in}{3.270636in}}%
\pgfpathlineto{\pgfqpoint{4.565559in}{3.477333in}}%
\pgfpathlineto{\pgfqpoint{4.367192in}{3.676173in}}%
\pgfpathlineto{\pgfqpoint{4.166788in}{3.873530in}}%
\pgfpathlineto{\pgfqpoint{3.966384in}{4.067406in}}%
\pgfpathlineto{\pgfqpoint{3.801852in}{4.224000in}}%
\pgfpathlineto{\pgfqpoint{3.798745in}{4.224000in}}%
\pgfpathlineto{\pgfqpoint{3.966384in}{4.064464in}}%
\pgfpathlineto{\pgfqpoint{4.313656in}{3.726134in}}%
\pgfpathlineto{\pgfqpoint{4.413932in}{3.626667in}}%
\pgfpathlineto{\pgfqpoint{4.607677in}{3.431624in}}%
\pgfpathlineto{\pgfqpoint{4.768000in}{3.267616in}}%
\pgfpathlineto{\pgfqpoint{4.768000in}{3.267616in}}%
\pgfusepath{fill}%
\end{pgfscope}%
\begin{pgfscope}%
\pgfpathrectangle{\pgfqpoint{0.800000in}{0.528000in}}{\pgfqpoint{3.968000in}{3.696000in}}%
\pgfusepath{clip}%
\pgfsetbuttcap%
\pgfsetroundjoin%
\definecolor{currentfill}{rgb}{0.150148,0.676631,0.506589}%
\pgfsetfillcolor{currentfill}%
\pgfsetlinewidth{0.000000pt}%
\definecolor{currentstroke}{rgb}{0.000000,0.000000,0.000000}%
\pgfsetstrokecolor{currentstroke}%
\pgfsetdash{}{0pt}%
\pgfpathmoveto{\pgfqpoint{4.768000in}{3.273656in}}%
\pgfpathlineto{\pgfqpoint{4.567596in}{3.478272in}}%
\pgfpathlineto{\pgfqpoint{4.367192in}{3.679137in}}%
\pgfpathlineto{\pgfqpoint{4.166788in}{3.876483in}}%
\pgfpathlineto{\pgfqpoint{3.966384in}{4.070348in}}%
\pgfpathlineto{\pgfqpoint{3.804960in}{4.224000in}}%
\pgfpathlineto{\pgfqpoint{3.801852in}{4.224000in}}%
\pgfpathlineto{\pgfqpoint{3.966384in}{4.067406in}}%
\pgfpathlineto{\pgfqpoint{4.315206in}{3.727578in}}%
\pgfpathlineto{\pgfqpoint{4.416903in}{3.626667in}}%
\pgfpathlineto{\pgfqpoint{4.607677in}{3.434635in}}%
\pgfpathlineto{\pgfqpoint{4.768000in}{3.270636in}}%
\pgfpathlineto{\pgfqpoint{4.768000in}{3.270636in}}%
\pgfusepath{fill}%
\end{pgfscope}%
\begin{pgfscope}%
\pgfpathrectangle{\pgfqpoint{0.800000in}{0.528000in}}{\pgfqpoint{3.968000in}{3.696000in}}%
\pgfusepath{clip}%
\pgfsetbuttcap%
\pgfsetroundjoin%
\definecolor{currentfill}{rgb}{0.153894,0.680203,0.504172}%
\pgfsetfillcolor{currentfill}%
\pgfsetlinewidth{0.000000pt}%
\definecolor{currentstroke}{rgb}{0.000000,0.000000,0.000000}%
\pgfsetstrokecolor{currentstroke}%
\pgfsetdash{}{0pt}%
\pgfpathmoveto{\pgfqpoint{4.768000in}{3.276677in}}%
\pgfpathlineto{\pgfqpoint{4.569599in}{3.479199in}}%
\pgfpathlineto{\pgfqpoint{4.487434in}{3.562013in}}%
\pgfpathlineto{\pgfqpoint{4.287030in}{3.761455in}}%
\pgfpathlineto{\pgfqpoint{3.945786in}{4.092814in}}%
\pgfpathlineto{\pgfqpoint{3.846141in}{4.187935in}}%
\pgfpathlineto{\pgfqpoint{3.806061in}{4.224000in}}%
\pgfpathlineto{\pgfqpoint{3.804960in}{4.224000in}}%
\pgfpathlineto{\pgfqpoint{3.806061in}{4.222961in}}%
\pgfpathlineto{\pgfqpoint{4.166788in}{3.876483in}}%
\pgfpathlineto{\pgfqpoint{4.494441in}{3.552000in}}%
\pgfpathlineto{\pgfqpoint{4.687838in}{3.355941in}}%
\pgfpathlineto{\pgfqpoint{4.768000in}{3.273656in}}%
\pgfpathlineto{\pgfqpoint{4.768000in}{3.273656in}}%
\pgfusepath{fill}%
\end{pgfscope}%
\begin{pgfscope}%
\pgfpathrectangle{\pgfqpoint{0.800000in}{0.528000in}}{\pgfqpoint{3.968000in}{3.696000in}}%
\pgfusepath{clip}%
\pgfsetbuttcap%
\pgfsetroundjoin%
\definecolor{currentfill}{rgb}{0.153894,0.680203,0.504172}%
\pgfsetfillcolor{currentfill}%
\pgfsetlinewidth{0.000000pt}%
\definecolor{currentstroke}{rgb}{0.000000,0.000000,0.000000}%
\pgfsetstrokecolor{currentstroke}%
\pgfsetdash{}{0pt}%
\pgfpathmoveto{\pgfqpoint{4.768000in}{3.279697in}}%
\pgfpathlineto{\pgfqpoint{4.574402in}{3.477333in}}%
\pgfpathlineto{\pgfqpoint{4.246949in}{3.803879in}}%
\pgfpathlineto{\pgfqpoint{3.907759in}{4.132060in}}%
\pgfpathlineto{\pgfqpoint{3.811109in}{4.224000in}}%
\pgfpathlineto{\pgfqpoint{3.808042in}{4.224000in}}%
\pgfpathlineto{\pgfqpoint{4.166788in}{3.879436in}}%
\pgfpathlineto{\pgfqpoint{4.497396in}{3.552000in}}%
\pgfpathlineto{\pgfqpoint{4.687838in}{3.358957in}}%
\pgfpathlineto{\pgfqpoint{4.768000in}{3.276677in}}%
\pgfpathlineto{\pgfqpoint{4.768000in}{3.276677in}}%
\pgfusepath{fill}%
\end{pgfscope}%
\begin{pgfscope}%
\pgfpathrectangle{\pgfqpoint{0.800000in}{0.528000in}}{\pgfqpoint{3.968000in}{3.696000in}}%
\pgfusepath{clip}%
\pgfsetbuttcap%
\pgfsetroundjoin%
\definecolor{currentfill}{rgb}{0.153894,0.680203,0.504172}%
\pgfsetfillcolor{currentfill}%
\pgfsetlinewidth{0.000000pt}%
\definecolor{currentstroke}{rgb}{0.000000,0.000000,0.000000}%
\pgfsetstrokecolor{currentstroke}%
\pgfsetdash{}{0pt}%
\pgfpathmoveto{\pgfqpoint{4.768000in}{3.282717in}}%
\pgfpathlineto{\pgfqpoint{4.577341in}{3.477333in}}%
\pgfpathlineto{\pgfqpoint{4.240332in}{3.813333in}}%
\pgfpathlineto{\pgfqpoint{4.046545in}{4.002034in}}%
\pgfpathlineto{\pgfqpoint{3.846141in}{4.193742in}}%
\pgfpathlineto{\pgfqpoint{3.814176in}{4.224000in}}%
\pgfpathlineto{\pgfqpoint{3.811109in}{4.224000in}}%
\pgfpathlineto{\pgfqpoint{4.166788in}{3.882389in}}%
\pgfpathlineto{\pgfqpoint{4.500351in}{3.552000in}}%
\pgfpathlineto{\pgfqpoint{4.687838in}{3.361973in}}%
\pgfpathlineto{\pgfqpoint{4.768000in}{3.279697in}}%
\pgfpathlineto{\pgfqpoint{4.768000in}{3.279697in}}%
\pgfusepath{fill}%
\end{pgfscope}%
\begin{pgfscope}%
\pgfpathrectangle{\pgfqpoint{0.800000in}{0.528000in}}{\pgfqpoint{3.968000in}{3.696000in}}%
\pgfusepath{clip}%
\pgfsetbuttcap%
\pgfsetroundjoin%
\definecolor{currentfill}{rgb}{0.153894,0.680203,0.504172}%
\pgfsetfillcolor{currentfill}%
\pgfsetlinewidth{0.000000pt}%
\definecolor{currentstroke}{rgb}{0.000000,0.000000,0.000000}%
\pgfsetstrokecolor{currentstroke}%
\pgfsetdash{}{0pt}%
\pgfpathmoveto{\pgfqpoint{4.768000in}{3.285737in}}%
\pgfpathlineto{\pgfqpoint{4.580280in}{3.477333in}}%
\pgfpathlineto{\pgfqpoint{4.243344in}{3.813333in}}%
\pgfpathlineto{\pgfqpoint{4.046545in}{4.004949in}}%
\pgfpathlineto{\pgfqpoint{3.846141in}{4.196646in}}%
\pgfpathlineto{\pgfqpoint{3.817244in}{4.224000in}}%
\pgfpathlineto{\pgfqpoint{3.814176in}{4.224000in}}%
\pgfpathlineto{\pgfqpoint{4.166788in}{3.885342in}}%
\pgfpathlineto{\pgfqpoint{4.503306in}{3.552000in}}%
\pgfpathlineto{\pgfqpoint{4.687838in}{3.364989in}}%
\pgfpathlineto{\pgfqpoint{4.768000in}{3.282717in}}%
\pgfpathlineto{\pgfqpoint{4.768000in}{3.282717in}}%
\pgfusepath{fill}%
\end{pgfscope}%
\begin{pgfscope}%
\pgfpathrectangle{\pgfqpoint{0.800000in}{0.528000in}}{\pgfqpoint{3.968000in}{3.696000in}}%
\pgfusepath{clip}%
\pgfsetbuttcap%
\pgfsetroundjoin%
\definecolor{currentfill}{rgb}{0.157851,0.683765,0.501686}%
\pgfsetfillcolor{currentfill}%
\pgfsetlinewidth{0.000000pt}%
\definecolor{currentstroke}{rgb}{0.000000,0.000000,0.000000}%
\pgfsetstrokecolor{currentstroke}%
\pgfsetdash{}{0pt}%
\pgfpathmoveto{\pgfqpoint{4.768000in}{3.288758in}}%
\pgfpathlineto{\pgfqpoint{4.583219in}{3.477333in}}%
\pgfpathlineto{\pgfqpoint{4.246357in}{3.813333in}}%
\pgfpathlineto{\pgfqpoint{4.046545in}{4.007863in}}%
\pgfpathlineto{\pgfqpoint{3.846141in}{4.199549in}}%
\pgfpathlineto{\pgfqpoint{3.820311in}{4.224000in}}%
\pgfpathlineto{\pgfqpoint{3.817244in}{4.224000in}}%
\pgfpathlineto{\pgfqpoint{4.167087in}{3.888000in}}%
\pgfpathlineto{\pgfqpoint{4.527515in}{3.530620in}}%
\pgfpathlineto{\pgfqpoint{4.768000in}{3.285737in}}%
\pgfpathlineto{\pgfqpoint{4.768000in}{3.285737in}}%
\pgfusepath{fill}%
\end{pgfscope}%
\begin{pgfscope}%
\pgfpathrectangle{\pgfqpoint{0.800000in}{0.528000in}}{\pgfqpoint{3.968000in}{3.696000in}}%
\pgfusepath{clip}%
\pgfsetbuttcap%
\pgfsetroundjoin%
\definecolor{currentfill}{rgb}{0.157851,0.683765,0.501686}%
\pgfsetfillcolor{currentfill}%
\pgfsetlinewidth{0.000000pt}%
\definecolor{currentstroke}{rgb}{0.000000,0.000000,0.000000}%
\pgfsetstrokecolor{currentstroke}%
\pgfsetdash{}{0pt}%
\pgfpathmoveto{\pgfqpoint{4.768000in}{3.291765in}}%
\pgfpathlineto{\pgfqpoint{4.567596in}{3.496122in}}%
\pgfpathlineto{\pgfqpoint{4.362744in}{3.701333in}}%
\pgfpathlineto{\pgfqpoint{4.166788in}{3.894133in}}%
\pgfpathlineto{\pgfqpoint{3.823378in}{4.224000in}}%
\pgfpathlineto{\pgfqpoint{3.820311in}{4.224000in}}%
\pgfpathlineto{\pgfqpoint{4.170078in}{3.888000in}}%
\pgfpathlineto{\pgfqpoint{4.527515in}{3.533593in}}%
\pgfpathlineto{\pgfqpoint{4.768000in}{3.288758in}}%
\pgfpathlineto{\pgfqpoint{4.768000in}{3.290667in}}%
\pgfpathlineto{\pgfqpoint{4.768000in}{3.290667in}}%
\pgfusepath{fill}%
\end{pgfscope}%
\begin{pgfscope}%
\pgfpathrectangle{\pgfqpoint{0.800000in}{0.528000in}}{\pgfqpoint{3.968000in}{3.696000in}}%
\pgfusepath{clip}%
\pgfsetbuttcap%
\pgfsetroundjoin%
\definecolor{currentfill}{rgb}{0.157851,0.683765,0.501686}%
\pgfsetfillcolor{currentfill}%
\pgfsetlinewidth{0.000000pt}%
\definecolor{currentstroke}{rgb}{0.000000,0.000000,0.000000}%
\pgfsetstrokecolor{currentstroke}%
\pgfsetdash{}{0pt}%
\pgfpathmoveto{\pgfqpoint{4.768000in}{3.294752in}}%
\pgfpathlineto{\pgfqpoint{4.567596in}{3.499097in}}%
\pgfpathlineto{\pgfqpoint{4.365732in}{3.701333in}}%
\pgfpathlineto{\pgfqpoint{4.166788in}{3.897053in}}%
\pgfpathlineto{\pgfqpoint{3.826446in}{4.224000in}}%
\pgfpathlineto{\pgfqpoint{3.823378in}{4.224000in}}%
\pgfpathlineto{\pgfqpoint{4.166788in}{3.894133in}}%
\pgfpathlineto{\pgfqpoint{4.367192in}{3.696921in}}%
\pgfpathlineto{\pgfqpoint{4.696252in}{3.365333in}}%
\pgfpathlineto{\pgfqpoint{4.768000in}{3.291765in}}%
\pgfpathlineto{\pgfqpoint{4.768000in}{3.291765in}}%
\pgfusepath{fill}%
\end{pgfscope}%
\begin{pgfscope}%
\pgfpathrectangle{\pgfqpoint{0.800000in}{0.528000in}}{\pgfqpoint{3.968000in}{3.696000in}}%
\pgfusepath{clip}%
\pgfsetbuttcap%
\pgfsetroundjoin%
\definecolor{currentfill}{rgb}{0.162016,0.687316,0.499129}%
\pgfsetfillcolor{currentfill}%
\pgfsetlinewidth{0.000000pt}%
\definecolor{currentstroke}{rgb}{0.000000,0.000000,0.000000}%
\pgfsetstrokecolor{currentstroke}%
\pgfsetdash{}{0pt}%
\pgfpathmoveto{\pgfqpoint{4.768000in}{3.297738in}}%
\pgfpathlineto{\pgfqpoint{4.567596in}{3.502072in}}%
\pgfpathlineto{\pgfqpoint{4.367192in}{3.702832in}}%
\pgfpathlineto{\pgfqpoint{4.166788in}{3.899974in}}%
\pgfpathlineto{\pgfqpoint{3.829513in}{4.224000in}}%
\pgfpathlineto{\pgfqpoint{3.826446in}{4.224000in}}%
\pgfpathlineto{\pgfqpoint{4.166788in}{3.897053in}}%
\pgfpathlineto{\pgfqpoint{4.367192in}{3.699885in}}%
\pgfpathlineto{\pgfqpoint{4.699167in}{3.365333in}}%
\pgfpathlineto{\pgfqpoint{4.768000in}{3.294752in}}%
\pgfpathlineto{\pgfqpoint{4.768000in}{3.294752in}}%
\pgfusepath{fill}%
\end{pgfscope}%
\begin{pgfscope}%
\pgfpathrectangle{\pgfqpoint{0.800000in}{0.528000in}}{\pgfqpoint{3.968000in}{3.696000in}}%
\pgfusepath{clip}%
\pgfsetbuttcap%
\pgfsetroundjoin%
\definecolor{currentfill}{rgb}{0.162016,0.687316,0.499129}%
\pgfsetfillcolor{currentfill}%
\pgfsetlinewidth{0.000000pt}%
\definecolor{currentstroke}{rgb}{0.000000,0.000000,0.000000}%
\pgfsetstrokecolor{currentstroke}%
\pgfsetdash{}{0pt}%
\pgfpathmoveto{\pgfqpoint{4.768000in}{3.300724in}}%
\pgfpathlineto{\pgfqpoint{4.567596in}{3.505048in}}%
\pgfpathlineto{\pgfqpoint{4.371652in}{3.701333in}}%
\pgfpathlineto{\pgfqpoint{4.206869in}{3.863745in}}%
\pgfpathlineto{\pgfqpoint{3.871947in}{4.186667in}}%
\pgfpathlineto{\pgfqpoint{3.832580in}{4.224000in}}%
\pgfpathlineto{\pgfqpoint{3.829513in}{4.224000in}}%
\pgfpathlineto{\pgfqpoint{4.166788in}{3.899974in}}%
\pgfpathlineto{\pgfqpoint{4.368701in}{3.701333in}}%
\pgfpathlineto{\pgfqpoint{4.702083in}{3.365333in}}%
\pgfpathlineto{\pgfqpoint{4.768000in}{3.297738in}}%
\pgfpathlineto{\pgfqpoint{4.768000in}{3.297738in}}%
\pgfusepath{fill}%
\end{pgfscope}%
\begin{pgfscope}%
\pgfpathrectangle{\pgfqpoint{0.800000in}{0.528000in}}{\pgfqpoint{3.968000in}{3.696000in}}%
\pgfusepath{clip}%
\pgfsetbuttcap%
\pgfsetroundjoin%
\definecolor{currentfill}{rgb}{0.162016,0.687316,0.499129}%
\pgfsetfillcolor{currentfill}%
\pgfsetlinewidth{0.000000pt}%
\definecolor{currentstroke}{rgb}{0.000000,0.000000,0.000000}%
\pgfsetstrokecolor{currentstroke}%
\pgfsetdash{}{0pt}%
\pgfpathmoveto{\pgfqpoint{4.768000in}{3.303710in}}%
\pgfpathlineto{\pgfqpoint{4.567596in}{3.508023in}}%
\pgfpathlineto{\pgfqpoint{4.374602in}{3.701333in}}%
\pgfpathlineto{\pgfqpoint{4.206869in}{3.866667in}}%
\pgfpathlineto{\pgfqpoint{3.875006in}{4.186667in}}%
\pgfpathlineto{\pgfqpoint{3.835648in}{4.224000in}}%
\pgfpathlineto{\pgfqpoint{3.832580in}{4.224000in}}%
\pgfpathlineto{\pgfqpoint{4.166788in}{3.902894in}}%
\pgfpathlineto{\pgfqpoint{4.371652in}{3.701333in}}%
\pgfpathlineto{\pgfqpoint{4.704999in}{3.365333in}}%
\pgfpathlineto{\pgfqpoint{4.768000in}{3.300724in}}%
\pgfpathlineto{\pgfqpoint{4.768000in}{3.300724in}}%
\pgfusepath{fill}%
\end{pgfscope}%
\begin{pgfscope}%
\pgfpathrectangle{\pgfqpoint{0.800000in}{0.528000in}}{\pgfqpoint{3.968000in}{3.696000in}}%
\pgfusepath{clip}%
\pgfsetbuttcap%
\pgfsetroundjoin%
\definecolor{currentfill}{rgb}{0.162016,0.687316,0.499129}%
\pgfsetfillcolor{currentfill}%
\pgfsetlinewidth{0.000000pt}%
\definecolor{currentstroke}{rgb}{0.000000,0.000000,0.000000}%
\pgfsetstrokecolor{currentstroke}%
\pgfsetdash{}{0pt}%
\pgfpathmoveto{\pgfqpoint{4.768000in}{3.306696in}}%
\pgfpathlineto{\pgfqpoint{4.563961in}{3.514667in}}%
\pgfpathlineto{\pgfqpoint{4.206869in}{3.869590in}}%
\pgfpathlineto{\pgfqpoint{3.878065in}{4.186667in}}%
\pgfpathlineto{\pgfqpoint{3.838715in}{4.224000in}}%
\pgfpathlineto{\pgfqpoint{3.835648in}{4.224000in}}%
\pgfpathlineto{\pgfqpoint{4.166788in}{3.905815in}}%
\pgfpathlineto{\pgfqpoint{4.367192in}{3.708695in}}%
\pgfpathlineto{\pgfqpoint{4.561014in}{3.514667in}}%
\pgfpathlineto{\pgfqpoint{4.736488in}{3.335982in}}%
\pgfpathlineto{\pgfqpoint{4.768000in}{3.303710in}}%
\pgfpathlineto{\pgfqpoint{4.768000in}{3.303710in}}%
\pgfusepath{fill}%
\end{pgfscope}%
\begin{pgfscope}%
\pgfpathrectangle{\pgfqpoint{0.800000in}{0.528000in}}{\pgfqpoint{3.968000in}{3.696000in}}%
\pgfusepath{clip}%
\pgfsetbuttcap%
\pgfsetroundjoin%
\definecolor{currentfill}{rgb}{0.166383,0.690856,0.496502}%
\pgfsetfillcolor{currentfill}%
\pgfsetlinewidth{0.000000pt}%
\definecolor{currentstroke}{rgb}{0.000000,0.000000,0.000000}%
\pgfsetstrokecolor{currentstroke}%
\pgfsetdash{}{0pt}%
\pgfpathmoveto{\pgfqpoint{4.768000in}{3.309683in}}%
\pgfpathlineto{\pgfqpoint{4.566909in}{3.514667in}}%
\pgfpathlineto{\pgfqpoint{4.206869in}{3.872513in}}%
\pgfpathlineto{\pgfqpoint{3.881123in}{4.186667in}}%
\pgfpathlineto{\pgfqpoint{3.841782in}{4.224000in}}%
\pgfpathlineto{\pgfqpoint{3.838715in}{4.224000in}}%
\pgfpathlineto{\pgfqpoint{4.166788in}{3.908735in}}%
\pgfpathlineto{\pgfqpoint{4.367192in}{3.711626in}}%
\pgfpathlineto{\pgfqpoint{4.567596in}{3.510998in}}%
\pgfpathlineto{\pgfqpoint{4.768000in}{3.306696in}}%
\pgfpathlineto{\pgfqpoint{4.768000in}{3.306696in}}%
\pgfusepath{fill}%
\end{pgfscope}%
\begin{pgfscope}%
\pgfpathrectangle{\pgfqpoint{0.800000in}{0.528000in}}{\pgfqpoint{3.968000in}{3.696000in}}%
\pgfusepath{clip}%
\pgfsetbuttcap%
\pgfsetroundjoin%
\definecolor{currentfill}{rgb}{0.166383,0.690856,0.496502}%
\pgfsetfillcolor{currentfill}%
\pgfsetlinewidth{0.000000pt}%
\definecolor{currentstroke}{rgb}{0.000000,0.000000,0.000000}%
\pgfsetstrokecolor{currentstroke}%
\pgfsetdash{}{0pt}%
\pgfpathmoveto{\pgfqpoint{4.768000in}{3.312669in}}%
\pgfpathlineto{\pgfqpoint{4.567596in}{3.516922in}}%
\pgfpathlineto{\pgfqpoint{4.232150in}{3.850667in}}%
\pgfpathlineto{\pgfqpoint{4.040154in}{4.037333in}}%
\pgfpathlineto{\pgfqpoint{3.844850in}{4.224000in}}%
\pgfpathlineto{\pgfqpoint{3.841782in}{4.224000in}}%
\pgfpathlineto{\pgfqpoint{4.166788in}{3.911656in}}%
\pgfpathlineto{\pgfqpoint{4.367192in}{3.714557in}}%
\pgfpathlineto{\pgfqpoint{4.567596in}{3.513973in}}%
\pgfpathlineto{\pgfqpoint{4.768000in}{3.309683in}}%
\pgfpathlineto{\pgfqpoint{4.768000in}{3.309683in}}%
\pgfusepath{fill}%
\end{pgfscope}%
\begin{pgfscope}%
\pgfpathrectangle{\pgfqpoint{0.800000in}{0.528000in}}{\pgfqpoint{3.968000in}{3.696000in}}%
\pgfusepath{clip}%
\pgfsetbuttcap%
\pgfsetroundjoin%
\definecolor{currentfill}{rgb}{0.166383,0.690856,0.496502}%
\pgfsetfillcolor{currentfill}%
\pgfsetlinewidth{0.000000pt}%
\definecolor{currentstroke}{rgb}{0.000000,0.000000,0.000000}%
\pgfsetstrokecolor{currentstroke}%
\pgfsetdash{}{0pt}%
\pgfpathmoveto{\pgfqpoint{4.768000in}{3.315655in}}%
\pgfpathlineto{\pgfqpoint{4.570259in}{3.517147in}}%
\pgfpathlineto{\pgfqpoint{4.473829in}{3.613994in}}%
\pgfpathlineto{\pgfqpoint{4.367192in}{3.720420in}}%
\pgfpathlineto{\pgfqpoint{4.166788in}{3.917497in}}%
\pgfpathlineto{\pgfqpoint{3.847895in}{4.224000in}}%
\pgfpathlineto{\pgfqpoint{3.844850in}{4.224000in}}%
\pgfpathlineto{\pgfqpoint{3.845487in}{4.223391in}}%
\pgfpathlineto{\pgfqpoint{3.926303in}{4.146544in}}%
\pgfpathlineto{\pgfqpoint{4.126707in}{3.953580in}}%
\pgfpathlineto{\pgfqpoint{4.472301in}{3.612571in}}%
\pgfpathlineto{\pgfqpoint{4.569828in}{3.514667in}}%
\pgfpathlineto{\pgfqpoint{4.768000in}{3.312669in}}%
\pgfpathlineto{\pgfqpoint{4.768000in}{3.312669in}}%
\pgfusepath{fill}%
\end{pgfscope}%
\begin{pgfscope}%
\pgfpathrectangle{\pgfqpoint{0.800000in}{0.528000in}}{\pgfqpoint{3.968000in}{3.696000in}}%
\pgfusepath{clip}%
\pgfsetbuttcap%
\pgfsetroundjoin%
\definecolor{currentfill}{rgb}{0.166383,0.690856,0.496502}%
\pgfsetfillcolor{currentfill}%
\pgfsetlinewidth{0.000000pt}%
\definecolor{currentstroke}{rgb}{0.000000,0.000000,0.000000}%
\pgfsetstrokecolor{currentstroke}%
\pgfsetdash{}{0pt}%
\pgfpathmoveto{\pgfqpoint{4.768000in}{3.318641in}}%
\pgfpathlineto{\pgfqpoint{4.575650in}{3.514667in}}%
\pgfpathlineto{\pgfqpoint{4.246949in}{3.842006in}}%
\pgfpathlineto{\pgfqpoint{4.046203in}{4.037333in}}%
\pgfpathlineto{\pgfqpoint{3.850923in}{4.224000in}}%
\pgfpathlineto{\pgfqpoint{3.847895in}{4.224000in}}%
\pgfpathlineto{\pgfqpoint{4.046545in}{4.034090in}}%
\pgfpathlineto{\pgfqpoint{4.246949in}{3.839081in}}%
\pgfpathlineto{\pgfqpoint{4.589399in}{3.497641in}}%
\pgfpathlineto{\pgfqpoint{4.687838in}{3.397792in}}%
\pgfpathlineto{\pgfqpoint{4.768000in}{3.315655in}}%
\pgfpathlineto{\pgfqpoint{4.768000in}{3.315655in}}%
\pgfusepath{fill}%
\end{pgfscope}%
\begin{pgfscope}%
\pgfpathrectangle{\pgfqpoint{0.800000in}{0.528000in}}{\pgfqpoint{3.968000in}{3.696000in}}%
\pgfusepath{clip}%
\pgfsetbuttcap%
\pgfsetroundjoin%
\definecolor{currentfill}{rgb}{0.170948,0.694384,0.493803}%
\pgfsetfillcolor{currentfill}%
\pgfsetlinewidth{0.000000pt}%
\definecolor{currentstroke}{rgb}{0.000000,0.000000,0.000000}%
\pgfsetstrokecolor{currentstroke}%
\pgfsetdash{}{0pt}%
\pgfpathmoveto{\pgfqpoint{4.768000in}{3.321627in}}%
\pgfpathlineto{\pgfqpoint{4.578562in}{3.514667in}}%
\pgfpathlineto{\pgfqpoint{4.241099in}{3.850667in}}%
\pgfpathlineto{\pgfqpoint{4.046545in}{4.039890in}}%
\pgfpathlineto{\pgfqpoint{3.853951in}{4.224000in}}%
\pgfpathlineto{\pgfqpoint{3.850923in}{4.224000in}}%
\pgfpathlineto{\pgfqpoint{4.046545in}{4.037004in}}%
\pgfpathlineto{\pgfqpoint{4.246949in}{3.842006in}}%
\pgfpathlineto{\pgfqpoint{4.590922in}{3.499060in}}%
\pgfpathlineto{\pgfqpoint{4.687838in}{3.400773in}}%
\pgfpathlineto{\pgfqpoint{4.768000in}{3.318641in}}%
\pgfpathlineto{\pgfqpoint{4.768000in}{3.318641in}}%
\pgfusepath{fill}%
\end{pgfscope}%
\begin{pgfscope}%
\pgfpathrectangle{\pgfqpoint{0.800000in}{0.528000in}}{\pgfqpoint{3.968000in}{3.696000in}}%
\pgfusepath{clip}%
\pgfsetbuttcap%
\pgfsetroundjoin%
\definecolor{currentfill}{rgb}{0.170948,0.694384,0.493803}%
\pgfsetfillcolor{currentfill}%
\pgfsetlinewidth{0.000000pt}%
\definecolor{currentstroke}{rgb}{0.000000,0.000000,0.000000}%
\pgfsetstrokecolor{currentstroke}%
\pgfsetdash{}{0pt}%
\pgfpathmoveto{\pgfqpoint{4.768000in}{3.324613in}}%
\pgfpathlineto{\pgfqpoint{4.581473in}{3.514667in}}%
\pgfpathlineto{\pgfqpoint{4.244082in}{3.850667in}}%
\pgfpathlineto{\pgfqpoint{4.046545in}{4.042772in}}%
\pgfpathlineto{\pgfqpoint{3.856979in}{4.224000in}}%
\pgfpathlineto{\pgfqpoint{3.853951in}{4.224000in}}%
\pgfpathlineto{\pgfqpoint{4.049194in}{4.037333in}}%
\pgfpathlineto{\pgfqpoint{4.246949in}{3.844931in}}%
\pgfpathlineto{\pgfqpoint{4.592445in}{3.500479in}}%
\pgfpathlineto{\pgfqpoint{4.688892in}{3.402667in}}%
\pgfpathlineto{\pgfqpoint{4.768000in}{3.321627in}}%
\pgfpathlineto{\pgfqpoint{4.768000in}{3.321627in}}%
\pgfusepath{fill}%
\end{pgfscope}%
\begin{pgfscope}%
\pgfpathrectangle{\pgfqpoint{0.800000in}{0.528000in}}{\pgfqpoint{3.968000in}{3.696000in}}%
\pgfusepath{clip}%
\pgfsetbuttcap%
\pgfsetroundjoin%
\definecolor{currentfill}{rgb}{0.170948,0.694384,0.493803}%
\pgfsetfillcolor{currentfill}%
\pgfsetlinewidth{0.000000pt}%
\definecolor{currentstroke}{rgb}{0.000000,0.000000,0.000000}%
\pgfsetstrokecolor{currentstroke}%
\pgfsetdash{}{0pt}%
\pgfpathmoveto{\pgfqpoint{4.768000in}{3.327600in}}%
\pgfpathlineto{\pgfqpoint{4.584384in}{3.514667in}}%
\pgfpathlineto{\pgfqpoint{4.246949in}{3.850779in}}%
\pgfpathlineto{\pgfqpoint{4.046545in}{4.045655in}}%
\pgfpathlineto{\pgfqpoint{3.860008in}{4.224000in}}%
\pgfpathlineto{\pgfqpoint{3.856979in}{4.224000in}}%
\pgfpathlineto{\pgfqpoint{4.052181in}{4.037333in}}%
\pgfpathlineto{\pgfqpoint{4.246949in}{3.847855in}}%
\pgfpathlineto{\pgfqpoint{4.593968in}{3.501897in}}%
\pgfpathlineto{\pgfqpoint{4.691780in}{3.402667in}}%
\pgfpathlineto{\pgfqpoint{4.768000in}{3.324613in}}%
\pgfpathlineto{\pgfqpoint{4.768000in}{3.324613in}}%
\pgfusepath{fill}%
\end{pgfscope}%
\begin{pgfscope}%
\pgfpathrectangle{\pgfqpoint{0.800000in}{0.528000in}}{\pgfqpoint{3.968000in}{3.696000in}}%
\pgfusepath{clip}%
\pgfsetbuttcap%
\pgfsetroundjoin%
\definecolor{currentfill}{rgb}{0.170948,0.694384,0.493803}%
\pgfsetfillcolor{currentfill}%
\pgfsetlinewidth{0.000000pt}%
\definecolor{currentstroke}{rgb}{0.000000,0.000000,0.000000}%
\pgfsetstrokecolor{currentstroke}%
\pgfsetdash{}{0pt}%
\pgfpathmoveto{\pgfqpoint{4.768000in}{3.330557in}}%
\pgfpathlineto{\pgfqpoint{4.438627in}{3.664000in}}%
\pgfpathlineto{\pgfqpoint{4.246949in}{3.853672in}}%
\pgfpathlineto{\pgfqpoint{4.046545in}{4.048537in}}%
\pgfpathlineto{\pgfqpoint{3.863036in}{4.224000in}}%
\pgfpathlineto{\pgfqpoint{3.860008in}{4.224000in}}%
\pgfpathlineto{\pgfqpoint{4.055168in}{4.037333in}}%
\pgfpathlineto{\pgfqpoint{4.247064in}{3.850667in}}%
\pgfpathlineto{\pgfqpoint{4.607677in}{3.491109in}}%
\pgfpathlineto{\pgfqpoint{4.768000in}{3.327600in}}%
\pgfpathlineto{\pgfqpoint{4.768000in}{3.328000in}}%
\pgfusepath{fill}%
\end{pgfscope}%
\begin{pgfscope}%
\pgfpathrectangle{\pgfqpoint{0.800000in}{0.528000in}}{\pgfqpoint{3.968000in}{3.696000in}}%
\pgfusepath{clip}%
\pgfsetbuttcap%
\pgfsetroundjoin%
\definecolor{currentfill}{rgb}{0.175707,0.697900,0.491033}%
\pgfsetfillcolor{currentfill}%
\pgfsetlinewidth{0.000000pt}%
\definecolor{currentstroke}{rgb}{0.000000,0.000000,0.000000}%
\pgfsetstrokecolor{currentstroke}%
\pgfsetdash{}{0pt}%
\pgfpathmoveto{\pgfqpoint{4.768000in}{3.333510in}}%
\pgfpathlineto{\pgfqpoint{4.441570in}{3.664000in}}%
\pgfpathlineto{\pgfqpoint{4.246949in}{3.856565in}}%
\pgfpathlineto{\pgfqpoint{4.046545in}{4.051419in}}%
\pgfpathlineto{\pgfqpoint{3.866064in}{4.224000in}}%
\pgfpathlineto{\pgfqpoint{3.863036in}{4.224000in}}%
\pgfpathlineto{\pgfqpoint{4.058154in}{4.037333in}}%
\pgfpathlineto{\pgfqpoint{4.250010in}{3.850667in}}%
\pgfpathlineto{\pgfqpoint{4.607677in}{3.494054in}}%
\pgfpathlineto{\pgfqpoint{4.768000in}{3.330557in}}%
\pgfpathlineto{\pgfqpoint{4.768000in}{3.330557in}}%
\pgfusepath{fill}%
\end{pgfscope}%
\begin{pgfscope}%
\pgfpathrectangle{\pgfqpoint{0.800000in}{0.528000in}}{\pgfqpoint{3.968000in}{3.696000in}}%
\pgfusepath{clip}%
\pgfsetbuttcap%
\pgfsetroundjoin%
\definecolor{currentfill}{rgb}{0.175707,0.697900,0.491033}%
\pgfsetfillcolor{currentfill}%
\pgfsetlinewidth{0.000000pt}%
\definecolor{currentstroke}{rgb}{0.000000,0.000000,0.000000}%
\pgfsetstrokecolor{currentstroke}%
\pgfsetdash{}{0pt}%
\pgfpathmoveto{\pgfqpoint{4.768000in}{3.336463in}}%
\pgfpathlineto{\pgfqpoint{4.444512in}{3.664000in}}%
\pgfpathlineto{\pgfqpoint{4.246949in}{3.859457in}}%
\pgfpathlineto{\pgfqpoint{4.046545in}{4.054302in}}%
\pgfpathlineto{\pgfqpoint{3.869092in}{4.224000in}}%
\pgfpathlineto{\pgfqpoint{3.866064in}{4.224000in}}%
\pgfpathlineto{\pgfqpoint{4.061141in}{4.037333in}}%
\pgfpathlineto{\pgfqpoint{4.252956in}{3.850667in}}%
\pgfpathlineto{\pgfqpoint{4.607677in}{3.496998in}}%
\pgfpathlineto{\pgfqpoint{4.768000in}{3.333510in}}%
\pgfpathlineto{\pgfqpoint{4.768000in}{3.333510in}}%
\pgfusepath{fill}%
\end{pgfscope}%
\begin{pgfscope}%
\pgfpathrectangle{\pgfqpoint{0.800000in}{0.528000in}}{\pgfqpoint{3.968000in}{3.696000in}}%
\pgfusepath{clip}%
\pgfsetbuttcap%
\pgfsetroundjoin%
\definecolor{currentfill}{rgb}{0.175707,0.697900,0.491033}%
\pgfsetfillcolor{currentfill}%
\pgfsetlinewidth{0.000000pt}%
\definecolor{currentstroke}{rgb}{0.000000,0.000000,0.000000}%
\pgfsetstrokecolor{currentstroke}%
\pgfsetdash{}{0pt}%
\pgfpathmoveto{\pgfqpoint{4.768000in}{3.339416in}}%
\pgfpathlineto{\pgfqpoint{4.447354in}{3.664100in}}%
\pgfpathlineto{\pgfqpoint{4.246949in}{3.862350in}}%
\pgfpathlineto{\pgfqpoint{4.046545in}{4.057184in}}%
\pgfpathlineto{\pgfqpoint{3.872121in}{4.224000in}}%
\pgfpathlineto{\pgfqpoint{3.869092in}{4.224000in}}%
\pgfpathlineto{\pgfqpoint{4.055448in}{4.045626in}}%
\pgfpathlineto{\pgfqpoint{4.166788in}{3.937803in}}%
\pgfpathlineto{\pgfqpoint{4.369457in}{3.738667in}}%
\pgfpathlineto{\pgfqpoint{4.567596in}{3.540458in}}%
\pgfpathlineto{\pgfqpoint{4.768000in}{3.336463in}}%
\pgfpathlineto{\pgfqpoint{4.768000in}{3.336463in}}%
\pgfusepath{fill}%
\end{pgfscope}%
\begin{pgfscope}%
\pgfpathrectangle{\pgfqpoint{0.800000in}{0.528000in}}{\pgfqpoint{3.968000in}{3.696000in}}%
\pgfusepath{clip}%
\pgfsetbuttcap%
\pgfsetroundjoin%
\definecolor{currentfill}{rgb}{0.180653,0.701402,0.488189}%
\pgfsetfillcolor{currentfill}%
\pgfsetlinewidth{0.000000pt}%
\definecolor{currentstroke}{rgb}{0.000000,0.000000,0.000000}%
\pgfsetstrokecolor{currentstroke}%
\pgfsetdash{}{0pt}%
\pgfpathmoveto{\pgfqpoint{4.768000in}{3.342368in}}%
\pgfpathlineto{\pgfqpoint{4.447354in}{3.667004in}}%
\pgfpathlineto{\pgfqpoint{4.246949in}{3.865243in}}%
\pgfpathlineto{\pgfqpoint{4.046545in}{4.060066in}}%
\pgfpathlineto{\pgfqpoint{3.875149in}{4.224000in}}%
\pgfpathlineto{\pgfqpoint{3.872121in}{4.224000in}}%
\pgfpathlineto{\pgfqpoint{4.046545in}{4.057184in}}%
\pgfpathlineto{\pgfqpoint{4.372379in}{3.738667in}}%
\pgfpathlineto{\pgfqpoint{4.567596in}{3.543400in}}%
\pgfpathlineto{\pgfqpoint{4.768000in}{3.339416in}}%
\pgfpathlineto{\pgfqpoint{4.768000in}{3.339416in}}%
\pgfusepath{fill}%
\end{pgfscope}%
\begin{pgfscope}%
\pgfpathrectangle{\pgfqpoint{0.800000in}{0.528000in}}{\pgfqpoint{3.968000in}{3.696000in}}%
\pgfusepath{clip}%
\pgfsetbuttcap%
\pgfsetroundjoin%
\definecolor{currentfill}{rgb}{0.180653,0.701402,0.488189}%
\pgfsetfillcolor{currentfill}%
\pgfsetlinewidth{0.000000pt}%
\definecolor{currentstroke}{rgb}{0.000000,0.000000,0.000000}%
\pgfsetstrokecolor{currentstroke}%
\pgfsetdash{}{0pt}%
\pgfpathmoveto{\pgfqpoint{4.768000in}{3.345321in}}%
\pgfpathlineto{\pgfqpoint{4.447354in}{3.669907in}}%
\pgfpathlineto{\pgfqpoint{4.246949in}{3.868136in}}%
\pgfpathlineto{\pgfqpoint{4.046545in}{4.062949in}}%
\pgfpathlineto{\pgfqpoint{3.878177in}{4.224000in}}%
\pgfpathlineto{\pgfqpoint{3.875149in}{4.224000in}}%
\pgfpathlineto{\pgfqpoint{4.046545in}{4.060066in}}%
\pgfpathlineto{\pgfqpoint{4.375301in}{3.738667in}}%
\pgfpathlineto{\pgfqpoint{4.567596in}{3.546342in}}%
\pgfpathlineto{\pgfqpoint{4.768000in}{3.342368in}}%
\pgfpathlineto{\pgfqpoint{4.768000in}{3.342368in}}%
\pgfusepath{fill}%
\end{pgfscope}%
\begin{pgfscope}%
\pgfpathrectangle{\pgfqpoint{0.800000in}{0.528000in}}{\pgfqpoint{3.968000in}{3.696000in}}%
\pgfusepath{clip}%
\pgfsetbuttcap%
\pgfsetroundjoin%
\definecolor{currentfill}{rgb}{0.180653,0.701402,0.488189}%
\pgfsetfillcolor{currentfill}%
\pgfsetlinewidth{0.000000pt}%
\definecolor{currentstroke}{rgb}{0.000000,0.000000,0.000000}%
\pgfsetstrokecolor{currentstroke}%
\pgfsetdash{}{0pt}%
\pgfpathmoveto{\pgfqpoint{4.768000in}{3.348274in}}%
\pgfpathlineto{\pgfqpoint{4.447354in}{3.672810in}}%
\pgfpathlineto{\pgfqpoint{4.246949in}{3.871029in}}%
\pgfpathlineto{\pgfqpoint{4.046545in}{4.065831in}}%
\pgfpathlineto{\pgfqpoint{3.881206in}{4.224000in}}%
\pgfpathlineto{\pgfqpoint{3.878177in}{4.224000in}}%
\pgfpathlineto{\pgfqpoint{4.046545in}{4.062949in}}%
\pgfpathlineto{\pgfqpoint{4.378224in}{3.738667in}}%
\pgfpathlineto{\pgfqpoint{4.567596in}{3.549284in}}%
\pgfpathlineto{\pgfqpoint{4.768000in}{3.345321in}}%
\pgfpathlineto{\pgfqpoint{4.768000in}{3.345321in}}%
\pgfusepath{fill}%
\end{pgfscope}%
\begin{pgfscope}%
\pgfpathrectangle{\pgfqpoint{0.800000in}{0.528000in}}{\pgfqpoint{3.968000in}{3.696000in}}%
\pgfusepath{clip}%
\pgfsetbuttcap%
\pgfsetroundjoin%
\definecolor{currentfill}{rgb}{0.180653,0.701402,0.488189}%
\pgfsetfillcolor{currentfill}%
\pgfsetlinewidth{0.000000pt}%
\definecolor{currentstroke}{rgb}{0.000000,0.000000,0.000000}%
\pgfsetstrokecolor{currentstroke}%
\pgfsetdash{}{0pt}%
\pgfpathmoveto{\pgfqpoint{4.768000in}{3.351227in}}%
\pgfpathlineto{\pgfqpoint{4.447354in}{3.675713in}}%
\pgfpathlineto{\pgfqpoint{4.246949in}{3.873921in}}%
\pgfpathlineto{\pgfqpoint{4.046545in}{4.068714in}}%
\pgfpathlineto{\pgfqpoint{3.884234in}{4.224000in}}%
\pgfpathlineto{\pgfqpoint{3.881206in}{4.224000in}}%
\pgfpathlineto{\pgfqpoint{4.046545in}{4.065831in}}%
\pgfpathlineto{\pgfqpoint{4.407273in}{3.712729in}}%
\pgfpathlineto{\pgfqpoint{4.740137in}{3.376713in}}%
\pgfpathlineto{\pgfqpoint{4.768000in}{3.348274in}}%
\pgfpathlineto{\pgfqpoint{4.768000in}{3.348274in}}%
\pgfusepath{fill}%
\end{pgfscope}%
\begin{pgfscope}%
\pgfpathrectangle{\pgfqpoint{0.800000in}{0.528000in}}{\pgfqpoint{3.968000in}{3.696000in}}%
\pgfusepath{clip}%
\pgfsetbuttcap%
\pgfsetroundjoin%
\definecolor{currentfill}{rgb}{0.185783,0.704891,0.485273}%
\pgfsetfillcolor{currentfill}%
\pgfsetlinewidth{0.000000pt}%
\definecolor{currentstroke}{rgb}{0.000000,0.000000,0.000000}%
\pgfsetstrokecolor{currentstroke}%
\pgfsetdash{}{0pt}%
\pgfpathmoveto{\pgfqpoint{4.768000in}{3.354180in}}%
\pgfpathlineto{\pgfqpoint{4.447354in}{3.678617in}}%
\pgfpathlineto{\pgfqpoint{4.246949in}{3.876814in}}%
\pgfpathlineto{\pgfqpoint{4.043355in}{4.074667in}}%
\pgfpathlineto{\pgfqpoint{3.886222in}{4.224000in}}%
\pgfpathlineto{\pgfqpoint{3.884234in}{4.224000in}}%
\pgfpathlineto{\pgfqpoint{4.046545in}{4.068714in}}%
\pgfpathlineto{\pgfqpoint{4.407273in}{3.715630in}}%
\pgfpathlineto{\pgfqpoint{4.741638in}{3.378111in}}%
\pgfpathlineto{\pgfqpoint{4.768000in}{3.351227in}}%
\pgfpathlineto{\pgfqpoint{4.768000in}{3.351227in}}%
\pgfusepath{fill}%
\end{pgfscope}%
\begin{pgfscope}%
\pgfpathrectangle{\pgfqpoint{0.800000in}{0.528000in}}{\pgfqpoint{3.968000in}{3.696000in}}%
\pgfusepath{clip}%
\pgfsetbuttcap%
\pgfsetroundjoin%
\definecolor{currentfill}{rgb}{0.185783,0.704891,0.485273}%
\pgfsetfillcolor{currentfill}%
\pgfsetlinewidth{0.000000pt}%
\definecolor{currentstroke}{rgb}{0.000000,0.000000,0.000000}%
\pgfsetstrokecolor{currentstroke}%
\pgfsetdash{}{0pt}%
\pgfpathmoveto{\pgfqpoint{4.768000in}{3.357133in}}%
\pgfpathlineto{\pgfqpoint{4.447354in}{3.681520in}}%
\pgfpathlineto{\pgfqpoint{4.246949in}{3.879707in}}%
\pgfpathlineto{\pgfqpoint{4.046350in}{4.074667in}}%
\pgfpathlineto{\pgfqpoint{3.890239in}{4.224000in}}%
\pgfpathlineto{\pgfqpoint{3.887249in}{4.224000in}}%
\pgfpathlineto{\pgfqpoint{4.246949in}{3.876814in}}%
\pgfpathlineto{\pgfqpoint{4.589855in}{3.535400in}}%
\pgfpathlineto{\pgfqpoint{4.687838in}{3.436177in}}%
\pgfpathlineto{\pgfqpoint{4.768000in}{3.354180in}}%
\pgfpathlineto{\pgfqpoint{4.768000in}{3.354180in}}%
\pgfusepath{fill}%
\end{pgfscope}%
\begin{pgfscope}%
\pgfpathrectangle{\pgfqpoint{0.800000in}{0.528000in}}{\pgfqpoint{3.968000in}{3.696000in}}%
\pgfusepath{clip}%
\pgfsetbuttcap%
\pgfsetroundjoin%
\definecolor{currentfill}{rgb}{0.185783,0.704891,0.485273}%
\pgfsetfillcolor{currentfill}%
\pgfsetlinewidth{0.000000pt}%
\definecolor{currentstroke}{rgb}{0.000000,0.000000,0.000000}%
\pgfsetstrokecolor{currentstroke}%
\pgfsetdash{}{0pt}%
\pgfpathmoveto{\pgfqpoint{4.768000in}{3.360086in}}%
\pgfpathlineto{\pgfqpoint{4.447354in}{3.684423in}}%
\pgfpathlineto{\pgfqpoint{4.241435in}{3.888000in}}%
\pgfpathlineto{\pgfqpoint{3.893229in}{4.224000in}}%
\pgfpathlineto{\pgfqpoint{3.890239in}{4.224000in}}%
\pgfpathlineto{\pgfqpoint{4.246949in}{3.879707in}}%
\pgfpathlineto{\pgfqpoint{4.591363in}{3.536804in}}%
\pgfpathlineto{\pgfqpoint{4.687838in}{3.439125in}}%
\pgfpathlineto{\pgfqpoint{4.768000in}{3.357133in}}%
\pgfpathlineto{\pgfqpoint{4.768000in}{3.357133in}}%
\pgfusepath{fill}%
\end{pgfscope}%
\begin{pgfscope}%
\pgfpathrectangle{\pgfqpoint{0.800000in}{0.528000in}}{\pgfqpoint{3.968000in}{3.696000in}}%
\pgfusepath{clip}%
\pgfsetbuttcap%
\pgfsetroundjoin%
\definecolor{currentfill}{rgb}{0.185783,0.704891,0.485273}%
\pgfsetfillcolor{currentfill}%
\pgfsetlinewidth{0.000000pt}%
\definecolor{currentstroke}{rgb}{0.000000,0.000000,0.000000}%
\pgfsetstrokecolor{currentstroke}%
\pgfsetdash{}{0pt}%
\pgfpathmoveto{\pgfqpoint{4.768000in}{3.363039in}}%
\pgfpathlineto{\pgfqpoint{4.447354in}{3.687327in}}%
\pgfpathlineto{\pgfqpoint{4.244389in}{3.888000in}}%
\pgfpathlineto{\pgfqpoint{3.896219in}{4.224000in}}%
\pgfpathlineto{\pgfqpoint{3.893229in}{4.224000in}}%
\pgfpathlineto{\pgfqpoint{4.246949in}{3.882600in}}%
\pgfpathlineto{\pgfqpoint{4.592870in}{3.538208in}}%
\pgfpathlineto{\pgfqpoint{4.689850in}{3.440000in}}%
\pgfpathlineto{\pgfqpoint{4.768000in}{3.360086in}}%
\pgfpathlineto{\pgfqpoint{4.768000in}{3.360086in}}%
\pgfusepath{fill}%
\end{pgfscope}%
\begin{pgfscope}%
\pgfpathrectangle{\pgfqpoint{0.800000in}{0.528000in}}{\pgfqpoint{3.968000in}{3.696000in}}%
\pgfusepath{clip}%
\pgfsetbuttcap%
\pgfsetroundjoin%
\definecolor{currentfill}{rgb}{0.191090,0.708366,0.482284}%
\pgfsetfillcolor{currentfill}%
\pgfsetlinewidth{0.000000pt}%
\definecolor{currentstroke}{rgb}{0.000000,0.000000,0.000000}%
\pgfsetstrokecolor{currentstroke}%
\pgfsetdash{}{0pt}%
\pgfpathmoveto{\pgfqpoint{4.768000in}{3.365984in}}%
\pgfpathlineto{\pgfqpoint{4.422147in}{3.715188in}}%
\pgfpathlineto{\pgfqpoint{4.323259in}{3.813333in}}%
\pgfpathlineto{\pgfqpoint{4.126707in}{4.005576in}}%
\pgfpathlineto{\pgfqpoint{3.899210in}{4.224000in}}%
\pgfpathlineto{\pgfqpoint{3.896219in}{4.224000in}}%
\pgfpathlineto{\pgfqpoint{4.246949in}{3.885493in}}%
\pgfpathlineto{\pgfqpoint{4.607677in}{3.526310in}}%
\pgfpathlineto{\pgfqpoint{4.768000in}{3.363039in}}%
\pgfpathlineto{\pgfqpoint{4.768000in}{3.365333in}}%
\pgfusepath{fill}%
\end{pgfscope}%
\begin{pgfscope}%
\pgfpathrectangle{\pgfqpoint{0.800000in}{0.528000in}}{\pgfqpoint{3.968000in}{3.696000in}}%
\pgfusepath{clip}%
\pgfsetbuttcap%
\pgfsetroundjoin%
\definecolor{currentfill}{rgb}{0.191090,0.708366,0.482284}%
\pgfsetfillcolor{currentfill}%
\pgfsetlinewidth{0.000000pt}%
\definecolor{currentstroke}{rgb}{0.000000,0.000000,0.000000}%
\pgfsetstrokecolor{currentstroke}%
\pgfsetdash{}{0pt}%
\pgfpathmoveto{\pgfqpoint{4.768000in}{3.368904in}}%
\pgfpathlineto{\pgfqpoint{4.423645in}{3.716583in}}%
\pgfpathlineto{\pgfqpoint{4.310381in}{3.828916in}}%
\pgfpathlineto{\pgfqpoint{4.126707in}{4.008431in}}%
\pgfpathlineto{\pgfqpoint{3.902200in}{4.224000in}}%
\pgfpathlineto{\pgfqpoint{3.899210in}{4.224000in}}%
\pgfpathlineto{\pgfqpoint{4.247338in}{3.888000in}}%
\pgfpathlineto{\pgfqpoint{4.607677in}{3.529222in}}%
\pgfpathlineto{\pgfqpoint{4.768000in}{3.365984in}}%
\pgfpathlineto{\pgfqpoint{4.768000in}{3.365984in}}%
\pgfusepath{fill}%
\end{pgfscope}%
\begin{pgfscope}%
\pgfpathrectangle{\pgfqpoint{0.800000in}{0.528000in}}{\pgfqpoint{3.968000in}{3.696000in}}%
\pgfusepath{clip}%
\pgfsetbuttcap%
\pgfsetroundjoin%
\definecolor{currentfill}{rgb}{0.191090,0.708366,0.482284}%
\pgfsetfillcolor{currentfill}%
\pgfsetlinewidth{0.000000pt}%
\definecolor{currentstroke}{rgb}{0.000000,0.000000,0.000000}%
\pgfsetstrokecolor{currentstroke}%
\pgfsetdash{}{0pt}%
\pgfpathmoveto{\pgfqpoint{4.768000in}{3.371825in}}%
\pgfpathlineto{\pgfqpoint{4.425143in}{3.717979in}}%
\pgfpathlineto{\pgfqpoint{4.327111in}{3.815308in}}%
\pgfpathlineto{\pgfqpoint{4.126707in}{4.011287in}}%
\pgfpathlineto{\pgfqpoint{3.905190in}{4.224000in}}%
\pgfpathlineto{\pgfqpoint{3.902200in}{4.224000in}}%
\pgfpathlineto{\pgfqpoint{4.250256in}{3.888000in}}%
\pgfpathlineto{\pgfqpoint{4.607677in}{3.532133in}}%
\pgfpathlineto{\pgfqpoint{4.768000in}{3.368904in}}%
\pgfpathlineto{\pgfqpoint{4.768000in}{3.368904in}}%
\pgfusepath{fill}%
\end{pgfscope}%
\begin{pgfscope}%
\pgfpathrectangle{\pgfqpoint{0.800000in}{0.528000in}}{\pgfqpoint{3.968000in}{3.696000in}}%
\pgfusepath{clip}%
\pgfsetbuttcap%
\pgfsetroundjoin%
\definecolor{currentfill}{rgb}{0.196571,0.711827,0.479221}%
\pgfsetfillcolor{currentfill}%
\pgfsetlinewidth{0.000000pt}%
\definecolor{currentstroke}{rgb}{0.000000,0.000000,0.000000}%
\pgfsetstrokecolor{currentstroke}%
\pgfsetdash{}{0pt}%
\pgfpathmoveto{\pgfqpoint{4.768000in}{3.374745in}}%
\pgfpathlineto{\pgfqpoint{4.426642in}{3.719374in}}%
\pgfpathlineto{\pgfqpoint{4.327111in}{3.818173in}}%
\pgfpathlineto{\pgfqpoint{4.126707in}{4.014142in}}%
\pgfpathlineto{\pgfqpoint{3.908180in}{4.224000in}}%
\pgfpathlineto{\pgfqpoint{3.905190in}{4.224000in}}%
\pgfpathlineto{\pgfqpoint{4.253174in}{3.888000in}}%
\pgfpathlineto{\pgfqpoint{4.607677in}{3.535045in}}%
\pgfpathlineto{\pgfqpoint{4.768000in}{3.371825in}}%
\pgfpathlineto{\pgfqpoint{4.768000in}{3.371825in}}%
\pgfusepath{fill}%
\end{pgfscope}%
\begin{pgfscope}%
\pgfpathrectangle{\pgfqpoint{0.800000in}{0.528000in}}{\pgfqpoint{3.968000in}{3.696000in}}%
\pgfusepath{clip}%
\pgfsetbuttcap%
\pgfsetroundjoin%
\definecolor{currentfill}{rgb}{0.196571,0.711827,0.479221}%
\pgfsetfillcolor{currentfill}%
\pgfsetlinewidth{0.000000pt}%
\definecolor{currentstroke}{rgb}{0.000000,0.000000,0.000000}%
\pgfsetstrokecolor{currentstroke}%
\pgfsetdash{}{0pt}%
\pgfpathmoveto{\pgfqpoint{4.768000in}{3.377665in}}%
\pgfpathlineto{\pgfqpoint{4.428140in}{3.720770in}}%
\pgfpathlineto{\pgfqpoint{4.327111in}{3.821039in}}%
\pgfpathlineto{\pgfqpoint{4.126707in}{4.016998in}}%
\pgfpathlineto{\pgfqpoint{3.911170in}{4.224000in}}%
\pgfpathlineto{\pgfqpoint{3.908180in}{4.224000in}}%
\pgfpathlineto{\pgfqpoint{4.246949in}{3.896966in}}%
\pgfpathlineto{\pgfqpoint{4.447354in}{3.698940in}}%
\pgfpathlineto{\pgfqpoint{4.647758in}{3.497364in}}%
\pgfpathlineto{\pgfqpoint{4.768000in}{3.374745in}}%
\pgfpathlineto{\pgfqpoint{4.768000in}{3.374745in}}%
\pgfusepath{fill}%
\end{pgfscope}%
\begin{pgfscope}%
\pgfpathrectangle{\pgfqpoint{0.800000in}{0.528000in}}{\pgfqpoint{3.968000in}{3.696000in}}%
\pgfusepath{clip}%
\pgfsetbuttcap%
\pgfsetroundjoin%
\definecolor{currentfill}{rgb}{0.196571,0.711827,0.479221}%
\pgfsetfillcolor{currentfill}%
\pgfsetlinewidth{0.000000pt}%
\definecolor{currentstroke}{rgb}{0.000000,0.000000,0.000000}%
\pgfsetstrokecolor{currentstroke}%
\pgfsetdash{}{0pt}%
\pgfpathmoveto{\pgfqpoint{4.768000in}{3.380586in}}%
\pgfpathlineto{\pgfqpoint{4.429638in}{3.722166in}}%
\pgfpathlineto{\pgfqpoint{4.327111in}{3.823904in}}%
\pgfpathlineto{\pgfqpoint{4.126707in}{4.019853in}}%
\pgfpathlineto{\pgfqpoint{3.914161in}{4.224000in}}%
\pgfpathlineto{\pgfqpoint{3.911170in}{4.224000in}}%
\pgfpathlineto{\pgfqpoint{4.246949in}{3.899827in}}%
\pgfpathlineto{\pgfqpoint{4.447859in}{3.701333in}}%
\pgfpathlineto{\pgfqpoint{4.647758in}{3.500278in}}%
\pgfpathlineto{\pgfqpoint{4.768000in}{3.377665in}}%
\pgfpathlineto{\pgfqpoint{4.768000in}{3.377665in}}%
\pgfusepath{fill}%
\end{pgfscope}%
\begin{pgfscope}%
\pgfpathrectangle{\pgfqpoint{0.800000in}{0.528000in}}{\pgfqpoint{3.968000in}{3.696000in}}%
\pgfusepath{clip}%
\pgfsetbuttcap%
\pgfsetroundjoin%
\definecolor{currentfill}{rgb}{0.196571,0.711827,0.479221}%
\pgfsetfillcolor{currentfill}%
\pgfsetlinewidth{0.000000pt}%
\definecolor{currentstroke}{rgb}{0.000000,0.000000,0.000000}%
\pgfsetstrokecolor{currentstroke}%
\pgfsetdash{}{0pt}%
\pgfpathmoveto{\pgfqpoint{4.768000in}{3.383506in}}%
\pgfpathlineto{\pgfqpoint{4.431136in}{3.723561in}}%
\pgfpathlineto{\pgfqpoint{4.327111in}{3.826770in}}%
\pgfpathlineto{\pgfqpoint{4.126707in}{4.022708in}}%
\pgfpathlineto{\pgfqpoint{3.917151in}{4.224000in}}%
\pgfpathlineto{\pgfqpoint{3.914161in}{4.224000in}}%
\pgfpathlineto{\pgfqpoint{4.246949in}{3.902689in}}%
\pgfpathlineto{\pgfqpoint{4.450738in}{3.701333in}}%
\pgfpathlineto{\pgfqpoint{4.647758in}{3.503191in}}%
\pgfpathlineto{\pgfqpoint{4.768000in}{3.380586in}}%
\pgfpathlineto{\pgfqpoint{4.768000in}{3.380586in}}%
\pgfusepath{fill}%
\end{pgfscope}%
\begin{pgfscope}%
\pgfpathrectangle{\pgfqpoint{0.800000in}{0.528000in}}{\pgfqpoint{3.968000in}{3.696000in}}%
\pgfusepath{clip}%
\pgfsetbuttcap%
\pgfsetroundjoin%
\definecolor{currentfill}{rgb}{0.202219,0.715272,0.476084}%
\pgfsetfillcolor{currentfill}%
\pgfsetlinewidth{0.000000pt}%
\definecolor{currentstroke}{rgb}{0.000000,0.000000,0.000000}%
\pgfsetstrokecolor{currentstroke}%
\pgfsetdash{}{0pt}%
\pgfpathmoveto{\pgfqpoint{4.768000in}{3.386426in}}%
\pgfpathlineto{\pgfqpoint{4.432634in}{3.724957in}}%
\pgfpathlineto{\pgfqpoint{4.327111in}{3.829636in}}%
\pgfpathlineto{\pgfqpoint{4.126707in}{4.025564in}}%
\pgfpathlineto{\pgfqpoint{3.920141in}{4.224000in}}%
\pgfpathlineto{\pgfqpoint{3.917151in}{4.224000in}}%
\pgfpathlineto{\pgfqpoint{4.246949in}{3.905550in}}%
\pgfpathlineto{\pgfqpoint{4.453618in}{3.701333in}}%
\pgfpathlineto{\pgfqpoint{4.647758in}{3.506105in}}%
\pgfpathlineto{\pgfqpoint{4.768000in}{3.383506in}}%
\pgfpathlineto{\pgfqpoint{4.768000in}{3.383506in}}%
\pgfusepath{fill}%
\end{pgfscope}%
\begin{pgfscope}%
\pgfpathrectangle{\pgfqpoint{0.800000in}{0.528000in}}{\pgfqpoint{3.968000in}{3.696000in}}%
\pgfusepath{clip}%
\pgfsetbuttcap%
\pgfsetroundjoin%
\definecolor{currentfill}{rgb}{0.202219,0.715272,0.476084}%
\pgfsetfillcolor{currentfill}%
\pgfsetlinewidth{0.000000pt}%
\definecolor{currentstroke}{rgb}{0.000000,0.000000,0.000000}%
\pgfsetstrokecolor{currentstroke}%
\pgfsetdash{}{0pt}%
\pgfpathmoveto{\pgfqpoint{4.768000in}{3.389346in}}%
\pgfpathlineto{\pgfqpoint{4.434133in}{3.726352in}}%
\pgfpathlineto{\pgfqpoint{4.327111in}{3.832501in}}%
\pgfpathlineto{\pgfqpoint{4.121991in}{4.032940in}}%
\pgfpathlineto{\pgfqpoint{4.040143in}{4.112000in}}%
\pgfpathlineto{\pgfqpoint{3.923131in}{4.224000in}}%
\pgfpathlineto{\pgfqpoint{3.920141in}{4.224000in}}%
\pgfpathlineto{\pgfqpoint{4.246949in}{3.908412in}}%
\pgfpathlineto{\pgfqpoint{4.447354in}{3.710453in}}%
\pgfpathlineto{\pgfqpoint{4.647758in}{3.509019in}}%
\pgfpathlineto{\pgfqpoint{4.768000in}{3.386426in}}%
\pgfpathlineto{\pgfqpoint{4.768000in}{3.386426in}}%
\pgfusepath{fill}%
\end{pgfscope}%
\begin{pgfscope}%
\pgfpathrectangle{\pgfqpoint{0.800000in}{0.528000in}}{\pgfqpoint{3.968000in}{3.696000in}}%
\pgfusepath{clip}%
\pgfsetbuttcap%
\pgfsetroundjoin%
\definecolor{currentfill}{rgb}{0.202219,0.715272,0.476084}%
\pgfsetfillcolor{currentfill}%
\pgfsetlinewidth{0.000000pt}%
\definecolor{currentstroke}{rgb}{0.000000,0.000000,0.000000}%
\pgfsetstrokecolor{currentstroke}%
\pgfsetdash{}{0pt}%
\pgfpathmoveto{\pgfqpoint{4.768000in}{3.392267in}}%
\pgfpathlineto{\pgfqpoint{4.435631in}{3.727748in}}%
\pgfpathlineto{\pgfqpoint{4.327111in}{3.835367in}}%
\pgfpathlineto{\pgfqpoint{4.120449in}{4.037333in}}%
\pgfpathlineto{\pgfqpoint{3.926121in}{4.224000in}}%
\pgfpathlineto{\pgfqpoint{3.923131in}{4.224000in}}%
\pgfpathlineto{\pgfqpoint{4.246949in}{3.911273in}}%
\pgfpathlineto{\pgfqpoint{4.447354in}{3.713325in}}%
\pgfpathlineto{\pgfqpoint{4.647758in}{3.511933in}}%
\pgfpathlineto{\pgfqpoint{4.768000in}{3.389346in}}%
\pgfpathlineto{\pgfqpoint{4.768000in}{3.389346in}}%
\pgfusepath{fill}%
\end{pgfscope}%
\begin{pgfscope}%
\pgfpathrectangle{\pgfqpoint{0.800000in}{0.528000in}}{\pgfqpoint{3.968000in}{3.696000in}}%
\pgfusepath{clip}%
\pgfsetbuttcap%
\pgfsetroundjoin%
\definecolor{currentfill}{rgb}{0.202219,0.715272,0.476084}%
\pgfsetfillcolor{currentfill}%
\pgfsetlinewidth{0.000000pt}%
\definecolor{currentstroke}{rgb}{0.000000,0.000000,0.000000}%
\pgfsetstrokecolor{currentstroke}%
\pgfsetdash{}{0pt}%
\pgfpathmoveto{\pgfqpoint{4.768000in}{3.395187in}}%
\pgfpathlineto{\pgfqpoint{4.437129in}{3.729143in}}%
\pgfpathlineto{\pgfqpoint{4.327111in}{3.838233in}}%
\pgfpathlineto{\pgfqpoint{4.123398in}{4.037333in}}%
\pgfpathlineto{\pgfqpoint{3.929076in}{4.224000in}}%
\pgfpathlineto{\pgfqpoint{3.926121in}{4.224000in}}%
\pgfpathlineto{\pgfqpoint{3.926303in}{4.223827in}}%
\pgfpathlineto{\pgfqpoint{4.126707in}{4.031275in}}%
\pgfpathlineto{\pgfqpoint{4.327111in}{3.835367in}}%
\pgfpathlineto{\pgfqpoint{4.649894in}{3.512677in}}%
\pgfpathlineto{\pgfqpoint{4.768000in}{3.392267in}}%
\pgfpathlineto{\pgfqpoint{4.768000in}{3.392267in}}%
\pgfusepath{fill}%
\end{pgfscope}%
\begin{pgfscope}%
\pgfpathrectangle{\pgfqpoint{0.800000in}{0.528000in}}{\pgfqpoint{3.968000in}{3.696000in}}%
\pgfusepath{clip}%
\pgfsetbuttcap%
\pgfsetroundjoin%
\definecolor{currentfill}{rgb}{0.208030,0.718701,0.472873}%
\pgfsetfillcolor{currentfill}%
\pgfsetlinewidth{0.000000pt}%
\definecolor{currentstroke}{rgb}{0.000000,0.000000,0.000000}%
\pgfsetstrokecolor{currentstroke}%
\pgfsetdash{}{0pt}%
\pgfpathmoveto{\pgfqpoint{4.768000in}{3.398107in}}%
\pgfpathlineto{\pgfqpoint{4.438627in}{3.730539in}}%
\pgfpathlineto{\pgfqpoint{4.327111in}{3.841098in}}%
\pgfpathlineto{\pgfqpoint{4.126348in}{4.037333in}}%
\pgfpathlineto{\pgfqpoint{3.932030in}{4.224000in}}%
\pgfpathlineto{\pgfqpoint{3.929076in}{4.224000in}}%
\pgfpathlineto{\pgfqpoint{4.126707in}{4.034130in}}%
\pgfpathlineto{\pgfqpoint{4.327111in}{3.838233in}}%
\pgfpathlineto{\pgfqpoint{4.684402in}{3.480534in}}%
\pgfpathlineto{\pgfqpoint{4.768000in}{3.395187in}}%
\pgfpathlineto{\pgfqpoint{4.768000in}{3.395187in}}%
\pgfusepath{fill}%
\end{pgfscope}%
\begin{pgfscope}%
\pgfpathrectangle{\pgfqpoint{0.800000in}{0.528000in}}{\pgfqpoint{3.968000in}{3.696000in}}%
\pgfusepath{clip}%
\pgfsetbuttcap%
\pgfsetroundjoin%
\definecolor{currentfill}{rgb}{0.208030,0.718701,0.472873}%
\pgfsetfillcolor{currentfill}%
\pgfsetlinewidth{0.000000pt}%
\definecolor{currentstroke}{rgb}{0.000000,0.000000,0.000000}%
\pgfsetstrokecolor{currentstroke}%
\pgfsetdash{}{0pt}%
\pgfpathmoveto{\pgfqpoint{4.768000in}{3.401028in}}%
\pgfpathlineto{\pgfqpoint{4.440126in}{3.731934in}}%
\pgfpathlineto{\pgfqpoint{4.327111in}{3.843964in}}%
\pgfpathlineto{\pgfqpoint{4.126707in}{4.039814in}}%
\pgfpathlineto{\pgfqpoint{3.934983in}{4.224000in}}%
\pgfpathlineto{\pgfqpoint{3.932030in}{4.224000in}}%
\pgfpathlineto{\pgfqpoint{4.126707in}{4.036985in}}%
\pgfpathlineto{\pgfqpoint{4.327111in}{3.841098in}}%
\pgfpathlineto{\pgfqpoint{4.653617in}{3.514667in}}%
\pgfpathlineto{\pgfqpoint{4.768000in}{3.398107in}}%
\pgfpathlineto{\pgfqpoint{4.768000in}{3.398107in}}%
\pgfusepath{fill}%
\end{pgfscope}%
\begin{pgfscope}%
\pgfpathrectangle{\pgfqpoint{0.800000in}{0.528000in}}{\pgfqpoint{3.968000in}{3.696000in}}%
\pgfusepath{clip}%
\pgfsetbuttcap%
\pgfsetroundjoin%
\definecolor{currentfill}{rgb}{0.208030,0.718701,0.472873}%
\pgfsetfillcolor{currentfill}%
\pgfsetlinewidth{0.000000pt}%
\definecolor{currentstroke}{rgb}{0.000000,0.000000,0.000000}%
\pgfsetstrokecolor{currentstroke}%
\pgfsetdash{}{0pt}%
\pgfpathmoveto{\pgfqpoint{4.768000in}{3.403934in}}%
\pgfpathlineto{\pgfqpoint{4.567596in}{3.607310in}}%
\pgfpathlineto{\pgfqpoint{4.361036in}{3.813333in}}%
\pgfpathlineto{\pgfqpoint{4.006465in}{4.158505in}}%
\pgfpathlineto{\pgfqpoint{3.937936in}{4.224000in}}%
\pgfpathlineto{\pgfqpoint{3.934983in}{4.224000in}}%
\pgfpathlineto{\pgfqpoint{4.129265in}{4.037333in}}%
\pgfpathlineto{\pgfqpoint{4.327111in}{3.843964in}}%
\pgfpathlineto{\pgfqpoint{4.656458in}{3.514667in}}%
\pgfpathlineto{\pgfqpoint{4.768000in}{3.401028in}}%
\pgfpathlineto{\pgfqpoint{4.768000in}{3.402667in}}%
\pgfusepath{fill}%
\end{pgfscope}%
\begin{pgfscope}%
\pgfpathrectangle{\pgfqpoint{0.800000in}{0.528000in}}{\pgfqpoint{3.968000in}{3.696000in}}%
\pgfusepath{clip}%
\pgfsetbuttcap%
\pgfsetroundjoin%
\definecolor{currentfill}{rgb}{0.208030,0.718701,0.472873}%
\pgfsetfillcolor{currentfill}%
\pgfsetlinewidth{0.000000pt}%
\definecolor{currentstroke}{rgb}{0.000000,0.000000,0.000000}%
\pgfsetstrokecolor{currentstroke}%
\pgfsetdash{}{0pt}%
\pgfpathmoveto{\pgfqpoint{4.768000in}{3.406822in}}%
\pgfpathlineto{\pgfqpoint{4.735531in}{3.440000in}}%
\pgfpathlineto{\pgfqpoint{4.731880in}{3.443690in}}%
\pgfpathlineto{\pgfqpoint{4.727919in}{3.447772in}}%
\pgfpathlineto{\pgfqpoint{4.698893in}{3.477333in}}%
\pgfpathlineto{\pgfqpoint{4.693582in}{3.482683in}}%
\pgfpathlineto{\pgfqpoint{4.687838in}{3.488583in}}%
\pgfpathlineto{\pgfqpoint{4.662141in}{3.514667in}}%
\pgfpathlineto{\pgfqpoint{4.655219in}{3.521617in}}%
\pgfpathlineto{\pgfqpoint{4.647758in}{3.529256in}}%
\pgfpathlineto{\pgfqpoint{4.625275in}{3.552000in}}%
\pgfpathlineto{\pgfqpoint{4.616792in}{3.560490in}}%
\pgfpathlineto{\pgfqpoint{4.607677in}{3.569791in}}%
\pgfpathlineto{\pgfqpoint{4.588294in}{3.589333in}}%
\pgfpathlineto{\pgfqpoint{4.578299in}{3.599303in}}%
\pgfpathlineto{\pgfqpoint{4.567596in}{3.610188in}}%
\pgfpathlineto{\pgfqpoint{4.551197in}{3.626667in}}%
\pgfpathlineto{\pgfqpoint{4.539741in}{3.638055in}}%
\pgfpathlineto{\pgfqpoint{4.527515in}{3.650447in}}%
\pgfpathlineto{\pgfqpoint{4.513983in}{3.664000in}}%
\pgfpathlineto{\pgfqpoint{4.501118in}{3.676746in}}%
\pgfpathlineto{\pgfqpoint{4.487434in}{3.690570in}}%
\pgfpathlineto{\pgfqpoint{4.476651in}{3.701333in}}%
\pgfpathlineto{\pgfqpoint{4.447354in}{3.730556in}}%
\pgfpathlineto{\pgfqpoint{4.443122in}{3.734725in}}%
\pgfpathlineto{\pgfqpoint{4.439200in}{3.738667in}}%
\pgfpathlineto{\pgfqpoint{4.407273in}{3.770405in}}%
\pgfpathlineto{\pgfqpoint{4.404349in}{3.773276in}}%
\pgfpathlineto{\pgfqpoint{4.401629in}{3.776000in}}%
\pgfpathlineto{\pgfqpoint{4.367192in}{3.810118in}}%
\pgfpathlineto{\pgfqpoint{4.365509in}{3.811765in}}%
\pgfpathlineto{\pgfqpoint{4.363938in}{3.813333in}}%
\pgfpathlineto{\pgfqpoint{4.345969in}{3.830898in}}%
\pgfpathlineto{\pgfqpoint{4.327111in}{3.849695in}}%
\pgfpathlineto{\pgfqpoint{4.326124in}{3.850667in}}%
\pgfpathlineto{\pgfqpoint{4.308645in}{3.867867in}}%
\pgfpathlineto{\pgfqpoint{4.288174in}{3.888000in}}%
\pgfpathlineto{\pgfqpoint{4.287615in}{3.888544in}}%
\pgfpathlineto{\pgfqpoint{4.287030in}{3.889124in}}%
\pgfpathlineto{\pgfqpoint{4.250090in}{3.925333in}}%
\pgfpathlineto{\pgfqpoint{4.246949in}{3.928409in}}%
\pgfpathlineto{\pgfqpoint{4.228910in}{3.945864in}}%
\pgfpathlineto{\pgfqpoint{4.211882in}{3.962667in}}%
\pgfpathlineto{\pgfqpoint{4.206869in}{3.967561in}}%
\pgfpathlineto{\pgfqpoint{4.173550in}{4.000000in}}%
\pgfpathlineto{\pgfqpoint{4.170226in}{4.003202in}}%
\pgfpathlineto{\pgfqpoint{4.166788in}{4.006579in}}%
\pgfpathlineto{\pgfqpoint{4.135092in}{4.037333in}}%
\pgfpathlineto{\pgfqpoint{4.130963in}{4.041298in}}%
\pgfpathlineto{\pgfqpoint{4.126707in}{4.045464in}}%
\pgfpathlineto{\pgfqpoint{4.111249in}{4.060268in}}%
\pgfpathlineto{\pgfqpoint{4.096508in}{4.074667in}}%
\pgfpathlineto{\pgfqpoint{4.086626in}{4.084216in}}%
\pgfpathlineto{\pgfqpoint{4.071894in}{4.098278in}}%
\pgfpathlineto{\pgfqpoint{4.057797in}{4.112000in}}%
\pgfpathlineto{\pgfqpoint{4.046545in}{4.122836in}}%
\pgfpathlineto{\pgfqpoint{4.018957in}{4.149333in}}%
\pgfpathlineto{\pgfqpoint{4.012774in}{4.155210in}}%
\pgfpathlineto{\pgfqpoint{4.006465in}{4.161324in}}%
\pgfpathlineto{\pgfqpoint{3.979988in}{4.186667in}}%
\pgfpathlineto{\pgfqpoint{3.973243in}{4.193055in}}%
\pgfpathlineto{\pgfqpoint{3.966384in}{4.199679in}}%
\pgfpathlineto{\pgfqpoint{3.940889in}{4.224000in}}%
\pgfpathlineto{\pgfqpoint{3.937936in}{4.224000in}}%
\pgfpathlineto{\pgfqpoint{3.966384in}{4.196863in}}%
\pgfpathlineto{\pgfqpoint{3.971758in}{4.191672in}}%
\pgfpathlineto{\pgfqpoint{3.977043in}{4.186667in}}%
\pgfpathlineto{\pgfqpoint{4.006465in}{4.158505in}}%
\pgfpathlineto{\pgfqpoint{4.011290in}{4.153828in}}%
\pgfpathlineto{\pgfqpoint{4.016020in}{4.149333in}}%
\pgfpathlineto{\pgfqpoint{4.046545in}{4.120015in}}%
\pgfpathlineto{\pgfqpoint{4.054868in}{4.112000in}}%
\pgfpathlineto{\pgfqpoint{4.070397in}{4.096883in}}%
\pgfpathlineto{\pgfqpoint{4.086626in}{4.081393in}}%
\pgfpathlineto{\pgfqpoint{4.093587in}{4.074667in}}%
\pgfpathlineto{\pgfqpoint{4.109754in}{4.058876in}}%
\pgfpathlineto{\pgfqpoint{4.126707in}{4.042639in}}%
\pgfpathlineto{\pgfqpoint{4.129484in}{4.039920in}}%
\pgfpathlineto{\pgfqpoint{4.132179in}{4.037333in}}%
\pgfpathlineto{\pgfqpoint{4.166788in}{4.003752in}}%
\pgfpathlineto{\pgfqpoint{4.168748in}{4.001826in}}%
\pgfpathlineto{\pgfqpoint{4.170644in}{4.000000in}}%
\pgfpathlineto{\pgfqpoint{4.206869in}{3.964732in}}%
\pgfpathlineto{\pgfqpoint{4.208984in}{3.962667in}}%
\pgfpathlineto{\pgfqpoint{4.227420in}{3.944475in}}%
\pgfpathlineto{\pgfqpoint{4.246949in}{3.925578in}}%
\pgfpathlineto{\pgfqpoint{4.247200in}{3.925333in}}%
\pgfpathlineto{\pgfqpoint{4.251980in}{3.920647in}}%
\pgfpathlineto{\pgfqpoint{4.285270in}{3.888000in}}%
\pgfpathlineto{\pgfqpoint{4.286123in}{3.887155in}}%
\pgfpathlineto{\pgfqpoint{4.287030in}{3.886273in}}%
\pgfpathlineto{\pgfqpoint{4.323214in}{3.850667in}}%
\pgfpathlineto{\pgfqpoint{4.327111in}{3.846829in}}%
\pgfpathlineto{\pgfqpoint{4.344482in}{3.829514in}}%
\pgfpathlineto{\pgfqpoint{4.361036in}{3.813333in}}%
\pgfpathlineto{\pgfqpoint{4.364007in}{3.810367in}}%
\pgfpathlineto{\pgfqpoint{4.367192in}{3.807250in}}%
\pgfpathlineto{\pgfqpoint{4.398735in}{3.776000in}}%
\pgfpathlineto{\pgfqpoint{4.402849in}{3.771879in}}%
\pgfpathlineto{\pgfqpoint{4.407273in}{3.767535in}}%
\pgfpathlineto{\pgfqpoint{4.436313in}{3.738667in}}%
\pgfpathlineto{\pgfqpoint{4.441624in}{3.733330in}}%
\pgfpathlineto{\pgfqpoint{4.447354in}{3.727684in}}%
\pgfpathlineto{\pgfqpoint{4.473772in}{3.701333in}}%
\pgfpathlineto{\pgfqpoint{4.487434in}{3.687696in}}%
\pgfpathlineto{\pgfqpoint{4.499638in}{3.675367in}}%
\pgfpathlineto{\pgfqpoint{4.511112in}{3.664000in}}%
\pgfpathlineto{\pgfqpoint{4.527515in}{3.647572in}}%
\pgfpathlineto{\pgfqpoint{4.538263in}{3.636678in}}%
\pgfpathlineto{\pgfqpoint{4.548333in}{3.626667in}}%
\pgfpathlineto{\pgfqpoint{4.567596in}{3.607310in}}%
\pgfpathlineto{\pgfqpoint{4.576822in}{3.597927in}}%
\pgfpathlineto{\pgfqpoint{4.585438in}{3.589333in}}%
\pgfpathlineto{\pgfqpoint{4.607677in}{3.566911in}}%
\pgfpathlineto{\pgfqpoint{4.615316in}{3.559116in}}%
\pgfpathlineto{\pgfqpoint{4.622426in}{3.552000in}}%
\pgfpathlineto{\pgfqpoint{4.647758in}{3.526374in}}%
\pgfpathlineto{\pgfqpoint{4.653745in}{3.520244in}}%
\pgfpathlineto{\pgfqpoint{4.659300in}{3.514667in}}%
\pgfpathlineto{\pgfqpoint{4.687838in}{3.485699in}}%
\pgfpathlineto{\pgfqpoint{4.692109in}{3.481312in}}%
\pgfpathlineto{\pgfqpoint{4.696058in}{3.477333in}}%
\pgfpathlineto{\pgfqpoint{4.727919in}{3.444886in}}%
\pgfpathlineto{\pgfqpoint{4.730409in}{3.442319in}}%
\pgfpathlineto{\pgfqpoint{4.732704in}{3.440000in}}%
\pgfpathlineto{\pgfqpoint{4.768000in}{3.403934in}}%
\pgfusepath{fill}%
\end{pgfscope}%
\begin{pgfscope}%
\pgfpathrectangle{\pgfqpoint{0.800000in}{0.528000in}}{\pgfqpoint{3.968000in}{3.696000in}}%
\pgfusepath{clip}%
\pgfsetbuttcap%
\pgfsetroundjoin%
\definecolor{currentfill}{rgb}{0.214000,0.722114,0.469588}%
\pgfsetfillcolor{currentfill}%
\pgfsetlinewidth{0.000000pt}%
\definecolor{currentstroke}{rgb}{0.000000,0.000000,0.000000}%
\pgfsetstrokecolor{currentstroke}%
\pgfsetdash{}{0pt}%
\pgfpathmoveto{\pgfqpoint{4.768000in}{3.409711in}}%
\pgfpathlineto{\pgfqpoint{4.738357in}{3.440000in}}%
\pgfpathlineto{\pgfqpoint{4.733351in}{3.445060in}}%
\pgfpathlineto{\pgfqpoint{4.727919in}{3.450658in}}%
\pgfpathlineto{\pgfqpoint{4.701727in}{3.477333in}}%
\pgfpathlineto{\pgfqpoint{4.695055in}{3.484055in}}%
\pgfpathlineto{\pgfqpoint{4.687838in}{3.491467in}}%
\pgfpathlineto{\pgfqpoint{4.664983in}{3.514667in}}%
\pgfpathlineto{\pgfqpoint{4.656693in}{3.522990in}}%
\pgfpathlineto{\pgfqpoint{4.647758in}{3.532138in}}%
\pgfpathlineto{\pgfqpoint{4.628124in}{3.552000in}}%
\pgfpathlineto{\pgfqpoint{4.618267in}{3.561864in}}%
\pgfpathlineto{\pgfqpoint{4.607677in}{3.572671in}}%
\pgfpathlineto{\pgfqpoint{4.591151in}{3.589333in}}%
\pgfpathlineto{\pgfqpoint{4.579776in}{3.600679in}}%
\pgfpathlineto{\pgfqpoint{4.567596in}{3.613066in}}%
\pgfpathlineto{\pgfqpoint{4.554061in}{3.626667in}}%
\pgfpathlineto{\pgfqpoint{4.541220in}{3.639432in}}%
\pgfpathlineto{\pgfqpoint{4.527515in}{3.653323in}}%
\pgfpathlineto{\pgfqpoint{4.516855in}{3.664000in}}%
\pgfpathlineto{\pgfqpoint{4.502598in}{3.678125in}}%
\pgfpathlineto{\pgfqpoint{4.487434in}{3.693444in}}%
\pgfpathlineto{\pgfqpoint{4.479530in}{3.701333in}}%
\pgfpathlineto{\pgfqpoint{4.447354in}{3.733428in}}%
\pgfpathlineto{\pgfqpoint{4.444620in}{3.736121in}}%
\pgfpathlineto{\pgfqpoint{4.442087in}{3.738667in}}%
\pgfpathlineto{\pgfqpoint{4.407273in}{3.773275in}}%
\pgfpathlineto{\pgfqpoint{4.405848in}{3.774673in}}%
\pgfpathlineto{\pgfqpoint{4.404524in}{3.776000in}}%
\pgfpathlineto{\pgfqpoint{4.367192in}{3.812986in}}%
\pgfpathlineto{\pgfqpoint{4.367010in}{3.813164in}}%
\pgfpathlineto{\pgfqpoint{4.366840in}{3.813333in}}%
\pgfpathlineto{\pgfqpoint{4.347455in}{3.832282in}}%
\pgfpathlineto{\pgfqpoint{4.329011in}{3.850667in}}%
\pgfpathlineto{\pgfqpoint{4.328084in}{3.851572in}}%
\pgfpathlineto{\pgfqpoint{4.327111in}{3.852540in}}%
\pgfpathlineto{\pgfqpoint{4.291057in}{3.888000in}}%
\pgfpathlineto{\pgfqpoint{4.289088in}{3.889916in}}%
\pgfpathlineto{\pgfqpoint{4.287030in}{3.891957in}}%
\pgfpathlineto{\pgfqpoint{4.252980in}{3.925333in}}%
\pgfpathlineto{\pgfqpoint{4.246949in}{3.931240in}}%
\pgfpathlineto{\pgfqpoint{4.230401in}{3.947252in}}%
\pgfpathlineto{\pgfqpoint{4.214780in}{3.962667in}}%
\pgfpathlineto{\pgfqpoint{4.206869in}{3.970390in}}%
\pgfpathlineto{\pgfqpoint{4.176455in}{4.000000in}}%
\pgfpathlineto{\pgfqpoint{4.171703in}{4.004578in}}%
\pgfpathlineto{\pgfqpoint{4.166788in}{4.009406in}}%
\pgfpathlineto{\pgfqpoint{4.138006in}{4.037333in}}%
\pgfpathlineto{\pgfqpoint{4.132442in}{4.042675in}}%
\pgfpathlineto{\pgfqpoint{4.126707in}{4.048289in}}%
\pgfpathlineto{\pgfqpoint{4.112745in}{4.061661in}}%
\pgfpathlineto{\pgfqpoint{4.099430in}{4.074667in}}%
\pgfpathlineto{\pgfqpoint{4.086626in}{4.087039in}}%
\pgfpathlineto{\pgfqpoint{4.073391in}{4.099672in}}%
\pgfpathlineto{\pgfqpoint{4.060726in}{4.112000in}}%
\pgfpathlineto{\pgfqpoint{4.046545in}{4.125657in}}%
\pgfpathlineto{\pgfqpoint{4.021894in}{4.149333in}}%
\pgfpathlineto{\pgfqpoint{4.014257in}{4.156591in}}%
\pgfpathlineto{\pgfqpoint{4.006465in}{4.164143in}}%
\pgfpathlineto{\pgfqpoint{3.982933in}{4.186667in}}%
\pgfpathlineto{\pgfqpoint{3.974727in}{4.194438in}}%
\pgfpathlineto{\pgfqpoint{3.966384in}{4.202496in}}%
\pgfpathlineto{\pgfqpoint{3.943842in}{4.224000in}}%
\pgfpathlineto{\pgfqpoint{3.940889in}{4.224000in}}%
\pgfpathlineto{\pgfqpoint{3.966384in}{4.199679in}}%
\pgfpathlineto{\pgfqpoint{3.973243in}{4.193055in}}%
\pgfpathlineto{\pgfqpoint{3.979988in}{4.186667in}}%
\pgfpathlineto{\pgfqpoint{4.006465in}{4.161324in}}%
\pgfpathlineto{\pgfqpoint{4.012774in}{4.155210in}}%
\pgfpathlineto{\pgfqpoint{4.018957in}{4.149333in}}%
\pgfpathlineto{\pgfqpoint{4.046545in}{4.122836in}}%
\pgfpathlineto{\pgfqpoint{4.057797in}{4.112000in}}%
\pgfpathlineto{\pgfqpoint{4.071894in}{4.098278in}}%
\pgfpathlineto{\pgfqpoint{4.086626in}{4.084216in}}%
\pgfpathlineto{\pgfqpoint{4.096508in}{4.074667in}}%
\pgfpathlineto{\pgfqpoint{4.111249in}{4.060268in}}%
\pgfpathlineto{\pgfqpoint{4.126707in}{4.045464in}}%
\pgfpathlineto{\pgfqpoint{4.130963in}{4.041298in}}%
\pgfpathlineto{\pgfqpoint{4.135092in}{4.037333in}}%
\pgfpathlineto{\pgfqpoint{4.166788in}{4.006579in}}%
\pgfpathlineto{\pgfqpoint{4.170226in}{4.003202in}}%
\pgfpathlineto{\pgfqpoint{4.173550in}{4.000000in}}%
\pgfpathlineto{\pgfqpoint{4.206869in}{3.967561in}}%
\pgfpathlineto{\pgfqpoint{4.211882in}{3.962667in}}%
\pgfpathlineto{\pgfqpoint{4.228910in}{3.945864in}}%
\pgfpathlineto{\pgfqpoint{4.246949in}{3.928409in}}%
\pgfpathlineto{\pgfqpoint{4.250090in}{3.925333in}}%
\pgfpathlineto{\pgfqpoint{4.287030in}{3.889124in}}%
\pgfpathlineto{\pgfqpoint{4.287615in}{3.888544in}}%
\pgfpathlineto{\pgfqpoint{4.288174in}{3.888000in}}%
\pgfpathlineto{\pgfqpoint{4.308645in}{3.867867in}}%
\pgfpathlineto{\pgfqpoint{4.326124in}{3.850667in}}%
\pgfpathlineto{\pgfqpoint{4.327111in}{3.849695in}}%
\pgfpathlineto{\pgfqpoint{4.345969in}{3.830898in}}%
\pgfpathlineto{\pgfqpoint{4.363938in}{3.813333in}}%
\pgfpathlineto{\pgfqpoint{4.365509in}{3.811765in}}%
\pgfpathlineto{\pgfqpoint{4.367192in}{3.810118in}}%
\pgfpathlineto{\pgfqpoint{4.401629in}{3.776000in}}%
\pgfpathlineto{\pgfqpoint{4.404349in}{3.773276in}}%
\pgfpathlineto{\pgfqpoint{4.407273in}{3.770405in}}%
\pgfpathlineto{\pgfqpoint{4.439200in}{3.738667in}}%
\pgfpathlineto{\pgfqpoint{4.443122in}{3.734725in}}%
\pgfpathlineto{\pgfqpoint{4.447354in}{3.730556in}}%
\pgfpathlineto{\pgfqpoint{4.476651in}{3.701333in}}%
\pgfpathlineto{\pgfqpoint{4.487434in}{3.690570in}}%
\pgfpathlineto{\pgfqpoint{4.501118in}{3.676746in}}%
\pgfpathlineto{\pgfqpoint{4.513983in}{3.664000in}}%
\pgfpathlineto{\pgfqpoint{4.527515in}{3.650447in}}%
\pgfpathlineto{\pgfqpoint{4.539741in}{3.638055in}}%
\pgfpathlineto{\pgfqpoint{4.551197in}{3.626667in}}%
\pgfpathlineto{\pgfqpoint{4.567596in}{3.610188in}}%
\pgfpathlineto{\pgfqpoint{4.578299in}{3.599303in}}%
\pgfpathlineto{\pgfqpoint{4.588294in}{3.589333in}}%
\pgfpathlineto{\pgfqpoint{4.607677in}{3.569791in}}%
\pgfpathlineto{\pgfqpoint{4.616792in}{3.560490in}}%
\pgfpathlineto{\pgfqpoint{4.625275in}{3.552000in}}%
\pgfpathlineto{\pgfqpoint{4.647758in}{3.529256in}}%
\pgfpathlineto{\pgfqpoint{4.655219in}{3.521617in}}%
\pgfpathlineto{\pgfqpoint{4.662141in}{3.514667in}}%
\pgfpathlineto{\pgfqpoint{4.687838in}{3.488583in}}%
\pgfpathlineto{\pgfqpoint{4.693582in}{3.482683in}}%
\pgfpathlineto{\pgfqpoint{4.698893in}{3.477333in}}%
\pgfpathlineto{\pgfqpoint{4.727919in}{3.447772in}}%
\pgfpathlineto{\pgfqpoint{4.731880in}{3.443690in}}%
\pgfpathlineto{\pgfqpoint{4.735531in}{3.440000in}}%
\pgfpathlineto{\pgfqpoint{4.768000in}{3.406822in}}%
\pgfusepath{fill}%
\end{pgfscope}%
\begin{pgfscope}%
\pgfpathrectangle{\pgfqpoint{0.800000in}{0.528000in}}{\pgfqpoint{3.968000in}{3.696000in}}%
\pgfusepath{clip}%
\pgfsetbuttcap%
\pgfsetroundjoin%
\definecolor{currentfill}{rgb}{0.214000,0.722114,0.469588}%
\pgfsetfillcolor{currentfill}%
\pgfsetlinewidth{0.000000pt}%
\definecolor{currentstroke}{rgb}{0.000000,0.000000,0.000000}%
\pgfsetstrokecolor{currentstroke}%
\pgfsetdash{}{0pt}%
\pgfpathmoveto{\pgfqpoint{4.768000in}{3.412599in}}%
\pgfpathlineto{\pgfqpoint{4.741184in}{3.440000in}}%
\pgfpathlineto{\pgfqpoint{4.734823in}{3.446430in}}%
\pgfpathlineto{\pgfqpoint{4.727919in}{3.453545in}}%
\pgfpathlineto{\pgfqpoint{4.704561in}{3.477333in}}%
\pgfpathlineto{\pgfqpoint{4.696527in}{3.485427in}}%
\pgfpathlineto{\pgfqpoint{4.687838in}{3.494352in}}%
\pgfpathlineto{\pgfqpoint{4.667824in}{3.514667in}}%
\pgfpathlineto{\pgfqpoint{4.658167in}{3.524363in}}%
\pgfpathlineto{\pgfqpoint{4.647758in}{3.535020in}}%
\pgfpathlineto{\pgfqpoint{4.630973in}{3.552000in}}%
\pgfpathlineto{\pgfqpoint{4.619743in}{3.563239in}}%
\pgfpathlineto{\pgfqpoint{4.607677in}{3.575551in}}%
\pgfpathlineto{\pgfqpoint{4.594007in}{3.589333in}}%
\pgfpathlineto{\pgfqpoint{4.581253in}{3.602054in}}%
\pgfpathlineto{\pgfqpoint{4.567596in}{3.615944in}}%
\pgfpathlineto{\pgfqpoint{4.556925in}{3.626667in}}%
\pgfpathlineto{\pgfqpoint{4.542698in}{3.640809in}}%
\pgfpathlineto{\pgfqpoint{4.527515in}{3.656199in}}%
\pgfpathlineto{\pgfqpoint{4.519726in}{3.664000in}}%
\pgfpathlineto{\pgfqpoint{4.504078in}{3.679503in}}%
\pgfpathlineto{\pgfqpoint{4.487434in}{3.696318in}}%
\pgfpathlineto{\pgfqpoint{4.482409in}{3.701333in}}%
\pgfpathlineto{\pgfqpoint{4.447354in}{3.736299in}}%
\pgfpathlineto{\pgfqpoint{4.446119in}{3.737516in}}%
\pgfpathlineto{\pgfqpoint{4.444974in}{3.738667in}}%
\pgfpathlineto{\pgfqpoint{4.409580in}{3.773851in}}%
\pgfpathlineto{\pgfqpoint{4.407417in}{3.776000in}}%
\pgfpathlineto{\pgfqpoint{4.407347in}{3.776069in}}%
\pgfpathlineto{\pgfqpoint{4.407273in}{3.776143in}}%
\pgfpathlineto{\pgfqpoint{4.369712in}{3.813333in}}%
\pgfpathlineto{\pgfqpoint{4.368484in}{3.814536in}}%
\pgfpathlineto{\pgfqpoint{4.367192in}{3.815826in}}%
\pgfpathlineto{\pgfqpoint{4.348941in}{3.833667in}}%
\pgfpathlineto{\pgfqpoint{4.331886in}{3.850667in}}%
\pgfpathlineto{\pgfqpoint{4.329555in}{3.852943in}}%
\pgfpathlineto{\pgfqpoint{4.327111in}{3.855375in}}%
\pgfpathlineto{\pgfqpoint{4.293939in}{3.888000in}}%
\pgfpathlineto{\pgfqpoint{4.290560in}{3.891288in}}%
\pgfpathlineto{\pgfqpoint{4.287030in}{3.894790in}}%
\pgfpathlineto{\pgfqpoint{4.255870in}{3.925333in}}%
\pgfpathlineto{\pgfqpoint{4.246949in}{3.934071in}}%
\pgfpathlineto{\pgfqpoint{4.231892in}{3.948641in}}%
\pgfpathlineto{\pgfqpoint{4.217678in}{3.962667in}}%
\pgfpathlineto{\pgfqpoint{4.206869in}{3.973219in}}%
\pgfpathlineto{\pgfqpoint{4.179361in}{4.000000in}}%
\pgfpathlineto{\pgfqpoint{4.173180in}{4.005954in}}%
\pgfpathlineto{\pgfqpoint{4.166788in}{4.012233in}}%
\pgfpathlineto{\pgfqpoint{4.140919in}{4.037333in}}%
\pgfpathlineto{\pgfqpoint{4.133921in}{4.044052in}}%
\pgfpathlineto{\pgfqpoint{4.126707in}{4.051114in}}%
\pgfpathlineto{\pgfqpoint{4.114240in}{4.063054in}}%
\pgfpathlineto{\pgfqpoint{4.102351in}{4.074667in}}%
\pgfpathlineto{\pgfqpoint{4.086626in}{4.089862in}}%
\pgfpathlineto{\pgfqpoint{4.074888in}{4.101066in}}%
\pgfpathlineto{\pgfqpoint{4.063655in}{4.112000in}}%
\pgfpathlineto{\pgfqpoint{4.046545in}{4.128478in}}%
\pgfpathlineto{\pgfqpoint{4.024831in}{4.149333in}}%
\pgfpathlineto{\pgfqpoint{4.015740in}{4.157973in}}%
\pgfpathlineto{\pgfqpoint{4.006465in}{4.166962in}}%
\pgfpathlineto{\pgfqpoint{3.985878in}{4.186667in}}%
\pgfpathlineto{\pgfqpoint{3.976212in}{4.195821in}}%
\pgfpathlineto{\pgfqpoint{3.966384in}{4.205313in}}%
\pgfpathlineto{\pgfqpoint{3.946795in}{4.224000in}}%
\pgfpathlineto{\pgfqpoint{3.943842in}{4.224000in}}%
\pgfpathlineto{\pgfqpoint{3.966384in}{4.202496in}}%
\pgfpathlineto{\pgfqpoint{3.974727in}{4.194438in}}%
\pgfpathlineto{\pgfqpoint{3.982933in}{4.186667in}}%
\pgfpathlineto{\pgfqpoint{4.006465in}{4.164143in}}%
\pgfpathlineto{\pgfqpoint{4.014257in}{4.156591in}}%
\pgfpathlineto{\pgfqpoint{4.021894in}{4.149333in}}%
\pgfpathlineto{\pgfqpoint{4.046545in}{4.125657in}}%
\pgfpathlineto{\pgfqpoint{4.060726in}{4.112000in}}%
\pgfpathlineto{\pgfqpoint{4.073391in}{4.099672in}}%
\pgfpathlineto{\pgfqpoint{4.086626in}{4.087039in}}%
\pgfpathlineto{\pgfqpoint{4.099430in}{4.074667in}}%
\pgfpathlineto{\pgfqpoint{4.112745in}{4.061661in}}%
\pgfpathlineto{\pgfqpoint{4.126707in}{4.048289in}}%
\pgfpathlineto{\pgfqpoint{4.132442in}{4.042675in}}%
\pgfpathlineto{\pgfqpoint{4.138006in}{4.037333in}}%
\pgfpathlineto{\pgfqpoint{4.166788in}{4.009406in}}%
\pgfpathlineto{\pgfqpoint{4.171703in}{4.004578in}}%
\pgfpathlineto{\pgfqpoint{4.176455in}{4.000000in}}%
\pgfpathlineto{\pgfqpoint{4.206869in}{3.970390in}}%
\pgfpathlineto{\pgfqpoint{4.214780in}{3.962667in}}%
\pgfpathlineto{\pgfqpoint{4.230401in}{3.947252in}}%
\pgfpathlineto{\pgfqpoint{4.246949in}{3.931240in}}%
\pgfpathlineto{\pgfqpoint{4.252980in}{3.925333in}}%
\pgfpathlineto{\pgfqpoint{4.287030in}{3.891957in}}%
\pgfpathlineto{\pgfqpoint{4.289088in}{3.889916in}}%
\pgfpathlineto{\pgfqpoint{4.291057in}{3.888000in}}%
\pgfpathlineto{\pgfqpoint{4.327111in}{3.852540in}}%
\pgfpathlineto{\pgfqpoint{4.328084in}{3.851572in}}%
\pgfpathlineto{\pgfqpoint{4.329011in}{3.850667in}}%
\pgfpathlineto{\pgfqpoint{4.347455in}{3.832282in}}%
\pgfpathlineto{\pgfqpoint{4.366840in}{3.813333in}}%
\pgfpathlineto{\pgfqpoint{4.367010in}{3.813164in}}%
\pgfpathlineto{\pgfqpoint{4.367192in}{3.812986in}}%
\pgfpathlineto{\pgfqpoint{4.404524in}{3.776000in}}%
\pgfpathlineto{\pgfqpoint{4.405848in}{3.774673in}}%
\pgfpathlineto{\pgfqpoint{4.407273in}{3.773275in}}%
\pgfpathlineto{\pgfqpoint{4.442087in}{3.738667in}}%
\pgfpathlineto{\pgfqpoint{4.444620in}{3.736121in}}%
\pgfpathlineto{\pgfqpoint{4.447354in}{3.733428in}}%
\pgfpathlineto{\pgfqpoint{4.479530in}{3.701333in}}%
\pgfpathlineto{\pgfqpoint{4.487434in}{3.693444in}}%
\pgfpathlineto{\pgfqpoint{4.502598in}{3.678125in}}%
\pgfpathlineto{\pgfqpoint{4.516855in}{3.664000in}}%
\pgfpathlineto{\pgfqpoint{4.527515in}{3.653323in}}%
\pgfpathlineto{\pgfqpoint{4.541220in}{3.639432in}}%
\pgfpathlineto{\pgfqpoint{4.554061in}{3.626667in}}%
\pgfpathlineto{\pgfqpoint{4.567596in}{3.613066in}}%
\pgfpathlineto{\pgfqpoint{4.579776in}{3.600679in}}%
\pgfpathlineto{\pgfqpoint{4.591151in}{3.589333in}}%
\pgfpathlineto{\pgfqpoint{4.607677in}{3.572671in}}%
\pgfpathlineto{\pgfqpoint{4.618267in}{3.561864in}}%
\pgfpathlineto{\pgfqpoint{4.628124in}{3.552000in}}%
\pgfpathlineto{\pgfqpoint{4.647758in}{3.532138in}}%
\pgfpathlineto{\pgfqpoint{4.656693in}{3.522990in}}%
\pgfpathlineto{\pgfqpoint{4.664983in}{3.514667in}}%
\pgfpathlineto{\pgfqpoint{4.687838in}{3.491467in}}%
\pgfpathlineto{\pgfqpoint{4.695055in}{3.484055in}}%
\pgfpathlineto{\pgfqpoint{4.701727in}{3.477333in}}%
\pgfpathlineto{\pgfqpoint{4.727919in}{3.450658in}}%
\pgfpathlineto{\pgfqpoint{4.733351in}{3.445060in}}%
\pgfpathlineto{\pgfqpoint{4.738357in}{3.440000in}}%
\pgfpathlineto{\pgfqpoint{4.768000in}{3.409711in}}%
\pgfusepath{fill}%
\end{pgfscope}%
\begin{pgfscope}%
\pgfpathrectangle{\pgfqpoint{0.800000in}{0.528000in}}{\pgfqpoint{3.968000in}{3.696000in}}%
\pgfusepath{clip}%
\pgfsetbuttcap%
\pgfsetroundjoin%
\definecolor{currentfill}{rgb}{0.214000,0.722114,0.469588}%
\pgfsetfillcolor{currentfill}%
\pgfsetlinewidth{0.000000pt}%
\definecolor{currentstroke}{rgb}{0.000000,0.000000,0.000000}%
\pgfsetstrokecolor{currentstroke}%
\pgfsetdash{}{0pt}%
\pgfpathmoveto{\pgfqpoint{4.768000in}{3.415487in}}%
\pgfpathlineto{\pgfqpoint{4.559789in}{3.626667in}}%
\pgfpathlineto{\pgfqpoint{4.206869in}{3.976047in}}%
\pgfpathlineto{\pgfqpoint{4.006465in}{4.169781in}}%
\pgfpathlineto{\pgfqpoint{3.949748in}{4.224000in}}%
\pgfpathlineto{\pgfqpoint{3.946795in}{4.224000in}}%
\pgfpathlineto{\pgfqpoint{4.140919in}{4.037333in}}%
\pgfpathlineto{\pgfqpoint{4.331886in}{3.850667in}}%
\pgfpathlineto{\pgfqpoint{4.687838in}{3.494352in}}%
\pgfpathlineto{\pgfqpoint{4.768000in}{3.412599in}}%
\pgfpathlineto{\pgfqpoint{4.768000in}{3.412599in}}%
\pgfusepath{fill}%
\end{pgfscope}%
\begin{pgfscope}%
\pgfpathrectangle{\pgfqpoint{0.800000in}{0.528000in}}{\pgfqpoint{3.968000in}{3.696000in}}%
\pgfusepath{clip}%
\pgfsetbuttcap%
\pgfsetroundjoin%
\definecolor{currentfill}{rgb}{0.220124,0.725509,0.466226}%
\pgfsetfillcolor{currentfill}%
\pgfsetlinewidth{0.000000pt}%
\definecolor{currentstroke}{rgb}{0.000000,0.000000,0.000000}%
\pgfsetstrokecolor{currentstroke}%
\pgfsetdash{}{0pt}%
\pgfpathmoveto{\pgfqpoint{4.768000in}{3.418376in}}%
\pgfpathlineto{\pgfqpoint{4.746838in}{3.440000in}}%
\pgfpathlineto{\pgfqpoint{4.737765in}{3.449171in}}%
\pgfpathlineto{\pgfqpoint{4.727919in}{3.459317in}}%
\pgfpathlineto{\pgfqpoint{4.710229in}{3.477333in}}%
\pgfpathlineto{\pgfqpoint{4.699472in}{3.488170in}}%
\pgfpathlineto{\pgfqpoint{4.687838in}{3.500120in}}%
\pgfpathlineto{\pgfqpoint{4.673507in}{3.514667in}}%
\pgfpathlineto{\pgfqpoint{4.661116in}{3.527109in}}%
\pgfpathlineto{\pgfqpoint{4.647758in}{3.540784in}}%
\pgfpathlineto{\pgfqpoint{4.636671in}{3.552000in}}%
\pgfpathlineto{\pgfqpoint{4.622694in}{3.565988in}}%
\pgfpathlineto{\pgfqpoint{4.607677in}{3.581311in}}%
\pgfpathlineto{\pgfqpoint{4.599720in}{3.589333in}}%
\pgfpathlineto{\pgfqpoint{4.584207in}{3.604806in}}%
\pgfpathlineto{\pgfqpoint{4.567596in}{3.621700in}}%
\pgfpathlineto{\pgfqpoint{4.562653in}{3.626667in}}%
\pgfpathlineto{\pgfqpoint{4.545656in}{3.643564in}}%
\pgfpathlineto{\pgfqpoint{4.527515in}{3.661951in}}%
\pgfpathlineto{\pgfqpoint{4.525469in}{3.664000in}}%
\pgfpathlineto{\pgfqpoint{4.507039in}{3.682260in}}%
\pgfpathlineto{\pgfqpoint{4.488159in}{3.701333in}}%
\pgfpathlineto{\pgfqpoint{4.487434in}{3.702058in}}%
\pgfpathlineto{\pgfqpoint{4.450707in}{3.738667in}}%
\pgfpathlineto{\pgfqpoint{4.449078in}{3.740273in}}%
\pgfpathlineto{\pgfqpoint{4.447354in}{3.742007in}}%
\pgfpathlineto{\pgfqpoint{4.413136in}{3.776000in}}%
\pgfpathlineto{\pgfqpoint{4.410283in}{3.778804in}}%
\pgfpathlineto{\pgfqpoint{4.407273in}{3.781821in}}%
\pgfpathlineto{\pgfqpoint{4.375446in}{3.813333in}}%
\pgfpathlineto{\pgfqpoint{4.371423in}{3.817274in}}%
\pgfpathlineto{\pgfqpoint{4.367192in}{3.821500in}}%
\pgfpathlineto{\pgfqpoint{4.351913in}{3.836435in}}%
\pgfpathlineto{\pgfqpoint{4.337636in}{3.850667in}}%
\pgfpathlineto{\pgfqpoint{4.332497in}{3.855684in}}%
\pgfpathlineto{\pgfqpoint{4.327111in}{3.861045in}}%
\pgfpathlineto{\pgfqpoint{4.299704in}{3.888000in}}%
\pgfpathlineto{\pgfqpoint{4.293506in}{3.894032in}}%
\pgfpathlineto{\pgfqpoint{4.287030in}{3.900456in}}%
\pgfpathlineto{\pgfqpoint{4.261650in}{3.925333in}}%
\pgfpathlineto{\pgfqpoint{4.246949in}{3.939733in}}%
\pgfpathlineto{\pgfqpoint{4.234873in}{3.951418in}}%
\pgfpathlineto{\pgfqpoint{4.223473in}{3.962667in}}%
\pgfpathlineto{\pgfqpoint{4.206869in}{3.978876in}}%
\pgfpathlineto{\pgfqpoint{4.185172in}{4.000000in}}%
\pgfpathlineto{\pgfqpoint{4.176135in}{4.008706in}}%
\pgfpathlineto{\pgfqpoint{4.166788in}{4.017886in}}%
\pgfpathlineto{\pgfqpoint{4.146746in}{4.037333in}}%
\pgfpathlineto{\pgfqpoint{4.136878in}{4.046807in}}%
\pgfpathlineto{\pgfqpoint{4.126707in}{4.056763in}}%
\pgfpathlineto{\pgfqpoint{4.117230in}{4.065840in}}%
\pgfpathlineto{\pgfqpoint{4.108193in}{4.074667in}}%
\pgfpathlineto{\pgfqpoint{4.086626in}{4.095508in}}%
\pgfpathlineto{\pgfqpoint{4.077881in}{4.103855in}}%
\pgfpathlineto{\pgfqpoint{4.069514in}{4.112000in}}%
\pgfpathlineto{\pgfqpoint{4.046545in}{4.134120in}}%
\pgfpathlineto{\pgfqpoint{4.030706in}{4.149333in}}%
\pgfpathlineto{\pgfqpoint{4.018707in}{4.160736in}}%
\pgfpathlineto{\pgfqpoint{4.006465in}{4.172599in}}%
\pgfpathlineto{\pgfqpoint{3.991768in}{4.186667in}}%
\pgfpathlineto{\pgfqpoint{3.979182in}{4.198587in}}%
\pgfpathlineto{\pgfqpoint{3.966384in}{4.210947in}}%
\pgfpathlineto{\pgfqpoint{3.952701in}{4.224000in}}%
\pgfpathlineto{\pgfqpoint{3.949748in}{4.224000in}}%
\pgfpathlineto{\pgfqpoint{3.966384in}{4.208130in}}%
\pgfpathlineto{\pgfqpoint{3.977697in}{4.197204in}}%
\pgfpathlineto{\pgfqpoint{3.988823in}{4.186667in}}%
\pgfpathlineto{\pgfqpoint{4.006465in}{4.169781in}}%
\pgfpathlineto{\pgfqpoint{4.017223in}{4.159355in}}%
\pgfpathlineto{\pgfqpoint{4.027769in}{4.149333in}}%
\pgfpathlineto{\pgfqpoint{4.046545in}{4.131299in}}%
\pgfpathlineto{\pgfqpoint{4.066584in}{4.112000in}}%
\pgfpathlineto{\pgfqpoint{4.076384in}{4.102460in}}%
\pgfpathlineto{\pgfqpoint{4.086626in}{4.092685in}}%
\pgfpathlineto{\pgfqpoint{4.105272in}{4.074667in}}%
\pgfpathlineto{\pgfqpoint{4.115735in}{4.064447in}}%
\pgfpathlineto{\pgfqpoint{4.126707in}{4.053939in}}%
\pgfpathlineto{\pgfqpoint{4.135399in}{4.045430in}}%
\pgfpathlineto{\pgfqpoint{4.143833in}{4.037333in}}%
\pgfpathlineto{\pgfqpoint{4.166788in}{4.015060in}}%
\pgfpathlineto{\pgfqpoint{4.174657in}{4.007330in}}%
\pgfpathlineto{\pgfqpoint{4.182267in}{4.000000in}}%
\pgfpathlineto{\pgfqpoint{4.206869in}{3.976047in}}%
\pgfpathlineto{\pgfqpoint{4.220576in}{3.962667in}}%
\pgfpathlineto{\pgfqpoint{4.233382in}{3.950030in}}%
\pgfpathlineto{\pgfqpoint{4.246949in}{3.936902in}}%
\pgfpathlineto{\pgfqpoint{4.258760in}{3.925333in}}%
\pgfpathlineto{\pgfqpoint{4.287030in}{3.897623in}}%
\pgfpathlineto{\pgfqpoint{4.292033in}{3.892660in}}%
\pgfpathlineto{\pgfqpoint{4.296822in}{3.888000in}}%
\pgfpathlineto{\pgfqpoint{4.327111in}{3.858210in}}%
\pgfpathlineto{\pgfqpoint{4.331026in}{3.854313in}}%
\pgfpathlineto{\pgfqpoint{4.334761in}{3.850667in}}%
\pgfpathlineto{\pgfqpoint{4.350427in}{3.835051in}}%
\pgfpathlineto{\pgfqpoint{4.367192in}{3.818663in}}%
\pgfpathlineto{\pgfqpoint{4.369953in}{3.815905in}}%
\pgfpathlineto{\pgfqpoint{4.372579in}{3.813333in}}%
\pgfpathlineto{\pgfqpoint{4.407273in}{3.778982in}}%
\pgfpathlineto{\pgfqpoint{4.408815in}{3.777436in}}%
\pgfpathlineto{\pgfqpoint{4.410276in}{3.776000in}}%
\pgfpathlineto{\pgfqpoint{4.447354in}{3.739166in}}%
\pgfpathlineto{\pgfqpoint{4.447611in}{3.738907in}}%
\pgfpathlineto{\pgfqpoint{4.447855in}{3.738667in}}%
\pgfpathlineto{\pgfqpoint{4.455000in}{3.731545in}}%
\pgfpathlineto{\pgfqpoint{4.485289in}{3.701333in}}%
\pgfpathlineto{\pgfqpoint{4.487434in}{3.699192in}}%
\pgfpathlineto{\pgfqpoint{4.505559in}{3.680882in}}%
\pgfpathlineto{\pgfqpoint{4.522598in}{3.664000in}}%
\pgfpathlineto{\pgfqpoint{4.527515in}{3.659075in}}%
\pgfpathlineto{\pgfqpoint{4.544177in}{3.642186in}}%
\pgfpathlineto{\pgfqpoint{4.559789in}{3.626667in}}%
\pgfpathlineto{\pgfqpoint{4.567596in}{3.618822in}}%
\pgfpathlineto{\pgfqpoint{4.582730in}{3.603430in}}%
\pgfpathlineto{\pgfqpoint{4.596864in}{3.589333in}}%
\pgfpathlineto{\pgfqpoint{4.607677in}{3.578431in}}%
\pgfpathlineto{\pgfqpoint{4.621218in}{3.564613in}}%
\pgfpathlineto{\pgfqpoint{4.633822in}{3.552000in}}%
\pgfpathlineto{\pgfqpoint{4.647758in}{3.537902in}}%
\pgfpathlineto{\pgfqpoint{4.659641in}{3.525736in}}%
\pgfpathlineto{\pgfqpoint{4.670666in}{3.514667in}}%
\pgfpathlineto{\pgfqpoint{4.687838in}{3.497236in}}%
\pgfpathlineto{\pgfqpoint{4.698000in}{3.486798in}}%
\pgfpathlineto{\pgfqpoint{4.707395in}{3.477333in}}%
\pgfpathlineto{\pgfqpoint{4.727919in}{3.456431in}}%
\pgfpathlineto{\pgfqpoint{4.736294in}{3.447800in}}%
\pgfpathlineto{\pgfqpoint{4.744011in}{3.440000in}}%
\pgfpathlineto{\pgfqpoint{4.768000in}{3.415487in}}%
\pgfusepath{fill}%
\end{pgfscope}%
\begin{pgfscope}%
\pgfpathrectangle{\pgfqpoint{0.800000in}{0.528000in}}{\pgfqpoint{3.968000in}{3.696000in}}%
\pgfusepath{clip}%
\pgfsetbuttcap%
\pgfsetroundjoin%
\definecolor{currentfill}{rgb}{0.220124,0.725509,0.466226}%
\pgfsetfillcolor{currentfill}%
\pgfsetlinewidth{0.000000pt}%
\definecolor{currentstroke}{rgb}{0.000000,0.000000,0.000000}%
\pgfsetstrokecolor{currentstroke}%
\pgfsetdash{}{0pt}%
\pgfpathmoveto{\pgfqpoint{4.768000in}{3.421264in}}%
\pgfpathlineto{\pgfqpoint{4.749664in}{3.440000in}}%
\pgfpathlineto{\pgfqpoint{4.739236in}{3.450541in}}%
\pgfpathlineto{\pgfqpoint{4.727919in}{3.462204in}}%
\pgfpathlineto{\pgfqpoint{4.713063in}{3.477333in}}%
\pgfpathlineto{\pgfqpoint{4.700945in}{3.489542in}}%
\pgfpathlineto{\pgfqpoint{4.687838in}{3.503004in}}%
\pgfpathlineto{\pgfqpoint{4.676349in}{3.514667in}}%
\pgfpathlineto{\pgfqpoint{4.662590in}{3.528482in}}%
\pgfpathlineto{\pgfqpoint{4.647758in}{3.543667in}}%
\pgfpathlineto{\pgfqpoint{4.639520in}{3.552000in}}%
\pgfpathlineto{\pgfqpoint{4.624169in}{3.567362in}}%
\pgfpathlineto{\pgfqpoint{4.607677in}{3.584191in}}%
\pgfpathlineto{\pgfqpoint{4.602577in}{3.589333in}}%
\pgfpathlineto{\pgfqpoint{4.585684in}{3.606182in}}%
\pgfpathlineto{\pgfqpoint{4.567596in}{3.624578in}}%
\pgfpathlineto{\pgfqpoint{4.565517in}{3.626667in}}%
\pgfpathlineto{\pgfqpoint{4.547134in}{3.644941in}}%
\pgfpathlineto{\pgfqpoint{4.528331in}{3.664000in}}%
\pgfpathlineto{\pgfqpoint{4.527936in}{3.664392in}}%
\pgfpathlineto{\pgfqpoint{4.527515in}{3.664818in}}%
\pgfpathlineto{\pgfqpoint{4.508519in}{3.683639in}}%
\pgfpathlineto{\pgfqpoint{4.491004in}{3.701333in}}%
\pgfpathlineto{\pgfqpoint{4.487434in}{3.704901in}}%
\pgfpathlineto{\pgfqpoint{4.453559in}{3.738667in}}%
\pgfpathlineto{\pgfqpoint{4.450545in}{3.741639in}}%
\pgfpathlineto{\pgfqpoint{4.447354in}{3.744848in}}%
\pgfpathlineto{\pgfqpoint{4.415996in}{3.776000in}}%
\pgfpathlineto{\pgfqpoint{4.411751in}{3.780172in}}%
\pgfpathlineto{\pgfqpoint{4.407273in}{3.784660in}}%
\pgfpathlineto{\pgfqpoint{4.378313in}{3.813333in}}%
\pgfpathlineto{\pgfqpoint{4.372893in}{3.818643in}}%
\pgfpathlineto{\pgfqpoint{4.367192in}{3.824337in}}%
\pgfpathlineto{\pgfqpoint{4.353399in}{3.837819in}}%
\pgfpathlineto{\pgfqpoint{4.340511in}{3.850667in}}%
\pgfpathlineto{\pgfqpoint{4.333969in}{3.857054in}}%
\pgfpathlineto{\pgfqpoint{4.327111in}{3.863880in}}%
\pgfpathlineto{\pgfqpoint{4.302587in}{3.888000in}}%
\pgfpathlineto{\pgfqpoint{4.294979in}{3.895403in}}%
\pgfpathlineto{\pgfqpoint{4.287030in}{3.903289in}}%
\pgfpathlineto{\pgfqpoint{4.264541in}{3.925333in}}%
\pgfpathlineto{\pgfqpoint{4.246949in}{3.942564in}}%
\pgfpathlineto{\pgfqpoint{4.236364in}{3.952807in}}%
\pgfpathlineto{\pgfqpoint{4.226371in}{3.962667in}}%
\pgfpathlineto{\pgfqpoint{4.206869in}{3.981705in}}%
\pgfpathlineto{\pgfqpoint{4.188078in}{4.000000in}}%
\pgfpathlineto{\pgfqpoint{4.177612in}{4.010082in}}%
\pgfpathlineto{\pgfqpoint{4.166788in}{4.020713in}}%
\pgfpathlineto{\pgfqpoint{4.149659in}{4.037333in}}%
\pgfpathlineto{\pgfqpoint{4.138357in}{4.048184in}}%
\pgfpathlineto{\pgfqpoint{4.126707in}{4.059588in}}%
\pgfpathlineto{\pgfqpoint{4.118726in}{4.067232in}}%
\pgfpathlineto{\pgfqpoint{4.111115in}{4.074667in}}%
\pgfpathlineto{\pgfqpoint{4.086626in}{4.098331in}}%
\pgfpathlineto{\pgfqpoint{4.079378in}{4.105249in}}%
\pgfpathlineto{\pgfqpoint{4.072443in}{4.112000in}}%
\pgfpathlineto{\pgfqpoint{4.046545in}{4.136941in}}%
\pgfpathlineto{\pgfqpoint{4.033643in}{4.149333in}}%
\pgfpathlineto{\pgfqpoint{4.020190in}{4.162118in}}%
\pgfpathlineto{\pgfqpoint{4.006465in}{4.175418in}}%
\pgfpathlineto{\pgfqpoint{3.994713in}{4.186667in}}%
\pgfpathlineto{\pgfqpoint{3.980667in}{4.199970in}}%
\pgfpathlineto{\pgfqpoint{3.966384in}{4.213764in}}%
\pgfpathlineto{\pgfqpoint{3.955654in}{4.224000in}}%
\pgfpathlineto{\pgfqpoint{3.952701in}{4.224000in}}%
\pgfpathlineto{\pgfqpoint{3.966384in}{4.210947in}}%
\pgfpathlineto{\pgfqpoint{3.979182in}{4.198587in}}%
\pgfpathlineto{\pgfqpoint{3.991768in}{4.186667in}}%
\pgfpathlineto{\pgfqpoint{4.006465in}{4.172599in}}%
\pgfpathlineto{\pgfqpoint{4.018707in}{4.160736in}}%
\pgfpathlineto{\pgfqpoint{4.030706in}{4.149333in}}%
\pgfpathlineto{\pgfqpoint{4.046545in}{4.134120in}}%
\pgfpathlineto{\pgfqpoint{4.069514in}{4.112000in}}%
\pgfpathlineto{\pgfqpoint{4.077881in}{4.103855in}}%
\pgfpathlineto{\pgfqpoint{4.086626in}{4.095508in}}%
\pgfpathlineto{\pgfqpoint{4.108193in}{4.074667in}}%
\pgfpathlineto{\pgfqpoint{4.117230in}{4.065840in}}%
\pgfpathlineto{\pgfqpoint{4.126707in}{4.056763in}}%
\pgfpathlineto{\pgfqpoint{4.136878in}{4.046807in}}%
\pgfpathlineto{\pgfqpoint{4.146746in}{4.037333in}}%
\pgfpathlineto{\pgfqpoint{4.166788in}{4.017886in}}%
\pgfpathlineto{\pgfqpoint{4.176135in}{4.008706in}}%
\pgfpathlineto{\pgfqpoint{4.185172in}{4.000000in}}%
\pgfpathlineto{\pgfqpoint{4.206869in}{3.978876in}}%
\pgfpathlineto{\pgfqpoint{4.223473in}{3.962667in}}%
\pgfpathlineto{\pgfqpoint{4.234873in}{3.951418in}}%
\pgfpathlineto{\pgfqpoint{4.246949in}{3.939733in}}%
\pgfpathlineto{\pgfqpoint{4.261650in}{3.925333in}}%
\pgfpathlineto{\pgfqpoint{4.287030in}{3.900456in}}%
\pgfpathlineto{\pgfqpoint{4.293506in}{3.894032in}}%
\pgfpathlineto{\pgfqpoint{4.299704in}{3.888000in}}%
\pgfpathlineto{\pgfqpoint{4.327111in}{3.861045in}}%
\pgfpathlineto{\pgfqpoint{4.332497in}{3.855684in}}%
\pgfpathlineto{\pgfqpoint{4.337636in}{3.850667in}}%
\pgfpathlineto{\pgfqpoint{4.351913in}{3.836435in}}%
\pgfpathlineto{\pgfqpoint{4.367192in}{3.821500in}}%
\pgfpathlineto{\pgfqpoint{4.371423in}{3.817274in}}%
\pgfpathlineto{\pgfqpoint{4.375446in}{3.813333in}}%
\pgfpathlineto{\pgfqpoint{4.407273in}{3.781821in}}%
\pgfpathlineto{\pgfqpoint{4.410283in}{3.778804in}}%
\pgfpathlineto{\pgfqpoint{4.413136in}{3.776000in}}%
\pgfpathlineto{\pgfqpoint{4.447354in}{3.742007in}}%
\pgfpathlineto{\pgfqpoint{4.449078in}{3.740273in}}%
\pgfpathlineto{\pgfqpoint{4.450707in}{3.738667in}}%
\pgfpathlineto{\pgfqpoint{4.487434in}{3.702058in}}%
\pgfpathlineto{\pgfqpoint{4.488159in}{3.701333in}}%
\pgfpathlineto{\pgfqpoint{4.507039in}{3.682260in}}%
\pgfpathlineto{\pgfqpoint{4.525469in}{3.664000in}}%
\pgfpathlineto{\pgfqpoint{4.527515in}{3.661951in}}%
\pgfpathlineto{\pgfqpoint{4.545656in}{3.643564in}}%
\pgfpathlineto{\pgfqpoint{4.562653in}{3.626667in}}%
\pgfpathlineto{\pgfqpoint{4.567596in}{3.621700in}}%
\pgfpathlineto{\pgfqpoint{4.584207in}{3.604806in}}%
\pgfpathlineto{\pgfqpoint{4.599720in}{3.589333in}}%
\pgfpathlineto{\pgfqpoint{4.607677in}{3.581311in}}%
\pgfpathlineto{\pgfqpoint{4.622694in}{3.565988in}}%
\pgfpathlineto{\pgfqpoint{4.636671in}{3.552000in}}%
\pgfpathlineto{\pgfqpoint{4.647758in}{3.540784in}}%
\pgfpathlineto{\pgfqpoint{4.661116in}{3.527109in}}%
\pgfpathlineto{\pgfqpoint{4.673507in}{3.514667in}}%
\pgfpathlineto{\pgfqpoint{4.687838in}{3.500120in}}%
\pgfpathlineto{\pgfqpoint{4.699472in}{3.488170in}}%
\pgfpathlineto{\pgfqpoint{4.710229in}{3.477333in}}%
\pgfpathlineto{\pgfqpoint{4.727919in}{3.459317in}}%
\pgfpathlineto{\pgfqpoint{4.737765in}{3.449171in}}%
\pgfpathlineto{\pgfqpoint{4.746838in}{3.440000in}}%
\pgfpathlineto{\pgfqpoint{4.768000in}{3.418376in}}%
\pgfusepath{fill}%
\end{pgfscope}%
\begin{pgfscope}%
\pgfpathrectangle{\pgfqpoint{0.800000in}{0.528000in}}{\pgfqpoint{3.968000in}{3.696000in}}%
\pgfusepath{clip}%
\pgfsetbuttcap%
\pgfsetroundjoin%
\definecolor{currentfill}{rgb}{0.220124,0.725509,0.466226}%
\pgfsetfillcolor{currentfill}%
\pgfsetlinewidth{0.000000pt}%
\definecolor{currentstroke}{rgb}{0.000000,0.000000,0.000000}%
\pgfsetstrokecolor{currentstroke}%
\pgfsetdash{}{0pt}%
\pgfpathmoveto{\pgfqpoint{4.768000in}{3.424153in}}%
\pgfpathlineto{\pgfqpoint{4.567596in}{3.627447in}}%
\pgfpathlineto{\pgfqpoint{4.206869in}{3.984534in}}%
\pgfpathlineto{\pgfqpoint{4.006465in}{4.178237in}}%
\pgfpathlineto{\pgfqpoint{3.958607in}{4.224000in}}%
\pgfpathlineto{\pgfqpoint{3.955654in}{4.224000in}}%
\pgfpathlineto{\pgfqpoint{4.138357in}{4.048184in}}%
\pgfpathlineto{\pgfqpoint{4.246949in}{3.942564in}}%
\pgfpathlineto{\pgfqpoint{4.453559in}{3.738667in}}%
\pgfpathlineto{\pgfqpoint{4.647758in}{3.543667in}}%
\pgfpathlineto{\pgfqpoint{4.768000in}{3.421264in}}%
\pgfpathlineto{\pgfqpoint{4.768000in}{3.421264in}}%
\pgfusepath{fill}%
\end{pgfscope}%
\begin{pgfscope}%
\pgfpathrectangle{\pgfqpoint{0.800000in}{0.528000in}}{\pgfqpoint{3.968000in}{3.696000in}}%
\pgfusepath{clip}%
\pgfsetbuttcap%
\pgfsetroundjoin%
\definecolor{currentfill}{rgb}{0.220124,0.725509,0.466226}%
\pgfsetfillcolor{currentfill}%
\pgfsetlinewidth{0.000000pt}%
\definecolor{currentstroke}{rgb}{0.000000,0.000000,0.000000}%
\pgfsetstrokecolor{currentstroke}%
\pgfsetdash{}{0pt}%
\pgfpathmoveto{\pgfqpoint{4.768000in}{3.427041in}}%
\pgfpathlineto{\pgfqpoint{4.567596in}{3.630294in}}%
\pgfpathlineto{\pgfqpoint{4.206869in}{3.987363in}}%
\pgfpathlineto{\pgfqpoint{4.000603in}{4.186667in}}%
\pgfpathlineto{\pgfqpoint{3.961560in}{4.224000in}}%
\pgfpathlineto{\pgfqpoint{3.958607in}{4.224000in}}%
\pgfpathlineto{\pgfqpoint{4.126707in}{4.062413in}}%
\pgfpathlineto{\pgfqpoint{4.487434in}{3.707744in}}%
\pgfpathlineto{\pgfqpoint{4.679190in}{3.514667in}}%
\pgfpathlineto{\pgfqpoint{4.768000in}{3.424153in}}%
\pgfpathlineto{\pgfqpoint{4.768000in}{3.424153in}}%
\pgfusepath{fill}%
\end{pgfscope}%
\begin{pgfscope}%
\pgfpathrectangle{\pgfqpoint{0.800000in}{0.528000in}}{\pgfqpoint{3.968000in}{3.696000in}}%
\pgfusepath{clip}%
\pgfsetbuttcap%
\pgfsetroundjoin%
\definecolor{currentfill}{rgb}{0.226397,0.728888,0.462789}%
\pgfsetfillcolor{currentfill}%
\pgfsetlinewidth{0.000000pt}%
\definecolor{currentstroke}{rgb}{0.000000,0.000000,0.000000}%
\pgfsetstrokecolor{currentstroke}%
\pgfsetdash{}{0pt}%
\pgfpathmoveto{\pgfqpoint{4.768000in}{3.429929in}}%
\pgfpathlineto{\pgfqpoint{4.567596in}{3.633141in}}%
\pgfpathlineto{\pgfqpoint{4.206869in}{3.990192in}}%
\pgfpathlineto{\pgfqpoint{4.003548in}{4.186667in}}%
\pgfpathlineto{\pgfqpoint{3.964513in}{4.224000in}}%
\pgfpathlineto{\pgfqpoint{3.961560in}{4.224000in}}%
\pgfpathlineto{\pgfqpoint{4.126707in}{4.065238in}}%
\pgfpathlineto{\pgfqpoint{4.487434in}{3.710587in}}%
\pgfpathlineto{\pgfqpoint{4.687838in}{3.508773in}}%
\pgfpathlineto{\pgfqpoint{4.768000in}{3.427041in}}%
\pgfpathlineto{\pgfqpoint{4.768000in}{3.427041in}}%
\pgfusepath{fill}%
\end{pgfscope}%
\begin{pgfscope}%
\pgfpathrectangle{\pgfqpoint{0.800000in}{0.528000in}}{\pgfqpoint{3.968000in}{3.696000in}}%
\pgfusepath{clip}%
\pgfsetbuttcap%
\pgfsetroundjoin%
\definecolor{currentfill}{rgb}{0.226397,0.728888,0.462789}%
\pgfsetfillcolor{currentfill}%
\pgfsetlinewidth{0.000000pt}%
\definecolor{currentstroke}{rgb}{0.000000,0.000000,0.000000}%
\pgfsetstrokecolor{currentstroke}%
\pgfsetdash{}{0pt}%
\pgfpathmoveto{\pgfqpoint{4.768000in}{3.432818in}}%
\pgfpathlineto{\pgfqpoint{4.576862in}{3.626667in}}%
\pgfpathlineto{\pgfqpoint{4.389782in}{3.813333in}}%
\pgfpathlineto{\pgfqpoint{4.045391in}{4.149333in}}%
\pgfpathlineto{\pgfqpoint{3.966384in}{4.224000in}}%
\pgfpathlineto{\pgfqpoint{3.964513in}{4.224000in}}%
\pgfpathlineto{\pgfqpoint{4.126707in}{4.068063in}}%
\pgfpathlineto{\pgfqpoint{4.487434in}{3.713430in}}%
\pgfpathlineto{\pgfqpoint{4.687838in}{3.511657in}}%
\pgfpathlineto{\pgfqpoint{4.768000in}{3.429929in}}%
\pgfpathlineto{\pgfqpoint{4.768000in}{3.429929in}}%
\pgfusepath{fill}%
\end{pgfscope}%
\begin{pgfscope}%
\pgfpathrectangle{\pgfqpoint{0.800000in}{0.528000in}}{\pgfqpoint{3.968000in}{3.696000in}}%
\pgfusepath{clip}%
\pgfsetbuttcap%
\pgfsetroundjoin%
\definecolor{currentfill}{rgb}{0.226397,0.728888,0.462789}%
\pgfsetfillcolor{currentfill}%
\pgfsetlinewidth{0.000000pt}%
\definecolor{currentstroke}{rgb}{0.000000,0.000000,0.000000}%
\pgfsetstrokecolor{currentstroke}%
\pgfsetdash{}{0pt}%
\pgfpathmoveto{\pgfqpoint{4.768000in}{3.435706in}}%
\pgfpathlineto{\pgfqpoint{4.579692in}{3.626667in}}%
\pgfpathlineto{\pgfqpoint{4.392650in}{3.813333in}}%
\pgfpathlineto{\pgfqpoint{4.046545in}{4.151027in}}%
\pgfpathlineto{\pgfqpoint{3.970369in}{4.224000in}}%
\pgfpathlineto{\pgfqpoint{3.967452in}{4.224000in}}%
\pgfpathlineto{\pgfqpoint{4.327111in}{3.875220in}}%
\pgfpathlineto{\pgfqpoint{4.668486in}{3.533974in}}%
\pgfpathlineto{\pgfqpoint{4.768000in}{3.432818in}}%
\pgfpathlineto{\pgfqpoint{4.768000in}{3.432818in}}%
\pgfusepath{fill}%
\end{pgfscope}%
\begin{pgfscope}%
\pgfpathrectangle{\pgfqpoint{0.800000in}{0.528000in}}{\pgfqpoint{3.968000in}{3.696000in}}%
\pgfusepath{clip}%
\pgfsetbuttcap%
\pgfsetroundjoin%
\definecolor{currentfill}{rgb}{0.226397,0.728888,0.462789}%
\pgfsetfillcolor{currentfill}%
\pgfsetlinewidth{0.000000pt}%
\definecolor{currentstroke}{rgb}{0.000000,0.000000,0.000000}%
\pgfsetstrokecolor{currentstroke}%
\pgfsetdash{}{0pt}%
\pgfpathmoveto{\pgfqpoint{4.768000in}{3.438595in}}%
\pgfpathlineto{\pgfqpoint{4.582521in}{3.626667in}}%
\pgfpathlineto{\pgfqpoint{4.395517in}{3.813333in}}%
\pgfpathlineto{\pgfqpoint{4.046545in}{4.153818in}}%
\pgfpathlineto{\pgfqpoint{3.973286in}{4.224000in}}%
\pgfpathlineto{\pgfqpoint{3.970369in}{4.224000in}}%
\pgfpathlineto{\pgfqpoint{4.327111in}{3.878055in}}%
\pgfpathlineto{\pgfqpoint{4.669960in}{3.535347in}}%
\pgfpathlineto{\pgfqpoint{4.768000in}{3.435706in}}%
\pgfpathlineto{\pgfqpoint{4.768000in}{3.435706in}}%
\pgfusepath{fill}%
\end{pgfscope}%
\begin{pgfscope}%
\pgfpathrectangle{\pgfqpoint{0.800000in}{0.528000in}}{\pgfqpoint{3.968000in}{3.696000in}}%
\pgfusepath{clip}%
\pgfsetbuttcap%
\pgfsetroundjoin%
\definecolor{currentfill}{rgb}{0.232815,0.732247,0.459277}%
\pgfsetfillcolor{currentfill}%
\pgfsetlinewidth{0.000000pt}%
\definecolor{currentstroke}{rgb}{0.000000,0.000000,0.000000}%
\pgfsetstrokecolor{currentstroke}%
\pgfsetdash{}{0pt}%
\pgfpathmoveto{\pgfqpoint{4.768000in}{3.441467in}}%
\pgfpathlineto{\pgfqpoint{4.407273in}{3.804532in}}%
\pgfpathlineto{\pgfqpoint{4.206869in}{4.001492in}}%
\pgfpathlineto{\pgfqpoint{3.976203in}{4.224000in}}%
\pgfpathlineto{\pgfqpoint{3.973286in}{4.224000in}}%
\pgfpathlineto{\pgfqpoint{4.327111in}{3.880890in}}%
\pgfpathlineto{\pgfqpoint{4.687838in}{3.520249in}}%
\pgfpathlineto{\pgfqpoint{4.768000in}{3.438595in}}%
\pgfpathlineto{\pgfqpoint{4.768000in}{3.440000in}}%
\pgfusepath{fill}%
\end{pgfscope}%
\begin{pgfscope}%
\pgfpathrectangle{\pgfqpoint{0.800000in}{0.528000in}}{\pgfqpoint{3.968000in}{3.696000in}}%
\pgfusepath{clip}%
\pgfsetbuttcap%
\pgfsetroundjoin%
\definecolor{currentfill}{rgb}{0.232815,0.732247,0.459277}%
\pgfsetfillcolor{currentfill}%
\pgfsetlinewidth{0.000000pt}%
\definecolor{currentstroke}{rgb}{0.000000,0.000000,0.000000}%
\pgfsetstrokecolor{currentstroke}%
\pgfsetdash{}{0pt}%
\pgfpathmoveto{\pgfqpoint{4.768000in}{3.444324in}}%
\pgfpathlineto{\pgfqpoint{4.751005in}{3.461503in}}%
\pgfpathlineto{\pgfqpoint{4.735644in}{3.477333in}}%
\pgfpathlineto{\pgfqpoint{4.727919in}{3.485208in}}%
\pgfpathlineto{\pgfqpoint{4.712726in}{3.500515in}}%
\pgfpathlineto{\pgfqpoint{4.698948in}{3.514667in}}%
\pgfpathlineto{\pgfqpoint{4.687838in}{3.525955in}}%
\pgfpathlineto{\pgfqpoint{4.674382in}{3.539466in}}%
\pgfpathlineto{\pgfqpoint{4.662140in}{3.552000in}}%
\pgfpathlineto{\pgfqpoint{4.647758in}{3.566565in}}%
\pgfpathlineto{\pgfqpoint{4.635974in}{3.578358in}}%
\pgfpathlineto{\pgfqpoint{4.625218in}{3.589333in}}%
\pgfpathlineto{\pgfqpoint{4.607677in}{3.607039in}}%
\pgfpathlineto{\pgfqpoint{4.597501in}{3.617188in}}%
\pgfpathlineto{\pgfqpoint{4.588181in}{3.626667in}}%
\pgfpathlineto{\pgfqpoint{4.567596in}{3.647377in}}%
\pgfpathlineto{\pgfqpoint{4.558963in}{3.655959in}}%
\pgfpathlineto{\pgfqpoint{4.551030in}{3.664000in}}%
\pgfpathlineto{\pgfqpoint{4.539647in}{3.675300in}}%
\pgfpathlineto{\pgfqpoint{4.527515in}{3.687578in}}%
\pgfpathlineto{\pgfqpoint{4.520359in}{3.694668in}}%
\pgfpathlineto{\pgfqpoint{4.513762in}{3.701333in}}%
\pgfpathlineto{\pgfqpoint{4.487434in}{3.727645in}}%
\pgfpathlineto{\pgfqpoint{4.476377in}{3.738667in}}%
\pgfpathlineto{\pgfqpoint{4.462279in}{3.752569in}}%
\pgfpathlineto{\pgfqpoint{4.447354in}{3.767576in}}%
\pgfpathlineto{\pgfqpoint{4.438873in}{3.776000in}}%
\pgfpathlineto{\pgfqpoint{4.423498in}{3.791113in}}%
\pgfpathlineto{\pgfqpoint{4.407273in}{3.807371in}}%
\pgfpathlineto{\pgfqpoint{4.401251in}{3.813333in}}%
\pgfpathlineto{\pgfqpoint{4.384651in}{3.829596in}}%
\pgfpathlineto{\pgfqpoint{4.367192in}{3.847033in}}%
\pgfpathlineto{\pgfqpoint{4.365288in}{3.848894in}}%
\pgfpathlineto{\pgfqpoint{4.363509in}{3.850667in}}%
\pgfpathlineto{\pgfqpoint{4.345739in}{3.868017in}}%
\pgfpathlineto{\pgfqpoint{4.327111in}{3.886560in}}%
\pgfpathlineto{\pgfqpoint{4.325647in}{3.888000in}}%
\pgfpathlineto{\pgfqpoint{4.306760in}{3.906378in}}%
\pgfpathlineto{\pgfqpoint{4.287654in}{3.925333in}}%
\pgfpathlineto{\pgfqpoint{4.287030in}{3.925946in}}%
\pgfpathlineto{\pgfqpoint{4.249523in}{3.962667in}}%
\pgfpathlineto{\pgfqpoint{4.248261in}{3.963889in}}%
\pgfpathlineto{\pgfqpoint{4.246949in}{3.965184in}}%
\pgfpathlineto{\pgfqpoint{4.211269in}{4.000000in}}%
\pgfpathlineto{\pgfqpoint{4.209108in}{4.002086in}}%
\pgfpathlineto{\pgfqpoint{4.206869in}{4.004291in}}%
\pgfpathlineto{\pgfqpoint{4.189430in}{4.021090in}}%
\pgfpathlineto{\pgfqpoint{4.172892in}{4.037333in}}%
\pgfpathlineto{\pgfqpoint{4.166788in}{4.043265in}}%
\pgfpathlineto{\pgfqpoint{4.150187in}{4.059203in}}%
\pgfpathlineto{\pgfqpoint{4.134390in}{4.074667in}}%
\pgfpathlineto{\pgfqpoint{4.126707in}{4.082108in}}%
\pgfpathlineto{\pgfqpoint{4.095763in}{4.112000in}}%
\pgfpathlineto{\pgfqpoint{4.091254in}{4.116310in}}%
\pgfpathlineto{\pgfqpoint{4.086626in}{4.120820in}}%
\pgfpathlineto{\pgfqpoint{4.057010in}{4.149333in}}%
\pgfpathlineto{\pgfqpoint{4.046545in}{4.159401in}}%
\pgfpathlineto{\pgfqpoint{4.032056in}{4.173170in}}%
\pgfpathlineto{\pgfqpoint{4.018129in}{4.186667in}}%
\pgfpathlineto{\pgfqpoint{4.012353in}{4.192151in}}%
\pgfpathlineto{\pgfqpoint{4.006465in}{4.197851in}}%
\pgfpathlineto{\pgfqpoint{3.992545in}{4.211034in}}%
\pgfpathlineto{\pgfqpoint{3.979120in}{4.224000in}}%
\pgfpathlineto{\pgfqpoint{3.976203in}{4.224000in}}%
\pgfpathlineto{\pgfqpoint{3.991060in}{4.209651in}}%
\pgfpathlineto{\pgfqpoint{4.006465in}{4.195062in}}%
\pgfpathlineto{\pgfqpoint{4.010884in}{4.190783in}}%
\pgfpathlineto{\pgfqpoint{4.015220in}{4.186667in}}%
\pgfpathlineto{\pgfqpoint{4.030573in}{4.171789in}}%
\pgfpathlineto{\pgfqpoint{4.046545in}{4.156610in}}%
\pgfpathlineto{\pgfqpoint{4.054108in}{4.149333in}}%
\pgfpathlineto{\pgfqpoint{4.086626in}{4.118027in}}%
\pgfpathlineto{\pgfqpoint{4.089788in}{4.114945in}}%
\pgfpathlineto{\pgfqpoint{4.092870in}{4.112000in}}%
\pgfpathlineto{\pgfqpoint{4.126707in}{4.079313in}}%
\pgfpathlineto{\pgfqpoint{4.131504in}{4.074667in}}%
\pgfpathlineto{\pgfqpoint{4.148708in}{4.057826in}}%
\pgfpathlineto{\pgfqpoint{4.166788in}{4.040468in}}%
\pgfpathlineto{\pgfqpoint{4.170014in}{4.037333in}}%
\pgfpathlineto{\pgfqpoint{4.187952in}{4.019714in}}%
\pgfpathlineto{\pgfqpoint{4.206869in}{4.001492in}}%
\pgfpathlineto{\pgfqpoint{4.207647in}{4.000725in}}%
\pgfpathlineto{\pgfqpoint{4.208398in}{4.000000in}}%
\pgfpathlineto{\pgfqpoint{4.240555in}{3.968623in}}%
\pgfpathlineto{\pgfqpoint{4.246656in}{3.962667in}}%
\pgfpathlineto{\pgfqpoint{4.246799in}{3.962526in}}%
\pgfpathlineto{\pgfqpoint{4.246949in}{3.962380in}}%
\pgfpathlineto{\pgfqpoint{4.284772in}{3.925333in}}%
\pgfpathlineto{\pgfqpoint{4.287030in}{3.923119in}}%
\pgfpathlineto{\pgfqpoint{4.305288in}{3.905006in}}%
\pgfpathlineto{\pgfqpoint{4.322764in}{3.888000in}}%
\pgfpathlineto{\pgfqpoint{4.327111in}{3.883725in}}%
\pgfpathlineto{\pgfqpoint{4.344267in}{3.866647in}}%
\pgfpathlineto{\pgfqpoint{4.360635in}{3.850667in}}%
\pgfpathlineto{\pgfqpoint{4.363802in}{3.847509in}}%
\pgfpathlineto{\pgfqpoint{4.367192in}{3.844196in}}%
\pgfpathlineto{\pgfqpoint{4.383181in}{3.828227in}}%
\pgfpathlineto{\pgfqpoint{4.398384in}{3.813333in}}%
\pgfpathlineto{\pgfqpoint{4.407273in}{3.804532in}}%
\pgfpathlineto{\pgfqpoint{4.422029in}{3.789745in}}%
\pgfpathlineto{\pgfqpoint{4.436014in}{3.776000in}}%
\pgfpathlineto{\pgfqpoint{4.447354in}{3.764735in}}%
\pgfpathlineto{\pgfqpoint{4.460812in}{3.751203in}}%
\pgfpathlineto{\pgfqpoint{4.473524in}{3.738667in}}%
\pgfpathlineto{\pgfqpoint{4.487434in}{3.724802in}}%
\pgfpathlineto{\pgfqpoint{4.510917in}{3.701333in}}%
\pgfpathlineto{\pgfqpoint{4.518879in}{3.693289in}}%
\pgfpathlineto{\pgfqpoint{4.527515in}{3.684733in}}%
\pgfpathlineto{\pgfqpoint{4.538183in}{3.673937in}}%
\pgfpathlineto{\pgfqpoint{4.548192in}{3.664000in}}%
\pgfpathlineto{\pgfqpoint{4.557484in}{3.654581in}}%
\pgfpathlineto{\pgfqpoint{4.567596in}{3.644530in}}%
\pgfpathlineto{\pgfqpoint{4.585351in}{3.626667in}}%
\pgfpathlineto{\pgfqpoint{4.596024in}{3.615813in}}%
\pgfpathlineto{\pgfqpoint{4.607677in}{3.604190in}}%
\pgfpathlineto{\pgfqpoint{4.622395in}{3.589333in}}%
\pgfpathlineto{\pgfqpoint{4.634499in}{3.576983in}}%
\pgfpathlineto{\pgfqpoint{4.647758in}{3.563714in}}%
\pgfpathlineto{\pgfqpoint{4.659324in}{3.552000in}}%
\pgfpathlineto{\pgfqpoint{4.672908in}{3.538093in}}%
\pgfpathlineto{\pgfqpoint{4.687838in}{3.523102in}}%
\pgfpathlineto{\pgfqpoint{4.696140in}{3.514667in}}%
\pgfpathlineto{\pgfqpoint{4.711253in}{3.499143in}}%
\pgfpathlineto{\pgfqpoint{4.727919in}{3.482353in}}%
\pgfpathlineto{\pgfqpoint{4.732843in}{3.477333in}}%
\pgfpathlineto{\pgfqpoint{4.749534in}{3.460133in}}%
\pgfpathlineto{\pgfqpoint{4.768000in}{3.441467in}}%
\pgfusepath{fill}%
\end{pgfscope}%
\begin{pgfscope}%
\pgfpathrectangle{\pgfqpoint{0.800000in}{0.528000in}}{\pgfqpoint{3.968000in}{3.696000in}}%
\pgfusepath{clip}%
\pgfsetbuttcap%
\pgfsetroundjoin%
\definecolor{currentfill}{rgb}{0.232815,0.732247,0.459277}%
\pgfsetfillcolor{currentfill}%
\pgfsetlinewidth{0.000000pt}%
\definecolor{currentstroke}{rgb}{0.000000,0.000000,0.000000}%
\pgfsetstrokecolor{currentstroke}%
\pgfsetdash{}{0pt}%
\pgfpathmoveto{\pgfqpoint{4.768000in}{3.447182in}}%
\pgfpathlineto{\pgfqpoint{4.752476in}{3.462873in}}%
\pgfpathlineto{\pgfqpoint{4.738445in}{3.477333in}}%
\pgfpathlineto{\pgfqpoint{4.727919in}{3.488063in}}%
\pgfpathlineto{\pgfqpoint{4.714198in}{3.501886in}}%
\pgfpathlineto{\pgfqpoint{4.701756in}{3.514667in}}%
\pgfpathlineto{\pgfqpoint{4.687838in}{3.528808in}}%
\pgfpathlineto{\pgfqpoint{4.675856in}{3.540839in}}%
\pgfpathlineto{\pgfqpoint{4.664955in}{3.552000in}}%
\pgfpathlineto{\pgfqpoint{4.647758in}{3.569416in}}%
\pgfpathlineto{\pgfqpoint{4.637450in}{3.579732in}}%
\pgfpathlineto{\pgfqpoint{4.628040in}{3.589333in}}%
\pgfpathlineto{\pgfqpoint{4.607677in}{3.609888in}}%
\pgfpathlineto{\pgfqpoint{4.598978in}{3.618564in}}%
\pgfpathlineto{\pgfqpoint{4.591011in}{3.626667in}}%
\pgfpathlineto{\pgfqpoint{4.567596in}{3.650224in}}%
\pgfpathlineto{\pgfqpoint{4.560441in}{3.657336in}}%
\pgfpathlineto{\pgfqpoint{4.553867in}{3.664000in}}%
\pgfpathlineto{\pgfqpoint{4.541111in}{3.676664in}}%
\pgfpathlineto{\pgfqpoint{4.527515in}{3.690423in}}%
\pgfpathlineto{\pgfqpoint{4.521839in}{3.696047in}}%
\pgfpathlineto{\pgfqpoint{4.516606in}{3.701333in}}%
\pgfpathlineto{\pgfqpoint{4.487434in}{3.730488in}}%
\pgfpathlineto{\pgfqpoint{4.479229in}{3.738667in}}%
\pgfpathlineto{\pgfqpoint{4.463746in}{3.753935in}}%
\pgfpathlineto{\pgfqpoint{4.447354in}{3.770416in}}%
\pgfpathlineto{\pgfqpoint{4.441733in}{3.776000in}}%
\pgfpathlineto{\pgfqpoint{4.424966in}{3.792480in}}%
\pgfpathlineto{\pgfqpoint{4.407273in}{3.810210in}}%
\pgfpathlineto{\pgfqpoint{4.404119in}{3.813333in}}%
\pgfpathlineto{\pgfqpoint{4.386121in}{3.830965in}}%
\pgfpathlineto{\pgfqpoint{4.367192in}{3.849870in}}%
\pgfpathlineto{\pgfqpoint{4.366774in}{3.850278in}}%
\pgfpathlineto{\pgfqpoint{4.366384in}{3.850667in}}%
\pgfpathlineto{\pgfqpoint{4.347210in}{3.869388in}}%
\pgfpathlineto{\pgfqpoint{4.328512in}{3.888000in}}%
\pgfpathlineto{\pgfqpoint{4.327111in}{3.889380in}}%
\pgfpathlineto{\pgfqpoint{4.308233in}{3.907749in}}%
\pgfpathlineto{\pgfqpoint{4.290510in}{3.925333in}}%
\pgfpathlineto{\pgfqpoint{4.287030in}{3.928749in}}%
\pgfpathlineto{\pgfqpoint{4.252386in}{3.962667in}}%
\pgfpathlineto{\pgfqpoint{4.249721in}{3.965248in}}%
\pgfpathlineto{\pgfqpoint{4.246949in}{3.967985in}}%
\pgfpathlineto{\pgfqpoint{4.214140in}{4.000000in}}%
\pgfpathlineto{\pgfqpoint{4.210569in}{4.003447in}}%
\pgfpathlineto{\pgfqpoint{4.206869in}{4.007090in}}%
\pgfpathlineto{\pgfqpoint{4.190907in}{4.022466in}}%
\pgfpathlineto{\pgfqpoint{4.175770in}{4.037333in}}%
\pgfpathlineto{\pgfqpoint{4.166788in}{4.046062in}}%
\pgfpathlineto{\pgfqpoint{4.151665in}{4.060581in}}%
\pgfpathlineto{\pgfqpoint{4.137276in}{4.074667in}}%
\pgfpathlineto{\pgfqpoint{4.126707in}{4.084903in}}%
\pgfpathlineto{\pgfqpoint{4.098657in}{4.112000in}}%
\pgfpathlineto{\pgfqpoint{4.092719in}{4.117675in}}%
\pgfpathlineto{\pgfqpoint{4.086626in}{4.123613in}}%
\pgfpathlineto{\pgfqpoint{4.059911in}{4.149333in}}%
\pgfpathlineto{\pgfqpoint{4.046545in}{4.162192in}}%
\pgfpathlineto{\pgfqpoint{4.033539in}{4.174552in}}%
\pgfpathlineto{\pgfqpoint{4.021038in}{4.186667in}}%
\pgfpathlineto{\pgfqpoint{4.013821in}{4.193519in}}%
\pgfpathlineto{\pgfqpoint{4.006465in}{4.200640in}}%
\pgfpathlineto{\pgfqpoint{3.994030in}{4.212417in}}%
\pgfpathlineto{\pgfqpoint{3.982036in}{4.224000in}}%
\pgfpathlineto{\pgfqpoint{3.979120in}{4.224000in}}%
\pgfpathlineto{\pgfqpoint{3.992545in}{4.211034in}}%
\pgfpathlineto{\pgfqpoint{4.006465in}{4.197851in}}%
\pgfpathlineto{\pgfqpoint{4.012353in}{4.192151in}}%
\pgfpathlineto{\pgfqpoint{4.018129in}{4.186667in}}%
\pgfpathlineto{\pgfqpoint{4.032056in}{4.173170in}}%
\pgfpathlineto{\pgfqpoint{4.046545in}{4.159401in}}%
\pgfpathlineto{\pgfqpoint{4.057010in}{4.149333in}}%
\pgfpathlineto{\pgfqpoint{4.086626in}{4.120820in}}%
\pgfpathlineto{\pgfqpoint{4.091254in}{4.116310in}}%
\pgfpathlineto{\pgfqpoint{4.095763in}{4.112000in}}%
\pgfpathlineto{\pgfqpoint{4.126707in}{4.082108in}}%
\pgfpathlineto{\pgfqpoint{4.134390in}{4.074667in}}%
\pgfpathlineto{\pgfqpoint{4.150187in}{4.059203in}}%
\pgfpathlineto{\pgfqpoint{4.166788in}{4.043265in}}%
\pgfpathlineto{\pgfqpoint{4.172892in}{4.037333in}}%
\pgfpathlineto{\pgfqpoint{4.189430in}{4.021090in}}%
\pgfpathlineto{\pgfqpoint{4.206869in}{4.004291in}}%
\pgfpathlineto{\pgfqpoint{4.209108in}{4.002086in}}%
\pgfpathlineto{\pgfqpoint{4.211269in}{4.000000in}}%
\pgfpathlineto{\pgfqpoint{4.246949in}{3.965184in}}%
\pgfpathlineto{\pgfqpoint{4.248261in}{3.963889in}}%
\pgfpathlineto{\pgfqpoint{4.249523in}{3.962667in}}%
\pgfpathlineto{\pgfqpoint{4.287030in}{3.925946in}}%
\pgfpathlineto{\pgfqpoint{4.287654in}{3.925333in}}%
\pgfpathlineto{\pgfqpoint{4.306760in}{3.906378in}}%
\pgfpathlineto{\pgfqpoint{4.325647in}{3.888000in}}%
\pgfpathlineto{\pgfqpoint{4.327111in}{3.886560in}}%
\pgfpathlineto{\pgfqpoint{4.345739in}{3.868017in}}%
\pgfpathlineto{\pgfqpoint{4.363509in}{3.850667in}}%
\pgfpathlineto{\pgfqpoint{4.365288in}{3.848894in}}%
\pgfpathlineto{\pgfqpoint{4.367192in}{3.847033in}}%
\pgfpathlineto{\pgfqpoint{4.384651in}{3.829596in}}%
\pgfpathlineto{\pgfqpoint{4.401251in}{3.813333in}}%
\pgfpathlineto{\pgfqpoint{4.407273in}{3.807371in}}%
\pgfpathlineto{\pgfqpoint{4.423498in}{3.791113in}}%
\pgfpathlineto{\pgfqpoint{4.438873in}{3.776000in}}%
\pgfpathlineto{\pgfqpoint{4.447354in}{3.767576in}}%
\pgfpathlineto{\pgfqpoint{4.462279in}{3.752569in}}%
\pgfpathlineto{\pgfqpoint{4.476377in}{3.738667in}}%
\pgfpathlineto{\pgfqpoint{4.487434in}{3.727645in}}%
\pgfpathlineto{\pgfqpoint{4.513762in}{3.701333in}}%
\pgfpathlineto{\pgfqpoint{4.520359in}{3.694668in}}%
\pgfpathlineto{\pgfqpoint{4.527515in}{3.687578in}}%
\pgfpathlineto{\pgfqpoint{4.539647in}{3.675300in}}%
\pgfpathlineto{\pgfqpoint{4.551030in}{3.664000in}}%
\pgfpathlineto{\pgfqpoint{4.558963in}{3.655959in}}%
\pgfpathlineto{\pgfqpoint{4.567596in}{3.647377in}}%
\pgfpathlineto{\pgfqpoint{4.588181in}{3.626667in}}%
\pgfpathlineto{\pgfqpoint{4.597501in}{3.617188in}}%
\pgfpathlineto{\pgfqpoint{4.607677in}{3.607039in}}%
\pgfpathlineto{\pgfqpoint{4.625218in}{3.589333in}}%
\pgfpathlineto{\pgfqpoint{4.635974in}{3.578358in}}%
\pgfpathlineto{\pgfqpoint{4.647758in}{3.566565in}}%
\pgfpathlineto{\pgfqpoint{4.662140in}{3.552000in}}%
\pgfpathlineto{\pgfqpoint{4.674382in}{3.539466in}}%
\pgfpathlineto{\pgfqpoint{4.687838in}{3.525955in}}%
\pgfpathlineto{\pgfqpoint{4.698948in}{3.514667in}}%
\pgfpathlineto{\pgfqpoint{4.712726in}{3.500515in}}%
\pgfpathlineto{\pgfqpoint{4.727919in}{3.485208in}}%
\pgfpathlineto{\pgfqpoint{4.735644in}{3.477333in}}%
\pgfpathlineto{\pgfqpoint{4.751005in}{3.461503in}}%
\pgfpathlineto{\pgfqpoint{4.768000in}{3.444324in}}%
\pgfusepath{fill}%
\end{pgfscope}%
\begin{pgfscope}%
\pgfpathrectangle{\pgfqpoint{0.800000in}{0.528000in}}{\pgfqpoint{3.968000in}{3.696000in}}%
\pgfusepath{clip}%
\pgfsetbuttcap%
\pgfsetroundjoin%
\definecolor{currentfill}{rgb}{0.239374,0.735588,0.455688}%
\pgfsetfillcolor{currentfill}%
\pgfsetlinewidth{0.000000pt}%
\definecolor{currentstroke}{rgb}{0.000000,0.000000,0.000000}%
\pgfsetstrokecolor{currentstroke}%
\pgfsetdash{}{0pt}%
\pgfpathmoveto{\pgfqpoint{4.768000in}{3.450039in}}%
\pgfpathlineto{\pgfqpoint{4.753947in}{3.464243in}}%
\pgfpathlineto{\pgfqpoint{4.741245in}{3.477333in}}%
\pgfpathlineto{\pgfqpoint{4.727919in}{3.490919in}}%
\pgfpathlineto{\pgfqpoint{4.715671in}{3.503258in}}%
\pgfpathlineto{\pgfqpoint{4.704564in}{3.514667in}}%
\pgfpathlineto{\pgfqpoint{4.687838in}{3.531661in}}%
\pgfpathlineto{\pgfqpoint{4.677330in}{3.542212in}}%
\pgfpathlineto{\pgfqpoint{4.667770in}{3.552000in}}%
\pgfpathlineto{\pgfqpoint{4.647758in}{3.572267in}}%
\pgfpathlineto{\pgfqpoint{4.638925in}{3.581106in}}%
\pgfpathlineto{\pgfqpoint{4.630863in}{3.589333in}}%
\pgfpathlineto{\pgfqpoint{4.607677in}{3.612737in}}%
\pgfpathlineto{\pgfqpoint{4.600455in}{3.619940in}}%
\pgfpathlineto{\pgfqpoint{4.593841in}{3.626667in}}%
\pgfpathlineto{\pgfqpoint{4.567596in}{3.653071in}}%
\pgfpathlineto{\pgfqpoint{4.561920in}{3.658713in}}%
\pgfpathlineto{\pgfqpoint{4.556704in}{3.664000in}}%
\pgfpathlineto{\pgfqpoint{4.542575in}{3.678027in}}%
\pgfpathlineto{\pgfqpoint{4.527515in}{3.693268in}}%
\pgfpathlineto{\pgfqpoint{4.523319in}{3.697425in}}%
\pgfpathlineto{\pgfqpoint{4.519451in}{3.701333in}}%
\pgfpathlineto{\pgfqpoint{4.487434in}{3.733331in}}%
\pgfpathlineto{\pgfqpoint{4.482081in}{3.738667in}}%
\pgfpathlineto{\pgfqpoint{4.465213in}{3.755302in}}%
\pgfpathlineto{\pgfqpoint{4.447354in}{3.773257in}}%
\pgfpathlineto{\pgfqpoint{4.444593in}{3.776000in}}%
\pgfpathlineto{\pgfqpoint{4.426434in}{3.793848in}}%
\pgfpathlineto{\pgfqpoint{4.407273in}{3.813049in}}%
\pgfpathlineto{\pgfqpoint{4.406986in}{3.813333in}}%
\pgfpathlineto{\pgfqpoint{4.387590in}{3.832334in}}%
\pgfpathlineto{\pgfqpoint{4.369234in}{3.850667in}}%
\pgfpathlineto{\pgfqpoint{4.367192in}{3.852685in}}%
\pgfpathlineto{\pgfqpoint{4.348681in}{3.870758in}}%
\pgfpathlineto{\pgfqpoint{4.331360in}{3.888000in}}%
\pgfpathlineto{\pgfqpoint{4.327111in}{3.892185in}}%
\pgfpathlineto{\pgfqpoint{4.309706in}{3.909121in}}%
\pgfpathlineto{\pgfqpoint{4.293365in}{3.925333in}}%
\pgfpathlineto{\pgfqpoint{4.287030in}{3.931552in}}%
\pgfpathlineto{\pgfqpoint{4.255249in}{3.962667in}}%
\pgfpathlineto{\pgfqpoint{4.251181in}{3.966608in}}%
\pgfpathlineto{\pgfqpoint{4.246949in}{3.970786in}}%
\pgfpathlineto{\pgfqpoint{4.217010in}{4.000000in}}%
\pgfpathlineto{\pgfqpoint{4.212030in}{4.004808in}}%
\pgfpathlineto{\pgfqpoint{4.206869in}{4.009889in}}%
\pgfpathlineto{\pgfqpoint{4.192384in}{4.023842in}}%
\pgfpathlineto{\pgfqpoint{4.178648in}{4.037333in}}%
\pgfpathlineto{\pgfqpoint{4.166788in}{4.048859in}}%
\pgfpathlineto{\pgfqpoint{4.153144in}{4.061958in}}%
\pgfpathlineto{\pgfqpoint{4.140162in}{4.074667in}}%
\pgfpathlineto{\pgfqpoint{4.126707in}{4.087698in}}%
\pgfpathlineto{\pgfqpoint{4.101550in}{4.112000in}}%
\pgfpathlineto{\pgfqpoint{4.094185in}{4.119040in}}%
\pgfpathlineto{\pgfqpoint{4.086626in}{4.126406in}}%
\pgfpathlineto{\pgfqpoint{4.062812in}{4.149333in}}%
\pgfpathlineto{\pgfqpoint{4.046545in}{4.164983in}}%
\pgfpathlineto{\pgfqpoint{4.035023in}{4.175934in}}%
\pgfpathlineto{\pgfqpoint{4.023947in}{4.186667in}}%
\pgfpathlineto{\pgfqpoint{4.015290in}{4.194887in}}%
\pgfpathlineto{\pgfqpoint{4.006465in}{4.203429in}}%
\pgfpathlineto{\pgfqpoint{3.995514in}{4.213800in}}%
\pgfpathlineto{\pgfqpoint{3.984953in}{4.224000in}}%
\pgfpathlineto{\pgfqpoint{3.982036in}{4.224000in}}%
\pgfpathlineto{\pgfqpoint{3.994030in}{4.212417in}}%
\pgfpathlineto{\pgfqpoint{4.006465in}{4.200640in}}%
\pgfpathlineto{\pgfqpoint{4.013821in}{4.193519in}}%
\pgfpathlineto{\pgfqpoint{4.021038in}{4.186667in}}%
\pgfpathlineto{\pgfqpoint{4.033539in}{4.174552in}}%
\pgfpathlineto{\pgfqpoint{4.046545in}{4.162192in}}%
\pgfpathlineto{\pgfqpoint{4.059911in}{4.149333in}}%
\pgfpathlineto{\pgfqpoint{4.086626in}{4.123613in}}%
\pgfpathlineto{\pgfqpoint{4.092719in}{4.117675in}}%
\pgfpathlineto{\pgfqpoint{4.098657in}{4.112000in}}%
\pgfpathlineto{\pgfqpoint{4.126707in}{4.084903in}}%
\pgfpathlineto{\pgfqpoint{4.137276in}{4.074667in}}%
\pgfpathlineto{\pgfqpoint{4.151665in}{4.060581in}}%
\pgfpathlineto{\pgfqpoint{4.166788in}{4.046062in}}%
\pgfpathlineto{\pgfqpoint{4.175770in}{4.037333in}}%
\pgfpathlineto{\pgfqpoint{4.190907in}{4.022466in}}%
\pgfpathlineto{\pgfqpoint{4.206869in}{4.007090in}}%
\pgfpathlineto{\pgfqpoint{4.210569in}{4.003447in}}%
\pgfpathlineto{\pgfqpoint{4.214140in}{4.000000in}}%
\pgfpathlineto{\pgfqpoint{4.246949in}{3.967985in}}%
\pgfpathlineto{\pgfqpoint{4.249721in}{3.965248in}}%
\pgfpathlineto{\pgfqpoint{4.252386in}{3.962667in}}%
\pgfpathlineto{\pgfqpoint{4.287030in}{3.928749in}}%
\pgfpathlineto{\pgfqpoint{4.290510in}{3.925333in}}%
\pgfpathlineto{\pgfqpoint{4.308233in}{3.907749in}}%
\pgfpathlineto{\pgfqpoint{4.327111in}{3.889380in}}%
\pgfpathlineto{\pgfqpoint{4.328512in}{3.888000in}}%
\pgfpathlineto{\pgfqpoint{4.347210in}{3.869388in}}%
\pgfpathlineto{\pgfqpoint{4.366384in}{3.850667in}}%
\pgfpathlineto{\pgfqpoint{4.366774in}{3.850278in}}%
\pgfpathlineto{\pgfqpoint{4.367192in}{3.849870in}}%
\pgfpathlineto{\pgfqpoint{4.386121in}{3.830965in}}%
\pgfpathlineto{\pgfqpoint{4.404119in}{3.813333in}}%
\pgfpathlineto{\pgfqpoint{4.407273in}{3.810210in}}%
\pgfpathlineto{\pgfqpoint{4.424966in}{3.792480in}}%
\pgfpathlineto{\pgfqpoint{4.441733in}{3.776000in}}%
\pgfpathlineto{\pgfqpoint{4.447354in}{3.770416in}}%
\pgfpathlineto{\pgfqpoint{4.463746in}{3.753935in}}%
\pgfpathlineto{\pgfqpoint{4.479229in}{3.738667in}}%
\pgfpathlineto{\pgfqpoint{4.487434in}{3.730488in}}%
\pgfpathlineto{\pgfqpoint{4.516606in}{3.701333in}}%
\pgfpathlineto{\pgfqpoint{4.521839in}{3.696047in}}%
\pgfpathlineto{\pgfqpoint{4.527515in}{3.690423in}}%
\pgfpathlineto{\pgfqpoint{4.541111in}{3.676664in}}%
\pgfpathlineto{\pgfqpoint{4.553867in}{3.664000in}}%
\pgfpathlineto{\pgfqpoint{4.560441in}{3.657336in}}%
\pgfpathlineto{\pgfqpoint{4.567596in}{3.650224in}}%
\pgfpathlineto{\pgfqpoint{4.591011in}{3.626667in}}%
\pgfpathlineto{\pgfqpoint{4.598978in}{3.618564in}}%
\pgfpathlineto{\pgfqpoint{4.607677in}{3.609888in}}%
\pgfpathlineto{\pgfqpoint{4.628040in}{3.589333in}}%
\pgfpathlineto{\pgfqpoint{4.637450in}{3.579732in}}%
\pgfpathlineto{\pgfqpoint{4.647758in}{3.569416in}}%
\pgfpathlineto{\pgfqpoint{4.664955in}{3.552000in}}%
\pgfpathlineto{\pgfqpoint{4.675856in}{3.540839in}}%
\pgfpathlineto{\pgfqpoint{4.687838in}{3.528808in}}%
\pgfpathlineto{\pgfqpoint{4.701756in}{3.514667in}}%
\pgfpathlineto{\pgfqpoint{4.714198in}{3.501886in}}%
\pgfpathlineto{\pgfqpoint{4.727919in}{3.488063in}}%
\pgfpathlineto{\pgfqpoint{4.738445in}{3.477333in}}%
\pgfpathlineto{\pgfqpoint{4.752476in}{3.462873in}}%
\pgfpathlineto{\pgfqpoint{4.768000in}{3.447182in}}%
\pgfusepath{fill}%
\end{pgfscope}%
\begin{pgfscope}%
\pgfpathrectangle{\pgfqpoint{0.800000in}{0.528000in}}{\pgfqpoint{3.968000in}{3.696000in}}%
\pgfusepath{clip}%
\pgfsetbuttcap%
\pgfsetroundjoin%
\definecolor{currentfill}{rgb}{0.239374,0.735588,0.455688}%
\pgfsetfillcolor{currentfill}%
\pgfsetlinewidth{0.000000pt}%
\definecolor{currentstroke}{rgb}{0.000000,0.000000,0.000000}%
\pgfsetstrokecolor{currentstroke}%
\pgfsetdash{}{0pt}%
\pgfpathmoveto{\pgfqpoint{4.768000in}{3.452896in}}%
\pgfpathlineto{\pgfqpoint{4.755418in}{3.465614in}}%
\pgfpathlineto{\pgfqpoint{4.744046in}{3.477333in}}%
\pgfpathlineto{\pgfqpoint{4.727919in}{3.493774in}}%
\pgfpathlineto{\pgfqpoint{4.717143in}{3.504630in}}%
\pgfpathlineto{\pgfqpoint{4.707372in}{3.514667in}}%
\pgfpathlineto{\pgfqpoint{4.687838in}{3.534515in}}%
\pgfpathlineto{\pgfqpoint{4.678804in}{3.543585in}}%
\pgfpathlineto{\pgfqpoint{4.670585in}{3.552000in}}%
\pgfpathlineto{\pgfqpoint{4.647758in}{3.575119in}}%
\pgfpathlineto{\pgfqpoint{4.640401in}{3.582481in}}%
\pgfpathlineto{\pgfqpoint{4.633685in}{3.589333in}}%
\pgfpathlineto{\pgfqpoint{4.607677in}{3.615586in}}%
\pgfpathlineto{\pgfqpoint{4.601932in}{3.621316in}}%
\pgfpathlineto{\pgfqpoint{4.596671in}{3.626667in}}%
\pgfpathlineto{\pgfqpoint{4.567596in}{3.655918in}}%
\pgfpathlineto{\pgfqpoint{4.563399in}{3.660090in}}%
\pgfpathlineto{\pgfqpoint{4.559541in}{3.664000in}}%
\pgfpathlineto{\pgfqpoint{4.544038in}{3.679391in}}%
\pgfpathlineto{\pgfqpoint{4.527515in}{3.696113in}}%
\pgfpathlineto{\pgfqpoint{4.524800in}{3.698804in}}%
\pgfpathlineto{\pgfqpoint{4.522296in}{3.701333in}}%
\pgfpathlineto{\pgfqpoint{4.487434in}{3.736174in}}%
\pgfpathlineto{\pgfqpoint{4.484933in}{3.738667in}}%
\pgfpathlineto{\pgfqpoint{4.466680in}{3.756668in}}%
\pgfpathlineto{\pgfqpoint{4.447451in}{3.776000in}}%
\pgfpathlineto{\pgfqpoint{4.447404in}{3.776047in}}%
\pgfpathlineto{\pgfqpoint{4.447354in}{3.776097in}}%
\pgfpathlineto{\pgfqpoint{4.427903in}{3.795216in}}%
\pgfpathlineto{\pgfqpoint{4.409822in}{3.813333in}}%
\pgfpathlineto{\pgfqpoint{4.407273in}{3.815861in}}%
\pgfpathlineto{\pgfqpoint{4.389060in}{3.833703in}}%
\pgfpathlineto{\pgfqpoint{4.372075in}{3.850667in}}%
\pgfpathlineto{\pgfqpoint{4.367192in}{3.855492in}}%
\pgfpathlineto{\pgfqpoint{4.350152in}{3.872128in}}%
\pgfpathlineto{\pgfqpoint{4.334208in}{3.888000in}}%
\pgfpathlineto{\pgfqpoint{4.327111in}{3.894989in}}%
\pgfpathlineto{\pgfqpoint{4.311179in}{3.910493in}}%
\pgfpathlineto{\pgfqpoint{4.296220in}{3.925333in}}%
\pgfpathlineto{\pgfqpoint{4.287030in}{3.934354in}}%
\pgfpathlineto{\pgfqpoint{4.258112in}{3.962667in}}%
\pgfpathlineto{\pgfqpoint{4.252640in}{3.967967in}}%
\pgfpathlineto{\pgfqpoint{4.246949in}{3.973587in}}%
\pgfpathlineto{\pgfqpoint{4.219881in}{4.000000in}}%
\pgfpathlineto{\pgfqpoint{4.213492in}{4.006169in}}%
\pgfpathlineto{\pgfqpoint{4.206869in}{4.012688in}}%
\pgfpathlineto{\pgfqpoint{4.193861in}{4.025218in}}%
\pgfpathlineto{\pgfqpoint{4.181526in}{4.037333in}}%
\pgfpathlineto{\pgfqpoint{4.166788in}{4.051656in}}%
\pgfpathlineto{\pgfqpoint{4.154623in}{4.063335in}}%
\pgfpathlineto{\pgfqpoint{4.143047in}{4.074667in}}%
\pgfpathlineto{\pgfqpoint{4.126707in}{4.090493in}}%
\pgfpathlineto{\pgfqpoint{4.104444in}{4.112000in}}%
\pgfpathlineto{\pgfqpoint{4.095650in}{4.120405in}}%
\pgfpathlineto{\pgfqpoint{4.086626in}{4.129199in}}%
\pgfpathlineto{\pgfqpoint{4.065713in}{4.149333in}}%
\pgfpathlineto{\pgfqpoint{4.046545in}{4.167774in}}%
\pgfpathlineto{\pgfqpoint{4.036506in}{4.177315in}}%
\pgfpathlineto{\pgfqpoint{4.026856in}{4.186667in}}%
\pgfpathlineto{\pgfqpoint{4.016758in}{4.196254in}}%
\pgfpathlineto{\pgfqpoint{4.006465in}{4.206218in}}%
\pgfpathlineto{\pgfqpoint{3.996999in}{4.215183in}}%
\pgfpathlineto{\pgfqpoint{3.987870in}{4.224000in}}%
\pgfpathlineto{\pgfqpoint{3.984953in}{4.224000in}}%
\pgfpathlineto{\pgfqpoint{3.995514in}{4.213800in}}%
\pgfpathlineto{\pgfqpoint{4.006465in}{4.203429in}}%
\pgfpathlineto{\pgfqpoint{4.015290in}{4.194887in}}%
\pgfpathlineto{\pgfqpoint{4.023947in}{4.186667in}}%
\pgfpathlineto{\pgfqpoint{4.035023in}{4.175934in}}%
\pgfpathlineto{\pgfqpoint{4.046545in}{4.164983in}}%
\pgfpathlineto{\pgfqpoint{4.062812in}{4.149333in}}%
\pgfpathlineto{\pgfqpoint{4.086626in}{4.126406in}}%
\pgfpathlineto{\pgfqpoint{4.094185in}{4.119040in}}%
\pgfpathlineto{\pgfqpoint{4.101550in}{4.112000in}}%
\pgfpathlineto{\pgfqpoint{4.126707in}{4.087698in}}%
\pgfpathlineto{\pgfqpoint{4.140162in}{4.074667in}}%
\pgfpathlineto{\pgfqpoint{4.153144in}{4.061958in}}%
\pgfpathlineto{\pgfqpoint{4.166788in}{4.048859in}}%
\pgfpathlineto{\pgfqpoint{4.178648in}{4.037333in}}%
\pgfpathlineto{\pgfqpoint{4.192384in}{4.023842in}}%
\pgfpathlineto{\pgfqpoint{4.206869in}{4.009889in}}%
\pgfpathlineto{\pgfqpoint{4.212030in}{4.004808in}}%
\pgfpathlineto{\pgfqpoint{4.217010in}{4.000000in}}%
\pgfpathlineto{\pgfqpoint{4.246949in}{3.970786in}}%
\pgfpathlineto{\pgfqpoint{4.251181in}{3.966608in}}%
\pgfpathlineto{\pgfqpoint{4.255249in}{3.962667in}}%
\pgfpathlineto{\pgfqpoint{4.287030in}{3.931552in}}%
\pgfpathlineto{\pgfqpoint{4.293365in}{3.925333in}}%
\pgfpathlineto{\pgfqpoint{4.309706in}{3.909121in}}%
\pgfpathlineto{\pgfqpoint{4.327111in}{3.892185in}}%
\pgfpathlineto{\pgfqpoint{4.331360in}{3.888000in}}%
\pgfpathlineto{\pgfqpoint{4.348681in}{3.870758in}}%
\pgfpathlineto{\pgfqpoint{4.367192in}{3.852685in}}%
\pgfpathlineto{\pgfqpoint{4.369234in}{3.850667in}}%
\pgfpathlineto{\pgfqpoint{4.387590in}{3.832334in}}%
\pgfpathlineto{\pgfqpoint{4.406986in}{3.813333in}}%
\pgfpathlineto{\pgfqpoint{4.407273in}{3.813049in}}%
\pgfpathlineto{\pgfqpoint{4.426434in}{3.793848in}}%
\pgfpathlineto{\pgfqpoint{4.444593in}{3.776000in}}%
\pgfpathlineto{\pgfqpoint{4.447354in}{3.773257in}}%
\pgfpathlineto{\pgfqpoint{4.465213in}{3.755302in}}%
\pgfpathlineto{\pgfqpoint{4.482081in}{3.738667in}}%
\pgfpathlineto{\pgfqpoint{4.487434in}{3.733331in}}%
\pgfpathlineto{\pgfqpoint{4.519451in}{3.701333in}}%
\pgfpathlineto{\pgfqpoint{4.523319in}{3.697425in}}%
\pgfpathlineto{\pgfqpoint{4.527515in}{3.693268in}}%
\pgfpathlineto{\pgfqpoint{4.542575in}{3.678027in}}%
\pgfpathlineto{\pgfqpoint{4.556704in}{3.664000in}}%
\pgfpathlineto{\pgfqpoint{4.561920in}{3.658713in}}%
\pgfpathlineto{\pgfqpoint{4.567596in}{3.653071in}}%
\pgfpathlineto{\pgfqpoint{4.593841in}{3.626667in}}%
\pgfpathlineto{\pgfqpoint{4.600455in}{3.619940in}}%
\pgfpathlineto{\pgfqpoint{4.607677in}{3.612737in}}%
\pgfpathlineto{\pgfqpoint{4.630863in}{3.589333in}}%
\pgfpathlineto{\pgfqpoint{4.638925in}{3.581106in}}%
\pgfpathlineto{\pgfqpoint{4.647758in}{3.572267in}}%
\pgfpathlineto{\pgfqpoint{4.667770in}{3.552000in}}%
\pgfpathlineto{\pgfqpoint{4.677330in}{3.542212in}}%
\pgfpathlineto{\pgfqpoint{4.687838in}{3.531661in}}%
\pgfpathlineto{\pgfqpoint{4.704564in}{3.514667in}}%
\pgfpathlineto{\pgfqpoint{4.715671in}{3.503258in}}%
\pgfpathlineto{\pgfqpoint{4.727919in}{3.490919in}}%
\pgfpathlineto{\pgfqpoint{4.741245in}{3.477333in}}%
\pgfpathlineto{\pgfqpoint{4.753947in}{3.464243in}}%
\pgfpathlineto{\pgfqpoint{4.768000in}{3.450039in}}%
\pgfusepath{fill}%
\end{pgfscope}%
\begin{pgfscope}%
\pgfpathrectangle{\pgfqpoint{0.800000in}{0.528000in}}{\pgfqpoint{3.968000in}{3.696000in}}%
\pgfusepath{clip}%
\pgfsetbuttcap%
\pgfsetroundjoin%
\definecolor{currentfill}{rgb}{0.239374,0.735588,0.455688}%
\pgfsetfillcolor{currentfill}%
\pgfsetlinewidth{0.000000pt}%
\definecolor{currentstroke}{rgb}{0.000000,0.000000,0.000000}%
\pgfsetstrokecolor{currentstroke}%
\pgfsetdash{}{0pt}%
\pgfpathmoveto{\pgfqpoint{4.768000in}{3.455753in}}%
\pgfpathlineto{\pgfqpoint{4.429371in}{3.796583in}}%
\pgfpathlineto{\pgfqpoint{4.327111in}{3.897794in}}%
\pgfpathlineto{\pgfqpoint{3.990787in}{4.224000in}}%
\pgfpathlineto{\pgfqpoint{3.987870in}{4.224000in}}%
\pgfpathlineto{\pgfqpoint{4.334208in}{3.888000in}}%
\pgfpathlineto{\pgfqpoint{4.687838in}{3.534515in}}%
\pgfpathlineto{\pgfqpoint{4.768000in}{3.452896in}}%
\pgfpathlineto{\pgfqpoint{4.768000in}{3.452896in}}%
\pgfusepath{fill}%
\end{pgfscope}%
\begin{pgfscope}%
\pgfpathrectangle{\pgfqpoint{0.800000in}{0.528000in}}{\pgfqpoint{3.968000in}{3.696000in}}%
\pgfusepath{clip}%
\pgfsetbuttcap%
\pgfsetroundjoin%
\definecolor{currentfill}{rgb}{0.239374,0.735588,0.455688}%
\pgfsetfillcolor{currentfill}%
\pgfsetlinewidth{0.000000pt}%
\definecolor{currentstroke}{rgb}{0.000000,0.000000,0.000000}%
\pgfsetstrokecolor{currentstroke}%
\pgfsetdash{}{0pt}%
\pgfpathmoveto{\pgfqpoint{4.768000in}{3.458610in}}%
\pgfpathlineto{\pgfqpoint{4.430839in}{3.797951in}}%
\pgfpathlineto{\pgfqpoint{4.327111in}{3.900599in}}%
\pgfpathlineto{\pgfqpoint{3.993703in}{4.224000in}}%
\pgfpathlineto{\pgfqpoint{3.990787in}{4.224000in}}%
\pgfpathlineto{\pgfqpoint{4.327111in}{3.897794in}}%
\pgfpathlineto{\pgfqpoint{4.527515in}{3.698958in}}%
\pgfpathlineto{\pgfqpoint{4.768000in}{3.455753in}}%
\pgfpathlineto{\pgfqpoint{4.768000in}{3.455753in}}%
\pgfusepath{fill}%
\end{pgfscope}%
\begin{pgfscope}%
\pgfpathrectangle{\pgfqpoint{0.800000in}{0.528000in}}{\pgfqpoint{3.968000in}{3.696000in}}%
\pgfusepath{clip}%
\pgfsetbuttcap%
\pgfsetroundjoin%
\definecolor{currentfill}{rgb}{0.246070,0.738910,0.452024}%
\pgfsetfillcolor{currentfill}%
\pgfsetlinewidth{0.000000pt}%
\definecolor{currentstroke}{rgb}{0.000000,0.000000,0.000000}%
\pgfsetstrokecolor{currentstroke}%
\pgfsetdash{}{0pt}%
\pgfpathmoveto{\pgfqpoint{4.768000in}{3.461468in}}%
\pgfpathlineto{\pgfqpoint{4.759831in}{3.469725in}}%
\pgfpathlineto{\pgfqpoint{4.752448in}{3.477333in}}%
\pgfpathlineto{\pgfqpoint{4.727919in}{3.502339in}}%
\pgfpathlineto{\pgfqpoint{4.721561in}{3.508745in}}%
\pgfpathlineto{\pgfqpoint{4.715796in}{3.514667in}}%
\pgfpathlineto{\pgfqpoint{4.687838in}{3.543074in}}%
\pgfpathlineto{\pgfqpoint{4.683227in}{3.547704in}}%
\pgfpathlineto{\pgfqpoint{4.679031in}{3.552000in}}%
\pgfpathlineto{\pgfqpoint{4.647758in}{3.583672in}}%
\pgfpathlineto{\pgfqpoint{4.644828in}{3.586604in}}%
\pgfpathlineto{\pgfqpoint{4.642153in}{3.589333in}}%
\pgfpathlineto{\pgfqpoint{4.607677in}{3.624133in}}%
\pgfpathlineto{\pgfqpoint{4.606363in}{3.625443in}}%
\pgfpathlineto{\pgfqpoint{4.605161in}{3.626667in}}%
\pgfpathlineto{\pgfqpoint{4.573747in}{3.658271in}}%
\pgfpathlineto{\pgfqpoint{4.568048in}{3.664000in}}%
\pgfpathlineto{\pgfqpoint{4.567596in}{3.664454in}}%
\pgfpathlineto{\pgfqpoint{4.548430in}{3.683481in}}%
\pgfpathlineto{\pgfqpoint{4.530791in}{3.701333in}}%
\pgfpathlineto{\pgfqpoint{4.527515in}{3.704613in}}%
\pgfpathlineto{\pgfqpoint{4.493418in}{3.738667in}}%
\pgfpathlineto{\pgfqpoint{4.490515in}{3.741536in}}%
\pgfpathlineto{\pgfqpoint{4.487434in}{3.744638in}}%
\pgfpathlineto{\pgfqpoint{4.471080in}{3.760767in}}%
\pgfpathlineto{\pgfqpoint{4.455929in}{3.776000in}}%
\pgfpathlineto{\pgfqpoint{4.451761in}{3.780105in}}%
\pgfpathlineto{\pgfqpoint{4.447354in}{3.784530in}}%
\pgfpathlineto{\pgfqpoint{4.432307in}{3.799319in}}%
\pgfpathlineto{\pgfqpoint{4.418322in}{3.813333in}}%
\pgfpathlineto{\pgfqpoint{4.407273in}{3.824288in}}%
\pgfpathlineto{\pgfqpoint{4.393469in}{3.837810in}}%
\pgfpathlineto{\pgfqpoint{4.380596in}{3.850667in}}%
\pgfpathlineto{\pgfqpoint{4.367192in}{3.863912in}}%
\pgfpathlineto{\pgfqpoint{4.354566in}{3.876240in}}%
\pgfpathlineto{\pgfqpoint{4.342752in}{3.888000in}}%
\pgfpathlineto{\pgfqpoint{4.327111in}{3.903404in}}%
\pgfpathlineto{\pgfqpoint{4.315597in}{3.914608in}}%
\pgfpathlineto{\pgfqpoint{4.304787in}{3.925333in}}%
\pgfpathlineto{\pgfqpoint{4.287030in}{3.942763in}}%
\pgfpathlineto{\pgfqpoint{4.266701in}{3.962667in}}%
\pgfpathlineto{\pgfqpoint{4.257019in}{3.972046in}}%
\pgfpathlineto{\pgfqpoint{4.246949in}{3.981990in}}%
\pgfpathlineto{\pgfqpoint{4.228492in}{4.000000in}}%
\pgfpathlineto{\pgfqpoint{4.217875in}{4.010252in}}%
\pgfpathlineto{\pgfqpoint{4.206869in}{4.021085in}}%
\pgfpathlineto{\pgfqpoint{4.198293in}{4.029346in}}%
\pgfpathlineto{\pgfqpoint{4.190161in}{4.037333in}}%
\pgfpathlineto{\pgfqpoint{4.166788in}{4.060047in}}%
\pgfpathlineto{\pgfqpoint{4.159059in}{4.067468in}}%
\pgfpathlineto{\pgfqpoint{4.151705in}{4.074667in}}%
\pgfpathlineto{\pgfqpoint{4.126707in}{4.098879in}}%
\pgfpathlineto{\pgfqpoint{4.113124in}{4.112000in}}%
\pgfpathlineto{\pgfqpoint{4.100047in}{4.124500in}}%
\pgfpathlineto{\pgfqpoint{4.086626in}{4.137579in}}%
\pgfpathlineto{\pgfqpoint{4.074417in}{4.149333in}}%
\pgfpathlineto{\pgfqpoint{4.046545in}{4.176148in}}%
\pgfpathlineto{\pgfqpoint{4.040956in}{4.181460in}}%
\pgfpathlineto{\pgfqpoint{4.035583in}{4.186667in}}%
\pgfpathlineto{\pgfqpoint{4.021163in}{4.200358in}}%
\pgfpathlineto{\pgfqpoint{4.006465in}{4.214586in}}%
\pgfpathlineto{\pgfqpoint{4.001453in}{4.219332in}}%
\pgfpathlineto{\pgfqpoint{3.996620in}{4.224000in}}%
\pgfpathlineto{\pgfqpoint{3.993703in}{4.224000in}}%
\pgfpathlineto{\pgfqpoint{3.999969in}{4.217949in}}%
\pgfpathlineto{\pgfqpoint{4.006465in}{4.211797in}}%
\pgfpathlineto{\pgfqpoint{4.019695in}{4.198990in}}%
\pgfpathlineto{\pgfqpoint{4.032674in}{4.186667in}}%
\pgfpathlineto{\pgfqpoint{4.039472in}{4.180078in}}%
\pgfpathlineto{\pgfqpoint{4.046545in}{4.173357in}}%
\pgfpathlineto{\pgfqpoint{4.071516in}{4.149333in}}%
\pgfpathlineto{\pgfqpoint{4.086626in}{4.134786in}}%
\pgfpathlineto{\pgfqpoint{4.098581in}{4.123135in}}%
\pgfpathlineto{\pgfqpoint{4.110230in}{4.112000in}}%
\pgfpathlineto{\pgfqpoint{4.126707in}{4.096084in}}%
\pgfpathlineto{\pgfqpoint{4.148819in}{4.074667in}}%
\pgfpathlineto{\pgfqpoint{4.157580in}{4.066090in}}%
\pgfpathlineto{\pgfqpoint{4.166788in}{4.057250in}}%
\pgfpathlineto{\pgfqpoint{4.187282in}{4.037333in}}%
\pgfpathlineto{\pgfqpoint{4.196816in}{4.027970in}}%
\pgfpathlineto{\pgfqpoint{4.206869in}{4.018286in}}%
\pgfpathlineto{\pgfqpoint{4.216414in}{4.008891in}}%
\pgfpathlineto{\pgfqpoint{4.225622in}{4.000000in}}%
\pgfpathlineto{\pgfqpoint{4.246949in}{3.979189in}}%
\pgfpathlineto{\pgfqpoint{4.255559in}{3.970686in}}%
\pgfpathlineto{\pgfqpoint{4.263838in}{3.962667in}}%
\pgfpathlineto{\pgfqpoint{4.287030in}{3.939960in}}%
\pgfpathlineto{\pgfqpoint{4.301931in}{3.925333in}}%
\pgfpathlineto{\pgfqpoint{4.314124in}{3.913237in}}%
\pgfpathlineto{\pgfqpoint{4.327111in}{3.900599in}}%
\pgfpathlineto{\pgfqpoint{4.339904in}{3.888000in}}%
\pgfpathlineto{\pgfqpoint{4.353095in}{3.874869in}}%
\pgfpathlineto{\pgfqpoint{4.367192in}{3.861106in}}%
\pgfpathlineto{\pgfqpoint{4.377756in}{3.850667in}}%
\pgfpathlineto{\pgfqpoint{4.392000in}{3.836441in}}%
\pgfpathlineto{\pgfqpoint{4.407273in}{3.821479in}}%
\pgfpathlineto{\pgfqpoint{4.415489in}{3.813333in}}%
\pgfpathlineto{\pgfqpoint{4.430839in}{3.797951in}}%
\pgfpathlineto{\pgfqpoint{4.447354in}{3.781719in}}%
\pgfpathlineto{\pgfqpoint{4.450308in}{3.778752in}}%
\pgfpathlineto{\pgfqpoint{4.453103in}{3.776000in}}%
\pgfpathlineto{\pgfqpoint{4.469613in}{3.759400in}}%
\pgfpathlineto{\pgfqpoint{4.487434in}{3.741826in}}%
\pgfpathlineto{\pgfqpoint{4.489064in}{3.740184in}}%
\pgfpathlineto{\pgfqpoint{4.490600in}{3.738667in}}%
\pgfpathlineto{\pgfqpoint{4.527515in}{3.701798in}}%
\pgfpathlineto{\pgfqpoint{4.527980in}{3.701333in}}%
\pgfpathlineto{\pgfqpoint{4.546966in}{3.682118in}}%
\pgfpathlineto{\pgfqpoint{4.565216in}{3.664000in}}%
\pgfpathlineto{\pgfqpoint{4.566356in}{3.662845in}}%
\pgfpathlineto{\pgfqpoint{4.567596in}{3.661612in}}%
\pgfpathlineto{\pgfqpoint{4.602331in}{3.626667in}}%
\pgfpathlineto{\pgfqpoint{4.604886in}{3.624068in}}%
\pgfpathlineto{\pgfqpoint{4.607677in}{3.621284in}}%
\pgfpathlineto{\pgfqpoint{4.639330in}{3.589333in}}%
\pgfpathlineto{\pgfqpoint{4.643352in}{3.585230in}}%
\pgfpathlineto{\pgfqpoint{4.647758in}{3.580821in}}%
\pgfpathlineto{\pgfqpoint{4.676216in}{3.552000in}}%
\pgfpathlineto{\pgfqpoint{4.681753in}{3.546331in}}%
\pgfpathlineto{\pgfqpoint{4.687838in}{3.540221in}}%
\pgfpathlineto{\pgfqpoint{4.712988in}{3.514667in}}%
\pgfpathlineto{\pgfqpoint{4.720089in}{3.507373in}}%
\pgfpathlineto{\pgfqpoint{4.727919in}{3.499484in}}%
\pgfpathlineto{\pgfqpoint{4.749647in}{3.477333in}}%
\pgfpathlineto{\pgfqpoint{4.758360in}{3.468354in}}%
\pgfpathlineto{\pgfqpoint{4.768000in}{3.458610in}}%
\pgfusepath{fill}%
\end{pgfscope}%
\begin{pgfscope}%
\pgfpathrectangle{\pgfqpoint{0.800000in}{0.528000in}}{\pgfqpoint{3.968000in}{3.696000in}}%
\pgfusepath{clip}%
\pgfsetbuttcap%
\pgfsetroundjoin%
\definecolor{currentfill}{rgb}{0.246070,0.738910,0.452024}%
\pgfsetfillcolor{currentfill}%
\pgfsetlinewidth{0.000000pt}%
\definecolor{currentstroke}{rgb}{0.000000,0.000000,0.000000}%
\pgfsetstrokecolor{currentstroke}%
\pgfsetdash{}{0pt}%
\pgfpathmoveto{\pgfqpoint{4.768000in}{3.464325in}}%
\pgfpathlineto{\pgfqpoint{4.761302in}{3.471095in}}%
\pgfpathlineto{\pgfqpoint{4.755249in}{3.477333in}}%
\pgfpathlineto{\pgfqpoint{4.727919in}{3.505194in}}%
\pgfpathlineto{\pgfqpoint{4.723034in}{3.510116in}}%
\pgfpathlineto{\pgfqpoint{4.718604in}{3.514667in}}%
\pgfpathlineto{\pgfqpoint{4.687838in}{3.545927in}}%
\pgfpathlineto{\pgfqpoint{4.684701in}{3.549077in}}%
\pgfpathlineto{\pgfqpoint{4.681846in}{3.552000in}}%
\pgfpathlineto{\pgfqpoint{4.647758in}{3.586523in}}%
\pgfpathlineto{\pgfqpoint{4.646303in}{3.587979in}}%
\pgfpathlineto{\pgfqpoint{4.644975in}{3.589333in}}%
\pgfpathlineto{\pgfqpoint{4.611729in}{3.622892in}}%
\pgfpathlineto{\pgfqpoint{4.607987in}{3.626667in}}%
\pgfpathlineto{\pgfqpoint{4.607837in}{3.626816in}}%
\pgfpathlineto{\pgfqpoint{4.607677in}{3.626979in}}%
\pgfpathlineto{\pgfqpoint{4.570852in}{3.664000in}}%
\pgfpathlineto{\pgfqpoint{4.567596in}{3.667271in}}%
\pgfpathlineto{\pgfqpoint{4.549894in}{3.684845in}}%
\pgfpathlineto{\pgfqpoint{4.533602in}{3.701333in}}%
\pgfpathlineto{\pgfqpoint{4.527515in}{3.707428in}}%
\pgfpathlineto{\pgfqpoint{4.496236in}{3.738667in}}%
\pgfpathlineto{\pgfqpoint{4.491965in}{3.742887in}}%
\pgfpathlineto{\pgfqpoint{4.487434in}{3.747451in}}%
\pgfpathlineto{\pgfqpoint{4.472547in}{3.762133in}}%
\pgfpathlineto{\pgfqpoint{4.458754in}{3.776000in}}%
\pgfpathlineto{\pgfqpoint{4.453213in}{3.781458in}}%
\pgfpathlineto{\pgfqpoint{4.447354in}{3.787341in}}%
\pgfpathlineto{\pgfqpoint{4.433776in}{3.800686in}}%
\pgfpathlineto{\pgfqpoint{4.421155in}{3.813333in}}%
\pgfpathlineto{\pgfqpoint{4.407273in}{3.827096in}}%
\pgfpathlineto{\pgfqpoint{4.394939in}{3.839179in}}%
\pgfpathlineto{\pgfqpoint{4.383437in}{3.850667in}}%
\pgfpathlineto{\pgfqpoint{4.367192in}{3.866719in}}%
\pgfpathlineto{\pgfqpoint{4.356037in}{3.877610in}}%
\pgfpathlineto{\pgfqpoint{4.345600in}{3.888000in}}%
\pgfpathlineto{\pgfqpoint{4.327111in}{3.906209in}}%
\pgfpathlineto{\pgfqpoint{4.317070in}{3.915980in}}%
\pgfpathlineto{\pgfqpoint{4.307642in}{3.925333in}}%
\pgfpathlineto{\pgfqpoint{4.287030in}{3.945566in}}%
\pgfpathlineto{\pgfqpoint{4.269563in}{3.962667in}}%
\pgfpathlineto{\pgfqpoint{4.258479in}{3.973405in}}%
\pgfpathlineto{\pgfqpoint{4.246949in}{3.984791in}}%
\pgfpathlineto{\pgfqpoint{4.231363in}{4.000000in}}%
\pgfpathlineto{\pgfqpoint{4.219336in}{4.011612in}}%
\pgfpathlineto{\pgfqpoint{4.206869in}{4.023884in}}%
\pgfpathlineto{\pgfqpoint{4.199770in}{4.030722in}}%
\pgfpathlineto{\pgfqpoint{4.193039in}{4.037333in}}%
\pgfpathlineto{\pgfqpoint{4.166788in}{4.062844in}}%
\pgfpathlineto{\pgfqpoint{4.160538in}{4.068845in}}%
\pgfpathlineto{\pgfqpoint{4.154591in}{4.074667in}}%
\pgfpathlineto{\pgfqpoint{4.126707in}{4.101674in}}%
\pgfpathlineto{\pgfqpoint{4.116017in}{4.112000in}}%
\pgfpathlineto{\pgfqpoint{4.101512in}{4.125866in}}%
\pgfpathlineto{\pgfqpoint{4.086626in}{4.140372in}}%
\pgfpathlineto{\pgfqpoint{4.077318in}{4.149333in}}%
\pgfpathlineto{\pgfqpoint{4.046545in}{4.178939in}}%
\pgfpathlineto{\pgfqpoint{4.042439in}{4.182842in}}%
\pgfpathlineto{\pgfqpoint{4.038492in}{4.186667in}}%
\pgfpathlineto{\pgfqpoint{4.022632in}{4.201725in}}%
\pgfpathlineto{\pgfqpoint{4.006465in}{4.217375in}}%
\pgfpathlineto{\pgfqpoint{4.002938in}{4.220715in}}%
\pgfpathlineto{\pgfqpoint{3.999537in}{4.224000in}}%
\pgfpathlineto{\pgfqpoint{3.996620in}{4.224000in}}%
\pgfpathlineto{\pgfqpoint{4.001453in}{4.219332in}}%
\pgfpathlineto{\pgfqpoint{4.006465in}{4.214586in}}%
\pgfpathlineto{\pgfqpoint{4.021163in}{4.200358in}}%
\pgfpathlineto{\pgfqpoint{4.035583in}{4.186667in}}%
\pgfpathlineto{\pgfqpoint{4.040956in}{4.181460in}}%
\pgfpathlineto{\pgfqpoint{4.046545in}{4.176148in}}%
\pgfpathlineto{\pgfqpoint{4.074417in}{4.149333in}}%
\pgfpathlineto{\pgfqpoint{4.086626in}{4.137579in}}%
\pgfpathlineto{\pgfqpoint{4.100047in}{4.124500in}}%
\pgfpathlineto{\pgfqpoint{4.113124in}{4.112000in}}%
\pgfpathlineto{\pgfqpoint{4.126707in}{4.098879in}}%
\pgfpathlineto{\pgfqpoint{4.151705in}{4.074667in}}%
\pgfpathlineto{\pgfqpoint{4.159059in}{4.067468in}}%
\pgfpathlineto{\pgfqpoint{4.166788in}{4.060047in}}%
\pgfpathlineto{\pgfqpoint{4.190161in}{4.037333in}}%
\pgfpathlineto{\pgfqpoint{4.198293in}{4.029346in}}%
\pgfpathlineto{\pgfqpoint{4.206869in}{4.021085in}}%
\pgfpathlineto{\pgfqpoint{4.217875in}{4.010252in}}%
\pgfpathlineto{\pgfqpoint{4.228492in}{4.000000in}}%
\pgfpathlineto{\pgfqpoint{4.246949in}{3.981990in}}%
\pgfpathlineto{\pgfqpoint{4.257019in}{3.972046in}}%
\pgfpathlineto{\pgfqpoint{4.266701in}{3.962667in}}%
\pgfpathlineto{\pgfqpoint{4.287030in}{3.942763in}}%
\pgfpathlineto{\pgfqpoint{4.304787in}{3.925333in}}%
\pgfpathlineto{\pgfqpoint{4.315597in}{3.914608in}}%
\pgfpathlineto{\pgfqpoint{4.327111in}{3.903404in}}%
\pgfpathlineto{\pgfqpoint{4.342752in}{3.888000in}}%
\pgfpathlineto{\pgfqpoint{4.354566in}{3.876240in}}%
\pgfpathlineto{\pgfqpoint{4.367192in}{3.863912in}}%
\pgfpathlineto{\pgfqpoint{4.380596in}{3.850667in}}%
\pgfpathlineto{\pgfqpoint{4.393469in}{3.837810in}}%
\pgfpathlineto{\pgfqpoint{4.407273in}{3.824288in}}%
\pgfpathlineto{\pgfqpoint{4.418322in}{3.813333in}}%
\pgfpathlineto{\pgfqpoint{4.432307in}{3.799319in}}%
\pgfpathlineto{\pgfqpoint{4.447354in}{3.784530in}}%
\pgfpathlineto{\pgfqpoint{4.451761in}{3.780105in}}%
\pgfpathlineto{\pgfqpoint{4.455929in}{3.776000in}}%
\pgfpathlineto{\pgfqpoint{4.471080in}{3.760767in}}%
\pgfpathlineto{\pgfqpoint{4.487434in}{3.744638in}}%
\pgfpathlineto{\pgfqpoint{4.490515in}{3.741536in}}%
\pgfpathlineto{\pgfqpoint{4.493418in}{3.738667in}}%
\pgfpathlineto{\pgfqpoint{4.527515in}{3.704613in}}%
\pgfpathlineto{\pgfqpoint{4.530791in}{3.701333in}}%
\pgfpathlineto{\pgfqpoint{4.548430in}{3.683481in}}%
\pgfpathlineto{\pgfqpoint{4.567596in}{3.664454in}}%
\pgfpathlineto{\pgfqpoint{4.568048in}{3.664000in}}%
\pgfpathlineto{\pgfqpoint{4.573747in}{3.658271in}}%
\pgfpathlineto{\pgfqpoint{4.605161in}{3.626667in}}%
\pgfpathlineto{\pgfqpoint{4.606363in}{3.625443in}}%
\pgfpathlineto{\pgfqpoint{4.607677in}{3.624133in}}%
\pgfpathlineto{\pgfqpoint{4.642153in}{3.589333in}}%
\pgfpathlineto{\pgfqpoint{4.644828in}{3.586604in}}%
\pgfpathlineto{\pgfqpoint{4.647758in}{3.583672in}}%
\pgfpathlineto{\pgfqpoint{4.679031in}{3.552000in}}%
\pgfpathlineto{\pgfqpoint{4.683227in}{3.547704in}}%
\pgfpathlineto{\pgfqpoint{4.687838in}{3.543074in}}%
\pgfpathlineto{\pgfqpoint{4.715796in}{3.514667in}}%
\pgfpathlineto{\pgfqpoint{4.721561in}{3.508745in}}%
\pgfpathlineto{\pgfqpoint{4.727919in}{3.502339in}}%
\pgfpathlineto{\pgfqpoint{4.752448in}{3.477333in}}%
\pgfpathlineto{\pgfqpoint{4.759831in}{3.469725in}}%
\pgfpathlineto{\pgfqpoint{4.768000in}{3.461468in}}%
\pgfusepath{fill}%
\end{pgfscope}%
\begin{pgfscope}%
\pgfpathrectangle{\pgfqpoint{0.800000in}{0.528000in}}{\pgfqpoint{3.968000in}{3.696000in}}%
\pgfusepath{clip}%
\pgfsetbuttcap%
\pgfsetroundjoin%
\definecolor{currentfill}{rgb}{0.246070,0.738910,0.452024}%
\pgfsetfillcolor{currentfill}%
\pgfsetlinewidth{0.000000pt}%
\definecolor{currentstroke}{rgb}{0.000000,0.000000,0.000000}%
\pgfsetstrokecolor{currentstroke}%
\pgfsetdash{}{0pt}%
\pgfpathmoveto{\pgfqpoint{4.768000in}{3.467182in}}%
\pgfpathlineto{\pgfqpoint{4.762773in}{3.472465in}}%
\pgfpathlineto{\pgfqpoint{4.758050in}{3.477333in}}%
\pgfpathlineto{\pgfqpoint{4.727919in}{3.508050in}}%
\pgfpathlineto{\pgfqpoint{4.724506in}{3.511488in}}%
\pgfpathlineto{\pgfqpoint{4.721412in}{3.514667in}}%
\pgfpathlineto{\pgfqpoint{4.687838in}{3.548780in}}%
\pgfpathlineto{\pgfqpoint{4.686175in}{3.550451in}}%
\pgfpathlineto{\pgfqpoint{4.684661in}{3.552000in}}%
\pgfpathlineto{\pgfqpoint{4.648259in}{3.588866in}}%
\pgfpathlineto{\pgfqpoint{4.647797in}{3.589333in}}%
\pgfpathlineto{\pgfqpoint{4.647758in}{3.589374in}}%
\pgfpathlineto{\pgfqpoint{4.610783in}{3.626667in}}%
\pgfpathlineto{\pgfqpoint{4.609284in}{3.628163in}}%
\pgfpathlineto{\pgfqpoint{4.607677in}{3.629798in}}%
\pgfpathlineto{\pgfqpoint{4.573656in}{3.664000in}}%
\pgfpathlineto{\pgfqpoint{4.567596in}{3.670087in}}%
\pgfpathlineto{\pgfqpoint{4.551358in}{3.686208in}}%
\pgfpathlineto{\pgfqpoint{4.536413in}{3.701333in}}%
\pgfpathlineto{\pgfqpoint{4.527515in}{3.710243in}}%
\pgfpathlineto{\pgfqpoint{4.499055in}{3.738667in}}%
\pgfpathlineto{\pgfqpoint{4.493416in}{3.744239in}}%
\pgfpathlineto{\pgfqpoint{4.487434in}{3.750264in}}%
\pgfpathlineto{\pgfqpoint{4.474014in}{3.763499in}}%
\pgfpathlineto{\pgfqpoint{4.461580in}{3.776000in}}%
\pgfpathlineto{\pgfqpoint{4.454665in}{3.782811in}}%
\pgfpathlineto{\pgfqpoint{4.447354in}{3.790151in}}%
\pgfpathlineto{\pgfqpoint{4.435244in}{3.802054in}}%
\pgfpathlineto{\pgfqpoint{4.423988in}{3.813333in}}%
\pgfpathlineto{\pgfqpoint{4.407273in}{3.829905in}}%
\pgfpathlineto{\pgfqpoint{4.396409in}{3.840548in}}%
\pgfpathlineto{\pgfqpoint{4.386277in}{3.850667in}}%
\pgfpathlineto{\pgfqpoint{4.367192in}{3.869526in}}%
\pgfpathlineto{\pgfqpoint{4.357509in}{3.878980in}}%
\pgfpathlineto{\pgfqpoint{4.348448in}{3.888000in}}%
\pgfpathlineto{\pgfqpoint{4.327111in}{3.909014in}}%
\pgfpathlineto{\pgfqpoint{4.318542in}{3.917352in}}%
\pgfpathlineto{\pgfqpoint{4.310498in}{3.925333in}}%
\pgfpathlineto{\pgfqpoint{4.287030in}{3.948369in}}%
\pgfpathlineto{\pgfqpoint{4.272426in}{3.962667in}}%
\pgfpathlineto{\pgfqpoint{4.259938in}{3.974765in}}%
\pgfpathlineto{\pgfqpoint{4.246949in}{3.987592in}}%
\pgfpathlineto{\pgfqpoint{4.234233in}{4.000000in}}%
\pgfpathlineto{\pgfqpoint{4.220797in}{4.012973in}}%
\pgfpathlineto{\pgfqpoint{4.206869in}{4.026683in}}%
\pgfpathlineto{\pgfqpoint{4.201248in}{4.032097in}}%
\pgfpathlineto{\pgfqpoint{4.195917in}{4.037333in}}%
\pgfpathlineto{\pgfqpoint{4.166788in}{4.065641in}}%
\pgfpathlineto{\pgfqpoint{4.162016in}{4.070222in}}%
\pgfpathlineto{\pgfqpoint{4.157476in}{4.074667in}}%
\pgfpathlineto{\pgfqpoint{4.126707in}{4.104469in}}%
\pgfpathlineto{\pgfqpoint{4.118911in}{4.112000in}}%
\pgfpathlineto{\pgfqpoint{4.102978in}{4.127231in}}%
\pgfpathlineto{\pgfqpoint{4.086626in}{4.143165in}}%
\pgfpathlineto{\pgfqpoint{4.080219in}{4.149333in}}%
\pgfpathlineto{\pgfqpoint{4.046545in}{4.181730in}}%
\pgfpathlineto{\pgfqpoint{4.043922in}{4.184223in}}%
\pgfpathlineto{\pgfqpoint{4.041401in}{4.186667in}}%
\pgfpathlineto{\pgfqpoint{4.024100in}{4.203093in}}%
\pgfpathlineto{\pgfqpoint{4.006465in}{4.220165in}}%
\pgfpathlineto{\pgfqpoint{4.004423in}{4.222098in}}%
\pgfpathlineto{\pgfqpoint{4.002454in}{4.224000in}}%
\pgfpathlineto{\pgfqpoint{3.999537in}{4.224000in}}%
\pgfpathlineto{\pgfqpoint{4.002938in}{4.220715in}}%
\pgfpathlineto{\pgfqpoint{4.006465in}{4.217375in}}%
\pgfpathlineto{\pgfqpoint{4.022632in}{4.201725in}}%
\pgfpathlineto{\pgfqpoint{4.038492in}{4.186667in}}%
\pgfpathlineto{\pgfqpoint{4.042439in}{4.182842in}}%
\pgfpathlineto{\pgfqpoint{4.046545in}{4.178939in}}%
\pgfpathlineto{\pgfqpoint{4.077318in}{4.149333in}}%
\pgfpathlineto{\pgfqpoint{4.086626in}{4.140372in}}%
\pgfpathlineto{\pgfqpoint{4.101512in}{4.125866in}}%
\pgfpathlineto{\pgfqpoint{4.116017in}{4.112000in}}%
\pgfpathlineto{\pgfqpoint{4.126707in}{4.101674in}}%
\pgfpathlineto{\pgfqpoint{4.154591in}{4.074667in}}%
\pgfpathlineto{\pgfqpoint{4.160538in}{4.068845in}}%
\pgfpathlineto{\pgfqpoint{4.166788in}{4.062844in}}%
\pgfpathlineto{\pgfqpoint{4.193039in}{4.037333in}}%
\pgfpathlineto{\pgfqpoint{4.199770in}{4.030722in}}%
\pgfpathlineto{\pgfqpoint{4.206869in}{4.023884in}}%
\pgfpathlineto{\pgfqpoint{4.219336in}{4.011612in}}%
\pgfpathlineto{\pgfqpoint{4.231363in}{4.000000in}}%
\pgfpathlineto{\pgfqpoint{4.246949in}{3.984791in}}%
\pgfpathlineto{\pgfqpoint{4.258479in}{3.973405in}}%
\pgfpathlineto{\pgfqpoint{4.269563in}{3.962667in}}%
\pgfpathlineto{\pgfqpoint{4.287030in}{3.945566in}}%
\pgfpathlineto{\pgfqpoint{4.307642in}{3.925333in}}%
\pgfpathlineto{\pgfqpoint{4.317070in}{3.915980in}}%
\pgfpathlineto{\pgfqpoint{4.327111in}{3.906209in}}%
\pgfpathlineto{\pgfqpoint{4.345600in}{3.888000in}}%
\pgfpathlineto{\pgfqpoint{4.356037in}{3.877610in}}%
\pgfpathlineto{\pgfqpoint{4.367192in}{3.866719in}}%
\pgfpathlineto{\pgfqpoint{4.383437in}{3.850667in}}%
\pgfpathlineto{\pgfqpoint{4.394939in}{3.839179in}}%
\pgfpathlineto{\pgfqpoint{4.407273in}{3.827096in}}%
\pgfpathlineto{\pgfqpoint{4.421155in}{3.813333in}}%
\pgfpathlineto{\pgfqpoint{4.433776in}{3.800686in}}%
\pgfpathlineto{\pgfqpoint{4.447354in}{3.787341in}}%
\pgfpathlineto{\pgfqpoint{4.453213in}{3.781458in}}%
\pgfpathlineto{\pgfqpoint{4.458754in}{3.776000in}}%
\pgfpathlineto{\pgfqpoint{4.472547in}{3.762133in}}%
\pgfpathlineto{\pgfqpoint{4.487434in}{3.747451in}}%
\pgfpathlineto{\pgfqpoint{4.491965in}{3.742887in}}%
\pgfpathlineto{\pgfqpoint{4.496236in}{3.738667in}}%
\pgfpathlineto{\pgfqpoint{4.527515in}{3.707428in}}%
\pgfpathlineto{\pgfqpoint{4.533602in}{3.701333in}}%
\pgfpathlineto{\pgfqpoint{4.549894in}{3.684845in}}%
\pgfpathlineto{\pgfqpoint{4.567596in}{3.667271in}}%
\pgfpathlineto{\pgfqpoint{4.570852in}{3.664000in}}%
\pgfpathlineto{\pgfqpoint{4.607677in}{3.626979in}}%
\pgfpathlineto{\pgfqpoint{4.607837in}{3.626816in}}%
\pgfpathlineto{\pgfqpoint{4.607987in}{3.626667in}}%
\pgfpathlineto{\pgfqpoint{4.611729in}{3.622892in}}%
\pgfpathlineto{\pgfqpoint{4.644975in}{3.589333in}}%
\pgfpathlineto{\pgfqpoint{4.646303in}{3.587979in}}%
\pgfpathlineto{\pgfqpoint{4.647758in}{3.586523in}}%
\pgfpathlineto{\pgfqpoint{4.681846in}{3.552000in}}%
\pgfpathlineto{\pgfqpoint{4.684701in}{3.549077in}}%
\pgfpathlineto{\pgfqpoint{4.687838in}{3.545927in}}%
\pgfpathlineto{\pgfqpoint{4.718604in}{3.514667in}}%
\pgfpathlineto{\pgfqpoint{4.723034in}{3.510116in}}%
\pgfpathlineto{\pgfqpoint{4.727919in}{3.505194in}}%
\pgfpathlineto{\pgfqpoint{4.755249in}{3.477333in}}%
\pgfpathlineto{\pgfqpoint{4.761302in}{3.471095in}}%
\pgfpathlineto{\pgfqpoint{4.768000in}{3.464325in}}%
\pgfusepath{fill}%
\end{pgfscope}%
\begin{pgfscope}%
\pgfpathrectangle{\pgfqpoint{0.800000in}{0.528000in}}{\pgfqpoint{3.968000in}{3.696000in}}%
\pgfusepath{clip}%
\pgfsetbuttcap%
\pgfsetroundjoin%
\definecolor{currentfill}{rgb}{0.246070,0.738910,0.452024}%
\pgfsetfillcolor{currentfill}%
\pgfsetlinewidth{0.000000pt}%
\definecolor{currentstroke}{rgb}{0.000000,0.000000,0.000000}%
\pgfsetstrokecolor{currentstroke}%
\pgfsetdash{}{0pt}%
\pgfpathmoveto{\pgfqpoint{4.768000in}{3.470039in}}%
\pgfpathlineto{\pgfqpoint{4.764245in}{3.473835in}}%
\pgfpathlineto{\pgfqpoint{4.760850in}{3.477333in}}%
\pgfpathlineto{\pgfqpoint{4.727919in}{3.510905in}}%
\pgfpathlineto{\pgfqpoint{4.725979in}{3.512859in}}%
\pgfpathlineto{\pgfqpoint{4.724220in}{3.514667in}}%
\pgfpathlineto{\pgfqpoint{4.687838in}{3.551633in}}%
\pgfpathlineto{\pgfqpoint{4.687649in}{3.551824in}}%
\pgfpathlineto{\pgfqpoint{4.687477in}{3.552000in}}%
\pgfpathlineto{\pgfqpoint{4.683332in}{3.556198in}}%
\pgfpathlineto{\pgfqpoint{4.650587in}{3.589333in}}%
\pgfpathlineto{\pgfqpoint{4.647758in}{3.592194in}}%
\pgfpathlineto{\pgfqpoint{4.613580in}{3.626667in}}%
\pgfpathlineto{\pgfqpoint{4.610730in}{3.629511in}}%
\pgfpathlineto{\pgfqpoint{4.607677in}{3.632617in}}%
\pgfpathlineto{\pgfqpoint{4.576459in}{3.664000in}}%
\pgfpathlineto{\pgfqpoint{4.567596in}{3.672904in}}%
\pgfpathlineto{\pgfqpoint{4.552822in}{3.687572in}}%
\pgfpathlineto{\pgfqpoint{4.539224in}{3.701333in}}%
\pgfpathlineto{\pgfqpoint{4.527515in}{3.713057in}}%
\pgfpathlineto{\pgfqpoint{4.501873in}{3.738667in}}%
\pgfpathlineto{\pgfqpoint{4.494867in}{3.745590in}}%
\pgfpathlineto{\pgfqpoint{4.487434in}{3.753077in}}%
\pgfpathlineto{\pgfqpoint{4.475480in}{3.764865in}}%
\pgfpathlineto{\pgfqpoint{4.464406in}{3.776000in}}%
\pgfpathlineto{\pgfqpoint{4.456118in}{3.784163in}}%
\pgfpathlineto{\pgfqpoint{4.447354in}{3.792962in}}%
\pgfpathlineto{\pgfqpoint{4.436712in}{3.803421in}}%
\pgfpathlineto{\pgfqpoint{4.426821in}{3.813333in}}%
\pgfpathlineto{\pgfqpoint{4.407273in}{3.832714in}}%
\pgfpathlineto{\pgfqpoint{4.397879in}{3.841917in}}%
\pgfpathlineto{\pgfqpoint{4.389118in}{3.850667in}}%
\pgfpathlineto{\pgfqpoint{4.367192in}{3.872333in}}%
\pgfpathlineto{\pgfqpoint{4.358980in}{3.880351in}}%
\pgfpathlineto{\pgfqpoint{4.351295in}{3.888000in}}%
\pgfpathlineto{\pgfqpoint{4.327111in}{3.911819in}}%
\pgfpathlineto{\pgfqpoint{4.320015in}{3.918724in}}%
\pgfpathlineto{\pgfqpoint{4.313353in}{3.925333in}}%
\pgfpathlineto{\pgfqpoint{4.287030in}{3.951172in}}%
\pgfpathlineto{\pgfqpoint{4.275289in}{3.962667in}}%
\pgfpathlineto{\pgfqpoint{4.261398in}{3.976124in}}%
\pgfpathlineto{\pgfqpoint{4.246949in}{3.990393in}}%
\pgfpathlineto{\pgfqpoint{4.237104in}{4.000000in}}%
\pgfpathlineto{\pgfqpoint{4.222258in}{4.014334in}}%
\pgfpathlineto{\pgfqpoint{4.206869in}{4.029482in}}%
\pgfpathlineto{\pgfqpoint{4.202725in}{4.033473in}}%
\pgfpathlineto{\pgfqpoint{4.198795in}{4.037333in}}%
\pgfpathlineto{\pgfqpoint{4.166788in}{4.068439in}}%
\pgfpathlineto{\pgfqpoint{4.163495in}{4.071600in}}%
\pgfpathlineto{\pgfqpoint{4.160362in}{4.074667in}}%
\pgfpathlineto{\pgfqpoint{4.126707in}{4.107264in}}%
\pgfpathlineto{\pgfqpoint{4.121804in}{4.112000in}}%
\pgfpathlineto{\pgfqpoint{4.104443in}{4.128596in}}%
\pgfpathlineto{\pgfqpoint{4.086626in}{4.145958in}}%
\pgfpathlineto{\pgfqpoint{4.083120in}{4.149333in}}%
\pgfpathlineto{\pgfqpoint{4.046545in}{4.184521in}}%
\pgfpathlineto{\pgfqpoint{4.045405in}{4.185605in}}%
\pgfpathlineto{\pgfqpoint{4.044310in}{4.186667in}}%
\pgfpathlineto{\pgfqpoint{4.025569in}{4.204461in}}%
\pgfpathlineto{\pgfqpoint{4.006465in}{4.222954in}}%
\pgfpathlineto{\pgfqpoint{4.005908in}{4.223481in}}%
\pgfpathlineto{\pgfqpoint{4.005371in}{4.224000in}}%
\pgfpathlineto{\pgfqpoint{4.002454in}{4.224000in}}%
\pgfpathlineto{\pgfqpoint{4.004423in}{4.222098in}}%
\pgfpathlineto{\pgfqpoint{4.006465in}{4.220165in}}%
\pgfpathlineto{\pgfqpoint{4.024100in}{4.203093in}}%
\pgfpathlineto{\pgfqpoint{4.041401in}{4.186667in}}%
\pgfpathlineto{\pgfqpoint{4.043922in}{4.184223in}}%
\pgfpathlineto{\pgfqpoint{4.046545in}{4.181730in}}%
\pgfpathlineto{\pgfqpoint{4.080219in}{4.149333in}}%
\pgfpathlineto{\pgfqpoint{4.086626in}{4.143165in}}%
\pgfpathlineto{\pgfqpoint{4.102978in}{4.127231in}}%
\pgfpathlineto{\pgfqpoint{4.118911in}{4.112000in}}%
\pgfpathlineto{\pgfqpoint{4.126707in}{4.104469in}}%
\pgfpathlineto{\pgfqpoint{4.157476in}{4.074667in}}%
\pgfpathlineto{\pgfqpoint{4.162016in}{4.070222in}}%
\pgfpathlineto{\pgfqpoint{4.166788in}{4.065641in}}%
\pgfpathlineto{\pgfqpoint{4.195917in}{4.037333in}}%
\pgfpathlineto{\pgfqpoint{4.201248in}{4.032097in}}%
\pgfpathlineto{\pgfqpoint{4.206869in}{4.026683in}}%
\pgfpathlineto{\pgfqpoint{4.220797in}{4.012973in}}%
\pgfpathlineto{\pgfqpoint{4.234233in}{4.000000in}}%
\pgfpathlineto{\pgfqpoint{4.246949in}{3.987592in}}%
\pgfpathlineto{\pgfqpoint{4.259938in}{3.974765in}}%
\pgfpathlineto{\pgfqpoint{4.272426in}{3.962667in}}%
\pgfpathlineto{\pgfqpoint{4.287030in}{3.948369in}}%
\pgfpathlineto{\pgfqpoint{4.310498in}{3.925333in}}%
\pgfpathlineto{\pgfqpoint{4.318542in}{3.917352in}}%
\pgfpathlineto{\pgfqpoint{4.327111in}{3.909014in}}%
\pgfpathlineto{\pgfqpoint{4.348448in}{3.888000in}}%
\pgfpathlineto{\pgfqpoint{4.357509in}{3.878980in}}%
\pgfpathlineto{\pgfqpoint{4.367192in}{3.869526in}}%
\pgfpathlineto{\pgfqpoint{4.386277in}{3.850667in}}%
\pgfpathlineto{\pgfqpoint{4.396409in}{3.840548in}}%
\pgfpathlineto{\pgfqpoint{4.407273in}{3.829905in}}%
\pgfpathlineto{\pgfqpoint{4.423988in}{3.813333in}}%
\pgfpathlineto{\pgfqpoint{4.435244in}{3.802054in}}%
\pgfpathlineto{\pgfqpoint{4.447354in}{3.790151in}}%
\pgfpathlineto{\pgfqpoint{4.454665in}{3.782811in}}%
\pgfpathlineto{\pgfqpoint{4.461580in}{3.776000in}}%
\pgfpathlineto{\pgfqpoint{4.474014in}{3.763499in}}%
\pgfpathlineto{\pgfqpoint{4.487434in}{3.750264in}}%
\pgfpathlineto{\pgfqpoint{4.493416in}{3.744239in}}%
\pgfpathlineto{\pgfqpoint{4.499055in}{3.738667in}}%
\pgfpathlineto{\pgfqpoint{4.527515in}{3.710243in}}%
\pgfpathlineto{\pgfqpoint{4.536413in}{3.701333in}}%
\pgfpathlineto{\pgfqpoint{4.551358in}{3.686208in}}%
\pgfpathlineto{\pgfqpoint{4.567596in}{3.670087in}}%
\pgfpathlineto{\pgfqpoint{4.573656in}{3.664000in}}%
\pgfpathlineto{\pgfqpoint{4.607677in}{3.629798in}}%
\pgfpathlineto{\pgfqpoint{4.609284in}{3.628163in}}%
\pgfpathlineto{\pgfqpoint{4.610783in}{3.626667in}}%
\pgfpathlineto{\pgfqpoint{4.647758in}{3.589374in}}%
\pgfpathlineto{\pgfqpoint{4.647797in}{3.589333in}}%
\pgfpathlineto{\pgfqpoint{4.648259in}{3.588866in}}%
\pgfpathlineto{\pgfqpoint{4.684661in}{3.552000in}}%
\pgfpathlineto{\pgfqpoint{4.686175in}{3.550451in}}%
\pgfpathlineto{\pgfqpoint{4.687838in}{3.548780in}}%
\pgfpathlineto{\pgfqpoint{4.721412in}{3.514667in}}%
\pgfpathlineto{\pgfqpoint{4.724506in}{3.511488in}}%
\pgfpathlineto{\pgfqpoint{4.727919in}{3.508050in}}%
\pgfpathlineto{\pgfqpoint{4.758050in}{3.477333in}}%
\pgfpathlineto{\pgfqpoint{4.762773in}{3.472465in}}%
\pgfpathlineto{\pgfqpoint{4.768000in}{3.467182in}}%
\pgfusepath{fill}%
\end{pgfscope}%
\begin{pgfscope}%
\pgfpathrectangle{\pgfqpoint{0.800000in}{0.528000in}}{\pgfqpoint{3.968000in}{3.696000in}}%
\pgfusepath{clip}%
\pgfsetbuttcap%
\pgfsetroundjoin%
\definecolor{currentfill}{rgb}{0.252899,0.742211,0.448284}%
\pgfsetfillcolor{currentfill}%
\pgfsetlinewidth{0.000000pt}%
\definecolor{currentstroke}{rgb}{0.000000,0.000000,0.000000}%
\pgfsetstrokecolor{currentstroke}%
\pgfsetdash{}{0pt}%
\pgfpathmoveto{\pgfqpoint{4.768000in}{3.472896in}}%
\pgfpathlineto{\pgfqpoint{4.765716in}{3.475206in}}%
\pgfpathlineto{\pgfqpoint{4.763651in}{3.477333in}}%
\pgfpathlineto{\pgfqpoint{4.727919in}{3.513760in}}%
\pgfpathlineto{\pgfqpoint{4.727452in}{3.514231in}}%
\pgfpathlineto{\pgfqpoint{4.727028in}{3.514667in}}%
\pgfpathlineto{\pgfqpoint{4.717215in}{3.524637in}}%
\pgfpathlineto{\pgfqpoint{4.690263in}{3.552000in}}%
\pgfpathlineto{\pgfqpoint{4.689097in}{3.553172in}}%
\pgfpathlineto{\pgfqpoint{4.687838in}{3.554460in}}%
\pgfpathlineto{\pgfqpoint{4.653376in}{3.589333in}}%
\pgfpathlineto{\pgfqpoint{4.647758in}{3.595015in}}%
\pgfpathlineto{\pgfqpoint{4.616377in}{3.626667in}}%
\pgfpathlineto{\pgfqpoint{4.612177in}{3.630858in}}%
\pgfpathlineto{\pgfqpoint{4.607677in}{3.635435in}}%
\pgfpathlineto{\pgfqpoint{4.579263in}{3.664000in}}%
\pgfpathlineto{\pgfqpoint{4.567596in}{3.675721in}}%
\pgfpathlineto{\pgfqpoint{4.554285in}{3.688935in}}%
\pgfpathlineto{\pgfqpoint{4.542035in}{3.701333in}}%
\pgfpathlineto{\pgfqpoint{4.527515in}{3.715872in}}%
\pgfpathlineto{\pgfqpoint{4.504691in}{3.738667in}}%
\pgfpathlineto{\pgfqpoint{4.496318in}{3.746941in}}%
\pgfpathlineto{\pgfqpoint{4.487434in}{3.755889in}}%
\pgfpathlineto{\pgfqpoint{4.476947in}{3.766232in}}%
\pgfpathlineto{\pgfqpoint{4.467231in}{3.776000in}}%
\pgfpathlineto{\pgfqpoint{4.457570in}{3.785516in}}%
\pgfpathlineto{\pgfqpoint{4.447354in}{3.795773in}}%
\pgfpathlineto{\pgfqpoint{4.438181in}{3.804789in}}%
\pgfpathlineto{\pgfqpoint{4.429654in}{3.813333in}}%
\pgfpathlineto{\pgfqpoint{4.407273in}{3.835523in}}%
\pgfpathlineto{\pgfqpoint{4.399349in}{3.843286in}}%
\pgfpathlineto{\pgfqpoint{4.391958in}{3.850667in}}%
\pgfpathlineto{\pgfqpoint{4.367192in}{3.875140in}}%
\pgfpathlineto{\pgfqpoint{4.360451in}{3.881721in}}%
\pgfpathlineto{\pgfqpoint{4.354143in}{3.888000in}}%
\pgfpathlineto{\pgfqpoint{4.327111in}{3.914624in}}%
\pgfpathlineto{\pgfqpoint{4.321488in}{3.920096in}}%
\pgfpathlineto{\pgfqpoint{4.316208in}{3.925333in}}%
\pgfpathlineto{\pgfqpoint{4.287030in}{3.953975in}}%
\pgfpathlineto{\pgfqpoint{4.278152in}{3.962667in}}%
\pgfpathlineto{\pgfqpoint{4.262857in}{3.977484in}}%
\pgfpathlineto{\pgfqpoint{4.246949in}{3.993194in}}%
\pgfpathlineto{\pgfqpoint{4.239974in}{4.000000in}}%
\pgfpathlineto{\pgfqpoint{4.223719in}{4.015695in}}%
\pgfpathlineto{\pgfqpoint{4.206869in}{4.032281in}}%
\pgfpathlineto{\pgfqpoint{4.204202in}{4.034849in}}%
\pgfpathlineto{\pgfqpoint{4.201673in}{4.037333in}}%
\pgfpathlineto{\pgfqpoint{4.166788in}{4.071236in}}%
\pgfpathlineto{\pgfqpoint{4.164974in}{4.072977in}}%
\pgfpathlineto{\pgfqpoint{4.163248in}{4.074667in}}%
\pgfpathlineto{\pgfqpoint{4.126707in}{4.110059in}}%
\pgfpathlineto{\pgfqpoint{4.124698in}{4.112000in}}%
\pgfpathlineto{\pgfqpoint{4.105909in}{4.129961in}}%
\pgfpathlineto{\pgfqpoint{4.086626in}{4.148751in}}%
\pgfpathlineto{\pgfqpoint{4.086022in}{4.149333in}}%
\pgfpathlineto{\pgfqpoint{4.067632in}{4.167026in}}%
\pgfpathlineto{\pgfqpoint{4.047210in}{4.186667in}}%
\pgfpathlineto{\pgfqpoint{4.046545in}{4.187306in}}%
\pgfpathlineto{\pgfqpoint{4.027037in}{4.205829in}}%
\pgfpathlineto{\pgfqpoint{4.008265in}{4.224000in}}%
\pgfpathlineto{\pgfqpoint{4.006465in}{4.224000in}}%
\pgfpathlineto{\pgfqpoint{4.005371in}{4.224000in}}%
\pgfpathlineto{\pgfqpoint{4.005908in}{4.223481in}}%
\pgfpathlineto{\pgfqpoint{4.006465in}{4.222954in}}%
\pgfpathlineto{\pgfqpoint{4.025569in}{4.204461in}}%
\pgfpathlineto{\pgfqpoint{4.044310in}{4.186667in}}%
\pgfpathlineto{\pgfqpoint{4.045405in}{4.185605in}}%
\pgfpathlineto{\pgfqpoint{4.046545in}{4.184521in}}%
\pgfpathlineto{\pgfqpoint{4.083120in}{4.149333in}}%
\pgfpathlineto{\pgfqpoint{4.086626in}{4.145958in}}%
\pgfpathlineto{\pgfqpoint{4.104443in}{4.128596in}}%
\pgfpathlineto{\pgfqpoint{4.121804in}{4.112000in}}%
\pgfpathlineto{\pgfqpoint{4.126707in}{4.107264in}}%
\pgfpathlineto{\pgfqpoint{4.160362in}{4.074667in}}%
\pgfpathlineto{\pgfqpoint{4.163495in}{4.071600in}}%
\pgfpathlineto{\pgfqpoint{4.166788in}{4.068439in}}%
\pgfpathlineto{\pgfqpoint{4.198795in}{4.037333in}}%
\pgfpathlineto{\pgfqpoint{4.202725in}{4.033473in}}%
\pgfpathlineto{\pgfqpoint{4.206869in}{4.029482in}}%
\pgfpathlineto{\pgfqpoint{4.222258in}{4.014334in}}%
\pgfpathlineto{\pgfqpoint{4.237104in}{4.000000in}}%
\pgfpathlineto{\pgfqpoint{4.246949in}{3.990393in}}%
\pgfpathlineto{\pgfqpoint{4.261398in}{3.976124in}}%
\pgfpathlineto{\pgfqpoint{4.275289in}{3.962667in}}%
\pgfpathlineto{\pgfqpoint{4.287030in}{3.951172in}}%
\pgfpathlineto{\pgfqpoint{4.313353in}{3.925333in}}%
\pgfpathlineto{\pgfqpoint{4.320015in}{3.918724in}}%
\pgfpathlineto{\pgfqpoint{4.327111in}{3.911819in}}%
\pgfpathlineto{\pgfqpoint{4.351295in}{3.888000in}}%
\pgfpathlineto{\pgfqpoint{4.358980in}{3.880351in}}%
\pgfpathlineto{\pgfqpoint{4.367192in}{3.872333in}}%
\pgfpathlineto{\pgfqpoint{4.389118in}{3.850667in}}%
\pgfpathlineto{\pgfqpoint{4.397879in}{3.841917in}}%
\pgfpathlineto{\pgfqpoint{4.407273in}{3.832714in}}%
\pgfpathlineto{\pgfqpoint{4.426821in}{3.813333in}}%
\pgfpathlineto{\pgfqpoint{4.436712in}{3.803421in}}%
\pgfpathlineto{\pgfqpoint{4.447354in}{3.792962in}}%
\pgfpathlineto{\pgfqpoint{4.456118in}{3.784163in}}%
\pgfpathlineto{\pgfqpoint{4.464406in}{3.776000in}}%
\pgfpathlineto{\pgfqpoint{4.475480in}{3.764865in}}%
\pgfpathlineto{\pgfqpoint{4.487434in}{3.753077in}}%
\pgfpathlineto{\pgfqpoint{4.494867in}{3.745590in}}%
\pgfpathlineto{\pgfqpoint{4.501873in}{3.738667in}}%
\pgfpathlineto{\pgfqpoint{4.527515in}{3.713057in}}%
\pgfpathlineto{\pgfqpoint{4.539224in}{3.701333in}}%
\pgfpathlineto{\pgfqpoint{4.552822in}{3.687572in}}%
\pgfpathlineto{\pgfqpoint{4.567596in}{3.672904in}}%
\pgfpathlineto{\pgfqpoint{4.576459in}{3.664000in}}%
\pgfpathlineto{\pgfqpoint{4.607677in}{3.632617in}}%
\pgfpathlineto{\pgfqpoint{4.610730in}{3.629511in}}%
\pgfpathlineto{\pgfqpoint{4.613580in}{3.626667in}}%
\pgfpathlineto{\pgfqpoint{4.647758in}{3.592194in}}%
\pgfpathlineto{\pgfqpoint{4.650587in}{3.589333in}}%
\pgfpathlineto{\pgfqpoint{4.683332in}{3.556198in}}%
\pgfpathlineto{\pgfqpoint{4.687477in}{3.552000in}}%
\pgfpathlineto{\pgfqpoint{4.687649in}{3.551824in}}%
\pgfpathlineto{\pgfqpoint{4.687838in}{3.551633in}}%
\pgfpathlineto{\pgfqpoint{4.724220in}{3.514667in}}%
\pgfpathlineto{\pgfqpoint{4.725979in}{3.512859in}}%
\pgfpathlineto{\pgfqpoint{4.727919in}{3.510905in}}%
\pgfpathlineto{\pgfqpoint{4.760850in}{3.477333in}}%
\pgfpathlineto{\pgfqpoint{4.764245in}{3.473835in}}%
\pgfpathlineto{\pgfqpoint{4.768000in}{3.470039in}}%
\pgfusepath{fill}%
\end{pgfscope}%
\begin{pgfscope}%
\pgfpathrectangle{\pgfqpoint{0.800000in}{0.528000in}}{\pgfqpoint{3.968000in}{3.696000in}}%
\pgfusepath{clip}%
\pgfsetbuttcap%
\pgfsetroundjoin%
\definecolor{currentfill}{rgb}{0.252899,0.742211,0.448284}%
\pgfsetfillcolor{currentfill}%
\pgfsetlinewidth{0.000000pt}%
\definecolor{currentstroke}{rgb}{0.000000,0.000000,0.000000}%
\pgfsetstrokecolor{currentstroke}%
\pgfsetdash{}{0pt}%
\pgfpathmoveto{\pgfqpoint{4.768000in}{3.475754in}}%
\pgfpathlineto{\pgfqpoint{4.767187in}{3.476576in}}%
\pgfpathlineto{\pgfqpoint{4.766452in}{3.477333in}}%
\pgfpathlineto{\pgfqpoint{4.750062in}{3.494042in}}%
\pgfpathlineto{\pgfqpoint{4.729813in}{3.514667in}}%
\pgfpathlineto{\pgfqpoint{4.728903in}{3.515583in}}%
\pgfpathlineto{\pgfqpoint{4.727919in}{3.516594in}}%
\pgfpathlineto{\pgfqpoint{4.693045in}{3.552000in}}%
\pgfpathlineto{\pgfqpoint{4.690540in}{3.554517in}}%
\pgfpathlineto{\pgfqpoint{4.687838in}{3.557283in}}%
\pgfpathlineto{\pgfqpoint{4.656166in}{3.589333in}}%
\pgfpathlineto{\pgfqpoint{4.647758in}{3.597836in}}%
\pgfpathlineto{\pgfqpoint{4.619173in}{3.626667in}}%
\pgfpathlineto{\pgfqpoint{4.613623in}{3.632205in}}%
\pgfpathlineto{\pgfqpoint{4.607677in}{3.638254in}}%
\pgfpathlineto{\pgfqpoint{4.582067in}{3.664000in}}%
\pgfpathlineto{\pgfqpoint{4.567596in}{3.678538in}}%
\pgfpathlineto{\pgfqpoint{4.555749in}{3.690299in}}%
\pgfpathlineto{\pgfqpoint{4.544846in}{3.701333in}}%
\pgfpathlineto{\pgfqpoint{4.527515in}{3.718687in}}%
\pgfpathlineto{\pgfqpoint{4.507510in}{3.738667in}}%
\pgfpathlineto{\pgfqpoint{4.497769in}{3.748293in}}%
\pgfpathlineto{\pgfqpoint{4.487434in}{3.758702in}}%
\pgfpathlineto{\pgfqpoint{4.478414in}{3.767598in}}%
\pgfpathlineto{\pgfqpoint{4.470057in}{3.776000in}}%
\pgfpathlineto{\pgfqpoint{4.459022in}{3.786869in}}%
\pgfpathlineto{\pgfqpoint{4.447354in}{3.798584in}}%
\pgfpathlineto{\pgfqpoint{4.439649in}{3.806157in}}%
\pgfpathlineto{\pgfqpoint{4.432487in}{3.813333in}}%
\pgfpathlineto{\pgfqpoint{4.407273in}{3.838332in}}%
\pgfpathlineto{\pgfqpoint{4.400818in}{3.844655in}}%
\pgfpathlineto{\pgfqpoint{4.394799in}{3.850667in}}%
\pgfpathlineto{\pgfqpoint{4.367192in}{3.877947in}}%
\pgfpathlineto{\pgfqpoint{4.361922in}{3.883092in}}%
\pgfpathlineto{\pgfqpoint{4.356991in}{3.888000in}}%
\pgfpathlineto{\pgfqpoint{4.327111in}{3.917429in}}%
\pgfpathlineto{\pgfqpoint{4.322961in}{3.921467in}}%
\pgfpathlineto{\pgfqpoint{4.319064in}{3.925333in}}%
\pgfpathlineto{\pgfqpoint{4.287030in}{3.956778in}}%
\pgfpathlineto{\pgfqpoint{4.281015in}{3.962667in}}%
\pgfpathlineto{\pgfqpoint{4.264317in}{3.978844in}}%
\pgfpathlineto{\pgfqpoint{4.246949in}{3.995995in}}%
\pgfpathlineto{\pgfqpoint{4.242845in}{4.000000in}}%
\pgfpathlineto{\pgfqpoint{4.225180in}{4.017056in}}%
\pgfpathlineto{\pgfqpoint{4.206869in}{4.035080in}}%
\pgfpathlineto{\pgfqpoint{4.205679in}{4.036225in}}%
\pgfpathlineto{\pgfqpoint{4.204551in}{4.037333in}}%
\pgfpathlineto{\pgfqpoint{4.166788in}{4.074033in}}%
\pgfpathlineto{\pgfqpoint{4.166453in}{4.074354in}}%
\pgfpathlineto{\pgfqpoint{4.166134in}{4.074667in}}%
\pgfpathlineto{\pgfqpoint{4.149717in}{4.090568in}}%
\pgfpathlineto{\pgfqpoint{4.127581in}{4.112000in}}%
\pgfpathlineto{\pgfqpoint{4.127150in}{4.112413in}}%
\pgfpathlineto{\pgfqpoint{4.126707in}{4.112845in}}%
\pgfpathlineto{\pgfqpoint{4.107374in}{4.131326in}}%
\pgfpathlineto{\pgfqpoint{4.088895in}{4.149333in}}%
\pgfpathlineto{\pgfqpoint{4.087775in}{4.150403in}}%
\pgfpathlineto{\pgfqpoint{4.086626in}{4.151521in}}%
\pgfpathlineto{\pgfqpoint{4.050084in}{4.186667in}}%
\pgfpathlineto{\pgfqpoint{4.046545in}{4.190068in}}%
\pgfpathlineto{\pgfqpoint{4.028505in}{4.207197in}}%
\pgfpathlineto{\pgfqpoint{4.011147in}{4.224000in}}%
\pgfpathlineto{\pgfqpoint{4.008265in}{4.224000in}}%
\pgfpathlineto{\pgfqpoint{4.027037in}{4.205829in}}%
\pgfpathlineto{\pgfqpoint{4.046545in}{4.187306in}}%
\pgfpathlineto{\pgfqpoint{4.047210in}{4.186667in}}%
\pgfpathlineto{\pgfqpoint{4.067632in}{4.167026in}}%
\pgfpathlineto{\pgfqpoint{4.086022in}{4.149333in}}%
\pgfpathlineto{\pgfqpoint{4.086626in}{4.148751in}}%
\pgfpathlineto{\pgfqpoint{4.105909in}{4.129961in}}%
\pgfpathlineto{\pgfqpoint{4.124698in}{4.112000in}}%
\pgfpathlineto{\pgfqpoint{4.126707in}{4.110059in}}%
\pgfpathlineto{\pgfqpoint{4.163248in}{4.074667in}}%
\pgfpathlineto{\pgfqpoint{4.164974in}{4.072977in}}%
\pgfpathlineto{\pgfqpoint{4.166788in}{4.071236in}}%
\pgfpathlineto{\pgfqpoint{4.201673in}{4.037333in}}%
\pgfpathlineto{\pgfqpoint{4.204202in}{4.034849in}}%
\pgfpathlineto{\pgfqpoint{4.206869in}{4.032281in}}%
\pgfpathlineto{\pgfqpoint{4.223719in}{4.015695in}}%
\pgfpathlineto{\pgfqpoint{4.239974in}{4.000000in}}%
\pgfpathlineto{\pgfqpoint{4.246949in}{3.993194in}}%
\pgfpathlineto{\pgfqpoint{4.262857in}{3.977484in}}%
\pgfpathlineto{\pgfqpoint{4.278152in}{3.962667in}}%
\pgfpathlineto{\pgfqpoint{4.287030in}{3.953975in}}%
\pgfpathlineto{\pgfqpoint{4.316208in}{3.925333in}}%
\pgfpathlineto{\pgfqpoint{4.321488in}{3.920096in}}%
\pgfpathlineto{\pgfqpoint{4.327111in}{3.914624in}}%
\pgfpathlineto{\pgfqpoint{4.354143in}{3.888000in}}%
\pgfpathlineto{\pgfqpoint{4.360451in}{3.881721in}}%
\pgfpathlineto{\pgfqpoint{4.367192in}{3.875140in}}%
\pgfpathlineto{\pgfqpoint{4.391958in}{3.850667in}}%
\pgfpathlineto{\pgfqpoint{4.399349in}{3.843286in}}%
\pgfpathlineto{\pgfqpoint{4.407273in}{3.835523in}}%
\pgfpathlineto{\pgfqpoint{4.429654in}{3.813333in}}%
\pgfpathlineto{\pgfqpoint{4.438181in}{3.804789in}}%
\pgfpathlineto{\pgfqpoint{4.447354in}{3.795773in}}%
\pgfpathlineto{\pgfqpoint{4.457570in}{3.785516in}}%
\pgfpathlineto{\pgfqpoint{4.467231in}{3.776000in}}%
\pgfpathlineto{\pgfqpoint{4.476947in}{3.766232in}}%
\pgfpathlineto{\pgfqpoint{4.487434in}{3.755889in}}%
\pgfpathlineto{\pgfqpoint{4.496318in}{3.746941in}}%
\pgfpathlineto{\pgfqpoint{4.504691in}{3.738667in}}%
\pgfpathlineto{\pgfqpoint{4.527515in}{3.715872in}}%
\pgfpathlineto{\pgfqpoint{4.542035in}{3.701333in}}%
\pgfpathlineto{\pgfqpoint{4.554285in}{3.688935in}}%
\pgfpathlineto{\pgfqpoint{4.567596in}{3.675721in}}%
\pgfpathlineto{\pgfqpoint{4.579263in}{3.664000in}}%
\pgfpathlineto{\pgfqpoint{4.607677in}{3.635435in}}%
\pgfpathlineto{\pgfqpoint{4.612177in}{3.630858in}}%
\pgfpathlineto{\pgfqpoint{4.616377in}{3.626667in}}%
\pgfpathlineto{\pgfqpoint{4.647758in}{3.595015in}}%
\pgfpathlineto{\pgfqpoint{4.653376in}{3.589333in}}%
\pgfpathlineto{\pgfqpoint{4.687838in}{3.554460in}}%
\pgfpathlineto{\pgfqpoint{4.689097in}{3.553172in}}%
\pgfpathlineto{\pgfqpoint{4.690263in}{3.552000in}}%
\pgfpathlineto{\pgfqpoint{4.717215in}{3.524637in}}%
\pgfpathlineto{\pgfqpoint{4.727028in}{3.514667in}}%
\pgfpathlineto{\pgfqpoint{4.727452in}{3.514231in}}%
\pgfpathlineto{\pgfqpoint{4.727919in}{3.513760in}}%
\pgfpathlineto{\pgfqpoint{4.763651in}{3.477333in}}%
\pgfpathlineto{\pgfqpoint{4.765716in}{3.475206in}}%
\pgfpathlineto{\pgfqpoint{4.768000in}{3.472896in}}%
\pgfusepath{fill}%
\end{pgfscope}%
\begin{pgfscope}%
\pgfpathrectangle{\pgfqpoint{0.800000in}{0.528000in}}{\pgfqpoint{3.968000in}{3.696000in}}%
\pgfusepath{clip}%
\pgfsetbuttcap%
\pgfsetroundjoin%
\definecolor{currentfill}{rgb}{0.252899,0.742211,0.448284}%
\pgfsetfillcolor{currentfill}%
\pgfsetlinewidth{0.000000pt}%
\definecolor{currentstroke}{rgb}{0.000000,0.000000,0.000000}%
\pgfsetstrokecolor{currentstroke}%
\pgfsetdash{}{0pt}%
\pgfpathmoveto{\pgfqpoint{4.768000in}{3.478597in}}%
\pgfpathlineto{\pgfqpoint{4.732588in}{3.514667in}}%
\pgfpathlineto{\pgfqpoint{4.730346in}{3.516927in}}%
\pgfpathlineto{\pgfqpoint{4.727919in}{3.519419in}}%
\pgfpathlineto{\pgfqpoint{4.695828in}{3.552000in}}%
\pgfpathlineto{\pgfqpoint{4.691984in}{3.555861in}}%
\pgfpathlineto{\pgfqpoint{4.687838in}{3.560105in}}%
\pgfpathlineto{\pgfqpoint{4.658955in}{3.589333in}}%
\pgfpathlineto{\pgfqpoint{4.647758in}{3.600657in}}%
\pgfpathlineto{\pgfqpoint{4.621970in}{3.626667in}}%
\pgfpathlineto{\pgfqpoint{4.615070in}{3.633553in}}%
\pgfpathlineto{\pgfqpoint{4.607677in}{3.641073in}}%
\pgfpathlineto{\pgfqpoint{4.584871in}{3.664000in}}%
\pgfpathlineto{\pgfqpoint{4.567596in}{3.681354in}}%
\pgfpathlineto{\pgfqpoint{4.557213in}{3.691662in}}%
\pgfpathlineto{\pgfqpoint{4.547657in}{3.701333in}}%
\pgfpathlineto{\pgfqpoint{4.527515in}{3.721502in}}%
\pgfpathlineto{\pgfqpoint{4.510328in}{3.738667in}}%
\pgfpathlineto{\pgfqpoint{4.499220in}{3.749644in}}%
\pgfpathlineto{\pgfqpoint{4.487434in}{3.761515in}}%
\pgfpathlineto{\pgfqpoint{4.479881in}{3.768964in}}%
\pgfpathlineto{\pgfqpoint{4.472883in}{3.776000in}}%
\pgfpathlineto{\pgfqpoint{4.460475in}{3.788222in}}%
\pgfpathlineto{\pgfqpoint{4.447354in}{3.801395in}}%
\pgfpathlineto{\pgfqpoint{4.441117in}{3.807524in}}%
\pgfpathlineto{\pgfqpoint{4.435320in}{3.813333in}}%
\pgfpathlineto{\pgfqpoint{4.407273in}{3.841141in}}%
\pgfpathlineto{\pgfqpoint{4.402288in}{3.846024in}}%
\pgfpathlineto{\pgfqpoint{4.397639in}{3.850667in}}%
\pgfpathlineto{\pgfqpoint{4.367192in}{3.880753in}}%
\pgfpathlineto{\pgfqpoint{4.363394in}{3.884462in}}%
\pgfpathlineto{\pgfqpoint{4.359839in}{3.888000in}}%
\pgfpathlineto{\pgfqpoint{4.327111in}{3.920233in}}%
\pgfpathlineto{\pgfqpoint{4.324433in}{3.922839in}}%
\pgfpathlineto{\pgfqpoint{4.321919in}{3.925333in}}%
\pgfpathlineto{\pgfqpoint{4.287030in}{3.959581in}}%
\pgfpathlineto{\pgfqpoint{4.283878in}{3.962667in}}%
\pgfpathlineto{\pgfqpoint{4.265776in}{3.980203in}}%
\pgfpathlineto{\pgfqpoint{4.246949in}{3.998796in}}%
\pgfpathlineto{\pgfqpoint{4.245715in}{4.000000in}}%
\pgfpathlineto{\pgfqpoint{4.226641in}{4.018417in}}%
\pgfpathlineto{\pgfqpoint{4.207423in}{4.037333in}}%
\pgfpathlineto{\pgfqpoint{4.207150in}{4.037596in}}%
\pgfpathlineto{\pgfqpoint{4.206869in}{4.037873in}}%
\pgfpathlineto{\pgfqpoint{4.168993in}{4.074667in}}%
\pgfpathlineto{\pgfqpoint{4.167908in}{4.075710in}}%
\pgfpathlineto{\pgfqpoint{4.166788in}{4.076807in}}%
\pgfpathlineto{\pgfqpoint{4.130439in}{4.112000in}}%
\pgfpathlineto{\pgfqpoint{4.128600in}{4.113763in}}%
\pgfpathlineto{\pgfqpoint{4.126707in}{4.115611in}}%
\pgfpathlineto{\pgfqpoint{4.108840in}{4.132691in}}%
\pgfpathlineto{\pgfqpoint{4.091761in}{4.149333in}}%
\pgfpathlineto{\pgfqpoint{4.089226in}{4.151755in}}%
\pgfpathlineto{\pgfqpoint{4.086626in}{4.154285in}}%
\pgfpathlineto{\pgfqpoint{4.052958in}{4.186667in}}%
\pgfpathlineto{\pgfqpoint{4.046545in}{4.192830in}}%
\pgfpathlineto{\pgfqpoint{4.029974in}{4.208564in}}%
\pgfpathlineto{\pgfqpoint{4.014028in}{4.224000in}}%
\pgfpathlineto{\pgfqpoint{4.011147in}{4.224000in}}%
\pgfpathlineto{\pgfqpoint{4.028505in}{4.207197in}}%
\pgfpathlineto{\pgfqpoint{4.046545in}{4.190068in}}%
\pgfpathlineto{\pgfqpoint{4.050084in}{4.186667in}}%
\pgfpathlineto{\pgfqpoint{4.086626in}{4.151521in}}%
\pgfpathlineto{\pgfqpoint{4.087775in}{4.150403in}}%
\pgfpathlineto{\pgfqpoint{4.088895in}{4.149333in}}%
\pgfpathlineto{\pgfqpoint{4.107374in}{4.131326in}}%
\pgfpathlineto{\pgfqpoint{4.126707in}{4.112845in}}%
\pgfpathlineto{\pgfqpoint{4.127150in}{4.112413in}}%
\pgfpathlineto{\pgfqpoint{4.127581in}{4.112000in}}%
\pgfpathlineto{\pgfqpoint{4.149717in}{4.090568in}}%
\pgfpathlineto{\pgfqpoint{4.166134in}{4.074667in}}%
\pgfpathlineto{\pgfqpoint{4.166453in}{4.074354in}}%
\pgfpathlineto{\pgfqpoint{4.166788in}{4.074033in}}%
\pgfpathlineto{\pgfqpoint{4.204551in}{4.037333in}}%
\pgfpathlineto{\pgfqpoint{4.205679in}{4.036225in}}%
\pgfpathlineto{\pgfqpoint{4.206869in}{4.035080in}}%
\pgfpathlineto{\pgfqpoint{4.225180in}{4.017056in}}%
\pgfpathlineto{\pgfqpoint{4.242845in}{4.000000in}}%
\pgfpathlineto{\pgfqpoint{4.246949in}{3.995995in}}%
\pgfpathlineto{\pgfqpoint{4.264317in}{3.978844in}}%
\pgfpathlineto{\pgfqpoint{4.281015in}{3.962667in}}%
\pgfpathlineto{\pgfqpoint{4.287030in}{3.956778in}}%
\pgfpathlineto{\pgfqpoint{4.319064in}{3.925333in}}%
\pgfpathlineto{\pgfqpoint{4.322961in}{3.921467in}}%
\pgfpathlineto{\pgfqpoint{4.327111in}{3.917429in}}%
\pgfpathlineto{\pgfqpoint{4.356991in}{3.888000in}}%
\pgfpathlineto{\pgfqpoint{4.361922in}{3.883092in}}%
\pgfpathlineto{\pgfqpoint{4.367192in}{3.877947in}}%
\pgfpathlineto{\pgfqpoint{4.394799in}{3.850667in}}%
\pgfpathlineto{\pgfqpoint{4.400818in}{3.844655in}}%
\pgfpathlineto{\pgfqpoint{4.407273in}{3.838332in}}%
\pgfpathlineto{\pgfqpoint{4.432487in}{3.813333in}}%
\pgfpathlineto{\pgfqpoint{4.439649in}{3.806157in}}%
\pgfpathlineto{\pgfqpoint{4.447354in}{3.798584in}}%
\pgfpathlineto{\pgfqpoint{4.459022in}{3.786869in}}%
\pgfpathlineto{\pgfqpoint{4.470057in}{3.776000in}}%
\pgfpathlineto{\pgfqpoint{4.478414in}{3.767598in}}%
\pgfpathlineto{\pgfqpoint{4.487434in}{3.758702in}}%
\pgfpathlineto{\pgfqpoint{4.497769in}{3.748293in}}%
\pgfpathlineto{\pgfqpoint{4.507510in}{3.738667in}}%
\pgfpathlineto{\pgfqpoint{4.527515in}{3.718687in}}%
\pgfpathlineto{\pgfqpoint{4.544846in}{3.701333in}}%
\pgfpathlineto{\pgfqpoint{4.555749in}{3.690299in}}%
\pgfpathlineto{\pgfqpoint{4.567596in}{3.678538in}}%
\pgfpathlineto{\pgfqpoint{4.582067in}{3.664000in}}%
\pgfpathlineto{\pgfqpoint{4.607677in}{3.638254in}}%
\pgfpathlineto{\pgfqpoint{4.613623in}{3.632205in}}%
\pgfpathlineto{\pgfqpoint{4.619173in}{3.626667in}}%
\pgfpathlineto{\pgfqpoint{4.647758in}{3.597836in}}%
\pgfpathlineto{\pgfqpoint{4.656166in}{3.589333in}}%
\pgfpathlineto{\pgfqpoint{4.687838in}{3.557283in}}%
\pgfpathlineto{\pgfqpoint{4.690540in}{3.554517in}}%
\pgfpathlineto{\pgfqpoint{4.693045in}{3.552000in}}%
\pgfpathlineto{\pgfqpoint{4.727919in}{3.516594in}}%
\pgfpathlineto{\pgfqpoint{4.728903in}{3.515583in}}%
\pgfpathlineto{\pgfqpoint{4.729813in}{3.514667in}}%
\pgfpathlineto{\pgfqpoint{4.750062in}{3.494042in}}%
\pgfpathlineto{\pgfqpoint{4.766452in}{3.477333in}}%
\pgfpathlineto{\pgfqpoint{4.767187in}{3.476576in}}%
\pgfpathlineto{\pgfqpoint{4.768000in}{3.475754in}}%
\pgfpathlineto{\pgfqpoint{4.768000in}{3.477333in}}%
\pgfusepath{fill}%
\end{pgfscope}%
\begin{pgfscope}%
\pgfpathrectangle{\pgfqpoint{0.800000in}{0.528000in}}{\pgfqpoint{3.968000in}{3.696000in}}%
\pgfusepath{clip}%
\pgfsetbuttcap%
\pgfsetroundjoin%
\definecolor{currentfill}{rgb}{0.252899,0.742211,0.448284}%
\pgfsetfillcolor{currentfill}%
\pgfsetlinewidth{0.000000pt}%
\definecolor{currentstroke}{rgb}{0.000000,0.000000,0.000000}%
\pgfsetstrokecolor{currentstroke}%
\pgfsetdash{}{0pt}%
\pgfpathmoveto{\pgfqpoint{4.768000in}{3.481424in}}%
\pgfpathlineto{\pgfqpoint{4.735363in}{3.514667in}}%
\pgfpathlineto{\pgfqpoint{4.731788in}{3.518270in}}%
\pgfpathlineto{\pgfqpoint{4.727919in}{3.522244in}}%
\pgfpathlineto{\pgfqpoint{4.698610in}{3.552000in}}%
\pgfpathlineto{\pgfqpoint{4.693427in}{3.557206in}}%
\pgfpathlineto{\pgfqpoint{4.687838in}{3.562928in}}%
\pgfpathlineto{\pgfqpoint{4.661745in}{3.589333in}}%
\pgfpathlineto{\pgfqpoint{4.647758in}{3.603477in}}%
\pgfpathlineto{\pgfqpoint{4.624766in}{3.626667in}}%
\pgfpathlineto{\pgfqpoint{4.616516in}{3.634900in}}%
\pgfpathlineto{\pgfqpoint{4.607677in}{3.643892in}}%
\pgfpathlineto{\pgfqpoint{4.587675in}{3.664000in}}%
\pgfpathlineto{\pgfqpoint{4.567596in}{3.684171in}}%
\pgfpathlineto{\pgfqpoint{4.558677in}{3.693026in}}%
\pgfpathlineto{\pgfqpoint{4.550468in}{3.701333in}}%
\pgfpathlineto{\pgfqpoint{4.527515in}{3.724317in}}%
\pgfpathlineto{\pgfqpoint{4.513147in}{3.738667in}}%
\pgfpathlineto{\pgfqpoint{4.500671in}{3.750996in}}%
\pgfpathlineto{\pgfqpoint{4.487434in}{3.764328in}}%
\pgfpathlineto{\pgfqpoint{4.481348in}{3.770330in}}%
\pgfpathlineto{\pgfqpoint{4.475708in}{3.776000in}}%
\pgfpathlineto{\pgfqpoint{4.461927in}{3.789574in}}%
\pgfpathlineto{\pgfqpoint{4.447354in}{3.804205in}}%
\pgfpathlineto{\pgfqpoint{4.442585in}{3.808892in}}%
\pgfpathlineto{\pgfqpoint{4.438153in}{3.813333in}}%
\pgfpathlineto{\pgfqpoint{4.407273in}{3.843949in}}%
\pgfpathlineto{\pgfqpoint{4.403758in}{3.847393in}}%
\pgfpathlineto{\pgfqpoint{4.400480in}{3.850667in}}%
\pgfpathlineto{\pgfqpoint{4.367192in}{3.883560in}}%
\pgfpathlineto{\pgfqpoint{4.364865in}{3.885832in}}%
\pgfpathlineto{\pgfqpoint{4.362687in}{3.888000in}}%
\pgfpathlineto{\pgfqpoint{4.327111in}{3.923038in}}%
\pgfpathlineto{\pgfqpoint{4.325906in}{3.924211in}}%
\pgfpathlineto{\pgfqpoint{4.324775in}{3.925333in}}%
\pgfpathlineto{\pgfqpoint{4.287030in}{3.962384in}}%
\pgfpathlineto{\pgfqpoint{4.286741in}{3.962667in}}%
\pgfpathlineto{\pgfqpoint{4.267236in}{3.981563in}}%
\pgfpathlineto{\pgfqpoint{4.248566in}{4.000000in}}%
\pgfpathlineto{\pgfqpoint{4.246949in}{4.001580in}}%
\pgfpathlineto{\pgfqpoint{4.228102in}{4.019778in}}%
\pgfpathlineto{\pgfqpoint{4.210266in}{4.037333in}}%
\pgfpathlineto{\pgfqpoint{4.208597in}{4.038943in}}%
\pgfpathlineto{\pgfqpoint{4.206869in}{4.040643in}}%
\pgfpathlineto{\pgfqpoint{4.171844in}{4.074667in}}%
\pgfpathlineto{\pgfqpoint{4.169356in}{4.077059in}}%
\pgfpathlineto{\pgfqpoint{4.166788in}{4.079575in}}%
\pgfpathlineto{\pgfqpoint{4.133298in}{4.112000in}}%
\pgfpathlineto{\pgfqpoint{4.130049in}{4.115113in}}%
\pgfpathlineto{\pgfqpoint{4.126707in}{4.118377in}}%
\pgfpathlineto{\pgfqpoint{4.110305in}{4.134056in}}%
\pgfpathlineto{\pgfqpoint{4.094628in}{4.149333in}}%
\pgfpathlineto{\pgfqpoint{4.090677in}{4.153106in}}%
\pgfpathlineto{\pgfqpoint{4.086626in}{4.157049in}}%
\pgfpathlineto{\pgfqpoint{4.055832in}{4.186667in}}%
\pgfpathlineto{\pgfqpoint{4.046545in}{4.195592in}}%
\pgfpathlineto{\pgfqpoint{4.031442in}{4.209932in}}%
\pgfpathlineto{\pgfqpoint{4.016909in}{4.224000in}}%
\pgfpathlineto{\pgfqpoint{4.014028in}{4.224000in}}%
\pgfpathlineto{\pgfqpoint{4.029974in}{4.208564in}}%
\pgfpathlineto{\pgfqpoint{4.046545in}{4.192830in}}%
\pgfpathlineto{\pgfqpoint{4.052958in}{4.186667in}}%
\pgfpathlineto{\pgfqpoint{4.086626in}{4.154285in}}%
\pgfpathlineto{\pgfqpoint{4.089226in}{4.151755in}}%
\pgfpathlineto{\pgfqpoint{4.091761in}{4.149333in}}%
\pgfpathlineto{\pgfqpoint{4.108840in}{4.132691in}}%
\pgfpathlineto{\pgfqpoint{4.126707in}{4.115611in}}%
\pgfpathlineto{\pgfqpoint{4.128600in}{4.113763in}}%
\pgfpathlineto{\pgfqpoint{4.130439in}{4.112000in}}%
\pgfpathlineto{\pgfqpoint{4.166788in}{4.076807in}}%
\pgfpathlineto{\pgfqpoint{4.167908in}{4.075710in}}%
\pgfpathlineto{\pgfqpoint{4.168993in}{4.074667in}}%
\pgfpathlineto{\pgfqpoint{4.206869in}{4.037873in}}%
\pgfpathlineto{\pgfqpoint{4.207150in}{4.037596in}}%
\pgfpathlineto{\pgfqpoint{4.207423in}{4.037333in}}%
\pgfpathlineto{\pgfqpoint{4.226641in}{4.018417in}}%
\pgfpathlineto{\pgfqpoint{4.245715in}{4.000000in}}%
\pgfpathlineto{\pgfqpoint{4.246949in}{3.998796in}}%
\pgfpathlineto{\pgfqpoint{4.265776in}{3.980203in}}%
\pgfpathlineto{\pgfqpoint{4.283878in}{3.962667in}}%
\pgfpathlineto{\pgfqpoint{4.287030in}{3.959581in}}%
\pgfpathlineto{\pgfqpoint{4.321919in}{3.925333in}}%
\pgfpathlineto{\pgfqpoint{4.324433in}{3.922839in}}%
\pgfpathlineto{\pgfqpoint{4.327111in}{3.920233in}}%
\pgfpathlineto{\pgfqpoint{4.359839in}{3.888000in}}%
\pgfpathlineto{\pgfqpoint{4.363394in}{3.884462in}}%
\pgfpathlineto{\pgfqpoint{4.367192in}{3.880753in}}%
\pgfpathlineto{\pgfqpoint{4.397639in}{3.850667in}}%
\pgfpathlineto{\pgfqpoint{4.402288in}{3.846024in}}%
\pgfpathlineto{\pgfqpoint{4.407273in}{3.841141in}}%
\pgfpathlineto{\pgfqpoint{4.435320in}{3.813333in}}%
\pgfpathlineto{\pgfqpoint{4.441117in}{3.807524in}}%
\pgfpathlineto{\pgfqpoint{4.447354in}{3.801395in}}%
\pgfpathlineto{\pgfqpoint{4.460475in}{3.788222in}}%
\pgfpathlineto{\pgfqpoint{4.472883in}{3.776000in}}%
\pgfpathlineto{\pgfqpoint{4.479881in}{3.768964in}}%
\pgfpathlineto{\pgfqpoint{4.487434in}{3.761515in}}%
\pgfpathlineto{\pgfqpoint{4.499220in}{3.749644in}}%
\pgfpathlineto{\pgfqpoint{4.510328in}{3.738667in}}%
\pgfpathlineto{\pgfqpoint{4.527515in}{3.721502in}}%
\pgfpathlineto{\pgfqpoint{4.547657in}{3.701333in}}%
\pgfpathlineto{\pgfqpoint{4.557213in}{3.691662in}}%
\pgfpathlineto{\pgfqpoint{4.567596in}{3.681354in}}%
\pgfpathlineto{\pgfqpoint{4.584871in}{3.664000in}}%
\pgfpathlineto{\pgfqpoint{4.607677in}{3.641073in}}%
\pgfpathlineto{\pgfqpoint{4.615070in}{3.633553in}}%
\pgfpathlineto{\pgfqpoint{4.621970in}{3.626667in}}%
\pgfpathlineto{\pgfqpoint{4.647758in}{3.600657in}}%
\pgfpathlineto{\pgfqpoint{4.658955in}{3.589333in}}%
\pgfpathlineto{\pgfqpoint{4.687838in}{3.560105in}}%
\pgfpathlineto{\pgfqpoint{4.691984in}{3.555861in}}%
\pgfpathlineto{\pgfqpoint{4.695828in}{3.552000in}}%
\pgfpathlineto{\pgfqpoint{4.727919in}{3.519419in}}%
\pgfpathlineto{\pgfqpoint{4.730346in}{3.516927in}}%
\pgfpathlineto{\pgfqpoint{4.732588in}{3.514667in}}%
\pgfpathlineto{\pgfqpoint{4.768000in}{3.478597in}}%
\pgfusepath{fill}%
\end{pgfscope}%
\begin{pgfscope}%
\pgfpathrectangle{\pgfqpoint{0.800000in}{0.528000in}}{\pgfqpoint{3.968000in}{3.696000in}}%
\pgfusepath{clip}%
\pgfsetbuttcap%
\pgfsetroundjoin%
\definecolor{currentfill}{rgb}{0.259857,0.745492,0.444467}%
\pgfsetfillcolor{currentfill}%
\pgfsetlinewidth{0.000000pt}%
\definecolor{currentstroke}{rgb}{0.000000,0.000000,0.000000}%
\pgfsetstrokecolor{currentstroke}%
\pgfsetdash{}{0pt}%
\pgfpathmoveto{\pgfqpoint{4.768000in}{3.484251in}}%
\pgfpathlineto{\pgfqpoint{4.738139in}{3.514667in}}%
\pgfpathlineto{\pgfqpoint{4.733230in}{3.519613in}}%
\pgfpathlineto{\pgfqpoint{4.727919in}{3.525068in}}%
\pgfpathlineto{\pgfqpoint{4.701392in}{3.552000in}}%
\pgfpathlineto{\pgfqpoint{4.694871in}{3.558551in}}%
\pgfpathlineto{\pgfqpoint{4.687838in}{3.565751in}}%
\pgfpathlineto{\pgfqpoint{4.664534in}{3.589333in}}%
\pgfpathlineto{\pgfqpoint{4.647758in}{3.606298in}}%
\pgfpathlineto{\pgfqpoint{4.627563in}{3.626667in}}%
\pgfpathlineto{\pgfqpoint{4.617963in}{3.636248in}}%
\pgfpathlineto{\pgfqpoint{4.607677in}{3.646710in}}%
\pgfpathlineto{\pgfqpoint{4.590478in}{3.664000in}}%
\pgfpathlineto{\pgfqpoint{4.567596in}{3.686988in}}%
\pgfpathlineto{\pgfqpoint{4.560141in}{3.694389in}}%
\pgfpathlineto{\pgfqpoint{4.553279in}{3.701333in}}%
\pgfpathlineto{\pgfqpoint{4.527515in}{3.727131in}}%
\pgfpathlineto{\pgfqpoint{4.515965in}{3.738667in}}%
\pgfpathlineto{\pgfqpoint{4.502121in}{3.752347in}}%
\pgfpathlineto{\pgfqpoint{4.487434in}{3.767141in}}%
\pgfpathlineto{\pgfqpoint{4.482814in}{3.771697in}}%
\pgfpathlineto{\pgfqpoint{4.478534in}{3.776000in}}%
\pgfpathlineto{\pgfqpoint{4.463379in}{3.790927in}}%
\pgfpathlineto{\pgfqpoint{4.447354in}{3.807016in}}%
\pgfpathlineto{\pgfqpoint{4.444054in}{3.810260in}}%
\pgfpathlineto{\pgfqpoint{4.440986in}{3.813333in}}%
\pgfpathlineto{\pgfqpoint{4.407273in}{3.846758in}}%
\pgfpathlineto{\pgfqpoint{4.405228in}{3.848762in}}%
\pgfpathlineto{\pgfqpoint{4.403320in}{3.850667in}}%
\pgfpathlineto{\pgfqpoint{4.367192in}{3.886367in}}%
\pgfpathlineto{\pgfqpoint{4.366336in}{3.887203in}}%
\pgfpathlineto{\pgfqpoint{4.365535in}{3.888000in}}%
\pgfpathlineto{\pgfqpoint{4.336653in}{3.916446in}}%
\pgfpathlineto{\pgfqpoint{4.327624in}{3.925333in}}%
\pgfpathlineto{\pgfqpoint{4.327373in}{3.925578in}}%
\pgfpathlineto{\pgfqpoint{4.327111in}{3.925838in}}%
\pgfpathlineto{\pgfqpoint{4.289574in}{3.962667in}}%
\pgfpathlineto{\pgfqpoint{4.288328in}{3.963876in}}%
\pgfpathlineto{\pgfqpoint{4.287030in}{3.965160in}}%
\pgfpathlineto{\pgfqpoint{4.268696in}{3.982922in}}%
\pgfpathlineto{\pgfqpoint{4.251402in}{4.000000in}}%
\pgfpathlineto{\pgfqpoint{4.246949in}{4.004351in}}%
\pgfpathlineto{\pgfqpoint{4.229563in}{4.021139in}}%
\pgfpathlineto{\pgfqpoint{4.213110in}{4.037333in}}%
\pgfpathlineto{\pgfqpoint{4.210044in}{4.040291in}}%
\pgfpathlineto{\pgfqpoint{4.206869in}{4.043412in}}%
\pgfpathlineto{\pgfqpoint{4.174695in}{4.074667in}}%
\pgfpathlineto{\pgfqpoint{4.170804in}{4.078407in}}%
\pgfpathlineto{\pgfqpoint{4.166788in}{4.082343in}}%
\pgfpathlineto{\pgfqpoint{4.136157in}{4.112000in}}%
\pgfpathlineto{\pgfqpoint{4.131499in}{4.116463in}}%
\pgfpathlineto{\pgfqpoint{4.126707in}{4.121143in}}%
\pgfpathlineto{\pgfqpoint{4.111770in}{4.135421in}}%
\pgfpathlineto{\pgfqpoint{4.097494in}{4.149333in}}%
\pgfpathlineto{\pgfqpoint{4.092128in}{4.154458in}}%
\pgfpathlineto{\pgfqpoint{4.086626in}{4.159813in}}%
\pgfpathlineto{\pgfqpoint{4.058705in}{4.186667in}}%
\pgfpathlineto{\pgfqpoint{4.046545in}{4.198354in}}%
\pgfpathlineto{\pgfqpoint{4.032911in}{4.211300in}}%
\pgfpathlineto{\pgfqpoint{4.019791in}{4.224000in}}%
\pgfpathlineto{\pgfqpoint{4.016909in}{4.224000in}}%
\pgfpathlineto{\pgfqpoint{4.031442in}{4.209932in}}%
\pgfpathlineto{\pgfqpoint{4.046545in}{4.195592in}}%
\pgfpathlineto{\pgfqpoint{4.055832in}{4.186667in}}%
\pgfpathlineto{\pgfqpoint{4.086626in}{4.157049in}}%
\pgfpathlineto{\pgfqpoint{4.090677in}{4.153106in}}%
\pgfpathlineto{\pgfqpoint{4.094628in}{4.149333in}}%
\pgfpathlineto{\pgfqpoint{4.110305in}{4.134056in}}%
\pgfpathlineto{\pgfqpoint{4.126707in}{4.118377in}}%
\pgfpathlineto{\pgfqpoint{4.130049in}{4.115113in}}%
\pgfpathlineto{\pgfqpoint{4.133298in}{4.112000in}}%
\pgfpathlineto{\pgfqpoint{4.166788in}{4.079575in}}%
\pgfpathlineto{\pgfqpoint{4.169356in}{4.077059in}}%
\pgfpathlineto{\pgfqpoint{4.171844in}{4.074667in}}%
\pgfpathlineto{\pgfqpoint{4.206869in}{4.040643in}}%
\pgfpathlineto{\pgfqpoint{4.208597in}{4.038943in}}%
\pgfpathlineto{\pgfqpoint{4.210266in}{4.037333in}}%
\pgfpathlineto{\pgfqpoint{4.228102in}{4.019778in}}%
\pgfpathlineto{\pgfqpoint{4.246949in}{4.001580in}}%
\pgfpathlineto{\pgfqpoint{4.248566in}{4.000000in}}%
\pgfpathlineto{\pgfqpoint{4.267236in}{3.981563in}}%
\pgfpathlineto{\pgfqpoint{4.286741in}{3.962667in}}%
\pgfpathlineto{\pgfqpoint{4.287030in}{3.962384in}}%
\pgfpathlineto{\pgfqpoint{4.324775in}{3.925333in}}%
\pgfpathlineto{\pgfqpoint{4.325906in}{3.924211in}}%
\pgfpathlineto{\pgfqpoint{4.327111in}{3.923038in}}%
\pgfpathlineto{\pgfqpoint{4.362687in}{3.888000in}}%
\pgfpathlineto{\pgfqpoint{4.364865in}{3.885832in}}%
\pgfpathlineto{\pgfqpoint{4.367192in}{3.883560in}}%
\pgfpathlineto{\pgfqpoint{4.400480in}{3.850667in}}%
\pgfpathlineto{\pgfqpoint{4.403758in}{3.847393in}}%
\pgfpathlineto{\pgfqpoint{4.407273in}{3.843949in}}%
\pgfpathlineto{\pgfqpoint{4.438153in}{3.813333in}}%
\pgfpathlineto{\pgfqpoint{4.442585in}{3.808892in}}%
\pgfpathlineto{\pgfqpoint{4.447354in}{3.804205in}}%
\pgfpathlineto{\pgfqpoint{4.461927in}{3.789574in}}%
\pgfpathlineto{\pgfqpoint{4.475708in}{3.776000in}}%
\pgfpathlineto{\pgfqpoint{4.481348in}{3.770330in}}%
\pgfpathlineto{\pgfqpoint{4.487434in}{3.764328in}}%
\pgfpathlineto{\pgfqpoint{4.500671in}{3.750996in}}%
\pgfpathlineto{\pgfqpoint{4.513147in}{3.738667in}}%
\pgfpathlineto{\pgfqpoint{4.527515in}{3.724317in}}%
\pgfpathlineto{\pgfqpoint{4.550468in}{3.701333in}}%
\pgfpathlineto{\pgfqpoint{4.558677in}{3.693026in}}%
\pgfpathlineto{\pgfqpoint{4.567596in}{3.684171in}}%
\pgfpathlineto{\pgfqpoint{4.587675in}{3.664000in}}%
\pgfpathlineto{\pgfqpoint{4.607677in}{3.643892in}}%
\pgfpathlineto{\pgfqpoint{4.616516in}{3.634900in}}%
\pgfpathlineto{\pgfqpoint{4.624766in}{3.626667in}}%
\pgfpathlineto{\pgfqpoint{4.647758in}{3.603477in}}%
\pgfpathlineto{\pgfqpoint{4.661745in}{3.589333in}}%
\pgfpathlineto{\pgfqpoint{4.687838in}{3.562928in}}%
\pgfpathlineto{\pgfqpoint{4.693427in}{3.557206in}}%
\pgfpathlineto{\pgfqpoint{4.698610in}{3.552000in}}%
\pgfpathlineto{\pgfqpoint{4.727919in}{3.522244in}}%
\pgfpathlineto{\pgfqpoint{4.731788in}{3.518270in}}%
\pgfpathlineto{\pgfqpoint{4.735363in}{3.514667in}}%
\pgfpathlineto{\pgfqpoint{4.768000in}{3.481424in}}%
\pgfusepath{fill}%
\end{pgfscope}%
\begin{pgfscope}%
\pgfpathrectangle{\pgfqpoint{0.800000in}{0.528000in}}{\pgfqpoint{3.968000in}{3.696000in}}%
\pgfusepath{clip}%
\pgfsetbuttcap%
\pgfsetroundjoin%
\definecolor{currentfill}{rgb}{0.259857,0.745492,0.444467}%
\pgfsetfillcolor{currentfill}%
\pgfsetlinewidth{0.000000pt}%
\definecolor{currentstroke}{rgb}{0.000000,0.000000,0.000000}%
\pgfsetstrokecolor{currentstroke}%
\pgfsetdash{}{0pt}%
\pgfpathmoveto{\pgfqpoint{4.768000in}{3.487077in}}%
\pgfpathlineto{\pgfqpoint{4.740914in}{3.514667in}}%
\pgfpathlineto{\pgfqpoint{4.734672in}{3.520957in}}%
\pgfpathlineto{\pgfqpoint{4.727919in}{3.527893in}}%
\pgfpathlineto{\pgfqpoint{4.704174in}{3.552000in}}%
\pgfpathlineto{\pgfqpoint{4.696315in}{3.559895in}}%
\pgfpathlineto{\pgfqpoint{4.687838in}{3.568574in}}%
\pgfpathlineto{\pgfqpoint{4.667323in}{3.589333in}}%
\pgfpathlineto{\pgfqpoint{4.647758in}{3.609119in}}%
\pgfpathlineto{\pgfqpoint{4.630360in}{3.626667in}}%
\pgfpathlineto{\pgfqpoint{4.619409in}{3.637595in}}%
\pgfpathlineto{\pgfqpoint{4.607677in}{3.649529in}}%
\pgfpathlineto{\pgfqpoint{4.593282in}{3.664000in}}%
\pgfpathlineto{\pgfqpoint{4.567596in}{3.689805in}}%
\pgfpathlineto{\pgfqpoint{4.561605in}{3.695753in}}%
\pgfpathlineto{\pgfqpoint{4.556090in}{3.701333in}}%
\pgfpathlineto{\pgfqpoint{4.527515in}{3.729946in}}%
\pgfpathlineto{\pgfqpoint{4.518783in}{3.738667in}}%
\pgfpathlineto{\pgfqpoint{4.503572in}{3.753698in}}%
\pgfpathlineto{\pgfqpoint{4.487434in}{3.769953in}}%
\pgfpathlineto{\pgfqpoint{4.484281in}{3.773063in}}%
\pgfpathlineto{\pgfqpoint{4.481360in}{3.776000in}}%
\pgfpathlineto{\pgfqpoint{4.464831in}{3.792280in}}%
\pgfpathlineto{\pgfqpoint{4.447354in}{3.809827in}}%
\pgfpathlineto{\pgfqpoint{4.445522in}{3.811627in}}%
\pgfpathlineto{\pgfqpoint{4.443819in}{3.813333in}}%
\pgfpathlineto{\pgfqpoint{4.407273in}{3.849567in}}%
\pgfpathlineto{\pgfqpoint{4.406697in}{3.850131in}}%
\pgfpathlineto{\pgfqpoint{4.406161in}{3.850667in}}%
\pgfpathlineto{\pgfqpoint{4.387895in}{3.868716in}}%
\pgfpathlineto{\pgfqpoint{4.368369in}{3.888000in}}%
\pgfpathlineto{\pgfqpoint{4.367795in}{3.888561in}}%
\pgfpathlineto{\pgfqpoint{4.367192in}{3.889162in}}%
\pgfpathlineto{\pgfqpoint{4.330445in}{3.925333in}}%
\pgfpathlineto{\pgfqpoint{4.328816in}{3.926921in}}%
\pgfpathlineto{\pgfqpoint{4.327111in}{3.928613in}}%
\pgfpathlineto{\pgfqpoint{4.292402in}{3.962667in}}%
\pgfpathlineto{\pgfqpoint{4.289772in}{3.965221in}}%
\pgfpathlineto{\pgfqpoint{4.287030in}{3.967934in}}%
\pgfpathlineto{\pgfqpoint{4.270155in}{3.984282in}}%
\pgfpathlineto{\pgfqpoint{4.254239in}{4.000000in}}%
\pgfpathlineto{\pgfqpoint{4.246949in}{4.007123in}}%
\pgfpathlineto{\pgfqpoint{4.231024in}{4.022500in}}%
\pgfpathlineto{\pgfqpoint{4.215954in}{4.037333in}}%
\pgfpathlineto{\pgfqpoint{4.211490in}{4.041638in}}%
\pgfpathlineto{\pgfqpoint{4.206869in}{4.046182in}}%
\pgfpathlineto{\pgfqpoint{4.177546in}{4.074667in}}%
\pgfpathlineto{\pgfqpoint{4.172252in}{4.079756in}}%
\pgfpathlineto{\pgfqpoint{4.166788in}{4.085110in}}%
\pgfpathlineto{\pgfqpoint{4.139015in}{4.112000in}}%
\pgfpathlineto{\pgfqpoint{4.132948in}{4.117813in}}%
\pgfpathlineto{\pgfqpoint{4.126707in}{4.123909in}}%
\pgfpathlineto{\pgfqpoint{4.113236in}{4.136786in}}%
\pgfpathlineto{\pgfqpoint{4.100360in}{4.149333in}}%
\pgfpathlineto{\pgfqpoint{4.093579in}{4.155809in}}%
\pgfpathlineto{\pgfqpoint{4.086626in}{4.162577in}}%
\pgfpathlineto{\pgfqpoint{4.061579in}{4.186667in}}%
\pgfpathlineto{\pgfqpoint{4.046545in}{4.201116in}}%
\pgfpathlineto{\pgfqpoint{4.034379in}{4.212668in}}%
\pgfpathlineto{\pgfqpoint{4.022672in}{4.224000in}}%
\pgfpathlineto{\pgfqpoint{4.019791in}{4.224000in}}%
\pgfpathlineto{\pgfqpoint{4.032911in}{4.211300in}}%
\pgfpathlineto{\pgfqpoint{4.046545in}{4.198354in}}%
\pgfpathlineto{\pgfqpoint{4.058705in}{4.186667in}}%
\pgfpathlineto{\pgfqpoint{4.086626in}{4.159813in}}%
\pgfpathlineto{\pgfqpoint{4.092128in}{4.154458in}}%
\pgfpathlineto{\pgfqpoint{4.097494in}{4.149333in}}%
\pgfpathlineto{\pgfqpoint{4.111770in}{4.135421in}}%
\pgfpathlineto{\pgfqpoint{4.126707in}{4.121143in}}%
\pgfpathlineto{\pgfqpoint{4.131499in}{4.116463in}}%
\pgfpathlineto{\pgfqpoint{4.136157in}{4.112000in}}%
\pgfpathlineto{\pgfqpoint{4.166788in}{4.082343in}}%
\pgfpathlineto{\pgfqpoint{4.170804in}{4.078407in}}%
\pgfpathlineto{\pgfqpoint{4.174695in}{4.074667in}}%
\pgfpathlineto{\pgfqpoint{4.206869in}{4.043412in}}%
\pgfpathlineto{\pgfqpoint{4.210044in}{4.040291in}}%
\pgfpathlineto{\pgfqpoint{4.213110in}{4.037333in}}%
\pgfpathlineto{\pgfqpoint{4.229563in}{4.021139in}}%
\pgfpathlineto{\pgfqpoint{4.246949in}{4.004351in}}%
\pgfpathlineto{\pgfqpoint{4.251402in}{4.000000in}}%
\pgfpathlineto{\pgfqpoint{4.268696in}{3.982922in}}%
\pgfpathlineto{\pgfqpoint{4.287030in}{3.965160in}}%
\pgfpathlineto{\pgfqpoint{4.288328in}{3.963876in}}%
\pgfpathlineto{\pgfqpoint{4.289574in}{3.962667in}}%
\pgfpathlineto{\pgfqpoint{4.327111in}{3.925838in}}%
\pgfpathlineto{\pgfqpoint{4.327373in}{3.925578in}}%
\pgfpathlineto{\pgfqpoint{4.327624in}{3.925333in}}%
\pgfpathlineto{\pgfqpoint{4.336653in}{3.916446in}}%
\pgfpathlineto{\pgfqpoint{4.365535in}{3.888000in}}%
\pgfpathlineto{\pgfqpoint{4.366336in}{3.887203in}}%
\pgfpathlineto{\pgfqpoint{4.367192in}{3.886367in}}%
\pgfpathlineto{\pgfqpoint{4.403320in}{3.850667in}}%
\pgfpathlineto{\pgfqpoint{4.405228in}{3.848762in}}%
\pgfpathlineto{\pgfqpoint{4.407273in}{3.846758in}}%
\pgfpathlineto{\pgfqpoint{4.440986in}{3.813333in}}%
\pgfpathlineto{\pgfqpoint{4.444054in}{3.810260in}}%
\pgfpathlineto{\pgfqpoint{4.447354in}{3.807016in}}%
\pgfpathlineto{\pgfqpoint{4.463379in}{3.790927in}}%
\pgfpathlineto{\pgfqpoint{4.478534in}{3.776000in}}%
\pgfpathlineto{\pgfqpoint{4.482814in}{3.771697in}}%
\pgfpathlineto{\pgfqpoint{4.487434in}{3.767141in}}%
\pgfpathlineto{\pgfqpoint{4.502121in}{3.752347in}}%
\pgfpathlineto{\pgfqpoint{4.515965in}{3.738667in}}%
\pgfpathlineto{\pgfqpoint{4.527515in}{3.727131in}}%
\pgfpathlineto{\pgfqpoint{4.553279in}{3.701333in}}%
\pgfpathlineto{\pgfqpoint{4.560141in}{3.694389in}}%
\pgfpathlineto{\pgfqpoint{4.567596in}{3.686988in}}%
\pgfpathlineto{\pgfqpoint{4.590478in}{3.664000in}}%
\pgfpathlineto{\pgfqpoint{4.607677in}{3.646710in}}%
\pgfpathlineto{\pgfqpoint{4.617963in}{3.636248in}}%
\pgfpathlineto{\pgfqpoint{4.627563in}{3.626667in}}%
\pgfpathlineto{\pgfqpoint{4.647758in}{3.606298in}}%
\pgfpathlineto{\pgfqpoint{4.664534in}{3.589333in}}%
\pgfpathlineto{\pgfqpoint{4.687838in}{3.565751in}}%
\pgfpathlineto{\pgfqpoint{4.694871in}{3.558551in}}%
\pgfpathlineto{\pgfqpoint{4.701392in}{3.552000in}}%
\pgfpathlineto{\pgfqpoint{4.727919in}{3.525068in}}%
\pgfpathlineto{\pgfqpoint{4.733230in}{3.519613in}}%
\pgfpathlineto{\pgfqpoint{4.738139in}{3.514667in}}%
\pgfpathlineto{\pgfqpoint{4.768000in}{3.484251in}}%
\pgfusepath{fill}%
\end{pgfscope}%
\begin{pgfscope}%
\pgfpathrectangle{\pgfqpoint{0.800000in}{0.528000in}}{\pgfqpoint{3.968000in}{3.696000in}}%
\pgfusepath{clip}%
\pgfsetbuttcap%
\pgfsetroundjoin%
\definecolor{currentfill}{rgb}{0.259857,0.745492,0.444467}%
\pgfsetfillcolor{currentfill}%
\pgfsetlinewidth{0.000000pt}%
\definecolor{currentstroke}{rgb}{0.000000,0.000000,0.000000}%
\pgfsetstrokecolor{currentstroke}%
\pgfsetdash{}{0pt}%
\pgfpathmoveto{\pgfqpoint{4.768000in}{3.489904in}}%
\pgfpathlineto{\pgfqpoint{4.743689in}{3.514667in}}%
\pgfpathlineto{\pgfqpoint{4.736114in}{3.522300in}}%
\pgfpathlineto{\pgfqpoint{4.727919in}{3.530718in}}%
\pgfpathlineto{\pgfqpoint{4.706957in}{3.552000in}}%
\pgfpathlineto{\pgfqpoint{4.697758in}{3.561240in}}%
\pgfpathlineto{\pgfqpoint{4.687838in}{3.571396in}}%
\pgfpathlineto{\pgfqpoint{4.670113in}{3.589333in}}%
\pgfpathlineto{\pgfqpoint{4.647758in}{3.611939in}}%
\pgfpathlineto{\pgfqpoint{4.633156in}{3.626667in}}%
\pgfpathlineto{\pgfqpoint{4.620856in}{3.638942in}}%
\pgfpathlineto{\pgfqpoint{4.607677in}{3.652348in}}%
\pgfpathlineto{\pgfqpoint{4.596086in}{3.664000in}}%
\pgfpathlineto{\pgfqpoint{4.567596in}{3.692621in}}%
\pgfpathlineto{\pgfqpoint{4.563068in}{3.697116in}}%
\pgfpathlineto{\pgfqpoint{4.558902in}{3.701333in}}%
\pgfpathlineto{\pgfqpoint{4.527515in}{3.732761in}}%
\pgfpathlineto{\pgfqpoint{4.521602in}{3.738667in}}%
\pgfpathlineto{\pgfqpoint{4.505023in}{3.755050in}}%
\pgfpathlineto{\pgfqpoint{4.487434in}{3.772766in}}%
\pgfpathlineto{\pgfqpoint{4.485748in}{3.774429in}}%
\pgfpathlineto{\pgfqpoint{4.484186in}{3.776000in}}%
\pgfpathlineto{\pgfqpoint{4.466284in}{3.793633in}}%
\pgfpathlineto{\pgfqpoint{4.447354in}{3.812638in}}%
\pgfpathlineto{\pgfqpoint{4.446990in}{3.812995in}}%
\pgfpathlineto{\pgfqpoint{4.446652in}{3.813333in}}%
\pgfpathlineto{\pgfqpoint{4.435766in}{3.824127in}}%
\pgfpathlineto{\pgfqpoint{4.408981in}{3.850667in}}%
\pgfpathlineto{\pgfqpoint{4.408149in}{3.851483in}}%
\pgfpathlineto{\pgfqpoint{4.407273in}{3.852358in}}%
\pgfpathlineto{\pgfqpoint{4.371183in}{3.888000in}}%
\pgfpathlineto{\pgfqpoint{4.369236in}{3.889904in}}%
\pgfpathlineto{\pgfqpoint{4.367192in}{3.891939in}}%
\pgfpathlineto{\pgfqpoint{4.333267in}{3.925333in}}%
\pgfpathlineto{\pgfqpoint{4.330258in}{3.928265in}}%
\pgfpathlineto{\pgfqpoint{4.327111in}{3.931389in}}%
\pgfpathlineto{\pgfqpoint{4.295231in}{3.962667in}}%
\pgfpathlineto{\pgfqpoint{4.291216in}{3.966565in}}%
\pgfpathlineto{\pgfqpoint{4.287030in}{3.970707in}}%
\pgfpathlineto{\pgfqpoint{4.271615in}{3.985641in}}%
\pgfpathlineto{\pgfqpoint{4.257075in}{4.000000in}}%
\pgfpathlineto{\pgfqpoint{4.246949in}{4.009895in}}%
\pgfpathlineto{\pgfqpoint{4.232485in}{4.023861in}}%
\pgfpathlineto{\pgfqpoint{4.218797in}{4.037333in}}%
\pgfpathlineto{\pgfqpoint{4.212937in}{4.042986in}}%
\pgfpathlineto{\pgfqpoint{4.206869in}{4.048952in}}%
\pgfpathlineto{\pgfqpoint{4.180397in}{4.074667in}}%
\pgfpathlineto{\pgfqpoint{4.173700in}{4.081105in}}%
\pgfpathlineto{\pgfqpoint{4.166788in}{4.087878in}}%
\pgfpathlineto{\pgfqpoint{4.141874in}{4.112000in}}%
\pgfpathlineto{\pgfqpoint{4.134398in}{4.119164in}}%
\pgfpathlineto{\pgfqpoint{4.126707in}{4.126674in}}%
\pgfpathlineto{\pgfqpoint{4.114701in}{4.138151in}}%
\pgfpathlineto{\pgfqpoint{4.103226in}{4.149333in}}%
\pgfpathlineto{\pgfqpoint{4.095030in}{4.157161in}}%
\pgfpathlineto{\pgfqpoint{4.086626in}{4.165341in}}%
\pgfpathlineto{\pgfqpoint{4.064453in}{4.186667in}}%
\pgfpathlineto{\pgfqpoint{4.046545in}{4.203878in}}%
\pgfpathlineto{\pgfqpoint{4.035848in}{4.214035in}}%
\pgfpathlineto{\pgfqpoint{4.025554in}{4.224000in}}%
\pgfpathlineto{\pgfqpoint{4.022672in}{4.224000in}}%
\pgfpathlineto{\pgfqpoint{4.034379in}{4.212668in}}%
\pgfpathlineto{\pgfqpoint{4.046545in}{4.201116in}}%
\pgfpathlineto{\pgfqpoint{4.061579in}{4.186667in}}%
\pgfpathlineto{\pgfqpoint{4.086626in}{4.162577in}}%
\pgfpathlineto{\pgfqpoint{4.093579in}{4.155809in}}%
\pgfpathlineto{\pgfqpoint{4.100360in}{4.149333in}}%
\pgfpathlineto{\pgfqpoint{4.113236in}{4.136786in}}%
\pgfpathlineto{\pgfqpoint{4.126707in}{4.123909in}}%
\pgfpathlineto{\pgfqpoint{4.132948in}{4.117813in}}%
\pgfpathlineto{\pgfqpoint{4.139015in}{4.112000in}}%
\pgfpathlineto{\pgfqpoint{4.166788in}{4.085110in}}%
\pgfpathlineto{\pgfqpoint{4.172252in}{4.079756in}}%
\pgfpathlineto{\pgfqpoint{4.177546in}{4.074667in}}%
\pgfpathlineto{\pgfqpoint{4.206869in}{4.046182in}}%
\pgfpathlineto{\pgfqpoint{4.211490in}{4.041638in}}%
\pgfpathlineto{\pgfqpoint{4.215954in}{4.037333in}}%
\pgfpathlineto{\pgfqpoint{4.231024in}{4.022500in}}%
\pgfpathlineto{\pgfqpoint{4.246949in}{4.007123in}}%
\pgfpathlineto{\pgfqpoint{4.254239in}{4.000000in}}%
\pgfpathlineto{\pgfqpoint{4.270155in}{3.984282in}}%
\pgfpathlineto{\pgfqpoint{4.287030in}{3.967934in}}%
\pgfpathlineto{\pgfqpoint{4.289772in}{3.965221in}}%
\pgfpathlineto{\pgfqpoint{4.292402in}{3.962667in}}%
\pgfpathlineto{\pgfqpoint{4.327111in}{3.928613in}}%
\pgfpathlineto{\pgfqpoint{4.328816in}{3.926921in}}%
\pgfpathlineto{\pgfqpoint{4.330445in}{3.925333in}}%
\pgfpathlineto{\pgfqpoint{4.367192in}{3.889162in}}%
\pgfpathlineto{\pgfqpoint{4.367795in}{3.888561in}}%
\pgfpathlineto{\pgfqpoint{4.368369in}{3.888000in}}%
\pgfpathlineto{\pgfqpoint{4.387895in}{3.868716in}}%
\pgfpathlineto{\pgfqpoint{4.406161in}{3.850667in}}%
\pgfpathlineto{\pgfqpoint{4.406697in}{3.850131in}}%
\pgfpathlineto{\pgfqpoint{4.407273in}{3.849567in}}%
\pgfpathlineto{\pgfqpoint{4.443819in}{3.813333in}}%
\pgfpathlineto{\pgfqpoint{4.445522in}{3.811627in}}%
\pgfpathlineto{\pgfqpoint{4.447354in}{3.809827in}}%
\pgfpathlineto{\pgfqpoint{4.464831in}{3.792280in}}%
\pgfpathlineto{\pgfqpoint{4.481360in}{3.776000in}}%
\pgfpathlineto{\pgfqpoint{4.484281in}{3.773063in}}%
\pgfpathlineto{\pgfqpoint{4.487434in}{3.769953in}}%
\pgfpathlineto{\pgfqpoint{4.503572in}{3.753698in}}%
\pgfpathlineto{\pgfqpoint{4.518783in}{3.738667in}}%
\pgfpathlineto{\pgfqpoint{4.527515in}{3.729946in}}%
\pgfpathlineto{\pgfqpoint{4.556090in}{3.701333in}}%
\pgfpathlineto{\pgfqpoint{4.561605in}{3.695753in}}%
\pgfpathlineto{\pgfqpoint{4.567596in}{3.689805in}}%
\pgfpathlineto{\pgfqpoint{4.593282in}{3.664000in}}%
\pgfpathlineto{\pgfqpoint{4.607677in}{3.649529in}}%
\pgfpathlineto{\pgfqpoint{4.619409in}{3.637595in}}%
\pgfpathlineto{\pgfqpoint{4.630360in}{3.626667in}}%
\pgfpathlineto{\pgfqpoint{4.647758in}{3.609119in}}%
\pgfpathlineto{\pgfqpoint{4.667323in}{3.589333in}}%
\pgfpathlineto{\pgfqpoint{4.687838in}{3.568574in}}%
\pgfpathlineto{\pgfqpoint{4.696315in}{3.559895in}}%
\pgfpathlineto{\pgfqpoint{4.704174in}{3.552000in}}%
\pgfpathlineto{\pgfqpoint{4.727919in}{3.527893in}}%
\pgfpathlineto{\pgfqpoint{4.734672in}{3.520957in}}%
\pgfpathlineto{\pgfqpoint{4.740914in}{3.514667in}}%
\pgfpathlineto{\pgfqpoint{4.768000in}{3.487077in}}%
\pgfusepath{fill}%
\end{pgfscope}%
\begin{pgfscope}%
\pgfpathrectangle{\pgfqpoint{0.800000in}{0.528000in}}{\pgfqpoint{3.968000in}{3.696000in}}%
\pgfusepath{clip}%
\pgfsetbuttcap%
\pgfsetroundjoin%
\definecolor{currentfill}{rgb}{0.266941,0.748751,0.440573}%
\pgfsetfillcolor{currentfill}%
\pgfsetlinewidth{0.000000pt}%
\definecolor{currentstroke}{rgb}{0.000000,0.000000,0.000000}%
\pgfsetstrokecolor{currentstroke}%
\pgfsetdash{}{0pt}%
\pgfpathmoveto{\pgfqpoint{4.768000in}{3.492731in}}%
\pgfpathlineto{\pgfqpoint{4.746464in}{3.514667in}}%
\pgfpathlineto{\pgfqpoint{4.737557in}{3.523644in}}%
\pgfpathlineto{\pgfqpoint{4.727919in}{3.533543in}}%
\pgfpathlineto{\pgfqpoint{4.709739in}{3.552000in}}%
\pgfpathlineto{\pgfqpoint{4.699202in}{3.562585in}}%
\pgfpathlineto{\pgfqpoint{4.687838in}{3.574219in}}%
\pgfpathlineto{\pgfqpoint{4.672902in}{3.589333in}}%
\pgfpathlineto{\pgfqpoint{4.647758in}{3.614760in}}%
\pgfpathlineto{\pgfqpoint{4.635953in}{3.626667in}}%
\pgfpathlineto{\pgfqpoint{4.622302in}{3.640290in}}%
\pgfpathlineto{\pgfqpoint{4.607677in}{3.655166in}}%
\pgfpathlineto{\pgfqpoint{4.598890in}{3.664000in}}%
\pgfpathlineto{\pgfqpoint{4.567596in}{3.695438in}}%
\pgfpathlineto{\pgfqpoint{4.564532in}{3.698480in}}%
\pgfpathlineto{\pgfqpoint{4.561713in}{3.701333in}}%
\pgfpathlineto{\pgfqpoint{4.527515in}{3.735576in}}%
\pgfpathlineto{\pgfqpoint{4.524420in}{3.738667in}}%
\pgfpathlineto{\pgfqpoint{4.506474in}{3.756401in}}%
\pgfpathlineto{\pgfqpoint{4.487434in}{3.775579in}}%
\pgfpathlineto{\pgfqpoint{4.487215in}{3.775795in}}%
\pgfpathlineto{\pgfqpoint{4.487011in}{3.776000in}}%
\pgfpathlineto{\pgfqpoint{4.467736in}{3.794985in}}%
\pgfpathlineto{\pgfqpoint{4.449460in}{3.813333in}}%
\pgfpathlineto{\pgfqpoint{4.447354in}{3.815426in}}%
\pgfpathlineto{\pgfqpoint{4.411788in}{3.850667in}}%
\pgfpathlineto{\pgfqpoint{4.409588in}{3.852823in}}%
\pgfpathlineto{\pgfqpoint{4.407273in}{3.855137in}}%
\pgfpathlineto{\pgfqpoint{4.373997in}{3.888000in}}%
\pgfpathlineto{\pgfqpoint{4.370676in}{3.891246in}}%
\pgfpathlineto{\pgfqpoint{4.367192in}{3.894716in}}%
\pgfpathlineto{\pgfqpoint{4.336089in}{3.925333in}}%
\pgfpathlineto{\pgfqpoint{4.331700in}{3.929608in}}%
\pgfpathlineto{\pgfqpoint{4.327111in}{3.934164in}}%
\pgfpathlineto{\pgfqpoint{4.298060in}{3.962667in}}%
\pgfpathlineto{\pgfqpoint{4.292660in}{3.967910in}}%
\pgfpathlineto{\pgfqpoint{4.287030in}{3.973481in}}%
\pgfpathlineto{\pgfqpoint{4.273074in}{3.987001in}}%
\pgfpathlineto{\pgfqpoint{4.259911in}{4.000000in}}%
\pgfpathlineto{\pgfqpoint{4.246949in}{4.012666in}}%
\pgfpathlineto{\pgfqpoint{4.233946in}{4.025221in}}%
\pgfpathlineto{\pgfqpoint{4.221641in}{4.037333in}}%
\pgfpathlineto{\pgfqpoint{4.214384in}{4.044333in}}%
\pgfpathlineto{\pgfqpoint{4.206869in}{4.051721in}}%
\pgfpathlineto{\pgfqpoint{4.183248in}{4.074667in}}%
\pgfpathlineto{\pgfqpoint{4.175148in}{4.082454in}}%
\pgfpathlineto{\pgfqpoint{4.166788in}{4.090646in}}%
\pgfpathlineto{\pgfqpoint{4.144733in}{4.112000in}}%
\pgfpathlineto{\pgfqpoint{4.135847in}{4.120514in}}%
\pgfpathlineto{\pgfqpoint{4.126707in}{4.129440in}}%
\pgfpathlineto{\pgfqpoint{4.116167in}{4.139516in}}%
\pgfpathlineto{\pgfqpoint{4.106092in}{4.149333in}}%
\pgfpathlineto{\pgfqpoint{4.096481in}{4.158512in}}%
\pgfpathlineto{\pgfqpoint{4.086626in}{4.168105in}}%
\pgfpathlineto{\pgfqpoint{4.067327in}{4.186667in}}%
\pgfpathlineto{\pgfqpoint{4.046545in}{4.206640in}}%
\pgfpathlineto{\pgfqpoint{4.037316in}{4.215403in}}%
\pgfpathlineto{\pgfqpoint{4.028435in}{4.224000in}}%
\pgfpathlineto{\pgfqpoint{4.025554in}{4.224000in}}%
\pgfpathlineto{\pgfqpoint{4.035848in}{4.214035in}}%
\pgfpathlineto{\pgfqpoint{4.046545in}{4.203878in}}%
\pgfpathlineto{\pgfqpoint{4.064453in}{4.186667in}}%
\pgfpathlineto{\pgfqpoint{4.086626in}{4.165341in}}%
\pgfpathlineto{\pgfqpoint{4.095030in}{4.157161in}}%
\pgfpathlineto{\pgfqpoint{4.103226in}{4.149333in}}%
\pgfpathlineto{\pgfqpoint{4.114701in}{4.138151in}}%
\pgfpathlineto{\pgfqpoint{4.126707in}{4.126674in}}%
\pgfpathlineto{\pgfqpoint{4.134398in}{4.119164in}}%
\pgfpathlineto{\pgfqpoint{4.141874in}{4.112000in}}%
\pgfpathlineto{\pgfqpoint{4.166788in}{4.087878in}}%
\pgfpathlineto{\pgfqpoint{4.173700in}{4.081105in}}%
\pgfpathlineto{\pgfqpoint{4.180397in}{4.074667in}}%
\pgfpathlineto{\pgfqpoint{4.206869in}{4.048952in}}%
\pgfpathlineto{\pgfqpoint{4.212937in}{4.042986in}}%
\pgfpathlineto{\pgfqpoint{4.218797in}{4.037333in}}%
\pgfpathlineto{\pgfqpoint{4.232485in}{4.023861in}}%
\pgfpathlineto{\pgfqpoint{4.246949in}{4.009895in}}%
\pgfpathlineto{\pgfqpoint{4.257075in}{4.000000in}}%
\pgfpathlineto{\pgfqpoint{4.271615in}{3.985641in}}%
\pgfpathlineto{\pgfqpoint{4.287030in}{3.970707in}}%
\pgfpathlineto{\pgfqpoint{4.291216in}{3.966565in}}%
\pgfpathlineto{\pgfqpoint{4.295231in}{3.962667in}}%
\pgfpathlineto{\pgfqpoint{4.327111in}{3.931389in}}%
\pgfpathlineto{\pgfqpoint{4.330258in}{3.928265in}}%
\pgfpathlineto{\pgfqpoint{4.333267in}{3.925333in}}%
\pgfpathlineto{\pgfqpoint{4.367192in}{3.891939in}}%
\pgfpathlineto{\pgfqpoint{4.369236in}{3.889904in}}%
\pgfpathlineto{\pgfqpoint{4.371183in}{3.888000in}}%
\pgfpathlineto{\pgfqpoint{4.407273in}{3.852358in}}%
\pgfpathlineto{\pgfqpoint{4.408149in}{3.851483in}}%
\pgfpathlineto{\pgfqpoint{4.408981in}{3.850667in}}%
\pgfpathlineto{\pgfqpoint{4.435766in}{3.824127in}}%
\pgfpathlineto{\pgfqpoint{4.446652in}{3.813333in}}%
\pgfpathlineto{\pgfqpoint{4.446990in}{3.812995in}}%
\pgfpathlineto{\pgfqpoint{4.447354in}{3.812638in}}%
\pgfpathlineto{\pgfqpoint{4.466284in}{3.793633in}}%
\pgfpathlineto{\pgfqpoint{4.484186in}{3.776000in}}%
\pgfpathlineto{\pgfqpoint{4.485748in}{3.774429in}}%
\pgfpathlineto{\pgfqpoint{4.487434in}{3.772766in}}%
\pgfpathlineto{\pgfqpoint{4.505023in}{3.755050in}}%
\pgfpathlineto{\pgfqpoint{4.521602in}{3.738667in}}%
\pgfpathlineto{\pgfqpoint{4.527515in}{3.732761in}}%
\pgfpathlineto{\pgfqpoint{4.558902in}{3.701333in}}%
\pgfpathlineto{\pgfqpoint{4.563068in}{3.697116in}}%
\pgfpathlineto{\pgfqpoint{4.567596in}{3.692621in}}%
\pgfpathlineto{\pgfqpoint{4.596086in}{3.664000in}}%
\pgfpathlineto{\pgfqpoint{4.607677in}{3.652348in}}%
\pgfpathlineto{\pgfqpoint{4.620856in}{3.638942in}}%
\pgfpathlineto{\pgfqpoint{4.633156in}{3.626667in}}%
\pgfpathlineto{\pgfqpoint{4.647758in}{3.611939in}}%
\pgfpathlineto{\pgfqpoint{4.670113in}{3.589333in}}%
\pgfpathlineto{\pgfqpoint{4.687838in}{3.571396in}}%
\pgfpathlineto{\pgfqpoint{4.697758in}{3.561240in}}%
\pgfpathlineto{\pgfqpoint{4.706957in}{3.552000in}}%
\pgfpathlineto{\pgfqpoint{4.727919in}{3.530718in}}%
\pgfpathlineto{\pgfqpoint{4.736114in}{3.522300in}}%
\pgfpathlineto{\pgfqpoint{4.743689in}{3.514667in}}%
\pgfpathlineto{\pgfqpoint{4.768000in}{3.489904in}}%
\pgfusepath{fill}%
\end{pgfscope}%
\begin{pgfscope}%
\pgfpathrectangle{\pgfqpoint{0.800000in}{0.528000in}}{\pgfqpoint{3.968000in}{3.696000in}}%
\pgfusepath{clip}%
\pgfsetbuttcap%
\pgfsetroundjoin%
\definecolor{currentfill}{rgb}{0.266941,0.748751,0.440573}%
\pgfsetfillcolor{currentfill}%
\pgfsetlinewidth{0.000000pt}%
\definecolor{currentstroke}{rgb}{0.000000,0.000000,0.000000}%
\pgfsetstrokecolor{currentstroke}%
\pgfsetdash{}{0pt}%
\pgfpathmoveto{\pgfqpoint{4.768000in}{3.495557in}}%
\pgfpathlineto{\pgfqpoint{4.749239in}{3.514667in}}%
\pgfpathlineto{\pgfqpoint{4.738999in}{3.524987in}}%
\pgfpathlineto{\pgfqpoint{4.727919in}{3.536367in}}%
\pgfpathlineto{\pgfqpoint{4.712521in}{3.552000in}}%
\pgfpathlineto{\pgfqpoint{4.700646in}{3.563929in}}%
\pgfpathlineto{\pgfqpoint{4.687838in}{3.577042in}}%
\pgfpathlineto{\pgfqpoint{4.675692in}{3.589333in}}%
\pgfpathlineto{\pgfqpoint{4.647758in}{3.617581in}}%
\pgfpathlineto{\pgfqpoint{4.638749in}{3.626667in}}%
\pgfpathlineto{\pgfqpoint{4.623749in}{3.641637in}}%
\pgfpathlineto{\pgfqpoint{4.607677in}{3.657985in}}%
\pgfpathlineto{\pgfqpoint{4.601694in}{3.664000in}}%
\pgfpathlineto{\pgfqpoint{4.567596in}{3.698255in}}%
\pgfpathlineto{\pgfqpoint{4.565996in}{3.699843in}}%
\pgfpathlineto{\pgfqpoint{4.564524in}{3.701333in}}%
\pgfpathlineto{\pgfqpoint{4.527515in}{3.738390in}}%
\pgfpathlineto{\pgfqpoint{4.527238in}{3.738667in}}%
\pgfpathlineto{\pgfqpoint{4.507925in}{3.757753in}}%
\pgfpathlineto{\pgfqpoint{4.489809in}{3.776000in}}%
\pgfpathlineto{\pgfqpoint{4.488656in}{3.777138in}}%
\pgfpathlineto{\pgfqpoint{4.487434in}{3.778367in}}%
\pgfpathlineto{\pgfqpoint{4.469188in}{3.796338in}}%
\pgfpathlineto{\pgfqpoint{4.452260in}{3.813333in}}%
\pgfpathlineto{\pgfqpoint{4.447354in}{3.818208in}}%
\pgfpathlineto{\pgfqpoint{4.414595in}{3.850667in}}%
\pgfpathlineto{\pgfqpoint{4.411028in}{3.854164in}}%
\pgfpathlineto{\pgfqpoint{4.407273in}{3.857917in}}%
\pgfpathlineto{\pgfqpoint{4.376812in}{3.888000in}}%
\pgfpathlineto{\pgfqpoint{4.372117in}{3.892588in}}%
\pgfpathlineto{\pgfqpoint{4.367192in}{3.897494in}}%
\pgfpathlineto{\pgfqpoint{4.338910in}{3.925333in}}%
\pgfpathlineto{\pgfqpoint{4.333143in}{3.930951in}}%
\pgfpathlineto{\pgfqpoint{4.327111in}{3.936940in}}%
\pgfpathlineto{\pgfqpoint{4.300889in}{3.962667in}}%
\pgfpathlineto{\pgfqpoint{4.294103in}{3.969255in}}%
\pgfpathlineto{\pgfqpoint{4.287030in}{3.976254in}}%
\pgfpathlineto{\pgfqpoint{4.274534in}{3.988360in}}%
\pgfpathlineto{\pgfqpoint{4.262748in}{4.000000in}}%
\pgfpathlineto{\pgfqpoint{4.246949in}{4.015438in}}%
\pgfpathlineto{\pgfqpoint{4.235407in}{4.026582in}}%
\pgfpathlineto{\pgfqpoint{4.224485in}{4.037333in}}%
\pgfpathlineto{\pgfqpoint{4.215830in}{4.045681in}}%
\pgfpathlineto{\pgfqpoint{4.206869in}{4.054491in}}%
\pgfpathlineto{\pgfqpoint{4.186100in}{4.074667in}}%
\pgfpathlineto{\pgfqpoint{4.176596in}{4.083803in}}%
\pgfpathlineto{\pgfqpoint{4.166788in}{4.093414in}}%
\pgfpathlineto{\pgfqpoint{4.147591in}{4.112000in}}%
\pgfpathlineto{\pgfqpoint{4.137297in}{4.121864in}}%
\pgfpathlineto{\pgfqpoint{4.126707in}{4.132206in}}%
\pgfpathlineto{\pgfqpoint{4.117632in}{4.140881in}}%
\pgfpathlineto{\pgfqpoint{4.108959in}{4.149333in}}%
\pgfpathlineto{\pgfqpoint{4.097932in}{4.159864in}}%
\pgfpathlineto{\pgfqpoint{4.086626in}{4.170869in}}%
\pgfpathlineto{\pgfqpoint{4.070201in}{4.186667in}}%
\pgfpathlineto{\pgfqpoint{4.046545in}{4.209402in}}%
\pgfpathlineto{\pgfqpoint{4.038784in}{4.216771in}}%
\pgfpathlineto{\pgfqpoint{4.031317in}{4.224000in}}%
\pgfpathlineto{\pgfqpoint{4.028435in}{4.224000in}}%
\pgfpathlineto{\pgfqpoint{4.037316in}{4.215403in}}%
\pgfpathlineto{\pgfqpoint{4.046545in}{4.206640in}}%
\pgfpathlineto{\pgfqpoint{4.067327in}{4.186667in}}%
\pgfpathlineto{\pgfqpoint{4.086626in}{4.168105in}}%
\pgfpathlineto{\pgfqpoint{4.096481in}{4.158512in}}%
\pgfpathlineto{\pgfqpoint{4.106092in}{4.149333in}}%
\pgfpathlineto{\pgfqpoint{4.116167in}{4.139516in}}%
\pgfpathlineto{\pgfqpoint{4.126707in}{4.129440in}}%
\pgfpathlineto{\pgfqpoint{4.135847in}{4.120514in}}%
\pgfpathlineto{\pgfqpoint{4.144733in}{4.112000in}}%
\pgfpathlineto{\pgfqpoint{4.166788in}{4.090646in}}%
\pgfpathlineto{\pgfqpoint{4.175148in}{4.082454in}}%
\pgfpathlineto{\pgfqpoint{4.183248in}{4.074667in}}%
\pgfpathlineto{\pgfqpoint{4.206869in}{4.051721in}}%
\pgfpathlineto{\pgfqpoint{4.214384in}{4.044333in}}%
\pgfpathlineto{\pgfqpoint{4.221641in}{4.037333in}}%
\pgfpathlineto{\pgfqpoint{4.233946in}{4.025221in}}%
\pgfpathlineto{\pgfqpoint{4.246949in}{4.012666in}}%
\pgfpathlineto{\pgfqpoint{4.259911in}{4.000000in}}%
\pgfpathlineto{\pgfqpoint{4.273074in}{3.987001in}}%
\pgfpathlineto{\pgfqpoint{4.287030in}{3.973481in}}%
\pgfpathlineto{\pgfqpoint{4.292660in}{3.967910in}}%
\pgfpathlineto{\pgfqpoint{4.298060in}{3.962667in}}%
\pgfpathlineto{\pgfqpoint{4.327111in}{3.934164in}}%
\pgfpathlineto{\pgfqpoint{4.331700in}{3.929608in}}%
\pgfpathlineto{\pgfqpoint{4.336089in}{3.925333in}}%
\pgfpathlineto{\pgfqpoint{4.367192in}{3.894716in}}%
\pgfpathlineto{\pgfqpoint{4.370676in}{3.891246in}}%
\pgfpathlineto{\pgfqpoint{4.373997in}{3.888000in}}%
\pgfpathlineto{\pgfqpoint{4.407273in}{3.855137in}}%
\pgfpathlineto{\pgfqpoint{4.409588in}{3.852823in}}%
\pgfpathlineto{\pgfqpoint{4.411788in}{3.850667in}}%
\pgfpathlineto{\pgfqpoint{4.447354in}{3.815426in}}%
\pgfpathlineto{\pgfqpoint{4.449460in}{3.813333in}}%
\pgfpathlineto{\pgfqpoint{4.467736in}{3.794985in}}%
\pgfpathlineto{\pgfqpoint{4.487011in}{3.776000in}}%
\pgfpathlineto{\pgfqpoint{4.487215in}{3.775795in}}%
\pgfpathlineto{\pgfqpoint{4.487434in}{3.775579in}}%
\pgfpathlineto{\pgfqpoint{4.506474in}{3.756401in}}%
\pgfpathlineto{\pgfqpoint{4.524420in}{3.738667in}}%
\pgfpathlineto{\pgfqpoint{4.527515in}{3.735576in}}%
\pgfpathlineto{\pgfqpoint{4.561713in}{3.701333in}}%
\pgfpathlineto{\pgfqpoint{4.564532in}{3.698480in}}%
\pgfpathlineto{\pgfqpoint{4.567596in}{3.695438in}}%
\pgfpathlineto{\pgfqpoint{4.598890in}{3.664000in}}%
\pgfpathlineto{\pgfqpoint{4.607677in}{3.655166in}}%
\pgfpathlineto{\pgfqpoint{4.622302in}{3.640290in}}%
\pgfpathlineto{\pgfqpoint{4.635953in}{3.626667in}}%
\pgfpathlineto{\pgfqpoint{4.647758in}{3.614760in}}%
\pgfpathlineto{\pgfqpoint{4.672902in}{3.589333in}}%
\pgfpathlineto{\pgfqpoint{4.687838in}{3.574219in}}%
\pgfpathlineto{\pgfqpoint{4.699202in}{3.562585in}}%
\pgfpathlineto{\pgfqpoint{4.709739in}{3.552000in}}%
\pgfpathlineto{\pgfqpoint{4.727919in}{3.533543in}}%
\pgfpathlineto{\pgfqpoint{4.737557in}{3.523644in}}%
\pgfpathlineto{\pgfqpoint{4.746464in}{3.514667in}}%
\pgfpathlineto{\pgfqpoint{4.768000in}{3.492731in}}%
\pgfusepath{fill}%
\end{pgfscope}%
\begin{pgfscope}%
\pgfpathrectangle{\pgfqpoint{0.800000in}{0.528000in}}{\pgfqpoint{3.968000in}{3.696000in}}%
\pgfusepath{clip}%
\pgfsetbuttcap%
\pgfsetroundjoin%
\definecolor{currentfill}{rgb}{0.266941,0.748751,0.440573}%
\pgfsetfillcolor{currentfill}%
\pgfsetlinewidth{0.000000pt}%
\definecolor{currentstroke}{rgb}{0.000000,0.000000,0.000000}%
\pgfsetstrokecolor{currentstroke}%
\pgfsetdash{}{0pt}%
\pgfpathmoveto{\pgfqpoint{4.768000in}{3.498384in}}%
\pgfpathlineto{\pgfqpoint{4.752014in}{3.514667in}}%
\pgfpathlineto{\pgfqpoint{4.740441in}{3.526330in}}%
\pgfpathlineto{\pgfqpoint{4.727919in}{3.539192in}}%
\pgfpathlineto{\pgfqpoint{4.715304in}{3.552000in}}%
\pgfpathlineto{\pgfqpoint{4.702089in}{3.565274in}}%
\pgfpathlineto{\pgfqpoint{4.687838in}{3.579864in}}%
\pgfpathlineto{\pgfqpoint{4.678481in}{3.589333in}}%
\pgfpathlineto{\pgfqpoint{4.647758in}{3.620402in}}%
\pgfpathlineto{\pgfqpoint{4.641546in}{3.626667in}}%
\pgfpathlineto{\pgfqpoint{4.625195in}{3.642984in}}%
\pgfpathlineto{\pgfqpoint{4.607677in}{3.660804in}}%
\pgfpathlineto{\pgfqpoint{4.604498in}{3.664000in}}%
\pgfpathlineto{\pgfqpoint{4.567596in}{3.701072in}}%
\pgfpathlineto{\pgfqpoint{4.567460in}{3.701207in}}%
\pgfpathlineto{\pgfqpoint{4.567335in}{3.701333in}}%
\pgfpathlineto{\pgfqpoint{4.563852in}{3.704821in}}%
\pgfpathlineto{\pgfqpoint{4.530027in}{3.738667in}}%
\pgfpathlineto{\pgfqpoint{4.527515in}{3.741178in}}%
\pgfpathlineto{\pgfqpoint{4.509376in}{3.759104in}}%
\pgfpathlineto{\pgfqpoint{4.492601in}{3.776000in}}%
\pgfpathlineto{\pgfqpoint{4.490093in}{3.778476in}}%
\pgfpathlineto{\pgfqpoint{4.487434in}{3.781150in}}%
\pgfpathlineto{\pgfqpoint{4.470641in}{3.797691in}}%
\pgfpathlineto{\pgfqpoint{4.455060in}{3.813333in}}%
\pgfpathlineto{\pgfqpoint{4.447354in}{3.820989in}}%
\pgfpathlineto{\pgfqpoint{4.417402in}{3.850667in}}%
\pgfpathlineto{\pgfqpoint{4.412467in}{3.855505in}}%
\pgfpathlineto{\pgfqpoint{4.407273in}{3.860696in}}%
\pgfpathlineto{\pgfqpoint{4.379626in}{3.888000in}}%
\pgfpathlineto{\pgfqpoint{4.373558in}{3.893930in}}%
\pgfpathlineto{\pgfqpoint{4.367192in}{3.900271in}}%
\pgfpathlineto{\pgfqpoint{4.341732in}{3.925333in}}%
\pgfpathlineto{\pgfqpoint{4.334585in}{3.932295in}}%
\pgfpathlineto{\pgfqpoint{4.327111in}{3.939715in}}%
\pgfpathlineto{\pgfqpoint{4.303718in}{3.962667in}}%
\pgfpathlineto{\pgfqpoint{4.295547in}{3.970600in}}%
\pgfpathlineto{\pgfqpoint{4.287030in}{3.979028in}}%
\pgfpathlineto{\pgfqpoint{4.275994in}{3.989720in}}%
\pgfpathlineto{\pgfqpoint{4.265584in}{4.000000in}}%
\pgfpathlineto{\pgfqpoint{4.246949in}{4.018210in}}%
\pgfpathlineto{\pgfqpoint{4.236868in}{4.027943in}}%
\pgfpathlineto{\pgfqpoint{4.227328in}{4.037333in}}%
\pgfpathlineto{\pgfqpoint{4.217277in}{4.047028in}}%
\pgfpathlineto{\pgfqpoint{4.206869in}{4.057261in}}%
\pgfpathlineto{\pgfqpoint{4.188951in}{4.074667in}}%
\pgfpathlineto{\pgfqpoint{4.178044in}{4.085152in}}%
\pgfpathlineto{\pgfqpoint{4.166788in}{4.096181in}}%
\pgfpathlineto{\pgfqpoint{4.150450in}{4.112000in}}%
\pgfpathlineto{\pgfqpoint{4.138746in}{4.123214in}}%
\pgfpathlineto{\pgfqpoint{4.126707in}{4.134972in}}%
\pgfpathlineto{\pgfqpoint{4.119098in}{4.142246in}}%
\pgfpathlineto{\pgfqpoint{4.111825in}{4.149333in}}%
\pgfpathlineto{\pgfqpoint{4.099383in}{4.161215in}}%
\pgfpathlineto{\pgfqpoint{4.086626in}{4.173633in}}%
\pgfpathlineto{\pgfqpoint{4.073074in}{4.186667in}}%
\pgfpathlineto{\pgfqpoint{4.046545in}{4.212164in}}%
\pgfpathlineto{\pgfqpoint{4.040253in}{4.218139in}}%
\pgfpathlineto{\pgfqpoint{4.034198in}{4.224000in}}%
\pgfpathlineto{\pgfqpoint{4.031317in}{4.224000in}}%
\pgfpathlineto{\pgfqpoint{4.038784in}{4.216771in}}%
\pgfpathlineto{\pgfqpoint{4.046545in}{4.209402in}}%
\pgfpathlineto{\pgfqpoint{4.070201in}{4.186667in}}%
\pgfpathlineto{\pgfqpoint{4.086626in}{4.170869in}}%
\pgfpathlineto{\pgfqpoint{4.097932in}{4.159864in}}%
\pgfpathlineto{\pgfqpoint{4.108959in}{4.149333in}}%
\pgfpathlineto{\pgfqpoint{4.117632in}{4.140881in}}%
\pgfpathlineto{\pgfqpoint{4.126707in}{4.132206in}}%
\pgfpathlineto{\pgfqpoint{4.137297in}{4.121864in}}%
\pgfpathlineto{\pgfqpoint{4.147591in}{4.112000in}}%
\pgfpathlineto{\pgfqpoint{4.166788in}{4.093414in}}%
\pgfpathlineto{\pgfqpoint{4.176596in}{4.083803in}}%
\pgfpathlineto{\pgfqpoint{4.186100in}{4.074667in}}%
\pgfpathlineto{\pgfqpoint{4.206869in}{4.054491in}}%
\pgfpathlineto{\pgfqpoint{4.215830in}{4.045681in}}%
\pgfpathlineto{\pgfqpoint{4.224485in}{4.037333in}}%
\pgfpathlineto{\pgfqpoint{4.235407in}{4.026582in}}%
\pgfpathlineto{\pgfqpoint{4.246949in}{4.015438in}}%
\pgfpathlineto{\pgfqpoint{4.262748in}{4.000000in}}%
\pgfpathlineto{\pgfqpoint{4.274534in}{3.988360in}}%
\pgfpathlineto{\pgfqpoint{4.287030in}{3.976254in}}%
\pgfpathlineto{\pgfqpoint{4.294103in}{3.969255in}}%
\pgfpathlineto{\pgfqpoint{4.300889in}{3.962667in}}%
\pgfpathlineto{\pgfqpoint{4.327111in}{3.936940in}}%
\pgfpathlineto{\pgfqpoint{4.333143in}{3.930951in}}%
\pgfpathlineto{\pgfqpoint{4.338910in}{3.925333in}}%
\pgfpathlineto{\pgfqpoint{4.367192in}{3.897494in}}%
\pgfpathlineto{\pgfqpoint{4.372117in}{3.892588in}}%
\pgfpathlineto{\pgfqpoint{4.376812in}{3.888000in}}%
\pgfpathlineto{\pgfqpoint{4.407273in}{3.857917in}}%
\pgfpathlineto{\pgfqpoint{4.411028in}{3.854164in}}%
\pgfpathlineto{\pgfqpoint{4.414595in}{3.850667in}}%
\pgfpathlineto{\pgfqpoint{4.447354in}{3.818208in}}%
\pgfpathlineto{\pgfqpoint{4.452260in}{3.813333in}}%
\pgfpathlineto{\pgfqpoint{4.469188in}{3.796338in}}%
\pgfpathlineto{\pgfqpoint{4.487434in}{3.778367in}}%
\pgfpathlineto{\pgfqpoint{4.488656in}{3.777138in}}%
\pgfpathlineto{\pgfqpoint{4.489809in}{3.776000in}}%
\pgfpathlineto{\pgfqpoint{4.507925in}{3.757753in}}%
\pgfpathlineto{\pgfqpoint{4.527238in}{3.738667in}}%
\pgfpathlineto{\pgfqpoint{4.527515in}{3.738390in}}%
\pgfpathlineto{\pgfqpoint{4.564524in}{3.701333in}}%
\pgfpathlineto{\pgfqpoint{4.565996in}{3.699843in}}%
\pgfpathlineto{\pgfqpoint{4.567596in}{3.698255in}}%
\pgfpathlineto{\pgfqpoint{4.601694in}{3.664000in}}%
\pgfpathlineto{\pgfqpoint{4.607677in}{3.657985in}}%
\pgfpathlineto{\pgfqpoint{4.623749in}{3.641637in}}%
\pgfpathlineto{\pgfqpoint{4.638749in}{3.626667in}}%
\pgfpathlineto{\pgfqpoint{4.647758in}{3.617581in}}%
\pgfpathlineto{\pgfqpoint{4.675692in}{3.589333in}}%
\pgfpathlineto{\pgfqpoint{4.687838in}{3.577042in}}%
\pgfpathlineto{\pgfqpoint{4.700646in}{3.563929in}}%
\pgfpathlineto{\pgfqpoint{4.712521in}{3.552000in}}%
\pgfpathlineto{\pgfqpoint{4.727919in}{3.536367in}}%
\pgfpathlineto{\pgfqpoint{4.738999in}{3.524987in}}%
\pgfpathlineto{\pgfqpoint{4.749239in}{3.514667in}}%
\pgfpathlineto{\pgfqpoint{4.768000in}{3.495557in}}%
\pgfusepath{fill}%
\end{pgfscope}%
\begin{pgfscope}%
\pgfpathrectangle{\pgfqpoint{0.800000in}{0.528000in}}{\pgfqpoint{3.968000in}{3.696000in}}%
\pgfusepath{clip}%
\pgfsetbuttcap%
\pgfsetroundjoin%
\definecolor{currentfill}{rgb}{0.266941,0.748751,0.440573}%
\pgfsetfillcolor{currentfill}%
\pgfsetlinewidth{0.000000pt}%
\definecolor{currentstroke}{rgb}{0.000000,0.000000,0.000000}%
\pgfsetstrokecolor{currentstroke}%
\pgfsetdash{}{0pt}%
\pgfpathmoveto{\pgfqpoint{4.768000in}{3.501211in}}%
\pgfpathlineto{\pgfqpoint{4.754790in}{3.514667in}}%
\pgfpathlineto{\pgfqpoint{4.741883in}{3.527674in}}%
\pgfpathlineto{\pgfqpoint{4.727919in}{3.542017in}}%
\pgfpathlineto{\pgfqpoint{4.718086in}{3.552000in}}%
\pgfpathlineto{\pgfqpoint{4.703533in}{3.566619in}}%
\pgfpathlineto{\pgfqpoint{4.687838in}{3.582687in}}%
\pgfpathlineto{\pgfqpoint{4.681270in}{3.589333in}}%
\pgfpathlineto{\pgfqpoint{4.647758in}{3.623222in}}%
\pgfpathlineto{\pgfqpoint{4.644343in}{3.626667in}}%
\pgfpathlineto{\pgfqpoint{4.626642in}{3.644332in}}%
\pgfpathlineto{\pgfqpoint{4.607677in}{3.663623in}}%
\pgfpathlineto{\pgfqpoint{4.607301in}{3.664000in}}%
\pgfpathlineto{\pgfqpoint{4.602522in}{3.668802in}}%
\pgfpathlineto{\pgfqpoint{4.570116in}{3.701333in}}%
\pgfpathlineto{\pgfqpoint{4.568897in}{3.702545in}}%
\pgfpathlineto{\pgfqpoint{4.567596in}{3.703861in}}%
\pgfpathlineto{\pgfqpoint{4.532812in}{3.738667in}}%
\pgfpathlineto{\pgfqpoint{4.527515in}{3.743963in}}%
\pgfpathlineto{\pgfqpoint{4.510826in}{3.760455in}}%
\pgfpathlineto{\pgfqpoint{4.495394in}{3.776000in}}%
\pgfpathlineto{\pgfqpoint{4.491529in}{3.779814in}}%
\pgfpathlineto{\pgfqpoint{4.487434in}{3.783933in}}%
\pgfpathlineto{\pgfqpoint{4.472093in}{3.799044in}}%
\pgfpathlineto{\pgfqpoint{4.457860in}{3.813333in}}%
\pgfpathlineto{\pgfqpoint{4.447354in}{3.823770in}}%
\pgfpathlineto{\pgfqpoint{4.420209in}{3.850667in}}%
\pgfpathlineto{\pgfqpoint{4.413907in}{3.856846in}}%
\pgfpathlineto{\pgfqpoint{4.407273in}{3.863475in}}%
\pgfpathlineto{\pgfqpoint{4.382440in}{3.888000in}}%
\pgfpathlineto{\pgfqpoint{4.374999in}{3.895272in}}%
\pgfpathlineto{\pgfqpoint{4.367192in}{3.903049in}}%
\pgfpathlineto{\pgfqpoint{4.344553in}{3.925333in}}%
\pgfpathlineto{\pgfqpoint{4.336027in}{3.933638in}}%
\pgfpathlineto{\pgfqpoint{4.327111in}{3.942491in}}%
\pgfpathlineto{\pgfqpoint{4.306547in}{3.962667in}}%
\pgfpathlineto{\pgfqpoint{4.296991in}{3.971945in}}%
\pgfpathlineto{\pgfqpoint{4.287030in}{3.981801in}}%
\pgfpathlineto{\pgfqpoint{4.277453in}{3.991079in}}%
\pgfpathlineto{\pgfqpoint{4.268420in}{4.000000in}}%
\pgfpathlineto{\pgfqpoint{4.246949in}{4.020981in}}%
\pgfpathlineto{\pgfqpoint{4.238329in}{4.029304in}}%
\pgfpathlineto{\pgfqpoint{4.230172in}{4.037333in}}%
\pgfpathlineto{\pgfqpoint{4.218724in}{4.048376in}}%
\pgfpathlineto{\pgfqpoint{4.206869in}{4.060030in}}%
\pgfpathlineto{\pgfqpoint{4.191802in}{4.074667in}}%
\pgfpathlineto{\pgfqpoint{4.179493in}{4.086500in}}%
\pgfpathlineto{\pgfqpoint{4.166788in}{4.098949in}}%
\pgfpathlineto{\pgfqpoint{4.153309in}{4.112000in}}%
\pgfpathlineto{\pgfqpoint{4.140196in}{4.124564in}}%
\pgfpathlineto{\pgfqpoint{4.126707in}{4.137738in}}%
\pgfpathlineto{\pgfqpoint{4.120563in}{4.143611in}}%
\pgfpathlineto{\pgfqpoint{4.114691in}{4.149333in}}%
\pgfpathlineto{\pgfqpoint{4.100834in}{4.162567in}}%
\pgfpathlineto{\pgfqpoint{4.086626in}{4.176397in}}%
\pgfpathlineto{\pgfqpoint{4.075948in}{4.186667in}}%
\pgfpathlineto{\pgfqpoint{4.046545in}{4.214926in}}%
\pgfpathlineto{\pgfqpoint{4.041721in}{4.219507in}}%
\pgfpathlineto{\pgfqpoint{4.037079in}{4.224000in}}%
\pgfpathlineto{\pgfqpoint{4.034198in}{4.224000in}}%
\pgfpathlineto{\pgfqpoint{4.040253in}{4.218139in}}%
\pgfpathlineto{\pgfqpoint{4.046545in}{4.212164in}}%
\pgfpathlineto{\pgfqpoint{4.073074in}{4.186667in}}%
\pgfpathlineto{\pgfqpoint{4.086626in}{4.173633in}}%
\pgfpathlineto{\pgfqpoint{4.099383in}{4.161215in}}%
\pgfpathlineto{\pgfqpoint{4.111825in}{4.149333in}}%
\pgfpathlineto{\pgfqpoint{4.119098in}{4.142246in}}%
\pgfpathlineto{\pgfqpoint{4.126707in}{4.134972in}}%
\pgfpathlineto{\pgfqpoint{4.138746in}{4.123214in}}%
\pgfpathlineto{\pgfqpoint{4.150450in}{4.112000in}}%
\pgfpathlineto{\pgfqpoint{4.166788in}{4.096181in}}%
\pgfpathlineto{\pgfqpoint{4.178044in}{4.085152in}}%
\pgfpathlineto{\pgfqpoint{4.188951in}{4.074667in}}%
\pgfpathlineto{\pgfqpoint{4.206869in}{4.057261in}}%
\pgfpathlineto{\pgfqpoint{4.217277in}{4.047028in}}%
\pgfpathlineto{\pgfqpoint{4.227328in}{4.037333in}}%
\pgfpathlineto{\pgfqpoint{4.236868in}{4.027943in}}%
\pgfpathlineto{\pgfqpoint{4.246949in}{4.018210in}}%
\pgfpathlineto{\pgfqpoint{4.265584in}{4.000000in}}%
\pgfpathlineto{\pgfqpoint{4.275994in}{3.989720in}}%
\pgfpathlineto{\pgfqpoint{4.287030in}{3.979028in}}%
\pgfpathlineto{\pgfqpoint{4.295547in}{3.970600in}}%
\pgfpathlineto{\pgfqpoint{4.303718in}{3.962667in}}%
\pgfpathlineto{\pgfqpoint{4.327111in}{3.939715in}}%
\pgfpathlineto{\pgfqpoint{4.334585in}{3.932295in}}%
\pgfpathlineto{\pgfqpoint{4.341732in}{3.925333in}}%
\pgfpathlineto{\pgfqpoint{4.367192in}{3.900271in}}%
\pgfpathlineto{\pgfqpoint{4.373558in}{3.893930in}}%
\pgfpathlineto{\pgfqpoint{4.379626in}{3.888000in}}%
\pgfpathlineto{\pgfqpoint{4.407273in}{3.860696in}}%
\pgfpathlineto{\pgfqpoint{4.412467in}{3.855505in}}%
\pgfpathlineto{\pgfqpoint{4.417402in}{3.850667in}}%
\pgfpathlineto{\pgfqpoint{4.447354in}{3.820989in}}%
\pgfpathlineto{\pgfqpoint{4.455060in}{3.813333in}}%
\pgfpathlineto{\pgfqpoint{4.470641in}{3.797691in}}%
\pgfpathlineto{\pgfqpoint{4.487434in}{3.781150in}}%
\pgfpathlineto{\pgfqpoint{4.490093in}{3.778476in}}%
\pgfpathlineto{\pgfqpoint{4.492601in}{3.776000in}}%
\pgfpathlineto{\pgfqpoint{4.509376in}{3.759104in}}%
\pgfpathlineto{\pgfqpoint{4.527515in}{3.741178in}}%
\pgfpathlineto{\pgfqpoint{4.530027in}{3.738667in}}%
\pgfpathlineto{\pgfqpoint{4.563852in}{3.704821in}}%
\pgfpathlineto{\pgfqpoint{4.567335in}{3.701333in}}%
\pgfpathlineto{\pgfqpoint{4.567460in}{3.701207in}}%
\pgfpathlineto{\pgfqpoint{4.567596in}{3.701072in}}%
\pgfpathlineto{\pgfqpoint{4.604498in}{3.664000in}}%
\pgfpathlineto{\pgfqpoint{4.607677in}{3.660804in}}%
\pgfpathlineto{\pgfqpoint{4.625195in}{3.642984in}}%
\pgfpathlineto{\pgfqpoint{4.641546in}{3.626667in}}%
\pgfpathlineto{\pgfqpoint{4.647758in}{3.620402in}}%
\pgfpathlineto{\pgfqpoint{4.678481in}{3.589333in}}%
\pgfpathlineto{\pgfqpoint{4.687838in}{3.579864in}}%
\pgfpathlineto{\pgfqpoint{4.702089in}{3.565274in}}%
\pgfpathlineto{\pgfqpoint{4.715304in}{3.552000in}}%
\pgfpathlineto{\pgfqpoint{4.727919in}{3.539192in}}%
\pgfpathlineto{\pgfqpoint{4.740441in}{3.526330in}}%
\pgfpathlineto{\pgfqpoint{4.752014in}{3.514667in}}%
\pgfpathlineto{\pgfqpoint{4.768000in}{3.498384in}}%
\pgfusepath{fill}%
\end{pgfscope}%
\begin{pgfscope}%
\pgfpathrectangle{\pgfqpoint{0.800000in}{0.528000in}}{\pgfqpoint{3.968000in}{3.696000in}}%
\pgfusepath{clip}%
\pgfsetbuttcap%
\pgfsetroundjoin%
\definecolor{currentfill}{rgb}{0.274149,0.751988,0.436601}%
\pgfsetfillcolor{currentfill}%
\pgfsetlinewidth{0.000000pt}%
\definecolor{currentstroke}{rgb}{0.000000,0.000000,0.000000}%
\pgfsetstrokecolor{currentstroke}%
\pgfsetdash{}{0pt}%
\pgfpathmoveto{\pgfqpoint{4.768000in}{3.504038in}}%
\pgfpathlineto{\pgfqpoint{4.757565in}{3.514667in}}%
\pgfpathlineto{\pgfqpoint{4.743326in}{3.529017in}}%
\pgfpathlineto{\pgfqpoint{4.727919in}{3.544841in}}%
\pgfpathlineto{\pgfqpoint{4.720868in}{3.552000in}}%
\pgfpathlineto{\pgfqpoint{4.704977in}{3.567963in}}%
\pgfpathlineto{\pgfqpoint{4.687838in}{3.585510in}}%
\pgfpathlineto{\pgfqpoint{4.684060in}{3.589333in}}%
\pgfpathlineto{\pgfqpoint{4.647758in}{3.626043in}}%
\pgfpathlineto{\pgfqpoint{4.647139in}{3.626667in}}%
\pgfpathlineto{\pgfqpoint{4.628088in}{3.645679in}}%
\pgfpathlineto{\pgfqpoint{4.610077in}{3.664000in}}%
\pgfpathlineto{\pgfqpoint{4.608917in}{3.665156in}}%
\pgfpathlineto{\pgfqpoint{4.607677in}{3.666416in}}%
\pgfpathlineto{\pgfqpoint{4.572894in}{3.701333in}}%
\pgfpathlineto{\pgfqpoint{4.570330in}{3.703880in}}%
\pgfpathlineto{\pgfqpoint{4.567596in}{3.706649in}}%
\pgfpathlineto{\pgfqpoint{4.535598in}{3.738667in}}%
\pgfpathlineto{\pgfqpoint{4.527515in}{3.746749in}}%
\pgfpathlineto{\pgfqpoint{4.512277in}{3.761807in}}%
\pgfpathlineto{\pgfqpoint{4.498186in}{3.776000in}}%
\pgfpathlineto{\pgfqpoint{4.492966in}{3.781152in}}%
\pgfpathlineto{\pgfqpoint{4.487434in}{3.786716in}}%
\pgfpathlineto{\pgfqpoint{4.473545in}{3.800396in}}%
\pgfpathlineto{\pgfqpoint{4.460659in}{3.813333in}}%
\pgfpathlineto{\pgfqpoint{4.447354in}{3.826551in}}%
\pgfpathlineto{\pgfqpoint{4.423016in}{3.850667in}}%
\pgfpathlineto{\pgfqpoint{4.415346in}{3.858187in}}%
\pgfpathlineto{\pgfqpoint{4.407273in}{3.866255in}}%
\pgfpathlineto{\pgfqpoint{4.385254in}{3.888000in}}%
\pgfpathlineto{\pgfqpoint{4.376440in}{3.896614in}}%
\pgfpathlineto{\pgfqpoint{4.367192in}{3.905826in}}%
\pgfpathlineto{\pgfqpoint{4.347375in}{3.925333in}}%
\pgfpathlineto{\pgfqpoint{4.337470in}{3.934982in}}%
\pgfpathlineto{\pgfqpoint{4.327111in}{3.945266in}}%
\pgfpathlineto{\pgfqpoint{4.309376in}{3.962667in}}%
\pgfpathlineto{\pgfqpoint{4.298435in}{3.973289in}}%
\pgfpathlineto{\pgfqpoint{4.287030in}{3.984575in}}%
\pgfpathlineto{\pgfqpoint{4.278913in}{3.992439in}}%
\pgfpathlineto{\pgfqpoint{4.271256in}{4.000000in}}%
\pgfpathlineto{\pgfqpoint{4.246949in}{4.023753in}}%
\pgfpathlineto{\pgfqpoint{4.239791in}{4.030665in}}%
\pgfpathlineto{\pgfqpoint{4.233016in}{4.037333in}}%
\pgfpathlineto{\pgfqpoint{4.220170in}{4.049723in}}%
\pgfpathlineto{\pgfqpoint{4.206869in}{4.062800in}}%
\pgfpathlineto{\pgfqpoint{4.194653in}{4.074667in}}%
\pgfpathlineto{\pgfqpoint{4.180941in}{4.087849in}}%
\pgfpathlineto{\pgfqpoint{4.166788in}{4.101717in}}%
\pgfpathlineto{\pgfqpoint{4.156167in}{4.112000in}}%
\pgfpathlineto{\pgfqpoint{4.141646in}{4.125914in}}%
\pgfpathlineto{\pgfqpoint{4.126707in}{4.140504in}}%
\pgfpathlineto{\pgfqpoint{4.122029in}{4.144976in}}%
\pgfpathlineto{\pgfqpoint{4.117557in}{4.149333in}}%
\pgfpathlineto{\pgfqpoint{4.102285in}{4.163919in}}%
\pgfpathlineto{\pgfqpoint{4.086626in}{4.179161in}}%
\pgfpathlineto{\pgfqpoint{4.078822in}{4.186667in}}%
\pgfpathlineto{\pgfqpoint{4.046545in}{4.217688in}}%
\pgfpathlineto{\pgfqpoint{4.043190in}{4.220874in}}%
\pgfpathlineto{\pgfqpoint{4.039961in}{4.224000in}}%
\pgfpathlineto{\pgfqpoint{4.037079in}{4.224000in}}%
\pgfpathlineto{\pgfqpoint{4.041721in}{4.219507in}}%
\pgfpathlineto{\pgfqpoint{4.046545in}{4.214926in}}%
\pgfpathlineto{\pgfqpoint{4.075948in}{4.186667in}}%
\pgfpathlineto{\pgfqpoint{4.086626in}{4.176397in}}%
\pgfpathlineto{\pgfqpoint{4.100834in}{4.162567in}}%
\pgfpathlineto{\pgfqpoint{4.114691in}{4.149333in}}%
\pgfpathlineto{\pgfqpoint{4.120563in}{4.143611in}}%
\pgfpathlineto{\pgfqpoint{4.126707in}{4.137738in}}%
\pgfpathlineto{\pgfqpoint{4.140196in}{4.124564in}}%
\pgfpathlineto{\pgfqpoint{4.153309in}{4.112000in}}%
\pgfpathlineto{\pgfqpoint{4.166788in}{4.098949in}}%
\pgfpathlineto{\pgfqpoint{4.179493in}{4.086500in}}%
\pgfpathlineto{\pgfqpoint{4.191802in}{4.074667in}}%
\pgfpathlineto{\pgfqpoint{4.206869in}{4.060030in}}%
\pgfpathlineto{\pgfqpoint{4.218724in}{4.048376in}}%
\pgfpathlineto{\pgfqpoint{4.230172in}{4.037333in}}%
\pgfpathlineto{\pgfqpoint{4.238329in}{4.029304in}}%
\pgfpathlineto{\pgfqpoint{4.246949in}{4.020981in}}%
\pgfpathlineto{\pgfqpoint{4.268420in}{4.000000in}}%
\pgfpathlineto{\pgfqpoint{4.277453in}{3.991079in}}%
\pgfpathlineto{\pgfqpoint{4.287030in}{3.981801in}}%
\pgfpathlineto{\pgfqpoint{4.296991in}{3.971945in}}%
\pgfpathlineto{\pgfqpoint{4.306547in}{3.962667in}}%
\pgfpathlineto{\pgfqpoint{4.327111in}{3.942491in}}%
\pgfpathlineto{\pgfqpoint{4.336027in}{3.933638in}}%
\pgfpathlineto{\pgfqpoint{4.344553in}{3.925333in}}%
\pgfpathlineto{\pgfqpoint{4.367192in}{3.903049in}}%
\pgfpathlineto{\pgfqpoint{4.374999in}{3.895272in}}%
\pgfpathlineto{\pgfqpoint{4.382440in}{3.888000in}}%
\pgfpathlineto{\pgfqpoint{4.407273in}{3.863475in}}%
\pgfpathlineto{\pgfqpoint{4.413907in}{3.856846in}}%
\pgfpathlineto{\pgfqpoint{4.420209in}{3.850667in}}%
\pgfpathlineto{\pgfqpoint{4.447354in}{3.823770in}}%
\pgfpathlineto{\pgfqpoint{4.457860in}{3.813333in}}%
\pgfpathlineto{\pgfqpoint{4.472093in}{3.799044in}}%
\pgfpathlineto{\pgfqpoint{4.487434in}{3.783933in}}%
\pgfpathlineto{\pgfqpoint{4.491529in}{3.779814in}}%
\pgfpathlineto{\pgfqpoint{4.495394in}{3.776000in}}%
\pgfpathlineto{\pgfqpoint{4.510826in}{3.760455in}}%
\pgfpathlineto{\pgfqpoint{4.527515in}{3.743963in}}%
\pgfpathlineto{\pgfqpoint{4.532812in}{3.738667in}}%
\pgfpathlineto{\pgfqpoint{4.567596in}{3.703861in}}%
\pgfpathlineto{\pgfqpoint{4.568897in}{3.702545in}}%
\pgfpathlineto{\pgfqpoint{4.570116in}{3.701333in}}%
\pgfpathlineto{\pgfqpoint{4.602522in}{3.668802in}}%
\pgfpathlineto{\pgfqpoint{4.607301in}{3.664000in}}%
\pgfpathlineto{\pgfqpoint{4.607677in}{3.663623in}}%
\pgfpathlineto{\pgfqpoint{4.626642in}{3.644332in}}%
\pgfpathlineto{\pgfqpoint{4.644343in}{3.626667in}}%
\pgfpathlineto{\pgfqpoint{4.647758in}{3.623222in}}%
\pgfpathlineto{\pgfqpoint{4.681270in}{3.589333in}}%
\pgfpathlineto{\pgfqpoint{4.687838in}{3.582687in}}%
\pgfpathlineto{\pgfqpoint{4.703533in}{3.566619in}}%
\pgfpathlineto{\pgfqpoint{4.718086in}{3.552000in}}%
\pgfpathlineto{\pgfqpoint{4.727919in}{3.542017in}}%
\pgfpathlineto{\pgfqpoint{4.741883in}{3.527674in}}%
\pgfpathlineto{\pgfqpoint{4.754790in}{3.514667in}}%
\pgfpathlineto{\pgfqpoint{4.768000in}{3.501211in}}%
\pgfusepath{fill}%
\end{pgfscope}%
\begin{pgfscope}%
\pgfpathrectangle{\pgfqpoint{0.800000in}{0.528000in}}{\pgfqpoint{3.968000in}{3.696000in}}%
\pgfusepath{clip}%
\pgfsetbuttcap%
\pgfsetroundjoin%
\definecolor{currentfill}{rgb}{0.274149,0.751988,0.436601}%
\pgfsetfillcolor{currentfill}%
\pgfsetlinewidth{0.000000pt}%
\definecolor{currentstroke}{rgb}{0.000000,0.000000,0.000000}%
\pgfsetstrokecolor{currentstroke}%
\pgfsetdash{}{0pt}%
\pgfpathmoveto{\pgfqpoint{4.768000in}{3.506864in}}%
\pgfpathlineto{\pgfqpoint{4.760340in}{3.514667in}}%
\pgfpathlineto{\pgfqpoint{4.744768in}{3.530360in}}%
\pgfpathlineto{\pgfqpoint{4.727919in}{3.547666in}}%
\pgfpathlineto{\pgfqpoint{4.723650in}{3.552000in}}%
\pgfpathlineto{\pgfqpoint{4.706420in}{3.569308in}}%
\pgfpathlineto{\pgfqpoint{4.687838in}{3.588332in}}%
\pgfpathlineto{\pgfqpoint{4.686849in}{3.589333in}}%
\pgfpathlineto{\pgfqpoint{4.675300in}{3.601012in}}%
\pgfpathlineto{\pgfqpoint{4.649910in}{3.626667in}}%
\pgfpathlineto{\pgfqpoint{4.648872in}{3.627705in}}%
\pgfpathlineto{\pgfqpoint{4.647758in}{3.628841in}}%
\pgfpathlineto{\pgfqpoint{4.629535in}{3.647026in}}%
\pgfpathlineto{\pgfqpoint{4.612848in}{3.664000in}}%
\pgfpathlineto{\pgfqpoint{4.610350in}{3.666490in}}%
\pgfpathlineto{\pgfqpoint{4.607677in}{3.669205in}}%
\pgfpathlineto{\pgfqpoint{4.575672in}{3.701333in}}%
\pgfpathlineto{\pgfqpoint{4.571764in}{3.705216in}}%
\pgfpathlineto{\pgfqpoint{4.567596in}{3.709436in}}%
\pgfpathlineto{\pgfqpoint{4.538383in}{3.738667in}}%
\pgfpathlineto{\pgfqpoint{4.527515in}{3.749534in}}%
\pgfpathlineto{\pgfqpoint{4.513728in}{3.763158in}}%
\pgfpathlineto{\pgfqpoint{4.500979in}{3.776000in}}%
\pgfpathlineto{\pgfqpoint{4.494403in}{3.782491in}}%
\pgfpathlineto{\pgfqpoint{4.487434in}{3.789499in}}%
\pgfpathlineto{\pgfqpoint{4.474998in}{3.801749in}}%
\pgfpathlineto{\pgfqpoint{4.463459in}{3.813333in}}%
\pgfpathlineto{\pgfqpoint{4.447354in}{3.829333in}}%
\pgfpathlineto{\pgfqpoint{4.425822in}{3.850667in}}%
\pgfpathlineto{\pgfqpoint{4.416786in}{3.859528in}}%
\pgfpathlineto{\pgfqpoint{4.407273in}{3.869034in}}%
\pgfpathlineto{\pgfqpoint{4.388069in}{3.888000in}}%
\pgfpathlineto{\pgfqpoint{4.377881in}{3.897956in}}%
\pgfpathlineto{\pgfqpoint{4.367192in}{3.908603in}}%
\pgfpathlineto{\pgfqpoint{4.350196in}{3.925333in}}%
\pgfpathlineto{\pgfqpoint{4.338912in}{3.936325in}}%
\pgfpathlineto{\pgfqpoint{4.327111in}{3.948042in}}%
\pgfpathlineto{\pgfqpoint{4.312205in}{3.962667in}}%
\pgfpathlineto{\pgfqpoint{4.299879in}{3.974634in}}%
\pgfpathlineto{\pgfqpoint{4.287030in}{3.987348in}}%
\pgfpathlineto{\pgfqpoint{4.280372in}{3.993798in}}%
\pgfpathlineto{\pgfqpoint{4.274093in}{4.000000in}}%
\pgfpathlineto{\pgfqpoint{4.246949in}{4.026524in}}%
\pgfpathlineto{\pgfqpoint{4.241252in}{4.032026in}}%
\pgfpathlineto{\pgfqpoint{4.235859in}{4.037333in}}%
\pgfpathlineto{\pgfqpoint{4.221617in}{4.051071in}}%
\pgfpathlineto{\pgfqpoint{4.206869in}{4.065570in}}%
\pgfpathlineto{\pgfqpoint{4.197504in}{4.074667in}}%
\pgfpathlineto{\pgfqpoint{4.182389in}{4.089198in}}%
\pgfpathlineto{\pgfqpoint{4.166788in}{4.104485in}}%
\pgfpathlineto{\pgfqpoint{4.159026in}{4.112000in}}%
\pgfpathlineto{\pgfqpoint{4.143095in}{4.127265in}}%
\pgfpathlineto{\pgfqpoint{4.126707in}{4.143270in}}%
\pgfpathlineto{\pgfqpoint{4.123494in}{4.146341in}}%
\pgfpathlineto{\pgfqpoint{4.120423in}{4.149333in}}%
\pgfpathlineto{\pgfqpoint{4.103736in}{4.165270in}}%
\pgfpathlineto{\pgfqpoint{4.086626in}{4.181925in}}%
\pgfpathlineto{\pgfqpoint{4.081696in}{4.186667in}}%
\pgfpathlineto{\pgfqpoint{4.046545in}{4.220450in}}%
\pgfpathlineto{\pgfqpoint{4.044658in}{4.222242in}}%
\pgfpathlineto{\pgfqpoint{4.042842in}{4.224000in}}%
\pgfpathlineto{\pgfqpoint{4.039961in}{4.224000in}}%
\pgfpathlineto{\pgfqpoint{4.043190in}{4.220874in}}%
\pgfpathlineto{\pgfqpoint{4.046545in}{4.217688in}}%
\pgfpathlineto{\pgfqpoint{4.078822in}{4.186667in}}%
\pgfpathlineto{\pgfqpoint{4.086626in}{4.179161in}}%
\pgfpathlineto{\pgfqpoint{4.102285in}{4.163919in}}%
\pgfpathlineto{\pgfqpoint{4.117557in}{4.149333in}}%
\pgfpathlineto{\pgfqpoint{4.122029in}{4.144976in}}%
\pgfpathlineto{\pgfqpoint{4.126707in}{4.140504in}}%
\pgfpathlineto{\pgfqpoint{4.141646in}{4.125914in}}%
\pgfpathlineto{\pgfqpoint{4.156167in}{4.112000in}}%
\pgfpathlineto{\pgfqpoint{4.166788in}{4.101717in}}%
\pgfpathlineto{\pgfqpoint{4.180941in}{4.087849in}}%
\pgfpathlineto{\pgfqpoint{4.194653in}{4.074667in}}%
\pgfpathlineto{\pgfqpoint{4.206869in}{4.062800in}}%
\pgfpathlineto{\pgfqpoint{4.220170in}{4.049723in}}%
\pgfpathlineto{\pgfqpoint{4.233016in}{4.037333in}}%
\pgfpathlineto{\pgfqpoint{4.239791in}{4.030665in}}%
\pgfpathlineto{\pgfqpoint{4.246949in}{4.023753in}}%
\pgfpathlineto{\pgfqpoint{4.271256in}{4.000000in}}%
\pgfpathlineto{\pgfqpoint{4.278913in}{3.992439in}}%
\pgfpathlineto{\pgfqpoint{4.287030in}{3.984575in}}%
\pgfpathlineto{\pgfqpoint{4.298435in}{3.973289in}}%
\pgfpathlineto{\pgfqpoint{4.309376in}{3.962667in}}%
\pgfpathlineto{\pgfqpoint{4.327111in}{3.945266in}}%
\pgfpathlineto{\pgfqpoint{4.337470in}{3.934982in}}%
\pgfpathlineto{\pgfqpoint{4.347375in}{3.925333in}}%
\pgfpathlineto{\pgfqpoint{4.367192in}{3.905826in}}%
\pgfpathlineto{\pgfqpoint{4.376440in}{3.896614in}}%
\pgfpathlineto{\pgfqpoint{4.385254in}{3.888000in}}%
\pgfpathlineto{\pgfqpoint{4.407273in}{3.866255in}}%
\pgfpathlineto{\pgfqpoint{4.415346in}{3.858187in}}%
\pgfpathlineto{\pgfqpoint{4.423016in}{3.850667in}}%
\pgfpathlineto{\pgfqpoint{4.447354in}{3.826551in}}%
\pgfpathlineto{\pgfqpoint{4.460659in}{3.813333in}}%
\pgfpathlineto{\pgfqpoint{4.473545in}{3.800396in}}%
\pgfpathlineto{\pgfqpoint{4.487434in}{3.786716in}}%
\pgfpathlineto{\pgfqpoint{4.492966in}{3.781152in}}%
\pgfpathlineto{\pgfqpoint{4.498186in}{3.776000in}}%
\pgfpathlineto{\pgfqpoint{4.512277in}{3.761807in}}%
\pgfpathlineto{\pgfqpoint{4.527515in}{3.746749in}}%
\pgfpathlineto{\pgfqpoint{4.535598in}{3.738667in}}%
\pgfpathlineto{\pgfqpoint{4.567596in}{3.706649in}}%
\pgfpathlineto{\pgfqpoint{4.570330in}{3.703880in}}%
\pgfpathlineto{\pgfqpoint{4.572894in}{3.701333in}}%
\pgfpathlineto{\pgfqpoint{4.607677in}{3.666416in}}%
\pgfpathlineto{\pgfqpoint{4.608917in}{3.665156in}}%
\pgfpathlineto{\pgfqpoint{4.610077in}{3.664000in}}%
\pgfpathlineto{\pgfqpoint{4.628088in}{3.645679in}}%
\pgfpathlineto{\pgfqpoint{4.647139in}{3.626667in}}%
\pgfpathlineto{\pgfqpoint{4.647758in}{3.626043in}}%
\pgfpathlineto{\pgfqpoint{4.684060in}{3.589333in}}%
\pgfpathlineto{\pgfqpoint{4.687838in}{3.585510in}}%
\pgfpathlineto{\pgfqpoint{4.704977in}{3.567963in}}%
\pgfpathlineto{\pgfqpoint{4.720868in}{3.552000in}}%
\pgfpathlineto{\pgfqpoint{4.727919in}{3.544841in}}%
\pgfpathlineto{\pgfqpoint{4.743326in}{3.529017in}}%
\pgfpathlineto{\pgfqpoint{4.757565in}{3.514667in}}%
\pgfpathlineto{\pgfqpoint{4.768000in}{3.504038in}}%
\pgfusepath{fill}%
\end{pgfscope}%
\begin{pgfscope}%
\pgfpathrectangle{\pgfqpoint{0.800000in}{0.528000in}}{\pgfqpoint{3.968000in}{3.696000in}}%
\pgfusepath{clip}%
\pgfsetbuttcap%
\pgfsetroundjoin%
\definecolor{currentfill}{rgb}{0.274149,0.751988,0.436601}%
\pgfsetfillcolor{currentfill}%
\pgfsetlinewidth{0.000000pt}%
\definecolor{currentstroke}{rgb}{0.000000,0.000000,0.000000}%
\pgfsetstrokecolor{currentstroke}%
\pgfsetdash{}{0pt}%
\pgfpathmoveto{\pgfqpoint{4.768000in}{3.509691in}}%
\pgfpathlineto{\pgfqpoint{4.763115in}{3.514667in}}%
\pgfpathlineto{\pgfqpoint{4.746210in}{3.531704in}}%
\pgfpathlineto{\pgfqpoint{4.727919in}{3.550491in}}%
\pgfpathlineto{\pgfqpoint{4.726433in}{3.552000in}}%
\pgfpathlineto{\pgfqpoint{4.707864in}{3.570653in}}%
\pgfpathlineto{\pgfqpoint{4.689618in}{3.589333in}}%
\pgfpathlineto{\pgfqpoint{4.687838in}{3.591136in}}%
\pgfpathlineto{\pgfqpoint{4.652675in}{3.626667in}}%
\pgfpathlineto{\pgfqpoint{4.650303in}{3.629038in}}%
\pgfpathlineto{\pgfqpoint{4.647758in}{3.631631in}}%
\pgfpathlineto{\pgfqpoint{4.630981in}{3.648374in}}%
\pgfpathlineto{\pgfqpoint{4.615619in}{3.664000in}}%
\pgfpathlineto{\pgfqpoint{4.611782in}{3.667824in}}%
\pgfpathlineto{\pgfqpoint{4.607677in}{3.671994in}}%
\pgfpathlineto{\pgfqpoint{4.578451in}{3.701333in}}%
\pgfpathlineto{\pgfqpoint{4.573198in}{3.706551in}}%
\pgfpathlineto{\pgfqpoint{4.567596in}{3.712223in}}%
\pgfpathlineto{\pgfqpoint{4.541168in}{3.738667in}}%
\pgfpathlineto{\pgfqpoint{4.527515in}{3.752319in}}%
\pgfpathlineto{\pgfqpoint{4.515179in}{3.764509in}}%
\pgfpathlineto{\pgfqpoint{4.503771in}{3.776000in}}%
\pgfpathlineto{\pgfqpoint{4.495839in}{3.783829in}}%
\pgfpathlineto{\pgfqpoint{4.487434in}{3.792282in}}%
\pgfpathlineto{\pgfqpoint{4.476450in}{3.803102in}}%
\pgfpathlineto{\pgfqpoint{4.466259in}{3.813333in}}%
\pgfpathlineto{\pgfqpoint{4.447354in}{3.832114in}}%
\pgfpathlineto{\pgfqpoint{4.428629in}{3.850667in}}%
\pgfpathlineto{\pgfqpoint{4.418225in}{3.860868in}}%
\pgfpathlineto{\pgfqpoint{4.407273in}{3.871813in}}%
\pgfpathlineto{\pgfqpoint{4.390883in}{3.888000in}}%
\pgfpathlineto{\pgfqpoint{4.379322in}{3.899299in}}%
\pgfpathlineto{\pgfqpoint{4.367192in}{3.911381in}}%
\pgfpathlineto{\pgfqpoint{4.353018in}{3.925333in}}%
\pgfpathlineto{\pgfqpoint{4.340355in}{3.937669in}}%
\pgfpathlineto{\pgfqpoint{4.327111in}{3.950817in}}%
\pgfpathlineto{\pgfqpoint{4.315033in}{3.962667in}}%
\pgfpathlineto{\pgfqpoint{4.301322in}{3.975979in}}%
\pgfpathlineto{\pgfqpoint{4.287030in}{3.990122in}}%
\pgfpathlineto{\pgfqpoint{4.281832in}{3.995158in}}%
\pgfpathlineto{\pgfqpoint{4.276929in}{4.000000in}}%
\pgfpathlineto{\pgfqpoint{4.246949in}{4.029296in}}%
\pgfpathlineto{\pgfqpoint{4.242713in}{4.033387in}}%
\pgfpathlineto{\pgfqpoint{4.238703in}{4.037333in}}%
\pgfpathlineto{\pgfqpoint{4.223064in}{4.052418in}}%
\pgfpathlineto{\pgfqpoint{4.206869in}{4.068339in}}%
\pgfpathlineto{\pgfqpoint{4.200355in}{4.074667in}}%
\pgfpathlineto{\pgfqpoint{4.183837in}{4.090547in}}%
\pgfpathlineto{\pgfqpoint{4.166788in}{4.107253in}}%
\pgfpathlineto{\pgfqpoint{4.161884in}{4.112000in}}%
\pgfpathlineto{\pgfqpoint{4.144545in}{4.128615in}}%
\pgfpathlineto{\pgfqpoint{4.126707in}{4.146035in}}%
\pgfpathlineto{\pgfqpoint{4.124960in}{4.147706in}}%
\pgfpathlineto{\pgfqpoint{4.123290in}{4.149333in}}%
\pgfpathlineto{\pgfqpoint{4.105187in}{4.166622in}}%
\pgfpathlineto{\pgfqpoint{4.086626in}{4.184689in}}%
\pgfpathlineto{\pgfqpoint{4.084570in}{4.186667in}}%
\pgfpathlineto{\pgfqpoint{4.046545in}{4.223212in}}%
\pgfpathlineto{\pgfqpoint{4.046127in}{4.223610in}}%
\pgfpathlineto{\pgfqpoint{4.045724in}{4.224000in}}%
\pgfpathlineto{\pgfqpoint{4.042842in}{4.224000in}}%
\pgfpathlineto{\pgfqpoint{4.044658in}{4.222242in}}%
\pgfpathlineto{\pgfqpoint{4.046545in}{4.220450in}}%
\pgfpathlineto{\pgfqpoint{4.081696in}{4.186667in}}%
\pgfpathlineto{\pgfqpoint{4.086626in}{4.181925in}}%
\pgfpathlineto{\pgfqpoint{4.103736in}{4.165270in}}%
\pgfpathlineto{\pgfqpoint{4.120423in}{4.149333in}}%
\pgfpathlineto{\pgfqpoint{4.123494in}{4.146341in}}%
\pgfpathlineto{\pgfqpoint{4.126707in}{4.143270in}}%
\pgfpathlineto{\pgfqpoint{4.143095in}{4.127265in}}%
\pgfpathlineto{\pgfqpoint{4.159026in}{4.112000in}}%
\pgfpathlineto{\pgfqpoint{4.166788in}{4.104485in}}%
\pgfpathlineto{\pgfqpoint{4.182389in}{4.089198in}}%
\pgfpathlineto{\pgfqpoint{4.197504in}{4.074667in}}%
\pgfpathlineto{\pgfqpoint{4.206869in}{4.065570in}}%
\pgfpathlineto{\pgfqpoint{4.221617in}{4.051071in}}%
\pgfpathlineto{\pgfqpoint{4.235859in}{4.037333in}}%
\pgfpathlineto{\pgfqpoint{4.241252in}{4.032026in}}%
\pgfpathlineto{\pgfqpoint{4.246949in}{4.026524in}}%
\pgfpathlineto{\pgfqpoint{4.274093in}{4.000000in}}%
\pgfpathlineto{\pgfqpoint{4.280372in}{3.993798in}}%
\pgfpathlineto{\pgfqpoint{4.287030in}{3.987348in}}%
\pgfpathlineto{\pgfqpoint{4.299879in}{3.974634in}}%
\pgfpathlineto{\pgfqpoint{4.312205in}{3.962667in}}%
\pgfpathlineto{\pgfqpoint{4.327111in}{3.948042in}}%
\pgfpathlineto{\pgfqpoint{4.338912in}{3.936325in}}%
\pgfpathlineto{\pgfqpoint{4.350196in}{3.925333in}}%
\pgfpathlineto{\pgfqpoint{4.367192in}{3.908603in}}%
\pgfpathlineto{\pgfqpoint{4.377881in}{3.897956in}}%
\pgfpathlineto{\pgfqpoint{4.388069in}{3.888000in}}%
\pgfpathlineto{\pgfqpoint{4.407273in}{3.869034in}}%
\pgfpathlineto{\pgfqpoint{4.416786in}{3.859528in}}%
\pgfpathlineto{\pgfqpoint{4.425822in}{3.850667in}}%
\pgfpathlineto{\pgfqpoint{4.447354in}{3.829333in}}%
\pgfpathlineto{\pgfqpoint{4.463459in}{3.813333in}}%
\pgfpathlineto{\pgfqpoint{4.474998in}{3.801749in}}%
\pgfpathlineto{\pgfqpoint{4.487434in}{3.789499in}}%
\pgfpathlineto{\pgfqpoint{4.494403in}{3.782491in}}%
\pgfpathlineto{\pgfqpoint{4.500979in}{3.776000in}}%
\pgfpathlineto{\pgfqpoint{4.513728in}{3.763158in}}%
\pgfpathlineto{\pgfqpoint{4.527515in}{3.749534in}}%
\pgfpathlineto{\pgfqpoint{4.538383in}{3.738667in}}%
\pgfpathlineto{\pgfqpoint{4.567596in}{3.709436in}}%
\pgfpathlineto{\pgfqpoint{4.571764in}{3.705216in}}%
\pgfpathlineto{\pgfqpoint{4.575672in}{3.701333in}}%
\pgfpathlineto{\pgfqpoint{4.607677in}{3.669205in}}%
\pgfpathlineto{\pgfqpoint{4.610350in}{3.666490in}}%
\pgfpathlineto{\pgfqpoint{4.612848in}{3.664000in}}%
\pgfpathlineto{\pgfqpoint{4.629535in}{3.647026in}}%
\pgfpathlineto{\pgfqpoint{4.647758in}{3.628841in}}%
\pgfpathlineto{\pgfqpoint{4.648872in}{3.627705in}}%
\pgfpathlineto{\pgfqpoint{4.649910in}{3.626667in}}%
\pgfpathlineto{\pgfqpoint{4.675300in}{3.601012in}}%
\pgfpathlineto{\pgfqpoint{4.686849in}{3.589333in}}%
\pgfpathlineto{\pgfqpoint{4.687838in}{3.588332in}}%
\pgfpathlineto{\pgfqpoint{4.706420in}{3.569308in}}%
\pgfpathlineto{\pgfqpoint{4.723650in}{3.552000in}}%
\pgfpathlineto{\pgfqpoint{4.727919in}{3.547666in}}%
\pgfpathlineto{\pgfqpoint{4.744768in}{3.530360in}}%
\pgfpathlineto{\pgfqpoint{4.760340in}{3.514667in}}%
\pgfpathlineto{\pgfqpoint{4.768000in}{3.506864in}}%
\pgfusepath{fill}%
\end{pgfscope}%
\begin{pgfscope}%
\pgfpathrectangle{\pgfqpoint{0.800000in}{0.528000in}}{\pgfqpoint{3.968000in}{3.696000in}}%
\pgfusepath{clip}%
\pgfsetbuttcap%
\pgfsetroundjoin%
\definecolor{currentfill}{rgb}{0.274149,0.751988,0.436601}%
\pgfsetfillcolor{currentfill}%
\pgfsetlinewidth{0.000000pt}%
\definecolor{currentstroke}{rgb}{0.000000,0.000000,0.000000}%
\pgfsetstrokecolor{currentstroke}%
\pgfsetdash{}{0pt}%
\pgfpathmoveto{\pgfqpoint{4.768000in}{3.512518in}}%
\pgfpathlineto{\pgfqpoint{4.765890in}{3.514667in}}%
\pgfpathlineto{\pgfqpoint{4.747652in}{3.533047in}}%
\pgfpathlineto{\pgfqpoint{4.729200in}{3.552000in}}%
\pgfpathlineto{\pgfqpoint{4.728584in}{3.552620in}}%
\pgfpathlineto{\pgfqpoint{4.727919in}{3.553302in}}%
\pgfpathlineto{\pgfqpoint{4.709308in}{3.571997in}}%
\pgfpathlineto{\pgfqpoint{4.692375in}{3.589333in}}%
\pgfpathlineto{\pgfqpoint{4.687838in}{3.593929in}}%
\pgfpathlineto{\pgfqpoint{4.655439in}{3.626667in}}%
\pgfpathlineto{\pgfqpoint{4.651734in}{3.630371in}}%
\pgfpathlineto{\pgfqpoint{4.647758in}{3.634422in}}%
\pgfpathlineto{\pgfqpoint{4.632428in}{3.649721in}}%
\pgfpathlineto{\pgfqpoint{4.618390in}{3.664000in}}%
\pgfpathlineto{\pgfqpoint{4.613215in}{3.669158in}}%
\pgfpathlineto{\pgfqpoint{4.607677in}{3.674783in}}%
\pgfpathlineto{\pgfqpoint{4.581229in}{3.701333in}}%
\pgfpathlineto{\pgfqpoint{4.574632in}{3.707887in}}%
\pgfpathlineto{\pgfqpoint{4.567596in}{3.715010in}}%
\pgfpathlineto{\pgfqpoint{4.543954in}{3.738667in}}%
\pgfpathlineto{\pgfqpoint{4.527515in}{3.755104in}}%
\pgfpathlineto{\pgfqpoint{4.516630in}{3.765861in}}%
\pgfpathlineto{\pgfqpoint{4.506564in}{3.776000in}}%
\pgfpathlineto{\pgfqpoint{4.497276in}{3.785167in}}%
\pgfpathlineto{\pgfqpoint{4.487434in}{3.795066in}}%
\pgfpathlineto{\pgfqpoint{4.477902in}{3.804455in}}%
\pgfpathlineto{\pgfqpoint{4.469058in}{3.813333in}}%
\pgfpathlineto{\pgfqpoint{4.447354in}{3.834895in}}%
\pgfpathlineto{\pgfqpoint{4.431436in}{3.850667in}}%
\pgfpathlineto{\pgfqpoint{4.419665in}{3.862209in}}%
\pgfpathlineto{\pgfqpoint{4.407273in}{3.874593in}}%
\pgfpathlineto{\pgfqpoint{4.393697in}{3.888000in}}%
\pgfpathlineto{\pgfqpoint{4.380763in}{3.900641in}}%
\pgfpathlineto{\pgfqpoint{4.367192in}{3.914158in}}%
\pgfpathlineto{\pgfqpoint{4.355839in}{3.925333in}}%
\pgfpathlineto{\pgfqpoint{4.341797in}{3.939012in}}%
\pgfpathlineto{\pgfqpoint{4.327111in}{3.953592in}}%
\pgfpathlineto{\pgfqpoint{4.317862in}{3.962667in}}%
\pgfpathlineto{\pgfqpoint{4.302766in}{3.977324in}}%
\pgfpathlineto{\pgfqpoint{4.287030in}{3.992895in}}%
\pgfpathlineto{\pgfqpoint{4.283292in}{3.996518in}}%
\pgfpathlineto{\pgfqpoint{4.279765in}{4.000000in}}%
\pgfpathlineto{\pgfqpoint{4.246949in}{4.032068in}}%
\pgfpathlineto{\pgfqpoint{4.244174in}{4.034748in}}%
\pgfpathlineto{\pgfqpoint{4.241547in}{4.037333in}}%
\pgfpathlineto{\pgfqpoint{4.224510in}{4.053766in}}%
\pgfpathlineto{\pgfqpoint{4.206869in}{4.071109in}}%
\pgfpathlineto{\pgfqpoint{4.203207in}{4.074667in}}%
\pgfpathlineto{\pgfqpoint{4.185285in}{4.091896in}}%
\pgfpathlineto{\pgfqpoint{4.166788in}{4.110020in}}%
\pgfpathlineto{\pgfqpoint{4.164743in}{4.112000in}}%
\pgfpathlineto{\pgfqpoint{4.145994in}{4.129965in}}%
\pgfpathlineto{\pgfqpoint{4.126707in}{4.148801in}}%
\pgfpathlineto{\pgfqpoint{4.126425in}{4.149071in}}%
\pgfpathlineto{\pgfqpoint{4.126156in}{4.149333in}}%
\pgfpathlineto{\pgfqpoint{4.106638in}{4.167973in}}%
\pgfpathlineto{\pgfqpoint{4.087434in}{4.186667in}}%
\pgfpathlineto{\pgfqpoint{4.087035in}{4.187047in}}%
\pgfpathlineto{\pgfqpoint{4.086626in}{4.187444in}}%
\pgfpathlineto{\pgfqpoint{4.048580in}{4.224000in}}%
\pgfpathlineto{\pgfqpoint{4.046545in}{4.224000in}}%
\pgfpathlineto{\pgfqpoint{4.045724in}{4.224000in}}%
\pgfpathlineto{\pgfqpoint{4.046127in}{4.223610in}}%
\pgfpathlineto{\pgfqpoint{4.046545in}{4.223212in}}%
\pgfpathlineto{\pgfqpoint{4.084570in}{4.186667in}}%
\pgfpathlineto{\pgfqpoint{4.086626in}{4.184689in}}%
\pgfpathlineto{\pgfqpoint{4.105187in}{4.166622in}}%
\pgfpathlineto{\pgfqpoint{4.123290in}{4.149333in}}%
\pgfpathlineto{\pgfqpoint{4.124960in}{4.147706in}}%
\pgfpathlineto{\pgfqpoint{4.126707in}{4.146035in}}%
\pgfpathlineto{\pgfqpoint{4.144545in}{4.128615in}}%
\pgfpathlineto{\pgfqpoint{4.161884in}{4.112000in}}%
\pgfpathlineto{\pgfqpoint{4.166788in}{4.107253in}}%
\pgfpathlineto{\pgfqpoint{4.183837in}{4.090547in}}%
\pgfpathlineto{\pgfqpoint{4.200355in}{4.074667in}}%
\pgfpathlineto{\pgfqpoint{4.206869in}{4.068339in}}%
\pgfpathlineto{\pgfqpoint{4.223064in}{4.052418in}}%
\pgfpathlineto{\pgfqpoint{4.238703in}{4.037333in}}%
\pgfpathlineto{\pgfqpoint{4.242713in}{4.033387in}}%
\pgfpathlineto{\pgfqpoint{4.246949in}{4.029296in}}%
\pgfpathlineto{\pgfqpoint{4.276929in}{4.000000in}}%
\pgfpathlineto{\pgfqpoint{4.281832in}{3.995158in}}%
\pgfpathlineto{\pgfqpoint{4.287030in}{3.990122in}}%
\pgfpathlineto{\pgfqpoint{4.301322in}{3.975979in}}%
\pgfpathlineto{\pgfqpoint{4.315033in}{3.962667in}}%
\pgfpathlineto{\pgfqpoint{4.327111in}{3.950817in}}%
\pgfpathlineto{\pgfqpoint{4.340355in}{3.937669in}}%
\pgfpathlineto{\pgfqpoint{4.353018in}{3.925333in}}%
\pgfpathlineto{\pgfqpoint{4.367192in}{3.911381in}}%
\pgfpathlineto{\pgfqpoint{4.379322in}{3.899299in}}%
\pgfpathlineto{\pgfqpoint{4.390883in}{3.888000in}}%
\pgfpathlineto{\pgfqpoint{4.407273in}{3.871813in}}%
\pgfpathlineto{\pgfqpoint{4.418225in}{3.860868in}}%
\pgfpathlineto{\pgfqpoint{4.428629in}{3.850667in}}%
\pgfpathlineto{\pgfqpoint{4.447354in}{3.832114in}}%
\pgfpathlineto{\pgfqpoint{4.466259in}{3.813333in}}%
\pgfpathlineto{\pgfqpoint{4.476450in}{3.803102in}}%
\pgfpathlineto{\pgfqpoint{4.487434in}{3.792282in}}%
\pgfpathlineto{\pgfqpoint{4.495839in}{3.783829in}}%
\pgfpathlineto{\pgfqpoint{4.503771in}{3.776000in}}%
\pgfpathlineto{\pgfqpoint{4.515179in}{3.764509in}}%
\pgfpathlineto{\pgfqpoint{4.527515in}{3.752319in}}%
\pgfpathlineto{\pgfqpoint{4.541168in}{3.738667in}}%
\pgfpathlineto{\pgfqpoint{4.567596in}{3.712223in}}%
\pgfpathlineto{\pgfqpoint{4.573198in}{3.706551in}}%
\pgfpathlineto{\pgfqpoint{4.578451in}{3.701333in}}%
\pgfpathlineto{\pgfqpoint{4.607677in}{3.671994in}}%
\pgfpathlineto{\pgfqpoint{4.611782in}{3.667824in}}%
\pgfpathlineto{\pgfqpoint{4.615619in}{3.664000in}}%
\pgfpathlineto{\pgfqpoint{4.630981in}{3.648374in}}%
\pgfpathlineto{\pgfqpoint{4.647758in}{3.631631in}}%
\pgfpathlineto{\pgfqpoint{4.650303in}{3.629038in}}%
\pgfpathlineto{\pgfqpoint{4.652675in}{3.626667in}}%
\pgfpathlineto{\pgfqpoint{4.687838in}{3.591136in}}%
\pgfpathlineto{\pgfqpoint{4.689618in}{3.589333in}}%
\pgfpathlineto{\pgfqpoint{4.707864in}{3.570653in}}%
\pgfpathlineto{\pgfqpoint{4.726433in}{3.552000in}}%
\pgfpathlineto{\pgfqpoint{4.727919in}{3.550491in}}%
\pgfpathlineto{\pgfqpoint{4.746210in}{3.531704in}}%
\pgfpathlineto{\pgfqpoint{4.763115in}{3.514667in}}%
\pgfpathlineto{\pgfqpoint{4.768000in}{3.509691in}}%
\pgfusepath{fill}%
\end{pgfscope}%
\begin{pgfscope}%
\pgfpathrectangle{\pgfqpoint{0.800000in}{0.528000in}}{\pgfqpoint{3.968000in}{3.696000in}}%
\pgfusepath{clip}%
\pgfsetbuttcap%
\pgfsetroundjoin%
\definecolor{currentfill}{rgb}{0.281477,0.755203,0.432552}%
\pgfsetfillcolor{currentfill}%
\pgfsetlinewidth{0.000000pt}%
\definecolor{currentstroke}{rgb}{0.000000,0.000000,0.000000}%
\pgfsetstrokecolor{currentstroke}%
\pgfsetdash{}{0pt}%
\pgfpathmoveto{\pgfqpoint{4.768000in}{3.515337in}}%
\pgfpathlineto{\pgfqpoint{4.749094in}{3.534390in}}%
\pgfpathlineto{\pgfqpoint{4.731950in}{3.552000in}}%
\pgfpathlineto{\pgfqpoint{4.730013in}{3.553950in}}%
\pgfpathlineto{\pgfqpoint{4.727919in}{3.556096in}}%
\pgfpathlineto{\pgfqpoint{4.710751in}{3.573342in}}%
\pgfpathlineto{\pgfqpoint{4.695132in}{3.589333in}}%
\pgfpathlineto{\pgfqpoint{4.687838in}{3.596722in}}%
\pgfpathlineto{\pgfqpoint{4.658203in}{3.626667in}}%
\pgfpathlineto{\pgfqpoint{4.653165in}{3.631704in}}%
\pgfpathlineto{\pgfqpoint{4.647758in}{3.637213in}}%
\pgfpathlineto{\pgfqpoint{4.633874in}{3.651069in}}%
\pgfpathlineto{\pgfqpoint{4.621161in}{3.664000in}}%
\pgfpathlineto{\pgfqpoint{4.614647in}{3.670492in}}%
\pgfpathlineto{\pgfqpoint{4.607677in}{3.677572in}}%
\pgfpathlineto{\pgfqpoint{4.584007in}{3.701333in}}%
\pgfpathlineto{\pgfqpoint{4.576066in}{3.709222in}}%
\pgfpathlineto{\pgfqpoint{4.567596in}{3.717797in}}%
\pgfpathlineto{\pgfqpoint{4.546739in}{3.738667in}}%
\pgfpathlineto{\pgfqpoint{4.527515in}{3.757889in}}%
\pgfpathlineto{\pgfqpoint{4.518081in}{3.767212in}}%
\pgfpathlineto{\pgfqpoint{4.509356in}{3.776000in}}%
\pgfpathlineto{\pgfqpoint{4.498713in}{3.786505in}}%
\pgfpathlineto{\pgfqpoint{4.487434in}{3.797849in}}%
\pgfpathlineto{\pgfqpoint{4.479354in}{3.805807in}}%
\pgfpathlineto{\pgfqpoint{4.471858in}{3.813333in}}%
\pgfpathlineto{\pgfqpoint{4.447354in}{3.837676in}}%
\pgfpathlineto{\pgfqpoint{4.434243in}{3.850667in}}%
\pgfpathlineto{\pgfqpoint{4.421104in}{3.863550in}}%
\pgfpathlineto{\pgfqpoint{4.407273in}{3.877372in}}%
\pgfpathlineto{\pgfqpoint{4.396511in}{3.888000in}}%
\pgfpathlineto{\pgfqpoint{4.382204in}{3.901983in}}%
\pgfpathlineto{\pgfqpoint{4.367192in}{3.916936in}}%
\pgfpathlineto{\pgfqpoint{4.358661in}{3.925333in}}%
\pgfpathlineto{\pgfqpoint{4.343239in}{3.940356in}}%
\pgfpathlineto{\pgfqpoint{4.327111in}{3.956368in}}%
\pgfpathlineto{\pgfqpoint{4.320691in}{3.962667in}}%
\pgfpathlineto{\pgfqpoint{4.304210in}{3.978669in}}%
\pgfpathlineto{\pgfqpoint{4.287030in}{3.995669in}}%
\pgfpathlineto{\pgfqpoint{4.284751in}{3.997877in}}%
\pgfpathlineto{\pgfqpoint{4.282601in}{4.000000in}}%
\pgfpathlineto{\pgfqpoint{4.246949in}{4.034839in}}%
\pgfpathlineto{\pgfqpoint{4.245635in}{4.036109in}}%
\pgfpathlineto{\pgfqpoint{4.244391in}{4.037333in}}%
\pgfpathlineto{\pgfqpoint{4.225957in}{4.055113in}}%
\pgfpathlineto{\pgfqpoint{4.206869in}{4.073879in}}%
\pgfpathlineto{\pgfqpoint{4.206058in}{4.074667in}}%
\pgfpathlineto{\pgfqpoint{4.186733in}{4.093245in}}%
\pgfpathlineto{\pgfqpoint{4.167592in}{4.112000in}}%
\pgfpathlineto{\pgfqpoint{4.167196in}{4.112380in}}%
\pgfpathlineto{\pgfqpoint{4.166788in}{4.112780in}}%
\pgfpathlineto{\pgfqpoint{4.147444in}{4.131315in}}%
\pgfpathlineto{\pgfqpoint{4.128994in}{4.149333in}}%
\pgfpathlineto{\pgfqpoint{4.127866in}{4.150413in}}%
\pgfpathlineto{\pgfqpoint{4.126707in}{4.151544in}}%
\pgfpathlineto{\pgfqpoint{4.108089in}{4.169325in}}%
\pgfpathlineto{\pgfqpoint{4.090273in}{4.186667in}}%
\pgfpathlineto{\pgfqpoint{4.088472in}{4.188386in}}%
\pgfpathlineto{\pgfqpoint{4.086626in}{4.190180in}}%
\pgfpathlineto{\pgfqpoint{4.051427in}{4.224000in}}%
\pgfpathlineto{\pgfqpoint{4.048580in}{4.224000in}}%
\pgfpathlineto{\pgfqpoint{4.086626in}{4.187444in}}%
\pgfpathlineto{\pgfqpoint{4.087035in}{4.187047in}}%
\pgfpathlineto{\pgfqpoint{4.087434in}{4.186667in}}%
\pgfpathlineto{\pgfqpoint{4.106638in}{4.167973in}}%
\pgfpathlineto{\pgfqpoint{4.126156in}{4.149333in}}%
\pgfpathlineto{\pgfqpoint{4.126425in}{4.149071in}}%
\pgfpathlineto{\pgfqpoint{4.126707in}{4.148801in}}%
\pgfpathlineto{\pgfqpoint{4.145994in}{4.129965in}}%
\pgfpathlineto{\pgfqpoint{4.164743in}{4.112000in}}%
\pgfpathlineto{\pgfqpoint{4.166788in}{4.110020in}}%
\pgfpathlineto{\pgfqpoint{4.185285in}{4.091896in}}%
\pgfpathlineto{\pgfqpoint{4.203207in}{4.074667in}}%
\pgfpathlineto{\pgfqpoint{4.206869in}{4.071109in}}%
\pgfpathlineto{\pgfqpoint{4.224510in}{4.053766in}}%
\pgfpathlineto{\pgfqpoint{4.241547in}{4.037333in}}%
\pgfpathlineto{\pgfqpoint{4.244174in}{4.034748in}}%
\pgfpathlineto{\pgfqpoint{4.246949in}{4.032068in}}%
\pgfpathlineto{\pgfqpoint{4.279765in}{4.000000in}}%
\pgfpathlineto{\pgfqpoint{4.283292in}{3.996518in}}%
\pgfpathlineto{\pgfqpoint{4.287030in}{3.992895in}}%
\pgfpathlineto{\pgfqpoint{4.302766in}{3.977324in}}%
\pgfpathlineto{\pgfqpoint{4.317862in}{3.962667in}}%
\pgfpathlineto{\pgfqpoint{4.327111in}{3.953592in}}%
\pgfpathlineto{\pgfqpoint{4.341797in}{3.939012in}}%
\pgfpathlineto{\pgfqpoint{4.355839in}{3.925333in}}%
\pgfpathlineto{\pgfqpoint{4.367192in}{3.914158in}}%
\pgfpathlineto{\pgfqpoint{4.380763in}{3.900641in}}%
\pgfpathlineto{\pgfqpoint{4.393697in}{3.888000in}}%
\pgfpathlineto{\pgfqpoint{4.407273in}{3.874593in}}%
\pgfpathlineto{\pgfqpoint{4.419665in}{3.862209in}}%
\pgfpathlineto{\pgfqpoint{4.431436in}{3.850667in}}%
\pgfpathlineto{\pgfqpoint{4.447354in}{3.834895in}}%
\pgfpathlineto{\pgfqpoint{4.469058in}{3.813333in}}%
\pgfpathlineto{\pgfqpoint{4.477902in}{3.804455in}}%
\pgfpathlineto{\pgfqpoint{4.487434in}{3.795066in}}%
\pgfpathlineto{\pgfqpoint{4.497276in}{3.785167in}}%
\pgfpathlineto{\pgfqpoint{4.506564in}{3.776000in}}%
\pgfpathlineto{\pgfqpoint{4.516630in}{3.765861in}}%
\pgfpathlineto{\pgfqpoint{4.527515in}{3.755104in}}%
\pgfpathlineto{\pgfqpoint{4.543954in}{3.738667in}}%
\pgfpathlineto{\pgfqpoint{4.567596in}{3.715010in}}%
\pgfpathlineto{\pgfqpoint{4.574632in}{3.707887in}}%
\pgfpathlineto{\pgfqpoint{4.581229in}{3.701333in}}%
\pgfpathlineto{\pgfqpoint{4.607677in}{3.674783in}}%
\pgfpathlineto{\pgfqpoint{4.613215in}{3.669158in}}%
\pgfpathlineto{\pgfqpoint{4.618390in}{3.664000in}}%
\pgfpathlineto{\pgfqpoint{4.632428in}{3.649721in}}%
\pgfpathlineto{\pgfqpoint{4.647758in}{3.634422in}}%
\pgfpathlineto{\pgfqpoint{4.651734in}{3.630371in}}%
\pgfpathlineto{\pgfqpoint{4.655439in}{3.626667in}}%
\pgfpathlineto{\pgfqpoint{4.687838in}{3.593929in}}%
\pgfpathlineto{\pgfqpoint{4.692375in}{3.589333in}}%
\pgfpathlineto{\pgfqpoint{4.709308in}{3.571997in}}%
\pgfpathlineto{\pgfqpoint{4.727919in}{3.553302in}}%
\pgfpathlineto{\pgfqpoint{4.728584in}{3.552620in}}%
\pgfpathlineto{\pgfqpoint{4.729200in}{3.552000in}}%
\pgfpathlineto{\pgfqpoint{4.747652in}{3.533047in}}%
\pgfpathlineto{\pgfqpoint{4.765890in}{3.514667in}}%
\pgfpathlineto{\pgfqpoint{4.768000in}{3.512518in}}%
\pgfpathlineto{\pgfqpoint{4.768000in}{3.514667in}}%
\pgfusepath{fill}%
\end{pgfscope}%
\begin{pgfscope}%
\pgfpathrectangle{\pgfqpoint{0.800000in}{0.528000in}}{\pgfqpoint{3.968000in}{3.696000in}}%
\pgfusepath{clip}%
\pgfsetbuttcap%
\pgfsetroundjoin%
\definecolor{currentfill}{rgb}{0.281477,0.755203,0.432552}%
\pgfsetfillcolor{currentfill}%
\pgfsetlinewidth{0.000000pt}%
\definecolor{currentstroke}{rgb}{0.000000,0.000000,0.000000}%
\pgfsetstrokecolor{currentstroke}%
\pgfsetdash{}{0pt}%
\pgfpathmoveto{\pgfqpoint{4.768000in}{3.518134in}}%
\pgfpathlineto{\pgfqpoint{4.750537in}{3.535734in}}%
\pgfpathlineto{\pgfqpoint{4.734700in}{3.552000in}}%
\pgfpathlineto{\pgfqpoint{4.731441in}{3.555280in}}%
\pgfpathlineto{\pgfqpoint{4.727919in}{3.558891in}}%
\pgfpathlineto{\pgfqpoint{4.712195in}{3.574687in}}%
\pgfpathlineto{\pgfqpoint{4.697889in}{3.589333in}}%
\pgfpathlineto{\pgfqpoint{4.687838in}{3.599515in}}%
\pgfpathlineto{\pgfqpoint{4.660967in}{3.626667in}}%
\pgfpathlineto{\pgfqpoint{4.654596in}{3.633037in}}%
\pgfpathlineto{\pgfqpoint{4.647758in}{3.640004in}}%
\pgfpathlineto{\pgfqpoint{4.635321in}{3.652416in}}%
\pgfpathlineto{\pgfqpoint{4.623933in}{3.664000in}}%
\pgfpathlineto{\pgfqpoint{4.616079in}{3.671827in}}%
\pgfpathlineto{\pgfqpoint{4.607677in}{3.680361in}}%
\pgfpathlineto{\pgfqpoint{4.586785in}{3.701333in}}%
\pgfpathlineto{\pgfqpoint{4.577500in}{3.710558in}}%
\pgfpathlineto{\pgfqpoint{4.567596in}{3.720584in}}%
\pgfpathlineto{\pgfqpoint{4.549524in}{3.738667in}}%
\pgfpathlineto{\pgfqpoint{4.527515in}{3.760674in}}%
\pgfpathlineto{\pgfqpoint{4.519532in}{3.768564in}}%
\pgfpathlineto{\pgfqpoint{4.512149in}{3.776000in}}%
\pgfpathlineto{\pgfqpoint{4.500149in}{3.787843in}}%
\pgfpathlineto{\pgfqpoint{4.487434in}{3.800632in}}%
\pgfpathlineto{\pgfqpoint{4.480807in}{3.807160in}}%
\pgfpathlineto{\pgfqpoint{4.474658in}{3.813333in}}%
\pgfpathlineto{\pgfqpoint{4.447354in}{3.840458in}}%
\pgfpathlineto{\pgfqpoint{4.437050in}{3.850667in}}%
\pgfpathlineto{\pgfqpoint{4.422544in}{3.864891in}}%
\pgfpathlineto{\pgfqpoint{4.407273in}{3.880151in}}%
\pgfpathlineto{\pgfqpoint{4.399325in}{3.888000in}}%
\pgfpathlineto{\pgfqpoint{4.383645in}{3.903325in}}%
\pgfpathlineto{\pgfqpoint{4.367192in}{3.919713in}}%
\pgfpathlineto{\pgfqpoint{4.361482in}{3.925333in}}%
\pgfpathlineto{\pgfqpoint{4.344682in}{3.941699in}}%
\pgfpathlineto{\pgfqpoint{4.327111in}{3.959143in}}%
\pgfpathlineto{\pgfqpoint{4.323520in}{3.962667in}}%
\pgfpathlineto{\pgfqpoint{4.305654in}{3.980014in}}%
\pgfpathlineto{\pgfqpoint{4.287030in}{3.998443in}}%
\pgfpathlineto{\pgfqpoint{4.286211in}{3.999237in}}%
\pgfpathlineto{\pgfqpoint{4.285438in}{4.000000in}}%
\pgfpathlineto{\pgfqpoint{4.253015in}{4.031684in}}%
\pgfpathlineto{\pgfqpoint{4.247231in}{4.037333in}}%
\pgfpathlineto{\pgfqpoint{4.246949in}{4.037608in}}%
\pgfpathlineto{\pgfqpoint{4.227404in}{4.056461in}}%
\pgfpathlineto{\pgfqpoint{4.208885in}{4.074667in}}%
\pgfpathlineto{\pgfqpoint{4.207894in}{4.075621in}}%
\pgfpathlineto{\pgfqpoint{4.206869in}{4.076628in}}%
\pgfpathlineto{\pgfqpoint{4.188181in}{4.094594in}}%
\pgfpathlineto{\pgfqpoint{4.170417in}{4.112000in}}%
\pgfpathlineto{\pgfqpoint{4.168630in}{4.113716in}}%
\pgfpathlineto{\pgfqpoint{4.166788in}{4.115519in}}%
\pgfpathlineto{\pgfqpoint{4.148893in}{4.132665in}}%
\pgfpathlineto{\pgfqpoint{4.131826in}{4.149333in}}%
\pgfpathlineto{\pgfqpoint{4.129302in}{4.151750in}}%
\pgfpathlineto{\pgfqpoint{4.126707in}{4.154281in}}%
\pgfpathlineto{\pgfqpoint{4.109540in}{4.170676in}}%
\pgfpathlineto{\pgfqpoint{4.093113in}{4.186667in}}%
\pgfpathlineto{\pgfqpoint{4.089908in}{4.189724in}}%
\pgfpathlineto{\pgfqpoint{4.086626in}{4.192915in}}%
\pgfpathlineto{\pgfqpoint{4.054274in}{4.224000in}}%
\pgfpathlineto{\pgfqpoint{4.051427in}{4.224000in}}%
\pgfpathlineto{\pgfqpoint{4.086626in}{4.190180in}}%
\pgfpathlineto{\pgfqpoint{4.088472in}{4.188386in}}%
\pgfpathlineto{\pgfqpoint{4.090273in}{4.186667in}}%
\pgfpathlineto{\pgfqpoint{4.108089in}{4.169325in}}%
\pgfpathlineto{\pgfqpoint{4.126707in}{4.151544in}}%
\pgfpathlineto{\pgfqpoint{4.127866in}{4.150413in}}%
\pgfpathlineto{\pgfqpoint{4.128994in}{4.149333in}}%
\pgfpathlineto{\pgfqpoint{4.147444in}{4.131315in}}%
\pgfpathlineto{\pgfqpoint{4.166788in}{4.112780in}}%
\pgfpathlineto{\pgfqpoint{4.167196in}{4.112380in}}%
\pgfpathlineto{\pgfqpoint{4.167592in}{4.112000in}}%
\pgfpathlineto{\pgfqpoint{4.186733in}{4.093245in}}%
\pgfpathlineto{\pgfqpoint{4.206058in}{4.074667in}}%
\pgfpathlineto{\pgfqpoint{4.206869in}{4.073879in}}%
\pgfpathlineto{\pgfqpoint{4.225957in}{4.055113in}}%
\pgfpathlineto{\pgfqpoint{4.244391in}{4.037333in}}%
\pgfpathlineto{\pgfqpoint{4.245635in}{4.036109in}}%
\pgfpathlineto{\pgfqpoint{4.246949in}{4.034839in}}%
\pgfpathlineto{\pgfqpoint{4.282601in}{4.000000in}}%
\pgfpathlineto{\pgfqpoint{4.284751in}{3.997877in}}%
\pgfpathlineto{\pgfqpoint{4.287030in}{3.995669in}}%
\pgfpathlineto{\pgfqpoint{4.304210in}{3.978669in}}%
\pgfpathlineto{\pgfqpoint{4.320691in}{3.962667in}}%
\pgfpathlineto{\pgfqpoint{4.327111in}{3.956368in}}%
\pgfpathlineto{\pgfqpoint{4.343239in}{3.940356in}}%
\pgfpathlineto{\pgfqpoint{4.358661in}{3.925333in}}%
\pgfpathlineto{\pgfqpoint{4.367192in}{3.916936in}}%
\pgfpathlineto{\pgfqpoint{4.382204in}{3.901983in}}%
\pgfpathlineto{\pgfqpoint{4.396511in}{3.888000in}}%
\pgfpathlineto{\pgfqpoint{4.407273in}{3.877372in}}%
\pgfpathlineto{\pgfqpoint{4.421104in}{3.863550in}}%
\pgfpathlineto{\pgfqpoint{4.434243in}{3.850667in}}%
\pgfpathlineto{\pgfqpoint{4.447354in}{3.837676in}}%
\pgfpathlineto{\pgfqpoint{4.471858in}{3.813333in}}%
\pgfpathlineto{\pgfqpoint{4.479354in}{3.805807in}}%
\pgfpathlineto{\pgfqpoint{4.487434in}{3.797849in}}%
\pgfpathlineto{\pgfqpoint{4.498713in}{3.786505in}}%
\pgfpathlineto{\pgfqpoint{4.509356in}{3.776000in}}%
\pgfpathlineto{\pgfqpoint{4.518081in}{3.767212in}}%
\pgfpathlineto{\pgfqpoint{4.527515in}{3.757889in}}%
\pgfpathlineto{\pgfqpoint{4.546739in}{3.738667in}}%
\pgfpathlineto{\pgfqpoint{4.567596in}{3.717797in}}%
\pgfpathlineto{\pgfqpoint{4.576066in}{3.709222in}}%
\pgfpathlineto{\pgfqpoint{4.584007in}{3.701333in}}%
\pgfpathlineto{\pgfqpoint{4.607677in}{3.677572in}}%
\pgfpathlineto{\pgfqpoint{4.614647in}{3.670492in}}%
\pgfpathlineto{\pgfqpoint{4.621161in}{3.664000in}}%
\pgfpathlineto{\pgfqpoint{4.633874in}{3.651069in}}%
\pgfpathlineto{\pgfqpoint{4.647758in}{3.637213in}}%
\pgfpathlineto{\pgfqpoint{4.653165in}{3.631704in}}%
\pgfpathlineto{\pgfqpoint{4.658203in}{3.626667in}}%
\pgfpathlineto{\pgfqpoint{4.687838in}{3.596722in}}%
\pgfpathlineto{\pgfqpoint{4.695132in}{3.589333in}}%
\pgfpathlineto{\pgfqpoint{4.710751in}{3.573342in}}%
\pgfpathlineto{\pgfqpoint{4.727919in}{3.556096in}}%
\pgfpathlineto{\pgfqpoint{4.730013in}{3.553950in}}%
\pgfpathlineto{\pgfqpoint{4.731950in}{3.552000in}}%
\pgfpathlineto{\pgfqpoint{4.749094in}{3.534390in}}%
\pgfpathlineto{\pgfqpoint{4.768000in}{3.515337in}}%
\pgfusepath{fill}%
\end{pgfscope}%
\begin{pgfscope}%
\pgfpathrectangle{\pgfqpoint{0.800000in}{0.528000in}}{\pgfqpoint{3.968000in}{3.696000in}}%
\pgfusepath{clip}%
\pgfsetbuttcap%
\pgfsetroundjoin%
\definecolor{currentfill}{rgb}{0.281477,0.755203,0.432552}%
\pgfsetfillcolor{currentfill}%
\pgfsetlinewidth{0.000000pt}%
\definecolor{currentstroke}{rgb}{0.000000,0.000000,0.000000}%
\pgfsetstrokecolor{currentstroke}%
\pgfsetdash{}{0pt}%
\pgfpathmoveto{\pgfqpoint{4.768000in}{3.520931in}}%
\pgfpathlineto{\pgfqpoint{4.751979in}{3.537077in}}%
\pgfpathlineto{\pgfqpoint{4.737450in}{3.552000in}}%
\pgfpathlineto{\pgfqpoint{4.732869in}{3.556610in}}%
\pgfpathlineto{\pgfqpoint{4.727919in}{3.561686in}}%
\pgfpathlineto{\pgfqpoint{4.713638in}{3.576032in}}%
\pgfpathlineto{\pgfqpoint{4.700646in}{3.589333in}}%
\pgfpathlineto{\pgfqpoint{4.687838in}{3.602308in}}%
\pgfpathlineto{\pgfqpoint{4.663731in}{3.626667in}}%
\pgfpathlineto{\pgfqpoint{4.656027in}{3.634369in}}%
\pgfpathlineto{\pgfqpoint{4.647758in}{3.642795in}}%
\pgfpathlineto{\pgfqpoint{4.636768in}{3.653763in}}%
\pgfpathlineto{\pgfqpoint{4.626704in}{3.664000in}}%
\pgfpathlineto{\pgfqpoint{4.617512in}{3.673161in}}%
\pgfpathlineto{\pgfqpoint{4.607677in}{3.683150in}}%
\pgfpathlineto{\pgfqpoint{4.589564in}{3.701333in}}%
\pgfpathlineto{\pgfqpoint{4.578933in}{3.711894in}}%
\pgfpathlineto{\pgfqpoint{4.567596in}{3.723371in}}%
\pgfpathlineto{\pgfqpoint{4.552310in}{3.738667in}}%
\pgfpathlineto{\pgfqpoint{4.527515in}{3.763459in}}%
\pgfpathlineto{\pgfqpoint{4.520982in}{3.769915in}}%
\pgfpathlineto{\pgfqpoint{4.514941in}{3.776000in}}%
\pgfpathlineto{\pgfqpoint{4.501586in}{3.789181in}}%
\pgfpathlineto{\pgfqpoint{4.487434in}{3.803415in}}%
\pgfpathlineto{\pgfqpoint{4.482259in}{3.808513in}}%
\pgfpathlineto{\pgfqpoint{4.477457in}{3.813333in}}%
\pgfpathlineto{\pgfqpoint{4.447354in}{3.843239in}}%
\pgfpathlineto{\pgfqpoint{4.439857in}{3.850667in}}%
\pgfpathlineto{\pgfqpoint{4.423983in}{3.866232in}}%
\pgfpathlineto{\pgfqpoint{4.407273in}{3.882930in}}%
\pgfpathlineto{\pgfqpoint{4.402140in}{3.888000in}}%
\pgfpathlineto{\pgfqpoint{4.385086in}{3.904667in}}%
\pgfpathlineto{\pgfqpoint{4.367192in}{3.922490in}}%
\pgfpathlineto{\pgfqpoint{4.364304in}{3.925333in}}%
\pgfpathlineto{\pgfqpoint{4.346124in}{3.943043in}}%
\pgfpathlineto{\pgfqpoint{4.327111in}{3.961919in}}%
\pgfpathlineto{\pgfqpoint{4.326349in}{3.962667in}}%
\pgfpathlineto{\pgfqpoint{4.307098in}{3.981358in}}%
\pgfpathlineto{\pgfqpoint{4.288259in}{4.000000in}}%
\pgfpathlineto{\pgfqpoint{4.287030in}{4.001203in}}%
\pgfpathlineto{\pgfqpoint{4.250041in}{4.037333in}}%
\pgfpathlineto{\pgfqpoint{4.246949in}{4.040351in}}%
\pgfpathlineto{\pgfqpoint{4.228850in}{4.057808in}}%
\pgfpathlineto{\pgfqpoint{4.211702in}{4.074667in}}%
\pgfpathlineto{\pgfqpoint{4.209326in}{4.076956in}}%
\pgfpathlineto{\pgfqpoint{4.206869in}{4.079369in}}%
\pgfpathlineto{\pgfqpoint{4.189629in}{4.095942in}}%
\pgfpathlineto{\pgfqpoint{4.173242in}{4.112000in}}%
\pgfpathlineto{\pgfqpoint{4.170064in}{4.115052in}}%
\pgfpathlineto{\pgfqpoint{4.166788in}{4.118258in}}%
\pgfpathlineto{\pgfqpoint{4.150343in}{4.134016in}}%
\pgfpathlineto{\pgfqpoint{4.134658in}{4.149333in}}%
\pgfpathlineto{\pgfqpoint{4.130737in}{4.153087in}}%
\pgfpathlineto{\pgfqpoint{4.126707in}{4.157019in}}%
\pgfpathlineto{\pgfqpoint{4.110991in}{4.172028in}}%
\pgfpathlineto{\pgfqpoint{4.095952in}{4.186667in}}%
\pgfpathlineto{\pgfqpoint{4.091345in}{4.191062in}}%
\pgfpathlineto{\pgfqpoint{4.086626in}{4.195651in}}%
\pgfpathlineto{\pgfqpoint{4.057121in}{4.224000in}}%
\pgfpathlineto{\pgfqpoint{4.054274in}{4.224000in}}%
\pgfpathlineto{\pgfqpoint{4.086626in}{4.192915in}}%
\pgfpathlineto{\pgfqpoint{4.089908in}{4.189724in}}%
\pgfpathlineto{\pgfqpoint{4.093113in}{4.186667in}}%
\pgfpathlineto{\pgfqpoint{4.109540in}{4.170676in}}%
\pgfpathlineto{\pgfqpoint{4.126707in}{4.154281in}}%
\pgfpathlineto{\pgfqpoint{4.129302in}{4.151750in}}%
\pgfpathlineto{\pgfqpoint{4.131826in}{4.149333in}}%
\pgfpathlineto{\pgfqpoint{4.148893in}{4.132665in}}%
\pgfpathlineto{\pgfqpoint{4.166788in}{4.115519in}}%
\pgfpathlineto{\pgfqpoint{4.168630in}{4.113716in}}%
\pgfpathlineto{\pgfqpoint{4.170417in}{4.112000in}}%
\pgfpathlineto{\pgfqpoint{4.188181in}{4.094594in}}%
\pgfpathlineto{\pgfqpoint{4.206869in}{4.076628in}}%
\pgfpathlineto{\pgfqpoint{4.207894in}{4.075621in}}%
\pgfpathlineto{\pgfqpoint{4.208885in}{4.074667in}}%
\pgfpathlineto{\pgfqpoint{4.227404in}{4.056461in}}%
\pgfpathlineto{\pgfqpoint{4.246949in}{4.037608in}}%
\pgfpathlineto{\pgfqpoint{4.247231in}{4.037333in}}%
\pgfpathlineto{\pgfqpoint{4.253015in}{4.031684in}}%
\pgfpathlineto{\pgfqpoint{4.285438in}{4.000000in}}%
\pgfpathlineto{\pgfqpoint{4.286211in}{3.999237in}}%
\pgfpathlineto{\pgfqpoint{4.287030in}{3.998443in}}%
\pgfpathlineto{\pgfqpoint{4.305654in}{3.980014in}}%
\pgfpathlineto{\pgfqpoint{4.323520in}{3.962667in}}%
\pgfpathlineto{\pgfqpoint{4.327111in}{3.959143in}}%
\pgfpathlineto{\pgfqpoint{4.344682in}{3.941699in}}%
\pgfpathlineto{\pgfqpoint{4.361482in}{3.925333in}}%
\pgfpathlineto{\pgfqpoint{4.367192in}{3.919713in}}%
\pgfpathlineto{\pgfqpoint{4.383645in}{3.903325in}}%
\pgfpathlineto{\pgfqpoint{4.399325in}{3.888000in}}%
\pgfpathlineto{\pgfqpoint{4.407273in}{3.880151in}}%
\pgfpathlineto{\pgfqpoint{4.422544in}{3.864891in}}%
\pgfpathlineto{\pgfqpoint{4.437050in}{3.850667in}}%
\pgfpathlineto{\pgfqpoint{4.447354in}{3.840458in}}%
\pgfpathlineto{\pgfqpoint{4.474658in}{3.813333in}}%
\pgfpathlineto{\pgfqpoint{4.480807in}{3.807160in}}%
\pgfpathlineto{\pgfqpoint{4.487434in}{3.800632in}}%
\pgfpathlineto{\pgfqpoint{4.500149in}{3.787843in}}%
\pgfpathlineto{\pgfqpoint{4.512149in}{3.776000in}}%
\pgfpathlineto{\pgfqpoint{4.519532in}{3.768564in}}%
\pgfpathlineto{\pgfqpoint{4.527515in}{3.760674in}}%
\pgfpathlineto{\pgfqpoint{4.549524in}{3.738667in}}%
\pgfpathlineto{\pgfqpoint{4.567596in}{3.720584in}}%
\pgfpathlineto{\pgfqpoint{4.577500in}{3.710558in}}%
\pgfpathlineto{\pgfqpoint{4.586785in}{3.701333in}}%
\pgfpathlineto{\pgfqpoint{4.607677in}{3.680361in}}%
\pgfpathlineto{\pgfqpoint{4.616079in}{3.671827in}}%
\pgfpathlineto{\pgfqpoint{4.623933in}{3.664000in}}%
\pgfpathlineto{\pgfqpoint{4.635321in}{3.652416in}}%
\pgfpathlineto{\pgfqpoint{4.647758in}{3.640004in}}%
\pgfpathlineto{\pgfqpoint{4.654596in}{3.633037in}}%
\pgfpathlineto{\pgfqpoint{4.660967in}{3.626667in}}%
\pgfpathlineto{\pgfqpoint{4.687838in}{3.599515in}}%
\pgfpathlineto{\pgfqpoint{4.697889in}{3.589333in}}%
\pgfpathlineto{\pgfqpoint{4.712195in}{3.574687in}}%
\pgfpathlineto{\pgfqpoint{4.727919in}{3.558891in}}%
\pgfpathlineto{\pgfqpoint{4.731441in}{3.555280in}}%
\pgfpathlineto{\pgfqpoint{4.734700in}{3.552000in}}%
\pgfpathlineto{\pgfqpoint{4.750537in}{3.535734in}}%
\pgfpathlineto{\pgfqpoint{4.768000in}{3.518134in}}%
\pgfusepath{fill}%
\end{pgfscope}%
\begin{pgfscope}%
\pgfpathrectangle{\pgfqpoint{0.800000in}{0.528000in}}{\pgfqpoint{3.968000in}{3.696000in}}%
\pgfusepath{clip}%
\pgfsetbuttcap%
\pgfsetroundjoin%
\definecolor{currentfill}{rgb}{0.281477,0.755203,0.432552}%
\pgfsetfillcolor{currentfill}%
\pgfsetlinewidth{0.000000pt}%
\definecolor{currentstroke}{rgb}{0.000000,0.000000,0.000000}%
\pgfsetstrokecolor{currentstroke}%
\pgfsetdash{}{0pt}%
\pgfpathmoveto{\pgfqpoint{4.768000in}{3.523728in}}%
\pgfpathlineto{\pgfqpoint{4.753421in}{3.538420in}}%
\pgfpathlineto{\pgfqpoint{4.740200in}{3.552000in}}%
\pgfpathlineto{\pgfqpoint{4.734297in}{3.557941in}}%
\pgfpathlineto{\pgfqpoint{4.727919in}{3.564481in}}%
\pgfpathlineto{\pgfqpoint{4.715082in}{3.577376in}}%
\pgfpathlineto{\pgfqpoint{4.703403in}{3.589333in}}%
\pgfpathlineto{\pgfqpoint{4.687838in}{3.605101in}}%
\pgfpathlineto{\pgfqpoint{4.666495in}{3.626667in}}%
\pgfpathlineto{\pgfqpoint{4.657458in}{3.635702in}}%
\pgfpathlineto{\pgfqpoint{4.647758in}{3.645586in}}%
\pgfpathlineto{\pgfqpoint{4.638214in}{3.655111in}}%
\pgfpathlineto{\pgfqpoint{4.629475in}{3.664000in}}%
\pgfpathlineto{\pgfqpoint{4.618944in}{3.674495in}}%
\pgfpathlineto{\pgfqpoint{4.607677in}{3.685939in}}%
\pgfpathlineto{\pgfqpoint{4.592342in}{3.701333in}}%
\pgfpathlineto{\pgfqpoint{4.580367in}{3.713229in}}%
\pgfpathlineto{\pgfqpoint{4.567596in}{3.726158in}}%
\pgfpathlineto{\pgfqpoint{4.555095in}{3.738667in}}%
\pgfpathlineto{\pgfqpoint{4.527515in}{3.766245in}}%
\pgfpathlineto{\pgfqpoint{4.522433in}{3.771266in}}%
\pgfpathlineto{\pgfqpoint{4.517734in}{3.776000in}}%
\pgfpathlineto{\pgfqpoint{4.503023in}{3.790520in}}%
\pgfpathlineto{\pgfqpoint{4.487434in}{3.806198in}}%
\pgfpathlineto{\pgfqpoint{4.483711in}{3.809865in}}%
\pgfpathlineto{\pgfqpoint{4.480257in}{3.813333in}}%
\pgfpathlineto{\pgfqpoint{4.447354in}{3.846020in}}%
\pgfpathlineto{\pgfqpoint{4.442664in}{3.850667in}}%
\pgfpathlineto{\pgfqpoint{4.425423in}{3.867573in}}%
\pgfpathlineto{\pgfqpoint{4.407273in}{3.885710in}}%
\pgfpathlineto{\pgfqpoint{4.404954in}{3.888000in}}%
\pgfpathlineto{\pgfqpoint{4.386527in}{3.906009in}}%
\pgfpathlineto{\pgfqpoint{4.367192in}{3.925268in}}%
\pgfpathlineto{\pgfqpoint{4.367125in}{3.925333in}}%
\pgfpathlineto{\pgfqpoint{4.347566in}{3.944386in}}%
\pgfpathlineto{\pgfqpoint{4.329153in}{3.962667in}}%
\pgfpathlineto{\pgfqpoint{4.327111in}{3.964673in}}%
\pgfpathlineto{\pgfqpoint{4.308541in}{3.982703in}}%
\pgfpathlineto{\pgfqpoint{4.291062in}{4.000000in}}%
\pgfpathlineto{\pgfqpoint{4.287030in}{4.003948in}}%
\pgfpathlineto{\pgfqpoint{4.252851in}{4.037333in}}%
\pgfpathlineto{\pgfqpoint{4.246949in}{4.043094in}}%
\pgfpathlineto{\pgfqpoint{4.230297in}{4.059156in}}%
\pgfpathlineto{\pgfqpoint{4.214519in}{4.074667in}}%
\pgfpathlineto{\pgfqpoint{4.210759in}{4.078290in}}%
\pgfpathlineto{\pgfqpoint{4.206869in}{4.082110in}}%
\pgfpathlineto{\pgfqpoint{4.191077in}{4.097291in}}%
\pgfpathlineto{\pgfqpoint{4.176066in}{4.112000in}}%
\pgfpathlineto{\pgfqpoint{4.171498in}{4.116387in}}%
\pgfpathlineto{\pgfqpoint{4.166788in}{4.120997in}}%
\pgfpathlineto{\pgfqpoint{4.151792in}{4.135366in}}%
\pgfpathlineto{\pgfqpoint{4.137491in}{4.149333in}}%
\pgfpathlineto{\pgfqpoint{4.132173in}{4.154424in}}%
\pgfpathlineto{\pgfqpoint{4.126707in}{4.159756in}}%
\pgfpathlineto{\pgfqpoint{4.112442in}{4.173379in}}%
\pgfpathlineto{\pgfqpoint{4.098791in}{4.186667in}}%
\pgfpathlineto{\pgfqpoint{4.092782in}{4.192400in}}%
\pgfpathlineto{\pgfqpoint{4.086626in}{4.198386in}}%
\pgfpathlineto{\pgfqpoint{4.059968in}{4.224000in}}%
\pgfpathlineto{\pgfqpoint{4.057121in}{4.224000in}}%
\pgfpathlineto{\pgfqpoint{4.086626in}{4.195651in}}%
\pgfpathlineto{\pgfqpoint{4.091345in}{4.191062in}}%
\pgfpathlineto{\pgfqpoint{4.095952in}{4.186667in}}%
\pgfpathlineto{\pgfqpoint{4.110991in}{4.172028in}}%
\pgfpathlineto{\pgfqpoint{4.126707in}{4.157019in}}%
\pgfpathlineto{\pgfqpoint{4.130737in}{4.153087in}}%
\pgfpathlineto{\pgfqpoint{4.134658in}{4.149333in}}%
\pgfpathlineto{\pgfqpoint{4.150343in}{4.134016in}}%
\pgfpathlineto{\pgfqpoint{4.166788in}{4.118258in}}%
\pgfpathlineto{\pgfqpoint{4.170064in}{4.115052in}}%
\pgfpathlineto{\pgfqpoint{4.173242in}{4.112000in}}%
\pgfpathlineto{\pgfqpoint{4.189629in}{4.095942in}}%
\pgfpathlineto{\pgfqpoint{4.206869in}{4.079369in}}%
\pgfpathlineto{\pgfqpoint{4.209326in}{4.076956in}}%
\pgfpathlineto{\pgfqpoint{4.211702in}{4.074667in}}%
\pgfpathlineto{\pgfqpoint{4.228850in}{4.057808in}}%
\pgfpathlineto{\pgfqpoint{4.246949in}{4.040351in}}%
\pgfpathlineto{\pgfqpoint{4.250041in}{4.037333in}}%
\pgfpathlineto{\pgfqpoint{4.287030in}{4.001203in}}%
\pgfpathlineto{\pgfqpoint{4.288259in}{4.000000in}}%
\pgfpathlineto{\pgfqpoint{4.307098in}{3.981358in}}%
\pgfpathlineto{\pgfqpoint{4.326349in}{3.962667in}}%
\pgfpathlineto{\pgfqpoint{4.327111in}{3.961919in}}%
\pgfpathlineto{\pgfqpoint{4.346124in}{3.943043in}}%
\pgfpathlineto{\pgfqpoint{4.364304in}{3.925333in}}%
\pgfpathlineto{\pgfqpoint{4.367192in}{3.922490in}}%
\pgfpathlineto{\pgfqpoint{4.385086in}{3.904667in}}%
\pgfpathlineto{\pgfqpoint{4.402140in}{3.888000in}}%
\pgfpathlineto{\pgfqpoint{4.407273in}{3.882930in}}%
\pgfpathlineto{\pgfqpoint{4.423983in}{3.866232in}}%
\pgfpathlineto{\pgfqpoint{4.439857in}{3.850667in}}%
\pgfpathlineto{\pgfqpoint{4.447354in}{3.843239in}}%
\pgfpathlineto{\pgfqpoint{4.477457in}{3.813333in}}%
\pgfpathlineto{\pgfqpoint{4.482259in}{3.808513in}}%
\pgfpathlineto{\pgfqpoint{4.487434in}{3.803415in}}%
\pgfpathlineto{\pgfqpoint{4.501586in}{3.789181in}}%
\pgfpathlineto{\pgfqpoint{4.514941in}{3.776000in}}%
\pgfpathlineto{\pgfqpoint{4.520982in}{3.769915in}}%
\pgfpathlineto{\pgfqpoint{4.527515in}{3.763459in}}%
\pgfpathlineto{\pgfqpoint{4.552310in}{3.738667in}}%
\pgfpathlineto{\pgfqpoint{4.567596in}{3.723371in}}%
\pgfpathlineto{\pgfqpoint{4.578933in}{3.711894in}}%
\pgfpathlineto{\pgfqpoint{4.589564in}{3.701333in}}%
\pgfpathlineto{\pgfqpoint{4.607677in}{3.683150in}}%
\pgfpathlineto{\pgfqpoint{4.617512in}{3.673161in}}%
\pgfpathlineto{\pgfqpoint{4.626704in}{3.664000in}}%
\pgfpathlineto{\pgfqpoint{4.636768in}{3.653763in}}%
\pgfpathlineto{\pgfqpoint{4.647758in}{3.642795in}}%
\pgfpathlineto{\pgfqpoint{4.656027in}{3.634369in}}%
\pgfpathlineto{\pgfqpoint{4.663731in}{3.626667in}}%
\pgfpathlineto{\pgfqpoint{4.687838in}{3.602308in}}%
\pgfpathlineto{\pgfqpoint{4.700646in}{3.589333in}}%
\pgfpathlineto{\pgfqpoint{4.713638in}{3.576032in}}%
\pgfpathlineto{\pgfqpoint{4.727919in}{3.561686in}}%
\pgfpathlineto{\pgfqpoint{4.732869in}{3.556610in}}%
\pgfpathlineto{\pgfqpoint{4.737450in}{3.552000in}}%
\pgfpathlineto{\pgfqpoint{4.751979in}{3.537077in}}%
\pgfpathlineto{\pgfqpoint{4.768000in}{3.520931in}}%
\pgfusepath{fill}%
\end{pgfscope}%
\begin{pgfscope}%
\pgfpathrectangle{\pgfqpoint{0.800000in}{0.528000in}}{\pgfqpoint{3.968000in}{3.696000in}}%
\pgfusepath{clip}%
\pgfsetbuttcap%
\pgfsetroundjoin%
\definecolor{currentfill}{rgb}{0.288921,0.758394,0.428426}%
\pgfsetfillcolor{currentfill}%
\pgfsetlinewidth{0.000000pt}%
\definecolor{currentstroke}{rgb}{0.000000,0.000000,0.000000}%
\pgfsetstrokecolor{currentstroke}%
\pgfsetdash{}{0pt}%
\pgfpathmoveto{\pgfqpoint{4.768000in}{3.526525in}}%
\pgfpathlineto{\pgfqpoint{4.754863in}{3.539764in}}%
\pgfpathlineto{\pgfqpoint{4.742950in}{3.552000in}}%
\pgfpathlineto{\pgfqpoint{4.735725in}{3.559271in}}%
\pgfpathlineto{\pgfqpoint{4.727919in}{3.567276in}}%
\pgfpathlineto{\pgfqpoint{4.716526in}{3.578721in}}%
\pgfpathlineto{\pgfqpoint{4.706160in}{3.589333in}}%
\pgfpathlineto{\pgfqpoint{4.687838in}{3.607893in}}%
\pgfpathlineto{\pgfqpoint{4.669259in}{3.626667in}}%
\pgfpathlineto{\pgfqpoint{4.658889in}{3.637035in}}%
\pgfpathlineto{\pgfqpoint{4.647758in}{3.648377in}}%
\pgfpathlineto{\pgfqpoint{4.639661in}{3.656458in}}%
\pgfpathlineto{\pgfqpoint{4.632246in}{3.664000in}}%
\pgfpathlineto{\pgfqpoint{4.620377in}{3.675829in}}%
\pgfpathlineto{\pgfqpoint{4.607677in}{3.688728in}}%
\pgfpathlineto{\pgfqpoint{4.595120in}{3.701333in}}%
\pgfpathlineto{\pgfqpoint{4.581801in}{3.714565in}}%
\pgfpathlineto{\pgfqpoint{4.567596in}{3.728945in}}%
\pgfpathlineto{\pgfqpoint{4.557880in}{3.738667in}}%
\pgfpathlineto{\pgfqpoint{4.527515in}{3.769030in}}%
\pgfpathlineto{\pgfqpoint{4.523884in}{3.772618in}}%
\pgfpathlineto{\pgfqpoint{4.520526in}{3.776000in}}%
\pgfpathlineto{\pgfqpoint{4.504459in}{3.791858in}}%
\pgfpathlineto{\pgfqpoint{4.487434in}{3.808982in}}%
\pgfpathlineto{\pgfqpoint{4.485164in}{3.811218in}}%
\pgfpathlineto{\pgfqpoint{4.483057in}{3.813333in}}%
\pgfpathlineto{\pgfqpoint{4.447354in}{3.848801in}}%
\pgfpathlineto{\pgfqpoint{4.445471in}{3.850667in}}%
\pgfpathlineto{\pgfqpoint{4.426862in}{3.868913in}}%
\pgfpathlineto{\pgfqpoint{4.407762in}{3.888000in}}%
\pgfpathlineto{\pgfqpoint{4.407273in}{3.888484in}}%
\pgfpathlineto{\pgfqpoint{4.387968in}{3.907352in}}%
\pgfpathlineto{\pgfqpoint{4.369915in}{3.925333in}}%
\pgfpathlineto{\pgfqpoint{4.368585in}{3.926631in}}%
\pgfpathlineto{\pgfqpoint{4.367192in}{3.928017in}}%
\pgfpathlineto{\pgfqpoint{4.349009in}{3.945730in}}%
\pgfpathlineto{\pgfqpoint{4.331949in}{3.962667in}}%
\pgfpathlineto{\pgfqpoint{4.327111in}{3.967420in}}%
\pgfpathlineto{\pgfqpoint{4.309985in}{3.984048in}}%
\pgfpathlineto{\pgfqpoint{4.293865in}{4.000000in}}%
\pgfpathlineto{\pgfqpoint{4.287030in}{4.006693in}}%
\pgfpathlineto{\pgfqpoint{4.255661in}{4.037333in}}%
\pgfpathlineto{\pgfqpoint{4.246949in}{4.045837in}}%
\pgfpathlineto{\pgfqpoint{4.231744in}{4.060503in}}%
\pgfpathlineto{\pgfqpoint{4.217337in}{4.074667in}}%
\pgfpathlineto{\pgfqpoint{4.212191in}{4.079625in}}%
\pgfpathlineto{\pgfqpoint{4.206869in}{4.084851in}}%
\pgfpathlineto{\pgfqpoint{4.192526in}{4.098640in}}%
\pgfpathlineto{\pgfqpoint{4.178891in}{4.112000in}}%
\pgfpathlineto{\pgfqpoint{4.172932in}{4.117723in}}%
\pgfpathlineto{\pgfqpoint{4.166788in}{4.123736in}}%
\pgfpathlineto{\pgfqpoint{4.153242in}{4.136716in}}%
\pgfpathlineto{\pgfqpoint{4.140323in}{4.149333in}}%
\pgfpathlineto{\pgfqpoint{4.133608in}{4.155761in}}%
\pgfpathlineto{\pgfqpoint{4.126707in}{4.162493in}}%
\pgfpathlineto{\pgfqpoint{4.113893in}{4.174731in}}%
\pgfpathlineto{\pgfqpoint{4.101631in}{4.186667in}}%
\pgfpathlineto{\pgfqpoint{4.094219in}{4.193739in}}%
\pgfpathlineto{\pgfqpoint{4.086626in}{4.201121in}}%
\pgfpathlineto{\pgfqpoint{4.062815in}{4.224000in}}%
\pgfpathlineto{\pgfqpoint{4.059968in}{4.224000in}}%
\pgfpathlineto{\pgfqpoint{4.086626in}{4.198386in}}%
\pgfpathlineto{\pgfqpoint{4.092782in}{4.192400in}}%
\pgfpathlineto{\pgfqpoint{4.098791in}{4.186667in}}%
\pgfpathlineto{\pgfqpoint{4.112442in}{4.173379in}}%
\pgfpathlineto{\pgfqpoint{4.126707in}{4.159756in}}%
\pgfpathlineto{\pgfqpoint{4.132173in}{4.154424in}}%
\pgfpathlineto{\pgfqpoint{4.137491in}{4.149333in}}%
\pgfpathlineto{\pgfqpoint{4.151792in}{4.135366in}}%
\pgfpathlineto{\pgfqpoint{4.166788in}{4.120997in}}%
\pgfpathlineto{\pgfqpoint{4.171498in}{4.116387in}}%
\pgfpathlineto{\pgfqpoint{4.176066in}{4.112000in}}%
\pgfpathlineto{\pgfqpoint{4.191077in}{4.097291in}}%
\pgfpathlineto{\pgfqpoint{4.206869in}{4.082110in}}%
\pgfpathlineto{\pgfqpoint{4.210759in}{4.078290in}}%
\pgfpathlineto{\pgfqpoint{4.214519in}{4.074667in}}%
\pgfpathlineto{\pgfqpoint{4.230297in}{4.059156in}}%
\pgfpathlineto{\pgfqpoint{4.246949in}{4.043094in}}%
\pgfpathlineto{\pgfqpoint{4.252851in}{4.037333in}}%
\pgfpathlineto{\pgfqpoint{4.287030in}{4.003948in}}%
\pgfpathlineto{\pgfqpoint{4.291062in}{4.000000in}}%
\pgfpathlineto{\pgfqpoint{4.308541in}{3.982703in}}%
\pgfpathlineto{\pgfqpoint{4.327111in}{3.964673in}}%
\pgfpathlineto{\pgfqpoint{4.329153in}{3.962667in}}%
\pgfpathlineto{\pgfqpoint{4.347566in}{3.944386in}}%
\pgfpathlineto{\pgfqpoint{4.367125in}{3.925333in}}%
\pgfpathlineto{\pgfqpoint{4.367192in}{3.925268in}}%
\pgfpathlineto{\pgfqpoint{4.386527in}{3.906009in}}%
\pgfpathlineto{\pgfqpoint{4.404954in}{3.888000in}}%
\pgfpathlineto{\pgfqpoint{4.407273in}{3.885710in}}%
\pgfpathlineto{\pgfqpoint{4.425423in}{3.867573in}}%
\pgfpathlineto{\pgfqpoint{4.442664in}{3.850667in}}%
\pgfpathlineto{\pgfqpoint{4.447354in}{3.846020in}}%
\pgfpathlineto{\pgfqpoint{4.480257in}{3.813333in}}%
\pgfpathlineto{\pgfqpoint{4.483711in}{3.809865in}}%
\pgfpathlineto{\pgfqpoint{4.487434in}{3.806198in}}%
\pgfpathlineto{\pgfqpoint{4.503023in}{3.790520in}}%
\pgfpathlineto{\pgfqpoint{4.517734in}{3.776000in}}%
\pgfpathlineto{\pgfqpoint{4.522433in}{3.771266in}}%
\pgfpathlineto{\pgfqpoint{4.527515in}{3.766245in}}%
\pgfpathlineto{\pgfqpoint{4.555095in}{3.738667in}}%
\pgfpathlineto{\pgfqpoint{4.567596in}{3.726158in}}%
\pgfpathlineto{\pgfqpoint{4.580367in}{3.713229in}}%
\pgfpathlineto{\pgfqpoint{4.592342in}{3.701333in}}%
\pgfpathlineto{\pgfqpoint{4.607677in}{3.685939in}}%
\pgfpathlineto{\pgfqpoint{4.618944in}{3.674495in}}%
\pgfpathlineto{\pgfqpoint{4.629475in}{3.664000in}}%
\pgfpathlineto{\pgfqpoint{4.638214in}{3.655111in}}%
\pgfpathlineto{\pgfqpoint{4.647758in}{3.645586in}}%
\pgfpathlineto{\pgfqpoint{4.657458in}{3.635702in}}%
\pgfpathlineto{\pgfqpoint{4.666495in}{3.626667in}}%
\pgfpathlineto{\pgfqpoint{4.687838in}{3.605101in}}%
\pgfpathlineto{\pgfqpoint{4.703403in}{3.589333in}}%
\pgfpathlineto{\pgfqpoint{4.715082in}{3.577376in}}%
\pgfpathlineto{\pgfqpoint{4.727919in}{3.564481in}}%
\pgfpathlineto{\pgfqpoint{4.734297in}{3.557941in}}%
\pgfpathlineto{\pgfqpoint{4.740200in}{3.552000in}}%
\pgfpathlineto{\pgfqpoint{4.753421in}{3.538420in}}%
\pgfpathlineto{\pgfqpoint{4.768000in}{3.523728in}}%
\pgfusepath{fill}%
\end{pgfscope}%
\begin{pgfscope}%
\pgfpathrectangle{\pgfqpoint{0.800000in}{0.528000in}}{\pgfqpoint{3.968000in}{3.696000in}}%
\pgfusepath{clip}%
\pgfsetbuttcap%
\pgfsetroundjoin%
\definecolor{currentfill}{rgb}{0.288921,0.758394,0.428426}%
\pgfsetfillcolor{currentfill}%
\pgfsetlinewidth{0.000000pt}%
\definecolor{currentstroke}{rgb}{0.000000,0.000000,0.000000}%
\pgfsetstrokecolor{currentstroke}%
\pgfsetdash{}{0pt}%
\pgfpathmoveto{\pgfqpoint{4.768000in}{3.529321in}}%
\pgfpathlineto{\pgfqpoint{4.756305in}{3.541107in}}%
\pgfpathlineto{\pgfqpoint{4.745700in}{3.552000in}}%
\pgfpathlineto{\pgfqpoint{4.737153in}{3.560601in}}%
\pgfpathlineto{\pgfqpoint{4.727919in}{3.570071in}}%
\pgfpathlineto{\pgfqpoint{4.717969in}{3.580066in}}%
\pgfpathlineto{\pgfqpoint{4.708917in}{3.589333in}}%
\pgfpathlineto{\pgfqpoint{4.687838in}{3.610686in}}%
\pgfpathlineto{\pgfqpoint{4.672023in}{3.626667in}}%
\pgfpathlineto{\pgfqpoint{4.660320in}{3.638368in}}%
\pgfpathlineto{\pgfqpoint{4.647758in}{3.651168in}}%
\pgfpathlineto{\pgfqpoint{4.641107in}{3.657805in}}%
\pgfpathlineto{\pgfqpoint{4.635017in}{3.664000in}}%
\pgfpathlineto{\pgfqpoint{4.621809in}{3.677164in}}%
\pgfpathlineto{\pgfqpoint{4.607677in}{3.691517in}}%
\pgfpathlineto{\pgfqpoint{4.597898in}{3.701333in}}%
\pgfpathlineto{\pgfqpoint{4.583235in}{3.715900in}}%
\pgfpathlineto{\pgfqpoint{4.567596in}{3.731732in}}%
\pgfpathlineto{\pgfqpoint{4.560666in}{3.738667in}}%
\pgfpathlineto{\pgfqpoint{4.527515in}{3.771815in}}%
\pgfpathlineto{\pgfqpoint{4.525335in}{3.773969in}}%
\pgfpathlineto{\pgfqpoint{4.523319in}{3.776000in}}%
\pgfpathlineto{\pgfqpoint{4.505896in}{3.793196in}}%
\pgfpathlineto{\pgfqpoint{4.487434in}{3.811765in}}%
\pgfpathlineto{\pgfqpoint{4.486616in}{3.812571in}}%
\pgfpathlineto{\pgfqpoint{4.485857in}{3.813333in}}%
\pgfpathlineto{\pgfqpoint{4.462137in}{3.836897in}}%
\pgfpathlineto{\pgfqpoint{4.448267in}{3.850667in}}%
\pgfpathlineto{\pgfqpoint{4.447354in}{3.851573in}}%
\pgfpathlineto{\pgfqpoint{4.428302in}{3.870254in}}%
\pgfpathlineto{\pgfqpoint{4.410543in}{3.888000in}}%
\pgfpathlineto{\pgfqpoint{4.407273in}{3.891234in}}%
\pgfpathlineto{\pgfqpoint{4.389409in}{3.908694in}}%
\pgfpathlineto{\pgfqpoint{4.372703in}{3.925333in}}%
\pgfpathlineto{\pgfqpoint{4.370012in}{3.927960in}}%
\pgfpathlineto{\pgfqpoint{4.367192in}{3.930766in}}%
\pgfpathlineto{\pgfqpoint{4.350451in}{3.947073in}}%
\pgfpathlineto{\pgfqpoint{4.334745in}{3.962667in}}%
\pgfpathlineto{\pgfqpoint{4.327111in}{3.970167in}}%
\pgfpathlineto{\pgfqpoint{4.311429in}{3.985393in}}%
\pgfpathlineto{\pgfqpoint{4.296668in}{4.000000in}}%
\pgfpathlineto{\pgfqpoint{4.287030in}{4.009438in}}%
\pgfpathlineto{\pgfqpoint{4.258471in}{4.037333in}}%
\pgfpathlineto{\pgfqpoint{4.246949in}{4.048580in}}%
\pgfpathlineto{\pgfqpoint{4.233190in}{4.061851in}}%
\pgfpathlineto{\pgfqpoint{4.220154in}{4.074667in}}%
\pgfpathlineto{\pgfqpoint{4.213624in}{4.080959in}}%
\pgfpathlineto{\pgfqpoint{4.206869in}{4.087592in}}%
\pgfpathlineto{\pgfqpoint{4.193974in}{4.099989in}}%
\pgfpathlineto{\pgfqpoint{4.181716in}{4.112000in}}%
\pgfpathlineto{\pgfqpoint{4.174366in}{4.119059in}}%
\pgfpathlineto{\pgfqpoint{4.166788in}{4.126476in}}%
\pgfpathlineto{\pgfqpoint{4.154691in}{4.138066in}}%
\pgfpathlineto{\pgfqpoint{4.143155in}{4.149333in}}%
\pgfpathlineto{\pgfqpoint{4.135043in}{4.157098in}}%
\pgfpathlineto{\pgfqpoint{4.126707in}{4.165230in}}%
\pgfpathlineto{\pgfqpoint{4.115344in}{4.176082in}}%
\pgfpathlineto{\pgfqpoint{4.104470in}{4.186667in}}%
\pgfpathlineto{\pgfqpoint{4.095656in}{4.195077in}}%
\pgfpathlineto{\pgfqpoint{4.086626in}{4.203857in}}%
\pgfpathlineto{\pgfqpoint{4.065662in}{4.224000in}}%
\pgfpathlineto{\pgfqpoint{4.062815in}{4.224000in}}%
\pgfpathlineto{\pgfqpoint{4.086626in}{4.201121in}}%
\pgfpathlineto{\pgfqpoint{4.094219in}{4.193739in}}%
\pgfpathlineto{\pgfqpoint{4.101631in}{4.186667in}}%
\pgfpathlineto{\pgfqpoint{4.113893in}{4.174731in}}%
\pgfpathlineto{\pgfqpoint{4.126707in}{4.162493in}}%
\pgfpathlineto{\pgfqpoint{4.133608in}{4.155761in}}%
\pgfpathlineto{\pgfqpoint{4.140323in}{4.149333in}}%
\pgfpathlineto{\pgfqpoint{4.153242in}{4.136716in}}%
\pgfpathlineto{\pgfqpoint{4.166788in}{4.123736in}}%
\pgfpathlineto{\pgfqpoint{4.172932in}{4.117723in}}%
\pgfpathlineto{\pgfqpoint{4.178891in}{4.112000in}}%
\pgfpathlineto{\pgfqpoint{4.192526in}{4.098640in}}%
\pgfpathlineto{\pgfqpoint{4.206869in}{4.084851in}}%
\pgfpathlineto{\pgfqpoint{4.212191in}{4.079625in}}%
\pgfpathlineto{\pgfqpoint{4.217337in}{4.074667in}}%
\pgfpathlineto{\pgfqpoint{4.231744in}{4.060503in}}%
\pgfpathlineto{\pgfqpoint{4.246949in}{4.045837in}}%
\pgfpathlineto{\pgfqpoint{4.255661in}{4.037333in}}%
\pgfpathlineto{\pgfqpoint{4.287030in}{4.006693in}}%
\pgfpathlineto{\pgfqpoint{4.293865in}{4.000000in}}%
\pgfpathlineto{\pgfqpoint{4.309985in}{3.984048in}}%
\pgfpathlineto{\pgfqpoint{4.327111in}{3.967420in}}%
\pgfpathlineto{\pgfqpoint{4.331949in}{3.962667in}}%
\pgfpathlineto{\pgfqpoint{4.349009in}{3.945730in}}%
\pgfpathlineto{\pgfqpoint{4.367192in}{3.928017in}}%
\pgfpathlineto{\pgfqpoint{4.368585in}{3.926631in}}%
\pgfpathlineto{\pgfqpoint{4.369915in}{3.925333in}}%
\pgfpathlineto{\pgfqpoint{4.387968in}{3.907352in}}%
\pgfpathlineto{\pgfqpoint{4.407273in}{3.888484in}}%
\pgfpathlineto{\pgfqpoint{4.407762in}{3.888000in}}%
\pgfpathlineto{\pgfqpoint{4.426862in}{3.868913in}}%
\pgfpathlineto{\pgfqpoint{4.445471in}{3.850667in}}%
\pgfpathlineto{\pgfqpoint{4.447354in}{3.848801in}}%
\pgfpathlineto{\pgfqpoint{4.483057in}{3.813333in}}%
\pgfpathlineto{\pgfqpoint{4.485164in}{3.811218in}}%
\pgfpathlineto{\pgfqpoint{4.487434in}{3.808982in}}%
\pgfpathlineto{\pgfqpoint{4.504459in}{3.791858in}}%
\pgfpathlineto{\pgfqpoint{4.520526in}{3.776000in}}%
\pgfpathlineto{\pgfqpoint{4.523884in}{3.772618in}}%
\pgfpathlineto{\pgfqpoint{4.527515in}{3.769030in}}%
\pgfpathlineto{\pgfqpoint{4.557880in}{3.738667in}}%
\pgfpathlineto{\pgfqpoint{4.567596in}{3.728945in}}%
\pgfpathlineto{\pgfqpoint{4.581801in}{3.714565in}}%
\pgfpathlineto{\pgfqpoint{4.595120in}{3.701333in}}%
\pgfpathlineto{\pgfqpoint{4.607677in}{3.688728in}}%
\pgfpathlineto{\pgfqpoint{4.620377in}{3.675829in}}%
\pgfpathlineto{\pgfqpoint{4.632246in}{3.664000in}}%
\pgfpathlineto{\pgfqpoint{4.639661in}{3.656458in}}%
\pgfpathlineto{\pgfqpoint{4.647758in}{3.648377in}}%
\pgfpathlineto{\pgfqpoint{4.658889in}{3.637035in}}%
\pgfpathlineto{\pgfqpoint{4.669259in}{3.626667in}}%
\pgfpathlineto{\pgfqpoint{4.687838in}{3.607893in}}%
\pgfpathlineto{\pgfqpoint{4.706160in}{3.589333in}}%
\pgfpathlineto{\pgfqpoint{4.716526in}{3.578721in}}%
\pgfpathlineto{\pgfqpoint{4.727919in}{3.567276in}}%
\pgfpathlineto{\pgfqpoint{4.735725in}{3.559271in}}%
\pgfpathlineto{\pgfqpoint{4.742950in}{3.552000in}}%
\pgfpathlineto{\pgfqpoint{4.754863in}{3.539764in}}%
\pgfpathlineto{\pgfqpoint{4.768000in}{3.526525in}}%
\pgfusepath{fill}%
\end{pgfscope}%
\begin{pgfscope}%
\pgfpathrectangle{\pgfqpoint{0.800000in}{0.528000in}}{\pgfqpoint{3.968000in}{3.696000in}}%
\pgfusepath{clip}%
\pgfsetbuttcap%
\pgfsetroundjoin%
\definecolor{currentfill}{rgb}{0.288921,0.758394,0.428426}%
\pgfsetfillcolor{currentfill}%
\pgfsetlinewidth{0.000000pt}%
\definecolor{currentstroke}{rgb}{0.000000,0.000000,0.000000}%
\pgfsetstrokecolor{currentstroke}%
\pgfsetdash{}{0pt}%
\pgfpathmoveto{\pgfqpoint{4.768000in}{3.532118in}}%
\pgfpathlineto{\pgfqpoint{4.757748in}{3.542450in}}%
\pgfpathlineto{\pgfqpoint{4.748450in}{3.552000in}}%
\pgfpathlineto{\pgfqpoint{4.738582in}{3.561932in}}%
\pgfpathlineto{\pgfqpoint{4.727919in}{3.572866in}}%
\pgfpathlineto{\pgfqpoint{4.719413in}{3.581410in}}%
\pgfpathlineto{\pgfqpoint{4.711674in}{3.589333in}}%
\pgfpathlineto{\pgfqpoint{4.687838in}{3.613479in}}%
\pgfpathlineto{\pgfqpoint{4.674787in}{3.626667in}}%
\pgfpathlineto{\pgfqpoint{4.661751in}{3.639701in}}%
\pgfpathlineto{\pgfqpoint{4.647758in}{3.653959in}}%
\pgfpathlineto{\pgfqpoint{4.642554in}{3.659153in}}%
\pgfpathlineto{\pgfqpoint{4.637788in}{3.664000in}}%
\pgfpathlineto{\pgfqpoint{4.623242in}{3.678498in}}%
\pgfpathlineto{\pgfqpoint{4.607677in}{3.694306in}}%
\pgfpathlineto{\pgfqpoint{4.600676in}{3.701333in}}%
\pgfpathlineto{\pgfqpoint{4.584669in}{3.717236in}}%
\pgfpathlineto{\pgfqpoint{4.567596in}{3.734519in}}%
\pgfpathlineto{\pgfqpoint{4.563451in}{3.738667in}}%
\pgfpathlineto{\pgfqpoint{4.527515in}{3.774600in}}%
\pgfpathlineto{\pgfqpoint{4.526786in}{3.775321in}}%
\pgfpathlineto{\pgfqpoint{4.526111in}{3.776000in}}%
\pgfpathlineto{\pgfqpoint{4.507333in}{3.794534in}}%
\pgfpathlineto{\pgfqpoint{4.488642in}{3.813333in}}%
\pgfpathlineto{\pgfqpoint{4.487434in}{3.814535in}}%
\pgfpathlineto{\pgfqpoint{4.451041in}{3.850667in}}%
\pgfpathlineto{\pgfqpoint{4.447354in}{3.854325in}}%
\pgfpathlineto{\pgfqpoint{4.429741in}{3.871595in}}%
\pgfpathlineto{\pgfqpoint{4.413325in}{3.888000in}}%
\pgfpathlineto{\pgfqpoint{4.407273in}{3.893985in}}%
\pgfpathlineto{\pgfqpoint{4.390850in}{3.910036in}}%
\pgfpathlineto{\pgfqpoint{4.375491in}{3.925333in}}%
\pgfpathlineto{\pgfqpoint{4.371439in}{3.929289in}}%
\pgfpathlineto{\pgfqpoint{4.367192in}{3.933514in}}%
\pgfpathlineto{\pgfqpoint{4.351893in}{3.948417in}}%
\pgfpathlineto{\pgfqpoint{4.337540in}{3.962667in}}%
\pgfpathlineto{\pgfqpoint{4.327111in}{3.972913in}}%
\pgfpathlineto{\pgfqpoint{4.312873in}{3.986738in}}%
\pgfpathlineto{\pgfqpoint{4.299470in}{4.000000in}}%
\pgfpathlineto{\pgfqpoint{4.287030in}{4.012183in}}%
\pgfpathlineto{\pgfqpoint{4.261281in}{4.037333in}}%
\pgfpathlineto{\pgfqpoint{4.246949in}{4.051322in}}%
\pgfpathlineto{\pgfqpoint{4.234637in}{4.063198in}}%
\pgfpathlineto{\pgfqpoint{4.222971in}{4.074667in}}%
\pgfpathlineto{\pgfqpoint{4.215057in}{4.082293in}}%
\pgfpathlineto{\pgfqpoint{4.206869in}{4.090333in}}%
\pgfpathlineto{\pgfqpoint{4.195422in}{4.101338in}}%
\pgfpathlineto{\pgfqpoint{4.184540in}{4.112000in}}%
\pgfpathlineto{\pgfqpoint{4.175800in}{4.120394in}}%
\pgfpathlineto{\pgfqpoint{4.166788in}{4.129215in}}%
\pgfpathlineto{\pgfqpoint{4.156141in}{4.139416in}}%
\pgfpathlineto{\pgfqpoint{4.145987in}{4.149333in}}%
\pgfpathlineto{\pgfqpoint{4.136479in}{4.158435in}}%
\pgfpathlineto{\pgfqpoint{4.126707in}{4.167968in}}%
\pgfpathlineto{\pgfqpoint{4.116795in}{4.177434in}}%
\pgfpathlineto{\pgfqpoint{4.107310in}{4.186667in}}%
\pgfpathlineto{\pgfqpoint{4.097092in}{4.196415in}}%
\pgfpathlineto{\pgfqpoint{4.086626in}{4.206592in}}%
\pgfpathlineto{\pgfqpoint{4.068509in}{4.224000in}}%
\pgfpathlineto{\pgfqpoint{4.065662in}{4.224000in}}%
\pgfpathlineto{\pgfqpoint{4.086626in}{4.203857in}}%
\pgfpathlineto{\pgfqpoint{4.095656in}{4.195077in}}%
\pgfpathlineto{\pgfqpoint{4.104470in}{4.186667in}}%
\pgfpathlineto{\pgfqpoint{4.115344in}{4.176082in}}%
\pgfpathlineto{\pgfqpoint{4.126707in}{4.165230in}}%
\pgfpathlineto{\pgfqpoint{4.135043in}{4.157098in}}%
\pgfpathlineto{\pgfqpoint{4.143155in}{4.149333in}}%
\pgfpathlineto{\pgfqpoint{4.154691in}{4.138066in}}%
\pgfpathlineto{\pgfqpoint{4.166788in}{4.126476in}}%
\pgfpathlineto{\pgfqpoint{4.174366in}{4.119059in}}%
\pgfpathlineto{\pgfqpoint{4.181716in}{4.112000in}}%
\pgfpathlineto{\pgfqpoint{4.193974in}{4.099989in}}%
\pgfpathlineto{\pgfqpoint{4.206869in}{4.087592in}}%
\pgfpathlineto{\pgfqpoint{4.213624in}{4.080959in}}%
\pgfpathlineto{\pgfqpoint{4.220154in}{4.074667in}}%
\pgfpathlineto{\pgfqpoint{4.233190in}{4.061851in}}%
\pgfpathlineto{\pgfqpoint{4.246949in}{4.048580in}}%
\pgfpathlineto{\pgfqpoint{4.258471in}{4.037333in}}%
\pgfpathlineto{\pgfqpoint{4.287030in}{4.009438in}}%
\pgfpathlineto{\pgfqpoint{4.296668in}{4.000000in}}%
\pgfpathlineto{\pgfqpoint{4.311429in}{3.985393in}}%
\pgfpathlineto{\pgfqpoint{4.327111in}{3.970167in}}%
\pgfpathlineto{\pgfqpoint{4.334745in}{3.962667in}}%
\pgfpathlineto{\pgfqpoint{4.350451in}{3.947073in}}%
\pgfpathlineto{\pgfqpoint{4.367192in}{3.930766in}}%
\pgfpathlineto{\pgfqpoint{4.370012in}{3.927960in}}%
\pgfpathlineto{\pgfqpoint{4.372703in}{3.925333in}}%
\pgfpathlineto{\pgfqpoint{4.389409in}{3.908694in}}%
\pgfpathlineto{\pgfqpoint{4.407273in}{3.891234in}}%
\pgfpathlineto{\pgfqpoint{4.410543in}{3.888000in}}%
\pgfpathlineto{\pgfqpoint{4.428302in}{3.870254in}}%
\pgfpathlineto{\pgfqpoint{4.447354in}{3.851573in}}%
\pgfpathlineto{\pgfqpoint{4.448267in}{3.850667in}}%
\pgfpathlineto{\pgfqpoint{4.462137in}{3.836897in}}%
\pgfpathlineto{\pgfqpoint{4.485857in}{3.813333in}}%
\pgfpathlineto{\pgfqpoint{4.486616in}{3.812571in}}%
\pgfpathlineto{\pgfqpoint{4.487434in}{3.811765in}}%
\pgfpathlineto{\pgfqpoint{4.505896in}{3.793196in}}%
\pgfpathlineto{\pgfqpoint{4.523319in}{3.776000in}}%
\pgfpathlineto{\pgfqpoint{4.525335in}{3.773969in}}%
\pgfpathlineto{\pgfqpoint{4.527515in}{3.771815in}}%
\pgfpathlineto{\pgfqpoint{4.560666in}{3.738667in}}%
\pgfpathlineto{\pgfqpoint{4.567596in}{3.731732in}}%
\pgfpathlineto{\pgfqpoint{4.583235in}{3.715900in}}%
\pgfpathlineto{\pgfqpoint{4.597898in}{3.701333in}}%
\pgfpathlineto{\pgfqpoint{4.607677in}{3.691517in}}%
\pgfpathlineto{\pgfqpoint{4.621809in}{3.677164in}}%
\pgfpathlineto{\pgfqpoint{4.635017in}{3.664000in}}%
\pgfpathlineto{\pgfqpoint{4.641107in}{3.657805in}}%
\pgfpathlineto{\pgfqpoint{4.647758in}{3.651168in}}%
\pgfpathlineto{\pgfqpoint{4.660320in}{3.638368in}}%
\pgfpathlineto{\pgfqpoint{4.672023in}{3.626667in}}%
\pgfpathlineto{\pgfqpoint{4.687838in}{3.610686in}}%
\pgfpathlineto{\pgfqpoint{4.708917in}{3.589333in}}%
\pgfpathlineto{\pgfqpoint{4.717969in}{3.580066in}}%
\pgfpathlineto{\pgfqpoint{4.727919in}{3.570071in}}%
\pgfpathlineto{\pgfqpoint{4.737153in}{3.560601in}}%
\pgfpathlineto{\pgfqpoint{4.745700in}{3.552000in}}%
\pgfpathlineto{\pgfqpoint{4.756305in}{3.541107in}}%
\pgfpathlineto{\pgfqpoint{4.768000in}{3.529321in}}%
\pgfusepath{fill}%
\end{pgfscope}%
\begin{pgfscope}%
\pgfpathrectangle{\pgfqpoint{0.800000in}{0.528000in}}{\pgfqpoint{3.968000in}{3.696000in}}%
\pgfusepath{clip}%
\pgfsetbuttcap%
\pgfsetroundjoin%
\definecolor{currentfill}{rgb}{0.296479,0.761561,0.424223}%
\pgfsetfillcolor{currentfill}%
\pgfsetlinewidth{0.000000pt}%
\definecolor{currentstroke}{rgb}{0.000000,0.000000,0.000000}%
\pgfsetstrokecolor{currentstroke}%
\pgfsetdash{}{0pt}%
\pgfpathmoveto{\pgfqpoint{4.768000in}{3.534915in}}%
\pgfpathlineto{\pgfqpoint{4.759190in}{3.543794in}}%
\pgfpathlineto{\pgfqpoint{4.751201in}{3.552000in}}%
\pgfpathlineto{\pgfqpoint{4.740010in}{3.563262in}}%
\pgfpathlineto{\pgfqpoint{4.727919in}{3.575661in}}%
\pgfpathlineto{\pgfqpoint{4.720857in}{3.582755in}}%
\pgfpathlineto{\pgfqpoint{4.714431in}{3.589333in}}%
\pgfpathlineto{\pgfqpoint{4.687838in}{3.616272in}}%
\pgfpathlineto{\pgfqpoint{4.677551in}{3.626667in}}%
\pgfpathlineto{\pgfqpoint{4.663182in}{3.641034in}}%
\pgfpathlineto{\pgfqpoint{4.647758in}{3.656750in}}%
\pgfpathlineto{\pgfqpoint{4.644000in}{3.660500in}}%
\pgfpathlineto{\pgfqpoint{4.640559in}{3.664000in}}%
\pgfpathlineto{\pgfqpoint{4.624674in}{3.679832in}}%
\pgfpathlineto{\pgfqpoint{4.607677in}{3.697095in}}%
\pgfpathlineto{\pgfqpoint{4.603455in}{3.701333in}}%
\pgfpathlineto{\pgfqpoint{4.586102in}{3.718571in}}%
\pgfpathlineto{\pgfqpoint{4.567596in}{3.737306in}}%
\pgfpathlineto{\pgfqpoint{4.566236in}{3.738667in}}%
\pgfpathlineto{\pgfqpoint{4.547743in}{3.757159in}}%
\pgfpathlineto{\pgfqpoint{4.528888in}{3.776000in}}%
\pgfpathlineto{\pgfqpoint{4.527515in}{3.777371in}}%
\pgfpathlineto{\pgfqpoint{4.508769in}{3.795872in}}%
\pgfpathlineto{\pgfqpoint{4.491409in}{3.813333in}}%
\pgfpathlineto{\pgfqpoint{4.487434in}{3.817290in}}%
\pgfpathlineto{\pgfqpoint{4.453816in}{3.850667in}}%
\pgfpathlineto{\pgfqpoint{4.447354in}{3.857078in}}%
\pgfpathlineto{\pgfqpoint{4.431181in}{3.872936in}}%
\pgfpathlineto{\pgfqpoint{4.416106in}{3.888000in}}%
\pgfpathlineto{\pgfqpoint{4.407273in}{3.896735in}}%
\pgfpathlineto{\pgfqpoint{4.392290in}{3.911378in}}%
\pgfpathlineto{\pgfqpoint{4.378280in}{3.925333in}}%
\pgfpathlineto{\pgfqpoint{4.372866in}{3.930618in}}%
\pgfpathlineto{\pgfqpoint{4.367192in}{3.936263in}}%
\pgfpathlineto{\pgfqpoint{4.353336in}{3.949760in}}%
\pgfpathlineto{\pgfqpoint{4.340336in}{3.962667in}}%
\pgfpathlineto{\pgfqpoint{4.327111in}{3.975660in}}%
\pgfpathlineto{\pgfqpoint{4.314317in}{3.988083in}}%
\pgfpathlineto{\pgfqpoint{4.302273in}{4.000000in}}%
\pgfpathlineto{\pgfqpoint{4.287030in}{4.014927in}}%
\pgfpathlineto{\pgfqpoint{4.264091in}{4.037333in}}%
\pgfpathlineto{\pgfqpoint{4.246949in}{4.054065in}}%
\pgfpathlineto{\pgfqpoint{4.236084in}{4.064546in}}%
\pgfpathlineto{\pgfqpoint{4.225789in}{4.074667in}}%
\pgfpathlineto{\pgfqpoint{4.216489in}{4.083628in}}%
\pgfpathlineto{\pgfqpoint{4.206869in}{4.093074in}}%
\pgfpathlineto{\pgfqpoint{4.196870in}{4.102687in}}%
\pgfpathlineto{\pgfqpoint{4.187365in}{4.112000in}}%
\pgfpathlineto{\pgfqpoint{4.177234in}{4.121730in}}%
\pgfpathlineto{\pgfqpoint{4.166788in}{4.131954in}}%
\pgfpathlineto{\pgfqpoint{4.157591in}{4.140766in}}%
\pgfpathlineto{\pgfqpoint{4.148819in}{4.149333in}}%
\pgfpathlineto{\pgfqpoint{4.137914in}{4.159772in}}%
\pgfpathlineto{\pgfqpoint{4.126707in}{4.170705in}}%
\pgfpathlineto{\pgfqpoint{4.118246in}{4.178785in}}%
\pgfpathlineto{\pgfqpoint{4.110149in}{4.186667in}}%
\pgfpathlineto{\pgfqpoint{4.098529in}{4.197754in}}%
\pgfpathlineto{\pgfqpoint{4.086626in}{4.209327in}}%
\pgfpathlineto{\pgfqpoint{4.071356in}{4.224000in}}%
\pgfpathlineto{\pgfqpoint{4.068509in}{4.224000in}}%
\pgfpathlineto{\pgfqpoint{4.086626in}{4.206592in}}%
\pgfpathlineto{\pgfqpoint{4.097092in}{4.196415in}}%
\pgfpathlineto{\pgfqpoint{4.107310in}{4.186667in}}%
\pgfpathlineto{\pgfqpoint{4.116795in}{4.177434in}}%
\pgfpathlineto{\pgfqpoint{4.126707in}{4.167968in}}%
\pgfpathlineto{\pgfqpoint{4.136479in}{4.158435in}}%
\pgfpathlineto{\pgfqpoint{4.145987in}{4.149333in}}%
\pgfpathlineto{\pgfqpoint{4.156141in}{4.139416in}}%
\pgfpathlineto{\pgfqpoint{4.166788in}{4.129215in}}%
\pgfpathlineto{\pgfqpoint{4.175800in}{4.120394in}}%
\pgfpathlineto{\pgfqpoint{4.184540in}{4.112000in}}%
\pgfpathlineto{\pgfqpoint{4.195422in}{4.101338in}}%
\pgfpathlineto{\pgfqpoint{4.206869in}{4.090333in}}%
\pgfpathlineto{\pgfqpoint{4.215057in}{4.082293in}}%
\pgfpathlineto{\pgfqpoint{4.222971in}{4.074667in}}%
\pgfpathlineto{\pgfqpoint{4.234637in}{4.063198in}}%
\pgfpathlineto{\pgfqpoint{4.246949in}{4.051322in}}%
\pgfpathlineto{\pgfqpoint{4.261281in}{4.037333in}}%
\pgfpathlineto{\pgfqpoint{4.287030in}{4.012183in}}%
\pgfpathlineto{\pgfqpoint{4.299470in}{4.000000in}}%
\pgfpathlineto{\pgfqpoint{4.312873in}{3.986738in}}%
\pgfpathlineto{\pgfqpoint{4.327111in}{3.972913in}}%
\pgfpathlineto{\pgfqpoint{4.337540in}{3.962667in}}%
\pgfpathlineto{\pgfqpoint{4.351893in}{3.948417in}}%
\pgfpathlineto{\pgfqpoint{4.367192in}{3.933514in}}%
\pgfpathlineto{\pgfqpoint{4.371439in}{3.929289in}}%
\pgfpathlineto{\pgfqpoint{4.375491in}{3.925333in}}%
\pgfpathlineto{\pgfqpoint{4.390850in}{3.910036in}}%
\pgfpathlineto{\pgfqpoint{4.407273in}{3.893985in}}%
\pgfpathlineto{\pgfqpoint{4.413325in}{3.888000in}}%
\pgfpathlineto{\pgfqpoint{4.429741in}{3.871595in}}%
\pgfpathlineto{\pgfqpoint{4.447354in}{3.854325in}}%
\pgfpathlineto{\pgfqpoint{4.451041in}{3.850667in}}%
\pgfpathlineto{\pgfqpoint{4.487434in}{3.814535in}}%
\pgfpathlineto{\pgfqpoint{4.488642in}{3.813333in}}%
\pgfpathlineto{\pgfqpoint{4.507333in}{3.794534in}}%
\pgfpathlineto{\pgfqpoint{4.526111in}{3.776000in}}%
\pgfpathlineto{\pgfqpoint{4.526786in}{3.775321in}}%
\pgfpathlineto{\pgfqpoint{4.527515in}{3.774600in}}%
\pgfpathlineto{\pgfqpoint{4.563451in}{3.738667in}}%
\pgfpathlineto{\pgfqpoint{4.567596in}{3.734519in}}%
\pgfpathlineto{\pgfqpoint{4.584669in}{3.717236in}}%
\pgfpathlineto{\pgfqpoint{4.600676in}{3.701333in}}%
\pgfpathlineto{\pgfqpoint{4.607677in}{3.694306in}}%
\pgfpathlineto{\pgfqpoint{4.623242in}{3.678498in}}%
\pgfpathlineto{\pgfqpoint{4.637788in}{3.664000in}}%
\pgfpathlineto{\pgfqpoint{4.642554in}{3.659153in}}%
\pgfpathlineto{\pgfqpoint{4.647758in}{3.653959in}}%
\pgfpathlineto{\pgfqpoint{4.661751in}{3.639701in}}%
\pgfpathlineto{\pgfqpoint{4.674787in}{3.626667in}}%
\pgfpathlineto{\pgfqpoint{4.687838in}{3.613479in}}%
\pgfpathlineto{\pgfqpoint{4.711674in}{3.589333in}}%
\pgfpathlineto{\pgfqpoint{4.719413in}{3.581410in}}%
\pgfpathlineto{\pgfqpoint{4.727919in}{3.572866in}}%
\pgfpathlineto{\pgfqpoint{4.738582in}{3.561932in}}%
\pgfpathlineto{\pgfqpoint{4.748450in}{3.552000in}}%
\pgfpathlineto{\pgfqpoint{4.757748in}{3.542450in}}%
\pgfpathlineto{\pgfqpoint{4.768000in}{3.532118in}}%
\pgfusepath{fill}%
\end{pgfscope}%
\begin{pgfscope}%
\pgfpathrectangle{\pgfqpoint{0.800000in}{0.528000in}}{\pgfqpoint{3.968000in}{3.696000in}}%
\pgfusepath{clip}%
\pgfsetbuttcap%
\pgfsetroundjoin%
\definecolor{currentfill}{rgb}{0.296479,0.761561,0.424223}%
\pgfsetfillcolor{currentfill}%
\pgfsetlinewidth{0.000000pt}%
\definecolor{currentstroke}{rgb}{0.000000,0.000000,0.000000}%
\pgfsetstrokecolor{currentstroke}%
\pgfsetdash{}{0pt}%
\pgfpathmoveto{\pgfqpoint{4.768000in}{3.537712in}}%
\pgfpathlineto{\pgfqpoint{4.760632in}{3.545137in}}%
\pgfpathlineto{\pgfqpoint{4.753951in}{3.552000in}}%
\pgfpathlineto{\pgfqpoint{4.741438in}{3.564592in}}%
\pgfpathlineto{\pgfqpoint{4.727919in}{3.578455in}}%
\pgfpathlineto{\pgfqpoint{4.722300in}{3.584100in}}%
\pgfpathlineto{\pgfqpoint{4.717188in}{3.589333in}}%
\pgfpathlineto{\pgfqpoint{4.687838in}{3.619065in}}%
\pgfpathlineto{\pgfqpoint{4.680315in}{3.626667in}}%
\pgfpathlineto{\pgfqpoint{4.664613in}{3.642367in}}%
\pgfpathlineto{\pgfqpoint{4.647758in}{3.659541in}}%
\pgfpathlineto{\pgfqpoint{4.645447in}{3.661847in}}%
\pgfpathlineto{\pgfqpoint{4.643330in}{3.664000in}}%
\pgfpathlineto{\pgfqpoint{4.626106in}{3.681166in}}%
\pgfpathlineto{\pgfqpoint{4.607677in}{3.699884in}}%
\pgfpathlineto{\pgfqpoint{4.606233in}{3.701333in}}%
\pgfpathlineto{\pgfqpoint{4.587536in}{3.719907in}}%
\pgfpathlineto{\pgfqpoint{4.569005in}{3.738667in}}%
\pgfpathlineto{\pgfqpoint{4.568323in}{3.739344in}}%
\pgfpathlineto{\pgfqpoint{4.567596in}{3.740079in}}%
\pgfpathlineto{\pgfqpoint{4.531648in}{3.776000in}}%
\pgfpathlineto{\pgfqpoint{4.527515in}{3.780127in}}%
\pgfpathlineto{\pgfqpoint{4.510206in}{3.797211in}}%
\pgfpathlineto{\pgfqpoint{4.494176in}{3.813333in}}%
\pgfpathlineto{\pgfqpoint{4.487434in}{3.820044in}}%
\pgfpathlineto{\pgfqpoint{4.456590in}{3.850667in}}%
\pgfpathlineto{\pgfqpoint{4.447354in}{3.859830in}}%
\pgfpathlineto{\pgfqpoint{4.432620in}{3.874277in}}%
\pgfpathlineto{\pgfqpoint{4.418887in}{3.888000in}}%
\pgfpathlineto{\pgfqpoint{4.407273in}{3.899486in}}%
\pgfpathlineto{\pgfqpoint{4.393731in}{3.912720in}}%
\pgfpathlineto{\pgfqpoint{4.381068in}{3.925333in}}%
\pgfpathlineto{\pgfqpoint{4.374293in}{3.931948in}}%
\pgfpathlineto{\pgfqpoint{4.367192in}{3.939011in}}%
\pgfpathlineto{\pgfqpoint{4.354778in}{3.951104in}}%
\pgfpathlineto{\pgfqpoint{4.343131in}{3.962667in}}%
\pgfpathlineto{\pgfqpoint{4.327111in}{3.978407in}}%
\pgfpathlineto{\pgfqpoint{4.315760in}{3.989427in}}%
\pgfpathlineto{\pgfqpoint{4.305076in}{4.000000in}}%
\pgfpathlineto{\pgfqpoint{4.287030in}{4.017672in}}%
\pgfpathlineto{\pgfqpoint{4.266901in}{4.037333in}}%
\pgfpathlineto{\pgfqpoint{4.246949in}{4.056808in}}%
\pgfpathlineto{\pgfqpoint{4.237530in}{4.065893in}}%
\pgfpathlineto{\pgfqpoint{4.228606in}{4.074667in}}%
\pgfpathlineto{\pgfqpoint{4.217922in}{4.084962in}}%
\pgfpathlineto{\pgfqpoint{4.206869in}{4.095815in}}%
\pgfpathlineto{\pgfqpoint{4.198318in}{4.104035in}}%
\pgfpathlineto{\pgfqpoint{4.190190in}{4.112000in}}%
\pgfpathlineto{\pgfqpoint{4.178668in}{4.123066in}}%
\pgfpathlineto{\pgfqpoint{4.166788in}{4.134693in}}%
\pgfpathlineto{\pgfqpoint{4.159040in}{4.142117in}}%
\pgfpathlineto{\pgfqpoint{4.151651in}{4.149333in}}%
\pgfpathlineto{\pgfqpoint{4.139349in}{4.161109in}}%
\pgfpathlineto{\pgfqpoint{4.126707in}{4.173442in}}%
\pgfpathlineto{\pgfqpoint{4.119697in}{4.180137in}}%
\pgfpathlineto{\pgfqpoint{4.112989in}{4.186667in}}%
\pgfpathlineto{\pgfqpoint{4.099966in}{4.199092in}}%
\pgfpathlineto{\pgfqpoint{4.086626in}{4.212063in}}%
\pgfpathlineto{\pgfqpoint{4.074202in}{4.224000in}}%
\pgfpathlineto{\pgfqpoint{4.071356in}{4.224000in}}%
\pgfpathlineto{\pgfqpoint{4.086626in}{4.209327in}}%
\pgfpathlineto{\pgfqpoint{4.098529in}{4.197754in}}%
\pgfpathlineto{\pgfqpoint{4.110149in}{4.186667in}}%
\pgfpathlineto{\pgfqpoint{4.118246in}{4.178785in}}%
\pgfpathlineto{\pgfqpoint{4.126707in}{4.170705in}}%
\pgfpathlineto{\pgfqpoint{4.137914in}{4.159772in}}%
\pgfpathlineto{\pgfqpoint{4.148819in}{4.149333in}}%
\pgfpathlineto{\pgfqpoint{4.157591in}{4.140766in}}%
\pgfpathlineto{\pgfqpoint{4.166788in}{4.131954in}}%
\pgfpathlineto{\pgfqpoint{4.177234in}{4.121730in}}%
\pgfpathlineto{\pgfqpoint{4.187365in}{4.112000in}}%
\pgfpathlineto{\pgfqpoint{4.196870in}{4.102687in}}%
\pgfpathlineto{\pgfqpoint{4.206869in}{4.093074in}}%
\pgfpathlineto{\pgfqpoint{4.216489in}{4.083628in}}%
\pgfpathlineto{\pgfqpoint{4.225789in}{4.074667in}}%
\pgfpathlineto{\pgfqpoint{4.236084in}{4.064546in}}%
\pgfpathlineto{\pgfqpoint{4.246949in}{4.054065in}}%
\pgfpathlineto{\pgfqpoint{4.264091in}{4.037333in}}%
\pgfpathlineto{\pgfqpoint{4.287030in}{4.014927in}}%
\pgfpathlineto{\pgfqpoint{4.302273in}{4.000000in}}%
\pgfpathlineto{\pgfqpoint{4.314317in}{3.988083in}}%
\pgfpathlineto{\pgfqpoint{4.327111in}{3.975660in}}%
\pgfpathlineto{\pgfqpoint{4.340336in}{3.962667in}}%
\pgfpathlineto{\pgfqpoint{4.353336in}{3.949760in}}%
\pgfpathlineto{\pgfqpoint{4.367192in}{3.936263in}}%
\pgfpathlineto{\pgfqpoint{4.372866in}{3.930618in}}%
\pgfpathlineto{\pgfqpoint{4.378280in}{3.925333in}}%
\pgfpathlineto{\pgfqpoint{4.392290in}{3.911378in}}%
\pgfpathlineto{\pgfqpoint{4.407273in}{3.896735in}}%
\pgfpathlineto{\pgfqpoint{4.416106in}{3.888000in}}%
\pgfpathlineto{\pgfqpoint{4.431181in}{3.872936in}}%
\pgfpathlineto{\pgfqpoint{4.447354in}{3.857078in}}%
\pgfpathlineto{\pgfqpoint{4.453816in}{3.850667in}}%
\pgfpathlineto{\pgfqpoint{4.487434in}{3.817290in}}%
\pgfpathlineto{\pgfqpoint{4.491409in}{3.813333in}}%
\pgfpathlineto{\pgfqpoint{4.508769in}{3.795872in}}%
\pgfpathlineto{\pgfqpoint{4.527515in}{3.777371in}}%
\pgfpathlineto{\pgfqpoint{4.528888in}{3.776000in}}%
\pgfpathlineto{\pgfqpoint{4.547743in}{3.757159in}}%
\pgfpathlineto{\pgfqpoint{4.566236in}{3.738667in}}%
\pgfpathlineto{\pgfqpoint{4.567596in}{3.737306in}}%
\pgfpathlineto{\pgfqpoint{4.586102in}{3.718571in}}%
\pgfpathlineto{\pgfqpoint{4.603455in}{3.701333in}}%
\pgfpathlineto{\pgfqpoint{4.607677in}{3.697095in}}%
\pgfpathlineto{\pgfqpoint{4.624674in}{3.679832in}}%
\pgfpathlineto{\pgfqpoint{4.640559in}{3.664000in}}%
\pgfpathlineto{\pgfqpoint{4.644000in}{3.660500in}}%
\pgfpathlineto{\pgfqpoint{4.647758in}{3.656750in}}%
\pgfpathlineto{\pgfqpoint{4.663182in}{3.641034in}}%
\pgfpathlineto{\pgfqpoint{4.677551in}{3.626667in}}%
\pgfpathlineto{\pgfqpoint{4.687838in}{3.616272in}}%
\pgfpathlineto{\pgfqpoint{4.714431in}{3.589333in}}%
\pgfpathlineto{\pgfqpoint{4.720857in}{3.582755in}}%
\pgfpathlineto{\pgfqpoint{4.727919in}{3.575661in}}%
\pgfpathlineto{\pgfqpoint{4.740010in}{3.563262in}}%
\pgfpathlineto{\pgfqpoint{4.751201in}{3.552000in}}%
\pgfpathlineto{\pgfqpoint{4.759190in}{3.543794in}}%
\pgfpathlineto{\pgfqpoint{4.768000in}{3.534915in}}%
\pgfusepath{fill}%
\end{pgfscope}%
\begin{pgfscope}%
\pgfpathrectangle{\pgfqpoint{0.800000in}{0.528000in}}{\pgfqpoint{3.968000in}{3.696000in}}%
\pgfusepath{clip}%
\pgfsetbuttcap%
\pgfsetroundjoin%
\definecolor{currentfill}{rgb}{0.296479,0.761561,0.424223}%
\pgfsetfillcolor{currentfill}%
\pgfsetlinewidth{0.000000pt}%
\definecolor{currentstroke}{rgb}{0.000000,0.000000,0.000000}%
\pgfsetstrokecolor{currentstroke}%
\pgfsetdash{}{0pt}%
\pgfpathmoveto{\pgfqpoint{4.768000in}{3.540509in}}%
\pgfpathlineto{\pgfqpoint{4.762074in}{3.546481in}}%
\pgfpathlineto{\pgfqpoint{4.756701in}{3.552000in}}%
\pgfpathlineto{\pgfqpoint{4.742866in}{3.565922in}}%
\pgfpathlineto{\pgfqpoint{4.727919in}{3.581250in}}%
\pgfpathlineto{\pgfqpoint{4.723744in}{3.585444in}}%
\pgfpathlineto{\pgfqpoint{4.719945in}{3.589333in}}%
\pgfpathlineto{\pgfqpoint{4.687838in}{3.621858in}}%
\pgfpathlineto{\pgfqpoint{4.683079in}{3.626667in}}%
\pgfpathlineto{\pgfqpoint{4.666044in}{3.643700in}}%
\pgfpathlineto{\pgfqpoint{4.647758in}{3.662332in}}%
\pgfpathlineto{\pgfqpoint{4.646893in}{3.663195in}}%
\pgfpathlineto{\pgfqpoint{4.646102in}{3.664000in}}%
\pgfpathlineto{\pgfqpoint{4.627539in}{3.682501in}}%
\pgfpathlineto{\pgfqpoint{4.608996in}{3.701333in}}%
\pgfpathlineto{\pgfqpoint{4.607677in}{3.702659in}}%
\pgfpathlineto{\pgfqpoint{4.588970in}{3.721242in}}%
\pgfpathlineto{\pgfqpoint{4.571758in}{3.738667in}}%
\pgfpathlineto{\pgfqpoint{4.569743in}{3.740666in}}%
\pgfpathlineto{\pgfqpoint{4.567596in}{3.742837in}}%
\pgfpathlineto{\pgfqpoint{4.534408in}{3.776000in}}%
\pgfpathlineto{\pgfqpoint{4.527515in}{3.782883in}}%
\pgfpathlineto{\pgfqpoint{4.511643in}{3.798549in}}%
\pgfpathlineto{\pgfqpoint{4.496943in}{3.813333in}}%
\pgfpathlineto{\pgfqpoint{4.487434in}{3.822798in}}%
\pgfpathlineto{\pgfqpoint{4.459364in}{3.850667in}}%
\pgfpathlineto{\pgfqpoint{4.447354in}{3.862582in}}%
\pgfpathlineto{\pgfqpoint{4.434060in}{3.875618in}}%
\pgfpathlineto{\pgfqpoint{4.421669in}{3.888000in}}%
\pgfpathlineto{\pgfqpoint{4.407273in}{3.902236in}}%
\pgfpathlineto{\pgfqpoint{4.395172in}{3.914062in}}%
\pgfpathlineto{\pgfqpoint{4.383857in}{3.925333in}}%
\pgfpathlineto{\pgfqpoint{4.375720in}{3.933277in}}%
\pgfpathlineto{\pgfqpoint{4.367192in}{3.941760in}}%
\pgfpathlineto{\pgfqpoint{4.356221in}{3.952447in}}%
\pgfpathlineto{\pgfqpoint{4.345927in}{3.962667in}}%
\pgfpathlineto{\pgfqpoint{4.327111in}{3.981153in}}%
\pgfpathlineto{\pgfqpoint{4.317204in}{3.990772in}}%
\pgfpathlineto{\pgfqpoint{4.307879in}{4.000000in}}%
\pgfpathlineto{\pgfqpoint{4.287030in}{4.020417in}}%
\pgfpathlineto{\pgfqpoint{4.269711in}{4.037333in}}%
\pgfpathlineto{\pgfqpoint{4.246949in}{4.059551in}}%
\pgfpathlineto{\pgfqpoint{4.238977in}{4.067241in}}%
\pgfpathlineto{\pgfqpoint{4.231423in}{4.074667in}}%
\pgfpathlineto{\pgfqpoint{4.219354in}{4.086296in}}%
\pgfpathlineto{\pgfqpoint{4.206869in}{4.098556in}}%
\pgfpathlineto{\pgfqpoint{4.199766in}{4.105384in}}%
\pgfpathlineto{\pgfqpoint{4.193014in}{4.112000in}}%
\pgfpathlineto{\pgfqpoint{4.180102in}{4.124401in}}%
\pgfpathlineto{\pgfqpoint{4.166788in}{4.137432in}}%
\pgfpathlineto{\pgfqpoint{4.160490in}{4.143467in}}%
\pgfpathlineto{\pgfqpoint{4.154483in}{4.149333in}}%
\pgfpathlineto{\pgfqpoint{4.140785in}{4.162446in}}%
\pgfpathlineto{\pgfqpoint{4.126707in}{4.176179in}}%
\pgfpathlineto{\pgfqpoint{4.121148in}{4.181488in}}%
\pgfpathlineto{\pgfqpoint{4.115828in}{4.186667in}}%
\pgfpathlineto{\pgfqpoint{4.101403in}{4.200430in}}%
\pgfpathlineto{\pgfqpoint{4.086626in}{4.214798in}}%
\pgfpathlineto{\pgfqpoint{4.077049in}{4.224000in}}%
\pgfpathlineto{\pgfqpoint{4.074202in}{4.224000in}}%
\pgfpathlineto{\pgfqpoint{4.086626in}{4.212063in}}%
\pgfpathlineto{\pgfqpoint{4.099966in}{4.199092in}}%
\pgfpathlineto{\pgfqpoint{4.112989in}{4.186667in}}%
\pgfpathlineto{\pgfqpoint{4.119697in}{4.180137in}}%
\pgfpathlineto{\pgfqpoint{4.126707in}{4.173442in}}%
\pgfpathlineto{\pgfqpoint{4.139349in}{4.161109in}}%
\pgfpathlineto{\pgfqpoint{4.151651in}{4.149333in}}%
\pgfpathlineto{\pgfqpoint{4.159040in}{4.142117in}}%
\pgfpathlineto{\pgfqpoint{4.166788in}{4.134693in}}%
\pgfpathlineto{\pgfqpoint{4.178668in}{4.123066in}}%
\pgfpathlineto{\pgfqpoint{4.190190in}{4.112000in}}%
\pgfpathlineto{\pgfqpoint{4.198318in}{4.104035in}}%
\pgfpathlineto{\pgfqpoint{4.206869in}{4.095815in}}%
\pgfpathlineto{\pgfqpoint{4.217922in}{4.084962in}}%
\pgfpathlineto{\pgfqpoint{4.228606in}{4.074667in}}%
\pgfpathlineto{\pgfqpoint{4.237530in}{4.065893in}}%
\pgfpathlineto{\pgfqpoint{4.246949in}{4.056808in}}%
\pgfpathlineto{\pgfqpoint{4.266901in}{4.037333in}}%
\pgfpathlineto{\pgfqpoint{4.287030in}{4.017672in}}%
\pgfpathlineto{\pgfqpoint{4.305076in}{4.000000in}}%
\pgfpathlineto{\pgfqpoint{4.315760in}{3.989427in}}%
\pgfpathlineto{\pgfqpoint{4.327111in}{3.978407in}}%
\pgfpathlineto{\pgfqpoint{4.343131in}{3.962667in}}%
\pgfpathlineto{\pgfqpoint{4.354778in}{3.951104in}}%
\pgfpathlineto{\pgfqpoint{4.367192in}{3.939011in}}%
\pgfpathlineto{\pgfqpoint{4.374293in}{3.931948in}}%
\pgfpathlineto{\pgfqpoint{4.381068in}{3.925333in}}%
\pgfpathlineto{\pgfqpoint{4.393731in}{3.912720in}}%
\pgfpathlineto{\pgfqpoint{4.407273in}{3.899486in}}%
\pgfpathlineto{\pgfqpoint{4.418887in}{3.888000in}}%
\pgfpathlineto{\pgfqpoint{4.432620in}{3.874277in}}%
\pgfpathlineto{\pgfqpoint{4.447354in}{3.859830in}}%
\pgfpathlineto{\pgfqpoint{4.456590in}{3.850667in}}%
\pgfpathlineto{\pgfqpoint{4.487434in}{3.820044in}}%
\pgfpathlineto{\pgfqpoint{4.494176in}{3.813333in}}%
\pgfpathlineto{\pgfqpoint{4.510206in}{3.797211in}}%
\pgfpathlineto{\pgfqpoint{4.527515in}{3.780127in}}%
\pgfpathlineto{\pgfqpoint{4.531648in}{3.776000in}}%
\pgfpathlineto{\pgfqpoint{4.567596in}{3.740079in}}%
\pgfpathlineto{\pgfqpoint{4.568323in}{3.739344in}}%
\pgfpathlineto{\pgfqpoint{4.569005in}{3.738667in}}%
\pgfpathlineto{\pgfqpoint{4.587536in}{3.719907in}}%
\pgfpathlineto{\pgfqpoint{4.606233in}{3.701333in}}%
\pgfpathlineto{\pgfqpoint{4.607677in}{3.699884in}}%
\pgfpathlineto{\pgfqpoint{4.626106in}{3.681166in}}%
\pgfpathlineto{\pgfqpoint{4.643330in}{3.664000in}}%
\pgfpathlineto{\pgfqpoint{4.645447in}{3.661847in}}%
\pgfpathlineto{\pgfqpoint{4.647758in}{3.659541in}}%
\pgfpathlineto{\pgfqpoint{4.664613in}{3.642367in}}%
\pgfpathlineto{\pgfqpoint{4.680315in}{3.626667in}}%
\pgfpathlineto{\pgfqpoint{4.687838in}{3.619065in}}%
\pgfpathlineto{\pgfqpoint{4.717188in}{3.589333in}}%
\pgfpathlineto{\pgfqpoint{4.722300in}{3.584100in}}%
\pgfpathlineto{\pgfqpoint{4.727919in}{3.578455in}}%
\pgfpathlineto{\pgfqpoint{4.741438in}{3.564592in}}%
\pgfpathlineto{\pgfqpoint{4.753951in}{3.552000in}}%
\pgfpathlineto{\pgfqpoint{4.760632in}{3.545137in}}%
\pgfpathlineto{\pgfqpoint{4.768000in}{3.537712in}}%
\pgfusepath{fill}%
\end{pgfscope}%
\begin{pgfscope}%
\pgfpathrectangle{\pgfqpoint{0.800000in}{0.528000in}}{\pgfqpoint{3.968000in}{3.696000in}}%
\pgfusepath{clip}%
\pgfsetbuttcap%
\pgfsetroundjoin%
\definecolor{currentfill}{rgb}{0.296479,0.761561,0.424223}%
\pgfsetfillcolor{currentfill}%
\pgfsetlinewidth{0.000000pt}%
\definecolor{currentstroke}{rgb}{0.000000,0.000000,0.000000}%
\pgfsetstrokecolor{currentstroke}%
\pgfsetdash{}{0pt}%
\pgfpathmoveto{\pgfqpoint{4.768000in}{3.543305in}}%
\pgfpathlineto{\pgfqpoint{4.763517in}{3.547824in}}%
\pgfpathlineto{\pgfqpoint{4.759451in}{3.552000in}}%
\pgfpathlineto{\pgfqpoint{4.744294in}{3.567253in}}%
\pgfpathlineto{\pgfqpoint{4.727919in}{3.584045in}}%
\pgfpathlineto{\pgfqpoint{4.725188in}{3.586789in}}%
\pgfpathlineto{\pgfqpoint{4.722703in}{3.589333in}}%
\pgfpathlineto{\pgfqpoint{4.687838in}{3.624651in}}%
\pgfpathlineto{\pgfqpoint{4.685843in}{3.626667in}}%
\pgfpathlineto{\pgfqpoint{4.667475in}{3.645033in}}%
\pgfpathlineto{\pgfqpoint{4.648860in}{3.664000in}}%
\pgfpathlineto{\pgfqpoint{4.648328in}{3.664531in}}%
\pgfpathlineto{\pgfqpoint{4.647758in}{3.665111in}}%
\pgfpathlineto{\pgfqpoint{4.628971in}{3.683835in}}%
\pgfpathlineto{\pgfqpoint{4.611742in}{3.701333in}}%
\pgfpathlineto{\pgfqpoint{4.607677in}{3.705419in}}%
\pgfpathlineto{\pgfqpoint{4.590404in}{3.722578in}}%
\pgfpathlineto{\pgfqpoint{4.574511in}{3.738667in}}%
\pgfpathlineto{\pgfqpoint{4.571163in}{3.741989in}}%
\pgfpathlineto{\pgfqpoint{4.567596in}{3.745595in}}%
\pgfpathlineto{\pgfqpoint{4.537168in}{3.776000in}}%
\pgfpathlineto{\pgfqpoint{4.527515in}{3.785639in}}%
\pgfpathlineto{\pgfqpoint{4.513079in}{3.799887in}}%
\pgfpathlineto{\pgfqpoint{4.499710in}{3.813333in}}%
\pgfpathlineto{\pgfqpoint{4.487434in}{3.825552in}}%
\pgfpathlineto{\pgfqpoint{4.462138in}{3.850667in}}%
\pgfpathlineto{\pgfqpoint{4.447354in}{3.865335in}}%
\pgfpathlineto{\pgfqpoint{4.435499in}{3.876958in}}%
\pgfpathlineto{\pgfqpoint{4.424450in}{3.888000in}}%
\pgfpathlineto{\pgfqpoint{4.407273in}{3.904987in}}%
\pgfpathlineto{\pgfqpoint{4.396613in}{3.915405in}}%
\pgfpathlineto{\pgfqpoint{4.386645in}{3.925333in}}%
\pgfpathlineto{\pgfqpoint{4.377147in}{3.934606in}}%
\pgfpathlineto{\pgfqpoint{4.367192in}{3.944508in}}%
\pgfpathlineto{\pgfqpoint{4.357663in}{3.953791in}}%
\pgfpathlineto{\pgfqpoint{4.348723in}{3.962667in}}%
\pgfpathlineto{\pgfqpoint{4.327111in}{3.983900in}}%
\pgfpathlineto{\pgfqpoint{4.318648in}{3.992117in}}%
\pgfpathlineto{\pgfqpoint{4.310682in}{4.000000in}}%
\pgfpathlineto{\pgfqpoint{4.287030in}{4.023162in}}%
\pgfpathlineto{\pgfqpoint{4.272521in}{4.037333in}}%
\pgfpathlineto{\pgfqpoint{4.246949in}{4.062294in}}%
\pgfpathlineto{\pgfqpoint{4.240424in}{4.068588in}}%
\pgfpathlineto{\pgfqpoint{4.234241in}{4.074667in}}%
\pgfpathlineto{\pgfqpoint{4.220787in}{4.087631in}}%
\pgfpathlineto{\pgfqpoint{4.206869in}{4.101297in}}%
\pgfpathlineto{\pgfqpoint{4.201214in}{4.106733in}}%
\pgfpathlineto{\pgfqpoint{4.195839in}{4.112000in}}%
\pgfpathlineto{\pgfqpoint{4.181536in}{4.125737in}}%
\pgfpathlineto{\pgfqpoint{4.166788in}{4.140171in}}%
\pgfpathlineto{\pgfqpoint{4.161939in}{4.144817in}}%
\pgfpathlineto{\pgfqpoint{4.157315in}{4.149333in}}%
\pgfpathlineto{\pgfqpoint{4.142220in}{4.163783in}}%
\pgfpathlineto{\pgfqpoint{4.126707in}{4.178917in}}%
\pgfpathlineto{\pgfqpoint{4.122599in}{4.182840in}}%
\pgfpathlineto{\pgfqpoint{4.118668in}{4.186667in}}%
\pgfpathlineto{\pgfqpoint{4.102840in}{4.201769in}}%
\pgfpathlineto{\pgfqpoint{4.086626in}{4.217534in}}%
\pgfpathlineto{\pgfqpoint{4.079896in}{4.224000in}}%
\pgfpathlineto{\pgfqpoint{4.077049in}{4.224000in}}%
\pgfpathlineto{\pgfqpoint{4.086626in}{4.214798in}}%
\pgfpathlineto{\pgfqpoint{4.101403in}{4.200430in}}%
\pgfpathlineto{\pgfqpoint{4.115828in}{4.186667in}}%
\pgfpathlineto{\pgfqpoint{4.121148in}{4.181488in}}%
\pgfpathlineto{\pgfqpoint{4.126707in}{4.176179in}}%
\pgfpathlineto{\pgfqpoint{4.140785in}{4.162446in}}%
\pgfpathlineto{\pgfqpoint{4.154483in}{4.149333in}}%
\pgfpathlineto{\pgfqpoint{4.160490in}{4.143467in}}%
\pgfpathlineto{\pgfqpoint{4.166788in}{4.137432in}}%
\pgfpathlineto{\pgfqpoint{4.180102in}{4.124401in}}%
\pgfpathlineto{\pgfqpoint{4.193014in}{4.112000in}}%
\pgfpathlineto{\pgfqpoint{4.199766in}{4.105384in}}%
\pgfpathlineto{\pgfqpoint{4.206869in}{4.098556in}}%
\pgfpathlineto{\pgfqpoint{4.219354in}{4.086296in}}%
\pgfpathlineto{\pgfqpoint{4.231423in}{4.074667in}}%
\pgfpathlineto{\pgfqpoint{4.238977in}{4.067241in}}%
\pgfpathlineto{\pgfqpoint{4.246949in}{4.059551in}}%
\pgfpathlineto{\pgfqpoint{4.269711in}{4.037333in}}%
\pgfpathlineto{\pgfqpoint{4.287030in}{4.020417in}}%
\pgfpathlineto{\pgfqpoint{4.307879in}{4.000000in}}%
\pgfpathlineto{\pgfqpoint{4.317204in}{3.990772in}}%
\pgfpathlineto{\pgfqpoint{4.327111in}{3.981153in}}%
\pgfpathlineto{\pgfqpoint{4.345927in}{3.962667in}}%
\pgfpathlineto{\pgfqpoint{4.356221in}{3.952447in}}%
\pgfpathlineto{\pgfqpoint{4.367192in}{3.941760in}}%
\pgfpathlineto{\pgfqpoint{4.375720in}{3.933277in}}%
\pgfpathlineto{\pgfqpoint{4.383857in}{3.925333in}}%
\pgfpathlineto{\pgfqpoint{4.395172in}{3.914062in}}%
\pgfpathlineto{\pgfqpoint{4.407273in}{3.902236in}}%
\pgfpathlineto{\pgfqpoint{4.421669in}{3.888000in}}%
\pgfpathlineto{\pgfqpoint{4.434060in}{3.875618in}}%
\pgfpathlineto{\pgfqpoint{4.447354in}{3.862582in}}%
\pgfpathlineto{\pgfqpoint{4.459364in}{3.850667in}}%
\pgfpathlineto{\pgfqpoint{4.487434in}{3.822798in}}%
\pgfpathlineto{\pgfqpoint{4.496943in}{3.813333in}}%
\pgfpathlineto{\pgfqpoint{4.511643in}{3.798549in}}%
\pgfpathlineto{\pgfqpoint{4.527515in}{3.782883in}}%
\pgfpathlineto{\pgfqpoint{4.534408in}{3.776000in}}%
\pgfpathlineto{\pgfqpoint{4.567596in}{3.742837in}}%
\pgfpathlineto{\pgfqpoint{4.569743in}{3.740666in}}%
\pgfpathlineto{\pgfqpoint{4.571758in}{3.738667in}}%
\pgfpathlineto{\pgfqpoint{4.588970in}{3.721242in}}%
\pgfpathlineto{\pgfqpoint{4.607677in}{3.702659in}}%
\pgfpathlineto{\pgfqpoint{4.608996in}{3.701333in}}%
\pgfpathlineto{\pgfqpoint{4.627539in}{3.682501in}}%
\pgfpathlineto{\pgfqpoint{4.646102in}{3.664000in}}%
\pgfpathlineto{\pgfqpoint{4.646893in}{3.663195in}}%
\pgfpathlineto{\pgfqpoint{4.647758in}{3.662332in}}%
\pgfpathlineto{\pgfqpoint{4.666044in}{3.643700in}}%
\pgfpathlineto{\pgfqpoint{4.683079in}{3.626667in}}%
\pgfpathlineto{\pgfqpoint{4.687838in}{3.621858in}}%
\pgfpathlineto{\pgfqpoint{4.719945in}{3.589333in}}%
\pgfpathlineto{\pgfqpoint{4.723744in}{3.585444in}}%
\pgfpathlineto{\pgfqpoint{4.727919in}{3.581250in}}%
\pgfpathlineto{\pgfqpoint{4.742866in}{3.565922in}}%
\pgfpathlineto{\pgfqpoint{4.756701in}{3.552000in}}%
\pgfpathlineto{\pgfqpoint{4.762074in}{3.546481in}}%
\pgfpathlineto{\pgfqpoint{4.768000in}{3.540509in}}%
\pgfusepath{fill}%
\end{pgfscope}%
\begin{pgfscope}%
\pgfpathrectangle{\pgfqpoint{0.800000in}{0.528000in}}{\pgfqpoint{3.968000in}{3.696000in}}%
\pgfusepath{clip}%
\pgfsetbuttcap%
\pgfsetroundjoin%
\definecolor{currentfill}{rgb}{0.304148,0.764704,0.419943}%
\pgfsetfillcolor{currentfill}%
\pgfsetlinewidth{0.000000pt}%
\definecolor{currentstroke}{rgb}{0.000000,0.000000,0.000000}%
\pgfsetstrokecolor{currentstroke}%
\pgfsetdash{}{0pt}%
\pgfpathmoveto{\pgfqpoint{4.768000in}{3.546102in}}%
\pgfpathlineto{\pgfqpoint{4.764959in}{3.549167in}}%
\pgfpathlineto{\pgfqpoint{4.762201in}{3.552000in}}%
\pgfpathlineto{\pgfqpoint{4.745723in}{3.568583in}}%
\pgfpathlineto{\pgfqpoint{4.727919in}{3.586840in}}%
\pgfpathlineto{\pgfqpoint{4.726631in}{3.588134in}}%
\pgfpathlineto{\pgfqpoint{4.725460in}{3.589333in}}%
\pgfpathlineto{\pgfqpoint{4.697368in}{3.617790in}}%
\pgfpathlineto{\pgfqpoint{4.688599in}{3.626667in}}%
\pgfpathlineto{\pgfqpoint{4.688232in}{3.627034in}}%
\pgfpathlineto{\pgfqpoint{4.687838in}{3.627436in}}%
\pgfpathlineto{\pgfqpoint{4.668906in}{3.646366in}}%
\pgfpathlineto{\pgfqpoint{4.651599in}{3.664000in}}%
\pgfpathlineto{\pgfqpoint{4.649745in}{3.665851in}}%
\pgfpathlineto{\pgfqpoint{4.647758in}{3.667873in}}%
\pgfpathlineto{\pgfqpoint{4.630404in}{3.685169in}}%
\pgfpathlineto{\pgfqpoint{4.614488in}{3.701333in}}%
\pgfpathlineto{\pgfqpoint{4.607677in}{3.708179in}}%
\pgfpathlineto{\pgfqpoint{4.591838in}{3.723913in}}%
\pgfpathlineto{\pgfqpoint{4.577265in}{3.738667in}}%
\pgfpathlineto{\pgfqpoint{4.572583in}{3.743312in}}%
\pgfpathlineto{\pgfqpoint{4.567596in}{3.748353in}}%
\pgfpathlineto{\pgfqpoint{4.539928in}{3.776000in}}%
\pgfpathlineto{\pgfqpoint{4.527515in}{3.788395in}}%
\pgfpathlineto{\pgfqpoint{4.514516in}{3.801225in}}%
\pgfpathlineto{\pgfqpoint{4.502478in}{3.813333in}}%
\pgfpathlineto{\pgfqpoint{4.487434in}{3.828307in}}%
\pgfpathlineto{\pgfqpoint{4.464912in}{3.850667in}}%
\pgfpathlineto{\pgfqpoint{4.447354in}{3.868087in}}%
\pgfpathlineto{\pgfqpoint{4.436939in}{3.878299in}}%
\pgfpathlineto{\pgfqpoint{4.427231in}{3.888000in}}%
\pgfpathlineto{\pgfqpoint{4.407273in}{3.907737in}}%
\pgfpathlineto{\pgfqpoint{4.398054in}{3.916747in}}%
\pgfpathlineto{\pgfqpoint{4.389433in}{3.925333in}}%
\pgfpathlineto{\pgfqpoint{4.378574in}{3.935935in}}%
\pgfpathlineto{\pgfqpoint{4.367192in}{3.947257in}}%
\pgfpathlineto{\pgfqpoint{4.359105in}{3.955134in}}%
\pgfpathlineto{\pgfqpoint{4.351518in}{3.962667in}}%
\pgfpathlineto{\pgfqpoint{4.327111in}{3.986646in}}%
\pgfpathlineto{\pgfqpoint{4.320092in}{3.993462in}}%
\pgfpathlineto{\pgfqpoint{4.313485in}{4.000000in}}%
\pgfpathlineto{\pgfqpoint{4.287030in}{4.025906in}}%
\pgfpathlineto{\pgfqpoint{4.275332in}{4.037333in}}%
\pgfpathlineto{\pgfqpoint{4.246949in}{4.065037in}}%
\pgfpathlineto{\pgfqpoint{4.241870in}{4.069936in}}%
\pgfpathlineto{\pgfqpoint{4.237058in}{4.074667in}}%
\pgfpathlineto{\pgfqpoint{4.222219in}{4.088965in}}%
\pgfpathlineto{\pgfqpoint{4.206869in}{4.104038in}}%
\pgfpathlineto{\pgfqpoint{4.202662in}{4.108082in}}%
\pgfpathlineto{\pgfqpoint{4.198664in}{4.112000in}}%
\pgfpathlineto{\pgfqpoint{4.182970in}{4.127073in}}%
\pgfpathlineto{\pgfqpoint{4.166788in}{4.142910in}}%
\pgfpathlineto{\pgfqpoint{4.163389in}{4.146167in}}%
\pgfpathlineto{\pgfqpoint{4.160147in}{4.149333in}}%
\pgfpathlineto{\pgfqpoint{4.143656in}{4.165120in}}%
\pgfpathlineto{\pgfqpoint{4.126707in}{4.181654in}}%
\pgfpathlineto{\pgfqpoint{4.124050in}{4.184192in}}%
\pgfpathlineto{\pgfqpoint{4.121507in}{4.186667in}}%
\pgfpathlineto{\pgfqpoint{4.104276in}{4.203107in}}%
\pgfpathlineto{\pgfqpoint{4.086626in}{4.220269in}}%
\pgfpathlineto{\pgfqpoint{4.082743in}{4.224000in}}%
\pgfpathlineto{\pgfqpoint{4.079896in}{4.224000in}}%
\pgfpathlineto{\pgfqpoint{4.086626in}{4.217534in}}%
\pgfpathlineto{\pgfqpoint{4.102840in}{4.201769in}}%
\pgfpathlineto{\pgfqpoint{4.118668in}{4.186667in}}%
\pgfpathlineto{\pgfqpoint{4.122599in}{4.182840in}}%
\pgfpathlineto{\pgfqpoint{4.126707in}{4.178917in}}%
\pgfpathlineto{\pgfqpoint{4.142220in}{4.163783in}}%
\pgfpathlineto{\pgfqpoint{4.157315in}{4.149333in}}%
\pgfpathlineto{\pgfqpoint{4.161939in}{4.144817in}}%
\pgfpathlineto{\pgfqpoint{4.166788in}{4.140171in}}%
\pgfpathlineto{\pgfqpoint{4.181536in}{4.125737in}}%
\pgfpathlineto{\pgfqpoint{4.195839in}{4.112000in}}%
\pgfpathlineto{\pgfqpoint{4.201214in}{4.106733in}}%
\pgfpathlineto{\pgfqpoint{4.206869in}{4.101297in}}%
\pgfpathlineto{\pgfqpoint{4.220787in}{4.087631in}}%
\pgfpathlineto{\pgfqpoint{4.234241in}{4.074667in}}%
\pgfpathlineto{\pgfqpoint{4.240424in}{4.068588in}}%
\pgfpathlineto{\pgfqpoint{4.246949in}{4.062294in}}%
\pgfpathlineto{\pgfqpoint{4.272521in}{4.037333in}}%
\pgfpathlineto{\pgfqpoint{4.287030in}{4.023162in}}%
\pgfpathlineto{\pgfqpoint{4.310682in}{4.000000in}}%
\pgfpathlineto{\pgfqpoint{4.318648in}{3.992117in}}%
\pgfpathlineto{\pgfqpoint{4.327111in}{3.983900in}}%
\pgfpathlineto{\pgfqpoint{4.348723in}{3.962667in}}%
\pgfpathlineto{\pgfqpoint{4.357663in}{3.953791in}}%
\pgfpathlineto{\pgfqpoint{4.367192in}{3.944508in}}%
\pgfpathlineto{\pgfqpoint{4.377147in}{3.934606in}}%
\pgfpathlineto{\pgfqpoint{4.386645in}{3.925333in}}%
\pgfpathlineto{\pgfqpoint{4.396613in}{3.915405in}}%
\pgfpathlineto{\pgfqpoint{4.407273in}{3.904987in}}%
\pgfpathlineto{\pgfqpoint{4.424450in}{3.888000in}}%
\pgfpathlineto{\pgfqpoint{4.435499in}{3.876958in}}%
\pgfpathlineto{\pgfqpoint{4.447354in}{3.865335in}}%
\pgfpathlineto{\pgfqpoint{4.462138in}{3.850667in}}%
\pgfpathlineto{\pgfqpoint{4.487434in}{3.825552in}}%
\pgfpathlineto{\pgfqpoint{4.499710in}{3.813333in}}%
\pgfpathlineto{\pgfqpoint{4.513079in}{3.799887in}}%
\pgfpathlineto{\pgfqpoint{4.527515in}{3.785639in}}%
\pgfpathlineto{\pgfqpoint{4.537168in}{3.776000in}}%
\pgfpathlineto{\pgfqpoint{4.567596in}{3.745595in}}%
\pgfpathlineto{\pgfqpoint{4.571163in}{3.741989in}}%
\pgfpathlineto{\pgfqpoint{4.574511in}{3.738667in}}%
\pgfpathlineto{\pgfqpoint{4.590404in}{3.722578in}}%
\pgfpathlineto{\pgfqpoint{4.607677in}{3.705419in}}%
\pgfpathlineto{\pgfqpoint{4.611742in}{3.701333in}}%
\pgfpathlineto{\pgfqpoint{4.628971in}{3.683835in}}%
\pgfpathlineto{\pgfqpoint{4.647758in}{3.665111in}}%
\pgfpathlineto{\pgfqpoint{4.648328in}{3.664531in}}%
\pgfpathlineto{\pgfqpoint{4.648860in}{3.664000in}}%
\pgfpathlineto{\pgfqpoint{4.667475in}{3.645033in}}%
\pgfpathlineto{\pgfqpoint{4.685843in}{3.626667in}}%
\pgfpathlineto{\pgfqpoint{4.687838in}{3.624651in}}%
\pgfpathlineto{\pgfqpoint{4.722703in}{3.589333in}}%
\pgfpathlineto{\pgfqpoint{4.725188in}{3.586789in}}%
\pgfpathlineto{\pgfqpoint{4.727919in}{3.584045in}}%
\pgfpathlineto{\pgfqpoint{4.744294in}{3.567253in}}%
\pgfpathlineto{\pgfqpoint{4.759451in}{3.552000in}}%
\pgfpathlineto{\pgfqpoint{4.763517in}{3.547824in}}%
\pgfpathlineto{\pgfqpoint{4.768000in}{3.543305in}}%
\pgfusepath{fill}%
\end{pgfscope}%
\begin{pgfscope}%
\pgfpathrectangle{\pgfqpoint{0.800000in}{0.528000in}}{\pgfqpoint{3.968000in}{3.696000in}}%
\pgfusepath{clip}%
\pgfsetbuttcap%
\pgfsetroundjoin%
\definecolor{currentfill}{rgb}{0.304148,0.764704,0.419943}%
\pgfsetfillcolor{currentfill}%
\pgfsetlinewidth{0.000000pt}%
\definecolor{currentstroke}{rgb}{0.000000,0.000000,0.000000}%
\pgfsetstrokecolor{currentstroke}%
\pgfsetdash{}{0pt}%
\pgfpathmoveto{\pgfqpoint{4.768000in}{3.548899in}}%
\pgfpathlineto{\pgfqpoint{4.766401in}{3.550511in}}%
\pgfpathlineto{\pgfqpoint{4.764951in}{3.552000in}}%
\pgfpathlineto{\pgfqpoint{4.747151in}{3.569913in}}%
\pgfpathlineto{\pgfqpoint{4.728213in}{3.589333in}}%
\pgfpathlineto{\pgfqpoint{4.728072in}{3.589475in}}%
\pgfpathlineto{\pgfqpoint{4.727919in}{3.589632in}}%
\pgfpathlineto{\pgfqpoint{4.691331in}{3.626667in}}%
\pgfpathlineto{\pgfqpoint{4.689648in}{3.628352in}}%
\pgfpathlineto{\pgfqpoint{4.687838in}{3.630199in}}%
\pgfpathlineto{\pgfqpoint{4.670337in}{3.647699in}}%
\pgfpathlineto{\pgfqpoint{4.654338in}{3.664000in}}%
\pgfpathlineto{\pgfqpoint{4.651162in}{3.667171in}}%
\pgfpathlineto{\pgfqpoint{4.647758in}{3.670635in}}%
\pgfpathlineto{\pgfqpoint{4.631836in}{3.686503in}}%
\pgfpathlineto{\pgfqpoint{4.617234in}{3.701333in}}%
\pgfpathlineto{\pgfqpoint{4.607677in}{3.710939in}}%
\pgfpathlineto{\pgfqpoint{4.593272in}{3.725249in}}%
\pgfpathlineto{\pgfqpoint{4.580018in}{3.738667in}}%
\pgfpathlineto{\pgfqpoint{4.574003in}{3.744634in}}%
\pgfpathlineto{\pgfqpoint{4.567596in}{3.751111in}}%
\pgfpathlineto{\pgfqpoint{4.542688in}{3.776000in}}%
\pgfpathlineto{\pgfqpoint{4.527515in}{3.791151in}}%
\pgfpathlineto{\pgfqpoint{4.515952in}{3.802563in}}%
\pgfpathlineto{\pgfqpoint{4.505245in}{3.813333in}}%
\pgfpathlineto{\pgfqpoint{4.487434in}{3.831061in}}%
\pgfpathlineto{\pgfqpoint{4.467686in}{3.850667in}}%
\pgfpathlineto{\pgfqpoint{4.447354in}{3.870839in}}%
\pgfpathlineto{\pgfqpoint{4.438378in}{3.879640in}}%
\pgfpathlineto{\pgfqpoint{4.430012in}{3.888000in}}%
\pgfpathlineto{\pgfqpoint{4.407273in}{3.910488in}}%
\pgfpathlineto{\pgfqpoint{4.399495in}{3.918089in}}%
\pgfpathlineto{\pgfqpoint{4.392222in}{3.925333in}}%
\pgfpathlineto{\pgfqpoint{4.380001in}{3.937264in}}%
\pgfpathlineto{\pgfqpoint{4.367192in}{3.950005in}}%
\pgfpathlineto{\pgfqpoint{4.360548in}{3.956478in}}%
\pgfpathlineto{\pgfqpoint{4.354314in}{3.962667in}}%
\pgfpathlineto{\pgfqpoint{4.327111in}{3.989393in}}%
\pgfpathlineto{\pgfqpoint{4.321536in}{3.994807in}}%
\pgfpathlineto{\pgfqpoint{4.316287in}{4.000000in}}%
\pgfpathlineto{\pgfqpoint{4.287030in}{4.028651in}}%
\pgfpathlineto{\pgfqpoint{4.278142in}{4.037333in}}%
\pgfpathlineto{\pgfqpoint{4.246949in}{4.067780in}}%
\pgfpathlineto{\pgfqpoint{4.243317in}{4.071283in}}%
\pgfpathlineto{\pgfqpoint{4.239876in}{4.074667in}}%
\pgfpathlineto{\pgfqpoint{4.223652in}{4.090299in}}%
\pgfpathlineto{\pgfqpoint{4.206869in}{4.106779in}}%
\pgfpathlineto{\pgfqpoint{4.204110in}{4.109431in}}%
\pgfpathlineto{\pgfqpoint{4.201488in}{4.112000in}}%
\pgfpathlineto{\pgfqpoint{4.184404in}{4.128408in}}%
\pgfpathlineto{\pgfqpoint{4.166788in}{4.145649in}}%
\pgfpathlineto{\pgfqpoint{4.164838in}{4.147517in}}%
\pgfpathlineto{\pgfqpoint{4.162979in}{4.149333in}}%
\pgfpathlineto{\pgfqpoint{4.145091in}{4.166457in}}%
\pgfpathlineto{\pgfqpoint{4.126707in}{4.184391in}}%
\pgfpathlineto{\pgfqpoint{4.125501in}{4.185543in}}%
\pgfpathlineto{\pgfqpoint{4.124347in}{4.186667in}}%
\pgfpathlineto{\pgfqpoint{4.105713in}{4.204445in}}%
\pgfpathlineto{\pgfqpoint{4.086626in}{4.223004in}}%
\pgfpathlineto{\pgfqpoint{4.085590in}{4.224000in}}%
\pgfpathlineto{\pgfqpoint{4.082743in}{4.224000in}}%
\pgfpathlineto{\pgfqpoint{4.086626in}{4.220269in}}%
\pgfpathlineto{\pgfqpoint{4.104276in}{4.203107in}}%
\pgfpathlineto{\pgfqpoint{4.121507in}{4.186667in}}%
\pgfpathlineto{\pgfqpoint{4.124050in}{4.184192in}}%
\pgfpathlineto{\pgfqpoint{4.126707in}{4.181654in}}%
\pgfpathlineto{\pgfqpoint{4.143656in}{4.165120in}}%
\pgfpathlineto{\pgfqpoint{4.160147in}{4.149333in}}%
\pgfpathlineto{\pgfqpoint{4.163389in}{4.146167in}}%
\pgfpathlineto{\pgfqpoint{4.166788in}{4.142910in}}%
\pgfpathlineto{\pgfqpoint{4.182970in}{4.127073in}}%
\pgfpathlineto{\pgfqpoint{4.198664in}{4.112000in}}%
\pgfpathlineto{\pgfqpoint{4.202662in}{4.108082in}}%
\pgfpathlineto{\pgfqpoint{4.206869in}{4.104038in}}%
\pgfpathlineto{\pgfqpoint{4.222219in}{4.088965in}}%
\pgfpathlineto{\pgfqpoint{4.237058in}{4.074667in}}%
\pgfpathlineto{\pgfqpoint{4.241870in}{4.069936in}}%
\pgfpathlineto{\pgfqpoint{4.246949in}{4.065037in}}%
\pgfpathlineto{\pgfqpoint{4.275332in}{4.037333in}}%
\pgfpathlineto{\pgfqpoint{4.287030in}{4.025906in}}%
\pgfpathlineto{\pgfqpoint{4.313485in}{4.000000in}}%
\pgfpathlineto{\pgfqpoint{4.320092in}{3.993462in}}%
\pgfpathlineto{\pgfqpoint{4.327111in}{3.986646in}}%
\pgfpathlineto{\pgfqpoint{4.351518in}{3.962667in}}%
\pgfpathlineto{\pgfqpoint{4.359105in}{3.955134in}}%
\pgfpathlineto{\pgfqpoint{4.367192in}{3.947257in}}%
\pgfpathlineto{\pgfqpoint{4.378574in}{3.935935in}}%
\pgfpathlineto{\pgfqpoint{4.389433in}{3.925333in}}%
\pgfpathlineto{\pgfqpoint{4.398054in}{3.916747in}}%
\pgfpathlineto{\pgfqpoint{4.407273in}{3.907737in}}%
\pgfpathlineto{\pgfqpoint{4.427231in}{3.888000in}}%
\pgfpathlineto{\pgfqpoint{4.436939in}{3.878299in}}%
\pgfpathlineto{\pgfqpoint{4.447354in}{3.868087in}}%
\pgfpathlineto{\pgfqpoint{4.464912in}{3.850667in}}%
\pgfpathlineto{\pgfqpoint{4.487434in}{3.828307in}}%
\pgfpathlineto{\pgfqpoint{4.502478in}{3.813333in}}%
\pgfpathlineto{\pgfqpoint{4.514516in}{3.801225in}}%
\pgfpathlineto{\pgfqpoint{4.527515in}{3.788395in}}%
\pgfpathlineto{\pgfqpoint{4.539928in}{3.776000in}}%
\pgfpathlineto{\pgfqpoint{4.567596in}{3.748353in}}%
\pgfpathlineto{\pgfqpoint{4.572583in}{3.743312in}}%
\pgfpathlineto{\pgfqpoint{4.577265in}{3.738667in}}%
\pgfpathlineto{\pgfqpoint{4.591838in}{3.723913in}}%
\pgfpathlineto{\pgfqpoint{4.607677in}{3.708179in}}%
\pgfpathlineto{\pgfqpoint{4.614488in}{3.701333in}}%
\pgfpathlineto{\pgfqpoint{4.630404in}{3.685169in}}%
\pgfpathlineto{\pgfqpoint{4.647758in}{3.667873in}}%
\pgfpathlineto{\pgfqpoint{4.649745in}{3.665851in}}%
\pgfpathlineto{\pgfqpoint{4.651599in}{3.664000in}}%
\pgfpathlineto{\pgfqpoint{4.668906in}{3.646366in}}%
\pgfpathlineto{\pgfqpoint{4.687838in}{3.627436in}}%
\pgfpathlineto{\pgfqpoint{4.688232in}{3.627034in}}%
\pgfpathlineto{\pgfqpoint{4.688599in}{3.626667in}}%
\pgfpathlineto{\pgfqpoint{4.697368in}{3.617790in}}%
\pgfpathlineto{\pgfqpoint{4.725460in}{3.589333in}}%
\pgfpathlineto{\pgfqpoint{4.726631in}{3.588134in}}%
\pgfpathlineto{\pgfqpoint{4.727919in}{3.586840in}}%
\pgfpathlineto{\pgfqpoint{4.745723in}{3.568583in}}%
\pgfpathlineto{\pgfqpoint{4.762201in}{3.552000in}}%
\pgfpathlineto{\pgfqpoint{4.764959in}{3.549167in}}%
\pgfpathlineto{\pgfqpoint{4.768000in}{3.546102in}}%
\pgfusepath{fill}%
\end{pgfscope}%
\begin{pgfscope}%
\pgfpathrectangle{\pgfqpoint{0.800000in}{0.528000in}}{\pgfqpoint{3.968000in}{3.696000in}}%
\pgfusepath{clip}%
\pgfsetbuttcap%
\pgfsetroundjoin%
\definecolor{currentfill}{rgb}{0.304148,0.764704,0.419943}%
\pgfsetfillcolor{currentfill}%
\pgfsetlinewidth{0.000000pt}%
\definecolor{currentstroke}{rgb}{0.000000,0.000000,0.000000}%
\pgfsetstrokecolor{currentstroke}%
\pgfsetdash{}{0pt}%
\pgfpathmoveto{\pgfqpoint{4.768000in}{3.551696in}}%
\pgfpathlineto{\pgfqpoint{4.767843in}{3.551854in}}%
\pgfpathlineto{\pgfqpoint{4.767701in}{3.552000in}}%
\pgfpathlineto{\pgfqpoint{4.748579in}{3.571244in}}%
\pgfpathlineto{\pgfqpoint{4.730939in}{3.589333in}}%
\pgfpathlineto{\pgfqpoint{4.729486in}{3.590793in}}%
\pgfpathlineto{\pgfqpoint{4.727919in}{3.592397in}}%
\pgfpathlineto{\pgfqpoint{4.694063in}{3.626667in}}%
\pgfpathlineto{\pgfqpoint{4.691064in}{3.629671in}}%
\pgfpathlineto{\pgfqpoint{4.687838in}{3.632963in}}%
\pgfpathlineto{\pgfqpoint{4.671768in}{3.649031in}}%
\pgfpathlineto{\pgfqpoint{4.657077in}{3.664000in}}%
\pgfpathlineto{\pgfqpoint{4.652579in}{3.668491in}}%
\pgfpathlineto{\pgfqpoint{4.647758in}{3.673397in}}%
\pgfpathlineto{\pgfqpoint{4.633268in}{3.687837in}}%
\pgfpathlineto{\pgfqpoint{4.619980in}{3.701333in}}%
\pgfpathlineto{\pgfqpoint{4.607677in}{3.713699in}}%
\pgfpathlineto{\pgfqpoint{4.594705in}{3.726585in}}%
\pgfpathlineto{\pgfqpoint{4.582771in}{3.738667in}}%
\pgfpathlineto{\pgfqpoint{4.575423in}{3.745957in}}%
\pgfpathlineto{\pgfqpoint{4.567596in}{3.753869in}}%
\pgfpathlineto{\pgfqpoint{4.545448in}{3.776000in}}%
\pgfpathlineto{\pgfqpoint{4.527515in}{3.793907in}}%
\pgfpathlineto{\pgfqpoint{4.517389in}{3.803901in}}%
\pgfpathlineto{\pgfqpoint{4.508012in}{3.813333in}}%
\pgfpathlineto{\pgfqpoint{4.487434in}{3.833815in}}%
\pgfpathlineto{\pgfqpoint{4.470461in}{3.850667in}}%
\pgfpathlineto{\pgfqpoint{4.447354in}{3.873592in}}%
\pgfpathlineto{\pgfqpoint{4.439818in}{3.880981in}}%
\pgfpathlineto{\pgfqpoint{4.432794in}{3.888000in}}%
\pgfpathlineto{\pgfqpoint{4.407273in}{3.913238in}}%
\pgfpathlineto{\pgfqpoint{4.400936in}{3.919431in}}%
\pgfpathlineto{\pgfqpoint{4.395010in}{3.925333in}}%
\pgfpathlineto{\pgfqpoint{4.381428in}{3.938593in}}%
\pgfpathlineto{\pgfqpoint{4.367192in}{3.952754in}}%
\pgfpathlineto{\pgfqpoint{4.361990in}{3.957821in}}%
\pgfpathlineto{\pgfqpoint{4.357109in}{3.962667in}}%
\pgfpathlineto{\pgfqpoint{4.327111in}{3.992140in}}%
\pgfpathlineto{\pgfqpoint{4.322979in}{3.996151in}}%
\pgfpathlineto{\pgfqpoint{4.319090in}{4.000000in}}%
\pgfpathlineto{\pgfqpoint{4.287030in}{4.031396in}}%
\pgfpathlineto{\pgfqpoint{4.280952in}{4.037333in}}%
\pgfpathlineto{\pgfqpoint{4.246949in}{4.070523in}}%
\pgfpathlineto{\pgfqpoint{4.244764in}{4.072631in}}%
\pgfpathlineto{\pgfqpoint{4.242693in}{4.074667in}}%
\pgfpathlineto{\pgfqpoint{4.225084in}{4.091634in}}%
\pgfpathlineto{\pgfqpoint{4.206869in}{4.109520in}}%
\pgfpathlineto{\pgfqpoint{4.205558in}{4.110780in}}%
\pgfpathlineto{\pgfqpoint{4.204313in}{4.112000in}}%
\pgfpathlineto{\pgfqpoint{4.185838in}{4.129744in}}%
\pgfpathlineto{\pgfqpoint{4.166788in}{4.148389in}}%
\pgfpathlineto{\pgfqpoint{4.166288in}{4.148868in}}%
\pgfpathlineto{\pgfqpoint{4.165811in}{4.149333in}}%
\pgfpathlineto{\pgfqpoint{4.146526in}{4.167794in}}%
\pgfpathlineto{\pgfqpoint{4.127180in}{4.186667in}}%
\pgfpathlineto{\pgfqpoint{4.126707in}{4.187124in}}%
\pgfpathlineto{\pgfqpoint{4.107150in}{4.205784in}}%
\pgfpathlineto{\pgfqpoint{4.088415in}{4.224000in}}%
\pgfpathlineto{\pgfqpoint{4.086626in}{4.224000in}}%
\pgfpathlineto{\pgfqpoint{4.085590in}{4.224000in}}%
\pgfpathlineto{\pgfqpoint{4.086626in}{4.223004in}}%
\pgfpathlineto{\pgfqpoint{4.105713in}{4.204445in}}%
\pgfpathlineto{\pgfqpoint{4.124347in}{4.186667in}}%
\pgfpathlineto{\pgfqpoint{4.125501in}{4.185543in}}%
\pgfpathlineto{\pgfqpoint{4.126707in}{4.184391in}}%
\pgfpathlineto{\pgfqpoint{4.145091in}{4.166457in}}%
\pgfpathlineto{\pgfqpoint{4.162979in}{4.149333in}}%
\pgfpathlineto{\pgfqpoint{4.164838in}{4.147517in}}%
\pgfpathlineto{\pgfqpoint{4.166788in}{4.145649in}}%
\pgfpathlineto{\pgfqpoint{4.184404in}{4.128408in}}%
\pgfpathlineto{\pgfqpoint{4.201488in}{4.112000in}}%
\pgfpathlineto{\pgfqpoint{4.204110in}{4.109431in}}%
\pgfpathlineto{\pgfqpoint{4.206869in}{4.106779in}}%
\pgfpathlineto{\pgfqpoint{4.223652in}{4.090299in}}%
\pgfpathlineto{\pgfqpoint{4.239876in}{4.074667in}}%
\pgfpathlineto{\pgfqpoint{4.243317in}{4.071283in}}%
\pgfpathlineto{\pgfqpoint{4.246949in}{4.067780in}}%
\pgfpathlineto{\pgfqpoint{4.278142in}{4.037333in}}%
\pgfpathlineto{\pgfqpoint{4.287030in}{4.028651in}}%
\pgfpathlineto{\pgfqpoint{4.316287in}{4.000000in}}%
\pgfpathlineto{\pgfqpoint{4.321536in}{3.994807in}}%
\pgfpathlineto{\pgfqpoint{4.327111in}{3.989393in}}%
\pgfpathlineto{\pgfqpoint{4.354314in}{3.962667in}}%
\pgfpathlineto{\pgfqpoint{4.360548in}{3.956478in}}%
\pgfpathlineto{\pgfqpoint{4.367192in}{3.950005in}}%
\pgfpathlineto{\pgfqpoint{4.380001in}{3.937264in}}%
\pgfpathlineto{\pgfqpoint{4.392222in}{3.925333in}}%
\pgfpathlineto{\pgfqpoint{4.399495in}{3.918089in}}%
\pgfpathlineto{\pgfqpoint{4.407273in}{3.910488in}}%
\pgfpathlineto{\pgfqpoint{4.430012in}{3.888000in}}%
\pgfpathlineto{\pgfqpoint{4.438378in}{3.879640in}}%
\pgfpathlineto{\pgfqpoint{4.447354in}{3.870839in}}%
\pgfpathlineto{\pgfqpoint{4.467686in}{3.850667in}}%
\pgfpathlineto{\pgfqpoint{4.487434in}{3.831061in}}%
\pgfpathlineto{\pgfqpoint{4.505245in}{3.813333in}}%
\pgfpathlineto{\pgfqpoint{4.515952in}{3.802563in}}%
\pgfpathlineto{\pgfqpoint{4.527515in}{3.791151in}}%
\pgfpathlineto{\pgfqpoint{4.542688in}{3.776000in}}%
\pgfpathlineto{\pgfqpoint{4.567596in}{3.751111in}}%
\pgfpathlineto{\pgfqpoint{4.574003in}{3.744634in}}%
\pgfpathlineto{\pgfqpoint{4.580018in}{3.738667in}}%
\pgfpathlineto{\pgfqpoint{4.593272in}{3.725249in}}%
\pgfpathlineto{\pgfqpoint{4.607677in}{3.710939in}}%
\pgfpathlineto{\pgfqpoint{4.617234in}{3.701333in}}%
\pgfpathlineto{\pgfqpoint{4.631836in}{3.686503in}}%
\pgfpathlineto{\pgfqpoint{4.647758in}{3.670635in}}%
\pgfpathlineto{\pgfqpoint{4.651162in}{3.667171in}}%
\pgfpathlineto{\pgfqpoint{4.654338in}{3.664000in}}%
\pgfpathlineto{\pgfqpoint{4.670337in}{3.647699in}}%
\pgfpathlineto{\pgfqpoint{4.687838in}{3.630199in}}%
\pgfpathlineto{\pgfqpoint{4.689648in}{3.628352in}}%
\pgfpathlineto{\pgfqpoint{4.691331in}{3.626667in}}%
\pgfpathlineto{\pgfqpoint{4.727919in}{3.589632in}}%
\pgfpathlineto{\pgfqpoint{4.728072in}{3.589475in}}%
\pgfpathlineto{\pgfqpoint{4.728213in}{3.589333in}}%
\pgfpathlineto{\pgfqpoint{4.747151in}{3.569913in}}%
\pgfpathlineto{\pgfqpoint{4.764951in}{3.552000in}}%
\pgfpathlineto{\pgfqpoint{4.766401in}{3.550511in}}%
\pgfpathlineto{\pgfqpoint{4.768000in}{3.548899in}}%
\pgfusepath{fill}%
\end{pgfscope}%
\begin{pgfscope}%
\pgfpathrectangle{\pgfqpoint{0.800000in}{0.528000in}}{\pgfqpoint{3.968000in}{3.696000in}}%
\pgfusepath{clip}%
\pgfsetbuttcap%
\pgfsetroundjoin%
\definecolor{currentfill}{rgb}{0.304148,0.764704,0.419943}%
\pgfsetfillcolor{currentfill}%
\pgfsetlinewidth{0.000000pt}%
\definecolor{currentstroke}{rgb}{0.000000,0.000000,0.000000}%
\pgfsetstrokecolor{currentstroke}%
\pgfsetdash{}{0pt}%
\pgfpathmoveto{\pgfqpoint{4.768000in}{3.554467in}}%
\pgfpathlineto{\pgfqpoint{4.750007in}{3.572574in}}%
\pgfpathlineto{\pgfqpoint{4.733664in}{3.589333in}}%
\pgfpathlineto{\pgfqpoint{4.730901in}{3.592110in}}%
\pgfpathlineto{\pgfqpoint{4.727919in}{3.595163in}}%
\pgfpathlineto{\pgfqpoint{4.696796in}{3.626667in}}%
\pgfpathlineto{\pgfqpoint{4.692480in}{3.630990in}}%
\pgfpathlineto{\pgfqpoint{4.687838in}{3.635727in}}%
\pgfpathlineto{\pgfqpoint{4.673199in}{3.650364in}}%
\pgfpathlineto{\pgfqpoint{4.659817in}{3.664000in}}%
\pgfpathlineto{\pgfqpoint{4.653997in}{3.669811in}}%
\pgfpathlineto{\pgfqpoint{4.647758in}{3.676159in}}%
\pgfpathlineto{\pgfqpoint{4.634701in}{3.689172in}}%
\pgfpathlineto{\pgfqpoint{4.622726in}{3.701333in}}%
\pgfpathlineto{\pgfqpoint{4.607677in}{3.716459in}}%
\pgfpathlineto{\pgfqpoint{4.596139in}{3.727920in}}%
\pgfpathlineto{\pgfqpoint{4.585524in}{3.738667in}}%
\pgfpathlineto{\pgfqpoint{4.576843in}{3.747280in}}%
\pgfpathlineto{\pgfqpoint{4.567596in}{3.756627in}}%
\pgfpathlineto{\pgfqpoint{4.548208in}{3.776000in}}%
\pgfpathlineto{\pgfqpoint{4.527515in}{3.796663in}}%
\pgfpathlineto{\pgfqpoint{4.518826in}{3.805240in}}%
\pgfpathlineto{\pgfqpoint{4.510779in}{3.813333in}}%
\pgfpathlineto{\pgfqpoint{4.487434in}{3.836569in}}%
\pgfpathlineto{\pgfqpoint{4.473235in}{3.850667in}}%
\pgfpathlineto{\pgfqpoint{4.447354in}{3.876344in}}%
\pgfpathlineto{\pgfqpoint{4.441257in}{3.882322in}}%
\pgfpathlineto{\pgfqpoint{4.435575in}{3.888000in}}%
\pgfpathlineto{\pgfqpoint{4.407273in}{3.915988in}}%
\pgfpathlineto{\pgfqpoint{4.402377in}{3.920773in}}%
\pgfpathlineto{\pgfqpoint{4.397799in}{3.925333in}}%
\pgfpathlineto{\pgfqpoint{4.382855in}{3.939922in}}%
\pgfpathlineto{\pgfqpoint{4.367192in}{3.955502in}}%
\pgfpathlineto{\pgfqpoint{4.363432in}{3.959165in}}%
\pgfpathlineto{\pgfqpoint{4.359905in}{3.962667in}}%
\pgfpathlineto{\pgfqpoint{4.327111in}{3.994886in}}%
\pgfpathlineto{\pgfqpoint{4.324423in}{3.997496in}}%
\pgfpathlineto{\pgfqpoint{4.321893in}{4.000000in}}%
\pgfpathlineto{\pgfqpoint{4.287030in}{4.034141in}}%
\pgfpathlineto{\pgfqpoint{4.283762in}{4.037333in}}%
\pgfpathlineto{\pgfqpoint{4.246949in}{4.073265in}}%
\pgfpathlineto{\pgfqpoint{4.246210in}{4.073978in}}%
\pgfpathlineto{\pgfqpoint{4.245510in}{4.074667in}}%
\pgfpathlineto{\pgfqpoint{4.226517in}{4.092968in}}%
\pgfpathlineto{\pgfqpoint{4.207135in}{4.112000in}}%
\pgfpathlineto{\pgfqpoint{4.206869in}{4.112258in}}%
\pgfpathlineto{\pgfqpoint{4.187272in}{4.131080in}}%
\pgfpathlineto{\pgfqpoint{4.168621in}{4.149333in}}%
\pgfpathlineto{\pgfqpoint{4.166788in}{4.151109in}}%
\pgfpathlineto{\pgfqpoint{4.147962in}{4.169131in}}%
\pgfpathlineto{\pgfqpoint{4.129986in}{4.186667in}}%
\pgfpathlineto{\pgfqpoint{4.126707in}{4.189833in}}%
\pgfpathlineto{\pgfqpoint{4.108587in}{4.207122in}}%
\pgfpathlineto{\pgfqpoint{4.091229in}{4.224000in}}%
\pgfpathlineto{\pgfqpoint{4.088415in}{4.224000in}}%
\pgfpathlineto{\pgfqpoint{4.107150in}{4.205784in}}%
\pgfpathlineto{\pgfqpoint{4.126707in}{4.187124in}}%
\pgfpathlineto{\pgfqpoint{4.127180in}{4.186667in}}%
\pgfpathlineto{\pgfqpoint{4.146526in}{4.167794in}}%
\pgfpathlineto{\pgfqpoint{4.165811in}{4.149333in}}%
\pgfpathlineto{\pgfqpoint{4.166288in}{4.148868in}}%
\pgfpathlineto{\pgfqpoint{4.166788in}{4.148389in}}%
\pgfpathlineto{\pgfqpoint{4.185838in}{4.129744in}}%
\pgfpathlineto{\pgfqpoint{4.204313in}{4.112000in}}%
\pgfpathlineto{\pgfqpoint{4.205558in}{4.110780in}}%
\pgfpathlineto{\pgfqpoint{4.206869in}{4.109520in}}%
\pgfpathlineto{\pgfqpoint{4.225084in}{4.091634in}}%
\pgfpathlineto{\pgfqpoint{4.242693in}{4.074667in}}%
\pgfpathlineto{\pgfqpoint{4.244764in}{4.072631in}}%
\pgfpathlineto{\pgfqpoint{4.246949in}{4.070523in}}%
\pgfpathlineto{\pgfqpoint{4.280952in}{4.037333in}}%
\pgfpathlineto{\pgfqpoint{4.287030in}{4.031396in}}%
\pgfpathlineto{\pgfqpoint{4.319090in}{4.000000in}}%
\pgfpathlineto{\pgfqpoint{4.322979in}{3.996151in}}%
\pgfpathlineto{\pgfqpoint{4.327111in}{3.992140in}}%
\pgfpathlineto{\pgfqpoint{4.357109in}{3.962667in}}%
\pgfpathlineto{\pgfqpoint{4.361990in}{3.957821in}}%
\pgfpathlineto{\pgfqpoint{4.367192in}{3.952754in}}%
\pgfpathlineto{\pgfqpoint{4.381428in}{3.938593in}}%
\pgfpathlineto{\pgfqpoint{4.395010in}{3.925333in}}%
\pgfpathlineto{\pgfqpoint{4.400936in}{3.919431in}}%
\pgfpathlineto{\pgfqpoint{4.407273in}{3.913238in}}%
\pgfpathlineto{\pgfqpoint{4.432794in}{3.888000in}}%
\pgfpathlineto{\pgfqpoint{4.439818in}{3.880981in}}%
\pgfpathlineto{\pgfqpoint{4.447354in}{3.873592in}}%
\pgfpathlineto{\pgfqpoint{4.470461in}{3.850667in}}%
\pgfpathlineto{\pgfqpoint{4.487434in}{3.833815in}}%
\pgfpathlineto{\pgfqpoint{4.508012in}{3.813333in}}%
\pgfpathlineto{\pgfqpoint{4.517389in}{3.803901in}}%
\pgfpathlineto{\pgfqpoint{4.527515in}{3.793907in}}%
\pgfpathlineto{\pgfqpoint{4.545448in}{3.776000in}}%
\pgfpathlineto{\pgfqpoint{4.567596in}{3.753869in}}%
\pgfpathlineto{\pgfqpoint{4.575423in}{3.745957in}}%
\pgfpathlineto{\pgfqpoint{4.582771in}{3.738667in}}%
\pgfpathlineto{\pgfqpoint{4.594705in}{3.726585in}}%
\pgfpathlineto{\pgfqpoint{4.607677in}{3.713699in}}%
\pgfpathlineto{\pgfqpoint{4.619980in}{3.701333in}}%
\pgfpathlineto{\pgfqpoint{4.633268in}{3.687837in}}%
\pgfpathlineto{\pgfqpoint{4.647758in}{3.673397in}}%
\pgfpathlineto{\pgfqpoint{4.652579in}{3.668491in}}%
\pgfpathlineto{\pgfqpoint{4.657077in}{3.664000in}}%
\pgfpathlineto{\pgfqpoint{4.671768in}{3.649031in}}%
\pgfpathlineto{\pgfqpoint{4.687838in}{3.632963in}}%
\pgfpathlineto{\pgfqpoint{4.691064in}{3.629671in}}%
\pgfpathlineto{\pgfqpoint{4.694063in}{3.626667in}}%
\pgfpathlineto{\pgfqpoint{4.727919in}{3.592397in}}%
\pgfpathlineto{\pgfqpoint{4.729486in}{3.590793in}}%
\pgfpathlineto{\pgfqpoint{4.730939in}{3.589333in}}%
\pgfpathlineto{\pgfqpoint{4.748579in}{3.571244in}}%
\pgfpathlineto{\pgfqpoint{4.767701in}{3.552000in}}%
\pgfpathlineto{\pgfqpoint{4.767843in}{3.551854in}}%
\pgfpathlineto{\pgfqpoint{4.768000in}{3.551696in}}%
\pgfpathlineto{\pgfqpoint{4.768000in}{3.552000in}}%
\pgfusepath{fill}%
\end{pgfscope}%
\begin{pgfscope}%
\pgfpathrectangle{\pgfqpoint{0.800000in}{0.528000in}}{\pgfqpoint{3.968000in}{3.696000in}}%
\pgfusepath{clip}%
\pgfsetbuttcap%
\pgfsetroundjoin%
\definecolor{currentfill}{rgb}{0.311925,0.767822,0.415586}%
\pgfsetfillcolor{currentfill}%
\pgfsetlinewidth{0.000000pt}%
\definecolor{currentstroke}{rgb}{0.000000,0.000000,0.000000}%
\pgfsetstrokecolor{currentstroke}%
\pgfsetdash{}{0pt}%
\pgfpathmoveto{\pgfqpoint{4.768000in}{3.557234in}}%
\pgfpathlineto{\pgfqpoint{4.751435in}{3.573904in}}%
\pgfpathlineto{\pgfqpoint{4.736390in}{3.589333in}}%
\pgfpathlineto{\pgfqpoint{4.732315in}{3.593428in}}%
\pgfpathlineto{\pgfqpoint{4.727919in}{3.597929in}}%
\pgfpathlineto{\pgfqpoint{4.699528in}{3.626667in}}%
\pgfpathlineto{\pgfqpoint{4.693896in}{3.632309in}}%
\pgfpathlineto{\pgfqpoint{4.687838in}{3.638491in}}%
\pgfpathlineto{\pgfqpoint{4.674630in}{3.651697in}}%
\pgfpathlineto{\pgfqpoint{4.662556in}{3.664000in}}%
\pgfpathlineto{\pgfqpoint{4.655414in}{3.671131in}}%
\pgfpathlineto{\pgfqpoint{4.647758in}{3.678921in}}%
\pgfpathlineto{\pgfqpoint{4.636133in}{3.690506in}}%
\pgfpathlineto{\pgfqpoint{4.625472in}{3.701333in}}%
\pgfpathlineto{\pgfqpoint{4.607677in}{3.719218in}}%
\pgfpathlineto{\pgfqpoint{4.597573in}{3.729256in}}%
\pgfpathlineto{\pgfqpoint{4.588277in}{3.738667in}}%
\pgfpathlineto{\pgfqpoint{4.578263in}{3.748602in}}%
\pgfpathlineto{\pgfqpoint{4.567596in}{3.759385in}}%
\pgfpathlineto{\pgfqpoint{4.550968in}{3.776000in}}%
\pgfpathlineto{\pgfqpoint{4.527515in}{3.799420in}}%
\pgfpathlineto{\pgfqpoint{4.520262in}{3.806578in}}%
\pgfpathlineto{\pgfqpoint{4.513546in}{3.813333in}}%
\pgfpathlineto{\pgfqpoint{4.487434in}{3.839323in}}%
\pgfpathlineto{\pgfqpoint{4.476009in}{3.850667in}}%
\pgfpathlineto{\pgfqpoint{4.447354in}{3.879096in}}%
\pgfpathlineto{\pgfqpoint{4.442697in}{3.883663in}}%
\pgfpathlineto{\pgfqpoint{4.438356in}{3.888000in}}%
\pgfpathlineto{\pgfqpoint{4.407273in}{3.918739in}}%
\pgfpathlineto{\pgfqpoint{4.403818in}{3.922115in}}%
\pgfpathlineto{\pgfqpoint{4.400587in}{3.925333in}}%
\pgfpathlineto{\pgfqpoint{4.384281in}{3.941251in}}%
\pgfpathlineto{\pgfqpoint{4.367192in}{3.958251in}}%
\pgfpathlineto{\pgfqpoint{4.364875in}{3.960508in}}%
\pgfpathlineto{\pgfqpoint{4.362701in}{3.962667in}}%
\pgfpathlineto{\pgfqpoint{4.327111in}{3.997633in}}%
\pgfpathlineto{\pgfqpoint{4.325867in}{3.998841in}}%
\pgfpathlineto{\pgfqpoint{4.324696in}{4.000000in}}%
\pgfpathlineto{\pgfqpoint{4.287030in}{4.036885in}}%
\pgfpathlineto{\pgfqpoint{4.286572in}{4.037333in}}%
\pgfpathlineto{\pgfqpoint{4.277005in}{4.046672in}}%
\pgfpathlineto{\pgfqpoint{4.248311in}{4.074667in}}%
\pgfpathlineto{\pgfqpoint{4.246949in}{4.075995in}}%
\pgfpathlineto{\pgfqpoint{4.227950in}{4.094303in}}%
\pgfpathlineto{\pgfqpoint{4.209926in}{4.112000in}}%
\pgfpathlineto{\pgfqpoint{4.206869in}{4.114971in}}%
\pgfpathlineto{\pgfqpoint{4.188706in}{4.132415in}}%
\pgfpathlineto{\pgfqpoint{4.171420in}{4.149333in}}%
\pgfpathlineto{\pgfqpoint{4.166788in}{4.153820in}}%
\pgfpathlineto{\pgfqpoint{4.149397in}{4.170468in}}%
\pgfpathlineto{\pgfqpoint{4.132792in}{4.186667in}}%
\pgfpathlineto{\pgfqpoint{4.126707in}{4.192542in}}%
\pgfpathlineto{\pgfqpoint{4.110024in}{4.208460in}}%
\pgfpathlineto{\pgfqpoint{4.094042in}{4.224000in}}%
\pgfpathlineto{\pgfqpoint{4.091229in}{4.224000in}}%
\pgfpathlineto{\pgfqpoint{4.108587in}{4.207122in}}%
\pgfpathlineto{\pgfqpoint{4.126707in}{4.189833in}}%
\pgfpathlineto{\pgfqpoint{4.129986in}{4.186667in}}%
\pgfpathlineto{\pgfqpoint{4.147962in}{4.169131in}}%
\pgfpathlineto{\pgfqpoint{4.166788in}{4.151109in}}%
\pgfpathlineto{\pgfqpoint{4.168621in}{4.149333in}}%
\pgfpathlineto{\pgfqpoint{4.187272in}{4.131080in}}%
\pgfpathlineto{\pgfqpoint{4.206869in}{4.112258in}}%
\pgfpathlineto{\pgfqpoint{4.207135in}{4.112000in}}%
\pgfpathlineto{\pgfqpoint{4.226517in}{4.092968in}}%
\pgfpathlineto{\pgfqpoint{4.245510in}{4.074667in}}%
\pgfpathlineto{\pgfqpoint{4.246210in}{4.073978in}}%
\pgfpathlineto{\pgfqpoint{4.246949in}{4.073265in}}%
\pgfpathlineto{\pgfqpoint{4.283762in}{4.037333in}}%
\pgfpathlineto{\pgfqpoint{4.287030in}{4.034141in}}%
\pgfpathlineto{\pgfqpoint{4.321893in}{4.000000in}}%
\pgfpathlineto{\pgfqpoint{4.324423in}{3.997496in}}%
\pgfpathlineto{\pgfqpoint{4.327111in}{3.994886in}}%
\pgfpathlineto{\pgfqpoint{4.359905in}{3.962667in}}%
\pgfpathlineto{\pgfqpoint{4.363432in}{3.959165in}}%
\pgfpathlineto{\pgfqpoint{4.367192in}{3.955502in}}%
\pgfpathlineto{\pgfqpoint{4.382855in}{3.939922in}}%
\pgfpathlineto{\pgfqpoint{4.397799in}{3.925333in}}%
\pgfpathlineto{\pgfqpoint{4.402377in}{3.920773in}}%
\pgfpathlineto{\pgfqpoint{4.407273in}{3.915988in}}%
\pgfpathlineto{\pgfqpoint{4.435575in}{3.888000in}}%
\pgfpathlineto{\pgfqpoint{4.441257in}{3.882322in}}%
\pgfpathlineto{\pgfqpoint{4.447354in}{3.876344in}}%
\pgfpathlineto{\pgfqpoint{4.473235in}{3.850667in}}%
\pgfpathlineto{\pgfqpoint{4.487434in}{3.836569in}}%
\pgfpathlineto{\pgfqpoint{4.510779in}{3.813333in}}%
\pgfpathlineto{\pgfqpoint{4.518826in}{3.805240in}}%
\pgfpathlineto{\pgfqpoint{4.527515in}{3.796663in}}%
\pgfpathlineto{\pgfqpoint{4.548208in}{3.776000in}}%
\pgfpathlineto{\pgfqpoint{4.567596in}{3.756627in}}%
\pgfpathlineto{\pgfqpoint{4.576843in}{3.747280in}}%
\pgfpathlineto{\pgfqpoint{4.585524in}{3.738667in}}%
\pgfpathlineto{\pgfqpoint{4.596139in}{3.727920in}}%
\pgfpathlineto{\pgfqpoint{4.607677in}{3.716459in}}%
\pgfpathlineto{\pgfqpoint{4.622726in}{3.701333in}}%
\pgfpathlineto{\pgfqpoint{4.634701in}{3.689172in}}%
\pgfpathlineto{\pgfqpoint{4.647758in}{3.676159in}}%
\pgfpathlineto{\pgfqpoint{4.653997in}{3.669811in}}%
\pgfpathlineto{\pgfqpoint{4.659817in}{3.664000in}}%
\pgfpathlineto{\pgfqpoint{4.673199in}{3.650364in}}%
\pgfpathlineto{\pgfqpoint{4.687838in}{3.635727in}}%
\pgfpathlineto{\pgfqpoint{4.692480in}{3.630990in}}%
\pgfpathlineto{\pgfqpoint{4.696796in}{3.626667in}}%
\pgfpathlineto{\pgfqpoint{4.727919in}{3.595163in}}%
\pgfpathlineto{\pgfqpoint{4.730901in}{3.592110in}}%
\pgfpathlineto{\pgfqpoint{4.733664in}{3.589333in}}%
\pgfpathlineto{\pgfqpoint{4.750007in}{3.572574in}}%
\pgfpathlineto{\pgfqpoint{4.768000in}{3.554467in}}%
\pgfusepath{fill}%
\end{pgfscope}%
\begin{pgfscope}%
\pgfpathrectangle{\pgfqpoint{0.800000in}{0.528000in}}{\pgfqpoint{3.968000in}{3.696000in}}%
\pgfusepath{clip}%
\pgfsetbuttcap%
\pgfsetroundjoin%
\definecolor{currentfill}{rgb}{0.311925,0.767822,0.415586}%
\pgfsetfillcolor{currentfill}%
\pgfsetlinewidth{0.000000pt}%
\definecolor{currentstroke}{rgb}{0.000000,0.000000,0.000000}%
\pgfsetstrokecolor{currentstroke}%
\pgfsetdash{}{0pt}%
\pgfpathmoveto{\pgfqpoint{4.768000in}{3.560002in}}%
\pgfpathlineto{\pgfqpoint{4.752864in}{3.575235in}}%
\pgfpathlineto{\pgfqpoint{4.739115in}{3.589333in}}%
\pgfpathlineto{\pgfqpoint{4.733730in}{3.594745in}}%
\pgfpathlineto{\pgfqpoint{4.727919in}{3.600694in}}%
\pgfpathlineto{\pgfqpoint{4.702260in}{3.626667in}}%
\pgfpathlineto{\pgfqpoint{4.695311in}{3.633627in}}%
\pgfpathlineto{\pgfqpoint{4.687838in}{3.641254in}}%
\pgfpathlineto{\pgfqpoint{4.676061in}{3.653030in}}%
\pgfpathlineto{\pgfqpoint{4.665295in}{3.664000in}}%
\pgfpathlineto{\pgfqpoint{4.656831in}{3.672452in}}%
\pgfpathlineto{\pgfqpoint{4.647758in}{3.681682in}}%
\pgfpathlineto{\pgfqpoint{4.637566in}{3.691840in}}%
\pgfpathlineto{\pgfqpoint{4.628218in}{3.701333in}}%
\pgfpathlineto{\pgfqpoint{4.607677in}{3.721978in}}%
\pgfpathlineto{\pgfqpoint{4.599007in}{3.730591in}}%
\pgfpathlineto{\pgfqpoint{4.591030in}{3.738667in}}%
\pgfpathlineto{\pgfqpoint{4.579683in}{3.749925in}}%
\pgfpathlineto{\pgfqpoint{4.567596in}{3.762143in}}%
\pgfpathlineto{\pgfqpoint{4.553728in}{3.776000in}}%
\pgfpathlineto{\pgfqpoint{4.527515in}{3.802176in}}%
\pgfpathlineto{\pgfqpoint{4.521699in}{3.807916in}}%
\pgfpathlineto{\pgfqpoint{4.516313in}{3.813333in}}%
\pgfpathlineto{\pgfqpoint{4.487434in}{3.842078in}}%
\pgfpathlineto{\pgfqpoint{4.478783in}{3.850667in}}%
\pgfpathlineto{\pgfqpoint{4.447354in}{3.881849in}}%
\pgfpathlineto{\pgfqpoint{4.444136in}{3.885003in}}%
\pgfpathlineto{\pgfqpoint{4.441138in}{3.888000in}}%
\pgfpathlineto{\pgfqpoint{4.407273in}{3.921489in}}%
\pgfpathlineto{\pgfqpoint{4.405259in}{3.923458in}}%
\pgfpathlineto{\pgfqpoint{4.403376in}{3.925333in}}%
\pgfpathlineto{\pgfqpoint{4.385708in}{3.942581in}}%
\pgfpathlineto{\pgfqpoint{4.367192in}{3.961000in}}%
\pgfpathlineto{\pgfqpoint{4.366317in}{3.961852in}}%
\pgfpathlineto{\pgfqpoint{4.365496in}{3.962667in}}%
\pgfpathlineto{\pgfqpoint{4.334551in}{3.993070in}}%
\pgfpathlineto{\pgfqpoint{4.327494in}{4.000000in}}%
\pgfpathlineto{\pgfqpoint{4.327307in}{4.000182in}}%
\pgfpathlineto{\pgfqpoint{4.327111in}{4.000376in}}%
\pgfpathlineto{\pgfqpoint{4.289354in}{4.037333in}}%
\pgfpathlineto{\pgfqpoint{4.287030in}{4.039607in}}%
\pgfpathlineto{\pgfqpoint{4.251096in}{4.074667in}}%
\pgfpathlineto{\pgfqpoint{4.246949in}{4.078709in}}%
\pgfpathlineto{\pgfqpoint{4.229382in}{4.095637in}}%
\pgfpathlineto{\pgfqpoint{4.212718in}{4.112000in}}%
\pgfpathlineto{\pgfqpoint{4.206869in}{4.117684in}}%
\pgfpathlineto{\pgfqpoint{4.190140in}{4.133751in}}%
\pgfpathlineto{\pgfqpoint{4.174219in}{4.149333in}}%
\pgfpathlineto{\pgfqpoint{4.166788in}{4.156531in}}%
\pgfpathlineto{\pgfqpoint{4.150833in}{4.171805in}}%
\pgfpathlineto{\pgfqpoint{4.135598in}{4.186667in}}%
\pgfpathlineto{\pgfqpoint{4.126707in}{4.195251in}}%
\pgfpathlineto{\pgfqpoint{4.111461in}{4.209799in}}%
\pgfpathlineto{\pgfqpoint{4.096855in}{4.224000in}}%
\pgfpathlineto{\pgfqpoint{4.094042in}{4.224000in}}%
\pgfpathlineto{\pgfqpoint{4.110024in}{4.208460in}}%
\pgfpathlineto{\pgfqpoint{4.126707in}{4.192542in}}%
\pgfpathlineto{\pgfqpoint{4.132792in}{4.186667in}}%
\pgfpathlineto{\pgfqpoint{4.149397in}{4.170468in}}%
\pgfpathlineto{\pgfqpoint{4.166788in}{4.153820in}}%
\pgfpathlineto{\pgfqpoint{4.171420in}{4.149333in}}%
\pgfpathlineto{\pgfqpoint{4.188706in}{4.132415in}}%
\pgfpathlineto{\pgfqpoint{4.206869in}{4.114971in}}%
\pgfpathlineto{\pgfqpoint{4.209926in}{4.112000in}}%
\pgfpathlineto{\pgfqpoint{4.227950in}{4.094303in}}%
\pgfpathlineto{\pgfqpoint{4.246949in}{4.075995in}}%
\pgfpathlineto{\pgfqpoint{4.248311in}{4.074667in}}%
\pgfpathlineto{\pgfqpoint{4.277005in}{4.046672in}}%
\pgfpathlineto{\pgfqpoint{4.286572in}{4.037333in}}%
\pgfpathlineto{\pgfqpoint{4.287030in}{4.036885in}}%
\pgfpathlineto{\pgfqpoint{4.324696in}{4.000000in}}%
\pgfpathlineto{\pgfqpoint{4.325867in}{3.998841in}}%
\pgfpathlineto{\pgfqpoint{4.327111in}{3.997633in}}%
\pgfpathlineto{\pgfqpoint{4.362701in}{3.962667in}}%
\pgfpathlineto{\pgfqpoint{4.364875in}{3.960508in}}%
\pgfpathlineto{\pgfqpoint{4.367192in}{3.958251in}}%
\pgfpathlineto{\pgfqpoint{4.384281in}{3.941251in}}%
\pgfpathlineto{\pgfqpoint{4.400587in}{3.925333in}}%
\pgfpathlineto{\pgfqpoint{4.403818in}{3.922115in}}%
\pgfpathlineto{\pgfqpoint{4.407273in}{3.918739in}}%
\pgfpathlineto{\pgfqpoint{4.438356in}{3.888000in}}%
\pgfpathlineto{\pgfqpoint{4.442697in}{3.883663in}}%
\pgfpathlineto{\pgfqpoint{4.447354in}{3.879096in}}%
\pgfpathlineto{\pgfqpoint{4.476009in}{3.850667in}}%
\pgfpathlineto{\pgfqpoint{4.487434in}{3.839323in}}%
\pgfpathlineto{\pgfqpoint{4.513546in}{3.813333in}}%
\pgfpathlineto{\pgfqpoint{4.520262in}{3.806578in}}%
\pgfpathlineto{\pgfqpoint{4.527515in}{3.799420in}}%
\pgfpathlineto{\pgfqpoint{4.550968in}{3.776000in}}%
\pgfpathlineto{\pgfqpoint{4.567596in}{3.759385in}}%
\pgfpathlineto{\pgfqpoint{4.578263in}{3.748602in}}%
\pgfpathlineto{\pgfqpoint{4.588277in}{3.738667in}}%
\pgfpathlineto{\pgfqpoint{4.597573in}{3.729256in}}%
\pgfpathlineto{\pgfqpoint{4.607677in}{3.719218in}}%
\pgfpathlineto{\pgfqpoint{4.625472in}{3.701333in}}%
\pgfpathlineto{\pgfqpoint{4.636133in}{3.690506in}}%
\pgfpathlineto{\pgfqpoint{4.647758in}{3.678921in}}%
\pgfpathlineto{\pgfqpoint{4.655414in}{3.671131in}}%
\pgfpathlineto{\pgfqpoint{4.662556in}{3.664000in}}%
\pgfpathlineto{\pgfqpoint{4.674630in}{3.651697in}}%
\pgfpathlineto{\pgfqpoint{4.687838in}{3.638491in}}%
\pgfpathlineto{\pgfqpoint{4.693896in}{3.632309in}}%
\pgfpathlineto{\pgfqpoint{4.699528in}{3.626667in}}%
\pgfpathlineto{\pgfqpoint{4.727919in}{3.597929in}}%
\pgfpathlineto{\pgfqpoint{4.732315in}{3.593428in}}%
\pgfpathlineto{\pgfqpoint{4.736390in}{3.589333in}}%
\pgfpathlineto{\pgfqpoint{4.751435in}{3.573904in}}%
\pgfpathlineto{\pgfqpoint{4.768000in}{3.557234in}}%
\pgfusepath{fill}%
\end{pgfscope}%
\begin{pgfscope}%
\pgfpathrectangle{\pgfqpoint{0.800000in}{0.528000in}}{\pgfqpoint{3.968000in}{3.696000in}}%
\pgfusepath{clip}%
\pgfsetbuttcap%
\pgfsetroundjoin%
\definecolor{currentfill}{rgb}{0.311925,0.767822,0.415586}%
\pgfsetfillcolor{currentfill}%
\pgfsetlinewidth{0.000000pt}%
\definecolor{currentstroke}{rgb}{0.000000,0.000000,0.000000}%
\pgfsetstrokecolor{currentstroke}%
\pgfsetdash{}{0pt}%
\pgfpathmoveto{\pgfqpoint{4.768000in}{3.562769in}}%
\pgfpathlineto{\pgfqpoint{4.754292in}{3.576565in}}%
\pgfpathlineto{\pgfqpoint{4.741840in}{3.589333in}}%
\pgfpathlineto{\pgfqpoint{4.735144in}{3.596063in}}%
\pgfpathlineto{\pgfqpoint{4.727919in}{3.603460in}}%
\pgfpathlineto{\pgfqpoint{4.704992in}{3.626667in}}%
\pgfpathlineto{\pgfqpoint{4.696727in}{3.634946in}}%
\pgfpathlineto{\pgfqpoint{4.687838in}{3.644018in}}%
\pgfpathlineto{\pgfqpoint{4.677492in}{3.654363in}}%
\pgfpathlineto{\pgfqpoint{4.668034in}{3.664000in}}%
\pgfpathlineto{\pgfqpoint{4.658248in}{3.673772in}}%
\pgfpathlineto{\pgfqpoint{4.647758in}{3.684444in}}%
\pgfpathlineto{\pgfqpoint{4.638998in}{3.693174in}}%
\pgfpathlineto{\pgfqpoint{4.630965in}{3.701333in}}%
\pgfpathlineto{\pgfqpoint{4.607677in}{3.724738in}}%
\pgfpathlineto{\pgfqpoint{4.600441in}{3.731927in}}%
\pgfpathlineto{\pgfqpoint{4.593783in}{3.738667in}}%
\pgfpathlineto{\pgfqpoint{4.581103in}{3.751247in}}%
\pgfpathlineto{\pgfqpoint{4.567596in}{3.764901in}}%
\pgfpathlineto{\pgfqpoint{4.556488in}{3.776000in}}%
\pgfpathlineto{\pgfqpoint{4.527515in}{3.804932in}}%
\pgfpathlineto{\pgfqpoint{4.523136in}{3.809254in}}%
\pgfpathlineto{\pgfqpoint{4.519080in}{3.813333in}}%
\pgfpathlineto{\pgfqpoint{4.487434in}{3.844832in}}%
\pgfpathlineto{\pgfqpoint{4.481557in}{3.850667in}}%
\pgfpathlineto{\pgfqpoint{4.447354in}{3.884601in}}%
\pgfpathlineto{\pgfqpoint{4.445576in}{3.886344in}}%
\pgfpathlineto{\pgfqpoint{4.443919in}{3.888000in}}%
\pgfpathlineto{\pgfqpoint{4.407273in}{3.924240in}}%
\pgfpathlineto{\pgfqpoint{4.406700in}{3.924800in}}%
\pgfpathlineto{\pgfqpoint{4.406164in}{3.925333in}}%
\pgfpathlineto{\pgfqpoint{4.387135in}{3.943910in}}%
\pgfpathlineto{\pgfqpoint{4.368279in}{3.962667in}}%
\pgfpathlineto{\pgfqpoint{4.367192in}{3.963737in}}%
\pgfpathlineto{\pgfqpoint{4.330264in}{4.000000in}}%
\pgfpathlineto{\pgfqpoint{4.328721in}{4.001500in}}%
\pgfpathlineto{\pgfqpoint{4.327111in}{4.003094in}}%
\pgfpathlineto{\pgfqpoint{4.292132in}{4.037333in}}%
\pgfpathlineto{\pgfqpoint{4.287030in}{4.042323in}}%
\pgfpathlineto{\pgfqpoint{4.253880in}{4.074667in}}%
\pgfpathlineto{\pgfqpoint{4.246949in}{4.081424in}}%
\pgfpathlineto{\pgfqpoint{4.230815in}{4.096971in}}%
\pgfpathlineto{\pgfqpoint{4.215509in}{4.112000in}}%
\pgfpathlineto{\pgfqpoint{4.206869in}{4.120397in}}%
\pgfpathlineto{\pgfqpoint{4.191574in}{4.135087in}}%
\pgfpathlineto{\pgfqpoint{4.177017in}{4.149333in}}%
\pgfpathlineto{\pgfqpoint{4.166788in}{4.159242in}}%
\pgfpathlineto{\pgfqpoint{4.152268in}{4.173142in}}%
\pgfpathlineto{\pgfqpoint{4.138404in}{4.186667in}}%
\pgfpathlineto{\pgfqpoint{4.126707in}{4.197961in}}%
\pgfpathlineto{\pgfqpoint{4.112897in}{4.211137in}}%
\pgfpathlineto{\pgfqpoint{4.099668in}{4.224000in}}%
\pgfpathlineto{\pgfqpoint{4.096855in}{4.224000in}}%
\pgfpathlineto{\pgfqpoint{4.111461in}{4.209799in}}%
\pgfpathlineto{\pgfqpoint{4.126707in}{4.195251in}}%
\pgfpathlineto{\pgfqpoint{4.135598in}{4.186667in}}%
\pgfpathlineto{\pgfqpoint{4.150833in}{4.171805in}}%
\pgfpathlineto{\pgfqpoint{4.166788in}{4.156531in}}%
\pgfpathlineto{\pgfqpoint{4.174219in}{4.149333in}}%
\pgfpathlineto{\pgfqpoint{4.190140in}{4.133751in}}%
\pgfpathlineto{\pgfqpoint{4.206869in}{4.117684in}}%
\pgfpathlineto{\pgfqpoint{4.212718in}{4.112000in}}%
\pgfpathlineto{\pgfqpoint{4.229382in}{4.095637in}}%
\pgfpathlineto{\pgfqpoint{4.246949in}{4.078709in}}%
\pgfpathlineto{\pgfqpoint{4.251096in}{4.074667in}}%
\pgfpathlineto{\pgfqpoint{4.287030in}{4.039607in}}%
\pgfpathlineto{\pgfqpoint{4.289354in}{4.037333in}}%
\pgfpathlineto{\pgfqpoint{4.327111in}{4.000376in}}%
\pgfpathlineto{\pgfqpoint{4.327307in}{4.000182in}}%
\pgfpathlineto{\pgfqpoint{4.327494in}{4.000000in}}%
\pgfpathlineto{\pgfqpoint{4.334551in}{3.993070in}}%
\pgfpathlineto{\pgfqpoint{4.365496in}{3.962667in}}%
\pgfpathlineto{\pgfqpoint{4.366317in}{3.961852in}}%
\pgfpathlineto{\pgfqpoint{4.367192in}{3.961000in}}%
\pgfpathlineto{\pgfqpoint{4.385708in}{3.942581in}}%
\pgfpathlineto{\pgfqpoint{4.403376in}{3.925333in}}%
\pgfpathlineto{\pgfqpoint{4.405259in}{3.923458in}}%
\pgfpathlineto{\pgfqpoint{4.407273in}{3.921489in}}%
\pgfpathlineto{\pgfqpoint{4.441138in}{3.888000in}}%
\pgfpathlineto{\pgfqpoint{4.444136in}{3.885003in}}%
\pgfpathlineto{\pgfqpoint{4.447354in}{3.881849in}}%
\pgfpathlineto{\pgfqpoint{4.478783in}{3.850667in}}%
\pgfpathlineto{\pgfqpoint{4.487434in}{3.842078in}}%
\pgfpathlineto{\pgfqpoint{4.516313in}{3.813333in}}%
\pgfpathlineto{\pgfqpoint{4.521699in}{3.807916in}}%
\pgfpathlineto{\pgfqpoint{4.527515in}{3.802176in}}%
\pgfpathlineto{\pgfqpoint{4.553728in}{3.776000in}}%
\pgfpathlineto{\pgfqpoint{4.567596in}{3.762143in}}%
\pgfpathlineto{\pgfqpoint{4.579683in}{3.749925in}}%
\pgfpathlineto{\pgfqpoint{4.591030in}{3.738667in}}%
\pgfpathlineto{\pgfqpoint{4.599007in}{3.730591in}}%
\pgfpathlineto{\pgfqpoint{4.607677in}{3.721978in}}%
\pgfpathlineto{\pgfqpoint{4.628218in}{3.701333in}}%
\pgfpathlineto{\pgfqpoint{4.637566in}{3.691840in}}%
\pgfpathlineto{\pgfqpoint{4.647758in}{3.681682in}}%
\pgfpathlineto{\pgfqpoint{4.656831in}{3.672452in}}%
\pgfpathlineto{\pgfqpoint{4.665295in}{3.664000in}}%
\pgfpathlineto{\pgfqpoint{4.676061in}{3.653030in}}%
\pgfpathlineto{\pgfqpoint{4.687838in}{3.641254in}}%
\pgfpathlineto{\pgfqpoint{4.695311in}{3.633627in}}%
\pgfpathlineto{\pgfqpoint{4.702260in}{3.626667in}}%
\pgfpathlineto{\pgfqpoint{4.727919in}{3.600694in}}%
\pgfpathlineto{\pgfqpoint{4.733730in}{3.594745in}}%
\pgfpathlineto{\pgfqpoint{4.739115in}{3.589333in}}%
\pgfpathlineto{\pgfqpoint{4.752864in}{3.575235in}}%
\pgfpathlineto{\pgfqpoint{4.768000in}{3.560002in}}%
\pgfusepath{fill}%
\end{pgfscope}%
\begin{pgfscope}%
\pgfpathrectangle{\pgfqpoint{0.800000in}{0.528000in}}{\pgfqpoint{3.968000in}{3.696000in}}%
\pgfusepath{clip}%
\pgfsetbuttcap%
\pgfsetroundjoin%
\definecolor{currentfill}{rgb}{0.319809,0.770914,0.411152}%
\pgfsetfillcolor{currentfill}%
\pgfsetlinewidth{0.000000pt}%
\definecolor{currentstroke}{rgb}{0.000000,0.000000,0.000000}%
\pgfsetstrokecolor{currentstroke}%
\pgfsetdash{}{0pt}%
\pgfpathmoveto{\pgfqpoint{4.768000in}{3.565537in}}%
\pgfpathlineto{\pgfqpoint{4.755720in}{3.577895in}}%
\pgfpathlineto{\pgfqpoint{4.744566in}{3.589333in}}%
\pgfpathlineto{\pgfqpoint{4.736558in}{3.597380in}}%
\pgfpathlineto{\pgfqpoint{4.727919in}{3.606226in}}%
\pgfpathlineto{\pgfqpoint{4.707725in}{3.626667in}}%
\pgfpathlineto{\pgfqpoint{4.698143in}{3.636265in}}%
\pgfpathlineto{\pgfqpoint{4.687838in}{3.646782in}}%
\pgfpathlineto{\pgfqpoint{4.678923in}{3.655696in}}%
\pgfpathlineto{\pgfqpoint{4.670773in}{3.664000in}}%
\pgfpathlineto{\pgfqpoint{4.659665in}{3.675092in}}%
\pgfpathlineto{\pgfqpoint{4.647758in}{3.687206in}}%
\pgfpathlineto{\pgfqpoint{4.640431in}{3.694509in}}%
\pgfpathlineto{\pgfqpoint{4.633711in}{3.701333in}}%
\pgfpathlineto{\pgfqpoint{4.607677in}{3.727498in}}%
\pgfpathlineto{\pgfqpoint{4.601875in}{3.733262in}}%
\pgfpathlineto{\pgfqpoint{4.596536in}{3.738667in}}%
\pgfpathlineto{\pgfqpoint{4.582523in}{3.752570in}}%
\pgfpathlineto{\pgfqpoint{4.567596in}{3.767659in}}%
\pgfpathlineto{\pgfqpoint{4.559249in}{3.776000in}}%
\pgfpathlineto{\pgfqpoint{4.527515in}{3.807688in}}%
\pgfpathlineto{\pgfqpoint{4.524572in}{3.810592in}}%
\pgfpathlineto{\pgfqpoint{4.521847in}{3.813333in}}%
\pgfpathlineto{\pgfqpoint{4.487434in}{3.847586in}}%
\pgfpathlineto{\pgfqpoint{4.484331in}{3.850667in}}%
\pgfpathlineto{\pgfqpoint{4.447354in}{3.887353in}}%
\pgfpathlineto{\pgfqpoint{4.447015in}{3.887685in}}%
\pgfpathlineto{\pgfqpoint{4.446700in}{3.888000in}}%
\pgfpathlineto{\pgfqpoint{4.436108in}{3.898475in}}%
\pgfpathlineto{\pgfqpoint{4.408933in}{3.925333in}}%
\pgfpathlineto{\pgfqpoint{4.407273in}{3.926973in}}%
\pgfpathlineto{\pgfqpoint{4.388562in}{3.945239in}}%
\pgfpathlineto{\pgfqpoint{4.371042in}{3.962667in}}%
\pgfpathlineto{\pgfqpoint{4.367192in}{3.966457in}}%
\pgfpathlineto{\pgfqpoint{4.333034in}{4.000000in}}%
\pgfpathlineto{\pgfqpoint{4.330136in}{4.002817in}}%
\pgfpathlineto{\pgfqpoint{4.327111in}{4.005813in}}%
\pgfpathlineto{\pgfqpoint{4.294909in}{4.037333in}}%
\pgfpathlineto{\pgfqpoint{4.287030in}{4.045040in}}%
\pgfpathlineto{\pgfqpoint{4.256665in}{4.074667in}}%
\pgfpathlineto{\pgfqpoint{4.246949in}{4.084139in}}%
\pgfpathlineto{\pgfqpoint{4.232247in}{4.098306in}}%
\pgfpathlineto{\pgfqpoint{4.218301in}{4.112000in}}%
\pgfpathlineto{\pgfqpoint{4.206869in}{4.123110in}}%
\pgfpathlineto{\pgfqpoint{4.193008in}{4.136422in}}%
\pgfpathlineto{\pgfqpoint{4.179816in}{4.149333in}}%
\pgfpathlineto{\pgfqpoint{4.166788in}{4.161954in}}%
\pgfpathlineto{\pgfqpoint{4.153703in}{4.174479in}}%
\pgfpathlineto{\pgfqpoint{4.141210in}{4.186667in}}%
\pgfpathlineto{\pgfqpoint{4.126707in}{4.200670in}}%
\pgfpathlineto{\pgfqpoint{4.114334in}{4.212475in}}%
\pgfpathlineto{\pgfqpoint{4.102481in}{4.224000in}}%
\pgfpathlineto{\pgfqpoint{4.099668in}{4.224000in}}%
\pgfpathlineto{\pgfqpoint{4.112897in}{4.211137in}}%
\pgfpathlineto{\pgfqpoint{4.126707in}{4.197961in}}%
\pgfpathlineto{\pgfqpoint{4.138404in}{4.186667in}}%
\pgfpathlineto{\pgfqpoint{4.152268in}{4.173142in}}%
\pgfpathlineto{\pgfqpoint{4.166788in}{4.159242in}}%
\pgfpathlineto{\pgfqpoint{4.177017in}{4.149333in}}%
\pgfpathlineto{\pgfqpoint{4.191574in}{4.135087in}}%
\pgfpathlineto{\pgfqpoint{4.206869in}{4.120397in}}%
\pgfpathlineto{\pgfqpoint{4.215509in}{4.112000in}}%
\pgfpathlineto{\pgfqpoint{4.230815in}{4.096971in}}%
\pgfpathlineto{\pgfqpoint{4.246949in}{4.081424in}}%
\pgfpathlineto{\pgfqpoint{4.253880in}{4.074667in}}%
\pgfpathlineto{\pgfqpoint{4.287030in}{4.042323in}}%
\pgfpathlineto{\pgfqpoint{4.292132in}{4.037333in}}%
\pgfpathlineto{\pgfqpoint{4.327111in}{4.003094in}}%
\pgfpathlineto{\pgfqpoint{4.328721in}{4.001500in}}%
\pgfpathlineto{\pgfqpoint{4.330264in}{4.000000in}}%
\pgfpathlineto{\pgfqpoint{4.367192in}{3.963737in}}%
\pgfpathlineto{\pgfqpoint{4.368279in}{3.962667in}}%
\pgfpathlineto{\pgfqpoint{4.387135in}{3.943910in}}%
\pgfpathlineto{\pgfqpoint{4.406164in}{3.925333in}}%
\pgfpathlineto{\pgfqpoint{4.406700in}{3.924800in}}%
\pgfpathlineto{\pgfqpoint{4.407273in}{3.924240in}}%
\pgfpathlineto{\pgfqpoint{4.443919in}{3.888000in}}%
\pgfpathlineto{\pgfqpoint{4.445576in}{3.886344in}}%
\pgfpathlineto{\pgfqpoint{4.447354in}{3.884601in}}%
\pgfpathlineto{\pgfqpoint{4.481557in}{3.850667in}}%
\pgfpathlineto{\pgfqpoint{4.487434in}{3.844832in}}%
\pgfpathlineto{\pgfqpoint{4.519080in}{3.813333in}}%
\pgfpathlineto{\pgfqpoint{4.523136in}{3.809254in}}%
\pgfpathlineto{\pgfqpoint{4.527515in}{3.804932in}}%
\pgfpathlineto{\pgfqpoint{4.556488in}{3.776000in}}%
\pgfpathlineto{\pgfqpoint{4.567596in}{3.764901in}}%
\pgfpathlineto{\pgfqpoint{4.581103in}{3.751247in}}%
\pgfpathlineto{\pgfqpoint{4.593783in}{3.738667in}}%
\pgfpathlineto{\pgfqpoint{4.600441in}{3.731927in}}%
\pgfpathlineto{\pgfqpoint{4.607677in}{3.724738in}}%
\pgfpathlineto{\pgfqpoint{4.630965in}{3.701333in}}%
\pgfpathlineto{\pgfqpoint{4.638998in}{3.693174in}}%
\pgfpathlineto{\pgfqpoint{4.647758in}{3.684444in}}%
\pgfpathlineto{\pgfqpoint{4.658248in}{3.673772in}}%
\pgfpathlineto{\pgfqpoint{4.668034in}{3.664000in}}%
\pgfpathlineto{\pgfqpoint{4.677492in}{3.654363in}}%
\pgfpathlineto{\pgfqpoint{4.687838in}{3.644018in}}%
\pgfpathlineto{\pgfqpoint{4.696727in}{3.634946in}}%
\pgfpathlineto{\pgfqpoint{4.704992in}{3.626667in}}%
\pgfpathlineto{\pgfqpoint{4.727919in}{3.603460in}}%
\pgfpathlineto{\pgfqpoint{4.735144in}{3.596063in}}%
\pgfpathlineto{\pgfqpoint{4.741840in}{3.589333in}}%
\pgfpathlineto{\pgfqpoint{4.754292in}{3.576565in}}%
\pgfpathlineto{\pgfqpoint{4.768000in}{3.562769in}}%
\pgfusepath{fill}%
\end{pgfscope}%
\begin{pgfscope}%
\pgfpathrectangle{\pgfqpoint{0.800000in}{0.528000in}}{\pgfqpoint{3.968000in}{3.696000in}}%
\pgfusepath{clip}%
\pgfsetbuttcap%
\pgfsetroundjoin%
\definecolor{currentfill}{rgb}{0.319809,0.770914,0.411152}%
\pgfsetfillcolor{currentfill}%
\pgfsetlinewidth{0.000000pt}%
\definecolor{currentstroke}{rgb}{0.000000,0.000000,0.000000}%
\pgfsetstrokecolor{currentstroke}%
\pgfsetdash{}{0pt}%
\pgfpathmoveto{\pgfqpoint{4.768000in}{3.568305in}}%
\pgfpathlineto{\pgfqpoint{4.757148in}{3.579225in}}%
\pgfpathlineto{\pgfqpoint{4.747291in}{3.589333in}}%
\pgfpathlineto{\pgfqpoint{4.737973in}{3.598698in}}%
\pgfpathlineto{\pgfqpoint{4.727919in}{3.608991in}}%
\pgfpathlineto{\pgfqpoint{4.710457in}{3.626667in}}%
\pgfpathlineto{\pgfqpoint{4.699559in}{3.637584in}}%
\pgfpathlineto{\pgfqpoint{4.687838in}{3.649546in}}%
\pgfpathlineto{\pgfqpoint{4.680354in}{3.657029in}}%
\pgfpathlineto{\pgfqpoint{4.673513in}{3.664000in}}%
\pgfpathlineto{\pgfqpoint{4.661083in}{3.676412in}}%
\pgfpathlineto{\pgfqpoint{4.647758in}{3.689968in}}%
\pgfpathlineto{\pgfqpoint{4.641863in}{3.695843in}}%
\pgfpathlineto{\pgfqpoint{4.636457in}{3.701333in}}%
\pgfpathlineto{\pgfqpoint{4.607677in}{3.730258in}}%
\pgfpathlineto{\pgfqpoint{4.603308in}{3.734598in}}%
\pgfpathlineto{\pgfqpoint{4.599289in}{3.738667in}}%
\pgfpathlineto{\pgfqpoint{4.583943in}{3.753893in}}%
\pgfpathlineto{\pgfqpoint{4.567596in}{3.770417in}}%
\pgfpathlineto{\pgfqpoint{4.562009in}{3.776000in}}%
\pgfpathlineto{\pgfqpoint{4.527515in}{3.810444in}}%
\pgfpathlineto{\pgfqpoint{4.526009in}{3.811931in}}%
\pgfpathlineto{\pgfqpoint{4.524614in}{3.813333in}}%
\pgfpathlineto{\pgfqpoint{4.487434in}{3.850340in}}%
\pgfpathlineto{\pgfqpoint{4.487106in}{3.850667in}}%
\pgfpathlineto{\pgfqpoint{4.482059in}{3.855673in}}%
\pgfpathlineto{\pgfqpoint{4.449457in}{3.888000in}}%
\pgfpathlineto{\pgfqpoint{4.448433in}{3.889005in}}%
\pgfpathlineto{\pgfqpoint{4.447354in}{3.890084in}}%
\pgfpathlineto{\pgfqpoint{4.411689in}{3.925333in}}%
\pgfpathlineto{\pgfqpoint{4.407273in}{3.929695in}}%
\pgfpathlineto{\pgfqpoint{4.389989in}{3.946568in}}%
\pgfpathlineto{\pgfqpoint{4.373805in}{3.962667in}}%
\pgfpathlineto{\pgfqpoint{4.367192in}{3.969178in}}%
\pgfpathlineto{\pgfqpoint{4.335805in}{4.000000in}}%
\pgfpathlineto{\pgfqpoint{4.331550in}{4.004135in}}%
\pgfpathlineto{\pgfqpoint{4.327111in}{4.008531in}}%
\pgfpathlineto{\pgfqpoint{4.297686in}{4.037333in}}%
\pgfpathlineto{\pgfqpoint{4.287030in}{4.047756in}}%
\pgfpathlineto{\pgfqpoint{4.259449in}{4.074667in}}%
\pgfpathlineto{\pgfqpoint{4.246949in}{4.086854in}}%
\pgfpathlineto{\pgfqpoint{4.233680in}{4.099640in}}%
\pgfpathlineto{\pgfqpoint{4.221092in}{4.112000in}}%
\pgfpathlineto{\pgfqpoint{4.206869in}{4.125823in}}%
\pgfpathlineto{\pgfqpoint{4.194442in}{4.137758in}}%
\pgfpathlineto{\pgfqpoint{4.182615in}{4.149333in}}%
\pgfpathlineto{\pgfqpoint{4.166788in}{4.164665in}}%
\pgfpathlineto{\pgfqpoint{4.155139in}{4.175816in}}%
\pgfpathlineto{\pgfqpoint{4.144016in}{4.186667in}}%
\pgfpathlineto{\pgfqpoint{4.126707in}{4.203379in}}%
\pgfpathlineto{\pgfqpoint{4.115771in}{4.213814in}}%
\pgfpathlineto{\pgfqpoint{4.105295in}{4.224000in}}%
\pgfpathlineto{\pgfqpoint{4.102481in}{4.224000in}}%
\pgfpathlineto{\pgfqpoint{4.114334in}{4.212475in}}%
\pgfpathlineto{\pgfqpoint{4.126707in}{4.200670in}}%
\pgfpathlineto{\pgfqpoint{4.141210in}{4.186667in}}%
\pgfpathlineto{\pgfqpoint{4.153703in}{4.174479in}}%
\pgfpathlineto{\pgfqpoint{4.166788in}{4.161954in}}%
\pgfpathlineto{\pgfqpoint{4.179816in}{4.149333in}}%
\pgfpathlineto{\pgfqpoint{4.193008in}{4.136422in}}%
\pgfpathlineto{\pgfqpoint{4.206869in}{4.123110in}}%
\pgfpathlineto{\pgfqpoint{4.218301in}{4.112000in}}%
\pgfpathlineto{\pgfqpoint{4.232247in}{4.098306in}}%
\pgfpathlineto{\pgfqpoint{4.246949in}{4.084139in}}%
\pgfpathlineto{\pgfqpoint{4.256665in}{4.074667in}}%
\pgfpathlineto{\pgfqpoint{4.287030in}{4.045040in}}%
\pgfpathlineto{\pgfqpoint{4.294909in}{4.037333in}}%
\pgfpathlineto{\pgfqpoint{4.327111in}{4.005813in}}%
\pgfpathlineto{\pgfqpoint{4.330136in}{4.002817in}}%
\pgfpathlineto{\pgfqpoint{4.333034in}{4.000000in}}%
\pgfpathlineto{\pgfqpoint{4.367192in}{3.966457in}}%
\pgfpathlineto{\pgfqpoint{4.371042in}{3.962667in}}%
\pgfpathlineto{\pgfqpoint{4.388562in}{3.945239in}}%
\pgfpathlineto{\pgfqpoint{4.407273in}{3.926973in}}%
\pgfpathlineto{\pgfqpoint{4.408933in}{3.925333in}}%
\pgfpathlineto{\pgfqpoint{4.436108in}{3.898475in}}%
\pgfpathlineto{\pgfqpoint{4.446700in}{3.888000in}}%
\pgfpathlineto{\pgfqpoint{4.447015in}{3.887685in}}%
\pgfpathlineto{\pgfqpoint{4.447354in}{3.887353in}}%
\pgfpathlineto{\pgfqpoint{4.484331in}{3.850667in}}%
\pgfpathlineto{\pgfqpoint{4.487434in}{3.847586in}}%
\pgfpathlineto{\pgfqpoint{4.521847in}{3.813333in}}%
\pgfpathlineto{\pgfqpoint{4.524572in}{3.810592in}}%
\pgfpathlineto{\pgfqpoint{4.527515in}{3.807688in}}%
\pgfpathlineto{\pgfqpoint{4.559249in}{3.776000in}}%
\pgfpathlineto{\pgfqpoint{4.567596in}{3.767659in}}%
\pgfpathlineto{\pgfqpoint{4.582523in}{3.752570in}}%
\pgfpathlineto{\pgfqpoint{4.596536in}{3.738667in}}%
\pgfpathlineto{\pgfqpoint{4.601875in}{3.733262in}}%
\pgfpathlineto{\pgfqpoint{4.607677in}{3.727498in}}%
\pgfpathlineto{\pgfqpoint{4.633711in}{3.701333in}}%
\pgfpathlineto{\pgfqpoint{4.640431in}{3.694509in}}%
\pgfpathlineto{\pgfqpoint{4.647758in}{3.687206in}}%
\pgfpathlineto{\pgfqpoint{4.659665in}{3.675092in}}%
\pgfpathlineto{\pgfqpoint{4.670773in}{3.664000in}}%
\pgfpathlineto{\pgfqpoint{4.678923in}{3.655696in}}%
\pgfpathlineto{\pgfqpoint{4.687838in}{3.646782in}}%
\pgfpathlineto{\pgfqpoint{4.698143in}{3.636265in}}%
\pgfpathlineto{\pgfqpoint{4.707725in}{3.626667in}}%
\pgfpathlineto{\pgfqpoint{4.727919in}{3.606226in}}%
\pgfpathlineto{\pgfqpoint{4.736558in}{3.597380in}}%
\pgfpathlineto{\pgfqpoint{4.744566in}{3.589333in}}%
\pgfpathlineto{\pgfqpoint{4.755720in}{3.577895in}}%
\pgfpathlineto{\pgfqpoint{4.768000in}{3.565537in}}%
\pgfusepath{fill}%
\end{pgfscope}%
\begin{pgfscope}%
\pgfpathrectangle{\pgfqpoint{0.800000in}{0.528000in}}{\pgfqpoint{3.968000in}{3.696000in}}%
\pgfusepath{clip}%
\pgfsetbuttcap%
\pgfsetroundjoin%
\definecolor{currentfill}{rgb}{0.319809,0.770914,0.411152}%
\pgfsetfillcolor{currentfill}%
\pgfsetlinewidth{0.000000pt}%
\definecolor{currentstroke}{rgb}{0.000000,0.000000,0.000000}%
\pgfsetstrokecolor{currentstroke}%
\pgfsetdash{}{0pt}%
\pgfpathmoveto{\pgfqpoint{4.768000in}{3.571072in}}%
\pgfpathlineto{\pgfqpoint{4.758576in}{3.580556in}}%
\pgfpathlineto{\pgfqpoint{4.750017in}{3.589333in}}%
\pgfpathlineto{\pgfqpoint{4.739387in}{3.600015in}}%
\pgfpathlineto{\pgfqpoint{4.727919in}{3.611757in}}%
\pgfpathlineto{\pgfqpoint{4.713189in}{3.626667in}}%
\pgfpathlineto{\pgfqpoint{4.700975in}{3.638903in}}%
\pgfpathlineto{\pgfqpoint{4.687838in}{3.652309in}}%
\pgfpathlineto{\pgfqpoint{4.681785in}{3.658362in}}%
\pgfpathlineto{\pgfqpoint{4.676252in}{3.664000in}}%
\pgfpathlineto{\pgfqpoint{4.662500in}{3.677732in}}%
\pgfpathlineto{\pgfqpoint{4.647758in}{3.692730in}}%
\pgfpathlineto{\pgfqpoint{4.643295in}{3.697177in}}%
\pgfpathlineto{\pgfqpoint{4.639203in}{3.701333in}}%
\pgfpathlineto{\pgfqpoint{4.607677in}{3.733018in}}%
\pgfpathlineto{\pgfqpoint{4.604742in}{3.735933in}}%
\pgfpathlineto{\pgfqpoint{4.602042in}{3.738667in}}%
\pgfpathlineto{\pgfqpoint{4.585362in}{3.755215in}}%
\pgfpathlineto{\pgfqpoint{4.567596in}{3.773175in}}%
\pgfpathlineto{\pgfqpoint{4.564769in}{3.776000in}}%
\pgfpathlineto{\pgfqpoint{4.527515in}{3.813200in}}%
\pgfpathlineto{\pgfqpoint{4.527446in}{3.813269in}}%
\pgfpathlineto{\pgfqpoint{4.527382in}{3.813333in}}%
\pgfpathlineto{\pgfqpoint{4.525433in}{3.815272in}}%
\pgfpathlineto{\pgfqpoint{4.489852in}{3.850667in}}%
\pgfpathlineto{\pgfqpoint{4.488676in}{3.851824in}}%
\pgfpathlineto{\pgfqpoint{4.487434in}{3.853070in}}%
\pgfpathlineto{\pgfqpoint{4.452206in}{3.888000in}}%
\pgfpathlineto{\pgfqpoint{4.449843in}{3.890319in}}%
\pgfpathlineto{\pgfqpoint{4.447354in}{3.892808in}}%
\pgfpathlineto{\pgfqpoint{4.414445in}{3.925333in}}%
\pgfpathlineto{\pgfqpoint{4.407273in}{3.932417in}}%
\pgfpathlineto{\pgfqpoint{4.391416in}{3.947897in}}%
\pgfpathlineto{\pgfqpoint{4.376568in}{3.962667in}}%
\pgfpathlineto{\pgfqpoint{4.367192in}{3.971898in}}%
\pgfpathlineto{\pgfqpoint{4.338575in}{4.000000in}}%
\pgfpathlineto{\pgfqpoint{4.332965in}{4.005453in}}%
\pgfpathlineto{\pgfqpoint{4.327111in}{4.011250in}}%
\pgfpathlineto{\pgfqpoint{4.300463in}{4.037333in}}%
\pgfpathlineto{\pgfqpoint{4.287030in}{4.050473in}}%
\pgfpathlineto{\pgfqpoint{4.262233in}{4.074667in}}%
\pgfpathlineto{\pgfqpoint{4.246949in}{4.089568in}}%
\pgfpathlineto{\pgfqpoint{4.235112in}{4.100974in}}%
\pgfpathlineto{\pgfqpoint{4.223884in}{4.112000in}}%
\pgfpathlineto{\pgfqpoint{4.206869in}{4.128536in}}%
\pgfpathlineto{\pgfqpoint{4.195876in}{4.139094in}}%
\pgfpathlineto{\pgfqpoint{4.185413in}{4.149333in}}%
\pgfpathlineto{\pgfqpoint{4.166788in}{4.167376in}}%
\pgfpathlineto{\pgfqpoint{4.156574in}{4.177153in}}%
\pgfpathlineto{\pgfqpoint{4.146822in}{4.186667in}}%
\pgfpathlineto{\pgfqpoint{4.126707in}{4.206088in}}%
\pgfpathlineto{\pgfqpoint{4.117208in}{4.215152in}}%
\pgfpathlineto{\pgfqpoint{4.108108in}{4.224000in}}%
\pgfpathlineto{\pgfqpoint{4.105295in}{4.224000in}}%
\pgfpathlineto{\pgfqpoint{4.115771in}{4.213814in}}%
\pgfpathlineto{\pgfqpoint{4.126707in}{4.203379in}}%
\pgfpathlineto{\pgfqpoint{4.144016in}{4.186667in}}%
\pgfpathlineto{\pgfqpoint{4.155139in}{4.175816in}}%
\pgfpathlineto{\pgfqpoint{4.166788in}{4.164665in}}%
\pgfpathlineto{\pgfqpoint{4.182615in}{4.149333in}}%
\pgfpathlineto{\pgfqpoint{4.194442in}{4.137758in}}%
\pgfpathlineto{\pgfqpoint{4.206869in}{4.125823in}}%
\pgfpathlineto{\pgfqpoint{4.221092in}{4.112000in}}%
\pgfpathlineto{\pgfqpoint{4.233680in}{4.099640in}}%
\pgfpathlineto{\pgfqpoint{4.246949in}{4.086854in}}%
\pgfpathlineto{\pgfqpoint{4.259449in}{4.074667in}}%
\pgfpathlineto{\pgfqpoint{4.287030in}{4.047756in}}%
\pgfpathlineto{\pgfqpoint{4.297686in}{4.037333in}}%
\pgfpathlineto{\pgfqpoint{4.327111in}{4.008531in}}%
\pgfpathlineto{\pgfqpoint{4.331550in}{4.004135in}}%
\pgfpathlineto{\pgfqpoint{4.335805in}{4.000000in}}%
\pgfpathlineto{\pgfqpoint{4.367192in}{3.969178in}}%
\pgfpathlineto{\pgfqpoint{4.373805in}{3.962667in}}%
\pgfpathlineto{\pgfqpoint{4.389989in}{3.946568in}}%
\pgfpathlineto{\pgfqpoint{4.407273in}{3.929695in}}%
\pgfpathlineto{\pgfqpoint{4.411689in}{3.925333in}}%
\pgfpathlineto{\pgfqpoint{4.447354in}{3.890084in}}%
\pgfpathlineto{\pgfqpoint{4.448433in}{3.889005in}}%
\pgfpathlineto{\pgfqpoint{4.449457in}{3.888000in}}%
\pgfpathlineto{\pgfqpoint{4.482059in}{3.855673in}}%
\pgfpathlineto{\pgfqpoint{4.487106in}{3.850667in}}%
\pgfpathlineto{\pgfqpoint{4.487434in}{3.850340in}}%
\pgfpathlineto{\pgfqpoint{4.524614in}{3.813333in}}%
\pgfpathlineto{\pgfqpoint{4.526009in}{3.811931in}}%
\pgfpathlineto{\pgfqpoint{4.527515in}{3.810444in}}%
\pgfpathlineto{\pgfqpoint{4.562009in}{3.776000in}}%
\pgfpathlineto{\pgfqpoint{4.567596in}{3.770417in}}%
\pgfpathlineto{\pgfqpoint{4.583943in}{3.753893in}}%
\pgfpathlineto{\pgfqpoint{4.599289in}{3.738667in}}%
\pgfpathlineto{\pgfqpoint{4.603308in}{3.734598in}}%
\pgfpathlineto{\pgfqpoint{4.607677in}{3.730258in}}%
\pgfpathlineto{\pgfqpoint{4.636457in}{3.701333in}}%
\pgfpathlineto{\pgfqpoint{4.641863in}{3.695843in}}%
\pgfpathlineto{\pgfqpoint{4.647758in}{3.689968in}}%
\pgfpathlineto{\pgfqpoint{4.661083in}{3.676412in}}%
\pgfpathlineto{\pgfqpoint{4.673513in}{3.664000in}}%
\pgfpathlineto{\pgfqpoint{4.680354in}{3.657029in}}%
\pgfpathlineto{\pgfqpoint{4.687838in}{3.649546in}}%
\pgfpathlineto{\pgfqpoint{4.699559in}{3.637584in}}%
\pgfpathlineto{\pgfqpoint{4.710457in}{3.626667in}}%
\pgfpathlineto{\pgfqpoint{4.727919in}{3.608991in}}%
\pgfpathlineto{\pgfqpoint{4.737973in}{3.598698in}}%
\pgfpathlineto{\pgfqpoint{4.747291in}{3.589333in}}%
\pgfpathlineto{\pgfqpoint{4.757148in}{3.579225in}}%
\pgfpathlineto{\pgfqpoint{4.768000in}{3.568305in}}%
\pgfusepath{fill}%
\end{pgfscope}%
\begin{pgfscope}%
\pgfpathrectangle{\pgfqpoint{0.800000in}{0.528000in}}{\pgfqpoint{3.968000in}{3.696000in}}%
\pgfusepath{clip}%
\pgfsetbuttcap%
\pgfsetroundjoin%
\definecolor{currentfill}{rgb}{0.319809,0.770914,0.411152}%
\pgfsetfillcolor{currentfill}%
\pgfsetlinewidth{0.000000pt}%
\definecolor{currentstroke}{rgb}{0.000000,0.000000,0.000000}%
\pgfsetstrokecolor{currentstroke}%
\pgfsetdash{}{0pt}%
\pgfpathmoveto{\pgfqpoint{4.768000in}{3.573840in}}%
\pgfpathlineto{\pgfqpoint{4.760005in}{3.581886in}}%
\pgfpathlineto{\pgfqpoint{4.752742in}{3.589333in}}%
\pgfpathlineto{\pgfqpoint{4.740802in}{3.601333in}}%
\pgfpathlineto{\pgfqpoint{4.727919in}{3.614523in}}%
\pgfpathlineto{\pgfqpoint{4.715922in}{3.626667in}}%
\pgfpathlineto{\pgfqpoint{4.702391in}{3.640221in}}%
\pgfpathlineto{\pgfqpoint{4.687838in}{3.655073in}}%
\pgfpathlineto{\pgfqpoint{4.683216in}{3.659695in}}%
\pgfpathlineto{\pgfqpoint{4.678991in}{3.664000in}}%
\pgfpathlineto{\pgfqpoint{4.663917in}{3.679052in}}%
\pgfpathlineto{\pgfqpoint{4.647758in}{3.695492in}}%
\pgfpathlineto{\pgfqpoint{4.644728in}{3.698511in}}%
\pgfpathlineto{\pgfqpoint{4.641949in}{3.701333in}}%
\pgfpathlineto{\pgfqpoint{4.607677in}{3.735778in}}%
\pgfpathlineto{\pgfqpoint{4.606176in}{3.737269in}}%
\pgfpathlineto{\pgfqpoint{4.604795in}{3.738667in}}%
\pgfpathlineto{\pgfqpoint{4.586782in}{3.756538in}}%
\pgfpathlineto{\pgfqpoint{4.567596in}{3.775933in}}%
\pgfpathlineto{\pgfqpoint{4.567529in}{3.776000in}}%
\pgfpathlineto{\pgfqpoint{4.566596in}{3.776931in}}%
\pgfpathlineto{\pgfqpoint{4.530118in}{3.813333in}}%
\pgfpathlineto{\pgfqpoint{4.528855in}{3.814581in}}%
\pgfpathlineto{\pgfqpoint{4.527515in}{3.815929in}}%
\pgfpathlineto{\pgfqpoint{4.492594in}{3.850667in}}%
\pgfpathlineto{\pgfqpoint{4.490086in}{3.853136in}}%
\pgfpathlineto{\pgfqpoint{4.487434in}{3.855795in}}%
\pgfpathlineto{\pgfqpoint{4.454955in}{3.888000in}}%
\pgfpathlineto{\pgfqpoint{4.451254in}{3.891633in}}%
\pgfpathlineto{\pgfqpoint{4.447354in}{3.895532in}}%
\pgfpathlineto{\pgfqpoint{4.417201in}{3.925333in}}%
\pgfpathlineto{\pgfqpoint{4.407273in}{3.935140in}}%
\pgfpathlineto{\pgfqpoint{4.392843in}{3.949226in}}%
\pgfpathlineto{\pgfqpoint{4.379331in}{3.962667in}}%
\pgfpathlineto{\pgfqpoint{4.367192in}{3.974618in}}%
\pgfpathlineto{\pgfqpoint{4.341345in}{4.000000in}}%
\pgfpathlineto{\pgfqpoint{4.334380in}{4.006770in}}%
\pgfpathlineto{\pgfqpoint{4.327111in}{4.013968in}}%
\pgfpathlineto{\pgfqpoint{4.303241in}{4.037333in}}%
\pgfpathlineto{\pgfqpoint{4.287030in}{4.053190in}}%
\pgfpathlineto{\pgfqpoint{4.265018in}{4.074667in}}%
\pgfpathlineto{\pgfqpoint{4.246949in}{4.092283in}}%
\pgfpathlineto{\pgfqpoint{4.236545in}{4.102309in}}%
\pgfpathlineto{\pgfqpoint{4.226675in}{4.112000in}}%
\pgfpathlineto{\pgfqpoint{4.206869in}{4.131249in}}%
\pgfpathlineto{\pgfqpoint{4.197310in}{4.140430in}}%
\pgfpathlineto{\pgfqpoint{4.188212in}{4.149333in}}%
\pgfpathlineto{\pgfqpoint{4.166788in}{4.170087in}}%
\pgfpathlineto{\pgfqpoint{4.158010in}{4.178490in}}%
\pgfpathlineto{\pgfqpoint{4.149628in}{4.186667in}}%
\pgfpathlineto{\pgfqpoint{4.126707in}{4.208797in}}%
\pgfpathlineto{\pgfqpoint{4.118645in}{4.216490in}}%
\pgfpathlineto{\pgfqpoint{4.110921in}{4.224000in}}%
\pgfpathlineto{\pgfqpoint{4.108108in}{4.224000in}}%
\pgfpathlineto{\pgfqpoint{4.117208in}{4.215152in}}%
\pgfpathlineto{\pgfqpoint{4.126707in}{4.206088in}}%
\pgfpathlineto{\pgfqpoint{4.146822in}{4.186667in}}%
\pgfpathlineto{\pgfqpoint{4.156574in}{4.177153in}}%
\pgfpathlineto{\pgfqpoint{4.166788in}{4.167376in}}%
\pgfpathlineto{\pgfqpoint{4.185413in}{4.149333in}}%
\pgfpathlineto{\pgfqpoint{4.195876in}{4.139094in}}%
\pgfpathlineto{\pgfqpoint{4.206869in}{4.128536in}}%
\pgfpathlineto{\pgfqpoint{4.223884in}{4.112000in}}%
\pgfpathlineto{\pgfqpoint{4.235112in}{4.100974in}}%
\pgfpathlineto{\pgfqpoint{4.246949in}{4.089568in}}%
\pgfpathlineto{\pgfqpoint{4.262233in}{4.074667in}}%
\pgfpathlineto{\pgfqpoint{4.287030in}{4.050473in}}%
\pgfpathlineto{\pgfqpoint{4.300463in}{4.037333in}}%
\pgfpathlineto{\pgfqpoint{4.327111in}{4.011250in}}%
\pgfpathlineto{\pgfqpoint{4.332965in}{4.005453in}}%
\pgfpathlineto{\pgfqpoint{4.338575in}{4.000000in}}%
\pgfpathlineto{\pgfqpoint{4.367192in}{3.971898in}}%
\pgfpathlineto{\pgfqpoint{4.376568in}{3.962667in}}%
\pgfpathlineto{\pgfqpoint{4.391416in}{3.947897in}}%
\pgfpathlineto{\pgfqpoint{4.407273in}{3.932417in}}%
\pgfpathlineto{\pgfqpoint{4.414445in}{3.925333in}}%
\pgfpathlineto{\pgfqpoint{4.447354in}{3.892808in}}%
\pgfpathlineto{\pgfqpoint{4.449843in}{3.890319in}}%
\pgfpathlineto{\pgfqpoint{4.452206in}{3.888000in}}%
\pgfpathlineto{\pgfqpoint{4.487434in}{3.853070in}}%
\pgfpathlineto{\pgfqpoint{4.488676in}{3.851824in}}%
\pgfpathlineto{\pgfqpoint{4.489852in}{3.850667in}}%
\pgfpathlineto{\pgfqpoint{4.525433in}{3.815272in}}%
\pgfpathlineto{\pgfqpoint{4.527382in}{3.813333in}}%
\pgfpathlineto{\pgfqpoint{4.527446in}{3.813269in}}%
\pgfpathlineto{\pgfqpoint{4.527515in}{3.813200in}}%
\pgfpathlineto{\pgfqpoint{4.564769in}{3.776000in}}%
\pgfpathlineto{\pgfqpoint{4.567596in}{3.773175in}}%
\pgfpathlineto{\pgfqpoint{4.585362in}{3.755215in}}%
\pgfpathlineto{\pgfqpoint{4.602042in}{3.738667in}}%
\pgfpathlineto{\pgfqpoint{4.604742in}{3.735933in}}%
\pgfpathlineto{\pgfqpoint{4.607677in}{3.733018in}}%
\pgfpathlineto{\pgfqpoint{4.639203in}{3.701333in}}%
\pgfpathlineto{\pgfqpoint{4.643295in}{3.697177in}}%
\pgfpathlineto{\pgfqpoint{4.647758in}{3.692730in}}%
\pgfpathlineto{\pgfqpoint{4.662500in}{3.677732in}}%
\pgfpathlineto{\pgfqpoint{4.676252in}{3.664000in}}%
\pgfpathlineto{\pgfqpoint{4.681785in}{3.658362in}}%
\pgfpathlineto{\pgfqpoint{4.687838in}{3.652309in}}%
\pgfpathlineto{\pgfqpoint{4.700975in}{3.638903in}}%
\pgfpathlineto{\pgfqpoint{4.713189in}{3.626667in}}%
\pgfpathlineto{\pgfqpoint{4.727919in}{3.611757in}}%
\pgfpathlineto{\pgfqpoint{4.739387in}{3.600015in}}%
\pgfpathlineto{\pgfqpoint{4.750017in}{3.589333in}}%
\pgfpathlineto{\pgfqpoint{4.758576in}{3.580556in}}%
\pgfpathlineto{\pgfqpoint{4.768000in}{3.571072in}}%
\pgfusepath{fill}%
\end{pgfscope}%
\begin{pgfscope}%
\pgfpathrectangle{\pgfqpoint{0.800000in}{0.528000in}}{\pgfqpoint{3.968000in}{3.696000in}}%
\pgfusepath{clip}%
\pgfsetbuttcap%
\pgfsetroundjoin%
\definecolor{currentfill}{rgb}{0.327796,0.773980,0.406640}%
\pgfsetfillcolor{currentfill}%
\pgfsetlinewidth{0.000000pt}%
\definecolor{currentstroke}{rgb}{0.000000,0.000000,0.000000}%
\pgfsetstrokecolor{currentstroke}%
\pgfsetdash{}{0pt}%
\pgfpathmoveto{\pgfqpoint{4.768000in}{3.576607in}}%
\pgfpathlineto{\pgfqpoint{4.761433in}{3.583216in}}%
\pgfpathlineto{\pgfqpoint{4.755468in}{3.589333in}}%
\pgfpathlineto{\pgfqpoint{4.742216in}{3.602650in}}%
\pgfpathlineto{\pgfqpoint{4.727919in}{3.617288in}}%
\pgfpathlineto{\pgfqpoint{4.718654in}{3.626667in}}%
\pgfpathlineto{\pgfqpoint{4.703806in}{3.641540in}}%
\pgfpathlineto{\pgfqpoint{4.687838in}{3.657837in}}%
\pgfpathlineto{\pgfqpoint{4.684647in}{3.661028in}}%
\pgfpathlineto{\pgfqpoint{4.681730in}{3.664000in}}%
\pgfpathlineto{\pgfqpoint{4.665334in}{3.680372in}}%
\pgfpathlineto{\pgfqpoint{4.647758in}{3.698253in}}%
\pgfpathlineto{\pgfqpoint{4.646160in}{3.699845in}}%
\pgfpathlineto{\pgfqpoint{4.644695in}{3.701333in}}%
\pgfpathlineto{\pgfqpoint{4.607677in}{3.738538in}}%
\pgfpathlineto{\pgfqpoint{4.607610in}{3.738604in}}%
\pgfpathlineto{\pgfqpoint{4.607548in}{3.738667in}}%
\pgfpathlineto{\pgfqpoint{4.588202in}{3.757861in}}%
\pgfpathlineto{\pgfqpoint{4.570258in}{3.776000in}}%
\pgfpathlineto{\pgfqpoint{4.568968in}{3.777278in}}%
\pgfpathlineto{\pgfqpoint{4.567596in}{3.778663in}}%
\pgfpathlineto{\pgfqpoint{4.532854in}{3.813333in}}%
\pgfpathlineto{\pgfqpoint{4.530263in}{3.815892in}}%
\pgfpathlineto{\pgfqpoint{4.527515in}{3.818657in}}%
\pgfpathlineto{\pgfqpoint{4.495336in}{3.850667in}}%
\pgfpathlineto{\pgfqpoint{4.491495in}{3.854449in}}%
\pgfpathlineto{\pgfqpoint{4.487434in}{3.858521in}}%
\pgfpathlineto{\pgfqpoint{4.457704in}{3.888000in}}%
\pgfpathlineto{\pgfqpoint{4.452664in}{3.892946in}}%
\pgfpathlineto{\pgfqpoint{4.447354in}{3.898256in}}%
\pgfpathlineto{\pgfqpoint{4.419957in}{3.925333in}}%
\pgfpathlineto{\pgfqpoint{4.407273in}{3.937862in}}%
\pgfpathlineto{\pgfqpoint{4.394270in}{3.950555in}}%
\pgfpathlineto{\pgfqpoint{4.382095in}{3.962667in}}%
\pgfpathlineto{\pgfqpoint{4.367192in}{3.977338in}}%
\pgfpathlineto{\pgfqpoint{4.344115in}{4.000000in}}%
\pgfpathlineto{\pgfqpoint{4.335794in}{4.008088in}}%
\pgfpathlineto{\pgfqpoint{4.327111in}{4.016686in}}%
\pgfpathlineto{\pgfqpoint{4.306018in}{4.037333in}}%
\pgfpathlineto{\pgfqpoint{4.287030in}{4.055906in}}%
\pgfpathlineto{\pgfqpoint{4.267802in}{4.074667in}}%
\pgfpathlineto{\pgfqpoint{4.246949in}{4.094998in}}%
\pgfpathlineto{\pgfqpoint{4.237978in}{4.103643in}}%
\pgfpathlineto{\pgfqpoint{4.229467in}{4.112000in}}%
\pgfpathlineto{\pgfqpoint{4.206869in}{4.133962in}}%
\pgfpathlineto{\pgfqpoint{4.198744in}{4.141765in}}%
\pgfpathlineto{\pgfqpoint{4.191011in}{4.149333in}}%
\pgfpathlineto{\pgfqpoint{4.166788in}{4.172798in}}%
\pgfpathlineto{\pgfqpoint{4.159445in}{4.179827in}}%
\pgfpathlineto{\pgfqpoint{4.152434in}{4.186667in}}%
\pgfpathlineto{\pgfqpoint{4.126707in}{4.211507in}}%
\pgfpathlineto{\pgfqpoint{4.120081in}{4.217828in}}%
\pgfpathlineto{\pgfqpoint{4.113734in}{4.224000in}}%
\pgfpathlineto{\pgfqpoint{4.110921in}{4.224000in}}%
\pgfpathlineto{\pgfqpoint{4.118645in}{4.216490in}}%
\pgfpathlineto{\pgfqpoint{4.126707in}{4.208797in}}%
\pgfpathlineto{\pgfqpoint{4.149628in}{4.186667in}}%
\pgfpathlineto{\pgfqpoint{4.158010in}{4.178490in}}%
\pgfpathlineto{\pgfqpoint{4.166788in}{4.170087in}}%
\pgfpathlineto{\pgfqpoint{4.188212in}{4.149333in}}%
\pgfpathlineto{\pgfqpoint{4.197310in}{4.140430in}}%
\pgfpathlineto{\pgfqpoint{4.206869in}{4.131249in}}%
\pgfpathlineto{\pgfqpoint{4.226675in}{4.112000in}}%
\pgfpathlineto{\pgfqpoint{4.236545in}{4.102309in}}%
\pgfpathlineto{\pgfqpoint{4.246949in}{4.092283in}}%
\pgfpathlineto{\pgfqpoint{4.265018in}{4.074667in}}%
\pgfpathlineto{\pgfqpoint{4.287030in}{4.053190in}}%
\pgfpathlineto{\pgfqpoint{4.303241in}{4.037333in}}%
\pgfpathlineto{\pgfqpoint{4.327111in}{4.013968in}}%
\pgfpathlineto{\pgfqpoint{4.334380in}{4.006770in}}%
\pgfpathlineto{\pgfqpoint{4.341345in}{4.000000in}}%
\pgfpathlineto{\pgfqpoint{4.367192in}{3.974618in}}%
\pgfpathlineto{\pgfqpoint{4.379331in}{3.962667in}}%
\pgfpathlineto{\pgfqpoint{4.392843in}{3.949226in}}%
\pgfpathlineto{\pgfqpoint{4.407273in}{3.935140in}}%
\pgfpathlineto{\pgfqpoint{4.417201in}{3.925333in}}%
\pgfpathlineto{\pgfqpoint{4.447354in}{3.895532in}}%
\pgfpathlineto{\pgfqpoint{4.451254in}{3.891633in}}%
\pgfpathlineto{\pgfqpoint{4.454955in}{3.888000in}}%
\pgfpathlineto{\pgfqpoint{4.487434in}{3.855795in}}%
\pgfpathlineto{\pgfqpoint{4.490086in}{3.853136in}}%
\pgfpathlineto{\pgfqpoint{4.492594in}{3.850667in}}%
\pgfpathlineto{\pgfqpoint{4.527515in}{3.815929in}}%
\pgfpathlineto{\pgfqpoint{4.528855in}{3.814581in}}%
\pgfpathlineto{\pgfqpoint{4.530118in}{3.813333in}}%
\pgfpathlineto{\pgfqpoint{4.566596in}{3.776931in}}%
\pgfpathlineto{\pgfqpoint{4.567529in}{3.776000in}}%
\pgfpathlineto{\pgfqpoint{4.567596in}{3.775933in}}%
\pgfpathlineto{\pgfqpoint{4.586782in}{3.756538in}}%
\pgfpathlineto{\pgfqpoint{4.604795in}{3.738667in}}%
\pgfpathlineto{\pgfqpoint{4.606176in}{3.737269in}}%
\pgfpathlineto{\pgfqpoint{4.607677in}{3.735778in}}%
\pgfpathlineto{\pgfqpoint{4.641949in}{3.701333in}}%
\pgfpathlineto{\pgfqpoint{4.644728in}{3.698511in}}%
\pgfpathlineto{\pgfqpoint{4.647758in}{3.695492in}}%
\pgfpathlineto{\pgfqpoint{4.663917in}{3.679052in}}%
\pgfpathlineto{\pgfqpoint{4.678991in}{3.664000in}}%
\pgfpathlineto{\pgfqpoint{4.683216in}{3.659695in}}%
\pgfpathlineto{\pgfqpoint{4.687838in}{3.655073in}}%
\pgfpathlineto{\pgfqpoint{4.702391in}{3.640221in}}%
\pgfpathlineto{\pgfqpoint{4.715922in}{3.626667in}}%
\pgfpathlineto{\pgfqpoint{4.727919in}{3.614523in}}%
\pgfpathlineto{\pgfqpoint{4.740802in}{3.601333in}}%
\pgfpathlineto{\pgfqpoint{4.752742in}{3.589333in}}%
\pgfpathlineto{\pgfqpoint{4.760005in}{3.581886in}}%
\pgfpathlineto{\pgfqpoint{4.768000in}{3.573840in}}%
\pgfusepath{fill}%
\end{pgfscope}%
\begin{pgfscope}%
\pgfpathrectangle{\pgfqpoint{0.800000in}{0.528000in}}{\pgfqpoint{3.968000in}{3.696000in}}%
\pgfusepath{clip}%
\pgfsetbuttcap%
\pgfsetroundjoin%
\definecolor{currentfill}{rgb}{0.327796,0.773980,0.406640}%
\pgfsetfillcolor{currentfill}%
\pgfsetlinewidth{0.000000pt}%
\definecolor{currentstroke}{rgb}{0.000000,0.000000,0.000000}%
\pgfsetstrokecolor{currentstroke}%
\pgfsetdash{}{0pt}%
\pgfpathmoveto{\pgfqpoint{4.768000in}{3.579375in}}%
\pgfpathlineto{\pgfqpoint{4.762861in}{3.584547in}}%
\pgfpathlineto{\pgfqpoint{4.758193in}{3.589333in}}%
\pgfpathlineto{\pgfqpoint{4.743631in}{3.603968in}}%
\pgfpathlineto{\pgfqpoint{4.727919in}{3.620054in}}%
\pgfpathlineto{\pgfqpoint{4.721386in}{3.626667in}}%
\pgfpathlineto{\pgfqpoint{4.705222in}{3.642859in}}%
\pgfpathlineto{\pgfqpoint{4.687838in}{3.660601in}}%
\pgfpathlineto{\pgfqpoint{4.686078in}{3.662361in}}%
\pgfpathlineto{\pgfqpoint{4.684469in}{3.664000in}}%
\pgfpathlineto{\pgfqpoint{4.666751in}{3.681692in}}%
\pgfpathlineto{\pgfqpoint{4.647758in}{3.701015in}}%
\pgfpathlineto{\pgfqpoint{4.647593in}{3.701180in}}%
\pgfpathlineto{\pgfqpoint{4.647441in}{3.701333in}}%
\pgfpathlineto{\pgfqpoint{4.643437in}{3.705358in}}%
\pgfpathlineto{\pgfqpoint{4.610271in}{3.738667in}}%
\pgfpathlineto{\pgfqpoint{4.607677in}{3.741271in}}%
\pgfpathlineto{\pgfqpoint{4.589622in}{3.759183in}}%
\pgfpathlineto{\pgfqpoint{4.572986in}{3.776000in}}%
\pgfpathlineto{\pgfqpoint{4.570374in}{3.778588in}}%
\pgfpathlineto{\pgfqpoint{4.567596in}{3.781393in}}%
\pgfpathlineto{\pgfqpoint{4.535589in}{3.813333in}}%
\pgfpathlineto{\pgfqpoint{4.531670in}{3.817204in}}%
\pgfpathlineto{\pgfqpoint{4.527515in}{3.821385in}}%
\pgfpathlineto{\pgfqpoint{4.498078in}{3.850667in}}%
\pgfpathlineto{\pgfqpoint{4.492904in}{3.855761in}}%
\pgfpathlineto{\pgfqpoint{4.487434in}{3.861247in}}%
\pgfpathlineto{\pgfqpoint{4.460453in}{3.888000in}}%
\pgfpathlineto{\pgfqpoint{4.454075in}{3.894260in}}%
\pgfpathlineto{\pgfqpoint{4.447354in}{3.900980in}}%
\pgfpathlineto{\pgfqpoint{4.422713in}{3.925333in}}%
\pgfpathlineto{\pgfqpoint{4.407273in}{3.940584in}}%
\pgfpathlineto{\pgfqpoint{4.395697in}{3.951884in}}%
\pgfpathlineto{\pgfqpoint{4.384858in}{3.962667in}}%
\pgfpathlineto{\pgfqpoint{4.367192in}{3.980059in}}%
\pgfpathlineto{\pgfqpoint{4.346885in}{4.000000in}}%
\pgfpathlineto{\pgfqpoint{4.337209in}{4.009405in}}%
\pgfpathlineto{\pgfqpoint{4.327111in}{4.019405in}}%
\pgfpathlineto{\pgfqpoint{4.308795in}{4.037333in}}%
\pgfpathlineto{\pgfqpoint{4.287030in}{4.058623in}}%
\pgfpathlineto{\pgfqpoint{4.270586in}{4.074667in}}%
\pgfpathlineto{\pgfqpoint{4.246949in}{4.097713in}}%
\pgfpathlineto{\pgfqpoint{4.239410in}{4.104977in}}%
\pgfpathlineto{\pgfqpoint{4.232258in}{4.112000in}}%
\pgfpathlineto{\pgfqpoint{4.206869in}{4.136675in}}%
\pgfpathlineto{\pgfqpoint{4.200178in}{4.143101in}}%
\pgfpathlineto{\pgfqpoint{4.193810in}{4.149333in}}%
\pgfpathlineto{\pgfqpoint{4.166788in}{4.175509in}}%
\pgfpathlineto{\pgfqpoint{4.160880in}{4.181164in}}%
\pgfpathlineto{\pgfqpoint{4.155240in}{4.186667in}}%
\pgfpathlineto{\pgfqpoint{4.126707in}{4.214216in}}%
\pgfpathlineto{\pgfqpoint{4.121518in}{4.219167in}}%
\pgfpathlineto{\pgfqpoint{4.116547in}{4.224000in}}%
\pgfpathlineto{\pgfqpoint{4.113734in}{4.224000in}}%
\pgfpathlineto{\pgfqpoint{4.120081in}{4.217828in}}%
\pgfpathlineto{\pgfqpoint{4.126707in}{4.211507in}}%
\pgfpathlineto{\pgfqpoint{4.152434in}{4.186667in}}%
\pgfpathlineto{\pgfqpoint{4.159445in}{4.179827in}}%
\pgfpathlineto{\pgfqpoint{4.166788in}{4.172798in}}%
\pgfpathlineto{\pgfqpoint{4.191011in}{4.149333in}}%
\pgfpathlineto{\pgfqpoint{4.198744in}{4.141765in}}%
\pgfpathlineto{\pgfqpoint{4.206869in}{4.133962in}}%
\pgfpathlineto{\pgfqpoint{4.229467in}{4.112000in}}%
\pgfpathlineto{\pgfqpoint{4.237978in}{4.103643in}}%
\pgfpathlineto{\pgfqpoint{4.246949in}{4.094998in}}%
\pgfpathlineto{\pgfqpoint{4.267802in}{4.074667in}}%
\pgfpathlineto{\pgfqpoint{4.287030in}{4.055906in}}%
\pgfpathlineto{\pgfqpoint{4.306018in}{4.037333in}}%
\pgfpathlineto{\pgfqpoint{4.327111in}{4.016686in}}%
\pgfpathlineto{\pgfqpoint{4.335794in}{4.008088in}}%
\pgfpathlineto{\pgfqpoint{4.344115in}{4.000000in}}%
\pgfpathlineto{\pgfqpoint{4.367192in}{3.977338in}}%
\pgfpathlineto{\pgfqpoint{4.382095in}{3.962667in}}%
\pgfpathlineto{\pgfqpoint{4.394270in}{3.950555in}}%
\pgfpathlineto{\pgfqpoint{4.407273in}{3.937862in}}%
\pgfpathlineto{\pgfqpoint{4.419957in}{3.925333in}}%
\pgfpathlineto{\pgfqpoint{4.447354in}{3.898256in}}%
\pgfpathlineto{\pgfqpoint{4.452664in}{3.892946in}}%
\pgfpathlineto{\pgfqpoint{4.457704in}{3.888000in}}%
\pgfpathlineto{\pgfqpoint{4.487434in}{3.858521in}}%
\pgfpathlineto{\pgfqpoint{4.491495in}{3.854449in}}%
\pgfpathlineto{\pgfqpoint{4.495336in}{3.850667in}}%
\pgfpathlineto{\pgfqpoint{4.527515in}{3.818657in}}%
\pgfpathlineto{\pgfqpoint{4.530263in}{3.815892in}}%
\pgfpathlineto{\pgfqpoint{4.532854in}{3.813333in}}%
\pgfpathlineto{\pgfqpoint{4.567596in}{3.778663in}}%
\pgfpathlineto{\pgfqpoint{4.568968in}{3.777278in}}%
\pgfpathlineto{\pgfqpoint{4.570258in}{3.776000in}}%
\pgfpathlineto{\pgfqpoint{4.588202in}{3.757861in}}%
\pgfpathlineto{\pgfqpoint{4.607548in}{3.738667in}}%
\pgfpathlineto{\pgfqpoint{4.607610in}{3.738604in}}%
\pgfpathlineto{\pgfqpoint{4.607677in}{3.738538in}}%
\pgfpathlineto{\pgfqpoint{4.644695in}{3.701333in}}%
\pgfpathlineto{\pgfqpoint{4.646160in}{3.699845in}}%
\pgfpathlineto{\pgfqpoint{4.647758in}{3.698253in}}%
\pgfpathlineto{\pgfqpoint{4.665334in}{3.680372in}}%
\pgfpathlineto{\pgfqpoint{4.681730in}{3.664000in}}%
\pgfpathlineto{\pgfqpoint{4.684647in}{3.661028in}}%
\pgfpathlineto{\pgfqpoint{4.687838in}{3.657837in}}%
\pgfpathlineto{\pgfqpoint{4.703806in}{3.641540in}}%
\pgfpathlineto{\pgfqpoint{4.718654in}{3.626667in}}%
\pgfpathlineto{\pgfqpoint{4.727919in}{3.617288in}}%
\pgfpathlineto{\pgfqpoint{4.742216in}{3.602650in}}%
\pgfpathlineto{\pgfqpoint{4.755468in}{3.589333in}}%
\pgfpathlineto{\pgfqpoint{4.761433in}{3.583216in}}%
\pgfpathlineto{\pgfqpoint{4.768000in}{3.576607in}}%
\pgfusepath{fill}%
\end{pgfscope}%
\begin{pgfscope}%
\pgfpathrectangle{\pgfqpoint{0.800000in}{0.528000in}}{\pgfqpoint{3.968000in}{3.696000in}}%
\pgfusepath{clip}%
\pgfsetbuttcap%
\pgfsetroundjoin%
\definecolor{currentfill}{rgb}{0.327796,0.773980,0.406640}%
\pgfsetfillcolor{currentfill}%
\pgfsetlinewidth{0.000000pt}%
\definecolor{currentstroke}{rgb}{0.000000,0.000000,0.000000}%
\pgfsetstrokecolor{currentstroke}%
\pgfsetdash{}{0pt}%
\pgfpathmoveto{\pgfqpoint{4.768000in}{3.582142in}}%
\pgfpathlineto{\pgfqpoint{4.764289in}{3.585877in}}%
\pgfpathlineto{\pgfqpoint{4.760919in}{3.589333in}}%
\pgfpathlineto{\pgfqpoint{4.745045in}{3.605285in}}%
\pgfpathlineto{\pgfqpoint{4.727919in}{3.622820in}}%
\pgfpathlineto{\pgfqpoint{4.724119in}{3.626667in}}%
\pgfpathlineto{\pgfqpoint{4.706638in}{3.644178in}}%
\pgfpathlineto{\pgfqpoint{4.687838in}{3.663364in}}%
\pgfpathlineto{\pgfqpoint{4.687509in}{3.663693in}}%
\pgfpathlineto{\pgfqpoint{4.687208in}{3.664000in}}%
\pgfpathlineto{\pgfqpoint{4.668169in}{3.683012in}}%
\pgfpathlineto{\pgfqpoint{4.650160in}{3.701333in}}%
\pgfpathlineto{\pgfqpoint{4.647758in}{3.703752in}}%
\pgfpathlineto{\pgfqpoint{4.612993in}{3.738667in}}%
\pgfpathlineto{\pgfqpoint{4.607677in}{3.744002in}}%
\pgfpathlineto{\pgfqpoint{4.591042in}{3.760506in}}%
\pgfpathlineto{\pgfqpoint{4.575715in}{3.776000in}}%
\pgfpathlineto{\pgfqpoint{4.571781in}{3.779898in}}%
\pgfpathlineto{\pgfqpoint{4.567596in}{3.784122in}}%
\pgfpathlineto{\pgfqpoint{4.538324in}{3.813333in}}%
\pgfpathlineto{\pgfqpoint{4.533078in}{3.818515in}}%
\pgfpathlineto{\pgfqpoint{4.527515in}{3.824112in}}%
\pgfpathlineto{\pgfqpoint{4.500820in}{3.850667in}}%
\pgfpathlineto{\pgfqpoint{4.494313in}{3.857074in}}%
\pgfpathlineto{\pgfqpoint{4.487434in}{3.863973in}}%
\pgfpathlineto{\pgfqpoint{4.463202in}{3.888000in}}%
\pgfpathlineto{\pgfqpoint{4.455485in}{3.895574in}}%
\pgfpathlineto{\pgfqpoint{4.447354in}{3.903704in}}%
\pgfpathlineto{\pgfqpoint{4.425469in}{3.925333in}}%
\pgfpathlineto{\pgfqpoint{4.407273in}{3.943306in}}%
\pgfpathlineto{\pgfqpoint{4.397124in}{3.953214in}}%
\pgfpathlineto{\pgfqpoint{4.387621in}{3.962667in}}%
\pgfpathlineto{\pgfqpoint{4.367192in}{3.982779in}}%
\pgfpathlineto{\pgfqpoint{4.349655in}{4.000000in}}%
\pgfpathlineto{\pgfqpoint{4.338623in}{4.010723in}}%
\pgfpathlineto{\pgfqpoint{4.327111in}{4.022123in}}%
\pgfpathlineto{\pgfqpoint{4.311572in}{4.037333in}}%
\pgfpathlineto{\pgfqpoint{4.287030in}{4.061339in}}%
\pgfpathlineto{\pgfqpoint{4.273371in}{4.074667in}}%
\pgfpathlineto{\pgfqpoint{4.246949in}{4.100427in}}%
\pgfpathlineto{\pgfqpoint{4.240843in}{4.106312in}}%
\pgfpathlineto{\pgfqpoint{4.235050in}{4.112000in}}%
\pgfpathlineto{\pgfqpoint{4.206869in}{4.139387in}}%
\pgfpathlineto{\pgfqpoint{4.201612in}{4.144437in}}%
\pgfpathlineto{\pgfqpoint{4.196608in}{4.149333in}}%
\pgfpathlineto{\pgfqpoint{4.166788in}{4.178220in}}%
\pgfpathlineto{\pgfqpoint{4.162316in}{4.182501in}}%
\pgfpathlineto{\pgfqpoint{4.158046in}{4.186667in}}%
\pgfpathlineto{\pgfqpoint{4.126707in}{4.216925in}}%
\pgfpathlineto{\pgfqpoint{4.122955in}{4.220505in}}%
\pgfpathlineto{\pgfqpoint{4.119361in}{4.224000in}}%
\pgfpathlineto{\pgfqpoint{4.116547in}{4.224000in}}%
\pgfpathlineto{\pgfqpoint{4.121518in}{4.219167in}}%
\pgfpathlineto{\pgfqpoint{4.126707in}{4.214216in}}%
\pgfpathlineto{\pgfqpoint{4.155240in}{4.186667in}}%
\pgfpathlineto{\pgfqpoint{4.160880in}{4.181164in}}%
\pgfpathlineto{\pgfqpoint{4.166788in}{4.175509in}}%
\pgfpathlineto{\pgfqpoint{4.193810in}{4.149333in}}%
\pgfpathlineto{\pgfqpoint{4.200178in}{4.143101in}}%
\pgfpathlineto{\pgfqpoint{4.206869in}{4.136675in}}%
\pgfpathlineto{\pgfqpoint{4.232258in}{4.112000in}}%
\pgfpathlineto{\pgfqpoint{4.239410in}{4.104977in}}%
\pgfpathlineto{\pgfqpoint{4.246949in}{4.097713in}}%
\pgfpathlineto{\pgfqpoint{4.270586in}{4.074667in}}%
\pgfpathlineto{\pgfqpoint{4.287030in}{4.058623in}}%
\pgfpathlineto{\pgfqpoint{4.308795in}{4.037333in}}%
\pgfpathlineto{\pgfqpoint{4.327111in}{4.019405in}}%
\pgfpathlineto{\pgfqpoint{4.337209in}{4.009405in}}%
\pgfpathlineto{\pgfqpoint{4.346885in}{4.000000in}}%
\pgfpathlineto{\pgfqpoint{4.367192in}{3.980059in}}%
\pgfpathlineto{\pgfqpoint{4.384858in}{3.962667in}}%
\pgfpathlineto{\pgfqpoint{4.395697in}{3.951884in}}%
\pgfpathlineto{\pgfqpoint{4.407273in}{3.940584in}}%
\pgfpathlineto{\pgfqpoint{4.422713in}{3.925333in}}%
\pgfpathlineto{\pgfqpoint{4.447354in}{3.900980in}}%
\pgfpathlineto{\pgfqpoint{4.454075in}{3.894260in}}%
\pgfpathlineto{\pgfqpoint{4.460453in}{3.888000in}}%
\pgfpathlineto{\pgfqpoint{4.487434in}{3.861247in}}%
\pgfpathlineto{\pgfqpoint{4.492904in}{3.855761in}}%
\pgfpathlineto{\pgfqpoint{4.498078in}{3.850667in}}%
\pgfpathlineto{\pgfqpoint{4.527515in}{3.821385in}}%
\pgfpathlineto{\pgfqpoint{4.531670in}{3.817204in}}%
\pgfpathlineto{\pgfqpoint{4.535589in}{3.813333in}}%
\pgfpathlineto{\pgfqpoint{4.567596in}{3.781393in}}%
\pgfpathlineto{\pgfqpoint{4.570374in}{3.778588in}}%
\pgfpathlineto{\pgfqpoint{4.572986in}{3.776000in}}%
\pgfpathlineto{\pgfqpoint{4.589622in}{3.759183in}}%
\pgfpathlineto{\pgfqpoint{4.607677in}{3.741271in}}%
\pgfpathlineto{\pgfqpoint{4.610271in}{3.738667in}}%
\pgfpathlineto{\pgfqpoint{4.643437in}{3.705358in}}%
\pgfpathlineto{\pgfqpoint{4.647441in}{3.701333in}}%
\pgfpathlineto{\pgfqpoint{4.647593in}{3.701180in}}%
\pgfpathlineto{\pgfqpoint{4.647758in}{3.701015in}}%
\pgfpathlineto{\pgfqpoint{4.666751in}{3.681692in}}%
\pgfpathlineto{\pgfqpoint{4.684469in}{3.664000in}}%
\pgfpathlineto{\pgfqpoint{4.686078in}{3.662361in}}%
\pgfpathlineto{\pgfqpoint{4.687838in}{3.660601in}}%
\pgfpathlineto{\pgfqpoint{4.705222in}{3.642859in}}%
\pgfpathlineto{\pgfqpoint{4.721386in}{3.626667in}}%
\pgfpathlineto{\pgfqpoint{4.727919in}{3.620054in}}%
\pgfpathlineto{\pgfqpoint{4.743631in}{3.603968in}}%
\pgfpathlineto{\pgfqpoint{4.758193in}{3.589333in}}%
\pgfpathlineto{\pgfqpoint{4.762861in}{3.584547in}}%
\pgfpathlineto{\pgfqpoint{4.768000in}{3.579375in}}%
\pgfusepath{fill}%
\end{pgfscope}%
\begin{pgfscope}%
\pgfpathrectangle{\pgfqpoint{0.800000in}{0.528000in}}{\pgfqpoint{3.968000in}{3.696000in}}%
\pgfusepath{clip}%
\pgfsetbuttcap%
\pgfsetroundjoin%
\definecolor{currentfill}{rgb}{0.327796,0.773980,0.406640}%
\pgfsetfillcolor{currentfill}%
\pgfsetlinewidth{0.000000pt}%
\definecolor{currentstroke}{rgb}{0.000000,0.000000,0.000000}%
\pgfsetstrokecolor{currentstroke}%
\pgfsetdash{}{0pt}%
\pgfpathmoveto{\pgfqpoint{4.768000in}{3.584910in}}%
\pgfpathlineto{\pgfqpoint{4.765717in}{3.587207in}}%
\pgfpathlineto{\pgfqpoint{4.763644in}{3.589333in}}%
\pgfpathlineto{\pgfqpoint{4.746460in}{3.606603in}}%
\pgfpathlineto{\pgfqpoint{4.727919in}{3.625585in}}%
\pgfpathlineto{\pgfqpoint{4.726851in}{3.626667in}}%
\pgfpathlineto{\pgfqpoint{4.708054in}{3.645496in}}%
\pgfpathlineto{\pgfqpoint{4.689924in}{3.664000in}}%
\pgfpathlineto{\pgfqpoint{4.687838in}{3.666106in}}%
\pgfpathlineto{\pgfqpoint{4.669586in}{3.684332in}}%
\pgfpathlineto{\pgfqpoint{4.652874in}{3.701333in}}%
\pgfpathlineto{\pgfqpoint{4.647758in}{3.706485in}}%
\pgfpathlineto{\pgfqpoint{4.615714in}{3.738667in}}%
\pgfpathlineto{\pgfqpoint{4.607677in}{3.746734in}}%
\pgfpathlineto{\pgfqpoint{4.592462in}{3.761828in}}%
\pgfpathlineto{\pgfqpoint{4.578443in}{3.776000in}}%
\pgfpathlineto{\pgfqpoint{4.573187in}{3.781208in}}%
\pgfpathlineto{\pgfqpoint{4.567596in}{3.786852in}}%
\pgfpathlineto{\pgfqpoint{4.541059in}{3.813333in}}%
\pgfpathlineto{\pgfqpoint{4.534486in}{3.819826in}}%
\pgfpathlineto{\pgfqpoint{4.527515in}{3.826840in}}%
\pgfpathlineto{\pgfqpoint{4.503562in}{3.850667in}}%
\pgfpathlineto{\pgfqpoint{4.495722in}{3.858386in}}%
\pgfpathlineto{\pgfqpoint{4.487434in}{3.866699in}}%
\pgfpathlineto{\pgfqpoint{4.465951in}{3.888000in}}%
\pgfpathlineto{\pgfqpoint{4.456895in}{3.896888in}}%
\pgfpathlineto{\pgfqpoint{4.447354in}{3.906428in}}%
\pgfpathlineto{\pgfqpoint{4.428226in}{3.925333in}}%
\pgfpathlineto{\pgfqpoint{4.407273in}{3.946028in}}%
\pgfpathlineto{\pgfqpoint{4.398551in}{3.954543in}}%
\pgfpathlineto{\pgfqpoint{4.390384in}{3.962667in}}%
\pgfpathlineto{\pgfqpoint{4.367192in}{3.985499in}}%
\pgfpathlineto{\pgfqpoint{4.352425in}{4.000000in}}%
\pgfpathlineto{\pgfqpoint{4.340038in}{4.012041in}}%
\pgfpathlineto{\pgfqpoint{4.327111in}{4.024842in}}%
\pgfpathlineto{\pgfqpoint{4.314349in}{4.037333in}}%
\pgfpathlineto{\pgfqpoint{4.287030in}{4.064056in}}%
\pgfpathlineto{\pgfqpoint{4.276155in}{4.074667in}}%
\pgfpathlineto{\pgfqpoint{4.246949in}{4.103142in}}%
\pgfpathlineto{\pgfqpoint{4.242275in}{4.107646in}}%
\pgfpathlineto{\pgfqpoint{4.237841in}{4.112000in}}%
\pgfpathlineto{\pgfqpoint{4.206869in}{4.142100in}}%
\pgfpathlineto{\pgfqpoint{4.203046in}{4.145772in}}%
\pgfpathlineto{\pgfqpoint{4.199407in}{4.149333in}}%
\pgfpathlineto{\pgfqpoint{4.166788in}{4.180931in}}%
\pgfpathlineto{\pgfqpoint{4.163751in}{4.183838in}}%
\pgfpathlineto{\pgfqpoint{4.160852in}{4.186667in}}%
\pgfpathlineto{\pgfqpoint{4.126707in}{4.219634in}}%
\pgfpathlineto{\pgfqpoint{4.124392in}{4.221843in}}%
\pgfpathlineto{\pgfqpoint{4.122174in}{4.224000in}}%
\pgfpathlineto{\pgfqpoint{4.119361in}{4.224000in}}%
\pgfpathlineto{\pgfqpoint{4.122955in}{4.220505in}}%
\pgfpathlineto{\pgfqpoint{4.126707in}{4.216925in}}%
\pgfpathlineto{\pgfqpoint{4.158046in}{4.186667in}}%
\pgfpathlineto{\pgfqpoint{4.162316in}{4.182501in}}%
\pgfpathlineto{\pgfqpoint{4.166788in}{4.178220in}}%
\pgfpathlineto{\pgfqpoint{4.196608in}{4.149333in}}%
\pgfpathlineto{\pgfqpoint{4.201612in}{4.144437in}}%
\pgfpathlineto{\pgfqpoint{4.206869in}{4.139387in}}%
\pgfpathlineto{\pgfqpoint{4.235050in}{4.112000in}}%
\pgfpathlineto{\pgfqpoint{4.240843in}{4.106312in}}%
\pgfpathlineto{\pgfqpoint{4.246949in}{4.100427in}}%
\pgfpathlineto{\pgfqpoint{4.273371in}{4.074667in}}%
\pgfpathlineto{\pgfqpoint{4.287030in}{4.061339in}}%
\pgfpathlineto{\pgfqpoint{4.311572in}{4.037333in}}%
\pgfpathlineto{\pgfqpoint{4.327111in}{4.022123in}}%
\pgfpathlineto{\pgfqpoint{4.338623in}{4.010723in}}%
\pgfpathlineto{\pgfqpoint{4.349655in}{4.000000in}}%
\pgfpathlineto{\pgfqpoint{4.367192in}{3.982779in}}%
\pgfpathlineto{\pgfqpoint{4.387621in}{3.962667in}}%
\pgfpathlineto{\pgfqpoint{4.397124in}{3.953214in}}%
\pgfpathlineto{\pgfqpoint{4.407273in}{3.943306in}}%
\pgfpathlineto{\pgfqpoint{4.425469in}{3.925333in}}%
\pgfpathlineto{\pgfqpoint{4.447354in}{3.903704in}}%
\pgfpathlineto{\pgfqpoint{4.455485in}{3.895574in}}%
\pgfpathlineto{\pgfqpoint{4.463202in}{3.888000in}}%
\pgfpathlineto{\pgfqpoint{4.487434in}{3.863973in}}%
\pgfpathlineto{\pgfqpoint{4.494313in}{3.857074in}}%
\pgfpathlineto{\pgfqpoint{4.500820in}{3.850667in}}%
\pgfpathlineto{\pgfqpoint{4.527515in}{3.824112in}}%
\pgfpathlineto{\pgfqpoint{4.533078in}{3.818515in}}%
\pgfpathlineto{\pgfqpoint{4.538324in}{3.813333in}}%
\pgfpathlineto{\pgfqpoint{4.567596in}{3.784122in}}%
\pgfpathlineto{\pgfqpoint{4.571781in}{3.779898in}}%
\pgfpathlineto{\pgfqpoint{4.575715in}{3.776000in}}%
\pgfpathlineto{\pgfqpoint{4.591042in}{3.760506in}}%
\pgfpathlineto{\pgfqpoint{4.607677in}{3.744002in}}%
\pgfpathlineto{\pgfqpoint{4.612993in}{3.738667in}}%
\pgfpathlineto{\pgfqpoint{4.647758in}{3.703752in}}%
\pgfpathlineto{\pgfqpoint{4.650160in}{3.701333in}}%
\pgfpathlineto{\pgfqpoint{4.668169in}{3.683012in}}%
\pgfpathlineto{\pgfqpoint{4.687208in}{3.664000in}}%
\pgfpathlineto{\pgfqpoint{4.687509in}{3.663693in}}%
\pgfpathlineto{\pgfqpoint{4.687838in}{3.663364in}}%
\pgfpathlineto{\pgfqpoint{4.706638in}{3.644178in}}%
\pgfpathlineto{\pgfqpoint{4.724119in}{3.626667in}}%
\pgfpathlineto{\pgfqpoint{4.727919in}{3.622820in}}%
\pgfpathlineto{\pgfqpoint{4.745045in}{3.605285in}}%
\pgfpathlineto{\pgfqpoint{4.760919in}{3.589333in}}%
\pgfpathlineto{\pgfqpoint{4.764289in}{3.585877in}}%
\pgfpathlineto{\pgfqpoint{4.768000in}{3.582142in}}%
\pgfusepath{fill}%
\end{pgfscope}%
\begin{pgfscope}%
\pgfpathrectangle{\pgfqpoint{0.800000in}{0.528000in}}{\pgfqpoint{3.968000in}{3.696000in}}%
\pgfusepath{clip}%
\pgfsetbuttcap%
\pgfsetroundjoin%
\definecolor{currentfill}{rgb}{0.335885,0.777018,0.402049}%
\pgfsetfillcolor{currentfill}%
\pgfsetlinewidth{0.000000pt}%
\definecolor{currentstroke}{rgb}{0.000000,0.000000,0.000000}%
\pgfsetstrokecolor{currentstroke}%
\pgfsetdash{}{0pt}%
\pgfpathmoveto{\pgfqpoint{4.768000in}{3.587678in}}%
\pgfpathlineto{\pgfqpoint{4.767146in}{3.588537in}}%
\pgfpathlineto{\pgfqpoint{4.766369in}{3.589333in}}%
\pgfpathlineto{\pgfqpoint{4.747874in}{3.607920in}}%
\pgfpathlineto{\pgfqpoint{4.729564in}{3.626667in}}%
\pgfpathlineto{\pgfqpoint{4.727919in}{3.628334in}}%
\pgfpathlineto{\pgfqpoint{4.709470in}{3.646815in}}%
\pgfpathlineto{\pgfqpoint{4.692632in}{3.664000in}}%
\pgfpathlineto{\pgfqpoint{4.687838in}{3.668841in}}%
\pgfpathlineto{\pgfqpoint{4.671003in}{3.685652in}}%
\pgfpathlineto{\pgfqpoint{4.655589in}{3.701333in}}%
\pgfpathlineto{\pgfqpoint{4.647758in}{3.709218in}}%
\pgfpathlineto{\pgfqpoint{4.618436in}{3.738667in}}%
\pgfpathlineto{\pgfqpoint{4.607677in}{3.749465in}}%
\pgfpathlineto{\pgfqpoint{4.593882in}{3.763151in}}%
\pgfpathlineto{\pgfqpoint{4.581171in}{3.776000in}}%
\pgfpathlineto{\pgfqpoint{4.574594in}{3.782518in}}%
\pgfpathlineto{\pgfqpoint{4.567596in}{3.789581in}}%
\pgfpathlineto{\pgfqpoint{4.543795in}{3.813333in}}%
\pgfpathlineto{\pgfqpoint{4.535894in}{3.821137in}}%
\pgfpathlineto{\pgfqpoint{4.527515in}{3.829568in}}%
\pgfpathlineto{\pgfqpoint{4.506305in}{3.850667in}}%
\pgfpathlineto{\pgfqpoint{4.497131in}{3.859699in}}%
\pgfpathlineto{\pgfqpoint{4.487434in}{3.869425in}}%
\pgfpathlineto{\pgfqpoint{4.468701in}{3.888000in}}%
\pgfpathlineto{\pgfqpoint{4.458306in}{3.898202in}}%
\pgfpathlineto{\pgfqpoint{4.447354in}{3.909152in}}%
\pgfpathlineto{\pgfqpoint{4.430982in}{3.925333in}}%
\pgfpathlineto{\pgfqpoint{4.407273in}{3.948750in}}%
\pgfpathlineto{\pgfqpoint{4.399978in}{3.955872in}}%
\pgfpathlineto{\pgfqpoint{4.393147in}{3.962667in}}%
\pgfpathlineto{\pgfqpoint{4.367192in}{3.988220in}}%
\pgfpathlineto{\pgfqpoint{4.355196in}{4.000000in}}%
\pgfpathlineto{\pgfqpoint{4.341453in}{4.013358in}}%
\pgfpathlineto{\pgfqpoint{4.327111in}{4.027560in}}%
\pgfpathlineto{\pgfqpoint{4.317127in}{4.037333in}}%
\pgfpathlineto{\pgfqpoint{4.287030in}{4.066773in}}%
\pgfpathlineto{\pgfqpoint{4.278939in}{4.074667in}}%
\pgfpathlineto{\pgfqpoint{4.246949in}{4.105857in}}%
\pgfpathlineto{\pgfqpoint{4.243708in}{4.108980in}}%
\pgfpathlineto{\pgfqpoint{4.240633in}{4.112000in}}%
\pgfpathlineto{\pgfqpoint{4.206869in}{4.144813in}}%
\pgfpathlineto{\pgfqpoint{4.204479in}{4.147108in}}%
\pgfpathlineto{\pgfqpoint{4.202206in}{4.149333in}}%
\pgfpathlineto{\pgfqpoint{4.166788in}{4.183642in}}%
\pgfpathlineto{\pgfqpoint{4.165186in}{4.185175in}}%
\pgfpathlineto{\pgfqpoint{4.163657in}{4.186667in}}%
\pgfpathlineto{\pgfqpoint{4.126707in}{4.222344in}}%
\pgfpathlineto{\pgfqpoint{4.125829in}{4.223182in}}%
\pgfpathlineto{\pgfqpoint{4.124987in}{4.224000in}}%
\pgfpathlineto{\pgfqpoint{4.122174in}{4.224000in}}%
\pgfpathlineto{\pgfqpoint{4.124392in}{4.221843in}}%
\pgfpathlineto{\pgfqpoint{4.126707in}{4.219634in}}%
\pgfpathlineto{\pgfqpoint{4.160852in}{4.186667in}}%
\pgfpathlineto{\pgfqpoint{4.163751in}{4.183838in}}%
\pgfpathlineto{\pgfqpoint{4.166788in}{4.180931in}}%
\pgfpathlineto{\pgfqpoint{4.199407in}{4.149333in}}%
\pgfpathlineto{\pgfqpoint{4.203046in}{4.145772in}}%
\pgfpathlineto{\pgfqpoint{4.206869in}{4.142100in}}%
\pgfpathlineto{\pgfqpoint{4.237841in}{4.112000in}}%
\pgfpathlineto{\pgfqpoint{4.242275in}{4.107646in}}%
\pgfpathlineto{\pgfqpoint{4.246949in}{4.103142in}}%
\pgfpathlineto{\pgfqpoint{4.276155in}{4.074667in}}%
\pgfpathlineto{\pgfqpoint{4.287030in}{4.064056in}}%
\pgfpathlineto{\pgfqpoint{4.314349in}{4.037333in}}%
\pgfpathlineto{\pgfqpoint{4.327111in}{4.024842in}}%
\pgfpathlineto{\pgfqpoint{4.340038in}{4.012041in}}%
\pgfpathlineto{\pgfqpoint{4.352425in}{4.000000in}}%
\pgfpathlineto{\pgfqpoint{4.367192in}{3.985499in}}%
\pgfpathlineto{\pgfqpoint{4.390384in}{3.962667in}}%
\pgfpathlineto{\pgfqpoint{4.398551in}{3.954543in}}%
\pgfpathlineto{\pgfqpoint{4.407273in}{3.946028in}}%
\pgfpathlineto{\pgfqpoint{4.428226in}{3.925333in}}%
\pgfpathlineto{\pgfqpoint{4.447354in}{3.906428in}}%
\pgfpathlineto{\pgfqpoint{4.456895in}{3.896888in}}%
\pgfpathlineto{\pgfqpoint{4.465951in}{3.888000in}}%
\pgfpathlineto{\pgfqpoint{4.487434in}{3.866699in}}%
\pgfpathlineto{\pgfqpoint{4.495722in}{3.858386in}}%
\pgfpathlineto{\pgfqpoint{4.503562in}{3.850667in}}%
\pgfpathlineto{\pgfqpoint{4.527515in}{3.826840in}}%
\pgfpathlineto{\pgfqpoint{4.534486in}{3.819826in}}%
\pgfpathlineto{\pgfqpoint{4.541059in}{3.813333in}}%
\pgfpathlineto{\pgfqpoint{4.567596in}{3.786852in}}%
\pgfpathlineto{\pgfqpoint{4.573187in}{3.781208in}}%
\pgfpathlineto{\pgfqpoint{4.578443in}{3.776000in}}%
\pgfpathlineto{\pgfqpoint{4.592462in}{3.761828in}}%
\pgfpathlineto{\pgfqpoint{4.607677in}{3.746734in}}%
\pgfpathlineto{\pgfqpoint{4.615714in}{3.738667in}}%
\pgfpathlineto{\pgfqpoint{4.647758in}{3.706485in}}%
\pgfpathlineto{\pgfqpoint{4.652874in}{3.701333in}}%
\pgfpathlineto{\pgfqpoint{4.669586in}{3.684332in}}%
\pgfpathlineto{\pgfqpoint{4.687838in}{3.666106in}}%
\pgfpathlineto{\pgfqpoint{4.689924in}{3.664000in}}%
\pgfpathlineto{\pgfqpoint{4.708054in}{3.645496in}}%
\pgfpathlineto{\pgfqpoint{4.726851in}{3.626667in}}%
\pgfpathlineto{\pgfqpoint{4.727919in}{3.625585in}}%
\pgfpathlineto{\pgfqpoint{4.746460in}{3.606603in}}%
\pgfpathlineto{\pgfqpoint{4.763644in}{3.589333in}}%
\pgfpathlineto{\pgfqpoint{4.765717in}{3.587207in}}%
\pgfpathlineto{\pgfqpoint{4.768000in}{3.584910in}}%
\pgfusepath{fill}%
\end{pgfscope}%
\begin{pgfscope}%
\pgfpathrectangle{\pgfqpoint{0.800000in}{0.528000in}}{\pgfqpoint{3.968000in}{3.696000in}}%
\pgfusepath{clip}%
\pgfsetbuttcap%
\pgfsetroundjoin%
\definecolor{currentfill}{rgb}{0.335885,0.777018,0.402049}%
\pgfsetfillcolor{currentfill}%
\pgfsetlinewidth{0.000000pt}%
\definecolor{currentstroke}{rgb}{0.000000,0.000000,0.000000}%
\pgfsetstrokecolor{currentstroke}%
\pgfsetdash{}{0pt}%
\pgfpathmoveto{\pgfqpoint{4.768000in}{3.590434in}}%
\pgfpathlineto{\pgfqpoint{4.749288in}{3.609238in}}%
\pgfpathlineto{\pgfqpoint{4.732265in}{3.626667in}}%
\pgfpathlineto{\pgfqpoint{4.727919in}{3.631071in}}%
\pgfpathlineto{\pgfqpoint{4.710885in}{3.648134in}}%
\pgfpathlineto{\pgfqpoint{4.695340in}{3.664000in}}%
\pgfpathlineto{\pgfqpoint{4.687838in}{3.671577in}}%
\pgfpathlineto{\pgfqpoint{4.672420in}{3.686972in}}%
\pgfpathlineto{\pgfqpoint{4.658304in}{3.701333in}}%
\pgfpathlineto{\pgfqpoint{4.647758in}{3.711952in}}%
\pgfpathlineto{\pgfqpoint{4.621158in}{3.738667in}}%
\pgfpathlineto{\pgfqpoint{4.607677in}{3.752196in}}%
\pgfpathlineto{\pgfqpoint{4.595302in}{3.764474in}}%
\pgfpathlineto{\pgfqpoint{4.583900in}{3.776000in}}%
\pgfpathlineto{\pgfqpoint{4.576000in}{3.783828in}}%
\pgfpathlineto{\pgfqpoint{4.567596in}{3.792311in}}%
\pgfpathlineto{\pgfqpoint{4.546530in}{3.813333in}}%
\pgfpathlineto{\pgfqpoint{4.537301in}{3.822449in}}%
\pgfpathlineto{\pgfqpoint{4.527515in}{3.832296in}}%
\pgfpathlineto{\pgfqpoint{4.509047in}{3.850667in}}%
\pgfpathlineto{\pgfqpoint{4.498540in}{3.861011in}}%
\pgfpathlineto{\pgfqpoint{4.487434in}{3.872151in}}%
\pgfpathlineto{\pgfqpoint{4.471450in}{3.888000in}}%
\pgfpathlineto{\pgfqpoint{4.459716in}{3.899515in}}%
\pgfpathlineto{\pgfqpoint{4.447354in}{3.911876in}}%
\pgfpathlineto{\pgfqpoint{4.433738in}{3.925333in}}%
\pgfpathlineto{\pgfqpoint{4.407273in}{3.951472in}}%
\pgfpathlineto{\pgfqpoint{4.401405in}{3.957201in}}%
\pgfpathlineto{\pgfqpoint{4.395910in}{3.962667in}}%
\pgfpathlineto{\pgfqpoint{4.367192in}{3.990940in}}%
\pgfpathlineto{\pgfqpoint{4.357966in}{4.000000in}}%
\pgfpathlineto{\pgfqpoint{4.342867in}{4.014676in}}%
\pgfpathlineto{\pgfqpoint{4.327111in}{4.030279in}}%
\pgfpathlineto{\pgfqpoint{4.319904in}{4.037333in}}%
\pgfpathlineto{\pgfqpoint{4.287030in}{4.069489in}}%
\pgfpathlineto{\pgfqpoint{4.281724in}{4.074667in}}%
\pgfpathlineto{\pgfqpoint{4.246949in}{4.108572in}}%
\pgfpathlineto{\pgfqpoint{4.245140in}{4.110315in}}%
\pgfpathlineto{\pgfqpoint{4.243424in}{4.112000in}}%
\pgfpathlineto{\pgfqpoint{4.206869in}{4.147526in}}%
\pgfpathlineto{\pgfqpoint{4.205913in}{4.148444in}}%
\pgfpathlineto{\pgfqpoint{4.205004in}{4.149333in}}%
\pgfpathlineto{\pgfqpoint{4.166788in}{4.186353in}}%
\pgfpathlineto{\pgfqpoint{4.166622in}{4.186512in}}%
\pgfpathlineto{\pgfqpoint{4.166463in}{4.186667in}}%
\pgfpathlineto{\pgfqpoint{4.157595in}{4.195229in}}%
\pgfpathlineto{\pgfqpoint{4.127787in}{4.224000in}}%
\pgfpathlineto{\pgfqpoint{4.126707in}{4.224000in}}%
\pgfpathlineto{\pgfqpoint{4.124987in}{4.224000in}}%
\pgfpathlineto{\pgfqpoint{4.125829in}{4.223182in}}%
\pgfpathlineto{\pgfqpoint{4.126707in}{4.222344in}}%
\pgfpathlineto{\pgfqpoint{4.163657in}{4.186667in}}%
\pgfpathlineto{\pgfqpoint{4.165186in}{4.185175in}}%
\pgfpathlineto{\pgfqpoint{4.166788in}{4.183642in}}%
\pgfpathlineto{\pgfqpoint{4.202206in}{4.149333in}}%
\pgfpathlineto{\pgfqpoint{4.204479in}{4.147108in}}%
\pgfpathlineto{\pgfqpoint{4.206869in}{4.144813in}}%
\pgfpathlineto{\pgfqpoint{4.240633in}{4.112000in}}%
\pgfpathlineto{\pgfqpoint{4.243708in}{4.108980in}}%
\pgfpathlineto{\pgfqpoint{4.246949in}{4.105857in}}%
\pgfpathlineto{\pgfqpoint{4.278939in}{4.074667in}}%
\pgfpathlineto{\pgfqpoint{4.287030in}{4.066773in}}%
\pgfpathlineto{\pgfqpoint{4.317127in}{4.037333in}}%
\pgfpathlineto{\pgfqpoint{4.327111in}{4.027560in}}%
\pgfpathlineto{\pgfqpoint{4.341453in}{4.013358in}}%
\pgfpathlineto{\pgfqpoint{4.355196in}{4.000000in}}%
\pgfpathlineto{\pgfqpoint{4.367192in}{3.988220in}}%
\pgfpathlineto{\pgfqpoint{4.393147in}{3.962667in}}%
\pgfpathlineto{\pgfqpoint{4.399978in}{3.955872in}}%
\pgfpathlineto{\pgfqpoint{4.407273in}{3.948750in}}%
\pgfpathlineto{\pgfqpoint{4.430982in}{3.925333in}}%
\pgfpathlineto{\pgfqpoint{4.447354in}{3.909152in}}%
\pgfpathlineto{\pgfqpoint{4.458306in}{3.898202in}}%
\pgfpathlineto{\pgfqpoint{4.468701in}{3.888000in}}%
\pgfpathlineto{\pgfqpoint{4.487434in}{3.869425in}}%
\pgfpathlineto{\pgfqpoint{4.497131in}{3.859699in}}%
\pgfpathlineto{\pgfqpoint{4.506305in}{3.850667in}}%
\pgfpathlineto{\pgfqpoint{4.527515in}{3.829568in}}%
\pgfpathlineto{\pgfqpoint{4.535894in}{3.821137in}}%
\pgfpathlineto{\pgfqpoint{4.543795in}{3.813333in}}%
\pgfpathlineto{\pgfqpoint{4.567596in}{3.789581in}}%
\pgfpathlineto{\pgfqpoint{4.574594in}{3.782518in}}%
\pgfpathlineto{\pgfqpoint{4.581171in}{3.776000in}}%
\pgfpathlineto{\pgfqpoint{4.593882in}{3.763151in}}%
\pgfpathlineto{\pgfqpoint{4.607677in}{3.749465in}}%
\pgfpathlineto{\pgfqpoint{4.618436in}{3.738667in}}%
\pgfpathlineto{\pgfqpoint{4.647758in}{3.709218in}}%
\pgfpathlineto{\pgfqpoint{4.655589in}{3.701333in}}%
\pgfpathlineto{\pgfqpoint{4.671003in}{3.685652in}}%
\pgfpathlineto{\pgfqpoint{4.687838in}{3.668841in}}%
\pgfpathlineto{\pgfqpoint{4.692632in}{3.664000in}}%
\pgfpathlineto{\pgfqpoint{4.709470in}{3.646815in}}%
\pgfpathlineto{\pgfqpoint{4.727919in}{3.628334in}}%
\pgfpathlineto{\pgfqpoint{4.729564in}{3.626667in}}%
\pgfpathlineto{\pgfqpoint{4.747874in}{3.607920in}}%
\pgfpathlineto{\pgfqpoint{4.766369in}{3.589333in}}%
\pgfpathlineto{\pgfqpoint{4.767146in}{3.588537in}}%
\pgfpathlineto{\pgfqpoint{4.768000in}{3.587678in}}%
\pgfpathlineto{\pgfqpoint{4.768000in}{3.589333in}}%
\pgfusepath{fill}%
\end{pgfscope}%
\begin{pgfscope}%
\pgfpathrectangle{\pgfqpoint{0.800000in}{0.528000in}}{\pgfqpoint{3.968000in}{3.696000in}}%
\pgfusepath{clip}%
\pgfsetbuttcap%
\pgfsetroundjoin%
\definecolor{currentfill}{rgb}{0.335885,0.777018,0.402049}%
\pgfsetfillcolor{currentfill}%
\pgfsetlinewidth{0.000000pt}%
\definecolor{currentstroke}{rgb}{0.000000,0.000000,0.000000}%
\pgfsetstrokecolor{currentstroke}%
\pgfsetdash{}{0pt}%
\pgfpathmoveto{\pgfqpoint{4.768000in}{3.593173in}}%
\pgfpathlineto{\pgfqpoint{4.750703in}{3.610555in}}%
\pgfpathlineto{\pgfqpoint{4.734967in}{3.626667in}}%
\pgfpathlineto{\pgfqpoint{4.727919in}{3.633808in}}%
\pgfpathlineto{\pgfqpoint{4.712301in}{3.649453in}}%
\pgfpathlineto{\pgfqpoint{4.698048in}{3.664000in}}%
\pgfpathlineto{\pgfqpoint{4.687838in}{3.674312in}}%
\pgfpathlineto{\pgfqpoint{4.673837in}{3.688292in}}%
\pgfpathlineto{\pgfqpoint{4.661019in}{3.701333in}}%
\pgfpathlineto{\pgfqpoint{4.647758in}{3.714685in}}%
\pgfpathlineto{\pgfqpoint{4.623879in}{3.738667in}}%
\pgfpathlineto{\pgfqpoint{4.607677in}{3.754928in}}%
\pgfpathlineto{\pgfqpoint{4.596722in}{3.765796in}}%
\pgfpathlineto{\pgfqpoint{4.586628in}{3.776000in}}%
\pgfpathlineto{\pgfqpoint{4.577406in}{3.785138in}}%
\pgfpathlineto{\pgfqpoint{4.567596in}{3.795041in}}%
\pgfpathlineto{\pgfqpoint{4.549265in}{3.813333in}}%
\pgfpathlineto{\pgfqpoint{4.538709in}{3.823760in}}%
\pgfpathlineto{\pgfqpoint{4.527515in}{3.835023in}}%
\pgfpathlineto{\pgfqpoint{4.511789in}{3.850667in}}%
\pgfpathlineto{\pgfqpoint{4.499949in}{3.862324in}}%
\pgfpathlineto{\pgfqpoint{4.487434in}{3.874876in}}%
\pgfpathlineto{\pgfqpoint{4.474199in}{3.888000in}}%
\pgfpathlineto{\pgfqpoint{4.461127in}{3.900829in}}%
\pgfpathlineto{\pgfqpoint{4.447354in}{3.914600in}}%
\pgfpathlineto{\pgfqpoint{4.436494in}{3.925333in}}%
\pgfpathlineto{\pgfqpoint{4.407273in}{3.954195in}}%
\pgfpathlineto{\pgfqpoint{4.402832in}{3.958530in}}%
\pgfpathlineto{\pgfqpoint{4.398673in}{3.962667in}}%
\pgfpathlineto{\pgfqpoint{4.367192in}{3.993660in}}%
\pgfpathlineto{\pgfqpoint{4.360736in}{4.000000in}}%
\pgfpathlineto{\pgfqpoint{4.344282in}{4.015994in}}%
\pgfpathlineto{\pgfqpoint{4.327111in}{4.032997in}}%
\pgfpathlineto{\pgfqpoint{4.322681in}{4.037333in}}%
\pgfpathlineto{\pgfqpoint{4.287030in}{4.072206in}}%
\pgfpathlineto{\pgfqpoint{4.284508in}{4.074667in}}%
\pgfpathlineto{\pgfqpoint{4.246949in}{4.111286in}}%
\pgfpathlineto{\pgfqpoint{4.246573in}{4.111649in}}%
\pgfpathlineto{\pgfqpoint{4.246216in}{4.112000in}}%
\pgfpathlineto{\pgfqpoint{4.229292in}{4.128447in}}%
\pgfpathlineto{\pgfqpoint{4.207792in}{4.149333in}}%
\pgfpathlineto{\pgfqpoint{4.207338in}{4.149770in}}%
\pgfpathlineto{\pgfqpoint{4.206869in}{4.150230in}}%
\pgfpathlineto{\pgfqpoint{4.169240in}{4.186667in}}%
\pgfpathlineto{\pgfqpoint{4.168032in}{4.187825in}}%
\pgfpathlineto{\pgfqpoint{4.166788in}{4.189040in}}%
\pgfpathlineto{\pgfqpoint{4.130568in}{4.224000in}}%
\pgfpathlineto{\pgfqpoint{4.127787in}{4.224000in}}%
\pgfpathlineto{\pgfqpoint{4.157595in}{4.195229in}}%
\pgfpathlineto{\pgfqpoint{4.166463in}{4.186667in}}%
\pgfpathlineto{\pgfqpoint{4.166622in}{4.186512in}}%
\pgfpathlineto{\pgfqpoint{4.166788in}{4.186353in}}%
\pgfpathlineto{\pgfqpoint{4.205004in}{4.149333in}}%
\pgfpathlineto{\pgfqpoint{4.205913in}{4.148444in}}%
\pgfpathlineto{\pgfqpoint{4.206869in}{4.147526in}}%
\pgfpathlineto{\pgfqpoint{4.243424in}{4.112000in}}%
\pgfpathlineto{\pgfqpoint{4.245140in}{4.110315in}}%
\pgfpathlineto{\pgfqpoint{4.246949in}{4.108572in}}%
\pgfpathlineto{\pgfqpoint{4.281724in}{4.074667in}}%
\pgfpathlineto{\pgfqpoint{4.287030in}{4.069489in}}%
\pgfpathlineto{\pgfqpoint{4.319904in}{4.037333in}}%
\pgfpathlineto{\pgfqpoint{4.327111in}{4.030279in}}%
\pgfpathlineto{\pgfqpoint{4.342867in}{4.014676in}}%
\pgfpathlineto{\pgfqpoint{4.357966in}{4.000000in}}%
\pgfpathlineto{\pgfqpoint{4.367192in}{3.990940in}}%
\pgfpathlineto{\pgfqpoint{4.395910in}{3.962667in}}%
\pgfpathlineto{\pgfqpoint{4.401405in}{3.957201in}}%
\pgfpathlineto{\pgfqpoint{4.407273in}{3.951472in}}%
\pgfpathlineto{\pgfqpoint{4.433738in}{3.925333in}}%
\pgfpathlineto{\pgfqpoint{4.447354in}{3.911876in}}%
\pgfpathlineto{\pgfqpoint{4.459716in}{3.899515in}}%
\pgfpathlineto{\pgfqpoint{4.471450in}{3.888000in}}%
\pgfpathlineto{\pgfqpoint{4.487434in}{3.872151in}}%
\pgfpathlineto{\pgfqpoint{4.498540in}{3.861011in}}%
\pgfpathlineto{\pgfqpoint{4.509047in}{3.850667in}}%
\pgfpathlineto{\pgfqpoint{4.527515in}{3.832296in}}%
\pgfpathlineto{\pgfqpoint{4.537301in}{3.822449in}}%
\pgfpathlineto{\pgfqpoint{4.546530in}{3.813333in}}%
\pgfpathlineto{\pgfqpoint{4.567596in}{3.792311in}}%
\pgfpathlineto{\pgfqpoint{4.576000in}{3.783828in}}%
\pgfpathlineto{\pgfqpoint{4.583900in}{3.776000in}}%
\pgfpathlineto{\pgfqpoint{4.595302in}{3.764474in}}%
\pgfpathlineto{\pgfqpoint{4.607677in}{3.752196in}}%
\pgfpathlineto{\pgfqpoint{4.621158in}{3.738667in}}%
\pgfpathlineto{\pgfqpoint{4.647758in}{3.711952in}}%
\pgfpathlineto{\pgfqpoint{4.658304in}{3.701333in}}%
\pgfpathlineto{\pgfqpoint{4.672420in}{3.686972in}}%
\pgfpathlineto{\pgfqpoint{4.687838in}{3.671577in}}%
\pgfpathlineto{\pgfqpoint{4.695340in}{3.664000in}}%
\pgfpathlineto{\pgfqpoint{4.710885in}{3.648134in}}%
\pgfpathlineto{\pgfqpoint{4.727919in}{3.631071in}}%
\pgfpathlineto{\pgfqpoint{4.732265in}{3.626667in}}%
\pgfpathlineto{\pgfqpoint{4.749288in}{3.609238in}}%
\pgfpathlineto{\pgfqpoint{4.768000in}{3.590434in}}%
\pgfusepath{fill}%
\end{pgfscope}%
\begin{pgfscope}%
\pgfpathrectangle{\pgfqpoint{0.800000in}{0.528000in}}{\pgfqpoint{3.968000in}{3.696000in}}%
\pgfusepath{clip}%
\pgfsetbuttcap%
\pgfsetroundjoin%
\definecolor{currentfill}{rgb}{0.335885,0.777018,0.402049}%
\pgfsetfillcolor{currentfill}%
\pgfsetlinewidth{0.000000pt}%
\definecolor{currentstroke}{rgb}{0.000000,0.000000,0.000000}%
\pgfsetstrokecolor{currentstroke}%
\pgfsetdash{}{0pt}%
\pgfpathmoveto{\pgfqpoint{4.768000in}{3.595911in}}%
\pgfpathlineto{\pgfqpoint{4.752117in}{3.611873in}}%
\pgfpathlineto{\pgfqpoint{4.737668in}{3.626667in}}%
\pgfpathlineto{\pgfqpoint{4.727919in}{3.636545in}}%
\pgfpathlineto{\pgfqpoint{4.713717in}{3.650771in}}%
\pgfpathlineto{\pgfqpoint{4.700756in}{3.664000in}}%
\pgfpathlineto{\pgfqpoint{4.687838in}{3.677047in}}%
\pgfpathlineto{\pgfqpoint{4.675255in}{3.689612in}}%
\pgfpathlineto{\pgfqpoint{4.663733in}{3.701333in}}%
\pgfpathlineto{\pgfqpoint{4.647758in}{3.717418in}}%
\pgfpathlineto{\pgfqpoint{4.626601in}{3.738667in}}%
\pgfpathlineto{\pgfqpoint{4.607677in}{3.757659in}}%
\pgfpathlineto{\pgfqpoint{4.598142in}{3.767119in}}%
\pgfpathlineto{\pgfqpoint{4.589357in}{3.776000in}}%
\pgfpathlineto{\pgfqpoint{4.578813in}{3.786448in}}%
\pgfpathlineto{\pgfqpoint{4.567596in}{3.797770in}}%
\pgfpathlineto{\pgfqpoint{4.552000in}{3.813333in}}%
\pgfpathlineto{\pgfqpoint{4.540117in}{3.825071in}}%
\pgfpathlineto{\pgfqpoint{4.527515in}{3.837751in}}%
\pgfpathlineto{\pgfqpoint{4.514531in}{3.850667in}}%
\pgfpathlineto{\pgfqpoint{4.501358in}{3.863636in}}%
\pgfpathlineto{\pgfqpoint{4.487434in}{3.877602in}}%
\pgfpathlineto{\pgfqpoint{4.476948in}{3.888000in}}%
\pgfpathlineto{\pgfqpoint{4.462537in}{3.902143in}}%
\pgfpathlineto{\pgfqpoint{4.447354in}{3.917324in}}%
\pgfpathlineto{\pgfqpoint{4.439250in}{3.925333in}}%
\pgfpathlineto{\pgfqpoint{4.407273in}{3.956917in}}%
\pgfpathlineto{\pgfqpoint{4.404259in}{3.959859in}}%
\pgfpathlineto{\pgfqpoint{4.401436in}{3.962667in}}%
\pgfpathlineto{\pgfqpoint{4.367192in}{3.996380in}}%
\pgfpathlineto{\pgfqpoint{4.363506in}{4.000000in}}%
\pgfpathlineto{\pgfqpoint{4.345696in}{4.017311in}}%
\pgfpathlineto{\pgfqpoint{4.327111in}{4.035716in}}%
\pgfpathlineto{\pgfqpoint{4.325458in}{4.037333in}}%
\pgfpathlineto{\pgfqpoint{4.292504in}{4.069568in}}%
\pgfpathlineto{\pgfqpoint{4.287289in}{4.074667in}}%
\pgfpathlineto{\pgfqpoint{4.287162in}{4.074790in}}%
\pgfpathlineto{\pgfqpoint{4.287030in}{4.074920in}}%
\pgfpathlineto{\pgfqpoint{4.248983in}{4.112000in}}%
\pgfpathlineto{\pgfqpoint{4.247984in}{4.112964in}}%
\pgfpathlineto{\pgfqpoint{4.246949in}{4.113981in}}%
\pgfpathlineto{\pgfqpoint{4.210558in}{4.149333in}}%
\pgfpathlineto{\pgfqpoint{4.208743in}{4.151079in}}%
\pgfpathlineto{\pgfqpoint{4.206869in}{4.152915in}}%
\pgfpathlineto{\pgfqpoint{4.172014in}{4.186667in}}%
\pgfpathlineto{\pgfqpoint{4.169438in}{4.189135in}}%
\pgfpathlineto{\pgfqpoint{4.166788in}{4.191723in}}%
\pgfpathlineto{\pgfqpoint{4.133348in}{4.224000in}}%
\pgfpathlineto{\pgfqpoint{4.130568in}{4.224000in}}%
\pgfpathlineto{\pgfqpoint{4.166788in}{4.189040in}}%
\pgfpathlineto{\pgfqpoint{4.168032in}{4.187825in}}%
\pgfpathlineto{\pgfqpoint{4.169240in}{4.186667in}}%
\pgfpathlineto{\pgfqpoint{4.206869in}{4.150230in}}%
\pgfpathlineto{\pgfqpoint{4.207338in}{4.149770in}}%
\pgfpathlineto{\pgfqpoint{4.207792in}{4.149333in}}%
\pgfpathlineto{\pgfqpoint{4.229292in}{4.128447in}}%
\pgfpathlineto{\pgfqpoint{4.246216in}{4.112000in}}%
\pgfpathlineto{\pgfqpoint{4.246573in}{4.111649in}}%
\pgfpathlineto{\pgfqpoint{4.246949in}{4.111286in}}%
\pgfpathlineto{\pgfqpoint{4.284508in}{4.074667in}}%
\pgfpathlineto{\pgfqpoint{4.287030in}{4.072206in}}%
\pgfpathlineto{\pgfqpoint{4.322681in}{4.037333in}}%
\pgfpathlineto{\pgfqpoint{4.327111in}{4.032997in}}%
\pgfpathlineto{\pgfqpoint{4.344282in}{4.015994in}}%
\pgfpathlineto{\pgfqpoint{4.360736in}{4.000000in}}%
\pgfpathlineto{\pgfqpoint{4.367192in}{3.993660in}}%
\pgfpathlineto{\pgfqpoint{4.398673in}{3.962667in}}%
\pgfpathlineto{\pgfqpoint{4.402832in}{3.958530in}}%
\pgfpathlineto{\pgfqpoint{4.407273in}{3.954195in}}%
\pgfpathlineto{\pgfqpoint{4.436494in}{3.925333in}}%
\pgfpathlineto{\pgfqpoint{4.447354in}{3.914600in}}%
\pgfpathlineto{\pgfqpoint{4.461127in}{3.900829in}}%
\pgfpathlineto{\pgfqpoint{4.474199in}{3.888000in}}%
\pgfpathlineto{\pgfqpoint{4.487434in}{3.874876in}}%
\pgfpathlineto{\pgfqpoint{4.499949in}{3.862324in}}%
\pgfpathlineto{\pgfqpoint{4.511789in}{3.850667in}}%
\pgfpathlineto{\pgfqpoint{4.527515in}{3.835023in}}%
\pgfpathlineto{\pgfqpoint{4.538709in}{3.823760in}}%
\pgfpathlineto{\pgfqpoint{4.549265in}{3.813333in}}%
\pgfpathlineto{\pgfqpoint{4.567596in}{3.795041in}}%
\pgfpathlineto{\pgfqpoint{4.577406in}{3.785138in}}%
\pgfpathlineto{\pgfqpoint{4.586628in}{3.776000in}}%
\pgfpathlineto{\pgfqpoint{4.596722in}{3.765796in}}%
\pgfpathlineto{\pgfqpoint{4.607677in}{3.754928in}}%
\pgfpathlineto{\pgfqpoint{4.623879in}{3.738667in}}%
\pgfpathlineto{\pgfqpoint{4.647758in}{3.714685in}}%
\pgfpathlineto{\pgfqpoint{4.661019in}{3.701333in}}%
\pgfpathlineto{\pgfqpoint{4.673837in}{3.688292in}}%
\pgfpathlineto{\pgfqpoint{4.687838in}{3.674312in}}%
\pgfpathlineto{\pgfqpoint{4.698048in}{3.664000in}}%
\pgfpathlineto{\pgfqpoint{4.712301in}{3.649453in}}%
\pgfpathlineto{\pgfqpoint{4.727919in}{3.633808in}}%
\pgfpathlineto{\pgfqpoint{4.734967in}{3.626667in}}%
\pgfpathlineto{\pgfqpoint{4.750703in}{3.610555in}}%
\pgfpathlineto{\pgfqpoint{4.768000in}{3.593173in}}%
\pgfusepath{fill}%
\end{pgfscope}%
\begin{pgfscope}%
\pgfpathrectangle{\pgfqpoint{0.800000in}{0.528000in}}{\pgfqpoint{3.968000in}{3.696000in}}%
\pgfusepath{clip}%
\pgfsetbuttcap%
\pgfsetroundjoin%
\definecolor{currentfill}{rgb}{0.344074,0.780029,0.397381}%
\pgfsetfillcolor{currentfill}%
\pgfsetlinewidth{0.000000pt}%
\definecolor{currentstroke}{rgb}{0.000000,0.000000,0.000000}%
\pgfsetstrokecolor{currentstroke}%
\pgfsetdash{}{0pt}%
\pgfpathmoveto{\pgfqpoint{4.768000in}{3.598650in}}%
\pgfpathlineto{\pgfqpoint{4.753532in}{3.613190in}}%
\pgfpathlineto{\pgfqpoint{4.740369in}{3.626667in}}%
\pgfpathlineto{\pgfqpoint{4.727919in}{3.639282in}}%
\pgfpathlineto{\pgfqpoint{4.715133in}{3.652090in}}%
\pgfpathlineto{\pgfqpoint{4.703464in}{3.664000in}}%
\pgfpathlineto{\pgfqpoint{4.687838in}{3.679782in}}%
\pgfpathlineto{\pgfqpoint{4.676672in}{3.690932in}}%
\pgfpathlineto{\pgfqpoint{4.666448in}{3.701333in}}%
\pgfpathlineto{\pgfqpoint{4.647758in}{3.720152in}}%
\pgfpathlineto{\pgfqpoint{4.629322in}{3.738667in}}%
\pgfpathlineto{\pgfqpoint{4.607677in}{3.760391in}}%
\pgfpathlineto{\pgfqpoint{4.599562in}{3.768442in}}%
\pgfpathlineto{\pgfqpoint{4.592085in}{3.776000in}}%
\pgfpathlineto{\pgfqpoint{4.580219in}{3.787758in}}%
\pgfpathlineto{\pgfqpoint{4.567596in}{3.800500in}}%
\pgfpathlineto{\pgfqpoint{4.554736in}{3.813333in}}%
\pgfpathlineto{\pgfqpoint{4.541525in}{3.826382in}}%
\pgfpathlineto{\pgfqpoint{4.527515in}{3.840479in}}%
\pgfpathlineto{\pgfqpoint{4.517273in}{3.850667in}}%
\pgfpathlineto{\pgfqpoint{4.502767in}{3.864949in}}%
\pgfpathlineto{\pgfqpoint{4.487434in}{3.880328in}}%
\pgfpathlineto{\pgfqpoint{4.479697in}{3.888000in}}%
\pgfpathlineto{\pgfqpoint{4.463948in}{3.903457in}}%
\pgfpathlineto{\pgfqpoint{4.447354in}{3.920048in}}%
\pgfpathlineto{\pgfqpoint{4.442006in}{3.925333in}}%
\pgfpathlineto{\pgfqpoint{4.407273in}{3.959639in}}%
\pgfpathlineto{\pgfqpoint{4.405686in}{3.961188in}}%
\pgfpathlineto{\pgfqpoint{4.404199in}{3.962667in}}%
\pgfpathlineto{\pgfqpoint{4.367192in}{3.999101in}}%
\pgfpathlineto{\pgfqpoint{4.366276in}{4.000000in}}%
\pgfpathlineto{\pgfqpoint{4.347111in}{4.018629in}}%
\pgfpathlineto{\pgfqpoint{4.328223in}{4.037333in}}%
\pgfpathlineto{\pgfqpoint{4.327678in}{4.037862in}}%
\pgfpathlineto{\pgfqpoint{4.327111in}{4.038423in}}%
\pgfpathlineto{\pgfqpoint{4.290041in}{4.074667in}}%
\pgfpathlineto{\pgfqpoint{4.288565in}{4.076096in}}%
\pgfpathlineto{\pgfqpoint{4.287030in}{4.077609in}}%
\pgfpathlineto{\pgfqpoint{4.251742in}{4.112000in}}%
\pgfpathlineto{\pgfqpoint{4.249388in}{4.114271in}}%
\pgfpathlineto{\pgfqpoint{4.246949in}{4.116668in}}%
\pgfpathlineto{\pgfqpoint{4.213324in}{4.149333in}}%
\pgfpathlineto{\pgfqpoint{4.210148in}{4.152388in}}%
\pgfpathlineto{\pgfqpoint{4.206869in}{4.155601in}}%
\pgfpathlineto{\pgfqpoint{4.174787in}{4.186667in}}%
\pgfpathlineto{\pgfqpoint{4.170845in}{4.190445in}}%
\pgfpathlineto{\pgfqpoint{4.166788in}{4.194407in}}%
\pgfpathlineto{\pgfqpoint{4.136128in}{4.224000in}}%
\pgfpathlineto{\pgfqpoint{4.133348in}{4.224000in}}%
\pgfpathlineto{\pgfqpoint{4.166788in}{4.191723in}}%
\pgfpathlineto{\pgfqpoint{4.169438in}{4.189135in}}%
\pgfpathlineto{\pgfqpoint{4.172014in}{4.186667in}}%
\pgfpathlineto{\pgfqpoint{4.206869in}{4.152915in}}%
\pgfpathlineto{\pgfqpoint{4.208743in}{4.151079in}}%
\pgfpathlineto{\pgfqpoint{4.210558in}{4.149333in}}%
\pgfpathlineto{\pgfqpoint{4.246949in}{4.113981in}}%
\pgfpathlineto{\pgfqpoint{4.247984in}{4.112964in}}%
\pgfpathlineto{\pgfqpoint{4.248983in}{4.112000in}}%
\pgfpathlineto{\pgfqpoint{4.287030in}{4.074920in}}%
\pgfpathlineto{\pgfqpoint{4.287162in}{4.074790in}}%
\pgfpathlineto{\pgfqpoint{4.287289in}{4.074667in}}%
\pgfpathlineto{\pgfqpoint{4.292504in}{4.069568in}}%
\pgfpathlineto{\pgfqpoint{4.325458in}{4.037333in}}%
\pgfpathlineto{\pgfqpoint{4.327111in}{4.035716in}}%
\pgfpathlineto{\pgfqpoint{4.345696in}{4.017311in}}%
\pgfpathlineto{\pgfqpoint{4.363506in}{4.000000in}}%
\pgfpathlineto{\pgfqpoint{4.367192in}{3.996380in}}%
\pgfpathlineto{\pgfqpoint{4.401436in}{3.962667in}}%
\pgfpathlineto{\pgfqpoint{4.404259in}{3.959859in}}%
\pgfpathlineto{\pgfqpoint{4.407273in}{3.956917in}}%
\pgfpathlineto{\pgfqpoint{4.439250in}{3.925333in}}%
\pgfpathlineto{\pgfqpoint{4.447354in}{3.917324in}}%
\pgfpathlineto{\pgfqpoint{4.462537in}{3.902143in}}%
\pgfpathlineto{\pgfqpoint{4.476948in}{3.888000in}}%
\pgfpathlineto{\pgfqpoint{4.487434in}{3.877602in}}%
\pgfpathlineto{\pgfqpoint{4.501358in}{3.863636in}}%
\pgfpathlineto{\pgfqpoint{4.514531in}{3.850667in}}%
\pgfpathlineto{\pgfqpoint{4.527515in}{3.837751in}}%
\pgfpathlineto{\pgfqpoint{4.540117in}{3.825071in}}%
\pgfpathlineto{\pgfqpoint{4.552000in}{3.813333in}}%
\pgfpathlineto{\pgfqpoint{4.567596in}{3.797770in}}%
\pgfpathlineto{\pgfqpoint{4.578813in}{3.786448in}}%
\pgfpathlineto{\pgfqpoint{4.589357in}{3.776000in}}%
\pgfpathlineto{\pgfqpoint{4.598142in}{3.767119in}}%
\pgfpathlineto{\pgfqpoint{4.607677in}{3.757659in}}%
\pgfpathlineto{\pgfqpoint{4.626601in}{3.738667in}}%
\pgfpathlineto{\pgfqpoint{4.647758in}{3.717418in}}%
\pgfpathlineto{\pgfqpoint{4.663733in}{3.701333in}}%
\pgfpathlineto{\pgfqpoint{4.675255in}{3.689612in}}%
\pgfpathlineto{\pgfqpoint{4.687838in}{3.677047in}}%
\pgfpathlineto{\pgfqpoint{4.700756in}{3.664000in}}%
\pgfpathlineto{\pgfqpoint{4.713717in}{3.650771in}}%
\pgfpathlineto{\pgfqpoint{4.727919in}{3.636545in}}%
\pgfpathlineto{\pgfqpoint{4.737668in}{3.626667in}}%
\pgfpathlineto{\pgfqpoint{4.752117in}{3.611873in}}%
\pgfpathlineto{\pgfqpoint{4.768000in}{3.595911in}}%
\pgfusepath{fill}%
\end{pgfscope}%
\begin{pgfscope}%
\pgfpathrectangle{\pgfqpoint{0.800000in}{0.528000in}}{\pgfqpoint{3.968000in}{3.696000in}}%
\pgfusepath{clip}%
\pgfsetbuttcap%
\pgfsetroundjoin%
\definecolor{currentfill}{rgb}{0.344074,0.780029,0.397381}%
\pgfsetfillcolor{currentfill}%
\pgfsetlinewidth{0.000000pt}%
\definecolor{currentstroke}{rgb}{0.000000,0.000000,0.000000}%
\pgfsetstrokecolor{currentstroke}%
\pgfsetdash{}{0pt}%
\pgfpathmoveto{\pgfqpoint{4.768000in}{3.601389in}}%
\pgfpathlineto{\pgfqpoint{4.754946in}{3.614508in}}%
\pgfpathlineto{\pgfqpoint{4.743070in}{3.626667in}}%
\pgfpathlineto{\pgfqpoint{4.727919in}{3.642019in}}%
\pgfpathlineto{\pgfqpoint{4.716549in}{3.653409in}}%
\pgfpathlineto{\pgfqpoint{4.706171in}{3.664000in}}%
\pgfpathlineto{\pgfqpoint{4.687838in}{3.682517in}}%
\pgfpathlineto{\pgfqpoint{4.678089in}{3.692252in}}%
\pgfpathlineto{\pgfqpoint{4.669163in}{3.701333in}}%
\pgfpathlineto{\pgfqpoint{4.647758in}{3.722885in}}%
\pgfpathlineto{\pgfqpoint{4.632044in}{3.738667in}}%
\pgfpathlineto{\pgfqpoint{4.607677in}{3.763122in}}%
\pgfpathlineto{\pgfqpoint{4.600982in}{3.769764in}}%
\pgfpathlineto{\pgfqpoint{4.594813in}{3.776000in}}%
\pgfpathlineto{\pgfqpoint{4.581625in}{3.789068in}}%
\pgfpathlineto{\pgfqpoint{4.567596in}{3.803229in}}%
\pgfpathlineto{\pgfqpoint{4.557471in}{3.813333in}}%
\pgfpathlineto{\pgfqpoint{4.542932in}{3.827694in}}%
\pgfpathlineto{\pgfqpoint{4.527515in}{3.843206in}}%
\pgfpathlineto{\pgfqpoint{4.520015in}{3.850667in}}%
\pgfpathlineto{\pgfqpoint{4.504177in}{3.866261in}}%
\pgfpathlineto{\pgfqpoint{4.487434in}{3.883054in}}%
\pgfpathlineto{\pgfqpoint{4.482446in}{3.888000in}}%
\pgfpathlineto{\pgfqpoint{4.465358in}{3.904771in}}%
\pgfpathlineto{\pgfqpoint{4.447354in}{3.922772in}}%
\pgfpathlineto{\pgfqpoint{4.444762in}{3.925333in}}%
\pgfpathlineto{\pgfqpoint{4.407273in}{3.962361in}}%
\pgfpathlineto{\pgfqpoint{4.407112in}{3.962517in}}%
\pgfpathlineto{\pgfqpoint{4.406962in}{3.962667in}}%
\pgfpathlineto{\pgfqpoint{4.401514in}{3.968031in}}%
\pgfpathlineto{\pgfqpoint{4.369025in}{4.000000in}}%
\pgfpathlineto{\pgfqpoint{4.368129in}{4.000873in}}%
\pgfpathlineto{\pgfqpoint{4.367192in}{4.001802in}}%
\pgfpathlineto{\pgfqpoint{4.348525in}{4.019946in}}%
\pgfpathlineto{\pgfqpoint{4.330968in}{4.037333in}}%
\pgfpathlineto{\pgfqpoint{4.329079in}{4.039167in}}%
\pgfpathlineto{\pgfqpoint{4.327111in}{4.041114in}}%
\pgfpathlineto{\pgfqpoint{4.292794in}{4.074667in}}%
\pgfpathlineto{\pgfqpoint{4.289967in}{4.077402in}}%
\pgfpathlineto{\pgfqpoint{4.287030in}{4.080298in}}%
\pgfpathlineto{\pgfqpoint{4.254501in}{4.112000in}}%
\pgfpathlineto{\pgfqpoint{4.250792in}{4.115579in}}%
\pgfpathlineto{\pgfqpoint{4.246949in}{4.119355in}}%
\pgfpathlineto{\pgfqpoint{4.216091in}{4.149333in}}%
\pgfpathlineto{\pgfqpoint{4.211553in}{4.153697in}}%
\pgfpathlineto{\pgfqpoint{4.206869in}{4.158286in}}%
\pgfpathlineto{\pgfqpoint{4.177560in}{4.186667in}}%
\pgfpathlineto{\pgfqpoint{4.172251in}{4.191756in}}%
\pgfpathlineto{\pgfqpoint{4.166788in}{4.197091in}}%
\pgfpathlineto{\pgfqpoint{4.138909in}{4.224000in}}%
\pgfpathlineto{\pgfqpoint{4.136128in}{4.224000in}}%
\pgfpathlineto{\pgfqpoint{4.166788in}{4.194407in}}%
\pgfpathlineto{\pgfqpoint{4.170845in}{4.190445in}}%
\pgfpathlineto{\pgfqpoint{4.174787in}{4.186667in}}%
\pgfpathlineto{\pgfqpoint{4.206869in}{4.155601in}}%
\pgfpathlineto{\pgfqpoint{4.210148in}{4.152388in}}%
\pgfpathlineto{\pgfqpoint{4.213324in}{4.149333in}}%
\pgfpathlineto{\pgfqpoint{4.246949in}{4.116668in}}%
\pgfpathlineto{\pgfqpoint{4.249388in}{4.114271in}}%
\pgfpathlineto{\pgfqpoint{4.251742in}{4.112000in}}%
\pgfpathlineto{\pgfqpoint{4.287030in}{4.077609in}}%
\pgfpathlineto{\pgfqpoint{4.288565in}{4.076096in}}%
\pgfpathlineto{\pgfqpoint{4.290041in}{4.074667in}}%
\pgfpathlineto{\pgfqpoint{4.327111in}{4.038423in}}%
\pgfpathlineto{\pgfqpoint{4.327678in}{4.037862in}}%
\pgfpathlineto{\pgfqpoint{4.328223in}{4.037333in}}%
\pgfpathlineto{\pgfqpoint{4.347111in}{4.018629in}}%
\pgfpathlineto{\pgfqpoint{4.366276in}{4.000000in}}%
\pgfpathlineto{\pgfqpoint{4.367192in}{3.999101in}}%
\pgfpathlineto{\pgfqpoint{4.404199in}{3.962667in}}%
\pgfpathlineto{\pgfqpoint{4.405686in}{3.961188in}}%
\pgfpathlineto{\pgfqpoint{4.407273in}{3.959639in}}%
\pgfpathlineto{\pgfqpoint{4.442006in}{3.925333in}}%
\pgfpathlineto{\pgfqpoint{4.447354in}{3.920048in}}%
\pgfpathlineto{\pgfqpoint{4.463948in}{3.903457in}}%
\pgfpathlineto{\pgfqpoint{4.479697in}{3.888000in}}%
\pgfpathlineto{\pgfqpoint{4.487434in}{3.880328in}}%
\pgfpathlineto{\pgfqpoint{4.502767in}{3.864949in}}%
\pgfpathlineto{\pgfqpoint{4.517273in}{3.850667in}}%
\pgfpathlineto{\pgfqpoint{4.527515in}{3.840479in}}%
\pgfpathlineto{\pgfqpoint{4.541525in}{3.826382in}}%
\pgfpathlineto{\pgfqpoint{4.554736in}{3.813333in}}%
\pgfpathlineto{\pgfqpoint{4.567596in}{3.800500in}}%
\pgfpathlineto{\pgfqpoint{4.580219in}{3.787758in}}%
\pgfpathlineto{\pgfqpoint{4.592085in}{3.776000in}}%
\pgfpathlineto{\pgfqpoint{4.599562in}{3.768442in}}%
\pgfpathlineto{\pgfqpoint{4.607677in}{3.760391in}}%
\pgfpathlineto{\pgfqpoint{4.629322in}{3.738667in}}%
\pgfpathlineto{\pgfqpoint{4.647758in}{3.720152in}}%
\pgfpathlineto{\pgfqpoint{4.666448in}{3.701333in}}%
\pgfpathlineto{\pgfqpoint{4.676672in}{3.690932in}}%
\pgfpathlineto{\pgfqpoint{4.687838in}{3.679782in}}%
\pgfpathlineto{\pgfqpoint{4.703464in}{3.664000in}}%
\pgfpathlineto{\pgfqpoint{4.715133in}{3.652090in}}%
\pgfpathlineto{\pgfqpoint{4.727919in}{3.639282in}}%
\pgfpathlineto{\pgfqpoint{4.740369in}{3.626667in}}%
\pgfpathlineto{\pgfqpoint{4.753532in}{3.613190in}}%
\pgfpathlineto{\pgfqpoint{4.768000in}{3.598650in}}%
\pgfusepath{fill}%
\end{pgfscope}%
\begin{pgfscope}%
\pgfpathrectangle{\pgfqpoint{0.800000in}{0.528000in}}{\pgfqpoint{3.968000in}{3.696000in}}%
\pgfusepath{clip}%
\pgfsetbuttcap%
\pgfsetroundjoin%
\definecolor{currentfill}{rgb}{0.344074,0.780029,0.397381}%
\pgfsetfillcolor{currentfill}%
\pgfsetlinewidth{0.000000pt}%
\definecolor{currentstroke}{rgb}{0.000000,0.000000,0.000000}%
\pgfsetstrokecolor{currentstroke}%
\pgfsetdash{}{0pt}%
\pgfpathmoveto{\pgfqpoint{4.768000in}{3.604128in}}%
\pgfpathlineto{\pgfqpoint{4.756361in}{3.615825in}}%
\pgfpathlineto{\pgfqpoint{4.745772in}{3.626667in}}%
\pgfpathlineto{\pgfqpoint{4.727919in}{3.644756in}}%
\pgfpathlineto{\pgfqpoint{4.717965in}{3.654728in}}%
\pgfpathlineto{\pgfqpoint{4.708879in}{3.664000in}}%
\pgfpathlineto{\pgfqpoint{4.687838in}{3.685252in}}%
\pgfpathlineto{\pgfqpoint{4.679506in}{3.693572in}}%
\pgfpathlineto{\pgfqpoint{4.671878in}{3.701333in}}%
\pgfpathlineto{\pgfqpoint{4.647758in}{3.725618in}}%
\pgfpathlineto{\pgfqpoint{4.634765in}{3.738667in}}%
\pgfpathlineto{\pgfqpoint{4.607677in}{3.765854in}}%
\pgfpathlineto{\pgfqpoint{4.602402in}{3.771087in}}%
\pgfpathlineto{\pgfqpoint{4.597542in}{3.776000in}}%
\pgfpathlineto{\pgfqpoint{4.583032in}{3.790378in}}%
\pgfpathlineto{\pgfqpoint{4.567596in}{3.805959in}}%
\pgfpathlineto{\pgfqpoint{4.560206in}{3.813333in}}%
\pgfpathlineto{\pgfqpoint{4.544340in}{3.829005in}}%
\pgfpathlineto{\pgfqpoint{4.527515in}{3.845934in}}%
\pgfpathlineto{\pgfqpoint{4.522758in}{3.850667in}}%
\pgfpathlineto{\pgfqpoint{4.505586in}{3.867574in}}%
\pgfpathlineto{\pgfqpoint{4.487434in}{3.885780in}}%
\pgfpathlineto{\pgfqpoint{4.485195in}{3.888000in}}%
\pgfpathlineto{\pgfqpoint{4.466769in}{3.906084in}}%
\pgfpathlineto{\pgfqpoint{4.447516in}{3.925333in}}%
\pgfpathlineto{\pgfqpoint{4.447437in}{3.925411in}}%
\pgfpathlineto{\pgfqpoint{4.447354in}{3.925494in}}%
\pgfpathlineto{\pgfqpoint{4.409697in}{3.962667in}}%
\pgfpathlineto{\pgfqpoint{4.408514in}{3.963823in}}%
\pgfpathlineto{\pgfqpoint{4.407273in}{3.965058in}}%
\pgfpathlineto{\pgfqpoint{4.371763in}{4.000000in}}%
\pgfpathlineto{\pgfqpoint{4.369529in}{4.002177in}}%
\pgfpathlineto{\pgfqpoint{4.367192in}{4.004495in}}%
\pgfpathlineto{\pgfqpoint{4.349940in}{4.021264in}}%
\pgfpathlineto{\pgfqpoint{4.333713in}{4.037333in}}%
\pgfpathlineto{\pgfqpoint{4.330481in}{4.040472in}}%
\pgfpathlineto{\pgfqpoint{4.327111in}{4.043804in}}%
\pgfpathlineto{\pgfqpoint{4.295546in}{4.074667in}}%
\pgfpathlineto{\pgfqpoint{4.291370in}{4.078709in}}%
\pgfpathlineto{\pgfqpoint{4.287030in}{4.082987in}}%
\pgfpathlineto{\pgfqpoint{4.257261in}{4.112000in}}%
\pgfpathlineto{\pgfqpoint{4.252196in}{4.116887in}}%
\pgfpathlineto{\pgfqpoint{4.246949in}{4.122042in}}%
\pgfpathlineto{\pgfqpoint{4.218857in}{4.149333in}}%
\pgfpathlineto{\pgfqpoint{4.212958in}{4.155006in}}%
\pgfpathlineto{\pgfqpoint{4.206869in}{4.160971in}}%
\pgfpathlineto{\pgfqpoint{4.180333in}{4.186667in}}%
\pgfpathlineto{\pgfqpoint{4.173658in}{4.193066in}}%
\pgfpathlineto{\pgfqpoint{4.166788in}{4.199774in}}%
\pgfpathlineto{\pgfqpoint{4.141689in}{4.224000in}}%
\pgfpathlineto{\pgfqpoint{4.138909in}{4.224000in}}%
\pgfpathlineto{\pgfqpoint{4.166788in}{4.197091in}}%
\pgfpathlineto{\pgfqpoint{4.172251in}{4.191756in}}%
\pgfpathlineto{\pgfqpoint{4.177560in}{4.186667in}}%
\pgfpathlineto{\pgfqpoint{4.206869in}{4.158286in}}%
\pgfpathlineto{\pgfqpoint{4.211553in}{4.153697in}}%
\pgfpathlineto{\pgfqpoint{4.216091in}{4.149333in}}%
\pgfpathlineto{\pgfqpoint{4.246949in}{4.119355in}}%
\pgfpathlineto{\pgfqpoint{4.250792in}{4.115579in}}%
\pgfpathlineto{\pgfqpoint{4.254501in}{4.112000in}}%
\pgfpathlineto{\pgfqpoint{4.287030in}{4.080298in}}%
\pgfpathlineto{\pgfqpoint{4.289967in}{4.077402in}}%
\pgfpathlineto{\pgfqpoint{4.292794in}{4.074667in}}%
\pgfpathlineto{\pgfqpoint{4.327111in}{4.041114in}}%
\pgfpathlineto{\pgfqpoint{4.329079in}{4.039167in}}%
\pgfpathlineto{\pgfqpoint{4.330968in}{4.037333in}}%
\pgfpathlineto{\pgfqpoint{4.348525in}{4.019946in}}%
\pgfpathlineto{\pgfqpoint{4.367192in}{4.001802in}}%
\pgfpathlineto{\pgfqpoint{4.368129in}{4.000873in}}%
\pgfpathlineto{\pgfqpoint{4.369025in}{4.000000in}}%
\pgfpathlineto{\pgfqpoint{4.401514in}{3.968031in}}%
\pgfpathlineto{\pgfqpoint{4.406962in}{3.962667in}}%
\pgfpathlineto{\pgfqpoint{4.407112in}{3.962517in}}%
\pgfpathlineto{\pgfqpoint{4.407273in}{3.962361in}}%
\pgfpathlineto{\pgfqpoint{4.444762in}{3.925333in}}%
\pgfpathlineto{\pgfqpoint{4.447354in}{3.922772in}}%
\pgfpathlineto{\pgfqpoint{4.465358in}{3.904771in}}%
\pgfpathlineto{\pgfqpoint{4.482446in}{3.888000in}}%
\pgfpathlineto{\pgfqpoint{4.487434in}{3.883054in}}%
\pgfpathlineto{\pgfqpoint{4.504177in}{3.866261in}}%
\pgfpathlineto{\pgfqpoint{4.520015in}{3.850667in}}%
\pgfpathlineto{\pgfqpoint{4.527515in}{3.843206in}}%
\pgfpathlineto{\pgfqpoint{4.542932in}{3.827694in}}%
\pgfpathlineto{\pgfqpoint{4.557471in}{3.813333in}}%
\pgfpathlineto{\pgfqpoint{4.567596in}{3.803229in}}%
\pgfpathlineto{\pgfqpoint{4.581625in}{3.789068in}}%
\pgfpathlineto{\pgfqpoint{4.594813in}{3.776000in}}%
\pgfpathlineto{\pgfqpoint{4.600982in}{3.769764in}}%
\pgfpathlineto{\pgfqpoint{4.607677in}{3.763122in}}%
\pgfpathlineto{\pgfqpoint{4.632044in}{3.738667in}}%
\pgfpathlineto{\pgfqpoint{4.647758in}{3.722885in}}%
\pgfpathlineto{\pgfqpoint{4.669163in}{3.701333in}}%
\pgfpathlineto{\pgfqpoint{4.678089in}{3.692252in}}%
\pgfpathlineto{\pgfqpoint{4.687838in}{3.682517in}}%
\pgfpathlineto{\pgfqpoint{4.706171in}{3.664000in}}%
\pgfpathlineto{\pgfqpoint{4.716549in}{3.653409in}}%
\pgfpathlineto{\pgfqpoint{4.727919in}{3.642019in}}%
\pgfpathlineto{\pgfqpoint{4.743070in}{3.626667in}}%
\pgfpathlineto{\pgfqpoint{4.754946in}{3.614508in}}%
\pgfpathlineto{\pgfqpoint{4.768000in}{3.601389in}}%
\pgfusepath{fill}%
\end{pgfscope}%
\begin{pgfscope}%
\pgfpathrectangle{\pgfqpoint{0.800000in}{0.528000in}}{\pgfqpoint{3.968000in}{3.696000in}}%
\pgfusepath{clip}%
\pgfsetbuttcap%
\pgfsetroundjoin%
\definecolor{currentfill}{rgb}{0.352360,0.783011,0.392636}%
\pgfsetfillcolor{currentfill}%
\pgfsetlinewidth{0.000000pt}%
\definecolor{currentstroke}{rgb}{0.000000,0.000000,0.000000}%
\pgfsetstrokecolor{currentstroke}%
\pgfsetdash{}{0pt}%
\pgfpathmoveto{\pgfqpoint{4.768000in}{3.606867in}}%
\pgfpathlineto{\pgfqpoint{4.757775in}{3.617143in}}%
\pgfpathlineto{\pgfqpoint{4.748473in}{3.626667in}}%
\pgfpathlineto{\pgfqpoint{4.727919in}{3.647493in}}%
\pgfpathlineto{\pgfqpoint{4.719380in}{3.656047in}}%
\pgfpathlineto{\pgfqpoint{4.711587in}{3.664000in}}%
\pgfpathlineto{\pgfqpoint{4.687838in}{3.687988in}}%
\pgfpathlineto{\pgfqpoint{4.680923in}{3.694892in}}%
\pgfpathlineto{\pgfqpoint{4.674592in}{3.701333in}}%
\pgfpathlineto{\pgfqpoint{4.647758in}{3.728352in}}%
\pgfpathlineto{\pgfqpoint{4.637487in}{3.738667in}}%
\pgfpathlineto{\pgfqpoint{4.607677in}{3.768585in}}%
\pgfpathlineto{\pgfqpoint{4.603822in}{3.772410in}}%
\pgfpathlineto{\pgfqpoint{4.600270in}{3.776000in}}%
\pgfpathlineto{\pgfqpoint{4.584438in}{3.791688in}}%
\pgfpathlineto{\pgfqpoint{4.567596in}{3.808688in}}%
\pgfpathlineto{\pgfqpoint{4.562941in}{3.813333in}}%
\pgfpathlineto{\pgfqpoint{4.545748in}{3.830316in}}%
\pgfpathlineto{\pgfqpoint{4.527515in}{3.848662in}}%
\pgfpathlineto{\pgfqpoint{4.525500in}{3.850667in}}%
\pgfpathlineto{\pgfqpoint{4.506995in}{3.868886in}}%
\pgfpathlineto{\pgfqpoint{4.487938in}{3.888000in}}%
\pgfpathlineto{\pgfqpoint{4.487693in}{3.888241in}}%
\pgfpathlineto{\pgfqpoint{4.487434in}{3.888500in}}%
\pgfpathlineto{\pgfqpoint{4.468179in}{3.907398in}}%
\pgfpathlineto{\pgfqpoint{4.450241in}{3.925333in}}%
\pgfpathlineto{\pgfqpoint{4.448834in}{3.926712in}}%
\pgfpathlineto{\pgfqpoint{4.447354in}{3.928191in}}%
\pgfpathlineto{\pgfqpoint{4.412429in}{3.962667in}}%
\pgfpathlineto{\pgfqpoint{4.409912in}{3.965125in}}%
\pgfpathlineto{\pgfqpoint{4.407273in}{3.967753in}}%
\pgfpathlineto{\pgfqpoint{4.374501in}{4.000000in}}%
\pgfpathlineto{\pgfqpoint{4.370928in}{4.003480in}}%
\pgfpathlineto{\pgfqpoint{4.367192in}{4.007188in}}%
\pgfpathlineto{\pgfqpoint{4.351355in}{4.022582in}}%
\pgfpathlineto{\pgfqpoint{4.336458in}{4.037333in}}%
\pgfpathlineto{\pgfqpoint{4.331882in}{4.041777in}}%
\pgfpathlineto{\pgfqpoint{4.327111in}{4.046495in}}%
\pgfpathlineto{\pgfqpoint{4.298298in}{4.074667in}}%
\pgfpathlineto{\pgfqpoint{4.292772in}{4.080015in}}%
\pgfpathlineto{\pgfqpoint{4.287030in}{4.085676in}}%
\pgfpathlineto{\pgfqpoint{4.260020in}{4.112000in}}%
\pgfpathlineto{\pgfqpoint{4.253599in}{4.118194in}}%
\pgfpathlineto{\pgfqpoint{4.246949in}{4.124729in}}%
\pgfpathlineto{\pgfqpoint{4.221623in}{4.149333in}}%
\pgfpathlineto{\pgfqpoint{4.214364in}{4.156314in}}%
\pgfpathlineto{\pgfqpoint{4.206869in}{4.163657in}}%
\pgfpathlineto{\pgfqpoint{4.183106in}{4.186667in}}%
\pgfpathlineto{\pgfqpoint{4.175064in}{4.194376in}}%
\pgfpathlineto{\pgfqpoint{4.166788in}{4.202458in}}%
\pgfpathlineto{\pgfqpoint{4.144469in}{4.224000in}}%
\pgfpathlineto{\pgfqpoint{4.141689in}{4.224000in}}%
\pgfpathlineto{\pgfqpoint{4.166788in}{4.199774in}}%
\pgfpathlineto{\pgfqpoint{4.173658in}{4.193066in}}%
\pgfpathlineto{\pgfqpoint{4.180333in}{4.186667in}}%
\pgfpathlineto{\pgfqpoint{4.206869in}{4.160971in}}%
\pgfpathlineto{\pgfqpoint{4.212958in}{4.155006in}}%
\pgfpathlineto{\pgfqpoint{4.218857in}{4.149333in}}%
\pgfpathlineto{\pgfqpoint{4.246949in}{4.122042in}}%
\pgfpathlineto{\pgfqpoint{4.252196in}{4.116887in}}%
\pgfpathlineto{\pgfqpoint{4.257261in}{4.112000in}}%
\pgfpathlineto{\pgfqpoint{4.287030in}{4.082987in}}%
\pgfpathlineto{\pgfqpoint{4.291370in}{4.078709in}}%
\pgfpathlineto{\pgfqpoint{4.295546in}{4.074667in}}%
\pgfpathlineto{\pgfqpoint{4.327111in}{4.043804in}}%
\pgfpathlineto{\pgfqpoint{4.330481in}{4.040472in}}%
\pgfpathlineto{\pgfqpoint{4.333713in}{4.037333in}}%
\pgfpathlineto{\pgfqpoint{4.349940in}{4.021264in}}%
\pgfpathlineto{\pgfqpoint{4.367192in}{4.004495in}}%
\pgfpathlineto{\pgfqpoint{4.369529in}{4.002177in}}%
\pgfpathlineto{\pgfqpoint{4.371763in}{4.000000in}}%
\pgfpathlineto{\pgfqpoint{4.407273in}{3.965058in}}%
\pgfpathlineto{\pgfqpoint{4.408514in}{3.963823in}}%
\pgfpathlineto{\pgfqpoint{4.409697in}{3.962667in}}%
\pgfpathlineto{\pgfqpoint{4.447354in}{3.925494in}}%
\pgfpathlineto{\pgfqpoint{4.447437in}{3.925411in}}%
\pgfpathlineto{\pgfqpoint{4.447516in}{3.925333in}}%
\pgfpathlineto{\pgfqpoint{4.466769in}{3.906084in}}%
\pgfpathlineto{\pgfqpoint{4.485195in}{3.888000in}}%
\pgfpathlineto{\pgfqpoint{4.487434in}{3.885780in}}%
\pgfpathlineto{\pgfqpoint{4.505586in}{3.867574in}}%
\pgfpathlineto{\pgfqpoint{4.522758in}{3.850667in}}%
\pgfpathlineto{\pgfqpoint{4.527515in}{3.845934in}}%
\pgfpathlineto{\pgfqpoint{4.544340in}{3.829005in}}%
\pgfpathlineto{\pgfqpoint{4.560206in}{3.813333in}}%
\pgfpathlineto{\pgfqpoint{4.567596in}{3.805959in}}%
\pgfpathlineto{\pgfqpoint{4.583032in}{3.790378in}}%
\pgfpathlineto{\pgfqpoint{4.597542in}{3.776000in}}%
\pgfpathlineto{\pgfqpoint{4.602402in}{3.771087in}}%
\pgfpathlineto{\pgfqpoint{4.607677in}{3.765854in}}%
\pgfpathlineto{\pgfqpoint{4.634765in}{3.738667in}}%
\pgfpathlineto{\pgfqpoint{4.647758in}{3.725618in}}%
\pgfpathlineto{\pgfqpoint{4.671878in}{3.701333in}}%
\pgfpathlineto{\pgfqpoint{4.679506in}{3.693572in}}%
\pgfpathlineto{\pgfqpoint{4.687838in}{3.685252in}}%
\pgfpathlineto{\pgfqpoint{4.708879in}{3.664000in}}%
\pgfpathlineto{\pgfqpoint{4.717965in}{3.654728in}}%
\pgfpathlineto{\pgfqpoint{4.727919in}{3.644756in}}%
\pgfpathlineto{\pgfqpoint{4.745772in}{3.626667in}}%
\pgfpathlineto{\pgfqpoint{4.756361in}{3.615825in}}%
\pgfpathlineto{\pgfqpoint{4.768000in}{3.604128in}}%
\pgfusepath{fill}%
\end{pgfscope}%
\begin{pgfscope}%
\pgfpathrectangle{\pgfqpoint{0.800000in}{0.528000in}}{\pgfqpoint{3.968000in}{3.696000in}}%
\pgfusepath{clip}%
\pgfsetbuttcap%
\pgfsetroundjoin%
\definecolor{currentfill}{rgb}{0.352360,0.783011,0.392636}%
\pgfsetfillcolor{currentfill}%
\pgfsetlinewidth{0.000000pt}%
\definecolor{currentstroke}{rgb}{0.000000,0.000000,0.000000}%
\pgfsetstrokecolor{currentstroke}%
\pgfsetdash{}{0pt}%
\pgfpathmoveto{\pgfqpoint{4.768000in}{3.609606in}}%
\pgfpathlineto{\pgfqpoint{4.759190in}{3.618460in}}%
\pgfpathlineto{\pgfqpoint{4.751174in}{3.626667in}}%
\pgfpathlineto{\pgfqpoint{4.727919in}{3.650230in}}%
\pgfpathlineto{\pgfqpoint{4.720796in}{3.657365in}}%
\pgfpathlineto{\pgfqpoint{4.714295in}{3.664000in}}%
\pgfpathlineto{\pgfqpoint{4.687838in}{3.690723in}}%
\pgfpathlineto{\pgfqpoint{4.682341in}{3.696212in}}%
\pgfpathlineto{\pgfqpoint{4.677307in}{3.701333in}}%
\pgfpathlineto{\pgfqpoint{4.647758in}{3.731085in}}%
\pgfpathlineto{\pgfqpoint{4.640208in}{3.738667in}}%
\pgfpathlineto{\pgfqpoint{4.607677in}{3.771317in}}%
\pgfpathlineto{\pgfqpoint{4.605242in}{3.773732in}}%
\pgfpathlineto{\pgfqpoint{4.602999in}{3.776000in}}%
\pgfpathlineto{\pgfqpoint{4.585845in}{3.792998in}}%
\pgfpathlineto{\pgfqpoint{4.567596in}{3.811418in}}%
\pgfpathlineto{\pgfqpoint{4.565677in}{3.813333in}}%
\pgfpathlineto{\pgfqpoint{4.547155in}{3.831627in}}%
\pgfpathlineto{\pgfqpoint{4.528234in}{3.850667in}}%
\pgfpathlineto{\pgfqpoint{4.527515in}{3.851382in}}%
\pgfpathlineto{\pgfqpoint{4.508404in}{3.870199in}}%
\pgfpathlineto{\pgfqpoint{4.490656in}{3.888000in}}%
\pgfpathlineto{\pgfqpoint{4.489089in}{3.889541in}}%
\pgfpathlineto{\pgfqpoint{4.487434in}{3.891199in}}%
\pgfpathlineto{\pgfqpoint{4.469590in}{3.908712in}}%
\pgfpathlineto{\pgfqpoint{4.452965in}{3.925333in}}%
\pgfpathlineto{\pgfqpoint{4.450231in}{3.928014in}}%
\pgfpathlineto{\pgfqpoint{4.447354in}{3.930887in}}%
\pgfpathlineto{\pgfqpoint{4.415160in}{3.962667in}}%
\pgfpathlineto{\pgfqpoint{4.411311in}{3.966428in}}%
\pgfpathlineto{\pgfqpoint{4.407273in}{3.970447in}}%
\pgfpathlineto{\pgfqpoint{4.377240in}{4.000000in}}%
\pgfpathlineto{\pgfqpoint{4.372328in}{4.004784in}}%
\pgfpathlineto{\pgfqpoint{4.367192in}{4.009880in}}%
\pgfpathlineto{\pgfqpoint{4.352769in}{4.023899in}}%
\pgfpathlineto{\pgfqpoint{4.339203in}{4.037333in}}%
\pgfpathlineto{\pgfqpoint{4.333283in}{4.043082in}}%
\pgfpathlineto{\pgfqpoint{4.327111in}{4.049186in}}%
\pgfpathlineto{\pgfqpoint{4.301050in}{4.074667in}}%
\pgfpathlineto{\pgfqpoint{4.294175in}{4.081321in}}%
\pgfpathlineto{\pgfqpoint{4.287030in}{4.088365in}}%
\pgfpathlineto{\pgfqpoint{4.262779in}{4.112000in}}%
\pgfpathlineto{\pgfqpoint{4.255003in}{4.119502in}}%
\pgfpathlineto{\pgfqpoint{4.246949in}{4.127417in}}%
\pgfpathlineto{\pgfqpoint{4.224389in}{4.149333in}}%
\pgfpathlineto{\pgfqpoint{4.215769in}{4.157623in}}%
\pgfpathlineto{\pgfqpoint{4.206869in}{4.166342in}}%
\pgfpathlineto{\pgfqpoint{4.185879in}{4.186667in}}%
\pgfpathlineto{\pgfqpoint{4.176471in}{4.195686in}}%
\pgfpathlineto{\pgfqpoint{4.166788in}{4.205141in}}%
\pgfpathlineto{\pgfqpoint{4.147250in}{4.224000in}}%
\pgfpathlineto{\pgfqpoint{4.144469in}{4.224000in}}%
\pgfpathlineto{\pgfqpoint{4.166788in}{4.202458in}}%
\pgfpathlineto{\pgfqpoint{4.175064in}{4.194376in}}%
\pgfpathlineto{\pgfqpoint{4.183106in}{4.186667in}}%
\pgfpathlineto{\pgfqpoint{4.206869in}{4.163657in}}%
\pgfpathlineto{\pgfqpoint{4.214364in}{4.156314in}}%
\pgfpathlineto{\pgfqpoint{4.221623in}{4.149333in}}%
\pgfpathlineto{\pgfqpoint{4.246949in}{4.124729in}}%
\pgfpathlineto{\pgfqpoint{4.253599in}{4.118194in}}%
\pgfpathlineto{\pgfqpoint{4.260020in}{4.112000in}}%
\pgfpathlineto{\pgfqpoint{4.287030in}{4.085676in}}%
\pgfpathlineto{\pgfqpoint{4.292772in}{4.080015in}}%
\pgfpathlineto{\pgfqpoint{4.298298in}{4.074667in}}%
\pgfpathlineto{\pgfqpoint{4.327111in}{4.046495in}}%
\pgfpathlineto{\pgfqpoint{4.331882in}{4.041777in}}%
\pgfpathlineto{\pgfqpoint{4.336458in}{4.037333in}}%
\pgfpathlineto{\pgfqpoint{4.351355in}{4.022582in}}%
\pgfpathlineto{\pgfqpoint{4.367192in}{4.007188in}}%
\pgfpathlineto{\pgfqpoint{4.370928in}{4.003480in}}%
\pgfpathlineto{\pgfqpoint{4.374501in}{4.000000in}}%
\pgfpathlineto{\pgfqpoint{4.407273in}{3.967753in}}%
\pgfpathlineto{\pgfqpoint{4.409912in}{3.965125in}}%
\pgfpathlineto{\pgfqpoint{4.412429in}{3.962667in}}%
\pgfpathlineto{\pgfqpoint{4.447354in}{3.928191in}}%
\pgfpathlineto{\pgfqpoint{4.448834in}{3.926712in}}%
\pgfpathlineto{\pgfqpoint{4.450241in}{3.925333in}}%
\pgfpathlineto{\pgfqpoint{4.468179in}{3.907398in}}%
\pgfpathlineto{\pgfqpoint{4.487434in}{3.888500in}}%
\pgfpathlineto{\pgfqpoint{4.487693in}{3.888241in}}%
\pgfpathlineto{\pgfqpoint{4.487938in}{3.888000in}}%
\pgfpathlineto{\pgfqpoint{4.506995in}{3.868886in}}%
\pgfpathlineto{\pgfqpoint{4.525500in}{3.850667in}}%
\pgfpathlineto{\pgfqpoint{4.527515in}{3.848662in}}%
\pgfpathlineto{\pgfqpoint{4.545748in}{3.830316in}}%
\pgfpathlineto{\pgfqpoint{4.562941in}{3.813333in}}%
\pgfpathlineto{\pgfqpoint{4.567596in}{3.808688in}}%
\pgfpathlineto{\pgfqpoint{4.584438in}{3.791688in}}%
\pgfpathlineto{\pgfqpoint{4.600270in}{3.776000in}}%
\pgfpathlineto{\pgfqpoint{4.603822in}{3.772410in}}%
\pgfpathlineto{\pgfqpoint{4.607677in}{3.768585in}}%
\pgfpathlineto{\pgfqpoint{4.637487in}{3.738667in}}%
\pgfpathlineto{\pgfqpoint{4.647758in}{3.728352in}}%
\pgfpathlineto{\pgfqpoint{4.674592in}{3.701333in}}%
\pgfpathlineto{\pgfqpoint{4.680923in}{3.694892in}}%
\pgfpathlineto{\pgfqpoint{4.687838in}{3.687988in}}%
\pgfpathlineto{\pgfqpoint{4.711587in}{3.664000in}}%
\pgfpathlineto{\pgfqpoint{4.719380in}{3.656047in}}%
\pgfpathlineto{\pgfqpoint{4.727919in}{3.647493in}}%
\pgfpathlineto{\pgfqpoint{4.748473in}{3.626667in}}%
\pgfpathlineto{\pgfqpoint{4.757775in}{3.617143in}}%
\pgfpathlineto{\pgfqpoint{4.768000in}{3.606867in}}%
\pgfusepath{fill}%
\end{pgfscope}%
\begin{pgfscope}%
\pgfpathrectangle{\pgfqpoint{0.800000in}{0.528000in}}{\pgfqpoint{3.968000in}{3.696000in}}%
\pgfusepath{clip}%
\pgfsetbuttcap%
\pgfsetroundjoin%
\definecolor{currentfill}{rgb}{0.352360,0.783011,0.392636}%
\pgfsetfillcolor{currentfill}%
\pgfsetlinewidth{0.000000pt}%
\definecolor{currentstroke}{rgb}{0.000000,0.000000,0.000000}%
\pgfsetstrokecolor{currentstroke}%
\pgfsetdash{}{0pt}%
\pgfpathmoveto{\pgfqpoint{4.768000in}{3.612345in}}%
\pgfpathlineto{\pgfqpoint{4.760604in}{3.619778in}}%
\pgfpathlineto{\pgfqpoint{4.753875in}{3.626667in}}%
\pgfpathlineto{\pgfqpoint{4.727919in}{3.652967in}}%
\pgfpathlineto{\pgfqpoint{4.722212in}{3.658684in}}%
\pgfpathlineto{\pgfqpoint{4.717003in}{3.664000in}}%
\pgfpathlineto{\pgfqpoint{4.687838in}{3.693458in}}%
\pgfpathlineto{\pgfqpoint{4.683758in}{3.697533in}}%
\pgfpathlineto{\pgfqpoint{4.680022in}{3.701333in}}%
\pgfpathlineto{\pgfqpoint{4.647758in}{3.733818in}}%
\pgfpathlineto{\pgfqpoint{4.642930in}{3.738667in}}%
\pgfpathlineto{\pgfqpoint{4.607677in}{3.774048in}}%
\pgfpathlineto{\pgfqpoint{4.606662in}{3.775055in}}%
\pgfpathlineto{\pgfqpoint{4.605727in}{3.776000in}}%
\pgfpathlineto{\pgfqpoint{4.587251in}{3.794308in}}%
\pgfpathlineto{\pgfqpoint{4.568403in}{3.813333in}}%
\pgfpathlineto{\pgfqpoint{4.567596in}{3.814139in}}%
\pgfpathlineto{\pgfqpoint{4.548563in}{3.832939in}}%
\pgfpathlineto{\pgfqpoint{4.530945in}{3.850667in}}%
\pgfpathlineto{\pgfqpoint{4.527515in}{3.854082in}}%
\pgfpathlineto{\pgfqpoint{4.509813in}{3.871511in}}%
\pgfpathlineto{\pgfqpoint{4.493374in}{3.888000in}}%
\pgfpathlineto{\pgfqpoint{4.490485in}{3.890841in}}%
\pgfpathlineto{\pgfqpoint{4.487434in}{3.893897in}}%
\pgfpathlineto{\pgfqpoint{4.471000in}{3.910026in}}%
\pgfpathlineto{\pgfqpoint{4.455690in}{3.925333in}}%
\pgfpathlineto{\pgfqpoint{4.451628in}{3.929315in}}%
\pgfpathlineto{\pgfqpoint{4.447354in}{3.933583in}}%
\pgfpathlineto{\pgfqpoint{4.417891in}{3.962667in}}%
\pgfpathlineto{\pgfqpoint{4.412709in}{3.967731in}}%
\pgfpathlineto{\pgfqpoint{4.407273in}{3.973142in}}%
\pgfpathlineto{\pgfqpoint{4.379978in}{4.000000in}}%
\pgfpathlineto{\pgfqpoint{4.373728in}{4.006088in}}%
\pgfpathlineto{\pgfqpoint{4.367192in}{4.012573in}}%
\pgfpathlineto{\pgfqpoint{4.354184in}{4.025217in}}%
\pgfpathlineto{\pgfqpoint{4.341948in}{4.037333in}}%
\pgfpathlineto{\pgfqpoint{4.334684in}{4.044387in}}%
\pgfpathlineto{\pgfqpoint{4.327111in}{4.051877in}}%
\pgfpathlineto{\pgfqpoint{4.303802in}{4.074667in}}%
\pgfpathlineto{\pgfqpoint{4.295577in}{4.082627in}}%
\pgfpathlineto{\pgfqpoint{4.287030in}{4.091054in}}%
\pgfpathlineto{\pgfqpoint{4.265538in}{4.112000in}}%
\pgfpathlineto{\pgfqpoint{4.256407in}{4.120809in}}%
\pgfpathlineto{\pgfqpoint{4.246949in}{4.130104in}}%
\pgfpathlineto{\pgfqpoint{4.227155in}{4.149333in}}%
\pgfpathlineto{\pgfqpoint{4.217174in}{4.158932in}}%
\pgfpathlineto{\pgfqpoint{4.206869in}{4.169027in}}%
\pgfpathlineto{\pgfqpoint{4.188653in}{4.186667in}}%
\pgfpathlineto{\pgfqpoint{4.177877in}{4.196996in}}%
\pgfpathlineto{\pgfqpoint{4.166788in}{4.207825in}}%
\pgfpathlineto{\pgfqpoint{4.150030in}{4.224000in}}%
\pgfpathlineto{\pgfqpoint{4.147250in}{4.224000in}}%
\pgfpathlineto{\pgfqpoint{4.166788in}{4.205141in}}%
\pgfpathlineto{\pgfqpoint{4.176471in}{4.195686in}}%
\pgfpathlineto{\pgfqpoint{4.185879in}{4.186667in}}%
\pgfpathlineto{\pgfqpoint{4.206869in}{4.166342in}}%
\pgfpathlineto{\pgfqpoint{4.215769in}{4.157623in}}%
\pgfpathlineto{\pgfqpoint{4.224389in}{4.149333in}}%
\pgfpathlineto{\pgfqpoint{4.246949in}{4.127417in}}%
\pgfpathlineto{\pgfqpoint{4.255003in}{4.119502in}}%
\pgfpathlineto{\pgfqpoint{4.262779in}{4.112000in}}%
\pgfpathlineto{\pgfqpoint{4.287030in}{4.088365in}}%
\pgfpathlineto{\pgfqpoint{4.294175in}{4.081321in}}%
\pgfpathlineto{\pgfqpoint{4.301050in}{4.074667in}}%
\pgfpathlineto{\pgfqpoint{4.327111in}{4.049186in}}%
\pgfpathlineto{\pgfqpoint{4.333283in}{4.043082in}}%
\pgfpathlineto{\pgfqpoint{4.339203in}{4.037333in}}%
\pgfpathlineto{\pgfqpoint{4.352769in}{4.023899in}}%
\pgfpathlineto{\pgfqpoint{4.367192in}{4.009880in}}%
\pgfpathlineto{\pgfqpoint{4.372328in}{4.004784in}}%
\pgfpathlineto{\pgfqpoint{4.377240in}{4.000000in}}%
\pgfpathlineto{\pgfqpoint{4.407273in}{3.970447in}}%
\pgfpathlineto{\pgfqpoint{4.411311in}{3.966428in}}%
\pgfpathlineto{\pgfqpoint{4.415160in}{3.962667in}}%
\pgfpathlineto{\pgfqpoint{4.447354in}{3.930887in}}%
\pgfpathlineto{\pgfqpoint{4.450231in}{3.928014in}}%
\pgfpathlineto{\pgfqpoint{4.452965in}{3.925333in}}%
\pgfpathlineto{\pgfqpoint{4.469590in}{3.908712in}}%
\pgfpathlineto{\pgfqpoint{4.487434in}{3.891199in}}%
\pgfpathlineto{\pgfqpoint{4.489089in}{3.889541in}}%
\pgfpathlineto{\pgfqpoint{4.490656in}{3.888000in}}%
\pgfpathlineto{\pgfqpoint{4.508404in}{3.870199in}}%
\pgfpathlineto{\pgfqpoint{4.527515in}{3.851382in}}%
\pgfpathlineto{\pgfqpoint{4.528234in}{3.850667in}}%
\pgfpathlineto{\pgfqpoint{4.547155in}{3.831627in}}%
\pgfpathlineto{\pgfqpoint{4.565677in}{3.813333in}}%
\pgfpathlineto{\pgfqpoint{4.567596in}{3.811418in}}%
\pgfpathlineto{\pgfqpoint{4.585845in}{3.792998in}}%
\pgfpathlineto{\pgfqpoint{4.602999in}{3.776000in}}%
\pgfpathlineto{\pgfqpoint{4.605242in}{3.773732in}}%
\pgfpathlineto{\pgfqpoint{4.607677in}{3.771317in}}%
\pgfpathlineto{\pgfqpoint{4.640208in}{3.738667in}}%
\pgfpathlineto{\pgfqpoint{4.647758in}{3.731085in}}%
\pgfpathlineto{\pgfqpoint{4.677307in}{3.701333in}}%
\pgfpathlineto{\pgfqpoint{4.682341in}{3.696212in}}%
\pgfpathlineto{\pgfqpoint{4.687838in}{3.690723in}}%
\pgfpathlineto{\pgfqpoint{4.714295in}{3.664000in}}%
\pgfpathlineto{\pgfqpoint{4.720796in}{3.657365in}}%
\pgfpathlineto{\pgfqpoint{4.727919in}{3.650230in}}%
\pgfpathlineto{\pgfqpoint{4.751174in}{3.626667in}}%
\pgfpathlineto{\pgfqpoint{4.759190in}{3.618460in}}%
\pgfpathlineto{\pgfqpoint{4.768000in}{3.609606in}}%
\pgfusepath{fill}%
\end{pgfscope}%
\begin{pgfscope}%
\pgfpathrectangle{\pgfqpoint{0.800000in}{0.528000in}}{\pgfqpoint{3.968000in}{3.696000in}}%
\pgfusepath{clip}%
\pgfsetbuttcap%
\pgfsetroundjoin%
\definecolor{currentfill}{rgb}{0.352360,0.783011,0.392636}%
\pgfsetfillcolor{currentfill}%
\pgfsetlinewidth{0.000000pt}%
\definecolor{currentstroke}{rgb}{0.000000,0.000000,0.000000}%
\pgfsetstrokecolor{currentstroke}%
\pgfsetdash{}{0pt}%
\pgfpathmoveto{\pgfqpoint{4.768000in}{3.615084in}}%
\pgfpathlineto{\pgfqpoint{4.762018in}{3.621095in}}%
\pgfpathlineto{\pgfqpoint{4.756577in}{3.626667in}}%
\pgfpathlineto{\pgfqpoint{4.727919in}{3.655704in}}%
\pgfpathlineto{\pgfqpoint{4.723628in}{3.660003in}}%
\pgfpathlineto{\pgfqpoint{4.719711in}{3.664000in}}%
\pgfpathlineto{\pgfqpoint{4.687838in}{3.696193in}}%
\pgfpathlineto{\pgfqpoint{4.685175in}{3.698853in}}%
\pgfpathlineto{\pgfqpoint{4.682737in}{3.701333in}}%
\pgfpathlineto{\pgfqpoint{4.647758in}{3.736552in}}%
\pgfpathlineto{\pgfqpoint{4.645652in}{3.738667in}}%
\pgfpathlineto{\pgfqpoint{4.618475in}{3.765942in}}%
\pgfpathlineto{\pgfqpoint{4.608447in}{3.776000in}}%
\pgfpathlineto{\pgfqpoint{4.607677in}{3.776772in}}%
\pgfpathlineto{\pgfqpoint{4.588657in}{3.795618in}}%
\pgfpathlineto{\pgfqpoint{4.571107in}{3.813333in}}%
\pgfpathlineto{\pgfqpoint{4.567596in}{3.816841in}}%
\pgfpathlineto{\pgfqpoint{4.549971in}{3.834250in}}%
\pgfpathlineto{\pgfqpoint{4.533655in}{3.850667in}}%
\pgfpathlineto{\pgfqpoint{4.527515in}{3.856782in}}%
\pgfpathlineto{\pgfqpoint{4.511222in}{3.872824in}}%
\pgfpathlineto{\pgfqpoint{4.496091in}{3.888000in}}%
\pgfpathlineto{\pgfqpoint{4.491880in}{3.892141in}}%
\pgfpathlineto{\pgfqpoint{4.487434in}{3.896595in}}%
\pgfpathlineto{\pgfqpoint{4.472411in}{3.911339in}}%
\pgfpathlineto{\pgfqpoint{4.458414in}{3.925333in}}%
\pgfpathlineto{\pgfqpoint{4.453025in}{3.930616in}}%
\pgfpathlineto{\pgfqpoint{4.447354in}{3.936279in}}%
\pgfpathlineto{\pgfqpoint{4.420623in}{3.962667in}}%
\pgfpathlineto{\pgfqpoint{4.414108in}{3.969033in}}%
\pgfpathlineto{\pgfqpoint{4.407273in}{3.975836in}}%
\pgfpathlineto{\pgfqpoint{4.382716in}{4.000000in}}%
\pgfpathlineto{\pgfqpoint{4.375128in}{4.007392in}}%
\pgfpathlineto{\pgfqpoint{4.367192in}{4.015266in}}%
\pgfpathlineto{\pgfqpoint{4.355598in}{4.026535in}}%
\pgfpathlineto{\pgfqpoint{4.344693in}{4.037333in}}%
\pgfpathlineto{\pgfqpoint{4.336085in}{4.045692in}}%
\pgfpathlineto{\pgfqpoint{4.327111in}{4.054568in}}%
\pgfpathlineto{\pgfqpoint{4.306554in}{4.074667in}}%
\pgfpathlineto{\pgfqpoint{4.296979in}{4.083934in}}%
\pgfpathlineto{\pgfqpoint{4.287030in}{4.093743in}}%
\pgfpathlineto{\pgfqpoint{4.268297in}{4.112000in}}%
\pgfpathlineto{\pgfqpoint{4.257811in}{4.122117in}}%
\pgfpathlineto{\pgfqpoint{4.246949in}{4.132791in}}%
\pgfpathlineto{\pgfqpoint{4.229921in}{4.149333in}}%
\pgfpathlineto{\pgfqpoint{4.218579in}{4.160241in}}%
\pgfpathlineto{\pgfqpoint{4.206869in}{4.171713in}}%
\pgfpathlineto{\pgfqpoint{4.191426in}{4.186667in}}%
\pgfpathlineto{\pgfqpoint{4.179284in}{4.198306in}}%
\pgfpathlineto{\pgfqpoint{4.166788in}{4.210508in}}%
\pgfpathlineto{\pgfqpoint{4.152810in}{4.224000in}}%
\pgfpathlineto{\pgfqpoint{4.150030in}{4.224000in}}%
\pgfpathlineto{\pgfqpoint{4.166788in}{4.207825in}}%
\pgfpathlineto{\pgfqpoint{4.177877in}{4.196996in}}%
\pgfpathlineto{\pgfqpoint{4.188653in}{4.186667in}}%
\pgfpathlineto{\pgfqpoint{4.206869in}{4.169027in}}%
\pgfpathlineto{\pgfqpoint{4.217174in}{4.158932in}}%
\pgfpathlineto{\pgfqpoint{4.227155in}{4.149333in}}%
\pgfpathlineto{\pgfqpoint{4.246949in}{4.130104in}}%
\pgfpathlineto{\pgfqpoint{4.256407in}{4.120809in}}%
\pgfpathlineto{\pgfqpoint{4.265538in}{4.112000in}}%
\pgfpathlineto{\pgfqpoint{4.287030in}{4.091054in}}%
\pgfpathlineto{\pgfqpoint{4.295577in}{4.082627in}}%
\pgfpathlineto{\pgfqpoint{4.303802in}{4.074667in}}%
\pgfpathlineto{\pgfqpoint{4.327111in}{4.051877in}}%
\pgfpathlineto{\pgfqpoint{4.334684in}{4.044387in}}%
\pgfpathlineto{\pgfqpoint{4.341948in}{4.037333in}}%
\pgfpathlineto{\pgfqpoint{4.354184in}{4.025217in}}%
\pgfpathlineto{\pgfqpoint{4.367192in}{4.012573in}}%
\pgfpathlineto{\pgfqpoint{4.373728in}{4.006088in}}%
\pgfpathlineto{\pgfqpoint{4.379978in}{4.000000in}}%
\pgfpathlineto{\pgfqpoint{4.407273in}{3.973142in}}%
\pgfpathlineto{\pgfqpoint{4.412709in}{3.967731in}}%
\pgfpathlineto{\pgfqpoint{4.417891in}{3.962667in}}%
\pgfpathlineto{\pgfqpoint{4.447354in}{3.933583in}}%
\pgfpathlineto{\pgfqpoint{4.451628in}{3.929315in}}%
\pgfpathlineto{\pgfqpoint{4.455690in}{3.925333in}}%
\pgfpathlineto{\pgfqpoint{4.471000in}{3.910026in}}%
\pgfpathlineto{\pgfqpoint{4.487434in}{3.893897in}}%
\pgfpathlineto{\pgfqpoint{4.490485in}{3.890841in}}%
\pgfpathlineto{\pgfqpoint{4.493374in}{3.888000in}}%
\pgfpathlineto{\pgfqpoint{4.509813in}{3.871511in}}%
\pgfpathlineto{\pgfqpoint{4.527515in}{3.854082in}}%
\pgfpathlineto{\pgfqpoint{4.530945in}{3.850667in}}%
\pgfpathlineto{\pgfqpoint{4.548563in}{3.832939in}}%
\pgfpathlineto{\pgfqpoint{4.567596in}{3.814139in}}%
\pgfpathlineto{\pgfqpoint{4.568403in}{3.813333in}}%
\pgfpathlineto{\pgfqpoint{4.587251in}{3.794308in}}%
\pgfpathlineto{\pgfqpoint{4.605727in}{3.776000in}}%
\pgfpathlineto{\pgfqpoint{4.606662in}{3.775055in}}%
\pgfpathlineto{\pgfqpoint{4.607677in}{3.774048in}}%
\pgfpathlineto{\pgfqpoint{4.642930in}{3.738667in}}%
\pgfpathlineto{\pgfqpoint{4.647758in}{3.733818in}}%
\pgfpathlineto{\pgfqpoint{4.680022in}{3.701333in}}%
\pgfpathlineto{\pgfqpoint{4.683758in}{3.697533in}}%
\pgfpathlineto{\pgfqpoint{4.687838in}{3.693458in}}%
\pgfpathlineto{\pgfqpoint{4.717003in}{3.664000in}}%
\pgfpathlineto{\pgfqpoint{4.722212in}{3.658684in}}%
\pgfpathlineto{\pgfqpoint{4.727919in}{3.652967in}}%
\pgfpathlineto{\pgfqpoint{4.753875in}{3.626667in}}%
\pgfpathlineto{\pgfqpoint{4.760604in}{3.619778in}}%
\pgfpathlineto{\pgfqpoint{4.768000in}{3.612345in}}%
\pgfusepath{fill}%
\end{pgfscope}%
\begin{pgfscope}%
\pgfpathrectangle{\pgfqpoint{0.800000in}{0.528000in}}{\pgfqpoint{3.968000in}{3.696000in}}%
\pgfusepath{clip}%
\pgfsetbuttcap%
\pgfsetroundjoin%
\definecolor{currentfill}{rgb}{0.360741,0.785964,0.387814}%
\pgfsetfillcolor{currentfill}%
\pgfsetlinewidth{0.000000pt}%
\definecolor{currentstroke}{rgb}{0.000000,0.000000,0.000000}%
\pgfsetstrokecolor{currentstroke}%
\pgfsetdash{}{0pt}%
\pgfpathmoveto{\pgfqpoint{4.768000in}{3.617823in}}%
\pgfpathlineto{\pgfqpoint{4.763433in}{3.622413in}}%
\pgfpathlineto{\pgfqpoint{4.759278in}{3.626667in}}%
\pgfpathlineto{\pgfqpoint{4.727919in}{3.658441in}}%
\pgfpathlineto{\pgfqpoint{4.725044in}{3.661322in}}%
\pgfpathlineto{\pgfqpoint{4.722419in}{3.664000in}}%
\pgfpathlineto{\pgfqpoint{4.687838in}{3.698928in}}%
\pgfpathlineto{\pgfqpoint{4.686592in}{3.700173in}}%
\pgfpathlineto{\pgfqpoint{4.685451in}{3.701333in}}%
\pgfpathlineto{\pgfqpoint{4.655958in}{3.731028in}}%
\pgfpathlineto{\pgfqpoint{4.648366in}{3.738667in}}%
\pgfpathlineto{\pgfqpoint{4.648072in}{3.738960in}}%
\pgfpathlineto{\pgfqpoint{4.647758in}{3.739279in}}%
\pgfpathlineto{\pgfqpoint{4.611144in}{3.776000in}}%
\pgfpathlineto{\pgfqpoint{4.607677in}{3.779475in}}%
\pgfpathlineto{\pgfqpoint{4.590064in}{3.796928in}}%
\pgfpathlineto{\pgfqpoint{4.573811in}{3.813333in}}%
\pgfpathlineto{\pgfqpoint{4.567596in}{3.819543in}}%
\pgfpathlineto{\pgfqpoint{4.551379in}{3.835561in}}%
\pgfpathlineto{\pgfqpoint{4.536366in}{3.850667in}}%
\pgfpathlineto{\pgfqpoint{4.527515in}{3.859482in}}%
\pgfpathlineto{\pgfqpoint{4.512631in}{3.874136in}}%
\pgfpathlineto{\pgfqpoint{4.498809in}{3.888000in}}%
\pgfpathlineto{\pgfqpoint{4.493276in}{3.893441in}}%
\pgfpathlineto{\pgfqpoint{4.487434in}{3.899293in}}%
\pgfpathlineto{\pgfqpoint{4.473821in}{3.912653in}}%
\pgfpathlineto{\pgfqpoint{4.461139in}{3.925333in}}%
\pgfpathlineto{\pgfqpoint{4.454422in}{3.931917in}}%
\pgfpathlineto{\pgfqpoint{4.447354in}{3.938976in}}%
\pgfpathlineto{\pgfqpoint{4.423354in}{3.962667in}}%
\pgfpathlineto{\pgfqpoint{4.415506in}{3.970336in}}%
\pgfpathlineto{\pgfqpoint{4.407273in}{3.978531in}}%
\pgfpathlineto{\pgfqpoint{4.385454in}{4.000000in}}%
\pgfpathlineto{\pgfqpoint{4.376527in}{4.008696in}}%
\pgfpathlineto{\pgfqpoint{4.367192in}{4.017958in}}%
\pgfpathlineto{\pgfqpoint{4.357013in}{4.027852in}}%
\pgfpathlineto{\pgfqpoint{4.347439in}{4.037333in}}%
\pgfpathlineto{\pgfqpoint{4.337486in}{4.046997in}}%
\pgfpathlineto{\pgfqpoint{4.327111in}{4.057258in}}%
\pgfpathlineto{\pgfqpoint{4.309306in}{4.074667in}}%
\pgfpathlineto{\pgfqpoint{4.298382in}{4.085240in}}%
\pgfpathlineto{\pgfqpoint{4.287030in}{4.096432in}}%
\pgfpathlineto{\pgfqpoint{4.271056in}{4.112000in}}%
\pgfpathlineto{\pgfqpoint{4.259215in}{4.123424in}}%
\pgfpathlineto{\pgfqpoint{4.246949in}{4.135478in}}%
\pgfpathlineto{\pgfqpoint{4.232687in}{4.149333in}}%
\pgfpathlineto{\pgfqpoint{4.219984in}{4.161550in}}%
\pgfpathlineto{\pgfqpoint{4.206869in}{4.174398in}}%
\pgfpathlineto{\pgfqpoint{4.194199in}{4.186667in}}%
\pgfpathlineto{\pgfqpoint{4.180690in}{4.199616in}}%
\pgfpathlineto{\pgfqpoint{4.166788in}{4.213192in}}%
\pgfpathlineto{\pgfqpoint{4.155590in}{4.224000in}}%
\pgfpathlineto{\pgfqpoint{4.152810in}{4.224000in}}%
\pgfpathlineto{\pgfqpoint{4.166788in}{4.210508in}}%
\pgfpathlineto{\pgfqpoint{4.179284in}{4.198306in}}%
\pgfpathlineto{\pgfqpoint{4.191426in}{4.186667in}}%
\pgfpathlineto{\pgfqpoint{4.206869in}{4.171713in}}%
\pgfpathlineto{\pgfqpoint{4.218579in}{4.160241in}}%
\pgfpathlineto{\pgfqpoint{4.229921in}{4.149333in}}%
\pgfpathlineto{\pgfqpoint{4.246949in}{4.132791in}}%
\pgfpathlineto{\pgfqpoint{4.257811in}{4.122117in}}%
\pgfpathlineto{\pgfqpoint{4.268297in}{4.112000in}}%
\pgfpathlineto{\pgfqpoint{4.287030in}{4.093743in}}%
\pgfpathlineto{\pgfqpoint{4.296979in}{4.083934in}}%
\pgfpathlineto{\pgfqpoint{4.306554in}{4.074667in}}%
\pgfpathlineto{\pgfqpoint{4.327111in}{4.054568in}}%
\pgfpathlineto{\pgfqpoint{4.336085in}{4.045692in}}%
\pgfpathlineto{\pgfqpoint{4.344693in}{4.037333in}}%
\pgfpathlineto{\pgfqpoint{4.355598in}{4.026535in}}%
\pgfpathlineto{\pgfqpoint{4.367192in}{4.015266in}}%
\pgfpathlineto{\pgfqpoint{4.375128in}{4.007392in}}%
\pgfpathlineto{\pgfqpoint{4.382716in}{4.000000in}}%
\pgfpathlineto{\pgfqpoint{4.407273in}{3.975836in}}%
\pgfpathlineto{\pgfqpoint{4.414108in}{3.969033in}}%
\pgfpathlineto{\pgfqpoint{4.420623in}{3.962667in}}%
\pgfpathlineto{\pgfqpoint{4.447354in}{3.936279in}}%
\pgfpathlineto{\pgfqpoint{4.453025in}{3.930616in}}%
\pgfpathlineto{\pgfqpoint{4.458414in}{3.925333in}}%
\pgfpathlineto{\pgfqpoint{4.472411in}{3.911339in}}%
\pgfpathlineto{\pgfqpoint{4.487434in}{3.896595in}}%
\pgfpathlineto{\pgfqpoint{4.491880in}{3.892141in}}%
\pgfpathlineto{\pgfqpoint{4.496091in}{3.888000in}}%
\pgfpathlineto{\pgfqpoint{4.511222in}{3.872824in}}%
\pgfpathlineto{\pgfqpoint{4.527515in}{3.856782in}}%
\pgfpathlineto{\pgfqpoint{4.533655in}{3.850667in}}%
\pgfpathlineto{\pgfqpoint{4.549971in}{3.834250in}}%
\pgfpathlineto{\pgfqpoint{4.567596in}{3.816841in}}%
\pgfpathlineto{\pgfqpoint{4.571107in}{3.813333in}}%
\pgfpathlineto{\pgfqpoint{4.588657in}{3.795618in}}%
\pgfpathlineto{\pgfqpoint{4.607677in}{3.776772in}}%
\pgfpathlineto{\pgfqpoint{4.608447in}{3.776000in}}%
\pgfpathlineto{\pgfqpoint{4.618475in}{3.765942in}}%
\pgfpathlineto{\pgfqpoint{4.645652in}{3.738667in}}%
\pgfpathlineto{\pgfqpoint{4.647758in}{3.736552in}}%
\pgfpathlineto{\pgfqpoint{4.682737in}{3.701333in}}%
\pgfpathlineto{\pgfqpoint{4.685175in}{3.698853in}}%
\pgfpathlineto{\pgfqpoint{4.687838in}{3.696193in}}%
\pgfpathlineto{\pgfqpoint{4.719711in}{3.664000in}}%
\pgfpathlineto{\pgfqpoint{4.723628in}{3.660003in}}%
\pgfpathlineto{\pgfqpoint{4.727919in}{3.655704in}}%
\pgfpathlineto{\pgfqpoint{4.756577in}{3.626667in}}%
\pgfpathlineto{\pgfqpoint{4.762018in}{3.621095in}}%
\pgfpathlineto{\pgfqpoint{4.768000in}{3.615084in}}%
\pgfusepath{fill}%
\end{pgfscope}%
\begin{pgfscope}%
\pgfpathrectangle{\pgfqpoint{0.800000in}{0.528000in}}{\pgfqpoint{3.968000in}{3.696000in}}%
\pgfusepath{clip}%
\pgfsetbuttcap%
\pgfsetroundjoin%
\definecolor{currentfill}{rgb}{0.360741,0.785964,0.387814}%
\pgfsetfillcolor{currentfill}%
\pgfsetlinewidth{0.000000pt}%
\definecolor{currentstroke}{rgb}{0.000000,0.000000,0.000000}%
\pgfsetstrokecolor{currentstroke}%
\pgfsetdash{}{0pt}%
\pgfpathmoveto{\pgfqpoint{4.768000in}{3.620562in}}%
\pgfpathlineto{\pgfqpoint{4.764847in}{3.623730in}}%
\pgfpathlineto{\pgfqpoint{4.761979in}{3.626667in}}%
\pgfpathlineto{\pgfqpoint{4.727919in}{3.661178in}}%
\pgfpathlineto{\pgfqpoint{4.726460in}{3.662640in}}%
\pgfpathlineto{\pgfqpoint{4.725127in}{3.664000in}}%
\pgfpathlineto{\pgfqpoint{4.692040in}{3.697420in}}%
\pgfpathlineto{\pgfqpoint{4.688162in}{3.701333in}}%
\pgfpathlineto{\pgfqpoint{4.688006in}{3.701490in}}%
\pgfpathlineto{\pgfqpoint{4.687838in}{3.701660in}}%
\pgfpathlineto{\pgfqpoint{4.651057in}{3.738667in}}%
\pgfpathlineto{\pgfqpoint{4.649462in}{3.740255in}}%
\pgfpathlineto{\pgfqpoint{4.647758in}{3.741984in}}%
\pgfpathlineto{\pgfqpoint{4.613841in}{3.776000in}}%
\pgfpathlineto{\pgfqpoint{4.607677in}{3.782179in}}%
\pgfpathlineto{\pgfqpoint{4.591470in}{3.798238in}}%
\pgfpathlineto{\pgfqpoint{4.576515in}{3.813333in}}%
\pgfpathlineto{\pgfqpoint{4.567596in}{3.822244in}}%
\pgfpathlineto{\pgfqpoint{4.552786in}{3.836872in}}%
\pgfpathlineto{\pgfqpoint{4.539077in}{3.850667in}}%
\pgfpathlineto{\pgfqpoint{4.527515in}{3.862182in}}%
\pgfpathlineto{\pgfqpoint{4.514040in}{3.875449in}}%
\pgfpathlineto{\pgfqpoint{4.501527in}{3.888000in}}%
\pgfpathlineto{\pgfqpoint{4.494672in}{3.894741in}}%
\pgfpathlineto{\pgfqpoint{4.487434in}{3.901991in}}%
\pgfpathlineto{\pgfqpoint{4.475232in}{3.913967in}}%
\pgfpathlineto{\pgfqpoint{4.463863in}{3.925333in}}%
\pgfpathlineto{\pgfqpoint{4.455819in}{3.933219in}}%
\pgfpathlineto{\pgfqpoint{4.447354in}{3.941672in}}%
\pgfpathlineto{\pgfqpoint{4.426085in}{3.962667in}}%
\pgfpathlineto{\pgfqpoint{4.416904in}{3.971638in}}%
\pgfpathlineto{\pgfqpoint{4.407273in}{3.981225in}}%
\pgfpathlineto{\pgfqpoint{4.388193in}{4.000000in}}%
\pgfpathlineto{\pgfqpoint{4.377927in}{4.009999in}}%
\pgfpathlineto{\pgfqpoint{4.367192in}{4.020651in}}%
\pgfpathlineto{\pgfqpoint{4.358428in}{4.029170in}}%
\pgfpathlineto{\pgfqpoint{4.350184in}{4.037333in}}%
\pgfpathlineto{\pgfqpoint{4.338887in}{4.048302in}}%
\pgfpathlineto{\pgfqpoint{4.327111in}{4.059949in}}%
\pgfpathlineto{\pgfqpoint{4.312058in}{4.074667in}}%
\pgfpathlineto{\pgfqpoint{4.299784in}{4.086546in}}%
\pgfpathlineto{\pgfqpoint{4.287030in}{4.099121in}}%
\pgfpathlineto{\pgfqpoint{4.273815in}{4.112000in}}%
\pgfpathlineto{\pgfqpoint{4.260618in}{4.124732in}}%
\pgfpathlineto{\pgfqpoint{4.246949in}{4.138165in}}%
\pgfpathlineto{\pgfqpoint{4.235453in}{4.149333in}}%
\pgfpathlineto{\pgfqpoint{4.221389in}{4.162859in}}%
\pgfpathlineto{\pgfqpoint{4.206869in}{4.177084in}}%
\pgfpathlineto{\pgfqpoint{4.196972in}{4.186667in}}%
\pgfpathlineto{\pgfqpoint{4.182097in}{4.200926in}}%
\pgfpathlineto{\pgfqpoint{4.166788in}{4.215876in}}%
\pgfpathlineto{\pgfqpoint{4.158371in}{4.224000in}}%
\pgfpathlineto{\pgfqpoint{4.155590in}{4.224000in}}%
\pgfpathlineto{\pgfqpoint{4.166788in}{4.213192in}}%
\pgfpathlineto{\pgfqpoint{4.180690in}{4.199616in}}%
\pgfpathlineto{\pgfqpoint{4.194199in}{4.186667in}}%
\pgfpathlineto{\pgfqpoint{4.206869in}{4.174398in}}%
\pgfpathlineto{\pgfqpoint{4.219984in}{4.161550in}}%
\pgfpathlineto{\pgfqpoint{4.232687in}{4.149333in}}%
\pgfpathlineto{\pgfqpoint{4.246949in}{4.135478in}}%
\pgfpathlineto{\pgfqpoint{4.259215in}{4.123424in}}%
\pgfpathlineto{\pgfqpoint{4.271056in}{4.112000in}}%
\pgfpathlineto{\pgfqpoint{4.287030in}{4.096432in}}%
\pgfpathlineto{\pgfqpoint{4.298382in}{4.085240in}}%
\pgfpathlineto{\pgfqpoint{4.309306in}{4.074667in}}%
\pgfpathlineto{\pgfqpoint{4.327111in}{4.057258in}}%
\pgfpathlineto{\pgfqpoint{4.337486in}{4.046997in}}%
\pgfpathlineto{\pgfqpoint{4.347439in}{4.037333in}}%
\pgfpathlineto{\pgfqpoint{4.357013in}{4.027852in}}%
\pgfpathlineto{\pgfqpoint{4.367192in}{4.017958in}}%
\pgfpathlineto{\pgfqpoint{4.376527in}{4.008696in}}%
\pgfpathlineto{\pgfqpoint{4.385454in}{4.000000in}}%
\pgfpathlineto{\pgfqpoint{4.407273in}{3.978531in}}%
\pgfpathlineto{\pgfqpoint{4.415506in}{3.970336in}}%
\pgfpathlineto{\pgfqpoint{4.423354in}{3.962667in}}%
\pgfpathlineto{\pgfqpoint{4.447354in}{3.938976in}}%
\pgfpathlineto{\pgfqpoint{4.454422in}{3.931917in}}%
\pgfpathlineto{\pgfqpoint{4.461139in}{3.925333in}}%
\pgfpathlineto{\pgfqpoint{4.473821in}{3.912653in}}%
\pgfpathlineto{\pgfqpoint{4.487434in}{3.899293in}}%
\pgfpathlineto{\pgfqpoint{4.493276in}{3.893441in}}%
\pgfpathlineto{\pgfqpoint{4.498809in}{3.888000in}}%
\pgfpathlineto{\pgfqpoint{4.512631in}{3.874136in}}%
\pgfpathlineto{\pgfqpoint{4.527515in}{3.859482in}}%
\pgfpathlineto{\pgfqpoint{4.536366in}{3.850667in}}%
\pgfpathlineto{\pgfqpoint{4.551379in}{3.835561in}}%
\pgfpathlineto{\pgfqpoint{4.567596in}{3.819543in}}%
\pgfpathlineto{\pgfqpoint{4.573811in}{3.813333in}}%
\pgfpathlineto{\pgfqpoint{4.590064in}{3.796928in}}%
\pgfpathlineto{\pgfqpoint{4.607677in}{3.779475in}}%
\pgfpathlineto{\pgfqpoint{4.611144in}{3.776000in}}%
\pgfpathlineto{\pgfqpoint{4.647758in}{3.739279in}}%
\pgfpathlineto{\pgfqpoint{4.648072in}{3.738960in}}%
\pgfpathlineto{\pgfqpoint{4.648366in}{3.738667in}}%
\pgfpathlineto{\pgfqpoint{4.655958in}{3.731028in}}%
\pgfpathlineto{\pgfqpoint{4.685451in}{3.701333in}}%
\pgfpathlineto{\pgfqpoint{4.686592in}{3.700173in}}%
\pgfpathlineto{\pgfqpoint{4.687838in}{3.698928in}}%
\pgfpathlineto{\pgfqpoint{4.722419in}{3.664000in}}%
\pgfpathlineto{\pgfqpoint{4.725044in}{3.661322in}}%
\pgfpathlineto{\pgfqpoint{4.727919in}{3.658441in}}%
\pgfpathlineto{\pgfqpoint{4.759278in}{3.626667in}}%
\pgfpathlineto{\pgfqpoint{4.763433in}{3.622413in}}%
\pgfpathlineto{\pgfqpoint{4.768000in}{3.617823in}}%
\pgfusepath{fill}%
\end{pgfscope}%
\begin{pgfscope}%
\pgfpathrectangle{\pgfqpoint{0.800000in}{0.528000in}}{\pgfqpoint{3.968000in}{3.696000in}}%
\pgfusepath{clip}%
\pgfsetbuttcap%
\pgfsetroundjoin%
\definecolor{currentfill}{rgb}{0.360741,0.785964,0.387814}%
\pgfsetfillcolor{currentfill}%
\pgfsetlinewidth{0.000000pt}%
\definecolor{currentstroke}{rgb}{0.000000,0.000000,0.000000}%
\pgfsetstrokecolor{currentstroke}%
\pgfsetdash{}{0pt}%
\pgfpathmoveto{\pgfqpoint{4.768000in}{3.623301in}}%
\pgfpathlineto{\pgfqpoint{4.766262in}{3.625048in}}%
\pgfpathlineto{\pgfqpoint{4.764680in}{3.626667in}}%
\pgfpathlineto{\pgfqpoint{4.727919in}{3.663915in}}%
\pgfpathlineto{\pgfqpoint{4.727875in}{3.663959in}}%
\pgfpathlineto{\pgfqpoint{4.727835in}{3.664000in}}%
\pgfpathlineto{\pgfqpoint{4.726841in}{3.665004in}}%
\pgfpathlineto{\pgfqpoint{4.690846in}{3.701333in}}%
\pgfpathlineto{\pgfqpoint{4.689395in}{3.702783in}}%
\pgfpathlineto{\pgfqpoint{4.687838in}{3.704367in}}%
\pgfpathlineto{\pgfqpoint{4.653748in}{3.738667in}}%
\pgfpathlineto{\pgfqpoint{4.650853in}{3.741550in}}%
\pgfpathlineto{\pgfqpoint{4.647758in}{3.744689in}}%
\pgfpathlineto{\pgfqpoint{4.616539in}{3.776000in}}%
\pgfpathlineto{\pgfqpoint{4.607677in}{3.784882in}}%
\pgfpathlineto{\pgfqpoint{4.592876in}{3.799548in}}%
\pgfpathlineto{\pgfqpoint{4.579219in}{3.813333in}}%
\pgfpathlineto{\pgfqpoint{4.567596in}{3.824946in}}%
\pgfpathlineto{\pgfqpoint{4.554194in}{3.838184in}}%
\pgfpathlineto{\pgfqpoint{4.541788in}{3.850667in}}%
\pgfpathlineto{\pgfqpoint{4.527515in}{3.864882in}}%
\pgfpathlineto{\pgfqpoint{4.515449in}{3.876761in}}%
\pgfpathlineto{\pgfqpoint{4.504244in}{3.888000in}}%
\pgfpathlineto{\pgfqpoint{4.496068in}{3.896041in}}%
\pgfpathlineto{\pgfqpoint{4.487434in}{3.904689in}}%
\pgfpathlineto{\pgfqpoint{4.476642in}{3.915281in}}%
\pgfpathlineto{\pgfqpoint{4.466588in}{3.925333in}}%
\pgfpathlineto{\pgfqpoint{4.457216in}{3.934520in}}%
\pgfpathlineto{\pgfqpoint{4.447354in}{3.944368in}}%
\pgfpathlineto{\pgfqpoint{4.428817in}{3.962667in}}%
\pgfpathlineto{\pgfqpoint{4.418303in}{3.972941in}}%
\pgfpathlineto{\pgfqpoint{4.407273in}{3.983919in}}%
\pgfpathlineto{\pgfqpoint{4.390931in}{4.000000in}}%
\pgfpathlineto{\pgfqpoint{4.379327in}{4.011303in}}%
\pgfpathlineto{\pgfqpoint{4.367192in}{4.023343in}}%
\pgfpathlineto{\pgfqpoint{4.359842in}{4.030487in}}%
\pgfpathlineto{\pgfqpoint{4.352929in}{4.037333in}}%
\pgfpathlineto{\pgfqpoint{4.340288in}{4.049607in}}%
\pgfpathlineto{\pgfqpoint{4.327111in}{4.062640in}}%
\pgfpathlineto{\pgfqpoint{4.314810in}{4.074667in}}%
\pgfpathlineto{\pgfqpoint{4.301187in}{4.087853in}}%
\pgfpathlineto{\pgfqpoint{4.287030in}{4.101810in}}%
\pgfpathlineto{\pgfqpoint{4.276574in}{4.112000in}}%
\pgfpathlineto{\pgfqpoint{4.262022in}{4.126040in}}%
\pgfpathlineto{\pgfqpoint{4.246949in}{4.140853in}}%
\pgfpathlineto{\pgfqpoint{4.238220in}{4.149333in}}%
\pgfpathlineto{\pgfqpoint{4.222795in}{4.164167in}}%
\pgfpathlineto{\pgfqpoint{4.206869in}{4.179769in}}%
\pgfpathlineto{\pgfqpoint{4.199745in}{4.186667in}}%
\pgfpathlineto{\pgfqpoint{4.183503in}{4.202236in}}%
\pgfpathlineto{\pgfqpoint{4.166788in}{4.218559in}}%
\pgfpathlineto{\pgfqpoint{4.161151in}{4.224000in}}%
\pgfpathlineto{\pgfqpoint{4.158371in}{4.224000in}}%
\pgfpathlineto{\pgfqpoint{4.166788in}{4.215876in}}%
\pgfpathlineto{\pgfqpoint{4.182097in}{4.200926in}}%
\pgfpathlineto{\pgfqpoint{4.196972in}{4.186667in}}%
\pgfpathlineto{\pgfqpoint{4.206869in}{4.177084in}}%
\pgfpathlineto{\pgfqpoint{4.221389in}{4.162859in}}%
\pgfpathlineto{\pgfqpoint{4.235453in}{4.149333in}}%
\pgfpathlineto{\pgfqpoint{4.246949in}{4.138165in}}%
\pgfpathlineto{\pgfqpoint{4.260618in}{4.124732in}}%
\pgfpathlineto{\pgfqpoint{4.273815in}{4.112000in}}%
\pgfpathlineto{\pgfqpoint{4.287030in}{4.099121in}}%
\pgfpathlineto{\pgfqpoint{4.299784in}{4.086546in}}%
\pgfpathlineto{\pgfqpoint{4.312058in}{4.074667in}}%
\pgfpathlineto{\pgfqpoint{4.327111in}{4.059949in}}%
\pgfpathlineto{\pgfqpoint{4.338887in}{4.048302in}}%
\pgfpathlineto{\pgfqpoint{4.350184in}{4.037333in}}%
\pgfpathlineto{\pgfqpoint{4.358428in}{4.029170in}}%
\pgfpathlineto{\pgfqpoint{4.367192in}{4.020651in}}%
\pgfpathlineto{\pgfqpoint{4.377927in}{4.009999in}}%
\pgfpathlineto{\pgfqpoint{4.388193in}{4.000000in}}%
\pgfpathlineto{\pgfqpoint{4.407273in}{3.981225in}}%
\pgfpathlineto{\pgfqpoint{4.416904in}{3.971638in}}%
\pgfpathlineto{\pgfqpoint{4.426085in}{3.962667in}}%
\pgfpathlineto{\pgfqpoint{4.447354in}{3.941672in}}%
\pgfpathlineto{\pgfqpoint{4.455819in}{3.933219in}}%
\pgfpathlineto{\pgfqpoint{4.463863in}{3.925333in}}%
\pgfpathlineto{\pgfqpoint{4.475232in}{3.913967in}}%
\pgfpathlineto{\pgfqpoint{4.487434in}{3.901991in}}%
\pgfpathlineto{\pgfqpoint{4.494672in}{3.894741in}}%
\pgfpathlineto{\pgfqpoint{4.501527in}{3.888000in}}%
\pgfpathlineto{\pgfqpoint{4.514040in}{3.875449in}}%
\pgfpathlineto{\pgfqpoint{4.527515in}{3.862182in}}%
\pgfpathlineto{\pgfqpoint{4.539077in}{3.850667in}}%
\pgfpathlineto{\pgfqpoint{4.552786in}{3.836872in}}%
\pgfpathlineto{\pgfqpoint{4.567596in}{3.822244in}}%
\pgfpathlineto{\pgfqpoint{4.576515in}{3.813333in}}%
\pgfpathlineto{\pgfqpoint{4.591470in}{3.798238in}}%
\pgfpathlineto{\pgfqpoint{4.607677in}{3.782179in}}%
\pgfpathlineto{\pgfqpoint{4.613841in}{3.776000in}}%
\pgfpathlineto{\pgfqpoint{4.647758in}{3.741984in}}%
\pgfpathlineto{\pgfqpoint{4.649462in}{3.740255in}}%
\pgfpathlineto{\pgfqpoint{4.651057in}{3.738667in}}%
\pgfpathlineto{\pgfqpoint{4.687838in}{3.701660in}}%
\pgfpathlineto{\pgfqpoint{4.688006in}{3.701490in}}%
\pgfpathlineto{\pgfqpoint{4.688162in}{3.701333in}}%
\pgfpathlineto{\pgfqpoint{4.692040in}{3.697420in}}%
\pgfpathlineto{\pgfqpoint{4.725127in}{3.664000in}}%
\pgfpathlineto{\pgfqpoint{4.726460in}{3.662640in}}%
\pgfpathlineto{\pgfqpoint{4.727919in}{3.661178in}}%
\pgfpathlineto{\pgfqpoint{4.761979in}{3.626667in}}%
\pgfpathlineto{\pgfqpoint{4.764847in}{3.623730in}}%
\pgfpathlineto{\pgfqpoint{4.768000in}{3.620562in}}%
\pgfusepath{fill}%
\end{pgfscope}%
\begin{pgfscope}%
\pgfpathrectangle{\pgfqpoint{0.800000in}{0.528000in}}{\pgfqpoint{3.968000in}{3.696000in}}%
\pgfusepath{clip}%
\pgfsetbuttcap%
\pgfsetroundjoin%
\definecolor{currentfill}{rgb}{0.360741,0.785964,0.387814}%
\pgfsetfillcolor{currentfill}%
\pgfsetlinewidth{0.000000pt}%
\definecolor{currentstroke}{rgb}{0.000000,0.000000,0.000000}%
\pgfsetstrokecolor{currentstroke}%
\pgfsetdash{}{0pt}%
\pgfpathmoveto{\pgfqpoint{4.768000in}{3.626040in}}%
\pgfpathlineto{\pgfqpoint{4.767676in}{3.626365in}}%
\pgfpathlineto{\pgfqpoint{4.767382in}{3.626667in}}%
\pgfpathlineto{\pgfqpoint{4.760341in}{3.633801in}}%
\pgfpathlineto{\pgfqpoint{4.730514in}{3.664000in}}%
\pgfpathlineto{\pgfqpoint{4.729264in}{3.665253in}}%
\pgfpathlineto{\pgfqpoint{4.727919in}{3.666625in}}%
\pgfpathlineto{\pgfqpoint{4.693531in}{3.701333in}}%
\pgfpathlineto{\pgfqpoint{4.690784in}{3.704077in}}%
\pgfpathlineto{\pgfqpoint{4.687838in}{3.707075in}}%
\pgfpathlineto{\pgfqpoint{4.656438in}{3.738667in}}%
\pgfpathlineto{\pgfqpoint{4.652243in}{3.742845in}}%
\pgfpathlineto{\pgfqpoint{4.647758in}{3.747395in}}%
\pgfpathlineto{\pgfqpoint{4.619236in}{3.776000in}}%
\pgfpathlineto{\pgfqpoint{4.607677in}{3.787586in}}%
\pgfpathlineto{\pgfqpoint{4.594283in}{3.800858in}}%
\pgfpathlineto{\pgfqpoint{4.581923in}{3.813333in}}%
\pgfpathlineto{\pgfqpoint{4.567596in}{3.827648in}}%
\pgfpathlineto{\pgfqpoint{4.555602in}{3.839495in}}%
\pgfpathlineto{\pgfqpoint{4.544499in}{3.850667in}}%
\pgfpathlineto{\pgfqpoint{4.527515in}{3.867582in}}%
\pgfpathlineto{\pgfqpoint{4.516859in}{3.878074in}}%
\pgfpathlineto{\pgfqpoint{4.506962in}{3.888000in}}%
\pgfpathlineto{\pgfqpoint{4.497463in}{3.897341in}}%
\pgfpathlineto{\pgfqpoint{4.487434in}{3.907387in}}%
\pgfpathlineto{\pgfqpoint{4.478052in}{3.916595in}}%
\pgfpathlineto{\pgfqpoint{4.469312in}{3.925333in}}%
\pgfpathlineto{\pgfqpoint{4.458613in}{3.935821in}}%
\pgfpathlineto{\pgfqpoint{4.447354in}{3.947064in}}%
\pgfpathlineto{\pgfqpoint{4.431548in}{3.962667in}}%
\pgfpathlineto{\pgfqpoint{4.419701in}{3.974243in}}%
\pgfpathlineto{\pgfqpoint{4.407273in}{3.986614in}}%
\pgfpathlineto{\pgfqpoint{4.393669in}{4.000000in}}%
\pgfpathlineto{\pgfqpoint{4.380727in}{4.012607in}}%
\pgfpathlineto{\pgfqpoint{4.367192in}{4.026036in}}%
\pgfpathlineto{\pgfqpoint{4.361257in}{4.031805in}}%
\pgfpathlineto{\pgfqpoint{4.355674in}{4.037333in}}%
\pgfpathlineto{\pgfqpoint{4.341689in}{4.050912in}}%
\pgfpathlineto{\pgfqpoint{4.327111in}{4.065331in}}%
\pgfpathlineto{\pgfqpoint{4.317563in}{4.074667in}}%
\pgfpathlineto{\pgfqpoint{4.302589in}{4.089159in}}%
\pgfpathlineto{\pgfqpoint{4.287030in}{4.104499in}}%
\pgfpathlineto{\pgfqpoint{4.279333in}{4.112000in}}%
\pgfpathlineto{\pgfqpoint{4.263426in}{4.127347in}}%
\pgfpathlineto{\pgfqpoint{4.246949in}{4.143540in}}%
\pgfpathlineto{\pgfqpoint{4.240986in}{4.149333in}}%
\pgfpathlineto{\pgfqpoint{4.224200in}{4.165476in}}%
\pgfpathlineto{\pgfqpoint{4.206869in}{4.182454in}}%
\pgfpathlineto{\pgfqpoint{4.202519in}{4.186667in}}%
\pgfpathlineto{\pgfqpoint{4.184910in}{4.203546in}}%
\pgfpathlineto{\pgfqpoint{4.166788in}{4.221243in}}%
\pgfpathlineto{\pgfqpoint{4.163931in}{4.224000in}}%
\pgfpathlineto{\pgfqpoint{4.161151in}{4.224000in}}%
\pgfpathlineto{\pgfqpoint{4.166788in}{4.218559in}}%
\pgfpathlineto{\pgfqpoint{4.183503in}{4.202236in}}%
\pgfpathlineto{\pgfqpoint{4.199745in}{4.186667in}}%
\pgfpathlineto{\pgfqpoint{4.206869in}{4.179769in}}%
\pgfpathlineto{\pgfqpoint{4.222795in}{4.164167in}}%
\pgfpathlineto{\pgfqpoint{4.238220in}{4.149333in}}%
\pgfpathlineto{\pgfqpoint{4.246949in}{4.140853in}}%
\pgfpathlineto{\pgfqpoint{4.262022in}{4.126040in}}%
\pgfpathlineto{\pgfqpoint{4.276574in}{4.112000in}}%
\pgfpathlineto{\pgfqpoint{4.287030in}{4.101810in}}%
\pgfpathlineto{\pgfqpoint{4.301187in}{4.087853in}}%
\pgfpathlineto{\pgfqpoint{4.314810in}{4.074667in}}%
\pgfpathlineto{\pgfqpoint{4.327111in}{4.062640in}}%
\pgfpathlineto{\pgfqpoint{4.340288in}{4.049607in}}%
\pgfpathlineto{\pgfqpoint{4.352929in}{4.037333in}}%
\pgfpathlineto{\pgfqpoint{4.359842in}{4.030487in}}%
\pgfpathlineto{\pgfqpoint{4.367192in}{4.023343in}}%
\pgfpathlineto{\pgfqpoint{4.379327in}{4.011303in}}%
\pgfpathlineto{\pgfqpoint{4.390931in}{4.000000in}}%
\pgfpathlineto{\pgfqpoint{4.407273in}{3.983919in}}%
\pgfpathlineto{\pgfqpoint{4.418303in}{3.972941in}}%
\pgfpathlineto{\pgfqpoint{4.428817in}{3.962667in}}%
\pgfpathlineto{\pgfqpoint{4.447354in}{3.944368in}}%
\pgfpathlineto{\pgfqpoint{4.457216in}{3.934520in}}%
\pgfpathlineto{\pgfqpoint{4.466588in}{3.925333in}}%
\pgfpathlineto{\pgfqpoint{4.476642in}{3.915281in}}%
\pgfpathlineto{\pgfqpoint{4.487434in}{3.904689in}}%
\pgfpathlineto{\pgfqpoint{4.496068in}{3.896041in}}%
\pgfpathlineto{\pgfqpoint{4.504244in}{3.888000in}}%
\pgfpathlineto{\pgfqpoint{4.515449in}{3.876761in}}%
\pgfpathlineto{\pgfqpoint{4.527515in}{3.864882in}}%
\pgfpathlineto{\pgfqpoint{4.541788in}{3.850667in}}%
\pgfpathlineto{\pgfqpoint{4.554194in}{3.838184in}}%
\pgfpathlineto{\pgfqpoint{4.567596in}{3.824946in}}%
\pgfpathlineto{\pgfqpoint{4.579219in}{3.813333in}}%
\pgfpathlineto{\pgfqpoint{4.592876in}{3.799548in}}%
\pgfpathlineto{\pgfqpoint{4.607677in}{3.784882in}}%
\pgfpathlineto{\pgfqpoint{4.616539in}{3.776000in}}%
\pgfpathlineto{\pgfqpoint{4.647758in}{3.744689in}}%
\pgfpathlineto{\pgfqpoint{4.650853in}{3.741550in}}%
\pgfpathlineto{\pgfqpoint{4.653748in}{3.738667in}}%
\pgfpathlineto{\pgfqpoint{4.687838in}{3.704367in}}%
\pgfpathlineto{\pgfqpoint{4.689395in}{3.702783in}}%
\pgfpathlineto{\pgfqpoint{4.690846in}{3.701333in}}%
\pgfpathlineto{\pgfqpoint{4.726841in}{3.665004in}}%
\pgfpathlineto{\pgfqpoint{4.727835in}{3.664000in}}%
\pgfpathlineto{\pgfqpoint{4.727875in}{3.663959in}}%
\pgfpathlineto{\pgfqpoint{4.727919in}{3.663915in}}%
\pgfpathlineto{\pgfqpoint{4.764680in}{3.626667in}}%
\pgfpathlineto{\pgfqpoint{4.766262in}{3.625048in}}%
\pgfpathlineto{\pgfqpoint{4.768000in}{3.623301in}}%
\pgfusepath{fill}%
\end{pgfscope}%
\begin{pgfscope}%
\pgfpathrectangle{\pgfqpoint{0.800000in}{0.528000in}}{\pgfqpoint{3.968000in}{3.696000in}}%
\pgfusepath{clip}%
\pgfsetbuttcap%
\pgfsetroundjoin%
\definecolor{currentfill}{rgb}{0.369214,0.788888,0.382914}%
\pgfsetfillcolor{currentfill}%
\pgfsetlinewidth{0.000000pt}%
\definecolor{currentstroke}{rgb}{0.000000,0.000000,0.000000}%
\pgfsetstrokecolor{currentstroke}%
\pgfsetdash{}{0pt}%
\pgfpathmoveto{\pgfqpoint{4.768000in}{3.628757in}}%
\pgfpathlineto{\pgfqpoint{4.733191in}{3.664000in}}%
\pgfpathlineto{\pgfqpoint{4.730652in}{3.666545in}}%
\pgfpathlineto{\pgfqpoint{4.727919in}{3.669334in}}%
\pgfpathlineto{\pgfqpoint{4.696215in}{3.701333in}}%
\pgfpathlineto{\pgfqpoint{4.692173in}{3.705371in}}%
\pgfpathlineto{\pgfqpoint{4.687838in}{3.709782in}}%
\pgfpathlineto{\pgfqpoint{4.659129in}{3.738667in}}%
\pgfpathlineto{\pgfqpoint{4.653634in}{3.744140in}}%
\pgfpathlineto{\pgfqpoint{4.647758in}{3.750100in}}%
\pgfpathlineto{\pgfqpoint{4.621934in}{3.776000in}}%
\pgfpathlineto{\pgfqpoint{4.607677in}{3.790289in}}%
\pgfpathlineto{\pgfqpoint{4.595689in}{3.802168in}}%
\pgfpathlineto{\pgfqpoint{4.584627in}{3.813333in}}%
\pgfpathlineto{\pgfqpoint{4.567596in}{3.830350in}}%
\pgfpathlineto{\pgfqpoint{4.557010in}{3.840806in}}%
\pgfpathlineto{\pgfqpoint{4.547210in}{3.850667in}}%
\pgfpathlineto{\pgfqpoint{4.527515in}{3.870281in}}%
\pgfpathlineto{\pgfqpoint{4.518268in}{3.879386in}}%
\pgfpathlineto{\pgfqpoint{4.509680in}{3.888000in}}%
\pgfpathlineto{\pgfqpoint{4.498859in}{3.898641in}}%
\pgfpathlineto{\pgfqpoint{4.487434in}{3.910085in}}%
\pgfpathlineto{\pgfqpoint{4.479463in}{3.917908in}}%
\pgfpathlineto{\pgfqpoint{4.472037in}{3.925333in}}%
\pgfpathlineto{\pgfqpoint{4.460010in}{3.937123in}}%
\pgfpathlineto{\pgfqpoint{4.447354in}{3.949761in}}%
\pgfpathlineto{\pgfqpoint{4.434279in}{3.962667in}}%
\pgfpathlineto{\pgfqpoint{4.421100in}{3.975546in}}%
\pgfpathlineto{\pgfqpoint{4.407273in}{3.989308in}}%
\pgfpathlineto{\pgfqpoint{4.396407in}{4.000000in}}%
\pgfpathlineto{\pgfqpoint{4.382126in}{4.013911in}}%
\pgfpathlineto{\pgfqpoint{4.367192in}{4.028729in}}%
\pgfpathlineto{\pgfqpoint{4.362671in}{4.033123in}}%
\pgfpathlineto{\pgfqpoint{4.358419in}{4.037333in}}%
\pgfpathlineto{\pgfqpoint{4.343090in}{4.052217in}}%
\pgfpathlineto{\pgfqpoint{4.327111in}{4.068022in}}%
\pgfpathlineto{\pgfqpoint{4.320315in}{4.074667in}}%
\pgfpathlineto{\pgfqpoint{4.303992in}{4.090465in}}%
\pgfpathlineto{\pgfqpoint{4.287030in}{4.107188in}}%
\pgfpathlineto{\pgfqpoint{4.282092in}{4.112000in}}%
\pgfpathlineto{\pgfqpoint{4.264830in}{4.128655in}}%
\pgfpathlineto{\pgfqpoint{4.246949in}{4.146227in}}%
\pgfpathlineto{\pgfqpoint{4.243752in}{4.149333in}}%
\pgfpathlineto{\pgfqpoint{4.225605in}{4.166785in}}%
\pgfpathlineto{\pgfqpoint{4.206869in}{4.185140in}}%
\pgfpathlineto{\pgfqpoint{4.205292in}{4.186667in}}%
\pgfpathlineto{\pgfqpoint{4.186316in}{4.204857in}}%
\pgfpathlineto{\pgfqpoint{4.166788in}{4.223926in}}%
\pgfpathlineto{\pgfqpoint{4.166712in}{4.224000in}}%
\pgfpathlineto{\pgfqpoint{4.163931in}{4.224000in}}%
\pgfpathlineto{\pgfqpoint{4.166788in}{4.221243in}}%
\pgfpathlineto{\pgfqpoint{4.184910in}{4.203546in}}%
\pgfpathlineto{\pgfqpoint{4.202519in}{4.186667in}}%
\pgfpathlineto{\pgfqpoint{4.206869in}{4.182454in}}%
\pgfpathlineto{\pgfqpoint{4.224200in}{4.165476in}}%
\pgfpathlineto{\pgfqpoint{4.240986in}{4.149333in}}%
\pgfpathlineto{\pgfqpoint{4.246949in}{4.143540in}}%
\pgfpathlineto{\pgfqpoint{4.263426in}{4.127347in}}%
\pgfpathlineto{\pgfqpoint{4.279333in}{4.112000in}}%
\pgfpathlineto{\pgfqpoint{4.287030in}{4.104499in}}%
\pgfpathlineto{\pgfqpoint{4.302589in}{4.089159in}}%
\pgfpathlineto{\pgfqpoint{4.317563in}{4.074667in}}%
\pgfpathlineto{\pgfqpoint{4.327111in}{4.065331in}}%
\pgfpathlineto{\pgfqpoint{4.341689in}{4.050912in}}%
\pgfpathlineto{\pgfqpoint{4.355674in}{4.037333in}}%
\pgfpathlineto{\pgfqpoint{4.361257in}{4.031805in}}%
\pgfpathlineto{\pgfqpoint{4.367192in}{4.026036in}}%
\pgfpathlineto{\pgfqpoint{4.380727in}{4.012607in}}%
\pgfpathlineto{\pgfqpoint{4.393669in}{4.000000in}}%
\pgfpathlineto{\pgfqpoint{4.407273in}{3.986614in}}%
\pgfpathlineto{\pgfqpoint{4.419701in}{3.974243in}}%
\pgfpathlineto{\pgfqpoint{4.431548in}{3.962667in}}%
\pgfpathlineto{\pgfqpoint{4.447354in}{3.947064in}}%
\pgfpathlineto{\pgfqpoint{4.458613in}{3.935821in}}%
\pgfpathlineto{\pgfqpoint{4.469312in}{3.925333in}}%
\pgfpathlineto{\pgfqpoint{4.478052in}{3.916595in}}%
\pgfpathlineto{\pgfqpoint{4.487434in}{3.907387in}}%
\pgfpathlineto{\pgfqpoint{4.497463in}{3.897341in}}%
\pgfpathlineto{\pgfqpoint{4.506962in}{3.888000in}}%
\pgfpathlineto{\pgfqpoint{4.516859in}{3.878074in}}%
\pgfpathlineto{\pgfqpoint{4.527515in}{3.867582in}}%
\pgfpathlineto{\pgfqpoint{4.544499in}{3.850667in}}%
\pgfpathlineto{\pgfqpoint{4.555602in}{3.839495in}}%
\pgfpathlineto{\pgfqpoint{4.567596in}{3.827648in}}%
\pgfpathlineto{\pgfqpoint{4.581923in}{3.813333in}}%
\pgfpathlineto{\pgfqpoint{4.594283in}{3.800858in}}%
\pgfpathlineto{\pgfqpoint{4.607677in}{3.787586in}}%
\pgfpathlineto{\pgfqpoint{4.619236in}{3.776000in}}%
\pgfpathlineto{\pgfqpoint{4.647758in}{3.747395in}}%
\pgfpathlineto{\pgfqpoint{4.652243in}{3.742845in}}%
\pgfpathlineto{\pgfqpoint{4.656438in}{3.738667in}}%
\pgfpathlineto{\pgfqpoint{4.687838in}{3.707075in}}%
\pgfpathlineto{\pgfqpoint{4.690784in}{3.704077in}}%
\pgfpathlineto{\pgfqpoint{4.693531in}{3.701333in}}%
\pgfpathlineto{\pgfqpoint{4.727919in}{3.666625in}}%
\pgfpathlineto{\pgfqpoint{4.729264in}{3.665253in}}%
\pgfpathlineto{\pgfqpoint{4.730514in}{3.664000in}}%
\pgfpathlineto{\pgfqpoint{4.760341in}{3.633801in}}%
\pgfpathlineto{\pgfqpoint{4.767382in}{3.626667in}}%
\pgfpathlineto{\pgfqpoint{4.767676in}{3.626365in}}%
\pgfpathlineto{\pgfqpoint{4.768000in}{3.626040in}}%
\pgfpathlineto{\pgfqpoint{4.768000in}{3.626667in}}%
\pgfusepath{fill}%
\end{pgfscope}%
\begin{pgfscope}%
\pgfpathrectangle{\pgfqpoint{0.800000in}{0.528000in}}{\pgfqpoint{3.968000in}{3.696000in}}%
\pgfusepath{clip}%
\pgfsetbuttcap%
\pgfsetroundjoin%
\definecolor{currentfill}{rgb}{0.369214,0.788888,0.382914}%
\pgfsetfillcolor{currentfill}%
\pgfsetlinewidth{0.000000pt}%
\definecolor{currentstroke}{rgb}{0.000000,0.000000,0.000000}%
\pgfsetstrokecolor{currentstroke}%
\pgfsetdash{}{0pt}%
\pgfpathmoveto{\pgfqpoint{4.768000in}{3.631468in}}%
\pgfpathlineto{\pgfqpoint{4.735869in}{3.664000in}}%
\pgfpathlineto{\pgfqpoint{4.732039in}{3.667838in}}%
\pgfpathlineto{\pgfqpoint{4.727919in}{3.672043in}}%
\pgfpathlineto{\pgfqpoint{4.698899in}{3.701333in}}%
\pgfpathlineto{\pgfqpoint{4.693562in}{3.706665in}}%
\pgfpathlineto{\pgfqpoint{4.687838in}{3.712489in}}%
\pgfpathlineto{\pgfqpoint{4.661820in}{3.738667in}}%
\pgfpathlineto{\pgfqpoint{4.655024in}{3.745435in}}%
\pgfpathlineto{\pgfqpoint{4.647758in}{3.752805in}}%
\pgfpathlineto{\pgfqpoint{4.624631in}{3.776000in}}%
\pgfpathlineto{\pgfqpoint{4.607677in}{3.792993in}}%
\pgfpathlineto{\pgfqpoint{4.597096in}{3.803478in}}%
\pgfpathlineto{\pgfqpoint{4.587332in}{3.813333in}}%
\pgfpathlineto{\pgfqpoint{4.567596in}{3.833051in}}%
\pgfpathlineto{\pgfqpoint{4.558417in}{3.842117in}}%
\pgfpathlineto{\pgfqpoint{4.549921in}{3.850667in}}%
\pgfpathlineto{\pgfqpoint{4.527515in}{3.872981in}}%
\pgfpathlineto{\pgfqpoint{4.519677in}{3.880699in}}%
\pgfpathlineto{\pgfqpoint{4.512398in}{3.888000in}}%
\pgfpathlineto{\pgfqpoint{4.500255in}{3.899942in}}%
\pgfpathlineto{\pgfqpoint{4.487434in}{3.912783in}}%
\pgfpathlineto{\pgfqpoint{4.480873in}{3.919222in}}%
\pgfpathlineto{\pgfqpoint{4.474761in}{3.925333in}}%
\pgfpathlineto{\pgfqpoint{4.461408in}{3.938424in}}%
\pgfpathlineto{\pgfqpoint{4.447354in}{3.952457in}}%
\pgfpathlineto{\pgfqpoint{4.437011in}{3.962667in}}%
\pgfpathlineto{\pgfqpoint{4.422498in}{3.976848in}}%
\pgfpathlineto{\pgfqpoint{4.407273in}{3.992003in}}%
\pgfpathlineto{\pgfqpoint{4.399145in}{4.000000in}}%
\pgfpathlineto{\pgfqpoint{4.383526in}{4.015215in}}%
\pgfpathlineto{\pgfqpoint{4.367192in}{4.031421in}}%
\pgfpathlineto{\pgfqpoint{4.364086in}{4.034440in}}%
\pgfpathlineto{\pgfqpoint{4.361164in}{4.037333in}}%
\pgfpathlineto{\pgfqpoint{4.344492in}{4.053522in}}%
\pgfpathlineto{\pgfqpoint{4.327111in}{4.070712in}}%
\pgfpathlineto{\pgfqpoint{4.323067in}{4.074667in}}%
\pgfpathlineto{\pgfqpoint{4.305394in}{4.091772in}}%
\pgfpathlineto{\pgfqpoint{4.287030in}{4.109877in}}%
\pgfpathlineto{\pgfqpoint{4.284852in}{4.112000in}}%
\pgfpathlineto{\pgfqpoint{4.266234in}{4.129962in}}%
\pgfpathlineto{\pgfqpoint{4.246949in}{4.148914in}}%
\pgfpathlineto{\pgfqpoint{4.246518in}{4.149333in}}%
\pgfpathlineto{\pgfqpoint{4.227010in}{4.168094in}}%
\pgfpathlineto{\pgfqpoint{4.208051in}{4.186667in}}%
\pgfpathlineto{\pgfqpoint{4.206869in}{4.187813in}}%
\pgfpathlineto{\pgfqpoint{4.187723in}{4.206167in}}%
\pgfpathlineto{\pgfqpoint{4.169461in}{4.224000in}}%
\pgfpathlineto{\pgfqpoint{4.166788in}{4.224000in}}%
\pgfpathlineto{\pgfqpoint{4.166712in}{4.224000in}}%
\pgfpathlineto{\pgfqpoint{4.166788in}{4.223926in}}%
\pgfpathlineto{\pgfqpoint{4.186316in}{4.204857in}}%
\pgfpathlineto{\pgfqpoint{4.205292in}{4.186667in}}%
\pgfpathlineto{\pgfqpoint{4.206869in}{4.185140in}}%
\pgfpathlineto{\pgfqpoint{4.225605in}{4.166785in}}%
\pgfpathlineto{\pgfqpoint{4.243752in}{4.149333in}}%
\pgfpathlineto{\pgfqpoint{4.246949in}{4.146227in}}%
\pgfpathlineto{\pgfqpoint{4.264830in}{4.128655in}}%
\pgfpathlineto{\pgfqpoint{4.282092in}{4.112000in}}%
\pgfpathlineto{\pgfqpoint{4.287030in}{4.107188in}}%
\pgfpathlineto{\pgfqpoint{4.303992in}{4.090465in}}%
\pgfpathlineto{\pgfqpoint{4.320315in}{4.074667in}}%
\pgfpathlineto{\pgfqpoint{4.327111in}{4.068022in}}%
\pgfpathlineto{\pgfqpoint{4.343090in}{4.052217in}}%
\pgfpathlineto{\pgfqpoint{4.358419in}{4.037333in}}%
\pgfpathlineto{\pgfqpoint{4.362671in}{4.033123in}}%
\pgfpathlineto{\pgfqpoint{4.367192in}{4.028729in}}%
\pgfpathlineto{\pgfqpoint{4.382126in}{4.013911in}}%
\pgfpathlineto{\pgfqpoint{4.396407in}{4.000000in}}%
\pgfpathlineto{\pgfqpoint{4.407273in}{3.989308in}}%
\pgfpathlineto{\pgfqpoint{4.421100in}{3.975546in}}%
\pgfpathlineto{\pgfqpoint{4.434279in}{3.962667in}}%
\pgfpathlineto{\pgfqpoint{4.447354in}{3.949761in}}%
\pgfpathlineto{\pgfqpoint{4.460010in}{3.937123in}}%
\pgfpathlineto{\pgfqpoint{4.472037in}{3.925333in}}%
\pgfpathlineto{\pgfqpoint{4.479463in}{3.917908in}}%
\pgfpathlineto{\pgfqpoint{4.487434in}{3.910085in}}%
\pgfpathlineto{\pgfqpoint{4.498859in}{3.898641in}}%
\pgfpathlineto{\pgfqpoint{4.509680in}{3.888000in}}%
\pgfpathlineto{\pgfqpoint{4.518268in}{3.879386in}}%
\pgfpathlineto{\pgfqpoint{4.527515in}{3.870281in}}%
\pgfpathlineto{\pgfqpoint{4.547210in}{3.850667in}}%
\pgfpathlineto{\pgfqpoint{4.557010in}{3.840806in}}%
\pgfpathlineto{\pgfqpoint{4.567596in}{3.830350in}}%
\pgfpathlineto{\pgfqpoint{4.584627in}{3.813333in}}%
\pgfpathlineto{\pgfqpoint{4.595689in}{3.802168in}}%
\pgfpathlineto{\pgfqpoint{4.607677in}{3.790289in}}%
\pgfpathlineto{\pgfqpoint{4.621934in}{3.776000in}}%
\pgfpathlineto{\pgfqpoint{4.647758in}{3.750100in}}%
\pgfpathlineto{\pgfqpoint{4.653634in}{3.744140in}}%
\pgfpathlineto{\pgfqpoint{4.659129in}{3.738667in}}%
\pgfpathlineto{\pgfqpoint{4.687838in}{3.709782in}}%
\pgfpathlineto{\pgfqpoint{4.692173in}{3.705371in}}%
\pgfpathlineto{\pgfqpoint{4.696215in}{3.701333in}}%
\pgfpathlineto{\pgfqpoint{4.727919in}{3.669334in}}%
\pgfpathlineto{\pgfqpoint{4.730652in}{3.666545in}}%
\pgfpathlineto{\pgfqpoint{4.733191in}{3.664000in}}%
\pgfpathlineto{\pgfqpoint{4.768000in}{3.628757in}}%
\pgfusepath{fill}%
\end{pgfscope}%
\begin{pgfscope}%
\pgfpathrectangle{\pgfqpoint{0.800000in}{0.528000in}}{\pgfqpoint{3.968000in}{3.696000in}}%
\pgfusepath{clip}%
\pgfsetbuttcap%
\pgfsetroundjoin%
\definecolor{currentfill}{rgb}{0.369214,0.788888,0.382914}%
\pgfsetfillcolor{currentfill}%
\pgfsetlinewidth{0.000000pt}%
\definecolor{currentstroke}{rgb}{0.000000,0.000000,0.000000}%
\pgfsetstrokecolor{currentstroke}%
\pgfsetdash{}{0pt}%
\pgfpathmoveto{\pgfqpoint{4.768000in}{3.634179in}}%
\pgfpathlineto{\pgfqpoint{4.738546in}{3.664000in}}%
\pgfpathlineto{\pgfqpoint{4.733427in}{3.669130in}}%
\pgfpathlineto{\pgfqpoint{4.727919in}{3.674752in}}%
\pgfpathlineto{\pgfqpoint{4.701583in}{3.701333in}}%
\pgfpathlineto{\pgfqpoint{4.694951in}{3.707959in}}%
\pgfpathlineto{\pgfqpoint{4.687838in}{3.715196in}}%
\pgfpathlineto{\pgfqpoint{4.664511in}{3.738667in}}%
\pgfpathlineto{\pgfqpoint{4.656414in}{3.746730in}}%
\pgfpathlineto{\pgfqpoint{4.647758in}{3.755511in}}%
\pgfpathlineto{\pgfqpoint{4.627328in}{3.776000in}}%
\pgfpathlineto{\pgfqpoint{4.607677in}{3.795696in}}%
\pgfpathlineto{\pgfqpoint{4.598502in}{3.804787in}}%
\pgfpathlineto{\pgfqpoint{4.590036in}{3.813333in}}%
\pgfpathlineto{\pgfqpoint{4.567596in}{3.835753in}}%
\pgfpathlineto{\pgfqpoint{4.559825in}{3.843429in}}%
\pgfpathlineto{\pgfqpoint{4.552632in}{3.850667in}}%
\pgfpathlineto{\pgfqpoint{4.527515in}{3.875681in}}%
\pgfpathlineto{\pgfqpoint{4.521086in}{3.882011in}}%
\pgfpathlineto{\pgfqpoint{4.515115in}{3.888000in}}%
\pgfpathlineto{\pgfqpoint{4.501650in}{3.901242in}}%
\pgfpathlineto{\pgfqpoint{4.487434in}{3.915481in}}%
\pgfpathlineto{\pgfqpoint{4.482284in}{3.920536in}}%
\pgfpathlineto{\pgfqpoint{4.477486in}{3.925333in}}%
\pgfpathlineto{\pgfqpoint{4.462805in}{3.939725in}}%
\pgfpathlineto{\pgfqpoint{4.447354in}{3.955153in}}%
\pgfpathlineto{\pgfqpoint{4.439742in}{3.962667in}}%
\pgfpathlineto{\pgfqpoint{4.423896in}{3.978151in}}%
\pgfpathlineto{\pgfqpoint{4.407273in}{3.994697in}}%
\pgfpathlineto{\pgfqpoint{4.401884in}{4.000000in}}%
\pgfpathlineto{\pgfqpoint{4.384926in}{4.016518in}}%
\pgfpathlineto{\pgfqpoint{4.367192in}{4.034114in}}%
\pgfpathlineto{\pgfqpoint{4.365501in}{4.035758in}}%
\pgfpathlineto{\pgfqpoint{4.363910in}{4.037333in}}%
\pgfpathlineto{\pgfqpoint{4.345893in}{4.054827in}}%
\pgfpathlineto{\pgfqpoint{4.327111in}{4.073403in}}%
\pgfpathlineto{\pgfqpoint{4.325819in}{4.074667in}}%
\pgfpathlineto{\pgfqpoint{4.306797in}{4.093078in}}%
\pgfpathlineto{\pgfqpoint{4.287604in}{4.112000in}}%
\pgfpathlineto{\pgfqpoint{4.287030in}{4.112560in}}%
\pgfpathlineto{\pgfqpoint{4.267637in}{4.131270in}}%
\pgfpathlineto{\pgfqpoint{4.249257in}{4.149333in}}%
\pgfpathlineto{\pgfqpoint{4.248123in}{4.150426in}}%
\pgfpathlineto{\pgfqpoint{4.246949in}{4.151578in}}%
\pgfpathlineto{\pgfqpoint{4.228415in}{4.169403in}}%
\pgfpathlineto{\pgfqpoint{4.210792in}{4.186667in}}%
\pgfpathlineto{\pgfqpoint{4.206869in}{4.190472in}}%
\pgfpathlineto{\pgfqpoint{4.189129in}{4.207477in}}%
\pgfpathlineto{\pgfqpoint{4.172209in}{4.224000in}}%
\pgfpathlineto{\pgfqpoint{4.169461in}{4.224000in}}%
\pgfpathlineto{\pgfqpoint{4.187723in}{4.206167in}}%
\pgfpathlineto{\pgfqpoint{4.206869in}{4.187813in}}%
\pgfpathlineto{\pgfqpoint{4.208051in}{4.186667in}}%
\pgfpathlineto{\pgfqpoint{4.227010in}{4.168094in}}%
\pgfpathlineto{\pgfqpoint{4.246518in}{4.149333in}}%
\pgfpathlineto{\pgfqpoint{4.246949in}{4.148914in}}%
\pgfpathlineto{\pgfqpoint{4.266234in}{4.129962in}}%
\pgfpathlineto{\pgfqpoint{4.284852in}{4.112000in}}%
\pgfpathlineto{\pgfqpoint{4.287030in}{4.109877in}}%
\pgfpathlineto{\pgfqpoint{4.305394in}{4.091772in}}%
\pgfpathlineto{\pgfqpoint{4.323067in}{4.074667in}}%
\pgfpathlineto{\pgfqpoint{4.327111in}{4.070712in}}%
\pgfpathlineto{\pgfqpoint{4.344492in}{4.053522in}}%
\pgfpathlineto{\pgfqpoint{4.361164in}{4.037333in}}%
\pgfpathlineto{\pgfqpoint{4.364086in}{4.034440in}}%
\pgfpathlineto{\pgfqpoint{4.367192in}{4.031421in}}%
\pgfpathlineto{\pgfqpoint{4.383526in}{4.015215in}}%
\pgfpathlineto{\pgfqpoint{4.399145in}{4.000000in}}%
\pgfpathlineto{\pgfqpoint{4.407273in}{3.992003in}}%
\pgfpathlineto{\pgfqpoint{4.422498in}{3.976848in}}%
\pgfpathlineto{\pgfqpoint{4.437011in}{3.962667in}}%
\pgfpathlineto{\pgfqpoint{4.447354in}{3.952457in}}%
\pgfpathlineto{\pgfqpoint{4.461408in}{3.938424in}}%
\pgfpathlineto{\pgfqpoint{4.474761in}{3.925333in}}%
\pgfpathlineto{\pgfqpoint{4.480873in}{3.919222in}}%
\pgfpathlineto{\pgfqpoint{4.487434in}{3.912783in}}%
\pgfpathlineto{\pgfqpoint{4.500255in}{3.899942in}}%
\pgfpathlineto{\pgfqpoint{4.512398in}{3.888000in}}%
\pgfpathlineto{\pgfqpoint{4.519677in}{3.880699in}}%
\pgfpathlineto{\pgfqpoint{4.527515in}{3.872981in}}%
\pgfpathlineto{\pgfqpoint{4.549921in}{3.850667in}}%
\pgfpathlineto{\pgfqpoint{4.558417in}{3.842117in}}%
\pgfpathlineto{\pgfqpoint{4.567596in}{3.833051in}}%
\pgfpathlineto{\pgfqpoint{4.587332in}{3.813333in}}%
\pgfpathlineto{\pgfqpoint{4.597096in}{3.803478in}}%
\pgfpathlineto{\pgfqpoint{4.607677in}{3.792993in}}%
\pgfpathlineto{\pgfqpoint{4.624631in}{3.776000in}}%
\pgfpathlineto{\pgfqpoint{4.647758in}{3.752805in}}%
\pgfpathlineto{\pgfqpoint{4.655024in}{3.745435in}}%
\pgfpathlineto{\pgfqpoint{4.661820in}{3.738667in}}%
\pgfpathlineto{\pgfqpoint{4.687838in}{3.712489in}}%
\pgfpathlineto{\pgfqpoint{4.693562in}{3.706665in}}%
\pgfpathlineto{\pgfqpoint{4.698899in}{3.701333in}}%
\pgfpathlineto{\pgfqpoint{4.727919in}{3.672043in}}%
\pgfpathlineto{\pgfqpoint{4.732039in}{3.667838in}}%
\pgfpathlineto{\pgfqpoint{4.735869in}{3.664000in}}%
\pgfpathlineto{\pgfqpoint{4.768000in}{3.631468in}}%
\pgfusepath{fill}%
\end{pgfscope}%
\begin{pgfscope}%
\pgfpathrectangle{\pgfqpoint{0.800000in}{0.528000in}}{\pgfqpoint{3.968000in}{3.696000in}}%
\pgfusepath{clip}%
\pgfsetbuttcap%
\pgfsetroundjoin%
\definecolor{currentfill}{rgb}{0.377779,0.791781,0.377939}%
\pgfsetfillcolor{currentfill}%
\pgfsetlinewidth{0.000000pt}%
\definecolor{currentstroke}{rgb}{0.000000,0.000000,0.000000}%
\pgfsetstrokecolor{currentstroke}%
\pgfsetdash{}{0pt}%
\pgfpathmoveto{\pgfqpoint{4.768000in}{3.636890in}}%
\pgfpathlineto{\pgfqpoint{4.741224in}{3.664000in}}%
\pgfpathlineto{\pgfqpoint{4.734815in}{3.670423in}}%
\pgfpathlineto{\pgfqpoint{4.727919in}{3.677461in}}%
\pgfpathlineto{\pgfqpoint{4.704267in}{3.701333in}}%
\pgfpathlineto{\pgfqpoint{4.696340in}{3.709253in}}%
\pgfpathlineto{\pgfqpoint{4.687838in}{3.717903in}}%
\pgfpathlineto{\pgfqpoint{4.667201in}{3.738667in}}%
\pgfpathlineto{\pgfqpoint{4.657805in}{3.748025in}}%
\pgfpathlineto{\pgfqpoint{4.647758in}{3.758216in}}%
\pgfpathlineto{\pgfqpoint{4.630026in}{3.776000in}}%
\pgfpathlineto{\pgfqpoint{4.607677in}{3.798400in}}%
\pgfpathlineto{\pgfqpoint{4.599908in}{3.806097in}}%
\pgfpathlineto{\pgfqpoint{4.592740in}{3.813333in}}%
\pgfpathlineto{\pgfqpoint{4.567596in}{3.838455in}}%
\pgfpathlineto{\pgfqpoint{4.561233in}{3.844740in}}%
\pgfpathlineto{\pgfqpoint{4.555343in}{3.850667in}}%
\pgfpathlineto{\pgfqpoint{4.527515in}{3.878381in}}%
\pgfpathlineto{\pgfqpoint{4.522495in}{3.883324in}}%
\pgfpathlineto{\pgfqpoint{4.517833in}{3.888000in}}%
\pgfpathlineto{\pgfqpoint{4.503046in}{3.902542in}}%
\pgfpathlineto{\pgfqpoint{4.487434in}{3.918179in}}%
\pgfpathlineto{\pgfqpoint{4.483694in}{3.921850in}}%
\pgfpathlineto{\pgfqpoint{4.480210in}{3.925333in}}%
\pgfpathlineto{\pgfqpoint{4.464202in}{3.941027in}}%
\pgfpathlineto{\pgfqpoint{4.447354in}{3.957849in}}%
\pgfpathlineto{\pgfqpoint{4.442473in}{3.962667in}}%
\pgfpathlineto{\pgfqpoint{4.425295in}{3.979453in}}%
\pgfpathlineto{\pgfqpoint{4.407273in}{3.997392in}}%
\pgfpathlineto{\pgfqpoint{4.404622in}{4.000000in}}%
\pgfpathlineto{\pgfqpoint{4.386326in}{4.017822in}}%
\pgfpathlineto{\pgfqpoint{4.367192in}{4.036806in}}%
\pgfpathlineto{\pgfqpoint{4.366915in}{4.037075in}}%
\pgfpathlineto{\pgfqpoint{4.366655in}{4.037333in}}%
\pgfpathlineto{\pgfqpoint{4.347294in}{4.056132in}}%
\pgfpathlineto{\pgfqpoint{4.328554in}{4.074667in}}%
\pgfpathlineto{\pgfqpoint{4.327111in}{4.076080in}}%
\pgfpathlineto{\pgfqpoint{4.308199in}{4.094384in}}%
\pgfpathlineto{\pgfqpoint{4.290331in}{4.112000in}}%
\pgfpathlineto{\pgfqpoint{4.287030in}{4.115222in}}%
\pgfpathlineto{\pgfqpoint{4.269041in}{4.132577in}}%
\pgfpathlineto{\pgfqpoint{4.251991in}{4.149333in}}%
\pgfpathlineto{\pgfqpoint{4.249514in}{4.151722in}}%
\pgfpathlineto{\pgfqpoint{4.246949in}{4.154239in}}%
\pgfpathlineto{\pgfqpoint{4.229820in}{4.170712in}}%
\pgfpathlineto{\pgfqpoint{4.213534in}{4.186667in}}%
\pgfpathlineto{\pgfqpoint{4.206869in}{4.193130in}}%
\pgfpathlineto{\pgfqpoint{4.190536in}{4.208787in}}%
\pgfpathlineto{\pgfqpoint{4.174957in}{4.224000in}}%
\pgfpathlineto{\pgfqpoint{4.172209in}{4.224000in}}%
\pgfpathlineto{\pgfqpoint{4.189129in}{4.207477in}}%
\pgfpathlineto{\pgfqpoint{4.206869in}{4.190472in}}%
\pgfpathlineto{\pgfqpoint{4.210792in}{4.186667in}}%
\pgfpathlineto{\pgfqpoint{4.228415in}{4.169403in}}%
\pgfpathlineto{\pgfqpoint{4.246949in}{4.151578in}}%
\pgfpathlineto{\pgfqpoint{4.248123in}{4.150426in}}%
\pgfpathlineto{\pgfqpoint{4.249257in}{4.149333in}}%
\pgfpathlineto{\pgfqpoint{4.267637in}{4.131270in}}%
\pgfpathlineto{\pgfqpoint{4.287030in}{4.112560in}}%
\pgfpathlineto{\pgfqpoint{4.287604in}{4.112000in}}%
\pgfpathlineto{\pgfqpoint{4.306797in}{4.093078in}}%
\pgfpathlineto{\pgfqpoint{4.325819in}{4.074667in}}%
\pgfpathlineto{\pgfqpoint{4.327111in}{4.073403in}}%
\pgfpathlineto{\pgfqpoint{4.345893in}{4.054827in}}%
\pgfpathlineto{\pgfqpoint{4.363910in}{4.037333in}}%
\pgfpathlineto{\pgfqpoint{4.365501in}{4.035758in}}%
\pgfpathlineto{\pgfqpoint{4.367192in}{4.034114in}}%
\pgfpathlineto{\pgfqpoint{4.384926in}{4.016518in}}%
\pgfpathlineto{\pgfqpoint{4.401884in}{4.000000in}}%
\pgfpathlineto{\pgfqpoint{4.407273in}{3.994697in}}%
\pgfpathlineto{\pgfqpoint{4.423896in}{3.978151in}}%
\pgfpathlineto{\pgfqpoint{4.439742in}{3.962667in}}%
\pgfpathlineto{\pgfqpoint{4.447354in}{3.955153in}}%
\pgfpathlineto{\pgfqpoint{4.462805in}{3.939725in}}%
\pgfpathlineto{\pgfqpoint{4.477486in}{3.925333in}}%
\pgfpathlineto{\pgfqpoint{4.482284in}{3.920536in}}%
\pgfpathlineto{\pgfqpoint{4.487434in}{3.915481in}}%
\pgfpathlineto{\pgfqpoint{4.501650in}{3.901242in}}%
\pgfpathlineto{\pgfqpoint{4.515115in}{3.888000in}}%
\pgfpathlineto{\pgfqpoint{4.521086in}{3.882011in}}%
\pgfpathlineto{\pgfqpoint{4.527515in}{3.875681in}}%
\pgfpathlineto{\pgfqpoint{4.552632in}{3.850667in}}%
\pgfpathlineto{\pgfqpoint{4.559825in}{3.843429in}}%
\pgfpathlineto{\pgfqpoint{4.567596in}{3.835753in}}%
\pgfpathlineto{\pgfqpoint{4.590036in}{3.813333in}}%
\pgfpathlineto{\pgfqpoint{4.598502in}{3.804787in}}%
\pgfpathlineto{\pgfqpoint{4.607677in}{3.795696in}}%
\pgfpathlineto{\pgfqpoint{4.627328in}{3.776000in}}%
\pgfpathlineto{\pgfqpoint{4.647758in}{3.755511in}}%
\pgfpathlineto{\pgfqpoint{4.656414in}{3.746730in}}%
\pgfpathlineto{\pgfqpoint{4.664511in}{3.738667in}}%
\pgfpathlineto{\pgfqpoint{4.687838in}{3.715196in}}%
\pgfpathlineto{\pgfqpoint{4.694951in}{3.707959in}}%
\pgfpathlineto{\pgfqpoint{4.701583in}{3.701333in}}%
\pgfpathlineto{\pgfqpoint{4.727919in}{3.674752in}}%
\pgfpathlineto{\pgfqpoint{4.733427in}{3.669130in}}%
\pgfpathlineto{\pgfqpoint{4.738546in}{3.664000in}}%
\pgfpathlineto{\pgfqpoint{4.768000in}{3.634179in}}%
\pgfusepath{fill}%
\end{pgfscope}%
\begin{pgfscope}%
\pgfpathrectangle{\pgfqpoint{0.800000in}{0.528000in}}{\pgfqpoint{3.968000in}{3.696000in}}%
\pgfusepath{clip}%
\pgfsetbuttcap%
\pgfsetroundjoin%
\definecolor{currentfill}{rgb}{0.377779,0.791781,0.377939}%
\pgfsetfillcolor{currentfill}%
\pgfsetlinewidth{0.000000pt}%
\definecolor{currentstroke}{rgb}{0.000000,0.000000,0.000000}%
\pgfsetstrokecolor{currentstroke}%
\pgfsetdash{}{0pt}%
\pgfpathmoveto{\pgfqpoint{4.768000in}{3.639600in}}%
\pgfpathlineto{\pgfqpoint{4.743901in}{3.664000in}}%
\pgfpathlineto{\pgfqpoint{4.736203in}{3.671716in}}%
\pgfpathlineto{\pgfqpoint{4.727919in}{3.680170in}}%
\pgfpathlineto{\pgfqpoint{4.706951in}{3.701333in}}%
\pgfpathlineto{\pgfqpoint{4.697729in}{3.710546in}}%
\pgfpathlineto{\pgfqpoint{4.687838in}{3.720611in}}%
\pgfpathlineto{\pgfqpoint{4.669892in}{3.738667in}}%
\pgfpathlineto{\pgfqpoint{4.659195in}{3.749320in}}%
\pgfpathlineto{\pgfqpoint{4.647758in}{3.760921in}}%
\pgfpathlineto{\pgfqpoint{4.632723in}{3.776000in}}%
\pgfpathlineto{\pgfqpoint{4.607677in}{3.801103in}}%
\pgfpathlineto{\pgfqpoint{4.601315in}{3.807407in}}%
\pgfpathlineto{\pgfqpoint{4.595444in}{3.813333in}}%
\pgfpathlineto{\pgfqpoint{4.567596in}{3.841156in}}%
\pgfpathlineto{\pgfqpoint{4.562641in}{3.846051in}}%
\pgfpathlineto{\pgfqpoint{4.558053in}{3.850667in}}%
\pgfpathlineto{\pgfqpoint{4.527515in}{3.881081in}}%
\pgfpathlineto{\pgfqpoint{4.523904in}{3.884636in}}%
\pgfpathlineto{\pgfqpoint{4.520551in}{3.888000in}}%
\pgfpathlineto{\pgfqpoint{4.504442in}{3.903842in}}%
\pgfpathlineto{\pgfqpoint{4.487434in}{3.920877in}}%
\pgfpathlineto{\pgfqpoint{4.485105in}{3.923163in}}%
\pgfpathlineto{\pgfqpoint{4.482935in}{3.925333in}}%
\pgfpathlineto{\pgfqpoint{4.465599in}{3.942328in}}%
\pgfpathlineto{\pgfqpoint{4.447354in}{3.960545in}}%
\pgfpathlineto{\pgfqpoint{4.445205in}{3.962667in}}%
\pgfpathlineto{\pgfqpoint{4.426693in}{3.980756in}}%
\pgfpathlineto{\pgfqpoint{4.407359in}{4.000000in}}%
\pgfpathlineto{\pgfqpoint{4.407273in}{4.000085in}}%
\pgfpathlineto{\pgfqpoint{4.387725in}{4.019126in}}%
\pgfpathlineto{\pgfqpoint{4.369375in}{4.037333in}}%
\pgfpathlineto{\pgfqpoint{4.368307in}{4.038372in}}%
\pgfpathlineto{\pgfqpoint{4.367192in}{4.039477in}}%
\pgfpathlineto{\pgfqpoint{4.348695in}{4.057438in}}%
\pgfpathlineto{\pgfqpoint{4.331275in}{4.074667in}}%
\pgfpathlineto{\pgfqpoint{4.327111in}{4.078743in}}%
\pgfpathlineto{\pgfqpoint{4.309601in}{4.095691in}}%
\pgfpathlineto{\pgfqpoint{4.293059in}{4.112000in}}%
\pgfpathlineto{\pgfqpoint{4.287030in}{4.117884in}}%
\pgfpathlineto{\pgfqpoint{4.270445in}{4.133885in}}%
\pgfpathlineto{\pgfqpoint{4.254726in}{4.149333in}}%
\pgfpathlineto{\pgfqpoint{4.250904in}{4.153017in}}%
\pgfpathlineto{\pgfqpoint{4.246949in}{4.156899in}}%
\pgfpathlineto{\pgfqpoint{4.231225in}{4.172020in}}%
\pgfpathlineto{\pgfqpoint{4.216275in}{4.186667in}}%
\pgfpathlineto{\pgfqpoint{4.206869in}{4.195789in}}%
\pgfpathlineto{\pgfqpoint{4.191943in}{4.210097in}}%
\pgfpathlineto{\pgfqpoint{4.177705in}{4.224000in}}%
\pgfpathlineto{\pgfqpoint{4.174957in}{4.224000in}}%
\pgfpathlineto{\pgfqpoint{4.190536in}{4.208787in}}%
\pgfpathlineto{\pgfqpoint{4.206869in}{4.193130in}}%
\pgfpathlineto{\pgfqpoint{4.213534in}{4.186667in}}%
\pgfpathlineto{\pgfqpoint{4.229820in}{4.170712in}}%
\pgfpathlineto{\pgfqpoint{4.246949in}{4.154239in}}%
\pgfpathlineto{\pgfqpoint{4.249514in}{4.151722in}}%
\pgfpathlineto{\pgfqpoint{4.251991in}{4.149333in}}%
\pgfpathlineto{\pgfqpoint{4.269041in}{4.132577in}}%
\pgfpathlineto{\pgfqpoint{4.287030in}{4.115222in}}%
\pgfpathlineto{\pgfqpoint{4.290331in}{4.112000in}}%
\pgfpathlineto{\pgfqpoint{4.308199in}{4.094384in}}%
\pgfpathlineto{\pgfqpoint{4.327111in}{4.076080in}}%
\pgfpathlineto{\pgfqpoint{4.328554in}{4.074667in}}%
\pgfpathlineto{\pgfqpoint{4.347294in}{4.056132in}}%
\pgfpathlineto{\pgfqpoint{4.366655in}{4.037333in}}%
\pgfpathlineto{\pgfqpoint{4.366915in}{4.037075in}}%
\pgfpathlineto{\pgfqpoint{4.367192in}{4.036806in}}%
\pgfpathlineto{\pgfqpoint{4.386326in}{4.017822in}}%
\pgfpathlineto{\pgfqpoint{4.404622in}{4.000000in}}%
\pgfpathlineto{\pgfqpoint{4.407273in}{3.997392in}}%
\pgfpathlineto{\pgfqpoint{4.425295in}{3.979453in}}%
\pgfpathlineto{\pgfqpoint{4.442473in}{3.962667in}}%
\pgfpathlineto{\pgfqpoint{4.447354in}{3.957849in}}%
\pgfpathlineto{\pgfqpoint{4.464202in}{3.941027in}}%
\pgfpathlineto{\pgfqpoint{4.480210in}{3.925333in}}%
\pgfpathlineto{\pgfqpoint{4.483694in}{3.921850in}}%
\pgfpathlineto{\pgfqpoint{4.487434in}{3.918179in}}%
\pgfpathlineto{\pgfqpoint{4.503046in}{3.902542in}}%
\pgfpathlineto{\pgfqpoint{4.517833in}{3.888000in}}%
\pgfpathlineto{\pgfqpoint{4.522495in}{3.883324in}}%
\pgfpathlineto{\pgfqpoint{4.527515in}{3.878381in}}%
\pgfpathlineto{\pgfqpoint{4.555343in}{3.850667in}}%
\pgfpathlineto{\pgfqpoint{4.561233in}{3.844740in}}%
\pgfpathlineto{\pgfqpoint{4.567596in}{3.838455in}}%
\pgfpathlineto{\pgfqpoint{4.592740in}{3.813333in}}%
\pgfpathlineto{\pgfqpoint{4.599908in}{3.806097in}}%
\pgfpathlineto{\pgfqpoint{4.607677in}{3.798400in}}%
\pgfpathlineto{\pgfqpoint{4.630026in}{3.776000in}}%
\pgfpathlineto{\pgfqpoint{4.647758in}{3.758216in}}%
\pgfpathlineto{\pgfqpoint{4.657805in}{3.748025in}}%
\pgfpathlineto{\pgfqpoint{4.667201in}{3.738667in}}%
\pgfpathlineto{\pgfqpoint{4.687838in}{3.717903in}}%
\pgfpathlineto{\pgfqpoint{4.696340in}{3.709253in}}%
\pgfpathlineto{\pgfqpoint{4.704267in}{3.701333in}}%
\pgfpathlineto{\pgfqpoint{4.727919in}{3.677461in}}%
\pgfpathlineto{\pgfqpoint{4.734815in}{3.670423in}}%
\pgfpathlineto{\pgfqpoint{4.741224in}{3.664000in}}%
\pgfpathlineto{\pgfqpoint{4.768000in}{3.636890in}}%
\pgfusepath{fill}%
\end{pgfscope}%
\begin{pgfscope}%
\pgfpathrectangle{\pgfqpoint{0.800000in}{0.528000in}}{\pgfqpoint{3.968000in}{3.696000in}}%
\pgfusepath{clip}%
\pgfsetbuttcap%
\pgfsetroundjoin%
\definecolor{currentfill}{rgb}{0.377779,0.791781,0.377939}%
\pgfsetfillcolor{currentfill}%
\pgfsetlinewidth{0.000000pt}%
\definecolor{currentstroke}{rgb}{0.000000,0.000000,0.000000}%
\pgfsetstrokecolor{currentstroke}%
\pgfsetdash{}{0pt}%
\pgfpathmoveto{\pgfqpoint{4.768000in}{3.642311in}}%
\pgfpathlineto{\pgfqpoint{4.746579in}{3.664000in}}%
\pgfpathlineto{\pgfqpoint{4.737590in}{3.673008in}}%
\pgfpathlineto{\pgfqpoint{4.727919in}{3.682879in}}%
\pgfpathlineto{\pgfqpoint{4.709635in}{3.701333in}}%
\pgfpathlineto{\pgfqpoint{4.699118in}{3.711840in}}%
\pgfpathlineto{\pgfqpoint{4.687838in}{3.723318in}}%
\pgfpathlineto{\pgfqpoint{4.672583in}{3.738667in}}%
\pgfpathlineto{\pgfqpoint{4.660585in}{3.750615in}}%
\pgfpathlineto{\pgfqpoint{4.647758in}{3.763627in}}%
\pgfpathlineto{\pgfqpoint{4.635421in}{3.776000in}}%
\pgfpathlineto{\pgfqpoint{4.607677in}{3.803807in}}%
\pgfpathlineto{\pgfqpoint{4.602721in}{3.808717in}}%
\pgfpathlineto{\pgfqpoint{4.598148in}{3.813333in}}%
\pgfpathlineto{\pgfqpoint{4.567596in}{3.843858in}}%
\pgfpathlineto{\pgfqpoint{4.564048in}{3.847362in}}%
\pgfpathlineto{\pgfqpoint{4.560764in}{3.850667in}}%
\pgfpathlineto{\pgfqpoint{4.527515in}{3.883781in}}%
\pgfpathlineto{\pgfqpoint{4.525313in}{3.885949in}}%
\pgfpathlineto{\pgfqpoint{4.523268in}{3.888000in}}%
\pgfpathlineto{\pgfqpoint{4.505838in}{3.905142in}}%
\pgfpathlineto{\pgfqpoint{4.487434in}{3.923575in}}%
\pgfpathlineto{\pgfqpoint{4.486515in}{3.924477in}}%
\pgfpathlineto{\pgfqpoint{4.485659in}{3.925333in}}%
\pgfpathlineto{\pgfqpoint{4.466996in}{3.943629in}}%
\pgfpathlineto{\pgfqpoint{4.447929in}{3.962667in}}%
\pgfpathlineto{\pgfqpoint{4.447354in}{3.963236in}}%
\pgfpathlineto{\pgfqpoint{4.428092in}{3.982058in}}%
\pgfpathlineto{\pgfqpoint{4.410066in}{4.000000in}}%
\pgfpathlineto{\pgfqpoint{4.407273in}{4.002752in}}%
\pgfpathlineto{\pgfqpoint{4.389125in}{4.020430in}}%
\pgfpathlineto{\pgfqpoint{4.372088in}{4.037333in}}%
\pgfpathlineto{\pgfqpoint{4.369694in}{4.039664in}}%
\pgfpathlineto{\pgfqpoint{4.367192in}{4.042143in}}%
\pgfpathlineto{\pgfqpoint{4.350096in}{4.058743in}}%
\pgfpathlineto{\pgfqpoint{4.333995in}{4.074667in}}%
\pgfpathlineto{\pgfqpoint{4.327111in}{4.081407in}}%
\pgfpathlineto{\pgfqpoint{4.311004in}{4.096997in}}%
\pgfpathlineto{\pgfqpoint{4.295786in}{4.112000in}}%
\pgfpathlineto{\pgfqpoint{4.287030in}{4.120546in}}%
\pgfpathlineto{\pgfqpoint{4.271849in}{4.135193in}}%
\pgfpathlineto{\pgfqpoint{4.257460in}{4.149333in}}%
\pgfpathlineto{\pgfqpoint{4.252295in}{4.154312in}}%
\pgfpathlineto{\pgfqpoint{4.246949in}{4.159559in}}%
\pgfpathlineto{\pgfqpoint{4.232631in}{4.173329in}}%
\pgfpathlineto{\pgfqpoint{4.219016in}{4.186667in}}%
\pgfpathlineto{\pgfqpoint{4.206869in}{4.198447in}}%
\pgfpathlineto{\pgfqpoint{4.193349in}{4.211407in}}%
\pgfpathlineto{\pgfqpoint{4.180453in}{4.224000in}}%
\pgfpathlineto{\pgfqpoint{4.177705in}{4.224000in}}%
\pgfpathlineto{\pgfqpoint{4.191943in}{4.210097in}}%
\pgfpathlineto{\pgfqpoint{4.206869in}{4.195789in}}%
\pgfpathlineto{\pgfqpoint{4.216275in}{4.186667in}}%
\pgfpathlineto{\pgfqpoint{4.231225in}{4.172020in}}%
\pgfpathlineto{\pgfqpoint{4.246949in}{4.156899in}}%
\pgfpathlineto{\pgfqpoint{4.250904in}{4.153017in}}%
\pgfpathlineto{\pgfqpoint{4.254726in}{4.149333in}}%
\pgfpathlineto{\pgfqpoint{4.270445in}{4.133885in}}%
\pgfpathlineto{\pgfqpoint{4.287030in}{4.117884in}}%
\pgfpathlineto{\pgfqpoint{4.293059in}{4.112000in}}%
\pgfpathlineto{\pgfqpoint{4.309601in}{4.095691in}}%
\pgfpathlineto{\pgfqpoint{4.327111in}{4.078743in}}%
\pgfpathlineto{\pgfqpoint{4.331275in}{4.074667in}}%
\pgfpathlineto{\pgfqpoint{4.348695in}{4.057438in}}%
\pgfpathlineto{\pgfqpoint{4.367192in}{4.039477in}}%
\pgfpathlineto{\pgfqpoint{4.368307in}{4.038372in}}%
\pgfpathlineto{\pgfqpoint{4.369375in}{4.037333in}}%
\pgfpathlineto{\pgfqpoint{4.387725in}{4.019126in}}%
\pgfpathlineto{\pgfqpoint{4.407273in}{4.000085in}}%
\pgfpathlineto{\pgfqpoint{4.407359in}{4.000000in}}%
\pgfpathlineto{\pgfqpoint{4.426693in}{3.980756in}}%
\pgfpathlineto{\pgfqpoint{4.445205in}{3.962667in}}%
\pgfpathlineto{\pgfqpoint{4.447354in}{3.960545in}}%
\pgfpathlineto{\pgfqpoint{4.465599in}{3.942328in}}%
\pgfpathlineto{\pgfqpoint{4.482935in}{3.925333in}}%
\pgfpathlineto{\pgfqpoint{4.485105in}{3.923163in}}%
\pgfpathlineto{\pgfqpoint{4.487434in}{3.920877in}}%
\pgfpathlineto{\pgfqpoint{4.504442in}{3.903842in}}%
\pgfpathlineto{\pgfqpoint{4.520551in}{3.888000in}}%
\pgfpathlineto{\pgfqpoint{4.523904in}{3.884636in}}%
\pgfpathlineto{\pgfqpoint{4.527515in}{3.881081in}}%
\pgfpathlineto{\pgfqpoint{4.558053in}{3.850667in}}%
\pgfpathlineto{\pgfqpoint{4.562641in}{3.846051in}}%
\pgfpathlineto{\pgfqpoint{4.567596in}{3.841156in}}%
\pgfpathlineto{\pgfqpoint{4.595444in}{3.813333in}}%
\pgfpathlineto{\pgfqpoint{4.601315in}{3.807407in}}%
\pgfpathlineto{\pgfqpoint{4.607677in}{3.801103in}}%
\pgfpathlineto{\pgfqpoint{4.632723in}{3.776000in}}%
\pgfpathlineto{\pgfqpoint{4.647758in}{3.760921in}}%
\pgfpathlineto{\pgfqpoint{4.659195in}{3.749320in}}%
\pgfpathlineto{\pgfqpoint{4.669892in}{3.738667in}}%
\pgfpathlineto{\pgfqpoint{4.687838in}{3.720611in}}%
\pgfpathlineto{\pgfqpoint{4.697729in}{3.710546in}}%
\pgfpathlineto{\pgfqpoint{4.706951in}{3.701333in}}%
\pgfpathlineto{\pgfqpoint{4.727919in}{3.680170in}}%
\pgfpathlineto{\pgfqpoint{4.736203in}{3.671716in}}%
\pgfpathlineto{\pgfqpoint{4.743901in}{3.664000in}}%
\pgfpathlineto{\pgfqpoint{4.768000in}{3.639600in}}%
\pgfusepath{fill}%
\end{pgfscope}%
\begin{pgfscope}%
\pgfpathrectangle{\pgfqpoint{0.800000in}{0.528000in}}{\pgfqpoint{3.968000in}{3.696000in}}%
\pgfusepath{clip}%
\pgfsetbuttcap%
\pgfsetroundjoin%
\definecolor{currentfill}{rgb}{0.377779,0.791781,0.377939}%
\pgfsetfillcolor{currentfill}%
\pgfsetlinewidth{0.000000pt}%
\definecolor{currentstroke}{rgb}{0.000000,0.000000,0.000000}%
\pgfsetstrokecolor{currentstroke}%
\pgfsetdash{}{0pt}%
\pgfpathmoveto{\pgfqpoint{4.768000in}{3.645022in}}%
\pgfpathlineto{\pgfqpoint{4.749256in}{3.664000in}}%
\pgfpathlineto{\pgfqpoint{4.738978in}{3.674301in}}%
\pgfpathlineto{\pgfqpoint{4.727919in}{3.685588in}}%
\pgfpathlineto{\pgfqpoint{4.712319in}{3.701333in}}%
\pgfpathlineto{\pgfqpoint{4.700508in}{3.713134in}}%
\pgfpathlineto{\pgfqpoint{4.687838in}{3.726025in}}%
\pgfpathlineto{\pgfqpoint{4.675274in}{3.738667in}}%
\pgfpathlineto{\pgfqpoint{4.661976in}{3.751910in}}%
\pgfpathlineto{\pgfqpoint{4.647758in}{3.766332in}}%
\pgfpathlineto{\pgfqpoint{4.638118in}{3.776000in}}%
\pgfpathlineto{\pgfqpoint{4.607677in}{3.806510in}}%
\pgfpathlineto{\pgfqpoint{4.604127in}{3.810027in}}%
\pgfpathlineto{\pgfqpoint{4.600852in}{3.813333in}}%
\pgfpathlineto{\pgfqpoint{4.567596in}{3.846560in}}%
\pgfpathlineto{\pgfqpoint{4.565456in}{3.848673in}}%
\pgfpathlineto{\pgfqpoint{4.563475in}{3.850667in}}%
\pgfpathlineto{\pgfqpoint{4.527515in}{3.886481in}}%
\pgfpathlineto{\pgfqpoint{4.526722in}{3.887261in}}%
\pgfpathlineto{\pgfqpoint{4.525986in}{3.888000in}}%
\pgfpathlineto{\pgfqpoint{4.507233in}{3.906442in}}%
\pgfpathlineto{\pgfqpoint{4.488373in}{3.925333in}}%
\pgfpathlineto{\pgfqpoint{4.487916in}{3.925782in}}%
\pgfpathlineto{\pgfqpoint{4.487434in}{3.926264in}}%
\pgfpathlineto{\pgfqpoint{4.468393in}{3.944930in}}%
\pgfpathlineto{\pgfqpoint{4.450630in}{3.962667in}}%
\pgfpathlineto{\pgfqpoint{4.447354in}{3.965905in}}%
\pgfpathlineto{\pgfqpoint{4.429490in}{3.983361in}}%
\pgfpathlineto{\pgfqpoint{4.412773in}{4.000000in}}%
\pgfpathlineto{\pgfqpoint{4.407273in}{4.005420in}}%
\pgfpathlineto{\pgfqpoint{4.390525in}{4.021734in}}%
\pgfpathlineto{\pgfqpoint{4.374802in}{4.037333in}}%
\pgfpathlineto{\pgfqpoint{4.371080in}{4.040955in}}%
\pgfpathlineto{\pgfqpoint{4.367192in}{4.044808in}}%
\pgfpathlineto{\pgfqpoint{4.351497in}{4.060048in}}%
\pgfpathlineto{\pgfqpoint{4.336716in}{4.074667in}}%
\pgfpathlineto{\pgfqpoint{4.327111in}{4.084071in}}%
\pgfpathlineto{\pgfqpoint{4.312406in}{4.098303in}}%
\pgfpathlineto{\pgfqpoint{4.298514in}{4.112000in}}%
\pgfpathlineto{\pgfqpoint{4.287030in}{4.123208in}}%
\pgfpathlineto{\pgfqpoint{4.273253in}{4.136500in}}%
\pgfpathlineto{\pgfqpoint{4.260194in}{4.149333in}}%
\pgfpathlineto{\pgfqpoint{4.253685in}{4.155607in}}%
\pgfpathlineto{\pgfqpoint{4.246949in}{4.162219in}}%
\pgfpathlineto{\pgfqpoint{4.234036in}{4.174638in}}%
\pgfpathlineto{\pgfqpoint{4.221757in}{4.186667in}}%
\pgfpathlineto{\pgfqpoint{4.206869in}{4.201105in}}%
\pgfpathlineto{\pgfqpoint{4.194756in}{4.212717in}}%
\pgfpathlineto{\pgfqpoint{4.183201in}{4.224000in}}%
\pgfpathlineto{\pgfqpoint{4.180453in}{4.224000in}}%
\pgfpathlineto{\pgfqpoint{4.193349in}{4.211407in}}%
\pgfpathlineto{\pgfqpoint{4.206869in}{4.198447in}}%
\pgfpathlineto{\pgfqpoint{4.219016in}{4.186667in}}%
\pgfpathlineto{\pgfqpoint{4.232631in}{4.173329in}}%
\pgfpathlineto{\pgfqpoint{4.246949in}{4.159559in}}%
\pgfpathlineto{\pgfqpoint{4.252295in}{4.154312in}}%
\pgfpathlineto{\pgfqpoint{4.257460in}{4.149333in}}%
\pgfpathlineto{\pgfqpoint{4.271849in}{4.135193in}}%
\pgfpathlineto{\pgfqpoint{4.287030in}{4.120546in}}%
\pgfpathlineto{\pgfqpoint{4.295786in}{4.112000in}}%
\pgfpathlineto{\pgfqpoint{4.311004in}{4.096997in}}%
\pgfpathlineto{\pgfqpoint{4.327111in}{4.081407in}}%
\pgfpathlineto{\pgfqpoint{4.333995in}{4.074667in}}%
\pgfpathlineto{\pgfqpoint{4.350096in}{4.058743in}}%
\pgfpathlineto{\pgfqpoint{4.367192in}{4.042143in}}%
\pgfpathlineto{\pgfqpoint{4.369694in}{4.039664in}}%
\pgfpathlineto{\pgfqpoint{4.372088in}{4.037333in}}%
\pgfpathlineto{\pgfqpoint{4.389125in}{4.020430in}}%
\pgfpathlineto{\pgfqpoint{4.407273in}{4.002752in}}%
\pgfpathlineto{\pgfqpoint{4.410066in}{4.000000in}}%
\pgfpathlineto{\pgfqpoint{4.428092in}{3.982058in}}%
\pgfpathlineto{\pgfqpoint{4.447354in}{3.963236in}}%
\pgfpathlineto{\pgfqpoint{4.447929in}{3.962667in}}%
\pgfpathlineto{\pgfqpoint{4.466996in}{3.943629in}}%
\pgfpathlineto{\pgfqpoint{4.485659in}{3.925333in}}%
\pgfpathlineto{\pgfqpoint{4.486515in}{3.924477in}}%
\pgfpathlineto{\pgfqpoint{4.487434in}{3.923575in}}%
\pgfpathlineto{\pgfqpoint{4.505838in}{3.905142in}}%
\pgfpathlineto{\pgfqpoint{4.523268in}{3.888000in}}%
\pgfpathlineto{\pgfqpoint{4.525313in}{3.885949in}}%
\pgfpathlineto{\pgfqpoint{4.527515in}{3.883781in}}%
\pgfpathlineto{\pgfqpoint{4.560764in}{3.850667in}}%
\pgfpathlineto{\pgfqpoint{4.564048in}{3.847362in}}%
\pgfpathlineto{\pgfqpoint{4.567596in}{3.843858in}}%
\pgfpathlineto{\pgfqpoint{4.598148in}{3.813333in}}%
\pgfpathlineto{\pgfqpoint{4.602721in}{3.808717in}}%
\pgfpathlineto{\pgfqpoint{4.607677in}{3.803807in}}%
\pgfpathlineto{\pgfqpoint{4.635421in}{3.776000in}}%
\pgfpathlineto{\pgfqpoint{4.647758in}{3.763627in}}%
\pgfpathlineto{\pgfqpoint{4.660585in}{3.750615in}}%
\pgfpathlineto{\pgfqpoint{4.672583in}{3.738667in}}%
\pgfpathlineto{\pgfqpoint{4.687838in}{3.723318in}}%
\pgfpathlineto{\pgfqpoint{4.699118in}{3.711840in}}%
\pgfpathlineto{\pgfqpoint{4.709635in}{3.701333in}}%
\pgfpathlineto{\pgfqpoint{4.727919in}{3.682879in}}%
\pgfpathlineto{\pgfqpoint{4.737590in}{3.673008in}}%
\pgfpathlineto{\pgfqpoint{4.746579in}{3.664000in}}%
\pgfpathlineto{\pgfqpoint{4.768000in}{3.642311in}}%
\pgfusepath{fill}%
\end{pgfscope}%
\begin{pgfscope}%
\pgfpathrectangle{\pgfqpoint{0.800000in}{0.528000in}}{\pgfqpoint{3.968000in}{3.696000in}}%
\pgfusepath{clip}%
\pgfsetbuttcap%
\pgfsetroundjoin%
\definecolor{currentfill}{rgb}{0.386433,0.794644,0.372886}%
\pgfsetfillcolor{currentfill}%
\pgfsetlinewidth{0.000000pt}%
\definecolor{currentstroke}{rgb}{0.000000,0.000000,0.000000}%
\pgfsetstrokecolor{currentstroke}%
\pgfsetdash{}{0pt}%
\pgfpathmoveto{\pgfqpoint{4.768000in}{3.647733in}}%
\pgfpathlineto{\pgfqpoint{4.751934in}{3.664000in}}%
\pgfpathlineto{\pgfqpoint{4.740366in}{3.675593in}}%
\pgfpathlineto{\pgfqpoint{4.727919in}{3.688297in}}%
\pgfpathlineto{\pgfqpoint{4.715003in}{3.701333in}}%
\pgfpathlineto{\pgfqpoint{4.701897in}{3.714428in}}%
\pgfpathlineto{\pgfqpoint{4.687838in}{3.728732in}}%
\pgfpathlineto{\pgfqpoint{4.677964in}{3.738667in}}%
\pgfpathlineto{\pgfqpoint{4.663366in}{3.753205in}}%
\pgfpathlineto{\pgfqpoint{4.647758in}{3.769038in}}%
\pgfpathlineto{\pgfqpoint{4.640816in}{3.776000in}}%
\pgfpathlineto{\pgfqpoint{4.607677in}{3.809214in}}%
\pgfpathlineto{\pgfqpoint{4.605534in}{3.811337in}}%
\pgfpathlineto{\pgfqpoint{4.603556in}{3.813333in}}%
\pgfpathlineto{\pgfqpoint{4.567596in}{3.849262in}}%
\pgfpathlineto{\pgfqpoint{4.566864in}{3.849985in}}%
\pgfpathlineto{\pgfqpoint{4.566186in}{3.850667in}}%
\pgfpathlineto{\pgfqpoint{4.545822in}{3.870948in}}%
\pgfpathlineto{\pgfqpoint{4.528690in}{3.888000in}}%
\pgfpathlineto{\pgfqpoint{4.527515in}{3.889169in}}%
\pgfpathlineto{\pgfqpoint{4.508629in}{3.907742in}}%
\pgfpathlineto{\pgfqpoint{4.491066in}{3.925333in}}%
\pgfpathlineto{\pgfqpoint{4.489299in}{3.927070in}}%
\pgfpathlineto{\pgfqpoint{4.487434in}{3.928935in}}%
\pgfpathlineto{\pgfqpoint{4.469790in}{3.946232in}}%
\pgfpathlineto{\pgfqpoint{4.453330in}{3.962667in}}%
\pgfpathlineto{\pgfqpoint{4.447354in}{3.968574in}}%
\pgfpathlineto{\pgfqpoint{4.430888in}{3.984664in}}%
\pgfpathlineto{\pgfqpoint{4.415480in}{4.000000in}}%
\pgfpathlineto{\pgfqpoint{4.407273in}{4.008087in}}%
\pgfpathlineto{\pgfqpoint{4.391925in}{4.023037in}}%
\pgfpathlineto{\pgfqpoint{4.377516in}{4.037333in}}%
\pgfpathlineto{\pgfqpoint{4.372467in}{4.042247in}}%
\pgfpathlineto{\pgfqpoint{4.367192in}{4.047474in}}%
\pgfpathlineto{\pgfqpoint{4.352898in}{4.061353in}}%
\pgfpathlineto{\pgfqpoint{4.339437in}{4.074667in}}%
\pgfpathlineto{\pgfqpoint{4.327111in}{4.086734in}}%
\pgfpathlineto{\pgfqpoint{4.313809in}{4.099610in}}%
\pgfpathlineto{\pgfqpoint{4.301241in}{4.112000in}}%
\pgfpathlineto{\pgfqpoint{4.287030in}{4.125870in}}%
\pgfpathlineto{\pgfqpoint{4.274656in}{4.137808in}}%
\pgfpathlineto{\pgfqpoint{4.262929in}{4.149333in}}%
\pgfpathlineto{\pgfqpoint{4.255076in}{4.156902in}}%
\pgfpathlineto{\pgfqpoint{4.246949in}{4.164879in}}%
\pgfpathlineto{\pgfqpoint{4.235441in}{4.175947in}}%
\pgfpathlineto{\pgfqpoint{4.224498in}{4.186667in}}%
\pgfpathlineto{\pgfqpoint{4.206869in}{4.203764in}}%
\pgfpathlineto{\pgfqpoint{4.196162in}{4.214027in}}%
\pgfpathlineto{\pgfqpoint{4.185949in}{4.224000in}}%
\pgfpathlineto{\pgfqpoint{4.183201in}{4.224000in}}%
\pgfpathlineto{\pgfqpoint{4.194756in}{4.212717in}}%
\pgfpathlineto{\pgfqpoint{4.206869in}{4.201105in}}%
\pgfpathlineto{\pgfqpoint{4.221757in}{4.186667in}}%
\pgfpathlineto{\pgfqpoint{4.234036in}{4.174638in}}%
\pgfpathlineto{\pgfqpoint{4.246949in}{4.162219in}}%
\pgfpathlineto{\pgfqpoint{4.253685in}{4.155607in}}%
\pgfpathlineto{\pgfqpoint{4.260194in}{4.149333in}}%
\pgfpathlineto{\pgfqpoint{4.273253in}{4.136500in}}%
\pgfpathlineto{\pgfqpoint{4.287030in}{4.123208in}}%
\pgfpathlineto{\pgfqpoint{4.298514in}{4.112000in}}%
\pgfpathlineto{\pgfqpoint{4.312406in}{4.098303in}}%
\pgfpathlineto{\pgfqpoint{4.327111in}{4.084071in}}%
\pgfpathlineto{\pgfqpoint{4.336716in}{4.074667in}}%
\pgfpathlineto{\pgfqpoint{4.351497in}{4.060048in}}%
\pgfpathlineto{\pgfqpoint{4.367192in}{4.044808in}}%
\pgfpathlineto{\pgfqpoint{4.371080in}{4.040955in}}%
\pgfpathlineto{\pgfqpoint{4.374802in}{4.037333in}}%
\pgfpathlineto{\pgfqpoint{4.390525in}{4.021734in}}%
\pgfpathlineto{\pgfqpoint{4.407273in}{4.005420in}}%
\pgfpathlineto{\pgfqpoint{4.412773in}{4.000000in}}%
\pgfpathlineto{\pgfqpoint{4.429490in}{3.983361in}}%
\pgfpathlineto{\pgfqpoint{4.447354in}{3.965905in}}%
\pgfpathlineto{\pgfqpoint{4.450630in}{3.962667in}}%
\pgfpathlineto{\pgfqpoint{4.468393in}{3.944930in}}%
\pgfpathlineto{\pgfqpoint{4.487434in}{3.926264in}}%
\pgfpathlineto{\pgfqpoint{4.487916in}{3.925782in}}%
\pgfpathlineto{\pgfqpoint{4.488373in}{3.925333in}}%
\pgfpathlineto{\pgfqpoint{4.507233in}{3.906442in}}%
\pgfpathlineto{\pgfqpoint{4.525986in}{3.888000in}}%
\pgfpathlineto{\pgfqpoint{4.526722in}{3.887261in}}%
\pgfpathlineto{\pgfqpoint{4.527515in}{3.886481in}}%
\pgfpathlineto{\pgfqpoint{4.563475in}{3.850667in}}%
\pgfpathlineto{\pgfqpoint{4.565456in}{3.848673in}}%
\pgfpathlineto{\pgfqpoint{4.567596in}{3.846560in}}%
\pgfpathlineto{\pgfqpoint{4.600852in}{3.813333in}}%
\pgfpathlineto{\pgfqpoint{4.604127in}{3.810027in}}%
\pgfpathlineto{\pgfqpoint{4.607677in}{3.806510in}}%
\pgfpathlineto{\pgfqpoint{4.638118in}{3.776000in}}%
\pgfpathlineto{\pgfqpoint{4.647758in}{3.766332in}}%
\pgfpathlineto{\pgfqpoint{4.661976in}{3.751910in}}%
\pgfpathlineto{\pgfqpoint{4.675274in}{3.738667in}}%
\pgfpathlineto{\pgfqpoint{4.687838in}{3.726025in}}%
\pgfpathlineto{\pgfqpoint{4.700508in}{3.713134in}}%
\pgfpathlineto{\pgfqpoint{4.712319in}{3.701333in}}%
\pgfpathlineto{\pgfqpoint{4.727919in}{3.685588in}}%
\pgfpathlineto{\pgfqpoint{4.738978in}{3.674301in}}%
\pgfpathlineto{\pgfqpoint{4.749256in}{3.664000in}}%
\pgfpathlineto{\pgfqpoint{4.768000in}{3.645022in}}%
\pgfusepath{fill}%
\end{pgfscope}%
\begin{pgfscope}%
\pgfpathrectangle{\pgfqpoint{0.800000in}{0.528000in}}{\pgfqpoint{3.968000in}{3.696000in}}%
\pgfusepath{clip}%
\pgfsetbuttcap%
\pgfsetroundjoin%
\definecolor{currentfill}{rgb}{0.386433,0.794644,0.372886}%
\pgfsetfillcolor{currentfill}%
\pgfsetlinewidth{0.000000pt}%
\definecolor{currentstroke}{rgb}{0.000000,0.000000,0.000000}%
\pgfsetstrokecolor{currentstroke}%
\pgfsetdash{}{0pt}%
\pgfpathmoveto{\pgfqpoint{4.768000in}{3.650444in}}%
\pgfpathlineto{\pgfqpoint{4.754611in}{3.664000in}}%
\pgfpathlineto{\pgfqpoint{4.741753in}{3.676886in}}%
\pgfpathlineto{\pgfqpoint{4.727919in}{3.691006in}}%
\pgfpathlineto{\pgfqpoint{4.717687in}{3.701333in}}%
\pgfpathlineto{\pgfqpoint{4.703286in}{3.715722in}}%
\pgfpathlineto{\pgfqpoint{4.687838in}{3.731439in}}%
\pgfpathlineto{\pgfqpoint{4.680655in}{3.738667in}}%
\pgfpathlineto{\pgfqpoint{4.664757in}{3.754500in}}%
\pgfpathlineto{\pgfqpoint{4.647758in}{3.771743in}}%
\pgfpathlineto{\pgfqpoint{4.643513in}{3.776000in}}%
\pgfpathlineto{\pgfqpoint{4.607677in}{3.811917in}}%
\pgfpathlineto{\pgfqpoint{4.606940in}{3.812647in}}%
\pgfpathlineto{\pgfqpoint{4.606261in}{3.813333in}}%
\pgfpathlineto{\pgfqpoint{4.586762in}{3.832814in}}%
\pgfpathlineto{\pgfqpoint{4.568882in}{3.850667in}}%
\pgfpathlineto{\pgfqpoint{4.568258in}{3.851284in}}%
\pgfpathlineto{\pgfqpoint{4.567596in}{3.851950in}}%
\pgfpathlineto{\pgfqpoint{4.531377in}{3.888000in}}%
\pgfpathlineto{\pgfqpoint{4.527515in}{3.891841in}}%
\pgfpathlineto{\pgfqpoint{4.510025in}{3.909042in}}%
\pgfpathlineto{\pgfqpoint{4.493760in}{3.925333in}}%
\pgfpathlineto{\pgfqpoint{4.490681in}{3.928358in}}%
\pgfpathlineto{\pgfqpoint{4.487434in}{3.931606in}}%
\pgfpathlineto{\pgfqpoint{4.471187in}{3.947533in}}%
\pgfpathlineto{\pgfqpoint{4.456030in}{3.962667in}}%
\pgfpathlineto{\pgfqpoint{4.447354in}{3.971243in}}%
\pgfpathlineto{\pgfqpoint{4.432287in}{3.985966in}}%
\pgfpathlineto{\pgfqpoint{4.418187in}{4.000000in}}%
\pgfpathlineto{\pgfqpoint{4.407273in}{4.010754in}}%
\pgfpathlineto{\pgfqpoint{4.393324in}{4.024341in}}%
\pgfpathlineto{\pgfqpoint{4.380230in}{4.037333in}}%
\pgfpathlineto{\pgfqpoint{4.373853in}{4.043538in}}%
\pgfpathlineto{\pgfqpoint{4.367192in}{4.050139in}}%
\pgfpathlineto{\pgfqpoint{4.354299in}{4.062658in}}%
\pgfpathlineto{\pgfqpoint{4.342157in}{4.074667in}}%
\pgfpathlineto{\pgfqpoint{4.327111in}{4.089398in}}%
\pgfpathlineto{\pgfqpoint{4.315211in}{4.100916in}}%
\pgfpathlineto{\pgfqpoint{4.303969in}{4.112000in}}%
\pgfpathlineto{\pgfqpoint{4.287030in}{4.128532in}}%
\pgfpathlineto{\pgfqpoint{4.276060in}{4.139115in}}%
\pgfpathlineto{\pgfqpoint{4.265663in}{4.149333in}}%
\pgfpathlineto{\pgfqpoint{4.256466in}{4.158198in}}%
\pgfpathlineto{\pgfqpoint{4.246949in}{4.167539in}}%
\pgfpathlineto{\pgfqpoint{4.236846in}{4.177256in}}%
\pgfpathlineto{\pgfqpoint{4.227240in}{4.186667in}}%
\pgfpathlineto{\pgfqpoint{4.206869in}{4.206422in}}%
\pgfpathlineto{\pgfqpoint{4.197569in}{4.215337in}}%
\pgfpathlineto{\pgfqpoint{4.188698in}{4.224000in}}%
\pgfpathlineto{\pgfqpoint{4.185949in}{4.224000in}}%
\pgfpathlineto{\pgfqpoint{4.196162in}{4.214027in}}%
\pgfpathlineto{\pgfqpoint{4.206869in}{4.203764in}}%
\pgfpathlineto{\pgfqpoint{4.224498in}{4.186667in}}%
\pgfpathlineto{\pgfqpoint{4.235441in}{4.175947in}}%
\pgfpathlineto{\pgfqpoint{4.246949in}{4.164879in}}%
\pgfpathlineto{\pgfqpoint{4.255076in}{4.156902in}}%
\pgfpathlineto{\pgfqpoint{4.262929in}{4.149333in}}%
\pgfpathlineto{\pgfqpoint{4.274656in}{4.137808in}}%
\pgfpathlineto{\pgfqpoint{4.287030in}{4.125870in}}%
\pgfpathlineto{\pgfqpoint{4.301241in}{4.112000in}}%
\pgfpathlineto{\pgfqpoint{4.313809in}{4.099610in}}%
\pgfpathlineto{\pgfqpoint{4.327111in}{4.086734in}}%
\pgfpathlineto{\pgfqpoint{4.339437in}{4.074667in}}%
\pgfpathlineto{\pgfqpoint{4.352898in}{4.061353in}}%
\pgfpathlineto{\pgfqpoint{4.367192in}{4.047474in}}%
\pgfpathlineto{\pgfqpoint{4.372467in}{4.042247in}}%
\pgfpathlineto{\pgfqpoint{4.377516in}{4.037333in}}%
\pgfpathlineto{\pgfqpoint{4.391925in}{4.023037in}}%
\pgfpathlineto{\pgfqpoint{4.407273in}{4.008087in}}%
\pgfpathlineto{\pgfqpoint{4.415480in}{4.000000in}}%
\pgfpathlineto{\pgfqpoint{4.430888in}{3.984664in}}%
\pgfpathlineto{\pgfqpoint{4.447354in}{3.968574in}}%
\pgfpathlineto{\pgfqpoint{4.453330in}{3.962667in}}%
\pgfpathlineto{\pgfqpoint{4.469790in}{3.946232in}}%
\pgfpathlineto{\pgfqpoint{4.487434in}{3.928935in}}%
\pgfpathlineto{\pgfqpoint{4.489299in}{3.927070in}}%
\pgfpathlineto{\pgfqpoint{4.491066in}{3.925333in}}%
\pgfpathlineto{\pgfqpoint{4.508629in}{3.907742in}}%
\pgfpathlineto{\pgfqpoint{4.527515in}{3.889169in}}%
\pgfpathlineto{\pgfqpoint{4.528690in}{3.888000in}}%
\pgfpathlineto{\pgfqpoint{4.545822in}{3.870948in}}%
\pgfpathlineto{\pgfqpoint{4.566186in}{3.850667in}}%
\pgfpathlineto{\pgfqpoint{4.566864in}{3.849985in}}%
\pgfpathlineto{\pgfqpoint{4.567596in}{3.849262in}}%
\pgfpathlineto{\pgfqpoint{4.603556in}{3.813333in}}%
\pgfpathlineto{\pgfqpoint{4.605534in}{3.811337in}}%
\pgfpathlineto{\pgfqpoint{4.607677in}{3.809214in}}%
\pgfpathlineto{\pgfqpoint{4.640816in}{3.776000in}}%
\pgfpathlineto{\pgfqpoint{4.647758in}{3.769038in}}%
\pgfpathlineto{\pgfqpoint{4.663366in}{3.753205in}}%
\pgfpathlineto{\pgfqpoint{4.677964in}{3.738667in}}%
\pgfpathlineto{\pgfqpoint{4.687838in}{3.728732in}}%
\pgfpathlineto{\pgfqpoint{4.701897in}{3.714428in}}%
\pgfpathlineto{\pgfqpoint{4.715003in}{3.701333in}}%
\pgfpathlineto{\pgfqpoint{4.727919in}{3.688297in}}%
\pgfpathlineto{\pgfqpoint{4.740366in}{3.675593in}}%
\pgfpathlineto{\pgfqpoint{4.751934in}{3.664000in}}%
\pgfpathlineto{\pgfqpoint{4.768000in}{3.647733in}}%
\pgfusepath{fill}%
\end{pgfscope}%
\begin{pgfscope}%
\pgfpathrectangle{\pgfqpoint{0.800000in}{0.528000in}}{\pgfqpoint{3.968000in}{3.696000in}}%
\pgfusepath{clip}%
\pgfsetbuttcap%
\pgfsetroundjoin%
\definecolor{currentfill}{rgb}{0.386433,0.794644,0.372886}%
\pgfsetfillcolor{currentfill}%
\pgfsetlinewidth{0.000000pt}%
\definecolor{currentstroke}{rgb}{0.000000,0.000000,0.000000}%
\pgfsetstrokecolor{currentstroke}%
\pgfsetdash{}{0pt}%
\pgfpathmoveto{\pgfqpoint{4.768000in}{3.653155in}}%
\pgfpathlineto{\pgfqpoint{4.757288in}{3.664000in}}%
\pgfpathlineto{\pgfqpoint{4.743141in}{3.678179in}}%
\pgfpathlineto{\pgfqpoint{4.727919in}{3.693715in}}%
\pgfpathlineto{\pgfqpoint{4.720371in}{3.701333in}}%
\pgfpathlineto{\pgfqpoint{4.704675in}{3.717015in}}%
\pgfpathlineto{\pgfqpoint{4.687838in}{3.734147in}}%
\pgfpathlineto{\pgfqpoint{4.683346in}{3.738667in}}%
\pgfpathlineto{\pgfqpoint{4.666147in}{3.755795in}}%
\pgfpathlineto{\pgfqpoint{4.647758in}{3.774448in}}%
\pgfpathlineto{\pgfqpoint{4.646210in}{3.776000in}}%
\pgfpathlineto{\pgfqpoint{4.625860in}{3.796396in}}%
\pgfpathlineto{\pgfqpoint{4.608950in}{3.813333in}}%
\pgfpathlineto{\pgfqpoint{4.608333in}{3.813945in}}%
\pgfpathlineto{\pgfqpoint{4.607677in}{3.814608in}}%
\pgfpathlineto{\pgfqpoint{4.571563in}{3.850667in}}%
\pgfpathlineto{\pgfqpoint{4.569638in}{3.852569in}}%
\pgfpathlineto{\pgfqpoint{4.567596in}{3.854625in}}%
\pgfpathlineto{\pgfqpoint{4.534064in}{3.888000in}}%
\pgfpathlineto{\pgfqpoint{4.527515in}{3.894514in}}%
\pgfpathlineto{\pgfqpoint{4.511420in}{3.910342in}}%
\pgfpathlineto{\pgfqpoint{4.496454in}{3.925333in}}%
\pgfpathlineto{\pgfqpoint{4.492064in}{3.929645in}}%
\pgfpathlineto{\pgfqpoint{4.487434in}{3.934276in}}%
\pgfpathlineto{\pgfqpoint{4.472584in}{3.948834in}}%
\pgfpathlineto{\pgfqpoint{4.458731in}{3.962667in}}%
\pgfpathlineto{\pgfqpoint{4.447354in}{3.973912in}}%
\pgfpathlineto{\pgfqpoint{4.433685in}{3.987269in}}%
\pgfpathlineto{\pgfqpoint{4.420894in}{4.000000in}}%
\pgfpathlineto{\pgfqpoint{4.407273in}{4.013421in}}%
\pgfpathlineto{\pgfqpoint{4.394724in}{4.025645in}}%
\pgfpathlineto{\pgfqpoint{4.382944in}{4.037333in}}%
\pgfpathlineto{\pgfqpoint{4.375240in}{4.044830in}}%
\pgfpathlineto{\pgfqpoint{4.367192in}{4.052805in}}%
\pgfpathlineto{\pgfqpoint{4.355700in}{4.063963in}}%
\pgfpathlineto{\pgfqpoint{4.344878in}{4.074667in}}%
\pgfpathlineto{\pgfqpoint{4.327111in}{4.092062in}}%
\pgfpathlineto{\pgfqpoint{4.316614in}{4.102222in}}%
\pgfpathlineto{\pgfqpoint{4.306696in}{4.112000in}}%
\pgfpathlineto{\pgfqpoint{4.287030in}{4.131193in}}%
\pgfpathlineto{\pgfqpoint{4.277464in}{4.140423in}}%
\pgfpathlineto{\pgfqpoint{4.268397in}{4.149333in}}%
\pgfpathlineto{\pgfqpoint{4.257857in}{4.159493in}}%
\pgfpathlineto{\pgfqpoint{4.246949in}{4.170200in}}%
\pgfpathlineto{\pgfqpoint{4.238251in}{4.178565in}}%
\pgfpathlineto{\pgfqpoint{4.229981in}{4.186667in}}%
\pgfpathlineto{\pgfqpoint{4.206869in}{4.209081in}}%
\pgfpathlineto{\pgfqpoint{4.198975in}{4.216647in}}%
\pgfpathlineto{\pgfqpoint{4.191446in}{4.224000in}}%
\pgfpathlineto{\pgfqpoint{4.188698in}{4.224000in}}%
\pgfpathlineto{\pgfqpoint{4.197569in}{4.215337in}}%
\pgfpathlineto{\pgfqpoint{4.206869in}{4.206422in}}%
\pgfpathlineto{\pgfqpoint{4.227240in}{4.186667in}}%
\pgfpathlineto{\pgfqpoint{4.236846in}{4.177256in}}%
\pgfpathlineto{\pgfqpoint{4.246949in}{4.167539in}}%
\pgfpathlineto{\pgfqpoint{4.256466in}{4.158198in}}%
\pgfpathlineto{\pgfqpoint{4.265663in}{4.149333in}}%
\pgfpathlineto{\pgfqpoint{4.276060in}{4.139115in}}%
\pgfpathlineto{\pgfqpoint{4.287030in}{4.128532in}}%
\pgfpathlineto{\pgfqpoint{4.303969in}{4.112000in}}%
\pgfpathlineto{\pgfqpoint{4.315211in}{4.100916in}}%
\pgfpathlineto{\pgfqpoint{4.327111in}{4.089398in}}%
\pgfpathlineto{\pgfqpoint{4.342157in}{4.074667in}}%
\pgfpathlineto{\pgfqpoint{4.354299in}{4.062658in}}%
\pgfpathlineto{\pgfqpoint{4.367192in}{4.050139in}}%
\pgfpathlineto{\pgfqpoint{4.373853in}{4.043538in}}%
\pgfpathlineto{\pgfqpoint{4.380230in}{4.037333in}}%
\pgfpathlineto{\pgfqpoint{4.393324in}{4.024341in}}%
\pgfpathlineto{\pgfqpoint{4.407273in}{4.010754in}}%
\pgfpathlineto{\pgfqpoint{4.418187in}{4.000000in}}%
\pgfpathlineto{\pgfqpoint{4.432287in}{3.985966in}}%
\pgfpathlineto{\pgfqpoint{4.447354in}{3.971243in}}%
\pgfpathlineto{\pgfqpoint{4.456030in}{3.962667in}}%
\pgfpathlineto{\pgfqpoint{4.471187in}{3.947533in}}%
\pgfpathlineto{\pgfqpoint{4.487434in}{3.931606in}}%
\pgfpathlineto{\pgfqpoint{4.490681in}{3.928358in}}%
\pgfpathlineto{\pgfqpoint{4.493760in}{3.925333in}}%
\pgfpathlineto{\pgfqpoint{4.510025in}{3.909042in}}%
\pgfpathlineto{\pgfqpoint{4.527515in}{3.891841in}}%
\pgfpathlineto{\pgfqpoint{4.531377in}{3.888000in}}%
\pgfpathlineto{\pgfqpoint{4.567596in}{3.851950in}}%
\pgfpathlineto{\pgfqpoint{4.568258in}{3.851284in}}%
\pgfpathlineto{\pgfqpoint{4.568882in}{3.850667in}}%
\pgfpathlineto{\pgfqpoint{4.586762in}{3.832814in}}%
\pgfpathlineto{\pgfqpoint{4.606261in}{3.813333in}}%
\pgfpathlineto{\pgfqpoint{4.606940in}{3.812647in}}%
\pgfpathlineto{\pgfqpoint{4.607677in}{3.811917in}}%
\pgfpathlineto{\pgfqpoint{4.643513in}{3.776000in}}%
\pgfpathlineto{\pgfqpoint{4.647758in}{3.771743in}}%
\pgfpathlineto{\pgfqpoint{4.664757in}{3.754500in}}%
\pgfpathlineto{\pgfqpoint{4.680655in}{3.738667in}}%
\pgfpathlineto{\pgfqpoint{4.687838in}{3.731439in}}%
\pgfpathlineto{\pgfqpoint{4.703286in}{3.715722in}}%
\pgfpathlineto{\pgfqpoint{4.717687in}{3.701333in}}%
\pgfpathlineto{\pgfqpoint{4.727919in}{3.691006in}}%
\pgfpathlineto{\pgfqpoint{4.741753in}{3.676886in}}%
\pgfpathlineto{\pgfqpoint{4.754611in}{3.664000in}}%
\pgfpathlineto{\pgfqpoint{4.768000in}{3.650444in}}%
\pgfusepath{fill}%
\end{pgfscope}%
\begin{pgfscope}%
\pgfpathrectangle{\pgfqpoint{0.800000in}{0.528000in}}{\pgfqpoint{3.968000in}{3.696000in}}%
\pgfusepath{clip}%
\pgfsetbuttcap%
\pgfsetroundjoin%
\definecolor{currentfill}{rgb}{0.386433,0.794644,0.372886}%
\pgfsetfillcolor{currentfill}%
\pgfsetlinewidth{0.000000pt}%
\definecolor{currentstroke}{rgb}{0.000000,0.000000,0.000000}%
\pgfsetstrokecolor{currentstroke}%
\pgfsetdash{}{0pt}%
\pgfpathmoveto{\pgfqpoint{4.768000in}{3.655866in}}%
\pgfpathlineto{\pgfqpoint{4.759966in}{3.664000in}}%
\pgfpathlineto{\pgfqpoint{4.744529in}{3.679471in}}%
\pgfpathlineto{\pgfqpoint{4.727919in}{3.696424in}}%
\pgfpathlineto{\pgfqpoint{4.723056in}{3.701333in}}%
\pgfpathlineto{\pgfqpoint{4.706064in}{3.718309in}}%
\pgfpathlineto{\pgfqpoint{4.687838in}{3.736854in}}%
\pgfpathlineto{\pgfqpoint{4.686036in}{3.738667in}}%
\pgfpathlineto{\pgfqpoint{4.667537in}{3.757091in}}%
\pgfpathlineto{\pgfqpoint{4.648895in}{3.776000in}}%
\pgfpathlineto{\pgfqpoint{4.647758in}{3.777142in}}%
\pgfpathlineto{\pgfqpoint{4.611624in}{3.813333in}}%
\pgfpathlineto{\pgfqpoint{4.609712in}{3.815229in}}%
\pgfpathlineto{\pgfqpoint{4.607677in}{3.817284in}}%
\pgfpathlineto{\pgfqpoint{4.574243in}{3.850667in}}%
\pgfpathlineto{\pgfqpoint{4.571018in}{3.853854in}}%
\pgfpathlineto{\pgfqpoint{4.567596in}{3.857299in}}%
\pgfpathlineto{\pgfqpoint{4.536751in}{3.888000in}}%
\pgfpathlineto{\pgfqpoint{4.527515in}{3.897187in}}%
\pgfpathlineto{\pgfqpoint{4.512816in}{3.911642in}}%
\pgfpathlineto{\pgfqpoint{4.499147in}{3.925333in}}%
\pgfpathlineto{\pgfqpoint{4.493446in}{3.930933in}}%
\pgfpathlineto{\pgfqpoint{4.487434in}{3.936947in}}%
\pgfpathlineto{\pgfqpoint{4.473981in}{3.950136in}}%
\pgfpathlineto{\pgfqpoint{4.461431in}{3.962667in}}%
\pgfpathlineto{\pgfqpoint{4.447354in}{3.976581in}}%
\pgfpathlineto{\pgfqpoint{4.435084in}{3.988571in}}%
\pgfpathlineto{\pgfqpoint{4.423601in}{4.000000in}}%
\pgfpathlineto{\pgfqpoint{4.407273in}{4.016089in}}%
\pgfpathlineto{\pgfqpoint{4.396124in}{4.026949in}}%
\pgfpathlineto{\pgfqpoint{4.385657in}{4.037333in}}%
\pgfpathlineto{\pgfqpoint{4.376626in}{4.046121in}}%
\pgfpathlineto{\pgfqpoint{4.367192in}{4.055470in}}%
\pgfpathlineto{\pgfqpoint{4.357101in}{4.065268in}}%
\pgfpathlineto{\pgfqpoint{4.347598in}{4.074667in}}%
\pgfpathlineto{\pgfqpoint{4.327111in}{4.094726in}}%
\pgfpathlineto{\pgfqpoint{4.318016in}{4.103528in}}%
\pgfpathlineto{\pgfqpoint{4.309423in}{4.112000in}}%
\pgfpathlineto{\pgfqpoint{4.287030in}{4.133855in}}%
\pgfpathlineto{\pgfqpoint{4.278868in}{4.141730in}}%
\pgfpathlineto{\pgfqpoint{4.271132in}{4.149333in}}%
\pgfpathlineto{\pgfqpoint{4.259247in}{4.160788in}}%
\pgfpathlineto{\pgfqpoint{4.246949in}{4.172860in}}%
\pgfpathlineto{\pgfqpoint{4.239656in}{4.179874in}}%
\pgfpathlineto{\pgfqpoint{4.232722in}{4.186667in}}%
\pgfpathlineto{\pgfqpoint{4.206869in}{4.211739in}}%
\pgfpathlineto{\pgfqpoint{4.200382in}{4.217958in}}%
\pgfpathlineto{\pgfqpoint{4.194194in}{4.224000in}}%
\pgfpathlineto{\pgfqpoint{4.191446in}{4.224000in}}%
\pgfpathlineto{\pgfqpoint{4.198975in}{4.216647in}}%
\pgfpathlineto{\pgfqpoint{4.206869in}{4.209081in}}%
\pgfpathlineto{\pgfqpoint{4.229981in}{4.186667in}}%
\pgfpathlineto{\pgfqpoint{4.238251in}{4.178565in}}%
\pgfpathlineto{\pgfqpoint{4.246949in}{4.170200in}}%
\pgfpathlineto{\pgfqpoint{4.257857in}{4.159493in}}%
\pgfpathlineto{\pgfqpoint{4.268397in}{4.149333in}}%
\pgfpathlineto{\pgfqpoint{4.277464in}{4.140423in}}%
\pgfpathlineto{\pgfqpoint{4.287030in}{4.131193in}}%
\pgfpathlineto{\pgfqpoint{4.306696in}{4.112000in}}%
\pgfpathlineto{\pgfqpoint{4.316614in}{4.102222in}}%
\pgfpathlineto{\pgfqpoint{4.327111in}{4.092062in}}%
\pgfpathlineto{\pgfqpoint{4.344878in}{4.074667in}}%
\pgfpathlineto{\pgfqpoint{4.355700in}{4.063963in}}%
\pgfpathlineto{\pgfqpoint{4.367192in}{4.052805in}}%
\pgfpathlineto{\pgfqpoint{4.375240in}{4.044830in}}%
\pgfpathlineto{\pgfqpoint{4.382944in}{4.037333in}}%
\pgfpathlineto{\pgfqpoint{4.394724in}{4.025645in}}%
\pgfpathlineto{\pgfqpoint{4.407273in}{4.013421in}}%
\pgfpathlineto{\pgfqpoint{4.420894in}{4.000000in}}%
\pgfpathlineto{\pgfqpoint{4.433685in}{3.987269in}}%
\pgfpathlineto{\pgfqpoint{4.447354in}{3.973912in}}%
\pgfpathlineto{\pgfqpoint{4.458731in}{3.962667in}}%
\pgfpathlineto{\pgfqpoint{4.472584in}{3.948834in}}%
\pgfpathlineto{\pgfqpoint{4.487434in}{3.934276in}}%
\pgfpathlineto{\pgfqpoint{4.492064in}{3.929645in}}%
\pgfpathlineto{\pgfqpoint{4.496454in}{3.925333in}}%
\pgfpathlineto{\pgfqpoint{4.511420in}{3.910342in}}%
\pgfpathlineto{\pgfqpoint{4.527515in}{3.894514in}}%
\pgfpathlineto{\pgfqpoint{4.534064in}{3.888000in}}%
\pgfpathlineto{\pgfqpoint{4.567596in}{3.854625in}}%
\pgfpathlineto{\pgfqpoint{4.569638in}{3.852569in}}%
\pgfpathlineto{\pgfqpoint{4.571563in}{3.850667in}}%
\pgfpathlineto{\pgfqpoint{4.607677in}{3.814608in}}%
\pgfpathlineto{\pgfqpoint{4.608333in}{3.813945in}}%
\pgfpathlineto{\pgfqpoint{4.608950in}{3.813333in}}%
\pgfpathlineto{\pgfqpoint{4.625860in}{3.796396in}}%
\pgfpathlineto{\pgfqpoint{4.646210in}{3.776000in}}%
\pgfpathlineto{\pgfqpoint{4.647758in}{3.774448in}}%
\pgfpathlineto{\pgfqpoint{4.666147in}{3.755795in}}%
\pgfpathlineto{\pgfqpoint{4.683346in}{3.738667in}}%
\pgfpathlineto{\pgfqpoint{4.687838in}{3.734147in}}%
\pgfpathlineto{\pgfqpoint{4.704675in}{3.717015in}}%
\pgfpathlineto{\pgfqpoint{4.720371in}{3.701333in}}%
\pgfpathlineto{\pgfqpoint{4.727919in}{3.693715in}}%
\pgfpathlineto{\pgfqpoint{4.743141in}{3.678179in}}%
\pgfpathlineto{\pgfqpoint{4.757288in}{3.664000in}}%
\pgfpathlineto{\pgfqpoint{4.768000in}{3.653155in}}%
\pgfusepath{fill}%
\end{pgfscope}%
\begin{pgfscope}%
\pgfpathrectangle{\pgfqpoint{0.800000in}{0.528000in}}{\pgfqpoint{3.968000in}{3.696000in}}%
\pgfusepath{clip}%
\pgfsetbuttcap%
\pgfsetroundjoin%
\definecolor{currentfill}{rgb}{0.395174,0.797475,0.367757}%
\pgfsetfillcolor{currentfill}%
\pgfsetlinewidth{0.000000pt}%
\definecolor{currentstroke}{rgb}{0.000000,0.000000,0.000000}%
\pgfsetstrokecolor{currentstroke}%
\pgfsetdash{}{0pt}%
\pgfpathmoveto{\pgfqpoint{4.768000in}{3.658577in}}%
\pgfpathlineto{\pgfqpoint{4.762643in}{3.664000in}}%
\pgfpathlineto{\pgfqpoint{4.745917in}{3.680764in}}%
\pgfpathlineto{\pgfqpoint{4.727919in}{3.699134in}}%
\pgfpathlineto{\pgfqpoint{4.725740in}{3.701333in}}%
\pgfpathlineto{\pgfqpoint{4.707453in}{3.719603in}}%
\pgfpathlineto{\pgfqpoint{4.688717in}{3.738667in}}%
\pgfpathlineto{\pgfqpoint{4.687838in}{3.739552in}}%
\pgfpathlineto{\pgfqpoint{4.668928in}{3.758386in}}%
\pgfpathlineto{\pgfqpoint{4.651562in}{3.776000in}}%
\pgfpathlineto{\pgfqpoint{4.647758in}{3.779820in}}%
\pgfpathlineto{\pgfqpoint{4.614298in}{3.813333in}}%
\pgfpathlineto{\pgfqpoint{4.611091in}{3.816513in}}%
\pgfpathlineto{\pgfqpoint{4.607677in}{3.819960in}}%
\pgfpathlineto{\pgfqpoint{4.576923in}{3.850667in}}%
\pgfpathlineto{\pgfqpoint{4.572398in}{3.855140in}}%
\pgfpathlineto{\pgfqpoint{4.567596in}{3.859973in}}%
\pgfpathlineto{\pgfqpoint{4.539438in}{3.888000in}}%
\pgfpathlineto{\pgfqpoint{4.527515in}{3.899859in}}%
\pgfpathlineto{\pgfqpoint{4.514212in}{3.912942in}}%
\pgfpathlineto{\pgfqpoint{4.501841in}{3.925333in}}%
\pgfpathlineto{\pgfqpoint{4.494829in}{3.932221in}}%
\pgfpathlineto{\pgfqpoint{4.487434in}{3.939618in}}%
\pgfpathlineto{\pgfqpoint{4.475378in}{3.951437in}}%
\pgfpathlineto{\pgfqpoint{4.464131in}{3.962667in}}%
\pgfpathlineto{\pgfqpoint{4.447354in}{3.979250in}}%
\pgfpathlineto{\pgfqpoint{4.436482in}{3.989874in}}%
\pgfpathlineto{\pgfqpoint{4.426308in}{4.000000in}}%
\pgfpathlineto{\pgfqpoint{4.407273in}{4.018756in}}%
\pgfpathlineto{\pgfqpoint{4.397524in}{4.028252in}}%
\pgfpathlineto{\pgfqpoint{4.388371in}{4.037333in}}%
\pgfpathlineto{\pgfqpoint{4.378013in}{4.047413in}}%
\pgfpathlineto{\pgfqpoint{4.367192in}{4.058136in}}%
\pgfpathlineto{\pgfqpoint{4.358502in}{4.066573in}}%
\pgfpathlineto{\pgfqpoint{4.350319in}{4.074667in}}%
\pgfpathlineto{\pgfqpoint{4.327111in}{4.097389in}}%
\pgfpathlineto{\pgfqpoint{4.319419in}{4.104835in}}%
\pgfpathlineto{\pgfqpoint{4.312151in}{4.112000in}}%
\pgfpathlineto{\pgfqpoint{4.287030in}{4.136517in}}%
\pgfpathlineto{\pgfqpoint{4.280272in}{4.143038in}}%
\pgfpathlineto{\pgfqpoint{4.273866in}{4.149333in}}%
\pgfpathlineto{\pgfqpoint{4.260638in}{4.162083in}}%
\pgfpathlineto{\pgfqpoint{4.246949in}{4.175520in}}%
\pgfpathlineto{\pgfqpoint{4.241062in}{4.181182in}}%
\pgfpathlineto{\pgfqpoint{4.235463in}{4.186667in}}%
\pgfpathlineto{\pgfqpoint{4.206869in}{4.214397in}}%
\pgfpathlineto{\pgfqpoint{4.201788in}{4.219268in}}%
\pgfpathlineto{\pgfqpoint{4.196942in}{4.224000in}}%
\pgfpathlineto{\pgfqpoint{4.194194in}{4.224000in}}%
\pgfpathlineto{\pgfqpoint{4.200382in}{4.217958in}}%
\pgfpathlineto{\pgfqpoint{4.206869in}{4.211739in}}%
\pgfpathlineto{\pgfqpoint{4.232722in}{4.186667in}}%
\pgfpathlineto{\pgfqpoint{4.239656in}{4.179874in}}%
\pgfpathlineto{\pgfqpoint{4.246949in}{4.172860in}}%
\pgfpathlineto{\pgfqpoint{4.259247in}{4.160788in}}%
\pgfpathlineto{\pgfqpoint{4.271132in}{4.149333in}}%
\pgfpathlineto{\pgfqpoint{4.278868in}{4.141730in}}%
\pgfpathlineto{\pgfqpoint{4.287030in}{4.133855in}}%
\pgfpathlineto{\pgfqpoint{4.309423in}{4.112000in}}%
\pgfpathlineto{\pgfqpoint{4.318016in}{4.103528in}}%
\pgfpathlineto{\pgfqpoint{4.327111in}{4.094726in}}%
\pgfpathlineto{\pgfqpoint{4.347598in}{4.074667in}}%
\pgfpathlineto{\pgfqpoint{4.357101in}{4.065268in}}%
\pgfpathlineto{\pgfqpoint{4.367192in}{4.055470in}}%
\pgfpathlineto{\pgfqpoint{4.376626in}{4.046121in}}%
\pgfpathlineto{\pgfqpoint{4.385657in}{4.037333in}}%
\pgfpathlineto{\pgfqpoint{4.396124in}{4.026949in}}%
\pgfpathlineto{\pgfqpoint{4.407273in}{4.016089in}}%
\pgfpathlineto{\pgfqpoint{4.423601in}{4.000000in}}%
\pgfpathlineto{\pgfqpoint{4.435084in}{3.988571in}}%
\pgfpathlineto{\pgfqpoint{4.447354in}{3.976581in}}%
\pgfpathlineto{\pgfqpoint{4.461431in}{3.962667in}}%
\pgfpathlineto{\pgfqpoint{4.473981in}{3.950136in}}%
\pgfpathlineto{\pgfqpoint{4.487434in}{3.936947in}}%
\pgfpathlineto{\pgfqpoint{4.493446in}{3.930933in}}%
\pgfpathlineto{\pgfqpoint{4.499147in}{3.925333in}}%
\pgfpathlineto{\pgfqpoint{4.512816in}{3.911642in}}%
\pgfpathlineto{\pgfqpoint{4.527515in}{3.897187in}}%
\pgfpathlineto{\pgfqpoint{4.536751in}{3.888000in}}%
\pgfpathlineto{\pgfqpoint{4.567596in}{3.857299in}}%
\pgfpathlineto{\pgfqpoint{4.571018in}{3.853854in}}%
\pgfpathlineto{\pgfqpoint{4.574243in}{3.850667in}}%
\pgfpathlineto{\pgfqpoint{4.607677in}{3.817284in}}%
\pgfpathlineto{\pgfqpoint{4.609712in}{3.815229in}}%
\pgfpathlineto{\pgfqpoint{4.611624in}{3.813333in}}%
\pgfpathlineto{\pgfqpoint{4.647758in}{3.777142in}}%
\pgfpathlineto{\pgfqpoint{4.648895in}{3.776000in}}%
\pgfpathlineto{\pgfqpoint{4.667537in}{3.757091in}}%
\pgfpathlineto{\pgfqpoint{4.686036in}{3.738667in}}%
\pgfpathlineto{\pgfqpoint{4.687838in}{3.736854in}}%
\pgfpathlineto{\pgfqpoint{4.706064in}{3.718309in}}%
\pgfpathlineto{\pgfqpoint{4.723056in}{3.701333in}}%
\pgfpathlineto{\pgfqpoint{4.727919in}{3.696424in}}%
\pgfpathlineto{\pgfqpoint{4.744529in}{3.679471in}}%
\pgfpathlineto{\pgfqpoint{4.759966in}{3.664000in}}%
\pgfpathlineto{\pgfqpoint{4.768000in}{3.655866in}}%
\pgfusepath{fill}%
\end{pgfscope}%
\begin{pgfscope}%
\pgfpathrectangle{\pgfqpoint{0.800000in}{0.528000in}}{\pgfqpoint{3.968000in}{3.696000in}}%
\pgfusepath{clip}%
\pgfsetbuttcap%
\pgfsetroundjoin%
\definecolor{currentfill}{rgb}{0.395174,0.797475,0.367757}%
\pgfsetfillcolor{currentfill}%
\pgfsetlinewidth{0.000000pt}%
\definecolor{currentstroke}{rgb}{0.000000,0.000000,0.000000}%
\pgfsetstrokecolor{currentstroke}%
\pgfsetdash{}{0pt}%
\pgfpathmoveto{\pgfqpoint{4.768000in}{3.661287in}}%
\pgfpathlineto{\pgfqpoint{4.765321in}{3.664000in}}%
\pgfpathlineto{\pgfqpoint{4.747304in}{3.682056in}}%
\pgfpathlineto{\pgfqpoint{4.728418in}{3.701333in}}%
\pgfpathlineto{\pgfqpoint{4.727919in}{3.701837in}}%
\pgfpathlineto{\pgfqpoint{4.708842in}{3.720897in}}%
\pgfpathlineto{\pgfqpoint{4.691378in}{3.738667in}}%
\pgfpathlineto{\pgfqpoint{4.687838in}{3.742232in}}%
\pgfpathlineto{\pgfqpoint{4.670318in}{3.759681in}}%
\pgfpathlineto{\pgfqpoint{4.654229in}{3.776000in}}%
\pgfpathlineto{\pgfqpoint{4.647758in}{3.782498in}}%
\pgfpathlineto{\pgfqpoint{4.616971in}{3.813333in}}%
\pgfpathlineto{\pgfqpoint{4.612469in}{3.817797in}}%
\pgfpathlineto{\pgfqpoint{4.607677in}{3.822637in}}%
\pgfpathlineto{\pgfqpoint{4.579604in}{3.850667in}}%
\pgfpathlineto{\pgfqpoint{4.573778in}{3.856425in}}%
\pgfpathlineto{\pgfqpoint{4.567596in}{3.862648in}}%
\pgfpathlineto{\pgfqpoint{4.542125in}{3.888000in}}%
\pgfpathlineto{\pgfqpoint{4.527515in}{3.902532in}}%
\pgfpathlineto{\pgfqpoint{4.515608in}{3.914242in}}%
\pgfpathlineto{\pgfqpoint{4.504534in}{3.925333in}}%
\pgfpathlineto{\pgfqpoint{4.496212in}{3.933509in}}%
\pgfpathlineto{\pgfqpoint{4.487434in}{3.942289in}}%
\pgfpathlineto{\pgfqpoint{4.476775in}{3.952738in}}%
\pgfpathlineto{\pgfqpoint{4.466832in}{3.962667in}}%
\pgfpathlineto{\pgfqpoint{4.447354in}{3.981919in}}%
\pgfpathlineto{\pgfqpoint{4.437880in}{3.991176in}}%
\pgfpathlineto{\pgfqpoint{4.429015in}{4.000000in}}%
\pgfpathlineto{\pgfqpoint{4.407273in}{4.021423in}}%
\pgfpathlineto{\pgfqpoint{4.398923in}{4.029556in}}%
\pgfpathlineto{\pgfqpoint{4.391085in}{4.037333in}}%
\pgfpathlineto{\pgfqpoint{4.379399in}{4.048704in}}%
\pgfpathlineto{\pgfqpoint{4.367192in}{4.060801in}}%
\pgfpathlineto{\pgfqpoint{4.359904in}{4.067878in}}%
\pgfpathlineto{\pgfqpoint{4.353040in}{4.074667in}}%
\pgfpathlineto{\pgfqpoint{4.327111in}{4.100053in}}%
\pgfpathlineto{\pgfqpoint{4.320821in}{4.106141in}}%
\pgfpathlineto{\pgfqpoint{4.314878in}{4.112000in}}%
\pgfpathlineto{\pgfqpoint{4.287030in}{4.139179in}}%
\pgfpathlineto{\pgfqpoint{4.281675in}{4.144346in}}%
\pgfpathlineto{\pgfqpoint{4.276600in}{4.149333in}}%
\pgfpathlineto{\pgfqpoint{4.262028in}{4.163378in}}%
\pgfpathlineto{\pgfqpoint{4.246949in}{4.178180in}}%
\pgfpathlineto{\pgfqpoint{4.242467in}{4.182491in}}%
\pgfpathlineto{\pgfqpoint{4.238204in}{4.186667in}}%
\pgfpathlineto{\pgfqpoint{4.206869in}{4.217056in}}%
\pgfpathlineto{\pgfqpoint{4.203195in}{4.220578in}}%
\pgfpathlineto{\pgfqpoint{4.199690in}{4.224000in}}%
\pgfpathlineto{\pgfqpoint{4.196942in}{4.224000in}}%
\pgfpathlineto{\pgfqpoint{4.201788in}{4.219268in}}%
\pgfpathlineto{\pgfqpoint{4.206869in}{4.214397in}}%
\pgfpathlineto{\pgfqpoint{4.235463in}{4.186667in}}%
\pgfpathlineto{\pgfqpoint{4.241062in}{4.181182in}}%
\pgfpathlineto{\pgfqpoint{4.246949in}{4.175520in}}%
\pgfpathlineto{\pgfqpoint{4.260638in}{4.162083in}}%
\pgfpathlineto{\pgfqpoint{4.273866in}{4.149333in}}%
\pgfpathlineto{\pgfqpoint{4.280272in}{4.143038in}}%
\pgfpathlineto{\pgfqpoint{4.287030in}{4.136517in}}%
\pgfpathlineto{\pgfqpoint{4.312151in}{4.112000in}}%
\pgfpathlineto{\pgfqpoint{4.319419in}{4.104835in}}%
\pgfpathlineto{\pgfqpoint{4.327111in}{4.097389in}}%
\pgfpathlineto{\pgfqpoint{4.350319in}{4.074667in}}%
\pgfpathlineto{\pgfqpoint{4.358502in}{4.066573in}}%
\pgfpathlineto{\pgfqpoint{4.367192in}{4.058136in}}%
\pgfpathlineto{\pgfqpoint{4.378013in}{4.047413in}}%
\pgfpathlineto{\pgfqpoint{4.388371in}{4.037333in}}%
\pgfpathlineto{\pgfqpoint{4.397524in}{4.028252in}}%
\pgfpathlineto{\pgfqpoint{4.407273in}{4.018756in}}%
\pgfpathlineto{\pgfqpoint{4.426308in}{4.000000in}}%
\pgfpathlineto{\pgfqpoint{4.436482in}{3.989874in}}%
\pgfpathlineto{\pgfqpoint{4.447354in}{3.979250in}}%
\pgfpathlineto{\pgfqpoint{4.464131in}{3.962667in}}%
\pgfpathlineto{\pgfqpoint{4.475378in}{3.951437in}}%
\pgfpathlineto{\pgfqpoint{4.487434in}{3.939618in}}%
\pgfpathlineto{\pgfqpoint{4.494829in}{3.932221in}}%
\pgfpathlineto{\pgfqpoint{4.501841in}{3.925333in}}%
\pgfpathlineto{\pgfqpoint{4.514212in}{3.912942in}}%
\pgfpathlineto{\pgfqpoint{4.527515in}{3.899859in}}%
\pgfpathlineto{\pgfqpoint{4.539438in}{3.888000in}}%
\pgfpathlineto{\pgfqpoint{4.567596in}{3.859973in}}%
\pgfpathlineto{\pgfqpoint{4.572398in}{3.855140in}}%
\pgfpathlineto{\pgfqpoint{4.576923in}{3.850667in}}%
\pgfpathlineto{\pgfqpoint{4.607677in}{3.819960in}}%
\pgfpathlineto{\pgfqpoint{4.611091in}{3.816513in}}%
\pgfpathlineto{\pgfqpoint{4.614298in}{3.813333in}}%
\pgfpathlineto{\pgfqpoint{4.647758in}{3.779820in}}%
\pgfpathlineto{\pgfqpoint{4.651562in}{3.776000in}}%
\pgfpathlineto{\pgfqpoint{4.668928in}{3.758386in}}%
\pgfpathlineto{\pgfqpoint{4.687838in}{3.739552in}}%
\pgfpathlineto{\pgfqpoint{4.688717in}{3.738667in}}%
\pgfpathlineto{\pgfqpoint{4.707453in}{3.719603in}}%
\pgfpathlineto{\pgfqpoint{4.725740in}{3.701333in}}%
\pgfpathlineto{\pgfqpoint{4.727919in}{3.699134in}}%
\pgfpathlineto{\pgfqpoint{4.745917in}{3.680764in}}%
\pgfpathlineto{\pgfqpoint{4.762643in}{3.664000in}}%
\pgfpathlineto{\pgfqpoint{4.768000in}{3.658577in}}%
\pgfusepath{fill}%
\end{pgfscope}%
\begin{pgfscope}%
\pgfpathrectangle{\pgfqpoint{0.800000in}{0.528000in}}{\pgfqpoint{3.968000in}{3.696000in}}%
\pgfusepath{clip}%
\pgfsetbuttcap%
\pgfsetroundjoin%
\definecolor{currentfill}{rgb}{0.395174,0.797475,0.367757}%
\pgfsetfillcolor{currentfill}%
\pgfsetlinewidth{0.000000pt}%
\definecolor{currentstroke}{rgb}{0.000000,0.000000,0.000000}%
\pgfsetstrokecolor{currentstroke}%
\pgfsetdash{}{0pt}%
\pgfpathmoveto{\pgfqpoint{4.768000in}{3.663998in}}%
\pgfpathlineto{\pgfqpoint{4.767998in}{3.664000in}}%
\pgfpathlineto{\pgfqpoint{4.748692in}{3.683349in}}%
\pgfpathlineto{\pgfqpoint{4.731072in}{3.701333in}}%
\pgfpathlineto{\pgfqpoint{4.727919in}{3.704519in}}%
\pgfpathlineto{\pgfqpoint{4.710231in}{3.722191in}}%
\pgfpathlineto{\pgfqpoint{4.694038in}{3.738667in}}%
\pgfpathlineto{\pgfqpoint{4.687838in}{3.744911in}}%
\pgfpathlineto{\pgfqpoint{4.671708in}{3.760976in}}%
\pgfpathlineto{\pgfqpoint{4.656896in}{3.776000in}}%
\pgfpathlineto{\pgfqpoint{4.647758in}{3.785176in}}%
\pgfpathlineto{\pgfqpoint{4.619645in}{3.813333in}}%
\pgfpathlineto{\pgfqpoint{4.613848in}{3.819082in}}%
\pgfpathlineto{\pgfqpoint{4.607677in}{3.825313in}}%
\pgfpathlineto{\pgfqpoint{4.582284in}{3.850667in}}%
\pgfpathlineto{\pgfqpoint{4.575158in}{3.857710in}}%
\pgfpathlineto{\pgfqpoint{4.567596in}{3.865322in}}%
\pgfpathlineto{\pgfqpoint{4.544812in}{3.888000in}}%
\pgfpathlineto{\pgfqpoint{4.527515in}{3.905204in}}%
\pgfpathlineto{\pgfqpoint{4.517003in}{3.915542in}}%
\pgfpathlineto{\pgfqpoint{4.507228in}{3.925333in}}%
\pgfpathlineto{\pgfqpoint{4.497594in}{3.934797in}}%
\pgfpathlineto{\pgfqpoint{4.487434in}{3.944960in}}%
\pgfpathlineto{\pgfqpoint{4.478172in}{3.954039in}}%
\pgfpathlineto{\pgfqpoint{4.469532in}{3.962667in}}%
\pgfpathlineto{\pgfqpoint{4.447354in}{3.984588in}}%
\pgfpathlineto{\pgfqpoint{4.439279in}{3.992479in}}%
\pgfpathlineto{\pgfqpoint{4.431722in}{4.000000in}}%
\pgfpathlineto{\pgfqpoint{4.407273in}{4.024091in}}%
\pgfpathlineto{\pgfqpoint{4.400323in}{4.030860in}}%
\pgfpathlineto{\pgfqpoint{4.393799in}{4.037333in}}%
\pgfpathlineto{\pgfqpoint{4.380786in}{4.049995in}}%
\pgfpathlineto{\pgfqpoint{4.367192in}{4.063467in}}%
\pgfpathlineto{\pgfqpoint{4.361305in}{4.069183in}}%
\pgfpathlineto{\pgfqpoint{4.355760in}{4.074667in}}%
\pgfpathlineto{\pgfqpoint{4.327111in}{4.102717in}}%
\pgfpathlineto{\pgfqpoint{4.322223in}{4.107447in}}%
\pgfpathlineto{\pgfqpoint{4.317606in}{4.112000in}}%
\pgfpathlineto{\pgfqpoint{4.287030in}{4.141841in}}%
\pgfpathlineto{\pgfqpoint{4.283079in}{4.145653in}}%
\pgfpathlineto{\pgfqpoint{4.279335in}{4.149333in}}%
\pgfpathlineto{\pgfqpoint{4.263419in}{4.164674in}}%
\pgfpathlineto{\pgfqpoint{4.246949in}{4.180840in}}%
\pgfpathlineto{\pgfqpoint{4.243872in}{4.183800in}}%
\pgfpathlineto{\pgfqpoint{4.240946in}{4.186667in}}%
\pgfpathlineto{\pgfqpoint{4.206869in}{4.219714in}}%
\pgfpathlineto{\pgfqpoint{4.204601in}{4.221888in}}%
\pgfpathlineto{\pgfqpoint{4.202438in}{4.224000in}}%
\pgfpathlineto{\pgfqpoint{4.199690in}{4.224000in}}%
\pgfpathlineto{\pgfqpoint{4.203195in}{4.220578in}}%
\pgfpathlineto{\pgfqpoint{4.206869in}{4.217056in}}%
\pgfpathlineto{\pgfqpoint{4.238204in}{4.186667in}}%
\pgfpathlineto{\pgfqpoint{4.242467in}{4.182491in}}%
\pgfpathlineto{\pgfqpoint{4.246949in}{4.178180in}}%
\pgfpathlineto{\pgfqpoint{4.262028in}{4.163378in}}%
\pgfpathlineto{\pgfqpoint{4.276600in}{4.149333in}}%
\pgfpathlineto{\pgfqpoint{4.281675in}{4.144346in}}%
\pgfpathlineto{\pgfqpoint{4.287030in}{4.139179in}}%
\pgfpathlineto{\pgfqpoint{4.314878in}{4.112000in}}%
\pgfpathlineto{\pgfqpoint{4.320821in}{4.106141in}}%
\pgfpathlineto{\pgfqpoint{4.327111in}{4.100053in}}%
\pgfpathlineto{\pgfqpoint{4.353040in}{4.074667in}}%
\pgfpathlineto{\pgfqpoint{4.359904in}{4.067878in}}%
\pgfpathlineto{\pgfqpoint{4.367192in}{4.060801in}}%
\pgfpathlineto{\pgfqpoint{4.379399in}{4.048704in}}%
\pgfpathlineto{\pgfqpoint{4.391085in}{4.037333in}}%
\pgfpathlineto{\pgfqpoint{4.398923in}{4.029556in}}%
\pgfpathlineto{\pgfqpoint{4.407273in}{4.021423in}}%
\pgfpathlineto{\pgfqpoint{4.429015in}{4.000000in}}%
\pgfpathlineto{\pgfqpoint{4.437880in}{3.991176in}}%
\pgfpathlineto{\pgfqpoint{4.447354in}{3.981919in}}%
\pgfpathlineto{\pgfqpoint{4.466832in}{3.962667in}}%
\pgfpathlineto{\pgfqpoint{4.476775in}{3.952738in}}%
\pgfpathlineto{\pgfqpoint{4.487434in}{3.942289in}}%
\pgfpathlineto{\pgfqpoint{4.496212in}{3.933509in}}%
\pgfpathlineto{\pgfqpoint{4.504534in}{3.925333in}}%
\pgfpathlineto{\pgfqpoint{4.515608in}{3.914242in}}%
\pgfpathlineto{\pgfqpoint{4.527515in}{3.902532in}}%
\pgfpathlineto{\pgfqpoint{4.542125in}{3.888000in}}%
\pgfpathlineto{\pgfqpoint{4.567596in}{3.862648in}}%
\pgfpathlineto{\pgfqpoint{4.573778in}{3.856425in}}%
\pgfpathlineto{\pgfqpoint{4.579604in}{3.850667in}}%
\pgfpathlineto{\pgfqpoint{4.607677in}{3.822637in}}%
\pgfpathlineto{\pgfqpoint{4.612469in}{3.817797in}}%
\pgfpathlineto{\pgfqpoint{4.616971in}{3.813333in}}%
\pgfpathlineto{\pgfqpoint{4.647758in}{3.782498in}}%
\pgfpathlineto{\pgfqpoint{4.654229in}{3.776000in}}%
\pgfpathlineto{\pgfqpoint{4.670318in}{3.759681in}}%
\pgfpathlineto{\pgfqpoint{4.687838in}{3.742232in}}%
\pgfpathlineto{\pgfqpoint{4.691378in}{3.738667in}}%
\pgfpathlineto{\pgfqpoint{4.708842in}{3.720897in}}%
\pgfpathlineto{\pgfqpoint{4.727919in}{3.701837in}}%
\pgfpathlineto{\pgfqpoint{4.728418in}{3.701333in}}%
\pgfpathlineto{\pgfqpoint{4.747304in}{3.682056in}}%
\pgfpathlineto{\pgfqpoint{4.765321in}{3.664000in}}%
\pgfpathlineto{\pgfqpoint{4.768000in}{3.661287in}}%
\pgfusepath{fill}%
\end{pgfscope}%
\begin{pgfscope}%
\pgfpathrectangle{\pgfqpoint{0.800000in}{0.528000in}}{\pgfqpoint{3.968000in}{3.696000in}}%
\pgfusepath{clip}%
\pgfsetbuttcap%
\pgfsetroundjoin%
\definecolor{currentfill}{rgb}{0.395174,0.797475,0.367757}%
\pgfsetfillcolor{currentfill}%
\pgfsetlinewidth{0.000000pt}%
\definecolor{currentstroke}{rgb}{0.000000,0.000000,0.000000}%
\pgfsetstrokecolor{currentstroke}%
\pgfsetdash{}{0pt}%
\pgfpathmoveto{\pgfqpoint{4.768000in}{3.666682in}}%
\pgfpathlineto{\pgfqpoint{4.750080in}{3.684642in}}%
\pgfpathlineto{\pgfqpoint{4.733726in}{3.701333in}}%
\pgfpathlineto{\pgfqpoint{4.727919in}{3.707201in}}%
\pgfpathlineto{\pgfqpoint{4.711620in}{3.723485in}}%
\pgfpathlineto{\pgfqpoint{4.696699in}{3.738667in}}%
\pgfpathlineto{\pgfqpoint{4.687838in}{3.747591in}}%
\pgfpathlineto{\pgfqpoint{4.673099in}{3.762271in}}%
\pgfpathlineto{\pgfqpoint{4.659564in}{3.776000in}}%
\pgfpathlineto{\pgfqpoint{4.647758in}{3.787854in}}%
\pgfpathlineto{\pgfqpoint{4.622319in}{3.813333in}}%
\pgfpathlineto{\pgfqpoint{4.615227in}{3.820366in}}%
\pgfpathlineto{\pgfqpoint{4.607677in}{3.827989in}}%
\pgfpathlineto{\pgfqpoint{4.584964in}{3.850667in}}%
\pgfpathlineto{\pgfqpoint{4.576538in}{3.858996in}}%
\pgfpathlineto{\pgfqpoint{4.567596in}{3.867997in}}%
\pgfpathlineto{\pgfqpoint{4.547499in}{3.888000in}}%
\pgfpathlineto{\pgfqpoint{4.527515in}{3.907877in}}%
\pgfpathlineto{\pgfqpoint{4.518399in}{3.916842in}}%
\pgfpathlineto{\pgfqpoint{4.509922in}{3.925333in}}%
\pgfpathlineto{\pgfqpoint{4.498977in}{3.936084in}}%
\pgfpathlineto{\pgfqpoint{4.487434in}{3.947631in}}%
\pgfpathlineto{\pgfqpoint{4.479569in}{3.955341in}}%
\pgfpathlineto{\pgfqpoint{4.472232in}{3.962667in}}%
\pgfpathlineto{\pgfqpoint{4.447354in}{3.987257in}}%
\pgfpathlineto{\pgfqpoint{4.440677in}{3.993781in}}%
\pgfpathlineto{\pgfqpoint{4.434430in}{4.000000in}}%
\pgfpathlineto{\pgfqpoint{4.407273in}{4.026758in}}%
\pgfpathlineto{\pgfqpoint{4.401723in}{4.032164in}}%
\pgfpathlineto{\pgfqpoint{4.396513in}{4.037333in}}%
\pgfpathlineto{\pgfqpoint{4.382172in}{4.051287in}}%
\pgfpathlineto{\pgfqpoint{4.367192in}{4.066132in}}%
\pgfpathlineto{\pgfqpoint{4.362706in}{4.070488in}}%
\pgfpathlineto{\pgfqpoint{4.358481in}{4.074667in}}%
\pgfpathlineto{\pgfqpoint{4.327111in}{4.105380in}}%
\pgfpathlineto{\pgfqpoint{4.323626in}{4.108754in}}%
\pgfpathlineto{\pgfqpoint{4.320333in}{4.112000in}}%
\pgfpathlineto{\pgfqpoint{4.287030in}{4.144503in}}%
\pgfpathlineto{\pgfqpoint{4.284483in}{4.146961in}}%
\pgfpathlineto{\pgfqpoint{4.282069in}{4.149333in}}%
\pgfpathlineto{\pgfqpoint{4.264809in}{4.165969in}}%
\pgfpathlineto{\pgfqpoint{4.246949in}{4.183500in}}%
\pgfpathlineto{\pgfqpoint{4.245277in}{4.185109in}}%
\pgfpathlineto{\pgfqpoint{4.243687in}{4.186667in}}%
\pgfpathlineto{\pgfqpoint{4.206869in}{4.222373in}}%
\pgfpathlineto{\pgfqpoint{4.206008in}{4.223198in}}%
\pgfpathlineto{\pgfqpoint{4.205186in}{4.224000in}}%
\pgfpathlineto{\pgfqpoint{4.202438in}{4.224000in}}%
\pgfpathlineto{\pgfqpoint{4.204601in}{4.221888in}}%
\pgfpathlineto{\pgfqpoint{4.206869in}{4.219714in}}%
\pgfpathlineto{\pgfqpoint{4.240946in}{4.186667in}}%
\pgfpathlineto{\pgfqpoint{4.243872in}{4.183800in}}%
\pgfpathlineto{\pgfqpoint{4.246949in}{4.180840in}}%
\pgfpathlineto{\pgfqpoint{4.263419in}{4.164674in}}%
\pgfpathlineto{\pgfqpoint{4.279335in}{4.149333in}}%
\pgfpathlineto{\pgfqpoint{4.283079in}{4.145653in}}%
\pgfpathlineto{\pgfqpoint{4.287030in}{4.141841in}}%
\pgfpathlineto{\pgfqpoint{4.317606in}{4.112000in}}%
\pgfpathlineto{\pgfqpoint{4.322223in}{4.107447in}}%
\pgfpathlineto{\pgfqpoint{4.327111in}{4.102717in}}%
\pgfpathlineto{\pgfqpoint{4.355760in}{4.074667in}}%
\pgfpathlineto{\pgfqpoint{4.361305in}{4.069183in}}%
\pgfpathlineto{\pgfqpoint{4.367192in}{4.063467in}}%
\pgfpathlineto{\pgfqpoint{4.380786in}{4.049995in}}%
\pgfpathlineto{\pgfqpoint{4.393799in}{4.037333in}}%
\pgfpathlineto{\pgfqpoint{4.400323in}{4.030860in}}%
\pgfpathlineto{\pgfqpoint{4.407273in}{4.024091in}}%
\pgfpathlineto{\pgfqpoint{4.431722in}{4.000000in}}%
\pgfpathlineto{\pgfqpoint{4.439279in}{3.992479in}}%
\pgfpathlineto{\pgfqpoint{4.447354in}{3.984588in}}%
\pgfpathlineto{\pgfqpoint{4.469532in}{3.962667in}}%
\pgfpathlineto{\pgfqpoint{4.478172in}{3.954039in}}%
\pgfpathlineto{\pgfqpoint{4.487434in}{3.944960in}}%
\pgfpathlineto{\pgfqpoint{4.497594in}{3.934797in}}%
\pgfpathlineto{\pgfqpoint{4.507228in}{3.925333in}}%
\pgfpathlineto{\pgfqpoint{4.517003in}{3.915542in}}%
\pgfpathlineto{\pgfqpoint{4.527515in}{3.905204in}}%
\pgfpathlineto{\pgfqpoint{4.544812in}{3.888000in}}%
\pgfpathlineto{\pgfqpoint{4.567596in}{3.865322in}}%
\pgfpathlineto{\pgfqpoint{4.575158in}{3.857710in}}%
\pgfpathlineto{\pgfqpoint{4.582284in}{3.850667in}}%
\pgfpathlineto{\pgfqpoint{4.607677in}{3.825313in}}%
\pgfpathlineto{\pgfqpoint{4.613848in}{3.819082in}}%
\pgfpathlineto{\pgfqpoint{4.619645in}{3.813333in}}%
\pgfpathlineto{\pgfqpoint{4.647758in}{3.785176in}}%
\pgfpathlineto{\pgfqpoint{4.656896in}{3.776000in}}%
\pgfpathlineto{\pgfqpoint{4.671708in}{3.760976in}}%
\pgfpathlineto{\pgfqpoint{4.687838in}{3.744911in}}%
\pgfpathlineto{\pgfqpoint{4.694038in}{3.738667in}}%
\pgfpathlineto{\pgfqpoint{4.710231in}{3.722191in}}%
\pgfpathlineto{\pgfqpoint{4.727919in}{3.704519in}}%
\pgfpathlineto{\pgfqpoint{4.731072in}{3.701333in}}%
\pgfpathlineto{\pgfqpoint{4.748692in}{3.683349in}}%
\pgfpathlineto{\pgfqpoint{4.767998in}{3.664000in}}%
\pgfpathlineto{\pgfqpoint{4.768000in}{3.663998in}}%
\pgfpathlineto{\pgfqpoint{4.768000in}{3.664000in}}%
\pgfusepath{fill}%
\end{pgfscope}%
\begin{pgfscope}%
\pgfpathrectangle{\pgfqpoint{0.800000in}{0.528000in}}{\pgfqpoint{3.968000in}{3.696000in}}%
\pgfusepath{clip}%
\pgfsetbuttcap%
\pgfsetroundjoin%
\definecolor{currentfill}{rgb}{0.404001,0.800275,0.362552}%
\pgfsetfillcolor{currentfill}%
\pgfsetlinewidth{0.000000pt}%
\definecolor{currentstroke}{rgb}{0.000000,0.000000,0.000000}%
\pgfsetstrokecolor{currentstroke}%
\pgfsetdash{}{0pt}%
\pgfpathmoveto{\pgfqpoint{4.768000in}{3.669365in}}%
\pgfpathlineto{\pgfqpoint{4.751467in}{3.685934in}}%
\pgfpathlineto{\pgfqpoint{4.736380in}{3.701333in}}%
\pgfpathlineto{\pgfqpoint{4.727919in}{3.709882in}}%
\pgfpathlineto{\pgfqpoint{4.713009in}{3.724778in}}%
\pgfpathlineto{\pgfqpoint{4.699360in}{3.738667in}}%
\pgfpathlineto{\pgfqpoint{4.687838in}{3.750271in}}%
\pgfpathlineto{\pgfqpoint{4.674489in}{3.763566in}}%
\pgfpathlineto{\pgfqpoint{4.662231in}{3.776000in}}%
\pgfpathlineto{\pgfqpoint{4.647758in}{3.790532in}}%
\pgfpathlineto{\pgfqpoint{4.624993in}{3.813333in}}%
\pgfpathlineto{\pgfqpoint{4.616605in}{3.821650in}}%
\pgfpathlineto{\pgfqpoint{4.607677in}{3.830665in}}%
\pgfpathlineto{\pgfqpoint{4.587644in}{3.850667in}}%
\pgfpathlineto{\pgfqpoint{4.577918in}{3.860281in}}%
\pgfpathlineto{\pgfqpoint{4.567596in}{3.870671in}}%
\pgfpathlineto{\pgfqpoint{4.550186in}{3.888000in}}%
\pgfpathlineto{\pgfqpoint{4.527515in}{3.910550in}}%
\pgfpathlineto{\pgfqpoint{4.519795in}{3.918142in}}%
\pgfpathlineto{\pgfqpoint{4.512615in}{3.925333in}}%
\pgfpathlineto{\pgfqpoint{4.500359in}{3.937372in}}%
\pgfpathlineto{\pgfqpoint{4.487434in}{3.950301in}}%
\pgfpathlineto{\pgfqpoint{4.480966in}{3.956642in}}%
\pgfpathlineto{\pgfqpoint{4.474933in}{3.962667in}}%
\pgfpathlineto{\pgfqpoint{4.447354in}{3.989926in}}%
\pgfpathlineto{\pgfqpoint{4.442076in}{3.995084in}}%
\pgfpathlineto{\pgfqpoint{4.437137in}{4.000000in}}%
\pgfpathlineto{\pgfqpoint{4.407273in}{4.029425in}}%
\pgfpathlineto{\pgfqpoint{4.403123in}{4.033468in}}%
\pgfpathlineto{\pgfqpoint{4.399226in}{4.037333in}}%
\pgfpathlineto{\pgfqpoint{4.383559in}{4.052578in}}%
\pgfpathlineto{\pgfqpoint{4.367192in}{4.068798in}}%
\pgfpathlineto{\pgfqpoint{4.364107in}{4.071793in}}%
\pgfpathlineto{\pgfqpoint{4.361201in}{4.074667in}}%
\pgfpathlineto{\pgfqpoint{4.327111in}{4.108044in}}%
\pgfpathlineto{\pgfqpoint{4.325028in}{4.110060in}}%
\pgfpathlineto{\pgfqpoint{4.323061in}{4.112000in}}%
\pgfpathlineto{\pgfqpoint{4.287030in}{4.147165in}}%
\pgfpathlineto{\pgfqpoint{4.285887in}{4.148268in}}%
\pgfpathlineto{\pgfqpoint{4.284803in}{4.149333in}}%
\pgfpathlineto{\pgfqpoint{4.266200in}{4.167264in}}%
\pgfpathlineto{\pgfqpoint{4.246949in}{4.186161in}}%
\pgfpathlineto{\pgfqpoint{4.246682in}{4.186418in}}%
\pgfpathlineto{\pgfqpoint{4.246428in}{4.186667in}}%
\pgfpathlineto{\pgfqpoint{4.233760in}{4.198952in}}%
\pgfpathlineto{\pgfqpoint{4.207922in}{4.224000in}}%
\pgfpathlineto{\pgfqpoint{4.206869in}{4.224000in}}%
\pgfpathlineto{\pgfqpoint{4.205186in}{4.224000in}}%
\pgfpathlineto{\pgfqpoint{4.206008in}{4.223198in}}%
\pgfpathlineto{\pgfqpoint{4.206869in}{4.222373in}}%
\pgfpathlineto{\pgfqpoint{4.243687in}{4.186667in}}%
\pgfpathlineto{\pgfqpoint{4.245277in}{4.185109in}}%
\pgfpathlineto{\pgfqpoint{4.246949in}{4.183500in}}%
\pgfpathlineto{\pgfqpoint{4.264809in}{4.165969in}}%
\pgfpathlineto{\pgfqpoint{4.282069in}{4.149333in}}%
\pgfpathlineto{\pgfqpoint{4.284483in}{4.146961in}}%
\pgfpathlineto{\pgfqpoint{4.287030in}{4.144503in}}%
\pgfpathlineto{\pgfqpoint{4.320333in}{4.112000in}}%
\pgfpathlineto{\pgfqpoint{4.323626in}{4.108754in}}%
\pgfpathlineto{\pgfqpoint{4.327111in}{4.105380in}}%
\pgfpathlineto{\pgfqpoint{4.358481in}{4.074667in}}%
\pgfpathlineto{\pgfqpoint{4.362706in}{4.070488in}}%
\pgfpathlineto{\pgfqpoint{4.367192in}{4.066132in}}%
\pgfpathlineto{\pgfqpoint{4.382172in}{4.051287in}}%
\pgfpathlineto{\pgfqpoint{4.396513in}{4.037333in}}%
\pgfpathlineto{\pgfqpoint{4.401723in}{4.032164in}}%
\pgfpathlineto{\pgfqpoint{4.407273in}{4.026758in}}%
\pgfpathlineto{\pgfqpoint{4.434430in}{4.000000in}}%
\pgfpathlineto{\pgfqpoint{4.440677in}{3.993781in}}%
\pgfpathlineto{\pgfqpoint{4.447354in}{3.987257in}}%
\pgfpathlineto{\pgfqpoint{4.472232in}{3.962667in}}%
\pgfpathlineto{\pgfqpoint{4.479569in}{3.955341in}}%
\pgfpathlineto{\pgfqpoint{4.487434in}{3.947631in}}%
\pgfpathlineto{\pgfqpoint{4.498977in}{3.936084in}}%
\pgfpathlineto{\pgfqpoint{4.509922in}{3.925333in}}%
\pgfpathlineto{\pgfqpoint{4.518399in}{3.916842in}}%
\pgfpathlineto{\pgfqpoint{4.527515in}{3.907877in}}%
\pgfpathlineto{\pgfqpoint{4.547499in}{3.888000in}}%
\pgfpathlineto{\pgfqpoint{4.567596in}{3.867997in}}%
\pgfpathlineto{\pgfqpoint{4.576538in}{3.858996in}}%
\pgfpathlineto{\pgfqpoint{4.584964in}{3.850667in}}%
\pgfpathlineto{\pgfqpoint{4.607677in}{3.827989in}}%
\pgfpathlineto{\pgfqpoint{4.615227in}{3.820366in}}%
\pgfpathlineto{\pgfqpoint{4.622319in}{3.813333in}}%
\pgfpathlineto{\pgfqpoint{4.647758in}{3.787854in}}%
\pgfpathlineto{\pgfqpoint{4.659564in}{3.776000in}}%
\pgfpathlineto{\pgfqpoint{4.673099in}{3.762271in}}%
\pgfpathlineto{\pgfqpoint{4.687838in}{3.747591in}}%
\pgfpathlineto{\pgfqpoint{4.696699in}{3.738667in}}%
\pgfpathlineto{\pgfqpoint{4.711620in}{3.723485in}}%
\pgfpathlineto{\pgfqpoint{4.727919in}{3.707201in}}%
\pgfpathlineto{\pgfqpoint{4.733726in}{3.701333in}}%
\pgfpathlineto{\pgfqpoint{4.750080in}{3.684642in}}%
\pgfpathlineto{\pgfqpoint{4.768000in}{3.666682in}}%
\pgfusepath{fill}%
\end{pgfscope}%
\begin{pgfscope}%
\pgfpathrectangle{\pgfqpoint{0.800000in}{0.528000in}}{\pgfqpoint{3.968000in}{3.696000in}}%
\pgfusepath{clip}%
\pgfsetbuttcap%
\pgfsetroundjoin%
\definecolor{currentfill}{rgb}{0.404001,0.800275,0.362552}%
\pgfsetfillcolor{currentfill}%
\pgfsetlinewidth{0.000000pt}%
\definecolor{currentstroke}{rgb}{0.000000,0.000000,0.000000}%
\pgfsetstrokecolor{currentstroke}%
\pgfsetdash{}{0pt}%
\pgfpathmoveto{\pgfqpoint{4.768000in}{3.672048in}}%
\pgfpathlineto{\pgfqpoint{4.752855in}{3.687227in}}%
\pgfpathlineto{\pgfqpoint{4.739035in}{3.701333in}}%
\pgfpathlineto{\pgfqpoint{4.727919in}{3.712564in}}%
\pgfpathlineto{\pgfqpoint{4.714398in}{3.726072in}}%
\pgfpathlineto{\pgfqpoint{4.702020in}{3.738667in}}%
\pgfpathlineto{\pgfqpoint{4.687838in}{3.752951in}}%
\pgfpathlineto{\pgfqpoint{4.675880in}{3.764861in}}%
\pgfpathlineto{\pgfqpoint{4.664898in}{3.776000in}}%
\pgfpathlineto{\pgfqpoint{4.647758in}{3.793210in}}%
\pgfpathlineto{\pgfqpoint{4.627666in}{3.813333in}}%
\pgfpathlineto{\pgfqpoint{4.617984in}{3.822934in}}%
\pgfpathlineto{\pgfqpoint{4.607677in}{3.833341in}}%
\pgfpathlineto{\pgfqpoint{4.590325in}{3.850667in}}%
\pgfpathlineto{\pgfqpoint{4.579298in}{3.861566in}}%
\pgfpathlineto{\pgfqpoint{4.567596in}{3.873345in}}%
\pgfpathlineto{\pgfqpoint{4.552873in}{3.888000in}}%
\pgfpathlineto{\pgfqpoint{4.527515in}{3.913222in}}%
\pgfpathlineto{\pgfqpoint{4.521190in}{3.919442in}}%
\pgfpathlineto{\pgfqpoint{4.515309in}{3.925333in}}%
\pgfpathlineto{\pgfqpoint{4.501742in}{3.938660in}}%
\pgfpathlineto{\pgfqpoint{4.487434in}{3.952972in}}%
\pgfpathlineto{\pgfqpoint{4.482363in}{3.957943in}}%
\pgfpathlineto{\pgfqpoint{4.477633in}{3.962667in}}%
\pgfpathlineto{\pgfqpoint{4.447354in}{3.992595in}}%
\pgfpathlineto{\pgfqpoint{4.443474in}{3.996386in}}%
\pgfpathlineto{\pgfqpoint{4.439844in}{4.000000in}}%
\pgfpathlineto{\pgfqpoint{4.407273in}{4.032092in}}%
\pgfpathlineto{\pgfqpoint{4.404522in}{4.034771in}}%
\pgfpathlineto{\pgfqpoint{4.401940in}{4.037333in}}%
\pgfpathlineto{\pgfqpoint{4.384946in}{4.053870in}}%
\pgfpathlineto{\pgfqpoint{4.367192in}{4.071463in}}%
\pgfpathlineto{\pgfqpoint{4.365508in}{4.073098in}}%
\pgfpathlineto{\pgfqpoint{4.363922in}{4.074667in}}%
\pgfpathlineto{\pgfqpoint{4.327111in}{4.110708in}}%
\pgfpathlineto{\pgfqpoint{4.326431in}{4.111366in}}%
\pgfpathlineto{\pgfqpoint{4.325788in}{4.112000in}}%
\pgfpathlineto{\pgfqpoint{4.298116in}{4.139008in}}%
\pgfpathlineto{\pgfqpoint{4.287532in}{4.149333in}}%
\pgfpathlineto{\pgfqpoint{4.287030in}{4.149822in}}%
\pgfpathlineto{\pgfqpoint{4.267590in}{4.168559in}}%
\pgfpathlineto{\pgfqpoint{4.249144in}{4.186667in}}%
\pgfpathlineto{\pgfqpoint{4.246949in}{4.188799in}}%
\pgfpathlineto{\pgfqpoint{4.210639in}{4.224000in}}%
\pgfpathlineto{\pgfqpoint{4.207922in}{4.224000in}}%
\pgfpathlineto{\pgfqpoint{4.233760in}{4.198952in}}%
\pgfpathlineto{\pgfqpoint{4.246428in}{4.186667in}}%
\pgfpathlineto{\pgfqpoint{4.246682in}{4.186418in}}%
\pgfpathlineto{\pgfqpoint{4.246949in}{4.186161in}}%
\pgfpathlineto{\pgfqpoint{4.266200in}{4.167264in}}%
\pgfpathlineto{\pgfqpoint{4.284803in}{4.149333in}}%
\pgfpathlineto{\pgfqpoint{4.285887in}{4.148268in}}%
\pgfpathlineto{\pgfqpoint{4.287030in}{4.147165in}}%
\pgfpathlineto{\pgfqpoint{4.323061in}{4.112000in}}%
\pgfpathlineto{\pgfqpoint{4.325028in}{4.110060in}}%
\pgfpathlineto{\pgfqpoint{4.327111in}{4.108044in}}%
\pgfpathlineto{\pgfqpoint{4.361201in}{4.074667in}}%
\pgfpathlineto{\pgfqpoint{4.364107in}{4.071793in}}%
\pgfpathlineto{\pgfqpoint{4.367192in}{4.068798in}}%
\pgfpathlineto{\pgfqpoint{4.383559in}{4.052578in}}%
\pgfpathlineto{\pgfqpoint{4.399226in}{4.037333in}}%
\pgfpathlineto{\pgfqpoint{4.403123in}{4.033468in}}%
\pgfpathlineto{\pgfqpoint{4.407273in}{4.029425in}}%
\pgfpathlineto{\pgfqpoint{4.437137in}{4.000000in}}%
\pgfpathlineto{\pgfqpoint{4.442076in}{3.995084in}}%
\pgfpathlineto{\pgfqpoint{4.447354in}{3.989926in}}%
\pgfpathlineto{\pgfqpoint{4.474933in}{3.962667in}}%
\pgfpathlineto{\pgfqpoint{4.480966in}{3.956642in}}%
\pgfpathlineto{\pgfqpoint{4.487434in}{3.950301in}}%
\pgfpathlineto{\pgfqpoint{4.500359in}{3.937372in}}%
\pgfpathlineto{\pgfqpoint{4.512615in}{3.925333in}}%
\pgfpathlineto{\pgfqpoint{4.519795in}{3.918142in}}%
\pgfpathlineto{\pgfqpoint{4.527515in}{3.910550in}}%
\pgfpathlineto{\pgfqpoint{4.550186in}{3.888000in}}%
\pgfpathlineto{\pgfqpoint{4.567596in}{3.870671in}}%
\pgfpathlineto{\pgfqpoint{4.577918in}{3.860281in}}%
\pgfpathlineto{\pgfqpoint{4.587644in}{3.850667in}}%
\pgfpathlineto{\pgfqpoint{4.607677in}{3.830665in}}%
\pgfpathlineto{\pgfqpoint{4.616605in}{3.821650in}}%
\pgfpathlineto{\pgfqpoint{4.624993in}{3.813333in}}%
\pgfpathlineto{\pgfqpoint{4.647758in}{3.790532in}}%
\pgfpathlineto{\pgfqpoint{4.662231in}{3.776000in}}%
\pgfpathlineto{\pgfqpoint{4.674489in}{3.763566in}}%
\pgfpathlineto{\pgfqpoint{4.687838in}{3.750271in}}%
\pgfpathlineto{\pgfqpoint{4.699360in}{3.738667in}}%
\pgfpathlineto{\pgfqpoint{4.713009in}{3.724778in}}%
\pgfpathlineto{\pgfqpoint{4.727919in}{3.709882in}}%
\pgfpathlineto{\pgfqpoint{4.736380in}{3.701333in}}%
\pgfpathlineto{\pgfqpoint{4.751467in}{3.685934in}}%
\pgfpathlineto{\pgfqpoint{4.768000in}{3.669365in}}%
\pgfusepath{fill}%
\end{pgfscope}%
\begin{pgfscope}%
\pgfpathrectangle{\pgfqpoint{0.800000in}{0.528000in}}{\pgfqpoint{3.968000in}{3.696000in}}%
\pgfusepath{clip}%
\pgfsetbuttcap%
\pgfsetroundjoin%
\definecolor{currentfill}{rgb}{0.404001,0.800275,0.362552}%
\pgfsetfillcolor{currentfill}%
\pgfsetlinewidth{0.000000pt}%
\definecolor{currentstroke}{rgb}{0.000000,0.000000,0.000000}%
\pgfsetstrokecolor{currentstroke}%
\pgfsetdash{}{0pt}%
\pgfpathmoveto{\pgfqpoint{4.768000in}{3.674732in}}%
\pgfpathlineto{\pgfqpoint{4.754243in}{3.688519in}}%
\pgfpathlineto{\pgfqpoint{4.741689in}{3.701333in}}%
\pgfpathlineto{\pgfqpoint{4.727919in}{3.715245in}}%
\pgfpathlineto{\pgfqpoint{4.715787in}{3.727366in}}%
\pgfpathlineto{\pgfqpoint{4.704681in}{3.738667in}}%
\pgfpathlineto{\pgfqpoint{4.687838in}{3.755631in}}%
\pgfpathlineto{\pgfqpoint{4.677270in}{3.766156in}}%
\pgfpathlineto{\pgfqpoint{4.667565in}{3.776000in}}%
\pgfpathlineto{\pgfqpoint{4.647758in}{3.795888in}}%
\pgfpathlineto{\pgfqpoint{4.630340in}{3.813333in}}%
\pgfpathlineto{\pgfqpoint{4.619363in}{3.824218in}}%
\pgfpathlineto{\pgfqpoint{4.607677in}{3.836018in}}%
\pgfpathlineto{\pgfqpoint{4.593005in}{3.850667in}}%
\pgfpathlineto{\pgfqpoint{4.580678in}{3.862852in}}%
\pgfpathlineto{\pgfqpoint{4.567596in}{3.876020in}}%
\pgfpathlineto{\pgfqpoint{4.555560in}{3.888000in}}%
\pgfpathlineto{\pgfqpoint{4.527515in}{3.915895in}}%
\pgfpathlineto{\pgfqpoint{4.522586in}{3.920742in}}%
\pgfpathlineto{\pgfqpoint{4.518003in}{3.925333in}}%
\pgfpathlineto{\pgfqpoint{4.503124in}{3.939948in}}%
\pgfpathlineto{\pgfqpoint{4.487434in}{3.955643in}}%
\pgfpathlineto{\pgfqpoint{4.483760in}{3.959245in}}%
\pgfpathlineto{\pgfqpoint{4.480333in}{3.962667in}}%
\pgfpathlineto{\pgfqpoint{4.447354in}{3.995265in}}%
\pgfpathlineto{\pgfqpoint{4.444872in}{3.997689in}}%
\pgfpathlineto{\pgfqpoint{4.442551in}{4.000000in}}%
\pgfpathlineto{\pgfqpoint{4.407273in}{4.034760in}}%
\pgfpathlineto{\pgfqpoint{4.405922in}{4.036075in}}%
\pgfpathlineto{\pgfqpoint{4.404654in}{4.037333in}}%
\pgfpathlineto{\pgfqpoint{4.386332in}{4.055161in}}%
\pgfpathlineto{\pgfqpoint{4.367192in}{4.074129in}}%
\pgfpathlineto{\pgfqpoint{4.366909in}{4.074403in}}%
\pgfpathlineto{\pgfqpoint{4.366643in}{4.074667in}}%
\pgfpathlineto{\pgfqpoint{4.355903in}{4.085182in}}%
\pgfpathlineto{\pgfqpoint{4.328500in}{4.112000in}}%
\pgfpathlineto{\pgfqpoint{4.327819in}{4.112659in}}%
\pgfpathlineto{\pgfqpoint{4.327111in}{4.113358in}}%
\pgfpathlineto{\pgfqpoint{4.290235in}{4.149333in}}%
\pgfpathlineto{\pgfqpoint{4.287030in}{4.152458in}}%
\pgfpathlineto{\pgfqpoint{4.268981in}{4.169854in}}%
\pgfpathlineto{\pgfqpoint{4.251854in}{4.186667in}}%
\pgfpathlineto{\pgfqpoint{4.246949in}{4.191433in}}%
\pgfpathlineto{\pgfqpoint{4.213356in}{4.224000in}}%
\pgfpathlineto{\pgfqpoint{4.210639in}{4.224000in}}%
\pgfpathlineto{\pgfqpoint{4.246949in}{4.188799in}}%
\pgfpathlineto{\pgfqpoint{4.249144in}{4.186667in}}%
\pgfpathlineto{\pgfqpoint{4.267590in}{4.168559in}}%
\pgfpathlineto{\pgfqpoint{4.287030in}{4.149822in}}%
\pgfpathlineto{\pgfqpoint{4.287532in}{4.149333in}}%
\pgfpathlineto{\pgfqpoint{4.298116in}{4.139008in}}%
\pgfpathlineto{\pgfqpoint{4.325788in}{4.112000in}}%
\pgfpathlineto{\pgfqpoint{4.326431in}{4.111366in}}%
\pgfpathlineto{\pgfqpoint{4.327111in}{4.110708in}}%
\pgfpathlineto{\pgfqpoint{4.363922in}{4.074667in}}%
\pgfpathlineto{\pgfqpoint{4.365508in}{4.073098in}}%
\pgfpathlineto{\pgfqpoint{4.367192in}{4.071463in}}%
\pgfpathlineto{\pgfqpoint{4.384946in}{4.053870in}}%
\pgfpathlineto{\pgfqpoint{4.401940in}{4.037333in}}%
\pgfpathlineto{\pgfqpoint{4.404522in}{4.034771in}}%
\pgfpathlineto{\pgfqpoint{4.407273in}{4.032092in}}%
\pgfpathlineto{\pgfqpoint{4.439844in}{4.000000in}}%
\pgfpathlineto{\pgfqpoint{4.443474in}{3.996386in}}%
\pgfpathlineto{\pgfqpoint{4.447354in}{3.992595in}}%
\pgfpathlineto{\pgfqpoint{4.477633in}{3.962667in}}%
\pgfpathlineto{\pgfqpoint{4.482363in}{3.957943in}}%
\pgfpathlineto{\pgfqpoint{4.487434in}{3.952972in}}%
\pgfpathlineto{\pgfqpoint{4.501742in}{3.938660in}}%
\pgfpathlineto{\pgfqpoint{4.515309in}{3.925333in}}%
\pgfpathlineto{\pgfqpoint{4.521190in}{3.919442in}}%
\pgfpathlineto{\pgfqpoint{4.527515in}{3.913222in}}%
\pgfpathlineto{\pgfqpoint{4.552873in}{3.888000in}}%
\pgfpathlineto{\pgfqpoint{4.567596in}{3.873345in}}%
\pgfpathlineto{\pgfqpoint{4.579298in}{3.861566in}}%
\pgfpathlineto{\pgfqpoint{4.590325in}{3.850667in}}%
\pgfpathlineto{\pgfqpoint{4.607677in}{3.833341in}}%
\pgfpathlineto{\pgfqpoint{4.617984in}{3.822934in}}%
\pgfpathlineto{\pgfqpoint{4.627666in}{3.813333in}}%
\pgfpathlineto{\pgfqpoint{4.647758in}{3.793210in}}%
\pgfpathlineto{\pgfqpoint{4.664898in}{3.776000in}}%
\pgfpathlineto{\pgfqpoint{4.675880in}{3.764861in}}%
\pgfpathlineto{\pgfqpoint{4.687838in}{3.752951in}}%
\pgfpathlineto{\pgfqpoint{4.702020in}{3.738667in}}%
\pgfpathlineto{\pgfqpoint{4.714398in}{3.726072in}}%
\pgfpathlineto{\pgfqpoint{4.727919in}{3.712564in}}%
\pgfpathlineto{\pgfqpoint{4.739035in}{3.701333in}}%
\pgfpathlineto{\pgfqpoint{4.752855in}{3.687227in}}%
\pgfpathlineto{\pgfqpoint{4.768000in}{3.672048in}}%
\pgfusepath{fill}%
\end{pgfscope}%
\begin{pgfscope}%
\pgfpathrectangle{\pgfqpoint{0.800000in}{0.528000in}}{\pgfqpoint{3.968000in}{3.696000in}}%
\pgfusepath{clip}%
\pgfsetbuttcap%
\pgfsetroundjoin%
\definecolor{currentfill}{rgb}{0.412913,0.803041,0.357269}%
\pgfsetfillcolor{currentfill}%
\pgfsetlinewidth{0.000000pt}%
\definecolor{currentstroke}{rgb}{0.000000,0.000000,0.000000}%
\pgfsetstrokecolor{currentstroke}%
\pgfsetdash{}{0pt}%
\pgfpathmoveto{\pgfqpoint{4.768000in}{3.677415in}}%
\pgfpathlineto{\pgfqpoint{4.755631in}{3.689812in}}%
\pgfpathlineto{\pgfqpoint{4.744343in}{3.701333in}}%
\pgfpathlineto{\pgfqpoint{4.727919in}{3.717927in}}%
\pgfpathlineto{\pgfqpoint{4.717176in}{3.728660in}}%
\pgfpathlineto{\pgfqpoint{4.707342in}{3.738667in}}%
\pgfpathlineto{\pgfqpoint{4.687838in}{3.758310in}}%
\pgfpathlineto{\pgfqpoint{4.678660in}{3.767451in}}%
\pgfpathlineto{\pgfqpoint{4.670232in}{3.776000in}}%
\pgfpathlineto{\pgfqpoint{4.647758in}{3.798566in}}%
\pgfpathlineto{\pgfqpoint{4.633014in}{3.813333in}}%
\pgfpathlineto{\pgfqpoint{4.620741in}{3.825502in}}%
\pgfpathlineto{\pgfqpoint{4.607677in}{3.838694in}}%
\pgfpathlineto{\pgfqpoint{4.595685in}{3.850667in}}%
\pgfpathlineto{\pgfqpoint{4.582058in}{3.864137in}}%
\pgfpathlineto{\pgfqpoint{4.567596in}{3.878694in}}%
\pgfpathlineto{\pgfqpoint{4.558247in}{3.888000in}}%
\pgfpathlineto{\pgfqpoint{4.527515in}{3.918567in}}%
\pgfpathlineto{\pgfqpoint{4.523982in}{3.922042in}}%
\pgfpathlineto{\pgfqpoint{4.520696in}{3.925333in}}%
\pgfpathlineto{\pgfqpoint{4.504507in}{3.941236in}}%
\pgfpathlineto{\pgfqpoint{4.487434in}{3.958314in}}%
\pgfpathlineto{\pgfqpoint{4.485157in}{3.960546in}}%
\pgfpathlineto{\pgfqpoint{4.483033in}{3.962667in}}%
\pgfpathlineto{\pgfqpoint{4.447354in}{3.997934in}}%
\pgfpathlineto{\pgfqpoint{4.446271in}{3.998992in}}%
\pgfpathlineto{\pgfqpoint{4.445258in}{4.000000in}}%
\pgfpathlineto{\pgfqpoint{4.409009in}{4.035716in}}%
\pgfpathlineto{\pgfqpoint{4.407367in}{4.037333in}}%
\pgfpathlineto{\pgfqpoint{4.407273in}{4.037426in}}%
\pgfpathlineto{\pgfqpoint{4.387719in}{4.056453in}}%
\pgfpathlineto{\pgfqpoint{4.369339in}{4.074667in}}%
\pgfpathlineto{\pgfqpoint{4.367192in}{4.076773in}}%
\pgfpathlineto{\pgfqpoint{4.331196in}{4.112000in}}%
\pgfpathlineto{\pgfqpoint{4.329194in}{4.113940in}}%
\pgfpathlineto{\pgfqpoint{4.327111in}{4.115995in}}%
\pgfpathlineto{\pgfqpoint{4.292938in}{4.149333in}}%
\pgfpathlineto{\pgfqpoint{4.287030in}{4.155093in}}%
\pgfpathlineto{\pgfqpoint{4.270371in}{4.171150in}}%
\pgfpathlineto{\pgfqpoint{4.254564in}{4.186667in}}%
\pgfpathlineto{\pgfqpoint{4.246949in}{4.194067in}}%
\pgfpathlineto{\pgfqpoint{4.216073in}{4.224000in}}%
\pgfpathlineto{\pgfqpoint{4.213356in}{4.224000in}}%
\pgfpathlineto{\pgfqpoint{4.246949in}{4.191433in}}%
\pgfpathlineto{\pgfqpoint{4.251854in}{4.186667in}}%
\pgfpathlineto{\pgfqpoint{4.268981in}{4.169854in}}%
\pgfpathlineto{\pgfqpoint{4.287030in}{4.152458in}}%
\pgfpathlineto{\pgfqpoint{4.290235in}{4.149333in}}%
\pgfpathlineto{\pgfqpoint{4.327111in}{4.113358in}}%
\pgfpathlineto{\pgfqpoint{4.327819in}{4.112659in}}%
\pgfpathlineto{\pgfqpoint{4.328500in}{4.112000in}}%
\pgfpathlineto{\pgfqpoint{4.355903in}{4.085182in}}%
\pgfpathlineto{\pgfqpoint{4.366643in}{4.074667in}}%
\pgfpathlineto{\pgfqpoint{4.366909in}{4.074403in}}%
\pgfpathlineto{\pgfqpoint{4.367192in}{4.074129in}}%
\pgfpathlineto{\pgfqpoint{4.386332in}{4.055161in}}%
\pgfpathlineto{\pgfqpoint{4.404654in}{4.037333in}}%
\pgfpathlineto{\pgfqpoint{4.405922in}{4.036075in}}%
\pgfpathlineto{\pgfqpoint{4.407273in}{4.034760in}}%
\pgfpathlineto{\pgfqpoint{4.442551in}{4.000000in}}%
\pgfpathlineto{\pgfqpoint{4.444872in}{3.997689in}}%
\pgfpathlineto{\pgfqpoint{4.447354in}{3.995265in}}%
\pgfpathlineto{\pgfqpoint{4.480333in}{3.962667in}}%
\pgfpathlineto{\pgfqpoint{4.483760in}{3.959245in}}%
\pgfpathlineto{\pgfqpoint{4.487434in}{3.955643in}}%
\pgfpathlineto{\pgfqpoint{4.503124in}{3.939948in}}%
\pgfpathlineto{\pgfqpoint{4.518003in}{3.925333in}}%
\pgfpathlineto{\pgfqpoint{4.522586in}{3.920742in}}%
\pgfpathlineto{\pgfqpoint{4.527515in}{3.915895in}}%
\pgfpathlineto{\pgfqpoint{4.555560in}{3.888000in}}%
\pgfpathlineto{\pgfqpoint{4.567596in}{3.876020in}}%
\pgfpathlineto{\pgfqpoint{4.580678in}{3.862852in}}%
\pgfpathlineto{\pgfqpoint{4.593005in}{3.850667in}}%
\pgfpathlineto{\pgfqpoint{4.607677in}{3.836018in}}%
\pgfpathlineto{\pgfqpoint{4.619363in}{3.824218in}}%
\pgfpathlineto{\pgfqpoint{4.630340in}{3.813333in}}%
\pgfpathlineto{\pgfqpoint{4.647758in}{3.795888in}}%
\pgfpathlineto{\pgfqpoint{4.667565in}{3.776000in}}%
\pgfpathlineto{\pgfqpoint{4.677270in}{3.766156in}}%
\pgfpathlineto{\pgfqpoint{4.687838in}{3.755631in}}%
\pgfpathlineto{\pgfqpoint{4.704681in}{3.738667in}}%
\pgfpathlineto{\pgfqpoint{4.715787in}{3.727366in}}%
\pgfpathlineto{\pgfqpoint{4.727919in}{3.715245in}}%
\pgfpathlineto{\pgfqpoint{4.741689in}{3.701333in}}%
\pgfpathlineto{\pgfqpoint{4.754243in}{3.688519in}}%
\pgfpathlineto{\pgfqpoint{4.768000in}{3.674732in}}%
\pgfusepath{fill}%
\end{pgfscope}%
\begin{pgfscope}%
\pgfpathrectangle{\pgfqpoint{0.800000in}{0.528000in}}{\pgfqpoint{3.968000in}{3.696000in}}%
\pgfusepath{clip}%
\pgfsetbuttcap%
\pgfsetroundjoin%
\definecolor{currentfill}{rgb}{0.412913,0.803041,0.357269}%
\pgfsetfillcolor{currentfill}%
\pgfsetlinewidth{0.000000pt}%
\definecolor{currentstroke}{rgb}{0.000000,0.000000,0.000000}%
\pgfsetstrokecolor{currentstroke}%
\pgfsetdash{}{0pt}%
\pgfpathmoveto{\pgfqpoint{4.768000in}{3.680099in}}%
\pgfpathlineto{\pgfqpoint{4.757018in}{3.691104in}}%
\pgfpathlineto{\pgfqpoint{4.746997in}{3.701333in}}%
\pgfpathlineto{\pgfqpoint{4.727919in}{3.720608in}}%
\pgfpathlineto{\pgfqpoint{4.718565in}{3.729954in}}%
\pgfpathlineto{\pgfqpoint{4.710002in}{3.738667in}}%
\pgfpathlineto{\pgfqpoint{4.687838in}{3.760990in}}%
\pgfpathlineto{\pgfqpoint{4.680051in}{3.768746in}}%
\pgfpathlineto{\pgfqpoint{4.672899in}{3.776000in}}%
\pgfpathlineto{\pgfqpoint{4.647758in}{3.801244in}}%
\pgfpathlineto{\pgfqpoint{4.635687in}{3.813333in}}%
\pgfpathlineto{\pgfqpoint{4.622120in}{3.826786in}}%
\pgfpathlineto{\pgfqpoint{4.607677in}{3.841370in}}%
\pgfpathlineto{\pgfqpoint{4.598366in}{3.850667in}}%
\pgfpathlineto{\pgfqpoint{4.583438in}{3.865423in}}%
\pgfpathlineto{\pgfqpoint{4.567596in}{3.881369in}}%
\pgfpathlineto{\pgfqpoint{4.560933in}{3.888000in}}%
\pgfpathlineto{\pgfqpoint{4.527515in}{3.921240in}}%
\pgfpathlineto{\pgfqpoint{4.525378in}{3.923342in}}%
\pgfpathlineto{\pgfqpoint{4.523390in}{3.925333in}}%
\pgfpathlineto{\pgfqpoint{4.505890in}{3.942523in}}%
\pgfpathlineto{\pgfqpoint{4.487434in}{3.960985in}}%
\pgfpathlineto{\pgfqpoint{4.486555in}{3.961847in}}%
\pgfpathlineto{\pgfqpoint{4.485734in}{3.962667in}}%
\pgfpathlineto{\pgfqpoint{4.457931in}{3.990148in}}%
\pgfpathlineto{\pgfqpoint{4.447958in}{4.000000in}}%
\pgfpathlineto{\pgfqpoint{4.447663in}{4.000288in}}%
\pgfpathlineto{\pgfqpoint{4.447354in}{4.000597in}}%
\pgfpathlineto{\pgfqpoint{4.410050in}{4.037333in}}%
\pgfpathlineto{\pgfqpoint{4.407273in}{4.040067in}}%
\pgfpathlineto{\pgfqpoint{4.389105in}{4.057744in}}%
\pgfpathlineto{\pgfqpoint{4.372028in}{4.074667in}}%
\pgfpathlineto{\pgfqpoint{4.367192in}{4.079412in}}%
\pgfpathlineto{\pgfqpoint{4.333893in}{4.112000in}}%
\pgfpathlineto{\pgfqpoint{4.330569in}{4.115221in}}%
\pgfpathlineto{\pgfqpoint{4.327111in}{4.118632in}}%
\pgfpathlineto{\pgfqpoint{4.295641in}{4.149333in}}%
\pgfpathlineto{\pgfqpoint{4.287030in}{4.157728in}}%
\pgfpathlineto{\pgfqpoint{4.271762in}{4.172445in}}%
\pgfpathlineto{\pgfqpoint{4.257274in}{4.186667in}}%
\pgfpathlineto{\pgfqpoint{4.246949in}{4.196700in}}%
\pgfpathlineto{\pgfqpoint{4.218789in}{4.224000in}}%
\pgfpathlineto{\pgfqpoint{4.216073in}{4.224000in}}%
\pgfpathlineto{\pgfqpoint{4.246949in}{4.194067in}}%
\pgfpathlineto{\pgfqpoint{4.254564in}{4.186667in}}%
\pgfpathlineto{\pgfqpoint{4.270371in}{4.171150in}}%
\pgfpathlineto{\pgfqpoint{4.287030in}{4.155093in}}%
\pgfpathlineto{\pgfqpoint{4.292938in}{4.149333in}}%
\pgfpathlineto{\pgfqpoint{4.327111in}{4.115995in}}%
\pgfpathlineto{\pgfqpoint{4.329194in}{4.113940in}}%
\pgfpathlineto{\pgfqpoint{4.331196in}{4.112000in}}%
\pgfpathlineto{\pgfqpoint{4.367192in}{4.076773in}}%
\pgfpathlineto{\pgfqpoint{4.369339in}{4.074667in}}%
\pgfpathlineto{\pgfqpoint{4.387719in}{4.056453in}}%
\pgfpathlineto{\pgfqpoint{4.407273in}{4.037426in}}%
\pgfpathlineto{\pgfqpoint{4.407367in}{4.037333in}}%
\pgfpathlineto{\pgfqpoint{4.409009in}{4.035716in}}%
\pgfpathlineto{\pgfqpoint{4.445258in}{4.000000in}}%
\pgfpathlineto{\pgfqpoint{4.446271in}{3.998992in}}%
\pgfpathlineto{\pgfqpoint{4.447354in}{3.997934in}}%
\pgfpathlineto{\pgfqpoint{4.483033in}{3.962667in}}%
\pgfpathlineto{\pgfqpoint{4.485157in}{3.960546in}}%
\pgfpathlineto{\pgfqpoint{4.487434in}{3.958314in}}%
\pgfpathlineto{\pgfqpoint{4.504507in}{3.941236in}}%
\pgfpathlineto{\pgfqpoint{4.520696in}{3.925333in}}%
\pgfpathlineto{\pgfqpoint{4.523982in}{3.922042in}}%
\pgfpathlineto{\pgfqpoint{4.527515in}{3.918567in}}%
\pgfpathlineto{\pgfqpoint{4.558247in}{3.888000in}}%
\pgfpathlineto{\pgfqpoint{4.567596in}{3.878694in}}%
\pgfpathlineto{\pgfqpoint{4.582058in}{3.864137in}}%
\pgfpathlineto{\pgfqpoint{4.595685in}{3.850667in}}%
\pgfpathlineto{\pgfqpoint{4.607677in}{3.838694in}}%
\pgfpathlineto{\pgfqpoint{4.620741in}{3.825502in}}%
\pgfpathlineto{\pgfqpoint{4.633014in}{3.813333in}}%
\pgfpathlineto{\pgfqpoint{4.647758in}{3.798566in}}%
\pgfpathlineto{\pgfqpoint{4.670232in}{3.776000in}}%
\pgfpathlineto{\pgfqpoint{4.678660in}{3.767451in}}%
\pgfpathlineto{\pgfqpoint{4.687838in}{3.758310in}}%
\pgfpathlineto{\pgfqpoint{4.707342in}{3.738667in}}%
\pgfpathlineto{\pgfqpoint{4.717176in}{3.728660in}}%
\pgfpathlineto{\pgfqpoint{4.727919in}{3.717927in}}%
\pgfpathlineto{\pgfqpoint{4.744343in}{3.701333in}}%
\pgfpathlineto{\pgfqpoint{4.755631in}{3.689812in}}%
\pgfpathlineto{\pgfqpoint{4.768000in}{3.677415in}}%
\pgfusepath{fill}%
\end{pgfscope}%
\begin{pgfscope}%
\pgfpathrectangle{\pgfqpoint{0.800000in}{0.528000in}}{\pgfqpoint{3.968000in}{3.696000in}}%
\pgfusepath{clip}%
\pgfsetbuttcap%
\pgfsetroundjoin%
\definecolor{currentfill}{rgb}{0.412913,0.803041,0.357269}%
\pgfsetfillcolor{currentfill}%
\pgfsetlinewidth{0.000000pt}%
\definecolor{currentstroke}{rgb}{0.000000,0.000000,0.000000}%
\pgfsetstrokecolor{currentstroke}%
\pgfsetdash{}{0pt}%
\pgfpathmoveto{\pgfqpoint{4.768000in}{3.682782in}}%
\pgfpathlineto{\pgfqpoint{4.758406in}{3.692397in}}%
\pgfpathlineto{\pgfqpoint{4.749651in}{3.701333in}}%
\pgfpathlineto{\pgfqpoint{4.727919in}{3.723290in}}%
\pgfpathlineto{\pgfqpoint{4.719954in}{3.731248in}}%
\pgfpathlineto{\pgfqpoint{4.712663in}{3.738667in}}%
\pgfpathlineto{\pgfqpoint{4.687838in}{3.763670in}}%
\pgfpathlineto{\pgfqpoint{4.681441in}{3.770041in}}%
\pgfpathlineto{\pgfqpoint{4.675566in}{3.776000in}}%
\pgfpathlineto{\pgfqpoint{4.647758in}{3.803922in}}%
\pgfpathlineto{\pgfqpoint{4.638361in}{3.813333in}}%
\pgfpathlineto{\pgfqpoint{4.623499in}{3.828071in}}%
\pgfpathlineto{\pgfqpoint{4.607677in}{3.844046in}}%
\pgfpathlineto{\pgfqpoint{4.601046in}{3.850667in}}%
\pgfpathlineto{\pgfqpoint{4.584818in}{3.866708in}}%
\pgfpathlineto{\pgfqpoint{4.567596in}{3.884043in}}%
\pgfpathlineto{\pgfqpoint{4.563620in}{3.888000in}}%
\pgfpathlineto{\pgfqpoint{4.527515in}{3.923913in}}%
\pgfpathlineto{\pgfqpoint{4.526773in}{3.924642in}}%
\pgfpathlineto{\pgfqpoint{4.526083in}{3.925333in}}%
\pgfpathlineto{\pgfqpoint{4.507272in}{3.943811in}}%
\pgfpathlineto{\pgfqpoint{4.488423in}{3.962667in}}%
\pgfpathlineto{\pgfqpoint{4.487434in}{3.963646in}}%
\pgfpathlineto{\pgfqpoint{4.450634in}{4.000000in}}%
\pgfpathlineto{\pgfqpoint{4.449034in}{4.001565in}}%
\pgfpathlineto{\pgfqpoint{4.447354in}{4.003239in}}%
\pgfpathlineto{\pgfqpoint{4.412733in}{4.037333in}}%
\pgfpathlineto{\pgfqpoint{4.407273in}{4.042707in}}%
\pgfpathlineto{\pgfqpoint{4.390492in}{4.059036in}}%
\pgfpathlineto{\pgfqpoint{4.374718in}{4.074667in}}%
\pgfpathlineto{\pgfqpoint{4.367192in}{4.082051in}}%
\pgfpathlineto{\pgfqpoint{4.336589in}{4.112000in}}%
\pgfpathlineto{\pgfqpoint{4.331944in}{4.116501in}}%
\pgfpathlineto{\pgfqpoint{4.327111in}{4.121269in}}%
\pgfpathlineto{\pgfqpoint{4.298344in}{4.149333in}}%
\pgfpathlineto{\pgfqpoint{4.287030in}{4.160364in}}%
\pgfpathlineto{\pgfqpoint{4.273152in}{4.173740in}}%
\pgfpathlineto{\pgfqpoint{4.259984in}{4.186667in}}%
\pgfpathlineto{\pgfqpoint{4.246949in}{4.199334in}}%
\pgfpathlineto{\pgfqpoint{4.221506in}{4.224000in}}%
\pgfpathlineto{\pgfqpoint{4.218789in}{4.224000in}}%
\pgfpathlineto{\pgfqpoint{4.246949in}{4.196700in}}%
\pgfpathlineto{\pgfqpoint{4.257274in}{4.186667in}}%
\pgfpathlineto{\pgfqpoint{4.271762in}{4.172445in}}%
\pgfpathlineto{\pgfqpoint{4.287030in}{4.157728in}}%
\pgfpathlineto{\pgfqpoint{4.295641in}{4.149333in}}%
\pgfpathlineto{\pgfqpoint{4.327111in}{4.118632in}}%
\pgfpathlineto{\pgfqpoint{4.330569in}{4.115221in}}%
\pgfpathlineto{\pgfqpoint{4.333893in}{4.112000in}}%
\pgfpathlineto{\pgfqpoint{4.367192in}{4.079412in}}%
\pgfpathlineto{\pgfqpoint{4.372028in}{4.074667in}}%
\pgfpathlineto{\pgfqpoint{4.389105in}{4.057744in}}%
\pgfpathlineto{\pgfqpoint{4.407273in}{4.040067in}}%
\pgfpathlineto{\pgfqpoint{4.410050in}{4.037333in}}%
\pgfpathlineto{\pgfqpoint{4.447354in}{4.000597in}}%
\pgfpathlineto{\pgfqpoint{4.447663in}{4.000288in}}%
\pgfpathlineto{\pgfqpoint{4.447958in}{4.000000in}}%
\pgfpathlineto{\pgfqpoint{4.457931in}{3.990148in}}%
\pgfpathlineto{\pgfqpoint{4.485734in}{3.962667in}}%
\pgfpathlineto{\pgfqpoint{4.486555in}{3.961847in}}%
\pgfpathlineto{\pgfqpoint{4.487434in}{3.960985in}}%
\pgfpathlineto{\pgfqpoint{4.505890in}{3.942523in}}%
\pgfpathlineto{\pgfqpoint{4.523390in}{3.925333in}}%
\pgfpathlineto{\pgfqpoint{4.525378in}{3.923342in}}%
\pgfpathlineto{\pgfqpoint{4.527515in}{3.921240in}}%
\pgfpathlineto{\pgfqpoint{4.560933in}{3.888000in}}%
\pgfpathlineto{\pgfqpoint{4.567596in}{3.881369in}}%
\pgfpathlineto{\pgfqpoint{4.583438in}{3.865423in}}%
\pgfpathlineto{\pgfqpoint{4.598366in}{3.850667in}}%
\pgfpathlineto{\pgfqpoint{4.607677in}{3.841370in}}%
\pgfpathlineto{\pgfqpoint{4.622120in}{3.826786in}}%
\pgfpathlineto{\pgfqpoint{4.635687in}{3.813333in}}%
\pgfpathlineto{\pgfqpoint{4.647758in}{3.801244in}}%
\pgfpathlineto{\pgfqpoint{4.672899in}{3.776000in}}%
\pgfpathlineto{\pgfqpoint{4.680051in}{3.768746in}}%
\pgfpathlineto{\pgfqpoint{4.687838in}{3.760990in}}%
\pgfpathlineto{\pgfqpoint{4.710002in}{3.738667in}}%
\pgfpathlineto{\pgfqpoint{4.718565in}{3.729954in}}%
\pgfpathlineto{\pgfqpoint{4.727919in}{3.720608in}}%
\pgfpathlineto{\pgfqpoint{4.746997in}{3.701333in}}%
\pgfpathlineto{\pgfqpoint{4.757018in}{3.691104in}}%
\pgfpathlineto{\pgfqpoint{4.768000in}{3.680099in}}%
\pgfusepath{fill}%
\end{pgfscope}%
\begin{pgfscope}%
\pgfpathrectangle{\pgfqpoint{0.800000in}{0.528000in}}{\pgfqpoint{3.968000in}{3.696000in}}%
\pgfusepath{clip}%
\pgfsetbuttcap%
\pgfsetroundjoin%
\definecolor{currentfill}{rgb}{0.412913,0.803041,0.357269}%
\pgfsetfillcolor{currentfill}%
\pgfsetlinewidth{0.000000pt}%
\definecolor{currentstroke}{rgb}{0.000000,0.000000,0.000000}%
\pgfsetstrokecolor{currentstroke}%
\pgfsetdash{}{0pt}%
\pgfpathmoveto{\pgfqpoint{4.768000in}{3.685465in}}%
\pgfpathlineto{\pgfqpoint{4.759794in}{3.693690in}}%
\pgfpathlineto{\pgfqpoint{4.752305in}{3.701333in}}%
\pgfpathlineto{\pgfqpoint{4.727919in}{3.725972in}}%
\pgfpathlineto{\pgfqpoint{4.721343in}{3.732541in}}%
\pgfpathlineto{\pgfqpoint{4.715323in}{3.738667in}}%
\pgfpathlineto{\pgfqpoint{4.687838in}{3.766350in}}%
\pgfpathlineto{\pgfqpoint{4.682831in}{3.771336in}}%
\pgfpathlineto{\pgfqpoint{4.678234in}{3.776000in}}%
\pgfpathlineto{\pgfqpoint{4.647758in}{3.806600in}}%
\pgfpathlineto{\pgfqpoint{4.641035in}{3.813333in}}%
\pgfpathlineto{\pgfqpoint{4.624877in}{3.829355in}}%
\pgfpathlineto{\pgfqpoint{4.607677in}{3.846722in}}%
\pgfpathlineto{\pgfqpoint{4.603726in}{3.850667in}}%
\pgfpathlineto{\pgfqpoint{4.586198in}{3.867993in}}%
\pgfpathlineto{\pgfqpoint{4.567596in}{3.886717in}}%
\pgfpathlineto{\pgfqpoint{4.566307in}{3.888000in}}%
\pgfpathlineto{\pgfqpoint{4.547320in}{3.906886in}}%
\pgfpathlineto{\pgfqpoint{4.528763in}{3.925333in}}%
\pgfpathlineto{\pgfqpoint{4.527515in}{3.926573in}}%
\pgfpathlineto{\pgfqpoint{4.508655in}{3.945099in}}%
\pgfpathlineto{\pgfqpoint{4.491093in}{3.962667in}}%
\pgfpathlineto{\pgfqpoint{4.487434in}{3.966290in}}%
\pgfpathlineto{\pgfqpoint{4.453311in}{4.000000in}}%
\pgfpathlineto{\pgfqpoint{4.450405in}{4.002842in}}%
\pgfpathlineto{\pgfqpoint{4.447354in}{4.005881in}}%
\pgfpathlineto{\pgfqpoint{4.415416in}{4.037333in}}%
\pgfpathlineto{\pgfqpoint{4.407273in}{4.045348in}}%
\pgfpathlineto{\pgfqpoint{4.391878in}{4.060327in}}%
\pgfpathlineto{\pgfqpoint{4.377408in}{4.074667in}}%
\pgfpathlineto{\pgfqpoint{4.367192in}{4.084690in}}%
\pgfpathlineto{\pgfqpoint{4.339286in}{4.112000in}}%
\pgfpathlineto{\pgfqpoint{4.333319in}{4.117782in}}%
\pgfpathlineto{\pgfqpoint{4.327111in}{4.123907in}}%
\pgfpathlineto{\pgfqpoint{4.301048in}{4.149333in}}%
\pgfpathlineto{\pgfqpoint{4.287030in}{4.162999in}}%
\pgfpathlineto{\pgfqpoint{4.274543in}{4.175035in}}%
\pgfpathlineto{\pgfqpoint{4.262694in}{4.186667in}}%
\pgfpathlineto{\pgfqpoint{4.246949in}{4.201968in}}%
\pgfpathlineto{\pgfqpoint{4.224223in}{4.224000in}}%
\pgfpathlineto{\pgfqpoint{4.221506in}{4.224000in}}%
\pgfpathlineto{\pgfqpoint{4.246949in}{4.199334in}}%
\pgfpathlineto{\pgfqpoint{4.259984in}{4.186667in}}%
\pgfpathlineto{\pgfqpoint{4.273152in}{4.173740in}}%
\pgfpathlineto{\pgfqpoint{4.287030in}{4.160364in}}%
\pgfpathlineto{\pgfqpoint{4.298344in}{4.149333in}}%
\pgfpathlineto{\pgfqpoint{4.327111in}{4.121269in}}%
\pgfpathlineto{\pgfqpoint{4.331944in}{4.116501in}}%
\pgfpathlineto{\pgfqpoint{4.336589in}{4.112000in}}%
\pgfpathlineto{\pgfqpoint{4.367192in}{4.082051in}}%
\pgfpathlineto{\pgfqpoint{4.374718in}{4.074667in}}%
\pgfpathlineto{\pgfqpoint{4.390492in}{4.059036in}}%
\pgfpathlineto{\pgfqpoint{4.407273in}{4.042707in}}%
\pgfpathlineto{\pgfqpoint{4.412733in}{4.037333in}}%
\pgfpathlineto{\pgfqpoint{4.447354in}{4.003239in}}%
\pgfpathlineto{\pgfqpoint{4.449034in}{4.001565in}}%
\pgfpathlineto{\pgfqpoint{4.450634in}{4.000000in}}%
\pgfpathlineto{\pgfqpoint{4.487434in}{3.963646in}}%
\pgfpathlineto{\pgfqpoint{4.488423in}{3.962667in}}%
\pgfpathlineto{\pgfqpoint{4.507272in}{3.943811in}}%
\pgfpathlineto{\pgfqpoint{4.526083in}{3.925333in}}%
\pgfpathlineto{\pgfqpoint{4.526773in}{3.924642in}}%
\pgfpathlineto{\pgfqpoint{4.527515in}{3.923913in}}%
\pgfpathlineto{\pgfqpoint{4.563620in}{3.888000in}}%
\pgfpathlineto{\pgfqpoint{4.567596in}{3.884043in}}%
\pgfpathlineto{\pgfqpoint{4.584818in}{3.866708in}}%
\pgfpathlineto{\pgfqpoint{4.601046in}{3.850667in}}%
\pgfpathlineto{\pgfqpoint{4.607677in}{3.844046in}}%
\pgfpathlineto{\pgfqpoint{4.623499in}{3.828071in}}%
\pgfpathlineto{\pgfqpoint{4.638361in}{3.813333in}}%
\pgfpathlineto{\pgfqpoint{4.647758in}{3.803922in}}%
\pgfpathlineto{\pgfqpoint{4.675566in}{3.776000in}}%
\pgfpathlineto{\pgfqpoint{4.681441in}{3.770041in}}%
\pgfpathlineto{\pgfqpoint{4.687838in}{3.763670in}}%
\pgfpathlineto{\pgfqpoint{4.712663in}{3.738667in}}%
\pgfpathlineto{\pgfqpoint{4.719954in}{3.731248in}}%
\pgfpathlineto{\pgfqpoint{4.727919in}{3.723290in}}%
\pgfpathlineto{\pgfqpoint{4.749651in}{3.701333in}}%
\pgfpathlineto{\pgfqpoint{4.758406in}{3.692397in}}%
\pgfpathlineto{\pgfqpoint{4.768000in}{3.682782in}}%
\pgfusepath{fill}%
\end{pgfscope}%
\begin{pgfscope}%
\pgfpathrectangle{\pgfqpoint{0.800000in}{0.528000in}}{\pgfqpoint{3.968000in}{3.696000in}}%
\pgfusepath{clip}%
\pgfsetbuttcap%
\pgfsetroundjoin%
\definecolor{currentfill}{rgb}{0.421908,0.805774,0.351910}%
\pgfsetfillcolor{currentfill}%
\pgfsetlinewidth{0.000000pt}%
\definecolor{currentstroke}{rgb}{0.000000,0.000000,0.000000}%
\pgfsetstrokecolor{currentstroke}%
\pgfsetdash{}{0pt}%
\pgfpathmoveto{\pgfqpoint{4.768000in}{3.688149in}}%
\pgfpathlineto{\pgfqpoint{4.761182in}{3.694982in}}%
\pgfpathlineto{\pgfqpoint{4.754959in}{3.701333in}}%
\pgfpathlineto{\pgfqpoint{4.727919in}{3.728653in}}%
\pgfpathlineto{\pgfqpoint{4.722732in}{3.733835in}}%
\pgfpathlineto{\pgfqpoint{4.717984in}{3.738667in}}%
\pgfpathlineto{\pgfqpoint{4.687838in}{3.769029in}}%
\pgfpathlineto{\pgfqpoint{4.684222in}{3.772631in}}%
\pgfpathlineto{\pgfqpoint{4.680901in}{3.776000in}}%
\pgfpathlineto{\pgfqpoint{4.647758in}{3.809278in}}%
\pgfpathlineto{\pgfqpoint{4.643709in}{3.813333in}}%
\pgfpathlineto{\pgfqpoint{4.626256in}{3.830639in}}%
\pgfpathlineto{\pgfqpoint{4.607677in}{3.849399in}}%
\pgfpathlineto{\pgfqpoint{4.606407in}{3.850667in}}%
\pgfpathlineto{\pgfqpoint{4.587578in}{3.869279in}}%
\pgfpathlineto{\pgfqpoint{4.568979in}{3.888000in}}%
\pgfpathlineto{\pgfqpoint{4.567596in}{3.889378in}}%
\pgfpathlineto{\pgfqpoint{4.531426in}{3.925333in}}%
\pgfpathlineto{\pgfqpoint{4.527515in}{3.929219in}}%
\pgfpathlineto{\pgfqpoint{4.510037in}{3.946387in}}%
\pgfpathlineto{\pgfqpoint{4.493763in}{3.962667in}}%
\pgfpathlineto{\pgfqpoint{4.487434in}{3.968934in}}%
\pgfpathlineto{\pgfqpoint{4.455987in}{4.000000in}}%
\pgfpathlineto{\pgfqpoint{4.451776in}{4.004119in}}%
\pgfpathlineto{\pgfqpoint{4.447354in}{4.008524in}}%
\pgfpathlineto{\pgfqpoint{4.418099in}{4.037333in}}%
\pgfpathlineto{\pgfqpoint{4.407273in}{4.047989in}}%
\pgfpathlineto{\pgfqpoint{4.393265in}{4.061619in}}%
\pgfpathlineto{\pgfqpoint{4.380098in}{4.074667in}}%
\pgfpathlineto{\pgfqpoint{4.367192in}{4.087328in}}%
\pgfpathlineto{\pgfqpoint{4.341982in}{4.112000in}}%
\pgfpathlineto{\pgfqpoint{4.334693in}{4.119063in}}%
\pgfpathlineto{\pgfqpoint{4.327111in}{4.126544in}}%
\pgfpathlineto{\pgfqpoint{4.303751in}{4.149333in}}%
\pgfpathlineto{\pgfqpoint{4.287030in}{4.165635in}}%
\pgfpathlineto{\pgfqpoint{4.275933in}{4.176330in}}%
\pgfpathlineto{\pgfqpoint{4.265404in}{4.186667in}}%
\pgfpathlineto{\pgfqpoint{4.246949in}{4.204602in}}%
\pgfpathlineto{\pgfqpoint{4.226939in}{4.224000in}}%
\pgfpathlineto{\pgfqpoint{4.224223in}{4.224000in}}%
\pgfpathlineto{\pgfqpoint{4.246949in}{4.201968in}}%
\pgfpathlineto{\pgfqpoint{4.262694in}{4.186667in}}%
\pgfpathlineto{\pgfqpoint{4.274543in}{4.175035in}}%
\pgfpathlineto{\pgfqpoint{4.287030in}{4.162999in}}%
\pgfpathlineto{\pgfqpoint{4.301048in}{4.149333in}}%
\pgfpathlineto{\pgfqpoint{4.327111in}{4.123907in}}%
\pgfpathlineto{\pgfqpoint{4.333319in}{4.117782in}}%
\pgfpathlineto{\pgfqpoint{4.339286in}{4.112000in}}%
\pgfpathlineto{\pgfqpoint{4.367192in}{4.084690in}}%
\pgfpathlineto{\pgfqpoint{4.377408in}{4.074667in}}%
\pgfpathlineto{\pgfqpoint{4.391878in}{4.060327in}}%
\pgfpathlineto{\pgfqpoint{4.407273in}{4.045348in}}%
\pgfpathlineto{\pgfqpoint{4.415416in}{4.037333in}}%
\pgfpathlineto{\pgfqpoint{4.447354in}{4.005881in}}%
\pgfpathlineto{\pgfqpoint{4.450405in}{4.002842in}}%
\pgfpathlineto{\pgfqpoint{4.453311in}{4.000000in}}%
\pgfpathlineto{\pgfqpoint{4.487434in}{3.966290in}}%
\pgfpathlineto{\pgfqpoint{4.491093in}{3.962667in}}%
\pgfpathlineto{\pgfqpoint{4.508655in}{3.945099in}}%
\pgfpathlineto{\pgfqpoint{4.527515in}{3.926573in}}%
\pgfpathlineto{\pgfqpoint{4.528763in}{3.925333in}}%
\pgfpathlineto{\pgfqpoint{4.547320in}{3.906886in}}%
\pgfpathlineto{\pgfqpoint{4.566307in}{3.888000in}}%
\pgfpathlineto{\pgfqpoint{4.567596in}{3.886717in}}%
\pgfpathlineto{\pgfqpoint{4.586198in}{3.867993in}}%
\pgfpathlineto{\pgfqpoint{4.603726in}{3.850667in}}%
\pgfpathlineto{\pgfqpoint{4.607677in}{3.846722in}}%
\pgfpathlineto{\pgfqpoint{4.624877in}{3.829355in}}%
\pgfpathlineto{\pgfqpoint{4.641035in}{3.813333in}}%
\pgfpathlineto{\pgfqpoint{4.647758in}{3.806600in}}%
\pgfpathlineto{\pgfqpoint{4.678234in}{3.776000in}}%
\pgfpathlineto{\pgfqpoint{4.682831in}{3.771336in}}%
\pgfpathlineto{\pgfqpoint{4.687838in}{3.766350in}}%
\pgfpathlineto{\pgfqpoint{4.715323in}{3.738667in}}%
\pgfpathlineto{\pgfqpoint{4.721343in}{3.732541in}}%
\pgfpathlineto{\pgfqpoint{4.727919in}{3.725972in}}%
\pgfpathlineto{\pgfqpoint{4.752305in}{3.701333in}}%
\pgfpathlineto{\pgfqpoint{4.759794in}{3.693690in}}%
\pgfpathlineto{\pgfqpoint{4.768000in}{3.685465in}}%
\pgfusepath{fill}%
\end{pgfscope}%
\begin{pgfscope}%
\pgfpathrectangle{\pgfqpoint{0.800000in}{0.528000in}}{\pgfqpoint{3.968000in}{3.696000in}}%
\pgfusepath{clip}%
\pgfsetbuttcap%
\pgfsetroundjoin%
\definecolor{currentfill}{rgb}{0.421908,0.805774,0.351910}%
\pgfsetfillcolor{currentfill}%
\pgfsetlinewidth{0.000000pt}%
\definecolor{currentstroke}{rgb}{0.000000,0.000000,0.000000}%
\pgfsetstrokecolor{currentstroke}%
\pgfsetdash{}{0pt}%
\pgfpathmoveto{\pgfqpoint{4.768000in}{3.690832in}}%
\pgfpathlineto{\pgfqpoint{4.762569in}{3.696275in}}%
\pgfpathlineto{\pgfqpoint{4.757613in}{3.701333in}}%
\pgfpathlineto{\pgfqpoint{4.727919in}{3.731335in}}%
\pgfpathlineto{\pgfqpoint{4.724121in}{3.735129in}}%
\pgfpathlineto{\pgfqpoint{4.720645in}{3.738667in}}%
\pgfpathlineto{\pgfqpoint{4.687838in}{3.771709in}}%
\pgfpathlineto{\pgfqpoint{4.685612in}{3.773926in}}%
\pgfpathlineto{\pgfqpoint{4.683568in}{3.776000in}}%
\pgfpathlineto{\pgfqpoint{4.647758in}{3.811956in}}%
\pgfpathlineto{\pgfqpoint{4.646382in}{3.813333in}}%
\pgfpathlineto{\pgfqpoint{4.627634in}{3.831923in}}%
\pgfpathlineto{\pgfqpoint{4.609071in}{3.850667in}}%
\pgfpathlineto{\pgfqpoint{4.607677in}{3.852061in}}%
\pgfpathlineto{\pgfqpoint{4.588958in}{3.870564in}}%
\pgfpathlineto{\pgfqpoint{4.571636in}{3.888000in}}%
\pgfpathlineto{\pgfqpoint{4.567596in}{3.892026in}}%
\pgfpathlineto{\pgfqpoint{4.534090in}{3.925333in}}%
\pgfpathlineto{\pgfqpoint{4.527515in}{3.931865in}}%
\pgfpathlineto{\pgfqpoint{4.511420in}{3.947675in}}%
\pgfpathlineto{\pgfqpoint{4.496433in}{3.962667in}}%
\pgfpathlineto{\pgfqpoint{4.487434in}{3.971578in}}%
\pgfpathlineto{\pgfqpoint{4.458664in}{4.000000in}}%
\pgfpathlineto{\pgfqpoint{4.453147in}{4.005396in}}%
\pgfpathlineto{\pgfqpoint{4.447354in}{4.011166in}}%
\pgfpathlineto{\pgfqpoint{4.420783in}{4.037333in}}%
\pgfpathlineto{\pgfqpoint{4.407273in}{4.050629in}}%
\pgfpathlineto{\pgfqpoint{4.394651in}{4.062910in}}%
\pgfpathlineto{\pgfqpoint{4.382788in}{4.074667in}}%
\pgfpathlineto{\pgfqpoint{4.367192in}{4.089967in}}%
\pgfpathlineto{\pgfqpoint{4.344679in}{4.112000in}}%
\pgfpathlineto{\pgfqpoint{4.336068in}{4.120343in}}%
\pgfpathlineto{\pgfqpoint{4.327111in}{4.129181in}}%
\pgfpathlineto{\pgfqpoint{4.306454in}{4.149333in}}%
\pgfpathlineto{\pgfqpoint{4.287030in}{4.168270in}}%
\pgfpathlineto{\pgfqpoint{4.277324in}{4.177626in}}%
\pgfpathlineto{\pgfqpoint{4.268114in}{4.186667in}}%
\pgfpathlineto{\pgfqpoint{4.246949in}{4.207235in}}%
\pgfpathlineto{\pgfqpoint{4.229656in}{4.224000in}}%
\pgfpathlineto{\pgfqpoint{4.226939in}{4.224000in}}%
\pgfpathlineto{\pgfqpoint{4.246949in}{4.204602in}}%
\pgfpathlineto{\pgfqpoint{4.265404in}{4.186667in}}%
\pgfpathlineto{\pgfqpoint{4.275933in}{4.176330in}}%
\pgfpathlineto{\pgfqpoint{4.287030in}{4.165635in}}%
\pgfpathlineto{\pgfqpoint{4.303751in}{4.149333in}}%
\pgfpathlineto{\pgfqpoint{4.327111in}{4.126544in}}%
\pgfpathlineto{\pgfqpoint{4.334693in}{4.119063in}}%
\pgfpathlineto{\pgfqpoint{4.341982in}{4.112000in}}%
\pgfpathlineto{\pgfqpoint{4.367192in}{4.087328in}}%
\pgfpathlineto{\pgfqpoint{4.380098in}{4.074667in}}%
\pgfpathlineto{\pgfqpoint{4.393265in}{4.061619in}}%
\pgfpathlineto{\pgfqpoint{4.407273in}{4.047989in}}%
\pgfpathlineto{\pgfqpoint{4.418099in}{4.037333in}}%
\pgfpathlineto{\pgfqpoint{4.447354in}{4.008524in}}%
\pgfpathlineto{\pgfqpoint{4.451776in}{4.004119in}}%
\pgfpathlineto{\pgfqpoint{4.455987in}{4.000000in}}%
\pgfpathlineto{\pgfqpoint{4.487434in}{3.968934in}}%
\pgfpathlineto{\pgfqpoint{4.493763in}{3.962667in}}%
\pgfpathlineto{\pgfqpoint{4.510037in}{3.946387in}}%
\pgfpathlineto{\pgfqpoint{4.527515in}{3.929219in}}%
\pgfpathlineto{\pgfqpoint{4.531426in}{3.925333in}}%
\pgfpathlineto{\pgfqpoint{4.567596in}{3.889378in}}%
\pgfpathlineto{\pgfqpoint{4.568979in}{3.888000in}}%
\pgfpathlineto{\pgfqpoint{4.587578in}{3.869279in}}%
\pgfpathlineto{\pgfqpoint{4.606407in}{3.850667in}}%
\pgfpathlineto{\pgfqpoint{4.607677in}{3.849399in}}%
\pgfpathlineto{\pgfqpoint{4.626256in}{3.830639in}}%
\pgfpathlineto{\pgfqpoint{4.643709in}{3.813333in}}%
\pgfpathlineto{\pgfqpoint{4.647758in}{3.809278in}}%
\pgfpathlineto{\pgfqpoint{4.680901in}{3.776000in}}%
\pgfpathlineto{\pgfqpoint{4.684222in}{3.772631in}}%
\pgfpathlineto{\pgfqpoint{4.687838in}{3.769029in}}%
\pgfpathlineto{\pgfqpoint{4.717984in}{3.738667in}}%
\pgfpathlineto{\pgfqpoint{4.722732in}{3.733835in}}%
\pgfpathlineto{\pgfqpoint{4.727919in}{3.728653in}}%
\pgfpathlineto{\pgfqpoint{4.754959in}{3.701333in}}%
\pgfpathlineto{\pgfqpoint{4.761182in}{3.694982in}}%
\pgfpathlineto{\pgfqpoint{4.768000in}{3.688149in}}%
\pgfusepath{fill}%
\end{pgfscope}%
\begin{pgfscope}%
\pgfpathrectangle{\pgfqpoint{0.800000in}{0.528000in}}{\pgfqpoint{3.968000in}{3.696000in}}%
\pgfusepath{clip}%
\pgfsetbuttcap%
\pgfsetroundjoin%
\definecolor{currentfill}{rgb}{0.421908,0.805774,0.351910}%
\pgfsetfillcolor{currentfill}%
\pgfsetlinewidth{0.000000pt}%
\definecolor{currentstroke}{rgb}{0.000000,0.000000,0.000000}%
\pgfsetstrokecolor{currentstroke}%
\pgfsetdash{}{0pt}%
\pgfpathmoveto{\pgfqpoint{4.768000in}{3.693516in}}%
\pgfpathlineto{\pgfqpoint{4.763957in}{3.697567in}}%
\pgfpathlineto{\pgfqpoint{4.760267in}{3.701333in}}%
\pgfpathlineto{\pgfqpoint{4.727919in}{3.734016in}}%
\pgfpathlineto{\pgfqpoint{4.725510in}{3.736423in}}%
\pgfpathlineto{\pgfqpoint{4.723305in}{3.738667in}}%
\pgfpathlineto{\pgfqpoint{4.687838in}{3.774389in}}%
\pgfpathlineto{\pgfqpoint{4.687003in}{3.775221in}}%
\pgfpathlineto{\pgfqpoint{4.686235in}{3.776000in}}%
\pgfpathlineto{\pgfqpoint{4.665668in}{3.796651in}}%
\pgfpathlineto{\pgfqpoint{4.649042in}{3.813333in}}%
\pgfpathlineto{\pgfqpoint{4.647758in}{3.814621in}}%
\pgfpathlineto{\pgfqpoint{4.629013in}{3.833207in}}%
\pgfpathlineto{\pgfqpoint{4.611722in}{3.850667in}}%
\pgfpathlineto{\pgfqpoint{4.607677in}{3.854710in}}%
\pgfpathlineto{\pgfqpoint{4.590338in}{3.871849in}}%
\pgfpathlineto{\pgfqpoint{4.574292in}{3.888000in}}%
\pgfpathlineto{\pgfqpoint{4.567596in}{3.894673in}}%
\pgfpathlineto{\pgfqpoint{4.536753in}{3.925333in}}%
\pgfpathlineto{\pgfqpoint{4.527515in}{3.934510in}}%
\pgfpathlineto{\pgfqpoint{4.512802in}{3.948963in}}%
\pgfpathlineto{\pgfqpoint{4.499103in}{3.962667in}}%
\pgfpathlineto{\pgfqpoint{4.487434in}{3.974222in}}%
\pgfpathlineto{\pgfqpoint{4.461341in}{4.000000in}}%
\pgfpathlineto{\pgfqpoint{4.454518in}{4.006673in}}%
\pgfpathlineto{\pgfqpoint{4.447354in}{4.013809in}}%
\pgfpathlineto{\pgfqpoint{4.423466in}{4.037333in}}%
\pgfpathlineto{\pgfqpoint{4.407273in}{4.053270in}}%
\pgfpathlineto{\pgfqpoint{4.396038in}{4.064202in}}%
\pgfpathlineto{\pgfqpoint{4.385478in}{4.074667in}}%
\pgfpathlineto{\pgfqpoint{4.367192in}{4.092606in}}%
\pgfpathlineto{\pgfqpoint{4.347375in}{4.112000in}}%
\pgfpathlineto{\pgfqpoint{4.337443in}{4.121624in}}%
\pgfpathlineto{\pgfqpoint{4.327111in}{4.131818in}}%
\pgfpathlineto{\pgfqpoint{4.309157in}{4.149333in}}%
\pgfpathlineto{\pgfqpoint{4.287030in}{4.170906in}}%
\pgfpathlineto{\pgfqpoint{4.278714in}{4.178921in}}%
\pgfpathlineto{\pgfqpoint{4.270824in}{4.186667in}}%
\pgfpathlineto{\pgfqpoint{4.246949in}{4.209869in}}%
\pgfpathlineto{\pgfqpoint{4.232373in}{4.224000in}}%
\pgfpathlineto{\pgfqpoint{4.229656in}{4.224000in}}%
\pgfpathlineto{\pgfqpoint{4.246949in}{4.207235in}}%
\pgfpathlineto{\pgfqpoint{4.268114in}{4.186667in}}%
\pgfpathlineto{\pgfqpoint{4.277324in}{4.177626in}}%
\pgfpathlineto{\pgfqpoint{4.287030in}{4.168270in}}%
\pgfpathlineto{\pgfqpoint{4.306454in}{4.149333in}}%
\pgfpathlineto{\pgfqpoint{4.327111in}{4.129181in}}%
\pgfpathlineto{\pgfqpoint{4.336068in}{4.120343in}}%
\pgfpathlineto{\pgfqpoint{4.344679in}{4.112000in}}%
\pgfpathlineto{\pgfqpoint{4.367192in}{4.089967in}}%
\pgfpathlineto{\pgfqpoint{4.382788in}{4.074667in}}%
\pgfpathlineto{\pgfqpoint{4.394651in}{4.062910in}}%
\pgfpathlineto{\pgfqpoint{4.407273in}{4.050629in}}%
\pgfpathlineto{\pgfqpoint{4.420783in}{4.037333in}}%
\pgfpathlineto{\pgfqpoint{4.447354in}{4.011166in}}%
\pgfpathlineto{\pgfqpoint{4.453147in}{4.005396in}}%
\pgfpathlineto{\pgfqpoint{4.458664in}{4.000000in}}%
\pgfpathlineto{\pgfqpoint{4.487434in}{3.971578in}}%
\pgfpathlineto{\pgfqpoint{4.496433in}{3.962667in}}%
\pgfpathlineto{\pgfqpoint{4.511420in}{3.947675in}}%
\pgfpathlineto{\pgfqpoint{4.527515in}{3.931865in}}%
\pgfpathlineto{\pgfqpoint{4.534090in}{3.925333in}}%
\pgfpathlineto{\pgfqpoint{4.567596in}{3.892026in}}%
\pgfpathlineto{\pgfqpoint{4.571636in}{3.888000in}}%
\pgfpathlineto{\pgfqpoint{4.588958in}{3.870564in}}%
\pgfpathlineto{\pgfqpoint{4.607677in}{3.852061in}}%
\pgfpathlineto{\pgfqpoint{4.609071in}{3.850667in}}%
\pgfpathlineto{\pgfqpoint{4.627634in}{3.831923in}}%
\pgfpathlineto{\pgfqpoint{4.646382in}{3.813333in}}%
\pgfpathlineto{\pgfqpoint{4.647758in}{3.811956in}}%
\pgfpathlineto{\pgfqpoint{4.683568in}{3.776000in}}%
\pgfpathlineto{\pgfqpoint{4.685612in}{3.773926in}}%
\pgfpathlineto{\pgfqpoint{4.687838in}{3.771709in}}%
\pgfpathlineto{\pgfqpoint{4.720645in}{3.738667in}}%
\pgfpathlineto{\pgfqpoint{4.724121in}{3.735129in}}%
\pgfpathlineto{\pgfqpoint{4.727919in}{3.731335in}}%
\pgfpathlineto{\pgfqpoint{4.757613in}{3.701333in}}%
\pgfpathlineto{\pgfqpoint{4.762569in}{3.696275in}}%
\pgfpathlineto{\pgfqpoint{4.768000in}{3.690832in}}%
\pgfusepath{fill}%
\end{pgfscope}%
\begin{pgfscope}%
\pgfpathrectangle{\pgfqpoint{0.800000in}{0.528000in}}{\pgfqpoint{3.968000in}{3.696000in}}%
\pgfusepath{clip}%
\pgfsetbuttcap%
\pgfsetroundjoin%
\definecolor{currentfill}{rgb}{0.421908,0.805774,0.351910}%
\pgfsetfillcolor{currentfill}%
\pgfsetlinewidth{0.000000pt}%
\definecolor{currentstroke}{rgb}{0.000000,0.000000,0.000000}%
\pgfsetstrokecolor{currentstroke}%
\pgfsetdash{}{0pt}%
\pgfpathmoveto{\pgfqpoint{4.768000in}{3.696199in}}%
\pgfpathlineto{\pgfqpoint{4.765345in}{3.698860in}}%
\pgfpathlineto{\pgfqpoint{4.762922in}{3.701333in}}%
\pgfpathlineto{\pgfqpoint{4.727919in}{3.736698in}}%
\pgfpathlineto{\pgfqpoint{4.726899in}{3.737717in}}%
\pgfpathlineto{\pgfqpoint{4.725966in}{3.738667in}}%
\pgfpathlineto{\pgfqpoint{4.701948in}{3.762857in}}%
\pgfpathlineto{\pgfqpoint{4.688890in}{3.776000in}}%
\pgfpathlineto{\pgfqpoint{4.688382in}{3.776506in}}%
\pgfpathlineto{\pgfqpoint{4.687838in}{3.777058in}}%
\pgfpathlineto{\pgfqpoint{4.651686in}{3.813333in}}%
\pgfpathlineto{\pgfqpoint{4.647758in}{3.817272in}}%
\pgfpathlineto{\pgfqpoint{4.630392in}{3.834491in}}%
\pgfpathlineto{\pgfqpoint{4.614372in}{3.850667in}}%
\pgfpathlineto{\pgfqpoint{4.607677in}{3.857359in}}%
\pgfpathlineto{\pgfqpoint{4.591717in}{3.873135in}}%
\pgfpathlineto{\pgfqpoint{4.576949in}{3.888000in}}%
\pgfpathlineto{\pgfqpoint{4.567596in}{3.897321in}}%
\pgfpathlineto{\pgfqpoint{4.539417in}{3.925333in}}%
\pgfpathlineto{\pgfqpoint{4.527515in}{3.937156in}}%
\pgfpathlineto{\pgfqpoint{4.514185in}{3.950250in}}%
\pgfpathlineto{\pgfqpoint{4.501773in}{3.962667in}}%
\pgfpathlineto{\pgfqpoint{4.487434in}{3.976866in}}%
\pgfpathlineto{\pgfqpoint{4.464017in}{4.000000in}}%
\pgfpathlineto{\pgfqpoint{4.455889in}{4.007950in}}%
\pgfpathlineto{\pgfqpoint{4.447354in}{4.016451in}}%
\pgfpathlineto{\pgfqpoint{4.426149in}{4.037333in}}%
\pgfpathlineto{\pgfqpoint{4.407273in}{4.055910in}}%
\pgfpathlineto{\pgfqpoint{4.397424in}{4.065493in}}%
\pgfpathlineto{\pgfqpoint{4.388167in}{4.074667in}}%
\pgfpathlineto{\pgfqpoint{4.367192in}{4.095245in}}%
\pgfpathlineto{\pgfqpoint{4.350071in}{4.112000in}}%
\pgfpathlineto{\pgfqpoint{4.338818in}{4.122904in}}%
\pgfpathlineto{\pgfqpoint{4.327111in}{4.134455in}}%
\pgfpathlineto{\pgfqpoint{4.311860in}{4.149333in}}%
\pgfpathlineto{\pgfqpoint{4.287030in}{4.173541in}}%
\pgfpathlineto{\pgfqpoint{4.280105in}{4.180216in}}%
\pgfpathlineto{\pgfqpoint{4.273533in}{4.186667in}}%
\pgfpathlineto{\pgfqpoint{4.246949in}{4.212503in}}%
\pgfpathlineto{\pgfqpoint{4.235090in}{4.224000in}}%
\pgfpathlineto{\pgfqpoint{4.232373in}{4.224000in}}%
\pgfpathlineto{\pgfqpoint{4.246949in}{4.209869in}}%
\pgfpathlineto{\pgfqpoint{4.270824in}{4.186667in}}%
\pgfpathlineto{\pgfqpoint{4.278714in}{4.178921in}}%
\pgfpathlineto{\pgfqpoint{4.287030in}{4.170906in}}%
\pgfpathlineto{\pgfqpoint{4.309157in}{4.149333in}}%
\pgfpathlineto{\pgfqpoint{4.327111in}{4.131818in}}%
\pgfpathlineto{\pgfqpoint{4.337443in}{4.121624in}}%
\pgfpathlineto{\pgfqpoint{4.347375in}{4.112000in}}%
\pgfpathlineto{\pgfqpoint{4.367192in}{4.092606in}}%
\pgfpathlineto{\pgfqpoint{4.385478in}{4.074667in}}%
\pgfpathlineto{\pgfqpoint{4.396038in}{4.064202in}}%
\pgfpathlineto{\pgfqpoint{4.407273in}{4.053270in}}%
\pgfpathlineto{\pgfqpoint{4.423466in}{4.037333in}}%
\pgfpathlineto{\pgfqpoint{4.447354in}{4.013809in}}%
\pgfpathlineto{\pgfqpoint{4.454518in}{4.006673in}}%
\pgfpathlineto{\pgfqpoint{4.461341in}{4.000000in}}%
\pgfpathlineto{\pgfqpoint{4.487434in}{3.974222in}}%
\pgfpathlineto{\pgfqpoint{4.499103in}{3.962667in}}%
\pgfpathlineto{\pgfqpoint{4.512802in}{3.948963in}}%
\pgfpathlineto{\pgfqpoint{4.527515in}{3.934510in}}%
\pgfpathlineto{\pgfqpoint{4.536753in}{3.925333in}}%
\pgfpathlineto{\pgfqpoint{4.567596in}{3.894673in}}%
\pgfpathlineto{\pgfqpoint{4.574292in}{3.888000in}}%
\pgfpathlineto{\pgfqpoint{4.590338in}{3.871849in}}%
\pgfpathlineto{\pgfqpoint{4.607677in}{3.854710in}}%
\pgfpathlineto{\pgfqpoint{4.611722in}{3.850667in}}%
\pgfpathlineto{\pgfqpoint{4.629013in}{3.833207in}}%
\pgfpathlineto{\pgfqpoint{4.647758in}{3.814621in}}%
\pgfpathlineto{\pgfqpoint{4.649042in}{3.813333in}}%
\pgfpathlineto{\pgfqpoint{4.665668in}{3.796651in}}%
\pgfpathlineto{\pgfqpoint{4.686235in}{3.776000in}}%
\pgfpathlineto{\pgfqpoint{4.687003in}{3.775221in}}%
\pgfpathlineto{\pgfqpoint{4.687838in}{3.774389in}}%
\pgfpathlineto{\pgfqpoint{4.723305in}{3.738667in}}%
\pgfpathlineto{\pgfqpoint{4.725510in}{3.736423in}}%
\pgfpathlineto{\pgfqpoint{4.727919in}{3.734016in}}%
\pgfpathlineto{\pgfqpoint{4.760267in}{3.701333in}}%
\pgfpathlineto{\pgfqpoint{4.763957in}{3.697567in}}%
\pgfpathlineto{\pgfqpoint{4.768000in}{3.693516in}}%
\pgfusepath{fill}%
\end{pgfscope}%
\begin{pgfscope}%
\pgfpathrectangle{\pgfqpoint{0.800000in}{0.528000in}}{\pgfqpoint{3.968000in}{3.696000in}}%
\pgfusepath{clip}%
\pgfsetbuttcap%
\pgfsetroundjoin%
\definecolor{currentfill}{rgb}{0.430983,0.808473,0.346476}%
\pgfsetfillcolor{currentfill}%
\pgfsetlinewidth{0.000000pt}%
\definecolor{currentstroke}{rgb}{0.000000,0.000000,0.000000}%
\pgfsetstrokecolor{currentstroke}%
\pgfsetdash{}{0pt}%
\pgfpathmoveto{\pgfqpoint{4.768000in}{3.698882in}}%
\pgfpathlineto{\pgfqpoint{4.766732in}{3.700153in}}%
\pgfpathlineto{\pgfqpoint{4.765576in}{3.701333in}}%
\pgfpathlineto{\pgfqpoint{4.736955in}{3.730251in}}%
\pgfpathlineto{\pgfqpoint{4.728619in}{3.738667in}}%
\pgfpathlineto{\pgfqpoint{4.728281in}{3.739004in}}%
\pgfpathlineto{\pgfqpoint{4.727919in}{3.739372in}}%
\pgfpathlineto{\pgfqpoint{4.691528in}{3.776000in}}%
\pgfpathlineto{\pgfqpoint{4.689745in}{3.777776in}}%
\pgfpathlineto{\pgfqpoint{4.687838in}{3.779711in}}%
\pgfpathlineto{\pgfqpoint{4.654329in}{3.813333in}}%
\pgfpathlineto{\pgfqpoint{4.647758in}{3.819923in}}%
\pgfpathlineto{\pgfqpoint{4.631770in}{3.835775in}}%
\pgfpathlineto{\pgfqpoint{4.617023in}{3.850667in}}%
\pgfpathlineto{\pgfqpoint{4.607677in}{3.860009in}}%
\pgfpathlineto{\pgfqpoint{4.593097in}{3.874420in}}%
\pgfpathlineto{\pgfqpoint{4.579606in}{3.888000in}}%
\pgfpathlineto{\pgfqpoint{4.567596in}{3.899968in}}%
\pgfpathlineto{\pgfqpoint{4.542080in}{3.925333in}}%
\pgfpathlineto{\pgfqpoint{4.527515in}{3.939802in}}%
\pgfpathlineto{\pgfqpoint{4.515568in}{3.951538in}}%
\pgfpathlineto{\pgfqpoint{4.504443in}{3.962667in}}%
\pgfpathlineto{\pgfqpoint{4.487434in}{3.979510in}}%
\pgfpathlineto{\pgfqpoint{4.466694in}{4.000000in}}%
\pgfpathlineto{\pgfqpoint{4.457260in}{4.009227in}}%
\pgfpathlineto{\pgfqpoint{4.447354in}{4.019093in}}%
\pgfpathlineto{\pgfqpoint{4.428832in}{4.037333in}}%
\pgfpathlineto{\pgfqpoint{4.407273in}{4.058551in}}%
\pgfpathlineto{\pgfqpoint{4.398811in}{4.066785in}}%
\pgfpathlineto{\pgfqpoint{4.390857in}{4.074667in}}%
\pgfpathlineto{\pgfqpoint{4.367192in}{4.097884in}}%
\pgfpathlineto{\pgfqpoint{4.352768in}{4.112000in}}%
\pgfpathlineto{\pgfqpoint{4.340193in}{4.124185in}}%
\pgfpathlineto{\pgfqpoint{4.327111in}{4.137092in}}%
\pgfpathlineto{\pgfqpoint{4.314564in}{4.149333in}}%
\pgfpathlineto{\pgfqpoint{4.287030in}{4.176176in}}%
\pgfpathlineto{\pgfqpoint{4.281495in}{4.181511in}}%
\pgfpathlineto{\pgfqpoint{4.276243in}{4.186667in}}%
\pgfpathlineto{\pgfqpoint{4.246949in}{4.215136in}}%
\pgfpathlineto{\pgfqpoint{4.237806in}{4.224000in}}%
\pgfpathlineto{\pgfqpoint{4.235090in}{4.224000in}}%
\pgfpathlineto{\pgfqpoint{4.246949in}{4.212503in}}%
\pgfpathlineto{\pgfqpoint{4.273533in}{4.186667in}}%
\pgfpathlineto{\pgfqpoint{4.280105in}{4.180216in}}%
\pgfpathlineto{\pgfqpoint{4.287030in}{4.173541in}}%
\pgfpathlineto{\pgfqpoint{4.311860in}{4.149333in}}%
\pgfpathlineto{\pgfqpoint{4.327111in}{4.134455in}}%
\pgfpathlineto{\pgfqpoint{4.338818in}{4.122904in}}%
\pgfpathlineto{\pgfqpoint{4.350071in}{4.112000in}}%
\pgfpathlineto{\pgfqpoint{4.367192in}{4.095245in}}%
\pgfpathlineto{\pgfqpoint{4.388167in}{4.074667in}}%
\pgfpathlineto{\pgfqpoint{4.397424in}{4.065493in}}%
\pgfpathlineto{\pgfqpoint{4.407273in}{4.055910in}}%
\pgfpathlineto{\pgfqpoint{4.426149in}{4.037333in}}%
\pgfpathlineto{\pgfqpoint{4.447354in}{4.016451in}}%
\pgfpathlineto{\pgfqpoint{4.455889in}{4.007950in}}%
\pgfpathlineto{\pgfqpoint{4.464017in}{4.000000in}}%
\pgfpathlineto{\pgfqpoint{4.487434in}{3.976866in}}%
\pgfpathlineto{\pgfqpoint{4.501773in}{3.962667in}}%
\pgfpathlineto{\pgfqpoint{4.514185in}{3.950250in}}%
\pgfpathlineto{\pgfqpoint{4.527515in}{3.937156in}}%
\pgfpathlineto{\pgfqpoint{4.539417in}{3.925333in}}%
\pgfpathlineto{\pgfqpoint{4.567596in}{3.897321in}}%
\pgfpathlineto{\pgfqpoint{4.576949in}{3.888000in}}%
\pgfpathlineto{\pgfqpoint{4.591717in}{3.873135in}}%
\pgfpathlineto{\pgfqpoint{4.607677in}{3.857359in}}%
\pgfpathlineto{\pgfqpoint{4.614372in}{3.850667in}}%
\pgfpathlineto{\pgfqpoint{4.630392in}{3.834491in}}%
\pgfpathlineto{\pgfqpoint{4.647758in}{3.817272in}}%
\pgfpathlineto{\pgfqpoint{4.651686in}{3.813333in}}%
\pgfpathlineto{\pgfqpoint{4.687838in}{3.777058in}}%
\pgfpathlineto{\pgfqpoint{4.688382in}{3.776506in}}%
\pgfpathlineto{\pgfqpoint{4.688890in}{3.776000in}}%
\pgfpathlineto{\pgfqpoint{4.701948in}{3.762857in}}%
\pgfpathlineto{\pgfqpoint{4.725966in}{3.738667in}}%
\pgfpathlineto{\pgfqpoint{4.726899in}{3.737717in}}%
\pgfpathlineto{\pgfqpoint{4.727919in}{3.736698in}}%
\pgfpathlineto{\pgfqpoint{4.762922in}{3.701333in}}%
\pgfpathlineto{\pgfqpoint{4.765345in}{3.698860in}}%
\pgfpathlineto{\pgfqpoint{4.768000in}{3.696199in}}%
\pgfusepath{fill}%
\end{pgfscope}%
\begin{pgfscope}%
\pgfpathrectangle{\pgfqpoint{0.800000in}{0.528000in}}{\pgfqpoint{3.968000in}{3.696000in}}%
\pgfusepath{clip}%
\pgfsetbuttcap%
\pgfsetroundjoin%
\definecolor{currentfill}{rgb}{0.430983,0.808473,0.346476}%
\pgfsetfillcolor{currentfill}%
\pgfsetlinewidth{0.000000pt}%
\definecolor{currentstroke}{rgb}{0.000000,0.000000,0.000000}%
\pgfsetstrokecolor{currentstroke}%
\pgfsetdash{}{0pt}%
\pgfpathmoveto{\pgfqpoint{4.768000in}{3.701563in}}%
\pgfpathlineto{\pgfqpoint{4.731250in}{3.738667in}}%
\pgfpathlineto{\pgfqpoint{4.729643in}{3.740273in}}%
\pgfpathlineto{\pgfqpoint{4.727919in}{3.742027in}}%
\pgfpathlineto{\pgfqpoint{4.694165in}{3.776000in}}%
\pgfpathlineto{\pgfqpoint{4.691109in}{3.779046in}}%
\pgfpathlineto{\pgfqpoint{4.687838in}{3.782364in}}%
\pgfpathlineto{\pgfqpoint{4.656973in}{3.813333in}}%
\pgfpathlineto{\pgfqpoint{4.647758in}{3.822574in}}%
\pgfpathlineto{\pgfqpoint{4.633149in}{3.837060in}}%
\pgfpathlineto{\pgfqpoint{4.619673in}{3.850667in}}%
\pgfpathlineto{\pgfqpoint{4.607677in}{3.862658in}}%
\pgfpathlineto{\pgfqpoint{4.594477in}{3.875705in}}%
\pgfpathlineto{\pgfqpoint{4.582263in}{3.888000in}}%
\pgfpathlineto{\pgfqpoint{4.567596in}{3.902616in}}%
\pgfpathlineto{\pgfqpoint{4.544743in}{3.925333in}}%
\pgfpathlineto{\pgfqpoint{4.527515in}{3.942448in}}%
\pgfpathlineto{\pgfqpoint{4.516950in}{3.952826in}}%
\pgfpathlineto{\pgfqpoint{4.507113in}{3.962667in}}%
\pgfpathlineto{\pgfqpoint{4.487434in}{3.982155in}}%
\pgfpathlineto{\pgfqpoint{4.469370in}{4.000000in}}%
\pgfpathlineto{\pgfqpoint{4.458631in}{4.010504in}}%
\pgfpathlineto{\pgfqpoint{4.447354in}{4.021736in}}%
\pgfpathlineto{\pgfqpoint{4.431515in}{4.037333in}}%
\pgfpathlineto{\pgfqpoint{4.407273in}{4.061192in}}%
\pgfpathlineto{\pgfqpoint{4.400197in}{4.068076in}}%
\pgfpathlineto{\pgfqpoint{4.393547in}{4.074667in}}%
\pgfpathlineto{\pgfqpoint{4.367192in}{4.100523in}}%
\pgfpathlineto{\pgfqpoint{4.355464in}{4.112000in}}%
\pgfpathlineto{\pgfqpoint{4.341568in}{4.125466in}}%
\pgfpathlineto{\pgfqpoint{4.327111in}{4.139730in}}%
\pgfpathlineto{\pgfqpoint{4.317267in}{4.149333in}}%
\pgfpathlineto{\pgfqpoint{4.287030in}{4.178812in}}%
\pgfpathlineto{\pgfqpoint{4.282886in}{4.182806in}}%
\pgfpathlineto{\pgfqpoint{4.278953in}{4.186667in}}%
\pgfpathlineto{\pgfqpoint{4.246949in}{4.217770in}}%
\pgfpathlineto{\pgfqpoint{4.240523in}{4.224000in}}%
\pgfpathlineto{\pgfqpoint{4.237806in}{4.224000in}}%
\pgfpathlineto{\pgfqpoint{4.246949in}{4.215136in}}%
\pgfpathlineto{\pgfqpoint{4.276243in}{4.186667in}}%
\pgfpathlineto{\pgfqpoint{4.281495in}{4.181511in}}%
\pgfpathlineto{\pgfqpoint{4.287030in}{4.176176in}}%
\pgfpathlineto{\pgfqpoint{4.314564in}{4.149333in}}%
\pgfpathlineto{\pgfqpoint{4.327111in}{4.137092in}}%
\pgfpathlineto{\pgfqpoint{4.340193in}{4.124185in}}%
\pgfpathlineto{\pgfqpoint{4.352768in}{4.112000in}}%
\pgfpathlineto{\pgfqpoint{4.367192in}{4.097884in}}%
\pgfpathlineto{\pgfqpoint{4.390857in}{4.074667in}}%
\pgfpathlineto{\pgfqpoint{4.398811in}{4.066785in}}%
\pgfpathlineto{\pgfqpoint{4.407273in}{4.058551in}}%
\pgfpathlineto{\pgfqpoint{4.428832in}{4.037333in}}%
\pgfpathlineto{\pgfqpoint{4.447354in}{4.019093in}}%
\pgfpathlineto{\pgfqpoint{4.457260in}{4.009227in}}%
\pgfpathlineto{\pgfqpoint{4.466694in}{4.000000in}}%
\pgfpathlineto{\pgfqpoint{4.487434in}{3.979510in}}%
\pgfpathlineto{\pgfqpoint{4.504443in}{3.962667in}}%
\pgfpathlineto{\pgfqpoint{4.515568in}{3.951538in}}%
\pgfpathlineto{\pgfqpoint{4.527515in}{3.939802in}}%
\pgfpathlineto{\pgfqpoint{4.542080in}{3.925333in}}%
\pgfpathlineto{\pgfqpoint{4.567596in}{3.899968in}}%
\pgfpathlineto{\pgfqpoint{4.579606in}{3.888000in}}%
\pgfpathlineto{\pgfqpoint{4.593097in}{3.874420in}}%
\pgfpathlineto{\pgfqpoint{4.607677in}{3.860009in}}%
\pgfpathlineto{\pgfqpoint{4.617023in}{3.850667in}}%
\pgfpathlineto{\pgfqpoint{4.631770in}{3.835775in}}%
\pgfpathlineto{\pgfqpoint{4.647758in}{3.819923in}}%
\pgfpathlineto{\pgfqpoint{4.654329in}{3.813333in}}%
\pgfpathlineto{\pgfqpoint{4.687838in}{3.779711in}}%
\pgfpathlineto{\pgfqpoint{4.689745in}{3.777776in}}%
\pgfpathlineto{\pgfqpoint{4.691528in}{3.776000in}}%
\pgfpathlineto{\pgfqpoint{4.727919in}{3.739372in}}%
\pgfpathlineto{\pgfqpoint{4.728281in}{3.739004in}}%
\pgfpathlineto{\pgfqpoint{4.728619in}{3.738667in}}%
\pgfpathlineto{\pgfqpoint{4.736955in}{3.730251in}}%
\pgfpathlineto{\pgfqpoint{4.765576in}{3.701333in}}%
\pgfpathlineto{\pgfqpoint{4.766732in}{3.700153in}}%
\pgfpathlineto{\pgfqpoint{4.768000in}{3.698882in}}%
\pgfpathlineto{\pgfqpoint{4.768000in}{3.701333in}}%
\pgfusepath{fill}%
\end{pgfscope}%
\begin{pgfscope}%
\pgfpathrectangle{\pgfqpoint{0.800000in}{0.528000in}}{\pgfqpoint{3.968000in}{3.696000in}}%
\pgfusepath{clip}%
\pgfsetbuttcap%
\pgfsetroundjoin%
\definecolor{currentfill}{rgb}{0.430983,0.808473,0.346476}%
\pgfsetfillcolor{currentfill}%
\pgfsetlinewidth{0.000000pt}%
\definecolor{currentstroke}{rgb}{0.000000,0.000000,0.000000}%
\pgfsetstrokecolor{currentstroke}%
\pgfsetdash{}{0pt}%
\pgfpathmoveto{\pgfqpoint{4.768000in}{3.704220in}}%
\pgfpathlineto{\pgfqpoint{4.733881in}{3.738667in}}%
\pgfpathlineto{\pgfqpoint{4.731005in}{3.741541in}}%
\pgfpathlineto{\pgfqpoint{4.727919in}{3.744682in}}%
\pgfpathlineto{\pgfqpoint{4.696803in}{3.776000in}}%
\pgfpathlineto{\pgfqpoint{4.692472in}{3.780316in}}%
\pgfpathlineto{\pgfqpoint{4.687838in}{3.785017in}}%
\pgfpathlineto{\pgfqpoint{4.659617in}{3.813333in}}%
\pgfpathlineto{\pgfqpoint{4.647758in}{3.825225in}}%
\pgfpathlineto{\pgfqpoint{4.634528in}{3.838344in}}%
\pgfpathlineto{\pgfqpoint{4.622323in}{3.850667in}}%
\pgfpathlineto{\pgfqpoint{4.607677in}{3.865308in}}%
\pgfpathlineto{\pgfqpoint{4.595857in}{3.876991in}}%
\pgfpathlineto{\pgfqpoint{4.584920in}{3.888000in}}%
\pgfpathlineto{\pgfqpoint{4.567596in}{3.905264in}}%
\pgfpathlineto{\pgfqpoint{4.547407in}{3.925333in}}%
\pgfpathlineto{\pgfqpoint{4.527515in}{3.945094in}}%
\pgfpathlineto{\pgfqpoint{4.518333in}{3.954114in}}%
\pgfpathlineto{\pgfqpoint{4.509783in}{3.962667in}}%
\pgfpathlineto{\pgfqpoint{4.487434in}{3.984799in}}%
\pgfpathlineto{\pgfqpoint{4.472047in}{4.000000in}}%
\pgfpathlineto{\pgfqpoint{4.460002in}{4.011781in}}%
\pgfpathlineto{\pgfqpoint{4.447354in}{4.024378in}}%
\pgfpathlineto{\pgfqpoint{4.434198in}{4.037333in}}%
\pgfpathlineto{\pgfqpoint{4.407273in}{4.063832in}}%
\pgfpathlineto{\pgfqpoint{4.401584in}{4.069368in}}%
\pgfpathlineto{\pgfqpoint{4.396237in}{4.074667in}}%
\pgfpathlineto{\pgfqpoint{4.367192in}{4.103162in}}%
\pgfpathlineto{\pgfqpoint{4.358161in}{4.112000in}}%
\pgfpathlineto{\pgfqpoint{4.342943in}{4.126746in}}%
\pgfpathlineto{\pgfqpoint{4.327111in}{4.142367in}}%
\pgfpathlineto{\pgfqpoint{4.319970in}{4.149333in}}%
\pgfpathlineto{\pgfqpoint{4.287030in}{4.181447in}}%
\pgfpathlineto{\pgfqpoint{4.284276in}{4.184102in}}%
\pgfpathlineto{\pgfqpoint{4.281663in}{4.186667in}}%
\pgfpathlineto{\pgfqpoint{4.246949in}{4.220404in}}%
\pgfpathlineto{\pgfqpoint{4.243240in}{4.224000in}}%
\pgfpathlineto{\pgfqpoint{4.240523in}{4.224000in}}%
\pgfpathlineto{\pgfqpoint{4.246949in}{4.217770in}}%
\pgfpathlineto{\pgfqpoint{4.278953in}{4.186667in}}%
\pgfpathlineto{\pgfqpoint{4.282886in}{4.182806in}}%
\pgfpathlineto{\pgfqpoint{4.287030in}{4.178812in}}%
\pgfpathlineto{\pgfqpoint{4.317267in}{4.149333in}}%
\pgfpathlineto{\pgfqpoint{4.327111in}{4.139730in}}%
\pgfpathlineto{\pgfqpoint{4.341568in}{4.125466in}}%
\pgfpathlineto{\pgfqpoint{4.355464in}{4.112000in}}%
\pgfpathlineto{\pgfqpoint{4.367192in}{4.100523in}}%
\pgfpathlineto{\pgfqpoint{4.393547in}{4.074667in}}%
\pgfpathlineto{\pgfqpoint{4.400197in}{4.068076in}}%
\pgfpathlineto{\pgfqpoint{4.407273in}{4.061192in}}%
\pgfpathlineto{\pgfqpoint{4.431515in}{4.037333in}}%
\pgfpathlineto{\pgfqpoint{4.447354in}{4.021736in}}%
\pgfpathlineto{\pgfqpoint{4.458631in}{4.010504in}}%
\pgfpathlineto{\pgfqpoint{4.469370in}{4.000000in}}%
\pgfpathlineto{\pgfqpoint{4.487434in}{3.982155in}}%
\pgfpathlineto{\pgfqpoint{4.507113in}{3.962667in}}%
\pgfpathlineto{\pgfqpoint{4.516950in}{3.952826in}}%
\pgfpathlineto{\pgfqpoint{4.527515in}{3.942448in}}%
\pgfpathlineto{\pgfqpoint{4.544743in}{3.925333in}}%
\pgfpathlineto{\pgfqpoint{4.567596in}{3.902616in}}%
\pgfpathlineto{\pgfqpoint{4.582263in}{3.888000in}}%
\pgfpathlineto{\pgfqpoint{4.594477in}{3.875705in}}%
\pgfpathlineto{\pgfqpoint{4.607677in}{3.862658in}}%
\pgfpathlineto{\pgfqpoint{4.619673in}{3.850667in}}%
\pgfpathlineto{\pgfqpoint{4.633149in}{3.837060in}}%
\pgfpathlineto{\pgfqpoint{4.647758in}{3.822574in}}%
\pgfpathlineto{\pgfqpoint{4.656973in}{3.813333in}}%
\pgfpathlineto{\pgfqpoint{4.687838in}{3.782364in}}%
\pgfpathlineto{\pgfqpoint{4.691109in}{3.779046in}}%
\pgfpathlineto{\pgfqpoint{4.694165in}{3.776000in}}%
\pgfpathlineto{\pgfqpoint{4.727919in}{3.742027in}}%
\pgfpathlineto{\pgfqpoint{4.729643in}{3.740273in}}%
\pgfpathlineto{\pgfqpoint{4.731250in}{3.738667in}}%
\pgfpathlineto{\pgfqpoint{4.768000in}{3.701563in}}%
\pgfusepath{fill}%
\end{pgfscope}%
\begin{pgfscope}%
\pgfpathrectangle{\pgfqpoint{0.800000in}{0.528000in}}{\pgfqpoint{3.968000in}{3.696000in}}%
\pgfusepath{clip}%
\pgfsetbuttcap%
\pgfsetroundjoin%
\definecolor{currentfill}{rgb}{0.430983,0.808473,0.346476}%
\pgfsetfillcolor{currentfill}%
\pgfsetlinewidth{0.000000pt}%
\definecolor{currentstroke}{rgb}{0.000000,0.000000,0.000000}%
\pgfsetstrokecolor{currentstroke}%
\pgfsetdash{}{0pt}%
\pgfpathmoveto{\pgfqpoint{4.768000in}{3.706876in}}%
\pgfpathlineto{\pgfqpoint{4.736512in}{3.738667in}}%
\pgfpathlineto{\pgfqpoint{4.732367in}{3.742810in}}%
\pgfpathlineto{\pgfqpoint{4.727919in}{3.747336in}}%
\pgfpathlineto{\pgfqpoint{4.699441in}{3.776000in}}%
\pgfpathlineto{\pgfqpoint{4.693835in}{3.781586in}}%
\pgfpathlineto{\pgfqpoint{4.687838in}{3.787670in}}%
\pgfpathlineto{\pgfqpoint{4.662261in}{3.813333in}}%
\pgfpathlineto{\pgfqpoint{4.647758in}{3.827877in}}%
\pgfpathlineto{\pgfqpoint{4.635906in}{3.839628in}}%
\pgfpathlineto{\pgfqpoint{4.624974in}{3.850667in}}%
\pgfpathlineto{\pgfqpoint{4.607677in}{3.867957in}}%
\pgfpathlineto{\pgfqpoint{4.597237in}{3.878276in}}%
\pgfpathlineto{\pgfqpoint{4.587577in}{3.888000in}}%
\pgfpathlineto{\pgfqpoint{4.567596in}{3.907911in}}%
\pgfpathlineto{\pgfqpoint{4.550070in}{3.925333in}}%
\pgfpathlineto{\pgfqpoint{4.527515in}{3.947740in}}%
\pgfpathlineto{\pgfqpoint{4.519715in}{3.955402in}}%
\pgfpathlineto{\pgfqpoint{4.512453in}{3.962667in}}%
\pgfpathlineto{\pgfqpoint{4.487434in}{3.987443in}}%
\pgfpathlineto{\pgfqpoint{4.474723in}{4.000000in}}%
\pgfpathlineto{\pgfqpoint{4.461373in}{4.013058in}}%
\pgfpathlineto{\pgfqpoint{4.447354in}{4.027021in}}%
\pgfpathlineto{\pgfqpoint{4.436882in}{4.037333in}}%
\pgfpathlineto{\pgfqpoint{4.407273in}{4.066473in}}%
\pgfpathlineto{\pgfqpoint{4.402970in}{4.070659in}}%
\pgfpathlineto{\pgfqpoint{4.398927in}{4.074667in}}%
\pgfpathlineto{\pgfqpoint{4.367192in}{4.105801in}}%
\pgfpathlineto{\pgfqpoint{4.360857in}{4.112000in}}%
\pgfpathlineto{\pgfqpoint{4.344317in}{4.128027in}}%
\pgfpathlineto{\pgfqpoint{4.327111in}{4.145004in}}%
\pgfpathlineto{\pgfqpoint{4.322673in}{4.149333in}}%
\pgfpathlineto{\pgfqpoint{4.287030in}{4.184083in}}%
\pgfpathlineto{\pgfqpoint{4.285667in}{4.185397in}}%
\pgfpathlineto{\pgfqpoint{4.284373in}{4.186667in}}%
\pgfpathlineto{\pgfqpoint{4.246949in}{4.223037in}}%
\pgfpathlineto{\pgfqpoint{4.245956in}{4.224000in}}%
\pgfpathlineto{\pgfqpoint{4.243240in}{4.224000in}}%
\pgfpathlineto{\pgfqpoint{4.246949in}{4.220404in}}%
\pgfpathlineto{\pgfqpoint{4.281663in}{4.186667in}}%
\pgfpathlineto{\pgfqpoint{4.284276in}{4.184102in}}%
\pgfpathlineto{\pgfqpoint{4.287030in}{4.181447in}}%
\pgfpathlineto{\pgfqpoint{4.319970in}{4.149333in}}%
\pgfpathlineto{\pgfqpoint{4.327111in}{4.142367in}}%
\pgfpathlineto{\pgfqpoint{4.342943in}{4.126746in}}%
\pgfpathlineto{\pgfqpoint{4.358161in}{4.112000in}}%
\pgfpathlineto{\pgfqpoint{4.367192in}{4.103162in}}%
\pgfpathlineto{\pgfqpoint{4.396237in}{4.074667in}}%
\pgfpathlineto{\pgfqpoint{4.401584in}{4.069368in}}%
\pgfpathlineto{\pgfqpoint{4.407273in}{4.063832in}}%
\pgfpathlineto{\pgfqpoint{4.434198in}{4.037333in}}%
\pgfpathlineto{\pgfqpoint{4.447354in}{4.024378in}}%
\pgfpathlineto{\pgfqpoint{4.460002in}{4.011781in}}%
\pgfpathlineto{\pgfqpoint{4.472047in}{4.000000in}}%
\pgfpathlineto{\pgfqpoint{4.487434in}{3.984799in}}%
\pgfpathlineto{\pgfqpoint{4.509783in}{3.962667in}}%
\pgfpathlineto{\pgfqpoint{4.518333in}{3.954114in}}%
\pgfpathlineto{\pgfqpoint{4.527515in}{3.945094in}}%
\pgfpathlineto{\pgfqpoint{4.547407in}{3.925333in}}%
\pgfpathlineto{\pgfqpoint{4.567596in}{3.905264in}}%
\pgfpathlineto{\pgfqpoint{4.584920in}{3.888000in}}%
\pgfpathlineto{\pgfqpoint{4.595857in}{3.876991in}}%
\pgfpathlineto{\pgfqpoint{4.607677in}{3.865308in}}%
\pgfpathlineto{\pgfqpoint{4.622323in}{3.850667in}}%
\pgfpathlineto{\pgfqpoint{4.634528in}{3.838344in}}%
\pgfpathlineto{\pgfqpoint{4.647758in}{3.825225in}}%
\pgfpathlineto{\pgfqpoint{4.659617in}{3.813333in}}%
\pgfpathlineto{\pgfqpoint{4.687838in}{3.785017in}}%
\pgfpathlineto{\pgfqpoint{4.692472in}{3.780316in}}%
\pgfpathlineto{\pgfqpoint{4.696803in}{3.776000in}}%
\pgfpathlineto{\pgfqpoint{4.727919in}{3.744682in}}%
\pgfpathlineto{\pgfqpoint{4.731005in}{3.741541in}}%
\pgfpathlineto{\pgfqpoint{4.733881in}{3.738667in}}%
\pgfpathlineto{\pgfqpoint{4.768000in}{3.704220in}}%
\pgfusepath{fill}%
\end{pgfscope}%
\begin{pgfscope}%
\pgfpathrectangle{\pgfqpoint{0.800000in}{0.528000in}}{\pgfqpoint{3.968000in}{3.696000in}}%
\pgfusepath{clip}%
\pgfsetbuttcap%
\pgfsetroundjoin%
\definecolor{currentfill}{rgb}{0.440137,0.811138,0.340967}%
\pgfsetfillcolor{currentfill}%
\pgfsetlinewidth{0.000000pt}%
\definecolor{currentstroke}{rgb}{0.000000,0.000000,0.000000}%
\pgfsetstrokecolor{currentstroke}%
\pgfsetdash{}{0pt}%
\pgfpathmoveto{\pgfqpoint{4.768000in}{3.709533in}}%
\pgfpathlineto{\pgfqpoint{4.739143in}{3.738667in}}%
\pgfpathlineto{\pgfqpoint{4.733729in}{3.744078in}}%
\pgfpathlineto{\pgfqpoint{4.727919in}{3.749991in}}%
\pgfpathlineto{\pgfqpoint{4.702078in}{3.776000in}}%
\pgfpathlineto{\pgfqpoint{4.695198in}{3.782856in}}%
\pgfpathlineto{\pgfqpoint{4.687838in}{3.790323in}}%
\pgfpathlineto{\pgfqpoint{4.664905in}{3.813333in}}%
\pgfpathlineto{\pgfqpoint{4.647758in}{3.830528in}}%
\pgfpathlineto{\pgfqpoint{4.637285in}{3.840912in}}%
\pgfpathlineto{\pgfqpoint{4.627624in}{3.850667in}}%
\pgfpathlineto{\pgfqpoint{4.607677in}{3.870606in}}%
\pgfpathlineto{\pgfqpoint{4.598617in}{3.879561in}}%
\pgfpathlineto{\pgfqpoint{4.590234in}{3.888000in}}%
\pgfpathlineto{\pgfqpoint{4.567596in}{3.910559in}}%
\pgfpathlineto{\pgfqpoint{4.552734in}{3.925333in}}%
\pgfpathlineto{\pgfqpoint{4.527515in}{3.950386in}}%
\pgfpathlineto{\pgfqpoint{4.521098in}{3.956689in}}%
\pgfpathlineto{\pgfqpoint{4.515123in}{3.962667in}}%
\pgfpathlineto{\pgfqpoint{4.487434in}{3.990087in}}%
\pgfpathlineto{\pgfqpoint{4.477400in}{4.000000in}}%
\pgfpathlineto{\pgfqpoint{4.462744in}{4.014335in}}%
\pgfpathlineto{\pgfqpoint{4.447354in}{4.029663in}}%
\pgfpathlineto{\pgfqpoint{4.439565in}{4.037333in}}%
\pgfpathlineto{\pgfqpoint{4.407273in}{4.069114in}}%
\pgfpathlineto{\pgfqpoint{4.404357in}{4.071951in}}%
\pgfpathlineto{\pgfqpoint{4.401616in}{4.074667in}}%
\pgfpathlineto{\pgfqpoint{4.367192in}{4.108440in}}%
\pgfpathlineto{\pgfqpoint{4.363554in}{4.112000in}}%
\pgfpathlineto{\pgfqpoint{4.345692in}{4.129308in}}%
\pgfpathlineto{\pgfqpoint{4.327111in}{4.147641in}}%
\pgfpathlineto{\pgfqpoint{4.325376in}{4.149333in}}%
\pgfpathlineto{\pgfqpoint{4.288213in}{4.185565in}}%
\pgfpathlineto{\pgfqpoint{4.287083in}{4.186667in}}%
\pgfpathlineto{\pgfqpoint{4.287057in}{4.186691in}}%
\pgfpathlineto{\pgfqpoint{4.287030in}{4.186718in}}%
\pgfpathlineto{\pgfqpoint{4.248654in}{4.224000in}}%
\pgfpathlineto{\pgfqpoint{4.246949in}{4.224000in}}%
\pgfpathlineto{\pgfqpoint{4.245956in}{4.224000in}}%
\pgfpathlineto{\pgfqpoint{4.246949in}{4.223037in}}%
\pgfpathlineto{\pgfqpoint{4.284373in}{4.186667in}}%
\pgfpathlineto{\pgfqpoint{4.285667in}{4.185397in}}%
\pgfpathlineto{\pgfqpoint{4.287030in}{4.184083in}}%
\pgfpathlineto{\pgfqpoint{4.322673in}{4.149333in}}%
\pgfpathlineto{\pgfqpoint{4.327111in}{4.145004in}}%
\pgfpathlineto{\pgfqpoint{4.344317in}{4.128027in}}%
\pgfpathlineto{\pgfqpoint{4.360857in}{4.112000in}}%
\pgfpathlineto{\pgfqpoint{4.367192in}{4.105801in}}%
\pgfpathlineto{\pgfqpoint{4.398927in}{4.074667in}}%
\pgfpathlineto{\pgfqpoint{4.402970in}{4.070659in}}%
\pgfpathlineto{\pgfqpoint{4.407273in}{4.066473in}}%
\pgfpathlineto{\pgfqpoint{4.436882in}{4.037333in}}%
\pgfpathlineto{\pgfqpoint{4.447354in}{4.027021in}}%
\pgfpathlineto{\pgfqpoint{4.461373in}{4.013058in}}%
\pgfpathlineto{\pgfqpoint{4.474723in}{4.000000in}}%
\pgfpathlineto{\pgfqpoint{4.487434in}{3.987443in}}%
\pgfpathlineto{\pgfqpoint{4.512453in}{3.962667in}}%
\pgfpathlineto{\pgfqpoint{4.519715in}{3.955402in}}%
\pgfpathlineto{\pgfqpoint{4.527515in}{3.947740in}}%
\pgfpathlineto{\pgfqpoint{4.550070in}{3.925333in}}%
\pgfpathlineto{\pgfqpoint{4.567596in}{3.907911in}}%
\pgfpathlineto{\pgfqpoint{4.587577in}{3.888000in}}%
\pgfpathlineto{\pgfqpoint{4.597237in}{3.878276in}}%
\pgfpathlineto{\pgfqpoint{4.607677in}{3.867957in}}%
\pgfpathlineto{\pgfqpoint{4.624974in}{3.850667in}}%
\pgfpathlineto{\pgfqpoint{4.635906in}{3.839628in}}%
\pgfpathlineto{\pgfqpoint{4.647758in}{3.827877in}}%
\pgfpathlineto{\pgfqpoint{4.662261in}{3.813333in}}%
\pgfpathlineto{\pgfqpoint{4.687838in}{3.787670in}}%
\pgfpathlineto{\pgfqpoint{4.693835in}{3.781586in}}%
\pgfpathlineto{\pgfqpoint{4.699441in}{3.776000in}}%
\pgfpathlineto{\pgfqpoint{4.727919in}{3.747336in}}%
\pgfpathlineto{\pgfqpoint{4.732367in}{3.742810in}}%
\pgfpathlineto{\pgfqpoint{4.736512in}{3.738667in}}%
\pgfpathlineto{\pgfqpoint{4.768000in}{3.706876in}}%
\pgfusepath{fill}%
\end{pgfscope}%
\begin{pgfscope}%
\pgfpathrectangle{\pgfqpoint{0.800000in}{0.528000in}}{\pgfqpoint{3.968000in}{3.696000in}}%
\pgfusepath{clip}%
\pgfsetbuttcap%
\pgfsetroundjoin%
\definecolor{currentfill}{rgb}{0.440137,0.811138,0.340967}%
\pgfsetfillcolor{currentfill}%
\pgfsetlinewidth{0.000000pt}%
\definecolor{currentstroke}{rgb}{0.000000,0.000000,0.000000}%
\pgfsetstrokecolor{currentstroke}%
\pgfsetdash{}{0pt}%
\pgfpathmoveto{\pgfqpoint{4.768000in}{3.712189in}}%
\pgfpathlineto{\pgfqpoint{4.741774in}{3.738667in}}%
\pgfpathlineto{\pgfqpoint{4.735091in}{3.745347in}}%
\pgfpathlineto{\pgfqpoint{4.727919in}{3.752646in}}%
\pgfpathlineto{\pgfqpoint{4.704716in}{3.776000in}}%
\pgfpathlineto{\pgfqpoint{4.696562in}{3.784125in}}%
\pgfpathlineto{\pgfqpoint{4.687838in}{3.792976in}}%
\pgfpathlineto{\pgfqpoint{4.667549in}{3.813333in}}%
\pgfpathlineto{\pgfqpoint{4.647758in}{3.833179in}}%
\pgfpathlineto{\pgfqpoint{4.638664in}{3.842196in}}%
\pgfpathlineto{\pgfqpoint{4.630275in}{3.850667in}}%
\pgfpathlineto{\pgfqpoint{4.607677in}{3.873256in}}%
\pgfpathlineto{\pgfqpoint{4.599997in}{3.880847in}}%
\pgfpathlineto{\pgfqpoint{4.592891in}{3.888000in}}%
\pgfpathlineto{\pgfqpoint{4.567596in}{3.913207in}}%
\pgfpathlineto{\pgfqpoint{4.555397in}{3.925333in}}%
\pgfpathlineto{\pgfqpoint{4.527515in}{3.953032in}}%
\pgfpathlineto{\pgfqpoint{4.522481in}{3.957977in}}%
\pgfpathlineto{\pgfqpoint{4.517793in}{3.962667in}}%
\pgfpathlineto{\pgfqpoint{4.487434in}{3.992731in}}%
\pgfpathlineto{\pgfqpoint{4.480076in}{4.000000in}}%
\pgfpathlineto{\pgfqpoint{4.464115in}{4.015612in}}%
\pgfpathlineto{\pgfqpoint{4.447354in}{4.032305in}}%
\pgfpathlineto{\pgfqpoint{4.442248in}{4.037333in}}%
\pgfpathlineto{\pgfqpoint{4.407273in}{4.071754in}}%
\pgfpathlineto{\pgfqpoint{4.405744in}{4.073242in}}%
\pgfpathlineto{\pgfqpoint{4.404306in}{4.074667in}}%
\pgfpathlineto{\pgfqpoint{4.367192in}{4.111079in}}%
\pgfpathlineto{\pgfqpoint{4.366250in}{4.112000in}}%
\pgfpathlineto{\pgfqpoint{4.347067in}{4.130588in}}%
\pgfpathlineto{\pgfqpoint{4.328069in}{4.149333in}}%
\pgfpathlineto{\pgfqpoint{4.327599in}{4.149788in}}%
\pgfpathlineto{\pgfqpoint{4.327111in}{4.150269in}}%
\pgfpathlineto{\pgfqpoint{4.289762in}{4.186667in}}%
\pgfpathlineto{\pgfqpoint{4.288420in}{4.187961in}}%
\pgfpathlineto{\pgfqpoint{4.287030in}{4.189327in}}%
\pgfpathlineto{\pgfqpoint{4.251340in}{4.224000in}}%
\pgfpathlineto{\pgfqpoint{4.248654in}{4.224000in}}%
\pgfpathlineto{\pgfqpoint{4.287030in}{4.186718in}}%
\pgfpathlineto{\pgfqpoint{4.287057in}{4.186691in}}%
\pgfpathlineto{\pgfqpoint{4.287083in}{4.186667in}}%
\pgfpathlineto{\pgfqpoint{4.288213in}{4.185565in}}%
\pgfpathlineto{\pgfqpoint{4.325376in}{4.149333in}}%
\pgfpathlineto{\pgfqpoint{4.327111in}{4.147641in}}%
\pgfpathlineto{\pgfqpoint{4.345692in}{4.129308in}}%
\pgfpathlineto{\pgfqpoint{4.363554in}{4.112000in}}%
\pgfpathlineto{\pgfqpoint{4.367192in}{4.108440in}}%
\pgfpathlineto{\pgfqpoint{4.401616in}{4.074667in}}%
\pgfpathlineto{\pgfqpoint{4.404357in}{4.071951in}}%
\pgfpathlineto{\pgfqpoint{4.407273in}{4.069114in}}%
\pgfpathlineto{\pgfqpoint{4.439565in}{4.037333in}}%
\pgfpathlineto{\pgfqpoint{4.447354in}{4.029663in}}%
\pgfpathlineto{\pgfqpoint{4.462744in}{4.014335in}}%
\pgfpathlineto{\pgfqpoint{4.477400in}{4.000000in}}%
\pgfpathlineto{\pgfqpoint{4.487434in}{3.990087in}}%
\pgfpathlineto{\pgfqpoint{4.515123in}{3.962667in}}%
\pgfpathlineto{\pgfqpoint{4.521098in}{3.956689in}}%
\pgfpathlineto{\pgfqpoint{4.527515in}{3.950386in}}%
\pgfpathlineto{\pgfqpoint{4.552734in}{3.925333in}}%
\pgfpathlineto{\pgfqpoint{4.567596in}{3.910559in}}%
\pgfpathlineto{\pgfqpoint{4.590234in}{3.888000in}}%
\pgfpathlineto{\pgfqpoint{4.598617in}{3.879561in}}%
\pgfpathlineto{\pgfqpoint{4.607677in}{3.870606in}}%
\pgfpathlineto{\pgfqpoint{4.627624in}{3.850667in}}%
\pgfpathlineto{\pgfqpoint{4.637285in}{3.840912in}}%
\pgfpathlineto{\pgfqpoint{4.647758in}{3.830528in}}%
\pgfpathlineto{\pgfqpoint{4.664905in}{3.813333in}}%
\pgfpathlineto{\pgfqpoint{4.687838in}{3.790323in}}%
\pgfpathlineto{\pgfqpoint{4.695198in}{3.782856in}}%
\pgfpathlineto{\pgfqpoint{4.702078in}{3.776000in}}%
\pgfpathlineto{\pgfqpoint{4.727919in}{3.749991in}}%
\pgfpathlineto{\pgfqpoint{4.733729in}{3.744078in}}%
\pgfpathlineto{\pgfqpoint{4.739143in}{3.738667in}}%
\pgfpathlineto{\pgfqpoint{4.768000in}{3.709533in}}%
\pgfusepath{fill}%
\end{pgfscope}%
\begin{pgfscope}%
\pgfpathrectangle{\pgfqpoint{0.800000in}{0.528000in}}{\pgfqpoint{3.968000in}{3.696000in}}%
\pgfusepath{clip}%
\pgfsetbuttcap%
\pgfsetroundjoin%
\definecolor{currentfill}{rgb}{0.440137,0.811138,0.340967}%
\pgfsetfillcolor{currentfill}%
\pgfsetlinewidth{0.000000pt}%
\definecolor{currentstroke}{rgb}{0.000000,0.000000,0.000000}%
\pgfsetstrokecolor{currentstroke}%
\pgfsetdash{}{0pt}%
\pgfpathmoveto{\pgfqpoint{4.768000in}{3.714846in}}%
\pgfpathlineto{\pgfqpoint{4.744406in}{3.738667in}}%
\pgfpathlineto{\pgfqpoint{4.736453in}{3.746616in}}%
\pgfpathlineto{\pgfqpoint{4.727919in}{3.755300in}}%
\pgfpathlineto{\pgfqpoint{4.707353in}{3.776000in}}%
\pgfpathlineto{\pgfqpoint{4.697925in}{3.785395in}}%
\pgfpathlineto{\pgfqpoint{4.687838in}{3.795629in}}%
\pgfpathlineto{\pgfqpoint{4.670193in}{3.813333in}}%
\pgfpathlineto{\pgfqpoint{4.647758in}{3.835830in}}%
\pgfpathlineto{\pgfqpoint{4.640042in}{3.843480in}}%
\pgfpathlineto{\pgfqpoint{4.632925in}{3.850667in}}%
\pgfpathlineto{\pgfqpoint{4.607677in}{3.875905in}}%
\pgfpathlineto{\pgfqpoint{4.601377in}{3.882132in}}%
\pgfpathlineto{\pgfqpoint{4.595548in}{3.888000in}}%
\pgfpathlineto{\pgfqpoint{4.567596in}{3.915854in}}%
\pgfpathlineto{\pgfqpoint{4.558061in}{3.925333in}}%
\pgfpathlineto{\pgfqpoint{4.527515in}{3.955678in}}%
\pgfpathlineto{\pgfqpoint{4.523863in}{3.959265in}}%
\pgfpathlineto{\pgfqpoint{4.520463in}{3.962667in}}%
\pgfpathlineto{\pgfqpoint{4.487434in}{3.995375in}}%
\pgfpathlineto{\pgfqpoint{4.482753in}{4.000000in}}%
\pgfpathlineto{\pgfqpoint{4.465486in}{4.016889in}}%
\pgfpathlineto{\pgfqpoint{4.447354in}{4.034948in}}%
\pgfpathlineto{\pgfqpoint{4.444931in}{4.037333in}}%
\pgfpathlineto{\pgfqpoint{4.407273in}{4.074395in}}%
\pgfpathlineto{\pgfqpoint{4.407130in}{4.074534in}}%
\pgfpathlineto{\pgfqpoint{4.406996in}{4.074667in}}%
\pgfpathlineto{\pgfqpoint{4.401803in}{4.079762in}}%
\pgfpathlineto{\pgfqpoint{4.368927in}{4.112000in}}%
\pgfpathlineto{\pgfqpoint{4.368078in}{4.112825in}}%
\pgfpathlineto{\pgfqpoint{4.367192in}{4.113701in}}%
\pgfpathlineto{\pgfqpoint{4.348442in}{4.131869in}}%
\pgfpathlineto{\pgfqpoint{4.330742in}{4.149333in}}%
\pgfpathlineto{\pgfqpoint{4.328961in}{4.151057in}}%
\pgfpathlineto{\pgfqpoint{4.327111in}{4.152880in}}%
\pgfpathlineto{\pgfqpoint{4.292441in}{4.186667in}}%
\pgfpathlineto{\pgfqpoint{4.289784in}{4.189231in}}%
\pgfpathlineto{\pgfqpoint{4.287030in}{4.191936in}}%
\pgfpathlineto{\pgfqpoint{4.254026in}{4.224000in}}%
\pgfpathlineto{\pgfqpoint{4.251340in}{4.224000in}}%
\pgfpathlineto{\pgfqpoint{4.287030in}{4.189327in}}%
\pgfpathlineto{\pgfqpoint{4.288420in}{4.187961in}}%
\pgfpathlineto{\pgfqpoint{4.289762in}{4.186667in}}%
\pgfpathlineto{\pgfqpoint{4.327111in}{4.150269in}}%
\pgfpathlineto{\pgfqpoint{4.327599in}{4.149788in}}%
\pgfpathlineto{\pgfqpoint{4.328069in}{4.149333in}}%
\pgfpathlineto{\pgfqpoint{4.347067in}{4.130588in}}%
\pgfpathlineto{\pgfqpoint{4.366250in}{4.112000in}}%
\pgfpathlineto{\pgfqpoint{4.367192in}{4.111079in}}%
\pgfpathlineto{\pgfqpoint{4.404306in}{4.074667in}}%
\pgfpathlineto{\pgfqpoint{4.405744in}{4.073242in}}%
\pgfpathlineto{\pgfqpoint{4.407273in}{4.071754in}}%
\pgfpathlineto{\pgfqpoint{4.442248in}{4.037333in}}%
\pgfpathlineto{\pgfqpoint{4.447354in}{4.032305in}}%
\pgfpathlineto{\pgfqpoint{4.464115in}{4.015612in}}%
\pgfpathlineto{\pgfqpoint{4.480076in}{4.000000in}}%
\pgfpathlineto{\pgfqpoint{4.487434in}{3.992731in}}%
\pgfpathlineto{\pgfqpoint{4.517793in}{3.962667in}}%
\pgfpathlineto{\pgfqpoint{4.522481in}{3.957977in}}%
\pgfpathlineto{\pgfqpoint{4.527515in}{3.953032in}}%
\pgfpathlineto{\pgfqpoint{4.555397in}{3.925333in}}%
\pgfpathlineto{\pgfqpoint{4.567596in}{3.913207in}}%
\pgfpathlineto{\pgfqpoint{4.592891in}{3.888000in}}%
\pgfpathlineto{\pgfqpoint{4.599997in}{3.880847in}}%
\pgfpathlineto{\pgfqpoint{4.607677in}{3.873256in}}%
\pgfpathlineto{\pgfqpoint{4.630275in}{3.850667in}}%
\pgfpathlineto{\pgfqpoint{4.638664in}{3.842196in}}%
\pgfpathlineto{\pgfqpoint{4.647758in}{3.833179in}}%
\pgfpathlineto{\pgfqpoint{4.667549in}{3.813333in}}%
\pgfpathlineto{\pgfqpoint{4.687838in}{3.792976in}}%
\pgfpathlineto{\pgfqpoint{4.696562in}{3.784125in}}%
\pgfpathlineto{\pgfqpoint{4.704716in}{3.776000in}}%
\pgfpathlineto{\pgfqpoint{4.727919in}{3.752646in}}%
\pgfpathlineto{\pgfqpoint{4.735091in}{3.745347in}}%
\pgfpathlineto{\pgfqpoint{4.741774in}{3.738667in}}%
\pgfpathlineto{\pgfqpoint{4.768000in}{3.712189in}}%
\pgfusepath{fill}%
\end{pgfscope}%
\begin{pgfscope}%
\pgfpathrectangle{\pgfqpoint{0.800000in}{0.528000in}}{\pgfqpoint{3.968000in}{3.696000in}}%
\pgfusepath{clip}%
\pgfsetbuttcap%
\pgfsetroundjoin%
\definecolor{currentfill}{rgb}{0.449368,0.813768,0.335384}%
\pgfsetfillcolor{currentfill}%
\pgfsetlinewidth{0.000000pt}%
\definecolor{currentstroke}{rgb}{0.000000,0.000000,0.000000}%
\pgfsetstrokecolor{currentstroke}%
\pgfsetdash{}{0pt}%
\pgfpathmoveto{\pgfqpoint{4.768000in}{3.717502in}}%
\pgfpathlineto{\pgfqpoint{4.747037in}{3.738667in}}%
\pgfpathlineto{\pgfqpoint{4.737815in}{3.747884in}}%
\pgfpathlineto{\pgfqpoint{4.727919in}{3.757955in}}%
\pgfpathlineto{\pgfqpoint{4.709991in}{3.776000in}}%
\pgfpathlineto{\pgfqpoint{4.699288in}{3.786665in}}%
\pgfpathlineto{\pgfqpoint{4.687838in}{3.798281in}}%
\pgfpathlineto{\pgfqpoint{4.672837in}{3.813333in}}%
\pgfpathlineto{\pgfqpoint{4.647758in}{3.838481in}}%
\pgfpathlineto{\pgfqpoint{4.641421in}{3.844764in}}%
\pgfpathlineto{\pgfqpoint{4.635576in}{3.850667in}}%
\pgfpathlineto{\pgfqpoint{4.607677in}{3.878555in}}%
\pgfpathlineto{\pgfqpoint{4.602757in}{3.883418in}}%
\pgfpathlineto{\pgfqpoint{4.598205in}{3.888000in}}%
\pgfpathlineto{\pgfqpoint{4.567596in}{3.918502in}}%
\pgfpathlineto{\pgfqpoint{4.560724in}{3.925333in}}%
\pgfpathlineto{\pgfqpoint{4.527515in}{3.958324in}}%
\pgfpathlineto{\pgfqpoint{4.525246in}{3.960553in}}%
\pgfpathlineto{\pgfqpoint{4.523133in}{3.962667in}}%
\pgfpathlineto{\pgfqpoint{4.487434in}{3.998019in}}%
\pgfpathlineto{\pgfqpoint{4.485430in}{4.000000in}}%
\pgfpathlineto{\pgfqpoint{4.466857in}{4.018166in}}%
\pgfpathlineto{\pgfqpoint{4.447611in}{4.037333in}}%
\pgfpathlineto{\pgfqpoint{4.447354in}{4.037588in}}%
\pgfpathlineto{\pgfqpoint{4.409659in}{4.074667in}}%
\pgfpathlineto{\pgfqpoint{4.408492in}{4.075803in}}%
\pgfpathlineto{\pgfqpoint{4.407273in}{4.077012in}}%
\pgfpathlineto{\pgfqpoint{4.371593in}{4.112000in}}%
\pgfpathlineto{\pgfqpoint{4.369438in}{4.114093in}}%
\pgfpathlineto{\pgfqpoint{4.367192in}{4.116313in}}%
\pgfpathlineto{\pgfqpoint{4.349817in}{4.133149in}}%
\pgfpathlineto{\pgfqpoint{4.333414in}{4.149333in}}%
\pgfpathlineto{\pgfqpoint{4.330323in}{4.152325in}}%
\pgfpathlineto{\pgfqpoint{4.327111in}{4.155491in}}%
\pgfpathlineto{\pgfqpoint{4.295121in}{4.186667in}}%
\pgfpathlineto{\pgfqpoint{4.291147in}{4.190501in}}%
\pgfpathlineto{\pgfqpoint{4.287030in}{4.194546in}}%
\pgfpathlineto{\pgfqpoint{4.256712in}{4.224000in}}%
\pgfpathlineto{\pgfqpoint{4.254026in}{4.224000in}}%
\pgfpathlineto{\pgfqpoint{4.287030in}{4.191936in}}%
\pgfpathlineto{\pgfqpoint{4.289784in}{4.189231in}}%
\pgfpathlineto{\pgfqpoint{4.292441in}{4.186667in}}%
\pgfpathlineto{\pgfqpoint{4.327111in}{4.152880in}}%
\pgfpathlineto{\pgfqpoint{4.328961in}{4.151057in}}%
\pgfpathlineto{\pgfqpoint{4.330742in}{4.149333in}}%
\pgfpathlineto{\pgfqpoint{4.348442in}{4.131869in}}%
\pgfpathlineto{\pgfqpoint{4.367192in}{4.113701in}}%
\pgfpathlineto{\pgfqpoint{4.368078in}{4.112825in}}%
\pgfpathlineto{\pgfqpoint{4.368927in}{4.112000in}}%
\pgfpathlineto{\pgfqpoint{4.401803in}{4.079762in}}%
\pgfpathlineto{\pgfqpoint{4.406996in}{4.074667in}}%
\pgfpathlineto{\pgfqpoint{4.407130in}{4.074534in}}%
\pgfpathlineto{\pgfqpoint{4.407273in}{4.074395in}}%
\pgfpathlineto{\pgfqpoint{4.444931in}{4.037333in}}%
\pgfpathlineto{\pgfqpoint{4.447354in}{4.034948in}}%
\pgfpathlineto{\pgfqpoint{4.465486in}{4.016889in}}%
\pgfpathlineto{\pgfqpoint{4.482753in}{4.000000in}}%
\pgfpathlineto{\pgfqpoint{4.487434in}{3.995375in}}%
\pgfpathlineto{\pgfqpoint{4.520463in}{3.962667in}}%
\pgfpathlineto{\pgfqpoint{4.523863in}{3.959265in}}%
\pgfpathlineto{\pgfqpoint{4.527515in}{3.955678in}}%
\pgfpathlineto{\pgfqpoint{4.558061in}{3.925333in}}%
\pgfpathlineto{\pgfqpoint{4.567596in}{3.915854in}}%
\pgfpathlineto{\pgfqpoint{4.595548in}{3.888000in}}%
\pgfpathlineto{\pgfqpoint{4.601377in}{3.882132in}}%
\pgfpathlineto{\pgfqpoint{4.607677in}{3.875905in}}%
\pgfpathlineto{\pgfqpoint{4.632925in}{3.850667in}}%
\pgfpathlineto{\pgfqpoint{4.640042in}{3.843480in}}%
\pgfpathlineto{\pgfqpoint{4.647758in}{3.835830in}}%
\pgfpathlineto{\pgfqpoint{4.670193in}{3.813333in}}%
\pgfpathlineto{\pgfqpoint{4.687838in}{3.795629in}}%
\pgfpathlineto{\pgfqpoint{4.697925in}{3.785395in}}%
\pgfpathlineto{\pgfqpoint{4.707353in}{3.776000in}}%
\pgfpathlineto{\pgfqpoint{4.727919in}{3.755300in}}%
\pgfpathlineto{\pgfqpoint{4.736453in}{3.746616in}}%
\pgfpathlineto{\pgfqpoint{4.744406in}{3.738667in}}%
\pgfpathlineto{\pgfqpoint{4.768000in}{3.714846in}}%
\pgfusepath{fill}%
\end{pgfscope}%
\begin{pgfscope}%
\pgfpathrectangle{\pgfqpoint{0.800000in}{0.528000in}}{\pgfqpoint{3.968000in}{3.696000in}}%
\pgfusepath{clip}%
\pgfsetbuttcap%
\pgfsetroundjoin%
\definecolor{currentfill}{rgb}{0.449368,0.813768,0.335384}%
\pgfsetfillcolor{currentfill}%
\pgfsetlinewidth{0.000000pt}%
\definecolor{currentstroke}{rgb}{0.000000,0.000000,0.000000}%
\pgfsetstrokecolor{currentstroke}%
\pgfsetdash{}{0pt}%
\pgfpathmoveto{\pgfqpoint{4.768000in}{3.720158in}}%
\pgfpathlineto{\pgfqpoint{4.749668in}{3.738667in}}%
\pgfpathlineto{\pgfqpoint{4.739177in}{3.749153in}}%
\pgfpathlineto{\pgfqpoint{4.727919in}{3.760610in}}%
\pgfpathlineto{\pgfqpoint{4.712628in}{3.776000in}}%
\pgfpathlineto{\pgfqpoint{4.700652in}{3.787935in}}%
\pgfpathlineto{\pgfqpoint{4.687838in}{3.800934in}}%
\pgfpathlineto{\pgfqpoint{4.675481in}{3.813333in}}%
\pgfpathlineto{\pgfqpoint{4.647758in}{3.841132in}}%
\pgfpathlineto{\pgfqpoint{4.642800in}{3.846049in}}%
\pgfpathlineto{\pgfqpoint{4.638226in}{3.850667in}}%
\pgfpathlineto{\pgfqpoint{4.607677in}{3.881204in}}%
\pgfpathlineto{\pgfqpoint{4.604137in}{3.884703in}}%
\pgfpathlineto{\pgfqpoint{4.600862in}{3.888000in}}%
\pgfpathlineto{\pgfqpoint{4.567596in}{3.921150in}}%
\pgfpathlineto{\pgfqpoint{4.563387in}{3.925333in}}%
\pgfpathlineto{\pgfqpoint{4.527515in}{3.960969in}}%
\pgfpathlineto{\pgfqpoint{4.526628in}{3.961841in}}%
\pgfpathlineto{\pgfqpoint{4.525802in}{3.962667in}}%
\pgfpathlineto{\pgfqpoint{4.498706in}{3.989501in}}%
\pgfpathlineto{\pgfqpoint{4.488099in}{4.000000in}}%
\pgfpathlineto{\pgfqpoint{4.487434in}{4.000657in}}%
\pgfpathlineto{\pgfqpoint{4.468228in}{4.019443in}}%
\pgfpathlineto{\pgfqpoint{4.450265in}{4.037333in}}%
\pgfpathlineto{\pgfqpoint{4.447354in}{4.040204in}}%
\pgfpathlineto{\pgfqpoint{4.412319in}{4.074667in}}%
\pgfpathlineto{\pgfqpoint{4.409852in}{4.077069in}}%
\pgfpathlineto{\pgfqpoint{4.407273in}{4.079627in}}%
\pgfpathlineto{\pgfqpoint{4.374260in}{4.112000in}}%
\pgfpathlineto{\pgfqpoint{4.370799in}{4.115360in}}%
\pgfpathlineto{\pgfqpoint{4.367192in}{4.118926in}}%
\pgfpathlineto{\pgfqpoint{4.351192in}{4.134430in}}%
\pgfpathlineto{\pgfqpoint{4.336087in}{4.149333in}}%
\pgfpathlineto{\pgfqpoint{4.331686in}{4.153594in}}%
\pgfpathlineto{\pgfqpoint{4.327111in}{4.158102in}}%
\pgfpathlineto{\pgfqpoint{4.297800in}{4.186667in}}%
\pgfpathlineto{\pgfqpoint{4.292511in}{4.191771in}}%
\pgfpathlineto{\pgfqpoint{4.287030in}{4.197155in}}%
\pgfpathlineto{\pgfqpoint{4.259398in}{4.224000in}}%
\pgfpathlineto{\pgfqpoint{4.256712in}{4.224000in}}%
\pgfpathlineto{\pgfqpoint{4.287030in}{4.194546in}}%
\pgfpathlineto{\pgfqpoint{4.291147in}{4.190501in}}%
\pgfpathlineto{\pgfqpoint{4.295121in}{4.186667in}}%
\pgfpathlineto{\pgfqpoint{4.327111in}{4.155491in}}%
\pgfpathlineto{\pgfqpoint{4.330323in}{4.152325in}}%
\pgfpathlineto{\pgfqpoint{4.333414in}{4.149333in}}%
\pgfpathlineto{\pgfqpoint{4.349817in}{4.133149in}}%
\pgfpathlineto{\pgfqpoint{4.367192in}{4.116313in}}%
\pgfpathlineto{\pgfqpoint{4.369438in}{4.114093in}}%
\pgfpathlineto{\pgfqpoint{4.371593in}{4.112000in}}%
\pgfpathlineto{\pgfqpoint{4.407273in}{4.077012in}}%
\pgfpathlineto{\pgfqpoint{4.408492in}{4.075803in}}%
\pgfpathlineto{\pgfqpoint{4.409659in}{4.074667in}}%
\pgfpathlineto{\pgfqpoint{4.447354in}{4.037588in}}%
\pgfpathlineto{\pgfqpoint{4.447611in}{4.037333in}}%
\pgfpathlineto{\pgfqpoint{4.466857in}{4.018166in}}%
\pgfpathlineto{\pgfqpoint{4.485430in}{4.000000in}}%
\pgfpathlineto{\pgfqpoint{4.487434in}{3.998019in}}%
\pgfpathlineto{\pgfqpoint{4.523133in}{3.962667in}}%
\pgfpathlineto{\pgfqpoint{4.525246in}{3.960553in}}%
\pgfpathlineto{\pgfqpoint{4.527515in}{3.958324in}}%
\pgfpathlineto{\pgfqpoint{4.560724in}{3.925333in}}%
\pgfpathlineto{\pgfqpoint{4.567596in}{3.918502in}}%
\pgfpathlineto{\pgfqpoint{4.598205in}{3.888000in}}%
\pgfpathlineto{\pgfqpoint{4.602757in}{3.883418in}}%
\pgfpathlineto{\pgfqpoint{4.607677in}{3.878555in}}%
\pgfpathlineto{\pgfqpoint{4.635576in}{3.850667in}}%
\pgfpathlineto{\pgfqpoint{4.641421in}{3.844764in}}%
\pgfpathlineto{\pgfqpoint{4.647758in}{3.838481in}}%
\pgfpathlineto{\pgfqpoint{4.672837in}{3.813333in}}%
\pgfpathlineto{\pgfqpoint{4.687838in}{3.798281in}}%
\pgfpathlineto{\pgfqpoint{4.699288in}{3.786665in}}%
\pgfpathlineto{\pgfqpoint{4.709991in}{3.776000in}}%
\pgfpathlineto{\pgfqpoint{4.727919in}{3.757955in}}%
\pgfpathlineto{\pgfqpoint{4.737815in}{3.747884in}}%
\pgfpathlineto{\pgfqpoint{4.747037in}{3.738667in}}%
\pgfpathlineto{\pgfqpoint{4.768000in}{3.717502in}}%
\pgfusepath{fill}%
\end{pgfscope}%
\begin{pgfscope}%
\pgfpathrectangle{\pgfqpoint{0.800000in}{0.528000in}}{\pgfqpoint{3.968000in}{3.696000in}}%
\pgfusepath{clip}%
\pgfsetbuttcap%
\pgfsetroundjoin%
\definecolor{currentfill}{rgb}{0.449368,0.813768,0.335384}%
\pgfsetfillcolor{currentfill}%
\pgfsetlinewidth{0.000000pt}%
\definecolor{currentstroke}{rgb}{0.000000,0.000000,0.000000}%
\pgfsetstrokecolor{currentstroke}%
\pgfsetdash{}{0pt}%
\pgfpathmoveto{\pgfqpoint{4.768000in}{3.722815in}}%
\pgfpathlineto{\pgfqpoint{4.752299in}{3.738667in}}%
\pgfpathlineto{\pgfqpoint{4.740539in}{3.750422in}}%
\pgfpathlineto{\pgfqpoint{4.727919in}{3.763265in}}%
\pgfpathlineto{\pgfqpoint{4.715266in}{3.776000in}}%
\pgfpathlineto{\pgfqpoint{4.702015in}{3.789205in}}%
\pgfpathlineto{\pgfqpoint{4.687838in}{3.803587in}}%
\pgfpathlineto{\pgfqpoint{4.678125in}{3.813333in}}%
\pgfpathlineto{\pgfqpoint{4.647758in}{3.843784in}}%
\pgfpathlineto{\pgfqpoint{4.644178in}{3.847333in}}%
\pgfpathlineto{\pgfqpoint{4.640876in}{3.850667in}}%
\pgfpathlineto{\pgfqpoint{4.607677in}{3.883853in}}%
\pgfpathlineto{\pgfqpoint{4.605517in}{3.885988in}}%
\pgfpathlineto{\pgfqpoint{4.603518in}{3.888000in}}%
\pgfpathlineto{\pgfqpoint{4.567596in}{3.923797in}}%
\pgfpathlineto{\pgfqpoint{4.566051in}{3.925333in}}%
\pgfpathlineto{\pgfqpoint{4.542824in}{3.948407in}}%
\pgfpathlineto{\pgfqpoint{4.528462in}{3.962667in}}%
\pgfpathlineto{\pgfqpoint{4.528001in}{3.963119in}}%
\pgfpathlineto{\pgfqpoint{4.527515in}{3.963606in}}%
\pgfpathlineto{\pgfqpoint{4.490745in}{4.000000in}}%
\pgfpathlineto{\pgfqpoint{4.487434in}{4.003275in}}%
\pgfpathlineto{\pgfqpoint{4.469599in}{4.020720in}}%
\pgfpathlineto{\pgfqpoint{4.452918in}{4.037333in}}%
\pgfpathlineto{\pgfqpoint{4.447354in}{4.042820in}}%
\pgfpathlineto{\pgfqpoint{4.414978in}{4.074667in}}%
\pgfpathlineto{\pgfqpoint{4.411212in}{4.078336in}}%
\pgfpathlineto{\pgfqpoint{4.407273in}{4.082241in}}%
\pgfpathlineto{\pgfqpoint{4.376926in}{4.112000in}}%
\pgfpathlineto{\pgfqpoint{4.372160in}{4.116628in}}%
\pgfpathlineto{\pgfqpoint{4.367192in}{4.121539in}}%
\pgfpathlineto{\pgfqpoint{4.352567in}{4.135711in}}%
\pgfpathlineto{\pgfqpoint{4.338760in}{4.149333in}}%
\pgfpathlineto{\pgfqpoint{4.333048in}{4.154863in}}%
\pgfpathlineto{\pgfqpoint{4.327111in}{4.160713in}}%
\pgfpathlineto{\pgfqpoint{4.300480in}{4.186667in}}%
\pgfpathlineto{\pgfqpoint{4.293874in}{4.193041in}}%
\pgfpathlineto{\pgfqpoint{4.287030in}{4.199765in}}%
\pgfpathlineto{\pgfqpoint{4.262084in}{4.224000in}}%
\pgfpathlineto{\pgfqpoint{4.259398in}{4.224000in}}%
\pgfpathlineto{\pgfqpoint{4.287030in}{4.197155in}}%
\pgfpathlineto{\pgfqpoint{4.292511in}{4.191771in}}%
\pgfpathlineto{\pgfqpoint{4.297800in}{4.186667in}}%
\pgfpathlineto{\pgfqpoint{4.327111in}{4.158102in}}%
\pgfpathlineto{\pgfqpoint{4.331686in}{4.153594in}}%
\pgfpathlineto{\pgfqpoint{4.336087in}{4.149333in}}%
\pgfpathlineto{\pgfqpoint{4.351192in}{4.134430in}}%
\pgfpathlineto{\pgfqpoint{4.367192in}{4.118926in}}%
\pgfpathlineto{\pgfqpoint{4.370799in}{4.115360in}}%
\pgfpathlineto{\pgfqpoint{4.374260in}{4.112000in}}%
\pgfpathlineto{\pgfqpoint{4.407273in}{4.079627in}}%
\pgfpathlineto{\pgfqpoint{4.409852in}{4.077069in}}%
\pgfpathlineto{\pgfqpoint{4.412319in}{4.074667in}}%
\pgfpathlineto{\pgfqpoint{4.447354in}{4.040204in}}%
\pgfpathlineto{\pgfqpoint{4.450265in}{4.037333in}}%
\pgfpathlineto{\pgfqpoint{4.468228in}{4.019443in}}%
\pgfpathlineto{\pgfqpoint{4.487434in}{4.000657in}}%
\pgfpathlineto{\pgfqpoint{4.488099in}{4.000000in}}%
\pgfpathlineto{\pgfqpoint{4.498706in}{3.989501in}}%
\pgfpathlineto{\pgfqpoint{4.525802in}{3.962667in}}%
\pgfpathlineto{\pgfqpoint{4.526628in}{3.961841in}}%
\pgfpathlineto{\pgfqpoint{4.527515in}{3.960969in}}%
\pgfpathlineto{\pgfqpoint{4.563387in}{3.925333in}}%
\pgfpathlineto{\pgfqpoint{4.567596in}{3.921150in}}%
\pgfpathlineto{\pgfqpoint{4.600862in}{3.888000in}}%
\pgfpathlineto{\pgfqpoint{4.604137in}{3.884703in}}%
\pgfpathlineto{\pgfqpoint{4.607677in}{3.881204in}}%
\pgfpathlineto{\pgfqpoint{4.638226in}{3.850667in}}%
\pgfpathlineto{\pgfqpoint{4.642800in}{3.846049in}}%
\pgfpathlineto{\pgfqpoint{4.647758in}{3.841132in}}%
\pgfpathlineto{\pgfqpoint{4.675481in}{3.813333in}}%
\pgfpathlineto{\pgfqpoint{4.687838in}{3.800934in}}%
\pgfpathlineto{\pgfqpoint{4.700652in}{3.787935in}}%
\pgfpathlineto{\pgfqpoint{4.712628in}{3.776000in}}%
\pgfpathlineto{\pgfqpoint{4.727919in}{3.760610in}}%
\pgfpathlineto{\pgfqpoint{4.739177in}{3.749153in}}%
\pgfpathlineto{\pgfqpoint{4.749668in}{3.738667in}}%
\pgfpathlineto{\pgfqpoint{4.768000in}{3.720158in}}%
\pgfusepath{fill}%
\end{pgfscope}%
\begin{pgfscope}%
\pgfpathrectangle{\pgfqpoint{0.800000in}{0.528000in}}{\pgfqpoint{3.968000in}{3.696000in}}%
\pgfusepath{clip}%
\pgfsetbuttcap%
\pgfsetroundjoin%
\definecolor{currentfill}{rgb}{0.449368,0.813768,0.335384}%
\pgfsetfillcolor{currentfill}%
\pgfsetlinewidth{0.000000pt}%
\definecolor{currentstroke}{rgb}{0.000000,0.000000,0.000000}%
\pgfsetstrokecolor{currentstroke}%
\pgfsetdash{}{0pt}%
\pgfpathmoveto{\pgfqpoint{4.768000in}{3.725471in}}%
\pgfpathlineto{\pgfqpoint{4.754930in}{3.738667in}}%
\pgfpathlineto{\pgfqpoint{4.741901in}{3.751690in}}%
\pgfpathlineto{\pgfqpoint{4.727919in}{3.765919in}}%
\pgfpathlineto{\pgfqpoint{4.717903in}{3.776000in}}%
\pgfpathlineto{\pgfqpoint{4.703378in}{3.790474in}}%
\pgfpathlineto{\pgfqpoint{4.687838in}{3.806240in}}%
\pgfpathlineto{\pgfqpoint{4.680769in}{3.813333in}}%
\pgfpathlineto{\pgfqpoint{4.647758in}{3.846435in}}%
\pgfpathlineto{\pgfqpoint{4.645557in}{3.848617in}}%
\pgfpathlineto{\pgfqpoint{4.643527in}{3.850667in}}%
\pgfpathlineto{\pgfqpoint{4.607677in}{3.886503in}}%
\pgfpathlineto{\pgfqpoint{4.606897in}{3.887274in}}%
\pgfpathlineto{\pgfqpoint{4.606175in}{3.888000in}}%
\pgfpathlineto{\pgfqpoint{4.584681in}{3.909419in}}%
\pgfpathlineto{\pgfqpoint{4.568702in}{3.925333in}}%
\pgfpathlineto{\pgfqpoint{4.567596in}{3.926434in}}%
\pgfpathlineto{\pgfqpoint{4.531102in}{3.962667in}}%
\pgfpathlineto{\pgfqpoint{4.529357in}{3.964382in}}%
\pgfpathlineto{\pgfqpoint{4.527515in}{3.966226in}}%
\pgfpathlineto{\pgfqpoint{4.493392in}{4.000000in}}%
\pgfpathlineto{\pgfqpoint{4.487434in}{4.005893in}}%
\pgfpathlineto{\pgfqpoint{4.470970in}{4.021997in}}%
\pgfpathlineto{\pgfqpoint{4.455571in}{4.037333in}}%
\pgfpathlineto{\pgfqpoint{4.447354in}{4.045436in}}%
\pgfpathlineto{\pgfqpoint{4.417638in}{4.074667in}}%
\pgfpathlineto{\pgfqpoint{4.412571in}{4.079602in}}%
\pgfpathlineto{\pgfqpoint{4.407273in}{4.084856in}}%
\pgfpathlineto{\pgfqpoint{4.379592in}{4.112000in}}%
\pgfpathlineto{\pgfqpoint{4.373521in}{4.117895in}}%
\pgfpathlineto{\pgfqpoint{4.367192in}{4.124152in}}%
\pgfpathlineto{\pgfqpoint{4.353942in}{4.136991in}}%
\pgfpathlineto{\pgfqpoint{4.341433in}{4.149333in}}%
\pgfpathlineto{\pgfqpoint{4.334410in}{4.156132in}}%
\pgfpathlineto{\pgfqpoint{4.327111in}{4.163325in}}%
\pgfpathlineto{\pgfqpoint{4.303159in}{4.186667in}}%
\pgfpathlineto{\pgfqpoint{4.295237in}{4.194311in}}%
\pgfpathlineto{\pgfqpoint{4.287030in}{4.202374in}}%
\pgfpathlineto{\pgfqpoint{4.264770in}{4.224000in}}%
\pgfpathlineto{\pgfqpoint{4.262084in}{4.224000in}}%
\pgfpathlineto{\pgfqpoint{4.287030in}{4.199765in}}%
\pgfpathlineto{\pgfqpoint{4.293874in}{4.193041in}}%
\pgfpathlineto{\pgfqpoint{4.300480in}{4.186667in}}%
\pgfpathlineto{\pgfqpoint{4.327111in}{4.160713in}}%
\pgfpathlineto{\pgfqpoint{4.333048in}{4.154863in}}%
\pgfpathlineto{\pgfqpoint{4.338760in}{4.149333in}}%
\pgfpathlineto{\pgfqpoint{4.352567in}{4.135711in}}%
\pgfpathlineto{\pgfqpoint{4.367192in}{4.121539in}}%
\pgfpathlineto{\pgfqpoint{4.372160in}{4.116628in}}%
\pgfpathlineto{\pgfqpoint{4.376926in}{4.112000in}}%
\pgfpathlineto{\pgfqpoint{4.407273in}{4.082241in}}%
\pgfpathlineto{\pgfqpoint{4.411212in}{4.078336in}}%
\pgfpathlineto{\pgfqpoint{4.414978in}{4.074667in}}%
\pgfpathlineto{\pgfqpoint{4.447354in}{4.042820in}}%
\pgfpathlineto{\pgfqpoint{4.452918in}{4.037333in}}%
\pgfpathlineto{\pgfqpoint{4.469599in}{4.020720in}}%
\pgfpathlineto{\pgfqpoint{4.487434in}{4.003275in}}%
\pgfpathlineto{\pgfqpoint{4.490745in}{4.000000in}}%
\pgfpathlineto{\pgfqpoint{4.527515in}{3.963606in}}%
\pgfpathlineto{\pgfqpoint{4.528001in}{3.963119in}}%
\pgfpathlineto{\pgfqpoint{4.528462in}{3.962667in}}%
\pgfpathlineto{\pgfqpoint{4.542824in}{3.948407in}}%
\pgfpathlineto{\pgfqpoint{4.566051in}{3.925333in}}%
\pgfpathlineto{\pgfqpoint{4.567596in}{3.923797in}}%
\pgfpathlineto{\pgfqpoint{4.603518in}{3.888000in}}%
\pgfpathlineto{\pgfqpoint{4.605517in}{3.885988in}}%
\pgfpathlineto{\pgfqpoint{4.607677in}{3.883853in}}%
\pgfpathlineto{\pgfqpoint{4.640876in}{3.850667in}}%
\pgfpathlineto{\pgfqpoint{4.644178in}{3.847333in}}%
\pgfpathlineto{\pgfqpoint{4.647758in}{3.843784in}}%
\pgfpathlineto{\pgfqpoint{4.678125in}{3.813333in}}%
\pgfpathlineto{\pgfqpoint{4.687838in}{3.803587in}}%
\pgfpathlineto{\pgfqpoint{4.702015in}{3.789205in}}%
\pgfpathlineto{\pgfqpoint{4.715266in}{3.776000in}}%
\pgfpathlineto{\pgfqpoint{4.727919in}{3.763265in}}%
\pgfpathlineto{\pgfqpoint{4.740539in}{3.750422in}}%
\pgfpathlineto{\pgfqpoint{4.752299in}{3.738667in}}%
\pgfpathlineto{\pgfqpoint{4.768000in}{3.722815in}}%
\pgfusepath{fill}%
\end{pgfscope}%
\begin{pgfscope}%
\pgfpathrectangle{\pgfqpoint{0.800000in}{0.528000in}}{\pgfqpoint{3.968000in}{3.696000in}}%
\pgfusepath{clip}%
\pgfsetbuttcap%
\pgfsetroundjoin%
\definecolor{currentfill}{rgb}{0.458674,0.816363,0.329727}%
\pgfsetfillcolor{currentfill}%
\pgfsetlinewidth{0.000000pt}%
\definecolor{currentstroke}{rgb}{0.000000,0.000000,0.000000}%
\pgfsetstrokecolor{currentstroke}%
\pgfsetdash{}{0pt}%
\pgfpathmoveto{\pgfqpoint{4.768000in}{3.728128in}}%
\pgfpathlineto{\pgfqpoint{4.757561in}{3.738667in}}%
\pgfpathlineto{\pgfqpoint{4.743263in}{3.752959in}}%
\pgfpathlineto{\pgfqpoint{4.727919in}{3.768574in}}%
\pgfpathlineto{\pgfqpoint{4.720541in}{3.776000in}}%
\pgfpathlineto{\pgfqpoint{4.704741in}{3.791744in}}%
\pgfpathlineto{\pgfqpoint{4.687838in}{3.808893in}}%
\pgfpathlineto{\pgfqpoint{4.683413in}{3.813333in}}%
\pgfpathlineto{\pgfqpoint{4.647758in}{3.849086in}}%
\pgfpathlineto{\pgfqpoint{4.646936in}{3.849901in}}%
\pgfpathlineto{\pgfqpoint{4.646177in}{3.850667in}}%
\pgfpathlineto{\pgfqpoint{4.624581in}{3.872254in}}%
\pgfpathlineto{\pgfqpoint{4.608820in}{3.888000in}}%
\pgfpathlineto{\pgfqpoint{4.608265in}{3.888548in}}%
\pgfpathlineto{\pgfqpoint{4.607677in}{3.889141in}}%
\pgfpathlineto{\pgfqpoint{4.571336in}{3.925333in}}%
\pgfpathlineto{\pgfqpoint{4.567596in}{3.929055in}}%
\pgfpathlineto{\pgfqpoint{4.533742in}{3.962667in}}%
\pgfpathlineto{\pgfqpoint{4.530713in}{3.965645in}}%
\pgfpathlineto{\pgfqpoint{4.527515in}{3.968845in}}%
\pgfpathlineto{\pgfqpoint{4.496039in}{4.000000in}}%
\pgfpathlineto{\pgfqpoint{4.487434in}{4.008511in}}%
\pgfpathlineto{\pgfqpoint{4.472341in}{4.023274in}}%
\pgfpathlineto{\pgfqpoint{4.458224in}{4.037333in}}%
\pgfpathlineto{\pgfqpoint{4.447354in}{4.048053in}}%
\pgfpathlineto{\pgfqpoint{4.420298in}{4.074667in}}%
\pgfpathlineto{\pgfqpoint{4.413931in}{4.080868in}}%
\pgfpathlineto{\pgfqpoint{4.407273in}{4.087471in}}%
\pgfpathlineto{\pgfqpoint{4.382258in}{4.112000in}}%
\pgfpathlineto{\pgfqpoint{4.374882in}{4.119163in}}%
\pgfpathlineto{\pgfqpoint{4.367192in}{4.126765in}}%
\pgfpathlineto{\pgfqpoint{4.355316in}{4.138272in}}%
\pgfpathlineto{\pgfqpoint{4.344106in}{4.149333in}}%
\pgfpathlineto{\pgfqpoint{4.335772in}{4.157400in}}%
\pgfpathlineto{\pgfqpoint{4.327111in}{4.165936in}}%
\pgfpathlineto{\pgfqpoint{4.305838in}{4.186667in}}%
\pgfpathlineto{\pgfqpoint{4.296601in}{4.195581in}}%
\pgfpathlineto{\pgfqpoint{4.287030in}{4.204984in}}%
\pgfpathlineto{\pgfqpoint{4.267456in}{4.224000in}}%
\pgfpathlineto{\pgfqpoint{4.264770in}{4.224000in}}%
\pgfpathlineto{\pgfqpoint{4.287030in}{4.202374in}}%
\pgfpathlineto{\pgfqpoint{4.295237in}{4.194311in}}%
\pgfpathlineto{\pgfqpoint{4.303159in}{4.186667in}}%
\pgfpathlineto{\pgfqpoint{4.327111in}{4.163325in}}%
\pgfpathlineto{\pgfqpoint{4.334410in}{4.156132in}}%
\pgfpathlineto{\pgfqpoint{4.341433in}{4.149333in}}%
\pgfpathlineto{\pgfqpoint{4.353942in}{4.136991in}}%
\pgfpathlineto{\pgfqpoint{4.367192in}{4.124152in}}%
\pgfpathlineto{\pgfqpoint{4.373521in}{4.117895in}}%
\pgfpathlineto{\pgfqpoint{4.379592in}{4.112000in}}%
\pgfpathlineto{\pgfqpoint{4.407273in}{4.084856in}}%
\pgfpathlineto{\pgfqpoint{4.412571in}{4.079602in}}%
\pgfpathlineto{\pgfqpoint{4.417638in}{4.074667in}}%
\pgfpathlineto{\pgfqpoint{4.447354in}{4.045436in}}%
\pgfpathlineto{\pgfqpoint{4.455571in}{4.037333in}}%
\pgfpathlineto{\pgfqpoint{4.470970in}{4.021997in}}%
\pgfpathlineto{\pgfqpoint{4.487434in}{4.005893in}}%
\pgfpathlineto{\pgfqpoint{4.493392in}{4.000000in}}%
\pgfpathlineto{\pgfqpoint{4.527515in}{3.966226in}}%
\pgfpathlineto{\pgfqpoint{4.529357in}{3.964382in}}%
\pgfpathlineto{\pgfqpoint{4.531102in}{3.962667in}}%
\pgfpathlineto{\pgfqpoint{4.567596in}{3.926434in}}%
\pgfpathlineto{\pgfqpoint{4.568702in}{3.925333in}}%
\pgfpathlineto{\pgfqpoint{4.584681in}{3.909419in}}%
\pgfpathlineto{\pgfqpoint{4.606175in}{3.888000in}}%
\pgfpathlineto{\pgfqpoint{4.606897in}{3.887274in}}%
\pgfpathlineto{\pgfqpoint{4.607677in}{3.886503in}}%
\pgfpathlineto{\pgfqpoint{4.643527in}{3.850667in}}%
\pgfpathlineto{\pgfqpoint{4.645557in}{3.848617in}}%
\pgfpathlineto{\pgfqpoint{4.647758in}{3.846435in}}%
\pgfpathlineto{\pgfqpoint{4.680769in}{3.813333in}}%
\pgfpathlineto{\pgfqpoint{4.687838in}{3.806240in}}%
\pgfpathlineto{\pgfqpoint{4.703378in}{3.790474in}}%
\pgfpathlineto{\pgfqpoint{4.717903in}{3.776000in}}%
\pgfpathlineto{\pgfqpoint{4.727919in}{3.765919in}}%
\pgfpathlineto{\pgfqpoint{4.741901in}{3.751690in}}%
\pgfpathlineto{\pgfqpoint{4.754930in}{3.738667in}}%
\pgfpathlineto{\pgfqpoint{4.768000in}{3.725471in}}%
\pgfusepath{fill}%
\end{pgfscope}%
\begin{pgfscope}%
\pgfpathrectangle{\pgfqpoint{0.800000in}{0.528000in}}{\pgfqpoint{3.968000in}{3.696000in}}%
\pgfusepath{clip}%
\pgfsetbuttcap%
\pgfsetroundjoin%
\definecolor{currentfill}{rgb}{0.458674,0.816363,0.329727}%
\pgfsetfillcolor{currentfill}%
\pgfsetlinewidth{0.000000pt}%
\definecolor{currentstroke}{rgb}{0.000000,0.000000,0.000000}%
\pgfsetstrokecolor{currentstroke}%
\pgfsetdash{}{0pt}%
\pgfpathmoveto{\pgfqpoint{4.768000in}{3.730784in}}%
\pgfpathlineto{\pgfqpoint{4.760193in}{3.738667in}}%
\pgfpathlineto{\pgfqpoint{4.744625in}{3.754227in}}%
\pgfpathlineto{\pgfqpoint{4.727919in}{3.771229in}}%
\pgfpathlineto{\pgfqpoint{4.723179in}{3.776000in}}%
\pgfpathlineto{\pgfqpoint{4.706105in}{3.793014in}}%
\pgfpathlineto{\pgfqpoint{4.687838in}{3.811546in}}%
\pgfpathlineto{\pgfqpoint{4.686057in}{3.813333in}}%
\pgfpathlineto{\pgfqpoint{4.662777in}{3.836677in}}%
\pgfpathlineto{\pgfqpoint{4.648816in}{3.850667in}}%
\pgfpathlineto{\pgfqpoint{4.648303in}{3.851175in}}%
\pgfpathlineto{\pgfqpoint{4.647758in}{3.851726in}}%
\pgfpathlineto{\pgfqpoint{4.611447in}{3.888000in}}%
\pgfpathlineto{\pgfqpoint{4.609619in}{3.889809in}}%
\pgfpathlineto{\pgfqpoint{4.607677in}{3.891764in}}%
\pgfpathlineto{\pgfqpoint{4.573970in}{3.925333in}}%
\pgfpathlineto{\pgfqpoint{4.567596in}{3.931677in}}%
\pgfpathlineto{\pgfqpoint{4.536383in}{3.962667in}}%
\pgfpathlineto{\pgfqpoint{4.532069in}{3.966908in}}%
\pgfpathlineto{\pgfqpoint{4.527515in}{3.971465in}}%
\pgfpathlineto{\pgfqpoint{4.498686in}{4.000000in}}%
\pgfpathlineto{\pgfqpoint{4.487434in}{4.011129in}}%
\pgfpathlineto{\pgfqpoint{4.473712in}{4.024551in}}%
\pgfpathlineto{\pgfqpoint{4.460877in}{4.037333in}}%
\pgfpathlineto{\pgfqpoint{4.447354in}{4.050669in}}%
\pgfpathlineto{\pgfqpoint{4.422957in}{4.074667in}}%
\pgfpathlineto{\pgfqpoint{4.415290in}{4.082135in}}%
\pgfpathlineto{\pgfqpoint{4.407273in}{4.090085in}}%
\pgfpathlineto{\pgfqpoint{4.384925in}{4.112000in}}%
\pgfpathlineto{\pgfqpoint{4.376243in}{4.120430in}}%
\pgfpathlineto{\pgfqpoint{4.367192in}{4.129378in}}%
\pgfpathlineto{\pgfqpoint{4.356691in}{4.139553in}}%
\pgfpathlineto{\pgfqpoint{4.346778in}{4.149333in}}%
\pgfpathlineto{\pgfqpoint{4.337134in}{4.158669in}}%
\pgfpathlineto{\pgfqpoint{4.327111in}{4.168547in}}%
\pgfpathlineto{\pgfqpoint{4.308518in}{4.186667in}}%
\pgfpathlineto{\pgfqpoint{4.297964in}{4.196851in}}%
\pgfpathlineto{\pgfqpoint{4.287030in}{4.207593in}}%
\pgfpathlineto{\pgfqpoint{4.270142in}{4.224000in}}%
\pgfpathlineto{\pgfqpoint{4.267456in}{4.224000in}}%
\pgfpathlineto{\pgfqpoint{4.287030in}{4.204984in}}%
\pgfpathlineto{\pgfqpoint{4.296601in}{4.195581in}}%
\pgfpathlineto{\pgfqpoint{4.305838in}{4.186667in}}%
\pgfpathlineto{\pgfqpoint{4.327111in}{4.165936in}}%
\pgfpathlineto{\pgfqpoint{4.335772in}{4.157400in}}%
\pgfpathlineto{\pgfqpoint{4.344106in}{4.149333in}}%
\pgfpathlineto{\pgfqpoint{4.355316in}{4.138272in}}%
\pgfpathlineto{\pgfqpoint{4.367192in}{4.126765in}}%
\pgfpathlineto{\pgfqpoint{4.374882in}{4.119163in}}%
\pgfpathlineto{\pgfqpoint{4.382258in}{4.112000in}}%
\pgfpathlineto{\pgfqpoint{4.407273in}{4.087471in}}%
\pgfpathlineto{\pgfqpoint{4.413931in}{4.080868in}}%
\pgfpathlineto{\pgfqpoint{4.420298in}{4.074667in}}%
\pgfpathlineto{\pgfqpoint{4.447354in}{4.048053in}}%
\pgfpathlineto{\pgfqpoint{4.458224in}{4.037333in}}%
\pgfpathlineto{\pgfqpoint{4.472341in}{4.023274in}}%
\pgfpathlineto{\pgfqpoint{4.487434in}{4.008511in}}%
\pgfpathlineto{\pgfqpoint{4.496039in}{4.000000in}}%
\pgfpathlineto{\pgfqpoint{4.527515in}{3.968845in}}%
\pgfpathlineto{\pgfqpoint{4.530713in}{3.965645in}}%
\pgfpathlineto{\pgfqpoint{4.533742in}{3.962667in}}%
\pgfpathlineto{\pgfqpoint{4.567596in}{3.929055in}}%
\pgfpathlineto{\pgfqpoint{4.571336in}{3.925333in}}%
\pgfpathlineto{\pgfqpoint{4.607677in}{3.889141in}}%
\pgfpathlineto{\pgfqpoint{4.608265in}{3.888548in}}%
\pgfpathlineto{\pgfqpoint{4.608820in}{3.888000in}}%
\pgfpathlineto{\pgfqpoint{4.624581in}{3.872254in}}%
\pgfpathlineto{\pgfqpoint{4.646177in}{3.850667in}}%
\pgfpathlineto{\pgfqpoint{4.646936in}{3.849901in}}%
\pgfpathlineto{\pgfqpoint{4.647758in}{3.849086in}}%
\pgfpathlineto{\pgfqpoint{4.683413in}{3.813333in}}%
\pgfpathlineto{\pgfqpoint{4.687838in}{3.808893in}}%
\pgfpathlineto{\pgfqpoint{4.704741in}{3.791744in}}%
\pgfpathlineto{\pgfqpoint{4.720541in}{3.776000in}}%
\pgfpathlineto{\pgfqpoint{4.727919in}{3.768574in}}%
\pgfpathlineto{\pgfqpoint{4.743263in}{3.752959in}}%
\pgfpathlineto{\pgfqpoint{4.757561in}{3.738667in}}%
\pgfpathlineto{\pgfqpoint{4.768000in}{3.728128in}}%
\pgfusepath{fill}%
\end{pgfscope}%
\begin{pgfscope}%
\pgfpathrectangle{\pgfqpoint{0.800000in}{0.528000in}}{\pgfqpoint{3.968000in}{3.696000in}}%
\pgfusepath{clip}%
\pgfsetbuttcap%
\pgfsetroundjoin%
\definecolor{currentfill}{rgb}{0.458674,0.816363,0.329727}%
\pgfsetfillcolor{currentfill}%
\pgfsetlinewidth{0.000000pt}%
\definecolor{currentstroke}{rgb}{0.000000,0.000000,0.000000}%
\pgfsetstrokecolor{currentstroke}%
\pgfsetdash{}{0pt}%
\pgfpathmoveto{\pgfqpoint{4.768000in}{3.733441in}}%
\pgfpathlineto{\pgfqpoint{4.762824in}{3.738667in}}%
\pgfpathlineto{\pgfqpoint{4.745987in}{3.755496in}}%
\pgfpathlineto{\pgfqpoint{4.727919in}{3.773883in}}%
\pgfpathlineto{\pgfqpoint{4.725816in}{3.776000in}}%
\pgfpathlineto{\pgfqpoint{4.707468in}{3.794284in}}%
\pgfpathlineto{\pgfqpoint{4.688692in}{3.813333in}}%
\pgfpathlineto{\pgfqpoint{4.687838in}{3.814190in}}%
\pgfpathlineto{\pgfqpoint{4.651437in}{3.850667in}}%
\pgfpathlineto{\pgfqpoint{4.649655in}{3.852434in}}%
\pgfpathlineto{\pgfqpoint{4.647758in}{3.854351in}}%
\pgfpathlineto{\pgfqpoint{4.614075in}{3.888000in}}%
\pgfpathlineto{\pgfqpoint{4.610972in}{3.891069in}}%
\pgfpathlineto{\pgfqpoint{4.607677in}{3.894387in}}%
\pgfpathlineto{\pgfqpoint{4.576604in}{3.925333in}}%
\pgfpathlineto{\pgfqpoint{4.567596in}{3.934298in}}%
\pgfpathlineto{\pgfqpoint{4.539023in}{3.962667in}}%
\pgfpathlineto{\pgfqpoint{4.533424in}{3.968171in}}%
\pgfpathlineto{\pgfqpoint{4.527515in}{3.974085in}}%
\pgfpathlineto{\pgfqpoint{4.501332in}{4.000000in}}%
\pgfpathlineto{\pgfqpoint{4.487434in}{4.013747in}}%
\pgfpathlineto{\pgfqpoint{4.475083in}{4.025828in}}%
\pgfpathlineto{\pgfqpoint{4.463531in}{4.037333in}}%
\pgfpathlineto{\pgfqpoint{4.447354in}{4.053285in}}%
\pgfpathlineto{\pgfqpoint{4.425617in}{4.074667in}}%
\pgfpathlineto{\pgfqpoint{4.416650in}{4.083401in}}%
\pgfpathlineto{\pgfqpoint{4.407273in}{4.092700in}}%
\pgfpathlineto{\pgfqpoint{4.387591in}{4.112000in}}%
\pgfpathlineto{\pgfqpoint{4.377604in}{4.121698in}}%
\pgfpathlineto{\pgfqpoint{4.367192in}{4.131991in}}%
\pgfpathlineto{\pgfqpoint{4.358066in}{4.140833in}}%
\pgfpathlineto{\pgfqpoint{4.349451in}{4.149333in}}%
\pgfpathlineto{\pgfqpoint{4.338496in}{4.159938in}}%
\pgfpathlineto{\pgfqpoint{4.327111in}{4.171158in}}%
\pgfpathlineto{\pgfqpoint{4.311197in}{4.186667in}}%
\pgfpathlineto{\pgfqpoint{4.299328in}{4.198121in}}%
\pgfpathlineto{\pgfqpoint{4.287030in}{4.210203in}}%
\pgfpathlineto{\pgfqpoint{4.272828in}{4.224000in}}%
\pgfpathlineto{\pgfqpoint{4.270142in}{4.224000in}}%
\pgfpathlineto{\pgfqpoint{4.287030in}{4.207593in}}%
\pgfpathlineto{\pgfqpoint{4.297964in}{4.196851in}}%
\pgfpathlineto{\pgfqpoint{4.308518in}{4.186667in}}%
\pgfpathlineto{\pgfqpoint{4.327111in}{4.168547in}}%
\pgfpathlineto{\pgfqpoint{4.337134in}{4.158669in}}%
\pgfpathlineto{\pgfqpoint{4.346778in}{4.149333in}}%
\pgfpathlineto{\pgfqpoint{4.356691in}{4.139553in}}%
\pgfpathlineto{\pgfqpoint{4.367192in}{4.129378in}}%
\pgfpathlineto{\pgfqpoint{4.376243in}{4.120430in}}%
\pgfpathlineto{\pgfqpoint{4.384925in}{4.112000in}}%
\pgfpathlineto{\pgfqpoint{4.407273in}{4.090085in}}%
\pgfpathlineto{\pgfqpoint{4.415290in}{4.082135in}}%
\pgfpathlineto{\pgfqpoint{4.422957in}{4.074667in}}%
\pgfpathlineto{\pgfqpoint{4.447354in}{4.050669in}}%
\pgfpathlineto{\pgfqpoint{4.460877in}{4.037333in}}%
\pgfpathlineto{\pgfqpoint{4.473712in}{4.024551in}}%
\pgfpathlineto{\pgfqpoint{4.487434in}{4.011129in}}%
\pgfpathlineto{\pgfqpoint{4.498686in}{4.000000in}}%
\pgfpathlineto{\pgfqpoint{4.527515in}{3.971465in}}%
\pgfpathlineto{\pgfqpoint{4.532069in}{3.966908in}}%
\pgfpathlineto{\pgfqpoint{4.536383in}{3.962667in}}%
\pgfpathlineto{\pgfqpoint{4.567596in}{3.931677in}}%
\pgfpathlineto{\pgfqpoint{4.573970in}{3.925333in}}%
\pgfpathlineto{\pgfqpoint{4.607677in}{3.891764in}}%
\pgfpathlineto{\pgfqpoint{4.609619in}{3.889809in}}%
\pgfpathlineto{\pgfqpoint{4.611447in}{3.888000in}}%
\pgfpathlineto{\pgfqpoint{4.647758in}{3.851726in}}%
\pgfpathlineto{\pgfqpoint{4.648303in}{3.851175in}}%
\pgfpathlineto{\pgfqpoint{4.648816in}{3.850667in}}%
\pgfpathlineto{\pgfqpoint{4.662777in}{3.836677in}}%
\pgfpathlineto{\pgfqpoint{4.686057in}{3.813333in}}%
\pgfpathlineto{\pgfqpoint{4.687838in}{3.811546in}}%
\pgfpathlineto{\pgfqpoint{4.706105in}{3.793014in}}%
\pgfpathlineto{\pgfqpoint{4.723179in}{3.776000in}}%
\pgfpathlineto{\pgfqpoint{4.727919in}{3.771229in}}%
\pgfpathlineto{\pgfqpoint{4.744625in}{3.754227in}}%
\pgfpathlineto{\pgfqpoint{4.760193in}{3.738667in}}%
\pgfpathlineto{\pgfqpoint{4.768000in}{3.730784in}}%
\pgfusepath{fill}%
\end{pgfscope}%
\begin{pgfscope}%
\pgfpathrectangle{\pgfqpoint{0.800000in}{0.528000in}}{\pgfqpoint{3.968000in}{3.696000in}}%
\pgfusepath{clip}%
\pgfsetbuttcap%
\pgfsetroundjoin%
\definecolor{currentfill}{rgb}{0.458674,0.816363,0.329727}%
\pgfsetfillcolor{currentfill}%
\pgfsetlinewidth{0.000000pt}%
\definecolor{currentstroke}{rgb}{0.000000,0.000000,0.000000}%
\pgfsetstrokecolor{currentstroke}%
\pgfsetdash{}{0pt}%
\pgfpathmoveto{\pgfqpoint{4.768000in}{3.736097in}}%
\pgfpathlineto{\pgfqpoint{4.765455in}{3.738667in}}%
\pgfpathlineto{\pgfqpoint{4.747349in}{3.756765in}}%
\pgfpathlineto{\pgfqpoint{4.728448in}{3.776000in}}%
\pgfpathlineto{\pgfqpoint{4.727919in}{3.776533in}}%
\pgfpathlineto{\pgfqpoint{4.708831in}{3.795554in}}%
\pgfpathlineto{\pgfqpoint{4.691307in}{3.813333in}}%
\pgfpathlineto{\pgfqpoint{4.687838in}{3.816817in}}%
\pgfpathlineto{\pgfqpoint{4.654058in}{3.850667in}}%
\pgfpathlineto{\pgfqpoint{4.651007in}{3.853694in}}%
\pgfpathlineto{\pgfqpoint{4.647758in}{3.856976in}}%
\pgfpathlineto{\pgfqpoint{4.616702in}{3.888000in}}%
\pgfpathlineto{\pgfqpoint{4.612325in}{3.892330in}}%
\pgfpathlineto{\pgfqpoint{4.607677in}{3.897010in}}%
\pgfpathlineto{\pgfqpoint{4.579237in}{3.925333in}}%
\pgfpathlineto{\pgfqpoint{4.567596in}{3.936920in}}%
\pgfpathlineto{\pgfqpoint{4.541663in}{3.962667in}}%
\pgfpathlineto{\pgfqpoint{4.534780in}{3.969434in}}%
\pgfpathlineto{\pgfqpoint{4.527515in}{3.976704in}}%
\pgfpathlineto{\pgfqpoint{4.503979in}{4.000000in}}%
\pgfpathlineto{\pgfqpoint{4.487434in}{4.016365in}}%
\pgfpathlineto{\pgfqpoint{4.476454in}{4.027105in}}%
\pgfpathlineto{\pgfqpoint{4.466184in}{4.037333in}}%
\pgfpathlineto{\pgfqpoint{4.447354in}{4.055901in}}%
\pgfpathlineto{\pgfqpoint{4.428277in}{4.074667in}}%
\pgfpathlineto{\pgfqpoint{4.418009in}{4.084667in}}%
\pgfpathlineto{\pgfqpoint{4.407273in}{4.095314in}}%
\pgfpathlineto{\pgfqpoint{4.390257in}{4.112000in}}%
\pgfpathlineto{\pgfqpoint{4.378964in}{4.122966in}}%
\pgfpathlineto{\pgfqpoint{4.367192in}{4.134603in}}%
\pgfpathlineto{\pgfqpoint{4.359441in}{4.142114in}}%
\pgfpathlineto{\pgfqpoint{4.352124in}{4.149333in}}%
\pgfpathlineto{\pgfqpoint{4.339858in}{4.161207in}}%
\pgfpathlineto{\pgfqpoint{4.327111in}{4.173769in}}%
\pgfpathlineto{\pgfqpoint{4.313877in}{4.186667in}}%
\pgfpathlineto{\pgfqpoint{4.300691in}{4.199391in}}%
\pgfpathlineto{\pgfqpoint{4.287030in}{4.212812in}}%
\pgfpathlineto{\pgfqpoint{4.275514in}{4.224000in}}%
\pgfpathlineto{\pgfqpoint{4.272828in}{4.224000in}}%
\pgfpathlineto{\pgfqpoint{4.287030in}{4.210203in}}%
\pgfpathlineto{\pgfqpoint{4.299328in}{4.198121in}}%
\pgfpathlineto{\pgfqpoint{4.311197in}{4.186667in}}%
\pgfpathlineto{\pgfqpoint{4.327111in}{4.171158in}}%
\pgfpathlineto{\pgfqpoint{4.338496in}{4.159938in}}%
\pgfpathlineto{\pgfqpoint{4.349451in}{4.149333in}}%
\pgfpathlineto{\pgfqpoint{4.358066in}{4.140833in}}%
\pgfpathlineto{\pgfqpoint{4.367192in}{4.131991in}}%
\pgfpathlineto{\pgfqpoint{4.377604in}{4.121698in}}%
\pgfpathlineto{\pgfqpoint{4.387591in}{4.112000in}}%
\pgfpathlineto{\pgfqpoint{4.407273in}{4.092700in}}%
\pgfpathlineto{\pgfqpoint{4.416650in}{4.083401in}}%
\pgfpathlineto{\pgfqpoint{4.425617in}{4.074667in}}%
\pgfpathlineto{\pgfqpoint{4.447354in}{4.053285in}}%
\pgfpathlineto{\pgfqpoint{4.463531in}{4.037333in}}%
\pgfpathlineto{\pgfqpoint{4.475083in}{4.025828in}}%
\pgfpathlineto{\pgfqpoint{4.487434in}{4.013747in}}%
\pgfpathlineto{\pgfqpoint{4.501332in}{4.000000in}}%
\pgfpathlineto{\pgfqpoint{4.527515in}{3.974085in}}%
\pgfpathlineto{\pgfqpoint{4.533424in}{3.968171in}}%
\pgfpathlineto{\pgfqpoint{4.539023in}{3.962667in}}%
\pgfpathlineto{\pgfqpoint{4.567596in}{3.934298in}}%
\pgfpathlineto{\pgfqpoint{4.576604in}{3.925333in}}%
\pgfpathlineto{\pgfqpoint{4.607677in}{3.894387in}}%
\pgfpathlineto{\pgfqpoint{4.610972in}{3.891069in}}%
\pgfpathlineto{\pgfqpoint{4.614075in}{3.888000in}}%
\pgfpathlineto{\pgfqpoint{4.647758in}{3.854351in}}%
\pgfpathlineto{\pgfqpoint{4.649655in}{3.852434in}}%
\pgfpathlineto{\pgfqpoint{4.651437in}{3.850667in}}%
\pgfpathlineto{\pgfqpoint{4.687838in}{3.814190in}}%
\pgfpathlineto{\pgfqpoint{4.688692in}{3.813333in}}%
\pgfpathlineto{\pgfqpoint{4.707468in}{3.794284in}}%
\pgfpathlineto{\pgfqpoint{4.725816in}{3.776000in}}%
\pgfpathlineto{\pgfqpoint{4.727919in}{3.773883in}}%
\pgfpathlineto{\pgfqpoint{4.745987in}{3.755496in}}%
\pgfpathlineto{\pgfqpoint{4.762824in}{3.738667in}}%
\pgfpathlineto{\pgfqpoint{4.768000in}{3.733441in}}%
\pgfusepath{fill}%
\end{pgfscope}%
\begin{pgfscope}%
\pgfpathrectangle{\pgfqpoint{0.800000in}{0.528000in}}{\pgfqpoint{3.968000in}{3.696000in}}%
\pgfusepath{clip}%
\pgfsetbuttcap%
\pgfsetroundjoin%
\definecolor{currentfill}{rgb}{0.468053,0.818921,0.323998}%
\pgfsetfillcolor{currentfill}%
\pgfsetlinewidth{0.000000pt}%
\definecolor{currentstroke}{rgb}{0.000000,0.000000,0.000000}%
\pgfsetstrokecolor{currentstroke}%
\pgfsetdash{}{0pt}%
\pgfpathmoveto{\pgfqpoint{4.768000in}{3.738753in}}%
\pgfpathlineto{\pgfqpoint{4.748711in}{3.758033in}}%
\pgfpathlineto{\pgfqpoint{4.731056in}{3.776000in}}%
\pgfpathlineto{\pgfqpoint{4.727919in}{3.779161in}}%
\pgfpathlineto{\pgfqpoint{4.710194in}{3.796824in}}%
\pgfpathlineto{\pgfqpoint{4.693921in}{3.813333in}}%
\pgfpathlineto{\pgfqpoint{4.687838in}{3.819444in}}%
\pgfpathlineto{\pgfqpoint{4.656679in}{3.850667in}}%
\pgfpathlineto{\pgfqpoint{4.652359in}{3.854953in}}%
\pgfpathlineto{\pgfqpoint{4.647758in}{3.859601in}}%
\pgfpathlineto{\pgfqpoint{4.619330in}{3.888000in}}%
\pgfpathlineto{\pgfqpoint{4.613678in}{3.893590in}}%
\pgfpathlineto{\pgfqpoint{4.607677in}{3.899633in}}%
\pgfpathlineto{\pgfqpoint{4.581871in}{3.925333in}}%
\pgfpathlineto{\pgfqpoint{4.567596in}{3.939541in}}%
\pgfpathlineto{\pgfqpoint{4.544304in}{3.962667in}}%
\pgfpathlineto{\pgfqpoint{4.536136in}{3.970696in}}%
\pgfpathlineto{\pgfqpoint{4.527515in}{3.979324in}}%
\pgfpathlineto{\pgfqpoint{4.506626in}{4.000000in}}%
\pgfpathlineto{\pgfqpoint{4.487434in}{4.018983in}}%
\pgfpathlineto{\pgfqpoint{4.477825in}{4.028382in}}%
\pgfpathlineto{\pgfqpoint{4.468837in}{4.037333in}}%
\pgfpathlineto{\pgfqpoint{4.447354in}{4.058518in}}%
\pgfpathlineto{\pgfqpoint{4.430937in}{4.074667in}}%
\pgfpathlineto{\pgfqpoint{4.419369in}{4.085934in}}%
\pgfpathlineto{\pgfqpoint{4.407273in}{4.097929in}}%
\pgfpathlineto{\pgfqpoint{4.392923in}{4.112000in}}%
\pgfpathlineto{\pgfqpoint{4.380325in}{4.124233in}}%
\pgfpathlineto{\pgfqpoint{4.367192in}{4.137216in}}%
\pgfpathlineto{\pgfqpoint{4.360816in}{4.143394in}}%
\pgfpathlineto{\pgfqpoint{4.354797in}{4.149333in}}%
\pgfpathlineto{\pgfqpoint{4.341220in}{4.162475in}}%
\pgfpathlineto{\pgfqpoint{4.327111in}{4.176380in}}%
\pgfpathlineto{\pgfqpoint{4.316556in}{4.186667in}}%
\pgfpathlineto{\pgfqpoint{4.302054in}{4.200661in}}%
\pgfpathlineto{\pgfqpoint{4.287030in}{4.215421in}}%
\pgfpathlineto{\pgfqpoint{4.278200in}{4.224000in}}%
\pgfpathlineto{\pgfqpoint{4.275514in}{4.224000in}}%
\pgfpathlineto{\pgfqpoint{4.287030in}{4.212812in}}%
\pgfpathlineto{\pgfqpoint{4.300691in}{4.199391in}}%
\pgfpathlineto{\pgfqpoint{4.313877in}{4.186667in}}%
\pgfpathlineto{\pgfqpoint{4.327111in}{4.173769in}}%
\pgfpathlineto{\pgfqpoint{4.339858in}{4.161207in}}%
\pgfpathlineto{\pgfqpoint{4.352124in}{4.149333in}}%
\pgfpathlineto{\pgfqpoint{4.359441in}{4.142114in}}%
\pgfpathlineto{\pgfqpoint{4.367192in}{4.134603in}}%
\pgfpathlineto{\pgfqpoint{4.378964in}{4.122966in}}%
\pgfpathlineto{\pgfqpoint{4.390257in}{4.112000in}}%
\pgfpathlineto{\pgfqpoint{4.407273in}{4.095314in}}%
\pgfpathlineto{\pgfqpoint{4.418009in}{4.084667in}}%
\pgfpathlineto{\pgfqpoint{4.428277in}{4.074667in}}%
\pgfpathlineto{\pgfqpoint{4.447354in}{4.055901in}}%
\pgfpathlineto{\pgfqpoint{4.466184in}{4.037333in}}%
\pgfpathlineto{\pgfqpoint{4.476454in}{4.027105in}}%
\pgfpathlineto{\pgfqpoint{4.487434in}{4.016365in}}%
\pgfpathlineto{\pgfqpoint{4.503979in}{4.000000in}}%
\pgfpathlineto{\pgfqpoint{4.527515in}{3.976704in}}%
\pgfpathlineto{\pgfqpoint{4.534780in}{3.969434in}}%
\pgfpathlineto{\pgfqpoint{4.541663in}{3.962667in}}%
\pgfpathlineto{\pgfqpoint{4.567596in}{3.936920in}}%
\pgfpathlineto{\pgfqpoint{4.579237in}{3.925333in}}%
\pgfpathlineto{\pgfqpoint{4.607677in}{3.897010in}}%
\pgfpathlineto{\pgfqpoint{4.612325in}{3.892330in}}%
\pgfpathlineto{\pgfqpoint{4.616702in}{3.888000in}}%
\pgfpathlineto{\pgfqpoint{4.647758in}{3.856976in}}%
\pgfpathlineto{\pgfqpoint{4.651007in}{3.853694in}}%
\pgfpathlineto{\pgfqpoint{4.654058in}{3.850667in}}%
\pgfpathlineto{\pgfqpoint{4.687838in}{3.816817in}}%
\pgfpathlineto{\pgfqpoint{4.691307in}{3.813333in}}%
\pgfpathlineto{\pgfqpoint{4.708831in}{3.795554in}}%
\pgfpathlineto{\pgfqpoint{4.727919in}{3.776533in}}%
\pgfpathlineto{\pgfqpoint{4.728448in}{3.776000in}}%
\pgfpathlineto{\pgfqpoint{4.747349in}{3.756765in}}%
\pgfpathlineto{\pgfqpoint{4.765455in}{3.738667in}}%
\pgfpathlineto{\pgfqpoint{4.768000in}{3.736097in}}%
\pgfpathlineto{\pgfqpoint{4.768000in}{3.738667in}}%
\pgfusepath{fill}%
\end{pgfscope}%
\begin{pgfscope}%
\pgfpathrectangle{\pgfqpoint{0.800000in}{0.528000in}}{\pgfqpoint{3.968000in}{3.696000in}}%
\pgfusepath{clip}%
\pgfsetbuttcap%
\pgfsetroundjoin%
\definecolor{currentfill}{rgb}{0.468053,0.818921,0.323998}%
\pgfsetfillcolor{currentfill}%
\pgfsetlinewidth{0.000000pt}%
\definecolor{currentstroke}{rgb}{0.000000,0.000000,0.000000}%
\pgfsetstrokecolor{currentstroke}%
\pgfsetdash{}{0pt}%
\pgfpathmoveto{\pgfqpoint{4.768000in}{3.741383in}}%
\pgfpathlineto{\pgfqpoint{4.750073in}{3.759302in}}%
\pgfpathlineto{\pgfqpoint{4.733665in}{3.776000in}}%
\pgfpathlineto{\pgfqpoint{4.727919in}{3.781789in}}%
\pgfpathlineto{\pgfqpoint{4.711558in}{3.798093in}}%
\pgfpathlineto{\pgfqpoint{4.696536in}{3.813333in}}%
\pgfpathlineto{\pgfqpoint{4.687838in}{3.822070in}}%
\pgfpathlineto{\pgfqpoint{4.659301in}{3.850667in}}%
\pgfpathlineto{\pgfqpoint{4.653711in}{3.856212in}}%
\pgfpathlineto{\pgfqpoint{4.647758in}{3.862226in}}%
\pgfpathlineto{\pgfqpoint{4.621957in}{3.888000in}}%
\pgfpathlineto{\pgfqpoint{4.615032in}{3.894851in}}%
\pgfpathlineto{\pgfqpoint{4.607677in}{3.902257in}}%
\pgfpathlineto{\pgfqpoint{4.584505in}{3.925333in}}%
\pgfpathlineto{\pgfqpoint{4.567596in}{3.942162in}}%
\pgfpathlineto{\pgfqpoint{4.546944in}{3.962667in}}%
\pgfpathlineto{\pgfqpoint{4.537492in}{3.971959in}}%
\pgfpathlineto{\pgfqpoint{4.527515in}{3.981944in}}%
\pgfpathlineto{\pgfqpoint{4.509273in}{4.000000in}}%
\pgfpathlineto{\pgfqpoint{4.487434in}{4.021601in}}%
\pgfpathlineto{\pgfqpoint{4.479196in}{4.029659in}}%
\pgfpathlineto{\pgfqpoint{4.471490in}{4.037333in}}%
\pgfpathlineto{\pgfqpoint{4.447354in}{4.061134in}}%
\pgfpathlineto{\pgfqpoint{4.433596in}{4.074667in}}%
\pgfpathlineto{\pgfqpoint{4.420729in}{4.087200in}}%
\pgfpathlineto{\pgfqpoint{4.407273in}{4.100543in}}%
\pgfpathlineto{\pgfqpoint{4.395590in}{4.112000in}}%
\pgfpathlineto{\pgfqpoint{4.381686in}{4.125501in}}%
\pgfpathlineto{\pgfqpoint{4.367192in}{4.139829in}}%
\pgfpathlineto{\pgfqpoint{4.362191in}{4.144675in}}%
\pgfpathlineto{\pgfqpoint{4.357470in}{4.149333in}}%
\pgfpathlineto{\pgfqpoint{4.342583in}{4.163744in}}%
\pgfpathlineto{\pgfqpoint{4.327111in}{4.178992in}}%
\pgfpathlineto{\pgfqpoint{4.319235in}{4.186667in}}%
\pgfpathlineto{\pgfqpoint{4.303418in}{4.201931in}}%
\pgfpathlineto{\pgfqpoint{4.287030in}{4.218031in}}%
\pgfpathlineto{\pgfqpoint{4.280886in}{4.224000in}}%
\pgfpathlineto{\pgfqpoint{4.278200in}{4.224000in}}%
\pgfpathlineto{\pgfqpoint{4.287030in}{4.215421in}}%
\pgfpathlineto{\pgfqpoint{4.302054in}{4.200661in}}%
\pgfpathlineto{\pgfqpoint{4.316556in}{4.186667in}}%
\pgfpathlineto{\pgfqpoint{4.327111in}{4.176380in}}%
\pgfpathlineto{\pgfqpoint{4.341220in}{4.162475in}}%
\pgfpathlineto{\pgfqpoint{4.354797in}{4.149333in}}%
\pgfpathlineto{\pgfqpoint{4.360816in}{4.143394in}}%
\pgfpathlineto{\pgfqpoint{4.367192in}{4.137216in}}%
\pgfpathlineto{\pgfqpoint{4.380325in}{4.124233in}}%
\pgfpathlineto{\pgfqpoint{4.392923in}{4.112000in}}%
\pgfpathlineto{\pgfqpoint{4.407273in}{4.097929in}}%
\pgfpathlineto{\pgfqpoint{4.419369in}{4.085934in}}%
\pgfpathlineto{\pgfqpoint{4.430937in}{4.074667in}}%
\pgfpathlineto{\pgfqpoint{4.447354in}{4.058518in}}%
\pgfpathlineto{\pgfqpoint{4.468837in}{4.037333in}}%
\pgfpathlineto{\pgfqpoint{4.477825in}{4.028382in}}%
\pgfpathlineto{\pgfqpoint{4.487434in}{4.018983in}}%
\pgfpathlineto{\pgfqpoint{4.506626in}{4.000000in}}%
\pgfpathlineto{\pgfqpoint{4.527515in}{3.979324in}}%
\pgfpathlineto{\pgfqpoint{4.536136in}{3.970696in}}%
\pgfpathlineto{\pgfqpoint{4.544304in}{3.962667in}}%
\pgfpathlineto{\pgfqpoint{4.567596in}{3.939541in}}%
\pgfpathlineto{\pgfqpoint{4.581871in}{3.925333in}}%
\pgfpathlineto{\pgfqpoint{4.607677in}{3.899633in}}%
\pgfpathlineto{\pgfqpoint{4.613678in}{3.893590in}}%
\pgfpathlineto{\pgfqpoint{4.619330in}{3.888000in}}%
\pgfpathlineto{\pgfqpoint{4.647758in}{3.859601in}}%
\pgfpathlineto{\pgfqpoint{4.652359in}{3.854953in}}%
\pgfpathlineto{\pgfqpoint{4.656679in}{3.850667in}}%
\pgfpathlineto{\pgfqpoint{4.687838in}{3.819444in}}%
\pgfpathlineto{\pgfqpoint{4.693921in}{3.813333in}}%
\pgfpathlineto{\pgfqpoint{4.710194in}{3.796824in}}%
\pgfpathlineto{\pgfqpoint{4.727919in}{3.779161in}}%
\pgfpathlineto{\pgfqpoint{4.731056in}{3.776000in}}%
\pgfpathlineto{\pgfqpoint{4.748711in}{3.758033in}}%
\pgfpathlineto{\pgfqpoint{4.768000in}{3.738753in}}%
\pgfusepath{fill}%
\end{pgfscope}%
\begin{pgfscope}%
\pgfpathrectangle{\pgfqpoint{0.800000in}{0.528000in}}{\pgfqpoint{3.968000in}{3.696000in}}%
\pgfusepath{clip}%
\pgfsetbuttcap%
\pgfsetroundjoin%
\definecolor{currentfill}{rgb}{0.468053,0.818921,0.323998}%
\pgfsetfillcolor{currentfill}%
\pgfsetlinewidth{0.000000pt}%
\definecolor{currentstroke}{rgb}{0.000000,0.000000,0.000000}%
\pgfsetstrokecolor{currentstroke}%
\pgfsetdash{}{0pt}%
\pgfpathmoveto{\pgfqpoint{4.768000in}{3.744013in}}%
\pgfpathlineto{\pgfqpoint{4.751435in}{3.760571in}}%
\pgfpathlineto{\pgfqpoint{4.736274in}{3.776000in}}%
\pgfpathlineto{\pgfqpoint{4.727919in}{3.784418in}}%
\pgfpathlineto{\pgfqpoint{4.712921in}{3.799363in}}%
\pgfpathlineto{\pgfqpoint{4.699151in}{3.813333in}}%
\pgfpathlineto{\pgfqpoint{4.687838in}{3.824697in}}%
\pgfpathlineto{\pgfqpoint{4.661922in}{3.850667in}}%
\pgfpathlineto{\pgfqpoint{4.655063in}{3.857472in}}%
\pgfpathlineto{\pgfqpoint{4.647758in}{3.864851in}}%
\pgfpathlineto{\pgfqpoint{4.624585in}{3.888000in}}%
\pgfpathlineto{\pgfqpoint{4.616385in}{3.896111in}}%
\pgfpathlineto{\pgfqpoint{4.607677in}{3.904880in}}%
\pgfpathlineto{\pgfqpoint{4.587139in}{3.925333in}}%
\pgfpathlineto{\pgfqpoint{4.567596in}{3.944784in}}%
\pgfpathlineto{\pgfqpoint{4.549584in}{3.962667in}}%
\pgfpathlineto{\pgfqpoint{4.538847in}{3.973222in}}%
\pgfpathlineto{\pgfqpoint{4.527515in}{3.984564in}}%
\pgfpathlineto{\pgfqpoint{4.511919in}{4.000000in}}%
\pgfpathlineto{\pgfqpoint{4.487434in}{4.024219in}}%
\pgfpathlineto{\pgfqpoint{4.480567in}{4.030936in}}%
\pgfpathlineto{\pgfqpoint{4.474143in}{4.037333in}}%
\pgfpathlineto{\pgfqpoint{4.447354in}{4.063750in}}%
\pgfpathlineto{\pgfqpoint{4.436256in}{4.074667in}}%
\pgfpathlineto{\pgfqpoint{4.422088in}{4.088467in}}%
\pgfpathlineto{\pgfqpoint{4.407273in}{4.103158in}}%
\pgfpathlineto{\pgfqpoint{4.398256in}{4.112000in}}%
\pgfpathlineto{\pgfqpoint{4.383047in}{4.126768in}}%
\pgfpathlineto{\pgfqpoint{4.367192in}{4.142442in}}%
\pgfpathlineto{\pgfqpoint{4.363566in}{4.145956in}}%
\pgfpathlineto{\pgfqpoint{4.360142in}{4.149333in}}%
\pgfpathlineto{\pgfqpoint{4.343945in}{4.165013in}}%
\pgfpathlineto{\pgfqpoint{4.327111in}{4.181603in}}%
\pgfpathlineto{\pgfqpoint{4.321915in}{4.186667in}}%
\pgfpathlineto{\pgfqpoint{4.304781in}{4.203201in}}%
\pgfpathlineto{\pgfqpoint{4.287030in}{4.220640in}}%
\pgfpathlineto{\pgfqpoint{4.283572in}{4.224000in}}%
\pgfpathlineto{\pgfqpoint{4.280886in}{4.224000in}}%
\pgfpathlineto{\pgfqpoint{4.287030in}{4.218031in}}%
\pgfpathlineto{\pgfqpoint{4.303418in}{4.201931in}}%
\pgfpathlineto{\pgfqpoint{4.319235in}{4.186667in}}%
\pgfpathlineto{\pgfqpoint{4.327111in}{4.178992in}}%
\pgfpathlineto{\pgfqpoint{4.342583in}{4.163744in}}%
\pgfpathlineto{\pgfqpoint{4.357470in}{4.149333in}}%
\pgfpathlineto{\pgfqpoint{4.362191in}{4.144675in}}%
\pgfpathlineto{\pgfqpoint{4.367192in}{4.139829in}}%
\pgfpathlineto{\pgfqpoint{4.381686in}{4.125501in}}%
\pgfpathlineto{\pgfqpoint{4.395590in}{4.112000in}}%
\pgfpathlineto{\pgfqpoint{4.407273in}{4.100543in}}%
\pgfpathlineto{\pgfqpoint{4.420729in}{4.087200in}}%
\pgfpathlineto{\pgfqpoint{4.433596in}{4.074667in}}%
\pgfpathlineto{\pgfqpoint{4.447354in}{4.061134in}}%
\pgfpathlineto{\pgfqpoint{4.471490in}{4.037333in}}%
\pgfpathlineto{\pgfqpoint{4.479196in}{4.029659in}}%
\pgfpathlineto{\pgfqpoint{4.487434in}{4.021601in}}%
\pgfpathlineto{\pgfqpoint{4.509273in}{4.000000in}}%
\pgfpathlineto{\pgfqpoint{4.527515in}{3.981944in}}%
\pgfpathlineto{\pgfqpoint{4.537492in}{3.971959in}}%
\pgfpathlineto{\pgfqpoint{4.546944in}{3.962667in}}%
\pgfpathlineto{\pgfqpoint{4.567596in}{3.942162in}}%
\pgfpathlineto{\pgfqpoint{4.584505in}{3.925333in}}%
\pgfpathlineto{\pgfqpoint{4.607677in}{3.902257in}}%
\pgfpathlineto{\pgfqpoint{4.615032in}{3.894851in}}%
\pgfpathlineto{\pgfqpoint{4.621957in}{3.888000in}}%
\pgfpathlineto{\pgfqpoint{4.647758in}{3.862226in}}%
\pgfpathlineto{\pgfqpoint{4.653711in}{3.856212in}}%
\pgfpathlineto{\pgfqpoint{4.659301in}{3.850667in}}%
\pgfpathlineto{\pgfqpoint{4.687838in}{3.822070in}}%
\pgfpathlineto{\pgfqpoint{4.696536in}{3.813333in}}%
\pgfpathlineto{\pgfqpoint{4.711558in}{3.798093in}}%
\pgfpathlineto{\pgfqpoint{4.727919in}{3.781789in}}%
\pgfpathlineto{\pgfqpoint{4.733665in}{3.776000in}}%
\pgfpathlineto{\pgfqpoint{4.750073in}{3.759302in}}%
\pgfpathlineto{\pgfqpoint{4.768000in}{3.741383in}}%
\pgfusepath{fill}%
\end{pgfscope}%
\begin{pgfscope}%
\pgfpathrectangle{\pgfqpoint{0.800000in}{0.528000in}}{\pgfqpoint{3.968000in}{3.696000in}}%
\pgfusepath{clip}%
\pgfsetbuttcap%
\pgfsetroundjoin%
\definecolor{currentfill}{rgb}{0.477504,0.821444,0.318195}%
\pgfsetfillcolor{currentfill}%
\pgfsetlinewidth{0.000000pt}%
\definecolor{currentstroke}{rgb}{0.000000,0.000000,0.000000}%
\pgfsetstrokecolor{currentstroke}%
\pgfsetdash{}{0pt}%
\pgfpathmoveto{\pgfqpoint{4.768000in}{3.746643in}}%
\pgfpathlineto{\pgfqpoint{4.752797in}{3.761839in}}%
\pgfpathlineto{\pgfqpoint{4.738882in}{3.776000in}}%
\pgfpathlineto{\pgfqpoint{4.727919in}{3.787046in}}%
\pgfpathlineto{\pgfqpoint{4.714284in}{3.800633in}}%
\pgfpathlineto{\pgfqpoint{4.701766in}{3.813333in}}%
\pgfpathlineto{\pgfqpoint{4.687838in}{3.827323in}}%
\pgfpathlineto{\pgfqpoint{4.664543in}{3.850667in}}%
\pgfpathlineto{\pgfqpoint{4.656415in}{3.858731in}}%
\pgfpathlineto{\pgfqpoint{4.647758in}{3.867476in}}%
\pgfpathlineto{\pgfqpoint{4.627212in}{3.888000in}}%
\pgfpathlineto{\pgfqpoint{4.617738in}{3.897372in}}%
\pgfpathlineto{\pgfqpoint{4.607677in}{3.907503in}}%
\pgfpathlineto{\pgfqpoint{4.589773in}{3.925333in}}%
\pgfpathlineto{\pgfqpoint{4.567596in}{3.947405in}}%
\pgfpathlineto{\pgfqpoint{4.552225in}{3.962667in}}%
\pgfpathlineto{\pgfqpoint{4.540203in}{3.974485in}}%
\pgfpathlineto{\pgfqpoint{4.527515in}{3.987183in}}%
\pgfpathlineto{\pgfqpoint{4.514566in}{4.000000in}}%
\pgfpathlineto{\pgfqpoint{4.487434in}{4.026837in}}%
\pgfpathlineto{\pgfqpoint{4.481938in}{4.032213in}}%
\pgfpathlineto{\pgfqpoint{4.476797in}{4.037333in}}%
\pgfpathlineto{\pgfqpoint{4.447354in}{4.066367in}}%
\pgfpathlineto{\pgfqpoint{4.438916in}{4.074667in}}%
\pgfpathlineto{\pgfqpoint{4.423448in}{4.089733in}}%
\pgfpathlineto{\pgfqpoint{4.407273in}{4.105772in}}%
\pgfpathlineto{\pgfqpoint{4.400922in}{4.112000in}}%
\pgfpathlineto{\pgfqpoint{4.384408in}{4.128036in}}%
\pgfpathlineto{\pgfqpoint{4.367192in}{4.145055in}}%
\pgfpathlineto{\pgfqpoint{4.364941in}{4.147236in}}%
\pgfpathlineto{\pgfqpoint{4.362815in}{4.149333in}}%
\pgfpathlineto{\pgfqpoint{4.345307in}{4.166282in}}%
\pgfpathlineto{\pgfqpoint{4.327111in}{4.184214in}}%
\pgfpathlineto{\pgfqpoint{4.324594in}{4.186667in}}%
\pgfpathlineto{\pgfqpoint{4.306145in}{4.204471in}}%
\pgfpathlineto{\pgfqpoint{4.287030in}{4.223250in}}%
\pgfpathlineto{\pgfqpoint{4.286258in}{4.224000in}}%
\pgfpathlineto{\pgfqpoint{4.283572in}{4.224000in}}%
\pgfpathlineto{\pgfqpoint{4.287030in}{4.220640in}}%
\pgfpathlineto{\pgfqpoint{4.304781in}{4.203201in}}%
\pgfpathlineto{\pgfqpoint{4.321915in}{4.186667in}}%
\pgfpathlineto{\pgfqpoint{4.327111in}{4.181603in}}%
\pgfpathlineto{\pgfqpoint{4.343945in}{4.165013in}}%
\pgfpathlineto{\pgfqpoint{4.360142in}{4.149333in}}%
\pgfpathlineto{\pgfqpoint{4.363566in}{4.145956in}}%
\pgfpathlineto{\pgfqpoint{4.367192in}{4.142442in}}%
\pgfpathlineto{\pgfqpoint{4.383047in}{4.126768in}}%
\pgfpathlineto{\pgfqpoint{4.398256in}{4.112000in}}%
\pgfpathlineto{\pgfqpoint{4.407273in}{4.103158in}}%
\pgfpathlineto{\pgfqpoint{4.422088in}{4.088467in}}%
\pgfpathlineto{\pgfqpoint{4.436256in}{4.074667in}}%
\pgfpathlineto{\pgfqpoint{4.447354in}{4.063750in}}%
\pgfpathlineto{\pgfqpoint{4.474143in}{4.037333in}}%
\pgfpathlineto{\pgfqpoint{4.480567in}{4.030936in}}%
\pgfpathlineto{\pgfqpoint{4.487434in}{4.024219in}}%
\pgfpathlineto{\pgfqpoint{4.511919in}{4.000000in}}%
\pgfpathlineto{\pgfqpoint{4.527515in}{3.984564in}}%
\pgfpathlineto{\pgfqpoint{4.538847in}{3.973222in}}%
\pgfpathlineto{\pgfqpoint{4.549584in}{3.962667in}}%
\pgfpathlineto{\pgfqpoint{4.567596in}{3.944784in}}%
\pgfpathlineto{\pgfqpoint{4.587139in}{3.925333in}}%
\pgfpathlineto{\pgfqpoint{4.607677in}{3.904880in}}%
\pgfpathlineto{\pgfqpoint{4.616385in}{3.896111in}}%
\pgfpathlineto{\pgfqpoint{4.624585in}{3.888000in}}%
\pgfpathlineto{\pgfqpoint{4.647758in}{3.864851in}}%
\pgfpathlineto{\pgfqpoint{4.655063in}{3.857472in}}%
\pgfpathlineto{\pgfqpoint{4.661922in}{3.850667in}}%
\pgfpathlineto{\pgfqpoint{4.687838in}{3.824697in}}%
\pgfpathlineto{\pgfqpoint{4.699151in}{3.813333in}}%
\pgfpathlineto{\pgfqpoint{4.712921in}{3.799363in}}%
\pgfpathlineto{\pgfqpoint{4.727919in}{3.784418in}}%
\pgfpathlineto{\pgfqpoint{4.736274in}{3.776000in}}%
\pgfpathlineto{\pgfqpoint{4.751435in}{3.760571in}}%
\pgfpathlineto{\pgfqpoint{4.768000in}{3.744013in}}%
\pgfusepath{fill}%
\end{pgfscope}%
\begin{pgfscope}%
\pgfpathrectangle{\pgfqpoint{0.800000in}{0.528000in}}{\pgfqpoint{3.968000in}{3.696000in}}%
\pgfusepath{clip}%
\pgfsetbuttcap%
\pgfsetroundjoin%
\definecolor{currentfill}{rgb}{0.477504,0.821444,0.318195}%
\pgfsetfillcolor{currentfill}%
\pgfsetlinewidth{0.000000pt}%
\definecolor{currentstroke}{rgb}{0.000000,0.000000,0.000000}%
\pgfsetstrokecolor{currentstroke}%
\pgfsetdash{}{0pt}%
\pgfpathmoveto{\pgfqpoint{4.768000in}{3.749273in}}%
\pgfpathlineto{\pgfqpoint{4.754159in}{3.763108in}}%
\pgfpathlineto{\pgfqpoint{4.741491in}{3.776000in}}%
\pgfpathlineto{\pgfqpoint{4.727919in}{3.789674in}}%
\pgfpathlineto{\pgfqpoint{4.715648in}{3.801903in}}%
\pgfpathlineto{\pgfqpoint{4.704381in}{3.813333in}}%
\pgfpathlineto{\pgfqpoint{4.687838in}{3.829950in}}%
\pgfpathlineto{\pgfqpoint{4.667164in}{3.850667in}}%
\pgfpathlineto{\pgfqpoint{4.657767in}{3.859990in}}%
\pgfpathlineto{\pgfqpoint{4.647758in}{3.870100in}}%
\pgfpathlineto{\pgfqpoint{4.629840in}{3.888000in}}%
\pgfpathlineto{\pgfqpoint{4.619091in}{3.898632in}}%
\pgfpathlineto{\pgfqpoint{4.607677in}{3.910126in}}%
\pgfpathlineto{\pgfqpoint{4.592407in}{3.925333in}}%
\pgfpathlineto{\pgfqpoint{4.567596in}{3.950027in}}%
\pgfpathlineto{\pgfqpoint{4.554865in}{3.962667in}}%
\pgfpathlineto{\pgfqpoint{4.541559in}{3.975748in}}%
\pgfpathlineto{\pgfqpoint{4.527515in}{3.989803in}}%
\pgfpathlineto{\pgfqpoint{4.517213in}{4.000000in}}%
\pgfpathlineto{\pgfqpoint{4.487434in}{4.029455in}}%
\pgfpathlineto{\pgfqpoint{4.483309in}{4.033490in}}%
\pgfpathlineto{\pgfqpoint{4.479450in}{4.037333in}}%
\pgfpathlineto{\pgfqpoint{4.447354in}{4.068983in}}%
\pgfpathlineto{\pgfqpoint{4.441575in}{4.074667in}}%
\pgfpathlineto{\pgfqpoint{4.424807in}{4.090999in}}%
\pgfpathlineto{\pgfqpoint{4.407273in}{4.108387in}}%
\pgfpathlineto{\pgfqpoint{4.403588in}{4.112000in}}%
\pgfpathlineto{\pgfqpoint{4.385769in}{4.129303in}}%
\pgfpathlineto{\pgfqpoint{4.367192in}{4.147668in}}%
\pgfpathlineto{\pgfqpoint{4.366315in}{4.148517in}}%
\pgfpathlineto{\pgfqpoint{4.365488in}{4.149333in}}%
\pgfpathlineto{\pgfqpoint{4.346669in}{4.167550in}}%
\pgfpathlineto{\pgfqpoint{4.327272in}{4.186667in}}%
\pgfpathlineto{\pgfqpoint{4.327111in}{4.186823in}}%
\pgfpathlineto{\pgfqpoint{4.307508in}{4.205741in}}%
\pgfpathlineto{\pgfqpoint{4.288923in}{4.224000in}}%
\pgfpathlineto{\pgfqpoint{4.287030in}{4.224000in}}%
\pgfpathlineto{\pgfqpoint{4.286258in}{4.224000in}}%
\pgfpathlineto{\pgfqpoint{4.287030in}{4.223250in}}%
\pgfpathlineto{\pgfqpoint{4.306145in}{4.204471in}}%
\pgfpathlineto{\pgfqpoint{4.324594in}{4.186667in}}%
\pgfpathlineto{\pgfqpoint{4.327111in}{4.184214in}}%
\pgfpathlineto{\pgfqpoint{4.345307in}{4.166282in}}%
\pgfpathlineto{\pgfqpoint{4.362815in}{4.149333in}}%
\pgfpathlineto{\pgfqpoint{4.364941in}{4.147236in}}%
\pgfpathlineto{\pgfqpoint{4.367192in}{4.145055in}}%
\pgfpathlineto{\pgfqpoint{4.384408in}{4.128036in}}%
\pgfpathlineto{\pgfqpoint{4.400922in}{4.112000in}}%
\pgfpathlineto{\pgfqpoint{4.407273in}{4.105772in}}%
\pgfpathlineto{\pgfqpoint{4.423448in}{4.089733in}}%
\pgfpathlineto{\pgfqpoint{4.438916in}{4.074667in}}%
\pgfpathlineto{\pgfqpoint{4.447354in}{4.066367in}}%
\pgfpathlineto{\pgfqpoint{4.476797in}{4.037333in}}%
\pgfpathlineto{\pgfqpoint{4.481938in}{4.032213in}}%
\pgfpathlineto{\pgfqpoint{4.487434in}{4.026837in}}%
\pgfpathlineto{\pgfqpoint{4.514566in}{4.000000in}}%
\pgfpathlineto{\pgfqpoint{4.527515in}{3.987183in}}%
\pgfpathlineto{\pgfqpoint{4.540203in}{3.974485in}}%
\pgfpathlineto{\pgfqpoint{4.552225in}{3.962667in}}%
\pgfpathlineto{\pgfqpoint{4.567596in}{3.947405in}}%
\pgfpathlineto{\pgfqpoint{4.589773in}{3.925333in}}%
\pgfpathlineto{\pgfqpoint{4.607677in}{3.907503in}}%
\pgfpathlineto{\pgfqpoint{4.617738in}{3.897372in}}%
\pgfpathlineto{\pgfqpoint{4.627212in}{3.888000in}}%
\pgfpathlineto{\pgfqpoint{4.647758in}{3.867476in}}%
\pgfpathlineto{\pgfqpoint{4.656415in}{3.858731in}}%
\pgfpathlineto{\pgfqpoint{4.664543in}{3.850667in}}%
\pgfpathlineto{\pgfqpoint{4.687838in}{3.827323in}}%
\pgfpathlineto{\pgfqpoint{4.701766in}{3.813333in}}%
\pgfpathlineto{\pgfqpoint{4.714284in}{3.800633in}}%
\pgfpathlineto{\pgfqpoint{4.727919in}{3.787046in}}%
\pgfpathlineto{\pgfqpoint{4.738882in}{3.776000in}}%
\pgfpathlineto{\pgfqpoint{4.752797in}{3.761839in}}%
\pgfpathlineto{\pgfqpoint{4.768000in}{3.746643in}}%
\pgfusepath{fill}%
\end{pgfscope}%
\begin{pgfscope}%
\pgfpathrectangle{\pgfqpoint{0.800000in}{0.528000in}}{\pgfqpoint{3.968000in}{3.696000in}}%
\pgfusepath{clip}%
\pgfsetbuttcap%
\pgfsetroundjoin%
\definecolor{currentfill}{rgb}{0.477504,0.821444,0.318195}%
\pgfsetfillcolor{currentfill}%
\pgfsetlinewidth{0.000000pt}%
\definecolor{currentstroke}{rgb}{0.000000,0.000000,0.000000}%
\pgfsetstrokecolor{currentstroke}%
\pgfsetdash{}{0pt}%
\pgfpathmoveto{\pgfqpoint{4.768000in}{3.751903in}}%
\pgfpathlineto{\pgfqpoint{4.755521in}{3.764377in}}%
\pgfpathlineto{\pgfqpoint{4.744099in}{3.776000in}}%
\pgfpathlineto{\pgfqpoint{4.727919in}{3.792302in}}%
\pgfpathlineto{\pgfqpoint{4.717011in}{3.803173in}}%
\pgfpathlineto{\pgfqpoint{4.706996in}{3.813333in}}%
\pgfpathlineto{\pgfqpoint{4.687838in}{3.832577in}}%
\pgfpathlineto{\pgfqpoint{4.669785in}{3.850667in}}%
\pgfpathlineto{\pgfqpoint{4.659119in}{3.861250in}}%
\pgfpathlineto{\pgfqpoint{4.647758in}{3.872725in}}%
\pgfpathlineto{\pgfqpoint{4.632467in}{3.888000in}}%
\pgfpathlineto{\pgfqpoint{4.620445in}{3.899893in}}%
\pgfpathlineto{\pgfqpoint{4.607677in}{3.912749in}}%
\pgfpathlineto{\pgfqpoint{4.595041in}{3.925333in}}%
\pgfpathlineto{\pgfqpoint{4.567596in}{3.952648in}}%
\pgfpathlineto{\pgfqpoint{4.557505in}{3.962667in}}%
\pgfpathlineto{\pgfqpoint{4.542915in}{3.977011in}}%
\pgfpathlineto{\pgfqpoint{4.527515in}{3.992423in}}%
\pgfpathlineto{\pgfqpoint{4.519860in}{4.000000in}}%
\pgfpathlineto{\pgfqpoint{4.487434in}{4.032073in}}%
\pgfpathlineto{\pgfqpoint{4.484680in}{4.034767in}}%
\pgfpathlineto{\pgfqpoint{4.482103in}{4.037333in}}%
\pgfpathlineto{\pgfqpoint{4.447354in}{4.071599in}}%
\pgfpathlineto{\pgfqpoint{4.444235in}{4.074667in}}%
\pgfpathlineto{\pgfqpoint{4.426167in}{4.092266in}}%
\pgfpathlineto{\pgfqpoint{4.407273in}{4.111002in}}%
\pgfpathlineto{\pgfqpoint{4.406255in}{4.112000in}}%
\pgfpathlineto{\pgfqpoint{4.387129in}{4.130571in}}%
\pgfpathlineto{\pgfqpoint{4.368150in}{4.149333in}}%
\pgfpathlineto{\pgfqpoint{4.367192in}{4.150271in}}%
\pgfpathlineto{\pgfqpoint{4.348031in}{4.168819in}}%
\pgfpathlineto{\pgfqpoint{4.329921in}{4.186667in}}%
\pgfpathlineto{\pgfqpoint{4.327111in}{4.189409in}}%
\pgfpathlineto{\pgfqpoint{4.308871in}{4.207011in}}%
\pgfpathlineto{\pgfqpoint{4.291579in}{4.224000in}}%
\pgfpathlineto{\pgfqpoint{4.288923in}{4.224000in}}%
\pgfpathlineto{\pgfqpoint{4.307508in}{4.205741in}}%
\pgfpathlineto{\pgfqpoint{4.327111in}{4.186823in}}%
\pgfpathlineto{\pgfqpoint{4.327272in}{4.186667in}}%
\pgfpathlineto{\pgfqpoint{4.346669in}{4.167550in}}%
\pgfpathlineto{\pgfqpoint{4.365488in}{4.149333in}}%
\pgfpathlineto{\pgfqpoint{4.366315in}{4.148517in}}%
\pgfpathlineto{\pgfqpoint{4.367192in}{4.147668in}}%
\pgfpathlineto{\pgfqpoint{4.385769in}{4.129303in}}%
\pgfpathlineto{\pgfqpoint{4.403588in}{4.112000in}}%
\pgfpathlineto{\pgfqpoint{4.407273in}{4.108387in}}%
\pgfpathlineto{\pgfqpoint{4.424807in}{4.090999in}}%
\pgfpathlineto{\pgfqpoint{4.441575in}{4.074667in}}%
\pgfpathlineto{\pgfqpoint{4.447354in}{4.068983in}}%
\pgfpathlineto{\pgfqpoint{4.479450in}{4.037333in}}%
\pgfpathlineto{\pgfqpoint{4.483309in}{4.033490in}}%
\pgfpathlineto{\pgfqpoint{4.487434in}{4.029455in}}%
\pgfpathlineto{\pgfqpoint{4.517213in}{4.000000in}}%
\pgfpathlineto{\pgfqpoint{4.527515in}{3.989803in}}%
\pgfpathlineto{\pgfqpoint{4.541559in}{3.975748in}}%
\pgfpathlineto{\pgfqpoint{4.554865in}{3.962667in}}%
\pgfpathlineto{\pgfqpoint{4.567596in}{3.950027in}}%
\pgfpathlineto{\pgfqpoint{4.592407in}{3.925333in}}%
\pgfpathlineto{\pgfqpoint{4.607677in}{3.910126in}}%
\pgfpathlineto{\pgfqpoint{4.619091in}{3.898632in}}%
\pgfpathlineto{\pgfqpoint{4.629840in}{3.888000in}}%
\pgfpathlineto{\pgfqpoint{4.647758in}{3.870100in}}%
\pgfpathlineto{\pgfqpoint{4.657767in}{3.859990in}}%
\pgfpathlineto{\pgfqpoint{4.667164in}{3.850667in}}%
\pgfpathlineto{\pgfqpoint{4.687838in}{3.829950in}}%
\pgfpathlineto{\pgfqpoint{4.704381in}{3.813333in}}%
\pgfpathlineto{\pgfqpoint{4.715648in}{3.801903in}}%
\pgfpathlineto{\pgfqpoint{4.727919in}{3.789674in}}%
\pgfpathlineto{\pgfqpoint{4.741491in}{3.776000in}}%
\pgfpathlineto{\pgfqpoint{4.754159in}{3.763108in}}%
\pgfpathlineto{\pgfqpoint{4.768000in}{3.749273in}}%
\pgfusepath{fill}%
\end{pgfscope}%
\begin{pgfscope}%
\pgfpathrectangle{\pgfqpoint{0.800000in}{0.528000in}}{\pgfqpoint{3.968000in}{3.696000in}}%
\pgfusepath{clip}%
\pgfsetbuttcap%
\pgfsetroundjoin%
\definecolor{currentfill}{rgb}{0.477504,0.821444,0.318195}%
\pgfsetfillcolor{currentfill}%
\pgfsetlinewidth{0.000000pt}%
\definecolor{currentstroke}{rgb}{0.000000,0.000000,0.000000}%
\pgfsetstrokecolor{currentstroke}%
\pgfsetdash{}{0pt}%
\pgfpathmoveto{\pgfqpoint{4.768000in}{3.754533in}}%
\pgfpathlineto{\pgfqpoint{4.756883in}{3.765645in}}%
\pgfpathlineto{\pgfqpoint{4.746708in}{3.776000in}}%
\pgfpathlineto{\pgfqpoint{4.727919in}{3.794931in}}%
\pgfpathlineto{\pgfqpoint{4.718374in}{3.804442in}}%
\pgfpathlineto{\pgfqpoint{4.709611in}{3.813333in}}%
\pgfpathlineto{\pgfqpoint{4.687838in}{3.835203in}}%
\pgfpathlineto{\pgfqpoint{4.672407in}{3.850667in}}%
\pgfpathlineto{\pgfqpoint{4.660471in}{3.862509in}}%
\pgfpathlineto{\pgfqpoint{4.647758in}{3.875350in}}%
\pgfpathlineto{\pgfqpoint{4.635095in}{3.888000in}}%
\pgfpathlineto{\pgfqpoint{4.621798in}{3.901153in}}%
\pgfpathlineto{\pgfqpoint{4.607677in}{3.915372in}}%
\pgfpathlineto{\pgfqpoint{4.597675in}{3.925333in}}%
\pgfpathlineto{\pgfqpoint{4.567596in}{3.955270in}}%
\pgfpathlineto{\pgfqpoint{4.560146in}{3.962667in}}%
\pgfpathlineto{\pgfqpoint{4.544271in}{3.978273in}}%
\pgfpathlineto{\pgfqpoint{4.527515in}{3.995042in}}%
\pgfpathlineto{\pgfqpoint{4.522506in}{4.000000in}}%
\pgfpathlineto{\pgfqpoint{4.487434in}{4.034691in}}%
\pgfpathlineto{\pgfqpoint{4.486051in}{4.036044in}}%
\pgfpathlineto{\pgfqpoint{4.484756in}{4.037333in}}%
\pgfpathlineto{\pgfqpoint{4.447354in}{4.074215in}}%
\pgfpathlineto{\pgfqpoint{4.446895in}{4.074667in}}%
\pgfpathlineto{\pgfqpoint{4.427527in}{4.093532in}}%
\pgfpathlineto{\pgfqpoint{4.408903in}{4.112000in}}%
\pgfpathlineto{\pgfqpoint{4.407273in}{4.113600in}}%
\pgfpathlineto{\pgfqpoint{4.388490in}{4.131838in}}%
\pgfpathlineto{\pgfqpoint{4.370793in}{4.149333in}}%
\pgfpathlineto{\pgfqpoint{4.367192in}{4.152859in}}%
\pgfpathlineto{\pgfqpoint{4.349393in}{4.170088in}}%
\pgfpathlineto{\pgfqpoint{4.332571in}{4.186667in}}%
\pgfpathlineto{\pgfqpoint{4.327111in}{4.191995in}}%
\pgfpathlineto{\pgfqpoint{4.310235in}{4.208280in}}%
\pgfpathlineto{\pgfqpoint{4.294235in}{4.224000in}}%
\pgfpathlineto{\pgfqpoint{4.291579in}{4.224000in}}%
\pgfpathlineto{\pgfqpoint{4.308871in}{4.207011in}}%
\pgfpathlineto{\pgfqpoint{4.327111in}{4.189409in}}%
\pgfpathlineto{\pgfqpoint{4.329921in}{4.186667in}}%
\pgfpathlineto{\pgfqpoint{4.348031in}{4.168819in}}%
\pgfpathlineto{\pgfqpoint{4.367192in}{4.150271in}}%
\pgfpathlineto{\pgfqpoint{4.368150in}{4.149333in}}%
\pgfpathlineto{\pgfqpoint{4.387129in}{4.130571in}}%
\pgfpathlineto{\pgfqpoint{4.406255in}{4.112000in}}%
\pgfpathlineto{\pgfqpoint{4.407273in}{4.111002in}}%
\pgfpathlineto{\pgfqpoint{4.426167in}{4.092266in}}%
\pgfpathlineto{\pgfqpoint{4.444235in}{4.074667in}}%
\pgfpathlineto{\pgfqpoint{4.447354in}{4.071599in}}%
\pgfpathlineto{\pgfqpoint{4.482103in}{4.037333in}}%
\pgfpathlineto{\pgfqpoint{4.484680in}{4.034767in}}%
\pgfpathlineto{\pgfqpoint{4.487434in}{4.032073in}}%
\pgfpathlineto{\pgfqpoint{4.519860in}{4.000000in}}%
\pgfpathlineto{\pgfqpoint{4.527515in}{3.992423in}}%
\pgfpathlineto{\pgfqpoint{4.542915in}{3.977011in}}%
\pgfpathlineto{\pgfqpoint{4.557505in}{3.962667in}}%
\pgfpathlineto{\pgfqpoint{4.567596in}{3.952648in}}%
\pgfpathlineto{\pgfqpoint{4.595041in}{3.925333in}}%
\pgfpathlineto{\pgfqpoint{4.607677in}{3.912749in}}%
\pgfpathlineto{\pgfqpoint{4.620445in}{3.899893in}}%
\pgfpathlineto{\pgfqpoint{4.632467in}{3.888000in}}%
\pgfpathlineto{\pgfqpoint{4.647758in}{3.872725in}}%
\pgfpathlineto{\pgfqpoint{4.659119in}{3.861250in}}%
\pgfpathlineto{\pgfqpoint{4.669785in}{3.850667in}}%
\pgfpathlineto{\pgfqpoint{4.687838in}{3.832577in}}%
\pgfpathlineto{\pgfqpoint{4.706996in}{3.813333in}}%
\pgfpathlineto{\pgfqpoint{4.717011in}{3.803173in}}%
\pgfpathlineto{\pgfqpoint{4.727919in}{3.792302in}}%
\pgfpathlineto{\pgfqpoint{4.744099in}{3.776000in}}%
\pgfpathlineto{\pgfqpoint{4.755521in}{3.764377in}}%
\pgfpathlineto{\pgfqpoint{4.768000in}{3.751903in}}%
\pgfusepath{fill}%
\end{pgfscope}%
\begin{pgfscope}%
\pgfpathrectangle{\pgfqpoint{0.800000in}{0.528000in}}{\pgfqpoint{3.968000in}{3.696000in}}%
\pgfusepath{clip}%
\pgfsetbuttcap%
\pgfsetroundjoin%
\definecolor{currentfill}{rgb}{0.487026,0.823929,0.312321}%
\pgfsetfillcolor{currentfill}%
\pgfsetlinewidth{0.000000pt}%
\definecolor{currentstroke}{rgb}{0.000000,0.000000,0.000000}%
\pgfsetstrokecolor{currentstroke}%
\pgfsetdash{}{0pt}%
\pgfpathmoveto{\pgfqpoint{4.768000in}{3.757163in}}%
\pgfpathlineto{\pgfqpoint{4.758245in}{3.766914in}}%
\pgfpathlineto{\pgfqpoint{4.749317in}{3.776000in}}%
\pgfpathlineto{\pgfqpoint{4.727919in}{3.797559in}}%
\pgfpathlineto{\pgfqpoint{4.719737in}{3.805712in}}%
\pgfpathlineto{\pgfqpoint{4.712226in}{3.813333in}}%
\pgfpathlineto{\pgfqpoint{4.687838in}{3.837830in}}%
\pgfpathlineto{\pgfqpoint{4.675028in}{3.850667in}}%
\pgfpathlineto{\pgfqpoint{4.661823in}{3.863768in}}%
\pgfpathlineto{\pgfqpoint{4.647758in}{3.877975in}}%
\pgfpathlineto{\pgfqpoint{4.637722in}{3.888000in}}%
\pgfpathlineto{\pgfqpoint{4.623151in}{3.902414in}}%
\pgfpathlineto{\pgfqpoint{4.607677in}{3.917995in}}%
\pgfpathlineto{\pgfqpoint{4.600309in}{3.925333in}}%
\pgfpathlineto{\pgfqpoint{4.567596in}{3.957891in}}%
\pgfpathlineto{\pgfqpoint{4.562786in}{3.962667in}}%
\pgfpathlineto{\pgfqpoint{4.545626in}{3.979536in}}%
\pgfpathlineto{\pgfqpoint{4.527515in}{3.997662in}}%
\pgfpathlineto{\pgfqpoint{4.525153in}{4.000000in}}%
\pgfpathlineto{\pgfqpoint{4.487434in}{4.037309in}}%
\pgfpathlineto{\pgfqpoint{4.487422in}{4.037321in}}%
\pgfpathlineto{\pgfqpoint{4.487410in}{4.037333in}}%
\pgfpathlineto{\pgfqpoint{4.486987in}{4.037750in}}%
\pgfpathlineto{\pgfqpoint{4.449530in}{4.074667in}}%
\pgfpathlineto{\pgfqpoint{4.448467in}{4.075704in}}%
\pgfpathlineto{\pgfqpoint{4.447354in}{4.076810in}}%
\pgfpathlineto{\pgfqpoint{4.428886in}{4.094798in}}%
\pgfpathlineto{\pgfqpoint{4.411539in}{4.112000in}}%
\pgfpathlineto{\pgfqpoint{4.407273in}{4.116189in}}%
\pgfpathlineto{\pgfqpoint{4.389851in}{4.133106in}}%
\pgfpathlineto{\pgfqpoint{4.373436in}{4.149333in}}%
\pgfpathlineto{\pgfqpoint{4.367192in}{4.155446in}}%
\pgfpathlineto{\pgfqpoint{4.350755in}{4.171357in}}%
\pgfpathlineto{\pgfqpoint{4.335220in}{4.186667in}}%
\pgfpathlineto{\pgfqpoint{4.327111in}{4.194580in}}%
\pgfpathlineto{\pgfqpoint{4.311598in}{4.209550in}}%
\pgfpathlineto{\pgfqpoint{4.296891in}{4.224000in}}%
\pgfpathlineto{\pgfqpoint{4.294235in}{4.224000in}}%
\pgfpathlineto{\pgfqpoint{4.310235in}{4.208280in}}%
\pgfpathlineto{\pgfqpoint{4.327111in}{4.191995in}}%
\pgfpathlineto{\pgfqpoint{4.332571in}{4.186667in}}%
\pgfpathlineto{\pgfqpoint{4.349393in}{4.170088in}}%
\pgfpathlineto{\pgfqpoint{4.367192in}{4.152859in}}%
\pgfpathlineto{\pgfqpoint{4.370793in}{4.149333in}}%
\pgfpathlineto{\pgfqpoint{4.388490in}{4.131838in}}%
\pgfpathlineto{\pgfqpoint{4.407273in}{4.113600in}}%
\pgfpathlineto{\pgfqpoint{4.408903in}{4.112000in}}%
\pgfpathlineto{\pgfqpoint{4.427527in}{4.093532in}}%
\pgfpathlineto{\pgfqpoint{4.446895in}{4.074667in}}%
\pgfpathlineto{\pgfqpoint{4.447354in}{4.074215in}}%
\pgfpathlineto{\pgfqpoint{4.484756in}{4.037333in}}%
\pgfpathlineto{\pgfqpoint{4.486051in}{4.036044in}}%
\pgfpathlineto{\pgfqpoint{4.487434in}{4.034691in}}%
\pgfpathlineto{\pgfqpoint{4.522506in}{4.000000in}}%
\pgfpathlineto{\pgfqpoint{4.527515in}{3.995042in}}%
\pgfpathlineto{\pgfqpoint{4.544271in}{3.978273in}}%
\pgfpathlineto{\pgfqpoint{4.560146in}{3.962667in}}%
\pgfpathlineto{\pgfqpoint{4.567596in}{3.955270in}}%
\pgfpathlineto{\pgfqpoint{4.597675in}{3.925333in}}%
\pgfpathlineto{\pgfqpoint{4.607677in}{3.915372in}}%
\pgfpathlineto{\pgfqpoint{4.621798in}{3.901153in}}%
\pgfpathlineto{\pgfqpoint{4.635095in}{3.888000in}}%
\pgfpathlineto{\pgfqpoint{4.647758in}{3.875350in}}%
\pgfpathlineto{\pgfqpoint{4.660471in}{3.862509in}}%
\pgfpathlineto{\pgfqpoint{4.672407in}{3.850667in}}%
\pgfpathlineto{\pgfqpoint{4.687838in}{3.835203in}}%
\pgfpathlineto{\pgfqpoint{4.709611in}{3.813333in}}%
\pgfpathlineto{\pgfqpoint{4.718374in}{3.804442in}}%
\pgfpathlineto{\pgfqpoint{4.727919in}{3.794931in}}%
\pgfpathlineto{\pgfqpoint{4.746708in}{3.776000in}}%
\pgfpathlineto{\pgfqpoint{4.756883in}{3.765645in}}%
\pgfpathlineto{\pgfqpoint{4.768000in}{3.754533in}}%
\pgfusepath{fill}%
\end{pgfscope}%
\begin{pgfscope}%
\pgfpathrectangle{\pgfqpoint{0.800000in}{0.528000in}}{\pgfqpoint{3.968000in}{3.696000in}}%
\pgfusepath{clip}%
\pgfsetbuttcap%
\pgfsetroundjoin%
\definecolor{currentfill}{rgb}{0.487026,0.823929,0.312321}%
\pgfsetfillcolor{currentfill}%
\pgfsetlinewidth{0.000000pt}%
\definecolor{currentstroke}{rgb}{0.000000,0.000000,0.000000}%
\pgfsetstrokecolor{currentstroke}%
\pgfsetdash{}{0pt}%
\pgfpathmoveto{\pgfqpoint{4.768000in}{3.759793in}}%
\pgfpathlineto{\pgfqpoint{4.759607in}{3.768182in}}%
\pgfpathlineto{\pgfqpoint{4.751925in}{3.776000in}}%
\pgfpathlineto{\pgfqpoint{4.727919in}{3.800187in}}%
\pgfpathlineto{\pgfqpoint{4.721101in}{3.806982in}}%
\pgfpathlineto{\pgfqpoint{4.714840in}{3.813333in}}%
\pgfpathlineto{\pgfqpoint{4.687838in}{3.840456in}}%
\pgfpathlineto{\pgfqpoint{4.677649in}{3.850667in}}%
\pgfpathlineto{\pgfqpoint{4.663175in}{3.865028in}}%
\pgfpathlineto{\pgfqpoint{4.647758in}{3.880600in}}%
\pgfpathlineto{\pgfqpoint{4.640350in}{3.888000in}}%
\pgfpathlineto{\pgfqpoint{4.624504in}{3.903674in}}%
\pgfpathlineto{\pgfqpoint{4.607677in}{3.920619in}}%
\pgfpathlineto{\pgfqpoint{4.602943in}{3.925333in}}%
\pgfpathlineto{\pgfqpoint{4.567596in}{3.960512in}}%
\pgfpathlineto{\pgfqpoint{4.565426in}{3.962667in}}%
\pgfpathlineto{\pgfqpoint{4.546982in}{3.980799in}}%
\pgfpathlineto{\pgfqpoint{4.527797in}{4.000000in}}%
\pgfpathlineto{\pgfqpoint{4.527660in}{4.000135in}}%
\pgfpathlineto{\pgfqpoint{4.527515in}{4.000279in}}%
\pgfpathlineto{\pgfqpoint{4.490034in}{4.037333in}}%
\pgfpathlineto{\pgfqpoint{4.488766in}{4.038574in}}%
\pgfpathlineto{\pgfqpoint{4.487434in}{4.039901in}}%
\pgfpathlineto{\pgfqpoint{4.452160in}{4.074667in}}%
\pgfpathlineto{\pgfqpoint{4.449813in}{4.076958in}}%
\pgfpathlineto{\pgfqpoint{4.447354in}{4.079401in}}%
\pgfpathlineto{\pgfqpoint{4.430246in}{4.096065in}}%
\pgfpathlineto{\pgfqpoint{4.414176in}{4.112000in}}%
\pgfpathlineto{\pgfqpoint{4.407273in}{4.118778in}}%
\pgfpathlineto{\pgfqpoint{4.391212in}{4.134374in}}%
\pgfpathlineto{\pgfqpoint{4.376079in}{4.149333in}}%
\pgfpathlineto{\pgfqpoint{4.367192in}{4.158033in}}%
\pgfpathlineto{\pgfqpoint{4.352117in}{4.172625in}}%
\pgfpathlineto{\pgfqpoint{4.337870in}{4.186667in}}%
\pgfpathlineto{\pgfqpoint{4.327111in}{4.197166in}}%
\pgfpathlineto{\pgfqpoint{4.312962in}{4.210820in}}%
\pgfpathlineto{\pgfqpoint{4.299547in}{4.224000in}}%
\pgfpathlineto{\pgfqpoint{4.296891in}{4.224000in}}%
\pgfpathlineto{\pgfqpoint{4.311598in}{4.209550in}}%
\pgfpathlineto{\pgfqpoint{4.327111in}{4.194580in}}%
\pgfpathlineto{\pgfqpoint{4.335220in}{4.186667in}}%
\pgfpathlineto{\pgfqpoint{4.350755in}{4.171357in}}%
\pgfpathlineto{\pgfqpoint{4.367192in}{4.155446in}}%
\pgfpathlineto{\pgfqpoint{4.373436in}{4.149333in}}%
\pgfpathlineto{\pgfqpoint{4.389851in}{4.133106in}}%
\pgfpathlineto{\pgfqpoint{4.407273in}{4.116189in}}%
\pgfpathlineto{\pgfqpoint{4.411539in}{4.112000in}}%
\pgfpathlineto{\pgfqpoint{4.428886in}{4.094798in}}%
\pgfpathlineto{\pgfqpoint{4.447354in}{4.076810in}}%
\pgfpathlineto{\pgfqpoint{4.448467in}{4.075704in}}%
\pgfpathlineto{\pgfqpoint{4.449530in}{4.074667in}}%
\pgfpathlineto{\pgfqpoint{4.486987in}{4.037750in}}%
\pgfpathlineto{\pgfqpoint{4.487410in}{4.037333in}}%
\pgfpathlineto{\pgfqpoint{4.487422in}{4.037321in}}%
\pgfpathlineto{\pgfqpoint{4.487434in}{4.037309in}}%
\pgfpathlineto{\pgfqpoint{4.525153in}{4.000000in}}%
\pgfpathlineto{\pgfqpoint{4.527515in}{3.997662in}}%
\pgfpathlineto{\pgfqpoint{4.545626in}{3.979536in}}%
\pgfpathlineto{\pgfqpoint{4.562786in}{3.962667in}}%
\pgfpathlineto{\pgfqpoint{4.567596in}{3.957891in}}%
\pgfpathlineto{\pgfqpoint{4.600309in}{3.925333in}}%
\pgfpathlineto{\pgfqpoint{4.607677in}{3.917995in}}%
\pgfpathlineto{\pgfqpoint{4.623151in}{3.902414in}}%
\pgfpathlineto{\pgfqpoint{4.637722in}{3.888000in}}%
\pgfpathlineto{\pgfqpoint{4.647758in}{3.877975in}}%
\pgfpathlineto{\pgfqpoint{4.661823in}{3.863768in}}%
\pgfpathlineto{\pgfqpoint{4.675028in}{3.850667in}}%
\pgfpathlineto{\pgfqpoint{4.687838in}{3.837830in}}%
\pgfpathlineto{\pgfqpoint{4.712226in}{3.813333in}}%
\pgfpathlineto{\pgfqpoint{4.719737in}{3.805712in}}%
\pgfpathlineto{\pgfqpoint{4.727919in}{3.797559in}}%
\pgfpathlineto{\pgfqpoint{4.749317in}{3.776000in}}%
\pgfpathlineto{\pgfqpoint{4.758245in}{3.766914in}}%
\pgfpathlineto{\pgfqpoint{4.768000in}{3.757163in}}%
\pgfusepath{fill}%
\end{pgfscope}%
\begin{pgfscope}%
\pgfpathrectangle{\pgfqpoint{0.800000in}{0.528000in}}{\pgfqpoint{3.968000in}{3.696000in}}%
\pgfusepath{clip}%
\pgfsetbuttcap%
\pgfsetroundjoin%
\definecolor{currentfill}{rgb}{0.487026,0.823929,0.312321}%
\pgfsetfillcolor{currentfill}%
\pgfsetlinewidth{0.000000pt}%
\definecolor{currentstroke}{rgb}{0.000000,0.000000,0.000000}%
\pgfsetstrokecolor{currentstroke}%
\pgfsetdash{}{0pt}%
\pgfpathmoveto{\pgfqpoint{4.768000in}{3.762423in}}%
\pgfpathlineto{\pgfqpoint{4.760969in}{3.769451in}}%
\pgfpathlineto{\pgfqpoint{4.754534in}{3.776000in}}%
\pgfpathlineto{\pgfqpoint{4.727919in}{3.802816in}}%
\pgfpathlineto{\pgfqpoint{4.722464in}{3.808252in}}%
\pgfpathlineto{\pgfqpoint{4.717455in}{3.813333in}}%
\pgfpathlineto{\pgfqpoint{4.687838in}{3.843083in}}%
\pgfpathlineto{\pgfqpoint{4.680270in}{3.850667in}}%
\pgfpathlineto{\pgfqpoint{4.664527in}{3.866287in}}%
\pgfpathlineto{\pgfqpoint{4.647758in}{3.883225in}}%
\pgfpathlineto{\pgfqpoint{4.642977in}{3.888000in}}%
\pgfpathlineto{\pgfqpoint{4.625858in}{3.904935in}}%
\pgfpathlineto{\pgfqpoint{4.607677in}{3.923242in}}%
\pgfpathlineto{\pgfqpoint{4.605577in}{3.925333in}}%
\pgfpathlineto{\pgfqpoint{4.574917in}{3.955847in}}%
\pgfpathlineto{\pgfqpoint{4.568061in}{3.962667in}}%
\pgfpathlineto{\pgfqpoint{4.567835in}{3.962889in}}%
\pgfpathlineto{\pgfqpoint{4.567596in}{3.963129in}}%
\pgfpathlineto{\pgfqpoint{4.548338in}{3.982062in}}%
\pgfpathlineto{\pgfqpoint{4.530414in}{4.000000in}}%
\pgfpathlineto{\pgfqpoint{4.529003in}{4.001386in}}%
\pgfpathlineto{\pgfqpoint{4.527515in}{4.002873in}}%
\pgfpathlineto{\pgfqpoint{4.492658in}{4.037333in}}%
\pgfpathlineto{\pgfqpoint{4.490111in}{4.039827in}}%
\pgfpathlineto{\pgfqpoint{4.487434in}{4.042494in}}%
\pgfpathlineto{\pgfqpoint{4.454791in}{4.074667in}}%
\pgfpathlineto{\pgfqpoint{4.451159in}{4.078211in}}%
\pgfpathlineto{\pgfqpoint{4.447354in}{4.081992in}}%
\pgfpathlineto{\pgfqpoint{4.431605in}{4.097331in}}%
\pgfpathlineto{\pgfqpoint{4.416812in}{4.112000in}}%
\pgfpathlineto{\pgfqpoint{4.407273in}{4.121367in}}%
\pgfpathlineto{\pgfqpoint{4.392573in}{4.135641in}}%
\pgfpathlineto{\pgfqpoint{4.378722in}{4.149333in}}%
\pgfpathlineto{\pgfqpoint{4.367192in}{4.160620in}}%
\pgfpathlineto{\pgfqpoint{4.353480in}{4.173894in}}%
\pgfpathlineto{\pgfqpoint{4.340519in}{4.186667in}}%
\pgfpathlineto{\pgfqpoint{4.327111in}{4.199752in}}%
\pgfpathlineto{\pgfqpoint{4.314325in}{4.212090in}}%
\pgfpathlineto{\pgfqpoint{4.302203in}{4.224000in}}%
\pgfpathlineto{\pgfqpoint{4.299547in}{4.224000in}}%
\pgfpathlineto{\pgfqpoint{4.312962in}{4.210820in}}%
\pgfpathlineto{\pgfqpoint{4.327111in}{4.197166in}}%
\pgfpathlineto{\pgfqpoint{4.337870in}{4.186667in}}%
\pgfpathlineto{\pgfqpoint{4.352117in}{4.172625in}}%
\pgfpathlineto{\pgfqpoint{4.367192in}{4.158033in}}%
\pgfpathlineto{\pgfqpoint{4.376079in}{4.149333in}}%
\pgfpathlineto{\pgfqpoint{4.391212in}{4.134374in}}%
\pgfpathlineto{\pgfqpoint{4.407273in}{4.118778in}}%
\pgfpathlineto{\pgfqpoint{4.414176in}{4.112000in}}%
\pgfpathlineto{\pgfqpoint{4.430246in}{4.096065in}}%
\pgfpathlineto{\pgfqpoint{4.447354in}{4.079401in}}%
\pgfpathlineto{\pgfqpoint{4.449813in}{4.076958in}}%
\pgfpathlineto{\pgfqpoint{4.452160in}{4.074667in}}%
\pgfpathlineto{\pgfqpoint{4.487434in}{4.039901in}}%
\pgfpathlineto{\pgfqpoint{4.488766in}{4.038574in}}%
\pgfpathlineto{\pgfqpoint{4.490034in}{4.037333in}}%
\pgfpathlineto{\pgfqpoint{4.527515in}{4.000279in}}%
\pgfpathlineto{\pgfqpoint{4.527660in}{4.000135in}}%
\pgfpathlineto{\pgfqpoint{4.527797in}{4.000000in}}%
\pgfpathlineto{\pgfqpoint{4.546982in}{3.980799in}}%
\pgfpathlineto{\pgfqpoint{4.565426in}{3.962667in}}%
\pgfpathlineto{\pgfqpoint{4.567596in}{3.960512in}}%
\pgfpathlineto{\pgfqpoint{4.602943in}{3.925333in}}%
\pgfpathlineto{\pgfqpoint{4.607677in}{3.920619in}}%
\pgfpathlineto{\pgfqpoint{4.624504in}{3.903674in}}%
\pgfpathlineto{\pgfqpoint{4.640350in}{3.888000in}}%
\pgfpathlineto{\pgfqpoint{4.647758in}{3.880600in}}%
\pgfpathlineto{\pgfqpoint{4.663175in}{3.865028in}}%
\pgfpathlineto{\pgfqpoint{4.677649in}{3.850667in}}%
\pgfpathlineto{\pgfqpoint{4.687838in}{3.840456in}}%
\pgfpathlineto{\pgfqpoint{4.714840in}{3.813333in}}%
\pgfpathlineto{\pgfqpoint{4.721101in}{3.806982in}}%
\pgfpathlineto{\pgfqpoint{4.727919in}{3.800187in}}%
\pgfpathlineto{\pgfqpoint{4.751925in}{3.776000in}}%
\pgfpathlineto{\pgfqpoint{4.759607in}{3.768182in}}%
\pgfpathlineto{\pgfqpoint{4.768000in}{3.759793in}}%
\pgfusepath{fill}%
\end{pgfscope}%
\begin{pgfscope}%
\pgfpathrectangle{\pgfqpoint{0.800000in}{0.528000in}}{\pgfqpoint{3.968000in}{3.696000in}}%
\pgfusepath{clip}%
\pgfsetbuttcap%
\pgfsetroundjoin%
\definecolor{currentfill}{rgb}{0.487026,0.823929,0.312321}%
\pgfsetfillcolor{currentfill}%
\pgfsetlinewidth{0.000000pt}%
\definecolor{currentstroke}{rgb}{0.000000,0.000000,0.000000}%
\pgfsetstrokecolor{currentstroke}%
\pgfsetdash{}{0pt}%
\pgfpathmoveto{\pgfqpoint{4.768000in}{3.765053in}}%
\pgfpathlineto{\pgfqpoint{4.762331in}{3.770720in}}%
\pgfpathlineto{\pgfqpoint{4.757142in}{3.776000in}}%
\pgfpathlineto{\pgfqpoint{4.727919in}{3.805444in}}%
\pgfpathlineto{\pgfqpoint{4.723827in}{3.809522in}}%
\pgfpathlineto{\pgfqpoint{4.720070in}{3.813333in}}%
\pgfpathlineto{\pgfqpoint{4.687838in}{3.845709in}}%
\pgfpathlineto{\pgfqpoint{4.682891in}{3.850667in}}%
\pgfpathlineto{\pgfqpoint{4.665879in}{3.867546in}}%
\pgfpathlineto{\pgfqpoint{4.647758in}{3.885850in}}%
\pgfpathlineto{\pgfqpoint{4.645605in}{3.888000in}}%
\pgfpathlineto{\pgfqpoint{4.627211in}{3.906195in}}%
\pgfpathlineto{\pgfqpoint{4.608205in}{3.925333in}}%
\pgfpathlineto{\pgfqpoint{4.607948in}{3.925586in}}%
\pgfpathlineto{\pgfqpoint{4.607677in}{3.925860in}}%
\pgfpathlineto{\pgfqpoint{4.570673in}{3.962667in}}%
\pgfpathlineto{\pgfqpoint{4.569177in}{3.964140in}}%
\pgfpathlineto{\pgfqpoint{4.567596in}{3.965725in}}%
\pgfpathlineto{\pgfqpoint{4.549694in}{3.983325in}}%
\pgfpathlineto{\pgfqpoint{4.533032in}{4.000000in}}%
\pgfpathlineto{\pgfqpoint{4.530346in}{4.002637in}}%
\pgfpathlineto{\pgfqpoint{4.527515in}{4.005467in}}%
\pgfpathlineto{\pgfqpoint{4.495282in}{4.037333in}}%
\pgfpathlineto{\pgfqpoint{4.491456in}{4.041079in}}%
\pgfpathlineto{\pgfqpoint{4.487434in}{4.045086in}}%
\pgfpathlineto{\pgfqpoint{4.457421in}{4.074667in}}%
\pgfpathlineto{\pgfqpoint{4.452505in}{4.079465in}}%
\pgfpathlineto{\pgfqpoint{4.447354in}{4.084582in}}%
\pgfpathlineto{\pgfqpoint{4.432965in}{4.098598in}}%
\pgfpathlineto{\pgfqpoint{4.419449in}{4.112000in}}%
\pgfpathlineto{\pgfqpoint{4.407273in}{4.123956in}}%
\pgfpathlineto{\pgfqpoint{4.393934in}{4.136909in}}%
\pgfpathlineto{\pgfqpoint{4.381365in}{4.149333in}}%
\pgfpathlineto{\pgfqpoint{4.367192in}{4.163208in}}%
\pgfpathlineto{\pgfqpoint{4.354842in}{4.175163in}}%
\pgfpathlineto{\pgfqpoint{4.343169in}{4.186667in}}%
\pgfpathlineto{\pgfqpoint{4.327111in}{4.202337in}}%
\pgfpathlineto{\pgfqpoint{4.315688in}{4.213360in}}%
\pgfpathlineto{\pgfqpoint{4.304859in}{4.224000in}}%
\pgfpathlineto{\pgfqpoint{4.302203in}{4.224000in}}%
\pgfpathlineto{\pgfqpoint{4.314325in}{4.212090in}}%
\pgfpathlineto{\pgfqpoint{4.327111in}{4.199752in}}%
\pgfpathlineto{\pgfqpoint{4.340519in}{4.186667in}}%
\pgfpathlineto{\pgfqpoint{4.353480in}{4.173894in}}%
\pgfpathlineto{\pgfqpoint{4.367192in}{4.160620in}}%
\pgfpathlineto{\pgfqpoint{4.378722in}{4.149333in}}%
\pgfpathlineto{\pgfqpoint{4.392573in}{4.135641in}}%
\pgfpathlineto{\pgfqpoint{4.407273in}{4.121367in}}%
\pgfpathlineto{\pgfqpoint{4.416812in}{4.112000in}}%
\pgfpathlineto{\pgfqpoint{4.431605in}{4.097331in}}%
\pgfpathlineto{\pgfqpoint{4.447354in}{4.081992in}}%
\pgfpathlineto{\pgfqpoint{4.451159in}{4.078211in}}%
\pgfpathlineto{\pgfqpoint{4.454791in}{4.074667in}}%
\pgfpathlineto{\pgfqpoint{4.487434in}{4.042494in}}%
\pgfpathlineto{\pgfqpoint{4.490111in}{4.039827in}}%
\pgfpathlineto{\pgfqpoint{4.492658in}{4.037333in}}%
\pgfpathlineto{\pgfqpoint{4.527515in}{4.002873in}}%
\pgfpathlineto{\pgfqpoint{4.529003in}{4.001386in}}%
\pgfpathlineto{\pgfqpoint{4.530414in}{4.000000in}}%
\pgfpathlineto{\pgfqpoint{4.548338in}{3.982062in}}%
\pgfpathlineto{\pgfqpoint{4.567596in}{3.963129in}}%
\pgfpathlineto{\pgfqpoint{4.567835in}{3.962889in}}%
\pgfpathlineto{\pgfqpoint{4.568061in}{3.962667in}}%
\pgfpathlineto{\pgfqpoint{4.574917in}{3.955847in}}%
\pgfpathlineto{\pgfqpoint{4.605577in}{3.925333in}}%
\pgfpathlineto{\pgfqpoint{4.607677in}{3.923242in}}%
\pgfpathlineto{\pgfqpoint{4.625858in}{3.904935in}}%
\pgfpathlineto{\pgfqpoint{4.642977in}{3.888000in}}%
\pgfpathlineto{\pgfqpoint{4.647758in}{3.883225in}}%
\pgfpathlineto{\pgfqpoint{4.664527in}{3.866287in}}%
\pgfpathlineto{\pgfqpoint{4.680270in}{3.850667in}}%
\pgfpathlineto{\pgfqpoint{4.687838in}{3.843083in}}%
\pgfpathlineto{\pgfqpoint{4.717455in}{3.813333in}}%
\pgfpathlineto{\pgfqpoint{4.722464in}{3.808252in}}%
\pgfpathlineto{\pgfqpoint{4.727919in}{3.802816in}}%
\pgfpathlineto{\pgfqpoint{4.754534in}{3.776000in}}%
\pgfpathlineto{\pgfqpoint{4.760969in}{3.769451in}}%
\pgfpathlineto{\pgfqpoint{4.768000in}{3.762423in}}%
\pgfusepath{fill}%
\end{pgfscope}%
\begin{pgfscope}%
\pgfpathrectangle{\pgfqpoint{0.800000in}{0.528000in}}{\pgfqpoint{3.968000in}{3.696000in}}%
\pgfusepath{clip}%
\pgfsetbuttcap%
\pgfsetroundjoin%
\definecolor{currentfill}{rgb}{0.496615,0.826376,0.306377}%
\pgfsetfillcolor{currentfill}%
\pgfsetlinewidth{0.000000pt}%
\definecolor{currentstroke}{rgb}{0.000000,0.000000,0.000000}%
\pgfsetstrokecolor{currentstroke}%
\pgfsetdash{}{0pt}%
\pgfpathmoveto{\pgfqpoint{4.768000in}{3.767683in}}%
\pgfpathlineto{\pgfqpoint{4.763693in}{3.771988in}}%
\pgfpathlineto{\pgfqpoint{4.759751in}{3.776000in}}%
\pgfpathlineto{\pgfqpoint{4.727919in}{3.808072in}}%
\pgfpathlineto{\pgfqpoint{4.725190in}{3.810792in}}%
\pgfpathlineto{\pgfqpoint{4.722685in}{3.813333in}}%
\pgfpathlineto{\pgfqpoint{4.687838in}{3.848336in}}%
\pgfpathlineto{\pgfqpoint{4.685513in}{3.850667in}}%
\pgfpathlineto{\pgfqpoint{4.667231in}{3.868805in}}%
\pgfpathlineto{\pgfqpoint{4.648227in}{3.888000in}}%
\pgfpathlineto{\pgfqpoint{4.647758in}{3.888470in}}%
\pgfpathlineto{\pgfqpoint{4.628564in}{3.907455in}}%
\pgfpathlineto{\pgfqpoint{4.610810in}{3.925333in}}%
\pgfpathlineto{\pgfqpoint{4.609289in}{3.926835in}}%
\pgfpathlineto{\pgfqpoint{4.607677in}{3.928457in}}%
\pgfpathlineto{\pgfqpoint{4.573284in}{3.962667in}}%
\pgfpathlineto{\pgfqpoint{4.570519in}{3.965390in}}%
\pgfpathlineto{\pgfqpoint{4.567596in}{3.968321in}}%
\pgfpathlineto{\pgfqpoint{4.551049in}{3.984588in}}%
\pgfpathlineto{\pgfqpoint{4.535649in}{4.000000in}}%
\pgfpathlineto{\pgfqpoint{4.531690in}{4.003888in}}%
\pgfpathlineto{\pgfqpoint{4.527515in}{4.008061in}}%
\pgfpathlineto{\pgfqpoint{4.497906in}{4.037333in}}%
\pgfpathlineto{\pgfqpoint{4.492800in}{4.042331in}}%
\pgfpathlineto{\pgfqpoint{4.487434in}{4.047678in}}%
\pgfpathlineto{\pgfqpoint{4.460051in}{4.074667in}}%
\pgfpathlineto{\pgfqpoint{4.453851in}{4.080718in}}%
\pgfpathlineto{\pgfqpoint{4.447354in}{4.087173in}}%
\pgfpathlineto{\pgfqpoint{4.434324in}{4.099864in}}%
\pgfpathlineto{\pgfqpoint{4.422086in}{4.112000in}}%
\pgfpathlineto{\pgfqpoint{4.407273in}{4.126545in}}%
\pgfpathlineto{\pgfqpoint{4.395295in}{4.138176in}}%
\pgfpathlineto{\pgfqpoint{4.384008in}{4.149333in}}%
\pgfpathlineto{\pgfqpoint{4.367192in}{4.165795in}}%
\pgfpathlineto{\pgfqpoint{4.356204in}{4.176432in}}%
\pgfpathlineto{\pgfqpoint{4.345818in}{4.186667in}}%
\pgfpathlineto{\pgfqpoint{4.327111in}{4.204923in}}%
\pgfpathlineto{\pgfqpoint{4.317052in}{4.214630in}}%
\pgfpathlineto{\pgfqpoint{4.307515in}{4.224000in}}%
\pgfpathlineto{\pgfqpoint{4.304859in}{4.224000in}}%
\pgfpathlineto{\pgfqpoint{4.315688in}{4.213360in}}%
\pgfpathlineto{\pgfqpoint{4.327111in}{4.202337in}}%
\pgfpathlineto{\pgfqpoint{4.343169in}{4.186667in}}%
\pgfpathlineto{\pgfqpoint{4.354842in}{4.175163in}}%
\pgfpathlineto{\pgfqpoint{4.367192in}{4.163208in}}%
\pgfpathlineto{\pgfqpoint{4.381365in}{4.149333in}}%
\pgfpathlineto{\pgfqpoint{4.393934in}{4.136909in}}%
\pgfpathlineto{\pgfqpoint{4.407273in}{4.123956in}}%
\pgfpathlineto{\pgfqpoint{4.419449in}{4.112000in}}%
\pgfpathlineto{\pgfqpoint{4.432965in}{4.098598in}}%
\pgfpathlineto{\pgfqpoint{4.447354in}{4.084582in}}%
\pgfpathlineto{\pgfqpoint{4.452505in}{4.079465in}}%
\pgfpathlineto{\pgfqpoint{4.457421in}{4.074667in}}%
\pgfpathlineto{\pgfqpoint{4.487434in}{4.045086in}}%
\pgfpathlineto{\pgfqpoint{4.491456in}{4.041079in}}%
\pgfpathlineto{\pgfqpoint{4.495282in}{4.037333in}}%
\pgfpathlineto{\pgfqpoint{4.527515in}{4.005467in}}%
\pgfpathlineto{\pgfqpoint{4.530346in}{4.002637in}}%
\pgfpathlineto{\pgfqpoint{4.533032in}{4.000000in}}%
\pgfpathlineto{\pgfqpoint{4.549694in}{3.983325in}}%
\pgfpathlineto{\pgfqpoint{4.567596in}{3.965725in}}%
\pgfpathlineto{\pgfqpoint{4.569177in}{3.964140in}}%
\pgfpathlineto{\pgfqpoint{4.570673in}{3.962667in}}%
\pgfpathlineto{\pgfqpoint{4.607677in}{3.925860in}}%
\pgfpathlineto{\pgfqpoint{4.607948in}{3.925586in}}%
\pgfpathlineto{\pgfqpoint{4.608205in}{3.925333in}}%
\pgfpathlineto{\pgfqpoint{4.627211in}{3.906195in}}%
\pgfpathlineto{\pgfqpoint{4.645605in}{3.888000in}}%
\pgfpathlineto{\pgfqpoint{4.647758in}{3.885850in}}%
\pgfpathlineto{\pgfqpoint{4.665879in}{3.867546in}}%
\pgfpathlineto{\pgfqpoint{4.682891in}{3.850667in}}%
\pgfpathlineto{\pgfqpoint{4.687838in}{3.845709in}}%
\pgfpathlineto{\pgfqpoint{4.720070in}{3.813333in}}%
\pgfpathlineto{\pgfqpoint{4.723827in}{3.809522in}}%
\pgfpathlineto{\pgfqpoint{4.727919in}{3.805444in}}%
\pgfpathlineto{\pgfqpoint{4.757142in}{3.776000in}}%
\pgfpathlineto{\pgfqpoint{4.762331in}{3.770720in}}%
\pgfpathlineto{\pgfqpoint{4.768000in}{3.765053in}}%
\pgfusepath{fill}%
\end{pgfscope}%
\begin{pgfscope}%
\pgfpathrectangle{\pgfqpoint{0.800000in}{0.528000in}}{\pgfqpoint{3.968000in}{3.696000in}}%
\pgfusepath{clip}%
\pgfsetbuttcap%
\pgfsetroundjoin%
\definecolor{currentfill}{rgb}{0.496615,0.826376,0.306377}%
\pgfsetfillcolor{currentfill}%
\pgfsetlinewidth{0.000000pt}%
\definecolor{currentstroke}{rgb}{0.000000,0.000000,0.000000}%
\pgfsetstrokecolor{currentstroke}%
\pgfsetdash{}{0pt}%
\pgfpathmoveto{\pgfqpoint{4.768000in}{3.770313in}}%
\pgfpathlineto{\pgfqpoint{4.765055in}{3.773257in}}%
\pgfpathlineto{\pgfqpoint{4.762360in}{3.776000in}}%
\pgfpathlineto{\pgfqpoint{4.727919in}{3.810701in}}%
\pgfpathlineto{\pgfqpoint{4.726554in}{3.812061in}}%
\pgfpathlineto{\pgfqpoint{4.725300in}{3.813333in}}%
\pgfpathlineto{\pgfqpoint{4.691891in}{3.846892in}}%
\pgfpathlineto{\pgfqpoint{4.688131in}{3.850667in}}%
\pgfpathlineto{\pgfqpoint{4.687989in}{3.850807in}}%
\pgfpathlineto{\pgfqpoint{4.687838in}{3.850960in}}%
\pgfpathlineto{\pgfqpoint{4.668583in}{3.870065in}}%
\pgfpathlineto{\pgfqpoint{4.650826in}{3.888000in}}%
\pgfpathlineto{\pgfqpoint{4.647758in}{3.891069in}}%
\pgfpathlineto{\pgfqpoint{4.629917in}{3.908716in}}%
\pgfpathlineto{\pgfqpoint{4.613415in}{3.925333in}}%
\pgfpathlineto{\pgfqpoint{4.610630in}{3.928084in}}%
\pgfpathlineto{\pgfqpoint{4.607677in}{3.931054in}}%
\pgfpathlineto{\pgfqpoint{4.575895in}{3.962667in}}%
\pgfpathlineto{\pgfqpoint{4.571862in}{3.966640in}}%
\pgfpathlineto{\pgfqpoint{4.567596in}{3.970916in}}%
\pgfpathlineto{\pgfqpoint{4.552405in}{3.985851in}}%
\pgfpathlineto{\pgfqpoint{4.538267in}{4.000000in}}%
\pgfpathlineto{\pgfqpoint{4.533033in}{4.005140in}}%
\pgfpathlineto{\pgfqpoint{4.527515in}{4.010655in}}%
\pgfpathlineto{\pgfqpoint{4.500529in}{4.037333in}}%
\pgfpathlineto{\pgfqpoint{4.494145in}{4.043584in}}%
\pgfpathlineto{\pgfqpoint{4.487434in}{4.050271in}}%
\pgfpathlineto{\pgfqpoint{4.462681in}{4.074667in}}%
\pgfpathlineto{\pgfqpoint{4.455197in}{4.081972in}}%
\pgfpathlineto{\pgfqpoint{4.447354in}{4.089764in}}%
\pgfpathlineto{\pgfqpoint{4.435684in}{4.101130in}}%
\pgfpathlineto{\pgfqpoint{4.424722in}{4.112000in}}%
\pgfpathlineto{\pgfqpoint{4.407273in}{4.129134in}}%
\pgfpathlineto{\pgfqpoint{4.396655in}{4.139444in}}%
\pgfpathlineto{\pgfqpoint{4.386652in}{4.149333in}}%
\pgfpathlineto{\pgfqpoint{4.367192in}{4.168382in}}%
\pgfpathlineto{\pgfqpoint{4.357566in}{4.177700in}}%
\pgfpathlineto{\pgfqpoint{4.348468in}{4.186667in}}%
\pgfpathlineto{\pgfqpoint{4.327111in}{4.207508in}}%
\pgfpathlineto{\pgfqpoint{4.318415in}{4.215900in}}%
\pgfpathlineto{\pgfqpoint{4.310171in}{4.224000in}}%
\pgfpathlineto{\pgfqpoint{4.307515in}{4.224000in}}%
\pgfpathlineto{\pgfqpoint{4.317052in}{4.214630in}}%
\pgfpathlineto{\pgfqpoint{4.327111in}{4.204923in}}%
\pgfpathlineto{\pgfqpoint{4.345818in}{4.186667in}}%
\pgfpathlineto{\pgfqpoint{4.356204in}{4.176432in}}%
\pgfpathlineto{\pgfqpoint{4.367192in}{4.165795in}}%
\pgfpathlineto{\pgfqpoint{4.384008in}{4.149333in}}%
\pgfpathlineto{\pgfqpoint{4.395295in}{4.138176in}}%
\pgfpathlineto{\pgfqpoint{4.407273in}{4.126545in}}%
\pgfpathlineto{\pgfqpoint{4.422086in}{4.112000in}}%
\pgfpathlineto{\pgfqpoint{4.434324in}{4.099864in}}%
\pgfpathlineto{\pgfqpoint{4.447354in}{4.087173in}}%
\pgfpathlineto{\pgfqpoint{4.453851in}{4.080718in}}%
\pgfpathlineto{\pgfqpoint{4.460051in}{4.074667in}}%
\pgfpathlineto{\pgfqpoint{4.487434in}{4.047678in}}%
\pgfpathlineto{\pgfqpoint{4.492800in}{4.042331in}}%
\pgfpathlineto{\pgfqpoint{4.497906in}{4.037333in}}%
\pgfpathlineto{\pgfqpoint{4.527515in}{4.008061in}}%
\pgfpathlineto{\pgfqpoint{4.531690in}{4.003888in}}%
\pgfpathlineto{\pgfqpoint{4.535649in}{4.000000in}}%
\pgfpathlineto{\pgfqpoint{4.551049in}{3.984588in}}%
\pgfpathlineto{\pgfqpoint{4.567596in}{3.968321in}}%
\pgfpathlineto{\pgfqpoint{4.570519in}{3.965390in}}%
\pgfpathlineto{\pgfqpoint{4.573284in}{3.962667in}}%
\pgfpathlineto{\pgfqpoint{4.607677in}{3.928457in}}%
\pgfpathlineto{\pgfqpoint{4.609289in}{3.926835in}}%
\pgfpathlineto{\pgfqpoint{4.610810in}{3.925333in}}%
\pgfpathlineto{\pgfqpoint{4.628564in}{3.907455in}}%
\pgfpathlineto{\pgfqpoint{4.647758in}{3.888470in}}%
\pgfpathlineto{\pgfqpoint{4.648227in}{3.888000in}}%
\pgfpathlineto{\pgfqpoint{4.667231in}{3.868805in}}%
\pgfpathlineto{\pgfqpoint{4.685513in}{3.850667in}}%
\pgfpathlineto{\pgfqpoint{4.687838in}{3.848336in}}%
\pgfpathlineto{\pgfqpoint{4.722685in}{3.813333in}}%
\pgfpathlineto{\pgfqpoint{4.725190in}{3.810792in}}%
\pgfpathlineto{\pgfqpoint{4.727919in}{3.808072in}}%
\pgfpathlineto{\pgfqpoint{4.759751in}{3.776000in}}%
\pgfpathlineto{\pgfqpoint{4.763693in}{3.771988in}}%
\pgfpathlineto{\pgfqpoint{4.768000in}{3.767683in}}%
\pgfusepath{fill}%
\end{pgfscope}%
\begin{pgfscope}%
\pgfpathrectangle{\pgfqpoint{0.800000in}{0.528000in}}{\pgfqpoint{3.968000in}{3.696000in}}%
\pgfusepath{clip}%
\pgfsetbuttcap%
\pgfsetroundjoin%
\definecolor{currentfill}{rgb}{0.496615,0.826376,0.306377}%
\pgfsetfillcolor{currentfill}%
\pgfsetlinewidth{0.000000pt}%
\definecolor{currentstroke}{rgb}{0.000000,0.000000,0.000000}%
\pgfsetstrokecolor{currentstroke}%
\pgfsetdash{}{0pt}%
\pgfpathmoveto{\pgfqpoint{4.768000in}{3.772943in}}%
\pgfpathlineto{\pgfqpoint{4.766417in}{3.774526in}}%
\pgfpathlineto{\pgfqpoint{4.764968in}{3.776000in}}%
\pgfpathlineto{\pgfqpoint{4.727919in}{3.813329in}}%
\pgfpathlineto{\pgfqpoint{4.727917in}{3.813331in}}%
\pgfpathlineto{\pgfqpoint{4.727915in}{3.813333in}}%
\pgfpathlineto{\pgfqpoint{4.727859in}{3.813389in}}%
\pgfpathlineto{\pgfqpoint{4.690723in}{3.850667in}}%
\pgfpathlineto{\pgfqpoint{4.689328in}{3.852054in}}%
\pgfpathlineto{\pgfqpoint{4.687838in}{3.853560in}}%
\pgfpathlineto{\pgfqpoint{4.669935in}{3.871324in}}%
\pgfpathlineto{\pgfqpoint{4.653425in}{3.888000in}}%
\pgfpathlineto{\pgfqpoint{4.647758in}{3.893668in}}%
\pgfpathlineto{\pgfqpoint{4.631271in}{3.909976in}}%
\pgfpathlineto{\pgfqpoint{4.616020in}{3.925333in}}%
\pgfpathlineto{\pgfqpoint{4.611971in}{3.929333in}}%
\pgfpathlineto{\pgfqpoint{4.607677in}{3.933652in}}%
\pgfpathlineto{\pgfqpoint{4.578506in}{3.962667in}}%
\pgfpathlineto{\pgfqpoint{4.573204in}{3.967890in}}%
\pgfpathlineto{\pgfqpoint{4.567596in}{3.973512in}}%
\pgfpathlineto{\pgfqpoint{4.553761in}{3.987113in}}%
\pgfpathlineto{\pgfqpoint{4.540885in}{4.000000in}}%
\pgfpathlineto{\pgfqpoint{4.534377in}{4.006391in}}%
\pgfpathlineto{\pgfqpoint{4.527515in}{4.013249in}}%
\pgfpathlineto{\pgfqpoint{4.503153in}{4.037333in}}%
\pgfpathlineto{\pgfqpoint{4.495489in}{4.044836in}}%
\pgfpathlineto{\pgfqpoint{4.487434in}{4.052863in}}%
\pgfpathlineto{\pgfqpoint{4.465312in}{4.074667in}}%
\pgfpathlineto{\pgfqpoint{4.456542in}{4.083226in}}%
\pgfpathlineto{\pgfqpoint{4.447354in}{4.092354in}}%
\pgfpathlineto{\pgfqpoint{4.437044in}{4.102397in}}%
\pgfpathlineto{\pgfqpoint{4.427359in}{4.112000in}}%
\pgfpathlineto{\pgfqpoint{4.407273in}{4.131723in}}%
\pgfpathlineto{\pgfqpoint{4.398016in}{4.140711in}}%
\pgfpathlineto{\pgfqpoint{4.389295in}{4.149333in}}%
\pgfpathlineto{\pgfqpoint{4.367192in}{4.170970in}}%
\pgfpathlineto{\pgfqpoint{4.358928in}{4.178969in}}%
\pgfpathlineto{\pgfqpoint{4.351117in}{4.186667in}}%
\pgfpathlineto{\pgfqpoint{4.327111in}{4.210094in}}%
\pgfpathlineto{\pgfqpoint{4.319779in}{4.217170in}}%
\pgfpathlineto{\pgfqpoint{4.312827in}{4.224000in}}%
\pgfpathlineto{\pgfqpoint{4.310171in}{4.224000in}}%
\pgfpathlineto{\pgfqpoint{4.318415in}{4.215900in}}%
\pgfpathlineto{\pgfqpoint{4.327111in}{4.207508in}}%
\pgfpathlineto{\pgfqpoint{4.348468in}{4.186667in}}%
\pgfpathlineto{\pgfqpoint{4.357566in}{4.177700in}}%
\pgfpathlineto{\pgfqpoint{4.367192in}{4.168382in}}%
\pgfpathlineto{\pgfqpoint{4.386652in}{4.149333in}}%
\pgfpathlineto{\pgfqpoint{4.396655in}{4.139444in}}%
\pgfpathlineto{\pgfqpoint{4.407273in}{4.129134in}}%
\pgfpathlineto{\pgfqpoint{4.424722in}{4.112000in}}%
\pgfpathlineto{\pgfqpoint{4.435684in}{4.101130in}}%
\pgfpathlineto{\pgfqpoint{4.447354in}{4.089764in}}%
\pgfpathlineto{\pgfqpoint{4.455197in}{4.081972in}}%
\pgfpathlineto{\pgfqpoint{4.462681in}{4.074667in}}%
\pgfpathlineto{\pgfqpoint{4.487434in}{4.050271in}}%
\pgfpathlineto{\pgfqpoint{4.494145in}{4.043584in}}%
\pgfpathlineto{\pgfqpoint{4.500529in}{4.037333in}}%
\pgfpathlineto{\pgfqpoint{4.527515in}{4.010655in}}%
\pgfpathlineto{\pgfqpoint{4.533033in}{4.005140in}}%
\pgfpathlineto{\pgfqpoint{4.538267in}{4.000000in}}%
\pgfpathlineto{\pgfqpoint{4.552405in}{3.985851in}}%
\pgfpathlineto{\pgfqpoint{4.567596in}{3.970916in}}%
\pgfpathlineto{\pgfqpoint{4.571862in}{3.966640in}}%
\pgfpathlineto{\pgfqpoint{4.575895in}{3.962667in}}%
\pgfpathlineto{\pgfqpoint{4.607677in}{3.931054in}}%
\pgfpathlineto{\pgfqpoint{4.610630in}{3.928084in}}%
\pgfpathlineto{\pgfqpoint{4.613415in}{3.925333in}}%
\pgfpathlineto{\pgfqpoint{4.629917in}{3.908716in}}%
\pgfpathlineto{\pgfqpoint{4.647758in}{3.891069in}}%
\pgfpathlineto{\pgfqpoint{4.650826in}{3.888000in}}%
\pgfpathlineto{\pgfqpoint{4.668583in}{3.870065in}}%
\pgfpathlineto{\pgfqpoint{4.687838in}{3.850960in}}%
\pgfpathlineto{\pgfqpoint{4.687989in}{3.850807in}}%
\pgfpathlineto{\pgfqpoint{4.688131in}{3.850667in}}%
\pgfpathlineto{\pgfqpoint{4.691891in}{3.846892in}}%
\pgfpathlineto{\pgfqpoint{4.725300in}{3.813333in}}%
\pgfpathlineto{\pgfqpoint{4.726554in}{3.812061in}}%
\pgfpathlineto{\pgfqpoint{4.727919in}{3.810701in}}%
\pgfpathlineto{\pgfqpoint{4.762360in}{3.776000in}}%
\pgfpathlineto{\pgfqpoint{4.765055in}{3.773257in}}%
\pgfpathlineto{\pgfqpoint{4.768000in}{3.770313in}}%
\pgfusepath{fill}%
\end{pgfscope}%
\begin{pgfscope}%
\pgfpathrectangle{\pgfqpoint{0.800000in}{0.528000in}}{\pgfqpoint{3.968000in}{3.696000in}}%
\pgfusepath{clip}%
\pgfsetbuttcap%
\pgfsetroundjoin%
\definecolor{currentfill}{rgb}{0.496615,0.826376,0.306377}%
\pgfsetfillcolor{currentfill}%
\pgfsetlinewidth{0.000000pt}%
\definecolor{currentstroke}{rgb}{0.000000,0.000000,0.000000}%
\pgfsetstrokecolor{currentstroke}%
\pgfsetdash{}{0pt}%
\pgfpathmoveto{\pgfqpoint{4.768000in}{3.775573in}}%
\pgfpathlineto{\pgfqpoint{4.767779in}{3.775794in}}%
\pgfpathlineto{\pgfqpoint{4.767577in}{3.776000in}}%
\pgfpathlineto{\pgfqpoint{4.762397in}{3.781219in}}%
\pgfpathlineto{\pgfqpoint{4.730501in}{3.813333in}}%
\pgfpathlineto{\pgfqpoint{4.729254in}{3.814577in}}%
\pgfpathlineto{\pgfqpoint{4.727919in}{3.815931in}}%
\pgfpathlineto{\pgfqpoint{4.693316in}{3.850667in}}%
\pgfpathlineto{\pgfqpoint{4.690666in}{3.853301in}}%
\pgfpathlineto{\pgfqpoint{4.687838in}{3.856161in}}%
\pgfpathlineto{\pgfqpoint{4.671287in}{3.872583in}}%
\pgfpathlineto{\pgfqpoint{4.656024in}{3.888000in}}%
\pgfpathlineto{\pgfqpoint{4.647758in}{3.896267in}}%
\pgfpathlineto{\pgfqpoint{4.632624in}{3.911237in}}%
\pgfpathlineto{\pgfqpoint{4.618625in}{3.925333in}}%
\pgfpathlineto{\pgfqpoint{4.613312in}{3.930582in}}%
\pgfpathlineto{\pgfqpoint{4.607677in}{3.936249in}}%
\pgfpathlineto{\pgfqpoint{4.581118in}{3.962667in}}%
\pgfpathlineto{\pgfqpoint{4.574546in}{3.969140in}}%
\pgfpathlineto{\pgfqpoint{4.567596in}{3.976108in}}%
\pgfpathlineto{\pgfqpoint{4.555117in}{3.988376in}}%
\pgfpathlineto{\pgfqpoint{4.543502in}{4.000000in}}%
\pgfpathlineto{\pgfqpoint{4.535720in}{4.007642in}}%
\pgfpathlineto{\pgfqpoint{4.527515in}{4.015843in}}%
\pgfpathlineto{\pgfqpoint{4.505777in}{4.037333in}}%
\pgfpathlineto{\pgfqpoint{4.496834in}{4.046089in}}%
\pgfpathlineto{\pgfqpoint{4.487434in}{4.055455in}}%
\pgfpathlineto{\pgfqpoint{4.467942in}{4.074667in}}%
\pgfpathlineto{\pgfqpoint{4.457888in}{4.084479in}}%
\pgfpathlineto{\pgfqpoint{4.447354in}{4.094945in}}%
\pgfpathlineto{\pgfqpoint{4.438403in}{4.103663in}}%
\pgfpathlineto{\pgfqpoint{4.429996in}{4.112000in}}%
\pgfpathlineto{\pgfqpoint{4.407273in}{4.134312in}}%
\pgfpathlineto{\pgfqpoint{4.399377in}{4.141979in}}%
\pgfpathlineto{\pgfqpoint{4.391938in}{4.149333in}}%
\pgfpathlineto{\pgfqpoint{4.367192in}{4.173557in}}%
\pgfpathlineto{\pgfqpoint{4.360290in}{4.180238in}}%
\pgfpathlineto{\pgfqpoint{4.353767in}{4.186667in}}%
\pgfpathlineto{\pgfqpoint{4.327111in}{4.212680in}}%
\pgfpathlineto{\pgfqpoint{4.321142in}{4.218440in}}%
\pgfpathlineto{\pgfqpoint{4.315483in}{4.224000in}}%
\pgfpathlineto{\pgfqpoint{4.312827in}{4.224000in}}%
\pgfpathlineto{\pgfqpoint{4.319779in}{4.217170in}}%
\pgfpathlineto{\pgfqpoint{4.327111in}{4.210094in}}%
\pgfpathlineto{\pgfqpoint{4.351117in}{4.186667in}}%
\pgfpathlineto{\pgfqpoint{4.358928in}{4.178969in}}%
\pgfpathlineto{\pgfqpoint{4.367192in}{4.170970in}}%
\pgfpathlineto{\pgfqpoint{4.389295in}{4.149333in}}%
\pgfpathlineto{\pgfqpoint{4.398016in}{4.140711in}}%
\pgfpathlineto{\pgfqpoint{4.407273in}{4.131723in}}%
\pgfpathlineto{\pgfqpoint{4.427359in}{4.112000in}}%
\pgfpathlineto{\pgfqpoint{4.437044in}{4.102397in}}%
\pgfpathlineto{\pgfqpoint{4.447354in}{4.092354in}}%
\pgfpathlineto{\pgfqpoint{4.456542in}{4.083226in}}%
\pgfpathlineto{\pgfqpoint{4.465312in}{4.074667in}}%
\pgfpathlineto{\pgfqpoint{4.487434in}{4.052863in}}%
\pgfpathlineto{\pgfqpoint{4.495489in}{4.044836in}}%
\pgfpathlineto{\pgfqpoint{4.503153in}{4.037333in}}%
\pgfpathlineto{\pgfqpoint{4.527515in}{4.013249in}}%
\pgfpathlineto{\pgfqpoint{4.534377in}{4.006391in}}%
\pgfpathlineto{\pgfqpoint{4.540885in}{4.000000in}}%
\pgfpathlineto{\pgfqpoint{4.553761in}{3.987113in}}%
\pgfpathlineto{\pgfqpoint{4.567596in}{3.973512in}}%
\pgfpathlineto{\pgfqpoint{4.573204in}{3.967890in}}%
\pgfpathlineto{\pgfqpoint{4.578506in}{3.962667in}}%
\pgfpathlineto{\pgfqpoint{4.607677in}{3.933652in}}%
\pgfpathlineto{\pgfqpoint{4.611971in}{3.929333in}}%
\pgfpathlineto{\pgfqpoint{4.616020in}{3.925333in}}%
\pgfpathlineto{\pgfqpoint{4.631271in}{3.909976in}}%
\pgfpathlineto{\pgfqpoint{4.647758in}{3.893668in}}%
\pgfpathlineto{\pgfqpoint{4.653425in}{3.888000in}}%
\pgfpathlineto{\pgfqpoint{4.669935in}{3.871324in}}%
\pgfpathlineto{\pgfqpoint{4.687838in}{3.853560in}}%
\pgfpathlineto{\pgfqpoint{4.689328in}{3.852054in}}%
\pgfpathlineto{\pgfqpoint{4.690723in}{3.850667in}}%
\pgfpathlineto{\pgfqpoint{4.727859in}{3.813389in}}%
\pgfpathlineto{\pgfqpoint{4.727915in}{3.813333in}}%
\pgfpathlineto{\pgfqpoint{4.727917in}{3.813331in}}%
\pgfpathlineto{\pgfqpoint{4.727919in}{3.813329in}}%
\pgfpathlineto{\pgfqpoint{4.764968in}{3.776000in}}%
\pgfpathlineto{\pgfqpoint{4.766417in}{3.774526in}}%
\pgfpathlineto{\pgfqpoint{4.768000in}{3.772943in}}%
\pgfusepath{fill}%
\end{pgfscope}%
\begin{pgfscope}%
\pgfpathrectangle{\pgfqpoint{0.800000in}{0.528000in}}{\pgfqpoint{3.968000in}{3.696000in}}%
\pgfusepath{clip}%
\pgfsetbuttcap%
\pgfsetroundjoin%
\definecolor{currentfill}{rgb}{0.506271,0.828786,0.300362}%
\pgfsetfillcolor{currentfill}%
\pgfsetlinewidth{0.000000pt}%
\definecolor{currentstroke}{rgb}{0.000000,0.000000,0.000000}%
\pgfsetstrokecolor{currentstroke}%
\pgfsetdash{}{0pt}%
\pgfpathmoveto{\pgfqpoint{4.768000in}{3.778182in}}%
\pgfpathlineto{\pgfqpoint{4.733088in}{3.813333in}}%
\pgfpathlineto{\pgfqpoint{4.730591in}{3.815822in}}%
\pgfpathlineto{\pgfqpoint{4.727919in}{3.818534in}}%
\pgfpathlineto{\pgfqpoint{4.695908in}{3.850667in}}%
\pgfpathlineto{\pgfqpoint{4.692004in}{3.854547in}}%
\pgfpathlineto{\pgfqpoint{4.687838in}{3.858762in}}%
\pgfpathlineto{\pgfqpoint{4.672639in}{3.873843in}}%
\pgfpathlineto{\pgfqpoint{4.658623in}{3.888000in}}%
\pgfpathlineto{\pgfqpoint{4.647758in}{3.898866in}}%
\pgfpathlineto{\pgfqpoint{4.633977in}{3.912497in}}%
\pgfpathlineto{\pgfqpoint{4.621230in}{3.925333in}}%
\pgfpathlineto{\pgfqpoint{4.614653in}{3.931831in}}%
\pgfpathlineto{\pgfqpoint{4.607677in}{3.938847in}}%
\pgfpathlineto{\pgfqpoint{4.583729in}{3.962667in}}%
\pgfpathlineto{\pgfqpoint{4.575888in}{3.970390in}}%
\pgfpathlineto{\pgfqpoint{4.567596in}{3.978703in}}%
\pgfpathlineto{\pgfqpoint{4.556472in}{3.989639in}}%
\pgfpathlineto{\pgfqpoint{4.546120in}{4.000000in}}%
\pgfpathlineto{\pgfqpoint{4.537063in}{4.008894in}}%
\pgfpathlineto{\pgfqpoint{4.527515in}{4.018437in}}%
\pgfpathlineto{\pgfqpoint{4.508401in}{4.037333in}}%
\pgfpathlineto{\pgfqpoint{4.498179in}{4.047341in}}%
\pgfpathlineto{\pgfqpoint{4.487434in}{4.058048in}}%
\pgfpathlineto{\pgfqpoint{4.470572in}{4.074667in}}%
\pgfpathlineto{\pgfqpoint{4.459234in}{4.085733in}}%
\pgfpathlineto{\pgfqpoint{4.447354in}{4.097536in}}%
\pgfpathlineto{\pgfqpoint{4.439763in}{4.104930in}}%
\pgfpathlineto{\pgfqpoint{4.432632in}{4.112000in}}%
\pgfpathlineto{\pgfqpoint{4.407273in}{4.136901in}}%
\pgfpathlineto{\pgfqpoint{4.400738in}{4.143247in}}%
\pgfpathlineto{\pgfqpoint{4.394581in}{4.149333in}}%
\pgfpathlineto{\pgfqpoint{4.367192in}{4.176144in}}%
\pgfpathlineto{\pgfqpoint{4.361652in}{4.181507in}}%
\pgfpathlineto{\pgfqpoint{4.356416in}{4.186667in}}%
\pgfpathlineto{\pgfqpoint{4.327111in}{4.215265in}}%
\pgfpathlineto{\pgfqpoint{4.322505in}{4.219710in}}%
\pgfpathlineto{\pgfqpoint{4.318139in}{4.224000in}}%
\pgfpathlineto{\pgfqpoint{4.315483in}{4.224000in}}%
\pgfpathlineto{\pgfqpoint{4.321142in}{4.218440in}}%
\pgfpathlineto{\pgfqpoint{4.327111in}{4.212680in}}%
\pgfpathlineto{\pgfqpoint{4.353767in}{4.186667in}}%
\pgfpathlineto{\pgfqpoint{4.360290in}{4.180238in}}%
\pgfpathlineto{\pgfqpoint{4.367192in}{4.173557in}}%
\pgfpathlineto{\pgfqpoint{4.391938in}{4.149333in}}%
\pgfpathlineto{\pgfqpoint{4.399377in}{4.141979in}}%
\pgfpathlineto{\pgfqpoint{4.407273in}{4.134312in}}%
\pgfpathlineto{\pgfqpoint{4.429996in}{4.112000in}}%
\pgfpathlineto{\pgfqpoint{4.438403in}{4.103663in}}%
\pgfpathlineto{\pgfqpoint{4.447354in}{4.094945in}}%
\pgfpathlineto{\pgfqpoint{4.457888in}{4.084479in}}%
\pgfpathlineto{\pgfqpoint{4.467942in}{4.074667in}}%
\pgfpathlineto{\pgfqpoint{4.487434in}{4.055455in}}%
\pgfpathlineto{\pgfqpoint{4.496834in}{4.046089in}}%
\pgfpathlineto{\pgfqpoint{4.505777in}{4.037333in}}%
\pgfpathlineto{\pgfqpoint{4.527515in}{4.015843in}}%
\pgfpathlineto{\pgfqpoint{4.535720in}{4.007642in}}%
\pgfpathlineto{\pgfqpoint{4.543502in}{4.000000in}}%
\pgfpathlineto{\pgfqpoint{4.555117in}{3.988376in}}%
\pgfpathlineto{\pgfqpoint{4.567596in}{3.976108in}}%
\pgfpathlineto{\pgfqpoint{4.574546in}{3.969140in}}%
\pgfpathlineto{\pgfqpoint{4.581118in}{3.962667in}}%
\pgfpathlineto{\pgfqpoint{4.607677in}{3.936249in}}%
\pgfpathlineto{\pgfqpoint{4.613312in}{3.930582in}}%
\pgfpathlineto{\pgfqpoint{4.618625in}{3.925333in}}%
\pgfpathlineto{\pgfqpoint{4.632624in}{3.911237in}}%
\pgfpathlineto{\pgfqpoint{4.647758in}{3.896267in}}%
\pgfpathlineto{\pgfqpoint{4.656024in}{3.888000in}}%
\pgfpathlineto{\pgfqpoint{4.671287in}{3.872583in}}%
\pgfpathlineto{\pgfqpoint{4.687838in}{3.856161in}}%
\pgfpathlineto{\pgfqpoint{4.690666in}{3.853301in}}%
\pgfpathlineto{\pgfqpoint{4.693316in}{3.850667in}}%
\pgfpathlineto{\pgfqpoint{4.727919in}{3.815931in}}%
\pgfpathlineto{\pgfqpoint{4.729254in}{3.814577in}}%
\pgfpathlineto{\pgfqpoint{4.730501in}{3.813333in}}%
\pgfpathlineto{\pgfqpoint{4.762397in}{3.781219in}}%
\pgfpathlineto{\pgfqpoint{4.767577in}{3.776000in}}%
\pgfpathlineto{\pgfqpoint{4.767779in}{3.775794in}}%
\pgfpathlineto{\pgfqpoint{4.768000in}{3.775573in}}%
\pgfpathlineto{\pgfqpoint{4.768000in}{3.776000in}}%
\pgfusepath{fill}%
\end{pgfscope}%
\begin{pgfscope}%
\pgfpathrectangle{\pgfqpoint{0.800000in}{0.528000in}}{\pgfqpoint{3.968000in}{3.696000in}}%
\pgfusepath{clip}%
\pgfsetbuttcap%
\pgfsetroundjoin%
\definecolor{currentfill}{rgb}{0.506271,0.828786,0.300362}%
\pgfsetfillcolor{currentfill}%
\pgfsetlinewidth{0.000000pt}%
\definecolor{currentstroke}{rgb}{0.000000,0.000000,0.000000}%
\pgfsetstrokecolor{currentstroke}%
\pgfsetdash{}{0pt}%
\pgfpathmoveto{\pgfqpoint{4.768000in}{3.780786in}}%
\pgfpathlineto{\pgfqpoint{4.735674in}{3.813333in}}%
\pgfpathlineto{\pgfqpoint{4.731929in}{3.817068in}}%
\pgfpathlineto{\pgfqpoint{4.727919in}{3.821136in}}%
\pgfpathlineto{\pgfqpoint{4.698501in}{3.850667in}}%
\pgfpathlineto{\pgfqpoint{4.693343in}{3.855794in}}%
\pgfpathlineto{\pgfqpoint{4.687838in}{3.861363in}}%
\pgfpathlineto{\pgfqpoint{4.673991in}{3.875102in}}%
\pgfpathlineto{\pgfqpoint{4.661221in}{3.888000in}}%
\pgfpathlineto{\pgfqpoint{4.647758in}{3.901465in}}%
\pgfpathlineto{\pgfqpoint{4.635330in}{3.913758in}}%
\pgfpathlineto{\pgfqpoint{4.623835in}{3.925333in}}%
\pgfpathlineto{\pgfqpoint{4.615994in}{3.933080in}}%
\pgfpathlineto{\pgfqpoint{4.607677in}{3.941444in}}%
\pgfpathlineto{\pgfqpoint{4.586340in}{3.962667in}}%
\pgfpathlineto{\pgfqpoint{4.577230in}{3.971640in}}%
\pgfpathlineto{\pgfqpoint{4.567596in}{3.981299in}}%
\pgfpathlineto{\pgfqpoint{4.557828in}{3.990902in}}%
\pgfpathlineto{\pgfqpoint{4.548737in}{4.000000in}}%
\pgfpathlineto{\pgfqpoint{4.538407in}{4.010145in}}%
\pgfpathlineto{\pgfqpoint{4.527515in}{4.021031in}}%
\pgfpathlineto{\pgfqpoint{4.511025in}{4.037333in}}%
\pgfpathlineto{\pgfqpoint{4.499523in}{4.048594in}}%
\pgfpathlineto{\pgfqpoint{4.487434in}{4.060640in}}%
\pgfpathlineto{\pgfqpoint{4.473203in}{4.074667in}}%
\pgfpathlineto{\pgfqpoint{4.460580in}{4.086986in}}%
\pgfpathlineto{\pgfqpoint{4.447354in}{4.100126in}}%
\pgfpathlineto{\pgfqpoint{4.441122in}{4.106196in}}%
\pgfpathlineto{\pgfqpoint{4.435269in}{4.112000in}}%
\pgfpathlineto{\pgfqpoint{4.407273in}{4.139490in}}%
\pgfpathlineto{\pgfqpoint{4.402099in}{4.144514in}}%
\pgfpathlineto{\pgfqpoint{4.397224in}{4.149333in}}%
\pgfpathlineto{\pgfqpoint{4.367192in}{4.178732in}}%
\pgfpathlineto{\pgfqpoint{4.363014in}{4.182775in}}%
\pgfpathlineto{\pgfqpoint{4.359066in}{4.186667in}}%
\pgfpathlineto{\pgfqpoint{4.327111in}{4.217851in}}%
\pgfpathlineto{\pgfqpoint{4.323869in}{4.220980in}}%
\pgfpathlineto{\pgfqpoint{4.320795in}{4.224000in}}%
\pgfpathlineto{\pgfqpoint{4.318139in}{4.224000in}}%
\pgfpathlineto{\pgfqpoint{4.322505in}{4.219710in}}%
\pgfpathlineto{\pgfqpoint{4.327111in}{4.215265in}}%
\pgfpathlineto{\pgfqpoint{4.356416in}{4.186667in}}%
\pgfpathlineto{\pgfqpoint{4.361652in}{4.181507in}}%
\pgfpathlineto{\pgfqpoint{4.367192in}{4.176144in}}%
\pgfpathlineto{\pgfqpoint{4.394581in}{4.149333in}}%
\pgfpathlineto{\pgfqpoint{4.400738in}{4.143247in}}%
\pgfpathlineto{\pgfqpoint{4.407273in}{4.136901in}}%
\pgfpathlineto{\pgfqpoint{4.432632in}{4.112000in}}%
\pgfpathlineto{\pgfqpoint{4.439763in}{4.104930in}}%
\pgfpathlineto{\pgfqpoint{4.447354in}{4.097536in}}%
\pgfpathlineto{\pgfqpoint{4.459234in}{4.085733in}}%
\pgfpathlineto{\pgfqpoint{4.470572in}{4.074667in}}%
\pgfpathlineto{\pgfqpoint{4.487434in}{4.058048in}}%
\pgfpathlineto{\pgfqpoint{4.498179in}{4.047341in}}%
\pgfpathlineto{\pgfqpoint{4.508401in}{4.037333in}}%
\pgfpathlineto{\pgfqpoint{4.527515in}{4.018437in}}%
\pgfpathlineto{\pgfqpoint{4.537063in}{4.008894in}}%
\pgfpathlineto{\pgfqpoint{4.546120in}{4.000000in}}%
\pgfpathlineto{\pgfqpoint{4.556472in}{3.989639in}}%
\pgfpathlineto{\pgfqpoint{4.567596in}{3.978703in}}%
\pgfpathlineto{\pgfqpoint{4.575888in}{3.970390in}}%
\pgfpathlineto{\pgfqpoint{4.583729in}{3.962667in}}%
\pgfpathlineto{\pgfqpoint{4.607677in}{3.938847in}}%
\pgfpathlineto{\pgfqpoint{4.614653in}{3.931831in}}%
\pgfpathlineto{\pgfqpoint{4.621230in}{3.925333in}}%
\pgfpathlineto{\pgfqpoint{4.633977in}{3.912497in}}%
\pgfpathlineto{\pgfqpoint{4.647758in}{3.898866in}}%
\pgfpathlineto{\pgfqpoint{4.658623in}{3.888000in}}%
\pgfpathlineto{\pgfqpoint{4.672639in}{3.873843in}}%
\pgfpathlineto{\pgfqpoint{4.687838in}{3.858762in}}%
\pgfpathlineto{\pgfqpoint{4.692004in}{3.854547in}}%
\pgfpathlineto{\pgfqpoint{4.695908in}{3.850667in}}%
\pgfpathlineto{\pgfqpoint{4.727919in}{3.818534in}}%
\pgfpathlineto{\pgfqpoint{4.730591in}{3.815822in}}%
\pgfpathlineto{\pgfqpoint{4.733088in}{3.813333in}}%
\pgfpathlineto{\pgfqpoint{4.768000in}{3.778182in}}%
\pgfusepath{fill}%
\end{pgfscope}%
\begin{pgfscope}%
\pgfpathrectangle{\pgfqpoint{0.800000in}{0.528000in}}{\pgfqpoint{3.968000in}{3.696000in}}%
\pgfusepath{clip}%
\pgfsetbuttcap%
\pgfsetroundjoin%
\definecolor{currentfill}{rgb}{0.506271,0.828786,0.300362}%
\pgfsetfillcolor{currentfill}%
\pgfsetlinewidth{0.000000pt}%
\definecolor{currentstroke}{rgb}{0.000000,0.000000,0.000000}%
\pgfsetstrokecolor{currentstroke}%
\pgfsetdash{}{0pt}%
\pgfpathmoveto{\pgfqpoint{4.768000in}{3.783390in}}%
\pgfpathlineto{\pgfqpoint{4.738261in}{3.813333in}}%
\pgfpathlineto{\pgfqpoint{4.733266in}{3.818313in}}%
\pgfpathlineto{\pgfqpoint{4.727919in}{3.823739in}}%
\pgfpathlineto{\pgfqpoint{4.701093in}{3.850667in}}%
\pgfpathlineto{\pgfqpoint{4.694681in}{3.857041in}}%
\pgfpathlineto{\pgfqpoint{4.687838in}{3.863964in}}%
\pgfpathlineto{\pgfqpoint{4.675343in}{3.876361in}}%
\pgfpathlineto{\pgfqpoint{4.663820in}{3.888000in}}%
\pgfpathlineto{\pgfqpoint{4.647758in}{3.904064in}}%
\pgfpathlineto{\pgfqpoint{4.636684in}{3.915018in}}%
\pgfpathlineto{\pgfqpoint{4.626440in}{3.925333in}}%
\pgfpathlineto{\pgfqpoint{4.617335in}{3.934329in}}%
\pgfpathlineto{\pgfqpoint{4.607677in}{3.944041in}}%
\pgfpathlineto{\pgfqpoint{4.588952in}{3.962667in}}%
\pgfpathlineto{\pgfqpoint{4.578572in}{3.972891in}}%
\pgfpathlineto{\pgfqpoint{4.567596in}{3.983895in}}%
\pgfpathlineto{\pgfqpoint{4.559184in}{3.992165in}}%
\pgfpathlineto{\pgfqpoint{4.551355in}{4.000000in}}%
\pgfpathlineto{\pgfqpoint{4.539750in}{4.011396in}}%
\pgfpathlineto{\pgfqpoint{4.527515in}{4.023625in}}%
\pgfpathlineto{\pgfqpoint{4.513649in}{4.037333in}}%
\pgfpathlineto{\pgfqpoint{4.500868in}{4.049846in}}%
\pgfpathlineto{\pgfqpoint{4.487434in}{4.063233in}}%
\pgfpathlineto{\pgfqpoint{4.475833in}{4.074667in}}%
\pgfpathlineto{\pgfqpoint{4.461926in}{4.088240in}}%
\pgfpathlineto{\pgfqpoint{4.447354in}{4.102717in}}%
\pgfpathlineto{\pgfqpoint{4.442482in}{4.107462in}}%
\pgfpathlineto{\pgfqpoint{4.437906in}{4.112000in}}%
\pgfpathlineto{\pgfqpoint{4.407273in}{4.142079in}}%
\pgfpathlineto{\pgfqpoint{4.403460in}{4.145782in}}%
\pgfpathlineto{\pgfqpoint{4.399867in}{4.149333in}}%
\pgfpathlineto{\pgfqpoint{4.367192in}{4.181319in}}%
\pgfpathlineto{\pgfqpoint{4.364377in}{4.184044in}}%
\pgfpathlineto{\pgfqpoint{4.361716in}{4.186667in}}%
\pgfpathlineto{\pgfqpoint{4.327111in}{4.220437in}}%
\pgfpathlineto{\pgfqpoint{4.325232in}{4.222250in}}%
\pgfpathlineto{\pgfqpoint{4.323451in}{4.224000in}}%
\pgfpathlineto{\pgfqpoint{4.320795in}{4.224000in}}%
\pgfpathlineto{\pgfqpoint{4.323869in}{4.220980in}}%
\pgfpathlineto{\pgfqpoint{4.327111in}{4.217851in}}%
\pgfpathlineto{\pgfqpoint{4.359066in}{4.186667in}}%
\pgfpathlineto{\pgfqpoint{4.363014in}{4.182775in}}%
\pgfpathlineto{\pgfqpoint{4.367192in}{4.178732in}}%
\pgfpathlineto{\pgfqpoint{4.397224in}{4.149333in}}%
\pgfpathlineto{\pgfqpoint{4.402099in}{4.144514in}}%
\pgfpathlineto{\pgfqpoint{4.407273in}{4.139490in}}%
\pgfpathlineto{\pgfqpoint{4.435269in}{4.112000in}}%
\pgfpathlineto{\pgfqpoint{4.441122in}{4.106196in}}%
\pgfpathlineto{\pgfqpoint{4.447354in}{4.100126in}}%
\pgfpathlineto{\pgfqpoint{4.460580in}{4.086986in}}%
\pgfpathlineto{\pgfqpoint{4.473203in}{4.074667in}}%
\pgfpathlineto{\pgfqpoint{4.487434in}{4.060640in}}%
\pgfpathlineto{\pgfqpoint{4.499523in}{4.048594in}}%
\pgfpathlineto{\pgfqpoint{4.511025in}{4.037333in}}%
\pgfpathlineto{\pgfqpoint{4.527515in}{4.021031in}}%
\pgfpathlineto{\pgfqpoint{4.538407in}{4.010145in}}%
\pgfpathlineto{\pgfqpoint{4.548737in}{4.000000in}}%
\pgfpathlineto{\pgfqpoint{4.557828in}{3.990902in}}%
\pgfpathlineto{\pgfqpoint{4.567596in}{3.981299in}}%
\pgfpathlineto{\pgfqpoint{4.577230in}{3.971640in}}%
\pgfpathlineto{\pgfqpoint{4.586340in}{3.962667in}}%
\pgfpathlineto{\pgfqpoint{4.607677in}{3.941444in}}%
\pgfpathlineto{\pgfqpoint{4.615994in}{3.933080in}}%
\pgfpathlineto{\pgfqpoint{4.623835in}{3.925333in}}%
\pgfpathlineto{\pgfqpoint{4.635330in}{3.913758in}}%
\pgfpathlineto{\pgfqpoint{4.647758in}{3.901465in}}%
\pgfpathlineto{\pgfqpoint{4.661221in}{3.888000in}}%
\pgfpathlineto{\pgfqpoint{4.673991in}{3.875102in}}%
\pgfpathlineto{\pgfqpoint{4.687838in}{3.861363in}}%
\pgfpathlineto{\pgfqpoint{4.693343in}{3.855794in}}%
\pgfpathlineto{\pgfqpoint{4.698501in}{3.850667in}}%
\pgfpathlineto{\pgfqpoint{4.727919in}{3.821136in}}%
\pgfpathlineto{\pgfqpoint{4.731929in}{3.817068in}}%
\pgfpathlineto{\pgfqpoint{4.735674in}{3.813333in}}%
\pgfpathlineto{\pgfqpoint{4.768000in}{3.780786in}}%
\pgfusepath{fill}%
\end{pgfscope}%
\begin{pgfscope}%
\pgfpathrectangle{\pgfqpoint{0.800000in}{0.528000in}}{\pgfqpoint{3.968000in}{3.696000in}}%
\pgfusepath{clip}%
\pgfsetbuttcap%
\pgfsetroundjoin%
\definecolor{currentfill}{rgb}{0.515992,0.831158,0.294279}%
\pgfsetfillcolor{currentfill}%
\pgfsetlinewidth{0.000000pt}%
\definecolor{currentstroke}{rgb}{0.000000,0.000000,0.000000}%
\pgfsetstrokecolor{currentstroke}%
\pgfsetdash{}{0pt}%
\pgfpathmoveto{\pgfqpoint{4.768000in}{3.785994in}}%
\pgfpathlineto{\pgfqpoint{4.740847in}{3.813333in}}%
\pgfpathlineto{\pgfqpoint{4.734603in}{3.819559in}}%
\pgfpathlineto{\pgfqpoint{4.727919in}{3.826341in}}%
\pgfpathlineto{\pgfqpoint{4.703686in}{3.850667in}}%
\pgfpathlineto{\pgfqpoint{4.696020in}{3.858287in}}%
\pgfpathlineto{\pgfqpoint{4.687838in}{3.866564in}}%
\pgfpathlineto{\pgfqpoint{4.676695in}{3.877621in}}%
\pgfpathlineto{\pgfqpoint{4.666419in}{3.888000in}}%
\pgfpathlineto{\pgfqpoint{4.647758in}{3.906663in}}%
\pgfpathlineto{\pgfqpoint{4.638037in}{3.916279in}}%
\pgfpathlineto{\pgfqpoint{4.629045in}{3.925333in}}%
\pgfpathlineto{\pgfqpoint{4.618676in}{3.935578in}}%
\pgfpathlineto{\pgfqpoint{4.607677in}{3.946639in}}%
\pgfpathlineto{\pgfqpoint{4.591563in}{3.962667in}}%
\pgfpathlineto{\pgfqpoint{4.579914in}{3.974141in}}%
\pgfpathlineto{\pgfqpoint{4.567596in}{3.986491in}}%
\pgfpathlineto{\pgfqpoint{4.560540in}{3.993428in}}%
\pgfpathlineto{\pgfqpoint{4.553973in}{4.000000in}}%
\pgfpathlineto{\pgfqpoint{4.541093in}{4.012647in}}%
\pgfpathlineto{\pgfqpoint{4.527515in}{4.026219in}}%
\pgfpathlineto{\pgfqpoint{4.516273in}{4.037333in}}%
\pgfpathlineto{\pgfqpoint{4.502213in}{4.051099in}}%
\pgfpathlineto{\pgfqpoint{4.487434in}{4.065825in}}%
\pgfpathlineto{\pgfqpoint{4.478463in}{4.074667in}}%
\pgfpathlineto{\pgfqpoint{4.463272in}{4.089494in}}%
\pgfpathlineto{\pgfqpoint{4.447354in}{4.105308in}}%
\pgfpathlineto{\pgfqpoint{4.443841in}{4.108729in}}%
\pgfpathlineto{\pgfqpoint{4.440542in}{4.112000in}}%
\pgfpathlineto{\pgfqpoint{4.407273in}{4.144668in}}%
\pgfpathlineto{\pgfqpoint{4.404821in}{4.147049in}}%
\pgfpathlineto{\pgfqpoint{4.402510in}{4.149333in}}%
\pgfpathlineto{\pgfqpoint{4.367192in}{4.183906in}}%
\pgfpathlineto{\pgfqpoint{4.365739in}{4.185313in}}%
\pgfpathlineto{\pgfqpoint{4.364365in}{4.186667in}}%
\pgfpathlineto{\pgfqpoint{4.327111in}{4.223022in}}%
\pgfpathlineto{\pgfqpoint{4.326596in}{4.223520in}}%
\pgfpathlineto{\pgfqpoint{4.326107in}{4.224000in}}%
\pgfpathlineto{\pgfqpoint{4.323451in}{4.224000in}}%
\pgfpathlineto{\pgfqpoint{4.325232in}{4.222250in}}%
\pgfpathlineto{\pgfqpoint{4.327111in}{4.220437in}}%
\pgfpathlineto{\pgfqpoint{4.361716in}{4.186667in}}%
\pgfpathlineto{\pgfqpoint{4.364377in}{4.184044in}}%
\pgfpathlineto{\pgfqpoint{4.367192in}{4.181319in}}%
\pgfpathlineto{\pgfqpoint{4.399867in}{4.149333in}}%
\pgfpathlineto{\pgfqpoint{4.403460in}{4.145782in}}%
\pgfpathlineto{\pgfqpoint{4.407273in}{4.142079in}}%
\pgfpathlineto{\pgfqpoint{4.437906in}{4.112000in}}%
\pgfpathlineto{\pgfqpoint{4.442482in}{4.107462in}}%
\pgfpathlineto{\pgfqpoint{4.447354in}{4.102717in}}%
\pgfpathlineto{\pgfqpoint{4.461926in}{4.088240in}}%
\pgfpathlineto{\pgfqpoint{4.475833in}{4.074667in}}%
\pgfpathlineto{\pgfqpoint{4.487434in}{4.063233in}}%
\pgfpathlineto{\pgfqpoint{4.500868in}{4.049846in}}%
\pgfpathlineto{\pgfqpoint{4.513649in}{4.037333in}}%
\pgfpathlineto{\pgfqpoint{4.527515in}{4.023625in}}%
\pgfpathlineto{\pgfqpoint{4.539750in}{4.011396in}}%
\pgfpathlineto{\pgfqpoint{4.551355in}{4.000000in}}%
\pgfpathlineto{\pgfqpoint{4.559184in}{3.992165in}}%
\pgfpathlineto{\pgfqpoint{4.567596in}{3.983895in}}%
\pgfpathlineto{\pgfqpoint{4.578572in}{3.972891in}}%
\pgfpathlineto{\pgfqpoint{4.588952in}{3.962667in}}%
\pgfpathlineto{\pgfqpoint{4.607677in}{3.944041in}}%
\pgfpathlineto{\pgfqpoint{4.617335in}{3.934329in}}%
\pgfpathlineto{\pgfqpoint{4.626440in}{3.925333in}}%
\pgfpathlineto{\pgfqpoint{4.636684in}{3.915018in}}%
\pgfpathlineto{\pgfqpoint{4.647758in}{3.904064in}}%
\pgfpathlineto{\pgfqpoint{4.663820in}{3.888000in}}%
\pgfpathlineto{\pgfqpoint{4.675343in}{3.876361in}}%
\pgfpathlineto{\pgfqpoint{4.687838in}{3.863964in}}%
\pgfpathlineto{\pgfqpoint{4.694681in}{3.857041in}}%
\pgfpathlineto{\pgfqpoint{4.701093in}{3.850667in}}%
\pgfpathlineto{\pgfqpoint{4.727919in}{3.823739in}}%
\pgfpathlineto{\pgfqpoint{4.733266in}{3.818313in}}%
\pgfpathlineto{\pgfqpoint{4.738261in}{3.813333in}}%
\pgfpathlineto{\pgfqpoint{4.768000in}{3.783390in}}%
\pgfusepath{fill}%
\end{pgfscope}%
\begin{pgfscope}%
\pgfpathrectangle{\pgfqpoint{0.800000in}{0.528000in}}{\pgfqpoint{3.968000in}{3.696000in}}%
\pgfusepath{clip}%
\pgfsetbuttcap%
\pgfsetroundjoin%
\definecolor{currentfill}{rgb}{0.515992,0.831158,0.294279}%
\pgfsetfillcolor{currentfill}%
\pgfsetlinewidth{0.000000pt}%
\definecolor{currentstroke}{rgb}{0.000000,0.000000,0.000000}%
\pgfsetstrokecolor{currentstroke}%
\pgfsetdash{}{0pt}%
\pgfpathmoveto{\pgfqpoint{4.768000in}{3.788598in}}%
\pgfpathlineto{\pgfqpoint{4.743433in}{3.813333in}}%
\pgfpathlineto{\pgfqpoint{4.735940in}{3.820804in}}%
\pgfpathlineto{\pgfqpoint{4.727919in}{3.828944in}}%
\pgfpathlineto{\pgfqpoint{4.706279in}{3.850667in}}%
\pgfpathlineto{\pgfqpoint{4.697358in}{3.859534in}}%
\pgfpathlineto{\pgfqpoint{4.687838in}{3.869165in}}%
\pgfpathlineto{\pgfqpoint{4.678047in}{3.878880in}}%
\pgfpathlineto{\pgfqpoint{4.669018in}{3.888000in}}%
\pgfpathlineto{\pgfqpoint{4.647758in}{3.909262in}}%
\pgfpathlineto{\pgfqpoint{4.639390in}{3.917539in}}%
\pgfpathlineto{\pgfqpoint{4.631650in}{3.925333in}}%
\pgfpathlineto{\pgfqpoint{4.620017in}{3.936827in}}%
\pgfpathlineto{\pgfqpoint{4.607677in}{3.949236in}}%
\pgfpathlineto{\pgfqpoint{4.594174in}{3.962667in}}%
\pgfpathlineto{\pgfqpoint{4.581256in}{3.975391in}}%
\pgfpathlineto{\pgfqpoint{4.567596in}{3.989086in}}%
\pgfpathlineto{\pgfqpoint{4.561896in}{3.994690in}}%
\pgfpathlineto{\pgfqpoint{4.556590in}{4.000000in}}%
\pgfpathlineto{\pgfqpoint{4.542437in}{4.013899in}}%
\pgfpathlineto{\pgfqpoint{4.527515in}{4.028813in}}%
\pgfpathlineto{\pgfqpoint{4.518897in}{4.037333in}}%
\pgfpathlineto{\pgfqpoint{4.503557in}{4.052351in}}%
\pgfpathlineto{\pgfqpoint{4.487434in}{4.068417in}}%
\pgfpathlineto{\pgfqpoint{4.481093in}{4.074667in}}%
\pgfpathlineto{\pgfqpoint{4.464618in}{4.090747in}}%
\pgfpathlineto{\pgfqpoint{4.447354in}{4.107898in}}%
\pgfpathlineto{\pgfqpoint{4.445201in}{4.109995in}}%
\pgfpathlineto{\pgfqpoint{4.443179in}{4.112000in}}%
\pgfpathlineto{\pgfqpoint{4.407273in}{4.147257in}}%
\pgfpathlineto{\pgfqpoint{4.406181in}{4.148317in}}%
\pgfpathlineto{\pgfqpoint{4.405153in}{4.149333in}}%
\pgfpathlineto{\pgfqpoint{4.367192in}{4.186493in}}%
\pgfpathlineto{\pgfqpoint{4.367101in}{4.186582in}}%
\pgfpathlineto{\pgfqpoint{4.367015in}{4.186667in}}%
\pgfpathlineto{\pgfqpoint{4.363297in}{4.190295in}}%
\pgfpathlineto{\pgfqpoint{4.328744in}{4.224000in}}%
\pgfpathlineto{\pgfqpoint{4.327111in}{4.224000in}}%
\pgfpathlineto{\pgfqpoint{4.326107in}{4.224000in}}%
\pgfpathlineto{\pgfqpoint{4.326596in}{4.223520in}}%
\pgfpathlineto{\pgfqpoint{4.327111in}{4.223022in}}%
\pgfpathlineto{\pgfqpoint{4.364365in}{4.186667in}}%
\pgfpathlineto{\pgfqpoint{4.365739in}{4.185313in}}%
\pgfpathlineto{\pgfqpoint{4.367192in}{4.183906in}}%
\pgfpathlineto{\pgfqpoint{4.402510in}{4.149333in}}%
\pgfpathlineto{\pgfqpoint{4.404821in}{4.147049in}}%
\pgfpathlineto{\pgfqpoint{4.407273in}{4.144668in}}%
\pgfpathlineto{\pgfqpoint{4.440542in}{4.112000in}}%
\pgfpathlineto{\pgfqpoint{4.443841in}{4.108729in}}%
\pgfpathlineto{\pgfqpoint{4.447354in}{4.105308in}}%
\pgfpathlineto{\pgfqpoint{4.463272in}{4.089494in}}%
\pgfpathlineto{\pgfqpoint{4.478463in}{4.074667in}}%
\pgfpathlineto{\pgfqpoint{4.487434in}{4.065825in}}%
\pgfpathlineto{\pgfqpoint{4.502213in}{4.051099in}}%
\pgfpathlineto{\pgfqpoint{4.516273in}{4.037333in}}%
\pgfpathlineto{\pgfqpoint{4.527515in}{4.026219in}}%
\pgfpathlineto{\pgfqpoint{4.541093in}{4.012647in}}%
\pgfpathlineto{\pgfqpoint{4.553973in}{4.000000in}}%
\pgfpathlineto{\pgfqpoint{4.560540in}{3.993428in}}%
\pgfpathlineto{\pgfqpoint{4.567596in}{3.986491in}}%
\pgfpathlineto{\pgfqpoint{4.579914in}{3.974141in}}%
\pgfpathlineto{\pgfqpoint{4.591563in}{3.962667in}}%
\pgfpathlineto{\pgfqpoint{4.607677in}{3.946639in}}%
\pgfpathlineto{\pgfqpoint{4.618676in}{3.935578in}}%
\pgfpathlineto{\pgfqpoint{4.629045in}{3.925333in}}%
\pgfpathlineto{\pgfqpoint{4.638037in}{3.916279in}}%
\pgfpathlineto{\pgfqpoint{4.647758in}{3.906663in}}%
\pgfpathlineto{\pgfqpoint{4.666419in}{3.888000in}}%
\pgfpathlineto{\pgfqpoint{4.676695in}{3.877621in}}%
\pgfpathlineto{\pgfqpoint{4.687838in}{3.866564in}}%
\pgfpathlineto{\pgfqpoint{4.696020in}{3.858287in}}%
\pgfpathlineto{\pgfqpoint{4.703686in}{3.850667in}}%
\pgfpathlineto{\pgfqpoint{4.727919in}{3.826341in}}%
\pgfpathlineto{\pgfqpoint{4.734603in}{3.819559in}}%
\pgfpathlineto{\pgfqpoint{4.740847in}{3.813333in}}%
\pgfpathlineto{\pgfqpoint{4.768000in}{3.785994in}}%
\pgfusepath{fill}%
\end{pgfscope}%
\begin{pgfscope}%
\pgfpathrectangle{\pgfqpoint{0.800000in}{0.528000in}}{\pgfqpoint{3.968000in}{3.696000in}}%
\pgfusepath{clip}%
\pgfsetbuttcap%
\pgfsetroundjoin%
\definecolor{currentfill}{rgb}{0.515992,0.831158,0.294279}%
\pgfsetfillcolor{currentfill}%
\pgfsetlinewidth{0.000000pt}%
\definecolor{currentstroke}{rgb}{0.000000,0.000000,0.000000}%
\pgfsetstrokecolor{currentstroke}%
\pgfsetdash{}{0pt}%
\pgfpathmoveto{\pgfqpoint{4.768000in}{3.791202in}}%
\pgfpathlineto{\pgfqpoint{4.746020in}{3.813333in}}%
\pgfpathlineto{\pgfqpoint{4.737277in}{3.822050in}}%
\pgfpathlineto{\pgfqpoint{4.727919in}{3.831546in}}%
\pgfpathlineto{\pgfqpoint{4.708871in}{3.850667in}}%
\pgfpathlineto{\pgfqpoint{4.698697in}{3.860781in}}%
\pgfpathlineto{\pgfqpoint{4.687838in}{3.871766in}}%
\pgfpathlineto{\pgfqpoint{4.679399in}{3.880139in}}%
\pgfpathlineto{\pgfqpoint{4.671617in}{3.888000in}}%
\pgfpathlineto{\pgfqpoint{4.647758in}{3.911862in}}%
\pgfpathlineto{\pgfqpoint{4.640743in}{3.918800in}}%
\pgfpathlineto{\pgfqpoint{4.634255in}{3.925333in}}%
\pgfpathlineto{\pgfqpoint{4.621357in}{3.938076in}}%
\pgfpathlineto{\pgfqpoint{4.607677in}{3.951834in}}%
\pgfpathlineto{\pgfqpoint{4.596786in}{3.962667in}}%
\pgfpathlineto{\pgfqpoint{4.582599in}{3.976641in}}%
\pgfpathlineto{\pgfqpoint{4.567596in}{3.991682in}}%
\pgfpathlineto{\pgfqpoint{4.563251in}{3.995953in}}%
\pgfpathlineto{\pgfqpoint{4.559208in}{4.000000in}}%
\pgfpathlineto{\pgfqpoint{4.543780in}{4.015150in}}%
\pgfpathlineto{\pgfqpoint{4.527515in}{4.031407in}}%
\pgfpathlineto{\pgfqpoint{4.521521in}{4.037333in}}%
\pgfpathlineto{\pgfqpoint{4.504902in}{4.053603in}}%
\pgfpathlineto{\pgfqpoint{4.487434in}{4.071010in}}%
\pgfpathlineto{\pgfqpoint{4.483724in}{4.074667in}}%
\pgfpathlineto{\pgfqpoint{4.465963in}{4.092001in}}%
\pgfpathlineto{\pgfqpoint{4.447354in}{4.110489in}}%
\pgfpathlineto{\pgfqpoint{4.446561in}{4.111261in}}%
\pgfpathlineto{\pgfqpoint{4.445816in}{4.112000in}}%
\pgfpathlineto{\pgfqpoint{4.417433in}{4.139870in}}%
\pgfpathlineto{\pgfqpoint{4.407790in}{4.149333in}}%
\pgfpathlineto{\pgfqpoint{4.407537in}{4.149580in}}%
\pgfpathlineto{\pgfqpoint{4.407273in}{4.149841in}}%
\pgfpathlineto{\pgfqpoint{4.369637in}{4.186667in}}%
\pgfpathlineto{\pgfqpoint{4.368439in}{4.187828in}}%
\pgfpathlineto{\pgfqpoint{4.367192in}{4.189057in}}%
\pgfpathlineto{\pgfqpoint{4.331371in}{4.224000in}}%
\pgfpathlineto{\pgfqpoint{4.328744in}{4.224000in}}%
\pgfpathlineto{\pgfqpoint{4.363297in}{4.190295in}}%
\pgfpathlineto{\pgfqpoint{4.367015in}{4.186667in}}%
\pgfpathlineto{\pgfqpoint{4.367101in}{4.186582in}}%
\pgfpathlineto{\pgfqpoint{4.367192in}{4.186493in}}%
\pgfpathlineto{\pgfqpoint{4.405153in}{4.149333in}}%
\pgfpathlineto{\pgfqpoint{4.406181in}{4.148317in}}%
\pgfpathlineto{\pgfqpoint{4.407273in}{4.147257in}}%
\pgfpathlineto{\pgfqpoint{4.443179in}{4.112000in}}%
\pgfpathlineto{\pgfqpoint{4.445201in}{4.109995in}}%
\pgfpathlineto{\pgfqpoint{4.447354in}{4.107898in}}%
\pgfpathlineto{\pgfqpoint{4.464618in}{4.090747in}}%
\pgfpathlineto{\pgfqpoint{4.481093in}{4.074667in}}%
\pgfpathlineto{\pgfqpoint{4.487434in}{4.068417in}}%
\pgfpathlineto{\pgfqpoint{4.503557in}{4.052351in}}%
\pgfpathlineto{\pgfqpoint{4.518897in}{4.037333in}}%
\pgfpathlineto{\pgfqpoint{4.527515in}{4.028813in}}%
\pgfpathlineto{\pgfqpoint{4.542437in}{4.013899in}}%
\pgfpathlineto{\pgfqpoint{4.556590in}{4.000000in}}%
\pgfpathlineto{\pgfqpoint{4.561896in}{3.994690in}}%
\pgfpathlineto{\pgfqpoint{4.567596in}{3.989086in}}%
\pgfpathlineto{\pgfqpoint{4.581256in}{3.975391in}}%
\pgfpathlineto{\pgfqpoint{4.594174in}{3.962667in}}%
\pgfpathlineto{\pgfqpoint{4.607677in}{3.949236in}}%
\pgfpathlineto{\pgfqpoint{4.620017in}{3.936827in}}%
\pgfpathlineto{\pgfqpoint{4.631650in}{3.925333in}}%
\pgfpathlineto{\pgfqpoint{4.639390in}{3.917539in}}%
\pgfpathlineto{\pgfqpoint{4.647758in}{3.909262in}}%
\pgfpathlineto{\pgfqpoint{4.669018in}{3.888000in}}%
\pgfpathlineto{\pgfqpoint{4.678047in}{3.878880in}}%
\pgfpathlineto{\pgfqpoint{4.687838in}{3.869165in}}%
\pgfpathlineto{\pgfqpoint{4.697358in}{3.859534in}}%
\pgfpathlineto{\pgfqpoint{4.706279in}{3.850667in}}%
\pgfpathlineto{\pgfqpoint{4.727919in}{3.828944in}}%
\pgfpathlineto{\pgfqpoint{4.735940in}{3.820804in}}%
\pgfpathlineto{\pgfqpoint{4.743433in}{3.813333in}}%
\pgfpathlineto{\pgfqpoint{4.768000in}{3.788598in}}%
\pgfusepath{fill}%
\end{pgfscope}%
\begin{pgfscope}%
\pgfpathrectangle{\pgfqpoint{0.800000in}{0.528000in}}{\pgfqpoint{3.968000in}{3.696000in}}%
\pgfusepath{clip}%
\pgfsetbuttcap%
\pgfsetroundjoin%
\definecolor{currentfill}{rgb}{0.515992,0.831158,0.294279}%
\pgfsetfillcolor{currentfill}%
\pgfsetlinewidth{0.000000pt}%
\definecolor{currentstroke}{rgb}{0.000000,0.000000,0.000000}%
\pgfsetstrokecolor{currentstroke}%
\pgfsetdash{}{0pt}%
\pgfpathmoveto{\pgfqpoint{4.768000in}{3.793807in}}%
\pgfpathlineto{\pgfqpoint{4.748606in}{3.813333in}}%
\pgfpathlineto{\pgfqpoint{4.738615in}{3.823296in}}%
\pgfpathlineto{\pgfqpoint{4.727919in}{3.834149in}}%
\pgfpathlineto{\pgfqpoint{4.711464in}{3.850667in}}%
\pgfpathlineto{\pgfqpoint{4.700035in}{3.862027in}}%
\pgfpathlineto{\pgfqpoint{4.687838in}{3.874367in}}%
\pgfpathlineto{\pgfqpoint{4.680751in}{3.881399in}}%
\pgfpathlineto{\pgfqpoint{4.674215in}{3.888000in}}%
\pgfpathlineto{\pgfqpoint{4.647758in}{3.914461in}}%
\pgfpathlineto{\pgfqpoint{4.642097in}{3.920060in}}%
\pgfpathlineto{\pgfqpoint{4.636860in}{3.925333in}}%
\pgfpathlineto{\pgfqpoint{4.622698in}{3.939325in}}%
\pgfpathlineto{\pgfqpoint{4.607677in}{3.954431in}}%
\pgfpathlineto{\pgfqpoint{4.599397in}{3.962667in}}%
\pgfpathlineto{\pgfqpoint{4.583941in}{3.977891in}}%
\pgfpathlineto{\pgfqpoint{4.567596in}{3.994278in}}%
\pgfpathlineto{\pgfqpoint{4.564607in}{3.997216in}}%
\pgfpathlineto{\pgfqpoint{4.561825in}{4.000000in}}%
\pgfpathlineto{\pgfqpoint{4.545124in}{4.016401in}}%
\pgfpathlineto{\pgfqpoint{4.527515in}{4.034001in}}%
\pgfpathlineto{\pgfqpoint{4.524145in}{4.037333in}}%
\pgfpathlineto{\pgfqpoint{4.506246in}{4.054856in}}%
\pgfpathlineto{\pgfqpoint{4.487434in}{4.073602in}}%
\pgfpathlineto{\pgfqpoint{4.486354in}{4.074667in}}%
\pgfpathlineto{\pgfqpoint{4.467309in}{4.093255in}}%
\pgfpathlineto{\pgfqpoint{4.448440in}{4.112000in}}%
\pgfpathlineto{\pgfqpoint{4.447909in}{4.112518in}}%
\pgfpathlineto{\pgfqpoint{4.447354in}{4.113069in}}%
\pgfpathlineto{\pgfqpoint{4.410404in}{4.149333in}}%
\pgfpathlineto{\pgfqpoint{4.408872in}{4.150823in}}%
\pgfpathlineto{\pgfqpoint{4.407273in}{4.152405in}}%
\pgfpathlineto{\pgfqpoint{4.372257in}{4.186667in}}%
\pgfpathlineto{\pgfqpoint{4.369775in}{4.189072in}}%
\pgfpathlineto{\pgfqpoint{4.367192in}{4.191620in}}%
\pgfpathlineto{\pgfqpoint{4.333998in}{4.224000in}}%
\pgfpathlineto{\pgfqpoint{4.331371in}{4.224000in}}%
\pgfpathlineto{\pgfqpoint{4.367192in}{4.189057in}}%
\pgfpathlineto{\pgfqpoint{4.368439in}{4.187828in}}%
\pgfpathlineto{\pgfqpoint{4.369637in}{4.186667in}}%
\pgfpathlineto{\pgfqpoint{4.407273in}{4.149841in}}%
\pgfpathlineto{\pgfqpoint{4.407537in}{4.149580in}}%
\pgfpathlineto{\pgfqpoint{4.407790in}{4.149333in}}%
\pgfpathlineto{\pgfqpoint{4.417433in}{4.139870in}}%
\pgfpathlineto{\pgfqpoint{4.445816in}{4.112000in}}%
\pgfpathlineto{\pgfqpoint{4.446561in}{4.111261in}}%
\pgfpathlineto{\pgfqpoint{4.447354in}{4.110489in}}%
\pgfpathlineto{\pgfqpoint{4.465963in}{4.092001in}}%
\pgfpathlineto{\pgfqpoint{4.483724in}{4.074667in}}%
\pgfpathlineto{\pgfqpoint{4.487434in}{4.071010in}}%
\pgfpathlineto{\pgfqpoint{4.504902in}{4.053603in}}%
\pgfpathlineto{\pgfqpoint{4.521521in}{4.037333in}}%
\pgfpathlineto{\pgfqpoint{4.527515in}{4.031407in}}%
\pgfpathlineto{\pgfqpoint{4.543780in}{4.015150in}}%
\pgfpathlineto{\pgfqpoint{4.559208in}{4.000000in}}%
\pgfpathlineto{\pgfqpoint{4.563251in}{3.995953in}}%
\pgfpathlineto{\pgfqpoint{4.567596in}{3.991682in}}%
\pgfpathlineto{\pgfqpoint{4.582599in}{3.976641in}}%
\pgfpathlineto{\pgfqpoint{4.596786in}{3.962667in}}%
\pgfpathlineto{\pgfqpoint{4.607677in}{3.951834in}}%
\pgfpathlineto{\pgfqpoint{4.621357in}{3.938076in}}%
\pgfpathlineto{\pgfqpoint{4.634255in}{3.925333in}}%
\pgfpathlineto{\pgfqpoint{4.640743in}{3.918800in}}%
\pgfpathlineto{\pgfqpoint{4.647758in}{3.911862in}}%
\pgfpathlineto{\pgfqpoint{4.671617in}{3.888000in}}%
\pgfpathlineto{\pgfqpoint{4.679399in}{3.880139in}}%
\pgfpathlineto{\pgfqpoint{4.687838in}{3.871766in}}%
\pgfpathlineto{\pgfqpoint{4.698697in}{3.860781in}}%
\pgfpathlineto{\pgfqpoint{4.708871in}{3.850667in}}%
\pgfpathlineto{\pgfqpoint{4.727919in}{3.831546in}}%
\pgfpathlineto{\pgfqpoint{4.737277in}{3.822050in}}%
\pgfpathlineto{\pgfqpoint{4.746020in}{3.813333in}}%
\pgfpathlineto{\pgfqpoint{4.768000in}{3.791202in}}%
\pgfusepath{fill}%
\end{pgfscope}%
\begin{pgfscope}%
\pgfpathrectangle{\pgfqpoint{0.800000in}{0.528000in}}{\pgfqpoint{3.968000in}{3.696000in}}%
\pgfusepath{clip}%
\pgfsetbuttcap%
\pgfsetroundjoin%
\definecolor{currentfill}{rgb}{0.525776,0.833491,0.288127}%
\pgfsetfillcolor{currentfill}%
\pgfsetlinewidth{0.000000pt}%
\definecolor{currentstroke}{rgb}{0.000000,0.000000,0.000000}%
\pgfsetstrokecolor{currentstroke}%
\pgfsetdash{}{0pt}%
\pgfpathmoveto{\pgfqpoint{4.768000in}{3.796411in}}%
\pgfpathlineto{\pgfqpoint{4.751193in}{3.813333in}}%
\pgfpathlineto{\pgfqpoint{4.739952in}{3.824541in}}%
\pgfpathlineto{\pgfqpoint{4.727919in}{3.836751in}}%
\pgfpathlineto{\pgfqpoint{4.714056in}{3.850667in}}%
\pgfpathlineto{\pgfqpoint{4.701373in}{3.863274in}}%
\pgfpathlineto{\pgfqpoint{4.687838in}{3.876967in}}%
\pgfpathlineto{\pgfqpoint{4.682103in}{3.882658in}}%
\pgfpathlineto{\pgfqpoint{4.676814in}{3.888000in}}%
\pgfpathlineto{\pgfqpoint{4.647758in}{3.917060in}}%
\pgfpathlineto{\pgfqpoint{4.643450in}{3.921321in}}%
\pgfpathlineto{\pgfqpoint{4.639465in}{3.925333in}}%
\pgfpathlineto{\pgfqpoint{4.624039in}{3.940574in}}%
\pgfpathlineto{\pgfqpoint{4.607677in}{3.957028in}}%
\pgfpathlineto{\pgfqpoint{4.602008in}{3.962667in}}%
\pgfpathlineto{\pgfqpoint{4.585283in}{3.979141in}}%
\pgfpathlineto{\pgfqpoint{4.567596in}{3.996873in}}%
\pgfpathlineto{\pgfqpoint{4.565963in}{3.998479in}}%
\pgfpathlineto{\pgfqpoint{4.564443in}{4.000000in}}%
\pgfpathlineto{\pgfqpoint{4.546467in}{4.017653in}}%
\pgfpathlineto{\pgfqpoint{4.527515in}{4.036595in}}%
\pgfpathlineto{\pgfqpoint{4.526769in}{4.037333in}}%
\pgfpathlineto{\pgfqpoint{4.507591in}{4.056108in}}%
\pgfpathlineto{\pgfqpoint{4.488967in}{4.074667in}}%
\pgfpathlineto{\pgfqpoint{4.488219in}{4.075398in}}%
\pgfpathlineto{\pgfqpoint{4.487434in}{4.076179in}}%
\pgfpathlineto{\pgfqpoint{4.468655in}{4.094508in}}%
\pgfpathlineto{\pgfqpoint{4.451048in}{4.112000in}}%
\pgfpathlineto{\pgfqpoint{4.449243in}{4.113760in}}%
\pgfpathlineto{\pgfqpoint{4.447354in}{4.115635in}}%
\pgfpathlineto{\pgfqpoint{4.413018in}{4.149333in}}%
\pgfpathlineto{\pgfqpoint{4.410207in}{4.152066in}}%
\pgfpathlineto{\pgfqpoint{4.407273in}{4.154969in}}%
\pgfpathlineto{\pgfqpoint{4.374877in}{4.186667in}}%
\pgfpathlineto{\pgfqpoint{4.371111in}{4.190317in}}%
\pgfpathlineto{\pgfqpoint{4.367192in}{4.194182in}}%
\pgfpathlineto{\pgfqpoint{4.336624in}{4.224000in}}%
\pgfpathlineto{\pgfqpoint{4.333998in}{4.224000in}}%
\pgfpathlineto{\pgfqpoint{4.367192in}{4.191620in}}%
\pgfpathlineto{\pgfqpoint{4.369775in}{4.189072in}}%
\pgfpathlineto{\pgfqpoint{4.372257in}{4.186667in}}%
\pgfpathlineto{\pgfqpoint{4.407273in}{4.152405in}}%
\pgfpathlineto{\pgfqpoint{4.408872in}{4.150823in}}%
\pgfpathlineto{\pgfqpoint{4.410404in}{4.149333in}}%
\pgfpathlineto{\pgfqpoint{4.447354in}{4.113069in}}%
\pgfpathlineto{\pgfqpoint{4.447909in}{4.112518in}}%
\pgfpathlineto{\pgfqpoint{4.448440in}{4.112000in}}%
\pgfpathlineto{\pgfqpoint{4.467309in}{4.093255in}}%
\pgfpathlineto{\pgfqpoint{4.486354in}{4.074667in}}%
\pgfpathlineto{\pgfqpoint{4.487434in}{4.073602in}}%
\pgfpathlineto{\pgfqpoint{4.506246in}{4.054856in}}%
\pgfpathlineto{\pgfqpoint{4.524145in}{4.037333in}}%
\pgfpathlineto{\pgfqpoint{4.527515in}{4.034001in}}%
\pgfpathlineto{\pgfqpoint{4.545124in}{4.016401in}}%
\pgfpathlineto{\pgfqpoint{4.561825in}{4.000000in}}%
\pgfpathlineto{\pgfqpoint{4.564607in}{3.997216in}}%
\pgfpathlineto{\pgfqpoint{4.567596in}{3.994278in}}%
\pgfpathlineto{\pgfqpoint{4.583941in}{3.977891in}}%
\pgfpathlineto{\pgfqpoint{4.599397in}{3.962667in}}%
\pgfpathlineto{\pgfqpoint{4.607677in}{3.954431in}}%
\pgfpathlineto{\pgfqpoint{4.622698in}{3.939325in}}%
\pgfpathlineto{\pgfqpoint{4.636860in}{3.925333in}}%
\pgfpathlineto{\pgfqpoint{4.642097in}{3.920060in}}%
\pgfpathlineto{\pgfqpoint{4.647758in}{3.914461in}}%
\pgfpathlineto{\pgfqpoint{4.674215in}{3.888000in}}%
\pgfpathlineto{\pgfqpoint{4.680751in}{3.881399in}}%
\pgfpathlineto{\pgfqpoint{4.687838in}{3.874367in}}%
\pgfpathlineto{\pgfqpoint{4.700035in}{3.862027in}}%
\pgfpathlineto{\pgfqpoint{4.711464in}{3.850667in}}%
\pgfpathlineto{\pgfqpoint{4.727919in}{3.834149in}}%
\pgfpathlineto{\pgfqpoint{4.738615in}{3.823296in}}%
\pgfpathlineto{\pgfqpoint{4.748606in}{3.813333in}}%
\pgfpathlineto{\pgfqpoint{4.768000in}{3.793807in}}%
\pgfusepath{fill}%
\end{pgfscope}%
\begin{pgfscope}%
\pgfpathrectangle{\pgfqpoint{0.800000in}{0.528000in}}{\pgfqpoint{3.968000in}{3.696000in}}%
\pgfusepath{clip}%
\pgfsetbuttcap%
\pgfsetroundjoin%
\definecolor{currentfill}{rgb}{0.525776,0.833491,0.288127}%
\pgfsetfillcolor{currentfill}%
\pgfsetlinewidth{0.000000pt}%
\definecolor{currentstroke}{rgb}{0.000000,0.000000,0.000000}%
\pgfsetstrokecolor{currentstroke}%
\pgfsetdash{}{0pt}%
\pgfpathmoveto{\pgfqpoint{4.768000in}{3.799015in}}%
\pgfpathlineto{\pgfqpoint{4.753779in}{3.813333in}}%
\pgfpathlineto{\pgfqpoint{4.741289in}{3.825787in}}%
\pgfpathlineto{\pgfqpoint{4.727919in}{3.839354in}}%
\pgfpathlineto{\pgfqpoint{4.716649in}{3.850667in}}%
\pgfpathlineto{\pgfqpoint{4.702712in}{3.864521in}}%
\pgfpathlineto{\pgfqpoint{4.687838in}{3.879568in}}%
\pgfpathlineto{\pgfqpoint{4.683455in}{3.883917in}}%
\pgfpathlineto{\pgfqpoint{4.679413in}{3.888000in}}%
\pgfpathlineto{\pgfqpoint{4.647758in}{3.919659in}}%
\pgfpathlineto{\pgfqpoint{4.644803in}{3.922581in}}%
\pgfpathlineto{\pgfqpoint{4.642070in}{3.925333in}}%
\pgfpathlineto{\pgfqpoint{4.625380in}{3.941823in}}%
\pgfpathlineto{\pgfqpoint{4.607677in}{3.959626in}}%
\pgfpathlineto{\pgfqpoint{4.604620in}{3.962667in}}%
\pgfpathlineto{\pgfqpoint{4.586625in}{3.980391in}}%
\pgfpathlineto{\pgfqpoint{4.567596in}{3.999469in}}%
\pgfpathlineto{\pgfqpoint{4.567319in}{3.999742in}}%
\pgfpathlineto{\pgfqpoint{4.567061in}{4.000000in}}%
\pgfpathlineto{\pgfqpoint{4.547810in}{4.018904in}}%
\pgfpathlineto{\pgfqpoint{4.529372in}{4.037333in}}%
\pgfpathlineto{\pgfqpoint{4.527515in}{4.039171in}}%
\pgfpathlineto{\pgfqpoint{4.508936in}{4.057361in}}%
\pgfpathlineto{\pgfqpoint{4.491569in}{4.074667in}}%
\pgfpathlineto{\pgfqpoint{4.489552in}{4.076639in}}%
\pgfpathlineto{\pgfqpoint{4.487434in}{4.078747in}}%
\pgfpathlineto{\pgfqpoint{4.470001in}{4.095762in}}%
\pgfpathlineto{\pgfqpoint{4.453656in}{4.112000in}}%
\pgfpathlineto{\pgfqpoint{4.450577in}{4.115002in}}%
\pgfpathlineto{\pgfqpoint{4.447354in}{4.118200in}}%
\pgfpathlineto{\pgfqpoint{4.415632in}{4.149333in}}%
\pgfpathlineto{\pgfqpoint{4.411542in}{4.153310in}}%
\pgfpathlineto{\pgfqpoint{4.407273in}{4.157533in}}%
\pgfpathlineto{\pgfqpoint{4.377498in}{4.186667in}}%
\pgfpathlineto{\pgfqpoint{4.372447in}{4.191561in}}%
\pgfpathlineto{\pgfqpoint{4.367192in}{4.196744in}}%
\pgfpathlineto{\pgfqpoint{4.339251in}{4.224000in}}%
\pgfpathlineto{\pgfqpoint{4.336624in}{4.224000in}}%
\pgfpathlineto{\pgfqpoint{4.367192in}{4.194182in}}%
\pgfpathlineto{\pgfqpoint{4.371111in}{4.190317in}}%
\pgfpathlineto{\pgfqpoint{4.374877in}{4.186667in}}%
\pgfpathlineto{\pgfqpoint{4.407273in}{4.154969in}}%
\pgfpathlineto{\pgfqpoint{4.410207in}{4.152066in}}%
\pgfpathlineto{\pgfqpoint{4.413018in}{4.149333in}}%
\pgfpathlineto{\pgfqpoint{4.447354in}{4.115635in}}%
\pgfpathlineto{\pgfqpoint{4.449243in}{4.113760in}}%
\pgfpathlineto{\pgfqpoint{4.451048in}{4.112000in}}%
\pgfpathlineto{\pgfqpoint{4.468655in}{4.094508in}}%
\pgfpathlineto{\pgfqpoint{4.487434in}{4.076179in}}%
\pgfpathlineto{\pgfqpoint{4.488219in}{4.075398in}}%
\pgfpathlineto{\pgfqpoint{4.488967in}{4.074667in}}%
\pgfpathlineto{\pgfqpoint{4.507591in}{4.056108in}}%
\pgfpathlineto{\pgfqpoint{4.526769in}{4.037333in}}%
\pgfpathlineto{\pgfqpoint{4.527515in}{4.036595in}}%
\pgfpathlineto{\pgfqpoint{4.546467in}{4.017653in}}%
\pgfpathlineto{\pgfqpoint{4.564443in}{4.000000in}}%
\pgfpathlineto{\pgfqpoint{4.565963in}{3.998479in}}%
\pgfpathlineto{\pgfqpoint{4.567596in}{3.996873in}}%
\pgfpathlineto{\pgfqpoint{4.585283in}{3.979141in}}%
\pgfpathlineto{\pgfqpoint{4.602008in}{3.962667in}}%
\pgfpathlineto{\pgfqpoint{4.607677in}{3.957028in}}%
\pgfpathlineto{\pgfqpoint{4.624039in}{3.940574in}}%
\pgfpathlineto{\pgfqpoint{4.639465in}{3.925333in}}%
\pgfpathlineto{\pgfqpoint{4.643450in}{3.921321in}}%
\pgfpathlineto{\pgfqpoint{4.647758in}{3.917060in}}%
\pgfpathlineto{\pgfqpoint{4.676814in}{3.888000in}}%
\pgfpathlineto{\pgfqpoint{4.682103in}{3.882658in}}%
\pgfpathlineto{\pgfqpoint{4.687838in}{3.876967in}}%
\pgfpathlineto{\pgfqpoint{4.701373in}{3.863274in}}%
\pgfpathlineto{\pgfqpoint{4.714056in}{3.850667in}}%
\pgfpathlineto{\pgfqpoint{4.727919in}{3.836751in}}%
\pgfpathlineto{\pgfqpoint{4.739952in}{3.824541in}}%
\pgfpathlineto{\pgfqpoint{4.751193in}{3.813333in}}%
\pgfpathlineto{\pgfqpoint{4.768000in}{3.796411in}}%
\pgfusepath{fill}%
\end{pgfscope}%
\begin{pgfscope}%
\pgfpathrectangle{\pgfqpoint{0.800000in}{0.528000in}}{\pgfqpoint{3.968000in}{3.696000in}}%
\pgfusepath{clip}%
\pgfsetbuttcap%
\pgfsetroundjoin%
\definecolor{currentfill}{rgb}{0.525776,0.833491,0.288127}%
\pgfsetfillcolor{currentfill}%
\pgfsetlinewidth{0.000000pt}%
\definecolor{currentstroke}{rgb}{0.000000,0.000000,0.000000}%
\pgfsetstrokecolor{currentstroke}%
\pgfsetdash{}{0pt}%
\pgfpathmoveto{\pgfqpoint{4.768000in}{3.801619in}}%
\pgfpathlineto{\pgfqpoint{4.756366in}{3.813333in}}%
\pgfpathlineto{\pgfqpoint{4.742626in}{3.827032in}}%
\pgfpathlineto{\pgfqpoint{4.727919in}{3.841956in}}%
\pgfpathlineto{\pgfqpoint{4.719242in}{3.850667in}}%
\pgfpathlineto{\pgfqpoint{4.704050in}{3.865767in}}%
\pgfpathlineto{\pgfqpoint{4.687838in}{3.882169in}}%
\pgfpathlineto{\pgfqpoint{4.684807in}{3.885177in}}%
\pgfpathlineto{\pgfqpoint{4.682012in}{3.888000in}}%
\pgfpathlineto{\pgfqpoint{4.647758in}{3.922258in}}%
\pgfpathlineto{\pgfqpoint{4.646156in}{3.923842in}}%
\pgfpathlineto{\pgfqpoint{4.644675in}{3.925333in}}%
\pgfpathlineto{\pgfqpoint{4.626721in}{3.943072in}}%
\pgfpathlineto{\pgfqpoint{4.607677in}{3.962223in}}%
\pgfpathlineto{\pgfqpoint{4.607231in}{3.962667in}}%
\pgfpathlineto{\pgfqpoint{4.587967in}{3.981641in}}%
\pgfpathlineto{\pgfqpoint{4.569656in}{4.000000in}}%
\pgfpathlineto{\pgfqpoint{4.568654in}{4.000985in}}%
\pgfpathlineto{\pgfqpoint{4.567596in}{4.002045in}}%
\pgfpathlineto{\pgfqpoint{4.549154in}{4.020155in}}%
\pgfpathlineto{\pgfqpoint{4.531967in}{4.037333in}}%
\pgfpathlineto{\pgfqpoint{4.527515in}{4.041740in}}%
\pgfpathlineto{\pgfqpoint{4.510280in}{4.058613in}}%
\pgfpathlineto{\pgfqpoint{4.494170in}{4.074667in}}%
\pgfpathlineto{\pgfqpoint{4.490884in}{4.077880in}}%
\pgfpathlineto{\pgfqpoint{4.487434in}{4.081314in}}%
\pgfpathlineto{\pgfqpoint{4.471347in}{4.097015in}}%
\pgfpathlineto{\pgfqpoint{4.456264in}{4.112000in}}%
\pgfpathlineto{\pgfqpoint{4.451910in}{4.116244in}}%
\pgfpathlineto{\pgfqpoint{4.447354in}{4.120766in}}%
\pgfpathlineto{\pgfqpoint{4.418246in}{4.149333in}}%
\pgfpathlineto{\pgfqpoint{4.412877in}{4.154553in}}%
\pgfpathlineto{\pgfqpoint{4.407273in}{4.160097in}}%
\pgfpathlineto{\pgfqpoint{4.380118in}{4.186667in}}%
\pgfpathlineto{\pgfqpoint{4.373783in}{4.192806in}}%
\pgfpathlineto{\pgfqpoint{4.367192in}{4.199306in}}%
\pgfpathlineto{\pgfqpoint{4.341878in}{4.224000in}}%
\pgfpathlineto{\pgfqpoint{4.339251in}{4.224000in}}%
\pgfpathlineto{\pgfqpoint{4.367192in}{4.196744in}}%
\pgfpathlineto{\pgfqpoint{4.372447in}{4.191561in}}%
\pgfpathlineto{\pgfqpoint{4.377498in}{4.186667in}}%
\pgfpathlineto{\pgfqpoint{4.407273in}{4.157533in}}%
\pgfpathlineto{\pgfqpoint{4.411542in}{4.153310in}}%
\pgfpathlineto{\pgfqpoint{4.415632in}{4.149333in}}%
\pgfpathlineto{\pgfqpoint{4.447354in}{4.118200in}}%
\pgfpathlineto{\pgfqpoint{4.450577in}{4.115002in}}%
\pgfpathlineto{\pgfqpoint{4.453656in}{4.112000in}}%
\pgfpathlineto{\pgfqpoint{4.470001in}{4.095762in}}%
\pgfpathlineto{\pgfqpoint{4.487434in}{4.078747in}}%
\pgfpathlineto{\pgfqpoint{4.489552in}{4.076639in}}%
\pgfpathlineto{\pgfqpoint{4.491569in}{4.074667in}}%
\pgfpathlineto{\pgfqpoint{4.508936in}{4.057361in}}%
\pgfpathlineto{\pgfqpoint{4.527515in}{4.039171in}}%
\pgfpathlineto{\pgfqpoint{4.529372in}{4.037333in}}%
\pgfpathlineto{\pgfqpoint{4.547810in}{4.018904in}}%
\pgfpathlineto{\pgfqpoint{4.567061in}{4.000000in}}%
\pgfpathlineto{\pgfqpoint{4.567319in}{3.999742in}}%
\pgfpathlineto{\pgfqpoint{4.567596in}{3.999469in}}%
\pgfpathlineto{\pgfqpoint{4.586625in}{3.980391in}}%
\pgfpathlineto{\pgfqpoint{4.604620in}{3.962667in}}%
\pgfpathlineto{\pgfqpoint{4.607677in}{3.959626in}}%
\pgfpathlineto{\pgfqpoint{4.625380in}{3.941823in}}%
\pgfpathlineto{\pgfqpoint{4.642070in}{3.925333in}}%
\pgfpathlineto{\pgfqpoint{4.644803in}{3.922581in}}%
\pgfpathlineto{\pgfqpoint{4.647758in}{3.919659in}}%
\pgfpathlineto{\pgfqpoint{4.679413in}{3.888000in}}%
\pgfpathlineto{\pgfqpoint{4.683455in}{3.883917in}}%
\pgfpathlineto{\pgfqpoint{4.687838in}{3.879568in}}%
\pgfpathlineto{\pgfqpoint{4.702712in}{3.864521in}}%
\pgfpathlineto{\pgfqpoint{4.716649in}{3.850667in}}%
\pgfpathlineto{\pgfqpoint{4.727919in}{3.839354in}}%
\pgfpathlineto{\pgfqpoint{4.741289in}{3.825787in}}%
\pgfpathlineto{\pgfqpoint{4.753779in}{3.813333in}}%
\pgfpathlineto{\pgfqpoint{4.768000in}{3.799015in}}%
\pgfusepath{fill}%
\end{pgfscope}%
\begin{pgfscope}%
\pgfpathrectangle{\pgfqpoint{0.800000in}{0.528000in}}{\pgfqpoint{3.968000in}{3.696000in}}%
\pgfusepath{clip}%
\pgfsetbuttcap%
\pgfsetroundjoin%
\definecolor{currentfill}{rgb}{0.525776,0.833491,0.288127}%
\pgfsetfillcolor{currentfill}%
\pgfsetlinewidth{0.000000pt}%
\definecolor{currentstroke}{rgb}{0.000000,0.000000,0.000000}%
\pgfsetstrokecolor{currentstroke}%
\pgfsetdash{}{0pt}%
\pgfpathmoveto{\pgfqpoint{4.768000in}{3.804223in}}%
\pgfpathlineto{\pgfqpoint{4.758952in}{3.813333in}}%
\pgfpathlineto{\pgfqpoint{4.743963in}{3.828278in}}%
\pgfpathlineto{\pgfqpoint{4.727919in}{3.844559in}}%
\pgfpathlineto{\pgfqpoint{4.721834in}{3.850667in}}%
\pgfpathlineto{\pgfqpoint{4.705389in}{3.867014in}}%
\pgfpathlineto{\pgfqpoint{4.687838in}{3.884770in}}%
\pgfpathlineto{\pgfqpoint{4.686159in}{3.886436in}}%
\pgfpathlineto{\pgfqpoint{4.684611in}{3.888000in}}%
\pgfpathlineto{\pgfqpoint{4.647758in}{3.924857in}}%
\pgfpathlineto{\pgfqpoint{4.647510in}{3.925102in}}%
\pgfpathlineto{\pgfqpoint{4.647280in}{3.925333in}}%
\pgfpathlineto{\pgfqpoint{4.628062in}{3.944321in}}%
\pgfpathlineto{\pgfqpoint{4.609819in}{3.962667in}}%
\pgfpathlineto{\pgfqpoint{4.607677in}{3.964800in}}%
\pgfpathlineto{\pgfqpoint{4.589309in}{3.982892in}}%
\pgfpathlineto{\pgfqpoint{4.572245in}{4.000000in}}%
\pgfpathlineto{\pgfqpoint{4.569984in}{4.002224in}}%
\pgfpathlineto{\pgfqpoint{4.567596in}{4.004615in}}%
\pgfpathlineto{\pgfqpoint{4.550497in}{4.021407in}}%
\pgfpathlineto{\pgfqpoint{4.534563in}{4.037333in}}%
\pgfpathlineto{\pgfqpoint{4.527515in}{4.044309in}}%
\pgfpathlineto{\pgfqpoint{4.511625in}{4.059866in}}%
\pgfpathlineto{\pgfqpoint{4.496772in}{4.074667in}}%
\pgfpathlineto{\pgfqpoint{4.492217in}{4.079121in}}%
\pgfpathlineto{\pgfqpoint{4.487434in}{4.083881in}}%
\pgfpathlineto{\pgfqpoint{4.472693in}{4.098269in}}%
\pgfpathlineto{\pgfqpoint{4.458871in}{4.112000in}}%
\pgfpathlineto{\pgfqpoint{4.453244in}{4.117487in}}%
\pgfpathlineto{\pgfqpoint{4.447354in}{4.123331in}}%
\pgfpathlineto{\pgfqpoint{4.420860in}{4.149333in}}%
\pgfpathlineto{\pgfqpoint{4.414211in}{4.155796in}}%
\pgfpathlineto{\pgfqpoint{4.407273in}{4.162661in}}%
\pgfpathlineto{\pgfqpoint{4.382738in}{4.186667in}}%
\pgfpathlineto{\pgfqpoint{4.375119in}{4.194050in}}%
\pgfpathlineto{\pgfqpoint{4.367192in}{4.201869in}}%
\pgfpathlineto{\pgfqpoint{4.344504in}{4.224000in}}%
\pgfpathlineto{\pgfqpoint{4.341878in}{4.224000in}}%
\pgfpathlineto{\pgfqpoint{4.367192in}{4.199306in}}%
\pgfpathlineto{\pgfqpoint{4.373783in}{4.192806in}}%
\pgfpathlineto{\pgfqpoint{4.380118in}{4.186667in}}%
\pgfpathlineto{\pgfqpoint{4.407273in}{4.160097in}}%
\pgfpathlineto{\pgfqpoint{4.412877in}{4.154553in}}%
\pgfpathlineto{\pgfqpoint{4.418246in}{4.149333in}}%
\pgfpathlineto{\pgfqpoint{4.447354in}{4.120766in}}%
\pgfpathlineto{\pgfqpoint{4.451910in}{4.116244in}}%
\pgfpathlineto{\pgfqpoint{4.456264in}{4.112000in}}%
\pgfpathlineto{\pgfqpoint{4.471347in}{4.097015in}}%
\pgfpathlineto{\pgfqpoint{4.487434in}{4.081314in}}%
\pgfpathlineto{\pgfqpoint{4.490884in}{4.077880in}}%
\pgfpathlineto{\pgfqpoint{4.494170in}{4.074667in}}%
\pgfpathlineto{\pgfqpoint{4.510280in}{4.058613in}}%
\pgfpathlineto{\pgfqpoint{4.527515in}{4.041740in}}%
\pgfpathlineto{\pgfqpoint{4.531967in}{4.037333in}}%
\pgfpathlineto{\pgfqpoint{4.549154in}{4.020155in}}%
\pgfpathlineto{\pgfqpoint{4.567596in}{4.002045in}}%
\pgfpathlineto{\pgfqpoint{4.568654in}{4.000985in}}%
\pgfpathlineto{\pgfqpoint{4.569656in}{4.000000in}}%
\pgfpathlineto{\pgfqpoint{4.587967in}{3.981641in}}%
\pgfpathlineto{\pgfqpoint{4.607231in}{3.962667in}}%
\pgfpathlineto{\pgfqpoint{4.607677in}{3.962223in}}%
\pgfpathlineto{\pgfqpoint{4.626721in}{3.943072in}}%
\pgfpathlineto{\pgfqpoint{4.644675in}{3.925333in}}%
\pgfpathlineto{\pgfqpoint{4.646156in}{3.923842in}}%
\pgfpathlineto{\pgfqpoint{4.647758in}{3.922258in}}%
\pgfpathlineto{\pgfqpoint{4.682012in}{3.888000in}}%
\pgfpathlineto{\pgfqpoint{4.684807in}{3.885177in}}%
\pgfpathlineto{\pgfqpoint{4.687838in}{3.882169in}}%
\pgfpathlineto{\pgfqpoint{4.704050in}{3.865767in}}%
\pgfpathlineto{\pgfqpoint{4.719242in}{3.850667in}}%
\pgfpathlineto{\pgfqpoint{4.727919in}{3.841956in}}%
\pgfpathlineto{\pgfqpoint{4.742626in}{3.827032in}}%
\pgfpathlineto{\pgfqpoint{4.756366in}{3.813333in}}%
\pgfpathlineto{\pgfqpoint{4.768000in}{3.801619in}}%
\pgfusepath{fill}%
\end{pgfscope}%
\begin{pgfscope}%
\pgfpathrectangle{\pgfqpoint{0.800000in}{0.528000in}}{\pgfqpoint{3.968000in}{3.696000in}}%
\pgfusepath{clip}%
\pgfsetbuttcap%
\pgfsetroundjoin%
\definecolor{currentfill}{rgb}{0.535621,0.835785,0.281908}%
\pgfsetfillcolor{currentfill}%
\pgfsetlinewidth{0.000000pt}%
\definecolor{currentstroke}{rgb}{0.000000,0.000000,0.000000}%
\pgfsetstrokecolor{currentstroke}%
\pgfsetdash{}{0pt}%
\pgfpathmoveto{\pgfqpoint{4.768000in}{3.806827in}}%
\pgfpathlineto{\pgfqpoint{4.761538in}{3.813333in}}%
\pgfpathlineto{\pgfqpoint{4.745301in}{3.829523in}}%
\pgfpathlineto{\pgfqpoint{4.727919in}{3.847161in}}%
\pgfpathlineto{\pgfqpoint{4.724427in}{3.850667in}}%
\pgfpathlineto{\pgfqpoint{4.706727in}{3.868261in}}%
\pgfpathlineto{\pgfqpoint{4.687838in}{3.887371in}}%
\pgfpathlineto{\pgfqpoint{4.687511in}{3.887695in}}%
\pgfpathlineto{\pgfqpoint{4.687209in}{3.888000in}}%
\pgfpathlineto{\pgfqpoint{4.678676in}{3.896535in}}%
\pgfpathlineto{\pgfqpoint{4.649862in}{3.925333in}}%
\pgfpathlineto{\pgfqpoint{4.648842in}{3.926343in}}%
\pgfpathlineto{\pgfqpoint{4.647758in}{3.927435in}}%
\pgfpathlineto{\pgfqpoint{4.629403in}{3.945570in}}%
\pgfpathlineto{\pgfqpoint{4.612401in}{3.962667in}}%
\pgfpathlineto{\pgfqpoint{4.607677in}{3.967372in}}%
\pgfpathlineto{\pgfqpoint{4.590651in}{3.984142in}}%
\pgfpathlineto{\pgfqpoint{4.574834in}{4.000000in}}%
\pgfpathlineto{\pgfqpoint{4.571314in}{4.003463in}}%
\pgfpathlineto{\pgfqpoint{4.567596in}{4.007186in}}%
\pgfpathlineto{\pgfqpoint{4.551840in}{4.022658in}}%
\pgfpathlineto{\pgfqpoint{4.537158in}{4.037333in}}%
\pgfpathlineto{\pgfqpoint{4.527515in}{4.046878in}}%
\pgfpathlineto{\pgfqpoint{4.512970in}{4.061118in}}%
\pgfpathlineto{\pgfqpoint{4.499373in}{4.074667in}}%
\pgfpathlineto{\pgfqpoint{4.493549in}{4.080362in}}%
\pgfpathlineto{\pgfqpoint{4.487434in}{4.086448in}}%
\pgfpathlineto{\pgfqpoint{4.474039in}{4.099523in}}%
\pgfpathlineto{\pgfqpoint{4.461479in}{4.112000in}}%
\pgfpathlineto{\pgfqpoint{4.454578in}{4.118729in}}%
\pgfpathlineto{\pgfqpoint{4.447354in}{4.125897in}}%
\pgfpathlineto{\pgfqpoint{4.423474in}{4.149333in}}%
\pgfpathlineto{\pgfqpoint{4.415546in}{4.157040in}}%
\pgfpathlineto{\pgfqpoint{4.407273in}{4.165224in}}%
\pgfpathlineto{\pgfqpoint{4.385359in}{4.186667in}}%
\pgfpathlineto{\pgfqpoint{4.376455in}{4.195295in}}%
\pgfpathlineto{\pgfqpoint{4.367192in}{4.204431in}}%
\pgfpathlineto{\pgfqpoint{4.347131in}{4.224000in}}%
\pgfpathlineto{\pgfqpoint{4.344504in}{4.224000in}}%
\pgfpathlineto{\pgfqpoint{4.367192in}{4.201869in}}%
\pgfpathlineto{\pgfqpoint{4.375119in}{4.194050in}}%
\pgfpathlineto{\pgfqpoint{4.382738in}{4.186667in}}%
\pgfpathlineto{\pgfqpoint{4.407273in}{4.162661in}}%
\pgfpathlineto{\pgfqpoint{4.414211in}{4.155796in}}%
\pgfpathlineto{\pgfqpoint{4.420860in}{4.149333in}}%
\pgfpathlineto{\pgfqpoint{4.447354in}{4.123331in}}%
\pgfpathlineto{\pgfqpoint{4.453244in}{4.117487in}}%
\pgfpathlineto{\pgfqpoint{4.458871in}{4.112000in}}%
\pgfpathlineto{\pgfqpoint{4.472693in}{4.098269in}}%
\pgfpathlineto{\pgfqpoint{4.487434in}{4.083881in}}%
\pgfpathlineto{\pgfqpoint{4.492217in}{4.079121in}}%
\pgfpathlineto{\pgfqpoint{4.496772in}{4.074667in}}%
\pgfpathlineto{\pgfqpoint{4.511625in}{4.059866in}}%
\pgfpathlineto{\pgfqpoint{4.527515in}{4.044309in}}%
\pgfpathlineto{\pgfqpoint{4.534563in}{4.037333in}}%
\pgfpathlineto{\pgfqpoint{4.550497in}{4.021407in}}%
\pgfpathlineto{\pgfqpoint{4.567596in}{4.004615in}}%
\pgfpathlineto{\pgfqpoint{4.569984in}{4.002224in}}%
\pgfpathlineto{\pgfqpoint{4.572245in}{4.000000in}}%
\pgfpathlineto{\pgfqpoint{4.589309in}{3.982892in}}%
\pgfpathlineto{\pgfqpoint{4.607677in}{3.964800in}}%
\pgfpathlineto{\pgfqpoint{4.609819in}{3.962667in}}%
\pgfpathlineto{\pgfqpoint{4.628062in}{3.944321in}}%
\pgfpathlineto{\pgfqpoint{4.647280in}{3.925333in}}%
\pgfpathlineto{\pgfqpoint{4.647510in}{3.925102in}}%
\pgfpathlineto{\pgfqpoint{4.647758in}{3.924857in}}%
\pgfpathlineto{\pgfqpoint{4.684611in}{3.888000in}}%
\pgfpathlineto{\pgfqpoint{4.686159in}{3.886436in}}%
\pgfpathlineto{\pgfqpoint{4.687838in}{3.884770in}}%
\pgfpathlineto{\pgfqpoint{4.705389in}{3.867014in}}%
\pgfpathlineto{\pgfqpoint{4.721834in}{3.850667in}}%
\pgfpathlineto{\pgfqpoint{4.727919in}{3.844559in}}%
\pgfpathlineto{\pgfqpoint{4.743963in}{3.828278in}}%
\pgfpathlineto{\pgfqpoint{4.758952in}{3.813333in}}%
\pgfpathlineto{\pgfqpoint{4.768000in}{3.804223in}}%
\pgfusepath{fill}%
\end{pgfscope}%
\begin{pgfscope}%
\pgfpathrectangle{\pgfqpoint{0.800000in}{0.528000in}}{\pgfqpoint{3.968000in}{3.696000in}}%
\pgfusepath{clip}%
\pgfsetbuttcap%
\pgfsetroundjoin%
\definecolor{currentfill}{rgb}{0.535621,0.835785,0.281908}%
\pgfsetfillcolor{currentfill}%
\pgfsetlinewidth{0.000000pt}%
\definecolor{currentstroke}{rgb}{0.000000,0.000000,0.000000}%
\pgfsetstrokecolor{currentstroke}%
\pgfsetdash{}{0pt}%
\pgfpathmoveto{\pgfqpoint{4.768000in}{3.809432in}}%
\pgfpathlineto{\pgfqpoint{4.764125in}{3.813333in}}%
\pgfpathlineto{\pgfqpoint{4.746638in}{3.830769in}}%
\pgfpathlineto{\pgfqpoint{4.727919in}{3.849764in}}%
\pgfpathlineto{\pgfqpoint{4.727019in}{3.850667in}}%
\pgfpathlineto{\pgfqpoint{4.708066in}{3.869507in}}%
\pgfpathlineto{\pgfqpoint{4.689787in}{3.888000in}}%
\pgfpathlineto{\pgfqpoint{4.687838in}{3.889952in}}%
\pgfpathlineto{\pgfqpoint{4.652439in}{3.925333in}}%
\pgfpathlineto{\pgfqpoint{4.650169in}{3.927580in}}%
\pgfpathlineto{\pgfqpoint{4.647758in}{3.930009in}}%
\pgfpathlineto{\pgfqpoint{4.630744in}{3.946819in}}%
\pgfpathlineto{\pgfqpoint{4.614984in}{3.962667in}}%
\pgfpathlineto{\pgfqpoint{4.607677in}{3.969944in}}%
\pgfpathlineto{\pgfqpoint{4.591994in}{3.985392in}}%
\pgfpathlineto{\pgfqpoint{4.577423in}{4.000000in}}%
\pgfpathlineto{\pgfqpoint{4.572644in}{4.004702in}}%
\pgfpathlineto{\pgfqpoint{4.567596in}{4.009756in}}%
\pgfpathlineto{\pgfqpoint{4.553184in}{4.023909in}}%
\pgfpathlineto{\pgfqpoint{4.539753in}{4.037333in}}%
\pgfpathlineto{\pgfqpoint{4.527515in}{4.049447in}}%
\pgfpathlineto{\pgfqpoint{4.514314in}{4.062371in}}%
\pgfpathlineto{\pgfqpoint{4.501975in}{4.074667in}}%
\pgfpathlineto{\pgfqpoint{4.494882in}{4.081603in}}%
\pgfpathlineto{\pgfqpoint{4.487434in}{4.089015in}}%
\pgfpathlineto{\pgfqpoint{4.475385in}{4.100776in}}%
\pgfpathlineto{\pgfqpoint{4.464087in}{4.112000in}}%
\pgfpathlineto{\pgfqpoint{4.455911in}{4.119971in}}%
\pgfpathlineto{\pgfqpoint{4.447354in}{4.128462in}}%
\pgfpathlineto{\pgfqpoint{4.426088in}{4.149333in}}%
\pgfpathlineto{\pgfqpoint{4.416881in}{4.158283in}}%
\pgfpathlineto{\pgfqpoint{4.407273in}{4.167788in}}%
\pgfpathlineto{\pgfqpoint{4.387979in}{4.186667in}}%
\pgfpathlineto{\pgfqpoint{4.377791in}{4.196539in}}%
\pgfpathlineto{\pgfqpoint{4.367192in}{4.206993in}}%
\pgfpathlineto{\pgfqpoint{4.349758in}{4.224000in}}%
\pgfpathlineto{\pgfqpoint{4.347131in}{4.224000in}}%
\pgfpathlineto{\pgfqpoint{4.367192in}{4.204431in}}%
\pgfpathlineto{\pgfqpoint{4.376455in}{4.195295in}}%
\pgfpathlineto{\pgfqpoint{4.385359in}{4.186667in}}%
\pgfpathlineto{\pgfqpoint{4.407273in}{4.165224in}}%
\pgfpathlineto{\pgfqpoint{4.415546in}{4.157040in}}%
\pgfpathlineto{\pgfqpoint{4.423474in}{4.149333in}}%
\pgfpathlineto{\pgfqpoint{4.447354in}{4.125897in}}%
\pgfpathlineto{\pgfqpoint{4.454578in}{4.118729in}}%
\pgfpathlineto{\pgfqpoint{4.461479in}{4.112000in}}%
\pgfpathlineto{\pgfqpoint{4.474039in}{4.099523in}}%
\pgfpathlineto{\pgfqpoint{4.487434in}{4.086448in}}%
\pgfpathlineto{\pgfqpoint{4.493549in}{4.080362in}}%
\pgfpathlineto{\pgfqpoint{4.499373in}{4.074667in}}%
\pgfpathlineto{\pgfqpoint{4.512970in}{4.061118in}}%
\pgfpathlineto{\pgfqpoint{4.527515in}{4.046878in}}%
\pgfpathlineto{\pgfqpoint{4.537158in}{4.037333in}}%
\pgfpathlineto{\pgfqpoint{4.551840in}{4.022658in}}%
\pgfpathlineto{\pgfqpoint{4.567596in}{4.007186in}}%
\pgfpathlineto{\pgfqpoint{4.571314in}{4.003463in}}%
\pgfpathlineto{\pgfqpoint{4.574834in}{4.000000in}}%
\pgfpathlineto{\pgfqpoint{4.590651in}{3.984142in}}%
\pgfpathlineto{\pgfqpoint{4.607677in}{3.967372in}}%
\pgfpathlineto{\pgfqpoint{4.612401in}{3.962667in}}%
\pgfpathlineto{\pgfqpoint{4.629403in}{3.945570in}}%
\pgfpathlineto{\pgfqpoint{4.647758in}{3.927435in}}%
\pgfpathlineto{\pgfqpoint{4.648842in}{3.926343in}}%
\pgfpathlineto{\pgfqpoint{4.649862in}{3.925333in}}%
\pgfpathlineto{\pgfqpoint{4.678676in}{3.896535in}}%
\pgfpathlineto{\pgfqpoint{4.687209in}{3.888000in}}%
\pgfpathlineto{\pgfqpoint{4.687511in}{3.887695in}}%
\pgfpathlineto{\pgfqpoint{4.687838in}{3.887371in}}%
\pgfpathlineto{\pgfqpoint{4.706727in}{3.868261in}}%
\pgfpathlineto{\pgfqpoint{4.724427in}{3.850667in}}%
\pgfpathlineto{\pgfqpoint{4.727919in}{3.847161in}}%
\pgfpathlineto{\pgfqpoint{4.745301in}{3.829523in}}%
\pgfpathlineto{\pgfqpoint{4.761538in}{3.813333in}}%
\pgfpathlineto{\pgfqpoint{4.768000in}{3.806827in}}%
\pgfusepath{fill}%
\end{pgfscope}%
\begin{pgfscope}%
\pgfpathrectangle{\pgfqpoint{0.800000in}{0.528000in}}{\pgfqpoint{3.968000in}{3.696000in}}%
\pgfusepath{clip}%
\pgfsetbuttcap%
\pgfsetroundjoin%
\definecolor{currentfill}{rgb}{0.535621,0.835785,0.281908}%
\pgfsetfillcolor{currentfill}%
\pgfsetlinewidth{0.000000pt}%
\definecolor{currentstroke}{rgb}{0.000000,0.000000,0.000000}%
\pgfsetstrokecolor{currentstroke}%
\pgfsetdash{}{0pt}%
\pgfpathmoveto{\pgfqpoint{4.768000in}{3.812036in}}%
\pgfpathlineto{\pgfqpoint{4.766711in}{3.813333in}}%
\pgfpathlineto{\pgfqpoint{4.747975in}{3.832014in}}%
\pgfpathlineto{\pgfqpoint{4.729594in}{3.850667in}}%
\pgfpathlineto{\pgfqpoint{4.727919in}{3.852349in}}%
\pgfpathlineto{\pgfqpoint{4.709404in}{3.870754in}}%
\pgfpathlineto{\pgfqpoint{4.692358in}{3.888000in}}%
\pgfpathlineto{\pgfqpoint{4.687838in}{3.892528in}}%
\pgfpathlineto{\pgfqpoint{4.655016in}{3.925333in}}%
\pgfpathlineto{\pgfqpoint{4.651497in}{3.928816in}}%
\pgfpathlineto{\pgfqpoint{4.647758in}{3.932583in}}%
\pgfpathlineto{\pgfqpoint{4.632085in}{3.948068in}}%
\pgfpathlineto{\pgfqpoint{4.617567in}{3.962667in}}%
\pgfpathlineto{\pgfqpoint{4.607677in}{3.972516in}}%
\pgfpathlineto{\pgfqpoint{4.593336in}{3.986642in}}%
\pgfpathlineto{\pgfqpoint{4.580012in}{4.000000in}}%
\pgfpathlineto{\pgfqpoint{4.573974in}{4.005941in}}%
\pgfpathlineto{\pgfqpoint{4.567596in}{4.012327in}}%
\pgfpathlineto{\pgfqpoint{4.554527in}{4.025160in}}%
\pgfpathlineto{\pgfqpoint{4.542348in}{4.037333in}}%
\pgfpathlineto{\pgfqpoint{4.527515in}{4.052015in}}%
\pgfpathlineto{\pgfqpoint{4.515659in}{4.063623in}}%
\pgfpathlineto{\pgfqpoint{4.504576in}{4.074667in}}%
\pgfpathlineto{\pgfqpoint{4.496214in}{4.082845in}}%
\pgfpathlineto{\pgfqpoint{4.487434in}{4.091583in}}%
\pgfpathlineto{\pgfqpoint{4.476730in}{4.102030in}}%
\pgfpathlineto{\pgfqpoint{4.466694in}{4.112000in}}%
\pgfpathlineto{\pgfqpoint{4.457245in}{4.121213in}}%
\pgfpathlineto{\pgfqpoint{4.447354in}{4.131028in}}%
\pgfpathlineto{\pgfqpoint{4.428702in}{4.149333in}}%
\pgfpathlineto{\pgfqpoint{4.418216in}{4.159527in}}%
\pgfpathlineto{\pgfqpoint{4.407273in}{4.170352in}}%
\pgfpathlineto{\pgfqpoint{4.390599in}{4.186667in}}%
\pgfpathlineto{\pgfqpoint{4.379127in}{4.197784in}}%
\pgfpathlineto{\pgfqpoint{4.367192in}{4.209555in}}%
\pgfpathlineto{\pgfqpoint{4.352384in}{4.224000in}}%
\pgfpathlineto{\pgfqpoint{4.349758in}{4.224000in}}%
\pgfpathlineto{\pgfqpoint{4.367192in}{4.206993in}}%
\pgfpathlineto{\pgfqpoint{4.377791in}{4.196539in}}%
\pgfpathlineto{\pgfqpoint{4.387979in}{4.186667in}}%
\pgfpathlineto{\pgfqpoint{4.407273in}{4.167788in}}%
\pgfpathlineto{\pgfqpoint{4.416881in}{4.158283in}}%
\pgfpathlineto{\pgfqpoint{4.426088in}{4.149333in}}%
\pgfpathlineto{\pgfqpoint{4.447354in}{4.128462in}}%
\pgfpathlineto{\pgfqpoint{4.455911in}{4.119971in}}%
\pgfpathlineto{\pgfqpoint{4.464087in}{4.112000in}}%
\pgfpathlineto{\pgfqpoint{4.475385in}{4.100776in}}%
\pgfpathlineto{\pgfqpoint{4.487434in}{4.089015in}}%
\pgfpathlineto{\pgfqpoint{4.494882in}{4.081603in}}%
\pgfpathlineto{\pgfqpoint{4.501975in}{4.074667in}}%
\pgfpathlineto{\pgfqpoint{4.514314in}{4.062371in}}%
\pgfpathlineto{\pgfqpoint{4.527515in}{4.049447in}}%
\pgfpathlineto{\pgfqpoint{4.539753in}{4.037333in}}%
\pgfpathlineto{\pgfqpoint{4.553184in}{4.023909in}}%
\pgfpathlineto{\pgfqpoint{4.567596in}{4.009756in}}%
\pgfpathlineto{\pgfqpoint{4.572644in}{4.004702in}}%
\pgfpathlineto{\pgfqpoint{4.577423in}{4.000000in}}%
\pgfpathlineto{\pgfqpoint{4.591994in}{3.985392in}}%
\pgfpathlineto{\pgfqpoint{4.607677in}{3.969944in}}%
\pgfpathlineto{\pgfqpoint{4.614984in}{3.962667in}}%
\pgfpathlineto{\pgfqpoint{4.630744in}{3.946819in}}%
\pgfpathlineto{\pgfqpoint{4.647758in}{3.930009in}}%
\pgfpathlineto{\pgfqpoint{4.650169in}{3.927580in}}%
\pgfpathlineto{\pgfqpoint{4.652439in}{3.925333in}}%
\pgfpathlineto{\pgfqpoint{4.687838in}{3.889952in}}%
\pgfpathlineto{\pgfqpoint{4.689787in}{3.888000in}}%
\pgfpathlineto{\pgfqpoint{4.708066in}{3.869507in}}%
\pgfpathlineto{\pgfqpoint{4.727019in}{3.850667in}}%
\pgfpathlineto{\pgfqpoint{4.727919in}{3.849764in}}%
\pgfpathlineto{\pgfqpoint{4.746638in}{3.830769in}}%
\pgfpathlineto{\pgfqpoint{4.764125in}{3.813333in}}%
\pgfpathlineto{\pgfqpoint{4.768000in}{3.809432in}}%
\pgfusepath{fill}%
\end{pgfscope}%
\begin{pgfscope}%
\pgfpathrectangle{\pgfqpoint{0.800000in}{0.528000in}}{\pgfqpoint{3.968000in}{3.696000in}}%
\pgfusepath{clip}%
\pgfsetbuttcap%
\pgfsetroundjoin%
\definecolor{currentfill}{rgb}{0.545524,0.838039,0.275626}%
\pgfsetfillcolor{currentfill}%
\pgfsetlinewidth{0.000000pt}%
\definecolor{currentstroke}{rgb}{0.000000,0.000000,0.000000}%
\pgfsetstrokecolor{currentstroke}%
\pgfsetdash{}{0pt}%
\pgfpathmoveto{\pgfqpoint{4.768000in}{3.814627in}}%
\pgfpathlineto{\pgfqpoint{4.749312in}{3.833260in}}%
\pgfpathlineto{\pgfqpoint{4.732158in}{3.850667in}}%
\pgfpathlineto{\pgfqpoint{4.727919in}{3.854927in}}%
\pgfpathlineto{\pgfqpoint{4.710742in}{3.872001in}}%
\pgfpathlineto{\pgfqpoint{4.694928in}{3.888000in}}%
\pgfpathlineto{\pgfqpoint{4.687838in}{3.895103in}}%
\pgfpathlineto{\pgfqpoint{4.657592in}{3.925333in}}%
\pgfpathlineto{\pgfqpoint{4.652824in}{3.930053in}}%
\pgfpathlineto{\pgfqpoint{4.647758in}{3.935157in}}%
\pgfpathlineto{\pgfqpoint{4.633426in}{3.949317in}}%
\pgfpathlineto{\pgfqpoint{4.620150in}{3.962667in}}%
\pgfpathlineto{\pgfqpoint{4.607677in}{3.975088in}}%
\pgfpathlineto{\pgfqpoint{4.594678in}{3.987892in}}%
\pgfpathlineto{\pgfqpoint{4.582601in}{4.000000in}}%
\pgfpathlineto{\pgfqpoint{4.575304in}{4.007180in}}%
\pgfpathlineto{\pgfqpoint{4.567596in}{4.014897in}}%
\pgfpathlineto{\pgfqpoint{4.555871in}{4.026412in}}%
\pgfpathlineto{\pgfqpoint{4.544944in}{4.037333in}}%
\pgfpathlineto{\pgfqpoint{4.527515in}{4.054584in}}%
\pgfpathlineto{\pgfqpoint{4.517003in}{4.064875in}}%
\pgfpathlineto{\pgfqpoint{4.507178in}{4.074667in}}%
\pgfpathlineto{\pgfqpoint{4.497546in}{4.084086in}}%
\pgfpathlineto{\pgfqpoint{4.487434in}{4.094150in}}%
\pgfpathlineto{\pgfqpoint{4.478076in}{4.103283in}}%
\pgfpathlineto{\pgfqpoint{4.469302in}{4.112000in}}%
\pgfpathlineto{\pgfqpoint{4.458579in}{4.122456in}}%
\pgfpathlineto{\pgfqpoint{4.447354in}{4.133594in}}%
\pgfpathlineto{\pgfqpoint{4.431316in}{4.149333in}}%
\pgfpathlineto{\pgfqpoint{4.419551in}{4.160770in}}%
\pgfpathlineto{\pgfqpoint{4.407273in}{4.172916in}}%
\pgfpathlineto{\pgfqpoint{4.393220in}{4.186667in}}%
\pgfpathlineto{\pgfqpoint{4.380463in}{4.199028in}}%
\pgfpathlineto{\pgfqpoint{4.367192in}{4.212118in}}%
\pgfpathlineto{\pgfqpoint{4.355011in}{4.224000in}}%
\pgfpathlineto{\pgfqpoint{4.352384in}{4.224000in}}%
\pgfpathlineto{\pgfqpoint{4.367192in}{4.209555in}}%
\pgfpathlineto{\pgfqpoint{4.379127in}{4.197784in}}%
\pgfpathlineto{\pgfqpoint{4.390599in}{4.186667in}}%
\pgfpathlineto{\pgfqpoint{4.407273in}{4.170352in}}%
\pgfpathlineto{\pgfqpoint{4.418216in}{4.159527in}}%
\pgfpathlineto{\pgfqpoint{4.428702in}{4.149333in}}%
\pgfpathlineto{\pgfqpoint{4.447354in}{4.131028in}}%
\pgfpathlineto{\pgfqpoint{4.457245in}{4.121213in}}%
\pgfpathlineto{\pgfqpoint{4.466694in}{4.112000in}}%
\pgfpathlineto{\pgfqpoint{4.476730in}{4.102030in}}%
\pgfpathlineto{\pgfqpoint{4.487434in}{4.091583in}}%
\pgfpathlineto{\pgfqpoint{4.496214in}{4.082845in}}%
\pgfpathlineto{\pgfqpoint{4.504576in}{4.074667in}}%
\pgfpathlineto{\pgfqpoint{4.515659in}{4.063623in}}%
\pgfpathlineto{\pgfqpoint{4.527515in}{4.052015in}}%
\pgfpathlineto{\pgfqpoint{4.542348in}{4.037333in}}%
\pgfpathlineto{\pgfqpoint{4.554527in}{4.025160in}}%
\pgfpathlineto{\pgfqpoint{4.567596in}{4.012327in}}%
\pgfpathlineto{\pgfqpoint{4.573974in}{4.005941in}}%
\pgfpathlineto{\pgfqpoint{4.580012in}{4.000000in}}%
\pgfpathlineto{\pgfqpoint{4.593336in}{3.986642in}}%
\pgfpathlineto{\pgfqpoint{4.607677in}{3.972516in}}%
\pgfpathlineto{\pgfqpoint{4.617567in}{3.962667in}}%
\pgfpathlineto{\pgfqpoint{4.632085in}{3.948068in}}%
\pgfpathlineto{\pgfqpoint{4.647758in}{3.932583in}}%
\pgfpathlineto{\pgfqpoint{4.651497in}{3.928816in}}%
\pgfpathlineto{\pgfqpoint{4.655016in}{3.925333in}}%
\pgfpathlineto{\pgfqpoint{4.687838in}{3.892528in}}%
\pgfpathlineto{\pgfqpoint{4.692358in}{3.888000in}}%
\pgfpathlineto{\pgfqpoint{4.709404in}{3.870754in}}%
\pgfpathlineto{\pgfqpoint{4.727919in}{3.852349in}}%
\pgfpathlineto{\pgfqpoint{4.729594in}{3.850667in}}%
\pgfpathlineto{\pgfqpoint{4.747975in}{3.832014in}}%
\pgfpathlineto{\pgfqpoint{4.766711in}{3.813333in}}%
\pgfpathlineto{\pgfqpoint{4.768000in}{3.812036in}}%
\pgfpathlineto{\pgfqpoint{4.768000in}{3.813333in}}%
\pgfusepath{fill}%
\end{pgfscope}%
\begin{pgfscope}%
\pgfpathrectangle{\pgfqpoint{0.800000in}{0.528000in}}{\pgfqpoint{3.968000in}{3.696000in}}%
\pgfusepath{clip}%
\pgfsetbuttcap%
\pgfsetroundjoin%
\definecolor{currentfill}{rgb}{0.545524,0.838039,0.275626}%
\pgfsetfillcolor{currentfill}%
\pgfsetlinewidth{0.000000pt}%
\definecolor{currentstroke}{rgb}{0.000000,0.000000,0.000000}%
\pgfsetstrokecolor{currentstroke}%
\pgfsetdash{}{0pt}%
\pgfpathmoveto{\pgfqpoint{4.768000in}{3.817206in}}%
\pgfpathlineto{\pgfqpoint{4.750649in}{3.834505in}}%
\pgfpathlineto{\pgfqpoint{4.734723in}{3.850667in}}%
\pgfpathlineto{\pgfqpoint{4.727919in}{3.857504in}}%
\pgfpathlineto{\pgfqpoint{4.712081in}{3.873247in}}%
\pgfpathlineto{\pgfqpoint{4.697499in}{3.888000in}}%
\pgfpathlineto{\pgfqpoint{4.687838in}{3.897678in}}%
\pgfpathlineto{\pgfqpoint{4.660169in}{3.925333in}}%
\pgfpathlineto{\pgfqpoint{4.654152in}{3.931289in}}%
\pgfpathlineto{\pgfqpoint{4.647758in}{3.937731in}}%
\pgfpathlineto{\pgfqpoint{4.634766in}{3.950566in}}%
\pgfpathlineto{\pgfqpoint{4.622733in}{3.962667in}}%
\pgfpathlineto{\pgfqpoint{4.607677in}{3.977660in}}%
\pgfpathlineto{\pgfqpoint{4.596020in}{3.989142in}}%
\pgfpathlineto{\pgfqpoint{4.585190in}{4.000000in}}%
\pgfpathlineto{\pgfqpoint{4.576634in}{4.008418in}}%
\pgfpathlineto{\pgfqpoint{4.567596in}{4.017468in}}%
\pgfpathlineto{\pgfqpoint{4.557214in}{4.027663in}}%
\pgfpathlineto{\pgfqpoint{4.547539in}{4.037333in}}%
\pgfpathlineto{\pgfqpoint{4.527515in}{4.057153in}}%
\pgfpathlineto{\pgfqpoint{4.518348in}{4.066128in}}%
\pgfpathlineto{\pgfqpoint{4.509779in}{4.074667in}}%
\pgfpathlineto{\pgfqpoint{4.498879in}{4.085327in}}%
\pgfpathlineto{\pgfqpoint{4.487434in}{4.096717in}}%
\pgfpathlineto{\pgfqpoint{4.479422in}{4.104537in}}%
\pgfpathlineto{\pgfqpoint{4.471910in}{4.112000in}}%
\pgfpathlineto{\pgfqpoint{4.459912in}{4.123698in}}%
\pgfpathlineto{\pgfqpoint{4.447354in}{4.136159in}}%
\pgfpathlineto{\pgfqpoint{4.433930in}{4.149333in}}%
\pgfpathlineto{\pgfqpoint{4.420886in}{4.162013in}}%
\pgfpathlineto{\pgfqpoint{4.407273in}{4.175480in}}%
\pgfpathlineto{\pgfqpoint{4.395840in}{4.186667in}}%
\pgfpathlineto{\pgfqpoint{4.381800in}{4.200273in}}%
\pgfpathlineto{\pgfqpoint{4.367192in}{4.214680in}}%
\pgfpathlineto{\pgfqpoint{4.357638in}{4.224000in}}%
\pgfpathlineto{\pgfqpoint{4.355011in}{4.224000in}}%
\pgfpathlineto{\pgfqpoint{4.367192in}{4.212118in}}%
\pgfpathlineto{\pgfqpoint{4.380463in}{4.199028in}}%
\pgfpathlineto{\pgfqpoint{4.393220in}{4.186667in}}%
\pgfpathlineto{\pgfqpoint{4.407273in}{4.172916in}}%
\pgfpathlineto{\pgfqpoint{4.419551in}{4.160770in}}%
\pgfpathlineto{\pgfqpoint{4.431316in}{4.149333in}}%
\pgfpathlineto{\pgfqpoint{4.447354in}{4.133594in}}%
\pgfpathlineto{\pgfqpoint{4.458579in}{4.122456in}}%
\pgfpathlineto{\pgfqpoint{4.469302in}{4.112000in}}%
\pgfpathlineto{\pgfqpoint{4.478076in}{4.103283in}}%
\pgfpathlineto{\pgfqpoint{4.487434in}{4.094150in}}%
\pgfpathlineto{\pgfqpoint{4.497546in}{4.084086in}}%
\pgfpathlineto{\pgfqpoint{4.507178in}{4.074667in}}%
\pgfpathlineto{\pgfqpoint{4.517003in}{4.064875in}}%
\pgfpathlineto{\pgfqpoint{4.527515in}{4.054584in}}%
\pgfpathlineto{\pgfqpoint{4.544944in}{4.037333in}}%
\pgfpathlineto{\pgfqpoint{4.555871in}{4.026412in}}%
\pgfpathlineto{\pgfqpoint{4.567596in}{4.014897in}}%
\pgfpathlineto{\pgfqpoint{4.575304in}{4.007180in}}%
\pgfpathlineto{\pgfqpoint{4.582601in}{4.000000in}}%
\pgfpathlineto{\pgfqpoint{4.594678in}{3.987892in}}%
\pgfpathlineto{\pgfqpoint{4.607677in}{3.975088in}}%
\pgfpathlineto{\pgfqpoint{4.620150in}{3.962667in}}%
\pgfpathlineto{\pgfqpoint{4.633426in}{3.949317in}}%
\pgfpathlineto{\pgfqpoint{4.647758in}{3.935157in}}%
\pgfpathlineto{\pgfqpoint{4.652824in}{3.930053in}}%
\pgfpathlineto{\pgfqpoint{4.657592in}{3.925333in}}%
\pgfpathlineto{\pgfqpoint{4.687838in}{3.895103in}}%
\pgfpathlineto{\pgfqpoint{4.694928in}{3.888000in}}%
\pgfpathlineto{\pgfqpoint{4.710742in}{3.872001in}}%
\pgfpathlineto{\pgfqpoint{4.727919in}{3.854927in}}%
\pgfpathlineto{\pgfqpoint{4.732158in}{3.850667in}}%
\pgfpathlineto{\pgfqpoint{4.749312in}{3.833260in}}%
\pgfpathlineto{\pgfqpoint{4.768000in}{3.814627in}}%
\pgfusepath{fill}%
\end{pgfscope}%
\begin{pgfscope}%
\pgfpathrectangle{\pgfqpoint{0.800000in}{0.528000in}}{\pgfqpoint{3.968000in}{3.696000in}}%
\pgfusepath{clip}%
\pgfsetbuttcap%
\pgfsetroundjoin%
\definecolor{currentfill}{rgb}{0.545524,0.838039,0.275626}%
\pgfsetfillcolor{currentfill}%
\pgfsetlinewidth{0.000000pt}%
\definecolor{currentstroke}{rgb}{0.000000,0.000000,0.000000}%
\pgfsetstrokecolor{currentstroke}%
\pgfsetdash{}{0pt}%
\pgfpathmoveto{\pgfqpoint{4.768000in}{3.819785in}}%
\pgfpathlineto{\pgfqpoint{4.751987in}{3.835751in}}%
\pgfpathlineto{\pgfqpoint{4.737288in}{3.850667in}}%
\pgfpathlineto{\pgfqpoint{4.727919in}{3.860081in}}%
\pgfpathlineto{\pgfqpoint{4.713419in}{3.874494in}}%
\pgfpathlineto{\pgfqpoint{4.700070in}{3.888000in}}%
\pgfpathlineto{\pgfqpoint{4.687838in}{3.900254in}}%
\pgfpathlineto{\pgfqpoint{4.662746in}{3.925333in}}%
\pgfpathlineto{\pgfqpoint{4.655480in}{3.932526in}}%
\pgfpathlineto{\pgfqpoint{4.647758in}{3.940304in}}%
\pgfpathlineto{\pgfqpoint{4.636107in}{3.951815in}}%
\pgfpathlineto{\pgfqpoint{4.625316in}{3.962667in}}%
\pgfpathlineto{\pgfqpoint{4.607677in}{3.980232in}}%
\pgfpathlineto{\pgfqpoint{4.597362in}{3.990392in}}%
\pgfpathlineto{\pgfqpoint{4.587779in}{4.000000in}}%
\pgfpathlineto{\pgfqpoint{4.577964in}{4.009657in}}%
\pgfpathlineto{\pgfqpoint{4.567596in}{4.020038in}}%
\pgfpathlineto{\pgfqpoint{4.558557in}{4.028914in}}%
\pgfpathlineto{\pgfqpoint{4.550134in}{4.037333in}}%
\pgfpathlineto{\pgfqpoint{4.527515in}{4.059722in}}%
\pgfpathlineto{\pgfqpoint{4.519693in}{4.067380in}}%
\pgfpathlineto{\pgfqpoint{4.512381in}{4.074667in}}%
\pgfpathlineto{\pgfqpoint{4.500211in}{4.086568in}}%
\pgfpathlineto{\pgfqpoint{4.487434in}{4.099284in}}%
\pgfpathlineto{\pgfqpoint{4.480768in}{4.105791in}}%
\pgfpathlineto{\pgfqpoint{4.474518in}{4.112000in}}%
\pgfpathlineto{\pgfqpoint{4.461246in}{4.124940in}}%
\pgfpathlineto{\pgfqpoint{4.447354in}{4.138725in}}%
\pgfpathlineto{\pgfqpoint{4.436544in}{4.149333in}}%
\pgfpathlineto{\pgfqpoint{4.422221in}{4.163257in}}%
\pgfpathlineto{\pgfqpoint{4.407273in}{4.178044in}}%
\pgfpathlineto{\pgfqpoint{4.398460in}{4.186667in}}%
\pgfpathlineto{\pgfqpoint{4.383136in}{4.201517in}}%
\pgfpathlineto{\pgfqpoint{4.367192in}{4.217242in}}%
\pgfpathlineto{\pgfqpoint{4.360264in}{4.224000in}}%
\pgfpathlineto{\pgfqpoint{4.357638in}{4.224000in}}%
\pgfpathlineto{\pgfqpoint{4.367192in}{4.214680in}}%
\pgfpathlineto{\pgfqpoint{4.381800in}{4.200273in}}%
\pgfpathlineto{\pgfqpoint{4.395840in}{4.186667in}}%
\pgfpathlineto{\pgfqpoint{4.407273in}{4.175480in}}%
\pgfpathlineto{\pgfqpoint{4.420886in}{4.162013in}}%
\pgfpathlineto{\pgfqpoint{4.433930in}{4.149333in}}%
\pgfpathlineto{\pgfqpoint{4.447354in}{4.136159in}}%
\pgfpathlineto{\pgfqpoint{4.459912in}{4.123698in}}%
\pgfpathlineto{\pgfqpoint{4.471910in}{4.112000in}}%
\pgfpathlineto{\pgfqpoint{4.479422in}{4.104537in}}%
\pgfpathlineto{\pgfqpoint{4.487434in}{4.096717in}}%
\pgfpathlineto{\pgfqpoint{4.498879in}{4.085327in}}%
\pgfpathlineto{\pgfqpoint{4.509779in}{4.074667in}}%
\pgfpathlineto{\pgfqpoint{4.518348in}{4.066128in}}%
\pgfpathlineto{\pgfqpoint{4.527515in}{4.057153in}}%
\pgfpathlineto{\pgfqpoint{4.547539in}{4.037333in}}%
\pgfpathlineto{\pgfqpoint{4.557214in}{4.027663in}}%
\pgfpathlineto{\pgfqpoint{4.567596in}{4.017468in}}%
\pgfpathlineto{\pgfqpoint{4.576634in}{4.008418in}}%
\pgfpathlineto{\pgfqpoint{4.585190in}{4.000000in}}%
\pgfpathlineto{\pgfqpoint{4.596020in}{3.989142in}}%
\pgfpathlineto{\pgfqpoint{4.607677in}{3.977660in}}%
\pgfpathlineto{\pgfqpoint{4.622733in}{3.962667in}}%
\pgfpathlineto{\pgfqpoint{4.634766in}{3.950566in}}%
\pgfpathlineto{\pgfqpoint{4.647758in}{3.937731in}}%
\pgfpathlineto{\pgfqpoint{4.654152in}{3.931289in}}%
\pgfpathlineto{\pgfqpoint{4.660169in}{3.925333in}}%
\pgfpathlineto{\pgfqpoint{4.687838in}{3.897678in}}%
\pgfpathlineto{\pgfqpoint{4.697499in}{3.888000in}}%
\pgfpathlineto{\pgfqpoint{4.712081in}{3.873247in}}%
\pgfpathlineto{\pgfqpoint{4.727919in}{3.857504in}}%
\pgfpathlineto{\pgfqpoint{4.734723in}{3.850667in}}%
\pgfpathlineto{\pgfqpoint{4.750649in}{3.834505in}}%
\pgfpathlineto{\pgfqpoint{4.768000in}{3.817206in}}%
\pgfusepath{fill}%
\end{pgfscope}%
\begin{pgfscope}%
\pgfpathrectangle{\pgfqpoint{0.800000in}{0.528000in}}{\pgfqpoint{3.968000in}{3.696000in}}%
\pgfusepath{clip}%
\pgfsetbuttcap%
\pgfsetroundjoin%
\definecolor{currentfill}{rgb}{0.545524,0.838039,0.275626}%
\pgfsetfillcolor{currentfill}%
\pgfsetlinewidth{0.000000pt}%
\definecolor{currentstroke}{rgb}{0.000000,0.000000,0.000000}%
\pgfsetstrokecolor{currentstroke}%
\pgfsetdash{}{0pt}%
\pgfpathmoveto{\pgfqpoint{4.768000in}{3.822364in}}%
\pgfpathlineto{\pgfqpoint{4.753324in}{3.836996in}}%
\pgfpathlineto{\pgfqpoint{4.739852in}{3.850667in}}%
\pgfpathlineto{\pgfqpoint{4.727919in}{3.862658in}}%
\pgfpathlineto{\pgfqpoint{4.714758in}{3.875741in}}%
\pgfpathlineto{\pgfqpoint{4.702640in}{3.888000in}}%
\pgfpathlineto{\pgfqpoint{4.687838in}{3.902829in}}%
\pgfpathlineto{\pgfqpoint{4.665323in}{3.925333in}}%
\pgfpathlineto{\pgfqpoint{4.656807in}{3.933763in}}%
\pgfpathlineto{\pgfqpoint{4.647758in}{3.942878in}}%
\pgfpathlineto{\pgfqpoint{4.637448in}{3.953064in}}%
\pgfpathlineto{\pgfqpoint{4.627899in}{3.962667in}}%
\pgfpathlineto{\pgfqpoint{4.607677in}{3.982805in}}%
\pgfpathlineto{\pgfqpoint{4.598704in}{3.991643in}}%
\pgfpathlineto{\pgfqpoint{4.590368in}{4.000000in}}%
\pgfpathlineto{\pgfqpoint{4.579294in}{4.010896in}}%
\pgfpathlineto{\pgfqpoint{4.567596in}{4.022609in}}%
\pgfpathlineto{\pgfqpoint{4.559901in}{4.030166in}}%
\pgfpathlineto{\pgfqpoint{4.552729in}{4.037333in}}%
\pgfpathlineto{\pgfqpoint{4.527515in}{4.062291in}}%
\pgfpathlineto{\pgfqpoint{4.521037in}{4.068633in}}%
\pgfpathlineto{\pgfqpoint{4.514982in}{4.074667in}}%
\pgfpathlineto{\pgfqpoint{4.501544in}{4.087809in}}%
\pgfpathlineto{\pgfqpoint{4.487434in}{4.101851in}}%
\pgfpathlineto{\pgfqpoint{4.482114in}{4.107044in}}%
\pgfpathlineto{\pgfqpoint{4.477125in}{4.112000in}}%
\pgfpathlineto{\pgfqpoint{4.462579in}{4.126182in}}%
\pgfpathlineto{\pgfqpoint{4.447354in}{4.141290in}}%
\pgfpathlineto{\pgfqpoint{4.439158in}{4.149333in}}%
\pgfpathlineto{\pgfqpoint{4.423556in}{4.164500in}}%
\pgfpathlineto{\pgfqpoint{4.407273in}{4.180608in}}%
\pgfpathlineto{\pgfqpoint{4.401081in}{4.186667in}}%
\pgfpathlineto{\pgfqpoint{4.384472in}{4.202762in}}%
\pgfpathlineto{\pgfqpoint{4.367192in}{4.219804in}}%
\pgfpathlineto{\pgfqpoint{4.362891in}{4.224000in}}%
\pgfpathlineto{\pgfqpoint{4.360264in}{4.224000in}}%
\pgfpathlineto{\pgfqpoint{4.367192in}{4.217242in}}%
\pgfpathlineto{\pgfqpoint{4.383136in}{4.201517in}}%
\pgfpathlineto{\pgfqpoint{4.398460in}{4.186667in}}%
\pgfpathlineto{\pgfqpoint{4.407273in}{4.178044in}}%
\pgfpathlineto{\pgfqpoint{4.422221in}{4.163257in}}%
\pgfpathlineto{\pgfqpoint{4.436544in}{4.149333in}}%
\pgfpathlineto{\pgfqpoint{4.447354in}{4.138725in}}%
\pgfpathlineto{\pgfqpoint{4.461246in}{4.124940in}}%
\pgfpathlineto{\pgfqpoint{4.474518in}{4.112000in}}%
\pgfpathlineto{\pgfqpoint{4.480768in}{4.105791in}}%
\pgfpathlineto{\pgfqpoint{4.487434in}{4.099284in}}%
\pgfpathlineto{\pgfqpoint{4.500211in}{4.086568in}}%
\pgfpathlineto{\pgfqpoint{4.512381in}{4.074667in}}%
\pgfpathlineto{\pgfqpoint{4.519693in}{4.067380in}}%
\pgfpathlineto{\pgfqpoint{4.527515in}{4.059722in}}%
\pgfpathlineto{\pgfqpoint{4.550134in}{4.037333in}}%
\pgfpathlineto{\pgfqpoint{4.558557in}{4.028914in}}%
\pgfpathlineto{\pgfqpoint{4.567596in}{4.020038in}}%
\pgfpathlineto{\pgfqpoint{4.577964in}{4.009657in}}%
\pgfpathlineto{\pgfqpoint{4.587779in}{4.000000in}}%
\pgfpathlineto{\pgfqpoint{4.597362in}{3.990392in}}%
\pgfpathlineto{\pgfqpoint{4.607677in}{3.980232in}}%
\pgfpathlineto{\pgfqpoint{4.625316in}{3.962667in}}%
\pgfpathlineto{\pgfqpoint{4.636107in}{3.951815in}}%
\pgfpathlineto{\pgfqpoint{4.647758in}{3.940304in}}%
\pgfpathlineto{\pgfqpoint{4.655480in}{3.932526in}}%
\pgfpathlineto{\pgfqpoint{4.662746in}{3.925333in}}%
\pgfpathlineto{\pgfqpoint{4.687838in}{3.900254in}}%
\pgfpathlineto{\pgfqpoint{4.700070in}{3.888000in}}%
\pgfpathlineto{\pgfqpoint{4.713419in}{3.874494in}}%
\pgfpathlineto{\pgfqpoint{4.727919in}{3.860081in}}%
\pgfpathlineto{\pgfqpoint{4.737288in}{3.850667in}}%
\pgfpathlineto{\pgfqpoint{4.751987in}{3.835751in}}%
\pgfpathlineto{\pgfqpoint{4.768000in}{3.819785in}}%
\pgfusepath{fill}%
\end{pgfscope}%
\begin{pgfscope}%
\pgfpathrectangle{\pgfqpoint{0.800000in}{0.528000in}}{\pgfqpoint{3.968000in}{3.696000in}}%
\pgfusepath{clip}%
\pgfsetbuttcap%
\pgfsetroundjoin%
\definecolor{currentfill}{rgb}{0.555484,0.840254,0.269281}%
\pgfsetfillcolor{currentfill}%
\pgfsetlinewidth{0.000000pt}%
\definecolor{currentstroke}{rgb}{0.000000,0.000000,0.000000}%
\pgfsetstrokecolor{currentstroke}%
\pgfsetdash{}{0pt}%
\pgfpathmoveto{\pgfqpoint{4.768000in}{3.824942in}}%
\pgfpathlineto{\pgfqpoint{4.754661in}{3.838242in}}%
\pgfpathlineto{\pgfqpoint{4.742417in}{3.850667in}}%
\pgfpathlineto{\pgfqpoint{4.727919in}{3.865235in}}%
\pgfpathlineto{\pgfqpoint{4.716096in}{3.876987in}}%
\pgfpathlineto{\pgfqpoint{4.705211in}{3.888000in}}%
\pgfpathlineto{\pgfqpoint{4.687838in}{3.905405in}}%
\pgfpathlineto{\pgfqpoint{4.667900in}{3.925333in}}%
\pgfpathlineto{\pgfqpoint{4.658135in}{3.934999in}}%
\pgfpathlineto{\pgfqpoint{4.647758in}{3.945452in}}%
\pgfpathlineto{\pgfqpoint{4.638789in}{3.954313in}}%
\pgfpathlineto{\pgfqpoint{4.630482in}{3.962667in}}%
\pgfpathlineto{\pgfqpoint{4.607677in}{3.985377in}}%
\pgfpathlineto{\pgfqpoint{4.600046in}{3.992893in}}%
\pgfpathlineto{\pgfqpoint{4.592957in}{4.000000in}}%
\pgfpathlineto{\pgfqpoint{4.580624in}{4.012135in}}%
\pgfpathlineto{\pgfqpoint{4.567596in}{4.025179in}}%
\pgfpathlineto{\pgfqpoint{4.561244in}{4.031417in}}%
\pgfpathlineto{\pgfqpoint{4.555325in}{4.037333in}}%
\pgfpathlineto{\pgfqpoint{4.527515in}{4.064860in}}%
\pgfpathlineto{\pgfqpoint{4.522382in}{4.069885in}}%
\pgfpathlineto{\pgfqpoint{4.517584in}{4.074667in}}%
\pgfpathlineto{\pgfqpoint{4.502876in}{4.089050in}}%
\pgfpathlineto{\pgfqpoint{4.487434in}{4.104418in}}%
\pgfpathlineto{\pgfqpoint{4.483460in}{4.108298in}}%
\pgfpathlineto{\pgfqpoint{4.479733in}{4.112000in}}%
\pgfpathlineto{\pgfqpoint{4.463913in}{4.127424in}}%
\pgfpathlineto{\pgfqpoint{4.447354in}{4.143856in}}%
\pgfpathlineto{\pgfqpoint{4.441772in}{4.149333in}}%
\pgfpathlineto{\pgfqpoint{4.424890in}{4.165743in}}%
\pgfpathlineto{\pgfqpoint{4.407273in}{4.183172in}}%
\pgfpathlineto{\pgfqpoint{4.403701in}{4.186667in}}%
\pgfpathlineto{\pgfqpoint{4.385808in}{4.204006in}}%
\pgfpathlineto{\pgfqpoint{4.367192in}{4.222367in}}%
\pgfpathlineto{\pgfqpoint{4.365518in}{4.224000in}}%
\pgfpathlineto{\pgfqpoint{4.362891in}{4.224000in}}%
\pgfpathlineto{\pgfqpoint{4.367192in}{4.219804in}}%
\pgfpathlineto{\pgfqpoint{4.384472in}{4.202762in}}%
\pgfpathlineto{\pgfqpoint{4.401081in}{4.186667in}}%
\pgfpathlineto{\pgfqpoint{4.407273in}{4.180608in}}%
\pgfpathlineto{\pgfqpoint{4.423556in}{4.164500in}}%
\pgfpathlineto{\pgfqpoint{4.439158in}{4.149333in}}%
\pgfpathlineto{\pgfqpoint{4.447354in}{4.141290in}}%
\pgfpathlineto{\pgfqpoint{4.462579in}{4.126182in}}%
\pgfpathlineto{\pgfqpoint{4.477125in}{4.112000in}}%
\pgfpathlineto{\pgfqpoint{4.482114in}{4.107044in}}%
\pgfpathlineto{\pgfqpoint{4.487434in}{4.101851in}}%
\pgfpathlineto{\pgfqpoint{4.501544in}{4.087809in}}%
\pgfpathlineto{\pgfqpoint{4.514982in}{4.074667in}}%
\pgfpathlineto{\pgfqpoint{4.521037in}{4.068633in}}%
\pgfpathlineto{\pgfqpoint{4.527515in}{4.062291in}}%
\pgfpathlineto{\pgfqpoint{4.552729in}{4.037333in}}%
\pgfpathlineto{\pgfqpoint{4.559901in}{4.030166in}}%
\pgfpathlineto{\pgfqpoint{4.567596in}{4.022609in}}%
\pgfpathlineto{\pgfqpoint{4.579294in}{4.010896in}}%
\pgfpathlineto{\pgfqpoint{4.590368in}{4.000000in}}%
\pgfpathlineto{\pgfqpoint{4.598704in}{3.991643in}}%
\pgfpathlineto{\pgfqpoint{4.607677in}{3.982805in}}%
\pgfpathlineto{\pgfqpoint{4.627899in}{3.962667in}}%
\pgfpathlineto{\pgfqpoint{4.637448in}{3.953064in}}%
\pgfpathlineto{\pgfqpoint{4.647758in}{3.942878in}}%
\pgfpathlineto{\pgfqpoint{4.656807in}{3.933763in}}%
\pgfpathlineto{\pgfqpoint{4.665323in}{3.925333in}}%
\pgfpathlineto{\pgfqpoint{4.687838in}{3.902829in}}%
\pgfpathlineto{\pgfqpoint{4.702640in}{3.888000in}}%
\pgfpathlineto{\pgfqpoint{4.714758in}{3.875741in}}%
\pgfpathlineto{\pgfqpoint{4.727919in}{3.862658in}}%
\pgfpathlineto{\pgfqpoint{4.739852in}{3.850667in}}%
\pgfpathlineto{\pgfqpoint{4.753324in}{3.836996in}}%
\pgfpathlineto{\pgfqpoint{4.768000in}{3.822364in}}%
\pgfusepath{fill}%
\end{pgfscope}%
\begin{pgfscope}%
\pgfpathrectangle{\pgfqpoint{0.800000in}{0.528000in}}{\pgfqpoint{3.968000in}{3.696000in}}%
\pgfusepath{clip}%
\pgfsetbuttcap%
\pgfsetroundjoin%
\definecolor{currentfill}{rgb}{0.555484,0.840254,0.269281}%
\pgfsetfillcolor{currentfill}%
\pgfsetlinewidth{0.000000pt}%
\definecolor{currentstroke}{rgb}{0.000000,0.000000,0.000000}%
\pgfsetstrokecolor{currentstroke}%
\pgfsetdash{}{0pt}%
\pgfpathmoveto{\pgfqpoint{4.768000in}{3.827521in}}%
\pgfpathlineto{\pgfqpoint{4.755998in}{3.839488in}}%
\pgfpathlineto{\pgfqpoint{4.744982in}{3.850667in}}%
\pgfpathlineto{\pgfqpoint{4.727919in}{3.867812in}}%
\pgfpathlineto{\pgfqpoint{4.717435in}{3.878234in}}%
\pgfpathlineto{\pgfqpoint{4.707782in}{3.888000in}}%
\pgfpathlineto{\pgfqpoint{4.687838in}{3.907980in}}%
\pgfpathlineto{\pgfqpoint{4.670476in}{3.925333in}}%
\pgfpathlineto{\pgfqpoint{4.659462in}{3.936236in}}%
\pgfpathlineto{\pgfqpoint{4.647758in}{3.948026in}}%
\pgfpathlineto{\pgfqpoint{4.640130in}{3.955562in}}%
\pgfpathlineto{\pgfqpoint{4.633065in}{3.962667in}}%
\pgfpathlineto{\pgfqpoint{4.607677in}{3.987949in}}%
\pgfpathlineto{\pgfqpoint{4.601389in}{3.994143in}}%
\pgfpathlineto{\pgfqpoint{4.595546in}{4.000000in}}%
\pgfpathlineto{\pgfqpoint{4.581954in}{4.013374in}}%
\pgfpathlineto{\pgfqpoint{4.567596in}{4.027750in}}%
\pgfpathlineto{\pgfqpoint{4.562587in}{4.032668in}}%
\pgfpathlineto{\pgfqpoint{4.557920in}{4.037333in}}%
\pgfpathlineto{\pgfqpoint{4.527515in}{4.067429in}}%
\pgfpathlineto{\pgfqpoint{4.523726in}{4.071138in}}%
\pgfpathlineto{\pgfqpoint{4.520185in}{4.074667in}}%
\pgfpathlineto{\pgfqpoint{4.504209in}{4.090291in}}%
\pgfpathlineto{\pgfqpoint{4.487434in}{4.106986in}}%
\pgfpathlineto{\pgfqpoint{4.484806in}{4.109551in}}%
\pgfpathlineto{\pgfqpoint{4.482341in}{4.112000in}}%
\pgfpathlineto{\pgfqpoint{4.465247in}{4.128667in}}%
\pgfpathlineto{\pgfqpoint{4.447354in}{4.146421in}}%
\pgfpathlineto{\pgfqpoint{4.444386in}{4.149333in}}%
\pgfpathlineto{\pgfqpoint{4.426225in}{4.166987in}}%
\pgfpathlineto{\pgfqpoint{4.407273in}{4.185736in}}%
\pgfpathlineto{\pgfqpoint{4.406321in}{4.186667in}}%
\pgfpathlineto{\pgfqpoint{4.387144in}{4.205251in}}%
\pgfpathlineto{\pgfqpoint{4.368134in}{4.224000in}}%
\pgfpathlineto{\pgfqpoint{4.367192in}{4.224000in}}%
\pgfpathlineto{\pgfqpoint{4.365518in}{4.224000in}}%
\pgfpathlineto{\pgfqpoint{4.367192in}{4.222367in}}%
\pgfpathlineto{\pgfqpoint{4.385808in}{4.204006in}}%
\pgfpathlineto{\pgfqpoint{4.403701in}{4.186667in}}%
\pgfpathlineto{\pgfqpoint{4.407273in}{4.183172in}}%
\pgfpathlineto{\pgfqpoint{4.424890in}{4.165743in}}%
\pgfpathlineto{\pgfqpoint{4.441772in}{4.149333in}}%
\pgfpathlineto{\pgfqpoint{4.447354in}{4.143856in}}%
\pgfpathlineto{\pgfqpoint{4.463913in}{4.127424in}}%
\pgfpathlineto{\pgfqpoint{4.479733in}{4.112000in}}%
\pgfpathlineto{\pgfqpoint{4.483460in}{4.108298in}}%
\pgfpathlineto{\pgfqpoint{4.487434in}{4.104418in}}%
\pgfpathlineto{\pgfqpoint{4.502876in}{4.089050in}}%
\pgfpathlineto{\pgfqpoint{4.517584in}{4.074667in}}%
\pgfpathlineto{\pgfqpoint{4.522382in}{4.069885in}}%
\pgfpathlineto{\pgfqpoint{4.527515in}{4.064860in}}%
\pgfpathlineto{\pgfqpoint{4.555325in}{4.037333in}}%
\pgfpathlineto{\pgfqpoint{4.561244in}{4.031417in}}%
\pgfpathlineto{\pgfqpoint{4.567596in}{4.025179in}}%
\pgfpathlineto{\pgfqpoint{4.580624in}{4.012135in}}%
\pgfpathlineto{\pgfqpoint{4.592957in}{4.000000in}}%
\pgfpathlineto{\pgfqpoint{4.600046in}{3.992893in}}%
\pgfpathlineto{\pgfqpoint{4.607677in}{3.985377in}}%
\pgfpathlineto{\pgfqpoint{4.630482in}{3.962667in}}%
\pgfpathlineto{\pgfqpoint{4.638789in}{3.954313in}}%
\pgfpathlineto{\pgfqpoint{4.647758in}{3.945452in}}%
\pgfpathlineto{\pgfqpoint{4.658135in}{3.934999in}}%
\pgfpathlineto{\pgfqpoint{4.667900in}{3.925333in}}%
\pgfpathlineto{\pgfqpoint{4.687838in}{3.905405in}}%
\pgfpathlineto{\pgfqpoint{4.705211in}{3.888000in}}%
\pgfpathlineto{\pgfqpoint{4.716096in}{3.876987in}}%
\pgfpathlineto{\pgfqpoint{4.727919in}{3.865235in}}%
\pgfpathlineto{\pgfqpoint{4.742417in}{3.850667in}}%
\pgfpathlineto{\pgfqpoint{4.754661in}{3.838242in}}%
\pgfpathlineto{\pgfqpoint{4.768000in}{3.824942in}}%
\pgfusepath{fill}%
\end{pgfscope}%
\begin{pgfscope}%
\pgfpathrectangle{\pgfqpoint{0.800000in}{0.528000in}}{\pgfqpoint{3.968000in}{3.696000in}}%
\pgfusepath{clip}%
\pgfsetbuttcap%
\pgfsetroundjoin%
\definecolor{currentfill}{rgb}{0.555484,0.840254,0.269281}%
\pgfsetfillcolor{currentfill}%
\pgfsetlinewidth{0.000000pt}%
\definecolor{currentstroke}{rgb}{0.000000,0.000000,0.000000}%
\pgfsetstrokecolor{currentstroke}%
\pgfsetdash{}{0pt}%
\pgfpathmoveto{\pgfqpoint{4.768000in}{3.830100in}}%
\pgfpathlineto{\pgfqpoint{4.757335in}{3.840733in}}%
\pgfpathlineto{\pgfqpoint{4.747546in}{3.850667in}}%
\pgfpathlineto{\pgfqpoint{4.727919in}{3.870389in}}%
\pgfpathlineto{\pgfqpoint{4.718773in}{3.879481in}}%
\pgfpathlineto{\pgfqpoint{4.710352in}{3.888000in}}%
\pgfpathlineto{\pgfqpoint{4.687838in}{3.910556in}}%
\pgfpathlineto{\pgfqpoint{4.673053in}{3.925333in}}%
\pgfpathlineto{\pgfqpoint{4.660790in}{3.937472in}}%
\pgfpathlineto{\pgfqpoint{4.647758in}{3.950600in}}%
\pgfpathlineto{\pgfqpoint{4.641471in}{3.956811in}}%
\pgfpathlineto{\pgfqpoint{4.635648in}{3.962667in}}%
\pgfpathlineto{\pgfqpoint{4.607677in}{3.990521in}}%
\pgfpathlineto{\pgfqpoint{4.602731in}{3.995393in}}%
\pgfpathlineto{\pgfqpoint{4.598135in}{4.000000in}}%
\pgfpathlineto{\pgfqpoint{4.583284in}{4.014612in}}%
\pgfpathlineto{\pgfqpoint{4.567596in}{4.030320in}}%
\pgfpathlineto{\pgfqpoint{4.563931in}{4.033919in}}%
\pgfpathlineto{\pgfqpoint{4.560515in}{4.037333in}}%
\pgfpathlineto{\pgfqpoint{4.527515in}{4.069997in}}%
\pgfpathlineto{\pgfqpoint{4.525071in}{4.072390in}}%
\pgfpathlineto{\pgfqpoint{4.522787in}{4.074667in}}%
\pgfpathlineto{\pgfqpoint{4.505541in}{4.091532in}}%
\pgfpathlineto{\pgfqpoint{4.487434in}{4.109553in}}%
\pgfpathlineto{\pgfqpoint{4.486151in}{4.110805in}}%
\pgfpathlineto{\pgfqpoint{4.484949in}{4.112000in}}%
\pgfpathlineto{\pgfqpoint{4.466580in}{4.129909in}}%
\pgfpathlineto{\pgfqpoint{4.447354in}{4.148987in}}%
\pgfpathlineto{\pgfqpoint{4.447000in}{4.149333in}}%
\pgfpathlineto{\pgfqpoint{4.427560in}{4.168230in}}%
\pgfpathlineto{\pgfqpoint{4.408923in}{4.186667in}}%
\pgfpathlineto{\pgfqpoint{4.407273in}{4.188284in}}%
\pgfpathlineto{\pgfqpoint{4.388480in}{4.206496in}}%
\pgfpathlineto{\pgfqpoint{4.370732in}{4.224000in}}%
\pgfpathlineto{\pgfqpoint{4.368134in}{4.224000in}}%
\pgfpathlineto{\pgfqpoint{4.387144in}{4.205251in}}%
\pgfpathlineto{\pgfqpoint{4.406321in}{4.186667in}}%
\pgfpathlineto{\pgfqpoint{4.407273in}{4.185736in}}%
\pgfpathlineto{\pgfqpoint{4.426225in}{4.166987in}}%
\pgfpathlineto{\pgfqpoint{4.444386in}{4.149333in}}%
\pgfpathlineto{\pgfqpoint{4.447354in}{4.146421in}}%
\pgfpathlineto{\pgfqpoint{4.465247in}{4.128667in}}%
\pgfpathlineto{\pgfqpoint{4.482341in}{4.112000in}}%
\pgfpathlineto{\pgfqpoint{4.484806in}{4.109551in}}%
\pgfpathlineto{\pgfqpoint{4.487434in}{4.106986in}}%
\pgfpathlineto{\pgfqpoint{4.504209in}{4.090291in}}%
\pgfpathlineto{\pgfqpoint{4.520185in}{4.074667in}}%
\pgfpathlineto{\pgfqpoint{4.523726in}{4.071138in}}%
\pgfpathlineto{\pgfqpoint{4.527515in}{4.067429in}}%
\pgfpathlineto{\pgfqpoint{4.557920in}{4.037333in}}%
\pgfpathlineto{\pgfqpoint{4.562587in}{4.032668in}}%
\pgfpathlineto{\pgfqpoint{4.567596in}{4.027750in}}%
\pgfpathlineto{\pgfqpoint{4.581954in}{4.013374in}}%
\pgfpathlineto{\pgfqpoint{4.595546in}{4.000000in}}%
\pgfpathlineto{\pgfqpoint{4.601389in}{3.994143in}}%
\pgfpathlineto{\pgfqpoint{4.607677in}{3.987949in}}%
\pgfpathlineto{\pgfqpoint{4.633065in}{3.962667in}}%
\pgfpathlineto{\pgfqpoint{4.640130in}{3.955562in}}%
\pgfpathlineto{\pgfqpoint{4.647758in}{3.948026in}}%
\pgfpathlineto{\pgfqpoint{4.659462in}{3.936236in}}%
\pgfpathlineto{\pgfqpoint{4.670476in}{3.925333in}}%
\pgfpathlineto{\pgfqpoint{4.687838in}{3.907980in}}%
\pgfpathlineto{\pgfqpoint{4.707782in}{3.888000in}}%
\pgfpathlineto{\pgfqpoint{4.717435in}{3.878234in}}%
\pgfpathlineto{\pgfqpoint{4.727919in}{3.867812in}}%
\pgfpathlineto{\pgfqpoint{4.744982in}{3.850667in}}%
\pgfpathlineto{\pgfqpoint{4.755998in}{3.839488in}}%
\pgfpathlineto{\pgfqpoint{4.768000in}{3.827521in}}%
\pgfusepath{fill}%
\end{pgfscope}%
\begin{pgfscope}%
\pgfpathrectangle{\pgfqpoint{0.800000in}{0.528000in}}{\pgfqpoint{3.968000in}{3.696000in}}%
\pgfusepath{clip}%
\pgfsetbuttcap%
\pgfsetroundjoin%
\definecolor{currentfill}{rgb}{0.555484,0.840254,0.269281}%
\pgfsetfillcolor{currentfill}%
\pgfsetlinewidth{0.000000pt}%
\definecolor{currentstroke}{rgb}{0.000000,0.000000,0.000000}%
\pgfsetstrokecolor{currentstroke}%
\pgfsetdash{}{0pt}%
\pgfpathmoveto{\pgfqpoint{4.768000in}{3.832679in}}%
\pgfpathlineto{\pgfqpoint{4.758673in}{3.841979in}}%
\pgfpathlineto{\pgfqpoint{4.750111in}{3.850667in}}%
\pgfpathlineto{\pgfqpoint{4.727919in}{3.872966in}}%
\pgfpathlineto{\pgfqpoint{4.720111in}{3.880728in}}%
\pgfpathlineto{\pgfqpoint{4.712923in}{3.888000in}}%
\pgfpathlineto{\pgfqpoint{4.687838in}{3.913131in}}%
\pgfpathlineto{\pgfqpoint{4.675630in}{3.925333in}}%
\pgfpathlineto{\pgfqpoint{4.662117in}{3.938709in}}%
\pgfpathlineto{\pgfqpoint{4.647758in}{3.953173in}}%
\pgfpathlineto{\pgfqpoint{4.642812in}{3.958060in}}%
\pgfpathlineto{\pgfqpoint{4.638231in}{3.962667in}}%
\pgfpathlineto{\pgfqpoint{4.607677in}{3.993093in}}%
\pgfpathlineto{\pgfqpoint{4.604073in}{3.996643in}}%
\pgfpathlineto{\pgfqpoint{4.600724in}{4.000000in}}%
\pgfpathlineto{\pgfqpoint{4.584614in}{4.015851in}}%
\pgfpathlineto{\pgfqpoint{4.567596in}{4.032891in}}%
\pgfpathlineto{\pgfqpoint{4.565274in}{4.035171in}}%
\pgfpathlineto{\pgfqpoint{4.563110in}{4.037333in}}%
\pgfpathlineto{\pgfqpoint{4.527515in}{4.072566in}}%
\pgfpathlineto{\pgfqpoint{4.526416in}{4.073643in}}%
\pgfpathlineto{\pgfqpoint{4.525388in}{4.074667in}}%
\pgfpathlineto{\pgfqpoint{4.506873in}{4.092773in}}%
\pgfpathlineto{\pgfqpoint{4.487555in}{4.112000in}}%
\pgfpathlineto{\pgfqpoint{4.487434in}{4.112119in}}%
\pgfpathlineto{\pgfqpoint{4.467914in}{4.131151in}}%
\pgfpathlineto{\pgfqpoint{4.449590in}{4.149333in}}%
\pgfpathlineto{\pgfqpoint{4.447354in}{4.151531in}}%
\pgfpathlineto{\pgfqpoint{4.428895in}{4.169473in}}%
\pgfpathlineto{\pgfqpoint{4.411515in}{4.186667in}}%
\pgfpathlineto{\pgfqpoint{4.407273in}{4.190823in}}%
\pgfpathlineto{\pgfqpoint{4.389816in}{4.207740in}}%
\pgfpathlineto{\pgfqpoint{4.373330in}{4.224000in}}%
\pgfpathlineto{\pgfqpoint{4.370732in}{4.224000in}}%
\pgfpathlineto{\pgfqpoint{4.388480in}{4.206496in}}%
\pgfpathlineto{\pgfqpoint{4.407273in}{4.188284in}}%
\pgfpathlineto{\pgfqpoint{4.408923in}{4.186667in}}%
\pgfpathlineto{\pgfqpoint{4.427560in}{4.168230in}}%
\pgfpathlineto{\pgfqpoint{4.447000in}{4.149333in}}%
\pgfpathlineto{\pgfqpoint{4.447354in}{4.148987in}}%
\pgfpathlineto{\pgfqpoint{4.466580in}{4.129909in}}%
\pgfpathlineto{\pgfqpoint{4.484949in}{4.112000in}}%
\pgfpathlineto{\pgfqpoint{4.486151in}{4.110805in}}%
\pgfpathlineto{\pgfqpoint{4.487434in}{4.109553in}}%
\pgfpathlineto{\pgfqpoint{4.505541in}{4.091532in}}%
\pgfpathlineto{\pgfqpoint{4.522787in}{4.074667in}}%
\pgfpathlineto{\pgfqpoint{4.525071in}{4.072390in}}%
\pgfpathlineto{\pgfqpoint{4.527515in}{4.069997in}}%
\pgfpathlineto{\pgfqpoint{4.560515in}{4.037333in}}%
\pgfpathlineto{\pgfqpoint{4.563931in}{4.033919in}}%
\pgfpathlineto{\pgfqpoint{4.567596in}{4.030320in}}%
\pgfpathlineto{\pgfqpoint{4.583284in}{4.014612in}}%
\pgfpathlineto{\pgfqpoint{4.598135in}{4.000000in}}%
\pgfpathlineto{\pgfqpoint{4.602731in}{3.995393in}}%
\pgfpathlineto{\pgfqpoint{4.607677in}{3.990521in}}%
\pgfpathlineto{\pgfqpoint{4.635648in}{3.962667in}}%
\pgfpathlineto{\pgfqpoint{4.641471in}{3.956811in}}%
\pgfpathlineto{\pgfqpoint{4.647758in}{3.950600in}}%
\pgfpathlineto{\pgfqpoint{4.660790in}{3.937472in}}%
\pgfpathlineto{\pgfqpoint{4.673053in}{3.925333in}}%
\pgfpathlineto{\pgfqpoint{4.687838in}{3.910556in}}%
\pgfpathlineto{\pgfqpoint{4.710352in}{3.888000in}}%
\pgfpathlineto{\pgfqpoint{4.718773in}{3.879481in}}%
\pgfpathlineto{\pgfqpoint{4.727919in}{3.870389in}}%
\pgfpathlineto{\pgfqpoint{4.747546in}{3.850667in}}%
\pgfpathlineto{\pgfqpoint{4.757335in}{3.840733in}}%
\pgfpathlineto{\pgfqpoint{4.768000in}{3.830100in}}%
\pgfusepath{fill}%
\end{pgfscope}%
\begin{pgfscope}%
\pgfpathrectangle{\pgfqpoint{0.800000in}{0.528000in}}{\pgfqpoint{3.968000in}{3.696000in}}%
\pgfusepath{clip}%
\pgfsetbuttcap%
\pgfsetroundjoin%
\definecolor{currentfill}{rgb}{0.565498,0.842430,0.262877}%
\pgfsetfillcolor{currentfill}%
\pgfsetlinewidth{0.000000pt}%
\definecolor{currentstroke}{rgb}{0.000000,0.000000,0.000000}%
\pgfsetstrokecolor{currentstroke}%
\pgfsetdash{}{0pt}%
\pgfpathmoveto{\pgfqpoint{4.768000in}{3.835257in}}%
\pgfpathlineto{\pgfqpoint{4.760010in}{3.843224in}}%
\pgfpathlineto{\pgfqpoint{4.752675in}{3.850667in}}%
\pgfpathlineto{\pgfqpoint{4.727919in}{3.875544in}}%
\pgfpathlineto{\pgfqpoint{4.721450in}{3.881974in}}%
\pgfpathlineto{\pgfqpoint{4.715494in}{3.888000in}}%
\pgfpathlineto{\pgfqpoint{4.687838in}{3.915707in}}%
\pgfpathlineto{\pgfqpoint{4.678207in}{3.925333in}}%
\pgfpathlineto{\pgfqpoint{4.663445in}{3.939945in}}%
\pgfpathlineto{\pgfqpoint{4.647758in}{3.955747in}}%
\pgfpathlineto{\pgfqpoint{4.644153in}{3.959309in}}%
\pgfpathlineto{\pgfqpoint{4.640814in}{3.962667in}}%
\pgfpathlineto{\pgfqpoint{4.607677in}{3.995665in}}%
\pgfpathlineto{\pgfqpoint{4.605415in}{3.997893in}}%
\pgfpathlineto{\pgfqpoint{4.603314in}{4.000000in}}%
\pgfpathlineto{\pgfqpoint{4.585944in}{4.017090in}}%
\pgfpathlineto{\pgfqpoint{4.567596in}{4.035461in}}%
\pgfpathlineto{\pgfqpoint{4.566618in}{4.036422in}}%
\pgfpathlineto{\pgfqpoint{4.565706in}{4.037333in}}%
\pgfpathlineto{\pgfqpoint{4.535540in}{4.067192in}}%
\pgfpathlineto{\pgfqpoint{4.527984in}{4.074667in}}%
\pgfpathlineto{\pgfqpoint{4.527515in}{4.075131in}}%
\pgfpathlineto{\pgfqpoint{4.508206in}{4.094014in}}%
\pgfpathlineto{\pgfqpoint{4.490134in}{4.112000in}}%
\pgfpathlineto{\pgfqpoint{4.487434in}{4.114661in}}%
\pgfpathlineto{\pgfqpoint{4.469248in}{4.132393in}}%
\pgfpathlineto{\pgfqpoint{4.452175in}{4.149333in}}%
\pgfpathlineto{\pgfqpoint{4.447354in}{4.154072in}}%
\pgfpathlineto{\pgfqpoint{4.430230in}{4.170717in}}%
\pgfpathlineto{\pgfqpoint{4.414107in}{4.186667in}}%
\pgfpathlineto{\pgfqpoint{4.407273in}{4.193362in}}%
\pgfpathlineto{\pgfqpoint{4.391152in}{4.208985in}}%
\pgfpathlineto{\pgfqpoint{4.375928in}{4.224000in}}%
\pgfpathlineto{\pgfqpoint{4.373330in}{4.224000in}}%
\pgfpathlineto{\pgfqpoint{4.389816in}{4.207740in}}%
\pgfpathlineto{\pgfqpoint{4.407273in}{4.190823in}}%
\pgfpathlineto{\pgfqpoint{4.411515in}{4.186667in}}%
\pgfpathlineto{\pgfqpoint{4.428895in}{4.169473in}}%
\pgfpathlineto{\pgfqpoint{4.447354in}{4.151531in}}%
\pgfpathlineto{\pgfqpoint{4.449590in}{4.149333in}}%
\pgfpathlineto{\pgfqpoint{4.467914in}{4.131151in}}%
\pgfpathlineto{\pgfqpoint{4.487434in}{4.112119in}}%
\pgfpathlineto{\pgfqpoint{4.487555in}{4.112000in}}%
\pgfpathlineto{\pgfqpoint{4.506873in}{4.092773in}}%
\pgfpathlineto{\pgfqpoint{4.525388in}{4.074667in}}%
\pgfpathlineto{\pgfqpoint{4.526416in}{4.073643in}}%
\pgfpathlineto{\pgfqpoint{4.527515in}{4.072566in}}%
\pgfpathlineto{\pgfqpoint{4.563110in}{4.037333in}}%
\pgfpathlineto{\pgfqpoint{4.565274in}{4.035171in}}%
\pgfpathlineto{\pgfqpoint{4.567596in}{4.032891in}}%
\pgfpathlineto{\pgfqpoint{4.584614in}{4.015851in}}%
\pgfpathlineto{\pgfqpoint{4.600724in}{4.000000in}}%
\pgfpathlineto{\pgfqpoint{4.604073in}{3.996643in}}%
\pgfpathlineto{\pgfqpoint{4.607677in}{3.993093in}}%
\pgfpathlineto{\pgfqpoint{4.638231in}{3.962667in}}%
\pgfpathlineto{\pgfqpoint{4.642812in}{3.958060in}}%
\pgfpathlineto{\pgfqpoint{4.647758in}{3.953173in}}%
\pgfpathlineto{\pgfqpoint{4.662117in}{3.938709in}}%
\pgfpathlineto{\pgfqpoint{4.675630in}{3.925333in}}%
\pgfpathlineto{\pgfqpoint{4.687838in}{3.913131in}}%
\pgfpathlineto{\pgfqpoint{4.712923in}{3.888000in}}%
\pgfpathlineto{\pgfqpoint{4.720111in}{3.880728in}}%
\pgfpathlineto{\pgfqpoint{4.727919in}{3.872966in}}%
\pgfpathlineto{\pgfqpoint{4.750111in}{3.850667in}}%
\pgfpathlineto{\pgfqpoint{4.758673in}{3.841979in}}%
\pgfpathlineto{\pgfqpoint{4.768000in}{3.832679in}}%
\pgfusepath{fill}%
\end{pgfscope}%
\begin{pgfscope}%
\pgfpathrectangle{\pgfqpoint{0.800000in}{0.528000in}}{\pgfqpoint{3.968000in}{3.696000in}}%
\pgfusepath{clip}%
\pgfsetbuttcap%
\pgfsetroundjoin%
\definecolor{currentfill}{rgb}{0.565498,0.842430,0.262877}%
\pgfsetfillcolor{currentfill}%
\pgfsetlinewidth{0.000000pt}%
\definecolor{currentstroke}{rgb}{0.000000,0.000000,0.000000}%
\pgfsetstrokecolor{currentstroke}%
\pgfsetdash{}{0pt}%
\pgfpathmoveto{\pgfqpoint{4.768000in}{3.837836in}}%
\pgfpathlineto{\pgfqpoint{4.761347in}{3.844470in}}%
\pgfpathlineto{\pgfqpoint{4.755240in}{3.850667in}}%
\pgfpathlineto{\pgfqpoint{4.727919in}{3.878121in}}%
\pgfpathlineto{\pgfqpoint{4.722788in}{3.883221in}}%
\pgfpathlineto{\pgfqpoint{4.718064in}{3.888000in}}%
\pgfpathlineto{\pgfqpoint{4.687838in}{3.918282in}}%
\pgfpathlineto{\pgfqpoint{4.680784in}{3.925333in}}%
\pgfpathlineto{\pgfqpoint{4.664772in}{3.941182in}}%
\pgfpathlineto{\pgfqpoint{4.647758in}{3.958321in}}%
\pgfpathlineto{\pgfqpoint{4.645494in}{3.960558in}}%
\pgfpathlineto{\pgfqpoint{4.643397in}{3.962667in}}%
\pgfpathlineto{\pgfqpoint{4.607677in}{3.998237in}}%
\pgfpathlineto{\pgfqpoint{4.606757in}{3.999143in}}%
\pgfpathlineto{\pgfqpoint{4.605903in}{4.000000in}}%
\pgfpathlineto{\pgfqpoint{4.587274in}{4.018329in}}%
\pgfpathlineto{\pgfqpoint{4.568293in}{4.037333in}}%
\pgfpathlineto{\pgfqpoint{4.567954in}{4.037667in}}%
\pgfpathlineto{\pgfqpoint{4.567596in}{4.038025in}}%
\pgfpathlineto{\pgfqpoint{4.530558in}{4.074667in}}%
\pgfpathlineto{\pgfqpoint{4.527515in}{4.077675in}}%
\pgfpathlineto{\pgfqpoint{4.509538in}{4.095255in}}%
\pgfpathlineto{\pgfqpoint{4.492714in}{4.112000in}}%
\pgfpathlineto{\pgfqpoint{4.487434in}{4.117204in}}%
\pgfpathlineto{\pgfqpoint{4.470581in}{4.133636in}}%
\pgfpathlineto{\pgfqpoint{4.454761in}{4.149333in}}%
\pgfpathlineto{\pgfqpoint{4.447354in}{4.156613in}}%
\pgfpathlineto{\pgfqpoint{4.431565in}{4.171960in}}%
\pgfpathlineto{\pgfqpoint{4.416699in}{4.186667in}}%
\pgfpathlineto{\pgfqpoint{4.407273in}{4.195902in}}%
\pgfpathlineto{\pgfqpoint{4.392488in}{4.210229in}}%
\pgfpathlineto{\pgfqpoint{4.378526in}{4.224000in}}%
\pgfpathlineto{\pgfqpoint{4.375928in}{4.224000in}}%
\pgfpathlineto{\pgfqpoint{4.391152in}{4.208985in}}%
\pgfpathlineto{\pgfqpoint{4.407273in}{4.193362in}}%
\pgfpathlineto{\pgfqpoint{4.414107in}{4.186667in}}%
\pgfpathlineto{\pgfqpoint{4.430230in}{4.170717in}}%
\pgfpathlineto{\pgfqpoint{4.447354in}{4.154072in}}%
\pgfpathlineto{\pgfqpoint{4.452175in}{4.149333in}}%
\pgfpathlineto{\pgfqpoint{4.469248in}{4.132393in}}%
\pgfpathlineto{\pgfqpoint{4.487434in}{4.114661in}}%
\pgfpathlineto{\pgfqpoint{4.490134in}{4.112000in}}%
\pgfpathlineto{\pgfqpoint{4.508206in}{4.094014in}}%
\pgfpathlineto{\pgfqpoint{4.527515in}{4.075131in}}%
\pgfpathlineto{\pgfqpoint{4.527984in}{4.074667in}}%
\pgfpathlineto{\pgfqpoint{4.535540in}{4.067192in}}%
\pgfpathlineto{\pgfqpoint{4.565706in}{4.037333in}}%
\pgfpathlineto{\pgfqpoint{4.566618in}{4.036422in}}%
\pgfpathlineto{\pgfqpoint{4.567596in}{4.035461in}}%
\pgfpathlineto{\pgfqpoint{4.585944in}{4.017090in}}%
\pgfpathlineto{\pgfqpoint{4.603314in}{4.000000in}}%
\pgfpathlineto{\pgfqpoint{4.605415in}{3.997893in}}%
\pgfpathlineto{\pgfqpoint{4.607677in}{3.995665in}}%
\pgfpathlineto{\pgfqpoint{4.640814in}{3.962667in}}%
\pgfpathlineto{\pgfqpoint{4.644153in}{3.959309in}}%
\pgfpathlineto{\pgfqpoint{4.647758in}{3.955747in}}%
\pgfpathlineto{\pgfqpoint{4.663445in}{3.939945in}}%
\pgfpathlineto{\pgfqpoint{4.678207in}{3.925333in}}%
\pgfpathlineto{\pgfqpoint{4.687838in}{3.915707in}}%
\pgfpathlineto{\pgfqpoint{4.715494in}{3.888000in}}%
\pgfpathlineto{\pgfqpoint{4.721450in}{3.881974in}}%
\pgfpathlineto{\pgfqpoint{4.727919in}{3.875544in}}%
\pgfpathlineto{\pgfqpoint{4.752675in}{3.850667in}}%
\pgfpathlineto{\pgfqpoint{4.760010in}{3.843224in}}%
\pgfpathlineto{\pgfqpoint{4.768000in}{3.835257in}}%
\pgfusepath{fill}%
\end{pgfscope}%
\begin{pgfscope}%
\pgfpathrectangle{\pgfqpoint{0.800000in}{0.528000in}}{\pgfqpoint{3.968000in}{3.696000in}}%
\pgfusepath{clip}%
\pgfsetbuttcap%
\pgfsetroundjoin%
\definecolor{currentfill}{rgb}{0.565498,0.842430,0.262877}%
\pgfsetfillcolor{currentfill}%
\pgfsetlinewidth{0.000000pt}%
\definecolor{currentstroke}{rgb}{0.000000,0.000000,0.000000}%
\pgfsetstrokecolor{currentstroke}%
\pgfsetdash{}{0pt}%
\pgfpathmoveto{\pgfqpoint{4.768000in}{3.840415in}}%
\pgfpathlineto{\pgfqpoint{4.762684in}{3.845715in}}%
\pgfpathlineto{\pgfqpoint{4.757805in}{3.850667in}}%
\pgfpathlineto{\pgfqpoint{4.727919in}{3.880698in}}%
\pgfpathlineto{\pgfqpoint{4.724127in}{3.884468in}}%
\pgfpathlineto{\pgfqpoint{4.720635in}{3.888000in}}%
\pgfpathlineto{\pgfqpoint{4.687838in}{3.920858in}}%
\pgfpathlineto{\pgfqpoint{4.683360in}{3.925333in}}%
\pgfpathlineto{\pgfqpoint{4.666100in}{3.942418in}}%
\pgfpathlineto{\pgfqpoint{4.647758in}{3.960895in}}%
\pgfpathlineto{\pgfqpoint{4.646834in}{3.961807in}}%
\pgfpathlineto{\pgfqpoint{4.645979in}{3.962667in}}%
\pgfpathlineto{\pgfqpoint{4.620252in}{3.988286in}}%
\pgfpathlineto{\pgfqpoint{4.608483in}{4.000000in}}%
\pgfpathlineto{\pgfqpoint{4.607677in}{4.000802in}}%
\pgfpathlineto{\pgfqpoint{4.588604in}{4.019568in}}%
\pgfpathlineto{\pgfqpoint{4.570861in}{4.037333in}}%
\pgfpathlineto{\pgfqpoint{4.569272in}{4.038895in}}%
\pgfpathlineto{\pgfqpoint{4.567596in}{4.040571in}}%
\pgfpathlineto{\pgfqpoint{4.533131in}{4.074667in}}%
\pgfpathlineto{\pgfqpoint{4.527515in}{4.080219in}}%
\pgfpathlineto{\pgfqpoint{4.510871in}{4.096496in}}%
\pgfpathlineto{\pgfqpoint{4.495293in}{4.112000in}}%
\pgfpathlineto{\pgfqpoint{4.487434in}{4.119746in}}%
\pgfpathlineto{\pgfqpoint{4.471915in}{4.134878in}}%
\pgfpathlineto{\pgfqpoint{4.457347in}{4.149333in}}%
\pgfpathlineto{\pgfqpoint{4.447354in}{4.159154in}}%
\pgfpathlineto{\pgfqpoint{4.432900in}{4.173204in}}%
\pgfpathlineto{\pgfqpoint{4.419290in}{4.186667in}}%
\pgfpathlineto{\pgfqpoint{4.407273in}{4.198441in}}%
\pgfpathlineto{\pgfqpoint{4.393824in}{4.211474in}}%
\pgfpathlineto{\pgfqpoint{4.381124in}{4.224000in}}%
\pgfpathlineto{\pgfqpoint{4.378526in}{4.224000in}}%
\pgfpathlineto{\pgfqpoint{4.392488in}{4.210229in}}%
\pgfpathlineto{\pgfqpoint{4.407273in}{4.195902in}}%
\pgfpathlineto{\pgfqpoint{4.416699in}{4.186667in}}%
\pgfpathlineto{\pgfqpoint{4.431565in}{4.171960in}}%
\pgfpathlineto{\pgfqpoint{4.447354in}{4.156613in}}%
\pgfpathlineto{\pgfqpoint{4.454761in}{4.149333in}}%
\pgfpathlineto{\pgfqpoint{4.470581in}{4.133636in}}%
\pgfpathlineto{\pgfqpoint{4.487434in}{4.117204in}}%
\pgfpathlineto{\pgfqpoint{4.492714in}{4.112000in}}%
\pgfpathlineto{\pgfqpoint{4.509538in}{4.095255in}}%
\pgfpathlineto{\pgfqpoint{4.527515in}{4.077675in}}%
\pgfpathlineto{\pgfqpoint{4.530558in}{4.074667in}}%
\pgfpathlineto{\pgfqpoint{4.567596in}{4.038025in}}%
\pgfpathlineto{\pgfqpoint{4.567954in}{4.037667in}}%
\pgfpathlineto{\pgfqpoint{4.568293in}{4.037333in}}%
\pgfpathlineto{\pgfqpoint{4.587274in}{4.018329in}}%
\pgfpathlineto{\pgfqpoint{4.605903in}{4.000000in}}%
\pgfpathlineto{\pgfqpoint{4.606757in}{3.999143in}}%
\pgfpathlineto{\pgfqpoint{4.607677in}{3.998237in}}%
\pgfpathlineto{\pgfqpoint{4.643397in}{3.962667in}}%
\pgfpathlineto{\pgfqpoint{4.645494in}{3.960558in}}%
\pgfpathlineto{\pgfqpoint{4.647758in}{3.958321in}}%
\pgfpathlineto{\pgfqpoint{4.664772in}{3.941182in}}%
\pgfpathlineto{\pgfqpoint{4.680784in}{3.925333in}}%
\pgfpathlineto{\pgfqpoint{4.687838in}{3.918282in}}%
\pgfpathlineto{\pgfqpoint{4.718064in}{3.888000in}}%
\pgfpathlineto{\pgfqpoint{4.722788in}{3.883221in}}%
\pgfpathlineto{\pgfqpoint{4.727919in}{3.878121in}}%
\pgfpathlineto{\pgfqpoint{4.755240in}{3.850667in}}%
\pgfpathlineto{\pgfqpoint{4.761347in}{3.844470in}}%
\pgfpathlineto{\pgfqpoint{4.768000in}{3.837836in}}%
\pgfusepath{fill}%
\end{pgfscope}%
\begin{pgfscope}%
\pgfpathrectangle{\pgfqpoint{0.800000in}{0.528000in}}{\pgfqpoint{3.968000in}{3.696000in}}%
\pgfusepath{clip}%
\pgfsetbuttcap%
\pgfsetroundjoin%
\definecolor{currentfill}{rgb}{0.565498,0.842430,0.262877}%
\pgfsetfillcolor{currentfill}%
\pgfsetlinewidth{0.000000pt}%
\definecolor{currentstroke}{rgb}{0.000000,0.000000,0.000000}%
\pgfsetstrokecolor{currentstroke}%
\pgfsetdash{}{0pt}%
\pgfpathmoveto{\pgfqpoint{4.768000in}{3.842994in}}%
\pgfpathlineto{\pgfqpoint{4.764021in}{3.846961in}}%
\pgfpathlineto{\pgfqpoint{4.760369in}{3.850667in}}%
\pgfpathlineto{\pgfqpoint{4.727919in}{3.883275in}}%
\pgfpathlineto{\pgfqpoint{4.725465in}{3.885714in}}%
\pgfpathlineto{\pgfqpoint{4.723206in}{3.888000in}}%
\pgfpathlineto{\pgfqpoint{4.687838in}{3.923433in}}%
\pgfpathlineto{\pgfqpoint{4.685937in}{3.925333in}}%
\pgfpathlineto{\pgfqpoint{4.667428in}{3.943655in}}%
\pgfpathlineto{\pgfqpoint{4.648554in}{3.962667in}}%
\pgfpathlineto{\pgfqpoint{4.647758in}{3.963461in}}%
\pgfpathlineto{\pgfqpoint{4.611044in}{4.000000in}}%
\pgfpathlineto{\pgfqpoint{4.607677in}{4.003349in}}%
\pgfpathlineto{\pgfqpoint{4.589934in}{4.020807in}}%
\pgfpathlineto{\pgfqpoint{4.573428in}{4.037333in}}%
\pgfpathlineto{\pgfqpoint{4.570590in}{4.040122in}}%
\pgfpathlineto{\pgfqpoint{4.567596in}{4.043116in}}%
\pgfpathlineto{\pgfqpoint{4.535704in}{4.074667in}}%
\pgfpathlineto{\pgfqpoint{4.527515in}{4.082763in}}%
\pgfpathlineto{\pgfqpoint{4.512203in}{4.097738in}}%
\pgfpathlineto{\pgfqpoint{4.497873in}{4.112000in}}%
\pgfpathlineto{\pgfqpoint{4.487434in}{4.122289in}}%
\pgfpathlineto{\pgfqpoint{4.473249in}{4.136120in}}%
\pgfpathlineto{\pgfqpoint{4.459932in}{4.149333in}}%
\pgfpathlineto{\pgfqpoint{4.447354in}{4.161695in}}%
\pgfpathlineto{\pgfqpoint{4.434234in}{4.174447in}}%
\pgfpathlineto{\pgfqpoint{4.421882in}{4.186667in}}%
\pgfpathlineto{\pgfqpoint{4.407273in}{4.200980in}}%
\pgfpathlineto{\pgfqpoint{4.395161in}{4.212718in}}%
\pgfpathlineto{\pgfqpoint{4.383721in}{4.224000in}}%
\pgfpathlineto{\pgfqpoint{4.381124in}{4.224000in}}%
\pgfpathlineto{\pgfqpoint{4.393824in}{4.211474in}}%
\pgfpathlineto{\pgfqpoint{4.407273in}{4.198441in}}%
\pgfpathlineto{\pgfqpoint{4.419290in}{4.186667in}}%
\pgfpathlineto{\pgfqpoint{4.432900in}{4.173204in}}%
\pgfpathlineto{\pgfqpoint{4.447354in}{4.159154in}}%
\pgfpathlineto{\pgfqpoint{4.457347in}{4.149333in}}%
\pgfpathlineto{\pgfqpoint{4.471915in}{4.134878in}}%
\pgfpathlineto{\pgfqpoint{4.487434in}{4.119746in}}%
\pgfpathlineto{\pgfqpoint{4.495293in}{4.112000in}}%
\pgfpathlineto{\pgfqpoint{4.510871in}{4.096496in}}%
\pgfpathlineto{\pgfqpoint{4.527515in}{4.080219in}}%
\pgfpathlineto{\pgfqpoint{4.533131in}{4.074667in}}%
\pgfpathlineto{\pgfqpoint{4.567596in}{4.040571in}}%
\pgfpathlineto{\pgfqpoint{4.569272in}{4.038895in}}%
\pgfpathlineto{\pgfqpoint{4.570861in}{4.037333in}}%
\pgfpathlineto{\pgfqpoint{4.588604in}{4.019568in}}%
\pgfpathlineto{\pgfqpoint{4.607677in}{4.000802in}}%
\pgfpathlineto{\pgfqpoint{4.608483in}{4.000000in}}%
\pgfpathlineto{\pgfqpoint{4.620252in}{3.988286in}}%
\pgfpathlineto{\pgfqpoint{4.645979in}{3.962667in}}%
\pgfpathlineto{\pgfqpoint{4.646834in}{3.961807in}}%
\pgfpathlineto{\pgfqpoint{4.647758in}{3.960895in}}%
\pgfpathlineto{\pgfqpoint{4.666100in}{3.942418in}}%
\pgfpathlineto{\pgfqpoint{4.683360in}{3.925333in}}%
\pgfpathlineto{\pgfqpoint{4.687838in}{3.920858in}}%
\pgfpathlineto{\pgfqpoint{4.720635in}{3.888000in}}%
\pgfpathlineto{\pgfqpoint{4.724127in}{3.884468in}}%
\pgfpathlineto{\pgfqpoint{4.727919in}{3.880698in}}%
\pgfpathlineto{\pgfqpoint{4.757805in}{3.850667in}}%
\pgfpathlineto{\pgfqpoint{4.762684in}{3.845715in}}%
\pgfpathlineto{\pgfqpoint{4.768000in}{3.840415in}}%
\pgfusepath{fill}%
\end{pgfscope}%
\begin{pgfscope}%
\pgfpathrectangle{\pgfqpoint{0.800000in}{0.528000in}}{\pgfqpoint{3.968000in}{3.696000in}}%
\pgfusepath{clip}%
\pgfsetbuttcap%
\pgfsetroundjoin%
\definecolor{currentfill}{rgb}{0.575563,0.844566,0.256415}%
\pgfsetfillcolor{currentfill}%
\pgfsetlinewidth{0.000000pt}%
\definecolor{currentstroke}{rgb}{0.000000,0.000000,0.000000}%
\pgfsetstrokecolor{currentstroke}%
\pgfsetdash{}{0pt}%
\pgfpathmoveto{\pgfqpoint{4.768000in}{3.845573in}}%
\pgfpathlineto{\pgfqpoint{4.765359in}{3.848206in}}%
\pgfpathlineto{\pgfqpoint{4.762934in}{3.850667in}}%
\pgfpathlineto{\pgfqpoint{4.727919in}{3.885852in}}%
\pgfpathlineto{\pgfqpoint{4.726804in}{3.886961in}}%
\pgfpathlineto{\pgfqpoint{4.725777in}{3.888000in}}%
\pgfpathlineto{\pgfqpoint{4.697429in}{3.916400in}}%
\pgfpathlineto{\pgfqpoint{4.688507in}{3.925333in}}%
\pgfpathlineto{\pgfqpoint{4.687838in}{3.926002in}}%
\pgfpathlineto{\pgfqpoint{4.668755in}{3.944892in}}%
\pgfpathlineto{\pgfqpoint{4.651109in}{3.962667in}}%
\pgfpathlineto{\pgfqpoint{4.647758in}{3.966010in}}%
\pgfpathlineto{\pgfqpoint{4.613605in}{4.000000in}}%
\pgfpathlineto{\pgfqpoint{4.607677in}{4.005897in}}%
\pgfpathlineto{\pgfqpoint{4.591264in}{4.022045in}}%
\pgfpathlineto{\pgfqpoint{4.575995in}{4.037333in}}%
\pgfpathlineto{\pgfqpoint{4.571908in}{4.041350in}}%
\pgfpathlineto{\pgfqpoint{4.567596in}{4.045662in}}%
\pgfpathlineto{\pgfqpoint{4.538278in}{4.074667in}}%
\pgfpathlineto{\pgfqpoint{4.527515in}{4.085307in}}%
\pgfpathlineto{\pgfqpoint{4.513536in}{4.098979in}}%
\pgfpathlineto{\pgfqpoint{4.500452in}{4.112000in}}%
\pgfpathlineto{\pgfqpoint{4.487434in}{4.124832in}}%
\pgfpathlineto{\pgfqpoint{4.474582in}{4.137362in}}%
\pgfpathlineto{\pgfqpoint{4.462518in}{4.149333in}}%
\pgfpathlineto{\pgfqpoint{4.447354in}{4.164236in}}%
\pgfpathlineto{\pgfqpoint{4.435569in}{4.175690in}}%
\pgfpathlineto{\pgfqpoint{4.424474in}{4.186667in}}%
\pgfpathlineto{\pgfqpoint{4.407273in}{4.203520in}}%
\pgfpathlineto{\pgfqpoint{4.396497in}{4.213963in}}%
\pgfpathlineto{\pgfqpoint{4.386319in}{4.224000in}}%
\pgfpathlineto{\pgfqpoint{4.383721in}{4.224000in}}%
\pgfpathlineto{\pgfqpoint{4.395161in}{4.212718in}}%
\pgfpathlineto{\pgfqpoint{4.407273in}{4.200980in}}%
\pgfpathlineto{\pgfqpoint{4.421882in}{4.186667in}}%
\pgfpathlineto{\pgfqpoint{4.434234in}{4.174447in}}%
\pgfpathlineto{\pgfqpoint{4.447354in}{4.161695in}}%
\pgfpathlineto{\pgfqpoint{4.459932in}{4.149333in}}%
\pgfpathlineto{\pgfqpoint{4.473249in}{4.136120in}}%
\pgfpathlineto{\pgfqpoint{4.487434in}{4.122289in}}%
\pgfpathlineto{\pgfqpoint{4.497873in}{4.112000in}}%
\pgfpathlineto{\pgfqpoint{4.512203in}{4.097738in}}%
\pgfpathlineto{\pgfqpoint{4.527515in}{4.082763in}}%
\pgfpathlineto{\pgfqpoint{4.535704in}{4.074667in}}%
\pgfpathlineto{\pgfqpoint{4.567596in}{4.043116in}}%
\pgfpathlineto{\pgfqpoint{4.570590in}{4.040122in}}%
\pgfpathlineto{\pgfqpoint{4.573428in}{4.037333in}}%
\pgfpathlineto{\pgfqpoint{4.589934in}{4.020807in}}%
\pgfpathlineto{\pgfqpoint{4.607677in}{4.003349in}}%
\pgfpathlineto{\pgfqpoint{4.611044in}{4.000000in}}%
\pgfpathlineto{\pgfqpoint{4.647758in}{3.963461in}}%
\pgfpathlineto{\pgfqpoint{4.648554in}{3.962667in}}%
\pgfpathlineto{\pgfqpoint{4.667428in}{3.943655in}}%
\pgfpathlineto{\pgfqpoint{4.685937in}{3.925333in}}%
\pgfpathlineto{\pgfqpoint{4.687838in}{3.923433in}}%
\pgfpathlineto{\pgfqpoint{4.723206in}{3.888000in}}%
\pgfpathlineto{\pgfqpoint{4.725465in}{3.885714in}}%
\pgfpathlineto{\pgfqpoint{4.727919in}{3.883275in}}%
\pgfpathlineto{\pgfqpoint{4.760369in}{3.850667in}}%
\pgfpathlineto{\pgfqpoint{4.764021in}{3.846961in}}%
\pgfpathlineto{\pgfqpoint{4.768000in}{3.842994in}}%
\pgfusepath{fill}%
\end{pgfscope}%
\begin{pgfscope}%
\pgfpathrectangle{\pgfqpoint{0.800000in}{0.528000in}}{\pgfqpoint{3.968000in}{3.696000in}}%
\pgfusepath{clip}%
\pgfsetbuttcap%
\pgfsetroundjoin%
\definecolor{currentfill}{rgb}{0.575563,0.844566,0.256415}%
\pgfsetfillcolor{currentfill}%
\pgfsetlinewidth{0.000000pt}%
\definecolor{currentstroke}{rgb}{0.000000,0.000000,0.000000}%
\pgfsetstrokecolor{currentstroke}%
\pgfsetdash{}{0pt}%
\pgfpathmoveto{\pgfqpoint{4.768000in}{3.848151in}}%
\pgfpathlineto{\pgfqpoint{4.766696in}{3.849452in}}%
\pgfpathlineto{\pgfqpoint{4.765499in}{3.850667in}}%
\pgfpathlineto{\pgfqpoint{4.733764in}{3.882556in}}%
\pgfpathlineto{\pgfqpoint{4.728343in}{3.888000in}}%
\pgfpathlineto{\pgfqpoint{4.728138in}{3.888204in}}%
\pgfpathlineto{\pgfqpoint{4.727919in}{3.888425in}}%
\pgfpathlineto{\pgfqpoint{4.691056in}{3.925333in}}%
\pgfpathlineto{\pgfqpoint{4.687838in}{3.928553in}}%
\pgfpathlineto{\pgfqpoint{4.670083in}{3.946128in}}%
\pgfpathlineto{\pgfqpoint{4.653664in}{3.962667in}}%
\pgfpathlineto{\pgfqpoint{4.647758in}{3.968559in}}%
\pgfpathlineto{\pgfqpoint{4.616166in}{4.000000in}}%
\pgfpathlineto{\pgfqpoint{4.607677in}{4.008444in}}%
\pgfpathlineto{\pgfqpoint{4.592594in}{4.023284in}}%
\pgfpathlineto{\pgfqpoint{4.578562in}{4.037333in}}%
\pgfpathlineto{\pgfqpoint{4.573226in}{4.042578in}}%
\pgfpathlineto{\pgfqpoint{4.567596in}{4.048208in}}%
\pgfpathlineto{\pgfqpoint{4.540851in}{4.074667in}}%
\pgfpathlineto{\pgfqpoint{4.527515in}{4.087851in}}%
\pgfpathlineto{\pgfqpoint{4.514868in}{4.100220in}}%
\pgfpathlineto{\pgfqpoint{4.503032in}{4.112000in}}%
\pgfpathlineto{\pgfqpoint{4.487434in}{4.127374in}}%
\pgfpathlineto{\pgfqpoint{4.475916in}{4.138604in}}%
\pgfpathlineto{\pgfqpoint{4.465103in}{4.149333in}}%
\pgfpathlineto{\pgfqpoint{4.447354in}{4.166777in}}%
\pgfpathlineto{\pgfqpoint{4.436904in}{4.176934in}}%
\pgfpathlineto{\pgfqpoint{4.427066in}{4.186667in}}%
\pgfpathlineto{\pgfqpoint{4.407273in}{4.206059in}}%
\pgfpathlineto{\pgfqpoint{4.397833in}{4.215207in}}%
\pgfpathlineto{\pgfqpoint{4.388917in}{4.224000in}}%
\pgfpathlineto{\pgfqpoint{4.386319in}{4.224000in}}%
\pgfpathlineto{\pgfqpoint{4.396497in}{4.213963in}}%
\pgfpathlineto{\pgfqpoint{4.407273in}{4.203520in}}%
\pgfpathlineto{\pgfqpoint{4.424474in}{4.186667in}}%
\pgfpathlineto{\pgfqpoint{4.435569in}{4.175690in}}%
\pgfpathlineto{\pgfqpoint{4.447354in}{4.164236in}}%
\pgfpathlineto{\pgfqpoint{4.462518in}{4.149333in}}%
\pgfpathlineto{\pgfqpoint{4.474582in}{4.137362in}}%
\pgfpathlineto{\pgfqpoint{4.487434in}{4.124832in}}%
\pgfpathlineto{\pgfqpoint{4.500452in}{4.112000in}}%
\pgfpathlineto{\pgfqpoint{4.513536in}{4.098979in}}%
\pgfpathlineto{\pgfqpoint{4.527515in}{4.085307in}}%
\pgfpathlineto{\pgfqpoint{4.538278in}{4.074667in}}%
\pgfpathlineto{\pgfqpoint{4.567596in}{4.045662in}}%
\pgfpathlineto{\pgfqpoint{4.571908in}{4.041350in}}%
\pgfpathlineto{\pgfqpoint{4.575995in}{4.037333in}}%
\pgfpathlineto{\pgfqpoint{4.591264in}{4.022045in}}%
\pgfpathlineto{\pgfqpoint{4.607677in}{4.005897in}}%
\pgfpathlineto{\pgfqpoint{4.613605in}{4.000000in}}%
\pgfpathlineto{\pgfqpoint{4.647758in}{3.966010in}}%
\pgfpathlineto{\pgfqpoint{4.651109in}{3.962667in}}%
\pgfpathlineto{\pgfqpoint{4.668755in}{3.944892in}}%
\pgfpathlineto{\pgfqpoint{4.687838in}{3.926002in}}%
\pgfpathlineto{\pgfqpoint{4.688507in}{3.925333in}}%
\pgfpathlineto{\pgfqpoint{4.697429in}{3.916400in}}%
\pgfpathlineto{\pgfqpoint{4.725777in}{3.888000in}}%
\pgfpathlineto{\pgfqpoint{4.726804in}{3.886961in}}%
\pgfpathlineto{\pgfqpoint{4.727919in}{3.885852in}}%
\pgfpathlineto{\pgfqpoint{4.762934in}{3.850667in}}%
\pgfpathlineto{\pgfqpoint{4.765359in}{3.848206in}}%
\pgfpathlineto{\pgfqpoint{4.768000in}{3.845573in}}%
\pgfusepath{fill}%
\end{pgfscope}%
\begin{pgfscope}%
\pgfpathrectangle{\pgfqpoint{0.800000in}{0.528000in}}{\pgfqpoint{3.968000in}{3.696000in}}%
\pgfusepath{clip}%
\pgfsetbuttcap%
\pgfsetroundjoin%
\definecolor{currentfill}{rgb}{0.575563,0.844566,0.256415}%
\pgfsetfillcolor{currentfill}%
\pgfsetlinewidth{0.000000pt}%
\definecolor{currentstroke}{rgb}{0.000000,0.000000,0.000000}%
\pgfsetstrokecolor{currentstroke}%
\pgfsetdash{}{0pt}%
\pgfpathmoveto{\pgfqpoint{4.768000in}{3.850730in}}%
\pgfpathlineto{\pgfqpoint{4.730886in}{3.888000in}}%
\pgfpathlineto{\pgfqpoint{4.729451in}{3.889427in}}%
\pgfpathlineto{\pgfqpoint{4.727919in}{3.890977in}}%
\pgfpathlineto{\pgfqpoint{4.693605in}{3.925333in}}%
\pgfpathlineto{\pgfqpoint{4.687838in}{3.931103in}}%
\pgfpathlineto{\pgfqpoint{4.671410in}{3.947365in}}%
\pgfpathlineto{\pgfqpoint{4.656219in}{3.962667in}}%
\pgfpathlineto{\pgfqpoint{4.647758in}{3.971108in}}%
\pgfpathlineto{\pgfqpoint{4.618728in}{4.000000in}}%
\pgfpathlineto{\pgfqpoint{4.607677in}{4.010991in}}%
\pgfpathlineto{\pgfqpoint{4.593924in}{4.024523in}}%
\pgfpathlineto{\pgfqpoint{4.581130in}{4.037333in}}%
\pgfpathlineto{\pgfqpoint{4.574544in}{4.043805in}}%
\pgfpathlineto{\pgfqpoint{4.567596in}{4.050754in}}%
\pgfpathlineto{\pgfqpoint{4.543424in}{4.074667in}}%
\pgfpathlineto{\pgfqpoint{4.527515in}{4.090395in}}%
\pgfpathlineto{\pgfqpoint{4.516200in}{4.101461in}}%
\pgfpathlineto{\pgfqpoint{4.505611in}{4.112000in}}%
\pgfpathlineto{\pgfqpoint{4.487434in}{4.129917in}}%
\pgfpathlineto{\pgfqpoint{4.477250in}{4.139847in}}%
\pgfpathlineto{\pgfqpoint{4.467689in}{4.149333in}}%
\pgfpathlineto{\pgfqpoint{4.447354in}{4.169317in}}%
\pgfpathlineto{\pgfqpoint{4.438239in}{4.178177in}}%
\pgfpathlineto{\pgfqpoint{4.429657in}{4.186667in}}%
\pgfpathlineto{\pgfqpoint{4.407273in}{4.208598in}}%
\pgfpathlineto{\pgfqpoint{4.399169in}{4.216452in}}%
\pgfpathlineto{\pgfqpoint{4.391515in}{4.224000in}}%
\pgfpathlineto{\pgfqpoint{4.388917in}{4.224000in}}%
\pgfpathlineto{\pgfqpoint{4.397833in}{4.215207in}}%
\pgfpathlineto{\pgfqpoint{4.407273in}{4.206059in}}%
\pgfpathlineto{\pgfqpoint{4.427066in}{4.186667in}}%
\pgfpathlineto{\pgfqpoint{4.436904in}{4.176934in}}%
\pgfpathlineto{\pgfqpoint{4.447354in}{4.166777in}}%
\pgfpathlineto{\pgfqpoint{4.465103in}{4.149333in}}%
\pgfpathlineto{\pgfqpoint{4.475916in}{4.138604in}}%
\pgfpathlineto{\pgfqpoint{4.487434in}{4.127374in}}%
\pgfpathlineto{\pgfqpoint{4.503032in}{4.112000in}}%
\pgfpathlineto{\pgfqpoint{4.514868in}{4.100220in}}%
\pgfpathlineto{\pgfqpoint{4.527515in}{4.087851in}}%
\pgfpathlineto{\pgfqpoint{4.540851in}{4.074667in}}%
\pgfpathlineto{\pgfqpoint{4.567596in}{4.048208in}}%
\pgfpathlineto{\pgfqpoint{4.573226in}{4.042578in}}%
\pgfpathlineto{\pgfqpoint{4.578562in}{4.037333in}}%
\pgfpathlineto{\pgfqpoint{4.592594in}{4.023284in}}%
\pgfpathlineto{\pgfqpoint{4.607677in}{4.008444in}}%
\pgfpathlineto{\pgfqpoint{4.616166in}{4.000000in}}%
\pgfpathlineto{\pgfqpoint{4.647758in}{3.968559in}}%
\pgfpathlineto{\pgfqpoint{4.653664in}{3.962667in}}%
\pgfpathlineto{\pgfqpoint{4.670083in}{3.946128in}}%
\pgfpathlineto{\pgfqpoint{4.687838in}{3.928553in}}%
\pgfpathlineto{\pgfqpoint{4.691056in}{3.925333in}}%
\pgfpathlineto{\pgfqpoint{4.727919in}{3.888425in}}%
\pgfpathlineto{\pgfqpoint{4.728138in}{3.888204in}}%
\pgfpathlineto{\pgfqpoint{4.728343in}{3.888000in}}%
\pgfpathlineto{\pgfqpoint{4.733764in}{3.882556in}}%
\pgfpathlineto{\pgfqpoint{4.765499in}{3.850667in}}%
\pgfpathlineto{\pgfqpoint{4.766696in}{3.849452in}}%
\pgfpathlineto{\pgfqpoint{4.768000in}{3.848151in}}%
\pgfpathlineto{\pgfqpoint{4.768000in}{3.850667in}}%
\pgfusepath{fill}%
\end{pgfscope}%
\begin{pgfscope}%
\pgfpathrectangle{\pgfqpoint{0.800000in}{0.528000in}}{\pgfqpoint{3.968000in}{3.696000in}}%
\pgfusepath{clip}%
\pgfsetbuttcap%
\pgfsetroundjoin%
\definecolor{currentfill}{rgb}{0.585678,0.846661,0.249897}%
\pgfsetfillcolor{currentfill}%
\pgfsetlinewidth{0.000000pt}%
\definecolor{currentstroke}{rgb}{0.000000,0.000000,0.000000}%
\pgfsetstrokecolor{currentstroke}%
\pgfsetdash{}{0pt}%
\pgfpathmoveto{\pgfqpoint{4.768000in}{3.853283in}}%
\pgfpathlineto{\pgfqpoint{4.733429in}{3.888000in}}%
\pgfpathlineto{\pgfqpoint{4.730764in}{3.890650in}}%
\pgfpathlineto{\pgfqpoint{4.727919in}{3.893530in}}%
\pgfpathlineto{\pgfqpoint{4.696154in}{3.925333in}}%
\pgfpathlineto{\pgfqpoint{4.687838in}{3.933654in}}%
\pgfpathlineto{\pgfqpoint{4.672738in}{3.948601in}}%
\pgfpathlineto{\pgfqpoint{4.658774in}{3.962667in}}%
\pgfpathlineto{\pgfqpoint{4.647758in}{3.973657in}}%
\pgfpathlineto{\pgfqpoint{4.621289in}{4.000000in}}%
\pgfpathlineto{\pgfqpoint{4.607677in}{4.013539in}}%
\pgfpathlineto{\pgfqpoint{4.595254in}{4.025762in}}%
\pgfpathlineto{\pgfqpoint{4.583697in}{4.037333in}}%
\pgfpathlineto{\pgfqpoint{4.575862in}{4.045033in}}%
\pgfpathlineto{\pgfqpoint{4.567596in}{4.053300in}}%
\pgfpathlineto{\pgfqpoint{4.545998in}{4.074667in}}%
\pgfpathlineto{\pgfqpoint{4.527515in}{4.092940in}}%
\pgfpathlineto{\pgfqpoint{4.517533in}{4.102702in}}%
\pgfpathlineto{\pgfqpoint{4.508190in}{4.112000in}}%
\pgfpathlineto{\pgfqpoint{4.487434in}{4.132459in}}%
\pgfpathlineto{\pgfqpoint{4.478583in}{4.141089in}}%
\pgfpathlineto{\pgfqpoint{4.470274in}{4.149333in}}%
\pgfpathlineto{\pgfqpoint{4.447354in}{4.171858in}}%
\pgfpathlineto{\pgfqpoint{4.439574in}{4.179420in}}%
\pgfpathlineto{\pgfqpoint{4.432249in}{4.186667in}}%
\pgfpathlineto{\pgfqpoint{4.407273in}{4.211138in}}%
\pgfpathlineto{\pgfqpoint{4.400505in}{4.217696in}}%
\pgfpathlineto{\pgfqpoint{4.394113in}{4.224000in}}%
\pgfpathlineto{\pgfqpoint{4.391515in}{4.224000in}}%
\pgfpathlineto{\pgfqpoint{4.399169in}{4.216452in}}%
\pgfpathlineto{\pgfqpoint{4.407273in}{4.208598in}}%
\pgfpathlineto{\pgfqpoint{4.429657in}{4.186667in}}%
\pgfpathlineto{\pgfqpoint{4.438239in}{4.178177in}}%
\pgfpathlineto{\pgfqpoint{4.447354in}{4.169317in}}%
\pgfpathlineto{\pgfqpoint{4.467689in}{4.149333in}}%
\pgfpathlineto{\pgfqpoint{4.477250in}{4.139847in}}%
\pgfpathlineto{\pgfqpoint{4.487434in}{4.129917in}}%
\pgfpathlineto{\pgfqpoint{4.505611in}{4.112000in}}%
\pgfpathlineto{\pgfqpoint{4.516200in}{4.101461in}}%
\pgfpathlineto{\pgfqpoint{4.527515in}{4.090395in}}%
\pgfpathlineto{\pgfqpoint{4.543424in}{4.074667in}}%
\pgfpathlineto{\pgfqpoint{4.567596in}{4.050754in}}%
\pgfpathlineto{\pgfqpoint{4.574544in}{4.043805in}}%
\pgfpathlineto{\pgfqpoint{4.581130in}{4.037333in}}%
\pgfpathlineto{\pgfqpoint{4.593924in}{4.024523in}}%
\pgfpathlineto{\pgfqpoint{4.607677in}{4.010991in}}%
\pgfpathlineto{\pgfqpoint{4.618728in}{4.000000in}}%
\pgfpathlineto{\pgfqpoint{4.647758in}{3.971108in}}%
\pgfpathlineto{\pgfqpoint{4.656219in}{3.962667in}}%
\pgfpathlineto{\pgfqpoint{4.671410in}{3.947365in}}%
\pgfpathlineto{\pgfqpoint{4.687838in}{3.931103in}}%
\pgfpathlineto{\pgfqpoint{4.693605in}{3.925333in}}%
\pgfpathlineto{\pgfqpoint{4.727919in}{3.890977in}}%
\pgfpathlineto{\pgfqpoint{4.729451in}{3.889427in}}%
\pgfpathlineto{\pgfqpoint{4.730886in}{3.888000in}}%
\pgfpathlineto{\pgfqpoint{4.768000in}{3.850730in}}%
\pgfusepath{fill}%
\end{pgfscope}%
\begin{pgfscope}%
\pgfpathrectangle{\pgfqpoint{0.800000in}{0.528000in}}{\pgfqpoint{3.968000in}{3.696000in}}%
\pgfusepath{clip}%
\pgfsetbuttcap%
\pgfsetroundjoin%
\definecolor{currentfill}{rgb}{0.585678,0.846661,0.249897}%
\pgfsetfillcolor{currentfill}%
\pgfsetlinewidth{0.000000pt}%
\definecolor{currentstroke}{rgb}{0.000000,0.000000,0.000000}%
\pgfsetstrokecolor{currentstroke}%
\pgfsetdash{}{0pt}%
\pgfpathmoveto{\pgfqpoint{4.768000in}{3.855837in}}%
\pgfpathlineto{\pgfqpoint{4.735972in}{3.888000in}}%
\pgfpathlineto{\pgfqpoint{4.732078in}{3.891874in}}%
\pgfpathlineto{\pgfqpoint{4.727919in}{3.896082in}}%
\pgfpathlineto{\pgfqpoint{4.698703in}{3.925333in}}%
\pgfpathlineto{\pgfqpoint{4.687838in}{3.936205in}}%
\pgfpathlineto{\pgfqpoint{4.674065in}{3.949838in}}%
\pgfpathlineto{\pgfqpoint{4.661329in}{3.962667in}}%
\pgfpathlineto{\pgfqpoint{4.647758in}{3.976206in}}%
\pgfpathlineto{\pgfqpoint{4.623850in}{4.000000in}}%
\pgfpathlineto{\pgfqpoint{4.607677in}{4.016086in}}%
\pgfpathlineto{\pgfqpoint{4.596584in}{4.027001in}}%
\pgfpathlineto{\pgfqpoint{4.586264in}{4.037333in}}%
\pgfpathlineto{\pgfqpoint{4.577180in}{4.046261in}}%
\pgfpathlineto{\pgfqpoint{4.567596in}{4.055845in}}%
\pgfpathlineto{\pgfqpoint{4.548571in}{4.074667in}}%
\pgfpathlineto{\pgfqpoint{4.527515in}{4.095484in}}%
\pgfpathlineto{\pgfqpoint{4.518865in}{4.103943in}}%
\pgfpathlineto{\pgfqpoint{4.510770in}{4.112000in}}%
\pgfpathlineto{\pgfqpoint{4.487434in}{4.135002in}}%
\pgfpathlineto{\pgfqpoint{4.479917in}{4.142331in}}%
\pgfpathlineto{\pgfqpoint{4.472860in}{4.149333in}}%
\pgfpathlineto{\pgfqpoint{4.447354in}{4.174399in}}%
\pgfpathlineto{\pgfqpoint{4.440909in}{4.180664in}}%
\pgfpathlineto{\pgfqpoint{4.434841in}{4.186667in}}%
\pgfpathlineto{\pgfqpoint{4.407273in}{4.213677in}}%
\pgfpathlineto{\pgfqpoint{4.401841in}{4.218941in}}%
\pgfpathlineto{\pgfqpoint{4.396711in}{4.224000in}}%
\pgfpathlineto{\pgfqpoint{4.394113in}{4.224000in}}%
\pgfpathlineto{\pgfqpoint{4.400505in}{4.217696in}}%
\pgfpathlineto{\pgfqpoint{4.407273in}{4.211138in}}%
\pgfpathlineto{\pgfqpoint{4.432249in}{4.186667in}}%
\pgfpathlineto{\pgfqpoint{4.439574in}{4.179420in}}%
\pgfpathlineto{\pgfqpoint{4.447354in}{4.171858in}}%
\pgfpathlineto{\pgfqpoint{4.470274in}{4.149333in}}%
\pgfpathlineto{\pgfqpoint{4.478583in}{4.141089in}}%
\pgfpathlineto{\pgfqpoint{4.487434in}{4.132459in}}%
\pgfpathlineto{\pgfqpoint{4.508190in}{4.112000in}}%
\pgfpathlineto{\pgfqpoint{4.517533in}{4.102702in}}%
\pgfpathlineto{\pgfqpoint{4.527515in}{4.092940in}}%
\pgfpathlineto{\pgfqpoint{4.545998in}{4.074667in}}%
\pgfpathlineto{\pgfqpoint{4.567596in}{4.053300in}}%
\pgfpathlineto{\pgfqpoint{4.575862in}{4.045033in}}%
\pgfpathlineto{\pgfqpoint{4.583697in}{4.037333in}}%
\pgfpathlineto{\pgfqpoint{4.595254in}{4.025762in}}%
\pgfpathlineto{\pgfqpoint{4.607677in}{4.013539in}}%
\pgfpathlineto{\pgfqpoint{4.621289in}{4.000000in}}%
\pgfpathlineto{\pgfqpoint{4.647758in}{3.973657in}}%
\pgfpathlineto{\pgfqpoint{4.658774in}{3.962667in}}%
\pgfpathlineto{\pgfqpoint{4.672738in}{3.948601in}}%
\pgfpathlineto{\pgfqpoint{4.687838in}{3.933654in}}%
\pgfpathlineto{\pgfqpoint{4.696154in}{3.925333in}}%
\pgfpathlineto{\pgfqpoint{4.727919in}{3.893530in}}%
\pgfpathlineto{\pgfqpoint{4.730764in}{3.890650in}}%
\pgfpathlineto{\pgfqpoint{4.733429in}{3.888000in}}%
\pgfpathlineto{\pgfqpoint{4.768000in}{3.853283in}}%
\pgfusepath{fill}%
\end{pgfscope}%
\begin{pgfscope}%
\pgfpathrectangle{\pgfqpoint{0.800000in}{0.528000in}}{\pgfqpoint{3.968000in}{3.696000in}}%
\pgfusepath{clip}%
\pgfsetbuttcap%
\pgfsetroundjoin%
\definecolor{currentfill}{rgb}{0.585678,0.846661,0.249897}%
\pgfsetfillcolor{currentfill}%
\pgfsetlinewidth{0.000000pt}%
\definecolor{currentstroke}{rgb}{0.000000,0.000000,0.000000}%
\pgfsetstrokecolor{currentstroke}%
\pgfsetdash{}{0pt}%
\pgfpathmoveto{\pgfqpoint{4.768000in}{3.858391in}}%
\pgfpathlineto{\pgfqpoint{4.738515in}{3.888000in}}%
\pgfpathlineto{\pgfqpoint{4.733391in}{3.893097in}}%
\pgfpathlineto{\pgfqpoint{4.727919in}{3.898634in}}%
\pgfpathlineto{\pgfqpoint{4.701252in}{3.925333in}}%
\pgfpathlineto{\pgfqpoint{4.687838in}{3.938755in}}%
\pgfpathlineto{\pgfqpoint{4.675393in}{3.951074in}}%
\pgfpathlineto{\pgfqpoint{4.663885in}{3.962667in}}%
\pgfpathlineto{\pgfqpoint{4.647758in}{3.978755in}}%
\pgfpathlineto{\pgfqpoint{4.626411in}{4.000000in}}%
\pgfpathlineto{\pgfqpoint{4.607677in}{4.018633in}}%
\pgfpathlineto{\pgfqpoint{4.597914in}{4.028239in}}%
\pgfpathlineto{\pgfqpoint{4.588831in}{4.037333in}}%
\pgfpathlineto{\pgfqpoint{4.578498in}{4.047488in}}%
\pgfpathlineto{\pgfqpoint{4.567596in}{4.058391in}}%
\pgfpathlineto{\pgfqpoint{4.551144in}{4.074667in}}%
\pgfpathlineto{\pgfqpoint{4.527515in}{4.098028in}}%
\pgfpathlineto{\pgfqpoint{4.520198in}{4.105184in}}%
\pgfpathlineto{\pgfqpoint{4.513349in}{4.112000in}}%
\pgfpathlineto{\pgfqpoint{4.487434in}{4.137544in}}%
\pgfpathlineto{\pgfqpoint{4.481250in}{4.143573in}}%
\pgfpathlineto{\pgfqpoint{4.475446in}{4.149333in}}%
\pgfpathlineto{\pgfqpoint{4.447354in}{4.176940in}}%
\pgfpathlineto{\pgfqpoint{4.442244in}{4.181907in}}%
\pgfpathlineto{\pgfqpoint{4.437432in}{4.186667in}}%
\pgfpathlineto{\pgfqpoint{4.407273in}{4.216216in}}%
\pgfpathlineto{\pgfqpoint{4.403177in}{4.220185in}}%
\pgfpathlineto{\pgfqpoint{4.399309in}{4.224000in}}%
\pgfpathlineto{\pgfqpoint{4.396711in}{4.224000in}}%
\pgfpathlineto{\pgfqpoint{4.401841in}{4.218941in}}%
\pgfpathlineto{\pgfqpoint{4.407273in}{4.213677in}}%
\pgfpathlineto{\pgfqpoint{4.434841in}{4.186667in}}%
\pgfpathlineto{\pgfqpoint{4.440909in}{4.180664in}}%
\pgfpathlineto{\pgfqpoint{4.447354in}{4.174399in}}%
\pgfpathlineto{\pgfqpoint{4.472860in}{4.149333in}}%
\pgfpathlineto{\pgfqpoint{4.479917in}{4.142331in}}%
\pgfpathlineto{\pgfqpoint{4.487434in}{4.135002in}}%
\pgfpathlineto{\pgfqpoint{4.510770in}{4.112000in}}%
\pgfpathlineto{\pgfqpoint{4.518865in}{4.103943in}}%
\pgfpathlineto{\pgfqpoint{4.527515in}{4.095484in}}%
\pgfpathlineto{\pgfqpoint{4.548571in}{4.074667in}}%
\pgfpathlineto{\pgfqpoint{4.567596in}{4.055845in}}%
\pgfpathlineto{\pgfqpoint{4.577180in}{4.046261in}}%
\pgfpathlineto{\pgfqpoint{4.586264in}{4.037333in}}%
\pgfpathlineto{\pgfqpoint{4.596584in}{4.027001in}}%
\pgfpathlineto{\pgfqpoint{4.607677in}{4.016086in}}%
\pgfpathlineto{\pgfqpoint{4.623850in}{4.000000in}}%
\pgfpathlineto{\pgfqpoint{4.647758in}{3.976206in}}%
\pgfpathlineto{\pgfqpoint{4.661329in}{3.962667in}}%
\pgfpathlineto{\pgfqpoint{4.674065in}{3.949838in}}%
\pgfpathlineto{\pgfqpoint{4.687838in}{3.936205in}}%
\pgfpathlineto{\pgfqpoint{4.698703in}{3.925333in}}%
\pgfpathlineto{\pgfqpoint{4.727919in}{3.896082in}}%
\pgfpathlineto{\pgfqpoint{4.732078in}{3.891874in}}%
\pgfpathlineto{\pgfqpoint{4.735972in}{3.888000in}}%
\pgfpathlineto{\pgfqpoint{4.768000in}{3.855837in}}%
\pgfusepath{fill}%
\end{pgfscope}%
\begin{pgfscope}%
\pgfpathrectangle{\pgfqpoint{0.800000in}{0.528000in}}{\pgfqpoint{3.968000in}{3.696000in}}%
\pgfusepath{clip}%
\pgfsetbuttcap%
\pgfsetroundjoin%
\definecolor{currentfill}{rgb}{0.585678,0.846661,0.249897}%
\pgfsetfillcolor{currentfill}%
\pgfsetlinewidth{0.000000pt}%
\definecolor{currentstroke}{rgb}{0.000000,0.000000,0.000000}%
\pgfsetstrokecolor{currentstroke}%
\pgfsetdash{}{0pt}%
\pgfpathmoveto{\pgfqpoint{4.768000in}{3.860945in}}%
\pgfpathlineto{\pgfqpoint{4.741059in}{3.888000in}}%
\pgfpathlineto{\pgfqpoint{4.734704in}{3.894320in}}%
\pgfpathlineto{\pgfqpoint{4.727919in}{3.901186in}}%
\pgfpathlineto{\pgfqpoint{4.703802in}{3.925333in}}%
\pgfpathlineto{\pgfqpoint{4.687838in}{3.941306in}}%
\pgfpathlineto{\pgfqpoint{4.676721in}{3.952311in}}%
\pgfpathlineto{\pgfqpoint{4.666440in}{3.962667in}}%
\pgfpathlineto{\pgfqpoint{4.647758in}{3.981304in}}%
\pgfpathlineto{\pgfqpoint{4.628972in}{4.000000in}}%
\pgfpathlineto{\pgfqpoint{4.607677in}{4.021181in}}%
\pgfpathlineto{\pgfqpoint{4.599244in}{4.029478in}}%
\pgfpathlineto{\pgfqpoint{4.591398in}{4.037333in}}%
\pgfpathlineto{\pgfqpoint{4.579817in}{4.048716in}}%
\pgfpathlineto{\pgfqpoint{4.567596in}{4.060937in}}%
\pgfpathlineto{\pgfqpoint{4.553718in}{4.074667in}}%
\pgfpathlineto{\pgfqpoint{4.527515in}{4.100572in}}%
\pgfpathlineto{\pgfqpoint{4.521530in}{4.106425in}}%
\pgfpathlineto{\pgfqpoint{4.515929in}{4.112000in}}%
\pgfpathlineto{\pgfqpoint{4.487434in}{4.140087in}}%
\pgfpathlineto{\pgfqpoint{4.482584in}{4.144816in}}%
\pgfpathlineto{\pgfqpoint{4.478031in}{4.149333in}}%
\pgfpathlineto{\pgfqpoint{4.447354in}{4.179481in}}%
\pgfpathlineto{\pgfqpoint{4.443579in}{4.183150in}}%
\pgfpathlineto{\pgfqpoint{4.440024in}{4.186667in}}%
\pgfpathlineto{\pgfqpoint{4.407273in}{4.218755in}}%
\pgfpathlineto{\pgfqpoint{4.404513in}{4.221430in}}%
\pgfpathlineto{\pgfqpoint{4.401907in}{4.224000in}}%
\pgfpathlineto{\pgfqpoint{4.399309in}{4.224000in}}%
\pgfpathlineto{\pgfqpoint{4.403177in}{4.220185in}}%
\pgfpathlineto{\pgfqpoint{4.407273in}{4.216216in}}%
\pgfpathlineto{\pgfqpoint{4.437432in}{4.186667in}}%
\pgfpathlineto{\pgfqpoint{4.442244in}{4.181907in}}%
\pgfpathlineto{\pgfqpoint{4.447354in}{4.176940in}}%
\pgfpathlineto{\pgfqpoint{4.475446in}{4.149333in}}%
\pgfpathlineto{\pgfqpoint{4.481250in}{4.143573in}}%
\pgfpathlineto{\pgfqpoint{4.487434in}{4.137544in}}%
\pgfpathlineto{\pgfqpoint{4.513349in}{4.112000in}}%
\pgfpathlineto{\pgfqpoint{4.520198in}{4.105184in}}%
\pgfpathlineto{\pgfqpoint{4.527515in}{4.098028in}}%
\pgfpathlineto{\pgfqpoint{4.551144in}{4.074667in}}%
\pgfpathlineto{\pgfqpoint{4.567596in}{4.058391in}}%
\pgfpathlineto{\pgfqpoint{4.578498in}{4.047488in}}%
\pgfpathlineto{\pgfqpoint{4.588831in}{4.037333in}}%
\pgfpathlineto{\pgfqpoint{4.597914in}{4.028239in}}%
\pgfpathlineto{\pgfqpoint{4.607677in}{4.018633in}}%
\pgfpathlineto{\pgfqpoint{4.626411in}{4.000000in}}%
\pgfpathlineto{\pgfqpoint{4.647758in}{3.978755in}}%
\pgfpathlineto{\pgfqpoint{4.663885in}{3.962667in}}%
\pgfpathlineto{\pgfqpoint{4.675393in}{3.951074in}}%
\pgfpathlineto{\pgfqpoint{4.687838in}{3.938755in}}%
\pgfpathlineto{\pgfqpoint{4.701252in}{3.925333in}}%
\pgfpathlineto{\pgfqpoint{4.727919in}{3.898634in}}%
\pgfpathlineto{\pgfqpoint{4.733391in}{3.893097in}}%
\pgfpathlineto{\pgfqpoint{4.738515in}{3.888000in}}%
\pgfpathlineto{\pgfqpoint{4.768000in}{3.858391in}}%
\pgfusepath{fill}%
\end{pgfscope}%
\begin{pgfscope}%
\pgfpathrectangle{\pgfqpoint{0.800000in}{0.528000in}}{\pgfqpoint{3.968000in}{3.696000in}}%
\pgfusepath{clip}%
\pgfsetbuttcap%
\pgfsetroundjoin%
\definecolor{currentfill}{rgb}{0.595839,0.848717,0.243329}%
\pgfsetfillcolor{currentfill}%
\pgfsetlinewidth{0.000000pt}%
\definecolor{currentstroke}{rgb}{0.000000,0.000000,0.000000}%
\pgfsetstrokecolor{currentstroke}%
\pgfsetdash{}{0pt}%
\pgfpathmoveto{\pgfqpoint{4.768000in}{3.863499in}}%
\pgfpathlineto{\pgfqpoint{4.743602in}{3.888000in}}%
\pgfpathlineto{\pgfqpoint{4.736018in}{3.895543in}}%
\pgfpathlineto{\pgfqpoint{4.727919in}{3.903739in}}%
\pgfpathlineto{\pgfqpoint{4.706351in}{3.925333in}}%
\pgfpathlineto{\pgfqpoint{4.687838in}{3.943857in}}%
\pgfpathlineto{\pgfqpoint{4.678048in}{3.953547in}}%
\pgfpathlineto{\pgfqpoint{4.668995in}{3.962667in}}%
\pgfpathlineto{\pgfqpoint{4.647758in}{3.983853in}}%
\pgfpathlineto{\pgfqpoint{4.631533in}{4.000000in}}%
\pgfpathlineto{\pgfqpoint{4.607677in}{4.023728in}}%
\pgfpathlineto{\pgfqpoint{4.600574in}{4.030717in}}%
\pgfpathlineto{\pgfqpoint{4.593966in}{4.037333in}}%
\pgfpathlineto{\pgfqpoint{4.581135in}{4.049944in}}%
\pgfpathlineto{\pgfqpoint{4.567596in}{4.063483in}}%
\pgfpathlineto{\pgfqpoint{4.556291in}{4.074667in}}%
\pgfpathlineto{\pgfqpoint{4.527515in}{4.103116in}}%
\pgfpathlineto{\pgfqpoint{4.522862in}{4.107666in}}%
\pgfpathlineto{\pgfqpoint{4.518508in}{4.112000in}}%
\pgfpathlineto{\pgfqpoint{4.487434in}{4.142629in}}%
\pgfpathlineto{\pgfqpoint{4.483918in}{4.146058in}}%
\pgfpathlineto{\pgfqpoint{4.480617in}{4.149333in}}%
\pgfpathlineto{\pgfqpoint{4.447354in}{4.182022in}}%
\pgfpathlineto{\pgfqpoint{4.444913in}{4.184394in}}%
\pgfpathlineto{\pgfqpoint{4.442616in}{4.186667in}}%
\pgfpathlineto{\pgfqpoint{4.407273in}{4.221295in}}%
\pgfpathlineto{\pgfqpoint{4.405849in}{4.222674in}}%
\pgfpathlineto{\pgfqpoint{4.404505in}{4.224000in}}%
\pgfpathlineto{\pgfqpoint{4.401907in}{4.224000in}}%
\pgfpathlineto{\pgfqpoint{4.404513in}{4.221430in}}%
\pgfpathlineto{\pgfqpoint{4.407273in}{4.218755in}}%
\pgfpathlineto{\pgfqpoint{4.440024in}{4.186667in}}%
\pgfpathlineto{\pgfqpoint{4.443579in}{4.183150in}}%
\pgfpathlineto{\pgfqpoint{4.447354in}{4.179481in}}%
\pgfpathlineto{\pgfqpoint{4.478031in}{4.149333in}}%
\pgfpathlineto{\pgfqpoint{4.482584in}{4.144816in}}%
\pgfpathlineto{\pgfqpoint{4.487434in}{4.140087in}}%
\pgfpathlineto{\pgfqpoint{4.515929in}{4.112000in}}%
\pgfpathlineto{\pgfqpoint{4.521530in}{4.106425in}}%
\pgfpathlineto{\pgfqpoint{4.527515in}{4.100572in}}%
\pgfpathlineto{\pgfqpoint{4.553718in}{4.074667in}}%
\pgfpathlineto{\pgfqpoint{4.567596in}{4.060937in}}%
\pgfpathlineto{\pgfqpoint{4.579817in}{4.048716in}}%
\pgfpathlineto{\pgfqpoint{4.591398in}{4.037333in}}%
\pgfpathlineto{\pgfqpoint{4.599244in}{4.029478in}}%
\pgfpathlineto{\pgfqpoint{4.607677in}{4.021181in}}%
\pgfpathlineto{\pgfqpoint{4.628972in}{4.000000in}}%
\pgfpathlineto{\pgfqpoint{4.647758in}{3.981304in}}%
\pgfpathlineto{\pgfqpoint{4.666440in}{3.962667in}}%
\pgfpathlineto{\pgfqpoint{4.676721in}{3.952311in}}%
\pgfpathlineto{\pgfqpoint{4.687838in}{3.941306in}}%
\pgfpathlineto{\pgfqpoint{4.703802in}{3.925333in}}%
\pgfpathlineto{\pgfqpoint{4.727919in}{3.901186in}}%
\pgfpathlineto{\pgfqpoint{4.734704in}{3.894320in}}%
\pgfpathlineto{\pgfqpoint{4.741059in}{3.888000in}}%
\pgfpathlineto{\pgfqpoint{4.768000in}{3.860945in}}%
\pgfusepath{fill}%
\end{pgfscope}%
\begin{pgfscope}%
\pgfpathrectangle{\pgfqpoint{0.800000in}{0.528000in}}{\pgfqpoint{3.968000in}{3.696000in}}%
\pgfusepath{clip}%
\pgfsetbuttcap%
\pgfsetroundjoin%
\definecolor{currentfill}{rgb}{0.595839,0.848717,0.243329}%
\pgfsetfillcolor{currentfill}%
\pgfsetlinewidth{0.000000pt}%
\definecolor{currentstroke}{rgb}{0.000000,0.000000,0.000000}%
\pgfsetstrokecolor{currentstroke}%
\pgfsetdash{}{0pt}%
\pgfpathmoveto{\pgfqpoint{4.768000in}{3.866053in}}%
\pgfpathlineto{\pgfqpoint{4.746145in}{3.888000in}}%
\pgfpathlineto{\pgfqpoint{4.737331in}{3.896767in}}%
\pgfpathlineto{\pgfqpoint{4.727919in}{3.906291in}}%
\pgfpathlineto{\pgfqpoint{4.708900in}{3.925333in}}%
\pgfpathlineto{\pgfqpoint{4.687838in}{3.946407in}}%
\pgfpathlineto{\pgfqpoint{4.679376in}{3.954784in}}%
\pgfpathlineto{\pgfqpoint{4.671550in}{3.962667in}}%
\pgfpathlineto{\pgfqpoint{4.647758in}{3.986402in}}%
\pgfpathlineto{\pgfqpoint{4.634095in}{4.000000in}}%
\pgfpathlineto{\pgfqpoint{4.607677in}{4.026276in}}%
\pgfpathlineto{\pgfqpoint{4.601904in}{4.031956in}}%
\pgfpathlineto{\pgfqpoint{4.596533in}{4.037333in}}%
\pgfpathlineto{\pgfqpoint{4.582453in}{4.051172in}}%
\pgfpathlineto{\pgfqpoint{4.567596in}{4.066028in}}%
\pgfpathlineto{\pgfqpoint{4.558864in}{4.074667in}}%
\pgfpathlineto{\pgfqpoint{4.527515in}{4.105660in}}%
\pgfpathlineto{\pgfqpoint{4.524195in}{4.108907in}}%
\pgfpathlineto{\pgfqpoint{4.521088in}{4.112000in}}%
\pgfpathlineto{\pgfqpoint{4.487434in}{4.145172in}}%
\pgfpathlineto{\pgfqpoint{4.485251in}{4.147300in}}%
\pgfpathlineto{\pgfqpoint{4.483202in}{4.149333in}}%
\pgfpathlineto{\pgfqpoint{4.447354in}{4.184563in}}%
\pgfpathlineto{\pgfqpoint{4.446248in}{4.185637in}}%
\pgfpathlineto{\pgfqpoint{4.445208in}{4.186667in}}%
\pgfpathlineto{\pgfqpoint{4.407273in}{4.223834in}}%
\pgfpathlineto{\pgfqpoint{4.407185in}{4.223919in}}%
\pgfpathlineto{\pgfqpoint{4.407103in}{4.224000in}}%
\pgfpathlineto{\pgfqpoint{4.404505in}{4.224000in}}%
\pgfpathlineto{\pgfqpoint{4.405849in}{4.222674in}}%
\pgfpathlineto{\pgfqpoint{4.407273in}{4.221295in}}%
\pgfpathlineto{\pgfqpoint{4.442616in}{4.186667in}}%
\pgfpathlineto{\pgfqpoint{4.444913in}{4.184394in}}%
\pgfpathlineto{\pgfqpoint{4.447354in}{4.182022in}}%
\pgfpathlineto{\pgfqpoint{4.480617in}{4.149333in}}%
\pgfpathlineto{\pgfqpoint{4.483918in}{4.146058in}}%
\pgfpathlineto{\pgfqpoint{4.487434in}{4.142629in}}%
\pgfpathlineto{\pgfqpoint{4.518508in}{4.112000in}}%
\pgfpathlineto{\pgfqpoint{4.522862in}{4.107666in}}%
\pgfpathlineto{\pgfqpoint{4.527515in}{4.103116in}}%
\pgfpathlineto{\pgfqpoint{4.556291in}{4.074667in}}%
\pgfpathlineto{\pgfqpoint{4.567596in}{4.063483in}}%
\pgfpathlineto{\pgfqpoint{4.581135in}{4.049944in}}%
\pgfpathlineto{\pgfqpoint{4.593966in}{4.037333in}}%
\pgfpathlineto{\pgfqpoint{4.600574in}{4.030717in}}%
\pgfpathlineto{\pgfqpoint{4.607677in}{4.023728in}}%
\pgfpathlineto{\pgfqpoint{4.631533in}{4.000000in}}%
\pgfpathlineto{\pgfqpoint{4.647758in}{3.983853in}}%
\pgfpathlineto{\pgfqpoint{4.668995in}{3.962667in}}%
\pgfpathlineto{\pgfqpoint{4.678048in}{3.953547in}}%
\pgfpathlineto{\pgfqpoint{4.687838in}{3.943857in}}%
\pgfpathlineto{\pgfqpoint{4.706351in}{3.925333in}}%
\pgfpathlineto{\pgfqpoint{4.727919in}{3.903739in}}%
\pgfpathlineto{\pgfqpoint{4.736018in}{3.895543in}}%
\pgfpathlineto{\pgfqpoint{4.743602in}{3.888000in}}%
\pgfpathlineto{\pgfqpoint{4.768000in}{3.863499in}}%
\pgfusepath{fill}%
\end{pgfscope}%
\begin{pgfscope}%
\pgfpathrectangle{\pgfqpoint{0.800000in}{0.528000in}}{\pgfqpoint{3.968000in}{3.696000in}}%
\pgfusepath{clip}%
\pgfsetbuttcap%
\pgfsetroundjoin%
\definecolor{currentfill}{rgb}{0.595839,0.848717,0.243329}%
\pgfsetfillcolor{currentfill}%
\pgfsetlinewidth{0.000000pt}%
\definecolor{currentstroke}{rgb}{0.000000,0.000000,0.000000}%
\pgfsetstrokecolor{currentstroke}%
\pgfsetdash{}{0pt}%
\pgfpathmoveto{\pgfqpoint{4.768000in}{3.868607in}}%
\pgfpathlineto{\pgfqpoint{4.748688in}{3.888000in}}%
\pgfpathlineto{\pgfqpoint{4.738644in}{3.897990in}}%
\pgfpathlineto{\pgfqpoint{4.727919in}{3.908843in}}%
\pgfpathlineto{\pgfqpoint{4.711449in}{3.925333in}}%
\pgfpathlineto{\pgfqpoint{4.687838in}{3.948958in}}%
\pgfpathlineto{\pgfqpoint{4.680703in}{3.956021in}}%
\pgfpathlineto{\pgfqpoint{4.674105in}{3.962667in}}%
\pgfpathlineto{\pgfqpoint{4.647758in}{3.988951in}}%
\pgfpathlineto{\pgfqpoint{4.636656in}{4.000000in}}%
\pgfpathlineto{\pgfqpoint{4.607677in}{4.028823in}}%
\pgfpathlineto{\pgfqpoint{4.603234in}{4.033195in}}%
\pgfpathlineto{\pgfqpoint{4.599100in}{4.037333in}}%
\pgfpathlineto{\pgfqpoint{4.583771in}{4.052399in}}%
\pgfpathlineto{\pgfqpoint{4.567596in}{4.068574in}}%
\pgfpathlineto{\pgfqpoint{4.561437in}{4.074667in}}%
\pgfpathlineto{\pgfqpoint{4.527515in}{4.108204in}}%
\pgfpathlineto{\pgfqpoint{4.525527in}{4.110148in}}%
\pgfpathlineto{\pgfqpoint{4.523667in}{4.112000in}}%
\pgfpathlineto{\pgfqpoint{4.487434in}{4.147714in}}%
\pgfpathlineto{\pgfqpoint{4.486585in}{4.148542in}}%
\pgfpathlineto{\pgfqpoint{4.485788in}{4.149333in}}%
\pgfpathlineto{\pgfqpoint{4.455878in}{4.178726in}}%
\pgfpathlineto{\pgfqpoint{4.447795in}{4.186667in}}%
\pgfpathlineto{\pgfqpoint{4.447579in}{4.186877in}}%
\pgfpathlineto{\pgfqpoint{4.447354in}{4.187100in}}%
\pgfpathlineto{\pgfqpoint{4.409675in}{4.224000in}}%
\pgfpathlineto{\pgfqpoint{4.407273in}{4.224000in}}%
\pgfpathlineto{\pgfqpoint{4.407103in}{4.224000in}}%
\pgfpathlineto{\pgfqpoint{4.407185in}{4.223919in}}%
\pgfpathlineto{\pgfqpoint{4.407273in}{4.223834in}}%
\pgfpathlineto{\pgfqpoint{4.445208in}{4.186667in}}%
\pgfpathlineto{\pgfqpoint{4.446248in}{4.185637in}}%
\pgfpathlineto{\pgfqpoint{4.447354in}{4.184563in}}%
\pgfpathlineto{\pgfqpoint{4.483202in}{4.149333in}}%
\pgfpathlineto{\pgfqpoint{4.485251in}{4.147300in}}%
\pgfpathlineto{\pgfqpoint{4.487434in}{4.145172in}}%
\pgfpathlineto{\pgfqpoint{4.521088in}{4.112000in}}%
\pgfpathlineto{\pgfqpoint{4.524195in}{4.108907in}}%
\pgfpathlineto{\pgfqpoint{4.527515in}{4.105660in}}%
\pgfpathlineto{\pgfqpoint{4.558864in}{4.074667in}}%
\pgfpathlineto{\pgfqpoint{4.567596in}{4.066028in}}%
\pgfpathlineto{\pgfqpoint{4.582453in}{4.051172in}}%
\pgfpathlineto{\pgfqpoint{4.596533in}{4.037333in}}%
\pgfpathlineto{\pgfqpoint{4.601904in}{4.031956in}}%
\pgfpathlineto{\pgfqpoint{4.607677in}{4.026276in}}%
\pgfpathlineto{\pgfqpoint{4.634095in}{4.000000in}}%
\pgfpathlineto{\pgfqpoint{4.647758in}{3.986402in}}%
\pgfpathlineto{\pgfqpoint{4.671550in}{3.962667in}}%
\pgfpathlineto{\pgfqpoint{4.679376in}{3.954784in}}%
\pgfpathlineto{\pgfqpoint{4.687838in}{3.946407in}}%
\pgfpathlineto{\pgfqpoint{4.708900in}{3.925333in}}%
\pgfpathlineto{\pgfqpoint{4.727919in}{3.906291in}}%
\pgfpathlineto{\pgfqpoint{4.737331in}{3.896767in}}%
\pgfpathlineto{\pgfqpoint{4.746145in}{3.888000in}}%
\pgfpathlineto{\pgfqpoint{4.768000in}{3.866053in}}%
\pgfusepath{fill}%
\end{pgfscope}%
\begin{pgfscope}%
\pgfpathrectangle{\pgfqpoint{0.800000in}{0.528000in}}{\pgfqpoint{3.968000in}{3.696000in}}%
\pgfusepath{clip}%
\pgfsetbuttcap%
\pgfsetroundjoin%
\definecolor{currentfill}{rgb}{0.595839,0.848717,0.243329}%
\pgfsetfillcolor{currentfill}%
\pgfsetlinewidth{0.000000pt}%
\definecolor{currentstroke}{rgb}{0.000000,0.000000,0.000000}%
\pgfsetstrokecolor{currentstroke}%
\pgfsetdash{}{0pt}%
\pgfpathmoveto{\pgfqpoint{4.768000in}{3.871161in}}%
\pgfpathlineto{\pgfqpoint{4.751231in}{3.888000in}}%
\pgfpathlineto{\pgfqpoint{4.739958in}{3.899213in}}%
\pgfpathlineto{\pgfqpoint{4.727919in}{3.911395in}}%
\pgfpathlineto{\pgfqpoint{4.713998in}{3.925333in}}%
\pgfpathlineto{\pgfqpoint{4.687838in}{3.951508in}}%
\pgfpathlineto{\pgfqpoint{4.682031in}{3.957257in}}%
\pgfpathlineto{\pgfqpoint{4.676660in}{3.962667in}}%
\pgfpathlineto{\pgfqpoint{4.647758in}{3.991500in}}%
\pgfpathlineto{\pgfqpoint{4.639217in}{4.000000in}}%
\pgfpathlineto{\pgfqpoint{4.607677in}{4.031370in}}%
\pgfpathlineto{\pgfqpoint{4.604564in}{4.034434in}}%
\pgfpathlineto{\pgfqpoint{4.601667in}{4.037333in}}%
\pgfpathlineto{\pgfqpoint{4.585089in}{4.053627in}}%
\pgfpathlineto{\pgfqpoint{4.567596in}{4.071120in}}%
\pgfpathlineto{\pgfqpoint{4.564011in}{4.074667in}}%
\pgfpathlineto{\pgfqpoint{4.527515in}{4.110749in}}%
\pgfpathlineto{\pgfqpoint{4.526860in}{4.111390in}}%
\pgfpathlineto{\pgfqpoint{4.526246in}{4.112000in}}%
\pgfpathlineto{\pgfqpoint{4.504459in}{4.133476in}}%
\pgfpathlineto{\pgfqpoint{4.488363in}{4.149333in}}%
\pgfpathlineto{\pgfqpoint{4.487910in}{4.149776in}}%
\pgfpathlineto{\pgfqpoint{4.487434in}{4.150248in}}%
\pgfpathlineto{\pgfqpoint{4.450358in}{4.186667in}}%
\pgfpathlineto{\pgfqpoint{4.448889in}{4.188097in}}%
\pgfpathlineto{\pgfqpoint{4.447354in}{4.189616in}}%
\pgfpathlineto{\pgfqpoint{4.412244in}{4.224000in}}%
\pgfpathlineto{\pgfqpoint{4.409675in}{4.224000in}}%
\pgfpathlineto{\pgfqpoint{4.447354in}{4.187100in}}%
\pgfpathlineto{\pgfqpoint{4.447579in}{4.186877in}}%
\pgfpathlineto{\pgfqpoint{4.447795in}{4.186667in}}%
\pgfpathlineto{\pgfqpoint{4.455878in}{4.178726in}}%
\pgfpathlineto{\pgfqpoint{4.485788in}{4.149333in}}%
\pgfpathlineto{\pgfqpoint{4.486585in}{4.148542in}}%
\pgfpathlineto{\pgfqpoint{4.487434in}{4.147714in}}%
\pgfpathlineto{\pgfqpoint{4.523667in}{4.112000in}}%
\pgfpathlineto{\pgfqpoint{4.525527in}{4.110148in}}%
\pgfpathlineto{\pgfqpoint{4.527515in}{4.108204in}}%
\pgfpathlineto{\pgfqpoint{4.561437in}{4.074667in}}%
\pgfpathlineto{\pgfqpoint{4.567596in}{4.068574in}}%
\pgfpathlineto{\pgfqpoint{4.583771in}{4.052399in}}%
\pgfpathlineto{\pgfqpoint{4.599100in}{4.037333in}}%
\pgfpathlineto{\pgfqpoint{4.603234in}{4.033195in}}%
\pgfpathlineto{\pgfqpoint{4.607677in}{4.028823in}}%
\pgfpathlineto{\pgfqpoint{4.636656in}{4.000000in}}%
\pgfpathlineto{\pgfqpoint{4.647758in}{3.988951in}}%
\pgfpathlineto{\pgfqpoint{4.674105in}{3.962667in}}%
\pgfpathlineto{\pgfqpoint{4.680703in}{3.956021in}}%
\pgfpathlineto{\pgfqpoint{4.687838in}{3.948958in}}%
\pgfpathlineto{\pgfqpoint{4.711449in}{3.925333in}}%
\pgfpathlineto{\pgfqpoint{4.727919in}{3.908843in}}%
\pgfpathlineto{\pgfqpoint{4.738644in}{3.897990in}}%
\pgfpathlineto{\pgfqpoint{4.748688in}{3.888000in}}%
\pgfpathlineto{\pgfqpoint{4.768000in}{3.868607in}}%
\pgfusepath{fill}%
\end{pgfscope}%
\begin{pgfscope}%
\pgfpathrectangle{\pgfqpoint{0.800000in}{0.528000in}}{\pgfqpoint{3.968000in}{3.696000in}}%
\pgfusepath{clip}%
\pgfsetbuttcap%
\pgfsetroundjoin%
\definecolor{currentfill}{rgb}{0.606045,0.850733,0.236712}%
\pgfsetfillcolor{currentfill}%
\pgfsetlinewidth{0.000000pt}%
\definecolor{currentstroke}{rgb}{0.000000,0.000000,0.000000}%
\pgfsetstrokecolor{currentstroke}%
\pgfsetdash{}{0pt}%
\pgfpathmoveto{\pgfqpoint{4.768000in}{3.873715in}}%
\pgfpathlineto{\pgfqpoint{4.753775in}{3.888000in}}%
\pgfpathlineto{\pgfqpoint{4.741271in}{3.900436in}}%
\pgfpathlineto{\pgfqpoint{4.727919in}{3.913948in}}%
\pgfpathlineto{\pgfqpoint{4.716547in}{3.925333in}}%
\pgfpathlineto{\pgfqpoint{4.687838in}{3.954059in}}%
\pgfpathlineto{\pgfqpoint{4.683358in}{3.958494in}}%
\pgfpathlineto{\pgfqpoint{4.679216in}{3.962667in}}%
\pgfpathlineto{\pgfqpoint{4.647758in}{3.994049in}}%
\pgfpathlineto{\pgfqpoint{4.641778in}{4.000000in}}%
\pgfpathlineto{\pgfqpoint{4.607677in}{4.033918in}}%
\pgfpathlineto{\pgfqpoint{4.605894in}{4.035672in}}%
\pgfpathlineto{\pgfqpoint{4.604235in}{4.037333in}}%
\pgfpathlineto{\pgfqpoint{4.586407in}{4.054855in}}%
\pgfpathlineto{\pgfqpoint{4.567596in}{4.073666in}}%
\pgfpathlineto{\pgfqpoint{4.566584in}{4.074667in}}%
\pgfpathlineto{\pgfqpoint{4.550110in}{4.090954in}}%
\pgfpathlineto{\pgfqpoint{4.528812in}{4.112000in}}%
\pgfpathlineto{\pgfqpoint{4.528180in}{4.112619in}}%
\pgfpathlineto{\pgfqpoint{4.527515in}{4.113280in}}%
\pgfpathlineto{\pgfqpoint{4.490921in}{4.149333in}}%
\pgfpathlineto{\pgfqpoint{4.489218in}{4.150995in}}%
\pgfpathlineto{\pgfqpoint{4.487434in}{4.152766in}}%
\pgfpathlineto{\pgfqpoint{4.452922in}{4.186667in}}%
\pgfpathlineto{\pgfqpoint{4.450199in}{4.189317in}}%
\pgfpathlineto{\pgfqpoint{4.447354in}{4.192133in}}%
\pgfpathlineto{\pgfqpoint{4.414814in}{4.224000in}}%
\pgfpathlineto{\pgfqpoint{4.412244in}{4.224000in}}%
\pgfpathlineto{\pgfqpoint{4.447354in}{4.189616in}}%
\pgfpathlineto{\pgfqpoint{4.448889in}{4.188097in}}%
\pgfpathlineto{\pgfqpoint{4.450358in}{4.186667in}}%
\pgfpathlineto{\pgfqpoint{4.487434in}{4.150248in}}%
\pgfpathlineto{\pgfqpoint{4.487910in}{4.149776in}}%
\pgfpathlineto{\pgfqpoint{4.488363in}{4.149333in}}%
\pgfpathlineto{\pgfqpoint{4.504459in}{4.133476in}}%
\pgfpathlineto{\pgfqpoint{4.526246in}{4.112000in}}%
\pgfpathlineto{\pgfqpoint{4.526860in}{4.111390in}}%
\pgfpathlineto{\pgfqpoint{4.527515in}{4.110749in}}%
\pgfpathlineto{\pgfqpoint{4.564011in}{4.074667in}}%
\pgfpathlineto{\pgfqpoint{4.567596in}{4.071120in}}%
\pgfpathlineto{\pgfqpoint{4.585089in}{4.053627in}}%
\pgfpathlineto{\pgfqpoint{4.601667in}{4.037333in}}%
\pgfpathlineto{\pgfqpoint{4.604564in}{4.034434in}}%
\pgfpathlineto{\pgfqpoint{4.607677in}{4.031370in}}%
\pgfpathlineto{\pgfqpoint{4.639217in}{4.000000in}}%
\pgfpathlineto{\pgfqpoint{4.647758in}{3.991500in}}%
\pgfpathlineto{\pgfqpoint{4.676660in}{3.962667in}}%
\pgfpathlineto{\pgfqpoint{4.682031in}{3.957257in}}%
\pgfpathlineto{\pgfqpoint{4.687838in}{3.951508in}}%
\pgfpathlineto{\pgfqpoint{4.713998in}{3.925333in}}%
\pgfpathlineto{\pgfqpoint{4.727919in}{3.911395in}}%
\pgfpathlineto{\pgfqpoint{4.739958in}{3.899213in}}%
\pgfpathlineto{\pgfqpoint{4.751231in}{3.888000in}}%
\pgfpathlineto{\pgfqpoint{4.768000in}{3.871161in}}%
\pgfusepath{fill}%
\end{pgfscope}%
\begin{pgfscope}%
\pgfpathrectangle{\pgfqpoint{0.800000in}{0.528000in}}{\pgfqpoint{3.968000in}{3.696000in}}%
\pgfusepath{clip}%
\pgfsetbuttcap%
\pgfsetroundjoin%
\definecolor{currentfill}{rgb}{0.606045,0.850733,0.236712}%
\pgfsetfillcolor{currentfill}%
\pgfsetlinewidth{0.000000pt}%
\definecolor{currentstroke}{rgb}{0.000000,0.000000,0.000000}%
\pgfsetstrokecolor{currentstroke}%
\pgfsetdash{}{0pt}%
\pgfpathmoveto{\pgfqpoint{4.768000in}{3.876269in}}%
\pgfpathlineto{\pgfqpoint{4.756318in}{3.888000in}}%
\pgfpathlineto{\pgfqpoint{4.742584in}{3.901660in}}%
\pgfpathlineto{\pgfqpoint{4.727919in}{3.916500in}}%
\pgfpathlineto{\pgfqpoint{4.719097in}{3.925333in}}%
\pgfpathlineto{\pgfqpoint{4.687838in}{3.956610in}}%
\pgfpathlineto{\pgfqpoint{4.684686in}{3.959730in}}%
\pgfpathlineto{\pgfqpoint{4.681771in}{3.962667in}}%
\pgfpathlineto{\pgfqpoint{4.647758in}{3.996598in}}%
\pgfpathlineto{\pgfqpoint{4.644339in}{4.000000in}}%
\pgfpathlineto{\pgfqpoint{4.607677in}{4.036465in}}%
\pgfpathlineto{\pgfqpoint{4.607224in}{4.036911in}}%
\pgfpathlineto{\pgfqpoint{4.606802in}{4.037333in}}%
\pgfpathlineto{\pgfqpoint{4.587725in}{4.056082in}}%
\pgfpathlineto{\pgfqpoint{4.569141in}{4.074667in}}%
\pgfpathlineto{\pgfqpoint{4.567596in}{4.076197in}}%
\pgfpathlineto{\pgfqpoint{4.531363in}{4.112000in}}%
\pgfpathlineto{\pgfqpoint{4.529487in}{4.113837in}}%
\pgfpathlineto{\pgfqpoint{4.527515in}{4.115800in}}%
\pgfpathlineto{\pgfqpoint{4.493479in}{4.149333in}}%
\pgfpathlineto{\pgfqpoint{4.490527in}{4.152214in}}%
\pgfpathlineto{\pgfqpoint{4.487434in}{4.155285in}}%
\pgfpathlineto{\pgfqpoint{4.455486in}{4.186667in}}%
\pgfpathlineto{\pgfqpoint{4.451508in}{4.190537in}}%
\pgfpathlineto{\pgfqpoint{4.447354in}{4.194650in}}%
\pgfpathlineto{\pgfqpoint{4.417384in}{4.224000in}}%
\pgfpathlineto{\pgfqpoint{4.414814in}{4.224000in}}%
\pgfpathlineto{\pgfqpoint{4.447354in}{4.192133in}}%
\pgfpathlineto{\pgfqpoint{4.450199in}{4.189317in}}%
\pgfpathlineto{\pgfqpoint{4.452922in}{4.186667in}}%
\pgfpathlineto{\pgfqpoint{4.487434in}{4.152766in}}%
\pgfpathlineto{\pgfqpoint{4.489218in}{4.150995in}}%
\pgfpathlineto{\pgfqpoint{4.490921in}{4.149333in}}%
\pgfpathlineto{\pgfqpoint{4.527515in}{4.113280in}}%
\pgfpathlineto{\pgfqpoint{4.528180in}{4.112619in}}%
\pgfpathlineto{\pgfqpoint{4.528812in}{4.112000in}}%
\pgfpathlineto{\pgfqpoint{4.550110in}{4.090954in}}%
\pgfpathlineto{\pgfqpoint{4.566584in}{4.074667in}}%
\pgfpathlineto{\pgfqpoint{4.567596in}{4.073666in}}%
\pgfpathlineto{\pgfqpoint{4.586407in}{4.054855in}}%
\pgfpathlineto{\pgfqpoint{4.604235in}{4.037333in}}%
\pgfpathlineto{\pgfqpoint{4.605894in}{4.035672in}}%
\pgfpathlineto{\pgfqpoint{4.607677in}{4.033918in}}%
\pgfpathlineto{\pgfqpoint{4.641778in}{4.000000in}}%
\pgfpathlineto{\pgfqpoint{4.647758in}{3.994049in}}%
\pgfpathlineto{\pgfqpoint{4.679216in}{3.962667in}}%
\pgfpathlineto{\pgfqpoint{4.683358in}{3.958494in}}%
\pgfpathlineto{\pgfqpoint{4.687838in}{3.954059in}}%
\pgfpathlineto{\pgfqpoint{4.716547in}{3.925333in}}%
\pgfpathlineto{\pgfqpoint{4.727919in}{3.913948in}}%
\pgfpathlineto{\pgfqpoint{4.741271in}{3.900436in}}%
\pgfpathlineto{\pgfqpoint{4.753775in}{3.888000in}}%
\pgfpathlineto{\pgfqpoint{4.768000in}{3.873715in}}%
\pgfusepath{fill}%
\end{pgfscope}%
\begin{pgfscope}%
\pgfpathrectangle{\pgfqpoint{0.800000in}{0.528000in}}{\pgfqpoint{3.968000in}{3.696000in}}%
\pgfusepath{clip}%
\pgfsetbuttcap%
\pgfsetroundjoin%
\definecolor{currentfill}{rgb}{0.606045,0.850733,0.236712}%
\pgfsetfillcolor{currentfill}%
\pgfsetlinewidth{0.000000pt}%
\definecolor{currentstroke}{rgb}{0.000000,0.000000,0.000000}%
\pgfsetstrokecolor{currentstroke}%
\pgfsetdash{}{0pt}%
\pgfpathmoveto{\pgfqpoint{4.768000in}{3.878822in}}%
\pgfpathlineto{\pgfqpoint{4.758861in}{3.888000in}}%
\pgfpathlineto{\pgfqpoint{4.743897in}{3.902883in}}%
\pgfpathlineto{\pgfqpoint{4.727919in}{3.919052in}}%
\pgfpathlineto{\pgfqpoint{4.721646in}{3.925333in}}%
\pgfpathlineto{\pgfqpoint{4.687838in}{3.959160in}}%
\pgfpathlineto{\pgfqpoint{4.686013in}{3.960967in}}%
\pgfpathlineto{\pgfqpoint{4.684326in}{3.962667in}}%
\pgfpathlineto{\pgfqpoint{4.647758in}{3.999147in}}%
\pgfpathlineto{\pgfqpoint{4.646901in}{4.000000in}}%
\pgfpathlineto{\pgfqpoint{4.634262in}{4.012570in}}%
\pgfpathlineto{\pgfqpoint{4.609351in}{4.037333in}}%
\pgfpathlineto{\pgfqpoint{4.607677in}{4.038997in}}%
\pgfpathlineto{\pgfqpoint{4.589043in}{4.057310in}}%
\pgfpathlineto{\pgfqpoint{4.571686in}{4.074667in}}%
\pgfpathlineto{\pgfqpoint{4.567596in}{4.078718in}}%
\pgfpathlineto{\pgfqpoint{4.533915in}{4.112000in}}%
\pgfpathlineto{\pgfqpoint{4.530795in}{4.115055in}}%
\pgfpathlineto{\pgfqpoint{4.527515in}{4.118320in}}%
\pgfpathlineto{\pgfqpoint{4.496037in}{4.149333in}}%
\pgfpathlineto{\pgfqpoint{4.491836in}{4.153433in}}%
\pgfpathlineto{\pgfqpoint{4.487434in}{4.157803in}}%
\pgfpathlineto{\pgfqpoint{4.458050in}{4.186667in}}%
\pgfpathlineto{\pgfqpoint{4.452818in}{4.191757in}}%
\pgfpathlineto{\pgfqpoint{4.447354in}{4.197167in}}%
\pgfpathlineto{\pgfqpoint{4.419954in}{4.224000in}}%
\pgfpathlineto{\pgfqpoint{4.417384in}{4.224000in}}%
\pgfpathlineto{\pgfqpoint{4.447354in}{4.194650in}}%
\pgfpathlineto{\pgfqpoint{4.451508in}{4.190537in}}%
\pgfpathlineto{\pgfqpoint{4.455486in}{4.186667in}}%
\pgfpathlineto{\pgfqpoint{4.487434in}{4.155285in}}%
\pgfpathlineto{\pgfqpoint{4.490527in}{4.152214in}}%
\pgfpathlineto{\pgfqpoint{4.493479in}{4.149333in}}%
\pgfpathlineto{\pgfqpoint{4.527515in}{4.115800in}}%
\pgfpathlineto{\pgfqpoint{4.529487in}{4.113837in}}%
\pgfpathlineto{\pgfqpoint{4.531363in}{4.112000in}}%
\pgfpathlineto{\pgfqpoint{4.567596in}{4.076197in}}%
\pgfpathlineto{\pgfqpoint{4.569141in}{4.074667in}}%
\pgfpathlineto{\pgfqpoint{4.587725in}{4.056082in}}%
\pgfpathlineto{\pgfqpoint{4.606802in}{4.037333in}}%
\pgfpathlineto{\pgfqpoint{4.607224in}{4.036911in}}%
\pgfpathlineto{\pgfqpoint{4.607677in}{4.036465in}}%
\pgfpathlineto{\pgfqpoint{4.644339in}{4.000000in}}%
\pgfpathlineto{\pgfqpoint{4.647758in}{3.996598in}}%
\pgfpathlineto{\pgfqpoint{4.681771in}{3.962667in}}%
\pgfpathlineto{\pgfqpoint{4.684686in}{3.959730in}}%
\pgfpathlineto{\pgfqpoint{4.687838in}{3.956610in}}%
\pgfpathlineto{\pgfqpoint{4.719097in}{3.925333in}}%
\pgfpathlineto{\pgfqpoint{4.727919in}{3.916500in}}%
\pgfpathlineto{\pgfqpoint{4.742584in}{3.901660in}}%
\pgfpathlineto{\pgfqpoint{4.756318in}{3.888000in}}%
\pgfpathlineto{\pgfqpoint{4.768000in}{3.876269in}}%
\pgfusepath{fill}%
\end{pgfscope}%
\begin{pgfscope}%
\pgfpathrectangle{\pgfqpoint{0.800000in}{0.528000in}}{\pgfqpoint{3.968000in}{3.696000in}}%
\pgfusepath{clip}%
\pgfsetbuttcap%
\pgfsetroundjoin%
\definecolor{currentfill}{rgb}{0.606045,0.850733,0.236712}%
\pgfsetfillcolor{currentfill}%
\pgfsetlinewidth{0.000000pt}%
\definecolor{currentstroke}{rgb}{0.000000,0.000000,0.000000}%
\pgfsetstrokecolor{currentstroke}%
\pgfsetdash{}{0pt}%
\pgfpathmoveto{\pgfqpoint{4.768000in}{3.881376in}}%
\pgfpathlineto{\pgfqpoint{4.761404in}{3.888000in}}%
\pgfpathlineto{\pgfqpoint{4.745211in}{3.904106in}}%
\pgfpathlineto{\pgfqpoint{4.727919in}{3.921604in}}%
\pgfpathlineto{\pgfqpoint{4.724195in}{3.925333in}}%
\pgfpathlineto{\pgfqpoint{4.687838in}{3.961711in}}%
\pgfpathlineto{\pgfqpoint{4.687341in}{3.962203in}}%
\pgfpathlineto{\pgfqpoint{4.686881in}{3.962667in}}%
\pgfpathlineto{\pgfqpoint{4.673399in}{3.976116in}}%
\pgfpathlineto{\pgfqpoint{4.649444in}{4.000000in}}%
\pgfpathlineto{\pgfqpoint{4.648625in}{4.000808in}}%
\pgfpathlineto{\pgfqpoint{4.647758in}{4.001680in}}%
\pgfpathlineto{\pgfqpoint{4.611891in}{4.037333in}}%
\pgfpathlineto{\pgfqpoint{4.607677in}{4.041520in}}%
\pgfpathlineto{\pgfqpoint{4.590361in}{4.058538in}}%
\pgfpathlineto{\pgfqpoint{4.574232in}{4.074667in}}%
\pgfpathlineto{\pgfqpoint{4.567596in}{4.081240in}}%
\pgfpathlineto{\pgfqpoint{4.536467in}{4.112000in}}%
\pgfpathlineto{\pgfqpoint{4.532102in}{4.116272in}}%
\pgfpathlineto{\pgfqpoint{4.527515in}{4.120840in}}%
\pgfpathlineto{\pgfqpoint{4.498594in}{4.149333in}}%
\pgfpathlineto{\pgfqpoint{4.493144in}{4.154652in}}%
\pgfpathlineto{\pgfqpoint{4.487434in}{4.160321in}}%
\pgfpathlineto{\pgfqpoint{4.460613in}{4.186667in}}%
\pgfpathlineto{\pgfqpoint{4.454128in}{4.192977in}}%
\pgfpathlineto{\pgfqpoint{4.447354in}{4.199683in}}%
\pgfpathlineto{\pgfqpoint{4.422524in}{4.224000in}}%
\pgfpathlineto{\pgfqpoint{4.419954in}{4.224000in}}%
\pgfpathlineto{\pgfqpoint{4.447354in}{4.197167in}}%
\pgfpathlineto{\pgfqpoint{4.452818in}{4.191757in}}%
\pgfpathlineto{\pgfqpoint{4.458050in}{4.186667in}}%
\pgfpathlineto{\pgfqpoint{4.487434in}{4.157803in}}%
\pgfpathlineto{\pgfqpoint{4.491836in}{4.153433in}}%
\pgfpathlineto{\pgfqpoint{4.496037in}{4.149333in}}%
\pgfpathlineto{\pgfqpoint{4.527515in}{4.118320in}}%
\pgfpathlineto{\pgfqpoint{4.530795in}{4.115055in}}%
\pgfpathlineto{\pgfqpoint{4.533915in}{4.112000in}}%
\pgfpathlineto{\pgfqpoint{4.567596in}{4.078718in}}%
\pgfpathlineto{\pgfqpoint{4.571686in}{4.074667in}}%
\pgfpathlineto{\pgfqpoint{4.589043in}{4.057310in}}%
\pgfpathlineto{\pgfqpoint{4.607677in}{4.038997in}}%
\pgfpathlineto{\pgfqpoint{4.609351in}{4.037333in}}%
\pgfpathlineto{\pgfqpoint{4.634262in}{4.012570in}}%
\pgfpathlineto{\pgfqpoint{4.646901in}{4.000000in}}%
\pgfpathlineto{\pgfqpoint{4.647758in}{3.999147in}}%
\pgfpathlineto{\pgfqpoint{4.684326in}{3.962667in}}%
\pgfpathlineto{\pgfqpoint{4.686013in}{3.960967in}}%
\pgfpathlineto{\pgfqpoint{4.687838in}{3.959160in}}%
\pgfpathlineto{\pgfqpoint{4.721646in}{3.925333in}}%
\pgfpathlineto{\pgfqpoint{4.727919in}{3.919052in}}%
\pgfpathlineto{\pgfqpoint{4.743897in}{3.902883in}}%
\pgfpathlineto{\pgfqpoint{4.758861in}{3.888000in}}%
\pgfpathlineto{\pgfqpoint{4.768000in}{3.878822in}}%
\pgfusepath{fill}%
\end{pgfscope}%
\begin{pgfscope}%
\pgfpathrectangle{\pgfqpoint{0.800000in}{0.528000in}}{\pgfqpoint{3.968000in}{3.696000in}}%
\pgfusepath{clip}%
\pgfsetbuttcap%
\pgfsetroundjoin%
\definecolor{currentfill}{rgb}{0.616293,0.852709,0.230052}%
\pgfsetfillcolor{currentfill}%
\pgfsetlinewidth{0.000000pt}%
\definecolor{currentstroke}{rgb}{0.000000,0.000000,0.000000}%
\pgfsetstrokecolor{currentstroke}%
\pgfsetdash{}{0pt}%
\pgfpathmoveto{\pgfqpoint{4.768000in}{3.883930in}}%
\pgfpathlineto{\pgfqpoint{4.763947in}{3.888000in}}%
\pgfpathlineto{\pgfqpoint{4.746524in}{3.905329in}}%
\pgfpathlineto{\pgfqpoint{4.727919in}{3.924157in}}%
\pgfpathlineto{\pgfqpoint{4.726744in}{3.925333in}}%
\pgfpathlineto{\pgfqpoint{4.710910in}{3.941177in}}%
\pgfpathlineto{\pgfqpoint{4.689419in}{3.962667in}}%
\pgfpathlineto{\pgfqpoint{4.688653in}{3.963425in}}%
\pgfpathlineto{\pgfqpoint{4.687838in}{3.964246in}}%
\pgfpathlineto{\pgfqpoint{4.651977in}{4.000000in}}%
\pgfpathlineto{\pgfqpoint{4.649929in}{4.002023in}}%
\pgfpathlineto{\pgfqpoint{4.647758in}{4.004205in}}%
\pgfpathlineto{\pgfqpoint{4.614431in}{4.037333in}}%
\pgfpathlineto{\pgfqpoint{4.607677in}{4.044043in}}%
\pgfpathlineto{\pgfqpoint{4.591679in}{4.059766in}}%
\pgfpathlineto{\pgfqpoint{4.576778in}{4.074667in}}%
\pgfpathlineto{\pgfqpoint{4.567596in}{4.083761in}}%
\pgfpathlineto{\pgfqpoint{4.539019in}{4.112000in}}%
\pgfpathlineto{\pgfqpoint{4.533410in}{4.117490in}}%
\pgfpathlineto{\pgfqpoint{4.527515in}{4.123360in}}%
\pgfpathlineto{\pgfqpoint{4.501152in}{4.149333in}}%
\pgfpathlineto{\pgfqpoint{4.494453in}{4.155871in}}%
\pgfpathlineto{\pgfqpoint{4.487434in}{4.162840in}}%
\pgfpathlineto{\pgfqpoint{4.463177in}{4.186667in}}%
\pgfpathlineto{\pgfqpoint{4.455438in}{4.194197in}}%
\pgfpathlineto{\pgfqpoint{4.447354in}{4.202200in}}%
\pgfpathlineto{\pgfqpoint{4.425094in}{4.224000in}}%
\pgfpathlineto{\pgfqpoint{4.422524in}{4.224000in}}%
\pgfpathlineto{\pgfqpoint{4.447354in}{4.199683in}}%
\pgfpathlineto{\pgfqpoint{4.454128in}{4.192977in}}%
\pgfpathlineto{\pgfqpoint{4.460613in}{4.186667in}}%
\pgfpathlineto{\pgfqpoint{4.487434in}{4.160321in}}%
\pgfpathlineto{\pgfqpoint{4.493144in}{4.154652in}}%
\pgfpathlineto{\pgfqpoint{4.498594in}{4.149333in}}%
\pgfpathlineto{\pgfqpoint{4.527515in}{4.120840in}}%
\pgfpathlineto{\pgfqpoint{4.532102in}{4.116272in}}%
\pgfpathlineto{\pgfqpoint{4.536467in}{4.112000in}}%
\pgfpathlineto{\pgfqpoint{4.567596in}{4.081240in}}%
\pgfpathlineto{\pgfqpoint{4.574232in}{4.074667in}}%
\pgfpathlineto{\pgfqpoint{4.590361in}{4.058538in}}%
\pgfpathlineto{\pgfqpoint{4.607677in}{4.041520in}}%
\pgfpathlineto{\pgfqpoint{4.611891in}{4.037333in}}%
\pgfpathlineto{\pgfqpoint{4.647758in}{4.001680in}}%
\pgfpathlineto{\pgfqpoint{4.648625in}{4.000808in}}%
\pgfpathlineto{\pgfqpoint{4.649444in}{4.000000in}}%
\pgfpathlineto{\pgfqpoint{4.673399in}{3.976116in}}%
\pgfpathlineto{\pgfqpoint{4.686881in}{3.962667in}}%
\pgfpathlineto{\pgfqpoint{4.687341in}{3.962203in}}%
\pgfpathlineto{\pgfqpoint{4.687838in}{3.961711in}}%
\pgfpathlineto{\pgfqpoint{4.724195in}{3.925333in}}%
\pgfpathlineto{\pgfqpoint{4.727919in}{3.921604in}}%
\pgfpathlineto{\pgfqpoint{4.745211in}{3.904106in}}%
\pgfpathlineto{\pgfqpoint{4.761404in}{3.888000in}}%
\pgfpathlineto{\pgfqpoint{4.768000in}{3.881376in}}%
\pgfusepath{fill}%
\end{pgfscope}%
\begin{pgfscope}%
\pgfpathrectangle{\pgfqpoint{0.800000in}{0.528000in}}{\pgfqpoint{3.968000in}{3.696000in}}%
\pgfusepath{clip}%
\pgfsetbuttcap%
\pgfsetroundjoin%
\definecolor{currentfill}{rgb}{0.616293,0.852709,0.230052}%
\pgfsetfillcolor{currentfill}%
\pgfsetlinewidth{0.000000pt}%
\definecolor{currentstroke}{rgb}{0.000000,0.000000,0.000000}%
\pgfsetstrokecolor{currentstroke}%
\pgfsetdash{}{0pt}%
\pgfpathmoveto{\pgfqpoint{4.768000in}{3.886484in}}%
\pgfpathlineto{\pgfqpoint{4.766490in}{3.888000in}}%
\pgfpathlineto{\pgfqpoint{4.747837in}{3.906553in}}%
\pgfpathlineto{\pgfqpoint{4.729279in}{3.925333in}}%
\pgfpathlineto{\pgfqpoint{4.727919in}{3.926696in}}%
\pgfpathlineto{\pgfqpoint{4.691947in}{3.962667in}}%
\pgfpathlineto{\pgfqpoint{4.689956in}{3.964639in}}%
\pgfpathlineto{\pgfqpoint{4.687838in}{3.966773in}}%
\pgfpathlineto{\pgfqpoint{4.654511in}{4.000000in}}%
\pgfpathlineto{\pgfqpoint{4.651233in}{4.003237in}}%
\pgfpathlineto{\pgfqpoint{4.647758in}{4.006729in}}%
\pgfpathlineto{\pgfqpoint{4.616970in}{4.037333in}}%
\pgfpathlineto{\pgfqpoint{4.607677in}{4.046566in}}%
\pgfpathlineto{\pgfqpoint{4.592997in}{4.060993in}}%
\pgfpathlineto{\pgfqpoint{4.579324in}{4.074667in}}%
\pgfpathlineto{\pgfqpoint{4.567596in}{4.086283in}}%
\pgfpathlineto{\pgfqpoint{4.541570in}{4.112000in}}%
\pgfpathlineto{\pgfqpoint{4.534717in}{4.118708in}}%
\pgfpathlineto{\pgfqpoint{4.527515in}{4.125880in}}%
\pgfpathlineto{\pgfqpoint{4.503710in}{4.149333in}}%
\pgfpathlineto{\pgfqpoint{4.495762in}{4.157090in}}%
\pgfpathlineto{\pgfqpoint{4.487434in}{4.165358in}}%
\pgfpathlineto{\pgfqpoint{4.465741in}{4.186667in}}%
\pgfpathlineto{\pgfqpoint{4.456748in}{4.195417in}}%
\pgfpathlineto{\pgfqpoint{4.447354in}{4.204717in}}%
\pgfpathlineto{\pgfqpoint{4.427664in}{4.224000in}}%
\pgfpathlineto{\pgfqpoint{4.425094in}{4.224000in}}%
\pgfpathlineto{\pgfqpoint{4.447354in}{4.202200in}}%
\pgfpathlineto{\pgfqpoint{4.455438in}{4.194197in}}%
\pgfpathlineto{\pgfqpoint{4.463177in}{4.186667in}}%
\pgfpathlineto{\pgfqpoint{4.487434in}{4.162840in}}%
\pgfpathlineto{\pgfqpoint{4.494453in}{4.155871in}}%
\pgfpathlineto{\pgfqpoint{4.501152in}{4.149333in}}%
\pgfpathlineto{\pgfqpoint{4.527515in}{4.123360in}}%
\pgfpathlineto{\pgfqpoint{4.533410in}{4.117490in}}%
\pgfpathlineto{\pgfqpoint{4.539019in}{4.112000in}}%
\pgfpathlineto{\pgfqpoint{4.567596in}{4.083761in}}%
\pgfpathlineto{\pgfqpoint{4.576778in}{4.074667in}}%
\pgfpathlineto{\pgfqpoint{4.591679in}{4.059766in}}%
\pgfpathlineto{\pgfqpoint{4.607677in}{4.044043in}}%
\pgfpathlineto{\pgfqpoint{4.614431in}{4.037333in}}%
\pgfpathlineto{\pgfqpoint{4.647758in}{4.004205in}}%
\pgfpathlineto{\pgfqpoint{4.649929in}{4.002023in}}%
\pgfpathlineto{\pgfqpoint{4.651977in}{4.000000in}}%
\pgfpathlineto{\pgfqpoint{4.687838in}{3.964246in}}%
\pgfpathlineto{\pgfqpoint{4.688653in}{3.963425in}}%
\pgfpathlineto{\pgfqpoint{4.689419in}{3.962667in}}%
\pgfpathlineto{\pgfqpoint{4.710910in}{3.941177in}}%
\pgfpathlineto{\pgfqpoint{4.726744in}{3.925333in}}%
\pgfpathlineto{\pgfqpoint{4.727919in}{3.924157in}}%
\pgfpathlineto{\pgfqpoint{4.746524in}{3.905329in}}%
\pgfpathlineto{\pgfqpoint{4.763947in}{3.888000in}}%
\pgfpathlineto{\pgfqpoint{4.768000in}{3.883930in}}%
\pgfusepath{fill}%
\end{pgfscope}%
\begin{pgfscope}%
\pgfpathrectangle{\pgfqpoint{0.800000in}{0.528000in}}{\pgfqpoint{3.968000in}{3.696000in}}%
\pgfusepath{clip}%
\pgfsetbuttcap%
\pgfsetroundjoin%
\definecolor{currentfill}{rgb}{0.616293,0.852709,0.230052}%
\pgfsetfillcolor{currentfill}%
\pgfsetlinewidth{0.000000pt}%
\definecolor{currentstroke}{rgb}{0.000000,0.000000,0.000000}%
\pgfsetstrokecolor{currentstroke}%
\pgfsetdash{}{0pt}%
\pgfpathmoveto{\pgfqpoint{4.768000in}{3.889028in}}%
\pgfpathlineto{\pgfqpoint{4.749151in}{3.907776in}}%
\pgfpathlineto{\pgfqpoint{4.731801in}{3.925333in}}%
\pgfpathlineto{\pgfqpoint{4.727919in}{3.929224in}}%
\pgfpathlineto{\pgfqpoint{4.694475in}{3.962667in}}%
\pgfpathlineto{\pgfqpoint{4.691259in}{3.965853in}}%
\pgfpathlineto{\pgfqpoint{4.687838in}{3.969299in}}%
\pgfpathlineto{\pgfqpoint{4.657045in}{4.000000in}}%
\pgfpathlineto{\pgfqpoint{4.652537in}{4.004452in}}%
\pgfpathlineto{\pgfqpoint{4.647758in}{4.009254in}}%
\pgfpathlineto{\pgfqpoint{4.619510in}{4.037333in}}%
\pgfpathlineto{\pgfqpoint{4.607677in}{4.049089in}}%
\pgfpathlineto{\pgfqpoint{4.594315in}{4.062221in}}%
\pgfpathlineto{\pgfqpoint{4.581869in}{4.074667in}}%
\pgfpathlineto{\pgfqpoint{4.567596in}{4.088804in}}%
\pgfpathlineto{\pgfqpoint{4.544122in}{4.112000in}}%
\pgfpathlineto{\pgfqpoint{4.536025in}{4.119926in}}%
\pgfpathlineto{\pgfqpoint{4.527515in}{4.128400in}}%
\pgfpathlineto{\pgfqpoint{4.506267in}{4.149333in}}%
\pgfpathlineto{\pgfqpoint{4.497070in}{4.158309in}}%
\pgfpathlineto{\pgfqpoint{4.487434in}{4.167876in}}%
\pgfpathlineto{\pgfqpoint{4.468305in}{4.186667in}}%
\pgfpathlineto{\pgfqpoint{4.458058in}{4.196637in}}%
\pgfpathlineto{\pgfqpoint{4.447354in}{4.207234in}}%
\pgfpathlineto{\pgfqpoint{4.430233in}{4.224000in}}%
\pgfpathlineto{\pgfqpoint{4.427664in}{4.224000in}}%
\pgfpathlineto{\pgfqpoint{4.447354in}{4.204717in}}%
\pgfpathlineto{\pgfqpoint{4.456748in}{4.195417in}}%
\pgfpathlineto{\pgfqpoint{4.465741in}{4.186667in}}%
\pgfpathlineto{\pgfqpoint{4.487434in}{4.165358in}}%
\pgfpathlineto{\pgfqpoint{4.495762in}{4.157090in}}%
\pgfpathlineto{\pgfqpoint{4.503710in}{4.149333in}}%
\pgfpathlineto{\pgfqpoint{4.527515in}{4.125880in}}%
\pgfpathlineto{\pgfqpoint{4.534717in}{4.118708in}}%
\pgfpathlineto{\pgfqpoint{4.541570in}{4.112000in}}%
\pgfpathlineto{\pgfqpoint{4.567596in}{4.086283in}}%
\pgfpathlineto{\pgfqpoint{4.579324in}{4.074667in}}%
\pgfpathlineto{\pgfqpoint{4.592997in}{4.060993in}}%
\pgfpathlineto{\pgfqpoint{4.607677in}{4.046566in}}%
\pgfpathlineto{\pgfqpoint{4.616970in}{4.037333in}}%
\pgfpathlineto{\pgfqpoint{4.647758in}{4.006729in}}%
\pgfpathlineto{\pgfqpoint{4.651233in}{4.003237in}}%
\pgfpathlineto{\pgfqpoint{4.654511in}{4.000000in}}%
\pgfpathlineto{\pgfqpoint{4.687838in}{3.966773in}}%
\pgfpathlineto{\pgfqpoint{4.689956in}{3.964639in}}%
\pgfpathlineto{\pgfqpoint{4.691947in}{3.962667in}}%
\pgfpathlineto{\pgfqpoint{4.727919in}{3.926696in}}%
\pgfpathlineto{\pgfqpoint{4.729279in}{3.925333in}}%
\pgfpathlineto{\pgfqpoint{4.747837in}{3.906553in}}%
\pgfpathlineto{\pgfqpoint{4.766490in}{3.888000in}}%
\pgfpathlineto{\pgfqpoint{4.768000in}{3.886484in}}%
\pgfpathlineto{\pgfqpoint{4.768000in}{3.888000in}}%
\pgfusepath{fill}%
\end{pgfscope}%
\begin{pgfscope}%
\pgfpathrectangle{\pgfqpoint{0.800000in}{0.528000in}}{\pgfqpoint{3.968000in}{3.696000in}}%
\pgfusepath{clip}%
\pgfsetbuttcap%
\pgfsetroundjoin%
\definecolor{currentfill}{rgb}{0.626579,0.854645,0.223353}%
\pgfsetfillcolor{currentfill}%
\pgfsetlinewidth{0.000000pt}%
\definecolor{currentstroke}{rgb}{0.000000,0.000000,0.000000}%
\pgfsetstrokecolor{currentstroke}%
\pgfsetdash{}{0pt}%
\pgfpathmoveto{\pgfqpoint{4.768000in}{3.891558in}}%
\pgfpathlineto{\pgfqpoint{4.750464in}{3.908999in}}%
\pgfpathlineto{\pgfqpoint{4.734323in}{3.925333in}}%
\pgfpathlineto{\pgfqpoint{4.727919in}{3.931752in}}%
\pgfpathlineto{\pgfqpoint{4.697003in}{3.962667in}}%
\pgfpathlineto{\pgfqpoint{4.692562in}{3.967066in}}%
\pgfpathlineto{\pgfqpoint{4.687838in}{3.971825in}}%
\pgfpathlineto{\pgfqpoint{4.659579in}{4.000000in}}%
\pgfpathlineto{\pgfqpoint{4.653841in}{4.005667in}}%
\pgfpathlineto{\pgfqpoint{4.647758in}{4.011779in}}%
\pgfpathlineto{\pgfqpoint{4.622050in}{4.037333in}}%
\pgfpathlineto{\pgfqpoint{4.607677in}{4.051612in}}%
\pgfpathlineto{\pgfqpoint{4.595633in}{4.063449in}}%
\pgfpathlineto{\pgfqpoint{4.584415in}{4.074667in}}%
\pgfpathlineto{\pgfqpoint{4.567596in}{4.091326in}}%
\pgfpathlineto{\pgfqpoint{4.546674in}{4.112000in}}%
\pgfpathlineto{\pgfqpoint{4.537332in}{4.121144in}}%
\pgfpathlineto{\pgfqpoint{4.527515in}{4.130920in}}%
\pgfpathlineto{\pgfqpoint{4.508825in}{4.149333in}}%
\pgfpathlineto{\pgfqpoint{4.498379in}{4.159528in}}%
\pgfpathlineto{\pgfqpoint{4.487434in}{4.170395in}}%
\pgfpathlineto{\pgfqpoint{4.470869in}{4.186667in}}%
\pgfpathlineto{\pgfqpoint{4.459368in}{4.197857in}}%
\pgfpathlineto{\pgfqpoint{4.447354in}{4.209750in}}%
\pgfpathlineto{\pgfqpoint{4.432803in}{4.224000in}}%
\pgfpathlineto{\pgfqpoint{4.430233in}{4.224000in}}%
\pgfpathlineto{\pgfqpoint{4.447354in}{4.207234in}}%
\pgfpathlineto{\pgfqpoint{4.458058in}{4.196637in}}%
\pgfpathlineto{\pgfqpoint{4.468305in}{4.186667in}}%
\pgfpathlineto{\pgfqpoint{4.487434in}{4.167876in}}%
\pgfpathlineto{\pgfqpoint{4.497070in}{4.158309in}}%
\pgfpathlineto{\pgfqpoint{4.506267in}{4.149333in}}%
\pgfpathlineto{\pgfqpoint{4.527515in}{4.128400in}}%
\pgfpathlineto{\pgfqpoint{4.536025in}{4.119926in}}%
\pgfpathlineto{\pgfqpoint{4.544122in}{4.112000in}}%
\pgfpathlineto{\pgfqpoint{4.567596in}{4.088804in}}%
\pgfpathlineto{\pgfqpoint{4.581869in}{4.074667in}}%
\pgfpathlineto{\pgfqpoint{4.594315in}{4.062221in}}%
\pgfpathlineto{\pgfqpoint{4.607677in}{4.049089in}}%
\pgfpathlineto{\pgfqpoint{4.619510in}{4.037333in}}%
\pgfpathlineto{\pgfqpoint{4.647758in}{4.009254in}}%
\pgfpathlineto{\pgfqpoint{4.652537in}{4.004452in}}%
\pgfpathlineto{\pgfqpoint{4.657045in}{4.000000in}}%
\pgfpathlineto{\pgfqpoint{4.687838in}{3.969299in}}%
\pgfpathlineto{\pgfqpoint{4.691259in}{3.965853in}}%
\pgfpathlineto{\pgfqpoint{4.694475in}{3.962667in}}%
\pgfpathlineto{\pgfqpoint{4.727919in}{3.929224in}}%
\pgfpathlineto{\pgfqpoint{4.731801in}{3.925333in}}%
\pgfpathlineto{\pgfqpoint{4.749151in}{3.907776in}}%
\pgfpathlineto{\pgfqpoint{4.768000in}{3.889028in}}%
\pgfusepath{fill}%
\end{pgfscope}%
\begin{pgfscope}%
\pgfpathrectangle{\pgfqpoint{0.800000in}{0.528000in}}{\pgfqpoint{3.968000in}{3.696000in}}%
\pgfusepath{clip}%
\pgfsetbuttcap%
\pgfsetroundjoin%
\definecolor{currentfill}{rgb}{0.626579,0.854645,0.223353}%
\pgfsetfillcolor{currentfill}%
\pgfsetlinewidth{0.000000pt}%
\definecolor{currentstroke}{rgb}{0.000000,0.000000,0.000000}%
\pgfsetstrokecolor{currentstroke}%
\pgfsetdash{}{0pt}%
\pgfpathmoveto{\pgfqpoint{4.768000in}{3.894087in}}%
\pgfpathlineto{\pgfqpoint{4.751777in}{3.910223in}}%
\pgfpathlineto{\pgfqpoint{4.736845in}{3.925333in}}%
\pgfpathlineto{\pgfqpoint{4.727919in}{3.934279in}}%
\pgfpathlineto{\pgfqpoint{4.699531in}{3.962667in}}%
\pgfpathlineto{\pgfqpoint{4.693864in}{3.968280in}}%
\pgfpathlineto{\pgfqpoint{4.687838in}{3.974352in}}%
\pgfpathlineto{\pgfqpoint{4.662113in}{4.000000in}}%
\pgfpathlineto{\pgfqpoint{4.655145in}{4.006881in}}%
\pgfpathlineto{\pgfqpoint{4.647758in}{4.014303in}}%
\pgfpathlineto{\pgfqpoint{4.624590in}{4.037333in}}%
\pgfpathlineto{\pgfqpoint{4.607677in}{4.054135in}}%
\pgfpathlineto{\pgfqpoint{4.596951in}{4.064676in}}%
\pgfpathlineto{\pgfqpoint{4.586961in}{4.074667in}}%
\pgfpathlineto{\pgfqpoint{4.567596in}{4.093847in}}%
\pgfpathlineto{\pgfqpoint{4.549226in}{4.112000in}}%
\pgfpathlineto{\pgfqpoint{4.538640in}{4.122362in}}%
\pgfpathlineto{\pgfqpoint{4.527515in}{4.133440in}}%
\pgfpathlineto{\pgfqpoint{4.511383in}{4.149333in}}%
\pgfpathlineto{\pgfqpoint{4.499688in}{4.160747in}}%
\pgfpathlineto{\pgfqpoint{4.487434in}{4.172913in}}%
\pgfpathlineto{\pgfqpoint{4.473432in}{4.186667in}}%
\pgfpathlineto{\pgfqpoint{4.460678in}{4.199077in}}%
\pgfpathlineto{\pgfqpoint{4.447354in}{4.212267in}}%
\pgfpathlineto{\pgfqpoint{4.435373in}{4.224000in}}%
\pgfpathlineto{\pgfqpoint{4.432803in}{4.224000in}}%
\pgfpathlineto{\pgfqpoint{4.447354in}{4.209750in}}%
\pgfpathlineto{\pgfqpoint{4.459368in}{4.197857in}}%
\pgfpathlineto{\pgfqpoint{4.470869in}{4.186667in}}%
\pgfpathlineto{\pgfqpoint{4.487434in}{4.170395in}}%
\pgfpathlineto{\pgfqpoint{4.498379in}{4.159528in}}%
\pgfpathlineto{\pgfqpoint{4.508825in}{4.149333in}}%
\pgfpathlineto{\pgfqpoint{4.527515in}{4.130920in}}%
\pgfpathlineto{\pgfqpoint{4.537332in}{4.121144in}}%
\pgfpathlineto{\pgfqpoint{4.546674in}{4.112000in}}%
\pgfpathlineto{\pgfqpoint{4.567596in}{4.091326in}}%
\pgfpathlineto{\pgfqpoint{4.584415in}{4.074667in}}%
\pgfpathlineto{\pgfqpoint{4.595633in}{4.063449in}}%
\pgfpathlineto{\pgfqpoint{4.607677in}{4.051612in}}%
\pgfpathlineto{\pgfqpoint{4.622050in}{4.037333in}}%
\pgfpathlineto{\pgfqpoint{4.647758in}{4.011779in}}%
\pgfpathlineto{\pgfqpoint{4.653841in}{4.005667in}}%
\pgfpathlineto{\pgfqpoint{4.659579in}{4.000000in}}%
\pgfpathlineto{\pgfqpoint{4.687838in}{3.971825in}}%
\pgfpathlineto{\pgfqpoint{4.692562in}{3.967066in}}%
\pgfpathlineto{\pgfqpoint{4.697003in}{3.962667in}}%
\pgfpathlineto{\pgfqpoint{4.727919in}{3.931752in}}%
\pgfpathlineto{\pgfqpoint{4.734323in}{3.925333in}}%
\pgfpathlineto{\pgfqpoint{4.750464in}{3.908999in}}%
\pgfpathlineto{\pgfqpoint{4.768000in}{3.891558in}}%
\pgfusepath{fill}%
\end{pgfscope}%
\begin{pgfscope}%
\pgfpathrectangle{\pgfqpoint{0.800000in}{0.528000in}}{\pgfqpoint{3.968000in}{3.696000in}}%
\pgfusepath{clip}%
\pgfsetbuttcap%
\pgfsetroundjoin%
\definecolor{currentfill}{rgb}{0.626579,0.854645,0.223353}%
\pgfsetfillcolor{currentfill}%
\pgfsetlinewidth{0.000000pt}%
\definecolor{currentstroke}{rgb}{0.000000,0.000000,0.000000}%
\pgfsetstrokecolor{currentstroke}%
\pgfsetdash{}{0pt}%
\pgfpathmoveto{\pgfqpoint{4.768000in}{3.896617in}}%
\pgfpathlineto{\pgfqpoint{4.753090in}{3.911446in}}%
\pgfpathlineto{\pgfqpoint{4.739367in}{3.925333in}}%
\pgfpathlineto{\pgfqpoint{4.727919in}{3.936807in}}%
\pgfpathlineto{\pgfqpoint{4.702059in}{3.962667in}}%
\pgfpathlineto{\pgfqpoint{4.695167in}{3.969493in}}%
\pgfpathlineto{\pgfqpoint{4.687838in}{3.976878in}}%
\pgfpathlineto{\pgfqpoint{4.664647in}{4.000000in}}%
\pgfpathlineto{\pgfqpoint{4.656449in}{4.008096in}}%
\pgfpathlineto{\pgfqpoint{4.647758in}{4.016828in}}%
\pgfpathlineto{\pgfqpoint{4.627130in}{4.037333in}}%
\pgfpathlineto{\pgfqpoint{4.607677in}{4.056658in}}%
\pgfpathlineto{\pgfqpoint{4.598269in}{4.065904in}}%
\pgfpathlineto{\pgfqpoint{4.589507in}{4.074667in}}%
\pgfpathlineto{\pgfqpoint{4.567596in}{4.096369in}}%
\pgfpathlineto{\pgfqpoint{4.551777in}{4.112000in}}%
\pgfpathlineto{\pgfqpoint{4.539947in}{4.123580in}}%
\pgfpathlineto{\pgfqpoint{4.527515in}{4.135960in}}%
\pgfpathlineto{\pgfqpoint{4.513941in}{4.149333in}}%
\pgfpathlineto{\pgfqpoint{4.500997in}{4.161966in}}%
\pgfpathlineto{\pgfqpoint{4.487434in}{4.175431in}}%
\pgfpathlineto{\pgfqpoint{4.475996in}{4.186667in}}%
\pgfpathlineto{\pgfqpoint{4.461987in}{4.200297in}}%
\pgfpathlineto{\pgfqpoint{4.447354in}{4.214784in}}%
\pgfpathlineto{\pgfqpoint{4.437943in}{4.224000in}}%
\pgfpathlineto{\pgfqpoint{4.435373in}{4.224000in}}%
\pgfpathlineto{\pgfqpoint{4.447354in}{4.212267in}}%
\pgfpathlineto{\pgfqpoint{4.460678in}{4.199077in}}%
\pgfpathlineto{\pgfqpoint{4.473432in}{4.186667in}}%
\pgfpathlineto{\pgfqpoint{4.487434in}{4.172913in}}%
\pgfpathlineto{\pgfqpoint{4.499688in}{4.160747in}}%
\pgfpathlineto{\pgfqpoint{4.511383in}{4.149333in}}%
\pgfpathlineto{\pgfqpoint{4.527515in}{4.133440in}}%
\pgfpathlineto{\pgfqpoint{4.538640in}{4.122362in}}%
\pgfpathlineto{\pgfqpoint{4.549226in}{4.112000in}}%
\pgfpathlineto{\pgfqpoint{4.567596in}{4.093847in}}%
\pgfpathlineto{\pgfqpoint{4.586961in}{4.074667in}}%
\pgfpathlineto{\pgfqpoint{4.596951in}{4.064676in}}%
\pgfpathlineto{\pgfqpoint{4.607677in}{4.054135in}}%
\pgfpathlineto{\pgfqpoint{4.624590in}{4.037333in}}%
\pgfpathlineto{\pgfqpoint{4.647758in}{4.014303in}}%
\pgfpathlineto{\pgfqpoint{4.655145in}{4.006881in}}%
\pgfpathlineto{\pgfqpoint{4.662113in}{4.000000in}}%
\pgfpathlineto{\pgfqpoint{4.687838in}{3.974352in}}%
\pgfpathlineto{\pgfqpoint{4.693864in}{3.968280in}}%
\pgfpathlineto{\pgfqpoint{4.699531in}{3.962667in}}%
\pgfpathlineto{\pgfqpoint{4.727919in}{3.934279in}}%
\pgfpathlineto{\pgfqpoint{4.736845in}{3.925333in}}%
\pgfpathlineto{\pgfqpoint{4.751777in}{3.910223in}}%
\pgfpathlineto{\pgfqpoint{4.768000in}{3.894087in}}%
\pgfusepath{fill}%
\end{pgfscope}%
\begin{pgfscope}%
\pgfpathrectangle{\pgfqpoint{0.800000in}{0.528000in}}{\pgfqpoint{3.968000in}{3.696000in}}%
\pgfusepath{clip}%
\pgfsetbuttcap%
\pgfsetroundjoin%
\definecolor{currentfill}{rgb}{0.626579,0.854645,0.223353}%
\pgfsetfillcolor{currentfill}%
\pgfsetlinewidth{0.000000pt}%
\definecolor{currentstroke}{rgb}{0.000000,0.000000,0.000000}%
\pgfsetstrokecolor{currentstroke}%
\pgfsetdash{}{0pt}%
\pgfpathmoveto{\pgfqpoint{4.768000in}{3.899146in}}%
\pgfpathlineto{\pgfqpoint{4.754404in}{3.912669in}}%
\pgfpathlineto{\pgfqpoint{4.741889in}{3.925333in}}%
\pgfpathlineto{\pgfqpoint{4.727919in}{3.939335in}}%
\pgfpathlineto{\pgfqpoint{4.704587in}{3.962667in}}%
\pgfpathlineto{\pgfqpoint{4.696470in}{3.970707in}}%
\pgfpathlineto{\pgfqpoint{4.687838in}{3.979404in}}%
\pgfpathlineto{\pgfqpoint{4.667181in}{4.000000in}}%
\pgfpathlineto{\pgfqpoint{4.657753in}{4.009310in}}%
\pgfpathlineto{\pgfqpoint{4.647758in}{4.019353in}}%
\pgfpathlineto{\pgfqpoint{4.629669in}{4.037333in}}%
\pgfpathlineto{\pgfqpoint{4.607677in}{4.059181in}}%
\pgfpathlineto{\pgfqpoint{4.599587in}{4.067132in}}%
\pgfpathlineto{\pgfqpoint{4.592052in}{4.074667in}}%
\pgfpathlineto{\pgfqpoint{4.567596in}{4.098890in}}%
\pgfpathlineto{\pgfqpoint{4.554329in}{4.112000in}}%
\pgfpathlineto{\pgfqpoint{4.541255in}{4.124798in}}%
\pgfpathlineto{\pgfqpoint{4.527515in}{4.138480in}}%
\pgfpathlineto{\pgfqpoint{4.516498in}{4.149333in}}%
\pgfpathlineto{\pgfqpoint{4.502305in}{4.163185in}}%
\pgfpathlineto{\pgfqpoint{4.487434in}{4.177950in}}%
\pgfpathlineto{\pgfqpoint{4.478560in}{4.186667in}}%
\pgfpathlineto{\pgfqpoint{4.463297in}{4.201517in}}%
\pgfpathlineto{\pgfqpoint{4.447354in}{4.217301in}}%
\pgfpathlineto{\pgfqpoint{4.440513in}{4.224000in}}%
\pgfpathlineto{\pgfqpoint{4.437943in}{4.224000in}}%
\pgfpathlineto{\pgfqpoint{4.447354in}{4.214784in}}%
\pgfpathlineto{\pgfqpoint{4.461987in}{4.200297in}}%
\pgfpathlineto{\pgfqpoint{4.475996in}{4.186667in}}%
\pgfpathlineto{\pgfqpoint{4.487434in}{4.175431in}}%
\pgfpathlineto{\pgfqpoint{4.500997in}{4.161966in}}%
\pgfpathlineto{\pgfqpoint{4.513941in}{4.149333in}}%
\pgfpathlineto{\pgfqpoint{4.527515in}{4.135960in}}%
\pgfpathlineto{\pgfqpoint{4.539947in}{4.123580in}}%
\pgfpathlineto{\pgfqpoint{4.551777in}{4.112000in}}%
\pgfpathlineto{\pgfqpoint{4.567596in}{4.096369in}}%
\pgfpathlineto{\pgfqpoint{4.589507in}{4.074667in}}%
\pgfpathlineto{\pgfqpoint{4.598269in}{4.065904in}}%
\pgfpathlineto{\pgfqpoint{4.607677in}{4.056658in}}%
\pgfpathlineto{\pgfqpoint{4.627130in}{4.037333in}}%
\pgfpathlineto{\pgfqpoint{4.647758in}{4.016828in}}%
\pgfpathlineto{\pgfqpoint{4.656449in}{4.008096in}}%
\pgfpathlineto{\pgfqpoint{4.664647in}{4.000000in}}%
\pgfpathlineto{\pgfqpoint{4.687838in}{3.976878in}}%
\pgfpathlineto{\pgfqpoint{4.695167in}{3.969493in}}%
\pgfpathlineto{\pgfqpoint{4.702059in}{3.962667in}}%
\pgfpathlineto{\pgfqpoint{4.727919in}{3.936807in}}%
\pgfpathlineto{\pgfqpoint{4.739367in}{3.925333in}}%
\pgfpathlineto{\pgfqpoint{4.753090in}{3.911446in}}%
\pgfpathlineto{\pgfqpoint{4.768000in}{3.896617in}}%
\pgfusepath{fill}%
\end{pgfscope}%
\begin{pgfscope}%
\pgfpathrectangle{\pgfqpoint{0.800000in}{0.528000in}}{\pgfqpoint{3.968000in}{3.696000in}}%
\pgfusepath{clip}%
\pgfsetbuttcap%
\pgfsetroundjoin%
\definecolor{currentfill}{rgb}{0.636902,0.856542,0.216620}%
\pgfsetfillcolor{currentfill}%
\pgfsetlinewidth{0.000000pt}%
\definecolor{currentstroke}{rgb}{0.000000,0.000000,0.000000}%
\pgfsetstrokecolor{currentstroke}%
\pgfsetdash{}{0pt}%
\pgfpathmoveto{\pgfqpoint{4.768000in}{3.901675in}}%
\pgfpathlineto{\pgfqpoint{4.755717in}{3.913892in}}%
\pgfpathlineto{\pgfqpoint{4.744411in}{3.925333in}}%
\pgfpathlineto{\pgfqpoint{4.727919in}{3.941863in}}%
\pgfpathlineto{\pgfqpoint{4.707115in}{3.962667in}}%
\pgfpathlineto{\pgfqpoint{4.697773in}{3.971920in}}%
\pgfpathlineto{\pgfqpoint{4.687838in}{3.981930in}}%
\pgfpathlineto{\pgfqpoint{4.669715in}{4.000000in}}%
\pgfpathlineto{\pgfqpoint{4.659057in}{4.010525in}}%
\pgfpathlineto{\pgfqpoint{4.647758in}{4.021877in}}%
\pgfpathlineto{\pgfqpoint{4.632209in}{4.037333in}}%
\pgfpathlineto{\pgfqpoint{4.607677in}{4.061704in}}%
\pgfpathlineto{\pgfqpoint{4.600905in}{4.068359in}}%
\pgfpathlineto{\pgfqpoint{4.594598in}{4.074667in}}%
\pgfpathlineto{\pgfqpoint{4.567596in}{4.101412in}}%
\pgfpathlineto{\pgfqpoint{4.556881in}{4.112000in}}%
\pgfpathlineto{\pgfqpoint{4.542562in}{4.126016in}}%
\pgfpathlineto{\pgfqpoint{4.527515in}{4.140999in}}%
\pgfpathlineto{\pgfqpoint{4.519056in}{4.149333in}}%
\pgfpathlineto{\pgfqpoint{4.503614in}{4.164404in}}%
\pgfpathlineto{\pgfqpoint{4.487434in}{4.180468in}}%
\pgfpathlineto{\pgfqpoint{4.481124in}{4.186667in}}%
\pgfpathlineto{\pgfqpoint{4.464607in}{4.202738in}}%
\pgfpathlineto{\pgfqpoint{4.447354in}{4.219817in}}%
\pgfpathlineto{\pgfqpoint{4.443083in}{4.224000in}}%
\pgfpathlineto{\pgfqpoint{4.440513in}{4.224000in}}%
\pgfpathlineto{\pgfqpoint{4.447354in}{4.217301in}}%
\pgfpathlineto{\pgfqpoint{4.463297in}{4.201517in}}%
\pgfpathlineto{\pgfqpoint{4.478560in}{4.186667in}}%
\pgfpathlineto{\pgfqpoint{4.487434in}{4.177950in}}%
\pgfpathlineto{\pgfqpoint{4.502305in}{4.163185in}}%
\pgfpathlineto{\pgfqpoint{4.516498in}{4.149333in}}%
\pgfpathlineto{\pgfqpoint{4.527515in}{4.138480in}}%
\pgfpathlineto{\pgfqpoint{4.541255in}{4.124798in}}%
\pgfpathlineto{\pgfqpoint{4.554329in}{4.112000in}}%
\pgfpathlineto{\pgfqpoint{4.567596in}{4.098890in}}%
\pgfpathlineto{\pgfqpoint{4.592052in}{4.074667in}}%
\pgfpathlineto{\pgfqpoint{4.599587in}{4.067132in}}%
\pgfpathlineto{\pgfqpoint{4.607677in}{4.059181in}}%
\pgfpathlineto{\pgfqpoint{4.629669in}{4.037333in}}%
\pgfpathlineto{\pgfqpoint{4.647758in}{4.019353in}}%
\pgfpathlineto{\pgfqpoint{4.657753in}{4.009310in}}%
\pgfpathlineto{\pgfqpoint{4.667181in}{4.000000in}}%
\pgfpathlineto{\pgfqpoint{4.687838in}{3.979404in}}%
\pgfpathlineto{\pgfqpoint{4.696470in}{3.970707in}}%
\pgfpathlineto{\pgfqpoint{4.704587in}{3.962667in}}%
\pgfpathlineto{\pgfqpoint{4.727919in}{3.939335in}}%
\pgfpathlineto{\pgfqpoint{4.741889in}{3.925333in}}%
\pgfpathlineto{\pgfqpoint{4.754404in}{3.912669in}}%
\pgfpathlineto{\pgfqpoint{4.768000in}{3.899146in}}%
\pgfusepath{fill}%
\end{pgfscope}%
\begin{pgfscope}%
\pgfpathrectangle{\pgfqpoint{0.800000in}{0.528000in}}{\pgfqpoint{3.968000in}{3.696000in}}%
\pgfusepath{clip}%
\pgfsetbuttcap%
\pgfsetroundjoin%
\definecolor{currentfill}{rgb}{0.636902,0.856542,0.216620}%
\pgfsetfillcolor{currentfill}%
\pgfsetlinewidth{0.000000pt}%
\definecolor{currentstroke}{rgb}{0.000000,0.000000,0.000000}%
\pgfsetstrokecolor{currentstroke}%
\pgfsetdash{}{0pt}%
\pgfpathmoveto{\pgfqpoint{4.768000in}{3.904205in}}%
\pgfpathlineto{\pgfqpoint{4.757030in}{3.915116in}}%
\pgfpathlineto{\pgfqpoint{4.746933in}{3.925333in}}%
\pgfpathlineto{\pgfqpoint{4.727919in}{3.944391in}}%
\pgfpathlineto{\pgfqpoint{4.709643in}{3.962667in}}%
\pgfpathlineto{\pgfqpoint{4.699076in}{3.973134in}}%
\pgfpathlineto{\pgfqpoint{4.687838in}{3.984457in}}%
\pgfpathlineto{\pgfqpoint{4.672248in}{4.000000in}}%
\pgfpathlineto{\pgfqpoint{4.660361in}{4.011740in}}%
\pgfpathlineto{\pgfqpoint{4.647758in}{4.024402in}}%
\pgfpathlineto{\pgfqpoint{4.634749in}{4.037333in}}%
\pgfpathlineto{\pgfqpoint{4.607677in}{4.064228in}}%
\pgfpathlineto{\pgfqpoint{4.602223in}{4.069587in}}%
\pgfpathlineto{\pgfqpoint{4.597144in}{4.074667in}}%
\pgfpathlineto{\pgfqpoint{4.567596in}{4.103933in}}%
\pgfpathlineto{\pgfqpoint{4.559432in}{4.112000in}}%
\pgfpathlineto{\pgfqpoint{4.543870in}{4.127233in}}%
\pgfpathlineto{\pgfqpoint{4.527515in}{4.143519in}}%
\pgfpathlineto{\pgfqpoint{4.521614in}{4.149333in}}%
\pgfpathlineto{\pgfqpoint{4.504923in}{4.165623in}}%
\pgfpathlineto{\pgfqpoint{4.487434in}{4.182986in}}%
\pgfpathlineto{\pgfqpoint{4.483688in}{4.186667in}}%
\pgfpathlineto{\pgfqpoint{4.465917in}{4.203958in}}%
\pgfpathlineto{\pgfqpoint{4.447354in}{4.222334in}}%
\pgfpathlineto{\pgfqpoint{4.445652in}{4.224000in}}%
\pgfpathlineto{\pgfqpoint{4.443083in}{4.224000in}}%
\pgfpathlineto{\pgfqpoint{4.447354in}{4.219817in}}%
\pgfpathlineto{\pgfqpoint{4.464607in}{4.202738in}}%
\pgfpathlineto{\pgfqpoint{4.481124in}{4.186667in}}%
\pgfpathlineto{\pgfqpoint{4.487434in}{4.180468in}}%
\pgfpathlineto{\pgfqpoint{4.503614in}{4.164404in}}%
\pgfpathlineto{\pgfqpoint{4.519056in}{4.149333in}}%
\pgfpathlineto{\pgfqpoint{4.527515in}{4.140999in}}%
\pgfpathlineto{\pgfqpoint{4.542562in}{4.126016in}}%
\pgfpathlineto{\pgfqpoint{4.556881in}{4.112000in}}%
\pgfpathlineto{\pgfqpoint{4.567596in}{4.101412in}}%
\pgfpathlineto{\pgfqpoint{4.594598in}{4.074667in}}%
\pgfpathlineto{\pgfqpoint{4.600905in}{4.068359in}}%
\pgfpathlineto{\pgfqpoint{4.607677in}{4.061704in}}%
\pgfpathlineto{\pgfqpoint{4.632209in}{4.037333in}}%
\pgfpathlineto{\pgfqpoint{4.647758in}{4.021877in}}%
\pgfpathlineto{\pgfqpoint{4.659057in}{4.010525in}}%
\pgfpathlineto{\pgfqpoint{4.669715in}{4.000000in}}%
\pgfpathlineto{\pgfqpoint{4.687838in}{3.981930in}}%
\pgfpathlineto{\pgfqpoint{4.697773in}{3.971920in}}%
\pgfpathlineto{\pgfqpoint{4.707115in}{3.962667in}}%
\pgfpathlineto{\pgfqpoint{4.727919in}{3.941863in}}%
\pgfpathlineto{\pgfqpoint{4.744411in}{3.925333in}}%
\pgfpathlineto{\pgfqpoint{4.755717in}{3.913892in}}%
\pgfpathlineto{\pgfqpoint{4.768000in}{3.901675in}}%
\pgfusepath{fill}%
\end{pgfscope}%
\begin{pgfscope}%
\pgfpathrectangle{\pgfqpoint{0.800000in}{0.528000in}}{\pgfqpoint{3.968000in}{3.696000in}}%
\pgfusepath{clip}%
\pgfsetbuttcap%
\pgfsetroundjoin%
\definecolor{currentfill}{rgb}{0.636902,0.856542,0.216620}%
\pgfsetfillcolor{currentfill}%
\pgfsetlinewidth{0.000000pt}%
\definecolor{currentstroke}{rgb}{0.000000,0.000000,0.000000}%
\pgfsetstrokecolor{currentstroke}%
\pgfsetdash{}{0pt}%
\pgfpathmoveto{\pgfqpoint{4.768000in}{3.906734in}}%
\pgfpathlineto{\pgfqpoint{4.758344in}{3.916339in}}%
\pgfpathlineto{\pgfqpoint{4.749455in}{3.925333in}}%
\pgfpathlineto{\pgfqpoint{4.727919in}{3.946919in}}%
\pgfpathlineto{\pgfqpoint{4.712171in}{3.962667in}}%
\pgfpathlineto{\pgfqpoint{4.700378in}{3.974347in}}%
\pgfpathlineto{\pgfqpoint{4.687838in}{3.986983in}}%
\pgfpathlineto{\pgfqpoint{4.674782in}{4.000000in}}%
\pgfpathlineto{\pgfqpoint{4.661665in}{4.012954in}}%
\pgfpathlineto{\pgfqpoint{4.647758in}{4.026927in}}%
\pgfpathlineto{\pgfqpoint{4.637289in}{4.037333in}}%
\pgfpathlineto{\pgfqpoint{4.607677in}{4.066751in}}%
\pgfpathlineto{\pgfqpoint{4.603541in}{4.070815in}}%
\pgfpathlineto{\pgfqpoint{4.599690in}{4.074667in}}%
\pgfpathlineto{\pgfqpoint{4.567596in}{4.106455in}}%
\pgfpathlineto{\pgfqpoint{4.561984in}{4.112000in}}%
\pgfpathlineto{\pgfqpoint{4.545177in}{4.128451in}}%
\pgfpathlineto{\pgfqpoint{4.527515in}{4.146039in}}%
\pgfpathlineto{\pgfqpoint{4.524172in}{4.149333in}}%
\pgfpathlineto{\pgfqpoint{4.506231in}{4.166842in}}%
\pgfpathlineto{\pgfqpoint{4.487434in}{4.185505in}}%
\pgfpathlineto{\pgfqpoint{4.486251in}{4.186667in}}%
\pgfpathlineto{\pgfqpoint{4.467227in}{4.205178in}}%
\pgfpathlineto{\pgfqpoint{4.448213in}{4.224000in}}%
\pgfpathlineto{\pgfqpoint{4.447354in}{4.224000in}}%
\pgfpathlineto{\pgfqpoint{4.445652in}{4.224000in}}%
\pgfpathlineto{\pgfqpoint{4.447354in}{4.222334in}}%
\pgfpathlineto{\pgfqpoint{4.465917in}{4.203958in}}%
\pgfpathlineto{\pgfqpoint{4.483688in}{4.186667in}}%
\pgfpathlineto{\pgfqpoint{4.487434in}{4.182986in}}%
\pgfpathlineto{\pgfqpoint{4.504923in}{4.165623in}}%
\pgfpathlineto{\pgfqpoint{4.521614in}{4.149333in}}%
\pgfpathlineto{\pgfqpoint{4.527515in}{4.143519in}}%
\pgfpathlineto{\pgfqpoint{4.543870in}{4.127233in}}%
\pgfpathlineto{\pgfqpoint{4.559432in}{4.112000in}}%
\pgfpathlineto{\pgfqpoint{4.567596in}{4.103933in}}%
\pgfpathlineto{\pgfqpoint{4.597144in}{4.074667in}}%
\pgfpathlineto{\pgfqpoint{4.602223in}{4.069587in}}%
\pgfpathlineto{\pgfqpoint{4.607677in}{4.064228in}}%
\pgfpathlineto{\pgfqpoint{4.634749in}{4.037333in}}%
\pgfpathlineto{\pgfqpoint{4.647758in}{4.024402in}}%
\pgfpathlineto{\pgfqpoint{4.660361in}{4.011740in}}%
\pgfpathlineto{\pgfqpoint{4.672248in}{4.000000in}}%
\pgfpathlineto{\pgfqpoint{4.687838in}{3.984457in}}%
\pgfpathlineto{\pgfqpoint{4.699076in}{3.973134in}}%
\pgfpathlineto{\pgfqpoint{4.709643in}{3.962667in}}%
\pgfpathlineto{\pgfqpoint{4.727919in}{3.944391in}}%
\pgfpathlineto{\pgfqpoint{4.746933in}{3.925333in}}%
\pgfpathlineto{\pgfqpoint{4.757030in}{3.915116in}}%
\pgfpathlineto{\pgfqpoint{4.768000in}{3.904205in}}%
\pgfusepath{fill}%
\end{pgfscope}%
\begin{pgfscope}%
\pgfpathrectangle{\pgfqpoint{0.800000in}{0.528000in}}{\pgfqpoint{3.968000in}{3.696000in}}%
\pgfusepath{clip}%
\pgfsetbuttcap%
\pgfsetroundjoin%
\definecolor{currentfill}{rgb}{0.636902,0.856542,0.216620}%
\pgfsetfillcolor{currentfill}%
\pgfsetlinewidth{0.000000pt}%
\definecolor{currentstroke}{rgb}{0.000000,0.000000,0.000000}%
\pgfsetstrokecolor{currentstroke}%
\pgfsetdash{}{0pt}%
\pgfpathmoveto{\pgfqpoint{4.768000in}{3.909264in}}%
\pgfpathlineto{\pgfqpoint{4.759657in}{3.917562in}}%
\pgfpathlineto{\pgfqpoint{4.751977in}{3.925333in}}%
\pgfpathlineto{\pgfqpoint{4.727919in}{3.949447in}}%
\pgfpathlineto{\pgfqpoint{4.714699in}{3.962667in}}%
\pgfpathlineto{\pgfqpoint{4.701681in}{3.975561in}}%
\pgfpathlineto{\pgfqpoint{4.687838in}{3.989509in}}%
\pgfpathlineto{\pgfqpoint{4.677316in}{4.000000in}}%
\pgfpathlineto{\pgfqpoint{4.662969in}{4.014169in}}%
\pgfpathlineto{\pgfqpoint{4.647758in}{4.029451in}}%
\pgfpathlineto{\pgfqpoint{4.639829in}{4.037333in}}%
\pgfpathlineto{\pgfqpoint{4.607677in}{4.069274in}}%
\pgfpathlineto{\pgfqpoint{4.604860in}{4.072043in}}%
\pgfpathlineto{\pgfqpoint{4.602235in}{4.074667in}}%
\pgfpathlineto{\pgfqpoint{4.567596in}{4.108976in}}%
\pgfpathlineto{\pgfqpoint{4.564536in}{4.112000in}}%
\pgfpathlineto{\pgfqpoint{4.546485in}{4.129669in}}%
\pgfpathlineto{\pgfqpoint{4.527515in}{4.148559in}}%
\pgfpathlineto{\pgfqpoint{4.526729in}{4.149333in}}%
\pgfpathlineto{\pgfqpoint{4.507540in}{4.168061in}}%
\pgfpathlineto{\pgfqpoint{4.488800in}{4.186667in}}%
\pgfpathlineto{\pgfqpoint{4.487434in}{4.188010in}}%
\pgfpathlineto{\pgfqpoint{4.468537in}{4.206398in}}%
\pgfpathlineto{\pgfqpoint{4.450755in}{4.224000in}}%
\pgfpathlineto{\pgfqpoint{4.448213in}{4.224000in}}%
\pgfpathlineto{\pgfqpoint{4.467227in}{4.205178in}}%
\pgfpathlineto{\pgfqpoint{4.486251in}{4.186667in}}%
\pgfpathlineto{\pgfqpoint{4.487434in}{4.185505in}}%
\pgfpathlineto{\pgfqpoint{4.506231in}{4.166842in}}%
\pgfpathlineto{\pgfqpoint{4.524172in}{4.149333in}}%
\pgfpathlineto{\pgfqpoint{4.527515in}{4.146039in}}%
\pgfpathlineto{\pgfqpoint{4.545177in}{4.128451in}}%
\pgfpathlineto{\pgfqpoint{4.561984in}{4.112000in}}%
\pgfpathlineto{\pgfqpoint{4.567596in}{4.106455in}}%
\pgfpathlineto{\pgfqpoint{4.599690in}{4.074667in}}%
\pgfpathlineto{\pgfqpoint{4.603541in}{4.070815in}}%
\pgfpathlineto{\pgfqpoint{4.607677in}{4.066751in}}%
\pgfpathlineto{\pgfqpoint{4.637289in}{4.037333in}}%
\pgfpathlineto{\pgfqpoint{4.647758in}{4.026927in}}%
\pgfpathlineto{\pgfqpoint{4.661665in}{4.012954in}}%
\pgfpathlineto{\pgfqpoint{4.674782in}{4.000000in}}%
\pgfpathlineto{\pgfqpoint{4.687838in}{3.986983in}}%
\pgfpathlineto{\pgfqpoint{4.700378in}{3.974347in}}%
\pgfpathlineto{\pgfqpoint{4.712171in}{3.962667in}}%
\pgfpathlineto{\pgfqpoint{4.727919in}{3.946919in}}%
\pgfpathlineto{\pgfqpoint{4.749455in}{3.925333in}}%
\pgfpathlineto{\pgfqpoint{4.758344in}{3.916339in}}%
\pgfpathlineto{\pgfqpoint{4.768000in}{3.906734in}}%
\pgfusepath{fill}%
\end{pgfscope}%
\begin{pgfscope}%
\pgfpathrectangle{\pgfqpoint{0.800000in}{0.528000in}}{\pgfqpoint{3.968000in}{3.696000in}}%
\pgfusepath{clip}%
\pgfsetbuttcap%
\pgfsetroundjoin%
\definecolor{currentfill}{rgb}{0.647257,0.858400,0.209861}%
\pgfsetfillcolor{currentfill}%
\pgfsetlinewidth{0.000000pt}%
\definecolor{currentstroke}{rgb}{0.000000,0.000000,0.000000}%
\pgfsetstrokecolor{currentstroke}%
\pgfsetdash{}{0pt}%
\pgfpathmoveto{\pgfqpoint{4.768000in}{3.911793in}}%
\pgfpathlineto{\pgfqpoint{4.760970in}{3.918785in}}%
\pgfpathlineto{\pgfqpoint{4.754500in}{3.925333in}}%
\pgfpathlineto{\pgfqpoint{4.727919in}{3.951975in}}%
\pgfpathlineto{\pgfqpoint{4.717227in}{3.962667in}}%
\pgfpathlineto{\pgfqpoint{4.702984in}{3.976774in}}%
\pgfpathlineto{\pgfqpoint{4.687838in}{3.992036in}}%
\pgfpathlineto{\pgfqpoint{4.679850in}{4.000000in}}%
\pgfpathlineto{\pgfqpoint{4.664273in}{4.015383in}}%
\pgfpathlineto{\pgfqpoint{4.647758in}{4.031976in}}%
\pgfpathlineto{\pgfqpoint{4.642368in}{4.037333in}}%
\pgfpathlineto{\pgfqpoint{4.607677in}{4.071797in}}%
\pgfpathlineto{\pgfqpoint{4.606178in}{4.073270in}}%
\pgfpathlineto{\pgfqpoint{4.604781in}{4.074667in}}%
\pgfpathlineto{\pgfqpoint{4.567596in}{4.111498in}}%
\pgfpathlineto{\pgfqpoint{4.567088in}{4.112000in}}%
\pgfpathlineto{\pgfqpoint{4.547792in}{4.130887in}}%
\pgfpathlineto{\pgfqpoint{4.529268in}{4.149333in}}%
\pgfpathlineto{\pgfqpoint{4.527515in}{4.151063in}}%
\pgfpathlineto{\pgfqpoint{4.508849in}{4.169280in}}%
\pgfpathlineto{\pgfqpoint{4.491337in}{4.186667in}}%
\pgfpathlineto{\pgfqpoint{4.487434in}{4.190505in}}%
\pgfpathlineto{\pgfqpoint{4.469847in}{4.207618in}}%
\pgfpathlineto{\pgfqpoint{4.453298in}{4.224000in}}%
\pgfpathlineto{\pgfqpoint{4.450755in}{4.224000in}}%
\pgfpathlineto{\pgfqpoint{4.468537in}{4.206398in}}%
\pgfpathlineto{\pgfqpoint{4.487434in}{4.188010in}}%
\pgfpathlineto{\pgfqpoint{4.488800in}{4.186667in}}%
\pgfpathlineto{\pgfqpoint{4.507540in}{4.168061in}}%
\pgfpathlineto{\pgfqpoint{4.526729in}{4.149333in}}%
\pgfpathlineto{\pgfqpoint{4.527515in}{4.148559in}}%
\pgfpathlineto{\pgfqpoint{4.546485in}{4.129669in}}%
\pgfpathlineto{\pgfqpoint{4.564536in}{4.112000in}}%
\pgfpathlineto{\pgfqpoint{4.567596in}{4.108976in}}%
\pgfpathlineto{\pgfqpoint{4.602235in}{4.074667in}}%
\pgfpathlineto{\pgfqpoint{4.604860in}{4.072043in}}%
\pgfpathlineto{\pgfqpoint{4.607677in}{4.069274in}}%
\pgfpathlineto{\pgfqpoint{4.639829in}{4.037333in}}%
\pgfpathlineto{\pgfqpoint{4.647758in}{4.029451in}}%
\pgfpathlineto{\pgfqpoint{4.662969in}{4.014169in}}%
\pgfpathlineto{\pgfqpoint{4.677316in}{4.000000in}}%
\pgfpathlineto{\pgfqpoint{4.687838in}{3.989509in}}%
\pgfpathlineto{\pgfqpoint{4.701681in}{3.975561in}}%
\pgfpathlineto{\pgfqpoint{4.714699in}{3.962667in}}%
\pgfpathlineto{\pgfqpoint{4.727919in}{3.949447in}}%
\pgfpathlineto{\pgfqpoint{4.751977in}{3.925333in}}%
\pgfpathlineto{\pgfqpoint{4.759657in}{3.917562in}}%
\pgfpathlineto{\pgfqpoint{4.768000in}{3.909264in}}%
\pgfusepath{fill}%
\end{pgfscope}%
\begin{pgfscope}%
\pgfpathrectangle{\pgfqpoint{0.800000in}{0.528000in}}{\pgfqpoint{3.968000in}{3.696000in}}%
\pgfusepath{clip}%
\pgfsetbuttcap%
\pgfsetroundjoin%
\definecolor{currentfill}{rgb}{0.647257,0.858400,0.209861}%
\pgfsetfillcolor{currentfill}%
\pgfsetlinewidth{0.000000pt}%
\definecolor{currentstroke}{rgb}{0.000000,0.000000,0.000000}%
\pgfsetstrokecolor{currentstroke}%
\pgfsetdash{}{0pt}%
\pgfpathmoveto{\pgfqpoint{4.768000in}{3.914323in}}%
\pgfpathlineto{\pgfqpoint{4.762283in}{3.920009in}}%
\pgfpathlineto{\pgfqpoint{4.757022in}{3.925333in}}%
\pgfpathlineto{\pgfqpoint{4.727919in}{3.954503in}}%
\pgfpathlineto{\pgfqpoint{4.719755in}{3.962667in}}%
\pgfpathlineto{\pgfqpoint{4.704287in}{3.977988in}}%
\pgfpathlineto{\pgfqpoint{4.687838in}{3.994562in}}%
\pgfpathlineto{\pgfqpoint{4.682384in}{4.000000in}}%
\pgfpathlineto{\pgfqpoint{4.665577in}{4.016598in}}%
\pgfpathlineto{\pgfqpoint{4.647758in}{4.034501in}}%
\pgfpathlineto{\pgfqpoint{4.644908in}{4.037333in}}%
\pgfpathlineto{\pgfqpoint{4.607677in}{4.074320in}}%
\pgfpathlineto{\pgfqpoint{4.607496in}{4.074498in}}%
\pgfpathlineto{\pgfqpoint{4.607327in}{4.074667in}}%
\pgfpathlineto{\pgfqpoint{4.601806in}{4.080135in}}%
\pgfpathlineto{\pgfqpoint{4.569618in}{4.112000in}}%
\pgfpathlineto{\pgfqpoint{4.568633in}{4.112966in}}%
\pgfpathlineto{\pgfqpoint{4.567596in}{4.114000in}}%
\pgfpathlineto{\pgfqpoint{4.549100in}{4.132105in}}%
\pgfpathlineto{\pgfqpoint{4.531799in}{4.149333in}}%
\pgfpathlineto{\pgfqpoint{4.527515in}{4.153559in}}%
\pgfpathlineto{\pgfqpoint{4.510157in}{4.170499in}}%
\pgfpathlineto{\pgfqpoint{4.493873in}{4.186667in}}%
\pgfpathlineto{\pgfqpoint{4.487434in}{4.192999in}}%
\pgfpathlineto{\pgfqpoint{4.471156in}{4.208838in}}%
\pgfpathlineto{\pgfqpoint{4.455840in}{4.224000in}}%
\pgfpathlineto{\pgfqpoint{4.453298in}{4.224000in}}%
\pgfpathlineto{\pgfqpoint{4.469847in}{4.207618in}}%
\pgfpathlineto{\pgfqpoint{4.487434in}{4.190505in}}%
\pgfpathlineto{\pgfqpoint{4.491337in}{4.186667in}}%
\pgfpathlineto{\pgfqpoint{4.508849in}{4.169280in}}%
\pgfpathlineto{\pgfqpoint{4.527515in}{4.151063in}}%
\pgfpathlineto{\pgfqpoint{4.529268in}{4.149333in}}%
\pgfpathlineto{\pgfqpoint{4.547792in}{4.130887in}}%
\pgfpathlineto{\pgfqpoint{4.567088in}{4.112000in}}%
\pgfpathlineto{\pgfqpoint{4.567596in}{4.111498in}}%
\pgfpathlineto{\pgfqpoint{4.604781in}{4.074667in}}%
\pgfpathlineto{\pgfqpoint{4.606178in}{4.073270in}}%
\pgfpathlineto{\pgfqpoint{4.607677in}{4.071797in}}%
\pgfpathlineto{\pgfqpoint{4.642368in}{4.037333in}}%
\pgfpathlineto{\pgfqpoint{4.647758in}{4.031976in}}%
\pgfpathlineto{\pgfqpoint{4.664273in}{4.015383in}}%
\pgfpathlineto{\pgfqpoint{4.679850in}{4.000000in}}%
\pgfpathlineto{\pgfqpoint{4.687838in}{3.992036in}}%
\pgfpathlineto{\pgfqpoint{4.702984in}{3.976774in}}%
\pgfpathlineto{\pgfqpoint{4.717227in}{3.962667in}}%
\pgfpathlineto{\pgfqpoint{4.727919in}{3.951975in}}%
\pgfpathlineto{\pgfqpoint{4.754500in}{3.925333in}}%
\pgfpathlineto{\pgfqpoint{4.760970in}{3.918785in}}%
\pgfpathlineto{\pgfqpoint{4.768000in}{3.911793in}}%
\pgfusepath{fill}%
\end{pgfscope}%
\begin{pgfscope}%
\pgfpathrectangle{\pgfqpoint{0.800000in}{0.528000in}}{\pgfqpoint{3.968000in}{3.696000in}}%
\pgfusepath{clip}%
\pgfsetbuttcap%
\pgfsetroundjoin%
\definecolor{currentfill}{rgb}{0.647257,0.858400,0.209861}%
\pgfsetfillcolor{currentfill}%
\pgfsetlinewidth{0.000000pt}%
\definecolor{currentstroke}{rgb}{0.000000,0.000000,0.000000}%
\pgfsetstrokecolor{currentstroke}%
\pgfsetdash{}{0pt}%
\pgfpathmoveto{\pgfqpoint{4.768000in}{3.916852in}}%
\pgfpathlineto{\pgfqpoint{4.763597in}{3.921232in}}%
\pgfpathlineto{\pgfqpoint{4.759544in}{3.925333in}}%
\pgfpathlineto{\pgfqpoint{4.727919in}{3.957030in}}%
\pgfpathlineto{\pgfqpoint{4.722283in}{3.962667in}}%
\pgfpathlineto{\pgfqpoint{4.705590in}{3.979201in}}%
\pgfpathlineto{\pgfqpoint{4.687838in}{3.997088in}}%
\pgfpathlineto{\pgfqpoint{4.684918in}{4.000000in}}%
\pgfpathlineto{\pgfqpoint{4.666881in}{4.017813in}}%
\pgfpathlineto{\pgfqpoint{4.647758in}{4.037026in}}%
\pgfpathlineto{\pgfqpoint{4.647448in}{4.037333in}}%
\pgfpathlineto{\pgfqpoint{4.642794in}{4.041956in}}%
\pgfpathlineto{\pgfqpoint{4.609849in}{4.074667in}}%
\pgfpathlineto{\pgfqpoint{4.608793in}{4.075706in}}%
\pgfpathlineto{\pgfqpoint{4.607677in}{4.076822in}}%
\pgfpathlineto{\pgfqpoint{4.572142in}{4.112000in}}%
\pgfpathlineto{\pgfqpoint{4.569928in}{4.114172in}}%
\pgfpathlineto{\pgfqpoint{4.567596in}{4.116498in}}%
\pgfpathlineto{\pgfqpoint{4.550407in}{4.133323in}}%
\pgfpathlineto{\pgfqpoint{4.534329in}{4.149333in}}%
\pgfpathlineto{\pgfqpoint{4.527515in}{4.156055in}}%
\pgfpathlineto{\pgfqpoint{4.511466in}{4.171718in}}%
\pgfpathlineto{\pgfqpoint{4.496410in}{4.186667in}}%
\pgfpathlineto{\pgfqpoint{4.487434in}{4.195494in}}%
\pgfpathlineto{\pgfqpoint{4.472466in}{4.210058in}}%
\pgfpathlineto{\pgfqpoint{4.458382in}{4.224000in}}%
\pgfpathlineto{\pgfqpoint{4.455840in}{4.224000in}}%
\pgfpathlineto{\pgfqpoint{4.471156in}{4.208838in}}%
\pgfpathlineto{\pgfqpoint{4.487434in}{4.192999in}}%
\pgfpathlineto{\pgfqpoint{4.493873in}{4.186667in}}%
\pgfpathlineto{\pgfqpoint{4.510157in}{4.170499in}}%
\pgfpathlineto{\pgfqpoint{4.527515in}{4.153559in}}%
\pgfpathlineto{\pgfqpoint{4.531799in}{4.149333in}}%
\pgfpathlineto{\pgfqpoint{4.549100in}{4.132105in}}%
\pgfpathlineto{\pgfqpoint{4.567596in}{4.114000in}}%
\pgfpathlineto{\pgfqpoint{4.568633in}{4.112966in}}%
\pgfpathlineto{\pgfqpoint{4.569618in}{4.112000in}}%
\pgfpathlineto{\pgfqpoint{4.601806in}{4.080135in}}%
\pgfpathlineto{\pgfqpoint{4.607327in}{4.074667in}}%
\pgfpathlineto{\pgfqpoint{4.607496in}{4.074498in}}%
\pgfpathlineto{\pgfqpoint{4.607677in}{4.074320in}}%
\pgfpathlineto{\pgfqpoint{4.644908in}{4.037333in}}%
\pgfpathlineto{\pgfqpoint{4.647758in}{4.034501in}}%
\pgfpathlineto{\pgfqpoint{4.665577in}{4.016598in}}%
\pgfpathlineto{\pgfqpoint{4.682384in}{4.000000in}}%
\pgfpathlineto{\pgfqpoint{4.687838in}{3.994562in}}%
\pgfpathlineto{\pgfqpoint{4.704287in}{3.977988in}}%
\pgfpathlineto{\pgfqpoint{4.719755in}{3.962667in}}%
\pgfpathlineto{\pgfqpoint{4.727919in}{3.954503in}}%
\pgfpathlineto{\pgfqpoint{4.757022in}{3.925333in}}%
\pgfpathlineto{\pgfqpoint{4.762283in}{3.920009in}}%
\pgfpathlineto{\pgfqpoint{4.768000in}{3.914323in}}%
\pgfusepath{fill}%
\end{pgfscope}%
\begin{pgfscope}%
\pgfpathrectangle{\pgfqpoint{0.800000in}{0.528000in}}{\pgfqpoint{3.968000in}{3.696000in}}%
\pgfusepath{clip}%
\pgfsetbuttcap%
\pgfsetroundjoin%
\definecolor{currentfill}{rgb}{0.657642,0.860219,0.203082}%
\pgfsetfillcolor{currentfill}%
\pgfsetlinewidth{0.000000pt}%
\definecolor{currentstroke}{rgb}{0.000000,0.000000,0.000000}%
\pgfsetstrokecolor{currentstroke}%
\pgfsetdash{}{0pt}%
\pgfpathmoveto{\pgfqpoint{4.768000in}{3.919382in}}%
\pgfpathlineto{\pgfqpoint{4.764910in}{3.922455in}}%
\pgfpathlineto{\pgfqpoint{4.762066in}{3.925333in}}%
\pgfpathlineto{\pgfqpoint{4.727919in}{3.959558in}}%
\pgfpathlineto{\pgfqpoint{4.724811in}{3.962667in}}%
\pgfpathlineto{\pgfqpoint{4.706893in}{3.980415in}}%
\pgfpathlineto{\pgfqpoint{4.687838in}{3.999614in}}%
\pgfpathlineto{\pgfqpoint{4.687452in}{4.000000in}}%
\pgfpathlineto{\pgfqpoint{4.668185in}{4.019027in}}%
\pgfpathlineto{\pgfqpoint{4.649964in}{4.037333in}}%
\pgfpathlineto{\pgfqpoint{4.648893in}{4.038390in}}%
\pgfpathlineto{\pgfqpoint{4.647758in}{4.039529in}}%
\pgfpathlineto{\pgfqpoint{4.612368in}{4.074667in}}%
\pgfpathlineto{\pgfqpoint{4.610086in}{4.076911in}}%
\pgfpathlineto{\pgfqpoint{4.607677in}{4.079322in}}%
\pgfpathlineto{\pgfqpoint{4.574667in}{4.112000in}}%
\pgfpathlineto{\pgfqpoint{4.571223in}{4.115378in}}%
\pgfpathlineto{\pgfqpoint{4.567596in}{4.118996in}}%
\pgfpathlineto{\pgfqpoint{4.551715in}{4.134541in}}%
\pgfpathlineto{\pgfqpoint{4.536860in}{4.149333in}}%
\pgfpathlineto{\pgfqpoint{4.527515in}{4.158551in}}%
\pgfpathlineto{\pgfqpoint{4.512775in}{4.172937in}}%
\pgfpathlineto{\pgfqpoint{4.498946in}{4.186667in}}%
\pgfpathlineto{\pgfqpoint{4.487434in}{4.197988in}}%
\pgfpathlineto{\pgfqpoint{4.473776in}{4.211278in}}%
\pgfpathlineto{\pgfqpoint{4.460925in}{4.224000in}}%
\pgfpathlineto{\pgfqpoint{4.458382in}{4.224000in}}%
\pgfpathlineto{\pgfqpoint{4.472466in}{4.210058in}}%
\pgfpathlineto{\pgfqpoint{4.487434in}{4.195494in}}%
\pgfpathlineto{\pgfqpoint{4.496410in}{4.186667in}}%
\pgfpathlineto{\pgfqpoint{4.511466in}{4.171718in}}%
\pgfpathlineto{\pgfqpoint{4.527515in}{4.156055in}}%
\pgfpathlineto{\pgfqpoint{4.534329in}{4.149333in}}%
\pgfpathlineto{\pgfqpoint{4.550407in}{4.133323in}}%
\pgfpathlineto{\pgfqpoint{4.567596in}{4.116498in}}%
\pgfpathlineto{\pgfqpoint{4.569928in}{4.114172in}}%
\pgfpathlineto{\pgfqpoint{4.572142in}{4.112000in}}%
\pgfpathlineto{\pgfqpoint{4.607677in}{4.076822in}}%
\pgfpathlineto{\pgfqpoint{4.608793in}{4.075706in}}%
\pgfpathlineto{\pgfqpoint{4.609849in}{4.074667in}}%
\pgfpathlineto{\pgfqpoint{4.642794in}{4.041956in}}%
\pgfpathlineto{\pgfqpoint{4.647448in}{4.037333in}}%
\pgfpathlineto{\pgfqpoint{4.647758in}{4.037026in}}%
\pgfpathlineto{\pgfqpoint{4.666881in}{4.017813in}}%
\pgfpathlineto{\pgfqpoint{4.684918in}{4.000000in}}%
\pgfpathlineto{\pgfqpoint{4.687838in}{3.997088in}}%
\pgfpathlineto{\pgfqpoint{4.705590in}{3.979201in}}%
\pgfpathlineto{\pgfqpoint{4.722283in}{3.962667in}}%
\pgfpathlineto{\pgfqpoint{4.727919in}{3.957030in}}%
\pgfpathlineto{\pgfqpoint{4.759544in}{3.925333in}}%
\pgfpathlineto{\pgfqpoint{4.763597in}{3.921232in}}%
\pgfpathlineto{\pgfqpoint{4.768000in}{3.916852in}}%
\pgfusepath{fill}%
\end{pgfscope}%
\begin{pgfscope}%
\pgfpathrectangle{\pgfqpoint{0.800000in}{0.528000in}}{\pgfqpoint{3.968000in}{3.696000in}}%
\pgfusepath{clip}%
\pgfsetbuttcap%
\pgfsetroundjoin%
\definecolor{currentfill}{rgb}{0.657642,0.860219,0.203082}%
\pgfsetfillcolor{currentfill}%
\pgfsetlinewidth{0.000000pt}%
\definecolor{currentstroke}{rgb}{0.000000,0.000000,0.000000}%
\pgfsetstrokecolor{currentstroke}%
\pgfsetdash{}{0pt}%
\pgfpathmoveto{\pgfqpoint{4.768000in}{3.921911in}}%
\pgfpathlineto{\pgfqpoint{4.766223in}{3.923678in}}%
\pgfpathlineto{\pgfqpoint{4.764588in}{3.925333in}}%
\pgfpathlineto{\pgfqpoint{4.727919in}{3.962086in}}%
\pgfpathlineto{\pgfqpoint{4.727339in}{3.962667in}}%
\pgfpathlineto{\pgfqpoint{4.708195in}{3.981628in}}%
\pgfpathlineto{\pgfqpoint{4.689963in}{4.000000in}}%
\pgfpathlineto{\pgfqpoint{4.688933in}{4.001019in}}%
\pgfpathlineto{\pgfqpoint{4.687838in}{4.002120in}}%
\pgfpathlineto{\pgfqpoint{4.669489in}{4.020242in}}%
\pgfpathlineto{\pgfqpoint{4.652477in}{4.037333in}}%
\pgfpathlineto{\pgfqpoint{4.650185in}{4.039594in}}%
\pgfpathlineto{\pgfqpoint{4.647758in}{4.042030in}}%
\pgfpathlineto{\pgfqpoint{4.614887in}{4.074667in}}%
\pgfpathlineto{\pgfqpoint{4.611380in}{4.078116in}}%
\pgfpathlineto{\pgfqpoint{4.607677in}{4.081821in}}%
\pgfpathlineto{\pgfqpoint{4.577192in}{4.112000in}}%
\pgfpathlineto{\pgfqpoint{4.572517in}{4.116584in}}%
\pgfpathlineto{\pgfqpoint{4.567596in}{4.121493in}}%
\pgfpathlineto{\pgfqpoint{4.553022in}{4.135759in}}%
\pgfpathlineto{\pgfqpoint{4.539390in}{4.149333in}}%
\pgfpathlineto{\pgfqpoint{4.527515in}{4.161047in}}%
\pgfpathlineto{\pgfqpoint{4.514083in}{4.174156in}}%
\pgfpathlineto{\pgfqpoint{4.501482in}{4.186667in}}%
\pgfpathlineto{\pgfqpoint{4.487434in}{4.200483in}}%
\pgfpathlineto{\pgfqpoint{4.475086in}{4.212498in}}%
\pgfpathlineto{\pgfqpoint{4.463467in}{4.224000in}}%
\pgfpathlineto{\pgfqpoint{4.460925in}{4.224000in}}%
\pgfpathlineto{\pgfqpoint{4.473776in}{4.211278in}}%
\pgfpathlineto{\pgfqpoint{4.487434in}{4.197988in}}%
\pgfpathlineto{\pgfqpoint{4.498946in}{4.186667in}}%
\pgfpathlineto{\pgfqpoint{4.512775in}{4.172937in}}%
\pgfpathlineto{\pgfqpoint{4.527515in}{4.158551in}}%
\pgfpathlineto{\pgfqpoint{4.536860in}{4.149333in}}%
\pgfpathlineto{\pgfqpoint{4.551715in}{4.134541in}}%
\pgfpathlineto{\pgfqpoint{4.567596in}{4.118996in}}%
\pgfpathlineto{\pgfqpoint{4.571223in}{4.115378in}}%
\pgfpathlineto{\pgfqpoint{4.574667in}{4.112000in}}%
\pgfpathlineto{\pgfqpoint{4.607677in}{4.079322in}}%
\pgfpathlineto{\pgfqpoint{4.610086in}{4.076911in}}%
\pgfpathlineto{\pgfqpoint{4.612368in}{4.074667in}}%
\pgfpathlineto{\pgfqpoint{4.647758in}{4.039529in}}%
\pgfpathlineto{\pgfqpoint{4.648893in}{4.038390in}}%
\pgfpathlineto{\pgfqpoint{4.649964in}{4.037333in}}%
\pgfpathlineto{\pgfqpoint{4.668185in}{4.019027in}}%
\pgfpathlineto{\pgfqpoint{4.687452in}{4.000000in}}%
\pgfpathlineto{\pgfqpoint{4.687838in}{3.999614in}}%
\pgfpathlineto{\pgfqpoint{4.706893in}{3.980415in}}%
\pgfpathlineto{\pgfqpoint{4.724811in}{3.962667in}}%
\pgfpathlineto{\pgfqpoint{4.727919in}{3.959558in}}%
\pgfpathlineto{\pgfqpoint{4.762066in}{3.925333in}}%
\pgfpathlineto{\pgfqpoint{4.764910in}{3.922455in}}%
\pgfpathlineto{\pgfqpoint{4.768000in}{3.919382in}}%
\pgfusepath{fill}%
\end{pgfscope}%
\begin{pgfscope}%
\pgfpathrectangle{\pgfqpoint{0.800000in}{0.528000in}}{\pgfqpoint{3.968000in}{3.696000in}}%
\pgfusepath{clip}%
\pgfsetbuttcap%
\pgfsetroundjoin%
\definecolor{currentfill}{rgb}{0.657642,0.860219,0.203082}%
\pgfsetfillcolor{currentfill}%
\pgfsetlinewidth{0.000000pt}%
\definecolor{currentstroke}{rgb}{0.000000,0.000000,0.000000}%
\pgfsetstrokecolor{currentstroke}%
\pgfsetdash{}{0pt}%
\pgfpathmoveto{\pgfqpoint{4.768000in}{3.924441in}}%
\pgfpathlineto{\pgfqpoint{4.767537in}{3.924902in}}%
\pgfpathlineto{\pgfqpoint{4.767110in}{3.925333in}}%
\pgfpathlineto{\pgfqpoint{4.755409in}{3.937061in}}%
\pgfpathlineto{\pgfqpoint{4.729846in}{3.962667in}}%
\pgfpathlineto{\pgfqpoint{4.727919in}{3.964596in}}%
\pgfpathlineto{\pgfqpoint{4.709498in}{3.982842in}}%
\pgfpathlineto{\pgfqpoint{4.692470in}{4.000000in}}%
\pgfpathlineto{\pgfqpoint{4.690224in}{4.002222in}}%
\pgfpathlineto{\pgfqpoint{4.687838in}{4.004623in}}%
\pgfpathlineto{\pgfqpoint{4.670793in}{4.021456in}}%
\pgfpathlineto{\pgfqpoint{4.654990in}{4.037333in}}%
\pgfpathlineto{\pgfqpoint{4.651478in}{4.040798in}}%
\pgfpathlineto{\pgfqpoint{4.647758in}{4.044531in}}%
\pgfpathlineto{\pgfqpoint{4.617406in}{4.074667in}}%
\pgfpathlineto{\pgfqpoint{4.612674in}{4.079321in}}%
\pgfpathlineto{\pgfqpoint{4.607677in}{4.084320in}}%
\pgfpathlineto{\pgfqpoint{4.579716in}{4.112000in}}%
\pgfpathlineto{\pgfqpoint{4.573812in}{4.117790in}}%
\pgfpathlineto{\pgfqpoint{4.567596in}{4.123991in}}%
\pgfpathlineto{\pgfqpoint{4.554330in}{4.136976in}}%
\pgfpathlineto{\pgfqpoint{4.541921in}{4.149333in}}%
\pgfpathlineto{\pgfqpoint{4.527515in}{4.163543in}}%
\pgfpathlineto{\pgfqpoint{4.515392in}{4.175375in}}%
\pgfpathlineto{\pgfqpoint{4.504019in}{4.186667in}}%
\pgfpathlineto{\pgfqpoint{4.487434in}{4.202978in}}%
\pgfpathlineto{\pgfqpoint{4.476396in}{4.213718in}}%
\pgfpathlineto{\pgfqpoint{4.466009in}{4.224000in}}%
\pgfpathlineto{\pgfqpoint{4.463467in}{4.224000in}}%
\pgfpathlineto{\pgfqpoint{4.475086in}{4.212498in}}%
\pgfpathlineto{\pgfqpoint{4.487434in}{4.200483in}}%
\pgfpathlineto{\pgfqpoint{4.501482in}{4.186667in}}%
\pgfpathlineto{\pgfqpoint{4.514083in}{4.174156in}}%
\pgfpathlineto{\pgfqpoint{4.527515in}{4.161047in}}%
\pgfpathlineto{\pgfqpoint{4.539390in}{4.149333in}}%
\pgfpathlineto{\pgfqpoint{4.553022in}{4.135759in}}%
\pgfpathlineto{\pgfqpoint{4.567596in}{4.121493in}}%
\pgfpathlineto{\pgfqpoint{4.572517in}{4.116584in}}%
\pgfpathlineto{\pgfqpoint{4.577192in}{4.112000in}}%
\pgfpathlineto{\pgfqpoint{4.607677in}{4.081821in}}%
\pgfpathlineto{\pgfqpoint{4.611380in}{4.078116in}}%
\pgfpathlineto{\pgfqpoint{4.614887in}{4.074667in}}%
\pgfpathlineto{\pgfqpoint{4.647758in}{4.042030in}}%
\pgfpathlineto{\pgfqpoint{4.650185in}{4.039594in}}%
\pgfpathlineto{\pgfqpoint{4.652477in}{4.037333in}}%
\pgfpathlineto{\pgfqpoint{4.669489in}{4.020242in}}%
\pgfpathlineto{\pgfqpoint{4.687838in}{4.002120in}}%
\pgfpathlineto{\pgfqpoint{4.688933in}{4.001019in}}%
\pgfpathlineto{\pgfqpoint{4.689963in}{4.000000in}}%
\pgfpathlineto{\pgfqpoint{4.708195in}{3.981628in}}%
\pgfpathlineto{\pgfqpoint{4.727339in}{3.962667in}}%
\pgfpathlineto{\pgfqpoint{4.727919in}{3.962086in}}%
\pgfpathlineto{\pgfqpoint{4.764588in}{3.925333in}}%
\pgfpathlineto{\pgfqpoint{4.766223in}{3.923678in}}%
\pgfpathlineto{\pgfqpoint{4.768000in}{3.921911in}}%
\pgfusepath{fill}%
\end{pgfscope}%
\begin{pgfscope}%
\pgfpathrectangle{\pgfqpoint{0.800000in}{0.528000in}}{\pgfqpoint{3.968000in}{3.696000in}}%
\pgfusepath{clip}%
\pgfsetbuttcap%
\pgfsetroundjoin%
\definecolor{currentfill}{rgb}{0.657642,0.860219,0.203082}%
\pgfsetfillcolor{currentfill}%
\pgfsetlinewidth{0.000000pt}%
\definecolor{currentstroke}{rgb}{0.000000,0.000000,0.000000}%
\pgfsetstrokecolor{currentstroke}%
\pgfsetdash{}{0pt}%
\pgfpathmoveto{\pgfqpoint{4.768000in}{3.926955in}}%
\pgfpathlineto{\pgfqpoint{4.732348in}{3.962667in}}%
\pgfpathlineto{\pgfqpoint{4.727919in}{3.967100in}}%
\pgfpathlineto{\pgfqpoint{4.710801in}{3.984055in}}%
\pgfpathlineto{\pgfqpoint{4.694977in}{4.000000in}}%
\pgfpathlineto{\pgfqpoint{4.691515in}{4.003425in}}%
\pgfpathlineto{\pgfqpoint{4.687838in}{4.007125in}}%
\pgfpathlineto{\pgfqpoint{4.672097in}{4.022671in}}%
\pgfpathlineto{\pgfqpoint{4.657503in}{4.037333in}}%
\pgfpathlineto{\pgfqpoint{4.652770in}{4.042002in}}%
\pgfpathlineto{\pgfqpoint{4.647758in}{4.047032in}}%
\pgfpathlineto{\pgfqpoint{4.619924in}{4.074667in}}%
\pgfpathlineto{\pgfqpoint{4.613967in}{4.080526in}}%
\pgfpathlineto{\pgfqpoint{4.607677in}{4.086820in}}%
\pgfpathlineto{\pgfqpoint{4.582241in}{4.112000in}}%
\pgfpathlineto{\pgfqpoint{4.575107in}{4.118996in}}%
\pgfpathlineto{\pgfqpoint{4.567596in}{4.126489in}}%
\pgfpathlineto{\pgfqpoint{4.555637in}{4.138194in}}%
\pgfpathlineto{\pgfqpoint{4.544451in}{4.149333in}}%
\pgfpathlineto{\pgfqpoint{4.527515in}{4.166040in}}%
\pgfpathlineto{\pgfqpoint{4.516701in}{4.176594in}}%
\pgfpathlineto{\pgfqpoint{4.506555in}{4.186667in}}%
\pgfpathlineto{\pgfqpoint{4.487434in}{4.205472in}}%
\pgfpathlineto{\pgfqpoint{4.477706in}{4.214938in}}%
\pgfpathlineto{\pgfqpoint{4.468552in}{4.224000in}}%
\pgfpathlineto{\pgfqpoint{4.466009in}{4.224000in}}%
\pgfpathlineto{\pgfqpoint{4.476396in}{4.213718in}}%
\pgfpathlineto{\pgfqpoint{4.487434in}{4.202978in}}%
\pgfpathlineto{\pgfqpoint{4.504019in}{4.186667in}}%
\pgfpathlineto{\pgfqpoint{4.515392in}{4.175375in}}%
\pgfpathlineto{\pgfqpoint{4.527515in}{4.163543in}}%
\pgfpathlineto{\pgfqpoint{4.541921in}{4.149333in}}%
\pgfpathlineto{\pgfqpoint{4.554330in}{4.136976in}}%
\pgfpathlineto{\pgfqpoint{4.567596in}{4.123991in}}%
\pgfpathlineto{\pgfqpoint{4.573812in}{4.117790in}}%
\pgfpathlineto{\pgfqpoint{4.579716in}{4.112000in}}%
\pgfpathlineto{\pgfqpoint{4.607677in}{4.084320in}}%
\pgfpathlineto{\pgfqpoint{4.612674in}{4.079321in}}%
\pgfpathlineto{\pgfqpoint{4.617406in}{4.074667in}}%
\pgfpathlineto{\pgfqpoint{4.647758in}{4.044531in}}%
\pgfpathlineto{\pgfqpoint{4.651478in}{4.040798in}}%
\pgfpathlineto{\pgfqpoint{4.654990in}{4.037333in}}%
\pgfpathlineto{\pgfqpoint{4.670793in}{4.021456in}}%
\pgfpathlineto{\pgfqpoint{4.687838in}{4.004623in}}%
\pgfpathlineto{\pgfqpoint{4.690224in}{4.002222in}}%
\pgfpathlineto{\pgfqpoint{4.692470in}{4.000000in}}%
\pgfpathlineto{\pgfqpoint{4.709498in}{3.982842in}}%
\pgfpathlineto{\pgfqpoint{4.727919in}{3.964596in}}%
\pgfpathlineto{\pgfqpoint{4.729846in}{3.962667in}}%
\pgfpathlineto{\pgfqpoint{4.755409in}{3.937061in}}%
\pgfpathlineto{\pgfqpoint{4.767110in}{3.925333in}}%
\pgfpathlineto{\pgfqpoint{4.767537in}{3.924902in}}%
\pgfpathlineto{\pgfqpoint{4.768000in}{3.924441in}}%
\pgfpathlineto{\pgfqpoint{4.768000in}{3.925333in}}%
\pgfusepath{fill}%
\end{pgfscope}%
\begin{pgfscope}%
\pgfpathrectangle{\pgfqpoint{0.800000in}{0.528000in}}{\pgfqpoint{3.968000in}{3.696000in}}%
\pgfusepath{clip}%
\pgfsetbuttcap%
\pgfsetroundjoin%
\definecolor{currentfill}{rgb}{0.668054,0.861999,0.196293}%
\pgfsetfillcolor{currentfill}%
\pgfsetlinewidth{0.000000pt}%
\definecolor{currentstroke}{rgb}{0.000000,0.000000,0.000000}%
\pgfsetstrokecolor{currentstroke}%
\pgfsetdash{}{0pt}%
\pgfpathmoveto{\pgfqpoint{4.768000in}{3.929460in}}%
\pgfpathlineto{\pgfqpoint{4.734849in}{3.962667in}}%
\pgfpathlineto{\pgfqpoint{4.727919in}{3.969604in}}%
\pgfpathlineto{\pgfqpoint{4.712104in}{3.985269in}}%
\pgfpathlineto{\pgfqpoint{4.697484in}{4.000000in}}%
\pgfpathlineto{\pgfqpoint{4.692807in}{4.004628in}}%
\pgfpathlineto{\pgfqpoint{4.687838in}{4.009628in}}%
\pgfpathlineto{\pgfqpoint{4.673401in}{4.023886in}}%
\pgfpathlineto{\pgfqpoint{4.660016in}{4.037333in}}%
\pgfpathlineto{\pgfqpoint{4.654063in}{4.043206in}}%
\pgfpathlineto{\pgfqpoint{4.647758in}{4.049533in}}%
\pgfpathlineto{\pgfqpoint{4.622443in}{4.074667in}}%
\pgfpathlineto{\pgfqpoint{4.615261in}{4.081731in}}%
\pgfpathlineto{\pgfqpoint{4.607677in}{4.089319in}}%
\pgfpathlineto{\pgfqpoint{4.584765in}{4.112000in}}%
\pgfpathlineto{\pgfqpoint{4.576402in}{4.120202in}}%
\pgfpathlineto{\pgfqpoint{4.567596in}{4.128986in}}%
\pgfpathlineto{\pgfqpoint{4.556945in}{4.139412in}}%
\pgfpathlineto{\pgfqpoint{4.546982in}{4.149333in}}%
\pgfpathlineto{\pgfqpoint{4.527515in}{4.168536in}}%
\pgfpathlineto{\pgfqpoint{4.518009in}{4.177813in}}%
\pgfpathlineto{\pgfqpoint{4.509092in}{4.186667in}}%
\pgfpathlineto{\pgfqpoint{4.487434in}{4.207967in}}%
\pgfpathlineto{\pgfqpoint{4.479016in}{4.216158in}}%
\pgfpathlineto{\pgfqpoint{4.471094in}{4.224000in}}%
\pgfpathlineto{\pgfqpoint{4.468552in}{4.224000in}}%
\pgfpathlineto{\pgfqpoint{4.477706in}{4.214938in}}%
\pgfpathlineto{\pgfqpoint{4.487434in}{4.205472in}}%
\pgfpathlineto{\pgfqpoint{4.506555in}{4.186667in}}%
\pgfpathlineto{\pgfqpoint{4.516701in}{4.176594in}}%
\pgfpathlineto{\pgfqpoint{4.527515in}{4.166040in}}%
\pgfpathlineto{\pgfqpoint{4.544451in}{4.149333in}}%
\pgfpathlineto{\pgfqpoint{4.555637in}{4.138194in}}%
\pgfpathlineto{\pgfqpoint{4.567596in}{4.126489in}}%
\pgfpathlineto{\pgfqpoint{4.575107in}{4.118996in}}%
\pgfpathlineto{\pgfqpoint{4.582241in}{4.112000in}}%
\pgfpathlineto{\pgfqpoint{4.607677in}{4.086820in}}%
\pgfpathlineto{\pgfqpoint{4.613967in}{4.080526in}}%
\pgfpathlineto{\pgfqpoint{4.619924in}{4.074667in}}%
\pgfpathlineto{\pgfqpoint{4.647758in}{4.047032in}}%
\pgfpathlineto{\pgfqpoint{4.652770in}{4.042002in}}%
\pgfpathlineto{\pgfqpoint{4.657503in}{4.037333in}}%
\pgfpathlineto{\pgfqpoint{4.672097in}{4.022671in}}%
\pgfpathlineto{\pgfqpoint{4.687838in}{4.007125in}}%
\pgfpathlineto{\pgfqpoint{4.691515in}{4.003425in}}%
\pgfpathlineto{\pgfqpoint{4.694977in}{4.000000in}}%
\pgfpathlineto{\pgfqpoint{4.710801in}{3.984055in}}%
\pgfpathlineto{\pgfqpoint{4.727919in}{3.967100in}}%
\pgfpathlineto{\pgfqpoint{4.732348in}{3.962667in}}%
\pgfpathlineto{\pgfqpoint{4.768000in}{3.926955in}}%
\pgfusepath{fill}%
\end{pgfscope}%
\begin{pgfscope}%
\pgfpathrectangle{\pgfqpoint{0.800000in}{0.528000in}}{\pgfqpoint{3.968000in}{3.696000in}}%
\pgfusepath{clip}%
\pgfsetbuttcap%
\pgfsetroundjoin%
\definecolor{currentfill}{rgb}{0.668054,0.861999,0.196293}%
\pgfsetfillcolor{currentfill}%
\pgfsetlinewidth{0.000000pt}%
\definecolor{currentstroke}{rgb}{0.000000,0.000000,0.000000}%
\pgfsetstrokecolor{currentstroke}%
\pgfsetdash{}{0pt}%
\pgfpathmoveto{\pgfqpoint{4.768000in}{3.931966in}}%
\pgfpathlineto{\pgfqpoint{4.737350in}{3.962667in}}%
\pgfpathlineto{\pgfqpoint{4.727919in}{3.972108in}}%
\pgfpathlineto{\pgfqpoint{4.713407in}{3.986482in}}%
\pgfpathlineto{\pgfqpoint{4.699991in}{4.000000in}}%
\pgfpathlineto{\pgfqpoint{4.694098in}{4.005831in}}%
\pgfpathlineto{\pgfqpoint{4.687838in}{4.012130in}}%
\pgfpathlineto{\pgfqpoint{4.674705in}{4.025100in}}%
\pgfpathlineto{\pgfqpoint{4.662529in}{4.037333in}}%
\pgfpathlineto{\pgfqpoint{4.655355in}{4.044410in}}%
\pgfpathlineto{\pgfqpoint{4.647758in}{4.052033in}}%
\pgfpathlineto{\pgfqpoint{4.624962in}{4.074667in}}%
\pgfpathlineto{\pgfqpoint{4.616555in}{4.082936in}}%
\pgfpathlineto{\pgfqpoint{4.607677in}{4.091818in}}%
\pgfpathlineto{\pgfqpoint{4.587290in}{4.112000in}}%
\pgfpathlineto{\pgfqpoint{4.577697in}{4.121408in}}%
\pgfpathlineto{\pgfqpoint{4.567596in}{4.131484in}}%
\pgfpathlineto{\pgfqpoint{4.558252in}{4.140630in}}%
\pgfpathlineto{\pgfqpoint{4.549512in}{4.149333in}}%
\pgfpathlineto{\pgfqpoint{4.527515in}{4.171032in}}%
\pgfpathlineto{\pgfqpoint{4.519318in}{4.179032in}}%
\pgfpathlineto{\pgfqpoint{4.511628in}{4.186667in}}%
\pgfpathlineto{\pgfqpoint{4.487434in}{4.210461in}}%
\pgfpathlineto{\pgfqpoint{4.480325in}{4.217378in}}%
\pgfpathlineto{\pgfqpoint{4.473637in}{4.224000in}}%
\pgfpathlineto{\pgfqpoint{4.471094in}{4.224000in}}%
\pgfpathlineto{\pgfqpoint{4.479016in}{4.216158in}}%
\pgfpathlineto{\pgfqpoint{4.487434in}{4.207967in}}%
\pgfpathlineto{\pgfqpoint{4.509092in}{4.186667in}}%
\pgfpathlineto{\pgfqpoint{4.518009in}{4.177813in}}%
\pgfpathlineto{\pgfqpoint{4.527515in}{4.168536in}}%
\pgfpathlineto{\pgfqpoint{4.546982in}{4.149333in}}%
\pgfpathlineto{\pgfqpoint{4.556945in}{4.139412in}}%
\pgfpathlineto{\pgfqpoint{4.567596in}{4.128986in}}%
\pgfpathlineto{\pgfqpoint{4.576402in}{4.120202in}}%
\pgfpathlineto{\pgfqpoint{4.584765in}{4.112000in}}%
\pgfpathlineto{\pgfqpoint{4.607677in}{4.089319in}}%
\pgfpathlineto{\pgfqpoint{4.615261in}{4.081731in}}%
\pgfpathlineto{\pgfqpoint{4.622443in}{4.074667in}}%
\pgfpathlineto{\pgfqpoint{4.647758in}{4.049533in}}%
\pgfpathlineto{\pgfqpoint{4.654063in}{4.043206in}}%
\pgfpathlineto{\pgfqpoint{4.660016in}{4.037333in}}%
\pgfpathlineto{\pgfqpoint{4.673401in}{4.023886in}}%
\pgfpathlineto{\pgfqpoint{4.687838in}{4.009628in}}%
\pgfpathlineto{\pgfqpoint{4.692807in}{4.004628in}}%
\pgfpathlineto{\pgfqpoint{4.697484in}{4.000000in}}%
\pgfpathlineto{\pgfqpoint{4.712104in}{3.985269in}}%
\pgfpathlineto{\pgfqpoint{4.727919in}{3.969604in}}%
\pgfpathlineto{\pgfqpoint{4.734849in}{3.962667in}}%
\pgfpathlineto{\pgfqpoint{4.768000in}{3.929460in}}%
\pgfusepath{fill}%
\end{pgfscope}%
\begin{pgfscope}%
\pgfpathrectangle{\pgfqpoint{0.800000in}{0.528000in}}{\pgfqpoint{3.968000in}{3.696000in}}%
\pgfusepath{clip}%
\pgfsetbuttcap%
\pgfsetroundjoin%
\definecolor{currentfill}{rgb}{0.668054,0.861999,0.196293}%
\pgfsetfillcolor{currentfill}%
\pgfsetlinewidth{0.000000pt}%
\definecolor{currentstroke}{rgb}{0.000000,0.000000,0.000000}%
\pgfsetstrokecolor{currentstroke}%
\pgfsetdash{}{0pt}%
\pgfpathmoveto{\pgfqpoint{4.768000in}{3.934471in}}%
\pgfpathlineto{\pgfqpoint{4.739852in}{3.962667in}}%
\pgfpathlineto{\pgfqpoint{4.727919in}{3.974612in}}%
\pgfpathlineto{\pgfqpoint{4.714709in}{3.987696in}}%
\pgfpathlineto{\pgfqpoint{4.702499in}{4.000000in}}%
\pgfpathlineto{\pgfqpoint{4.695389in}{4.007033in}}%
\pgfpathlineto{\pgfqpoint{4.687838in}{4.014632in}}%
\pgfpathlineto{\pgfqpoint{4.676009in}{4.026315in}}%
\pgfpathlineto{\pgfqpoint{4.665042in}{4.037333in}}%
\pgfpathlineto{\pgfqpoint{4.656648in}{4.045614in}}%
\pgfpathlineto{\pgfqpoint{4.647758in}{4.054534in}}%
\pgfpathlineto{\pgfqpoint{4.627481in}{4.074667in}}%
\pgfpathlineto{\pgfqpoint{4.617848in}{4.084141in}}%
\pgfpathlineto{\pgfqpoint{4.607677in}{4.094317in}}%
\pgfpathlineto{\pgfqpoint{4.589815in}{4.112000in}}%
\pgfpathlineto{\pgfqpoint{4.578991in}{4.122614in}}%
\pgfpathlineto{\pgfqpoint{4.567596in}{4.133982in}}%
\pgfpathlineto{\pgfqpoint{4.559560in}{4.141848in}}%
\pgfpathlineto{\pgfqpoint{4.552043in}{4.149333in}}%
\pgfpathlineto{\pgfqpoint{4.527515in}{4.173528in}}%
\pgfpathlineto{\pgfqpoint{4.520627in}{4.180250in}}%
\pgfpathlineto{\pgfqpoint{4.514165in}{4.186667in}}%
\pgfpathlineto{\pgfqpoint{4.487434in}{4.212956in}}%
\pgfpathlineto{\pgfqpoint{4.481635in}{4.218599in}}%
\pgfpathlineto{\pgfqpoint{4.476179in}{4.224000in}}%
\pgfpathlineto{\pgfqpoint{4.473637in}{4.224000in}}%
\pgfpathlineto{\pgfqpoint{4.480325in}{4.217378in}}%
\pgfpathlineto{\pgfqpoint{4.487434in}{4.210461in}}%
\pgfpathlineto{\pgfqpoint{4.511628in}{4.186667in}}%
\pgfpathlineto{\pgfqpoint{4.519318in}{4.179032in}}%
\pgfpathlineto{\pgfqpoint{4.527515in}{4.171032in}}%
\pgfpathlineto{\pgfqpoint{4.549512in}{4.149333in}}%
\pgfpathlineto{\pgfqpoint{4.558252in}{4.140630in}}%
\pgfpathlineto{\pgfqpoint{4.567596in}{4.131484in}}%
\pgfpathlineto{\pgfqpoint{4.577697in}{4.121408in}}%
\pgfpathlineto{\pgfqpoint{4.587290in}{4.112000in}}%
\pgfpathlineto{\pgfqpoint{4.607677in}{4.091818in}}%
\pgfpathlineto{\pgfqpoint{4.616555in}{4.082936in}}%
\pgfpathlineto{\pgfqpoint{4.624962in}{4.074667in}}%
\pgfpathlineto{\pgfqpoint{4.647758in}{4.052033in}}%
\pgfpathlineto{\pgfqpoint{4.655355in}{4.044410in}}%
\pgfpathlineto{\pgfqpoint{4.662529in}{4.037333in}}%
\pgfpathlineto{\pgfqpoint{4.674705in}{4.025100in}}%
\pgfpathlineto{\pgfqpoint{4.687838in}{4.012130in}}%
\pgfpathlineto{\pgfqpoint{4.694098in}{4.005831in}}%
\pgfpathlineto{\pgfqpoint{4.699991in}{4.000000in}}%
\pgfpathlineto{\pgfqpoint{4.713407in}{3.986482in}}%
\pgfpathlineto{\pgfqpoint{4.727919in}{3.972108in}}%
\pgfpathlineto{\pgfqpoint{4.737350in}{3.962667in}}%
\pgfpathlineto{\pgfqpoint{4.768000in}{3.931966in}}%
\pgfusepath{fill}%
\end{pgfscope}%
\begin{pgfscope}%
\pgfpathrectangle{\pgfqpoint{0.800000in}{0.528000in}}{\pgfqpoint{3.968000in}{3.696000in}}%
\pgfusepath{clip}%
\pgfsetbuttcap%
\pgfsetroundjoin%
\definecolor{currentfill}{rgb}{0.668054,0.861999,0.196293}%
\pgfsetfillcolor{currentfill}%
\pgfsetlinewidth{0.000000pt}%
\definecolor{currentstroke}{rgb}{0.000000,0.000000,0.000000}%
\pgfsetstrokecolor{currentstroke}%
\pgfsetdash{}{0pt}%
\pgfpathmoveto{\pgfqpoint{4.768000in}{3.936977in}}%
\pgfpathlineto{\pgfqpoint{4.742353in}{3.962667in}}%
\pgfpathlineto{\pgfqpoint{4.727919in}{3.977116in}}%
\pgfpathlineto{\pgfqpoint{4.716012in}{3.988909in}}%
\pgfpathlineto{\pgfqpoint{4.705006in}{4.000000in}}%
\pgfpathlineto{\pgfqpoint{4.696681in}{4.008236in}}%
\pgfpathlineto{\pgfqpoint{4.687838in}{4.017135in}}%
\pgfpathlineto{\pgfqpoint{4.677313in}{4.027529in}}%
\pgfpathlineto{\pgfqpoint{4.667555in}{4.037333in}}%
\pgfpathlineto{\pgfqpoint{4.657940in}{4.046818in}}%
\pgfpathlineto{\pgfqpoint{4.647758in}{4.057035in}}%
\pgfpathlineto{\pgfqpoint{4.630000in}{4.074667in}}%
\pgfpathlineto{\pgfqpoint{4.619142in}{4.085346in}}%
\pgfpathlineto{\pgfqpoint{4.607677in}{4.096817in}}%
\pgfpathlineto{\pgfqpoint{4.592339in}{4.112000in}}%
\pgfpathlineto{\pgfqpoint{4.580286in}{4.123820in}}%
\pgfpathlineto{\pgfqpoint{4.567596in}{4.136480in}}%
\pgfpathlineto{\pgfqpoint{4.560867in}{4.143066in}}%
\pgfpathlineto{\pgfqpoint{4.554573in}{4.149333in}}%
\pgfpathlineto{\pgfqpoint{4.527515in}{4.176024in}}%
\pgfpathlineto{\pgfqpoint{4.521935in}{4.181469in}}%
\pgfpathlineto{\pgfqpoint{4.516701in}{4.186667in}}%
\pgfpathlineto{\pgfqpoint{4.487434in}{4.215451in}}%
\pgfpathlineto{\pgfqpoint{4.482945in}{4.219819in}}%
\pgfpathlineto{\pgfqpoint{4.478721in}{4.224000in}}%
\pgfpathlineto{\pgfqpoint{4.476179in}{4.224000in}}%
\pgfpathlineto{\pgfqpoint{4.481635in}{4.218599in}}%
\pgfpathlineto{\pgfqpoint{4.487434in}{4.212956in}}%
\pgfpathlineto{\pgfqpoint{4.514165in}{4.186667in}}%
\pgfpathlineto{\pgfqpoint{4.520627in}{4.180250in}}%
\pgfpathlineto{\pgfqpoint{4.527515in}{4.173528in}}%
\pgfpathlineto{\pgfqpoint{4.552043in}{4.149333in}}%
\pgfpathlineto{\pgfqpoint{4.559560in}{4.141848in}}%
\pgfpathlineto{\pgfqpoint{4.567596in}{4.133982in}}%
\pgfpathlineto{\pgfqpoint{4.578991in}{4.122614in}}%
\pgfpathlineto{\pgfqpoint{4.589815in}{4.112000in}}%
\pgfpathlineto{\pgfqpoint{4.607677in}{4.094317in}}%
\pgfpathlineto{\pgfqpoint{4.617848in}{4.084141in}}%
\pgfpathlineto{\pgfqpoint{4.627481in}{4.074667in}}%
\pgfpathlineto{\pgfqpoint{4.647758in}{4.054534in}}%
\pgfpathlineto{\pgfqpoint{4.656648in}{4.045614in}}%
\pgfpathlineto{\pgfqpoint{4.665042in}{4.037333in}}%
\pgfpathlineto{\pgfqpoint{4.676009in}{4.026315in}}%
\pgfpathlineto{\pgfqpoint{4.687838in}{4.014632in}}%
\pgfpathlineto{\pgfqpoint{4.695389in}{4.007033in}}%
\pgfpathlineto{\pgfqpoint{4.702499in}{4.000000in}}%
\pgfpathlineto{\pgfqpoint{4.714709in}{3.987696in}}%
\pgfpathlineto{\pgfqpoint{4.727919in}{3.974612in}}%
\pgfpathlineto{\pgfqpoint{4.739852in}{3.962667in}}%
\pgfpathlineto{\pgfqpoint{4.768000in}{3.934471in}}%
\pgfusepath{fill}%
\end{pgfscope}%
\begin{pgfscope}%
\pgfpathrectangle{\pgfqpoint{0.800000in}{0.528000in}}{\pgfqpoint{3.968000in}{3.696000in}}%
\pgfusepath{clip}%
\pgfsetbuttcap%
\pgfsetroundjoin%
\definecolor{currentfill}{rgb}{0.678489,0.863742,0.189503}%
\pgfsetfillcolor{currentfill}%
\pgfsetlinewidth{0.000000pt}%
\definecolor{currentstroke}{rgb}{0.000000,0.000000,0.000000}%
\pgfsetstrokecolor{currentstroke}%
\pgfsetdash{}{0pt}%
\pgfpathmoveto{\pgfqpoint{4.768000in}{3.939482in}}%
\pgfpathlineto{\pgfqpoint{4.744854in}{3.962667in}}%
\pgfpathlineto{\pgfqpoint{4.727919in}{3.979619in}}%
\pgfpathlineto{\pgfqpoint{4.717315in}{3.990123in}}%
\pgfpathlineto{\pgfqpoint{4.707513in}{4.000000in}}%
\pgfpathlineto{\pgfqpoint{4.697972in}{4.009439in}}%
\pgfpathlineto{\pgfqpoint{4.687838in}{4.019637in}}%
\pgfpathlineto{\pgfqpoint{4.678617in}{4.028744in}}%
\pgfpathlineto{\pgfqpoint{4.670068in}{4.037333in}}%
\pgfpathlineto{\pgfqpoint{4.659233in}{4.048022in}}%
\pgfpathlineto{\pgfqpoint{4.647758in}{4.059536in}}%
\pgfpathlineto{\pgfqpoint{4.632518in}{4.074667in}}%
\pgfpathlineto{\pgfqpoint{4.620436in}{4.086551in}}%
\pgfpathlineto{\pgfqpoint{4.607677in}{4.099316in}}%
\pgfpathlineto{\pgfqpoint{4.594864in}{4.112000in}}%
\pgfpathlineto{\pgfqpoint{4.581581in}{4.125026in}}%
\pgfpathlineto{\pgfqpoint{4.567596in}{4.138977in}}%
\pgfpathlineto{\pgfqpoint{4.562175in}{4.144284in}}%
\pgfpathlineto{\pgfqpoint{4.557104in}{4.149333in}}%
\pgfpathlineto{\pgfqpoint{4.527515in}{4.178520in}}%
\pgfpathlineto{\pgfqpoint{4.523244in}{4.182688in}}%
\pgfpathlineto{\pgfqpoint{4.519237in}{4.186667in}}%
\pgfpathlineto{\pgfqpoint{4.487434in}{4.217945in}}%
\pgfpathlineto{\pgfqpoint{4.484255in}{4.221039in}}%
\pgfpathlineto{\pgfqpoint{4.481264in}{4.224000in}}%
\pgfpathlineto{\pgfqpoint{4.478721in}{4.224000in}}%
\pgfpathlineto{\pgfqpoint{4.482945in}{4.219819in}}%
\pgfpathlineto{\pgfqpoint{4.487434in}{4.215451in}}%
\pgfpathlineto{\pgfqpoint{4.516701in}{4.186667in}}%
\pgfpathlineto{\pgfqpoint{4.521935in}{4.181469in}}%
\pgfpathlineto{\pgfqpoint{4.527515in}{4.176024in}}%
\pgfpathlineto{\pgfqpoint{4.554573in}{4.149333in}}%
\pgfpathlineto{\pgfqpoint{4.560867in}{4.143066in}}%
\pgfpathlineto{\pgfqpoint{4.567596in}{4.136480in}}%
\pgfpathlineto{\pgfqpoint{4.580286in}{4.123820in}}%
\pgfpathlineto{\pgfqpoint{4.592339in}{4.112000in}}%
\pgfpathlineto{\pgfqpoint{4.607677in}{4.096817in}}%
\pgfpathlineto{\pgfqpoint{4.619142in}{4.085346in}}%
\pgfpathlineto{\pgfqpoint{4.630000in}{4.074667in}}%
\pgfpathlineto{\pgfqpoint{4.647758in}{4.057035in}}%
\pgfpathlineto{\pgfqpoint{4.657940in}{4.046818in}}%
\pgfpathlineto{\pgfqpoint{4.667555in}{4.037333in}}%
\pgfpathlineto{\pgfqpoint{4.677313in}{4.027529in}}%
\pgfpathlineto{\pgfqpoint{4.687838in}{4.017135in}}%
\pgfpathlineto{\pgfqpoint{4.696681in}{4.008236in}}%
\pgfpathlineto{\pgfqpoint{4.705006in}{4.000000in}}%
\pgfpathlineto{\pgfqpoint{4.716012in}{3.988909in}}%
\pgfpathlineto{\pgfqpoint{4.727919in}{3.977116in}}%
\pgfpathlineto{\pgfqpoint{4.742353in}{3.962667in}}%
\pgfpathlineto{\pgfqpoint{4.768000in}{3.936977in}}%
\pgfusepath{fill}%
\end{pgfscope}%
\begin{pgfscope}%
\pgfpathrectangle{\pgfqpoint{0.800000in}{0.528000in}}{\pgfqpoint{3.968000in}{3.696000in}}%
\pgfusepath{clip}%
\pgfsetbuttcap%
\pgfsetroundjoin%
\definecolor{currentfill}{rgb}{0.678489,0.863742,0.189503}%
\pgfsetfillcolor{currentfill}%
\pgfsetlinewidth{0.000000pt}%
\definecolor{currentstroke}{rgb}{0.000000,0.000000,0.000000}%
\pgfsetstrokecolor{currentstroke}%
\pgfsetdash{}{0pt}%
\pgfpathmoveto{\pgfqpoint{4.768000in}{3.941988in}}%
\pgfpathlineto{\pgfqpoint{4.747356in}{3.962667in}}%
\pgfpathlineto{\pgfqpoint{4.727919in}{3.982123in}}%
\pgfpathlineto{\pgfqpoint{4.718618in}{3.991336in}}%
\pgfpathlineto{\pgfqpoint{4.710020in}{4.000000in}}%
\pgfpathlineto{\pgfqpoint{4.699264in}{4.010642in}}%
\pgfpathlineto{\pgfqpoint{4.687838in}{4.022140in}}%
\pgfpathlineto{\pgfqpoint{4.679921in}{4.029959in}}%
\pgfpathlineto{\pgfqpoint{4.672581in}{4.037333in}}%
\pgfpathlineto{\pgfqpoint{4.660525in}{4.049226in}}%
\pgfpathlineto{\pgfqpoint{4.647758in}{4.062037in}}%
\pgfpathlineto{\pgfqpoint{4.635037in}{4.074667in}}%
\pgfpathlineto{\pgfqpoint{4.621729in}{4.087756in}}%
\pgfpathlineto{\pgfqpoint{4.607677in}{4.101815in}}%
\pgfpathlineto{\pgfqpoint{4.597389in}{4.112000in}}%
\pgfpathlineto{\pgfqpoint{4.582876in}{4.126233in}}%
\pgfpathlineto{\pgfqpoint{4.567596in}{4.141475in}}%
\pgfpathlineto{\pgfqpoint{4.563482in}{4.145502in}}%
\pgfpathlineto{\pgfqpoint{4.559634in}{4.149333in}}%
\pgfpathlineto{\pgfqpoint{4.527515in}{4.181016in}}%
\pgfpathlineto{\pgfqpoint{4.524553in}{4.183907in}}%
\pgfpathlineto{\pgfqpoint{4.521774in}{4.186667in}}%
\pgfpathlineto{\pgfqpoint{4.487434in}{4.220440in}}%
\pgfpathlineto{\pgfqpoint{4.485565in}{4.222259in}}%
\pgfpathlineto{\pgfqpoint{4.483806in}{4.224000in}}%
\pgfpathlineto{\pgfqpoint{4.481264in}{4.224000in}}%
\pgfpathlineto{\pgfqpoint{4.484255in}{4.221039in}}%
\pgfpathlineto{\pgfqpoint{4.487434in}{4.217945in}}%
\pgfpathlineto{\pgfqpoint{4.519237in}{4.186667in}}%
\pgfpathlineto{\pgfqpoint{4.523244in}{4.182688in}}%
\pgfpathlineto{\pgfqpoint{4.527515in}{4.178520in}}%
\pgfpathlineto{\pgfqpoint{4.557104in}{4.149333in}}%
\pgfpathlineto{\pgfqpoint{4.562175in}{4.144284in}}%
\pgfpathlineto{\pgfqpoint{4.567596in}{4.138977in}}%
\pgfpathlineto{\pgfqpoint{4.581581in}{4.125026in}}%
\pgfpathlineto{\pgfqpoint{4.594864in}{4.112000in}}%
\pgfpathlineto{\pgfqpoint{4.607677in}{4.099316in}}%
\pgfpathlineto{\pgfqpoint{4.620436in}{4.086551in}}%
\pgfpathlineto{\pgfqpoint{4.632518in}{4.074667in}}%
\pgfpathlineto{\pgfqpoint{4.647758in}{4.059536in}}%
\pgfpathlineto{\pgfqpoint{4.659233in}{4.048022in}}%
\pgfpathlineto{\pgfqpoint{4.670068in}{4.037333in}}%
\pgfpathlineto{\pgfqpoint{4.678617in}{4.028744in}}%
\pgfpathlineto{\pgfqpoint{4.687838in}{4.019637in}}%
\pgfpathlineto{\pgfqpoint{4.697972in}{4.009439in}}%
\pgfpathlineto{\pgfqpoint{4.707513in}{4.000000in}}%
\pgfpathlineto{\pgfqpoint{4.717315in}{3.990123in}}%
\pgfpathlineto{\pgfqpoint{4.727919in}{3.979619in}}%
\pgfpathlineto{\pgfqpoint{4.744854in}{3.962667in}}%
\pgfpathlineto{\pgfqpoint{4.768000in}{3.939482in}}%
\pgfusepath{fill}%
\end{pgfscope}%
\begin{pgfscope}%
\pgfpathrectangle{\pgfqpoint{0.800000in}{0.528000in}}{\pgfqpoint{3.968000in}{3.696000in}}%
\pgfusepath{clip}%
\pgfsetbuttcap%
\pgfsetroundjoin%
\definecolor{currentfill}{rgb}{0.678489,0.863742,0.189503}%
\pgfsetfillcolor{currentfill}%
\pgfsetlinewidth{0.000000pt}%
\definecolor{currentstroke}{rgb}{0.000000,0.000000,0.000000}%
\pgfsetstrokecolor{currentstroke}%
\pgfsetdash{}{0pt}%
\pgfpathmoveto{\pgfqpoint{4.768000in}{3.944494in}}%
\pgfpathlineto{\pgfqpoint{4.749857in}{3.962667in}}%
\pgfpathlineto{\pgfqpoint{4.727919in}{3.984627in}}%
\pgfpathlineto{\pgfqpoint{4.719921in}{3.992550in}}%
\pgfpathlineto{\pgfqpoint{4.712527in}{4.000000in}}%
\pgfpathlineto{\pgfqpoint{4.700555in}{4.011845in}}%
\pgfpathlineto{\pgfqpoint{4.687838in}{4.024642in}}%
\pgfpathlineto{\pgfqpoint{4.681225in}{4.031173in}}%
\pgfpathlineto{\pgfqpoint{4.675094in}{4.037333in}}%
\pgfpathlineto{\pgfqpoint{4.661818in}{4.050430in}}%
\pgfpathlineto{\pgfqpoint{4.647758in}{4.064538in}}%
\pgfpathlineto{\pgfqpoint{4.637556in}{4.074667in}}%
\pgfpathlineto{\pgfqpoint{4.623023in}{4.088961in}}%
\pgfpathlineto{\pgfqpoint{4.607677in}{4.104314in}}%
\pgfpathlineto{\pgfqpoint{4.599913in}{4.112000in}}%
\pgfpathlineto{\pgfqpoint{4.584171in}{4.127439in}}%
\pgfpathlineto{\pgfqpoint{4.567596in}{4.143973in}}%
\pgfpathlineto{\pgfqpoint{4.564790in}{4.146719in}}%
\pgfpathlineto{\pgfqpoint{4.562165in}{4.149333in}}%
\pgfpathlineto{\pgfqpoint{4.527515in}{4.183513in}}%
\pgfpathlineto{\pgfqpoint{4.525862in}{4.185126in}}%
\pgfpathlineto{\pgfqpoint{4.524310in}{4.186667in}}%
\pgfpathlineto{\pgfqpoint{4.487434in}{4.222934in}}%
\pgfpathlineto{\pgfqpoint{4.486875in}{4.223479in}}%
\pgfpathlineto{\pgfqpoint{4.486348in}{4.224000in}}%
\pgfpathlineto{\pgfqpoint{4.483806in}{4.224000in}}%
\pgfpathlineto{\pgfqpoint{4.485565in}{4.222259in}}%
\pgfpathlineto{\pgfqpoint{4.487434in}{4.220440in}}%
\pgfpathlineto{\pgfqpoint{4.521774in}{4.186667in}}%
\pgfpathlineto{\pgfqpoint{4.524553in}{4.183907in}}%
\pgfpathlineto{\pgfqpoint{4.527515in}{4.181016in}}%
\pgfpathlineto{\pgfqpoint{4.559634in}{4.149333in}}%
\pgfpathlineto{\pgfqpoint{4.563482in}{4.145502in}}%
\pgfpathlineto{\pgfqpoint{4.567596in}{4.141475in}}%
\pgfpathlineto{\pgfqpoint{4.582876in}{4.126233in}}%
\pgfpathlineto{\pgfqpoint{4.597389in}{4.112000in}}%
\pgfpathlineto{\pgfqpoint{4.607677in}{4.101815in}}%
\pgfpathlineto{\pgfqpoint{4.621729in}{4.087756in}}%
\pgfpathlineto{\pgfqpoint{4.635037in}{4.074667in}}%
\pgfpathlineto{\pgfqpoint{4.647758in}{4.062037in}}%
\pgfpathlineto{\pgfqpoint{4.660525in}{4.049226in}}%
\pgfpathlineto{\pgfqpoint{4.672581in}{4.037333in}}%
\pgfpathlineto{\pgfqpoint{4.679921in}{4.029959in}}%
\pgfpathlineto{\pgfqpoint{4.687838in}{4.022140in}}%
\pgfpathlineto{\pgfqpoint{4.699264in}{4.010642in}}%
\pgfpathlineto{\pgfqpoint{4.710020in}{4.000000in}}%
\pgfpathlineto{\pgfqpoint{4.718618in}{3.991336in}}%
\pgfpathlineto{\pgfqpoint{4.727919in}{3.982123in}}%
\pgfpathlineto{\pgfqpoint{4.747356in}{3.962667in}}%
\pgfpathlineto{\pgfqpoint{4.768000in}{3.941988in}}%
\pgfusepath{fill}%
\end{pgfscope}%
\begin{pgfscope}%
\pgfpathrectangle{\pgfqpoint{0.800000in}{0.528000in}}{\pgfqpoint{3.968000in}{3.696000in}}%
\pgfusepath{clip}%
\pgfsetbuttcap%
\pgfsetroundjoin%
\definecolor{currentfill}{rgb}{0.678489,0.863742,0.189503}%
\pgfsetfillcolor{currentfill}%
\pgfsetlinewidth{0.000000pt}%
\definecolor{currentstroke}{rgb}{0.000000,0.000000,0.000000}%
\pgfsetstrokecolor{currentstroke}%
\pgfsetdash{}{0pt}%
\pgfpathmoveto{\pgfqpoint{4.768000in}{3.946999in}}%
\pgfpathlineto{\pgfqpoint{4.752358in}{3.962667in}}%
\pgfpathlineto{\pgfqpoint{4.727919in}{3.987131in}}%
\pgfpathlineto{\pgfqpoint{4.721224in}{3.993763in}}%
\pgfpathlineto{\pgfqpoint{4.715034in}{4.000000in}}%
\pgfpathlineto{\pgfqpoint{4.701846in}{4.013048in}}%
\pgfpathlineto{\pgfqpoint{4.687838in}{4.027144in}}%
\pgfpathlineto{\pgfqpoint{4.682529in}{4.032388in}}%
\pgfpathlineto{\pgfqpoint{4.677607in}{4.037333in}}%
\pgfpathlineto{\pgfqpoint{4.663110in}{4.051634in}}%
\pgfpathlineto{\pgfqpoint{4.647758in}{4.067038in}}%
\pgfpathlineto{\pgfqpoint{4.640075in}{4.074667in}}%
\pgfpathlineto{\pgfqpoint{4.624317in}{4.090166in}}%
\pgfpathlineto{\pgfqpoint{4.607677in}{4.106814in}}%
\pgfpathlineto{\pgfqpoint{4.602438in}{4.112000in}}%
\pgfpathlineto{\pgfqpoint{4.585466in}{4.128645in}}%
\pgfpathlineto{\pgfqpoint{4.567596in}{4.146470in}}%
\pgfpathlineto{\pgfqpoint{4.566097in}{4.147937in}}%
\pgfpathlineto{\pgfqpoint{4.564695in}{4.149333in}}%
\pgfpathlineto{\pgfqpoint{4.527515in}{4.186009in}}%
\pgfpathlineto{\pgfqpoint{4.527170in}{4.186345in}}%
\pgfpathlineto{\pgfqpoint{4.526847in}{4.186667in}}%
\pgfpathlineto{\pgfqpoint{4.514884in}{4.198432in}}%
\pgfpathlineto{\pgfqpoint{4.488875in}{4.224000in}}%
\pgfpathlineto{\pgfqpoint{4.487434in}{4.224000in}}%
\pgfpathlineto{\pgfqpoint{4.486348in}{4.224000in}}%
\pgfpathlineto{\pgfqpoint{4.486875in}{4.223479in}}%
\pgfpathlineto{\pgfqpoint{4.487434in}{4.222934in}}%
\pgfpathlineto{\pgfqpoint{4.524310in}{4.186667in}}%
\pgfpathlineto{\pgfqpoint{4.525862in}{4.185126in}}%
\pgfpathlineto{\pgfqpoint{4.527515in}{4.183513in}}%
\pgfpathlineto{\pgfqpoint{4.562165in}{4.149333in}}%
\pgfpathlineto{\pgfqpoint{4.564790in}{4.146719in}}%
\pgfpathlineto{\pgfqpoint{4.567596in}{4.143973in}}%
\pgfpathlineto{\pgfqpoint{4.584171in}{4.127439in}}%
\pgfpathlineto{\pgfqpoint{4.599913in}{4.112000in}}%
\pgfpathlineto{\pgfqpoint{4.607677in}{4.104314in}}%
\pgfpathlineto{\pgfqpoint{4.623023in}{4.088961in}}%
\pgfpathlineto{\pgfqpoint{4.637556in}{4.074667in}}%
\pgfpathlineto{\pgfqpoint{4.647758in}{4.064538in}}%
\pgfpathlineto{\pgfqpoint{4.661818in}{4.050430in}}%
\pgfpathlineto{\pgfqpoint{4.675094in}{4.037333in}}%
\pgfpathlineto{\pgfqpoint{4.681225in}{4.031173in}}%
\pgfpathlineto{\pgfqpoint{4.687838in}{4.024642in}}%
\pgfpathlineto{\pgfqpoint{4.700555in}{4.011845in}}%
\pgfpathlineto{\pgfqpoint{4.712527in}{4.000000in}}%
\pgfpathlineto{\pgfqpoint{4.719921in}{3.992550in}}%
\pgfpathlineto{\pgfqpoint{4.727919in}{3.984627in}}%
\pgfpathlineto{\pgfqpoint{4.749857in}{3.962667in}}%
\pgfpathlineto{\pgfqpoint{4.768000in}{3.944494in}}%
\pgfusepath{fill}%
\end{pgfscope}%
\begin{pgfscope}%
\pgfpathrectangle{\pgfqpoint{0.800000in}{0.528000in}}{\pgfqpoint{3.968000in}{3.696000in}}%
\pgfusepath{clip}%
\pgfsetbuttcap%
\pgfsetroundjoin%
\definecolor{currentfill}{rgb}{0.688944,0.865448,0.182725}%
\pgfsetfillcolor{currentfill}%
\pgfsetlinewidth{0.000000pt}%
\definecolor{currentstroke}{rgb}{0.000000,0.000000,0.000000}%
\pgfsetstrokecolor{currentstroke}%
\pgfsetdash{}{0pt}%
\pgfpathmoveto{\pgfqpoint{4.768000in}{3.949505in}}%
\pgfpathlineto{\pgfqpoint{4.754860in}{3.962667in}}%
\pgfpathlineto{\pgfqpoint{4.727919in}{3.989635in}}%
\pgfpathlineto{\pgfqpoint{4.722526in}{3.994977in}}%
\pgfpathlineto{\pgfqpoint{4.717541in}{4.000000in}}%
\pgfpathlineto{\pgfqpoint{4.703138in}{4.014251in}}%
\pgfpathlineto{\pgfqpoint{4.687838in}{4.029647in}}%
\pgfpathlineto{\pgfqpoint{4.683833in}{4.033602in}}%
\pgfpathlineto{\pgfqpoint{4.680119in}{4.037333in}}%
\pgfpathlineto{\pgfqpoint{4.664403in}{4.052837in}}%
\pgfpathlineto{\pgfqpoint{4.647758in}{4.069539in}}%
\pgfpathlineto{\pgfqpoint{4.642593in}{4.074667in}}%
\pgfpathlineto{\pgfqpoint{4.625610in}{4.091371in}}%
\pgfpathlineto{\pgfqpoint{4.607677in}{4.109313in}}%
\pgfpathlineto{\pgfqpoint{4.604962in}{4.112000in}}%
\pgfpathlineto{\pgfqpoint{4.586760in}{4.129851in}}%
\pgfpathlineto{\pgfqpoint{4.567596in}{4.148968in}}%
\pgfpathlineto{\pgfqpoint{4.567405in}{4.149155in}}%
\pgfpathlineto{\pgfqpoint{4.567226in}{4.149333in}}%
\pgfpathlineto{\pgfqpoint{4.560956in}{4.155519in}}%
\pgfpathlineto{\pgfqpoint{4.529363in}{4.186667in}}%
\pgfpathlineto{\pgfqpoint{4.528461in}{4.187548in}}%
\pgfpathlineto{\pgfqpoint{4.527515in}{4.188488in}}%
\pgfpathlineto{\pgfqpoint{4.491391in}{4.224000in}}%
\pgfpathlineto{\pgfqpoint{4.488875in}{4.224000in}}%
\pgfpathlineto{\pgfqpoint{4.514884in}{4.198432in}}%
\pgfpathlineto{\pgfqpoint{4.526847in}{4.186667in}}%
\pgfpathlineto{\pgfqpoint{4.527170in}{4.186345in}}%
\pgfpathlineto{\pgfqpoint{4.527515in}{4.186009in}}%
\pgfpathlineto{\pgfqpoint{4.564695in}{4.149333in}}%
\pgfpathlineto{\pgfqpoint{4.566097in}{4.147937in}}%
\pgfpathlineto{\pgfqpoint{4.567596in}{4.146470in}}%
\pgfpathlineto{\pgfqpoint{4.585466in}{4.128645in}}%
\pgfpathlineto{\pgfqpoint{4.602438in}{4.112000in}}%
\pgfpathlineto{\pgfqpoint{4.607677in}{4.106814in}}%
\pgfpathlineto{\pgfqpoint{4.624317in}{4.090166in}}%
\pgfpathlineto{\pgfqpoint{4.640075in}{4.074667in}}%
\pgfpathlineto{\pgfqpoint{4.647758in}{4.067038in}}%
\pgfpathlineto{\pgfqpoint{4.663110in}{4.051634in}}%
\pgfpathlineto{\pgfqpoint{4.677607in}{4.037333in}}%
\pgfpathlineto{\pgfqpoint{4.682529in}{4.032388in}}%
\pgfpathlineto{\pgfqpoint{4.687838in}{4.027144in}}%
\pgfpathlineto{\pgfqpoint{4.701846in}{4.013048in}}%
\pgfpathlineto{\pgfqpoint{4.715034in}{4.000000in}}%
\pgfpathlineto{\pgfqpoint{4.721224in}{3.993763in}}%
\pgfpathlineto{\pgfqpoint{4.727919in}{3.987131in}}%
\pgfpathlineto{\pgfqpoint{4.752358in}{3.962667in}}%
\pgfpathlineto{\pgfqpoint{4.768000in}{3.946999in}}%
\pgfusepath{fill}%
\end{pgfscope}%
\begin{pgfscope}%
\pgfpathrectangle{\pgfqpoint{0.800000in}{0.528000in}}{\pgfqpoint{3.968000in}{3.696000in}}%
\pgfusepath{clip}%
\pgfsetbuttcap%
\pgfsetroundjoin%
\definecolor{currentfill}{rgb}{0.688944,0.865448,0.182725}%
\pgfsetfillcolor{currentfill}%
\pgfsetlinewidth{0.000000pt}%
\definecolor{currentstroke}{rgb}{0.000000,0.000000,0.000000}%
\pgfsetstrokecolor{currentstroke}%
\pgfsetdash{}{0pt}%
\pgfpathmoveto{\pgfqpoint{4.768000in}{3.952010in}}%
\pgfpathlineto{\pgfqpoint{4.757361in}{3.962667in}}%
\pgfpathlineto{\pgfqpoint{4.727919in}{3.992139in}}%
\pgfpathlineto{\pgfqpoint{4.723829in}{3.996190in}}%
\pgfpathlineto{\pgfqpoint{4.720049in}{4.000000in}}%
\pgfpathlineto{\pgfqpoint{4.704429in}{4.015453in}}%
\pgfpathlineto{\pgfqpoint{4.687838in}{4.032149in}}%
\pgfpathlineto{\pgfqpoint{4.685137in}{4.034817in}}%
\pgfpathlineto{\pgfqpoint{4.682632in}{4.037333in}}%
\pgfpathlineto{\pgfqpoint{4.665695in}{4.054041in}}%
\pgfpathlineto{\pgfqpoint{4.647758in}{4.072040in}}%
\pgfpathlineto{\pgfqpoint{4.645112in}{4.074667in}}%
\pgfpathlineto{\pgfqpoint{4.626904in}{4.092576in}}%
\pgfpathlineto{\pgfqpoint{4.607677in}{4.111812in}}%
\pgfpathlineto{\pgfqpoint{4.607487in}{4.112000in}}%
\pgfpathlineto{\pgfqpoint{4.588055in}{4.131057in}}%
\pgfpathlineto{\pgfqpoint{4.569734in}{4.149333in}}%
\pgfpathlineto{\pgfqpoint{4.568692in}{4.150354in}}%
\pgfpathlineto{\pgfqpoint{4.567596in}{4.151446in}}%
\pgfpathlineto{\pgfqpoint{4.531873in}{4.186667in}}%
\pgfpathlineto{\pgfqpoint{4.529746in}{4.188744in}}%
\pgfpathlineto{\pgfqpoint{4.527515in}{4.190961in}}%
\pgfpathlineto{\pgfqpoint{4.493906in}{4.224000in}}%
\pgfpathlineto{\pgfqpoint{4.491391in}{4.224000in}}%
\pgfpathlineto{\pgfqpoint{4.527515in}{4.188488in}}%
\pgfpathlineto{\pgfqpoint{4.528461in}{4.187548in}}%
\pgfpathlineto{\pgfqpoint{4.529363in}{4.186667in}}%
\pgfpathlineto{\pgfqpoint{4.560956in}{4.155519in}}%
\pgfpathlineto{\pgfqpoint{4.567226in}{4.149333in}}%
\pgfpathlineto{\pgfqpoint{4.567405in}{4.149155in}}%
\pgfpathlineto{\pgfqpoint{4.567596in}{4.148968in}}%
\pgfpathlineto{\pgfqpoint{4.586760in}{4.129851in}}%
\pgfpathlineto{\pgfqpoint{4.604962in}{4.112000in}}%
\pgfpathlineto{\pgfqpoint{4.607677in}{4.109313in}}%
\pgfpathlineto{\pgfqpoint{4.625610in}{4.091371in}}%
\pgfpathlineto{\pgfqpoint{4.642593in}{4.074667in}}%
\pgfpathlineto{\pgfqpoint{4.647758in}{4.069539in}}%
\pgfpathlineto{\pgfqpoint{4.664403in}{4.052837in}}%
\pgfpathlineto{\pgfqpoint{4.680119in}{4.037333in}}%
\pgfpathlineto{\pgfqpoint{4.683833in}{4.033602in}}%
\pgfpathlineto{\pgfqpoint{4.687838in}{4.029647in}}%
\pgfpathlineto{\pgfqpoint{4.703138in}{4.014251in}}%
\pgfpathlineto{\pgfqpoint{4.717541in}{4.000000in}}%
\pgfpathlineto{\pgfqpoint{4.722526in}{3.994977in}}%
\pgfpathlineto{\pgfqpoint{4.727919in}{3.989635in}}%
\pgfpathlineto{\pgfqpoint{4.754860in}{3.962667in}}%
\pgfpathlineto{\pgfqpoint{4.768000in}{3.949505in}}%
\pgfusepath{fill}%
\end{pgfscope}%
\begin{pgfscope}%
\pgfpathrectangle{\pgfqpoint{0.800000in}{0.528000in}}{\pgfqpoint{3.968000in}{3.696000in}}%
\pgfusepath{clip}%
\pgfsetbuttcap%
\pgfsetroundjoin%
\definecolor{currentfill}{rgb}{0.688944,0.865448,0.182725}%
\pgfsetfillcolor{currentfill}%
\pgfsetlinewidth{0.000000pt}%
\definecolor{currentstroke}{rgb}{0.000000,0.000000,0.000000}%
\pgfsetstrokecolor{currentstroke}%
\pgfsetdash{}{0pt}%
\pgfpathmoveto{\pgfqpoint{4.768000in}{3.954516in}}%
\pgfpathlineto{\pgfqpoint{4.759863in}{3.962667in}}%
\pgfpathlineto{\pgfqpoint{4.727919in}{3.994643in}}%
\pgfpathlineto{\pgfqpoint{4.725132in}{3.997404in}}%
\pgfpathlineto{\pgfqpoint{4.722556in}{4.000000in}}%
\pgfpathlineto{\pgfqpoint{4.705720in}{4.016656in}}%
\pgfpathlineto{\pgfqpoint{4.687838in}{4.034652in}}%
\pgfpathlineto{\pgfqpoint{4.686441in}{4.036032in}}%
\pgfpathlineto{\pgfqpoint{4.685145in}{4.037333in}}%
\pgfpathlineto{\pgfqpoint{4.666988in}{4.055245in}}%
\pgfpathlineto{\pgfqpoint{4.647758in}{4.074541in}}%
\pgfpathlineto{\pgfqpoint{4.647631in}{4.074667in}}%
\pgfpathlineto{\pgfqpoint{4.628198in}{4.093781in}}%
\pgfpathlineto{\pgfqpoint{4.609987in}{4.112000in}}%
\pgfpathlineto{\pgfqpoint{4.608863in}{4.113105in}}%
\pgfpathlineto{\pgfqpoint{4.607677in}{4.114290in}}%
\pgfpathlineto{\pgfqpoint{4.589350in}{4.132263in}}%
\pgfpathlineto{\pgfqpoint{4.572238in}{4.149333in}}%
\pgfpathlineto{\pgfqpoint{4.569975in}{4.151550in}}%
\pgfpathlineto{\pgfqpoint{4.567596in}{4.153920in}}%
\pgfpathlineto{\pgfqpoint{4.534383in}{4.186667in}}%
\pgfpathlineto{\pgfqpoint{4.531030in}{4.189941in}}%
\pgfpathlineto{\pgfqpoint{4.527515in}{4.193433in}}%
\pgfpathlineto{\pgfqpoint{4.496422in}{4.224000in}}%
\pgfpathlineto{\pgfqpoint{4.493906in}{4.224000in}}%
\pgfpathlineto{\pgfqpoint{4.527515in}{4.190961in}}%
\pgfpathlineto{\pgfqpoint{4.529746in}{4.188744in}}%
\pgfpathlineto{\pgfqpoint{4.531873in}{4.186667in}}%
\pgfpathlineto{\pgfqpoint{4.567596in}{4.151446in}}%
\pgfpathlineto{\pgfqpoint{4.568692in}{4.150354in}}%
\pgfpathlineto{\pgfqpoint{4.569734in}{4.149333in}}%
\pgfpathlineto{\pgfqpoint{4.588055in}{4.131057in}}%
\pgfpathlineto{\pgfqpoint{4.607487in}{4.112000in}}%
\pgfpathlineto{\pgfqpoint{4.607677in}{4.111812in}}%
\pgfpathlineto{\pgfqpoint{4.626904in}{4.092576in}}%
\pgfpathlineto{\pgfqpoint{4.645112in}{4.074667in}}%
\pgfpathlineto{\pgfqpoint{4.647758in}{4.072040in}}%
\pgfpathlineto{\pgfqpoint{4.665695in}{4.054041in}}%
\pgfpathlineto{\pgfqpoint{4.682632in}{4.037333in}}%
\pgfpathlineto{\pgfqpoint{4.685137in}{4.034817in}}%
\pgfpathlineto{\pgfqpoint{4.687838in}{4.032149in}}%
\pgfpathlineto{\pgfqpoint{4.704429in}{4.015453in}}%
\pgfpathlineto{\pgfqpoint{4.720049in}{4.000000in}}%
\pgfpathlineto{\pgfqpoint{4.723829in}{3.996190in}}%
\pgfpathlineto{\pgfqpoint{4.727919in}{3.992139in}}%
\pgfpathlineto{\pgfqpoint{4.757361in}{3.962667in}}%
\pgfpathlineto{\pgfqpoint{4.768000in}{3.952010in}}%
\pgfusepath{fill}%
\end{pgfscope}%
\begin{pgfscope}%
\pgfpathrectangle{\pgfqpoint{0.800000in}{0.528000in}}{\pgfqpoint{3.968000in}{3.696000in}}%
\pgfusepath{clip}%
\pgfsetbuttcap%
\pgfsetroundjoin%
\definecolor{currentfill}{rgb}{0.699415,0.867117,0.175971}%
\pgfsetfillcolor{currentfill}%
\pgfsetlinewidth{0.000000pt}%
\definecolor{currentstroke}{rgb}{0.000000,0.000000,0.000000}%
\pgfsetstrokecolor{currentstroke}%
\pgfsetdash{}{0pt}%
\pgfpathmoveto{\pgfqpoint{4.768000in}{3.957021in}}%
\pgfpathlineto{\pgfqpoint{4.762364in}{3.962667in}}%
\pgfpathlineto{\pgfqpoint{4.727919in}{3.997147in}}%
\pgfpathlineto{\pgfqpoint{4.726435in}{3.998617in}}%
\pgfpathlineto{\pgfqpoint{4.725063in}{4.000000in}}%
\pgfpathlineto{\pgfqpoint{4.707012in}{4.017859in}}%
\pgfpathlineto{\pgfqpoint{4.687838in}{4.037154in}}%
\pgfpathlineto{\pgfqpoint{4.687745in}{4.037246in}}%
\pgfpathlineto{\pgfqpoint{4.687658in}{4.037333in}}%
\pgfpathlineto{\pgfqpoint{4.668280in}{4.056449in}}%
\pgfpathlineto{\pgfqpoint{4.650125in}{4.074667in}}%
\pgfpathlineto{\pgfqpoint{4.647758in}{4.077020in}}%
\pgfpathlineto{\pgfqpoint{4.629491in}{4.094986in}}%
\pgfpathlineto{\pgfqpoint{4.612485in}{4.112000in}}%
\pgfpathlineto{\pgfqpoint{4.610145in}{4.114299in}}%
\pgfpathlineto{\pgfqpoint{4.607677in}{4.116766in}}%
\pgfpathlineto{\pgfqpoint{4.590645in}{4.133469in}}%
\pgfpathlineto{\pgfqpoint{4.574741in}{4.149333in}}%
\pgfpathlineto{\pgfqpoint{4.571259in}{4.152745in}}%
\pgfpathlineto{\pgfqpoint{4.567596in}{4.156395in}}%
\pgfpathlineto{\pgfqpoint{4.536892in}{4.186667in}}%
\pgfpathlineto{\pgfqpoint{4.532315in}{4.191138in}}%
\pgfpathlineto{\pgfqpoint{4.527515in}{4.195906in}}%
\pgfpathlineto{\pgfqpoint{4.498937in}{4.224000in}}%
\pgfpathlineto{\pgfqpoint{4.496422in}{4.224000in}}%
\pgfpathlineto{\pgfqpoint{4.527515in}{4.193433in}}%
\pgfpathlineto{\pgfqpoint{4.531030in}{4.189941in}}%
\pgfpathlineto{\pgfqpoint{4.534383in}{4.186667in}}%
\pgfpathlineto{\pgfqpoint{4.567596in}{4.153920in}}%
\pgfpathlineto{\pgfqpoint{4.569975in}{4.151550in}}%
\pgfpathlineto{\pgfqpoint{4.572238in}{4.149333in}}%
\pgfpathlineto{\pgfqpoint{4.589350in}{4.132263in}}%
\pgfpathlineto{\pgfqpoint{4.607677in}{4.114290in}}%
\pgfpathlineto{\pgfqpoint{4.608863in}{4.113105in}}%
\pgfpathlineto{\pgfqpoint{4.609987in}{4.112000in}}%
\pgfpathlineto{\pgfqpoint{4.628198in}{4.093781in}}%
\pgfpathlineto{\pgfqpoint{4.647631in}{4.074667in}}%
\pgfpathlineto{\pgfqpoint{4.647758in}{4.074541in}}%
\pgfpathlineto{\pgfqpoint{4.666988in}{4.055245in}}%
\pgfpathlineto{\pgfqpoint{4.685145in}{4.037333in}}%
\pgfpathlineto{\pgfqpoint{4.686441in}{4.036032in}}%
\pgfpathlineto{\pgfqpoint{4.687838in}{4.034652in}}%
\pgfpathlineto{\pgfqpoint{4.705720in}{4.016656in}}%
\pgfpathlineto{\pgfqpoint{4.722556in}{4.000000in}}%
\pgfpathlineto{\pgfqpoint{4.725132in}{3.997404in}}%
\pgfpathlineto{\pgfqpoint{4.727919in}{3.994643in}}%
\pgfpathlineto{\pgfqpoint{4.759863in}{3.962667in}}%
\pgfpathlineto{\pgfqpoint{4.768000in}{3.954516in}}%
\pgfusepath{fill}%
\end{pgfscope}%
\begin{pgfscope}%
\pgfpathrectangle{\pgfqpoint{0.800000in}{0.528000in}}{\pgfqpoint{3.968000in}{3.696000in}}%
\pgfusepath{clip}%
\pgfsetbuttcap%
\pgfsetroundjoin%
\definecolor{currentfill}{rgb}{0.699415,0.867117,0.175971}%
\pgfsetfillcolor{currentfill}%
\pgfsetlinewidth{0.000000pt}%
\definecolor{currentstroke}{rgb}{0.000000,0.000000,0.000000}%
\pgfsetstrokecolor{currentstroke}%
\pgfsetdash{}{0pt}%
\pgfpathmoveto{\pgfqpoint{4.768000in}{3.959527in}}%
\pgfpathlineto{\pgfqpoint{4.764865in}{3.962667in}}%
\pgfpathlineto{\pgfqpoint{4.727919in}{3.999651in}}%
\pgfpathlineto{\pgfqpoint{4.727738in}{3.999831in}}%
\pgfpathlineto{\pgfqpoint{4.727570in}{4.000000in}}%
\pgfpathlineto{\pgfqpoint{4.708303in}{4.019062in}}%
\pgfpathlineto{\pgfqpoint{4.690147in}{4.037333in}}%
\pgfpathlineto{\pgfqpoint{4.687838in}{4.039635in}}%
\pgfpathlineto{\pgfqpoint{4.669573in}{4.057653in}}%
\pgfpathlineto{\pgfqpoint{4.652617in}{4.074667in}}%
\pgfpathlineto{\pgfqpoint{4.647758in}{4.079497in}}%
\pgfpathlineto{\pgfqpoint{4.630785in}{4.096191in}}%
\pgfpathlineto{\pgfqpoint{4.614983in}{4.112000in}}%
\pgfpathlineto{\pgfqpoint{4.611428in}{4.115494in}}%
\pgfpathlineto{\pgfqpoint{4.607677in}{4.119242in}}%
\pgfpathlineto{\pgfqpoint{4.591940in}{4.134675in}}%
\pgfpathlineto{\pgfqpoint{4.577245in}{4.149333in}}%
\pgfpathlineto{\pgfqpoint{4.572542in}{4.153941in}}%
\pgfpathlineto{\pgfqpoint{4.567596in}{4.158869in}}%
\pgfpathlineto{\pgfqpoint{4.539402in}{4.186667in}}%
\pgfpathlineto{\pgfqpoint{4.533600in}{4.192334in}}%
\pgfpathlineto{\pgfqpoint{4.527515in}{4.198379in}}%
\pgfpathlineto{\pgfqpoint{4.501452in}{4.224000in}}%
\pgfpathlineto{\pgfqpoint{4.498937in}{4.224000in}}%
\pgfpathlineto{\pgfqpoint{4.527515in}{4.195906in}}%
\pgfpathlineto{\pgfqpoint{4.532315in}{4.191138in}}%
\pgfpathlineto{\pgfqpoint{4.536892in}{4.186667in}}%
\pgfpathlineto{\pgfqpoint{4.567596in}{4.156395in}}%
\pgfpathlineto{\pgfqpoint{4.571259in}{4.152745in}}%
\pgfpathlineto{\pgfqpoint{4.574741in}{4.149333in}}%
\pgfpathlineto{\pgfqpoint{4.590645in}{4.133469in}}%
\pgfpathlineto{\pgfqpoint{4.607677in}{4.116766in}}%
\pgfpathlineto{\pgfqpoint{4.610145in}{4.114299in}}%
\pgfpathlineto{\pgfqpoint{4.612485in}{4.112000in}}%
\pgfpathlineto{\pgfqpoint{4.629491in}{4.094986in}}%
\pgfpathlineto{\pgfqpoint{4.647758in}{4.077020in}}%
\pgfpathlineto{\pgfqpoint{4.650125in}{4.074667in}}%
\pgfpathlineto{\pgfqpoint{4.668280in}{4.056449in}}%
\pgfpathlineto{\pgfqpoint{4.687658in}{4.037333in}}%
\pgfpathlineto{\pgfqpoint{4.687745in}{4.037246in}}%
\pgfpathlineto{\pgfqpoint{4.687838in}{4.037154in}}%
\pgfpathlineto{\pgfqpoint{4.707012in}{4.017859in}}%
\pgfpathlineto{\pgfqpoint{4.725063in}{4.000000in}}%
\pgfpathlineto{\pgfqpoint{4.726435in}{3.998617in}}%
\pgfpathlineto{\pgfqpoint{4.727919in}{3.997147in}}%
\pgfpathlineto{\pgfqpoint{4.762364in}{3.962667in}}%
\pgfpathlineto{\pgfqpoint{4.768000in}{3.957021in}}%
\pgfusepath{fill}%
\end{pgfscope}%
\begin{pgfscope}%
\pgfpathrectangle{\pgfqpoint{0.800000in}{0.528000in}}{\pgfqpoint{3.968000in}{3.696000in}}%
\pgfusepath{clip}%
\pgfsetbuttcap%
\pgfsetroundjoin%
\definecolor{currentfill}{rgb}{0.699415,0.867117,0.175971}%
\pgfsetfillcolor{currentfill}%
\pgfsetlinewidth{0.000000pt}%
\definecolor{currentstroke}{rgb}{0.000000,0.000000,0.000000}%
\pgfsetstrokecolor{currentstroke}%
\pgfsetdash{}{0pt}%
\pgfpathmoveto{\pgfqpoint{4.768000in}{3.962032in}}%
\pgfpathlineto{\pgfqpoint{4.767367in}{3.962667in}}%
\pgfpathlineto{\pgfqpoint{4.758890in}{3.971153in}}%
\pgfpathlineto{\pgfqpoint{4.730055in}{4.000000in}}%
\pgfpathlineto{\pgfqpoint{4.727919in}{4.002135in}}%
\pgfpathlineto{\pgfqpoint{4.709595in}{4.020265in}}%
\pgfpathlineto{\pgfqpoint{4.692633in}{4.037333in}}%
\pgfpathlineto{\pgfqpoint{4.687838in}{4.042114in}}%
\pgfpathlineto{\pgfqpoint{4.670865in}{4.058857in}}%
\pgfpathlineto{\pgfqpoint{4.655109in}{4.074667in}}%
\pgfpathlineto{\pgfqpoint{4.647758in}{4.081974in}}%
\pgfpathlineto{\pgfqpoint{4.632079in}{4.097396in}}%
\pgfpathlineto{\pgfqpoint{4.617481in}{4.112000in}}%
\pgfpathlineto{\pgfqpoint{4.612710in}{4.116688in}}%
\pgfpathlineto{\pgfqpoint{4.607677in}{4.121718in}}%
\pgfpathlineto{\pgfqpoint{4.593234in}{4.135881in}}%
\pgfpathlineto{\pgfqpoint{4.579749in}{4.149333in}}%
\pgfpathlineto{\pgfqpoint{4.573826in}{4.155136in}}%
\pgfpathlineto{\pgfqpoint{4.567596in}{4.161343in}}%
\pgfpathlineto{\pgfqpoint{4.541911in}{4.186667in}}%
\pgfpathlineto{\pgfqpoint{4.534884in}{4.193531in}}%
\pgfpathlineto{\pgfqpoint{4.527515in}{4.200852in}}%
\pgfpathlineto{\pgfqpoint{4.503968in}{4.224000in}}%
\pgfpathlineto{\pgfqpoint{4.501452in}{4.224000in}}%
\pgfpathlineto{\pgfqpoint{4.527515in}{4.198379in}}%
\pgfpathlineto{\pgfqpoint{4.533600in}{4.192334in}}%
\pgfpathlineto{\pgfqpoint{4.539402in}{4.186667in}}%
\pgfpathlineto{\pgfqpoint{4.567596in}{4.158869in}}%
\pgfpathlineto{\pgfqpoint{4.572542in}{4.153941in}}%
\pgfpathlineto{\pgfqpoint{4.577245in}{4.149333in}}%
\pgfpathlineto{\pgfqpoint{4.591940in}{4.134675in}}%
\pgfpathlineto{\pgfqpoint{4.607677in}{4.119242in}}%
\pgfpathlineto{\pgfqpoint{4.611428in}{4.115494in}}%
\pgfpathlineto{\pgfqpoint{4.614983in}{4.112000in}}%
\pgfpathlineto{\pgfqpoint{4.630785in}{4.096191in}}%
\pgfpathlineto{\pgfqpoint{4.647758in}{4.079497in}}%
\pgfpathlineto{\pgfqpoint{4.652617in}{4.074667in}}%
\pgfpathlineto{\pgfqpoint{4.669573in}{4.057653in}}%
\pgfpathlineto{\pgfqpoint{4.687838in}{4.039635in}}%
\pgfpathlineto{\pgfqpoint{4.690147in}{4.037333in}}%
\pgfpathlineto{\pgfqpoint{4.708303in}{4.019062in}}%
\pgfpathlineto{\pgfqpoint{4.727570in}{4.000000in}}%
\pgfpathlineto{\pgfqpoint{4.727738in}{3.999831in}}%
\pgfpathlineto{\pgfqpoint{4.727919in}{3.999651in}}%
\pgfpathlineto{\pgfqpoint{4.764865in}{3.962667in}}%
\pgfpathlineto{\pgfqpoint{4.768000in}{3.959527in}}%
\pgfusepath{fill}%
\end{pgfscope}%
\begin{pgfscope}%
\pgfpathrectangle{\pgfqpoint{0.800000in}{0.528000in}}{\pgfqpoint{3.968000in}{3.696000in}}%
\pgfusepath{clip}%
\pgfsetbuttcap%
\pgfsetroundjoin%
\definecolor{currentfill}{rgb}{0.699415,0.867117,0.175971}%
\pgfsetfillcolor{currentfill}%
\pgfsetlinewidth{0.000000pt}%
\definecolor{currentstroke}{rgb}{0.000000,0.000000,0.000000}%
\pgfsetstrokecolor{currentstroke}%
\pgfsetdash{}{0pt}%
\pgfpathmoveto{\pgfqpoint{4.768000in}{3.964520in}}%
\pgfpathlineto{\pgfqpoint{4.732536in}{4.000000in}}%
\pgfpathlineto{\pgfqpoint{4.727919in}{4.004615in}}%
\pgfpathlineto{\pgfqpoint{4.710886in}{4.021468in}}%
\pgfpathlineto{\pgfqpoint{4.695120in}{4.037333in}}%
\pgfpathlineto{\pgfqpoint{4.687838in}{4.044593in}}%
\pgfpathlineto{\pgfqpoint{4.672158in}{4.060061in}}%
\pgfpathlineto{\pgfqpoint{4.657602in}{4.074667in}}%
\pgfpathlineto{\pgfqpoint{4.647758in}{4.084452in}}%
\pgfpathlineto{\pgfqpoint{4.633372in}{4.098601in}}%
\pgfpathlineto{\pgfqpoint{4.619980in}{4.112000in}}%
\pgfpathlineto{\pgfqpoint{4.613992in}{4.117883in}}%
\pgfpathlineto{\pgfqpoint{4.607677in}{4.124193in}}%
\pgfpathlineto{\pgfqpoint{4.594529in}{4.137087in}}%
\pgfpathlineto{\pgfqpoint{4.582253in}{4.149333in}}%
\pgfpathlineto{\pgfqpoint{4.575109in}{4.156332in}}%
\pgfpathlineto{\pgfqpoint{4.567596in}{4.163818in}}%
\pgfpathlineto{\pgfqpoint{4.544421in}{4.186667in}}%
\pgfpathlineto{\pgfqpoint{4.536169in}{4.194727in}}%
\pgfpathlineto{\pgfqpoint{4.527515in}{4.203325in}}%
\pgfpathlineto{\pgfqpoint{4.506483in}{4.224000in}}%
\pgfpathlineto{\pgfqpoint{4.503968in}{4.224000in}}%
\pgfpathlineto{\pgfqpoint{4.527515in}{4.200852in}}%
\pgfpathlineto{\pgfqpoint{4.534884in}{4.193531in}}%
\pgfpathlineto{\pgfqpoint{4.541911in}{4.186667in}}%
\pgfpathlineto{\pgfqpoint{4.567596in}{4.161343in}}%
\pgfpathlineto{\pgfqpoint{4.573826in}{4.155136in}}%
\pgfpathlineto{\pgfqpoint{4.579749in}{4.149333in}}%
\pgfpathlineto{\pgfqpoint{4.593234in}{4.135881in}}%
\pgfpathlineto{\pgfqpoint{4.607677in}{4.121718in}}%
\pgfpathlineto{\pgfqpoint{4.612710in}{4.116688in}}%
\pgfpathlineto{\pgfqpoint{4.617481in}{4.112000in}}%
\pgfpathlineto{\pgfqpoint{4.632079in}{4.097396in}}%
\pgfpathlineto{\pgfqpoint{4.647758in}{4.081974in}}%
\pgfpathlineto{\pgfqpoint{4.655109in}{4.074667in}}%
\pgfpathlineto{\pgfqpoint{4.670865in}{4.058857in}}%
\pgfpathlineto{\pgfqpoint{4.687838in}{4.042114in}}%
\pgfpathlineto{\pgfqpoint{4.692633in}{4.037333in}}%
\pgfpathlineto{\pgfqpoint{4.709595in}{4.020265in}}%
\pgfpathlineto{\pgfqpoint{4.727919in}{4.002135in}}%
\pgfpathlineto{\pgfqpoint{4.730055in}{4.000000in}}%
\pgfpathlineto{\pgfqpoint{4.758890in}{3.971153in}}%
\pgfpathlineto{\pgfqpoint{4.767367in}{3.962667in}}%
\pgfpathlineto{\pgfqpoint{4.768000in}{3.962032in}}%
\pgfpathlineto{\pgfqpoint{4.768000in}{3.962667in}}%
\pgfusepath{fill}%
\end{pgfscope}%
\begin{pgfscope}%
\pgfpathrectangle{\pgfqpoint{0.800000in}{0.528000in}}{\pgfqpoint{3.968000in}{3.696000in}}%
\pgfusepath{clip}%
\pgfsetbuttcap%
\pgfsetroundjoin%
\definecolor{currentfill}{rgb}{0.709898,0.868751,0.169257}%
\pgfsetfillcolor{currentfill}%
\pgfsetlinewidth{0.000000pt}%
\definecolor{currentstroke}{rgb}{0.000000,0.000000,0.000000}%
\pgfsetstrokecolor{currentstroke}%
\pgfsetdash{}{0pt}%
\pgfpathmoveto{\pgfqpoint{4.768000in}{3.967002in}}%
\pgfpathlineto{\pgfqpoint{4.735016in}{4.000000in}}%
\pgfpathlineto{\pgfqpoint{4.727919in}{4.007096in}}%
\pgfpathlineto{\pgfqpoint{4.712177in}{4.022671in}}%
\pgfpathlineto{\pgfqpoint{4.697607in}{4.037333in}}%
\pgfpathlineto{\pgfqpoint{4.687838in}{4.047071in}}%
\pgfpathlineto{\pgfqpoint{4.673450in}{4.061265in}}%
\pgfpathlineto{\pgfqpoint{4.660094in}{4.074667in}}%
\pgfpathlineto{\pgfqpoint{4.647758in}{4.086929in}}%
\pgfpathlineto{\pgfqpoint{4.634666in}{4.099806in}}%
\pgfpathlineto{\pgfqpoint{4.622478in}{4.112000in}}%
\pgfpathlineto{\pgfqpoint{4.615275in}{4.119077in}}%
\pgfpathlineto{\pgfqpoint{4.607677in}{4.126669in}}%
\pgfpathlineto{\pgfqpoint{4.595824in}{4.138293in}}%
\pgfpathlineto{\pgfqpoint{4.584757in}{4.149333in}}%
\pgfpathlineto{\pgfqpoint{4.576393in}{4.157527in}}%
\pgfpathlineto{\pgfqpoint{4.567596in}{4.166292in}}%
\pgfpathlineto{\pgfqpoint{4.546931in}{4.186667in}}%
\pgfpathlineto{\pgfqpoint{4.537454in}{4.195924in}}%
\pgfpathlineto{\pgfqpoint{4.527515in}{4.205797in}}%
\pgfpathlineto{\pgfqpoint{4.508999in}{4.224000in}}%
\pgfpathlineto{\pgfqpoint{4.506483in}{4.224000in}}%
\pgfpathlineto{\pgfqpoint{4.527515in}{4.203325in}}%
\pgfpathlineto{\pgfqpoint{4.536169in}{4.194727in}}%
\pgfpathlineto{\pgfqpoint{4.544421in}{4.186667in}}%
\pgfpathlineto{\pgfqpoint{4.567596in}{4.163818in}}%
\pgfpathlineto{\pgfqpoint{4.575109in}{4.156332in}}%
\pgfpathlineto{\pgfqpoint{4.582253in}{4.149333in}}%
\pgfpathlineto{\pgfqpoint{4.594529in}{4.137087in}}%
\pgfpathlineto{\pgfqpoint{4.607677in}{4.124193in}}%
\pgfpathlineto{\pgfqpoint{4.613992in}{4.117883in}}%
\pgfpathlineto{\pgfqpoint{4.619980in}{4.112000in}}%
\pgfpathlineto{\pgfqpoint{4.633372in}{4.098601in}}%
\pgfpathlineto{\pgfqpoint{4.647758in}{4.084452in}}%
\pgfpathlineto{\pgfqpoint{4.657602in}{4.074667in}}%
\pgfpathlineto{\pgfqpoint{4.672158in}{4.060061in}}%
\pgfpathlineto{\pgfqpoint{4.687838in}{4.044593in}}%
\pgfpathlineto{\pgfqpoint{4.695120in}{4.037333in}}%
\pgfpathlineto{\pgfqpoint{4.710886in}{4.021468in}}%
\pgfpathlineto{\pgfqpoint{4.727919in}{4.004615in}}%
\pgfpathlineto{\pgfqpoint{4.732536in}{4.000000in}}%
\pgfpathlineto{\pgfqpoint{4.768000in}{3.964520in}}%
\pgfusepath{fill}%
\end{pgfscope}%
\begin{pgfscope}%
\pgfpathrectangle{\pgfqpoint{0.800000in}{0.528000in}}{\pgfqpoint{3.968000in}{3.696000in}}%
\pgfusepath{clip}%
\pgfsetbuttcap%
\pgfsetroundjoin%
\definecolor{currentfill}{rgb}{0.709898,0.868751,0.169257}%
\pgfsetfillcolor{currentfill}%
\pgfsetlinewidth{0.000000pt}%
\definecolor{currentstroke}{rgb}{0.000000,0.000000,0.000000}%
\pgfsetstrokecolor{currentstroke}%
\pgfsetdash{}{0pt}%
\pgfpathmoveto{\pgfqpoint{4.768000in}{3.969484in}}%
\pgfpathlineto{\pgfqpoint{4.737497in}{4.000000in}}%
\pgfpathlineto{\pgfqpoint{4.727919in}{4.009576in}}%
\pgfpathlineto{\pgfqpoint{4.713469in}{4.023873in}}%
\pgfpathlineto{\pgfqpoint{4.700093in}{4.037333in}}%
\pgfpathlineto{\pgfqpoint{4.687838in}{4.049550in}}%
\pgfpathlineto{\pgfqpoint{4.674743in}{4.062469in}}%
\pgfpathlineto{\pgfqpoint{4.662586in}{4.074667in}}%
\pgfpathlineto{\pgfqpoint{4.647758in}{4.089407in}}%
\pgfpathlineto{\pgfqpoint{4.635960in}{4.101011in}}%
\pgfpathlineto{\pgfqpoint{4.624976in}{4.112000in}}%
\pgfpathlineto{\pgfqpoint{4.616557in}{4.120272in}}%
\pgfpathlineto{\pgfqpoint{4.607677in}{4.129145in}}%
\pgfpathlineto{\pgfqpoint{4.597119in}{4.139499in}}%
\pgfpathlineto{\pgfqpoint{4.587261in}{4.149333in}}%
\pgfpathlineto{\pgfqpoint{4.577676in}{4.158723in}}%
\pgfpathlineto{\pgfqpoint{4.567596in}{4.168766in}}%
\pgfpathlineto{\pgfqpoint{4.549440in}{4.186667in}}%
\pgfpathlineto{\pgfqpoint{4.538738in}{4.197121in}}%
\pgfpathlineto{\pgfqpoint{4.527515in}{4.208270in}}%
\pgfpathlineto{\pgfqpoint{4.511514in}{4.224000in}}%
\pgfpathlineto{\pgfqpoint{4.508999in}{4.224000in}}%
\pgfpathlineto{\pgfqpoint{4.527515in}{4.205797in}}%
\pgfpathlineto{\pgfqpoint{4.537454in}{4.195924in}}%
\pgfpathlineto{\pgfqpoint{4.546931in}{4.186667in}}%
\pgfpathlineto{\pgfqpoint{4.567596in}{4.166292in}}%
\pgfpathlineto{\pgfqpoint{4.576393in}{4.157527in}}%
\pgfpathlineto{\pgfqpoint{4.584757in}{4.149333in}}%
\pgfpathlineto{\pgfqpoint{4.595824in}{4.138293in}}%
\pgfpathlineto{\pgfqpoint{4.607677in}{4.126669in}}%
\pgfpathlineto{\pgfqpoint{4.615275in}{4.119077in}}%
\pgfpathlineto{\pgfqpoint{4.622478in}{4.112000in}}%
\pgfpathlineto{\pgfqpoint{4.634666in}{4.099806in}}%
\pgfpathlineto{\pgfqpoint{4.647758in}{4.086929in}}%
\pgfpathlineto{\pgfqpoint{4.660094in}{4.074667in}}%
\pgfpathlineto{\pgfqpoint{4.673450in}{4.061265in}}%
\pgfpathlineto{\pgfqpoint{4.687838in}{4.047071in}}%
\pgfpathlineto{\pgfqpoint{4.697607in}{4.037333in}}%
\pgfpathlineto{\pgfqpoint{4.712177in}{4.022671in}}%
\pgfpathlineto{\pgfqpoint{4.727919in}{4.007096in}}%
\pgfpathlineto{\pgfqpoint{4.735016in}{4.000000in}}%
\pgfpathlineto{\pgfqpoint{4.768000in}{3.967002in}}%
\pgfusepath{fill}%
\end{pgfscope}%
\begin{pgfscope}%
\pgfpathrectangle{\pgfqpoint{0.800000in}{0.528000in}}{\pgfqpoint{3.968000in}{3.696000in}}%
\pgfusepath{clip}%
\pgfsetbuttcap%
\pgfsetroundjoin%
\definecolor{currentfill}{rgb}{0.709898,0.868751,0.169257}%
\pgfsetfillcolor{currentfill}%
\pgfsetlinewidth{0.000000pt}%
\definecolor{currentstroke}{rgb}{0.000000,0.000000,0.000000}%
\pgfsetstrokecolor{currentstroke}%
\pgfsetdash{}{0pt}%
\pgfpathmoveto{\pgfqpoint{4.768000in}{3.971966in}}%
\pgfpathlineto{\pgfqpoint{4.739978in}{4.000000in}}%
\pgfpathlineto{\pgfqpoint{4.727919in}{4.012057in}}%
\pgfpathlineto{\pgfqpoint{4.714760in}{4.025076in}}%
\pgfpathlineto{\pgfqpoint{4.702580in}{4.037333in}}%
\pgfpathlineto{\pgfqpoint{4.687838in}{4.052029in}}%
\pgfpathlineto{\pgfqpoint{4.676035in}{4.063673in}}%
\pgfpathlineto{\pgfqpoint{4.665079in}{4.074667in}}%
\pgfpathlineto{\pgfqpoint{4.647758in}{4.091884in}}%
\pgfpathlineto{\pgfqpoint{4.637253in}{4.102216in}}%
\pgfpathlineto{\pgfqpoint{4.627474in}{4.112000in}}%
\pgfpathlineto{\pgfqpoint{4.617839in}{4.121466in}}%
\pgfpathlineto{\pgfqpoint{4.607677in}{4.131621in}}%
\pgfpathlineto{\pgfqpoint{4.598414in}{4.140705in}}%
\pgfpathlineto{\pgfqpoint{4.589764in}{4.149333in}}%
\pgfpathlineto{\pgfqpoint{4.578960in}{4.159918in}}%
\pgfpathlineto{\pgfqpoint{4.567596in}{4.171241in}}%
\pgfpathlineto{\pgfqpoint{4.551950in}{4.186667in}}%
\pgfpathlineto{\pgfqpoint{4.540023in}{4.198317in}}%
\pgfpathlineto{\pgfqpoint{4.527515in}{4.210743in}}%
\pgfpathlineto{\pgfqpoint{4.514030in}{4.224000in}}%
\pgfpathlineto{\pgfqpoint{4.511514in}{4.224000in}}%
\pgfpathlineto{\pgfqpoint{4.527515in}{4.208270in}}%
\pgfpathlineto{\pgfqpoint{4.538738in}{4.197121in}}%
\pgfpathlineto{\pgfqpoint{4.549440in}{4.186667in}}%
\pgfpathlineto{\pgfqpoint{4.567596in}{4.168766in}}%
\pgfpathlineto{\pgfqpoint{4.577676in}{4.158723in}}%
\pgfpathlineto{\pgfqpoint{4.587261in}{4.149333in}}%
\pgfpathlineto{\pgfqpoint{4.597119in}{4.139499in}}%
\pgfpathlineto{\pgfqpoint{4.607677in}{4.129145in}}%
\pgfpathlineto{\pgfqpoint{4.616557in}{4.120272in}}%
\pgfpathlineto{\pgfqpoint{4.624976in}{4.112000in}}%
\pgfpathlineto{\pgfqpoint{4.635960in}{4.101011in}}%
\pgfpathlineto{\pgfqpoint{4.647758in}{4.089407in}}%
\pgfpathlineto{\pgfqpoint{4.662586in}{4.074667in}}%
\pgfpathlineto{\pgfqpoint{4.674743in}{4.062469in}}%
\pgfpathlineto{\pgfqpoint{4.687838in}{4.049550in}}%
\pgfpathlineto{\pgfqpoint{4.700093in}{4.037333in}}%
\pgfpathlineto{\pgfqpoint{4.713469in}{4.023873in}}%
\pgfpathlineto{\pgfqpoint{4.727919in}{4.009576in}}%
\pgfpathlineto{\pgfqpoint{4.737497in}{4.000000in}}%
\pgfpathlineto{\pgfqpoint{4.768000in}{3.969484in}}%
\pgfusepath{fill}%
\end{pgfscope}%
\begin{pgfscope}%
\pgfpathrectangle{\pgfqpoint{0.800000in}{0.528000in}}{\pgfqpoint{3.968000in}{3.696000in}}%
\pgfusepath{clip}%
\pgfsetbuttcap%
\pgfsetroundjoin%
\definecolor{currentfill}{rgb}{0.709898,0.868751,0.169257}%
\pgfsetfillcolor{currentfill}%
\pgfsetlinewidth{0.000000pt}%
\definecolor{currentstroke}{rgb}{0.000000,0.000000,0.000000}%
\pgfsetstrokecolor{currentstroke}%
\pgfsetdash{}{0pt}%
\pgfpathmoveto{\pgfqpoint{4.768000in}{3.974448in}}%
\pgfpathlineto{\pgfqpoint{4.742459in}{4.000000in}}%
\pgfpathlineto{\pgfqpoint{4.727919in}{4.014537in}}%
\pgfpathlineto{\pgfqpoint{4.716051in}{4.026279in}}%
\pgfpathlineto{\pgfqpoint{4.705067in}{4.037333in}}%
\pgfpathlineto{\pgfqpoint{4.687838in}{4.054508in}}%
\pgfpathlineto{\pgfqpoint{4.677328in}{4.064877in}}%
\pgfpathlineto{\pgfqpoint{4.667571in}{4.074667in}}%
\pgfpathlineto{\pgfqpoint{4.647758in}{4.094361in}}%
\pgfpathlineto{\pgfqpoint{4.638547in}{4.103421in}}%
\pgfpathlineto{\pgfqpoint{4.629972in}{4.112000in}}%
\pgfpathlineto{\pgfqpoint{4.619122in}{4.122660in}}%
\pgfpathlineto{\pgfqpoint{4.607677in}{4.134097in}}%
\pgfpathlineto{\pgfqpoint{4.599709in}{4.141911in}}%
\pgfpathlineto{\pgfqpoint{4.592268in}{4.149333in}}%
\pgfpathlineto{\pgfqpoint{4.580243in}{4.161114in}}%
\pgfpathlineto{\pgfqpoint{4.567596in}{4.173715in}}%
\pgfpathlineto{\pgfqpoint{4.554460in}{4.186667in}}%
\pgfpathlineto{\pgfqpoint{4.541308in}{4.199514in}}%
\pgfpathlineto{\pgfqpoint{4.527515in}{4.213216in}}%
\pgfpathlineto{\pgfqpoint{4.516545in}{4.224000in}}%
\pgfpathlineto{\pgfqpoint{4.514030in}{4.224000in}}%
\pgfpathlineto{\pgfqpoint{4.527515in}{4.210743in}}%
\pgfpathlineto{\pgfqpoint{4.540023in}{4.198317in}}%
\pgfpathlineto{\pgfqpoint{4.551950in}{4.186667in}}%
\pgfpathlineto{\pgfqpoint{4.567596in}{4.171241in}}%
\pgfpathlineto{\pgfqpoint{4.578960in}{4.159918in}}%
\pgfpathlineto{\pgfqpoint{4.589764in}{4.149333in}}%
\pgfpathlineto{\pgfqpoint{4.598414in}{4.140705in}}%
\pgfpathlineto{\pgfqpoint{4.607677in}{4.131621in}}%
\pgfpathlineto{\pgfqpoint{4.617839in}{4.121466in}}%
\pgfpathlineto{\pgfqpoint{4.627474in}{4.112000in}}%
\pgfpathlineto{\pgfqpoint{4.637253in}{4.102216in}}%
\pgfpathlineto{\pgfqpoint{4.647758in}{4.091884in}}%
\pgfpathlineto{\pgfqpoint{4.665079in}{4.074667in}}%
\pgfpathlineto{\pgfqpoint{4.676035in}{4.063673in}}%
\pgfpathlineto{\pgfqpoint{4.687838in}{4.052029in}}%
\pgfpathlineto{\pgfqpoint{4.702580in}{4.037333in}}%
\pgfpathlineto{\pgfqpoint{4.714760in}{4.025076in}}%
\pgfpathlineto{\pgfqpoint{4.727919in}{4.012057in}}%
\pgfpathlineto{\pgfqpoint{4.739978in}{4.000000in}}%
\pgfpathlineto{\pgfqpoint{4.768000in}{3.971966in}}%
\pgfusepath{fill}%
\end{pgfscope}%
\begin{pgfscope}%
\pgfpathrectangle{\pgfqpoint{0.800000in}{0.528000in}}{\pgfqpoint{3.968000in}{3.696000in}}%
\pgfusepath{clip}%
\pgfsetbuttcap%
\pgfsetroundjoin%
\definecolor{currentfill}{rgb}{0.720391,0.870350,0.162603}%
\pgfsetfillcolor{currentfill}%
\pgfsetlinewidth{0.000000pt}%
\definecolor{currentstroke}{rgb}{0.000000,0.000000,0.000000}%
\pgfsetstrokecolor{currentstroke}%
\pgfsetdash{}{0pt}%
\pgfpathmoveto{\pgfqpoint{4.768000in}{3.976930in}}%
\pgfpathlineto{\pgfqpoint{4.744940in}{4.000000in}}%
\pgfpathlineto{\pgfqpoint{4.727919in}{4.017018in}}%
\pgfpathlineto{\pgfqpoint{4.717343in}{4.027482in}}%
\pgfpathlineto{\pgfqpoint{4.707553in}{4.037333in}}%
\pgfpathlineto{\pgfqpoint{4.687838in}{4.056987in}}%
\pgfpathlineto{\pgfqpoint{4.678620in}{4.066081in}}%
\pgfpathlineto{\pgfqpoint{4.670063in}{4.074667in}}%
\pgfpathlineto{\pgfqpoint{4.647758in}{4.096839in}}%
\pgfpathlineto{\pgfqpoint{4.639841in}{4.104626in}}%
\pgfpathlineto{\pgfqpoint{4.632470in}{4.112000in}}%
\pgfpathlineto{\pgfqpoint{4.620404in}{4.123855in}}%
\pgfpathlineto{\pgfqpoint{4.607677in}{4.136573in}}%
\pgfpathlineto{\pgfqpoint{4.601003in}{4.143117in}}%
\pgfpathlineto{\pgfqpoint{4.594772in}{4.149333in}}%
\pgfpathlineto{\pgfqpoint{4.581527in}{4.162309in}}%
\pgfpathlineto{\pgfqpoint{4.567596in}{4.176189in}}%
\pgfpathlineto{\pgfqpoint{4.556969in}{4.186667in}}%
\pgfpathlineto{\pgfqpoint{4.542592in}{4.200710in}}%
\pgfpathlineto{\pgfqpoint{4.527515in}{4.215689in}}%
\pgfpathlineto{\pgfqpoint{4.519061in}{4.224000in}}%
\pgfpathlineto{\pgfqpoint{4.516545in}{4.224000in}}%
\pgfpathlineto{\pgfqpoint{4.527515in}{4.213216in}}%
\pgfpathlineto{\pgfqpoint{4.541308in}{4.199514in}}%
\pgfpathlineto{\pgfqpoint{4.554460in}{4.186667in}}%
\pgfpathlineto{\pgfqpoint{4.567596in}{4.173715in}}%
\pgfpathlineto{\pgfqpoint{4.580243in}{4.161114in}}%
\pgfpathlineto{\pgfqpoint{4.592268in}{4.149333in}}%
\pgfpathlineto{\pgfqpoint{4.599709in}{4.141911in}}%
\pgfpathlineto{\pgfqpoint{4.607677in}{4.134097in}}%
\pgfpathlineto{\pgfqpoint{4.619122in}{4.122660in}}%
\pgfpathlineto{\pgfqpoint{4.629972in}{4.112000in}}%
\pgfpathlineto{\pgfqpoint{4.638547in}{4.103421in}}%
\pgfpathlineto{\pgfqpoint{4.647758in}{4.094361in}}%
\pgfpathlineto{\pgfqpoint{4.667571in}{4.074667in}}%
\pgfpathlineto{\pgfqpoint{4.677328in}{4.064877in}}%
\pgfpathlineto{\pgfqpoint{4.687838in}{4.054508in}}%
\pgfpathlineto{\pgfqpoint{4.705067in}{4.037333in}}%
\pgfpathlineto{\pgfqpoint{4.716051in}{4.026279in}}%
\pgfpathlineto{\pgfqpoint{4.727919in}{4.014537in}}%
\pgfpathlineto{\pgfqpoint{4.742459in}{4.000000in}}%
\pgfpathlineto{\pgfqpoint{4.768000in}{3.974448in}}%
\pgfusepath{fill}%
\end{pgfscope}%
\begin{pgfscope}%
\pgfpathrectangle{\pgfqpoint{0.800000in}{0.528000in}}{\pgfqpoint{3.968000in}{3.696000in}}%
\pgfusepath{clip}%
\pgfsetbuttcap%
\pgfsetroundjoin%
\definecolor{currentfill}{rgb}{0.720391,0.870350,0.162603}%
\pgfsetfillcolor{currentfill}%
\pgfsetlinewidth{0.000000pt}%
\definecolor{currentstroke}{rgb}{0.000000,0.000000,0.000000}%
\pgfsetstrokecolor{currentstroke}%
\pgfsetdash{}{0pt}%
\pgfpathmoveto{\pgfqpoint{4.768000in}{3.979412in}}%
\pgfpathlineto{\pgfqpoint{4.747421in}{4.000000in}}%
\pgfpathlineto{\pgfqpoint{4.727919in}{4.019498in}}%
\pgfpathlineto{\pgfqpoint{4.718634in}{4.028685in}}%
\pgfpathlineto{\pgfqpoint{4.710040in}{4.037333in}}%
\pgfpathlineto{\pgfqpoint{4.687838in}{4.059466in}}%
\pgfpathlineto{\pgfqpoint{4.679913in}{4.067285in}}%
\pgfpathlineto{\pgfqpoint{4.672556in}{4.074667in}}%
\pgfpathlineto{\pgfqpoint{4.647758in}{4.099316in}}%
\pgfpathlineto{\pgfqpoint{4.641134in}{4.105831in}}%
\pgfpathlineto{\pgfqpoint{4.634968in}{4.112000in}}%
\pgfpathlineto{\pgfqpoint{4.621687in}{4.125049in}}%
\pgfpathlineto{\pgfqpoint{4.607677in}{4.139049in}}%
\pgfpathlineto{\pgfqpoint{4.602298in}{4.144323in}}%
\pgfpathlineto{\pgfqpoint{4.597276in}{4.149333in}}%
\pgfpathlineto{\pgfqpoint{4.582810in}{4.163505in}}%
\pgfpathlineto{\pgfqpoint{4.567596in}{4.178664in}}%
\pgfpathlineto{\pgfqpoint{4.559479in}{4.186667in}}%
\pgfpathlineto{\pgfqpoint{4.543877in}{4.201907in}}%
\pgfpathlineto{\pgfqpoint{4.527515in}{4.218162in}}%
\pgfpathlineto{\pgfqpoint{4.521576in}{4.224000in}}%
\pgfpathlineto{\pgfqpoint{4.519061in}{4.224000in}}%
\pgfpathlineto{\pgfqpoint{4.527515in}{4.215689in}}%
\pgfpathlineto{\pgfqpoint{4.542592in}{4.200710in}}%
\pgfpathlineto{\pgfqpoint{4.556969in}{4.186667in}}%
\pgfpathlineto{\pgfqpoint{4.567596in}{4.176189in}}%
\pgfpathlineto{\pgfqpoint{4.581527in}{4.162309in}}%
\pgfpathlineto{\pgfqpoint{4.594772in}{4.149333in}}%
\pgfpathlineto{\pgfqpoint{4.601003in}{4.143117in}}%
\pgfpathlineto{\pgfqpoint{4.607677in}{4.136573in}}%
\pgfpathlineto{\pgfqpoint{4.620404in}{4.123855in}}%
\pgfpathlineto{\pgfqpoint{4.632470in}{4.112000in}}%
\pgfpathlineto{\pgfqpoint{4.639841in}{4.104626in}}%
\pgfpathlineto{\pgfqpoint{4.647758in}{4.096839in}}%
\pgfpathlineto{\pgfqpoint{4.670063in}{4.074667in}}%
\pgfpathlineto{\pgfqpoint{4.678620in}{4.066081in}}%
\pgfpathlineto{\pgfqpoint{4.687838in}{4.056987in}}%
\pgfpathlineto{\pgfqpoint{4.707553in}{4.037333in}}%
\pgfpathlineto{\pgfqpoint{4.717343in}{4.027482in}}%
\pgfpathlineto{\pgfqpoint{4.727919in}{4.017018in}}%
\pgfpathlineto{\pgfqpoint{4.744940in}{4.000000in}}%
\pgfpathlineto{\pgfqpoint{4.768000in}{3.976930in}}%
\pgfusepath{fill}%
\end{pgfscope}%
\begin{pgfscope}%
\pgfpathrectangle{\pgfqpoint{0.800000in}{0.528000in}}{\pgfqpoint{3.968000in}{3.696000in}}%
\pgfusepath{clip}%
\pgfsetbuttcap%
\pgfsetroundjoin%
\definecolor{currentfill}{rgb}{0.720391,0.870350,0.162603}%
\pgfsetfillcolor{currentfill}%
\pgfsetlinewidth{0.000000pt}%
\definecolor{currentstroke}{rgb}{0.000000,0.000000,0.000000}%
\pgfsetstrokecolor{currentstroke}%
\pgfsetdash{}{0pt}%
\pgfpathmoveto{\pgfqpoint{4.768000in}{3.981895in}}%
\pgfpathlineto{\pgfqpoint{4.749902in}{4.000000in}}%
\pgfpathlineto{\pgfqpoint{4.727919in}{4.021979in}}%
\pgfpathlineto{\pgfqpoint{4.719926in}{4.029888in}}%
\pgfpathlineto{\pgfqpoint{4.712527in}{4.037333in}}%
\pgfpathlineto{\pgfqpoint{4.687838in}{4.061945in}}%
\pgfpathlineto{\pgfqpoint{4.681205in}{4.068488in}}%
\pgfpathlineto{\pgfqpoint{4.675048in}{4.074667in}}%
\pgfpathlineto{\pgfqpoint{4.647758in}{4.101794in}}%
\pgfpathlineto{\pgfqpoint{4.642428in}{4.107036in}}%
\pgfpathlineto{\pgfqpoint{4.637466in}{4.112000in}}%
\pgfpathlineto{\pgfqpoint{4.622969in}{4.126244in}}%
\pgfpathlineto{\pgfqpoint{4.607677in}{4.141525in}}%
\pgfpathlineto{\pgfqpoint{4.603593in}{4.145530in}}%
\pgfpathlineto{\pgfqpoint{4.599780in}{4.149333in}}%
\pgfpathlineto{\pgfqpoint{4.584094in}{4.164700in}}%
\pgfpathlineto{\pgfqpoint{4.567596in}{4.181138in}}%
\pgfpathlineto{\pgfqpoint{4.561989in}{4.186667in}}%
\pgfpathlineto{\pgfqpoint{4.545162in}{4.203103in}}%
\pgfpathlineto{\pgfqpoint{4.527515in}{4.220634in}}%
\pgfpathlineto{\pgfqpoint{4.524092in}{4.224000in}}%
\pgfpathlineto{\pgfqpoint{4.521576in}{4.224000in}}%
\pgfpathlineto{\pgfqpoint{4.527515in}{4.218162in}}%
\pgfpathlineto{\pgfqpoint{4.543877in}{4.201907in}}%
\pgfpathlineto{\pgfqpoint{4.559479in}{4.186667in}}%
\pgfpathlineto{\pgfqpoint{4.567596in}{4.178664in}}%
\pgfpathlineto{\pgfqpoint{4.582810in}{4.163505in}}%
\pgfpathlineto{\pgfqpoint{4.597276in}{4.149333in}}%
\pgfpathlineto{\pgfqpoint{4.602298in}{4.144323in}}%
\pgfpathlineto{\pgfqpoint{4.607677in}{4.139049in}}%
\pgfpathlineto{\pgfqpoint{4.621687in}{4.125049in}}%
\pgfpathlineto{\pgfqpoint{4.634968in}{4.112000in}}%
\pgfpathlineto{\pgfqpoint{4.641134in}{4.105831in}}%
\pgfpathlineto{\pgfqpoint{4.647758in}{4.099316in}}%
\pgfpathlineto{\pgfqpoint{4.672556in}{4.074667in}}%
\pgfpathlineto{\pgfqpoint{4.679913in}{4.067285in}}%
\pgfpathlineto{\pgfqpoint{4.687838in}{4.059466in}}%
\pgfpathlineto{\pgfqpoint{4.710040in}{4.037333in}}%
\pgfpathlineto{\pgfqpoint{4.718634in}{4.028685in}}%
\pgfpathlineto{\pgfqpoint{4.727919in}{4.019498in}}%
\pgfpathlineto{\pgfqpoint{4.747421in}{4.000000in}}%
\pgfpathlineto{\pgfqpoint{4.768000in}{3.979412in}}%
\pgfusepath{fill}%
\end{pgfscope}%
\begin{pgfscope}%
\pgfpathrectangle{\pgfqpoint{0.800000in}{0.528000in}}{\pgfqpoint{3.968000in}{3.696000in}}%
\pgfusepath{clip}%
\pgfsetbuttcap%
\pgfsetroundjoin%
\definecolor{currentfill}{rgb}{0.730889,0.871916,0.156029}%
\pgfsetfillcolor{currentfill}%
\pgfsetlinewidth{0.000000pt}%
\definecolor{currentstroke}{rgb}{0.000000,0.000000,0.000000}%
\pgfsetstrokecolor{currentstroke}%
\pgfsetdash{}{0pt}%
\pgfpathmoveto{\pgfqpoint{4.768000in}{3.984377in}}%
\pgfpathlineto{\pgfqpoint{4.752383in}{4.000000in}}%
\pgfpathlineto{\pgfqpoint{4.727919in}{4.024459in}}%
\pgfpathlineto{\pgfqpoint{4.721217in}{4.031090in}}%
\pgfpathlineto{\pgfqpoint{4.715013in}{4.037333in}}%
\pgfpathlineto{\pgfqpoint{4.687838in}{4.064424in}}%
\pgfpathlineto{\pgfqpoint{4.682498in}{4.069692in}}%
\pgfpathlineto{\pgfqpoint{4.677541in}{4.074667in}}%
\pgfpathlineto{\pgfqpoint{4.647758in}{4.104271in}}%
\pgfpathlineto{\pgfqpoint{4.643722in}{4.108241in}}%
\pgfpathlineto{\pgfqpoint{4.639964in}{4.112000in}}%
\pgfpathlineto{\pgfqpoint{4.624251in}{4.127438in}}%
\pgfpathlineto{\pgfqpoint{4.607677in}{4.144000in}}%
\pgfpathlineto{\pgfqpoint{4.604888in}{4.146736in}}%
\pgfpathlineto{\pgfqpoint{4.602284in}{4.149333in}}%
\pgfpathlineto{\pgfqpoint{4.585377in}{4.165896in}}%
\pgfpathlineto{\pgfqpoint{4.567596in}{4.183612in}}%
\pgfpathlineto{\pgfqpoint{4.564498in}{4.186667in}}%
\pgfpathlineto{\pgfqpoint{4.546446in}{4.204300in}}%
\pgfpathlineto{\pgfqpoint{4.527515in}{4.223107in}}%
\pgfpathlineto{\pgfqpoint{4.526607in}{4.224000in}}%
\pgfpathlineto{\pgfqpoint{4.524092in}{4.224000in}}%
\pgfpathlineto{\pgfqpoint{4.527515in}{4.220634in}}%
\pgfpathlineto{\pgfqpoint{4.545162in}{4.203103in}}%
\pgfpathlineto{\pgfqpoint{4.561989in}{4.186667in}}%
\pgfpathlineto{\pgfqpoint{4.567596in}{4.181138in}}%
\pgfpathlineto{\pgfqpoint{4.584094in}{4.164700in}}%
\pgfpathlineto{\pgfqpoint{4.599780in}{4.149333in}}%
\pgfpathlineto{\pgfqpoint{4.603593in}{4.145530in}}%
\pgfpathlineto{\pgfqpoint{4.607677in}{4.141525in}}%
\pgfpathlineto{\pgfqpoint{4.622969in}{4.126244in}}%
\pgfpathlineto{\pgfqpoint{4.637466in}{4.112000in}}%
\pgfpathlineto{\pgfqpoint{4.642428in}{4.107036in}}%
\pgfpathlineto{\pgfqpoint{4.647758in}{4.101794in}}%
\pgfpathlineto{\pgfqpoint{4.675048in}{4.074667in}}%
\pgfpathlineto{\pgfqpoint{4.681205in}{4.068488in}}%
\pgfpathlineto{\pgfqpoint{4.687838in}{4.061945in}}%
\pgfpathlineto{\pgfqpoint{4.712527in}{4.037333in}}%
\pgfpathlineto{\pgfqpoint{4.719926in}{4.029888in}}%
\pgfpathlineto{\pgfqpoint{4.727919in}{4.021979in}}%
\pgfpathlineto{\pgfqpoint{4.749902in}{4.000000in}}%
\pgfpathlineto{\pgfqpoint{4.768000in}{3.981895in}}%
\pgfusepath{fill}%
\end{pgfscope}%
\begin{pgfscope}%
\pgfpathrectangle{\pgfqpoint{0.800000in}{0.528000in}}{\pgfqpoint{3.968000in}{3.696000in}}%
\pgfusepath{clip}%
\pgfsetbuttcap%
\pgfsetroundjoin%
\definecolor{currentfill}{rgb}{0.730889,0.871916,0.156029}%
\pgfsetfillcolor{currentfill}%
\pgfsetlinewidth{0.000000pt}%
\definecolor{currentstroke}{rgb}{0.000000,0.000000,0.000000}%
\pgfsetstrokecolor{currentstroke}%
\pgfsetdash{}{0pt}%
\pgfpathmoveto{\pgfqpoint{4.768000in}{3.986859in}}%
\pgfpathlineto{\pgfqpoint{4.754864in}{4.000000in}}%
\pgfpathlineto{\pgfqpoint{4.727919in}{4.026940in}}%
\pgfpathlineto{\pgfqpoint{4.722508in}{4.032293in}}%
\pgfpathlineto{\pgfqpoint{4.717500in}{4.037333in}}%
\pgfpathlineto{\pgfqpoint{4.687838in}{4.066903in}}%
\pgfpathlineto{\pgfqpoint{4.683791in}{4.070896in}}%
\pgfpathlineto{\pgfqpoint{4.680033in}{4.074667in}}%
\pgfpathlineto{\pgfqpoint{4.647758in}{4.106749in}}%
\pgfpathlineto{\pgfqpoint{4.645015in}{4.109446in}}%
\pgfpathlineto{\pgfqpoint{4.642462in}{4.112000in}}%
\pgfpathlineto{\pgfqpoint{4.625534in}{4.128633in}}%
\pgfpathlineto{\pgfqpoint{4.607677in}{4.146476in}}%
\pgfpathlineto{\pgfqpoint{4.606183in}{4.147942in}}%
\pgfpathlineto{\pgfqpoint{4.604788in}{4.149333in}}%
\pgfpathlineto{\pgfqpoint{4.586661in}{4.167091in}}%
\pgfpathlineto{\pgfqpoint{4.567596in}{4.186087in}}%
\pgfpathlineto{\pgfqpoint{4.567008in}{4.186667in}}%
\pgfpathlineto{\pgfqpoint{4.547731in}{4.205497in}}%
\pgfpathlineto{\pgfqpoint{4.529106in}{4.224000in}}%
\pgfpathlineto{\pgfqpoint{4.527515in}{4.224000in}}%
\pgfpathlineto{\pgfqpoint{4.526607in}{4.224000in}}%
\pgfpathlineto{\pgfqpoint{4.527515in}{4.223107in}}%
\pgfpathlineto{\pgfqpoint{4.546446in}{4.204300in}}%
\pgfpathlineto{\pgfqpoint{4.564498in}{4.186667in}}%
\pgfpathlineto{\pgfqpoint{4.567596in}{4.183612in}}%
\pgfpathlineto{\pgfqpoint{4.585377in}{4.165896in}}%
\pgfpathlineto{\pgfqpoint{4.602284in}{4.149333in}}%
\pgfpathlineto{\pgfqpoint{4.604888in}{4.146736in}}%
\pgfpathlineto{\pgfqpoint{4.607677in}{4.144000in}}%
\pgfpathlineto{\pgfqpoint{4.624251in}{4.127438in}}%
\pgfpathlineto{\pgfqpoint{4.639964in}{4.112000in}}%
\pgfpathlineto{\pgfqpoint{4.643722in}{4.108241in}}%
\pgfpathlineto{\pgfqpoint{4.647758in}{4.104271in}}%
\pgfpathlineto{\pgfqpoint{4.677541in}{4.074667in}}%
\pgfpathlineto{\pgfqpoint{4.682498in}{4.069692in}}%
\pgfpathlineto{\pgfqpoint{4.687838in}{4.064424in}}%
\pgfpathlineto{\pgfqpoint{4.715013in}{4.037333in}}%
\pgfpathlineto{\pgfqpoint{4.721217in}{4.031090in}}%
\pgfpathlineto{\pgfqpoint{4.727919in}{4.024459in}}%
\pgfpathlineto{\pgfqpoint{4.752383in}{4.000000in}}%
\pgfpathlineto{\pgfqpoint{4.768000in}{3.984377in}}%
\pgfusepath{fill}%
\end{pgfscope}%
\begin{pgfscope}%
\pgfpathrectangle{\pgfqpoint{0.800000in}{0.528000in}}{\pgfqpoint{3.968000in}{3.696000in}}%
\pgfusepath{clip}%
\pgfsetbuttcap%
\pgfsetroundjoin%
\definecolor{currentfill}{rgb}{0.730889,0.871916,0.156029}%
\pgfsetfillcolor{currentfill}%
\pgfsetlinewidth{0.000000pt}%
\definecolor{currentstroke}{rgb}{0.000000,0.000000,0.000000}%
\pgfsetstrokecolor{currentstroke}%
\pgfsetdash{}{0pt}%
\pgfpathmoveto{\pgfqpoint{4.768000in}{3.989341in}}%
\pgfpathlineto{\pgfqpoint{4.757345in}{4.000000in}}%
\pgfpathlineto{\pgfqpoint{4.727919in}{4.029420in}}%
\pgfpathlineto{\pgfqpoint{4.723800in}{4.033496in}}%
\pgfpathlineto{\pgfqpoint{4.719987in}{4.037333in}}%
\pgfpathlineto{\pgfqpoint{4.687838in}{4.069382in}}%
\pgfpathlineto{\pgfqpoint{4.685083in}{4.072100in}}%
\pgfpathlineto{\pgfqpoint{4.682525in}{4.074667in}}%
\pgfpathlineto{\pgfqpoint{4.647758in}{4.109226in}}%
\pgfpathlineto{\pgfqpoint{4.646309in}{4.110651in}}%
\pgfpathlineto{\pgfqpoint{4.644960in}{4.112000in}}%
\pgfpathlineto{\pgfqpoint{4.626816in}{4.129827in}}%
\pgfpathlineto{\pgfqpoint{4.607677in}{4.148952in}}%
\pgfpathlineto{\pgfqpoint{4.607477in}{4.149148in}}%
\pgfpathlineto{\pgfqpoint{4.607291in}{4.149333in}}%
\pgfpathlineto{\pgfqpoint{4.587944in}{4.168287in}}%
\pgfpathlineto{\pgfqpoint{4.569497in}{4.186667in}}%
\pgfpathlineto{\pgfqpoint{4.567596in}{4.188544in}}%
\pgfpathlineto{\pgfqpoint{4.549015in}{4.206693in}}%
\pgfpathlineto{\pgfqpoint{4.531595in}{4.224000in}}%
\pgfpathlineto{\pgfqpoint{4.529106in}{4.224000in}}%
\pgfpathlineto{\pgfqpoint{4.547731in}{4.205497in}}%
\pgfpathlineto{\pgfqpoint{4.567008in}{4.186667in}}%
\pgfpathlineto{\pgfqpoint{4.567596in}{4.186087in}}%
\pgfpathlineto{\pgfqpoint{4.586661in}{4.167091in}}%
\pgfpathlineto{\pgfqpoint{4.604788in}{4.149333in}}%
\pgfpathlineto{\pgfqpoint{4.606183in}{4.147942in}}%
\pgfpathlineto{\pgfqpoint{4.607677in}{4.146476in}}%
\pgfpathlineto{\pgfqpoint{4.625534in}{4.128633in}}%
\pgfpathlineto{\pgfqpoint{4.642462in}{4.112000in}}%
\pgfpathlineto{\pgfqpoint{4.645015in}{4.109446in}}%
\pgfpathlineto{\pgfqpoint{4.647758in}{4.106749in}}%
\pgfpathlineto{\pgfqpoint{4.680033in}{4.074667in}}%
\pgfpathlineto{\pgfqpoint{4.683791in}{4.070896in}}%
\pgfpathlineto{\pgfqpoint{4.687838in}{4.066903in}}%
\pgfpathlineto{\pgfqpoint{4.717500in}{4.037333in}}%
\pgfpathlineto{\pgfqpoint{4.722508in}{4.032293in}}%
\pgfpathlineto{\pgfqpoint{4.727919in}{4.026940in}}%
\pgfpathlineto{\pgfqpoint{4.754864in}{4.000000in}}%
\pgfpathlineto{\pgfqpoint{4.768000in}{3.986859in}}%
\pgfusepath{fill}%
\end{pgfscope}%
\begin{pgfscope}%
\pgfpathrectangle{\pgfqpoint{0.800000in}{0.528000in}}{\pgfqpoint{3.968000in}{3.696000in}}%
\pgfusepath{clip}%
\pgfsetbuttcap%
\pgfsetroundjoin%
\definecolor{currentfill}{rgb}{0.730889,0.871916,0.156029}%
\pgfsetfillcolor{currentfill}%
\pgfsetlinewidth{0.000000pt}%
\definecolor{currentstroke}{rgb}{0.000000,0.000000,0.000000}%
\pgfsetstrokecolor{currentstroke}%
\pgfsetdash{}{0pt}%
\pgfpathmoveto{\pgfqpoint{4.768000in}{3.991823in}}%
\pgfpathlineto{\pgfqpoint{4.759826in}{4.000000in}}%
\pgfpathlineto{\pgfqpoint{4.727919in}{4.031901in}}%
\pgfpathlineto{\pgfqpoint{4.725091in}{4.034699in}}%
\pgfpathlineto{\pgfqpoint{4.722473in}{4.037333in}}%
\pgfpathlineto{\pgfqpoint{4.687838in}{4.071861in}}%
\pgfpathlineto{\pgfqpoint{4.686376in}{4.073304in}}%
\pgfpathlineto{\pgfqpoint{4.685018in}{4.074667in}}%
\pgfpathlineto{\pgfqpoint{4.647758in}{4.111703in}}%
\pgfpathlineto{\pgfqpoint{4.647603in}{4.111856in}}%
\pgfpathlineto{\pgfqpoint{4.647458in}{4.112000in}}%
\pgfpathlineto{\pgfqpoint{4.628098in}{4.131022in}}%
\pgfpathlineto{\pgfqpoint{4.609773in}{4.149333in}}%
\pgfpathlineto{\pgfqpoint{4.607677in}{4.151409in}}%
\pgfpathlineto{\pgfqpoint{4.589228in}{4.169482in}}%
\pgfpathlineto{\pgfqpoint{4.571981in}{4.186667in}}%
\pgfpathlineto{\pgfqpoint{4.567596in}{4.190995in}}%
\pgfpathlineto{\pgfqpoint{4.550300in}{4.207890in}}%
\pgfpathlineto{\pgfqpoint{4.534084in}{4.224000in}}%
\pgfpathlineto{\pgfqpoint{4.531595in}{4.224000in}}%
\pgfpathlineto{\pgfqpoint{4.549015in}{4.206693in}}%
\pgfpathlineto{\pgfqpoint{4.567596in}{4.188544in}}%
\pgfpathlineto{\pgfqpoint{4.569497in}{4.186667in}}%
\pgfpathlineto{\pgfqpoint{4.587944in}{4.168287in}}%
\pgfpathlineto{\pgfqpoint{4.607291in}{4.149333in}}%
\pgfpathlineto{\pgfqpoint{4.607477in}{4.149148in}}%
\pgfpathlineto{\pgfqpoint{4.607677in}{4.148952in}}%
\pgfpathlineto{\pgfqpoint{4.626816in}{4.129827in}}%
\pgfpathlineto{\pgfqpoint{4.644960in}{4.112000in}}%
\pgfpathlineto{\pgfqpoint{4.646309in}{4.110651in}}%
\pgfpathlineto{\pgfqpoint{4.647758in}{4.109226in}}%
\pgfpathlineto{\pgfqpoint{4.682525in}{4.074667in}}%
\pgfpathlineto{\pgfqpoint{4.685083in}{4.072100in}}%
\pgfpathlineto{\pgfqpoint{4.687838in}{4.069382in}}%
\pgfpathlineto{\pgfqpoint{4.719987in}{4.037333in}}%
\pgfpathlineto{\pgfqpoint{4.723800in}{4.033496in}}%
\pgfpathlineto{\pgfqpoint{4.727919in}{4.029420in}}%
\pgfpathlineto{\pgfqpoint{4.757345in}{4.000000in}}%
\pgfpathlineto{\pgfqpoint{4.768000in}{3.989341in}}%
\pgfusepath{fill}%
\end{pgfscope}%
\begin{pgfscope}%
\pgfpathrectangle{\pgfqpoint{0.800000in}{0.528000in}}{\pgfqpoint{3.968000in}{3.696000in}}%
\pgfusepath{clip}%
\pgfsetbuttcap%
\pgfsetroundjoin%
\definecolor{currentfill}{rgb}{0.741388,0.873449,0.149561}%
\pgfsetfillcolor{currentfill}%
\pgfsetlinewidth{0.000000pt}%
\definecolor{currentstroke}{rgb}{0.000000,0.000000,0.000000}%
\pgfsetstrokecolor{currentstroke}%
\pgfsetdash{}{0pt}%
\pgfpathmoveto{\pgfqpoint{4.768000in}{3.994305in}}%
\pgfpathlineto{\pgfqpoint{4.762307in}{4.000000in}}%
\pgfpathlineto{\pgfqpoint{4.727919in}{4.034381in}}%
\pgfpathlineto{\pgfqpoint{4.726382in}{4.035902in}}%
\pgfpathlineto{\pgfqpoint{4.724960in}{4.037333in}}%
\pgfpathlineto{\pgfqpoint{4.687838in}{4.074340in}}%
\pgfpathlineto{\pgfqpoint{4.687668in}{4.074508in}}%
\pgfpathlineto{\pgfqpoint{4.687510in}{4.074667in}}%
\pgfpathlineto{\pgfqpoint{4.682620in}{4.079528in}}%
\pgfpathlineto{\pgfqpoint{4.649934in}{4.112000in}}%
\pgfpathlineto{\pgfqpoint{4.647758in}{4.114161in}}%
\pgfpathlineto{\pgfqpoint{4.629381in}{4.132216in}}%
\pgfpathlineto{\pgfqpoint{4.612251in}{4.149333in}}%
\pgfpathlineto{\pgfqpoint{4.607677in}{4.153862in}}%
\pgfpathlineto{\pgfqpoint{4.590511in}{4.170678in}}%
\pgfpathlineto{\pgfqpoint{4.574464in}{4.186667in}}%
\pgfpathlineto{\pgfqpoint{4.567596in}{4.193446in}}%
\pgfpathlineto{\pgfqpoint{4.551585in}{4.209086in}}%
\pgfpathlineto{\pgfqpoint{4.536573in}{4.224000in}}%
\pgfpathlineto{\pgfqpoint{4.534084in}{4.224000in}}%
\pgfpathlineto{\pgfqpoint{4.550300in}{4.207890in}}%
\pgfpathlineto{\pgfqpoint{4.567596in}{4.190995in}}%
\pgfpathlineto{\pgfqpoint{4.571981in}{4.186667in}}%
\pgfpathlineto{\pgfqpoint{4.589228in}{4.169482in}}%
\pgfpathlineto{\pgfqpoint{4.607677in}{4.151409in}}%
\pgfpathlineto{\pgfqpoint{4.609773in}{4.149333in}}%
\pgfpathlineto{\pgfqpoint{4.628098in}{4.131022in}}%
\pgfpathlineto{\pgfqpoint{4.647458in}{4.112000in}}%
\pgfpathlineto{\pgfqpoint{4.647603in}{4.111856in}}%
\pgfpathlineto{\pgfqpoint{4.647758in}{4.111703in}}%
\pgfpathlineto{\pgfqpoint{4.685018in}{4.074667in}}%
\pgfpathlineto{\pgfqpoint{4.686376in}{4.073304in}}%
\pgfpathlineto{\pgfqpoint{4.687838in}{4.071861in}}%
\pgfpathlineto{\pgfqpoint{4.722473in}{4.037333in}}%
\pgfpathlineto{\pgfqpoint{4.725091in}{4.034699in}}%
\pgfpathlineto{\pgfqpoint{4.727919in}{4.031901in}}%
\pgfpathlineto{\pgfqpoint{4.759826in}{4.000000in}}%
\pgfpathlineto{\pgfqpoint{4.768000in}{3.991823in}}%
\pgfusepath{fill}%
\end{pgfscope}%
\begin{pgfscope}%
\pgfpathrectangle{\pgfqpoint{0.800000in}{0.528000in}}{\pgfqpoint{3.968000in}{3.696000in}}%
\pgfusepath{clip}%
\pgfsetbuttcap%
\pgfsetroundjoin%
\definecolor{currentfill}{rgb}{0.741388,0.873449,0.149561}%
\pgfsetfillcolor{currentfill}%
\pgfsetlinewidth{0.000000pt}%
\definecolor{currentstroke}{rgb}{0.000000,0.000000,0.000000}%
\pgfsetstrokecolor{currentstroke}%
\pgfsetdash{}{0pt}%
\pgfpathmoveto{\pgfqpoint{4.768000in}{3.996787in}}%
\pgfpathlineto{\pgfqpoint{4.764788in}{4.000000in}}%
\pgfpathlineto{\pgfqpoint{4.727919in}{4.036862in}}%
\pgfpathlineto{\pgfqpoint{4.727674in}{4.037105in}}%
\pgfpathlineto{\pgfqpoint{4.727447in}{4.037333in}}%
\pgfpathlineto{\pgfqpoint{4.720721in}{4.044038in}}%
\pgfpathlineto{\pgfqpoint{4.689980in}{4.074667in}}%
\pgfpathlineto{\pgfqpoint{4.688940in}{4.075693in}}%
\pgfpathlineto{\pgfqpoint{4.687838in}{4.076799in}}%
\pgfpathlineto{\pgfqpoint{4.652406in}{4.112000in}}%
\pgfpathlineto{\pgfqpoint{4.647758in}{4.116615in}}%
\pgfpathlineto{\pgfqpoint{4.630663in}{4.133411in}}%
\pgfpathlineto{\pgfqpoint{4.614729in}{4.149333in}}%
\pgfpathlineto{\pgfqpoint{4.607677in}{4.156315in}}%
\pgfpathlineto{\pgfqpoint{4.591795in}{4.171873in}}%
\pgfpathlineto{\pgfqpoint{4.576948in}{4.186667in}}%
\pgfpathlineto{\pgfqpoint{4.567596in}{4.195898in}}%
\pgfpathlineto{\pgfqpoint{4.552869in}{4.210283in}}%
\pgfpathlineto{\pgfqpoint{4.539062in}{4.224000in}}%
\pgfpathlineto{\pgfqpoint{4.536573in}{4.224000in}}%
\pgfpathlineto{\pgfqpoint{4.551585in}{4.209086in}}%
\pgfpathlineto{\pgfqpoint{4.567596in}{4.193446in}}%
\pgfpathlineto{\pgfqpoint{4.574464in}{4.186667in}}%
\pgfpathlineto{\pgfqpoint{4.590511in}{4.170678in}}%
\pgfpathlineto{\pgfqpoint{4.607677in}{4.153862in}}%
\pgfpathlineto{\pgfqpoint{4.612251in}{4.149333in}}%
\pgfpathlineto{\pgfqpoint{4.629381in}{4.132216in}}%
\pgfpathlineto{\pgfqpoint{4.647758in}{4.114161in}}%
\pgfpathlineto{\pgfqpoint{4.649934in}{4.112000in}}%
\pgfpathlineto{\pgfqpoint{4.682620in}{4.079528in}}%
\pgfpathlineto{\pgfqpoint{4.687510in}{4.074667in}}%
\pgfpathlineto{\pgfqpoint{4.687668in}{4.074508in}}%
\pgfpathlineto{\pgfqpoint{4.687838in}{4.074340in}}%
\pgfpathlineto{\pgfqpoint{4.724960in}{4.037333in}}%
\pgfpathlineto{\pgfqpoint{4.726382in}{4.035902in}}%
\pgfpathlineto{\pgfqpoint{4.727919in}{4.034381in}}%
\pgfpathlineto{\pgfqpoint{4.762307in}{4.000000in}}%
\pgfpathlineto{\pgfqpoint{4.768000in}{3.994305in}}%
\pgfusepath{fill}%
\end{pgfscope}%
\begin{pgfscope}%
\pgfpathrectangle{\pgfqpoint{0.800000in}{0.528000in}}{\pgfqpoint{3.968000in}{3.696000in}}%
\pgfusepath{clip}%
\pgfsetbuttcap%
\pgfsetroundjoin%
\definecolor{currentfill}{rgb}{0.741388,0.873449,0.149561}%
\pgfsetfillcolor{currentfill}%
\pgfsetlinewidth{0.000000pt}%
\definecolor{currentstroke}{rgb}{0.000000,0.000000,0.000000}%
\pgfsetstrokecolor{currentstroke}%
\pgfsetdash{}{0pt}%
\pgfpathmoveto{\pgfqpoint{4.768000in}{3.999269in}}%
\pgfpathlineto{\pgfqpoint{4.767269in}{4.000000in}}%
\pgfpathlineto{\pgfqpoint{4.757309in}{4.009958in}}%
\pgfpathlineto{\pgfqpoint{4.729912in}{4.037333in}}%
\pgfpathlineto{\pgfqpoint{4.728946in}{4.038290in}}%
\pgfpathlineto{\pgfqpoint{4.727919in}{4.039324in}}%
\pgfpathlineto{\pgfqpoint{4.692446in}{4.074667in}}%
\pgfpathlineto{\pgfqpoint{4.690209in}{4.076875in}}%
\pgfpathlineto{\pgfqpoint{4.687838in}{4.079255in}}%
\pgfpathlineto{\pgfqpoint{4.654878in}{4.112000in}}%
\pgfpathlineto{\pgfqpoint{4.647758in}{4.119069in}}%
\pgfpathlineto{\pgfqpoint{4.631946in}{4.134605in}}%
\pgfpathlineto{\pgfqpoint{4.617206in}{4.149333in}}%
\pgfpathlineto{\pgfqpoint{4.607677in}{4.158768in}}%
\pgfpathlineto{\pgfqpoint{4.593078in}{4.173069in}}%
\pgfpathlineto{\pgfqpoint{4.579431in}{4.186667in}}%
\pgfpathlineto{\pgfqpoint{4.567596in}{4.198349in}}%
\pgfpathlineto{\pgfqpoint{4.554154in}{4.211479in}}%
\pgfpathlineto{\pgfqpoint{4.541551in}{4.224000in}}%
\pgfpathlineto{\pgfqpoint{4.539062in}{4.224000in}}%
\pgfpathlineto{\pgfqpoint{4.552869in}{4.210283in}}%
\pgfpathlineto{\pgfqpoint{4.567596in}{4.195898in}}%
\pgfpathlineto{\pgfqpoint{4.576948in}{4.186667in}}%
\pgfpathlineto{\pgfqpoint{4.591795in}{4.171873in}}%
\pgfpathlineto{\pgfqpoint{4.607677in}{4.156315in}}%
\pgfpathlineto{\pgfqpoint{4.614729in}{4.149333in}}%
\pgfpathlineto{\pgfqpoint{4.630663in}{4.133411in}}%
\pgfpathlineto{\pgfqpoint{4.647758in}{4.116615in}}%
\pgfpathlineto{\pgfqpoint{4.652406in}{4.112000in}}%
\pgfpathlineto{\pgfqpoint{4.687838in}{4.076799in}}%
\pgfpathlineto{\pgfqpoint{4.688940in}{4.075693in}}%
\pgfpathlineto{\pgfqpoint{4.689980in}{4.074667in}}%
\pgfpathlineto{\pgfqpoint{4.720721in}{4.044038in}}%
\pgfpathlineto{\pgfqpoint{4.727447in}{4.037333in}}%
\pgfpathlineto{\pgfqpoint{4.727674in}{4.037105in}}%
\pgfpathlineto{\pgfqpoint{4.727919in}{4.036862in}}%
\pgfpathlineto{\pgfqpoint{4.764788in}{4.000000in}}%
\pgfpathlineto{\pgfqpoint{4.768000in}{3.996787in}}%
\pgfusepath{fill}%
\end{pgfscope}%
\begin{pgfscope}%
\pgfpathrectangle{\pgfqpoint{0.800000in}{0.528000in}}{\pgfqpoint{3.968000in}{3.696000in}}%
\pgfusepath{clip}%
\pgfsetbuttcap%
\pgfsetroundjoin%
\definecolor{currentfill}{rgb}{0.741388,0.873449,0.149561}%
\pgfsetfillcolor{currentfill}%
\pgfsetlinewidth{0.000000pt}%
\definecolor{currentstroke}{rgb}{0.000000,0.000000,0.000000}%
\pgfsetstrokecolor{currentstroke}%
\pgfsetdash{}{0pt}%
\pgfpathmoveto{\pgfqpoint{4.768000in}{4.001735in}}%
\pgfpathlineto{\pgfqpoint{4.732373in}{4.037333in}}%
\pgfpathlineto{\pgfqpoint{4.730214in}{4.039471in}}%
\pgfpathlineto{\pgfqpoint{4.727919in}{4.041781in}}%
\pgfpathlineto{\pgfqpoint{4.694913in}{4.074667in}}%
\pgfpathlineto{\pgfqpoint{4.691478in}{4.078057in}}%
\pgfpathlineto{\pgfqpoint{4.687838in}{4.081711in}}%
\pgfpathlineto{\pgfqpoint{4.657350in}{4.112000in}}%
\pgfpathlineto{\pgfqpoint{4.647758in}{4.121524in}}%
\pgfpathlineto{\pgfqpoint{4.633228in}{4.135800in}}%
\pgfpathlineto{\pgfqpoint{4.619684in}{4.149333in}}%
\pgfpathlineto{\pgfqpoint{4.607677in}{4.161220in}}%
\pgfpathlineto{\pgfqpoint{4.594362in}{4.174265in}}%
\pgfpathlineto{\pgfqpoint{4.581914in}{4.186667in}}%
\pgfpathlineto{\pgfqpoint{4.567596in}{4.200801in}}%
\pgfpathlineto{\pgfqpoint{4.555439in}{4.212676in}}%
\pgfpathlineto{\pgfqpoint{4.544040in}{4.224000in}}%
\pgfpathlineto{\pgfqpoint{4.541551in}{4.224000in}}%
\pgfpathlineto{\pgfqpoint{4.554154in}{4.211479in}}%
\pgfpathlineto{\pgfqpoint{4.567596in}{4.198349in}}%
\pgfpathlineto{\pgfqpoint{4.579431in}{4.186667in}}%
\pgfpathlineto{\pgfqpoint{4.593078in}{4.173069in}}%
\pgfpathlineto{\pgfqpoint{4.607677in}{4.158768in}}%
\pgfpathlineto{\pgfqpoint{4.617206in}{4.149333in}}%
\pgfpathlineto{\pgfqpoint{4.631946in}{4.134605in}}%
\pgfpathlineto{\pgfqpoint{4.647758in}{4.119069in}}%
\pgfpathlineto{\pgfqpoint{4.654878in}{4.112000in}}%
\pgfpathlineto{\pgfqpoint{4.687838in}{4.079255in}}%
\pgfpathlineto{\pgfqpoint{4.690209in}{4.076875in}}%
\pgfpathlineto{\pgfqpoint{4.692446in}{4.074667in}}%
\pgfpathlineto{\pgfqpoint{4.727919in}{4.039324in}}%
\pgfpathlineto{\pgfqpoint{4.728946in}{4.038290in}}%
\pgfpathlineto{\pgfqpoint{4.729912in}{4.037333in}}%
\pgfpathlineto{\pgfqpoint{4.757309in}{4.009958in}}%
\pgfpathlineto{\pgfqpoint{4.767269in}{4.000000in}}%
\pgfpathlineto{\pgfqpoint{4.768000in}{3.999269in}}%
\pgfpathlineto{\pgfqpoint{4.768000in}{4.000000in}}%
\pgfusepath{fill}%
\end{pgfscope}%
\begin{pgfscope}%
\pgfpathrectangle{\pgfqpoint{0.800000in}{0.528000in}}{\pgfqpoint{3.968000in}{3.696000in}}%
\pgfusepath{clip}%
\pgfsetbuttcap%
\pgfsetroundjoin%
\definecolor{currentfill}{rgb}{0.751884,0.874951,0.143228}%
\pgfsetfillcolor{currentfill}%
\pgfsetlinewidth{0.000000pt}%
\definecolor{currentstroke}{rgb}{0.000000,0.000000,0.000000}%
\pgfsetstrokecolor{currentstroke}%
\pgfsetdash{}{0pt}%
\pgfpathmoveto{\pgfqpoint{4.768000in}{4.004193in}}%
\pgfpathlineto{\pgfqpoint{4.734834in}{4.037333in}}%
\pgfpathlineto{\pgfqpoint{4.731482in}{4.040652in}}%
\pgfpathlineto{\pgfqpoint{4.727919in}{4.044239in}}%
\pgfpathlineto{\pgfqpoint{4.697379in}{4.074667in}}%
\pgfpathlineto{\pgfqpoint{4.692747in}{4.079239in}}%
\pgfpathlineto{\pgfqpoint{4.687838in}{4.084167in}}%
\pgfpathlineto{\pgfqpoint{4.659822in}{4.112000in}}%
\pgfpathlineto{\pgfqpoint{4.647758in}{4.123978in}}%
\pgfpathlineto{\pgfqpoint{4.634510in}{4.136994in}}%
\pgfpathlineto{\pgfqpoint{4.622162in}{4.149333in}}%
\pgfpathlineto{\pgfqpoint{4.607677in}{4.163673in}}%
\pgfpathlineto{\pgfqpoint{4.595645in}{4.175460in}}%
\pgfpathlineto{\pgfqpoint{4.584398in}{4.186667in}}%
\pgfpathlineto{\pgfqpoint{4.567596in}{4.203252in}}%
\pgfpathlineto{\pgfqpoint{4.556723in}{4.213873in}}%
\pgfpathlineto{\pgfqpoint{4.546529in}{4.224000in}}%
\pgfpathlineto{\pgfqpoint{4.544040in}{4.224000in}}%
\pgfpathlineto{\pgfqpoint{4.555439in}{4.212676in}}%
\pgfpathlineto{\pgfqpoint{4.567596in}{4.200801in}}%
\pgfpathlineto{\pgfqpoint{4.581914in}{4.186667in}}%
\pgfpathlineto{\pgfqpoint{4.594362in}{4.174265in}}%
\pgfpathlineto{\pgfqpoint{4.607677in}{4.161220in}}%
\pgfpathlineto{\pgfqpoint{4.619684in}{4.149333in}}%
\pgfpathlineto{\pgfqpoint{4.633228in}{4.135800in}}%
\pgfpathlineto{\pgfqpoint{4.647758in}{4.121524in}}%
\pgfpathlineto{\pgfqpoint{4.657350in}{4.112000in}}%
\pgfpathlineto{\pgfqpoint{4.687838in}{4.081711in}}%
\pgfpathlineto{\pgfqpoint{4.691478in}{4.078057in}}%
\pgfpathlineto{\pgfqpoint{4.694913in}{4.074667in}}%
\pgfpathlineto{\pgfqpoint{4.727919in}{4.041781in}}%
\pgfpathlineto{\pgfqpoint{4.730214in}{4.039471in}}%
\pgfpathlineto{\pgfqpoint{4.732373in}{4.037333in}}%
\pgfpathlineto{\pgfqpoint{4.768000in}{4.001735in}}%
\pgfusepath{fill}%
\end{pgfscope}%
\begin{pgfscope}%
\pgfpathrectangle{\pgfqpoint{0.800000in}{0.528000in}}{\pgfqpoint{3.968000in}{3.696000in}}%
\pgfusepath{clip}%
\pgfsetbuttcap%
\pgfsetroundjoin%
\definecolor{currentfill}{rgb}{0.751884,0.874951,0.143228}%
\pgfsetfillcolor{currentfill}%
\pgfsetlinewidth{0.000000pt}%
\definecolor{currentstroke}{rgb}{0.000000,0.000000,0.000000}%
\pgfsetstrokecolor{currentstroke}%
\pgfsetdash{}{0pt}%
\pgfpathmoveto{\pgfqpoint{4.768000in}{4.006652in}}%
\pgfpathlineto{\pgfqpoint{4.737295in}{4.037333in}}%
\pgfpathlineto{\pgfqpoint{4.732750in}{4.041833in}}%
\pgfpathlineto{\pgfqpoint{4.727919in}{4.046696in}}%
\pgfpathlineto{\pgfqpoint{4.699846in}{4.074667in}}%
\pgfpathlineto{\pgfqpoint{4.694016in}{4.080421in}}%
\pgfpathlineto{\pgfqpoint{4.687838in}{4.086623in}}%
\pgfpathlineto{\pgfqpoint{4.662294in}{4.112000in}}%
\pgfpathlineto{\pgfqpoint{4.647758in}{4.126433in}}%
\pgfpathlineto{\pgfqpoint{4.635793in}{4.138189in}}%
\pgfpathlineto{\pgfqpoint{4.624640in}{4.149333in}}%
\pgfpathlineto{\pgfqpoint{4.607677in}{4.166126in}}%
\pgfpathlineto{\pgfqpoint{4.596929in}{4.176656in}}%
\pgfpathlineto{\pgfqpoint{4.586881in}{4.186667in}}%
\pgfpathlineto{\pgfqpoint{4.567596in}{4.205704in}}%
\pgfpathlineto{\pgfqpoint{4.558008in}{4.215069in}}%
\pgfpathlineto{\pgfqpoint{4.549018in}{4.224000in}}%
\pgfpathlineto{\pgfqpoint{4.546529in}{4.224000in}}%
\pgfpathlineto{\pgfqpoint{4.556723in}{4.213873in}}%
\pgfpathlineto{\pgfqpoint{4.567596in}{4.203252in}}%
\pgfpathlineto{\pgfqpoint{4.584398in}{4.186667in}}%
\pgfpathlineto{\pgfqpoint{4.595645in}{4.175460in}}%
\pgfpathlineto{\pgfqpoint{4.607677in}{4.163673in}}%
\pgfpathlineto{\pgfqpoint{4.622162in}{4.149333in}}%
\pgfpathlineto{\pgfqpoint{4.634510in}{4.136994in}}%
\pgfpathlineto{\pgfqpoint{4.647758in}{4.123978in}}%
\pgfpathlineto{\pgfqpoint{4.659822in}{4.112000in}}%
\pgfpathlineto{\pgfqpoint{4.687838in}{4.084167in}}%
\pgfpathlineto{\pgfqpoint{4.692747in}{4.079239in}}%
\pgfpathlineto{\pgfqpoint{4.697379in}{4.074667in}}%
\pgfpathlineto{\pgfqpoint{4.727919in}{4.044239in}}%
\pgfpathlineto{\pgfqpoint{4.731482in}{4.040652in}}%
\pgfpathlineto{\pgfqpoint{4.734834in}{4.037333in}}%
\pgfpathlineto{\pgfqpoint{4.768000in}{4.004193in}}%
\pgfusepath{fill}%
\end{pgfscope}%
\begin{pgfscope}%
\pgfpathrectangle{\pgfqpoint{0.800000in}{0.528000in}}{\pgfqpoint{3.968000in}{3.696000in}}%
\pgfusepath{clip}%
\pgfsetbuttcap%
\pgfsetroundjoin%
\definecolor{currentfill}{rgb}{0.751884,0.874951,0.143228}%
\pgfsetfillcolor{currentfill}%
\pgfsetlinewidth{0.000000pt}%
\definecolor{currentstroke}{rgb}{0.000000,0.000000,0.000000}%
\pgfsetstrokecolor{currentstroke}%
\pgfsetdash{}{0pt}%
\pgfpathmoveto{\pgfqpoint{4.768000in}{4.009111in}}%
\pgfpathlineto{\pgfqpoint{4.739756in}{4.037333in}}%
\pgfpathlineto{\pgfqpoint{4.734018in}{4.043014in}}%
\pgfpathlineto{\pgfqpoint{4.727919in}{4.049154in}}%
\pgfpathlineto{\pgfqpoint{4.702312in}{4.074667in}}%
\pgfpathlineto{\pgfqpoint{4.695286in}{4.081603in}}%
\pgfpathlineto{\pgfqpoint{4.687838in}{4.089079in}}%
\pgfpathlineto{\pgfqpoint{4.664766in}{4.112000in}}%
\pgfpathlineto{\pgfqpoint{4.647758in}{4.128887in}}%
\pgfpathlineto{\pgfqpoint{4.637075in}{4.139383in}}%
\pgfpathlineto{\pgfqpoint{4.627117in}{4.149333in}}%
\pgfpathlineto{\pgfqpoint{4.607677in}{4.168579in}}%
\pgfpathlineto{\pgfqpoint{4.598212in}{4.177851in}}%
\pgfpathlineto{\pgfqpoint{4.589365in}{4.186667in}}%
\pgfpathlineto{\pgfqpoint{4.567596in}{4.208155in}}%
\pgfpathlineto{\pgfqpoint{4.559293in}{4.216266in}}%
\pgfpathlineto{\pgfqpoint{4.551507in}{4.224000in}}%
\pgfpathlineto{\pgfqpoint{4.549018in}{4.224000in}}%
\pgfpathlineto{\pgfqpoint{4.558008in}{4.215069in}}%
\pgfpathlineto{\pgfqpoint{4.567596in}{4.205704in}}%
\pgfpathlineto{\pgfqpoint{4.586881in}{4.186667in}}%
\pgfpathlineto{\pgfqpoint{4.596929in}{4.176656in}}%
\pgfpathlineto{\pgfqpoint{4.607677in}{4.166126in}}%
\pgfpathlineto{\pgfqpoint{4.624640in}{4.149333in}}%
\pgfpathlineto{\pgfqpoint{4.635793in}{4.138189in}}%
\pgfpathlineto{\pgfqpoint{4.647758in}{4.126433in}}%
\pgfpathlineto{\pgfqpoint{4.662294in}{4.112000in}}%
\pgfpathlineto{\pgfqpoint{4.687838in}{4.086623in}}%
\pgfpathlineto{\pgfqpoint{4.694016in}{4.080421in}}%
\pgfpathlineto{\pgfqpoint{4.699846in}{4.074667in}}%
\pgfpathlineto{\pgfqpoint{4.727919in}{4.046696in}}%
\pgfpathlineto{\pgfqpoint{4.732750in}{4.041833in}}%
\pgfpathlineto{\pgfqpoint{4.737295in}{4.037333in}}%
\pgfpathlineto{\pgfqpoint{4.768000in}{4.006652in}}%
\pgfusepath{fill}%
\end{pgfscope}%
\begin{pgfscope}%
\pgfpathrectangle{\pgfqpoint{0.800000in}{0.528000in}}{\pgfqpoint{3.968000in}{3.696000in}}%
\pgfusepath{clip}%
\pgfsetbuttcap%
\pgfsetroundjoin%
\definecolor{currentfill}{rgb}{0.751884,0.874951,0.143228}%
\pgfsetfillcolor{currentfill}%
\pgfsetlinewidth{0.000000pt}%
\definecolor{currentstroke}{rgb}{0.000000,0.000000,0.000000}%
\pgfsetstrokecolor{currentstroke}%
\pgfsetdash{}{0pt}%
\pgfpathmoveto{\pgfqpoint{4.768000in}{4.011570in}}%
\pgfpathlineto{\pgfqpoint{4.742217in}{4.037333in}}%
\pgfpathlineto{\pgfqpoint{4.735286in}{4.044195in}}%
\pgfpathlineto{\pgfqpoint{4.727919in}{4.051611in}}%
\pgfpathlineto{\pgfqpoint{4.704779in}{4.074667in}}%
\pgfpathlineto{\pgfqpoint{4.696555in}{4.082785in}}%
\pgfpathlineto{\pgfqpoint{4.687838in}{4.091535in}}%
\pgfpathlineto{\pgfqpoint{4.667238in}{4.112000in}}%
\pgfpathlineto{\pgfqpoint{4.647758in}{4.131342in}}%
\pgfpathlineto{\pgfqpoint{4.638357in}{4.140578in}}%
\pgfpathlineto{\pgfqpoint{4.629595in}{4.149333in}}%
\pgfpathlineto{\pgfqpoint{4.607677in}{4.171032in}}%
\pgfpathlineto{\pgfqpoint{4.599496in}{4.179047in}}%
\pgfpathlineto{\pgfqpoint{4.591848in}{4.186667in}}%
\pgfpathlineto{\pgfqpoint{4.567596in}{4.210606in}}%
\pgfpathlineto{\pgfqpoint{4.560577in}{4.217462in}}%
\pgfpathlineto{\pgfqpoint{4.553997in}{4.224000in}}%
\pgfpathlineto{\pgfqpoint{4.551507in}{4.224000in}}%
\pgfpathlineto{\pgfqpoint{4.559293in}{4.216266in}}%
\pgfpathlineto{\pgfqpoint{4.567596in}{4.208155in}}%
\pgfpathlineto{\pgfqpoint{4.589365in}{4.186667in}}%
\pgfpathlineto{\pgfqpoint{4.598212in}{4.177851in}}%
\pgfpathlineto{\pgfqpoint{4.607677in}{4.168579in}}%
\pgfpathlineto{\pgfqpoint{4.627117in}{4.149333in}}%
\pgfpathlineto{\pgfqpoint{4.637075in}{4.139383in}}%
\pgfpathlineto{\pgfqpoint{4.647758in}{4.128887in}}%
\pgfpathlineto{\pgfqpoint{4.664766in}{4.112000in}}%
\pgfpathlineto{\pgfqpoint{4.687838in}{4.089079in}}%
\pgfpathlineto{\pgfqpoint{4.695286in}{4.081603in}}%
\pgfpathlineto{\pgfqpoint{4.702312in}{4.074667in}}%
\pgfpathlineto{\pgfqpoint{4.727919in}{4.049154in}}%
\pgfpathlineto{\pgfqpoint{4.734018in}{4.043014in}}%
\pgfpathlineto{\pgfqpoint{4.739756in}{4.037333in}}%
\pgfpathlineto{\pgfqpoint{4.768000in}{4.009111in}}%
\pgfusepath{fill}%
\end{pgfscope}%
\begin{pgfscope}%
\pgfpathrectangle{\pgfqpoint{0.800000in}{0.528000in}}{\pgfqpoint{3.968000in}{3.696000in}}%
\pgfusepath{clip}%
\pgfsetbuttcap%
\pgfsetroundjoin%
\definecolor{currentfill}{rgb}{0.762373,0.876424,0.137064}%
\pgfsetfillcolor{currentfill}%
\pgfsetlinewidth{0.000000pt}%
\definecolor{currentstroke}{rgb}{0.000000,0.000000,0.000000}%
\pgfsetstrokecolor{currentstroke}%
\pgfsetdash{}{0pt}%
\pgfpathmoveto{\pgfqpoint{4.768000in}{4.014029in}}%
\pgfpathlineto{\pgfqpoint{4.744678in}{4.037333in}}%
\pgfpathlineto{\pgfqpoint{4.736554in}{4.045376in}}%
\pgfpathlineto{\pgfqpoint{4.727919in}{4.054068in}}%
\pgfpathlineto{\pgfqpoint{4.707245in}{4.074667in}}%
\pgfpathlineto{\pgfqpoint{4.697824in}{4.083967in}}%
\pgfpathlineto{\pgfqpoint{4.687838in}{4.093991in}}%
\pgfpathlineto{\pgfqpoint{4.669711in}{4.112000in}}%
\pgfpathlineto{\pgfqpoint{4.647758in}{4.133796in}}%
\pgfpathlineto{\pgfqpoint{4.639640in}{4.141772in}}%
\pgfpathlineto{\pgfqpoint{4.632073in}{4.149333in}}%
\pgfpathlineto{\pgfqpoint{4.607677in}{4.173485in}}%
\pgfpathlineto{\pgfqpoint{4.600779in}{4.180242in}}%
\pgfpathlineto{\pgfqpoint{4.594331in}{4.186667in}}%
\pgfpathlineto{\pgfqpoint{4.567596in}{4.213058in}}%
\pgfpathlineto{\pgfqpoint{4.561862in}{4.218659in}}%
\pgfpathlineto{\pgfqpoint{4.556486in}{4.224000in}}%
\pgfpathlineto{\pgfqpoint{4.553997in}{4.224000in}}%
\pgfpathlineto{\pgfqpoint{4.560577in}{4.217462in}}%
\pgfpathlineto{\pgfqpoint{4.567596in}{4.210606in}}%
\pgfpathlineto{\pgfqpoint{4.591848in}{4.186667in}}%
\pgfpathlineto{\pgfqpoint{4.599496in}{4.179047in}}%
\pgfpathlineto{\pgfqpoint{4.607677in}{4.171032in}}%
\pgfpathlineto{\pgfqpoint{4.629595in}{4.149333in}}%
\pgfpathlineto{\pgfqpoint{4.638357in}{4.140578in}}%
\pgfpathlineto{\pgfqpoint{4.647758in}{4.131342in}}%
\pgfpathlineto{\pgfqpoint{4.667238in}{4.112000in}}%
\pgfpathlineto{\pgfqpoint{4.687838in}{4.091535in}}%
\pgfpathlineto{\pgfqpoint{4.696555in}{4.082785in}}%
\pgfpathlineto{\pgfqpoint{4.704779in}{4.074667in}}%
\pgfpathlineto{\pgfqpoint{4.727919in}{4.051611in}}%
\pgfpathlineto{\pgfqpoint{4.735286in}{4.044195in}}%
\pgfpathlineto{\pgfqpoint{4.742217in}{4.037333in}}%
\pgfpathlineto{\pgfqpoint{4.768000in}{4.011570in}}%
\pgfusepath{fill}%
\end{pgfscope}%
\begin{pgfscope}%
\pgfpathrectangle{\pgfqpoint{0.800000in}{0.528000in}}{\pgfqpoint{3.968000in}{3.696000in}}%
\pgfusepath{clip}%
\pgfsetbuttcap%
\pgfsetroundjoin%
\definecolor{currentfill}{rgb}{0.762373,0.876424,0.137064}%
\pgfsetfillcolor{currentfill}%
\pgfsetlinewidth{0.000000pt}%
\definecolor{currentstroke}{rgb}{0.000000,0.000000,0.000000}%
\pgfsetstrokecolor{currentstroke}%
\pgfsetdash{}{0pt}%
\pgfpathmoveto{\pgfqpoint{4.768000in}{4.016488in}}%
\pgfpathlineto{\pgfqpoint{4.747139in}{4.037333in}}%
\pgfpathlineto{\pgfqpoint{4.737822in}{4.046557in}}%
\pgfpathlineto{\pgfqpoint{4.727919in}{4.056526in}}%
\pgfpathlineto{\pgfqpoint{4.709712in}{4.074667in}}%
\pgfpathlineto{\pgfqpoint{4.699093in}{4.085150in}}%
\pgfpathlineto{\pgfqpoint{4.687838in}{4.096447in}}%
\pgfpathlineto{\pgfqpoint{4.672183in}{4.112000in}}%
\pgfpathlineto{\pgfqpoint{4.647758in}{4.136251in}}%
\pgfpathlineto{\pgfqpoint{4.640922in}{4.142967in}}%
\pgfpathlineto{\pgfqpoint{4.634551in}{4.149333in}}%
\pgfpathlineto{\pgfqpoint{4.607677in}{4.175938in}}%
\pgfpathlineto{\pgfqpoint{4.602063in}{4.181438in}}%
\pgfpathlineto{\pgfqpoint{4.596815in}{4.186667in}}%
\pgfpathlineto{\pgfqpoint{4.567596in}{4.215509in}}%
\pgfpathlineto{\pgfqpoint{4.563147in}{4.219856in}}%
\pgfpathlineto{\pgfqpoint{4.558975in}{4.224000in}}%
\pgfpathlineto{\pgfqpoint{4.556486in}{4.224000in}}%
\pgfpathlineto{\pgfqpoint{4.561862in}{4.218659in}}%
\pgfpathlineto{\pgfqpoint{4.567596in}{4.213058in}}%
\pgfpathlineto{\pgfqpoint{4.594331in}{4.186667in}}%
\pgfpathlineto{\pgfqpoint{4.600779in}{4.180242in}}%
\pgfpathlineto{\pgfqpoint{4.607677in}{4.173485in}}%
\pgfpathlineto{\pgfqpoint{4.632073in}{4.149333in}}%
\pgfpathlineto{\pgfqpoint{4.639640in}{4.141772in}}%
\pgfpathlineto{\pgfqpoint{4.647758in}{4.133796in}}%
\pgfpathlineto{\pgfqpoint{4.669711in}{4.112000in}}%
\pgfpathlineto{\pgfqpoint{4.687838in}{4.093991in}}%
\pgfpathlineto{\pgfqpoint{4.697824in}{4.083967in}}%
\pgfpathlineto{\pgfqpoint{4.707245in}{4.074667in}}%
\pgfpathlineto{\pgfqpoint{4.727919in}{4.054068in}}%
\pgfpathlineto{\pgfqpoint{4.736554in}{4.045376in}}%
\pgfpathlineto{\pgfqpoint{4.744678in}{4.037333in}}%
\pgfpathlineto{\pgfqpoint{4.768000in}{4.014029in}}%
\pgfusepath{fill}%
\end{pgfscope}%
\begin{pgfscope}%
\pgfpathrectangle{\pgfqpoint{0.800000in}{0.528000in}}{\pgfqpoint{3.968000in}{3.696000in}}%
\pgfusepath{clip}%
\pgfsetbuttcap%
\pgfsetroundjoin%
\definecolor{currentfill}{rgb}{0.762373,0.876424,0.137064}%
\pgfsetfillcolor{currentfill}%
\pgfsetlinewidth{0.000000pt}%
\definecolor{currentstroke}{rgb}{0.000000,0.000000,0.000000}%
\pgfsetstrokecolor{currentstroke}%
\pgfsetdash{}{0pt}%
\pgfpathmoveto{\pgfqpoint{4.768000in}{4.018947in}}%
\pgfpathlineto{\pgfqpoint{4.749600in}{4.037333in}}%
\pgfpathlineto{\pgfqpoint{4.739090in}{4.047738in}}%
\pgfpathlineto{\pgfqpoint{4.727919in}{4.058983in}}%
\pgfpathlineto{\pgfqpoint{4.712178in}{4.074667in}}%
\pgfpathlineto{\pgfqpoint{4.700362in}{4.086332in}}%
\pgfpathlineto{\pgfqpoint{4.687838in}{4.098903in}}%
\pgfpathlineto{\pgfqpoint{4.674655in}{4.112000in}}%
\pgfpathlineto{\pgfqpoint{4.647758in}{4.138705in}}%
\pgfpathlineto{\pgfqpoint{4.642205in}{4.144161in}}%
\pgfpathlineto{\pgfqpoint{4.637028in}{4.149333in}}%
\pgfpathlineto{\pgfqpoint{4.607677in}{4.178391in}}%
\pgfpathlineto{\pgfqpoint{4.603346in}{4.182633in}}%
\pgfpathlineto{\pgfqpoint{4.599298in}{4.186667in}}%
\pgfpathlineto{\pgfqpoint{4.567596in}{4.217961in}}%
\pgfpathlineto{\pgfqpoint{4.564431in}{4.221052in}}%
\pgfpathlineto{\pgfqpoint{4.561464in}{4.224000in}}%
\pgfpathlineto{\pgfqpoint{4.558975in}{4.224000in}}%
\pgfpathlineto{\pgfqpoint{4.563147in}{4.219856in}}%
\pgfpathlineto{\pgfqpoint{4.567596in}{4.215509in}}%
\pgfpathlineto{\pgfqpoint{4.596815in}{4.186667in}}%
\pgfpathlineto{\pgfqpoint{4.602063in}{4.181438in}}%
\pgfpathlineto{\pgfqpoint{4.607677in}{4.175938in}}%
\pgfpathlineto{\pgfqpoint{4.634551in}{4.149333in}}%
\pgfpathlineto{\pgfqpoint{4.640922in}{4.142967in}}%
\pgfpathlineto{\pgfqpoint{4.647758in}{4.136251in}}%
\pgfpathlineto{\pgfqpoint{4.672183in}{4.112000in}}%
\pgfpathlineto{\pgfqpoint{4.687838in}{4.096447in}}%
\pgfpathlineto{\pgfqpoint{4.699093in}{4.085150in}}%
\pgfpathlineto{\pgfqpoint{4.709712in}{4.074667in}}%
\pgfpathlineto{\pgfqpoint{4.727919in}{4.056526in}}%
\pgfpathlineto{\pgfqpoint{4.737822in}{4.046557in}}%
\pgfpathlineto{\pgfqpoint{4.747139in}{4.037333in}}%
\pgfpathlineto{\pgfqpoint{4.768000in}{4.016488in}}%
\pgfusepath{fill}%
\end{pgfscope}%
\begin{pgfscope}%
\pgfpathrectangle{\pgfqpoint{0.800000in}{0.528000in}}{\pgfqpoint{3.968000in}{3.696000in}}%
\pgfusepath{clip}%
\pgfsetbuttcap%
\pgfsetroundjoin%
\definecolor{currentfill}{rgb}{0.772852,0.877868,0.131109}%
\pgfsetfillcolor{currentfill}%
\pgfsetlinewidth{0.000000pt}%
\definecolor{currentstroke}{rgb}{0.000000,0.000000,0.000000}%
\pgfsetstrokecolor{currentstroke}%
\pgfsetdash{}{0pt}%
\pgfpathmoveto{\pgfqpoint{4.768000in}{4.021406in}}%
\pgfpathlineto{\pgfqpoint{4.752060in}{4.037333in}}%
\pgfpathlineto{\pgfqpoint{4.740358in}{4.048919in}}%
\pgfpathlineto{\pgfqpoint{4.727919in}{4.061441in}}%
\pgfpathlineto{\pgfqpoint{4.714645in}{4.074667in}}%
\pgfpathlineto{\pgfqpoint{4.701631in}{4.087514in}}%
\pgfpathlineto{\pgfqpoint{4.687838in}{4.101358in}}%
\pgfpathlineto{\pgfqpoint{4.677127in}{4.112000in}}%
\pgfpathlineto{\pgfqpoint{4.647758in}{4.141159in}}%
\pgfpathlineto{\pgfqpoint{4.643487in}{4.145355in}}%
\pgfpathlineto{\pgfqpoint{4.639506in}{4.149333in}}%
\pgfpathlineto{\pgfqpoint{4.607677in}{4.180844in}}%
\pgfpathlineto{\pgfqpoint{4.604630in}{4.183829in}}%
\pgfpathlineto{\pgfqpoint{4.601782in}{4.186667in}}%
\pgfpathlineto{\pgfqpoint{4.567596in}{4.220412in}}%
\pgfpathlineto{\pgfqpoint{4.565716in}{4.222249in}}%
\pgfpathlineto{\pgfqpoint{4.563953in}{4.224000in}}%
\pgfpathlineto{\pgfqpoint{4.561464in}{4.224000in}}%
\pgfpathlineto{\pgfqpoint{4.564431in}{4.221052in}}%
\pgfpathlineto{\pgfqpoint{4.567596in}{4.217961in}}%
\pgfpathlineto{\pgfqpoint{4.599298in}{4.186667in}}%
\pgfpathlineto{\pgfqpoint{4.603346in}{4.182633in}}%
\pgfpathlineto{\pgfqpoint{4.607677in}{4.178391in}}%
\pgfpathlineto{\pgfqpoint{4.637028in}{4.149333in}}%
\pgfpathlineto{\pgfqpoint{4.642205in}{4.144161in}}%
\pgfpathlineto{\pgfqpoint{4.647758in}{4.138705in}}%
\pgfpathlineto{\pgfqpoint{4.674655in}{4.112000in}}%
\pgfpathlineto{\pgfqpoint{4.687838in}{4.098903in}}%
\pgfpathlineto{\pgfqpoint{4.700362in}{4.086332in}}%
\pgfpathlineto{\pgfqpoint{4.712178in}{4.074667in}}%
\pgfpathlineto{\pgfqpoint{4.727919in}{4.058983in}}%
\pgfpathlineto{\pgfqpoint{4.739090in}{4.047738in}}%
\pgfpathlineto{\pgfqpoint{4.749600in}{4.037333in}}%
\pgfpathlineto{\pgfqpoint{4.768000in}{4.018947in}}%
\pgfusepath{fill}%
\end{pgfscope}%
\begin{pgfscope}%
\pgfpathrectangle{\pgfqpoint{0.800000in}{0.528000in}}{\pgfqpoint{3.968000in}{3.696000in}}%
\pgfusepath{clip}%
\pgfsetbuttcap%
\pgfsetroundjoin%
\definecolor{currentfill}{rgb}{0.772852,0.877868,0.131109}%
\pgfsetfillcolor{currentfill}%
\pgfsetlinewidth{0.000000pt}%
\definecolor{currentstroke}{rgb}{0.000000,0.000000,0.000000}%
\pgfsetstrokecolor{currentstroke}%
\pgfsetdash{}{0pt}%
\pgfpathmoveto{\pgfqpoint{4.768000in}{4.023865in}}%
\pgfpathlineto{\pgfqpoint{4.754521in}{4.037333in}}%
\pgfpathlineto{\pgfqpoint{4.741626in}{4.050100in}}%
\pgfpathlineto{\pgfqpoint{4.727919in}{4.063898in}}%
\pgfpathlineto{\pgfqpoint{4.717111in}{4.074667in}}%
\pgfpathlineto{\pgfqpoint{4.702900in}{4.088696in}}%
\pgfpathlineto{\pgfqpoint{4.687838in}{4.103814in}}%
\pgfpathlineto{\pgfqpoint{4.679599in}{4.112000in}}%
\pgfpathlineto{\pgfqpoint{4.647758in}{4.143614in}}%
\pgfpathlineto{\pgfqpoint{4.644769in}{4.146550in}}%
\pgfpathlineto{\pgfqpoint{4.641984in}{4.149333in}}%
\pgfpathlineto{\pgfqpoint{4.607677in}{4.183297in}}%
\pgfpathlineto{\pgfqpoint{4.605914in}{4.185024in}}%
\pgfpathlineto{\pgfqpoint{4.604265in}{4.186667in}}%
\pgfpathlineto{\pgfqpoint{4.567596in}{4.222864in}}%
\pgfpathlineto{\pgfqpoint{4.567000in}{4.223445in}}%
\pgfpathlineto{\pgfqpoint{4.566442in}{4.224000in}}%
\pgfpathlineto{\pgfqpoint{4.563953in}{4.224000in}}%
\pgfpathlineto{\pgfqpoint{4.565716in}{4.222249in}}%
\pgfpathlineto{\pgfqpoint{4.567596in}{4.220412in}}%
\pgfpathlineto{\pgfqpoint{4.601782in}{4.186667in}}%
\pgfpathlineto{\pgfqpoint{4.604630in}{4.183829in}}%
\pgfpathlineto{\pgfqpoint{4.607677in}{4.180844in}}%
\pgfpathlineto{\pgfqpoint{4.639506in}{4.149333in}}%
\pgfpathlineto{\pgfqpoint{4.643487in}{4.145355in}}%
\pgfpathlineto{\pgfqpoint{4.647758in}{4.141159in}}%
\pgfpathlineto{\pgfqpoint{4.677127in}{4.112000in}}%
\pgfpathlineto{\pgfqpoint{4.687838in}{4.101358in}}%
\pgfpathlineto{\pgfqpoint{4.701631in}{4.087514in}}%
\pgfpathlineto{\pgfqpoint{4.714645in}{4.074667in}}%
\pgfpathlineto{\pgfqpoint{4.727919in}{4.061441in}}%
\pgfpathlineto{\pgfqpoint{4.740358in}{4.048919in}}%
\pgfpathlineto{\pgfqpoint{4.752060in}{4.037333in}}%
\pgfpathlineto{\pgfqpoint{4.768000in}{4.021406in}}%
\pgfusepath{fill}%
\end{pgfscope}%
\begin{pgfscope}%
\pgfpathrectangle{\pgfqpoint{0.800000in}{0.528000in}}{\pgfqpoint{3.968000in}{3.696000in}}%
\pgfusepath{clip}%
\pgfsetbuttcap%
\pgfsetroundjoin%
\definecolor{currentfill}{rgb}{0.772852,0.877868,0.131109}%
\pgfsetfillcolor{currentfill}%
\pgfsetlinewidth{0.000000pt}%
\definecolor{currentstroke}{rgb}{0.000000,0.000000,0.000000}%
\pgfsetstrokecolor{currentstroke}%
\pgfsetdash{}{0pt}%
\pgfpathmoveto{\pgfqpoint{4.768000in}{4.026324in}}%
\pgfpathlineto{\pgfqpoint{4.756982in}{4.037333in}}%
\pgfpathlineto{\pgfqpoint{4.742894in}{4.051281in}}%
\pgfpathlineto{\pgfqpoint{4.727919in}{4.066356in}}%
\pgfpathlineto{\pgfqpoint{4.719578in}{4.074667in}}%
\pgfpathlineto{\pgfqpoint{4.704169in}{4.089878in}}%
\pgfpathlineto{\pgfqpoint{4.687838in}{4.106270in}}%
\pgfpathlineto{\pgfqpoint{4.682071in}{4.112000in}}%
\pgfpathlineto{\pgfqpoint{4.647758in}{4.146068in}}%
\pgfpathlineto{\pgfqpoint{4.646052in}{4.147744in}}%
\pgfpathlineto{\pgfqpoint{4.644462in}{4.149333in}}%
\pgfpathlineto{\pgfqpoint{4.607677in}{4.185750in}}%
\pgfpathlineto{\pgfqpoint{4.607197in}{4.186220in}}%
\pgfpathlineto{\pgfqpoint{4.606749in}{4.186667in}}%
\pgfpathlineto{\pgfqpoint{4.591218in}{4.201998in}}%
\pgfpathlineto{\pgfqpoint{4.568917in}{4.224000in}}%
\pgfpathlineto{\pgfqpoint{4.567596in}{4.224000in}}%
\pgfpathlineto{\pgfqpoint{4.566442in}{4.224000in}}%
\pgfpathlineto{\pgfqpoint{4.567000in}{4.223445in}}%
\pgfpathlineto{\pgfqpoint{4.567596in}{4.222864in}}%
\pgfpathlineto{\pgfqpoint{4.604265in}{4.186667in}}%
\pgfpathlineto{\pgfqpoint{4.605914in}{4.185024in}}%
\pgfpathlineto{\pgfqpoint{4.607677in}{4.183297in}}%
\pgfpathlineto{\pgfqpoint{4.641984in}{4.149333in}}%
\pgfpathlineto{\pgfqpoint{4.644769in}{4.146550in}}%
\pgfpathlineto{\pgfqpoint{4.647758in}{4.143614in}}%
\pgfpathlineto{\pgfqpoint{4.679599in}{4.112000in}}%
\pgfpathlineto{\pgfqpoint{4.687838in}{4.103814in}}%
\pgfpathlineto{\pgfqpoint{4.702900in}{4.088696in}}%
\pgfpathlineto{\pgfqpoint{4.717111in}{4.074667in}}%
\pgfpathlineto{\pgfqpoint{4.727919in}{4.063898in}}%
\pgfpathlineto{\pgfqpoint{4.741626in}{4.050100in}}%
\pgfpathlineto{\pgfqpoint{4.754521in}{4.037333in}}%
\pgfpathlineto{\pgfqpoint{4.768000in}{4.023865in}}%
\pgfusepath{fill}%
\end{pgfscope}%
\begin{pgfscope}%
\pgfpathrectangle{\pgfqpoint{0.800000in}{0.528000in}}{\pgfqpoint{3.968000in}{3.696000in}}%
\pgfusepath{clip}%
\pgfsetbuttcap%
\pgfsetroundjoin%
\definecolor{currentfill}{rgb}{0.772852,0.877868,0.131109}%
\pgfsetfillcolor{currentfill}%
\pgfsetlinewidth{0.000000pt}%
\definecolor{currentstroke}{rgb}{0.000000,0.000000,0.000000}%
\pgfsetstrokecolor{currentstroke}%
\pgfsetdash{}{0pt}%
\pgfpathmoveto{\pgfqpoint{4.768000in}{4.028783in}}%
\pgfpathlineto{\pgfqpoint{4.759443in}{4.037333in}}%
\pgfpathlineto{\pgfqpoint{4.744162in}{4.052462in}}%
\pgfpathlineto{\pgfqpoint{4.727919in}{4.068813in}}%
\pgfpathlineto{\pgfqpoint{4.722044in}{4.074667in}}%
\pgfpathlineto{\pgfqpoint{4.705438in}{4.091060in}}%
\pgfpathlineto{\pgfqpoint{4.687838in}{4.108726in}}%
\pgfpathlineto{\pgfqpoint{4.684543in}{4.112000in}}%
\pgfpathlineto{\pgfqpoint{4.647758in}{4.148523in}}%
\pgfpathlineto{\pgfqpoint{4.647334in}{4.148939in}}%
\pgfpathlineto{\pgfqpoint{4.646939in}{4.149333in}}%
\pgfpathlineto{\pgfqpoint{4.633918in}{4.162224in}}%
\pgfpathlineto{\pgfqpoint{4.609216in}{4.186667in}}%
\pgfpathlineto{\pgfqpoint{4.608466in}{4.187402in}}%
\pgfpathlineto{\pgfqpoint{4.607677in}{4.188189in}}%
\pgfpathlineto{\pgfqpoint{4.571381in}{4.224000in}}%
\pgfpathlineto{\pgfqpoint{4.568917in}{4.224000in}}%
\pgfpathlineto{\pgfqpoint{4.591218in}{4.201998in}}%
\pgfpathlineto{\pgfqpoint{4.606749in}{4.186667in}}%
\pgfpathlineto{\pgfqpoint{4.607197in}{4.186220in}}%
\pgfpathlineto{\pgfqpoint{4.607677in}{4.185750in}}%
\pgfpathlineto{\pgfqpoint{4.644462in}{4.149333in}}%
\pgfpathlineto{\pgfqpoint{4.646052in}{4.147744in}}%
\pgfpathlineto{\pgfqpoint{4.647758in}{4.146068in}}%
\pgfpathlineto{\pgfqpoint{4.682071in}{4.112000in}}%
\pgfpathlineto{\pgfqpoint{4.687838in}{4.106270in}}%
\pgfpathlineto{\pgfqpoint{4.704169in}{4.089878in}}%
\pgfpathlineto{\pgfqpoint{4.719578in}{4.074667in}}%
\pgfpathlineto{\pgfqpoint{4.727919in}{4.066356in}}%
\pgfpathlineto{\pgfqpoint{4.742894in}{4.051281in}}%
\pgfpathlineto{\pgfqpoint{4.756982in}{4.037333in}}%
\pgfpathlineto{\pgfqpoint{4.768000in}{4.026324in}}%
\pgfusepath{fill}%
\end{pgfscope}%
\begin{pgfscope}%
\pgfpathrectangle{\pgfqpoint{0.800000in}{0.528000in}}{\pgfqpoint{3.968000in}{3.696000in}}%
\pgfusepath{clip}%
\pgfsetbuttcap%
\pgfsetroundjoin%
\definecolor{currentfill}{rgb}{0.783315,0.879285,0.125405}%
\pgfsetfillcolor{currentfill}%
\pgfsetlinewidth{0.000000pt}%
\definecolor{currentstroke}{rgb}{0.000000,0.000000,0.000000}%
\pgfsetstrokecolor{currentstroke}%
\pgfsetdash{}{0pt}%
\pgfpathmoveto{\pgfqpoint{4.768000in}{4.031242in}}%
\pgfpathlineto{\pgfqpoint{4.761904in}{4.037333in}}%
\pgfpathlineto{\pgfqpoint{4.745430in}{4.053643in}}%
\pgfpathlineto{\pgfqpoint{4.727919in}{4.071271in}}%
\pgfpathlineto{\pgfqpoint{4.724511in}{4.074667in}}%
\pgfpathlineto{\pgfqpoint{4.706707in}{4.092242in}}%
\pgfpathlineto{\pgfqpoint{4.687838in}{4.111182in}}%
\pgfpathlineto{\pgfqpoint{4.687015in}{4.112000in}}%
\pgfpathlineto{\pgfqpoint{4.674529in}{4.124397in}}%
\pgfpathlineto{\pgfqpoint{4.649400in}{4.149333in}}%
\pgfpathlineto{\pgfqpoint{4.648601in}{4.150119in}}%
\pgfpathlineto{\pgfqpoint{4.647758in}{4.150962in}}%
\pgfpathlineto{\pgfqpoint{4.611674in}{4.186667in}}%
\pgfpathlineto{\pgfqpoint{4.609726in}{4.188576in}}%
\pgfpathlineto{\pgfqpoint{4.607677in}{4.190619in}}%
\pgfpathlineto{\pgfqpoint{4.573844in}{4.224000in}}%
\pgfpathlineto{\pgfqpoint{4.571381in}{4.224000in}}%
\pgfpathlineto{\pgfqpoint{4.607677in}{4.188189in}}%
\pgfpathlineto{\pgfqpoint{4.608466in}{4.187402in}}%
\pgfpathlineto{\pgfqpoint{4.609216in}{4.186667in}}%
\pgfpathlineto{\pgfqpoint{4.633918in}{4.162224in}}%
\pgfpathlineto{\pgfqpoint{4.646939in}{4.149333in}}%
\pgfpathlineto{\pgfqpoint{4.647334in}{4.148939in}}%
\pgfpathlineto{\pgfqpoint{4.647758in}{4.148523in}}%
\pgfpathlineto{\pgfqpoint{4.684543in}{4.112000in}}%
\pgfpathlineto{\pgfqpoint{4.687838in}{4.108726in}}%
\pgfpathlineto{\pgfqpoint{4.705438in}{4.091060in}}%
\pgfpathlineto{\pgfqpoint{4.722044in}{4.074667in}}%
\pgfpathlineto{\pgfqpoint{4.727919in}{4.068813in}}%
\pgfpathlineto{\pgfqpoint{4.744162in}{4.052462in}}%
\pgfpathlineto{\pgfqpoint{4.759443in}{4.037333in}}%
\pgfpathlineto{\pgfqpoint{4.768000in}{4.028783in}}%
\pgfusepath{fill}%
\end{pgfscope}%
\begin{pgfscope}%
\pgfpathrectangle{\pgfqpoint{0.800000in}{0.528000in}}{\pgfqpoint{3.968000in}{3.696000in}}%
\pgfusepath{clip}%
\pgfsetbuttcap%
\pgfsetroundjoin%
\definecolor{currentfill}{rgb}{0.783315,0.879285,0.125405}%
\pgfsetfillcolor{currentfill}%
\pgfsetlinewidth{0.000000pt}%
\definecolor{currentstroke}{rgb}{0.000000,0.000000,0.000000}%
\pgfsetstrokecolor{currentstroke}%
\pgfsetdash{}{0pt}%
\pgfpathmoveto{\pgfqpoint{4.768000in}{4.033701in}}%
\pgfpathlineto{\pgfqpoint{4.764365in}{4.037333in}}%
\pgfpathlineto{\pgfqpoint{4.746697in}{4.054824in}}%
\pgfpathlineto{\pgfqpoint{4.727919in}{4.073728in}}%
\pgfpathlineto{\pgfqpoint{4.726977in}{4.074667in}}%
\pgfpathlineto{\pgfqpoint{4.707976in}{4.093424in}}%
\pgfpathlineto{\pgfqpoint{4.689470in}{4.112000in}}%
\pgfpathlineto{\pgfqpoint{4.688678in}{4.112782in}}%
\pgfpathlineto{\pgfqpoint{4.687838in}{4.113623in}}%
\pgfpathlineto{\pgfqpoint{4.651852in}{4.149333in}}%
\pgfpathlineto{\pgfqpoint{4.649860in}{4.151292in}}%
\pgfpathlineto{\pgfqpoint{4.647758in}{4.153394in}}%
\pgfpathlineto{\pgfqpoint{4.614131in}{4.186667in}}%
\pgfpathlineto{\pgfqpoint{4.610987in}{4.189750in}}%
\pgfpathlineto{\pgfqpoint{4.607677in}{4.193049in}}%
\pgfpathlineto{\pgfqpoint{4.576307in}{4.224000in}}%
\pgfpathlineto{\pgfqpoint{4.573844in}{4.224000in}}%
\pgfpathlineto{\pgfqpoint{4.607677in}{4.190619in}}%
\pgfpathlineto{\pgfqpoint{4.609726in}{4.188576in}}%
\pgfpathlineto{\pgfqpoint{4.611674in}{4.186667in}}%
\pgfpathlineto{\pgfqpoint{4.647758in}{4.150962in}}%
\pgfpathlineto{\pgfqpoint{4.648601in}{4.150119in}}%
\pgfpathlineto{\pgfqpoint{4.649400in}{4.149333in}}%
\pgfpathlineto{\pgfqpoint{4.674529in}{4.124397in}}%
\pgfpathlineto{\pgfqpoint{4.687015in}{4.112000in}}%
\pgfpathlineto{\pgfqpoint{4.687838in}{4.111182in}}%
\pgfpathlineto{\pgfqpoint{4.706707in}{4.092242in}}%
\pgfpathlineto{\pgfqpoint{4.724511in}{4.074667in}}%
\pgfpathlineto{\pgfqpoint{4.727919in}{4.071271in}}%
\pgfpathlineto{\pgfqpoint{4.745430in}{4.053643in}}%
\pgfpathlineto{\pgfqpoint{4.761904in}{4.037333in}}%
\pgfpathlineto{\pgfqpoint{4.768000in}{4.031242in}}%
\pgfusepath{fill}%
\end{pgfscope}%
\begin{pgfscope}%
\pgfpathrectangle{\pgfqpoint{0.800000in}{0.528000in}}{\pgfqpoint{3.968000in}{3.696000in}}%
\pgfusepath{clip}%
\pgfsetbuttcap%
\pgfsetroundjoin%
\definecolor{currentfill}{rgb}{0.783315,0.879285,0.125405}%
\pgfsetfillcolor{currentfill}%
\pgfsetlinewidth{0.000000pt}%
\definecolor{currentstroke}{rgb}{0.000000,0.000000,0.000000}%
\pgfsetstrokecolor{currentstroke}%
\pgfsetdash{}{0pt}%
\pgfpathmoveto{\pgfqpoint{4.768000in}{4.036160in}}%
\pgfpathlineto{\pgfqpoint{4.766826in}{4.037333in}}%
\pgfpathlineto{\pgfqpoint{4.747965in}{4.056005in}}%
\pgfpathlineto{\pgfqpoint{4.729428in}{4.074667in}}%
\pgfpathlineto{\pgfqpoint{4.728696in}{4.075390in}}%
\pgfpathlineto{\pgfqpoint{4.727919in}{4.076172in}}%
\pgfpathlineto{\pgfqpoint{4.709245in}{4.094606in}}%
\pgfpathlineto{\pgfqpoint{4.691917in}{4.112000in}}%
\pgfpathlineto{\pgfqpoint{4.689936in}{4.113954in}}%
\pgfpathlineto{\pgfqpoint{4.687838in}{4.116057in}}%
\pgfpathlineto{\pgfqpoint{4.654304in}{4.149333in}}%
\pgfpathlineto{\pgfqpoint{4.651120in}{4.152465in}}%
\pgfpathlineto{\pgfqpoint{4.647758in}{4.155826in}}%
\pgfpathlineto{\pgfqpoint{4.616589in}{4.186667in}}%
\pgfpathlineto{\pgfqpoint{4.612247in}{4.190924in}}%
\pgfpathlineto{\pgfqpoint{4.607677in}{4.195480in}}%
\pgfpathlineto{\pgfqpoint{4.578771in}{4.224000in}}%
\pgfpathlineto{\pgfqpoint{4.576307in}{4.224000in}}%
\pgfpathlineto{\pgfqpoint{4.607677in}{4.193049in}}%
\pgfpathlineto{\pgfqpoint{4.610987in}{4.189750in}}%
\pgfpathlineto{\pgfqpoint{4.614131in}{4.186667in}}%
\pgfpathlineto{\pgfqpoint{4.647758in}{4.153394in}}%
\pgfpathlineto{\pgfqpoint{4.649860in}{4.151292in}}%
\pgfpathlineto{\pgfqpoint{4.651852in}{4.149333in}}%
\pgfpathlineto{\pgfqpoint{4.687838in}{4.113623in}}%
\pgfpathlineto{\pgfqpoint{4.688678in}{4.112782in}}%
\pgfpathlineto{\pgfqpoint{4.689470in}{4.112000in}}%
\pgfpathlineto{\pgfqpoint{4.707976in}{4.093424in}}%
\pgfpathlineto{\pgfqpoint{4.726977in}{4.074667in}}%
\pgfpathlineto{\pgfqpoint{4.727919in}{4.073728in}}%
\pgfpathlineto{\pgfqpoint{4.746697in}{4.054824in}}%
\pgfpathlineto{\pgfqpoint{4.764365in}{4.037333in}}%
\pgfpathlineto{\pgfqpoint{4.768000in}{4.033701in}}%
\pgfusepath{fill}%
\end{pgfscope}%
\begin{pgfscope}%
\pgfpathrectangle{\pgfqpoint{0.800000in}{0.528000in}}{\pgfqpoint{3.968000in}{3.696000in}}%
\pgfusepath{clip}%
\pgfsetbuttcap%
\pgfsetroundjoin%
\definecolor{currentfill}{rgb}{0.783315,0.879285,0.125405}%
\pgfsetfillcolor{currentfill}%
\pgfsetlinewidth{0.000000pt}%
\definecolor{currentstroke}{rgb}{0.000000,0.000000,0.000000}%
\pgfsetstrokecolor{currentstroke}%
\pgfsetdash{}{0pt}%
\pgfpathmoveto{\pgfqpoint{4.768000in}{4.038607in}}%
\pgfpathlineto{\pgfqpoint{4.749233in}{4.057186in}}%
\pgfpathlineto{\pgfqpoint{4.731869in}{4.074667in}}%
\pgfpathlineto{\pgfqpoint{4.729953in}{4.076561in}}%
\pgfpathlineto{\pgfqpoint{4.727919in}{4.078606in}}%
\pgfpathlineto{\pgfqpoint{4.710514in}{4.095788in}}%
\pgfpathlineto{\pgfqpoint{4.694364in}{4.112000in}}%
\pgfpathlineto{\pgfqpoint{4.691194in}{4.115126in}}%
\pgfpathlineto{\pgfqpoint{4.687838in}{4.118490in}}%
\pgfpathlineto{\pgfqpoint{4.656756in}{4.149333in}}%
\pgfpathlineto{\pgfqpoint{4.652379in}{4.153638in}}%
\pgfpathlineto{\pgfqpoint{4.647758in}{4.158258in}}%
\pgfpathlineto{\pgfqpoint{4.619047in}{4.186667in}}%
\pgfpathlineto{\pgfqpoint{4.613508in}{4.192098in}}%
\pgfpathlineto{\pgfqpoint{4.607677in}{4.197910in}}%
\pgfpathlineto{\pgfqpoint{4.581234in}{4.224000in}}%
\pgfpathlineto{\pgfqpoint{4.578771in}{4.224000in}}%
\pgfpathlineto{\pgfqpoint{4.607677in}{4.195480in}}%
\pgfpathlineto{\pgfqpoint{4.612247in}{4.190924in}}%
\pgfpathlineto{\pgfqpoint{4.616589in}{4.186667in}}%
\pgfpathlineto{\pgfqpoint{4.647758in}{4.155826in}}%
\pgfpathlineto{\pgfqpoint{4.651120in}{4.152465in}}%
\pgfpathlineto{\pgfqpoint{4.654304in}{4.149333in}}%
\pgfpathlineto{\pgfqpoint{4.687838in}{4.116057in}}%
\pgfpathlineto{\pgfqpoint{4.689936in}{4.113954in}}%
\pgfpathlineto{\pgfqpoint{4.691917in}{4.112000in}}%
\pgfpathlineto{\pgfqpoint{4.709245in}{4.094606in}}%
\pgfpathlineto{\pgfqpoint{4.727919in}{4.076172in}}%
\pgfpathlineto{\pgfqpoint{4.728696in}{4.075390in}}%
\pgfpathlineto{\pgfqpoint{4.729428in}{4.074667in}}%
\pgfpathlineto{\pgfqpoint{4.747965in}{4.056005in}}%
\pgfpathlineto{\pgfqpoint{4.766826in}{4.037333in}}%
\pgfpathlineto{\pgfqpoint{4.768000in}{4.036160in}}%
\pgfpathlineto{\pgfqpoint{4.768000in}{4.037333in}}%
\pgfusepath{fill}%
\end{pgfscope}%
\begin{pgfscope}%
\pgfpathrectangle{\pgfqpoint{0.800000in}{0.528000in}}{\pgfqpoint{3.968000in}{3.696000in}}%
\pgfusepath{clip}%
\pgfsetbuttcap%
\pgfsetroundjoin%
\definecolor{currentfill}{rgb}{0.793760,0.880678,0.120005}%
\pgfsetfillcolor{currentfill}%
\pgfsetlinewidth{0.000000pt}%
\definecolor{currentstroke}{rgb}{0.000000,0.000000,0.000000}%
\pgfsetstrokecolor{currentstroke}%
\pgfsetdash{}{0pt}%
\pgfpathmoveto{\pgfqpoint{4.768000in}{4.041044in}}%
\pgfpathlineto{\pgfqpoint{4.750501in}{4.058368in}}%
\pgfpathlineto{\pgfqpoint{4.734310in}{4.074667in}}%
\pgfpathlineto{\pgfqpoint{4.731210in}{4.077732in}}%
\pgfpathlineto{\pgfqpoint{4.727919in}{4.081041in}}%
\pgfpathlineto{\pgfqpoint{4.711783in}{4.096970in}}%
\pgfpathlineto{\pgfqpoint{4.696810in}{4.112000in}}%
\pgfpathlineto{\pgfqpoint{4.692452in}{4.116298in}}%
\pgfpathlineto{\pgfqpoint{4.687838in}{4.120923in}}%
\pgfpathlineto{\pgfqpoint{4.659209in}{4.149333in}}%
\pgfpathlineto{\pgfqpoint{4.653638in}{4.154811in}}%
\pgfpathlineto{\pgfqpoint{4.647758in}{4.160690in}}%
\pgfpathlineto{\pgfqpoint{4.621504in}{4.186667in}}%
\pgfpathlineto{\pgfqpoint{4.614768in}{4.193272in}}%
\pgfpathlineto{\pgfqpoint{4.607677in}{4.200341in}}%
\pgfpathlineto{\pgfqpoint{4.583697in}{4.224000in}}%
\pgfpathlineto{\pgfqpoint{4.581234in}{4.224000in}}%
\pgfpathlineto{\pgfqpoint{4.607677in}{4.197910in}}%
\pgfpathlineto{\pgfqpoint{4.613508in}{4.192098in}}%
\pgfpathlineto{\pgfqpoint{4.619047in}{4.186667in}}%
\pgfpathlineto{\pgfqpoint{4.647758in}{4.158258in}}%
\pgfpathlineto{\pgfqpoint{4.652379in}{4.153638in}}%
\pgfpathlineto{\pgfqpoint{4.656756in}{4.149333in}}%
\pgfpathlineto{\pgfqpoint{4.687838in}{4.118490in}}%
\pgfpathlineto{\pgfqpoint{4.691194in}{4.115126in}}%
\pgfpathlineto{\pgfqpoint{4.694364in}{4.112000in}}%
\pgfpathlineto{\pgfqpoint{4.710514in}{4.095788in}}%
\pgfpathlineto{\pgfqpoint{4.727919in}{4.078606in}}%
\pgfpathlineto{\pgfqpoint{4.729953in}{4.076561in}}%
\pgfpathlineto{\pgfqpoint{4.731869in}{4.074667in}}%
\pgfpathlineto{\pgfqpoint{4.749233in}{4.057186in}}%
\pgfpathlineto{\pgfqpoint{4.768000in}{4.038607in}}%
\pgfusepath{fill}%
\end{pgfscope}%
\begin{pgfscope}%
\pgfpathrectangle{\pgfqpoint{0.800000in}{0.528000in}}{\pgfqpoint{3.968000in}{3.696000in}}%
\pgfusepath{clip}%
\pgfsetbuttcap%
\pgfsetroundjoin%
\definecolor{currentfill}{rgb}{0.793760,0.880678,0.120005}%
\pgfsetfillcolor{currentfill}%
\pgfsetlinewidth{0.000000pt}%
\definecolor{currentstroke}{rgb}{0.000000,0.000000,0.000000}%
\pgfsetstrokecolor{currentstroke}%
\pgfsetdash{}{0pt}%
\pgfpathmoveto{\pgfqpoint{4.768000in}{4.043480in}}%
\pgfpathlineto{\pgfqpoint{4.751769in}{4.059549in}}%
\pgfpathlineto{\pgfqpoint{4.736751in}{4.074667in}}%
\pgfpathlineto{\pgfqpoint{4.732467in}{4.078903in}}%
\pgfpathlineto{\pgfqpoint{4.727919in}{4.083476in}}%
\pgfpathlineto{\pgfqpoint{4.713052in}{4.098152in}}%
\pgfpathlineto{\pgfqpoint{4.699257in}{4.112000in}}%
\pgfpathlineto{\pgfqpoint{4.693710in}{4.117470in}}%
\pgfpathlineto{\pgfqpoint{4.687838in}{4.123357in}}%
\pgfpathlineto{\pgfqpoint{4.661661in}{4.149333in}}%
\pgfpathlineto{\pgfqpoint{4.654897in}{4.155984in}}%
\pgfpathlineto{\pgfqpoint{4.647758in}{4.163122in}}%
\pgfpathlineto{\pgfqpoint{4.623962in}{4.186667in}}%
\pgfpathlineto{\pgfqpoint{4.616028in}{4.194446in}}%
\pgfpathlineto{\pgfqpoint{4.607677in}{4.202771in}}%
\pgfpathlineto{\pgfqpoint{4.586160in}{4.224000in}}%
\pgfpathlineto{\pgfqpoint{4.583697in}{4.224000in}}%
\pgfpathlineto{\pgfqpoint{4.607677in}{4.200341in}}%
\pgfpathlineto{\pgfqpoint{4.614768in}{4.193272in}}%
\pgfpathlineto{\pgfqpoint{4.621504in}{4.186667in}}%
\pgfpathlineto{\pgfqpoint{4.647758in}{4.160690in}}%
\pgfpathlineto{\pgfqpoint{4.653638in}{4.154811in}}%
\pgfpathlineto{\pgfqpoint{4.659209in}{4.149333in}}%
\pgfpathlineto{\pgfqpoint{4.687838in}{4.120923in}}%
\pgfpathlineto{\pgfqpoint{4.692452in}{4.116298in}}%
\pgfpathlineto{\pgfqpoint{4.696810in}{4.112000in}}%
\pgfpathlineto{\pgfqpoint{4.711783in}{4.096970in}}%
\pgfpathlineto{\pgfqpoint{4.727919in}{4.081041in}}%
\pgfpathlineto{\pgfqpoint{4.731210in}{4.077732in}}%
\pgfpathlineto{\pgfqpoint{4.734310in}{4.074667in}}%
\pgfpathlineto{\pgfqpoint{4.750501in}{4.058368in}}%
\pgfpathlineto{\pgfqpoint{4.768000in}{4.041044in}}%
\pgfusepath{fill}%
\end{pgfscope}%
\begin{pgfscope}%
\pgfpathrectangle{\pgfqpoint{0.800000in}{0.528000in}}{\pgfqpoint{3.968000in}{3.696000in}}%
\pgfusepath{clip}%
\pgfsetbuttcap%
\pgfsetroundjoin%
\definecolor{currentfill}{rgb}{0.793760,0.880678,0.120005}%
\pgfsetfillcolor{currentfill}%
\pgfsetlinewidth{0.000000pt}%
\definecolor{currentstroke}{rgb}{0.000000,0.000000,0.000000}%
\pgfsetstrokecolor{currentstroke}%
\pgfsetdash{}{0pt}%
\pgfpathmoveto{\pgfqpoint{4.768000in}{4.045916in}}%
\pgfpathlineto{\pgfqpoint{4.753037in}{4.060730in}}%
\pgfpathlineto{\pgfqpoint{4.739193in}{4.074667in}}%
\pgfpathlineto{\pgfqpoint{4.733725in}{4.080074in}}%
\pgfpathlineto{\pgfqpoint{4.727919in}{4.085911in}}%
\pgfpathlineto{\pgfqpoint{4.714321in}{4.099334in}}%
\pgfpathlineto{\pgfqpoint{4.701704in}{4.112000in}}%
\pgfpathlineto{\pgfqpoint{4.694969in}{4.118641in}}%
\pgfpathlineto{\pgfqpoint{4.687838in}{4.125790in}}%
\pgfpathlineto{\pgfqpoint{4.664113in}{4.149333in}}%
\pgfpathlineto{\pgfqpoint{4.656157in}{4.157157in}}%
\pgfpathlineto{\pgfqpoint{4.647758in}{4.165553in}}%
\pgfpathlineto{\pgfqpoint{4.626420in}{4.186667in}}%
\pgfpathlineto{\pgfqpoint{4.617289in}{4.195620in}}%
\pgfpathlineto{\pgfqpoint{4.607677in}{4.205201in}}%
\pgfpathlineto{\pgfqpoint{4.588624in}{4.224000in}}%
\pgfpathlineto{\pgfqpoint{4.586160in}{4.224000in}}%
\pgfpathlineto{\pgfqpoint{4.607677in}{4.202771in}}%
\pgfpathlineto{\pgfqpoint{4.616028in}{4.194446in}}%
\pgfpathlineto{\pgfqpoint{4.623962in}{4.186667in}}%
\pgfpathlineto{\pgfqpoint{4.647758in}{4.163122in}}%
\pgfpathlineto{\pgfqpoint{4.654897in}{4.155984in}}%
\pgfpathlineto{\pgfqpoint{4.661661in}{4.149333in}}%
\pgfpathlineto{\pgfqpoint{4.687838in}{4.123357in}}%
\pgfpathlineto{\pgfqpoint{4.693710in}{4.117470in}}%
\pgfpathlineto{\pgfqpoint{4.699257in}{4.112000in}}%
\pgfpathlineto{\pgfqpoint{4.713052in}{4.098152in}}%
\pgfpathlineto{\pgfqpoint{4.727919in}{4.083476in}}%
\pgfpathlineto{\pgfqpoint{4.732467in}{4.078903in}}%
\pgfpathlineto{\pgfqpoint{4.736751in}{4.074667in}}%
\pgfpathlineto{\pgfqpoint{4.751769in}{4.059549in}}%
\pgfpathlineto{\pgfqpoint{4.768000in}{4.043480in}}%
\pgfusepath{fill}%
\end{pgfscope}%
\begin{pgfscope}%
\pgfpathrectangle{\pgfqpoint{0.800000in}{0.528000in}}{\pgfqpoint{3.968000in}{3.696000in}}%
\pgfusepath{clip}%
\pgfsetbuttcap%
\pgfsetroundjoin%
\definecolor{currentfill}{rgb}{0.793760,0.880678,0.120005}%
\pgfsetfillcolor{currentfill}%
\pgfsetlinewidth{0.000000pt}%
\definecolor{currentstroke}{rgb}{0.000000,0.000000,0.000000}%
\pgfsetstrokecolor{currentstroke}%
\pgfsetdash{}{0pt}%
\pgfpathmoveto{\pgfqpoint{4.768000in}{4.048353in}}%
\pgfpathlineto{\pgfqpoint{4.754305in}{4.061911in}}%
\pgfpathlineto{\pgfqpoint{4.741634in}{4.074667in}}%
\pgfpathlineto{\pgfqpoint{4.734982in}{4.081245in}}%
\pgfpathlineto{\pgfqpoint{4.727919in}{4.088346in}}%
\pgfpathlineto{\pgfqpoint{4.715590in}{4.100516in}}%
\pgfpathlineto{\pgfqpoint{4.704150in}{4.112000in}}%
\pgfpathlineto{\pgfqpoint{4.696227in}{4.119813in}}%
\pgfpathlineto{\pgfqpoint{4.687838in}{4.128223in}}%
\pgfpathlineto{\pgfqpoint{4.666565in}{4.149333in}}%
\pgfpathlineto{\pgfqpoint{4.657416in}{4.158330in}}%
\pgfpathlineto{\pgfqpoint{4.647758in}{4.167985in}}%
\pgfpathlineto{\pgfqpoint{4.628878in}{4.186667in}}%
\pgfpathlineto{\pgfqpoint{4.618549in}{4.196794in}}%
\pgfpathlineto{\pgfqpoint{4.607677in}{4.207632in}}%
\pgfpathlineto{\pgfqpoint{4.591087in}{4.224000in}}%
\pgfpathlineto{\pgfqpoint{4.588624in}{4.224000in}}%
\pgfpathlineto{\pgfqpoint{4.607677in}{4.205201in}}%
\pgfpathlineto{\pgfqpoint{4.617289in}{4.195620in}}%
\pgfpathlineto{\pgfqpoint{4.626420in}{4.186667in}}%
\pgfpathlineto{\pgfqpoint{4.647758in}{4.165553in}}%
\pgfpathlineto{\pgfqpoint{4.656157in}{4.157157in}}%
\pgfpathlineto{\pgfqpoint{4.664113in}{4.149333in}}%
\pgfpathlineto{\pgfqpoint{4.687838in}{4.125790in}}%
\pgfpathlineto{\pgfqpoint{4.694969in}{4.118641in}}%
\pgfpathlineto{\pgfqpoint{4.701704in}{4.112000in}}%
\pgfpathlineto{\pgfqpoint{4.714321in}{4.099334in}}%
\pgfpathlineto{\pgfqpoint{4.727919in}{4.085911in}}%
\pgfpathlineto{\pgfqpoint{4.733725in}{4.080074in}}%
\pgfpathlineto{\pgfqpoint{4.739193in}{4.074667in}}%
\pgfpathlineto{\pgfqpoint{4.753037in}{4.060730in}}%
\pgfpathlineto{\pgfqpoint{4.768000in}{4.045916in}}%
\pgfusepath{fill}%
\end{pgfscope}%
\begin{pgfscope}%
\pgfpathrectangle{\pgfqpoint{0.800000in}{0.528000in}}{\pgfqpoint{3.968000in}{3.696000in}}%
\pgfusepath{clip}%
\pgfsetbuttcap%
\pgfsetroundjoin%
\definecolor{currentfill}{rgb}{0.804182,0.882046,0.114965}%
\pgfsetfillcolor{currentfill}%
\pgfsetlinewidth{0.000000pt}%
\definecolor{currentstroke}{rgb}{0.000000,0.000000,0.000000}%
\pgfsetstrokecolor{currentstroke}%
\pgfsetdash{}{0pt}%
\pgfpathmoveto{\pgfqpoint{4.768000in}{4.050789in}}%
\pgfpathlineto{\pgfqpoint{4.755573in}{4.063092in}}%
\pgfpathlineto{\pgfqpoint{4.744075in}{4.074667in}}%
\pgfpathlineto{\pgfqpoint{4.736239in}{4.082416in}}%
\pgfpathlineto{\pgfqpoint{4.727919in}{4.090781in}}%
\pgfpathlineto{\pgfqpoint{4.716860in}{4.101698in}}%
\pgfpathlineto{\pgfqpoint{4.706597in}{4.112000in}}%
\pgfpathlineto{\pgfqpoint{4.697485in}{4.120985in}}%
\pgfpathlineto{\pgfqpoint{4.687838in}{4.130657in}}%
\pgfpathlineto{\pgfqpoint{4.669017in}{4.149333in}}%
\pgfpathlineto{\pgfqpoint{4.658675in}{4.159503in}}%
\pgfpathlineto{\pgfqpoint{4.647758in}{4.170417in}}%
\pgfpathlineto{\pgfqpoint{4.631335in}{4.186667in}}%
\pgfpathlineto{\pgfqpoint{4.619809in}{4.197968in}}%
\pgfpathlineto{\pgfqpoint{4.607677in}{4.210062in}}%
\pgfpathlineto{\pgfqpoint{4.593550in}{4.224000in}}%
\pgfpathlineto{\pgfqpoint{4.591087in}{4.224000in}}%
\pgfpathlineto{\pgfqpoint{4.607677in}{4.207632in}}%
\pgfpathlineto{\pgfqpoint{4.618549in}{4.196794in}}%
\pgfpathlineto{\pgfqpoint{4.628878in}{4.186667in}}%
\pgfpathlineto{\pgfqpoint{4.647758in}{4.167985in}}%
\pgfpathlineto{\pgfqpoint{4.657416in}{4.158330in}}%
\pgfpathlineto{\pgfqpoint{4.666565in}{4.149333in}}%
\pgfpathlineto{\pgfqpoint{4.687838in}{4.128223in}}%
\pgfpathlineto{\pgfqpoint{4.696227in}{4.119813in}}%
\pgfpathlineto{\pgfqpoint{4.704150in}{4.112000in}}%
\pgfpathlineto{\pgfqpoint{4.715590in}{4.100516in}}%
\pgfpathlineto{\pgfqpoint{4.727919in}{4.088346in}}%
\pgfpathlineto{\pgfqpoint{4.734982in}{4.081245in}}%
\pgfpathlineto{\pgfqpoint{4.741634in}{4.074667in}}%
\pgfpathlineto{\pgfqpoint{4.754305in}{4.061911in}}%
\pgfpathlineto{\pgfqpoint{4.768000in}{4.048353in}}%
\pgfusepath{fill}%
\end{pgfscope}%
\begin{pgfscope}%
\pgfpathrectangle{\pgfqpoint{0.800000in}{0.528000in}}{\pgfqpoint{3.968000in}{3.696000in}}%
\pgfusepath{clip}%
\pgfsetbuttcap%
\pgfsetroundjoin%
\definecolor{currentfill}{rgb}{0.804182,0.882046,0.114965}%
\pgfsetfillcolor{currentfill}%
\pgfsetlinewidth{0.000000pt}%
\definecolor{currentstroke}{rgb}{0.000000,0.000000,0.000000}%
\pgfsetstrokecolor{currentstroke}%
\pgfsetdash{}{0pt}%
\pgfpathmoveto{\pgfqpoint{4.768000in}{4.053225in}}%
\pgfpathlineto{\pgfqpoint{4.756841in}{4.064273in}}%
\pgfpathlineto{\pgfqpoint{4.746516in}{4.074667in}}%
\pgfpathlineto{\pgfqpoint{4.737496in}{4.083587in}}%
\pgfpathlineto{\pgfqpoint{4.727919in}{4.093216in}}%
\pgfpathlineto{\pgfqpoint{4.718129in}{4.102881in}}%
\pgfpathlineto{\pgfqpoint{4.709044in}{4.112000in}}%
\pgfpathlineto{\pgfqpoint{4.698743in}{4.122157in}}%
\pgfpathlineto{\pgfqpoint{4.687838in}{4.133090in}}%
\pgfpathlineto{\pgfqpoint{4.671470in}{4.149333in}}%
\pgfpathlineto{\pgfqpoint{4.659935in}{4.160676in}}%
\pgfpathlineto{\pgfqpoint{4.647758in}{4.172849in}}%
\pgfpathlineto{\pgfqpoint{4.633793in}{4.186667in}}%
\pgfpathlineto{\pgfqpoint{4.621070in}{4.199142in}}%
\pgfpathlineto{\pgfqpoint{4.607677in}{4.212493in}}%
\pgfpathlineto{\pgfqpoint{4.596014in}{4.224000in}}%
\pgfpathlineto{\pgfqpoint{4.593550in}{4.224000in}}%
\pgfpathlineto{\pgfqpoint{4.607677in}{4.210062in}}%
\pgfpathlineto{\pgfqpoint{4.619809in}{4.197968in}}%
\pgfpathlineto{\pgfqpoint{4.631335in}{4.186667in}}%
\pgfpathlineto{\pgfqpoint{4.647758in}{4.170417in}}%
\pgfpathlineto{\pgfqpoint{4.658675in}{4.159503in}}%
\pgfpathlineto{\pgfqpoint{4.669017in}{4.149333in}}%
\pgfpathlineto{\pgfqpoint{4.687838in}{4.130657in}}%
\pgfpathlineto{\pgfqpoint{4.697485in}{4.120985in}}%
\pgfpathlineto{\pgfqpoint{4.706597in}{4.112000in}}%
\pgfpathlineto{\pgfqpoint{4.716860in}{4.101698in}}%
\pgfpathlineto{\pgfqpoint{4.727919in}{4.090781in}}%
\pgfpathlineto{\pgfqpoint{4.736239in}{4.082416in}}%
\pgfpathlineto{\pgfqpoint{4.744075in}{4.074667in}}%
\pgfpathlineto{\pgfqpoint{4.755573in}{4.063092in}}%
\pgfpathlineto{\pgfqpoint{4.768000in}{4.050789in}}%
\pgfusepath{fill}%
\end{pgfscope}%
\begin{pgfscope}%
\pgfpathrectangle{\pgfqpoint{0.800000in}{0.528000in}}{\pgfqpoint{3.968000in}{3.696000in}}%
\pgfusepath{clip}%
\pgfsetbuttcap%
\pgfsetroundjoin%
\definecolor{currentfill}{rgb}{0.804182,0.882046,0.114965}%
\pgfsetfillcolor{currentfill}%
\pgfsetlinewidth{0.000000pt}%
\definecolor{currentstroke}{rgb}{0.000000,0.000000,0.000000}%
\pgfsetstrokecolor{currentstroke}%
\pgfsetdash{}{0pt}%
\pgfpathmoveto{\pgfqpoint{4.768000in}{4.055662in}}%
\pgfpathlineto{\pgfqpoint{4.758109in}{4.065454in}}%
\pgfpathlineto{\pgfqpoint{4.748957in}{4.074667in}}%
\pgfpathlineto{\pgfqpoint{4.738753in}{4.084758in}}%
\pgfpathlineto{\pgfqpoint{4.727919in}{4.095650in}}%
\pgfpathlineto{\pgfqpoint{4.719398in}{4.104063in}}%
\pgfpathlineto{\pgfqpoint{4.711490in}{4.112000in}}%
\pgfpathlineto{\pgfqpoint{4.700001in}{4.123329in}}%
\pgfpathlineto{\pgfqpoint{4.687838in}{4.135523in}}%
\pgfpathlineto{\pgfqpoint{4.673922in}{4.149333in}}%
\pgfpathlineto{\pgfqpoint{4.661194in}{4.161849in}}%
\pgfpathlineto{\pgfqpoint{4.647758in}{4.175281in}}%
\pgfpathlineto{\pgfqpoint{4.636251in}{4.186667in}}%
\pgfpathlineto{\pgfqpoint{4.622330in}{4.200316in}}%
\pgfpathlineto{\pgfqpoint{4.607677in}{4.214923in}}%
\pgfpathlineto{\pgfqpoint{4.598477in}{4.224000in}}%
\pgfpathlineto{\pgfqpoint{4.596014in}{4.224000in}}%
\pgfpathlineto{\pgfqpoint{4.607677in}{4.212493in}}%
\pgfpathlineto{\pgfqpoint{4.621070in}{4.199142in}}%
\pgfpathlineto{\pgfqpoint{4.633793in}{4.186667in}}%
\pgfpathlineto{\pgfqpoint{4.647758in}{4.172849in}}%
\pgfpathlineto{\pgfqpoint{4.659935in}{4.160676in}}%
\pgfpathlineto{\pgfqpoint{4.671470in}{4.149333in}}%
\pgfpathlineto{\pgfqpoint{4.687838in}{4.133090in}}%
\pgfpathlineto{\pgfqpoint{4.698743in}{4.122157in}}%
\pgfpathlineto{\pgfqpoint{4.709044in}{4.112000in}}%
\pgfpathlineto{\pgfqpoint{4.718129in}{4.102881in}}%
\pgfpathlineto{\pgfqpoint{4.727919in}{4.093216in}}%
\pgfpathlineto{\pgfqpoint{4.737496in}{4.083587in}}%
\pgfpathlineto{\pgfqpoint{4.746516in}{4.074667in}}%
\pgfpathlineto{\pgfqpoint{4.756841in}{4.064273in}}%
\pgfpathlineto{\pgfqpoint{4.768000in}{4.053225in}}%
\pgfusepath{fill}%
\end{pgfscope}%
\begin{pgfscope}%
\pgfpathrectangle{\pgfqpoint{0.800000in}{0.528000in}}{\pgfqpoint{3.968000in}{3.696000in}}%
\pgfusepath{clip}%
\pgfsetbuttcap%
\pgfsetroundjoin%
\definecolor{currentfill}{rgb}{0.814576,0.883393,0.110347}%
\pgfsetfillcolor{currentfill}%
\pgfsetlinewidth{0.000000pt}%
\definecolor{currentstroke}{rgb}{0.000000,0.000000,0.000000}%
\pgfsetstrokecolor{currentstroke}%
\pgfsetdash{}{0pt}%
\pgfpathmoveto{\pgfqpoint{4.768000in}{4.058098in}}%
\pgfpathlineto{\pgfqpoint{4.759377in}{4.066635in}}%
\pgfpathlineto{\pgfqpoint{4.751398in}{4.074667in}}%
\pgfpathlineto{\pgfqpoint{4.740010in}{4.085929in}}%
\pgfpathlineto{\pgfqpoint{4.727919in}{4.098085in}}%
\pgfpathlineto{\pgfqpoint{4.720667in}{4.105245in}}%
\pgfpathlineto{\pgfqpoint{4.713937in}{4.112000in}}%
\pgfpathlineto{\pgfqpoint{4.701260in}{4.124501in}}%
\pgfpathlineto{\pgfqpoint{4.687838in}{4.137957in}}%
\pgfpathlineto{\pgfqpoint{4.676374in}{4.149333in}}%
\pgfpathlineto{\pgfqpoint{4.662453in}{4.163021in}}%
\pgfpathlineto{\pgfqpoint{4.647758in}{4.177713in}}%
\pgfpathlineto{\pgfqpoint{4.638709in}{4.186667in}}%
\pgfpathlineto{\pgfqpoint{4.623591in}{4.201490in}}%
\pgfpathlineto{\pgfqpoint{4.607677in}{4.217354in}}%
\pgfpathlineto{\pgfqpoint{4.600940in}{4.224000in}}%
\pgfpathlineto{\pgfqpoint{4.598477in}{4.224000in}}%
\pgfpathlineto{\pgfqpoint{4.607677in}{4.214923in}}%
\pgfpathlineto{\pgfqpoint{4.622330in}{4.200316in}}%
\pgfpathlineto{\pgfqpoint{4.636251in}{4.186667in}}%
\pgfpathlineto{\pgfqpoint{4.647758in}{4.175281in}}%
\pgfpathlineto{\pgfqpoint{4.661194in}{4.161849in}}%
\pgfpathlineto{\pgfqpoint{4.673922in}{4.149333in}}%
\pgfpathlineto{\pgfqpoint{4.687838in}{4.135523in}}%
\pgfpathlineto{\pgfqpoint{4.700001in}{4.123329in}}%
\pgfpathlineto{\pgfqpoint{4.711490in}{4.112000in}}%
\pgfpathlineto{\pgfqpoint{4.719398in}{4.104063in}}%
\pgfpathlineto{\pgfqpoint{4.727919in}{4.095650in}}%
\pgfpathlineto{\pgfqpoint{4.738753in}{4.084758in}}%
\pgfpathlineto{\pgfqpoint{4.748957in}{4.074667in}}%
\pgfpathlineto{\pgfqpoint{4.758109in}{4.065454in}}%
\pgfpathlineto{\pgfqpoint{4.768000in}{4.055662in}}%
\pgfusepath{fill}%
\end{pgfscope}%
\begin{pgfscope}%
\pgfpathrectangle{\pgfqpoint{0.800000in}{0.528000in}}{\pgfqpoint{3.968000in}{3.696000in}}%
\pgfusepath{clip}%
\pgfsetbuttcap%
\pgfsetroundjoin%
\definecolor{currentfill}{rgb}{0.814576,0.883393,0.110347}%
\pgfsetfillcolor{currentfill}%
\pgfsetlinewidth{0.000000pt}%
\definecolor{currentstroke}{rgb}{0.000000,0.000000,0.000000}%
\pgfsetstrokecolor{currentstroke}%
\pgfsetdash{}{0pt}%
\pgfpathmoveto{\pgfqpoint{4.768000in}{4.060534in}}%
\pgfpathlineto{\pgfqpoint{4.760645in}{4.067816in}}%
\pgfpathlineto{\pgfqpoint{4.753840in}{4.074667in}}%
\pgfpathlineto{\pgfqpoint{4.741267in}{4.087100in}}%
\pgfpathlineto{\pgfqpoint{4.727919in}{4.100520in}}%
\pgfpathlineto{\pgfqpoint{4.721936in}{4.106427in}}%
\pgfpathlineto{\pgfqpoint{4.716384in}{4.112000in}}%
\pgfpathlineto{\pgfqpoint{4.702518in}{4.125673in}}%
\pgfpathlineto{\pgfqpoint{4.687838in}{4.140390in}}%
\pgfpathlineto{\pgfqpoint{4.678826in}{4.149333in}}%
\pgfpathlineto{\pgfqpoint{4.663712in}{4.164194in}}%
\pgfpathlineto{\pgfqpoint{4.647758in}{4.180145in}}%
\pgfpathlineto{\pgfqpoint{4.641166in}{4.186667in}}%
\pgfpathlineto{\pgfqpoint{4.624851in}{4.202664in}}%
\pgfpathlineto{\pgfqpoint{4.607677in}{4.219784in}}%
\pgfpathlineto{\pgfqpoint{4.603404in}{4.224000in}}%
\pgfpathlineto{\pgfqpoint{4.600940in}{4.224000in}}%
\pgfpathlineto{\pgfqpoint{4.607677in}{4.217354in}}%
\pgfpathlineto{\pgfqpoint{4.623591in}{4.201490in}}%
\pgfpathlineto{\pgfqpoint{4.638709in}{4.186667in}}%
\pgfpathlineto{\pgfqpoint{4.647758in}{4.177713in}}%
\pgfpathlineto{\pgfqpoint{4.662453in}{4.163021in}}%
\pgfpathlineto{\pgfqpoint{4.676374in}{4.149333in}}%
\pgfpathlineto{\pgfqpoint{4.687838in}{4.137957in}}%
\pgfpathlineto{\pgfqpoint{4.701260in}{4.124501in}}%
\pgfpathlineto{\pgfqpoint{4.713937in}{4.112000in}}%
\pgfpathlineto{\pgfqpoint{4.720667in}{4.105245in}}%
\pgfpathlineto{\pgfqpoint{4.727919in}{4.098085in}}%
\pgfpathlineto{\pgfqpoint{4.740010in}{4.085929in}}%
\pgfpathlineto{\pgfqpoint{4.751398in}{4.074667in}}%
\pgfpathlineto{\pgfqpoint{4.759377in}{4.066635in}}%
\pgfpathlineto{\pgfqpoint{4.768000in}{4.058098in}}%
\pgfusepath{fill}%
\end{pgfscope}%
\begin{pgfscope}%
\pgfpathrectangle{\pgfqpoint{0.800000in}{0.528000in}}{\pgfqpoint{3.968000in}{3.696000in}}%
\pgfusepath{clip}%
\pgfsetbuttcap%
\pgfsetroundjoin%
\definecolor{currentfill}{rgb}{0.814576,0.883393,0.110347}%
\pgfsetfillcolor{currentfill}%
\pgfsetlinewidth{0.000000pt}%
\definecolor{currentstroke}{rgb}{0.000000,0.000000,0.000000}%
\pgfsetstrokecolor{currentstroke}%
\pgfsetdash{}{0pt}%
\pgfpathmoveto{\pgfqpoint{4.768000in}{4.062971in}}%
\pgfpathlineto{\pgfqpoint{4.761913in}{4.068997in}}%
\pgfpathlineto{\pgfqpoint{4.756281in}{4.074667in}}%
\pgfpathlineto{\pgfqpoint{4.742524in}{4.088271in}}%
\pgfpathlineto{\pgfqpoint{4.727919in}{4.102955in}}%
\pgfpathlineto{\pgfqpoint{4.723205in}{4.107609in}}%
\pgfpathlineto{\pgfqpoint{4.718830in}{4.112000in}}%
\pgfpathlineto{\pgfqpoint{4.703776in}{4.126845in}}%
\pgfpathlineto{\pgfqpoint{4.687838in}{4.142824in}}%
\pgfpathlineto{\pgfqpoint{4.681278in}{4.149333in}}%
\pgfpathlineto{\pgfqpoint{4.664972in}{4.165367in}}%
\pgfpathlineto{\pgfqpoint{4.647758in}{4.182577in}}%
\pgfpathlineto{\pgfqpoint{4.643624in}{4.186667in}}%
\pgfpathlineto{\pgfqpoint{4.626111in}{4.203838in}}%
\pgfpathlineto{\pgfqpoint{4.607677in}{4.222214in}}%
\pgfpathlineto{\pgfqpoint{4.605867in}{4.224000in}}%
\pgfpathlineto{\pgfqpoint{4.603404in}{4.224000in}}%
\pgfpathlineto{\pgfqpoint{4.607677in}{4.219784in}}%
\pgfpathlineto{\pgfqpoint{4.624851in}{4.202664in}}%
\pgfpathlineto{\pgfqpoint{4.641166in}{4.186667in}}%
\pgfpathlineto{\pgfqpoint{4.647758in}{4.180145in}}%
\pgfpathlineto{\pgfqpoint{4.663712in}{4.164194in}}%
\pgfpathlineto{\pgfqpoint{4.678826in}{4.149333in}}%
\pgfpathlineto{\pgfqpoint{4.687838in}{4.140390in}}%
\pgfpathlineto{\pgfqpoint{4.702518in}{4.125673in}}%
\pgfpathlineto{\pgfqpoint{4.716384in}{4.112000in}}%
\pgfpathlineto{\pgfqpoint{4.721936in}{4.106427in}}%
\pgfpathlineto{\pgfqpoint{4.727919in}{4.100520in}}%
\pgfpathlineto{\pgfqpoint{4.741267in}{4.087100in}}%
\pgfpathlineto{\pgfqpoint{4.753840in}{4.074667in}}%
\pgfpathlineto{\pgfqpoint{4.760645in}{4.067816in}}%
\pgfpathlineto{\pgfqpoint{4.768000in}{4.060534in}}%
\pgfusepath{fill}%
\end{pgfscope}%
\begin{pgfscope}%
\pgfpathrectangle{\pgfqpoint{0.800000in}{0.528000in}}{\pgfqpoint{3.968000in}{3.696000in}}%
\pgfusepath{clip}%
\pgfsetbuttcap%
\pgfsetroundjoin%
\definecolor{currentfill}{rgb}{0.814576,0.883393,0.110347}%
\pgfsetfillcolor{currentfill}%
\pgfsetlinewidth{0.000000pt}%
\definecolor{currentstroke}{rgb}{0.000000,0.000000,0.000000}%
\pgfsetstrokecolor{currentstroke}%
\pgfsetdash{}{0pt}%
\pgfpathmoveto{\pgfqpoint{4.768000in}{4.065407in}}%
\pgfpathlineto{\pgfqpoint{4.763181in}{4.070178in}}%
\pgfpathlineto{\pgfqpoint{4.758722in}{4.074667in}}%
\pgfpathlineto{\pgfqpoint{4.743781in}{4.089442in}}%
\pgfpathlineto{\pgfqpoint{4.727919in}{4.105390in}}%
\pgfpathlineto{\pgfqpoint{4.724474in}{4.108791in}}%
\pgfpathlineto{\pgfqpoint{4.721277in}{4.112000in}}%
\pgfpathlineto{\pgfqpoint{4.705034in}{4.128017in}}%
\pgfpathlineto{\pgfqpoint{4.687838in}{4.145257in}}%
\pgfpathlineto{\pgfqpoint{4.683730in}{4.149333in}}%
\pgfpathlineto{\pgfqpoint{4.666231in}{4.166540in}}%
\pgfpathlineto{\pgfqpoint{4.647758in}{4.185008in}}%
\pgfpathlineto{\pgfqpoint{4.646082in}{4.186667in}}%
\pgfpathlineto{\pgfqpoint{4.627372in}{4.205011in}}%
\pgfpathlineto{\pgfqpoint{4.608324in}{4.224000in}}%
\pgfpathlineto{\pgfqpoint{4.607677in}{4.224000in}}%
\pgfpathlineto{\pgfqpoint{4.605867in}{4.224000in}}%
\pgfpathlineto{\pgfqpoint{4.607677in}{4.222214in}}%
\pgfpathlineto{\pgfqpoint{4.626111in}{4.203838in}}%
\pgfpathlineto{\pgfqpoint{4.643624in}{4.186667in}}%
\pgfpathlineto{\pgfqpoint{4.647758in}{4.182577in}}%
\pgfpathlineto{\pgfqpoint{4.664972in}{4.165367in}}%
\pgfpathlineto{\pgfqpoint{4.681278in}{4.149333in}}%
\pgfpathlineto{\pgfqpoint{4.687838in}{4.142824in}}%
\pgfpathlineto{\pgfqpoint{4.703776in}{4.126845in}}%
\pgfpathlineto{\pgfqpoint{4.718830in}{4.112000in}}%
\pgfpathlineto{\pgfqpoint{4.723205in}{4.107609in}}%
\pgfpathlineto{\pgfqpoint{4.727919in}{4.102955in}}%
\pgfpathlineto{\pgfqpoint{4.742524in}{4.088271in}}%
\pgfpathlineto{\pgfqpoint{4.756281in}{4.074667in}}%
\pgfpathlineto{\pgfqpoint{4.761913in}{4.068997in}}%
\pgfpathlineto{\pgfqpoint{4.768000in}{4.062971in}}%
\pgfusepath{fill}%
\end{pgfscope}%
\begin{pgfscope}%
\pgfpathrectangle{\pgfqpoint{0.800000in}{0.528000in}}{\pgfqpoint{3.968000in}{3.696000in}}%
\pgfusepath{clip}%
\pgfsetbuttcap%
\pgfsetroundjoin%
\definecolor{currentfill}{rgb}{0.824940,0.884720,0.106217}%
\pgfsetfillcolor{currentfill}%
\pgfsetlinewidth{0.000000pt}%
\definecolor{currentstroke}{rgb}{0.000000,0.000000,0.000000}%
\pgfsetstrokecolor{currentstroke}%
\pgfsetdash{}{0pt}%
\pgfpathmoveto{\pgfqpoint{4.768000in}{4.067843in}}%
\pgfpathlineto{\pgfqpoint{4.764449in}{4.071359in}}%
\pgfpathlineto{\pgfqpoint{4.761163in}{4.074667in}}%
\pgfpathlineto{\pgfqpoint{4.745039in}{4.090613in}}%
\pgfpathlineto{\pgfqpoint{4.727919in}{4.107825in}}%
\pgfpathlineto{\pgfqpoint{4.725743in}{4.109973in}}%
\pgfpathlineto{\pgfqpoint{4.723724in}{4.112000in}}%
\pgfpathlineto{\pgfqpoint{4.706292in}{4.129189in}}%
\pgfpathlineto{\pgfqpoint{4.687838in}{4.147690in}}%
\pgfpathlineto{\pgfqpoint{4.686183in}{4.149333in}}%
\pgfpathlineto{\pgfqpoint{4.667490in}{4.167713in}}%
\pgfpathlineto{\pgfqpoint{4.648531in}{4.186667in}}%
\pgfpathlineto{\pgfqpoint{4.647758in}{4.187433in}}%
\pgfpathlineto{\pgfqpoint{4.628632in}{4.206185in}}%
\pgfpathlineto{\pgfqpoint{4.610762in}{4.224000in}}%
\pgfpathlineto{\pgfqpoint{4.608324in}{4.224000in}}%
\pgfpathlineto{\pgfqpoint{4.627372in}{4.205011in}}%
\pgfpathlineto{\pgfqpoint{4.646082in}{4.186667in}}%
\pgfpathlineto{\pgfqpoint{4.647758in}{4.185008in}}%
\pgfpathlineto{\pgfqpoint{4.666231in}{4.166540in}}%
\pgfpathlineto{\pgfqpoint{4.683730in}{4.149333in}}%
\pgfpathlineto{\pgfqpoint{4.687838in}{4.145257in}}%
\pgfpathlineto{\pgfqpoint{4.705034in}{4.128017in}}%
\pgfpathlineto{\pgfqpoint{4.721277in}{4.112000in}}%
\pgfpathlineto{\pgfqpoint{4.724474in}{4.108791in}}%
\pgfpathlineto{\pgfqpoint{4.727919in}{4.105390in}}%
\pgfpathlineto{\pgfqpoint{4.743781in}{4.089442in}}%
\pgfpathlineto{\pgfqpoint{4.758722in}{4.074667in}}%
\pgfpathlineto{\pgfqpoint{4.763181in}{4.070178in}}%
\pgfpathlineto{\pgfqpoint{4.768000in}{4.065407in}}%
\pgfusepath{fill}%
\end{pgfscope}%
\begin{pgfscope}%
\pgfpathrectangle{\pgfqpoint{0.800000in}{0.528000in}}{\pgfqpoint{3.968000in}{3.696000in}}%
\pgfusepath{clip}%
\pgfsetbuttcap%
\pgfsetroundjoin%
\definecolor{currentfill}{rgb}{0.824940,0.884720,0.106217}%
\pgfsetfillcolor{currentfill}%
\pgfsetlinewidth{0.000000pt}%
\definecolor{currentstroke}{rgb}{0.000000,0.000000,0.000000}%
\pgfsetstrokecolor{currentstroke}%
\pgfsetdash{}{0pt}%
\pgfpathmoveto{\pgfqpoint{4.768000in}{4.070280in}}%
\pgfpathlineto{\pgfqpoint{4.765717in}{4.072540in}}%
\pgfpathlineto{\pgfqpoint{4.763604in}{4.074667in}}%
\pgfpathlineto{\pgfqpoint{4.746296in}{4.091783in}}%
\pgfpathlineto{\pgfqpoint{4.727919in}{4.110259in}}%
\pgfpathlineto{\pgfqpoint{4.727012in}{4.111155in}}%
\pgfpathlineto{\pgfqpoint{4.726170in}{4.112000in}}%
\pgfpathlineto{\pgfqpoint{4.707551in}{4.130361in}}%
\pgfpathlineto{\pgfqpoint{4.688627in}{4.149333in}}%
\pgfpathlineto{\pgfqpoint{4.687838in}{4.150116in}}%
\pgfpathlineto{\pgfqpoint{4.668749in}{4.168886in}}%
\pgfpathlineto{\pgfqpoint{4.650964in}{4.186667in}}%
\pgfpathlineto{\pgfqpoint{4.647758in}{4.189843in}}%
\pgfpathlineto{\pgfqpoint{4.629892in}{4.207359in}}%
\pgfpathlineto{\pgfqpoint{4.613200in}{4.224000in}}%
\pgfpathlineto{\pgfqpoint{4.610762in}{4.224000in}}%
\pgfpathlineto{\pgfqpoint{4.628632in}{4.206185in}}%
\pgfpathlineto{\pgfqpoint{4.647758in}{4.187433in}}%
\pgfpathlineto{\pgfqpoint{4.648531in}{4.186667in}}%
\pgfpathlineto{\pgfqpoint{4.667490in}{4.167713in}}%
\pgfpathlineto{\pgfqpoint{4.686183in}{4.149333in}}%
\pgfpathlineto{\pgfqpoint{4.687838in}{4.147690in}}%
\pgfpathlineto{\pgfqpoint{4.706292in}{4.129189in}}%
\pgfpathlineto{\pgfqpoint{4.723724in}{4.112000in}}%
\pgfpathlineto{\pgfqpoint{4.725743in}{4.109973in}}%
\pgfpathlineto{\pgfqpoint{4.727919in}{4.107825in}}%
\pgfpathlineto{\pgfqpoint{4.745039in}{4.090613in}}%
\pgfpathlineto{\pgfqpoint{4.761163in}{4.074667in}}%
\pgfpathlineto{\pgfqpoint{4.764449in}{4.071359in}}%
\pgfpathlineto{\pgfqpoint{4.768000in}{4.067843in}}%
\pgfusepath{fill}%
\end{pgfscope}%
\begin{pgfscope}%
\pgfpathrectangle{\pgfqpoint{0.800000in}{0.528000in}}{\pgfqpoint{3.968000in}{3.696000in}}%
\pgfusepath{clip}%
\pgfsetbuttcap%
\pgfsetroundjoin%
\definecolor{currentfill}{rgb}{0.824940,0.884720,0.106217}%
\pgfsetfillcolor{currentfill}%
\pgfsetlinewidth{0.000000pt}%
\definecolor{currentstroke}{rgb}{0.000000,0.000000,0.000000}%
\pgfsetstrokecolor{currentstroke}%
\pgfsetdash{}{0pt}%
\pgfpathmoveto{\pgfqpoint{4.768000in}{4.072716in}}%
\pgfpathlineto{\pgfqpoint{4.766985in}{4.073721in}}%
\pgfpathlineto{\pgfqpoint{4.766045in}{4.074667in}}%
\pgfpathlineto{\pgfqpoint{4.747553in}{4.092954in}}%
\pgfpathlineto{\pgfqpoint{4.728610in}{4.112000in}}%
\pgfpathlineto{\pgfqpoint{4.727919in}{4.112688in}}%
\pgfpathlineto{\pgfqpoint{4.708809in}{4.131533in}}%
\pgfpathlineto{\pgfqpoint{4.691054in}{4.149333in}}%
\pgfpathlineto{\pgfqpoint{4.687838in}{4.152528in}}%
\pgfpathlineto{\pgfqpoint{4.670009in}{4.170059in}}%
\pgfpathlineto{\pgfqpoint{4.653397in}{4.186667in}}%
\pgfpathlineto{\pgfqpoint{4.647758in}{4.192253in}}%
\pgfpathlineto{\pgfqpoint{4.631153in}{4.208533in}}%
\pgfpathlineto{\pgfqpoint{4.615638in}{4.224000in}}%
\pgfpathlineto{\pgfqpoint{4.613200in}{4.224000in}}%
\pgfpathlineto{\pgfqpoint{4.629892in}{4.207359in}}%
\pgfpathlineto{\pgfqpoint{4.647758in}{4.189843in}}%
\pgfpathlineto{\pgfqpoint{4.650964in}{4.186667in}}%
\pgfpathlineto{\pgfqpoint{4.668749in}{4.168886in}}%
\pgfpathlineto{\pgfqpoint{4.687838in}{4.150116in}}%
\pgfpathlineto{\pgfqpoint{4.688627in}{4.149333in}}%
\pgfpathlineto{\pgfqpoint{4.707551in}{4.130361in}}%
\pgfpathlineto{\pgfqpoint{4.726170in}{4.112000in}}%
\pgfpathlineto{\pgfqpoint{4.727012in}{4.111155in}}%
\pgfpathlineto{\pgfqpoint{4.727919in}{4.110259in}}%
\pgfpathlineto{\pgfqpoint{4.746296in}{4.091783in}}%
\pgfpathlineto{\pgfqpoint{4.763604in}{4.074667in}}%
\pgfpathlineto{\pgfqpoint{4.765717in}{4.072540in}}%
\pgfpathlineto{\pgfqpoint{4.768000in}{4.070280in}}%
\pgfusepath{fill}%
\end{pgfscope}%
\begin{pgfscope}%
\pgfpathrectangle{\pgfqpoint{0.800000in}{0.528000in}}{\pgfqpoint{3.968000in}{3.696000in}}%
\pgfusepath{clip}%
\pgfsetbuttcap%
\pgfsetroundjoin%
\definecolor{currentfill}{rgb}{0.824940,0.884720,0.106217}%
\pgfsetfillcolor{currentfill}%
\pgfsetlinewidth{0.000000pt}%
\definecolor{currentstroke}{rgb}{0.000000,0.000000,0.000000}%
\pgfsetstrokecolor{currentstroke}%
\pgfsetdash{}{0pt}%
\pgfpathmoveto{\pgfqpoint{4.768000in}{4.075148in}}%
\pgfpathlineto{\pgfqpoint{4.748810in}{4.094125in}}%
\pgfpathlineto{\pgfqpoint{4.731031in}{4.112000in}}%
\pgfpathlineto{\pgfqpoint{4.727919in}{4.115101in}}%
\pgfpathlineto{\pgfqpoint{4.710067in}{4.132705in}}%
\pgfpathlineto{\pgfqpoint{4.693481in}{4.149333in}}%
\pgfpathlineto{\pgfqpoint{4.687838in}{4.154939in}}%
\pgfpathlineto{\pgfqpoint{4.671268in}{4.171232in}}%
\pgfpathlineto{\pgfqpoint{4.655829in}{4.186667in}}%
\pgfpathlineto{\pgfqpoint{4.647758in}{4.194663in}}%
\pgfpathlineto{\pgfqpoint{4.632413in}{4.209707in}}%
\pgfpathlineto{\pgfqpoint{4.618076in}{4.224000in}}%
\pgfpathlineto{\pgfqpoint{4.615638in}{4.224000in}}%
\pgfpathlineto{\pgfqpoint{4.631153in}{4.208533in}}%
\pgfpathlineto{\pgfqpoint{4.647758in}{4.192253in}}%
\pgfpathlineto{\pgfqpoint{4.653397in}{4.186667in}}%
\pgfpathlineto{\pgfqpoint{4.670009in}{4.170059in}}%
\pgfpathlineto{\pgfqpoint{4.687838in}{4.152528in}}%
\pgfpathlineto{\pgfqpoint{4.691054in}{4.149333in}}%
\pgfpathlineto{\pgfqpoint{4.708809in}{4.131533in}}%
\pgfpathlineto{\pgfqpoint{4.727919in}{4.112688in}}%
\pgfpathlineto{\pgfqpoint{4.728610in}{4.112000in}}%
\pgfpathlineto{\pgfqpoint{4.747553in}{4.092954in}}%
\pgfpathlineto{\pgfqpoint{4.766045in}{4.074667in}}%
\pgfpathlineto{\pgfqpoint{4.766985in}{4.073721in}}%
\pgfpathlineto{\pgfqpoint{4.768000in}{4.072716in}}%
\pgfpathlineto{\pgfqpoint{4.768000in}{4.074667in}}%
\pgfusepath{fill}%
\end{pgfscope}%
\begin{pgfscope}%
\pgfpathrectangle{\pgfqpoint{0.800000in}{0.528000in}}{\pgfqpoint{3.968000in}{3.696000in}}%
\pgfusepath{clip}%
\pgfsetbuttcap%
\pgfsetroundjoin%
\definecolor{currentfill}{rgb}{0.835270,0.886029,0.102646}%
\pgfsetfillcolor{currentfill}%
\pgfsetlinewidth{0.000000pt}%
\definecolor{currentstroke}{rgb}{0.000000,0.000000,0.000000}%
\pgfsetstrokecolor{currentstroke}%
\pgfsetdash{}{0pt}%
\pgfpathmoveto{\pgfqpoint{4.768000in}{4.077562in}}%
\pgfpathlineto{\pgfqpoint{4.750067in}{4.095296in}}%
\pgfpathlineto{\pgfqpoint{4.733453in}{4.112000in}}%
\pgfpathlineto{\pgfqpoint{4.727919in}{4.117513in}}%
\pgfpathlineto{\pgfqpoint{4.711325in}{4.133877in}}%
\pgfpathlineto{\pgfqpoint{4.695908in}{4.149333in}}%
\pgfpathlineto{\pgfqpoint{4.687838in}{4.157350in}}%
\pgfpathlineto{\pgfqpoint{4.672527in}{4.172405in}}%
\pgfpathlineto{\pgfqpoint{4.658262in}{4.186667in}}%
\pgfpathlineto{\pgfqpoint{4.647758in}{4.197072in}}%
\pgfpathlineto{\pgfqpoint{4.633673in}{4.210881in}}%
\pgfpathlineto{\pgfqpoint{4.620514in}{4.224000in}}%
\pgfpathlineto{\pgfqpoint{4.618076in}{4.224000in}}%
\pgfpathlineto{\pgfqpoint{4.632413in}{4.209707in}}%
\pgfpathlineto{\pgfqpoint{4.647758in}{4.194663in}}%
\pgfpathlineto{\pgfqpoint{4.655829in}{4.186667in}}%
\pgfpathlineto{\pgfqpoint{4.671268in}{4.171232in}}%
\pgfpathlineto{\pgfqpoint{4.687838in}{4.154939in}}%
\pgfpathlineto{\pgfqpoint{4.693481in}{4.149333in}}%
\pgfpathlineto{\pgfqpoint{4.710067in}{4.132705in}}%
\pgfpathlineto{\pgfqpoint{4.727919in}{4.115101in}}%
\pgfpathlineto{\pgfqpoint{4.731031in}{4.112000in}}%
\pgfpathlineto{\pgfqpoint{4.748810in}{4.094125in}}%
\pgfpathlineto{\pgfqpoint{4.768000in}{4.075148in}}%
\pgfusepath{fill}%
\end{pgfscope}%
\begin{pgfscope}%
\pgfpathrectangle{\pgfqpoint{0.800000in}{0.528000in}}{\pgfqpoint{3.968000in}{3.696000in}}%
\pgfusepath{clip}%
\pgfsetbuttcap%
\pgfsetroundjoin%
\definecolor{currentfill}{rgb}{0.835270,0.886029,0.102646}%
\pgfsetfillcolor{currentfill}%
\pgfsetlinewidth{0.000000pt}%
\definecolor{currentstroke}{rgb}{0.000000,0.000000,0.000000}%
\pgfsetstrokecolor{currentstroke}%
\pgfsetdash{}{0pt}%
\pgfpathmoveto{\pgfqpoint{4.768000in}{4.079976in}}%
\pgfpathlineto{\pgfqpoint{4.751324in}{4.096467in}}%
\pgfpathlineto{\pgfqpoint{4.735875in}{4.112000in}}%
\pgfpathlineto{\pgfqpoint{4.727919in}{4.119926in}}%
\pgfpathlineto{\pgfqpoint{4.712583in}{4.135049in}}%
\pgfpathlineto{\pgfqpoint{4.698335in}{4.149333in}}%
\pgfpathlineto{\pgfqpoint{4.687838in}{4.159761in}}%
\pgfpathlineto{\pgfqpoint{4.673787in}{4.173578in}}%
\pgfpathlineto{\pgfqpoint{4.660694in}{4.186667in}}%
\pgfpathlineto{\pgfqpoint{4.647758in}{4.199482in}}%
\pgfpathlineto{\pgfqpoint{4.634934in}{4.212055in}}%
\pgfpathlineto{\pgfqpoint{4.622952in}{4.224000in}}%
\pgfpathlineto{\pgfqpoint{4.620514in}{4.224000in}}%
\pgfpathlineto{\pgfqpoint{4.633673in}{4.210881in}}%
\pgfpathlineto{\pgfqpoint{4.647758in}{4.197072in}}%
\pgfpathlineto{\pgfqpoint{4.658262in}{4.186667in}}%
\pgfpathlineto{\pgfqpoint{4.672527in}{4.172405in}}%
\pgfpathlineto{\pgfqpoint{4.687838in}{4.157350in}}%
\pgfpathlineto{\pgfqpoint{4.695908in}{4.149333in}}%
\pgfpathlineto{\pgfqpoint{4.711325in}{4.133877in}}%
\pgfpathlineto{\pgfqpoint{4.727919in}{4.117513in}}%
\pgfpathlineto{\pgfqpoint{4.733453in}{4.112000in}}%
\pgfpathlineto{\pgfqpoint{4.750067in}{4.095296in}}%
\pgfpathlineto{\pgfqpoint{4.768000in}{4.077562in}}%
\pgfusepath{fill}%
\end{pgfscope}%
\begin{pgfscope}%
\pgfpathrectangle{\pgfqpoint{0.800000in}{0.528000in}}{\pgfqpoint{3.968000in}{3.696000in}}%
\pgfusepath{clip}%
\pgfsetbuttcap%
\pgfsetroundjoin%
\definecolor{currentfill}{rgb}{0.835270,0.886029,0.102646}%
\pgfsetfillcolor{currentfill}%
\pgfsetlinewidth{0.000000pt}%
\definecolor{currentstroke}{rgb}{0.000000,0.000000,0.000000}%
\pgfsetstrokecolor{currentstroke}%
\pgfsetdash{}{0pt}%
\pgfpathmoveto{\pgfqpoint{4.768000in}{4.082390in}}%
\pgfpathlineto{\pgfqpoint{4.752581in}{4.097638in}}%
\pgfpathlineto{\pgfqpoint{4.738297in}{4.112000in}}%
\pgfpathlineto{\pgfqpoint{4.727919in}{4.122339in}}%
\pgfpathlineto{\pgfqpoint{4.713841in}{4.136221in}}%
\pgfpathlineto{\pgfqpoint{4.700762in}{4.149333in}}%
\pgfpathlineto{\pgfqpoint{4.687838in}{4.162172in}}%
\pgfpathlineto{\pgfqpoint{4.675046in}{4.174751in}}%
\pgfpathlineto{\pgfqpoint{4.663127in}{4.186667in}}%
\pgfpathlineto{\pgfqpoint{4.647758in}{4.201892in}}%
\pgfpathlineto{\pgfqpoint{4.636194in}{4.213229in}}%
\pgfpathlineto{\pgfqpoint{4.625390in}{4.224000in}}%
\pgfpathlineto{\pgfqpoint{4.622952in}{4.224000in}}%
\pgfpathlineto{\pgfqpoint{4.634934in}{4.212055in}}%
\pgfpathlineto{\pgfqpoint{4.647758in}{4.199482in}}%
\pgfpathlineto{\pgfqpoint{4.660694in}{4.186667in}}%
\pgfpathlineto{\pgfqpoint{4.673787in}{4.173578in}}%
\pgfpathlineto{\pgfqpoint{4.687838in}{4.159761in}}%
\pgfpathlineto{\pgfqpoint{4.698335in}{4.149333in}}%
\pgfpathlineto{\pgfqpoint{4.712583in}{4.135049in}}%
\pgfpathlineto{\pgfqpoint{4.727919in}{4.119926in}}%
\pgfpathlineto{\pgfqpoint{4.735875in}{4.112000in}}%
\pgfpathlineto{\pgfqpoint{4.751324in}{4.096467in}}%
\pgfpathlineto{\pgfqpoint{4.768000in}{4.079976in}}%
\pgfusepath{fill}%
\end{pgfscope}%
\begin{pgfscope}%
\pgfpathrectangle{\pgfqpoint{0.800000in}{0.528000in}}{\pgfqpoint{3.968000in}{3.696000in}}%
\pgfusepath{clip}%
\pgfsetbuttcap%
\pgfsetroundjoin%
\definecolor{currentfill}{rgb}{0.845561,0.887322,0.099702}%
\pgfsetfillcolor{currentfill}%
\pgfsetlinewidth{0.000000pt}%
\definecolor{currentstroke}{rgb}{0.000000,0.000000,0.000000}%
\pgfsetstrokecolor{currentstroke}%
\pgfsetdash{}{0pt}%
\pgfpathmoveto{\pgfqpoint{4.768000in}{4.084804in}}%
\pgfpathlineto{\pgfqpoint{4.753838in}{4.098809in}}%
\pgfpathlineto{\pgfqpoint{4.740718in}{4.112000in}}%
\pgfpathlineto{\pgfqpoint{4.727919in}{4.124751in}}%
\pgfpathlineto{\pgfqpoint{4.715100in}{4.137393in}}%
\pgfpathlineto{\pgfqpoint{4.703189in}{4.149333in}}%
\pgfpathlineto{\pgfqpoint{4.687838in}{4.164584in}}%
\pgfpathlineto{\pgfqpoint{4.676305in}{4.175924in}}%
\pgfpathlineto{\pgfqpoint{4.665560in}{4.186667in}}%
\pgfpathlineto{\pgfqpoint{4.647758in}{4.204301in}}%
\pgfpathlineto{\pgfqpoint{4.637455in}{4.214403in}}%
\pgfpathlineto{\pgfqpoint{4.627828in}{4.224000in}}%
\pgfpathlineto{\pgfqpoint{4.625390in}{4.224000in}}%
\pgfpathlineto{\pgfqpoint{4.636194in}{4.213229in}}%
\pgfpathlineto{\pgfqpoint{4.647758in}{4.201892in}}%
\pgfpathlineto{\pgfqpoint{4.663127in}{4.186667in}}%
\pgfpathlineto{\pgfqpoint{4.675046in}{4.174751in}}%
\pgfpathlineto{\pgfqpoint{4.687838in}{4.162172in}}%
\pgfpathlineto{\pgfqpoint{4.700762in}{4.149333in}}%
\pgfpathlineto{\pgfqpoint{4.713841in}{4.136221in}}%
\pgfpathlineto{\pgfqpoint{4.727919in}{4.122339in}}%
\pgfpathlineto{\pgfqpoint{4.738297in}{4.112000in}}%
\pgfpathlineto{\pgfqpoint{4.752581in}{4.097638in}}%
\pgfpathlineto{\pgfqpoint{4.768000in}{4.082390in}}%
\pgfusepath{fill}%
\end{pgfscope}%
\begin{pgfscope}%
\pgfpathrectangle{\pgfqpoint{0.800000in}{0.528000in}}{\pgfqpoint{3.968000in}{3.696000in}}%
\pgfusepath{clip}%
\pgfsetbuttcap%
\pgfsetroundjoin%
\definecolor{currentfill}{rgb}{0.845561,0.887322,0.099702}%
\pgfsetfillcolor{currentfill}%
\pgfsetlinewidth{0.000000pt}%
\definecolor{currentstroke}{rgb}{0.000000,0.000000,0.000000}%
\pgfsetstrokecolor{currentstroke}%
\pgfsetdash{}{0pt}%
\pgfpathmoveto{\pgfqpoint{4.768000in}{4.087218in}}%
\pgfpathlineto{\pgfqpoint{4.755095in}{4.099980in}}%
\pgfpathlineto{\pgfqpoint{4.743140in}{4.112000in}}%
\pgfpathlineto{\pgfqpoint{4.727919in}{4.127164in}}%
\pgfpathlineto{\pgfqpoint{4.716358in}{4.138565in}}%
\pgfpathlineto{\pgfqpoint{4.705617in}{4.149333in}}%
\pgfpathlineto{\pgfqpoint{4.687838in}{4.166995in}}%
\pgfpathlineto{\pgfqpoint{4.677564in}{4.177097in}}%
\pgfpathlineto{\pgfqpoint{4.667992in}{4.186667in}}%
\pgfpathlineto{\pgfqpoint{4.647758in}{4.206711in}}%
\pgfpathlineto{\pgfqpoint{4.638715in}{4.215577in}}%
\pgfpathlineto{\pgfqpoint{4.630266in}{4.224000in}}%
\pgfpathlineto{\pgfqpoint{4.627828in}{4.224000in}}%
\pgfpathlineto{\pgfqpoint{4.637455in}{4.214403in}}%
\pgfpathlineto{\pgfqpoint{4.647758in}{4.204301in}}%
\pgfpathlineto{\pgfqpoint{4.665560in}{4.186667in}}%
\pgfpathlineto{\pgfqpoint{4.676305in}{4.175924in}}%
\pgfpathlineto{\pgfqpoint{4.687838in}{4.164584in}}%
\pgfpathlineto{\pgfqpoint{4.703189in}{4.149333in}}%
\pgfpathlineto{\pgfqpoint{4.715100in}{4.137393in}}%
\pgfpathlineto{\pgfqpoint{4.727919in}{4.124751in}}%
\pgfpathlineto{\pgfqpoint{4.740718in}{4.112000in}}%
\pgfpathlineto{\pgfqpoint{4.753838in}{4.098809in}}%
\pgfpathlineto{\pgfqpoint{4.768000in}{4.084804in}}%
\pgfusepath{fill}%
\end{pgfscope}%
\begin{pgfscope}%
\pgfpathrectangle{\pgfqpoint{0.800000in}{0.528000in}}{\pgfqpoint{3.968000in}{3.696000in}}%
\pgfusepath{clip}%
\pgfsetbuttcap%
\pgfsetroundjoin%
\definecolor{currentfill}{rgb}{0.845561,0.887322,0.099702}%
\pgfsetfillcolor{currentfill}%
\pgfsetlinewidth{0.000000pt}%
\definecolor{currentstroke}{rgb}{0.000000,0.000000,0.000000}%
\pgfsetstrokecolor{currentstroke}%
\pgfsetdash{}{0pt}%
\pgfpathmoveto{\pgfqpoint{4.768000in}{4.089632in}}%
\pgfpathlineto{\pgfqpoint{4.756352in}{4.101151in}}%
\pgfpathlineto{\pgfqpoint{4.745562in}{4.112000in}}%
\pgfpathlineto{\pgfqpoint{4.727919in}{4.129577in}}%
\pgfpathlineto{\pgfqpoint{4.717616in}{4.139736in}}%
\pgfpathlineto{\pgfqpoint{4.708044in}{4.149333in}}%
\pgfpathlineto{\pgfqpoint{4.687838in}{4.169406in}}%
\pgfpathlineto{\pgfqpoint{4.678824in}{4.178270in}}%
\pgfpathlineto{\pgfqpoint{4.670425in}{4.186667in}}%
\pgfpathlineto{\pgfqpoint{4.647758in}{4.209121in}}%
\pgfpathlineto{\pgfqpoint{4.639975in}{4.216751in}}%
\pgfpathlineto{\pgfqpoint{4.632704in}{4.224000in}}%
\pgfpathlineto{\pgfqpoint{4.630266in}{4.224000in}}%
\pgfpathlineto{\pgfqpoint{4.638715in}{4.215577in}}%
\pgfpathlineto{\pgfqpoint{4.647758in}{4.206711in}}%
\pgfpathlineto{\pgfqpoint{4.667992in}{4.186667in}}%
\pgfpathlineto{\pgfqpoint{4.677564in}{4.177097in}}%
\pgfpathlineto{\pgfqpoint{4.687838in}{4.166995in}}%
\pgfpathlineto{\pgfqpoint{4.705617in}{4.149333in}}%
\pgfpathlineto{\pgfqpoint{4.716358in}{4.138565in}}%
\pgfpathlineto{\pgfqpoint{4.727919in}{4.127164in}}%
\pgfpathlineto{\pgfqpoint{4.743140in}{4.112000in}}%
\pgfpathlineto{\pgfqpoint{4.755095in}{4.099980in}}%
\pgfpathlineto{\pgfqpoint{4.768000in}{4.087218in}}%
\pgfusepath{fill}%
\end{pgfscope}%
\begin{pgfscope}%
\pgfpathrectangle{\pgfqpoint{0.800000in}{0.528000in}}{\pgfqpoint{3.968000in}{3.696000in}}%
\pgfusepath{clip}%
\pgfsetbuttcap%
\pgfsetroundjoin%
\definecolor{currentfill}{rgb}{0.845561,0.887322,0.099702}%
\pgfsetfillcolor{currentfill}%
\pgfsetlinewidth{0.000000pt}%
\definecolor{currentstroke}{rgb}{0.000000,0.000000,0.000000}%
\pgfsetstrokecolor{currentstroke}%
\pgfsetdash{}{0pt}%
\pgfpathmoveto{\pgfqpoint{4.768000in}{4.092047in}}%
\pgfpathlineto{\pgfqpoint{4.757610in}{4.102322in}}%
\pgfpathlineto{\pgfqpoint{4.747984in}{4.112000in}}%
\pgfpathlineto{\pgfqpoint{4.727919in}{4.131989in}}%
\pgfpathlineto{\pgfqpoint{4.718874in}{4.140908in}}%
\pgfpathlineto{\pgfqpoint{4.710471in}{4.149333in}}%
\pgfpathlineto{\pgfqpoint{4.687838in}{4.171817in}}%
\pgfpathlineto{\pgfqpoint{4.680083in}{4.179443in}}%
\pgfpathlineto{\pgfqpoint{4.672857in}{4.186667in}}%
\pgfpathlineto{\pgfqpoint{4.647758in}{4.211531in}}%
\pgfpathlineto{\pgfqpoint{4.641236in}{4.217925in}}%
\pgfpathlineto{\pgfqpoint{4.635142in}{4.224000in}}%
\pgfpathlineto{\pgfqpoint{4.632704in}{4.224000in}}%
\pgfpathlineto{\pgfqpoint{4.639975in}{4.216751in}}%
\pgfpathlineto{\pgfqpoint{4.647758in}{4.209121in}}%
\pgfpathlineto{\pgfqpoint{4.670425in}{4.186667in}}%
\pgfpathlineto{\pgfqpoint{4.678824in}{4.178270in}}%
\pgfpathlineto{\pgfqpoint{4.687838in}{4.169406in}}%
\pgfpathlineto{\pgfqpoint{4.708044in}{4.149333in}}%
\pgfpathlineto{\pgfqpoint{4.717616in}{4.139736in}}%
\pgfpathlineto{\pgfqpoint{4.727919in}{4.129577in}}%
\pgfpathlineto{\pgfqpoint{4.745562in}{4.112000in}}%
\pgfpathlineto{\pgfqpoint{4.756352in}{4.101151in}}%
\pgfpathlineto{\pgfqpoint{4.768000in}{4.089632in}}%
\pgfusepath{fill}%
\end{pgfscope}%
\begin{pgfscope}%
\pgfpathrectangle{\pgfqpoint{0.800000in}{0.528000in}}{\pgfqpoint{3.968000in}{3.696000in}}%
\pgfusepath{clip}%
\pgfsetbuttcap%
\pgfsetroundjoin%
\definecolor{currentfill}{rgb}{0.855810,0.888601,0.097452}%
\pgfsetfillcolor{currentfill}%
\pgfsetlinewidth{0.000000pt}%
\definecolor{currentstroke}{rgb}{0.000000,0.000000,0.000000}%
\pgfsetstrokecolor{currentstroke}%
\pgfsetdash{}{0pt}%
\pgfpathmoveto{\pgfqpoint{4.768000in}{4.094461in}}%
\pgfpathlineto{\pgfqpoint{4.758867in}{4.103493in}}%
\pgfpathlineto{\pgfqpoint{4.750405in}{4.112000in}}%
\pgfpathlineto{\pgfqpoint{4.727919in}{4.134402in}}%
\pgfpathlineto{\pgfqpoint{4.720132in}{4.142080in}}%
\pgfpathlineto{\pgfqpoint{4.712898in}{4.149333in}}%
\pgfpathlineto{\pgfqpoint{4.687838in}{4.174228in}}%
\pgfpathlineto{\pgfqpoint{4.681342in}{4.180616in}}%
\pgfpathlineto{\pgfqpoint{4.675290in}{4.186667in}}%
\pgfpathlineto{\pgfqpoint{4.647758in}{4.213940in}}%
\pgfpathlineto{\pgfqpoint{4.642496in}{4.219099in}}%
\pgfpathlineto{\pgfqpoint{4.637580in}{4.224000in}}%
\pgfpathlineto{\pgfqpoint{4.635142in}{4.224000in}}%
\pgfpathlineto{\pgfqpoint{4.641236in}{4.217925in}}%
\pgfpathlineto{\pgfqpoint{4.647758in}{4.211531in}}%
\pgfpathlineto{\pgfqpoint{4.672857in}{4.186667in}}%
\pgfpathlineto{\pgfqpoint{4.680083in}{4.179443in}}%
\pgfpathlineto{\pgfqpoint{4.687838in}{4.171817in}}%
\pgfpathlineto{\pgfqpoint{4.710471in}{4.149333in}}%
\pgfpathlineto{\pgfqpoint{4.718874in}{4.140908in}}%
\pgfpathlineto{\pgfqpoint{4.727919in}{4.131989in}}%
\pgfpathlineto{\pgfqpoint{4.747984in}{4.112000in}}%
\pgfpathlineto{\pgfqpoint{4.757610in}{4.102322in}}%
\pgfpathlineto{\pgfqpoint{4.768000in}{4.092047in}}%
\pgfusepath{fill}%
\end{pgfscope}%
\begin{pgfscope}%
\pgfpathrectangle{\pgfqpoint{0.800000in}{0.528000in}}{\pgfqpoint{3.968000in}{3.696000in}}%
\pgfusepath{clip}%
\pgfsetbuttcap%
\pgfsetroundjoin%
\definecolor{currentfill}{rgb}{0.855810,0.888601,0.097452}%
\pgfsetfillcolor{currentfill}%
\pgfsetlinewidth{0.000000pt}%
\definecolor{currentstroke}{rgb}{0.000000,0.000000,0.000000}%
\pgfsetstrokecolor{currentstroke}%
\pgfsetdash{}{0pt}%
\pgfpathmoveto{\pgfqpoint{4.768000in}{4.096875in}}%
\pgfpathlineto{\pgfqpoint{4.760124in}{4.104664in}}%
\pgfpathlineto{\pgfqpoint{4.752827in}{4.112000in}}%
\pgfpathlineto{\pgfqpoint{4.727919in}{4.136814in}}%
\pgfpathlineto{\pgfqpoint{4.721391in}{4.143252in}}%
\pgfpathlineto{\pgfqpoint{4.715325in}{4.149333in}}%
\pgfpathlineto{\pgfqpoint{4.687838in}{4.176640in}}%
\pgfpathlineto{\pgfqpoint{4.682602in}{4.181789in}}%
\pgfpathlineto{\pgfqpoint{4.677722in}{4.186667in}}%
\pgfpathlineto{\pgfqpoint{4.647758in}{4.216350in}}%
\pgfpathlineto{\pgfqpoint{4.643756in}{4.220273in}}%
\pgfpathlineto{\pgfqpoint{4.640018in}{4.224000in}}%
\pgfpathlineto{\pgfqpoint{4.637580in}{4.224000in}}%
\pgfpathlineto{\pgfqpoint{4.642496in}{4.219099in}}%
\pgfpathlineto{\pgfqpoint{4.647758in}{4.213940in}}%
\pgfpathlineto{\pgfqpoint{4.675290in}{4.186667in}}%
\pgfpathlineto{\pgfqpoint{4.681342in}{4.180616in}}%
\pgfpathlineto{\pgfqpoint{4.687838in}{4.174228in}}%
\pgfpathlineto{\pgfqpoint{4.712898in}{4.149333in}}%
\pgfpathlineto{\pgfqpoint{4.720132in}{4.142080in}}%
\pgfpathlineto{\pgfqpoint{4.727919in}{4.134402in}}%
\pgfpathlineto{\pgfqpoint{4.750405in}{4.112000in}}%
\pgfpathlineto{\pgfqpoint{4.758867in}{4.103493in}}%
\pgfpathlineto{\pgfqpoint{4.768000in}{4.094461in}}%
\pgfusepath{fill}%
\end{pgfscope}%
\begin{pgfscope}%
\pgfpathrectangle{\pgfqpoint{0.800000in}{0.528000in}}{\pgfqpoint{3.968000in}{3.696000in}}%
\pgfusepath{clip}%
\pgfsetbuttcap%
\pgfsetroundjoin%
\definecolor{currentfill}{rgb}{0.855810,0.888601,0.097452}%
\pgfsetfillcolor{currentfill}%
\pgfsetlinewidth{0.000000pt}%
\definecolor{currentstroke}{rgb}{0.000000,0.000000,0.000000}%
\pgfsetstrokecolor{currentstroke}%
\pgfsetdash{}{0pt}%
\pgfpathmoveto{\pgfqpoint{4.768000in}{4.099289in}}%
\pgfpathlineto{\pgfqpoint{4.761381in}{4.105835in}}%
\pgfpathlineto{\pgfqpoint{4.755249in}{4.112000in}}%
\pgfpathlineto{\pgfqpoint{4.727919in}{4.139227in}}%
\pgfpathlineto{\pgfqpoint{4.722649in}{4.144424in}}%
\pgfpathlineto{\pgfqpoint{4.717752in}{4.149333in}}%
\pgfpathlineto{\pgfqpoint{4.687838in}{4.179051in}}%
\pgfpathlineto{\pgfqpoint{4.683861in}{4.182962in}}%
\pgfpathlineto{\pgfqpoint{4.680155in}{4.186667in}}%
\pgfpathlineto{\pgfqpoint{4.647758in}{4.218760in}}%
\pgfpathlineto{\pgfqpoint{4.645017in}{4.221447in}}%
\pgfpathlineto{\pgfqpoint{4.642456in}{4.224000in}}%
\pgfpathlineto{\pgfqpoint{4.640018in}{4.224000in}}%
\pgfpathlineto{\pgfqpoint{4.643756in}{4.220273in}}%
\pgfpathlineto{\pgfqpoint{4.647758in}{4.216350in}}%
\pgfpathlineto{\pgfqpoint{4.677722in}{4.186667in}}%
\pgfpathlineto{\pgfqpoint{4.682602in}{4.181789in}}%
\pgfpathlineto{\pgfqpoint{4.687838in}{4.176640in}}%
\pgfpathlineto{\pgfqpoint{4.715325in}{4.149333in}}%
\pgfpathlineto{\pgfqpoint{4.721391in}{4.143252in}}%
\pgfpathlineto{\pgfqpoint{4.727919in}{4.136814in}}%
\pgfpathlineto{\pgfqpoint{4.752827in}{4.112000in}}%
\pgfpathlineto{\pgfqpoint{4.760124in}{4.104664in}}%
\pgfpathlineto{\pgfqpoint{4.768000in}{4.096875in}}%
\pgfusepath{fill}%
\end{pgfscope}%
\begin{pgfscope}%
\pgfpathrectangle{\pgfqpoint{0.800000in}{0.528000in}}{\pgfqpoint{3.968000in}{3.696000in}}%
\pgfusepath{clip}%
\pgfsetbuttcap%
\pgfsetroundjoin%
\definecolor{currentfill}{rgb}{0.855810,0.888601,0.097452}%
\pgfsetfillcolor{currentfill}%
\pgfsetlinewidth{0.000000pt}%
\definecolor{currentstroke}{rgb}{0.000000,0.000000,0.000000}%
\pgfsetstrokecolor{currentstroke}%
\pgfsetdash{}{0pt}%
\pgfpathmoveto{\pgfqpoint{4.768000in}{4.101703in}}%
\pgfpathlineto{\pgfqpoint{4.762638in}{4.107006in}}%
\pgfpathlineto{\pgfqpoint{4.757670in}{4.112000in}}%
\pgfpathlineto{\pgfqpoint{4.727919in}{4.141640in}}%
\pgfpathlineto{\pgfqpoint{4.723907in}{4.145596in}}%
\pgfpathlineto{\pgfqpoint{4.720179in}{4.149333in}}%
\pgfpathlineto{\pgfqpoint{4.687838in}{4.181462in}}%
\pgfpathlineto{\pgfqpoint{4.685120in}{4.184135in}}%
\pgfpathlineto{\pgfqpoint{4.682588in}{4.186667in}}%
\pgfpathlineto{\pgfqpoint{4.647758in}{4.221170in}}%
\pgfpathlineto{\pgfqpoint{4.646277in}{4.222621in}}%
\pgfpathlineto{\pgfqpoint{4.644894in}{4.224000in}}%
\pgfpathlineto{\pgfqpoint{4.642456in}{4.224000in}}%
\pgfpathlineto{\pgfqpoint{4.645017in}{4.221447in}}%
\pgfpathlineto{\pgfqpoint{4.647758in}{4.218760in}}%
\pgfpathlineto{\pgfqpoint{4.680155in}{4.186667in}}%
\pgfpathlineto{\pgfqpoint{4.683861in}{4.182962in}}%
\pgfpathlineto{\pgfqpoint{4.687838in}{4.179051in}}%
\pgfpathlineto{\pgfqpoint{4.717752in}{4.149333in}}%
\pgfpathlineto{\pgfqpoint{4.722649in}{4.144424in}}%
\pgfpathlineto{\pgfqpoint{4.727919in}{4.139227in}}%
\pgfpathlineto{\pgfqpoint{4.755249in}{4.112000in}}%
\pgfpathlineto{\pgfqpoint{4.761381in}{4.105835in}}%
\pgfpathlineto{\pgfqpoint{4.768000in}{4.099289in}}%
\pgfusepath{fill}%
\end{pgfscope}%
\begin{pgfscope}%
\pgfpathrectangle{\pgfqpoint{0.800000in}{0.528000in}}{\pgfqpoint{3.968000in}{3.696000in}}%
\pgfusepath{clip}%
\pgfsetbuttcap%
\pgfsetroundjoin%
\definecolor{currentfill}{rgb}{0.866013,0.889868,0.095953}%
\pgfsetfillcolor{currentfill}%
\pgfsetlinewidth{0.000000pt}%
\definecolor{currentstroke}{rgb}{0.000000,0.000000,0.000000}%
\pgfsetstrokecolor{currentstroke}%
\pgfsetdash{}{0pt}%
\pgfpathmoveto{\pgfqpoint{4.768000in}{4.104117in}}%
\pgfpathlineto{\pgfqpoint{4.763895in}{4.108176in}}%
\pgfpathlineto{\pgfqpoint{4.760092in}{4.112000in}}%
\pgfpathlineto{\pgfqpoint{4.727919in}{4.144052in}}%
\pgfpathlineto{\pgfqpoint{4.725165in}{4.146768in}}%
\pgfpathlineto{\pgfqpoint{4.722607in}{4.149333in}}%
\pgfpathlineto{\pgfqpoint{4.687838in}{4.183873in}}%
\pgfpathlineto{\pgfqpoint{4.686379in}{4.185308in}}%
\pgfpathlineto{\pgfqpoint{4.685020in}{4.186667in}}%
\pgfpathlineto{\pgfqpoint{4.647758in}{4.223579in}}%
\pgfpathlineto{\pgfqpoint{4.647538in}{4.223795in}}%
\pgfpathlineto{\pgfqpoint{4.647332in}{4.224000in}}%
\pgfpathlineto{\pgfqpoint{4.644894in}{4.224000in}}%
\pgfpathlineto{\pgfqpoint{4.646277in}{4.222621in}}%
\pgfpathlineto{\pgfqpoint{4.647758in}{4.221170in}}%
\pgfpathlineto{\pgfqpoint{4.682588in}{4.186667in}}%
\pgfpathlineto{\pgfqpoint{4.685120in}{4.184135in}}%
\pgfpathlineto{\pgfqpoint{4.687838in}{4.181462in}}%
\pgfpathlineto{\pgfqpoint{4.720179in}{4.149333in}}%
\pgfpathlineto{\pgfqpoint{4.723907in}{4.145596in}}%
\pgfpathlineto{\pgfqpoint{4.727919in}{4.141640in}}%
\pgfpathlineto{\pgfqpoint{4.757670in}{4.112000in}}%
\pgfpathlineto{\pgfqpoint{4.762638in}{4.107006in}}%
\pgfpathlineto{\pgfqpoint{4.768000in}{4.101703in}}%
\pgfusepath{fill}%
\end{pgfscope}%
\begin{pgfscope}%
\pgfpathrectangle{\pgfqpoint{0.800000in}{0.528000in}}{\pgfqpoint{3.968000in}{3.696000in}}%
\pgfusepath{clip}%
\pgfsetbuttcap%
\pgfsetroundjoin%
\definecolor{currentfill}{rgb}{0.866013,0.889868,0.095953}%
\pgfsetfillcolor{currentfill}%
\pgfsetlinewidth{0.000000pt}%
\definecolor{currentstroke}{rgb}{0.000000,0.000000,0.000000}%
\pgfsetstrokecolor{currentstroke}%
\pgfsetdash{}{0pt}%
\pgfpathmoveto{\pgfqpoint{4.768000in}{4.106531in}}%
\pgfpathlineto{\pgfqpoint{4.765152in}{4.109347in}}%
\pgfpathlineto{\pgfqpoint{4.762514in}{4.112000in}}%
\pgfpathlineto{\pgfqpoint{4.727919in}{4.146465in}}%
\pgfpathlineto{\pgfqpoint{4.726423in}{4.147940in}}%
\pgfpathlineto{\pgfqpoint{4.725034in}{4.149333in}}%
\pgfpathlineto{\pgfqpoint{4.687838in}{4.186284in}}%
\pgfpathlineto{\pgfqpoint{4.687639in}{4.186481in}}%
\pgfpathlineto{\pgfqpoint{4.687453in}{4.186667in}}%
\pgfpathlineto{\pgfqpoint{4.681380in}{4.192683in}}%
\pgfpathlineto{\pgfqpoint{4.649750in}{4.224000in}}%
\pgfpathlineto{\pgfqpoint{4.647758in}{4.224000in}}%
\pgfpathlineto{\pgfqpoint{4.647332in}{4.224000in}}%
\pgfpathlineto{\pgfqpoint{4.647538in}{4.223795in}}%
\pgfpathlineto{\pgfqpoint{4.647758in}{4.223579in}}%
\pgfpathlineto{\pgfqpoint{4.685020in}{4.186667in}}%
\pgfpathlineto{\pgfqpoint{4.686379in}{4.185308in}}%
\pgfpathlineto{\pgfqpoint{4.687838in}{4.183873in}}%
\pgfpathlineto{\pgfqpoint{4.722607in}{4.149333in}}%
\pgfpathlineto{\pgfqpoint{4.725165in}{4.146768in}}%
\pgfpathlineto{\pgfqpoint{4.727919in}{4.144052in}}%
\pgfpathlineto{\pgfqpoint{4.760092in}{4.112000in}}%
\pgfpathlineto{\pgfqpoint{4.763895in}{4.108176in}}%
\pgfpathlineto{\pgfqpoint{4.768000in}{4.104117in}}%
\pgfusepath{fill}%
\end{pgfscope}%
\begin{pgfscope}%
\pgfpathrectangle{\pgfqpoint{0.800000in}{0.528000in}}{\pgfqpoint{3.968000in}{3.696000in}}%
\pgfusepath{clip}%
\pgfsetbuttcap%
\pgfsetroundjoin%
\definecolor{currentfill}{rgb}{0.866013,0.889868,0.095953}%
\pgfsetfillcolor{currentfill}%
\pgfsetlinewidth{0.000000pt}%
\definecolor{currentstroke}{rgb}{0.000000,0.000000,0.000000}%
\pgfsetstrokecolor{currentstroke}%
\pgfsetdash{}{0pt}%
\pgfpathmoveto{\pgfqpoint{4.768000in}{4.108945in}}%
\pgfpathlineto{\pgfqpoint{4.766409in}{4.110518in}}%
\pgfpathlineto{\pgfqpoint{4.764936in}{4.112000in}}%
\pgfpathlineto{\pgfqpoint{4.727919in}{4.148878in}}%
\pgfpathlineto{\pgfqpoint{4.727682in}{4.149112in}}%
\pgfpathlineto{\pgfqpoint{4.727461in}{4.149333in}}%
\pgfpathlineto{\pgfqpoint{4.720573in}{4.156176in}}%
\pgfpathlineto{\pgfqpoint{4.689864in}{4.186667in}}%
\pgfpathlineto{\pgfqpoint{4.688879in}{4.187636in}}%
\pgfpathlineto{\pgfqpoint{4.687838in}{4.188677in}}%
\pgfpathlineto{\pgfqpoint{4.652163in}{4.224000in}}%
\pgfpathlineto{\pgfqpoint{4.649750in}{4.224000in}}%
\pgfpathlineto{\pgfqpoint{4.681380in}{4.192683in}}%
\pgfpathlineto{\pgfqpoint{4.687453in}{4.186667in}}%
\pgfpathlineto{\pgfqpoint{4.687639in}{4.186481in}}%
\pgfpathlineto{\pgfqpoint{4.687838in}{4.186284in}}%
\pgfpathlineto{\pgfqpoint{4.725034in}{4.149333in}}%
\pgfpathlineto{\pgfqpoint{4.726423in}{4.147940in}}%
\pgfpathlineto{\pgfqpoint{4.727919in}{4.146465in}}%
\pgfpathlineto{\pgfqpoint{4.762514in}{4.112000in}}%
\pgfpathlineto{\pgfqpoint{4.765152in}{4.109347in}}%
\pgfpathlineto{\pgfqpoint{4.768000in}{4.106531in}}%
\pgfusepath{fill}%
\end{pgfscope}%
\begin{pgfscope}%
\pgfpathrectangle{\pgfqpoint{0.800000in}{0.528000in}}{\pgfqpoint{3.968000in}{3.696000in}}%
\pgfusepath{clip}%
\pgfsetbuttcap%
\pgfsetroundjoin%
\definecolor{currentfill}{rgb}{0.866013,0.889868,0.095953}%
\pgfsetfillcolor{currentfill}%
\pgfsetlinewidth{0.000000pt}%
\definecolor{currentstroke}{rgb}{0.000000,0.000000,0.000000}%
\pgfsetstrokecolor{currentstroke}%
\pgfsetdash{}{0pt}%
\pgfpathmoveto{\pgfqpoint{4.768000in}{4.111359in}}%
\pgfpathlineto{\pgfqpoint{4.767666in}{4.111689in}}%
\pgfpathlineto{\pgfqpoint{4.767357in}{4.112000in}}%
\pgfpathlineto{\pgfqpoint{4.758120in}{4.121203in}}%
\pgfpathlineto{\pgfqpoint{4.729868in}{4.149333in}}%
\pgfpathlineto{\pgfqpoint{4.728922in}{4.150267in}}%
\pgfpathlineto{\pgfqpoint{4.727919in}{4.151273in}}%
\pgfpathlineto{\pgfqpoint{4.692272in}{4.186667in}}%
\pgfpathlineto{\pgfqpoint{4.690116in}{4.188788in}}%
\pgfpathlineto{\pgfqpoint{4.687838in}{4.191067in}}%
\pgfpathlineto{\pgfqpoint{4.654576in}{4.224000in}}%
\pgfpathlineto{\pgfqpoint{4.652163in}{4.224000in}}%
\pgfpathlineto{\pgfqpoint{4.687838in}{4.188677in}}%
\pgfpathlineto{\pgfqpoint{4.688879in}{4.187636in}}%
\pgfpathlineto{\pgfqpoint{4.689864in}{4.186667in}}%
\pgfpathlineto{\pgfqpoint{4.720573in}{4.156176in}}%
\pgfpathlineto{\pgfqpoint{4.727461in}{4.149333in}}%
\pgfpathlineto{\pgfqpoint{4.727682in}{4.149112in}}%
\pgfpathlineto{\pgfqpoint{4.727919in}{4.148878in}}%
\pgfpathlineto{\pgfqpoint{4.764936in}{4.112000in}}%
\pgfpathlineto{\pgfqpoint{4.766409in}{4.110518in}}%
\pgfpathlineto{\pgfqpoint{4.768000in}{4.108945in}}%
\pgfusepath{fill}%
\end{pgfscope}%
\begin{pgfscope}%
\pgfpathrectangle{\pgfqpoint{0.800000in}{0.528000in}}{\pgfqpoint{3.968000in}{3.696000in}}%
\pgfusepath{clip}%
\pgfsetbuttcap%
\pgfsetroundjoin%
\definecolor{currentfill}{rgb}{0.876168,0.891125,0.095250}%
\pgfsetfillcolor{currentfill}%
\pgfsetlinewidth{0.000000pt}%
\definecolor{currentstroke}{rgb}{0.000000,0.000000,0.000000}%
\pgfsetstrokecolor{currentstroke}%
\pgfsetdash{}{0pt}%
\pgfpathmoveto{\pgfqpoint{4.768000in}{4.113757in}}%
\pgfpathlineto{\pgfqpoint{4.732271in}{4.149333in}}%
\pgfpathlineto{\pgfqpoint{4.730158in}{4.151418in}}%
\pgfpathlineto{\pgfqpoint{4.727919in}{4.153664in}}%
\pgfpathlineto{\pgfqpoint{4.694680in}{4.186667in}}%
\pgfpathlineto{\pgfqpoint{4.691353in}{4.189941in}}%
\pgfpathlineto{\pgfqpoint{4.687838in}{4.193456in}}%
\pgfpathlineto{\pgfqpoint{4.656989in}{4.224000in}}%
\pgfpathlineto{\pgfqpoint{4.654576in}{4.224000in}}%
\pgfpathlineto{\pgfqpoint{4.687838in}{4.191067in}}%
\pgfpathlineto{\pgfqpoint{4.690116in}{4.188788in}}%
\pgfpathlineto{\pgfqpoint{4.692272in}{4.186667in}}%
\pgfpathlineto{\pgfqpoint{4.727919in}{4.151273in}}%
\pgfpathlineto{\pgfqpoint{4.728922in}{4.150267in}}%
\pgfpathlineto{\pgfqpoint{4.729868in}{4.149333in}}%
\pgfpathlineto{\pgfqpoint{4.758120in}{4.121203in}}%
\pgfpathlineto{\pgfqpoint{4.767357in}{4.112000in}}%
\pgfpathlineto{\pgfqpoint{4.767666in}{4.111689in}}%
\pgfpathlineto{\pgfqpoint{4.768000in}{4.111359in}}%
\pgfpathlineto{\pgfqpoint{4.768000in}{4.112000in}}%
\pgfusepath{fill}%
\end{pgfscope}%
\begin{pgfscope}%
\pgfpathrectangle{\pgfqpoint{0.800000in}{0.528000in}}{\pgfqpoint{3.968000in}{3.696000in}}%
\pgfusepath{clip}%
\pgfsetbuttcap%
\pgfsetroundjoin%
\definecolor{currentfill}{rgb}{0.876168,0.891125,0.095250}%
\pgfsetfillcolor{currentfill}%
\pgfsetlinewidth{0.000000pt}%
\definecolor{currentstroke}{rgb}{0.000000,0.000000,0.000000}%
\pgfsetstrokecolor{currentstroke}%
\pgfsetdash{}{0pt}%
\pgfpathmoveto{\pgfqpoint{4.768000in}{4.116150in}}%
\pgfpathlineto{\pgfqpoint{4.734673in}{4.149333in}}%
\pgfpathlineto{\pgfqpoint{4.731394in}{4.152570in}}%
\pgfpathlineto{\pgfqpoint{4.727919in}{4.156054in}}%
\pgfpathlineto{\pgfqpoint{4.697088in}{4.186667in}}%
\pgfpathlineto{\pgfqpoint{4.692590in}{4.191093in}}%
\pgfpathlineto{\pgfqpoint{4.687838in}{4.195845in}}%
\pgfpathlineto{\pgfqpoint{4.659403in}{4.224000in}}%
\pgfpathlineto{\pgfqpoint{4.656989in}{4.224000in}}%
\pgfpathlineto{\pgfqpoint{4.687838in}{4.193456in}}%
\pgfpathlineto{\pgfqpoint{4.691353in}{4.189941in}}%
\pgfpathlineto{\pgfqpoint{4.694680in}{4.186667in}}%
\pgfpathlineto{\pgfqpoint{4.727919in}{4.153664in}}%
\pgfpathlineto{\pgfqpoint{4.730158in}{4.151418in}}%
\pgfpathlineto{\pgfqpoint{4.732271in}{4.149333in}}%
\pgfpathlineto{\pgfqpoint{4.768000in}{4.113757in}}%
\pgfusepath{fill}%
\end{pgfscope}%
\begin{pgfscope}%
\pgfpathrectangle{\pgfqpoint{0.800000in}{0.528000in}}{\pgfqpoint{3.968000in}{3.696000in}}%
\pgfusepath{clip}%
\pgfsetbuttcap%
\pgfsetroundjoin%
\definecolor{currentfill}{rgb}{0.876168,0.891125,0.095250}%
\pgfsetfillcolor{currentfill}%
\pgfsetlinewidth{0.000000pt}%
\definecolor{currentstroke}{rgb}{0.000000,0.000000,0.000000}%
\pgfsetstrokecolor{currentstroke}%
\pgfsetdash{}{0pt}%
\pgfpathmoveto{\pgfqpoint{4.768000in}{4.118542in}}%
\pgfpathlineto{\pgfqpoint{4.737076in}{4.149333in}}%
\pgfpathlineto{\pgfqpoint{4.732630in}{4.153721in}}%
\pgfpathlineto{\pgfqpoint{4.727919in}{4.158445in}}%
\pgfpathlineto{\pgfqpoint{4.699496in}{4.186667in}}%
\pgfpathlineto{\pgfqpoint{4.693827in}{4.192245in}}%
\pgfpathlineto{\pgfqpoint{4.687838in}{4.198235in}}%
\pgfpathlineto{\pgfqpoint{4.661816in}{4.224000in}}%
\pgfpathlineto{\pgfqpoint{4.659403in}{4.224000in}}%
\pgfpathlineto{\pgfqpoint{4.687838in}{4.195845in}}%
\pgfpathlineto{\pgfqpoint{4.692590in}{4.191093in}}%
\pgfpathlineto{\pgfqpoint{4.697088in}{4.186667in}}%
\pgfpathlineto{\pgfqpoint{4.727919in}{4.156054in}}%
\pgfpathlineto{\pgfqpoint{4.731394in}{4.152570in}}%
\pgfpathlineto{\pgfqpoint{4.734673in}{4.149333in}}%
\pgfpathlineto{\pgfqpoint{4.768000in}{4.116150in}}%
\pgfusepath{fill}%
\end{pgfscope}%
\begin{pgfscope}%
\pgfpathrectangle{\pgfqpoint{0.800000in}{0.528000in}}{\pgfqpoint{3.968000in}{3.696000in}}%
\pgfusepath{clip}%
\pgfsetbuttcap%
\pgfsetroundjoin%
\definecolor{currentfill}{rgb}{0.886271,0.892374,0.095374}%
\pgfsetfillcolor{currentfill}%
\pgfsetlinewidth{0.000000pt}%
\definecolor{currentstroke}{rgb}{0.000000,0.000000,0.000000}%
\pgfsetstrokecolor{currentstroke}%
\pgfsetdash{}{0pt}%
\pgfpathmoveto{\pgfqpoint{4.768000in}{4.120934in}}%
\pgfpathlineto{\pgfqpoint{4.739478in}{4.149333in}}%
\pgfpathlineto{\pgfqpoint{4.733866in}{4.154872in}}%
\pgfpathlineto{\pgfqpoint{4.727919in}{4.160836in}}%
\pgfpathlineto{\pgfqpoint{4.701904in}{4.186667in}}%
\pgfpathlineto{\pgfqpoint{4.695064in}{4.193397in}}%
\pgfpathlineto{\pgfqpoint{4.687838in}{4.200624in}}%
\pgfpathlineto{\pgfqpoint{4.664229in}{4.224000in}}%
\pgfpathlineto{\pgfqpoint{4.661816in}{4.224000in}}%
\pgfpathlineto{\pgfqpoint{4.687838in}{4.198235in}}%
\pgfpathlineto{\pgfqpoint{4.693827in}{4.192245in}}%
\pgfpathlineto{\pgfqpoint{4.699496in}{4.186667in}}%
\pgfpathlineto{\pgfqpoint{4.727919in}{4.158445in}}%
\pgfpathlineto{\pgfqpoint{4.732630in}{4.153721in}}%
\pgfpathlineto{\pgfqpoint{4.737076in}{4.149333in}}%
\pgfpathlineto{\pgfqpoint{4.768000in}{4.118542in}}%
\pgfusepath{fill}%
\end{pgfscope}%
\begin{pgfscope}%
\pgfpathrectangle{\pgfqpoint{0.800000in}{0.528000in}}{\pgfqpoint{3.968000in}{3.696000in}}%
\pgfusepath{clip}%
\pgfsetbuttcap%
\pgfsetroundjoin%
\definecolor{currentfill}{rgb}{0.886271,0.892374,0.095374}%
\pgfsetfillcolor{currentfill}%
\pgfsetlinewidth{0.000000pt}%
\definecolor{currentstroke}{rgb}{0.000000,0.000000,0.000000}%
\pgfsetstrokecolor{currentstroke}%
\pgfsetdash{}{0pt}%
\pgfpathmoveto{\pgfqpoint{4.768000in}{4.123327in}}%
\pgfpathlineto{\pgfqpoint{4.741881in}{4.149333in}}%
\pgfpathlineto{\pgfqpoint{4.735102in}{4.156023in}}%
\pgfpathlineto{\pgfqpoint{4.727919in}{4.163227in}}%
\pgfpathlineto{\pgfqpoint{4.704312in}{4.186667in}}%
\pgfpathlineto{\pgfqpoint{4.696301in}{4.194549in}}%
\pgfpathlineto{\pgfqpoint{4.687838in}{4.203014in}}%
\pgfpathlineto{\pgfqpoint{4.666643in}{4.224000in}}%
\pgfpathlineto{\pgfqpoint{4.664229in}{4.224000in}}%
\pgfpathlineto{\pgfqpoint{4.687838in}{4.200624in}}%
\pgfpathlineto{\pgfqpoint{4.695064in}{4.193397in}}%
\pgfpathlineto{\pgfqpoint{4.701904in}{4.186667in}}%
\pgfpathlineto{\pgfqpoint{4.727919in}{4.160836in}}%
\pgfpathlineto{\pgfqpoint{4.733866in}{4.154872in}}%
\pgfpathlineto{\pgfqpoint{4.739478in}{4.149333in}}%
\pgfpathlineto{\pgfqpoint{4.768000in}{4.120934in}}%
\pgfusepath{fill}%
\end{pgfscope}%
\begin{pgfscope}%
\pgfpathrectangle{\pgfqpoint{0.800000in}{0.528000in}}{\pgfqpoint{3.968000in}{3.696000in}}%
\pgfusepath{clip}%
\pgfsetbuttcap%
\pgfsetroundjoin%
\definecolor{currentfill}{rgb}{0.886271,0.892374,0.095374}%
\pgfsetfillcolor{currentfill}%
\pgfsetlinewidth{0.000000pt}%
\definecolor{currentstroke}{rgb}{0.000000,0.000000,0.000000}%
\pgfsetstrokecolor{currentstroke}%
\pgfsetdash{}{0pt}%
\pgfpathmoveto{\pgfqpoint{4.768000in}{4.125719in}}%
\pgfpathlineto{\pgfqpoint{4.744284in}{4.149333in}}%
\pgfpathlineto{\pgfqpoint{4.736337in}{4.157175in}}%
\pgfpathlineto{\pgfqpoint{4.727919in}{4.165618in}}%
\pgfpathlineto{\pgfqpoint{4.706720in}{4.186667in}}%
\pgfpathlineto{\pgfqpoint{4.697538in}{4.195702in}}%
\pgfpathlineto{\pgfqpoint{4.687838in}{4.205403in}}%
\pgfpathlineto{\pgfqpoint{4.669056in}{4.224000in}}%
\pgfpathlineto{\pgfqpoint{4.666643in}{4.224000in}}%
\pgfpathlineto{\pgfqpoint{4.687838in}{4.203014in}}%
\pgfpathlineto{\pgfqpoint{4.696301in}{4.194549in}}%
\pgfpathlineto{\pgfqpoint{4.704312in}{4.186667in}}%
\pgfpathlineto{\pgfqpoint{4.727919in}{4.163227in}}%
\pgfpathlineto{\pgfqpoint{4.735102in}{4.156023in}}%
\pgfpathlineto{\pgfqpoint{4.741881in}{4.149333in}}%
\pgfpathlineto{\pgfqpoint{4.768000in}{4.123327in}}%
\pgfusepath{fill}%
\end{pgfscope}%
\begin{pgfscope}%
\pgfpathrectangle{\pgfqpoint{0.800000in}{0.528000in}}{\pgfqpoint{3.968000in}{3.696000in}}%
\pgfusepath{clip}%
\pgfsetbuttcap%
\pgfsetroundjoin%
\definecolor{currentfill}{rgb}{0.886271,0.892374,0.095374}%
\pgfsetfillcolor{currentfill}%
\pgfsetlinewidth{0.000000pt}%
\definecolor{currentstroke}{rgb}{0.000000,0.000000,0.000000}%
\pgfsetstrokecolor{currentstroke}%
\pgfsetdash{}{0pt}%
\pgfpathmoveto{\pgfqpoint{4.768000in}{4.128111in}}%
\pgfpathlineto{\pgfqpoint{4.746686in}{4.149333in}}%
\pgfpathlineto{\pgfqpoint{4.737573in}{4.158326in}}%
\pgfpathlineto{\pgfqpoint{4.727919in}{4.168009in}}%
\pgfpathlineto{\pgfqpoint{4.709128in}{4.186667in}}%
\pgfpathlineto{\pgfqpoint{4.698775in}{4.196854in}}%
\pgfpathlineto{\pgfqpoint{4.687838in}{4.207793in}}%
\pgfpathlineto{\pgfqpoint{4.671469in}{4.224000in}}%
\pgfpathlineto{\pgfqpoint{4.669056in}{4.224000in}}%
\pgfpathlineto{\pgfqpoint{4.687838in}{4.205403in}}%
\pgfpathlineto{\pgfqpoint{4.697538in}{4.195702in}}%
\pgfpathlineto{\pgfqpoint{4.706720in}{4.186667in}}%
\pgfpathlineto{\pgfqpoint{4.727919in}{4.165618in}}%
\pgfpathlineto{\pgfqpoint{4.736337in}{4.157175in}}%
\pgfpathlineto{\pgfqpoint{4.744284in}{4.149333in}}%
\pgfpathlineto{\pgfqpoint{4.768000in}{4.125719in}}%
\pgfusepath{fill}%
\end{pgfscope}%
\begin{pgfscope}%
\pgfpathrectangle{\pgfqpoint{0.800000in}{0.528000in}}{\pgfqpoint{3.968000in}{3.696000in}}%
\pgfusepath{clip}%
\pgfsetbuttcap%
\pgfsetroundjoin%
\definecolor{currentfill}{rgb}{0.896320,0.893616,0.096335}%
\pgfsetfillcolor{currentfill}%
\pgfsetlinewidth{0.000000pt}%
\definecolor{currentstroke}{rgb}{0.000000,0.000000,0.000000}%
\pgfsetstrokecolor{currentstroke}%
\pgfsetdash{}{0pt}%
\pgfpathmoveto{\pgfqpoint{4.768000in}{4.130503in}}%
\pgfpathlineto{\pgfqpoint{4.749089in}{4.149333in}}%
\pgfpathlineto{\pgfqpoint{4.738809in}{4.159477in}}%
\pgfpathlineto{\pgfqpoint{4.727919in}{4.170399in}}%
\pgfpathlineto{\pgfqpoint{4.711536in}{4.186667in}}%
\pgfpathlineto{\pgfqpoint{4.700012in}{4.198006in}}%
\pgfpathlineto{\pgfqpoint{4.687838in}{4.210182in}}%
\pgfpathlineto{\pgfqpoint{4.673882in}{4.224000in}}%
\pgfpathlineto{\pgfqpoint{4.671469in}{4.224000in}}%
\pgfpathlineto{\pgfqpoint{4.687838in}{4.207793in}}%
\pgfpathlineto{\pgfqpoint{4.698775in}{4.196854in}}%
\pgfpathlineto{\pgfqpoint{4.709128in}{4.186667in}}%
\pgfpathlineto{\pgfqpoint{4.727919in}{4.168009in}}%
\pgfpathlineto{\pgfqpoint{4.737573in}{4.158326in}}%
\pgfpathlineto{\pgfqpoint{4.746686in}{4.149333in}}%
\pgfpathlineto{\pgfqpoint{4.768000in}{4.128111in}}%
\pgfusepath{fill}%
\end{pgfscope}%
\begin{pgfscope}%
\pgfpathrectangle{\pgfqpoint{0.800000in}{0.528000in}}{\pgfqpoint{3.968000in}{3.696000in}}%
\pgfusepath{clip}%
\pgfsetbuttcap%
\pgfsetroundjoin%
\definecolor{currentfill}{rgb}{0.896320,0.893616,0.096335}%
\pgfsetfillcolor{currentfill}%
\pgfsetlinewidth{0.000000pt}%
\definecolor{currentstroke}{rgb}{0.000000,0.000000,0.000000}%
\pgfsetstrokecolor{currentstroke}%
\pgfsetdash{}{0pt}%
\pgfpathmoveto{\pgfqpoint{4.768000in}{4.132896in}}%
\pgfpathlineto{\pgfqpoint{4.751491in}{4.149333in}}%
\pgfpathlineto{\pgfqpoint{4.740045in}{4.160628in}}%
\pgfpathlineto{\pgfqpoint{4.727919in}{4.172790in}}%
\pgfpathlineto{\pgfqpoint{4.713944in}{4.186667in}}%
\pgfpathlineto{\pgfqpoint{4.701249in}{4.199158in}}%
\pgfpathlineto{\pgfqpoint{4.687838in}{4.212571in}}%
\pgfpathlineto{\pgfqpoint{4.676296in}{4.224000in}}%
\pgfpathlineto{\pgfqpoint{4.673882in}{4.224000in}}%
\pgfpathlineto{\pgfqpoint{4.687838in}{4.210182in}}%
\pgfpathlineto{\pgfqpoint{4.700012in}{4.198006in}}%
\pgfpathlineto{\pgfqpoint{4.711536in}{4.186667in}}%
\pgfpathlineto{\pgfqpoint{4.727919in}{4.170399in}}%
\pgfpathlineto{\pgfqpoint{4.738809in}{4.159477in}}%
\pgfpathlineto{\pgfqpoint{4.749089in}{4.149333in}}%
\pgfpathlineto{\pgfqpoint{4.768000in}{4.130503in}}%
\pgfusepath{fill}%
\end{pgfscope}%
\begin{pgfscope}%
\pgfpathrectangle{\pgfqpoint{0.800000in}{0.528000in}}{\pgfqpoint{3.968000in}{3.696000in}}%
\pgfusepath{clip}%
\pgfsetbuttcap%
\pgfsetroundjoin%
\definecolor{currentfill}{rgb}{0.896320,0.893616,0.096335}%
\pgfsetfillcolor{currentfill}%
\pgfsetlinewidth{0.000000pt}%
\definecolor{currentstroke}{rgb}{0.000000,0.000000,0.000000}%
\pgfsetstrokecolor{currentstroke}%
\pgfsetdash{}{0pt}%
\pgfpathmoveto{\pgfqpoint{4.768000in}{4.135288in}}%
\pgfpathlineto{\pgfqpoint{4.753894in}{4.149333in}}%
\pgfpathlineto{\pgfqpoint{4.741281in}{4.161779in}}%
\pgfpathlineto{\pgfqpoint{4.727919in}{4.175181in}}%
\pgfpathlineto{\pgfqpoint{4.716352in}{4.186667in}}%
\pgfpathlineto{\pgfqpoint{4.702486in}{4.200310in}}%
\pgfpathlineto{\pgfqpoint{4.687838in}{4.214961in}}%
\pgfpathlineto{\pgfqpoint{4.678709in}{4.224000in}}%
\pgfpathlineto{\pgfqpoint{4.676296in}{4.224000in}}%
\pgfpathlineto{\pgfqpoint{4.687838in}{4.212571in}}%
\pgfpathlineto{\pgfqpoint{4.701249in}{4.199158in}}%
\pgfpathlineto{\pgfqpoint{4.713944in}{4.186667in}}%
\pgfpathlineto{\pgfqpoint{4.727919in}{4.172790in}}%
\pgfpathlineto{\pgfqpoint{4.740045in}{4.160628in}}%
\pgfpathlineto{\pgfqpoint{4.751491in}{4.149333in}}%
\pgfpathlineto{\pgfqpoint{4.768000in}{4.132896in}}%
\pgfusepath{fill}%
\end{pgfscope}%
\begin{pgfscope}%
\pgfpathrectangle{\pgfqpoint{0.800000in}{0.528000in}}{\pgfqpoint{3.968000in}{3.696000in}}%
\pgfusepath{clip}%
\pgfsetbuttcap%
\pgfsetroundjoin%
\definecolor{currentfill}{rgb}{0.896320,0.893616,0.096335}%
\pgfsetfillcolor{currentfill}%
\pgfsetlinewidth{0.000000pt}%
\definecolor{currentstroke}{rgb}{0.000000,0.000000,0.000000}%
\pgfsetstrokecolor{currentstroke}%
\pgfsetdash{}{0pt}%
\pgfpathmoveto{\pgfqpoint{4.768000in}{4.137680in}}%
\pgfpathlineto{\pgfqpoint{4.756297in}{4.149333in}}%
\pgfpathlineto{\pgfqpoint{4.742517in}{4.162931in}}%
\pgfpathlineto{\pgfqpoint{4.727919in}{4.177572in}}%
\pgfpathlineto{\pgfqpoint{4.718760in}{4.186667in}}%
\pgfpathlineto{\pgfqpoint{4.703723in}{4.201463in}}%
\pgfpathlineto{\pgfqpoint{4.687838in}{4.217350in}}%
\pgfpathlineto{\pgfqpoint{4.681122in}{4.224000in}}%
\pgfpathlineto{\pgfqpoint{4.678709in}{4.224000in}}%
\pgfpathlineto{\pgfqpoint{4.687838in}{4.214961in}}%
\pgfpathlineto{\pgfqpoint{4.702486in}{4.200310in}}%
\pgfpathlineto{\pgfqpoint{4.716352in}{4.186667in}}%
\pgfpathlineto{\pgfqpoint{4.727919in}{4.175181in}}%
\pgfpathlineto{\pgfqpoint{4.741281in}{4.161779in}}%
\pgfpathlineto{\pgfqpoint{4.753894in}{4.149333in}}%
\pgfpathlineto{\pgfqpoint{4.768000in}{4.135288in}}%
\pgfusepath{fill}%
\end{pgfscope}%
\begin{pgfscope}%
\pgfpathrectangle{\pgfqpoint{0.800000in}{0.528000in}}{\pgfqpoint{3.968000in}{3.696000in}}%
\pgfusepath{clip}%
\pgfsetbuttcap%
\pgfsetroundjoin%
\definecolor{currentfill}{rgb}{0.906311,0.894855,0.098125}%
\pgfsetfillcolor{currentfill}%
\pgfsetlinewidth{0.000000pt}%
\definecolor{currentstroke}{rgb}{0.000000,0.000000,0.000000}%
\pgfsetstrokecolor{currentstroke}%
\pgfsetdash{}{0pt}%
\pgfpathmoveto{\pgfqpoint{4.768000in}{4.140073in}}%
\pgfpathlineto{\pgfqpoint{4.758699in}{4.149333in}}%
\pgfpathlineto{\pgfqpoint{4.743753in}{4.164082in}}%
\pgfpathlineto{\pgfqpoint{4.727919in}{4.179963in}}%
\pgfpathlineto{\pgfqpoint{4.721168in}{4.186667in}}%
\pgfpathlineto{\pgfqpoint{4.704960in}{4.202615in}}%
\pgfpathlineto{\pgfqpoint{4.687838in}{4.219740in}}%
\pgfpathlineto{\pgfqpoint{4.683536in}{4.224000in}}%
\pgfpathlineto{\pgfqpoint{4.681122in}{4.224000in}}%
\pgfpathlineto{\pgfqpoint{4.687838in}{4.217350in}}%
\pgfpathlineto{\pgfqpoint{4.703723in}{4.201463in}}%
\pgfpathlineto{\pgfqpoint{4.718760in}{4.186667in}}%
\pgfpathlineto{\pgfqpoint{4.727919in}{4.177572in}}%
\pgfpathlineto{\pgfqpoint{4.742517in}{4.162931in}}%
\pgfpathlineto{\pgfqpoint{4.756297in}{4.149333in}}%
\pgfpathlineto{\pgfqpoint{4.768000in}{4.137680in}}%
\pgfusepath{fill}%
\end{pgfscope}%
\begin{pgfscope}%
\pgfpathrectangle{\pgfqpoint{0.800000in}{0.528000in}}{\pgfqpoint{3.968000in}{3.696000in}}%
\pgfusepath{clip}%
\pgfsetbuttcap%
\pgfsetroundjoin%
\definecolor{currentfill}{rgb}{0.906311,0.894855,0.098125}%
\pgfsetfillcolor{currentfill}%
\pgfsetlinewidth{0.000000pt}%
\definecolor{currentstroke}{rgb}{0.000000,0.000000,0.000000}%
\pgfsetstrokecolor{currentstroke}%
\pgfsetdash{}{0pt}%
\pgfpathmoveto{\pgfqpoint{4.768000in}{4.142465in}}%
\pgfpathlineto{\pgfqpoint{4.761102in}{4.149333in}}%
\pgfpathlineto{\pgfqpoint{4.744989in}{4.165233in}}%
\pgfpathlineto{\pgfqpoint{4.727919in}{4.182354in}}%
\pgfpathlineto{\pgfqpoint{4.723575in}{4.186667in}}%
\pgfpathlineto{\pgfqpoint{4.706197in}{4.203767in}}%
\pgfpathlineto{\pgfqpoint{4.687838in}{4.222129in}}%
\pgfpathlineto{\pgfqpoint{4.685949in}{4.224000in}}%
\pgfpathlineto{\pgfqpoint{4.683536in}{4.224000in}}%
\pgfpathlineto{\pgfqpoint{4.687838in}{4.219740in}}%
\pgfpathlineto{\pgfqpoint{4.704960in}{4.202615in}}%
\pgfpathlineto{\pgfqpoint{4.721168in}{4.186667in}}%
\pgfpathlineto{\pgfqpoint{4.727919in}{4.179963in}}%
\pgfpathlineto{\pgfqpoint{4.743753in}{4.164082in}}%
\pgfpathlineto{\pgfqpoint{4.758699in}{4.149333in}}%
\pgfpathlineto{\pgfqpoint{4.768000in}{4.140073in}}%
\pgfusepath{fill}%
\end{pgfscope}%
\begin{pgfscope}%
\pgfpathrectangle{\pgfqpoint{0.800000in}{0.528000in}}{\pgfqpoint{3.968000in}{3.696000in}}%
\pgfusepath{clip}%
\pgfsetbuttcap%
\pgfsetroundjoin%
\definecolor{currentfill}{rgb}{0.906311,0.894855,0.098125}%
\pgfsetfillcolor{currentfill}%
\pgfsetlinewidth{0.000000pt}%
\definecolor{currentstroke}{rgb}{0.000000,0.000000,0.000000}%
\pgfsetstrokecolor{currentstroke}%
\pgfsetdash{}{0pt}%
\pgfpathmoveto{\pgfqpoint{4.768000in}{4.144857in}}%
\pgfpathlineto{\pgfqpoint{4.763504in}{4.149333in}}%
\pgfpathlineto{\pgfqpoint{4.746225in}{4.166384in}}%
\pgfpathlineto{\pgfqpoint{4.727919in}{4.184745in}}%
\pgfpathlineto{\pgfqpoint{4.725983in}{4.186667in}}%
\pgfpathlineto{\pgfqpoint{4.707434in}{4.204919in}}%
\pgfpathlineto{\pgfqpoint{4.688357in}{4.224000in}}%
\pgfpathlineto{\pgfqpoint{4.687838in}{4.224000in}}%
\pgfpathlineto{\pgfqpoint{4.685949in}{4.224000in}}%
\pgfpathlineto{\pgfqpoint{4.687838in}{4.222129in}}%
\pgfpathlineto{\pgfqpoint{4.706197in}{4.203767in}}%
\pgfpathlineto{\pgfqpoint{4.723575in}{4.186667in}}%
\pgfpathlineto{\pgfqpoint{4.727919in}{4.182354in}}%
\pgfpathlineto{\pgfqpoint{4.744989in}{4.165233in}}%
\pgfpathlineto{\pgfqpoint{4.761102in}{4.149333in}}%
\pgfpathlineto{\pgfqpoint{4.768000in}{4.142465in}}%
\pgfusepath{fill}%
\end{pgfscope}%
\begin{pgfscope}%
\pgfpathrectangle{\pgfqpoint{0.800000in}{0.528000in}}{\pgfqpoint{3.968000in}{3.696000in}}%
\pgfusepath{clip}%
\pgfsetbuttcap%
\pgfsetroundjoin%
\definecolor{currentfill}{rgb}{0.916242,0.896091,0.100717}%
\pgfsetfillcolor{currentfill}%
\pgfsetlinewidth{0.000000pt}%
\definecolor{currentstroke}{rgb}{0.000000,0.000000,0.000000}%
\pgfsetstrokecolor{currentstroke}%
\pgfsetdash{}{0pt}%
\pgfpathmoveto{\pgfqpoint{4.768000in}{4.147249in}}%
\pgfpathlineto{\pgfqpoint{4.765907in}{4.149333in}}%
\pgfpathlineto{\pgfqpoint{4.747461in}{4.167536in}}%
\pgfpathlineto{\pgfqpoint{4.728387in}{4.186667in}}%
\pgfpathlineto{\pgfqpoint{4.727919in}{4.187131in}}%
\pgfpathlineto{\pgfqpoint{4.708671in}{4.206072in}}%
\pgfpathlineto{\pgfqpoint{4.690746in}{4.224000in}}%
\pgfpathlineto{\pgfqpoint{4.688357in}{4.224000in}}%
\pgfpathlineto{\pgfqpoint{4.707434in}{4.204919in}}%
\pgfpathlineto{\pgfqpoint{4.725983in}{4.186667in}}%
\pgfpathlineto{\pgfqpoint{4.727919in}{4.184745in}}%
\pgfpathlineto{\pgfqpoint{4.746225in}{4.166384in}}%
\pgfpathlineto{\pgfqpoint{4.763504in}{4.149333in}}%
\pgfpathlineto{\pgfqpoint{4.768000in}{4.144857in}}%
\pgfusepath{fill}%
\end{pgfscope}%
\begin{pgfscope}%
\pgfpathrectangle{\pgfqpoint{0.800000in}{0.528000in}}{\pgfqpoint{3.968000in}{3.696000in}}%
\pgfusepath{clip}%
\pgfsetbuttcap%
\pgfsetroundjoin%
\definecolor{currentfill}{rgb}{0.916242,0.896091,0.100717}%
\pgfsetfillcolor{currentfill}%
\pgfsetlinewidth{0.000000pt}%
\definecolor{currentstroke}{rgb}{0.000000,0.000000,0.000000}%
\pgfsetstrokecolor{currentstroke}%
\pgfsetdash{}{0pt}%
\pgfpathmoveto{\pgfqpoint{4.768000in}{4.149639in}}%
\pgfpathlineto{\pgfqpoint{4.748697in}{4.168687in}}%
\pgfpathlineto{\pgfqpoint{4.730770in}{4.186667in}}%
\pgfpathlineto{\pgfqpoint{4.727919in}{4.189501in}}%
\pgfpathlineto{\pgfqpoint{4.709908in}{4.207224in}}%
\pgfpathlineto{\pgfqpoint{4.693135in}{4.224000in}}%
\pgfpathlineto{\pgfqpoint{4.690746in}{4.224000in}}%
\pgfpathlineto{\pgfqpoint{4.708671in}{4.206072in}}%
\pgfpathlineto{\pgfqpoint{4.727919in}{4.187131in}}%
\pgfpathlineto{\pgfqpoint{4.728387in}{4.186667in}}%
\pgfpathlineto{\pgfqpoint{4.747461in}{4.167536in}}%
\pgfpathlineto{\pgfqpoint{4.765907in}{4.149333in}}%
\pgfpathlineto{\pgfqpoint{4.768000in}{4.147249in}}%
\pgfpathlineto{\pgfqpoint{4.768000in}{4.149333in}}%
\pgfusepath{fill}%
\end{pgfscope}%
\begin{pgfscope}%
\pgfpathrectangle{\pgfqpoint{0.800000in}{0.528000in}}{\pgfqpoint{3.968000in}{3.696000in}}%
\pgfusepath{clip}%
\pgfsetbuttcap%
\pgfsetroundjoin%
\definecolor{currentfill}{rgb}{0.916242,0.896091,0.100717}%
\pgfsetfillcolor{currentfill}%
\pgfsetlinewidth{0.000000pt}%
\definecolor{currentstroke}{rgb}{0.000000,0.000000,0.000000}%
\pgfsetstrokecolor{currentstroke}%
\pgfsetdash{}{0pt}%
\pgfpathmoveto{\pgfqpoint{4.768000in}{4.152010in}}%
\pgfpathlineto{\pgfqpoint{4.749933in}{4.169838in}}%
\pgfpathlineto{\pgfqpoint{4.733154in}{4.186667in}}%
\pgfpathlineto{\pgfqpoint{4.727919in}{4.191870in}}%
\pgfpathlineto{\pgfqpoint{4.711145in}{4.208376in}}%
\pgfpathlineto{\pgfqpoint{4.695524in}{4.224000in}}%
\pgfpathlineto{\pgfqpoint{4.693135in}{4.224000in}}%
\pgfpathlineto{\pgfqpoint{4.709908in}{4.207224in}}%
\pgfpathlineto{\pgfqpoint{4.727919in}{4.189501in}}%
\pgfpathlineto{\pgfqpoint{4.730770in}{4.186667in}}%
\pgfpathlineto{\pgfqpoint{4.748697in}{4.168687in}}%
\pgfpathlineto{\pgfqpoint{4.768000in}{4.149639in}}%
\pgfusepath{fill}%
\end{pgfscope}%
\begin{pgfscope}%
\pgfpathrectangle{\pgfqpoint{0.800000in}{0.528000in}}{\pgfqpoint{3.968000in}{3.696000in}}%
\pgfusepath{clip}%
\pgfsetbuttcap%
\pgfsetroundjoin%
\definecolor{currentfill}{rgb}{0.916242,0.896091,0.100717}%
\pgfsetfillcolor{currentfill}%
\pgfsetlinewidth{0.000000pt}%
\definecolor{currentstroke}{rgb}{0.000000,0.000000,0.000000}%
\pgfsetstrokecolor{currentstroke}%
\pgfsetdash{}{0pt}%
\pgfpathmoveto{\pgfqpoint{4.768000in}{4.154381in}}%
\pgfpathlineto{\pgfqpoint{4.751169in}{4.170989in}}%
\pgfpathlineto{\pgfqpoint{4.735538in}{4.186667in}}%
\pgfpathlineto{\pgfqpoint{4.727919in}{4.194240in}}%
\pgfpathlineto{\pgfqpoint{4.712382in}{4.209528in}}%
\pgfpathlineto{\pgfqpoint{4.697913in}{4.224000in}}%
\pgfpathlineto{\pgfqpoint{4.695524in}{4.224000in}}%
\pgfpathlineto{\pgfqpoint{4.711145in}{4.208376in}}%
\pgfpathlineto{\pgfqpoint{4.727919in}{4.191870in}}%
\pgfpathlineto{\pgfqpoint{4.733154in}{4.186667in}}%
\pgfpathlineto{\pgfqpoint{4.749933in}{4.169838in}}%
\pgfpathlineto{\pgfqpoint{4.768000in}{4.152010in}}%
\pgfusepath{fill}%
\end{pgfscope}%
\begin{pgfscope}%
\pgfpathrectangle{\pgfqpoint{0.800000in}{0.528000in}}{\pgfqpoint{3.968000in}{3.696000in}}%
\pgfusepath{clip}%
\pgfsetbuttcap%
\pgfsetroundjoin%
\definecolor{currentfill}{rgb}{0.926106,0.897330,0.104071}%
\pgfsetfillcolor{currentfill}%
\pgfsetlinewidth{0.000000pt}%
\definecolor{currentstroke}{rgb}{0.000000,0.000000,0.000000}%
\pgfsetstrokecolor{currentstroke}%
\pgfsetdash{}{0pt}%
\pgfpathmoveto{\pgfqpoint{4.768000in}{4.156751in}}%
\pgfpathlineto{\pgfqpoint{4.752405in}{4.172141in}}%
\pgfpathlineto{\pgfqpoint{4.737922in}{4.186667in}}%
\pgfpathlineto{\pgfqpoint{4.727919in}{4.196609in}}%
\pgfpathlineto{\pgfqpoint{4.713619in}{4.210680in}}%
\pgfpathlineto{\pgfqpoint{4.700302in}{4.224000in}}%
\pgfpathlineto{\pgfqpoint{4.697913in}{4.224000in}}%
\pgfpathlineto{\pgfqpoint{4.712382in}{4.209528in}}%
\pgfpathlineto{\pgfqpoint{4.727919in}{4.194240in}}%
\pgfpathlineto{\pgfqpoint{4.735538in}{4.186667in}}%
\pgfpathlineto{\pgfqpoint{4.751169in}{4.170989in}}%
\pgfpathlineto{\pgfqpoint{4.768000in}{4.154381in}}%
\pgfusepath{fill}%
\end{pgfscope}%
\begin{pgfscope}%
\pgfpathrectangle{\pgfqpoint{0.800000in}{0.528000in}}{\pgfqpoint{3.968000in}{3.696000in}}%
\pgfusepath{clip}%
\pgfsetbuttcap%
\pgfsetroundjoin%
\definecolor{currentfill}{rgb}{0.926106,0.897330,0.104071}%
\pgfsetfillcolor{currentfill}%
\pgfsetlinewidth{0.000000pt}%
\definecolor{currentstroke}{rgb}{0.000000,0.000000,0.000000}%
\pgfsetstrokecolor{currentstroke}%
\pgfsetdash{}{0pt}%
\pgfpathmoveto{\pgfqpoint{4.768000in}{4.159122in}}%
\pgfpathlineto{\pgfqpoint{4.753641in}{4.173292in}}%
\pgfpathlineto{\pgfqpoint{4.740305in}{4.186667in}}%
\pgfpathlineto{\pgfqpoint{4.727919in}{4.198978in}}%
\pgfpathlineto{\pgfqpoint{4.714856in}{4.211833in}}%
\pgfpathlineto{\pgfqpoint{4.702691in}{4.224000in}}%
\pgfpathlineto{\pgfqpoint{4.700302in}{4.224000in}}%
\pgfpathlineto{\pgfqpoint{4.713619in}{4.210680in}}%
\pgfpathlineto{\pgfqpoint{4.727919in}{4.196609in}}%
\pgfpathlineto{\pgfqpoint{4.737922in}{4.186667in}}%
\pgfpathlineto{\pgfqpoint{4.752405in}{4.172141in}}%
\pgfpathlineto{\pgfqpoint{4.768000in}{4.156751in}}%
\pgfusepath{fill}%
\end{pgfscope}%
\begin{pgfscope}%
\pgfpathrectangle{\pgfqpoint{0.800000in}{0.528000in}}{\pgfqpoint{3.968000in}{3.696000in}}%
\pgfusepath{clip}%
\pgfsetbuttcap%
\pgfsetroundjoin%
\definecolor{currentfill}{rgb}{0.926106,0.897330,0.104071}%
\pgfsetfillcolor{currentfill}%
\pgfsetlinewidth{0.000000pt}%
\definecolor{currentstroke}{rgb}{0.000000,0.000000,0.000000}%
\pgfsetstrokecolor{currentstroke}%
\pgfsetdash{}{0pt}%
\pgfpathmoveto{\pgfqpoint{4.768000in}{4.161493in}}%
\pgfpathlineto{\pgfqpoint{4.754877in}{4.174443in}}%
\pgfpathlineto{\pgfqpoint{4.742689in}{4.186667in}}%
\pgfpathlineto{\pgfqpoint{4.727919in}{4.201348in}}%
\pgfpathlineto{\pgfqpoint{4.716093in}{4.212985in}}%
\pgfpathlineto{\pgfqpoint{4.705080in}{4.224000in}}%
\pgfpathlineto{\pgfqpoint{4.702691in}{4.224000in}}%
\pgfpathlineto{\pgfqpoint{4.714856in}{4.211833in}}%
\pgfpathlineto{\pgfqpoint{4.727919in}{4.198978in}}%
\pgfpathlineto{\pgfqpoint{4.740305in}{4.186667in}}%
\pgfpathlineto{\pgfqpoint{4.753641in}{4.173292in}}%
\pgfpathlineto{\pgfqpoint{4.768000in}{4.159122in}}%
\pgfusepath{fill}%
\end{pgfscope}%
\begin{pgfscope}%
\pgfpathrectangle{\pgfqpoint{0.800000in}{0.528000in}}{\pgfqpoint{3.968000in}{3.696000in}}%
\pgfusepath{clip}%
\pgfsetbuttcap%
\pgfsetroundjoin%
\definecolor{currentfill}{rgb}{0.926106,0.897330,0.104071}%
\pgfsetfillcolor{currentfill}%
\pgfsetlinewidth{0.000000pt}%
\definecolor{currentstroke}{rgb}{0.000000,0.000000,0.000000}%
\pgfsetstrokecolor{currentstroke}%
\pgfsetdash{}{0pt}%
\pgfpathmoveto{\pgfqpoint{4.768000in}{4.163864in}}%
\pgfpathlineto{\pgfqpoint{4.756113in}{4.175594in}}%
\pgfpathlineto{\pgfqpoint{4.745073in}{4.186667in}}%
\pgfpathlineto{\pgfqpoint{4.727919in}{4.203717in}}%
\pgfpathlineto{\pgfqpoint{4.717330in}{4.214137in}}%
\pgfpathlineto{\pgfqpoint{4.707469in}{4.224000in}}%
\pgfpathlineto{\pgfqpoint{4.705080in}{4.224000in}}%
\pgfpathlineto{\pgfqpoint{4.716093in}{4.212985in}}%
\pgfpathlineto{\pgfqpoint{4.727919in}{4.201348in}}%
\pgfpathlineto{\pgfqpoint{4.742689in}{4.186667in}}%
\pgfpathlineto{\pgfqpoint{4.754877in}{4.174443in}}%
\pgfpathlineto{\pgfqpoint{4.768000in}{4.161493in}}%
\pgfusepath{fill}%
\end{pgfscope}%
\begin{pgfscope}%
\pgfpathrectangle{\pgfqpoint{0.800000in}{0.528000in}}{\pgfqpoint{3.968000in}{3.696000in}}%
\pgfusepath{clip}%
\pgfsetbuttcap%
\pgfsetroundjoin%
\definecolor{currentfill}{rgb}{0.935904,0.898570,0.108131}%
\pgfsetfillcolor{currentfill}%
\pgfsetlinewidth{0.000000pt}%
\definecolor{currentstroke}{rgb}{0.000000,0.000000,0.000000}%
\pgfsetstrokecolor{currentstroke}%
\pgfsetdash{}{0pt}%
\pgfpathmoveto{\pgfqpoint{4.768000in}{4.166235in}}%
\pgfpathlineto{\pgfqpoint{4.757349in}{4.176745in}}%
\pgfpathlineto{\pgfqpoint{4.747457in}{4.186667in}}%
\pgfpathlineto{\pgfqpoint{4.727919in}{4.206087in}}%
\pgfpathlineto{\pgfqpoint{4.718567in}{4.215289in}}%
\pgfpathlineto{\pgfqpoint{4.709858in}{4.224000in}}%
\pgfpathlineto{\pgfqpoint{4.707469in}{4.224000in}}%
\pgfpathlineto{\pgfqpoint{4.717330in}{4.214137in}}%
\pgfpathlineto{\pgfqpoint{4.727919in}{4.203717in}}%
\pgfpathlineto{\pgfqpoint{4.745073in}{4.186667in}}%
\pgfpathlineto{\pgfqpoint{4.756113in}{4.175594in}}%
\pgfpathlineto{\pgfqpoint{4.768000in}{4.163864in}}%
\pgfusepath{fill}%
\end{pgfscope}%
\begin{pgfscope}%
\pgfpathrectangle{\pgfqpoint{0.800000in}{0.528000in}}{\pgfqpoint{3.968000in}{3.696000in}}%
\pgfusepath{clip}%
\pgfsetbuttcap%
\pgfsetroundjoin%
\definecolor{currentfill}{rgb}{0.935904,0.898570,0.108131}%
\pgfsetfillcolor{currentfill}%
\pgfsetlinewidth{0.000000pt}%
\definecolor{currentstroke}{rgb}{0.000000,0.000000,0.000000}%
\pgfsetstrokecolor{currentstroke}%
\pgfsetdash{}{0pt}%
\pgfpathmoveto{\pgfqpoint{4.768000in}{4.168606in}}%
\pgfpathlineto{\pgfqpoint{4.758585in}{4.177897in}}%
\pgfpathlineto{\pgfqpoint{4.749841in}{4.186667in}}%
\pgfpathlineto{\pgfqpoint{4.727919in}{4.208456in}}%
\pgfpathlineto{\pgfqpoint{4.719804in}{4.216441in}}%
\pgfpathlineto{\pgfqpoint{4.712247in}{4.224000in}}%
\pgfpathlineto{\pgfqpoint{4.709858in}{4.224000in}}%
\pgfpathlineto{\pgfqpoint{4.718567in}{4.215289in}}%
\pgfpathlineto{\pgfqpoint{4.727919in}{4.206087in}}%
\pgfpathlineto{\pgfqpoint{4.747457in}{4.186667in}}%
\pgfpathlineto{\pgfqpoint{4.757349in}{4.176745in}}%
\pgfpathlineto{\pgfqpoint{4.768000in}{4.166235in}}%
\pgfusepath{fill}%
\end{pgfscope}%
\begin{pgfscope}%
\pgfpathrectangle{\pgfqpoint{0.800000in}{0.528000in}}{\pgfqpoint{3.968000in}{3.696000in}}%
\pgfusepath{clip}%
\pgfsetbuttcap%
\pgfsetroundjoin%
\definecolor{currentfill}{rgb}{0.935904,0.898570,0.108131}%
\pgfsetfillcolor{currentfill}%
\pgfsetlinewidth{0.000000pt}%
\definecolor{currentstroke}{rgb}{0.000000,0.000000,0.000000}%
\pgfsetstrokecolor{currentstroke}%
\pgfsetdash{}{0pt}%
\pgfpathmoveto{\pgfqpoint{4.768000in}{4.170977in}}%
\pgfpathlineto{\pgfqpoint{4.759821in}{4.179048in}}%
\pgfpathlineto{\pgfqpoint{4.752224in}{4.186667in}}%
\pgfpathlineto{\pgfqpoint{4.727919in}{4.210826in}}%
\pgfpathlineto{\pgfqpoint{4.721041in}{4.217594in}}%
\pgfpathlineto{\pgfqpoint{4.714636in}{4.224000in}}%
\pgfpathlineto{\pgfqpoint{4.712247in}{4.224000in}}%
\pgfpathlineto{\pgfqpoint{4.719804in}{4.216441in}}%
\pgfpathlineto{\pgfqpoint{4.727919in}{4.208456in}}%
\pgfpathlineto{\pgfqpoint{4.749841in}{4.186667in}}%
\pgfpathlineto{\pgfqpoint{4.758585in}{4.177897in}}%
\pgfpathlineto{\pgfqpoint{4.768000in}{4.168606in}}%
\pgfusepath{fill}%
\end{pgfscope}%
\begin{pgfscope}%
\pgfpathrectangle{\pgfqpoint{0.800000in}{0.528000in}}{\pgfqpoint{3.968000in}{3.696000in}}%
\pgfusepath{clip}%
\pgfsetbuttcap%
\pgfsetroundjoin%
\definecolor{currentfill}{rgb}{0.935904,0.898570,0.108131}%
\pgfsetfillcolor{currentfill}%
\pgfsetlinewidth{0.000000pt}%
\definecolor{currentstroke}{rgb}{0.000000,0.000000,0.000000}%
\pgfsetstrokecolor{currentstroke}%
\pgfsetdash{}{0pt}%
\pgfpathmoveto{\pgfqpoint{4.768000in}{4.173347in}}%
\pgfpathlineto{\pgfqpoint{4.761056in}{4.180199in}}%
\pgfpathlineto{\pgfqpoint{4.754608in}{4.186667in}}%
\pgfpathlineto{\pgfqpoint{4.727919in}{4.213195in}}%
\pgfpathlineto{\pgfqpoint{4.722278in}{4.218746in}}%
\pgfpathlineto{\pgfqpoint{4.717025in}{4.224000in}}%
\pgfpathlineto{\pgfqpoint{4.714636in}{4.224000in}}%
\pgfpathlineto{\pgfqpoint{4.721041in}{4.217594in}}%
\pgfpathlineto{\pgfqpoint{4.727919in}{4.210826in}}%
\pgfpathlineto{\pgfqpoint{4.752224in}{4.186667in}}%
\pgfpathlineto{\pgfqpoint{4.759821in}{4.179048in}}%
\pgfpathlineto{\pgfqpoint{4.768000in}{4.170977in}}%
\pgfusepath{fill}%
\end{pgfscope}%
\begin{pgfscope}%
\pgfpathrectangle{\pgfqpoint{0.800000in}{0.528000in}}{\pgfqpoint{3.968000in}{3.696000in}}%
\pgfusepath{clip}%
\pgfsetbuttcap%
\pgfsetroundjoin%
\definecolor{currentfill}{rgb}{0.945636,0.899815,0.112838}%
\pgfsetfillcolor{currentfill}%
\pgfsetlinewidth{0.000000pt}%
\definecolor{currentstroke}{rgb}{0.000000,0.000000,0.000000}%
\pgfsetstrokecolor{currentstroke}%
\pgfsetdash{}{0pt}%
\pgfpathmoveto{\pgfqpoint{4.768000in}{4.175718in}}%
\pgfpathlineto{\pgfqpoint{4.762292in}{4.181350in}}%
\pgfpathlineto{\pgfqpoint{4.756992in}{4.186667in}}%
\pgfpathlineto{\pgfqpoint{4.727919in}{4.215565in}}%
\pgfpathlineto{\pgfqpoint{4.723515in}{4.219898in}}%
\pgfpathlineto{\pgfqpoint{4.719414in}{4.224000in}}%
\pgfpathlineto{\pgfqpoint{4.717025in}{4.224000in}}%
\pgfpathlineto{\pgfqpoint{4.722278in}{4.218746in}}%
\pgfpathlineto{\pgfqpoint{4.727919in}{4.213195in}}%
\pgfpathlineto{\pgfqpoint{4.754608in}{4.186667in}}%
\pgfpathlineto{\pgfqpoint{4.761056in}{4.180199in}}%
\pgfpathlineto{\pgfqpoint{4.768000in}{4.173347in}}%
\pgfusepath{fill}%
\end{pgfscope}%
\begin{pgfscope}%
\pgfpathrectangle{\pgfqpoint{0.800000in}{0.528000in}}{\pgfqpoint{3.968000in}{3.696000in}}%
\pgfusepath{clip}%
\pgfsetbuttcap%
\pgfsetroundjoin%
\definecolor{currentfill}{rgb}{0.945636,0.899815,0.112838}%
\pgfsetfillcolor{currentfill}%
\pgfsetlinewidth{0.000000pt}%
\definecolor{currentstroke}{rgb}{0.000000,0.000000,0.000000}%
\pgfsetstrokecolor{currentstroke}%
\pgfsetdash{}{0pt}%
\pgfpathmoveto{\pgfqpoint{4.768000in}{4.178089in}}%
\pgfpathlineto{\pgfqpoint{4.763528in}{4.182502in}}%
\pgfpathlineto{\pgfqpoint{4.759376in}{4.186667in}}%
\pgfpathlineto{\pgfqpoint{4.727919in}{4.217934in}}%
\pgfpathlineto{\pgfqpoint{4.724752in}{4.221050in}}%
\pgfpathlineto{\pgfqpoint{4.721803in}{4.224000in}}%
\pgfpathlineto{\pgfqpoint{4.719414in}{4.224000in}}%
\pgfpathlineto{\pgfqpoint{4.723515in}{4.219898in}}%
\pgfpathlineto{\pgfqpoint{4.727919in}{4.215565in}}%
\pgfpathlineto{\pgfqpoint{4.756992in}{4.186667in}}%
\pgfpathlineto{\pgfqpoint{4.762292in}{4.181350in}}%
\pgfpathlineto{\pgfqpoint{4.768000in}{4.175718in}}%
\pgfusepath{fill}%
\end{pgfscope}%
\begin{pgfscope}%
\pgfpathrectangle{\pgfqpoint{0.800000in}{0.528000in}}{\pgfqpoint{3.968000in}{3.696000in}}%
\pgfusepath{clip}%
\pgfsetbuttcap%
\pgfsetroundjoin%
\definecolor{currentfill}{rgb}{0.945636,0.899815,0.112838}%
\pgfsetfillcolor{currentfill}%
\pgfsetlinewidth{0.000000pt}%
\definecolor{currentstroke}{rgb}{0.000000,0.000000,0.000000}%
\pgfsetstrokecolor{currentstroke}%
\pgfsetdash{}{0pt}%
\pgfpathmoveto{\pgfqpoint{4.768000in}{4.180460in}}%
\pgfpathlineto{\pgfqpoint{4.764764in}{4.183653in}}%
\pgfpathlineto{\pgfqpoint{4.761759in}{4.186667in}}%
\pgfpathlineto{\pgfqpoint{4.727919in}{4.220303in}}%
\pgfpathlineto{\pgfqpoint{4.725989in}{4.222202in}}%
\pgfpathlineto{\pgfqpoint{4.724192in}{4.224000in}}%
\pgfpathlineto{\pgfqpoint{4.721803in}{4.224000in}}%
\pgfpathlineto{\pgfqpoint{4.724752in}{4.221050in}}%
\pgfpathlineto{\pgfqpoint{4.727919in}{4.217934in}}%
\pgfpathlineto{\pgfqpoint{4.759376in}{4.186667in}}%
\pgfpathlineto{\pgfqpoint{4.763528in}{4.182502in}}%
\pgfpathlineto{\pgfqpoint{4.768000in}{4.178089in}}%
\pgfusepath{fill}%
\end{pgfscope}%
\begin{pgfscope}%
\pgfpathrectangle{\pgfqpoint{0.800000in}{0.528000in}}{\pgfqpoint{3.968000in}{3.696000in}}%
\pgfusepath{clip}%
\pgfsetbuttcap%
\pgfsetroundjoin%
\definecolor{currentfill}{rgb}{0.955300,0.901065,0.118128}%
\pgfsetfillcolor{currentfill}%
\pgfsetlinewidth{0.000000pt}%
\definecolor{currentstroke}{rgb}{0.000000,0.000000,0.000000}%
\pgfsetstrokecolor{currentstroke}%
\pgfsetdash{}{0pt}%
\pgfpathmoveto{\pgfqpoint{4.768000in}{4.182831in}}%
\pgfpathlineto{\pgfqpoint{4.766000in}{4.184804in}}%
\pgfpathlineto{\pgfqpoint{4.764143in}{4.186667in}}%
\pgfpathlineto{\pgfqpoint{4.727919in}{4.222673in}}%
\pgfpathlineto{\pgfqpoint{4.727226in}{4.223355in}}%
\pgfpathlineto{\pgfqpoint{4.726581in}{4.224000in}}%
\pgfpathlineto{\pgfqpoint{4.724192in}{4.224000in}}%
\pgfpathlineto{\pgfqpoint{4.725989in}{4.222202in}}%
\pgfpathlineto{\pgfqpoint{4.727919in}{4.220303in}}%
\pgfpathlineto{\pgfqpoint{4.761759in}{4.186667in}}%
\pgfpathlineto{\pgfqpoint{4.764764in}{4.183653in}}%
\pgfpathlineto{\pgfqpoint{4.768000in}{4.180460in}}%
\pgfusepath{fill}%
\end{pgfscope}%
\begin{pgfscope}%
\pgfpathrectangle{\pgfqpoint{0.800000in}{0.528000in}}{\pgfqpoint{3.968000in}{3.696000in}}%
\pgfusepath{clip}%
\pgfsetbuttcap%
\pgfsetroundjoin%
\definecolor{currentfill}{rgb}{0.955300,0.901065,0.118128}%
\pgfsetfillcolor{currentfill}%
\pgfsetlinewidth{0.000000pt}%
\definecolor{currentstroke}{rgb}{0.000000,0.000000,0.000000}%
\pgfsetstrokecolor{currentstroke}%
\pgfsetdash{}{0pt}%
\pgfpathmoveto{\pgfqpoint{4.768000in}{4.185202in}}%
\pgfpathlineto{\pgfqpoint{4.767236in}{4.185955in}}%
\pgfpathlineto{\pgfqpoint{4.766527in}{4.186667in}}%
\pgfpathlineto{\pgfqpoint{4.744587in}{4.208475in}}%
\pgfpathlineto{\pgfqpoint{4.728960in}{4.224000in}}%
\pgfpathlineto{\pgfqpoint{4.727919in}{4.224000in}}%
\pgfpathlineto{\pgfqpoint{4.726581in}{4.224000in}}%
\pgfpathlineto{\pgfqpoint{4.727226in}{4.223355in}}%
\pgfpathlineto{\pgfqpoint{4.727919in}{4.222673in}}%
\pgfpathlineto{\pgfqpoint{4.764143in}{4.186667in}}%
\pgfpathlineto{\pgfqpoint{4.766000in}{4.184804in}}%
\pgfpathlineto{\pgfqpoint{4.768000in}{4.182831in}}%
\pgfusepath{fill}%
\end{pgfscope}%
\begin{pgfscope}%
\pgfpathrectangle{\pgfqpoint{0.800000in}{0.528000in}}{\pgfqpoint{3.968000in}{3.696000in}}%
\pgfusepath{clip}%
\pgfsetbuttcap%
\pgfsetroundjoin%
\definecolor{currentfill}{rgb}{0.955300,0.901065,0.118128}%
\pgfsetfillcolor{currentfill}%
\pgfsetlinewidth{0.000000pt}%
\definecolor{currentstroke}{rgb}{0.000000,0.000000,0.000000}%
\pgfsetstrokecolor{currentstroke}%
\pgfsetdash{}{0pt}%
\pgfpathmoveto{\pgfqpoint{4.768000in}{4.187564in}}%
\pgfpathlineto{\pgfqpoint{4.731325in}{4.224000in}}%
\pgfpathlineto{\pgfqpoint{4.728960in}{4.224000in}}%
\pgfpathlineto{\pgfqpoint{4.744587in}{4.208475in}}%
\pgfpathlineto{\pgfqpoint{4.766527in}{4.186667in}}%
\pgfpathlineto{\pgfqpoint{4.767236in}{4.185955in}}%
\pgfpathlineto{\pgfqpoint{4.768000in}{4.185202in}}%
\pgfpathlineto{\pgfqpoint{4.768000in}{4.186667in}}%
\pgfusepath{fill}%
\end{pgfscope}%
\begin{pgfscope}%
\pgfpathrectangle{\pgfqpoint{0.800000in}{0.528000in}}{\pgfqpoint{3.968000in}{3.696000in}}%
\pgfusepath{clip}%
\pgfsetbuttcap%
\pgfsetroundjoin%
\definecolor{currentfill}{rgb}{0.955300,0.901065,0.118128}%
\pgfsetfillcolor{currentfill}%
\pgfsetlinewidth{0.000000pt}%
\definecolor{currentstroke}{rgb}{0.000000,0.000000,0.000000}%
\pgfsetstrokecolor{currentstroke}%
\pgfsetdash{}{0pt}%
\pgfpathmoveto{\pgfqpoint{4.768000in}{4.189914in}}%
\pgfpathlineto{\pgfqpoint{4.733690in}{4.224000in}}%
\pgfpathlineto{\pgfqpoint{4.731325in}{4.224000in}}%
\pgfpathlineto{\pgfqpoint{4.768000in}{4.187564in}}%
\pgfusepath{fill}%
\end{pgfscope}%
\begin{pgfscope}%
\pgfpathrectangle{\pgfqpoint{0.800000in}{0.528000in}}{\pgfqpoint{3.968000in}{3.696000in}}%
\pgfusepath{clip}%
\pgfsetbuttcap%
\pgfsetroundjoin%
\definecolor{currentfill}{rgb}{0.964894,0.902323,0.123941}%
\pgfsetfillcolor{currentfill}%
\pgfsetlinewidth{0.000000pt}%
\definecolor{currentstroke}{rgb}{0.000000,0.000000,0.000000}%
\pgfsetstrokecolor{currentstroke}%
\pgfsetdash{}{0pt}%
\pgfpathmoveto{\pgfqpoint{4.768000in}{4.192264in}}%
\pgfpathlineto{\pgfqpoint{4.736055in}{4.224000in}}%
\pgfpathlineto{\pgfqpoint{4.733690in}{4.224000in}}%
\pgfpathlineto{\pgfqpoint{4.768000in}{4.189914in}}%
\pgfusepath{fill}%
\end{pgfscope}%
\begin{pgfscope}%
\pgfpathrectangle{\pgfqpoint{0.800000in}{0.528000in}}{\pgfqpoint{3.968000in}{3.696000in}}%
\pgfusepath{clip}%
\pgfsetbuttcap%
\pgfsetroundjoin%
\definecolor{currentfill}{rgb}{0.964894,0.902323,0.123941}%
\pgfsetfillcolor{currentfill}%
\pgfsetlinewidth{0.000000pt}%
\definecolor{currentstroke}{rgb}{0.000000,0.000000,0.000000}%
\pgfsetstrokecolor{currentstroke}%
\pgfsetdash{}{0pt}%
\pgfpathmoveto{\pgfqpoint{4.768000in}{4.194614in}}%
\pgfpathlineto{\pgfqpoint{4.738421in}{4.224000in}}%
\pgfpathlineto{\pgfqpoint{4.736055in}{4.224000in}}%
\pgfpathlineto{\pgfqpoint{4.768000in}{4.192264in}}%
\pgfusepath{fill}%
\end{pgfscope}%
\begin{pgfscope}%
\pgfpathrectangle{\pgfqpoint{0.800000in}{0.528000in}}{\pgfqpoint{3.968000in}{3.696000in}}%
\pgfusepath{clip}%
\pgfsetbuttcap%
\pgfsetroundjoin%
\definecolor{currentfill}{rgb}{0.964894,0.902323,0.123941}%
\pgfsetfillcolor{currentfill}%
\pgfsetlinewidth{0.000000pt}%
\definecolor{currentstroke}{rgb}{0.000000,0.000000,0.000000}%
\pgfsetstrokecolor{currentstroke}%
\pgfsetdash{}{0pt}%
\pgfpathmoveto{\pgfqpoint{4.768000in}{4.196964in}}%
\pgfpathlineto{\pgfqpoint{4.740786in}{4.224000in}}%
\pgfpathlineto{\pgfqpoint{4.738421in}{4.224000in}}%
\pgfpathlineto{\pgfqpoint{4.768000in}{4.194614in}}%
\pgfusepath{fill}%
\end{pgfscope}%
\begin{pgfscope}%
\pgfpathrectangle{\pgfqpoint{0.800000in}{0.528000in}}{\pgfqpoint{3.968000in}{3.696000in}}%
\pgfusepath{clip}%
\pgfsetbuttcap%
\pgfsetroundjoin%
\definecolor{currentfill}{rgb}{0.964894,0.902323,0.123941}%
\pgfsetfillcolor{currentfill}%
\pgfsetlinewidth{0.000000pt}%
\definecolor{currentstroke}{rgb}{0.000000,0.000000,0.000000}%
\pgfsetstrokecolor{currentstroke}%
\pgfsetdash{}{0pt}%
\pgfpathmoveto{\pgfqpoint{4.768000in}{4.199313in}}%
\pgfpathlineto{\pgfqpoint{4.743151in}{4.224000in}}%
\pgfpathlineto{\pgfqpoint{4.740786in}{4.224000in}}%
\pgfpathlineto{\pgfqpoint{4.768000in}{4.196964in}}%
\pgfusepath{fill}%
\end{pgfscope}%
\begin{pgfscope}%
\pgfpathrectangle{\pgfqpoint{0.800000in}{0.528000in}}{\pgfqpoint{3.968000in}{3.696000in}}%
\pgfusepath{clip}%
\pgfsetbuttcap%
\pgfsetroundjoin%
\definecolor{currentfill}{rgb}{0.974417,0.903590,0.130215}%
\pgfsetfillcolor{currentfill}%
\pgfsetlinewidth{0.000000pt}%
\definecolor{currentstroke}{rgb}{0.000000,0.000000,0.000000}%
\pgfsetstrokecolor{currentstroke}%
\pgfsetdash{}{0pt}%
\pgfpathmoveto{\pgfqpoint{4.768000in}{4.201663in}}%
\pgfpathlineto{\pgfqpoint{4.745516in}{4.224000in}}%
\pgfpathlineto{\pgfqpoint{4.743151in}{4.224000in}}%
\pgfpathlineto{\pgfqpoint{4.768000in}{4.199313in}}%
\pgfusepath{fill}%
\end{pgfscope}%
\begin{pgfscope}%
\pgfpathrectangle{\pgfqpoint{0.800000in}{0.528000in}}{\pgfqpoint{3.968000in}{3.696000in}}%
\pgfusepath{clip}%
\pgfsetbuttcap%
\pgfsetroundjoin%
\definecolor{currentfill}{rgb}{0.974417,0.903590,0.130215}%
\pgfsetfillcolor{currentfill}%
\pgfsetlinewidth{0.000000pt}%
\definecolor{currentstroke}{rgb}{0.000000,0.000000,0.000000}%
\pgfsetstrokecolor{currentstroke}%
\pgfsetdash{}{0pt}%
\pgfpathmoveto{\pgfqpoint{4.768000in}{4.204013in}}%
\pgfpathlineto{\pgfqpoint{4.747882in}{4.224000in}}%
\pgfpathlineto{\pgfqpoint{4.745516in}{4.224000in}}%
\pgfpathlineto{\pgfqpoint{4.768000in}{4.201663in}}%
\pgfusepath{fill}%
\end{pgfscope}%
\begin{pgfscope}%
\pgfpathrectangle{\pgfqpoint{0.800000in}{0.528000in}}{\pgfqpoint{3.968000in}{3.696000in}}%
\pgfusepath{clip}%
\pgfsetbuttcap%
\pgfsetroundjoin%
\definecolor{currentfill}{rgb}{0.974417,0.903590,0.130215}%
\pgfsetfillcolor{currentfill}%
\pgfsetlinewidth{0.000000pt}%
\definecolor{currentstroke}{rgb}{0.000000,0.000000,0.000000}%
\pgfsetstrokecolor{currentstroke}%
\pgfsetdash{}{0pt}%
\pgfpathmoveto{\pgfqpoint{4.768000in}{4.206363in}}%
\pgfpathlineto{\pgfqpoint{4.750247in}{4.224000in}}%
\pgfpathlineto{\pgfqpoint{4.747882in}{4.224000in}}%
\pgfpathlineto{\pgfqpoint{4.768000in}{4.204013in}}%
\pgfusepath{fill}%
\end{pgfscope}%
\begin{pgfscope}%
\pgfpathrectangle{\pgfqpoint{0.800000in}{0.528000in}}{\pgfqpoint{3.968000in}{3.696000in}}%
\pgfusepath{clip}%
\pgfsetbuttcap%
\pgfsetroundjoin%
\definecolor{currentfill}{rgb}{0.974417,0.903590,0.130215}%
\pgfsetfillcolor{currentfill}%
\pgfsetlinewidth{0.000000pt}%
\definecolor{currentstroke}{rgb}{0.000000,0.000000,0.000000}%
\pgfsetstrokecolor{currentstroke}%
\pgfsetdash{}{0pt}%
\pgfpathmoveto{\pgfqpoint{4.768000in}{4.208713in}}%
\pgfpathlineto{\pgfqpoint{4.752612in}{4.224000in}}%
\pgfpathlineto{\pgfqpoint{4.750247in}{4.224000in}}%
\pgfpathlineto{\pgfqpoint{4.768000in}{4.206363in}}%
\pgfusepath{fill}%
\end{pgfscope}%
\begin{pgfscope}%
\pgfpathrectangle{\pgfqpoint{0.800000in}{0.528000in}}{\pgfqpoint{3.968000in}{3.696000in}}%
\pgfusepath{clip}%
\pgfsetbuttcap%
\pgfsetroundjoin%
\definecolor{currentfill}{rgb}{0.983868,0.904867,0.136897}%
\pgfsetfillcolor{currentfill}%
\pgfsetlinewidth{0.000000pt}%
\definecolor{currentstroke}{rgb}{0.000000,0.000000,0.000000}%
\pgfsetstrokecolor{currentstroke}%
\pgfsetdash{}{0pt}%
\pgfpathmoveto{\pgfqpoint{4.768000in}{4.211062in}}%
\pgfpathlineto{\pgfqpoint{4.754977in}{4.224000in}}%
\pgfpathlineto{\pgfqpoint{4.752612in}{4.224000in}}%
\pgfpathlineto{\pgfqpoint{4.768000in}{4.208713in}}%
\pgfusepath{fill}%
\end{pgfscope}%
\begin{pgfscope}%
\pgfpathrectangle{\pgfqpoint{0.800000in}{0.528000in}}{\pgfqpoint{3.968000in}{3.696000in}}%
\pgfusepath{clip}%
\pgfsetbuttcap%
\pgfsetroundjoin%
\definecolor{currentfill}{rgb}{0.983868,0.904867,0.136897}%
\pgfsetfillcolor{currentfill}%
\pgfsetlinewidth{0.000000pt}%
\definecolor{currentstroke}{rgb}{0.000000,0.000000,0.000000}%
\pgfsetstrokecolor{currentstroke}%
\pgfsetdash{}{0pt}%
\pgfpathmoveto{\pgfqpoint{4.768000in}{4.213412in}}%
\pgfpathlineto{\pgfqpoint{4.757343in}{4.224000in}}%
\pgfpathlineto{\pgfqpoint{4.754977in}{4.224000in}}%
\pgfpathlineto{\pgfqpoint{4.768000in}{4.211062in}}%
\pgfusepath{fill}%
\end{pgfscope}%
\begin{pgfscope}%
\pgfpathrectangle{\pgfqpoint{0.800000in}{0.528000in}}{\pgfqpoint{3.968000in}{3.696000in}}%
\pgfusepath{clip}%
\pgfsetbuttcap%
\pgfsetroundjoin%
\definecolor{currentfill}{rgb}{0.983868,0.904867,0.136897}%
\pgfsetfillcolor{currentfill}%
\pgfsetlinewidth{0.000000pt}%
\definecolor{currentstroke}{rgb}{0.000000,0.000000,0.000000}%
\pgfsetstrokecolor{currentstroke}%
\pgfsetdash{}{0pt}%
\pgfpathmoveto{\pgfqpoint{4.768000in}{4.215762in}}%
\pgfpathlineto{\pgfqpoint{4.759708in}{4.224000in}}%
\pgfpathlineto{\pgfqpoint{4.757343in}{4.224000in}}%
\pgfpathlineto{\pgfqpoint{4.768000in}{4.213412in}}%
\pgfusepath{fill}%
\end{pgfscope}%
\begin{pgfscope}%
\pgfpathrectangle{\pgfqpoint{0.800000in}{0.528000in}}{\pgfqpoint{3.968000in}{3.696000in}}%
\pgfusepath{clip}%
\pgfsetbuttcap%
\pgfsetroundjoin%
\definecolor{currentfill}{rgb}{0.993248,0.906157,0.143936}%
\pgfsetfillcolor{currentfill}%
\pgfsetlinewidth{0.000000pt}%
\definecolor{currentstroke}{rgb}{0.000000,0.000000,0.000000}%
\pgfsetstrokecolor{currentstroke}%
\pgfsetdash{}{0pt}%
\pgfpathmoveto{\pgfqpoint{4.768000in}{4.218112in}}%
\pgfpathlineto{\pgfqpoint{4.762073in}{4.224000in}}%
\pgfpathlineto{\pgfqpoint{4.759708in}{4.224000in}}%
\pgfpathlineto{\pgfqpoint{4.768000in}{4.215762in}}%
\pgfusepath{fill}%
\end{pgfscope}%
\begin{pgfscope}%
\pgfpathrectangle{\pgfqpoint{0.800000in}{0.528000in}}{\pgfqpoint{3.968000in}{3.696000in}}%
\pgfusepath{clip}%
\pgfsetbuttcap%
\pgfsetroundjoin%
\definecolor{currentfill}{rgb}{0.993248,0.906157,0.143936}%
\pgfsetfillcolor{currentfill}%
\pgfsetlinewidth{0.000000pt}%
\definecolor{currentstroke}{rgb}{0.000000,0.000000,0.000000}%
\pgfsetstrokecolor{currentstroke}%
\pgfsetdash{}{0pt}%
\pgfpathmoveto{\pgfqpoint{4.768000in}{4.220462in}}%
\pgfpathlineto{\pgfqpoint{4.764438in}{4.224000in}}%
\pgfpathlineto{\pgfqpoint{4.762073in}{4.224000in}}%
\pgfpathlineto{\pgfqpoint{4.768000in}{4.218112in}}%
\pgfusepath{fill}%
\end{pgfscope}%
\begin{pgfscope}%
\pgfpathrectangle{\pgfqpoint{0.800000in}{0.528000in}}{\pgfqpoint{3.968000in}{3.696000in}}%
\pgfusepath{clip}%
\pgfsetbuttcap%
\pgfsetroundjoin%
\definecolor{currentfill}{rgb}{0.993248,0.906157,0.143936}%
\pgfsetfillcolor{currentfill}%
\pgfsetlinewidth{0.000000pt}%
\definecolor{currentstroke}{rgb}{0.000000,0.000000,0.000000}%
\pgfsetstrokecolor{currentstroke}%
\pgfsetdash{}{0pt}%
\pgfpathmoveto{\pgfqpoint{4.768000in}{4.222811in}}%
\pgfpathlineto{\pgfqpoint{4.766804in}{4.224000in}}%
\pgfpathlineto{\pgfqpoint{4.764438in}{4.224000in}}%
\pgfpathlineto{\pgfqpoint{4.768000in}{4.220462in}}%
\pgfusepath{fill}%
\end{pgfscope}%
\begin{pgfscope}%
\pgfpathrectangle{\pgfqpoint{0.800000in}{0.528000in}}{\pgfqpoint{3.968000in}{3.696000in}}%
\pgfusepath{clip}%
\pgfsetbuttcap%
\pgfsetroundjoin%
\definecolor{currentfill}{rgb}{0.993248,0.906157,0.143936}%
\pgfsetfillcolor{currentfill}%
\pgfsetlinewidth{0.000000pt}%
\definecolor{currentstroke}{rgb}{0.000000,0.000000,0.000000}%
\pgfsetstrokecolor{currentstroke}%
\pgfsetdash{}{0pt}%
\pgfpathmoveto{\pgfqpoint{4.766804in}{4.224000in}}%
\pgfpathlineto{\pgfqpoint{4.768000in}{4.222811in}}%
\pgfpathlineto{\pgfqpoint{4.768000in}{4.224000in}}%
\pgfusepath{fill}%
\end{pgfscope}%
\begin{pgfscope}%
\pgfsetbuttcap%
\pgfsetroundjoin%
\definecolor{currentfill}{rgb}{0.000000,0.000000,0.000000}%
\pgfsetfillcolor{currentfill}%
\pgfsetlinewidth{0.803000pt}%
\definecolor{currentstroke}{rgb}{0.000000,0.000000,0.000000}%
\pgfsetstrokecolor{currentstroke}%
\pgfsetdash{}{0pt}%
\pgfsys@defobject{currentmarker}{\pgfqpoint{0.000000in}{-0.048611in}}{\pgfqpoint{0.000000in}{0.000000in}}{%
\pgfpathmoveto{\pgfqpoint{0.000000in}{0.000000in}}%
\pgfpathlineto{\pgfqpoint{0.000000in}{-0.048611in}}%
\pgfusepath{stroke,fill}%
}%
\begin{pgfscope}%
\pgfsys@transformshift{0.800000in}{0.528000in}%
\pgfsys@useobject{currentmarker}{}%
\end{pgfscope}%
\end{pgfscope}%
\begin{pgfscope}%
\definecolor{textcolor}{rgb}{0.000000,0.000000,0.000000}%
\pgfsetstrokecolor{textcolor}%
\pgfsetfillcolor{textcolor}%
\pgftext[x=0.800000in,y=0.430778in,,top]{\color{textcolor}\rmfamily\fontsize{10.000000}{12.000000}\selectfont \(\displaystyle 0.500\)}%
\end{pgfscope}%
\begin{pgfscope}%
\pgfsetbuttcap%
\pgfsetroundjoin%
\definecolor{currentfill}{rgb}{0.000000,0.000000,0.000000}%
\pgfsetfillcolor{currentfill}%
\pgfsetlinewidth{0.803000pt}%
\definecolor{currentstroke}{rgb}{0.000000,0.000000,0.000000}%
\pgfsetstrokecolor{currentstroke}%
\pgfsetdash{}{0pt}%
\pgfsys@defobject{currentmarker}{\pgfqpoint{0.000000in}{-0.048611in}}{\pgfqpoint{0.000000in}{0.000000in}}{%
\pgfpathmoveto{\pgfqpoint{0.000000in}{0.000000in}}%
\pgfpathlineto{\pgfqpoint{0.000000in}{-0.048611in}}%
\pgfusepath{stroke,fill}%
}%
\begin{pgfscope}%
\pgfsys@transformshift{1.296000in}{0.528000in}%
\pgfsys@useobject{currentmarker}{}%
\end{pgfscope}%
\end{pgfscope}%
\begin{pgfscope}%
\definecolor{textcolor}{rgb}{0.000000,0.000000,0.000000}%
\pgfsetstrokecolor{textcolor}%
\pgfsetfillcolor{textcolor}%
\pgftext[x=1.296000in,y=0.430778in,,top]{\color{textcolor}\rmfamily\fontsize{10.000000}{12.000000}\selectfont \(\displaystyle 0.525\)}%
\end{pgfscope}%
\begin{pgfscope}%
\pgfsetbuttcap%
\pgfsetroundjoin%
\definecolor{currentfill}{rgb}{0.000000,0.000000,0.000000}%
\pgfsetfillcolor{currentfill}%
\pgfsetlinewidth{0.803000pt}%
\definecolor{currentstroke}{rgb}{0.000000,0.000000,0.000000}%
\pgfsetstrokecolor{currentstroke}%
\pgfsetdash{}{0pt}%
\pgfsys@defobject{currentmarker}{\pgfqpoint{0.000000in}{-0.048611in}}{\pgfqpoint{0.000000in}{0.000000in}}{%
\pgfpathmoveto{\pgfqpoint{0.000000in}{0.000000in}}%
\pgfpathlineto{\pgfqpoint{0.000000in}{-0.048611in}}%
\pgfusepath{stroke,fill}%
}%
\begin{pgfscope}%
\pgfsys@transformshift{1.792000in}{0.528000in}%
\pgfsys@useobject{currentmarker}{}%
\end{pgfscope}%
\end{pgfscope}%
\begin{pgfscope}%
\definecolor{textcolor}{rgb}{0.000000,0.000000,0.000000}%
\pgfsetstrokecolor{textcolor}%
\pgfsetfillcolor{textcolor}%
\pgftext[x=1.792000in,y=0.430778in,,top]{\color{textcolor}\rmfamily\fontsize{10.000000}{12.000000}\selectfont \(\displaystyle 0.550\)}%
\end{pgfscope}%
\begin{pgfscope}%
\pgfsetbuttcap%
\pgfsetroundjoin%
\definecolor{currentfill}{rgb}{0.000000,0.000000,0.000000}%
\pgfsetfillcolor{currentfill}%
\pgfsetlinewidth{0.803000pt}%
\definecolor{currentstroke}{rgb}{0.000000,0.000000,0.000000}%
\pgfsetstrokecolor{currentstroke}%
\pgfsetdash{}{0pt}%
\pgfsys@defobject{currentmarker}{\pgfqpoint{0.000000in}{-0.048611in}}{\pgfqpoint{0.000000in}{0.000000in}}{%
\pgfpathmoveto{\pgfqpoint{0.000000in}{0.000000in}}%
\pgfpathlineto{\pgfqpoint{0.000000in}{-0.048611in}}%
\pgfusepath{stroke,fill}%
}%
\begin{pgfscope}%
\pgfsys@transformshift{2.288000in}{0.528000in}%
\pgfsys@useobject{currentmarker}{}%
\end{pgfscope}%
\end{pgfscope}%
\begin{pgfscope}%
\definecolor{textcolor}{rgb}{0.000000,0.000000,0.000000}%
\pgfsetstrokecolor{textcolor}%
\pgfsetfillcolor{textcolor}%
\pgftext[x=2.288000in,y=0.430778in,,top]{\color{textcolor}\rmfamily\fontsize{10.000000}{12.000000}\selectfont \(\displaystyle 0.575\)}%
\end{pgfscope}%
\begin{pgfscope}%
\pgfsetbuttcap%
\pgfsetroundjoin%
\definecolor{currentfill}{rgb}{0.000000,0.000000,0.000000}%
\pgfsetfillcolor{currentfill}%
\pgfsetlinewidth{0.803000pt}%
\definecolor{currentstroke}{rgb}{0.000000,0.000000,0.000000}%
\pgfsetstrokecolor{currentstroke}%
\pgfsetdash{}{0pt}%
\pgfsys@defobject{currentmarker}{\pgfqpoint{0.000000in}{-0.048611in}}{\pgfqpoint{0.000000in}{0.000000in}}{%
\pgfpathmoveto{\pgfqpoint{0.000000in}{0.000000in}}%
\pgfpathlineto{\pgfqpoint{0.000000in}{-0.048611in}}%
\pgfusepath{stroke,fill}%
}%
\begin{pgfscope}%
\pgfsys@transformshift{2.784000in}{0.528000in}%
\pgfsys@useobject{currentmarker}{}%
\end{pgfscope}%
\end{pgfscope}%
\begin{pgfscope}%
\definecolor{textcolor}{rgb}{0.000000,0.000000,0.000000}%
\pgfsetstrokecolor{textcolor}%
\pgfsetfillcolor{textcolor}%
\pgftext[x=2.784000in,y=0.430778in,,top]{\color{textcolor}\rmfamily\fontsize{10.000000}{12.000000}\selectfont \(\displaystyle 0.600\)}%
\end{pgfscope}%
\begin{pgfscope}%
\pgfsetbuttcap%
\pgfsetroundjoin%
\definecolor{currentfill}{rgb}{0.000000,0.000000,0.000000}%
\pgfsetfillcolor{currentfill}%
\pgfsetlinewidth{0.803000pt}%
\definecolor{currentstroke}{rgb}{0.000000,0.000000,0.000000}%
\pgfsetstrokecolor{currentstroke}%
\pgfsetdash{}{0pt}%
\pgfsys@defobject{currentmarker}{\pgfqpoint{0.000000in}{-0.048611in}}{\pgfqpoint{0.000000in}{0.000000in}}{%
\pgfpathmoveto{\pgfqpoint{0.000000in}{0.000000in}}%
\pgfpathlineto{\pgfqpoint{0.000000in}{-0.048611in}}%
\pgfusepath{stroke,fill}%
}%
\begin{pgfscope}%
\pgfsys@transformshift{3.280000in}{0.528000in}%
\pgfsys@useobject{currentmarker}{}%
\end{pgfscope}%
\end{pgfscope}%
\begin{pgfscope}%
\definecolor{textcolor}{rgb}{0.000000,0.000000,0.000000}%
\pgfsetstrokecolor{textcolor}%
\pgfsetfillcolor{textcolor}%
\pgftext[x=3.280000in,y=0.430778in,,top]{\color{textcolor}\rmfamily\fontsize{10.000000}{12.000000}\selectfont \(\displaystyle 0.625\)}%
\end{pgfscope}%
\begin{pgfscope}%
\pgfsetbuttcap%
\pgfsetroundjoin%
\definecolor{currentfill}{rgb}{0.000000,0.000000,0.000000}%
\pgfsetfillcolor{currentfill}%
\pgfsetlinewidth{0.803000pt}%
\definecolor{currentstroke}{rgb}{0.000000,0.000000,0.000000}%
\pgfsetstrokecolor{currentstroke}%
\pgfsetdash{}{0pt}%
\pgfsys@defobject{currentmarker}{\pgfqpoint{0.000000in}{-0.048611in}}{\pgfqpoint{0.000000in}{0.000000in}}{%
\pgfpathmoveto{\pgfqpoint{0.000000in}{0.000000in}}%
\pgfpathlineto{\pgfqpoint{0.000000in}{-0.048611in}}%
\pgfusepath{stroke,fill}%
}%
\begin{pgfscope}%
\pgfsys@transformshift{3.776000in}{0.528000in}%
\pgfsys@useobject{currentmarker}{}%
\end{pgfscope}%
\end{pgfscope}%
\begin{pgfscope}%
\definecolor{textcolor}{rgb}{0.000000,0.000000,0.000000}%
\pgfsetstrokecolor{textcolor}%
\pgfsetfillcolor{textcolor}%
\pgftext[x=3.776000in,y=0.430778in,,top]{\color{textcolor}\rmfamily\fontsize{10.000000}{12.000000}\selectfont \(\displaystyle 0.650\)}%
\end{pgfscope}%
\begin{pgfscope}%
\pgfsetbuttcap%
\pgfsetroundjoin%
\definecolor{currentfill}{rgb}{0.000000,0.000000,0.000000}%
\pgfsetfillcolor{currentfill}%
\pgfsetlinewidth{0.803000pt}%
\definecolor{currentstroke}{rgb}{0.000000,0.000000,0.000000}%
\pgfsetstrokecolor{currentstroke}%
\pgfsetdash{}{0pt}%
\pgfsys@defobject{currentmarker}{\pgfqpoint{0.000000in}{-0.048611in}}{\pgfqpoint{0.000000in}{0.000000in}}{%
\pgfpathmoveto{\pgfqpoint{0.000000in}{0.000000in}}%
\pgfpathlineto{\pgfqpoint{0.000000in}{-0.048611in}}%
\pgfusepath{stroke,fill}%
}%
\begin{pgfscope}%
\pgfsys@transformshift{4.272000in}{0.528000in}%
\pgfsys@useobject{currentmarker}{}%
\end{pgfscope}%
\end{pgfscope}%
\begin{pgfscope}%
\definecolor{textcolor}{rgb}{0.000000,0.000000,0.000000}%
\pgfsetstrokecolor{textcolor}%
\pgfsetfillcolor{textcolor}%
\pgftext[x=4.272000in,y=0.430778in,,top]{\color{textcolor}\rmfamily\fontsize{10.000000}{12.000000}\selectfont \(\displaystyle 0.675\)}%
\end{pgfscope}%
\begin{pgfscope}%
\pgfsetbuttcap%
\pgfsetroundjoin%
\definecolor{currentfill}{rgb}{0.000000,0.000000,0.000000}%
\pgfsetfillcolor{currentfill}%
\pgfsetlinewidth{0.803000pt}%
\definecolor{currentstroke}{rgb}{0.000000,0.000000,0.000000}%
\pgfsetstrokecolor{currentstroke}%
\pgfsetdash{}{0pt}%
\pgfsys@defobject{currentmarker}{\pgfqpoint{0.000000in}{-0.048611in}}{\pgfqpoint{0.000000in}{0.000000in}}{%
\pgfpathmoveto{\pgfqpoint{0.000000in}{0.000000in}}%
\pgfpathlineto{\pgfqpoint{0.000000in}{-0.048611in}}%
\pgfusepath{stroke,fill}%
}%
\begin{pgfscope}%
\pgfsys@transformshift{4.768000in}{0.528000in}%
\pgfsys@useobject{currentmarker}{}%
\end{pgfscope}%
\end{pgfscope}%
\begin{pgfscope}%
\definecolor{textcolor}{rgb}{0.000000,0.000000,0.000000}%
\pgfsetstrokecolor{textcolor}%
\pgfsetfillcolor{textcolor}%
\pgftext[x=4.768000in,y=0.430778in,,top]{\color{textcolor}\rmfamily\fontsize{10.000000}{12.000000}\selectfont \(\displaystyle 0.700\)}%
\end{pgfscope}%
\begin{pgfscope}%
\pgfsetbuttcap%
\pgfsetroundjoin%
\definecolor{currentfill}{rgb}{0.000000,0.000000,0.000000}%
\pgfsetfillcolor{currentfill}%
\pgfsetlinewidth{0.803000pt}%
\definecolor{currentstroke}{rgb}{0.000000,0.000000,0.000000}%
\pgfsetstrokecolor{currentstroke}%
\pgfsetdash{}{0pt}%
\pgfsys@defobject{currentmarker}{\pgfqpoint{-0.048611in}{0.000000in}}{\pgfqpoint{0.000000in}{0.000000in}}{%
\pgfpathmoveto{\pgfqpoint{0.000000in}{0.000000in}}%
\pgfpathlineto{\pgfqpoint{-0.048611in}{0.000000in}}%
\pgfusepath{stroke,fill}%
}%
\begin{pgfscope}%
\pgfsys@transformshift{0.800000in}{0.836000in}%
\pgfsys@useobject{currentmarker}{}%
\end{pgfscope}%
\end{pgfscope}%
\begin{pgfscope}%
\definecolor{textcolor}{rgb}{0.000000,0.000000,0.000000}%
\pgfsetstrokecolor{textcolor}%
\pgfsetfillcolor{textcolor}%
\pgftext[x=0.525308in,y=0.787775in,left,base]{\color{textcolor}\rmfamily\fontsize{10.000000}{12.000000}\selectfont \(\displaystyle 0.6\)}%
\end{pgfscope}%
\begin{pgfscope}%
\pgfsetbuttcap%
\pgfsetroundjoin%
\definecolor{currentfill}{rgb}{0.000000,0.000000,0.000000}%
\pgfsetfillcolor{currentfill}%
\pgfsetlinewidth{0.803000pt}%
\definecolor{currentstroke}{rgb}{0.000000,0.000000,0.000000}%
\pgfsetstrokecolor{currentstroke}%
\pgfsetdash{}{0pt}%
\pgfsys@defobject{currentmarker}{\pgfqpoint{-0.048611in}{0.000000in}}{\pgfqpoint{0.000000in}{0.000000in}}{%
\pgfpathmoveto{\pgfqpoint{0.000000in}{0.000000in}}%
\pgfpathlineto{\pgfqpoint{-0.048611in}{0.000000in}}%
\pgfusepath{stroke,fill}%
}%
\begin{pgfscope}%
\pgfsys@transformshift{0.800000in}{1.452000in}%
\pgfsys@useobject{currentmarker}{}%
\end{pgfscope}%
\end{pgfscope}%
\begin{pgfscope}%
\definecolor{textcolor}{rgb}{0.000000,0.000000,0.000000}%
\pgfsetstrokecolor{textcolor}%
\pgfsetfillcolor{textcolor}%
\pgftext[x=0.525308in,y=1.403775in,left,base]{\color{textcolor}\rmfamily\fontsize{10.000000}{12.000000}\selectfont \(\displaystyle 0.8\)}%
\end{pgfscope}%
\begin{pgfscope}%
\pgfsetbuttcap%
\pgfsetroundjoin%
\definecolor{currentfill}{rgb}{0.000000,0.000000,0.000000}%
\pgfsetfillcolor{currentfill}%
\pgfsetlinewidth{0.803000pt}%
\definecolor{currentstroke}{rgb}{0.000000,0.000000,0.000000}%
\pgfsetstrokecolor{currentstroke}%
\pgfsetdash{}{0pt}%
\pgfsys@defobject{currentmarker}{\pgfqpoint{-0.048611in}{0.000000in}}{\pgfqpoint{0.000000in}{0.000000in}}{%
\pgfpathmoveto{\pgfqpoint{0.000000in}{0.000000in}}%
\pgfpathlineto{\pgfqpoint{-0.048611in}{0.000000in}}%
\pgfusepath{stroke,fill}%
}%
\begin{pgfscope}%
\pgfsys@transformshift{0.800000in}{2.068000in}%
\pgfsys@useobject{currentmarker}{}%
\end{pgfscope}%
\end{pgfscope}%
\begin{pgfscope}%
\definecolor{textcolor}{rgb}{0.000000,0.000000,0.000000}%
\pgfsetstrokecolor{textcolor}%
\pgfsetfillcolor{textcolor}%
\pgftext[x=0.525308in,y=2.019775in,left,base]{\color{textcolor}\rmfamily\fontsize{10.000000}{12.000000}\selectfont \(\displaystyle 1.0\)}%
\end{pgfscope}%
\begin{pgfscope}%
\pgfsetbuttcap%
\pgfsetroundjoin%
\definecolor{currentfill}{rgb}{0.000000,0.000000,0.000000}%
\pgfsetfillcolor{currentfill}%
\pgfsetlinewidth{0.803000pt}%
\definecolor{currentstroke}{rgb}{0.000000,0.000000,0.000000}%
\pgfsetstrokecolor{currentstroke}%
\pgfsetdash{}{0pt}%
\pgfsys@defobject{currentmarker}{\pgfqpoint{-0.048611in}{0.000000in}}{\pgfqpoint{0.000000in}{0.000000in}}{%
\pgfpathmoveto{\pgfqpoint{0.000000in}{0.000000in}}%
\pgfpathlineto{\pgfqpoint{-0.048611in}{0.000000in}}%
\pgfusepath{stroke,fill}%
}%
\begin{pgfscope}%
\pgfsys@transformshift{0.800000in}{2.684000in}%
\pgfsys@useobject{currentmarker}{}%
\end{pgfscope}%
\end{pgfscope}%
\begin{pgfscope}%
\definecolor{textcolor}{rgb}{0.000000,0.000000,0.000000}%
\pgfsetstrokecolor{textcolor}%
\pgfsetfillcolor{textcolor}%
\pgftext[x=0.525308in,y=2.635775in,left,base]{\color{textcolor}\rmfamily\fontsize{10.000000}{12.000000}\selectfont \(\displaystyle 1.2\)}%
\end{pgfscope}%
\begin{pgfscope}%
\pgfsetbuttcap%
\pgfsetroundjoin%
\definecolor{currentfill}{rgb}{0.000000,0.000000,0.000000}%
\pgfsetfillcolor{currentfill}%
\pgfsetlinewidth{0.803000pt}%
\definecolor{currentstroke}{rgb}{0.000000,0.000000,0.000000}%
\pgfsetstrokecolor{currentstroke}%
\pgfsetdash{}{0pt}%
\pgfsys@defobject{currentmarker}{\pgfqpoint{-0.048611in}{0.000000in}}{\pgfqpoint{0.000000in}{0.000000in}}{%
\pgfpathmoveto{\pgfqpoint{0.000000in}{0.000000in}}%
\pgfpathlineto{\pgfqpoint{-0.048611in}{0.000000in}}%
\pgfusepath{stroke,fill}%
}%
\begin{pgfscope}%
\pgfsys@transformshift{0.800000in}{3.300000in}%
\pgfsys@useobject{currentmarker}{}%
\end{pgfscope}%
\end{pgfscope}%
\begin{pgfscope}%
\definecolor{textcolor}{rgb}{0.000000,0.000000,0.000000}%
\pgfsetstrokecolor{textcolor}%
\pgfsetfillcolor{textcolor}%
\pgftext[x=0.525308in,y=3.251775in,left,base]{\color{textcolor}\rmfamily\fontsize{10.000000}{12.000000}\selectfont \(\displaystyle 1.4\)}%
\end{pgfscope}%
\begin{pgfscope}%
\pgfsetbuttcap%
\pgfsetroundjoin%
\definecolor{currentfill}{rgb}{0.000000,0.000000,0.000000}%
\pgfsetfillcolor{currentfill}%
\pgfsetlinewidth{0.803000pt}%
\definecolor{currentstroke}{rgb}{0.000000,0.000000,0.000000}%
\pgfsetstrokecolor{currentstroke}%
\pgfsetdash{}{0pt}%
\pgfsys@defobject{currentmarker}{\pgfqpoint{-0.048611in}{0.000000in}}{\pgfqpoint{0.000000in}{0.000000in}}{%
\pgfpathmoveto{\pgfqpoint{0.000000in}{0.000000in}}%
\pgfpathlineto{\pgfqpoint{-0.048611in}{0.000000in}}%
\pgfusepath{stroke,fill}%
}%
\begin{pgfscope}%
\pgfsys@transformshift{0.800000in}{3.916000in}%
\pgfsys@useobject{currentmarker}{}%
\end{pgfscope}%
\end{pgfscope}%
\begin{pgfscope}%
\definecolor{textcolor}{rgb}{0.000000,0.000000,0.000000}%
\pgfsetstrokecolor{textcolor}%
\pgfsetfillcolor{textcolor}%
\pgftext[x=0.525308in,y=3.867775in,left,base]{\color{textcolor}\rmfamily\fontsize{10.000000}{12.000000}\selectfont \(\displaystyle 1.6\)}%
\end{pgfscope}%
\begin{pgfscope}%
\pgfsetrectcap%
\pgfsetmiterjoin%
\pgfsetlinewidth{0.803000pt}%
\definecolor{currentstroke}{rgb}{0.000000,0.000000,0.000000}%
\pgfsetstrokecolor{currentstroke}%
\pgfsetdash{}{0pt}%
\pgfpathmoveto{\pgfqpoint{0.800000in}{0.528000in}}%
\pgfpathlineto{\pgfqpoint{0.800000in}{4.224000in}}%
\pgfusepath{stroke}%
\end{pgfscope}%
\begin{pgfscope}%
\pgfsetrectcap%
\pgfsetmiterjoin%
\pgfsetlinewidth{0.803000pt}%
\definecolor{currentstroke}{rgb}{0.000000,0.000000,0.000000}%
\pgfsetstrokecolor{currentstroke}%
\pgfsetdash{}{0pt}%
\pgfpathmoveto{\pgfqpoint{4.768000in}{0.528000in}}%
\pgfpathlineto{\pgfqpoint{4.768000in}{4.224000in}}%
\pgfusepath{stroke}%
\end{pgfscope}%
\begin{pgfscope}%
\pgfsetrectcap%
\pgfsetmiterjoin%
\pgfsetlinewidth{0.803000pt}%
\definecolor{currentstroke}{rgb}{0.000000,0.000000,0.000000}%
\pgfsetstrokecolor{currentstroke}%
\pgfsetdash{}{0pt}%
\pgfpathmoveto{\pgfqpoint{0.800000in}{0.528000in}}%
\pgfpathlineto{\pgfqpoint{4.768000in}{0.528000in}}%
\pgfusepath{stroke}%
\end{pgfscope}%
\begin{pgfscope}%
\pgfsetrectcap%
\pgfsetmiterjoin%
\pgfsetlinewidth{0.803000pt}%
\definecolor{currentstroke}{rgb}{0.000000,0.000000,0.000000}%
\pgfsetstrokecolor{currentstroke}%
\pgfsetdash{}{0pt}%
\pgfpathmoveto{\pgfqpoint{0.800000in}{4.224000in}}%
\pgfpathlineto{\pgfqpoint{4.768000in}{4.224000in}}%
\pgfusepath{stroke}%
\end{pgfscope}%
\begin{pgfscope}%
\pgfpathrectangle{\pgfqpoint{5.016000in}{0.528000in}}{\pgfqpoint{0.184800in}{3.696000in}}%
\pgfusepath{clip}%
\pgfsetbuttcap%
\pgfsetmiterjoin%
\definecolor{currentfill}{rgb}{1.000000,1.000000,1.000000}%
\pgfsetfillcolor{currentfill}%
\pgfsetlinewidth{0.010037pt}%
\definecolor{currentstroke}{rgb}{1.000000,1.000000,1.000000}%
\pgfsetstrokecolor{currentstroke}%
\pgfsetdash{}{0pt}%
\pgfpathmoveto{\pgfqpoint{5.016000in}{0.528000in}}%
\pgfpathlineto{\pgfqpoint{5.016000in}{0.531878in}}%
\pgfpathlineto{\pgfqpoint{5.016000in}{4.220122in}}%
\pgfpathlineto{\pgfqpoint{5.016000in}{4.224000in}}%
\pgfpathlineto{\pgfqpoint{5.200800in}{4.224000in}}%
\pgfpathlineto{\pgfqpoint{5.200800in}{4.220122in}}%
\pgfpathlineto{\pgfqpoint{5.200800in}{0.531878in}}%
\pgfpathlineto{\pgfqpoint{5.200800in}{0.528000in}}%
\pgfpathclose%
\pgfusepath{stroke,fill}%
\end{pgfscope}%
\begin{pgfscope}%
\pgfsys@transformshift{5.020000in}{0.530000in}%
\pgftext[left,bottom]{\pgfimage[interpolate=true,width=0.180000in,height=3.690000in]{CP2b_Contour_Plot-img0.png}}%
\end{pgfscope}%
\begin{pgfscope}%
\pgfsetbuttcap%
\pgfsetroundjoin%
\definecolor{currentfill}{rgb}{0.000000,0.000000,0.000000}%
\pgfsetfillcolor{currentfill}%
\pgfsetlinewidth{0.803000pt}%
\definecolor{currentstroke}{rgb}{0.000000,0.000000,0.000000}%
\pgfsetstrokecolor{currentstroke}%
\pgfsetdash{}{0pt}%
\pgfsys@defobject{currentmarker}{\pgfqpoint{0.000000in}{0.000000in}}{\pgfqpoint{0.048611in}{0.000000in}}{%
\pgfpathmoveto{\pgfqpoint{0.000000in}{0.000000in}}%
\pgfpathlineto{\pgfqpoint{0.048611in}{0.000000in}}%
\pgfusepath{stroke,fill}%
}%
\begin{pgfscope}%
\pgfsys@transformshift{5.200800in}{0.528000in}%
\pgfsys@useobject{currentmarker}{}%
\end{pgfscope}%
\end{pgfscope}%
\begin{pgfscope}%
\definecolor{textcolor}{rgb}{0.000000,0.000000,0.000000}%
\pgfsetstrokecolor{textcolor}%
\pgfsetfillcolor{textcolor}%
\pgftext[x=5.298022in,y=0.479775in,left,base]{\color{textcolor}\rmfamily\fontsize{10.000000}{12.000000}\selectfont \(\displaystyle 0.60\)}%
\end{pgfscope}%
\begin{pgfscope}%
\pgfsetbuttcap%
\pgfsetroundjoin%
\definecolor{currentfill}{rgb}{0.000000,0.000000,0.000000}%
\pgfsetfillcolor{currentfill}%
\pgfsetlinewidth{0.803000pt}%
\definecolor{currentstroke}{rgb}{0.000000,0.000000,0.000000}%
\pgfsetstrokecolor{currentstroke}%
\pgfsetdash{}{0pt}%
\pgfsys@defobject{currentmarker}{\pgfqpoint{0.000000in}{0.000000in}}{\pgfqpoint{0.048611in}{0.000000in}}{%
\pgfpathmoveto{\pgfqpoint{0.000000in}{0.000000in}}%
\pgfpathlineto{\pgfqpoint{0.048611in}{0.000000in}}%
\pgfusepath{stroke,fill}%
}%
\begin{pgfscope}%
\pgfsys@transformshift{5.200800in}{0.900315in}%
\pgfsys@useobject{currentmarker}{}%
\end{pgfscope}%
\end{pgfscope}%
\begin{pgfscope}%
\definecolor{textcolor}{rgb}{0.000000,0.000000,0.000000}%
\pgfsetstrokecolor{textcolor}%
\pgfsetfillcolor{textcolor}%
\pgftext[x=5.298022in,y=0.852090in,left,base]{\color{textcolor}\rmfamily\fontsize{10.000000}{12.000000}\selectfont \(\displaystyle 4.44\)}%
\end{pgfscope}%
\begin{pgfscope}%
\pgfsetbuttcap%
\pgfsetroundjoin%
\definecolor{currentfill}{rgb}{0.000000,0.000000,0.000000}%
\pgfsetfillcolor{currentfill}%
\pgfsetlinewidth{0.803000pt}%
\definecolor{currentstroke}{rgb}{0.000000,0.000000,0.000000}%
\pgfsetstrokecolor{currentstroke}%
\pgfsetdash{}{0pt}%
\pgfsys@defobject{currentmarker}{\pgfqpoint{0.000000in}{0.000000in}}{\pgfqpoint{0.048611in}{0.000000in}}{%
\pgfpathmoveto{\pgfqpoint{0.000000in}{0.000000in}}%
\pgfpathlineto{\pgfqpoint{0.048611in}{0.000000in}}%
\pgfusepath{stroke,fill}%
}%
\begin{pgfscope}%
\pgfsys@transformshift{5.200800in}{1.272630in}%
\pgfsys@useobject{currentmarker}{}%
\end{pgfscope}%
\end{pgfscope}%
\begin{pgfscope}%
\definecolor{textcolor}{rgb}{0.000000,0.000000,0.000000}%
\pgfsetstrokecolor{textcolor}%
\pgfsetfillcolor{textcolor}%
\pgftext[x=5.298022in,y=1.224404in,left,base]{\color{textcolor}\rmfamily\fontsize{10.000000}{12.000000}\selectfont \(\displaystyle 8.28\)}%
\end{pgfscope}%
\begin{pgfscope}%
\pgfsetbuttcap%
\pgfsetroundjoin%
\definecolor{currentfill}{rgb}{0.000000,0.000000,0.000000}%
\pgfsetfillcolor{currentfill}%
\pgfsetlinewidth{0.803000pt}%
\definecolor{currentstroke}{rgb}{0.000000,0.000000,0.000000}%
\pgfsetstrokecolor{currentstroke}%
\pgfsetdash{}{0pt}%
\pgfsys@defobject{currentmarker}{\pgfqpoint{0.000000in}{0.000000in}}{\pgfqpoint{0.048611in}{0.000000in}}{%
\pgfpathmoveto{\pgfqpoint{0.000000in}{0.000000in}}%
\pgfpathlineto{\pgfqpoint{0.048611in}{0.000000in}}%
\pgfusepath{stroke,fill}%
}%
\begin{pgfscope}%
\pgfsys@transformshift{5.200800in}{1.644944in}%
\pgfsys@useobject{currentmarker}{}%
\end{pgfscope}%
\end{pgfscope}%
\begin{pgfscope}%
\definecolor{textcolor}{rgb}{0.000000,0.000000,0.000000}%
\pgfsetstrokecolor{textcolor}%
\pgfsetfillcolor{textcolor}%
\pgftext[x=5.298022in,y=1.596719in,left,base]{\color{textcolor}\rmfamily\fontsize{10.000000}{12.000000}\selectfont \(\displaystyle 12.12\)}%
\end{pgfscope}%
\begin{pgfscope}%
\pgfsetbuttcap%
\pgfsetroundjoin%
\definecolor{currentfill}{rgb}{0.000000,0.000000,0.000000}%
\pgfsetfillcolor{currentfill}%
\pgfsetlinewidth{0.803000pt}%
\definecolor{currentstroke}{rgb}{0.000000,0.000000,0.000000}%
\pgfsetstrokecolor{currentstroke}%
\pgfsetdash{}{0pt}%
\pgfsys@defobject{currentmarker}{\pgfqpoint{0.000000in}{0.000000in}}{\pgfqpoint{0.048611in}{0.000000in}}{%
\pgfpathmoveto{\pgfqpoint{0.000000in}{0.000000in}}%
\pgfpathlineto{\pgfqpoint{0.048611in}{0.000000in}}%
\pgfusepath{stroke,fill}%
}%
\begin{pgfscope}%
\pgfsys@transformshift{5.200800in}{2.017259in}%
\pgfsys@useobject{currentmarker}{}%
\end{pgfscope}%
\end{pgfscope}%
\begin{pgfscope}%
\definecolor{textcolor}{rgb}{0.000000,0.000000,0.000000}%
\pgfsetstrokecolor{textcolor}%
\pgfsetfillcolor{textcolor}%
\pgftext[x=5.298022in,y=1.969034in,left,base]{\color{textcolor}\rmfamily\fontsize{10.000000}{12.000000}\selectfont \(\displaystyle 15.96\)}%
\end{pgfscope}%
\begin{pgfscope}%
\pgfsetbuttcap%
\pgfsetroundjoin%
\definecolor{currentfill}{rgb}{0.000000,0.000000,0.000000}%
\pgfsetfillcolor{currentfill}%
\pgfsetlinewidth{0.803000pt}%
\definecolor{currentstroke}{rgb}{0.000000,0.000000,0.000000}%
\pgfsetstrokecolor{currentstroke}%
\pgfsetdash{}{0pt}%
\pgfsys@defobject{currentmarker}{\pgfqpoint{0.000000in}{0.000000in}}{\pgfqpoint{0.048611in}{0.000000in}}{%
\pgfpathmoveto{\pgfqpoint{0.000000in}{0.000000in}}%
\pgfpathlineto{\pgfqpoint{0.048611in}{0.000000in}}%
\pgfusepath{stroke,fill}%
}%
\begin{pgfscope}%
\pgfsys@transformshift{5.200800in}{2.389574in}%
\pgfsys@useobject{currentmarker}{}%
\end{pgfscope}%
\end{pgfscope}%
\begin{pgfscope}%
\definecolor{textcolor}{rgb}{0.000000,0.000000,0.000000}%
\pgfsetstrokecolor{textcolor}%
\pgfsetfillcolor{textcolor}%
\pgftext[x=5.298022in,y=2.341349in,left,base]{\color{textcolor}\rmfamily\fontsize{10.000000}{12.000000}\selectfont \(\displaystyle 19.80\)}%
\end{pgfscope}%
\begin{pgfscope}%
\pgfsetbuttcap%
\pgfsetroundjoin%
\definecolor{currentfill}{rgb}{0.000000,0.000000,0.000000}%
\pgfsetfillcolor{currentfill}%
\pgfsetlinewidth{0.803000pt}%
\definecolor{currentstroke}{rgb}{0.000000,0.000000,0.000000}%
\pgfsetstrokecolor{currentstroke}%
\pgfsetdash{}{0pt}%
\pgfsys@defobject{currentmarker}{\pgfqpoint{0.000000in}{0.000000in}}{\pgfqpoint{0.048611in}{0.000000in}}{%
\pgfpathmoveto{\pgfqpoint{0.000000in}{0.000000in}}%
\pgfpathlineto{\pgfqpoint{0.048611in}{0.000000in}}%
\pgfusepath{stroke,fill}%
}%
\begin{pgfscope}%
\pgfsys@transformshift{5.200800in}{2.761889in}%
\pgfsys@useobject{currentmarker}{}%
\end{pgfscope}%
\end{pgfscope}%
\begin{pgfscope}%
\definecolor{textcolor}{rgb}{0.000000,0.000000,0.000000}%
\pgfsetstrokecolor{textcolor}%
\pgfsetfillcolor{textcolor}%
\pgftext[x=5.298022in,y=2.713663in,left,base]{\color{textcolor}\rmfamily\fontsize{10.000000}{12.000000}\selectfont \(\displaystyle 23.64\)}%
\end{pgfscope}%
\begin{pgfscope}%
\pgfsetbuttcap%
\pgfsetroundjoin%
\definecolor{currentfill}{rgb}{0.000000,0.000000,0.000000}%
\pgfsetfillcolor{currentfill}%
\pgfsetlinewidth{0.803000pt}%
\definecolor{currentstroke}{rgb}{0.000000,0.000000,0.000000}%
\pgfsetstrokecolor{currentstroke}%
\pgfsetdash{}{0pt}%
\pgfsys@defobject{currentmarker}{\pgfqpoint{0.000000in}{0.000000in}}{\pgfqpoint{0.048611in}{0.000000in}}{%
\pgfpathmoveto{\pgfqpoint{0.000000in}{0.000000in}}%
\pgfpathlineto{\pgfqpoint{0.048611in}{0.000000in}}%
\pgfusepath{stroke,fill}%
}%
\begin{pgfscope}%
\pgfsys@transformshift{5.200800in}{3.134204in}%
\pgfsys@useobject{currentmarker}{}%
\end{pgfscope}%
\end{pgfscope}%
\begin{pgfscope}%
\definecolor{textcolor}{rgb}{0.000000,0.000000,0.000000}%
\pgfsetstrokecolor{textcolor}%
\pgfsetfillcolor{textcolor}%
\pgftext[x=5.298022in,y=3.085978in,left,base]{\color{textcolor}\rmfamily\fontsize{10.000000}{12.000000}\selectfont \(\displaystyle 27.48\)}%
\end{pgfscope}%
\begin{pgfscope}%
\pgfsetbuttcap%
\pgfsetroundjoin%
\definecolor{currentfill}{rgb}{0.000000,0.000000,0.000000}%
\pgfsetfillcolor{currentfill}%
\pgfsetlinewidth{0.803000pt}%
\definecolor{currentstroke}{rgb}{0.000000,0.000000,0.000000}%
\pgfsetstrokecolor{currentstroke}%
\pgfsetdash{}{0pt}%
\pgfsys@defobject{currentmarker}{\pgfqpoint{0.000000in}{0.000000in}}{\pgfqpoint{0.048611in}{0.000000in}}{%
\pgfpathmoveto{\pgfqpoint{0.000000in}{0.000000in}}%
\pgfpathlineto{\pgfqpoint{0.048611in}{0.000000in}}%
\pgfusepath{stroke,fill}%
}%
\begin{pgfscope}%
\pgfsys@transformshift{5.200800in}{3.506518in}%
\pgfsys@useobject{currentmarker}{}%
\end{pgfscope}%
\end{pgfscope}%
\begin{pgfscope}%
\definecolor{textcolor}{rgb}{0.000000,0.000000,0.000000}%
\pgfsetstrokecolor{textcolor}%
\pgfsetfillcolor{textcolor}%
\pgftext[x=5.298022in,y=3.458293in,left,base]{\color{textcolor}\rmfamily\fontsize{10.000000}{12.000000}\selectfont \(\displaystyle 31.32\)}%
\end{pgfscope}%
\begin{pgfscope}%
\pgfsetbuttcap%
\pgfsetroundjoin%
\definecolor{currentfill}{rgb}{0.000000,0.000000,0.000000}%
\pgfsetfillcolor{currentfill}%
\pgfsetlinewidth{0.803000pt}%
\definecolor{currentstroke}{rgb}{0.000000,0.000000,0.000000}%
\pgfsetstrokecolor{currentstroke}%
\pgfsetdash{}{0pt}%
\pgfsys@defobject{currentmarker}{\pgfqpoint{0.000000in}{0.000000in}}{\pgfqpoint{0.048611in}{0.000000in}}{%
\pgfpathmoveto{\pgfqpoint{0.000000in}{0.000000in}}%
\pgfpathlineto{\pgfqpoint{0.048611in}{0.000000in}}%
\pgfusepath{stroke,fill}%
}%
\begin{pgfscope}%
\pgfsys@transformshift{5.200800in}{3.878833in}%
\pgfsys@useobject{currentmarker}{}%
\end{pgfscope}%
\end{pgfscope}%
\begin{pgfscope}%
\definecolor{textcolor}{rgb}{0.000000,0.000000,0.000000}%
\pgfsetstrokecolor{textcolor}%
\pgfsetfillcolor{textcolor}%
\pgftext[x=5.298022in,y=3.830608in,left,base]{\color{textcolor}\rmfamily\fontsize{10.000000}{12.000000}\selectfont \(\displaystyle 35.16\)}%
\end{pgfscope}%
\begin{pgfscope}%
\pgfsetbuttcap%
\pgfsetmiterjoin%
\pgfsetlinewidth{0.803000pt}%
\definecolor{currentstroke}{rgb}{0.000000,0.000000,0.000000}%
\pgfsetstrokecolor{currentstroke}%
\pgfsetdash{}{0pt}%
\pgfpathmoveto{\pgfqpoint{5.016000in}{0.528000in}}%
\pgfpathlineto{\pgfqpoint{5.016000in}{0.531878in}}%
\pgfpathlineto{\pgfqpoint{5.016000in}{4.220122in}}%
\pgfpathlineto{\pgfqpoint{5.016000in}{4.224000in}}%
\pgfpathlineto{\pgfqpoint{5.200800in}{4.224000in}}%
\pgfpathlineto{\pgfqpoint{5.200800in}{4.220122in}}%
\pgfpathlineto{\pgfqpoint{5.200800in}{0.531878in}}%
\pgfpathlineto{\pgfqpoint{5.200800in}{0.528000in}}%
\pgfpathclose%
\pgfusepath{stroke}%
\end{pgfscope}%
\end{pgfpicture}%
\makeatother%
\endgroup%
}
            \caption{Weighted Contour Plot for LinearWithErrors.txt}
            \label{fig:Weighted Contour}
        \end{center}
    \end{figure}
    
    \newpage\noindent
    Lastly, in Figure \ref{fig:CP1c_Contour_Plot} we have a contour plot that corresponds to 
    the data from LinearNoErrors.txt. It's clear that the range of possible values of 
    $\frac{\chi^2}{\texttt{dof}}$ for the weighted fit is much larger than for the unweighted fit. 
    We've plotted the two contours using the same scale for the value of $\frac{\chi^2}{\texttt{dof}}$ 
    in order to show just how much bigger the range is for the weighted fit. 

    \begin{figure}[H]
        \begin{center}
           \scalebox{.7}{%% Creator: Matplotlib, PGF backend
%%
%% To include the figure in your LaTeX document, write
%%   \input{<filename>.pgf}
%%
%% Make sure the required packages are loaded in your preamble
%%   \usepackage{pgf}
%%
%% Figures using additional raster images can only be included by \input if
%% they are in the same directory as the main LaTeX file. For loading figures
%% from other directories you can use the `import` package
%%   \usepackage{import}
%% and then include the figures with
%%   \import{<path to file>}{<filename>.pgf}
%%
%% Matplotlib used the following preamble
%%
\begingroup%
\makeatletter%
\begin{pgfpicture}%
\pgfpathrectangle{\pgfpointorigin}{\pgfqpoint{6.400000in}{4.800000in}}%
\pgfusepath{use as bounding box, clip}%
\begin{pgfscope}%
\pgfsetbuttcap%
\pgfsetmiterjoin%
\definecolor{currentfill}{rgb}{1.000000,1.000000,1.000000}%
\pgfsetfillcolor{currentfill}%
\pgfsetlinewidth{0.000000pt}%
\definecolor{currentstroke}{rgb}{1.000000,1.000000,1.000000}%
\pgfsetstrokecolor{currentstroke}%
\pgfsetdash{}{0pt}%
\pgfpathmoveto{\pgfqpoint{0.000000in}{0.000000in}}%
\pgfpathlineto{\pgfqpoint{6.400000in}{0.000000in}}%
\pgfpathlineto{\pgfqpoint{6.400000in}{4.800000in}}%
\pgfpathlineto{\pgfqpoint{0.000000in}{4.800000in}}%
\pgfpathclose%
\pgfusepath{fill}%
\end{pgfscope}%
\begin{pgfscope}%
\pgfsetbuttcap%
\pgfsetmiterjoin%
\definecolor{currentfill}{rgb}{1.000000,1.000000,1.000000}%
\pgfsetfillcolor{currentfill}%
\pgfsetlinewidth{0.000000pt}%
\definecolor{currentstroke}{rgb}{0.000000,0.000000,0.000000}%
\pgfsetstrokecolor{currentstroke}%
\pgfsetstrokeopacity{0.000000}%
\pgfsetdash{}{0pt}%
\pgfpathmoveto{\pgfqpoint{0.800000in}{0.528000in}}%
\pgfpathlineto{\pgfqpoint{4.768000in}{0.528000in}}%
\pgfpathlineto{\pgfqpoint{4.768000in}{4.224000in}}%
\pgfpathlineto{\pgfqpoint{0.800000in}{4.224000in}}%
\pgfpathclose%
\pgfusepath{fill}%
\end{pgfscope}%
\begin{pgfscope}%
\pgfpathrectangle{\pgfqpoint{0.800000in}{0.528000in}}{\pgfqpoint{3.968000in}{3.696000in}}%
\pgfusepath{clip}%
\pgfsetbuttcap%
\pgfsetroundjoin%
\definecolor{currentfill}{rgb}{0.000000,0.000000,0.553476}%
\pgfsetfillcolor{currentfill}%
\pgfsetlinewidth{1.003750pt}%
\definecolor{currentstroke}{rgb}{0.000000,0.000000,0.553476}%
\pgfsetstrokecolor{currentstroke}%
\pgfsetdash{}{0pt}%
\pgfpathmoveto{\pgfqpoint{4.768000in}{2.487834in}}%
\pgfpathlineto{\pgfqpoint{4.447354in}{2.887644in}}%
\pgfpathlineto{\pgfqpoint{4.086626in}{3.320855in}}%
\pgfpathlineto{\pgfqpoint{3.758037in}{3.701333in}}%
\pgfpathlineto{\pgfqpoint{3.592644in}{3.888000in}}%
\pgfpathlineto{\pgfqpoint{3.287791in}{4.224000in}}%
\pgfpathlineto{\pgfqpoint{0.800000in}{4.224000in}}%
\pgfpathlineto{\pgfqpoint{0.800000in}{1.604087in}}%
\pgfpathlineto{\pgfqpoint{0.980655in}{1.386667in}}%
\pgfpathlineto{\pgfqpoint{1.138536in}{1.200000in}}%
\pgfpathlineto{\pgfqpoint{1.299080in}{1.013333in}}%
\pgfpathlineto{\pgfqpoint{1.462341in}{0.826667in}}%
\pgfpathlineto{\pgfqpoint{1.616298in}{0.653675in}}%
\pgfpathlineto{\pgfqpoint{1.695598in}{0.565333in}}%
\pgfpathlineto{\pgfqpoint{1.729354in}{0.528000in}}%
\pgfpathlineto{\pgfqpoint{4.768000in}{0.528000in}}%
\pgfpathlineto{\pgfqpoint{4.768000in}{2.469333in}}%
\pgfpathlineto{\pgfqpoint{4.768000in}{2.469333in}}%
\pgfusepath{stroke,fill}%
\end{pgfscope}%
\begin{pgfscope}%
\pgfpathrectangle{\pgfqpoint{0.800000in}{0.528000in}}{\pgfqpoint{3.968000in}{3.696000in}}%
\pgfusepath{clip}%
\pgfsetbuttcap%
\pgfsetroundjoin%
\definecolor{currentfill}{rgb}{0.000000,0.000000,0.660428}%
\pgfsetfillcolor{currentfill}%
\pgfsetlinewidth{1.003750pt}%
\definecolor{currentstroke}{rgb}{0.000000,0.000000,0.660428}%
\pgfsetstrokecolor{currentstroke}%
\pgfsetdash{}{0pt}%
\pgfpathmoveto{\pgfqpoint{1.729354in}{0.528000in}}%
\pgfpathlineto{\pgfqpoint{1.594913in}{0.677333in}}%
\pgfpathlineto{\pgfqpoint{1.234500in}{1.088000in}}%
\pgfpathlineto{\pgfqpoint{1.075022in}{1.274667in}}%
\pgfpathlineto{\pgfqpoint{0.910968in}{1.469972in}}%
\pgfpathlineto{\pgfqpoint{0.800000in}{1.604087in}}%
\pgfpathlineto{\pgfqpoint{0.800000in}{0.528000in}}%
\pgfpathlineto{\pgfqpoint{1.721859in}{0.528000in}}%
\pgfpathmoveto{\pgfqpoint{4.768000in}{3.784956in}}%
\pgfpathlineto{\pgfqpoint{4.552107in}{4.037333in}}%
\pgfpathlineto{\pgfqpoint{4.390267in}{4.224000in}}%
\pgfpathlineto{\pgfqpoint{3.287791in}{4.224000in}}%
\pgfpathlineto{\pgfqpoint{3.458300in}{4.037333in}}%
\pgfpathlineto{\pgfqpoint{3.625932in}{3.850667in}}%
\pgfpathlineto{\pgfqpoint{3.779530in}{3.676622in}}%
\pgfpathlineto{\pgfqpoint{3.855917in}{3.589333in}}%
\pgfpathlineto{\pgfqpoint{4.206869in}{3.178350in}}%
\pgfpathlineto{\pgfqpoint{4.392412in}{2.954667in}}%
\pgfpathlineto{\pgfqpoint{4.727919in}{2.538646in}}%
\pgfpathlineto{\pgfqpoint{4.768000in}{2.487834in}}%
\pgfpathlineto{\pgfqpoint{4.768000in}{3.776000in}}%
\pgfpathlineto{\pgfqpoint{4.768000in}{3.776000in}}%
\pgfusepath{stroke,fill}%
\end{pgfscope}%
\begin{pgfscope}%
\pgfpathrectangle{\pgfqpoint{0.800000in}{0.528000in}}{\pgfqpoint{3.968000in}{3.696000in}}%
\pgfusepath{clip}%
\pgfsetbuttcap%
\pgfsetroundjoin%
\definecolor{currentfill}{rgb}{0.000000,0.000000,0.785205}%
\pgfsetfillcolor{currentfill}%
\pgfsetlinewidth{1.003750pt}%
\definecolor{currentstroke}{rgb}{0.000000,0.000000,0.785205}%
\pgfsetstrokecolor{currentstroke}%
\pgfsetdash{}{0pt}%
\pgfpathmoveto{\pgfqpoint{4.390267in}{4.224000in}}%
\pgfpathlineto{\pgfqpoint{4.397846in}{4.215219in}}%
\pgfpathlineto{\pgfqpoint{4.407273in}{4.204509in}}%
\pgfpathlineto{\pgfqpoint{4.422786in}{4.186667in}}%
\pgfpathlineto{\pgfqpoint{4.433719in}{4.173967in}}%
\pgfpathlineto{\pgfqpoint{4.447354in}{4.158436in}}%
\pgfpathlineto{\pgfqpoint{4.455248in}{4.149333in}}%
\pgfpathlineto{\pgfqpoint{4.469551in}{4.132676in}}%
\pgfpathlineto{\pgfqpoint{4.487434in}{4.112252in}}%
\pgfpathlineto{\pgfqpoint{4.487652in}{4.112000in}}%
\pgfpathlineto{\pgfqpoint{4.505342in}{4.091346in}}%
\pgfpathlineto{\pgfqpoint{4.519909in}{4.074667in}}%
\pgfpathlineto{\pgfqpoint{4.523282in}{4.070723in}}%
\pgfpathlineto{\pgfqpoint{4.527515in}{4.065871in}}%
\pgfpathlineto{\pgfqpoint{4.552107in}{4.037333in}}%
\pgfpathlineto{\pgfqpoint{4.558965in}{4.029294in}}%
\pgfpathlineto{\pgfqpoint{4.567596in}{4.019375in}}%
\pgfpathlineto{\pgfqpoint{4.576801in}{4.008574in}}%
\pgfpathlineto{\pgfqpoint{4.584250in}{4.000000in}}%
\pgfpathlineto{\pgfqpoint{4.594608in}{3.987828in}}%
\pgfpathlineto{\pgfqpoint{4.607677in}{3.972769in}}%
\pgfpathlineto{\pgfqpoint{4.616338in}{3.962667in}}%
\pgfpathlineto{\pgfqpoint{4.630211in}{3.946323in}}%
\pgfpathlineto{\pgfqpoint{4.647758in}{3.926052in}}%
\pgfpathlineto{\pgfqpoint{4.648098in}{3.925650in}}%
\pgfpathlineto{\pgfqpoint{4.648372in}{3.925333in}}%
\pgfpathlineto{\pgfqpoint{4.650788in}{3.922511in}}%
\pgfpathlineto{\pgfqpoint{4.680262in}{3.888000in}}%
\pgfpathlineto{\pgfqpoint{4.683599in}{3.884052in}}%
\pgfpathlineto{\pgfqpoint{4.687838in}{3.879136in}}%
\pgfpathlineto{\pgfqpoint{4.701296in}{3.863201in}}%
\pgfpathlineto{\pgfqpoint{4.712090in}{3.850667in}}%
\pgfpathlineto{\pgfqpoint{4.727919in}{3.832101in}}%
\pgfpathlineto{\pgfqpoint{4.736778in}{3.821585in}}%
\pgfpathlineto{\pgfqpoint{4.743867in}{3.813333in}}%
\pgfpathlineto{\pgfqpoint{4.768000in}{3.784956in}}%
\pgfpathlineto{\pgfqpoint{4.768000in}{3.813333in}}%
\pgfpathlineto{\pgfqpoint{4.768000in}{3.850667in}}%
\pgfpathlineto{\pgfqpoint{4.768000in}{3.888000in}}%
\pgfpathlineto{\pgfqpoint{4.768000in}{3.925333in}}%
\pgfpathlineto{\pgfqpoint{4.768000in}{3.962667in}}%
\pgfpathlineto{\pgfqpoint{4.768000in}{4.000000in}}%
\pgfpathlineto{\pgfqpoint{4.768000in}{4.037333in}}%
\pgfpathlineto{\pgfqpoint{4.768000in}{4.074667in}}%
\pgfpathlineto{\pgfqpoint{4.768000in}{4.112000in}}%
\pgfpathlineto{\pgfqpoint{4.768000in}{4.149333in}}%
\pgfpathlineto{\pgfqpoint{4.768000in}{4.186667in}}%
\pgfpathlineto{\pgfqpoint{4.768000in}{4.224000in}}%
\pgfpathlineto{\pgfqpoint{4.727919in}{4.224000in}}%
\pgfpathlineto{\pgfqpoint{4.687838in}{4.224000in}}%
\pgfpathlineto{\pgfqpoint{4.647758in}{4.224000in}}%
\pgfpathlineto{\pgfqpoint{4.607677in}{4.224000in}}%
\pgfpathlineto{\pgfqpoint{4.567596in}{4.224000in}}%
\pgfpathlineto{\pgfqpoint{4.527515in}{4.224000in}}%
\pgfpathlineto{\pgfqpoint{4.487434in}{4.224000in}}%
\pgfpathlineto{\pgfqpoint{4.447354in}{4.224000in}}%
\pgfpathlineto{\pgfqpoint{4.407273in}{4.224000in}}%
\pgfusepath{stroke,fill}%
\end{pgfscope}%
\begin{pgfscope}%
\pgfpathrectangle{\pgfqpoint{0.800000in}{0.528000in}}{\pgfqpoint{3.968000in}{3.696000in}}%
\pgfusepath{clip}%
\pgfsetbuttcap%
\pgfsetroundjoin%
\definecolor{currentfill}{rgb}{0.000000,0.000000,0.892157}%
\pgfsetfillcolor{currentfill}%
\pgfsetlinewidth{1.003750pt}%
\definecolor{currentstroke}{rgb}{0.000000,0.000000,0.892157}%
\pgfsetstrokecolor{currentstroke}%
\pgfsetdash{}{0pt}%
\pgfsys@defobject{currentmarker}{\pgfqpoint{infin}{infin}}{\pgfqpoint{-infin}{-infin}}{%
\pgfusepath{stroke,fill}%
}%
\begin{pgfscope}%
\pgfsys@transformshift{0.000000in}{0.000000in}%
\pgfsys@useobject{currentmarker}{}%
\end{pgfscope}%
\end{pgfscope}%
\begin{pgfscope}%
\pgfpathrectangle{\pgfqpoint{0.800000in}{0.528000in}}{\pgfqpoint{3.968000in}{3.696000in}}%
\pgfusepath{clip}%
\pgfsetbuttcap%
\pgfsetroundjoin%
\definecolor{currentfill}{rgb}{0.000000,0.000000,0.999109}%
\pgfsetfillcolor{currentfill}%
\pgfsetlinewidth{1.003750pt}%
\definecolor{currentstroke}{rgb}{0.000000,0.000000,0.999109}%
\pgfsetstrokecolor{currentstroke}%
\pgfsetdash{}{0pt}%
\pgfsys@defobject{currentmarker}{\pgfqpoint{infin}{infin}}{\pgfqpoint{-infin}{-infin}}{%
\pgfusepath{stroke,fill}%
}%
\begin{pgfscope}%
\pgfsys@transformshift{0.000000in}{0.000000in}%
\pgfsys@useobject{currentmarker}{}%
\end{pgfscope}%
\end{pgfscope}%
\begin{pgfscope}%
\pgfpathrectangle{\pgfqpoint{0.800000in}{0.528000in}}{\pgfqpoint{3.968000in}{3.696000in}}%
\pgfusepath{clip}%
\pgfsetbuttcap%
\pgfsetroundjoin%
\definecolor{currentfill}{rgb}{0.000000,0.049020,1.000000}%
\pgfsetfillcolor{currentfill}%
\pgfsetlinewidth{1.003750pt}%
\definecolor{currentstroke}{rgb}{0.000000,0.049020,1.000000}%
\pgfsetstrokecolor{currentstroke}%
\pgfsetdash{}{0pt}%
\pgfsys@defobject{currentmarker}{\pgfqpoint{infin}{infin}}{\pgfqpoint{-infin}{-infin}}{%
\pgfusepath{stroke,fill}%
}%
\begin{pgfscope}%
\pgfsys@transformshift{0.000000in}{0.000000in}%
\pgfsys@useobject{currentmarker}{}%
\end{pgfscope}%
\end{pgfscope}%
\begin{pgfscope}%
\pgfpathrectangle{\pgfqpoint{0.800000in}{0.528000in}}{\pgfqpoint{3.968000in}{3.696000in}}%
\pgfusepath{clip}%
\pgfsetbuttcap%
\pgfsetroundjoin%
\definecolor{currentfill}{rgb}{0.000000,0.143137,1.000000}%
\pgfsetfillcolor{currentfill}%
\pgfsetlinewidth{1.003750pt}%
\definecolor{currentstroke}{rgb}{0.000000,0.143137,1.000000}%
\pgfsetstrokecolor{currentstroke}%
\pgfsetdash{}{0pt}%
\pgfsys@defobject{currentmarker}{\pgfqpoint{infin}{infin}}{\pgfqpoint{-infin}{-infin}}{%
\pgfusepath{stroke,fill}%
}%
\begin{pgfscope}%
\pgfsys@transformshift{0.000000in}{0.000000in}%
\pgfsys@useobject{currentmarker}{}%
\end{pgfscope}%
\end{pgfscope}%
\begin{pgfscope}%
\pgfpathrectangle{\pgfqpoint{0.800000in}{0.528000in}}{\pgfqpoint{3.968000in}{3.696000in}}%
\pgfusepath{clip}%
\pgfsetbuttcap%
\pgfsetroundjoin%
\definecolor{currentfill}{rgb}{0.000000,0.252941,1.000000}%
\pgfsetfillcolor{currentfill}%
\pgfsetlinewidth{1.003750pt}%
\definecolor{currentstroke}{rgb}{0.000000,0.252941,1.000000}%
\pgfsetstrokecolor{currentstroke}%
\pgfsetdash{}{0pt}%
\pgfsys@defobject{currentmarker}{\pgfqpoint{infin}{infin}}{\pgfqpoint{-infin}{-infin}}{%
\pgfusepath{stroke,fill}%
}%
\begin{pgfscope}%
\pgfsys@transformshift{0.000000in}{0.000000in}%
\pgfsys@useobject{currentmarker}{}%
\end{pgfscope}%
\end{pgfscope}%
\begin{pgfscope}%
\pgfpathrectangle{\pgfqpoint{0.800000in}{0.528000in}}{\pgfqpoint{3.968000in}{3.696000in}}%
\pgfusepath{clip}%
\pgfsetbuttcap%
\pgfsetroundjoin%
\definecolor{currentfill}{rgb}{0.000000,0.347059,1.000000}%
\pgfsetfillcolor{currentfill}%
\pgfsetlinewidth{1.003750pt}%
\definecolor{currentstroke}{rgb}{0.000000,0.347059,1.000000}%
\pgfsetstrokecolor{currentstroke}%
\pgfsetdash{}{0pt}%
\pgfsys@defobject{currentmarker}{\pgfqpoint{infin}{infin}}{\pgfqpoint{-infin}{-infin}}{%
\pgfusepath{stroke,fill}%
}%
\begin{pgfscope}%
\pgfsys@transformshift{0.000000in}{0.000000in}%
\pgfsys@useobject{currentmarker}{}%
\end{pgfscope}%
\end{pgfscope}%
\begin{pgfscope}%
\pgfpathrectangle{\pgfqpoint{0.800000in}{0.528000in}}{\pgfqpoint{3.968000in}{3.696000in}}%
\pgfusepath{clip}%
\pgfsetbuttcap%
\pgfsetroundjoin%
\definecolor{currentfill}{rgb}{0.000000,0.441176,1.000000}%
\pgfsetfillcolor{currentfill}%
\pgfsetlinewidth{1.003750pt}%
\definecolor{currentstroke}{rgb}{0.000000,0.441176,1.000000}%
\pgfsetstrokecolor{currentstroke}%
\pgfsetdash{}{0pt}%
\pgfsys@defobject{currentmarker}{\pgfqpoint{infin}{infin}}{\pgfqpoint{-infin}{-infin}}{%
\pgfusepath{stroke,fill}%
}%
\begin{pgfscope}%
\pgfsys@transformshift{0.000000in}{0.000000in}%
\pgfsys@useobject{currentmarker}{}%
\end{pgfscope}%
\end{pgfscope}%
\begin{pgfscope}%
\pgfpathrectangle{\pgfqpoint{0.800000in}{0.528000in}}{\pgfqpoint{3.968000in}{3.696000in}}%
\pgfusepath{clip}%
\pgfsetbuttcap%
\pgfsetroundjoin%
\definecolor{currentfill}{rgb}{0.000000,0.550980,1.000000}%
\pgfsetfillcolor{currentfill}%
\pgfsetlinewidth{1.003750pt}%
\definecolor{currentstroke}{rgb}{0.000000,0.550980,1.000000}%
\pgfsetstrokecolor{currentstroke}%
\pgfsetdash{}{0pt}%
\pgfsys@defobject{currentmarker}{\pgfqpoint{infin}{infin}}{\pgfqpoint{-infin}{-infin}}{%
\pgfusepath{stroke,fill}%
}%
\begin{pgfscope}%
\pgfsys@transformshift{0.000000in}{0.000000in}%
\pgfsys@useobject{currentmarker}{}%
\end{pgfscope}%
\end{pgfscope}%
\begin{pgfscope}%
\pgfpathrectangle{\pgfqpoint{0.800000in}{0.528000in}}{\pgfqpoint{3.968000in}{3.696000in}}%
\pgfusepath{clip}%
\pgfsetbuttcap%
\pgfsetroundjoin%
\definecolor{currentfill}{rgb}{0.000000,0.645098,1.000000}%
\pgfsetfillcolor{currentfill}%
\pgfsetlinewidth{1.003750pt}%
\definecolor{currentstroke}{rgb}{0.000000,0.645098,1.000000}%
\pgfsetstrokecolor{currentstroke}%
\pgfsetdash{}{0pt}%
\pgfsys@defobject{currentmarker}{\pgfqpoint{infin}{infin}}{\pgfqpoint{-infin}{-infin}}{%
\pgfusepath{stroke,fill}%
}%
\begin{pgfscope}%
\pgfsys@transformshift{0.000000in}{0.000000in}%
\pgfsys@useobject{currentmarker}{}%
\end{pgfscope}%
\end{pgfscope}%
\begin{pgfscope}%
\pgfpathrectangle{\pgfqpoint{0.800000in}{0.528000in}}{\pgfqpoint{3.968000in}{3.696000in}}%
\pgfusepath{clip}%
\pgfsetbuttcap%
\pgfsetroundjoin%
\definecolor{currentfill}{rgb}{0.000000,0.754902,1.000000}%
\pgfsetfillcolor{currentfill}%
\pgfsetlinewidth{1.003750pt}%
\definecolor{currentstroke}{rgb}{0.000000,0.754902,1.000000}%
\pgfsetstrokecolor{currentstroke}%
\pgfsetdash{}{0pt}%
\pgfsys@defobject{currentmarker}{\pgfqpoint{infin}{infin}}{\pgfqpoint{-infin}{-infin}}{%
\pgfusepath{stroke,fill}%
}%
\begin{pgfscope}%
\pgfsys@transformshift{0.000000in}{0.000000in}%
\pgfsys@useobject{currentmarker}{}%
\end{pgfscope}%
\end{pgfscope}%
\begin{pgfscope}%
\pgfpathrectangle{\pgfqpoint{0.800000in}{0.528000in}}{\pgfqpoint{3.968000in}{3.696000in}}%
\pgfusepath{clip}%
\pgfsetbuttcap%
\pgfsetroundjoin%
\definecolor{currentfill}{rgb}{0.000000,0.849020,1.000000}%
\pgfsetfillcolor{currentfill}%
\pgfsetlinewidth{1.003750pt}%
\definecolor{currentstroke}{rgb}{0.000000,0.849020,1.000000}%
\pgfsetstrokecolor{currentstroke}%
\pgfsetdash{}{0pt}%
\pgfsys@defobject{currentmarker}{\pgfqpoint{infin}{infin}}{\pgfqpoint{-infin}{-infin}}{%
\pgfusepath{stroke,fill}%
}%
\begin{pgfscope}%
\pgfsys@transformshift{0.000000in}{0.000000in}%
\pgfsys@useobject{currentmarker}{}%
\end{pgfscope}%
\end{pgfscope}%
\begin{pgfscope}%
\pgfpathrectangle{\pgfqpoint{0.800000in}{0.528000in}}{\pgfqpoint{3.968000in}{3.696000in}}%
\pgfusepath{clip}%
\pgfsetbuttcap%
\pgfsetroundjoin%
\definecolor{currentfill}{rgb}{0.034788,0.943137,0.932954}%
\pgfsetfillcolor{currentfill}%
\pgfsetlinewidth{1.003750pt}%
\definecolor{currentstroke}{rgb}{0.034788,0.943137,0.932954}%
\pgfsetstrokecolor{currentstroke}%
\pgfsetdash{}{0pt}%
\pgfsys@defobject{currentmarker}{\pgfqpoint{infin}{infin}}{\pgfqpoint{-infin}{-infin}}{%
\pgfusepath{stroke,fill}%
}%
\begin{pgfscope}%
\pgfsys@transformshift{0.000000in}{0.000000in}%
\pgfsys@useobject{currentmarker}{}%
\end{pgfscope}%
\end{pgfscope}%
\begin{pgfscope}%
\pgfpathrectangle{\pgfqpoint{0.800000in}{0.528000in}}{\pgfqpoint{3.968000in}{3.696000in}}%
\pgfusepath{clip}%
\pgfsetbuttcap%
\pgfsetroundjoin%
\definecolor{currentfill}{rgb}{0.123340,1.000000,0.844402}%
\pgfsetfillcolor{currentfill}%
\pgfsetlinewidth{1.003750pt}%
\definecolor{currentstroke}{rgb}{0.123340,1.000000,0.844402}%
\pgfsetstrokecolor{currentstroke}%
\pgfsetdash{}{0pt}%
\pgfsys@defobject{currentmarker}{\pgfqpoint{infin}{infin}}{\pgfqpoint{-infin}{-infin}}{%
\pgfusepath{stroke,fill}%
}%
\begin{pgfscope}%
\pgfsys@transformshift{0.000000in}{0.000000in}%
\pgfsys@useobject{currentmarker}{}%
\end{pgfscope}%
\end{pgfscope}%
\begin{pgfscope}%
\pgfpathrectangle{\pgfqpoint{0.800000in}{0.528000in}}{\pgfqpoint{3.968000in}{3.696000in}}%
\pgfusepath{clip}%
\pgfsetbuttcap%
\pgfsetroundjoin%
\definecolor{currentfill}{rgb}{0.199241,1.000000,0.768501}%
\pgfsetfillcolor{currentfill}%
\pgfsetlinewidth{1.003750pt}%
\definecolor{currentstroke}{rgb}{0.199241,1.000000,0.768501}%
\pgfsetstrokecolor{currentstroke}%
\pgfsetdash{}{0pt}%
\pgfsys@defobject{currentmarker}{\pgfqpoint{infin}{infin}}{\pgfqpoint{-infin}{-infin}}{%
\pgfusepath{stroke,fill}%
}%
\begin{pgfscope}%
\pgfsys@transformshift{0.000000in}{0.000000in}%
\pgfsys@useobject{currentmarker}{}%
\end{pgfscope}%
\end{pgfscope}%
\begin{pgfscope}%
\pgfpathrectangle{\pgfqpoint{0.800000in}{0.528000in}}{\pgfqpoint{3.968000in}{3.696000in}}%
\pgfusepath{clip}%
\pgfsetbuttcap%
\pgfsetroundjoin%
\definecolor{currentfill}{rgb}{0.287793,1.000000,0.679949}%
\pgfsetfillcolor{currentfill}%
\pgfsetlinewidth{1.003750pt}%
\definecolor{currentstroke}{rgb}{0.287793,1.000000,0.679949}%
\pgfsetstrokecolor{currentstroke}%
\pgfsetdash{}{0pt}%
\pgfsys@defobject{currentmarker}{\pgfqpoint{infin}{infin}}{\pgfqpoint{-infin}{-infin}}{%
\pgfusepath{stroke,fill}%
}%
\begin{pgfscope}%
\pgfsys@transformshift{0.000000in}{0.000000in}%
\pgfsys@useobject{currentmarker}{}%
\end{pgfscope}%
\end{pgfscope}%
\begin{pgfscope}%
\pgfpathrectangle{\pgfqpoint{0.800000in}{0.528000in}}{\pgfqpoint{3.968000in}{3.696000in}}%
\pgfusepath{clip}%
\pgfsetbuttcap%
\pgfsetroundjoin%
\definecolor{currentfill}{rgb}{0.363694,1.000000,0.604048}%
\pgfsetfillcolor{currentfill}%
\pgfsetlinewidth{1.003750pt}%
\definecolor{currentstroke}{rgb}{0.363694,1.000000,0.604048}%
\pgfsetstrokecolor{currentstroke}%
\pgfsetdash{}{0pt}%
\pgfsys@defobject{currentmarker}{\pgfqpoint{infin}{infin}}{\pgfqpoint{-infin}{-infin}}{%
\pgfusepath{stroke,fill}%
}%
\begin{pgfscope}%
\pgfsys@transformshift{0.000000in}{0.000000in}%
\pgfsys@useobject{currentmarker}{}%
\end{pgfscope}%
\end{pgfscope}%
\begin{pgfscope}%
\pgfpathrectangle{\pgfqpoint{0.800000in}{0.528000in}}{\pgfqpoint{3.968000in}{3.696000in}}%
\pgfusepath{clip}%
\pgfsetbuttcap%
\pgfsetroundjoin%
\definecolor{currentfill}{rgb}{0.439595,1.000000,0.528147}%
\pgfsetfillcolor{currentfill}%
\pgfsetlinewidth{1.003750pt}%
\definecolor{currentstroke}{rgb}{0.439595,1.000000,0.528147}%
\pgfsetstrokecolor{currentstroke}%
\pgfsetdash{}{0pt}%
\pgfsys@defobject{currentmarker}{\pgfqpoint{infin}{infin}}{\pgfqpoint{-infin}{-infin}}{%
\pgfusepath{stroke,fill}%
}%
\begin{pgfscope}%
\pgfsys@transformshift{0.000000in}{0.000000in}%
\pgfsys@useobject{currentmarker}{}%
\end{pgfscope}%
\end{pgfscope}%
\begin{pgfscope}%
\pgfpathrectangle{\pgfqpoint{0.800000in}{0.528000in}}{\pgfqpoint{3.968000in}{3.696000in}}%
\pgfusepath{clip}%
\pgfsetbuttcap%
\pgfsetroundjoin%
\definecolor{currentfill}{rgb}{0.528147,1.000000,0.439595}%
\pgfsetfillcolor{currentfill}%
\pgfsetlinewidth{1.003750pt}%
\definecolor{currentstroke}{rgb}{0.528147,1.000000,0.439595}%
\pgfsetstrokecolor{currentstroke}%
\pgfsetdash{}{0pt}%
\pgfsys@defobject{currentmarker}{\pgfqpoint{infin}{infin}}{\pgfqpoint{-infin}{-infin}}{%
\pgfusepath{stroke,fill}%
}%
\begin{pgfscope}%
\pgfsys@transformshift{0.000000in}{0.000000in}%
\pgfsys@useobject{currentmarker}{}%
\end{pgfscope}%
\end{pgfscope}%
\begin{pgfscope}%
\pgfpathrectangle{\pgfqpoint{0.800000in}{0.528000in}}{\pgfqpoint{3.968000in}{3.696000in}}%
\pgfusepath{clip}%
\pgfsetbuttcap%
\pgfsetroundjoin%
\definecolor{currentfill}{rgb}{0.604048,1.000000,0.363694}%
\pgfsetfillcolor{currentfill}%
\pgfsetlinewidth{1.003750pt}%
\definecolor{currentstroke}{rgb}{0.604048,1.000000,0.363694}%
\pgfsetstrokecolor{currentstroke}%
\pgfsetdash{}{0pt}%
\pgfsys@defobject{currentmarker}{\pgfqpoint{infin}{infin}}{\pgfqpoint{-infin}{-infin}}{%
\pgfusepath{stroke,fill}%
}%
\begin{pgfscope}%
\pgfsys@transformshift{0.000000in}{0.000000in}%
\pgfsys@useobject{currentmarker}{}%
\end{pgfscope}%
\end{pgfscope}%
\begin{pgfscope}%
\pgfpathrectangle{\pgfqpoint{0.800000in}{0.528000in}}{\pgfqpoint{3.968000in}{3.696000in}}%
\pgfusepath{clip}%
\pgfsetbuttcap%
\pgfsetroundjoin%
\definecolor{currentfill}{rgb}{0.692600,1.000000,0.275142}%
\pgfsetfillcolor{currentfill}%
\pgfsetlinewidth{1.003750pt}%
\definecolor{currentstroke}{rgb}{0.692600,1.000000,0.275142}%
\pgfsetstrokecolor{currentstroke}%
\pgfsetdash{}{0pt}%
\pgfsys@defobject{currentmarker}{\pgfqpoint{infin}{infin}}{\pgfqpoint{-infin}{-infin}}{%
\pgfusepath{stroke,fill}%
}%
\begin{pgfscope}%
\pgfsys@transformshift{0.000000in}{0.000000in}%
\pgfsys@useobject{currentmarker}{}%
\end{pgfscope}%
\end{pgfscope}%
\begin{pgfscope}%
\pgfpathrectangle{\pgfqpoint{0.800000in}{0.528000in}}{\pgfqpoint{3.968000in}{3.696000in}}%
\pgfusepath{clip}%
\pgfsetbuttcap%
\pgfsetroundjoin%
\definecolor{currentfill}{rgb}{0.768501,1.000000,0.199241}%
\pgfsetfillcolor{currentfill}%
\pgfsetlinewidth{1.003750pt}%
\definecolor{currentstroke}{rgb}{0.768501,1.000000,0.199241}%
\pgfsetstrokecolor{currentstroke}%
\pgfsetdash{}{0pt}%
\pgfsys@defobject{currentmarker}{\pgfqpoint{infin}{infin}}{\pgfqpoint{-infin}{-infin}}{%
\pgfusepath{stroke,fill}%
}%
\begin{pgfscope}%
\pgfsys@transformshift{0.000000in}{0.000000in}%
\pgfsys@useobject{currentmarker}{}%
\end{pgfscope}%
\end{pgfscope}%
\begin{pgfscope}%
\pgfpathrectangle{\pgfqpoint{0.800000in}{0.528000in}}{\pgfqpoint{3.968000in}{3.696000in}}%
\pgfusepath{clip}%
\pgfsetbuttcap%
\pgfsetroundjoin%
\definecolor{currentfill}{rgb}{0.844402,1.000000,0.123340}%
\pgfsetfillcolor{currentfill}%
\pgfsetlinewidth{1.003750pt}%
\definecolor{currentstroke}{rgb}{0.844402,1.000000,0.123340}%
\pgfsetstrokecolor{currentstroke}%
\pgfsetdash{}{0pt}%
\pgfsys@defobject{currentmarker}{\pgfqpoint{infin}{infin}}{\pgfqpoint{-infin}{-infin}}{%
\pgfusepath{stroke,fill}%
}%
\begin{pgfscope}%
\pgfsys@transformshift{0.000000in}{0.000000in}%
\pgfsys@useobject{currentmarker}{}%
\end{pgfscope}%
\end{pgfscope}%
\begin{pgfscope}%
\pgfpathrectangle{\pgfqpoint{0.800000in}{0.528000in}}{\pgfqpoint{3.968000in}{3.696000in}}%
\pgfusepath{clip}%
\pgfsetbuttcap%
\pgfsetroundjoin%
\definecolor{currentfill}{rgb}{0.932954,1.000000,0.034788}%
\pgfsetfillcolor{currentfill}%
\pgfsetlinewidth{1.003750pt}%
\definecolor{currentstroke}{rgb}{0.932954,1.000000,0.034788}%
\pgfsetstrokecolor{currentstroke}%
\pgfsetdash{}{0pt}%
\pgfsys@defobject{currentmarker}{\pgfqpoint{infin}{infin}}{\pgfqpoint{-infin}{-infin}}{%
\pgfusepath{stroke,fill}%
}%
\begin{pgfscope}%
\pgfsys@transformshift{0.000000in}{0.000000in}%
\pgfsys@useobject{currentmarker}{}%
\end{pgfscope}%
\end{pgfscope}%
\begin{pgfscope}%
\pgfpathrectangle{\pgfqpoint{0.800000in}{0.528000in}}{\pgfqpoint{3.968000in}{3.696000in}}%
\pgfusepath{clip}%
\pgfsetbuttcap%
\pgfsetroundjoin%
\definecolor{currentfill}{rgb}{1.000000,0.915759,0.000000}%
\pgfsetfillcolor{currentfill}%
\pgfsetlinewidth{1.003750pt}%
\definecolor{currentstroke}{rgb}{1.000000,0.915759,0.000000}%
\pgfsetstrokecolor{currentstroke}%
\pgfsetdash{}{0pt}%
\pgfsys@defobject{currentmarker}{\pgfqpoint{infin}{infin}}{\pgfqpoint{-infin}{-infin}}{%
\pgfusepath{stroke,fill}%
}%
\begin{pgfscope}%
\pgfsys@transformshift{0.000000in}{0.000000in}%
\pgfsys@useobject{currentmarker}{}%
\end{pgfscope}%
\end{pgfscope}%
\begin{pgfscope}%
\pgfpathrectangle{\pgfqpoint{0.800000in}{0.528000in}}{\pgfqpoint{3.968000in}{3.696000in}}%
\pgfusepath{clip}%
\pgfsetbuttcap%
\pgfsetroundjoin%
\definecolor{currentfill}{rgb}{1.000000,0.814089,0.000000}%
\pgfsetfillcolor{currentfill}%
\pgfsetlinewidth{1.003750pt}%
\definecolor{currentstroke}{rgb}{1.000000,0.814089,0.000000}%
\pgfsetstrokecolor{currentstroke}%
\pgfsetdash{}{0pt}%
\pgfsys@defobject{currentmarker}{\pgfqpoint{infin}{infin}}{\pgfqpoint{-infin}{-infin}}{%
\pgfusepath{stroke,fill}%
}%
\begin{pgfscope}%
\pgfsys@transformshift{0.000000in}{0.000000in}%
\pgfsys@useobject{currentmarker}{}%
\end{pgfscope}%
\end{pgfscope}%
\begin{pgfscope}%
\pgfpathrectangle{\pgfqpoint{0.800000in}{0.528000in}}{\pgfqpoint{3.968000in}{3.696000in}}%
\pgfusepath{clip}%
\pgfsetbuttcap%
\pgfsetroundjoin%
\definecolor{currentfill}{rgb}{1.000000,0.726943,0.000000}%
\pgfsetfillcolor{currentfill}%
\pgfsetlinewidth{1.003750pt}%
\definecolor{currentstroke}{rgb}{1.000000,0.726943,0.000000}%
\pgfsetstrokecolor{currentstroke}%
\pgfsetdash{}{0pt}%
\pgfsys@defobject{currentmarker}{\pgfqpoint{infin}{infin}}{\pgfqpoint{-infin}{-infin}}{%
\pgfusepath{stroke,fill}%
}%
\begin{pgfscope}%
\pgfsys@transformshift{0.000000in}{0.000000in}%
\pgfsys@useobject{currentmarker}{}%
\end{pgfscope}%
\end{pgfscope}%
\begin{pgfscope}%
\pgfpathrectangle{\pgfqpoint{0.800000in}{0.528000in}}{\pgfqpoint{3.968000in}{3.696000in}}%
\pgfusepath{clip}%
\pgfsetbuttcap%
\pgfsetroundjoin%
\definecolor{currentfill}{rgb}{1.000000,0.639797,0.000000}%
\pgfsetfillcolor{currentfill}%
\pgfsetlinewidth{1.003750pt}%
\definecolor{currentstroke}{rgb}{1.000000,0.639797,0.000000}%
\pgfsetstrokecolor{currentstroke}%
\pgfsetdash{}{0pt}%
\pgfsys@defobject{currentmarker}{\pgfqpoint{infin}{infin}}{\pgfqpoint{-infin}{-infin}}{%
\pgfusepath{stroke,fill}%
}%
\begin{pgfscope}%
\pgfsys@transformshift{0.000000in}{0.000000in}%
\pgfsys@useobject{currentmarker}{}%
\end{pgfscope}%
\end{pgfscope}%
\begin{pgfscope}%
\pgfpathrectangle{\pgfqpoint{0.800000in}{0.528000in}}{\pgfqpoint{3.968000in}{3.696000in}}%
\pgfusepath{clip}%
\pgfsetbuttcap%
\pgfsetroundjoin%
\definecolor{currentfill}{rgb}{1.000000,0.538126,0.000000}%
\pgfsetfillcolor{currentfill}%
\pgfsetlinewidth{1.003750pt}%
\definecolor{currentstroke}{rgb}{1.000000,0.538126,0.000000}%
\pgfsetstrokecolor{currentstroke}%
\pgfsetdash{}{0pt}%
\pgfsys@defobject{currentmarker}{\pgfqpoint{infin}{infin}}{\pgfqpoint{-infin}{-infin}}{%
\pgfusepath{stroke,fill}%
}%
\begin{pgfscope}%
\pgfsys@transformshift{0.000000in}{0.000000in}%
\pgfsys@useobject{currentmarker}{}%
\end{pgfscope}%
\end{pgfscope}%
\begin{pgfscope}%
\pgfpathrectangle{\pgfqpoint{0.800000in}{0.528000in}}{\pgfqpoint{3.968000in}{3.696000in}}%
\pgfusepath{clip}%
\pgfsetbuttcap%
\pgfsetroundjoin%
\definecolor{currentfill}{rgb}{1.000000,0.450980,0.000000}%
\pgfsetfillcolor{currentfill}%
\pgfsetlinewidth{1.003750pt}%
\definecolor{currentstroke}{rgb}{1.000000,0.450980,0.000000}%
\pgfsetstrokecolor{currentstroke}%
\pgfsetdash{}{0pt}%
\pgfsys@defobject{currentmarker}{\pgfqpoint{infin}{infin}}{\pgfqpoint{-infin}{-infin}}{%
\pgfusepath{stroke,fill}%
}%
\begin{pgfscope}%
\pgfsys@transformshift{0.000000in}{0.000000in}%
\pgfsys@useobject{currentmarker}{}%
\end{pgfscope}%
\end{pgfscope}%
\begin{pgfscope}%
\pgfpathrectangle{\pgfqpoint{0.800000in}{0.528000in}}{\pgfqpoint{3.968000in}{3.696000in}}%
\pgfusepath{clip}%
\pgfsetbuttcap%
\pgfsetroundjoin%
\definecolor{currentfill}{rgb}{1.000000,0.349310,0.000000}%
\pgfsetfillcolor{currentfill}%
\pgfsetlinewidth{1.003750pt}%
\definecolor{currentstroke}{rgb}{1.000000,0.349310,0.000000}%
\pgfsetstrokecolor{currentstroke}%
\pgfsetdash{}{0pt}%
\pgfsys@defobject{currentmarker}{\pgfqpoint{infin}{infin}}{\pgfqpoint{-infin}{-infin}}{%
\pgfusepath{stroke,fill}%
}%
\begin{pgfscope}%
\pgfsys@transformshift{0.000000in}{0.000000in}%
\pgfsys@useobject{currentmarker}{}%
\end{pgfscope}%
\end{pgfscope}%
\begin{pgfscope}%
\pgfpathrectangle{\pgfqpoint{0.800000in}{0.528000in}}{\pgfqpoint{3.968000in}{3.696000in}}%
\pgfusepath{clip}%
\pgfsetbuttcap%
\pgfsetroundjoin%
\definecolor{currentfill}{rgb}{1.000000,0.262164,0.000000}%
\pgfsetfillcolor{currentfill}%
\pgfsetlinewidth{1.003750pt}%
\definecolor{currentstroke}{rgb}{1.000000,0.262164,0.000000}%
\pgfsetstrokecolor{currentstroke}%
\pgfsetdash{}{0pt}%
\pgfsys@defobject{currentmarker}{\pgfqpoint{infin}{infin}}{\pgfqpoint{-infin}{-infin}}{%
\pgfusepath{stroke,fill}%
}%
\begin{pgfscope}%
\pgfsys@transformshift{0.000000in}{0.000000in}%
\pgfsys@useobject{currentmarker}{}%
\end{pgfscope}%
\end{pgfscope}%
\begin{pgfscope}%
\pgfpathrectangle{\pgfqpoint{0.800000in}{0.528000in}}{\pgfqpoint{3.968000in}{3.696000in}}%
\pgfusepath{clip}%
\pgfsetbuttcap%
\pgfsetroundjoin%
\definecolor{currentfill}{rgb}{1.000000,0.175018,0.000000}%
\pgfsetfillcolor{currentfill}%
\pgfsetlinewidth{1.003750pt}%
\definecolor{currentstroke}{rgb}{1.000000,0.175018,0.000000}%
\pgfsetstrokecolor{currentstroke}%
\pgfsetdash{}{0pt}%
\pgfsys@defobject{currentmarker}{\pgfqpoint{infin}{infin}}{\pgfqpoint{-infin}{-infin}}{%
\pgfusepath{stroke,fill}%
}%
\begin{pgfscope}%
\pgfsys@transformshift{0.000000in}{0.000000in}%
\pgfsys@useobject{currentmarker}{}%
\end{pgfscope}%
\end{pgfscope}%
\begin{pgfscope}%
\pgfpathrectangle{\pgfqpoint{0.800000in}{0.528000in}}{\pgfqpoint{3.968000in}{3.696000in}}%
\pgfusepath{clip}%
\pgfsetbuttcap%
\pgfsetroundjoin%
\definecolor{currentfill}{rgb}{0.999109,0.073348,0.000000}%
\pgfsetfillcolor{currentfill}%
\pgfsetlinewidth{1.003750pt}%
\definecolor{currentstroke}{rgb}{0.999109,0.073348,0.000000}%
\pgfsetstrokecolor{currentstroke}%
\pgfsetdash{}{0pt}%
\pgfsys@defobject{currentmarker}{\pgfqpoint{infin}{infin}}{\pgfqpoint{-infin}{-infin}}{%
\pgfusepath{stroke,fill}%
}%
\begin{pgfscope}%
\pgfsys@transformshift{0.000000in}{0.000000in}%
\pgfsys@useobject{currentmarker}{}%
\end{pgfscope}%
\end{pgfscope}%
\begin{pgfscope}%
\pgfpathrectangle{\pgfqpoint{0.800000in}{0.528000in}}{\pgfqpoint{3.968000in}{3.696000in}}%
\pgfusepath{clip}%
\pgfsetbuttcap%
\pgfsetroundjoin%
\definecolor{currentfill}{rgb}{0.892157,0.000000,0.000000}%
\pgfsetfillcolor{currentfill}%
\pgfsetlinewidth{1.003750pt}%
\definecolor{currentstroke}{rgb}{0.892157,0.000000,0.000000}%
\pgfsetstrokecolor{currentstroke}%
\pgfsetdash{}{0pt}%
\pgfsys@defobject{currentmarker}{\pgfqpoint{infin}{infin}}{\pgfqpoint{-infin}{-infin}}{%
\pgfusepath{stroke,fill}%
}%
\begin{pgfscope}%
\pgfsys@transformshift{0.000000in}{0.000000in}%
\pgfsys@useobject{currentmarker}{}%
\end{pgfscope}%
\end{pgfscope}%
\begin{pgfscope}%
\pgfpathrectangle{\pgfqpoint{0.800000in}{0.528000in}}{\pgfqpoint{3.968000in}{3.696000in}}%
\pgfusepath{clip}%
\pgfsetbuttcap%
\pgfsetroundjoin%
\definecolor{currentfill}{rgb}{0.767380,0.000000,0.000000}%
\pgfsetfillcolor{currentfill}%
\pgfsetlinewidth{1.003750pt}%
\definecolor{currentstroke}{rgb}{0.767380,0.000000,0.000000}%
\pgfsetstrokecolor{currentstroke}%
\pgfsetdash{}{0pt}%
\pgfsys@defobject{currentmarker}{\pgfqpoint{infin}{infin}}{\pgfqpoint{-infin}{-infin}}{%
\pgfusepath{stroke,fill}%
}%
\begin{pgfscope}%
\pgfsys@transformshift{0.000000in}{0.000000in}%
\pgfsys@useobject{currentmarker}{}%
\end{pgfscope}%
\end{pgfscope}%
\begin{pgfscope}%
\pgfpathrectangle{\pgfqpoint{0.800000in}{0.528000in}}{\pgfqpoint{3.968000in}{3.696000in}}%
\pgfusepath{clip}%
\pgfsetbuttcap%
\pgfsetroundjoin%
\definecolor{currentfill}{rgb}{0.660428,0.000000,0.000000}%
\pgfsetfillcolor{currentfill}%
\pgfsetlinewidth{1.003750pt}%
\definecolor{currentstroke}{rgb}{0.660428,0.000000,0.000000}%
\pgfsetstrokecolor{currentstroke}%
\pgfsetdash{}{0pt}%
\pgfsys@defobject{currentmarker}{\pgfqpoint{infin}{infin}}{\pgfqpoint{-infin}{-infin}}{%
\pgfusepath{stroke,fill}%
}%
\begin{pgfscope}%
\pgfsys@transformshift{0.000000in}{0.000000in}%
\pgfsys@useobject{currentmarker}{}%
\end{pgfscope}%
\end{pgfscope}%
\begin{pgfscope}%
\pgfpathrectangle{\pgfqpoint{0.800000in}{0.528000in}}{\pgfqpoint{3.968000in}{3.696000in}}%
\pgfusepath{clip}%
\pgfsetbuttcap%
\pgfsetroundjoin%
\definecolor{currentfill}{rgb}{0.553476,0.000000,0.000000}%
\pgfsetfillcolor{currentfill}%
\pgfsetlinewidth{1.003750pt}%
\definecolor{currentstroke}{rgb}{0.553476,0.000000,0.000000}%
\pgfsetstrokecolor{currentstroke}%
\pgfsetdash{}{0pt}%
\pgfsys@defobject{currentmarker}{\pgfqpoint{infin}{infin}}{\pgfqpoint{-infin}{-infin}}{%
\pgfusepath{stroke,fill}%
}%
\begin{pgfscope}%
\pgfsys@transformshift{0.000000in}{0.000000in}%
\pgfsys@useobject{currentmarker}{}%
\end{pgfscope}%
\end{pgfscope}%
\begin{pgfscope}%
\pgfsetbuttcap%
\pgfsetroundjoin%
\definecolor{currentfill}{rgb}{0.000000,0.000000,0.000000}%
\pgfsetfillcolor{currentfill}%
\pgfsetlinewidth{0.803000pt}%
\definecolor{currentstroke}{rgb}{0.000000,0.000000,0.000000}%
\pgfsetstrokecolor{currentstroke}%
\pgfsetdash{}{0pt}%
\pgfsys@defobject{currentmarker}{\pgfqpoint{0.000000in}{-0.048611in}}{\pgfqpoint{0.000000in}{0.000000in}}{%
\pgfpathmoveto{\pgfqpoint{0.000000in}{0.000000in}}%
\pgfpathlineto{\pgfqpoint{0.000000in}{-0.048611in}}%
\pgfusepath{stroke,fill}%
}%
\begin{pgfscope}%
\pgfsys@transformshift{0.800000in}{0.528000in}%
\pgfsys@useobject{currentmarker}{}%
\end{pgfscope}%
\end{pgfscope}%
\begin{pgfscope}%
\definecolor{textcolor}{rgb}{0.000000,0.000000,0.000000}%
\pgfsetstrokecolor{textcolor}%
\pgfsetfillcolor{textcolor}%
\pgftext[x=0.800000in,y=0.430778in,,top]{\color{textcolor}\rmfamily\fontsize{10.000000}{12.000000}\selectfont \(\displaystyle 0.500\)}%
\end{pgfscope}%
\begin{pgfscope}%
\pgfsetbuttcap%
\pgfsetroundjoin%
\definecolor{currentfill}{rgb}{0.000000,0.000000,0.000000}%
\pgfsetfillcolor{currentfill}%
\pgfsetlinewidth{0.803000pt}%
\definecolor{currentstroke}{rgb}{0.000000,0.000000,0.000000}%
\pgfsetstrokecolor{currentstroke}%
\pgfsetdash{}{0pt}%
\pgfsys@defobject{currentmarker}{\pgfqpoint{0.000000in}{-0.048611in}}{\pgfqpoint{0.000000in}{0.000000in}}{%
\pgfpathmoveto{\pgfqpoint{0.000000in}{0.000000in}}%
\pgfpathlineto{\pgfqpoint{0.000000in}{-0.048611in}}%
\pgfusepath{stroke,fill}%
}%
\begin{pgfscope}%
\pgfsys@transformshift{1.296000in}{0.528000in}%
\pgfsys@useobject{currentmarker}{}%
\end{pgfscope}%
\end{pgfscope}%
\begin{pgfscope}%
\definecolor{textcolor}{rgb}{0.000000,0.000000,0.000000}%
\pgfsetstrokecolor{textcolor}%
\pgfsetfillcolor{textcolor}%
\pgftext[x=1.296000in,y=0.430778in,,top]{\color{textcolor}\rmfamily\fontsize{10.000000}{12.000000}\selectfont \(\displaystyle 0.525\)}%
\end{pgfscope}%
\begin{pgfscope}%
\pgfsetbuttcap%
\pgfsetroundjoin%
\definecolor{currentfill}{rgb}{0.000000,0.000000,0.000000}%
\pgfsetfillcolor{currentfill}%
\pgfsetlinewidth{0.803000pt}%
\definecolor{currentstroke}{rgb}{0.000000,0.000000,0.000000}%
\pgfsetstrokecolor{currentstroke}%
\pgfsetdash{}{0pt}%
\pgfsys@defobject{currentmarker}{\pgfqpoint{0.000000in}{-0.048611in}}{\pgfqpoint{0.000000in}{0.000000in}}{%
\pgfpathmoveto{\pgfqpoint{0.000000in}{0.000000in}}%
\pgfpathlineto{\pgfqpoint{0.000000in}{-0.048611in}}%
\pgfusepath{stroke,fill}%
}%
\begin{pgfscope}%
\pgfsys@transformshift{1.792000in}{0.528000in}%
\pgfsys@useobject{currentmarker}{}%
\end{pgfscope}%
\end{pgfscope}%
\begin{pgfscope}%
\definecolor{textcolor}{rgb}{0.000000,0.000000,0.000000}%
\pgfsetstrokecolor{textcolor}%
\pgfsetfillcolor{textcolor}%
\pgftext[x=1.792000in,y=0.430778in,,top]{\color{textcolor}\rmfamily\fontsize{10.000000}{12.000000}\selectfont \(\displaystyle 0.550\)}%
\end{pgfscope}%
\begin{pgfscope}%
\pgfsetbuttcap%
\pgfsetroundjoin%
\definecolor{currentfill}{rgb}{0.000000,0.000000,0.000000}%
\pgfsetfillcolor{currentfill}%
\pgfsetlinewidth{0.803000pt}%
\definecolor{currentstroke}{rgb}{0.000000,0.000000,0.000000}%
\pgfsetstrokecolor{currentstroke}%
\pgfsetdash{}{0pt}%
\pgfsys@defobject{currentmarker}{\pgfqpoint{0.000000in}{-0.048611in}}{\pgfqpoint{0.000000in}{0.000000in}}{%
\pgfpathmoveto{\pgfqpoint{0.000000in}{0.000000in}}%
\pgfpathlineto{\pgfqpoint{0.000000in}{-0.048611in}}%
\pgfusepath{stroke,fill}%
}%
\begin{pgfscope}%
\pgfsys@transformshift{2.288000in}{0.528000in}%
\pgfsys@useobject{currentmarker}{}%
\end{pgfscope}%
\end{pgfscope}%
\begin{pgfscope}%
\definecolor{textcolor}{rgb}{0.000000,0.000000,0.000000}%
\pgfsetstrokecolor{textcolor}%
\pgfsetfillcolor{textcolor}%
\pgftext[x=2.288000in,y=0.430778in,,top]{\color{textcolor}\rmfamily\fontsize{10.000000}{12.000000}\selectfont \(\displaystyle 0.575\)}%
\end{pgfscope}%
\begin{pgfscope}%
\pgfsetbuttcap%
\pgfsetroundjoin%
\definecolor{currentfill}{rgb}{0.000000,0.000000,0.000000}%
\pgfsetfillcolor{currentfill}%
\pgfsetlinewidth{0.803000pt}%
\definecolor{currentstroke}{rgb}{0.000000,0.000000,0.000000}%
\pgfsetstrokecolor{currentstroke}%
\pgfsetdash{}{0pt}%
\pgfsys@defobject{currentmarker}{\pgfqpoint{0.000000in}{-0.048611in}}{\pgfqpoint{0.000000in}{0.000000in}}{%
\pgfpathmoveto{\pgfqpoint{0.000000in}{0.000000in}}%
\pgfpathlineto{\pgfqpoint{0.000000in}{-0.048611in}}%
\pgfusepath{stroke,fill}%
}%
\begin{pgfscope}%
\pgfsys@transformshift{2.784000in}{0.528000in}%
\pgfsys@useobject{currentmarker}{}%
\end{pgfscope}%
\end{pgfscope}%
\begin{pgfscope}%
\definecolor{textcolor}{rgb}{0.000000,0.000000,0.000000}%
\pgfsetstrokecolor{textcolor}%
\pgfsetfillcolor{textcolor}%
\pgftext[x=2.784000in,y=0.430778in,,top]{\color{textcolor}\rmfamily\fontsize{10.000000}{12.000000}\selectfont \(\displaystyle 0.600\)}%
\end{pgfscope}%
\begin{pgfscope}%
\pgfsetbuttcap%
\pgfsetroundjoin%
\definecolor{currentfill}{rgb}{0.000000,0.000000,0.000000}%
\pgfsetfillcolor{currentfill}%
\pgfsetlinewidth{0.803000pt}%
\definecolor{currentstroke}{rgb}{0.000000,0.000000,0.000000}%
\pgfsetstrokecolor{currentstroke}%
\pgfsetdash{}{0pt}%
\pgfsys@defobject{currentmarker}{\pgfqpoint{0.000000in}{-0.048611in}}{\pgfqpoint{0.000000in}{0.000000in}}{%
\pgfpathmoveto{\pgfqpoint{0.000000in}{0.000000in}}%
\pgfpathlineto{\pgfqpoint{0.000000in}{-0.048611in}}%
\pgfusepath{stroke,fill}%
}%
\begin{pgfscope}%
\pgfsys@transformshift{3.280000in}{0.528000in}%
\pgfsys@useobject{currentmarker}{}%
\end{pgfscope}%
\end{pgfscope}%
\begin{pgfscope}%
\definecolor{textcolor}{rgb}{0.000000,0.000000,0.000000}%
\pgfsetstrokecolor{textcolor}%
\pgfsetfillcolor{textcolor}%
\pgftext[x=3.280000in,y=0.430778in,,top]{\color{textcolor}\rmfamily\fontsize{10.000000}{12.000000}\selectfont \(\displaystyle 0.625\)}%
\end{pgfscope}%
\begin{pgfscope}%
\pgfsetbuttcap%
\pgfsetroundjoin%
\definecolor{currentfill}{rgb}{0.000000,0.000000,0.000000}%
\pgfsetfillcolor{currentfill}%
\pgfsetlinewidth{0.803000pt}%
\definecolor{currentstroke}{rgb}{0.000000,0.000000,0.000000}%
\pgfsetstrokecolor{currentstroke}%
\pgfsetdash{}{0pt}%
\pgfsys@defobject{currentmarker}{\pgfqpoint{0.000000in}{-0.048611in}}{\pgfqpoint{0.000000in}{0.000000in}}{%
\pgfpathmoveto{\pgfqpoint{0.000000in}{0.000000in}}%
\pgfpathlineto{\pgfqpoint{0.000000in}{-0.048611in}}%
\pgfusepath{stroke,fill}%
}%
\begin{pgfscope}%
\pgfsys@transformshift{3.776000in}{0.528000in}%
\pgfsys@useobject{currentmarker}{}%
\end{pgfscope}%
\end{pgfscope}%
\begin{pgfscope}%
\definecolor{textcolor}{rgb}{0.000000,0.000000,0.000000}%
\pgfsetstrokecolor{textcolor}%
\pgfsetfillcolor{textcolor}%
\pgftext[x=3.776000in,y=0.430778in,,top]{\color{textcolor}\rmfamily\fontsize{10.000000}{12.000000}\selectfont \(\displaystyle 0.650\)}%
\end{pgfscope}%
\begin{pgfscope}%
\pgfsetbuttcap%
\pgfsetroundjoin%
\definecolor{currentfill}{rgb}{0.000000,0.000000,0.000000}%
\pgfsetfillcolor{currentfill}%
\pgfsetlinewidth{0.803000pt}%
\definecolor{currentstroke}{rgb}{0.000000,0.000000,0.000000}%
\pgfsetstrokecolor{currentstroke}%
\pgfsetdash{}{0pt}%
\pgfsys@defobject{currentmarker}{\pgfqpoint{0.000000in}{-0.048611in}}{\pgfqpoint{0.000000in}{0.000000in}}{%
\pgfpathmoveto{\pgfqpoint{0.000000in}{0.000000in}}%
\pgfpathlineto{\pgfqpoint{0.000000in}{-0.048611in}}%
\pgfusepath{stroke,fill}%
}%
\begin{pgfscope}%
\pgfsys@transformshift{4.272000in}{0.528000in}%
\pgfsys@useobject{currentmarker}{}%
\end{pgfscope}%
\end{pgfscope}%
\begin{pgfscope}%
\definecolor{textcolor}{rgb}{0.000000,0.000000,0.000000}%
\pgfsetstrokecolor{textcolor}%
\pgfsetfillcolor{textcolor}%
\pgftext[x=4.272000in,y=0.430778in,,top]{\color{textcolor}\rmfamily\fontsize{10.000000}{12.000000}\selectfont \(\displaystyle 0.675\)}%
\end{pgfscope}%
\begin{pgfscope}%
\pgfsetbuttcap%
\pgfsetroundjoin%
\definecolor{currentfill}{rgb}{0.000000,0.000000,0.000000}%
\pgfsetfillcolor{currentfill}%
\pgfsetlinewidth{0.803000pt}%
\definecolor{currentstroke}{rgb}{0.000000,0.000000,0.000000}%
\pgfsetstrokecolor{currentstroke}%
\pgfsetdash{}{0pt}%
\pgfsys@defobject{currentmarker}{\pgfqpoint{0.000000in}{-0.048611in}}{\pgfqpoint{0.000000in}{0.000000in}}{%
\pgfpathmoveto{\pgfqpoint{0.000000in}{0.000000in}}%
\pgfpathlineto{\pgfqpoint{0.000000in}{-0.048611in}}%
\pgfusepath{stroke,fill}%
}%
\begin{pgfscope}%
\pgfsys@transformshift{4.768000in}{0.528000in}%
\pgfsys@useobject{currentmarker}{}%
\end{pgfscope}%
\end{pgfscope}%
\begin{pgfscope}%
\definecolor{textcolor}{rgb}{0.000000,0.000000,0.000000}%
\pgfsetstrokecolor{textcolor}%
\pgfsetfillcolor{textcolor}%
\pgftext[x=4.768000in,y=0.430778in,,top]{\color{textcolor}\rmfamily\fontsize{10.000000}{12.000000}\selectfont \(\displaystyle 0.700\)}%
\end{pgfscope}%
\begin{pgfscope}%
\definecolor{textcolor}{rgb}{0.000000,0.000000,0.000000}%
\pgfsetstrokecolor{textcolor}%
\pgfsetfillcolor{textcolor}%
\pgftext[x=2.784000in,y=0.251766in,,top]{\color{textcolor}\rmfamily\fontsize{10.000000}{12.000000}\selectfont m}%
\end{pgfscope}%
\begin{pgfscope}%
\pgfsetbuttcap%
\pgfsetroundjoin%
\definecolor{currentfill}{rgb}{0.000000,0.000000,0.000000}%
\pgfsetfillcolor{currentfill}%
\pgfsetlinewidth{0.803000pt}%
\definecolor{currentstroke}{rgb}{0.000000,0.000000,0.000000}%
\pgfsetstrokecolor{currentstroke}%
\pgfsetdash{}{0pt}%
\pgfsys@defobject{currentmarker}{\pgfqpoint{-0.048611in}{0.000000in}}{\pgfqpoint{0.000000in}{0.000000in}}{%
\pgfpathmoveto{\pgfqpoint{0.000000in}{0.000000in}}%
\pgfpathlineto{\pgfqpoint{-0.048611in}{0.000000in}}%
\pgfusepath{stroke,fill}%
}%
\begin{pgfscope}%
\pgfsys@transformshift{0.800000in}{0.836000in}%
\pgfsys@useobject{currentmarker}{}%
\end{pgfscope}%
\end{pgfscope}%
\begin{pgfscope}%
\definecolor{textcolor}{rgb}{0.000000,0.000000,0.000000}%
\pgfsetstrokecolor{textcolor}%
\pgfsetfillcolor{textcolor}%
\pgftext[x=0.525308in,y=0.787775in,left,base]{\color{textcolor}\rmfamily\fontsize{10.000000}{12.000000}\selectfont \(\displaystyle 0.6\)}%
\end{pgfscope}%
\begin{pgfscope}%
\pgfsetbuttcap%
\pgfsetroundjoin%
\definecolor{currentfill}{rgb}{0.000000,0.000000,0.000000}%
\pgfsetfillcolor{currentfill}%
\pgfsetlinewidth{0.803000pt}%
\definecolor{currentstroke}{rgb}{0.000000,0.000000,0.000000}%
\pgfsetstrokecolor{currentstroke}%
\pgfsetdash{}{0pt}%
\pgfsys@defobject{currentmarker}{\pgfqpoint{-0.048611in}{0.000000in}}{\pgfqpoint{0.000000in}{0.000000in}}{%
\pgfpathmoveto{\pgfqpoint{0.000000in}{0.000000in}}%
\pgfpathlineto{\pgfqpoint{-0.048611in}{0.000000in}}%
\pgfusepath{stroke,fill}%
}%
\begin{pgfscope}%
\pgfsys@transformshift{0.800000in}{1.452000in}%
\pgfsys@useobject{currentmarker}{}%
\end{pgfscope}%
\end{pgfscope}%
\begin{pgfscope}%
\definecolor{textcolor}{rgb}{0.000000,0.000000,0.000000}%
\pgfsetstrokecolor{textcolor}%
\pgfsetfillcolor{textcolor}%
\pgftext[x=0.525308in,y=1.403775in,left,base]{\color{textcolor}\rmfamily\fontsize{10.000000}{12.000000}\selectfont \(\displaystyle 0.8\)}%
\end{pgfscope}%
\begin{pgfscope}%
\pgfsetbuttcap%
\pgfsetroundjoin%
\definecolor{currentfill}{rgb}{0.000000,0.000000,0.000000}%
\pgfsetfillcolor{currentfill}%
\pgfsetlinewidth{0.803000pt}%
\definecolor{currentstroke}{rgb}{0.000000,0.000000,0.000000}%
\pgfsetstrokecolor{currentstroke}%
\pgfsetdash{}{0pt}%
\pgfsys@defobject{currentmarker}{\pgfqpoint{-0.048611in}{0.000000in}}{\pgfqpoint{0.000000in}{0.000000in}}{%
\pgfpathmoveto{\pgfqpoint{0.000000in}{0.000000in}}%
\pgfpathlineto{\pgfqpoint{-0.048611in}{0.000000in}}%
\pgfusepath{stroke,fill}%
}%
\begin{pgfscope}%
\pgfsys@transformshift{0.800000in}{2.068000in}%
\pgfsys@useobject{currentmarker}{}%
\end{pgfscope}%
\end{pgfscope}%
\begin{pgfscope}%
\definecolor{textcolor}{rgb}{0.000000,0.000000,0.000000}%
\pgfsetstrokecolor{textcolor}%
\pgfsetfillcolor{textcolor}%
\pgftext[x=0.525308in,y=2.019775in,left,base]{\color{textcolor}\rmfamily\fontsize{10.000000}{12.000000}\selectfont \(\displaystyle 1.0\)}%
\end{pgfscope}%
\begin{pgfscope}%
\pgfsetbuttcap%
\pgfsetroundjoin%
\definecolor{currentfill}{rgb}{0.000000,0.000000,0.000000}%
\pgfsetfillcolor{currentfill}%
\pgfsetlinewidth{0.803000pt}%
\definecolor{currentstroke}{rgb}{0.000000,0.000000,0.000000}%
\pgfsetstrokecolor{currentstroke}%
\pgfsetdash{}{0pt}%
\pgfsys@defobject{currentmarker}{\pgfqpoint{-0.048611in}{0.000000in}}{\pgfqpoint{0.000000in}{0.000000in}}{%
\pgfpathmoveto{\pgfqpoint{0.000000in}{0.000000in}}%
\pgfpathlineto{\pgfqpoint{-0.048611in}{0.000000in}}%
\pgfusepath{stroke,fill}%
}%
\begin{pgfscope}%
\pgfsys@transformshift{0.800000in}{2.684000in}%
\pgfsys@useobject{currentmarker}{}%
\end{pgfscope}%
\end{pgfscope}%
\begin{pgfscope}%
\definecolor{textcolor}{rgb}{0.000000,0.000000,0.000000}%
\pgfsetstrokecolor{textcolor}%
\pgfsetfillcolor{textcolor}%
\pgftext[x=0.525308in,y=2.635775in,left,base]{\color{textcolor}\rmfamily\fontsize{10.000000}{12.000000}\selectfont \(\displaystyle 1.2\)}%
\end{pgfscope}%
\begin{pgfscope}%
\pgfsetbuttcap%
\pgfsetroundjoin%
\definecolor{currentfill}{rgb}{0.000000,0.000000,0.000000}%
\pgfsetfillcolor{currentfill}%
\pgfsetlinewidth{0.803000pt}%
\definecolor{currentstroke}{rgb}{0.000000,0.000000,0.000000}%
\pgfsetstrokecolor{currentstroke}%
\pgfsetdash{}{0pt}%
\pgfsys@defobject{currentmarker}{\pgfqpoint{-0.048611in}{0.000000in}}{\pgfqpoint{0.000000in}{0.000000in}}{%
\pgfpathmoveto{\pgfqpoint{0.000000in}{0.000000in}}%
\pgfpathlineto{\pgfqpoint{-0.048611in}{0.000000in}}%
\pgfusepath{stroke,fill}%
}%
\begin{pgfscope}%
\pgfsys@transformshift{0.800000in}{3.300000in}%
\pgfsys@useobject{currentmarker}{}%
\end{pgfscope}%
\end{pgfscope}%
\begin{pgfscope}%
\definecolor{textcolor}{rgb}{0.000000,0.000000,0.000000}%
\pgfsetstrokecolor{textcolor}%
\pgfsetfillcolor{textcolor}%
\pgftext[x=0.525308in,y=3.251775in,left,base]{\color{textcolor}\rmfamily\fontsize{10.000000}{12.000000}\selectfont \(\displaystyle 1.4\)}%
\end{pgfscope}%
\begin{pgfscope}%
\pgfsetbuttcap%
\pgfsetroundjoin%
\definecolor{currentfill}{rgb}{0.000000,0.000000,0.000000}%
\pgfsetfillcolor{currentfill}%
\pgfsetlinewidth{0.803000pt}%
\definecolor{currentstroke}{rgb}{0.000000,0.000000,0.000000}%
\pgfsetstrokecolor{currentstroke}%
\pgfsetdash{}{0pt}%
\pgfsys@defobject{currentmarker}{\pgfqpoint{-0.048611in}{0.000000in}}{\pgfqpoint{0.000000in}{0.000000in}}{%
\pgfpathmoveto{\pgfqpoint{0.000000in}{0.000000in}}%
\pgfpathlineto{\pgfqpoint{-0.048611in}{0.000000in}}%
\pgfusepath{stroke,fill}%
}%
\begin{pgfscope}%
\pgfsys@transformshift{0.800000in}{3.916000in}%
\pgfsys@useobject{currentmarker}{}%
\end{pgfscope}%
\end{pgfscope}%
\begin{pgfscope}%
\definecolor{textcolor}{rgb}{0.000000,0.000000,0.000000}%
\pgfsetstrokecolor{textcolor}%
\pgfsetfillcolor{textcolor}%
\pgftext[x=0.525308in,y=3.867775in,left,base]{\color{textcolor}\rmfamily\fontsize{10.000000}{12.000000}\selectfont \(\displaystyle 1.6\)}%
\end{pgfscope}%
\begin{pgfscope}%
\definecolor{textcolor}{rgb}{0.000000,0.000000,0.000000}%
\pgfsetstrokecolor{textcolor}%
\pgfsetfillcolor{textcolor}%
\pgftext[x=0.469752in,y=2.376000in,,bottom]{\color{textcolor}\rmfamily\fontsize{10.000000}{12.000000}\selectfont c}%
\end{pgfscope}%
\begin{pgfscope}%
\pgfsetrectcap%
\pgfsetmiterjoin%
\pgfsetlinewidth{0.803000pt}%
\definecolor{currentstroke}{rgb}{0.000000,0.000000,0.000000}%
\pgfsetstrokecolor{currentstroke}%
\pgfsetdash{}{0pt}%
\pgfpathmoveto{\pgfqpoint{0.800000in}{0.528000in}}%
\pgfpathlineto{\pgfqpoint{0.800000in}{4.224000in}}%
\pgfusepath{stroke}%
\end{pgfscope}%
\begin{pgfscope}%
\pgfsetrectcap%
\pgfsetmiterjoin%
\pgfsetlinewidth{0.803000pt}%
\definecolor{currentstroke}{rgb}{0.000000,0.000000,0.000000}%
\pgfsetstrokecolor{currentstroke}%
\pgfsetdash{}{0pt}%
\pgfpathmoveto{\pgfqpoint{4.768000in}{0.528000in}}%
\pgfpathlineto{\pgfqpoint{4.768000in}{4.224000in}}%
\pgfusepath{stroke}%
\end{pgfscope}%
\begin{pgfscope}%
\pgfsetrectcap%
\pgfsetmiterjoin%
\pgfsetlinewidth{0.803000pt}%
\definecolor{currentstroke}{rgb}{0.000000,0.000000,0.000000}%
\pgfsetstrokecolor{currentstroke}%
\pgfsetdash{}{0pt}%
\pgfpathmoveto{\pgfqpoint{0.800000in}{0.528000in}}%
\pgfpathlineto{\pgfqpoint{4.768000in}{0.528000in}}%
\pgfusepath{stroke}%
\end{pgfscope}%
\begin{pgfscope}%
\pgfsetrectcap%
\pgfsetmiterjoin%
\pgfsetlinewidth{0.803000pt}%
\definecolor{currentstroke}{rgb}{0.000000,0.000000,0.000000}%
\pgfsetstrokecolor{currentstroke}%
\pgfsetdash{}{0pt}%
\pgfpathmoveto{\pgfqpoint{0.800000in}{4.224000in}}%
\pgfpathlineto{\pgfqpoint{4.768000in}{4.224000in}}%
\pgfusepath{stroke}%
\end{pgfscope}%
\begin{pgfscope}%
\definecolor{textcolor}{rgb}{0.000000,0.000000,0.000000}%
\pgfsetstrokecolor{textcolor}%
\pgfsetfillcolor{textcolor}%
\pgftext[x=2.784000in,y=4.432333in,,base]{\color{textcolor}\rmfamily\fontsize{12.000000}{14.400000}\selectfont \(\displaystyle \frac{\chi^2}{\texttt{dof}}\) for various values of m and c}%
\end{pgfscope}%
\begin{pgfscope}%
\pgfpathrectangle{\pgfqpoint{5.016000in}{0.528000in}}{\pgfqpoint{0.184800in}{3.696000in}}%
\pgfusepath{clip}%
\pgfsetbuttcap%
\pgfsetmiterjoin%
\definecolor{currentfill}{rgb}{1.000000,1.000000,1.000000}%
\pgfsetfillcolor{currentfill}%
\pgfsetlinewidth{0.010037pt}%
\definecolor{currentstroke}{rgb}{1.000000,1.000000,1.000000}%
\pgfsetstrokecolor{currentstroke}%
\pgfsetdash{}{0pt}%
\pgfpathmoveto{\pgfqpoint{5.016000in}{0.528000in}}%
\pgfpathlineto{\pgfqpoint{5.016000in}{0.620400in}}%
\pgfpathlineto{\pgfqpoint{5.016000in}{4.131600in}}%
\pgfpathlineto{\pgfqpoint{5.016000in}{4.224000in}}%
\pgfpathlineto{\pgfqpoint{5.200800in}{4.224000in}}%
\pgfpathlineto{\pgfqpoint{5.200800in}{4.131600in}}%
\pgfpathlineto{\pgfqpoint{5.200800in}{0.620400in}}%
\pgfpathlineto{\pgfqpoint{5.200800in}{0.528000in}}%
\pgfpathclose%
\pgfusepath{stroke,fill}%
\end{pgfscope}%
\begin{pgfscope}%
\pgfpathrectangle{\pgfqpoint{5.016000in}{0.528000in}}{\pgfqpoint{0.184800in}{3.696000in}}%
\pgfusepath{clip}%
\pgfsetbuttcap%
\pgfsetroundjoin%
\definecolor{currentfill}{rgb}{0.000000,0.000000,0.553476}%
\pgfsetfillcolor{currentfill}%
\pgfsetlinewidth{0.000000pt}%
\definecolor{currentstroke}{rgb}{0.000000,0.000000,0.000000}%
\pgfsetstrokecolor{currentstroke}%
\pgfsetdash{}{0pt}%
\pgfpathmoveto{\pgfqpoint{5.016000in}{0.528000in}}%
\pgfpathlineto{\pgfqpoint{5.200800in}{0.528000in}}%
\pgfpathlineto{\pgfqpoint{5.200800in}{0.620400in}}%
\pgfpathlineto{\pgfqpoint{5.016000in}{0.620400in}}%
\pgfpathlineto{\pgfqpoint{5.016000in}{0.528000in}}%
\pgfusepath{fill}%
\end{pgfscope}%
\begin{pgfscope}%
\pgfpathrectangle{\pgfqpoint{5.016000in}{0.528000in}}{\pgfqpoint{0.184800in}{3.696000in}}%
\pgfusepath{clip}%
\pgfsetbuttcap%
\pgfsetroundjoin%
\definecolor{currentfill}{rgb}{0.000000,0.000000,0.660428}%
\pgfsetfillcolor{currentfill}%
\pgfsetlinewidth{0.000000pt}%
\definecolor{currentstroke}{rgb}{0.000000,0.000000,0.000000}%
\pgfsetstrokecolor{currentstroke}%
\pgfsetdash{}{0pt}%
\pgfpathmoveto{\pgfqpoint{5.016000in}{0.620400in}}%
\pgfpathlineto{\pgfqpoint{5.200800in}{0.620400in}}%
\pgfpathlineto{\pgfqpoint{5.200800in}{0.712800in}}%
\pgfpathlineto{\pgfqpoint{5.016000in}{0.712800in}}%
\pgfpathlineto{\pgfqpoint{5.016000in}{0.620400in}}%
\pgfusepath{fill}%
\end{pgfscope}%
\begin{pgfscope}%
\pgfpathrectangle{\pgfqpoint{5.016000in}{0.528000in}}{\pgfqpoint{0.184800in}{3.696000in}}%
\pgfusepath{clip}%
\pgfsetbuttcap%
\pgfsetroundjoin%
\definecolor{currentfill}{rgb}{0.000000,0.000000,0.785205}%
\pgfsetfillcolor{currentfill}%
\pgfsetlinewidth{0.000000pt}%
\definecolor{currentstroke}{rgb}{0.000000,0.000000,0.000000}%
\pgfsetstrokecolor{currentstroke}%
\pgfsetdash{}{0pt}%
\pgfpathmoveto{\pgfqpoint{5.016000in}{0.712800in}}%
\pgfpathlineto{\pgfqpoint{5.200800in}{0.712800in}}%
\pgfpathlineto{\pgfqpoint{5.200800in}{0.805200in}}%
\pgfpathlineto{\pgfqpoint{5.016000in}{0.805200in}}%
\pgfpathlineto{\pgfqpoint{5.016000in}{0.712800in}}%
\pgfusepath{fill}%
\end{pgfscope}%
\begin{pgfscope}%
\pgfpathrectangle{\pgfqpoint{5.016000in}{0.528000in}}{\pgfqpoint{0.184800in}{3.696000in}}%
\pgfusepath{clip}%
\pgfsetbuttcap%
\pgfsetroundjoin%
\definecolor{currentfill}{rgb}{0.000000,0.000000,0.892157}%
\pgfsetfillcolor{currentfill}%
\pgfsetlinewidth{0.000000pt}%
\definecolor{currentstroke}{rgb}{0.000000,0.000000,0.000000}%
\pgfsetstrokecolor{currentstroke}%
\pgfsetdash{}{0pt}%
\pgfpathmoveto{\pgfqpoint{5.016000in}{0.805200in}}%
\pgfpathlineto{\pgfqpoint{5.200800in}{0.805200in}}%
\pgfpathlineto{\pgfqpoint{5.200800in}{0.897600in}}%
\pgfpathlineto{\pgfqpoint{5.016000in}{0.897600in}}%
\pgfpathlineto{\pgfqpoint{5.016000in}{0.805200in}}%
\pgfusepath{fill}%
\end{pgfscope}%
\begin{pgfscope}%
\pgfpathrectangle{\pgfqpoint{5.016000in}{0.528000in}}{\pgfqpoint{0.184800in}{3.696000in}}%
\pgfusepath{clip}%
\pgfsetbuttcap%
\pgfsetroundjoin%
\definecolor{currentfill}{rgb}{0.000000,0.000000,0.999109}%
\pgfsetfillcolor{currentfill}%
\pgfsetlinewidth{0.000000pt}%
\definecolor{currentstroke}{rgb}{0.000000,0.000000,0.000000}%
\pgfsetstrokecolor{currentstroke}%
\pgfsetdash{}{0pt}%
\pgfpathmoveto{\pgfqpoint{5.016000in}{0.897600in}}%
\pgfpathlineto{\pgfqpoint{5.200800in}{0.897600in}}%
\pgfpathlineto{\pgfqpoint{5.200800in}{0.990000in}}%
\pgfpathlineto{\pgfqpoint{5.016000in}{0.990000in}}%
\pgfpathlineto{\pgfqpoint{5.016000in}{0.897600in}}%
\pgfusepath{fill}%
\end{pgfscope}%
\begin{pgfscope}%
\pgfpathrectangle{\pgfqpoint{5.016000in}{0.528000in}}{\pgfqpoint{0.184800in}{3.696000in}}%
\pgfusepath{clip}%
\pgfsetbuttcap%
\pgfsetroundjoin%
\definecolor{currentfill}{rgb}{0.000000,0.049020,1.000000}%
\pgfsetfillcolor{currentfill}%
\pgfsetlinewidth{0.000000pt}%
\definecolor{currentstroke}{rgb}{0.000000,0.000000,0.000000}%
\pgfsetstrokecolor{currentstroke}%
\pgfsetdash{}{0pt}%
\pgfpathmoveto{\pgfqpoint{5.016000in}{0.990000in}}%
\pgfpathlineto{\pgfqpoint{5.200800in}{0.990000in}}%
\pgfpathlineto{\pgfqpoint{5.200800in}{1.082400in}}%
\pgfpathlineto{\pgfqpoint{5.016000in}{1.082400in}}%
\pgfpathlineto{\pgfqpoint{5.016000in}{0.990000in}}%
\pgfusepath{fill}%
\end{pgfscope}%
\begin{pgfscope}%
\pgfpathrectangle{\pgfqpoint{5.016000in}{0.528000in}}{\pgfqpoint{0.184800in}{3.696000in}}%
\pgfusepath{clip}%
\pgfsetbuttcap%
\pgfsetroundjoin%
\definecolor{currentfill}{rgb}{0.000000,0.143137,1.000000}%
\pgfsetfillcolor{currentfill}%
\pgfsetlinewidth{0.000000pt}%
\definecolor{currentstroke}{rgb}{0.000000,0.000000,0.000000}%
\pgfsetstrokecolor{currentstroke}%
\pgfsetdash{}{0pt}%
\pgfpathmoveto{\pgfqpoint{5.016000in}{1.082400in}}%
\pgfpathlineto{\pgfqpoint{5.200800in}{1.082400in}}%
\pgfpathlineto{\pgfqpoint{5.200800in}{1.174800in}}%
\pgfpathlineto{\pgfqpoint{5.016000in}{1.174800in}}%
\pgfpathlineto{\pgfqpoint{5.016000in}{1.082400in}}%
\pgfusepath{fill}%
\end{pgfscope}%
\begin{pgfscope}%
\pgfpathrectangle{\pgfqpoint{5.016000in}{0.528000in}}{\pgfqpoint{0.184800in}{3.696000in}}%
\pgfusepath{clip}%
\pgfsetbuttcap%
\pgfsetroundjoin%
\definecolor{currentfill}{rgb}{0.000000,0.252941,1.000000}%
\pgfsetfillcolor{currentfill}%
\pgfsetlinewidth{0.000000pt}%
\definecolor{currentstroke}{rgb}{0.000000,0.000000,0.000000}%
\pgfsetstrokecolor{currentstroke}%
\pgfsetdash{}{0pt}%
\pgfpathmoveto{\pgfqpoint{5.016000in}{1.174800in}}%
\pgfpathlineto{\pgfqpoint{5.200800in}{1.174800in}}%
\pgfpathlineto{\pgfqpoint{5.200800in}{1.267200in}}%
\pgfpathlineto{\pgfqpoint{5.016000in}{1.267200in}}%
\pgfpathlineto{\pgfqpoint{5.016000in}{1.174800in}}%
\pgfusepath{fill}%
\end{pgfscope}%
\begin{pgfscope}%
\pgfpathrectangle{\pgfqpoint{5.016000in}{0.528000in}}{\pgfqpoint{0.184800in}{3.696000in}}%
\pgfusepath{clip}%
\pgfsetbuttcap%
\pgfsetroundjoin%
\definecolor{currentfill}{rgb}{0.000000,0.347059,1.000000}%
\pgfsetfillcolor{currentfill}%
\pgfsetlinewidth{0.000000pt}%
\definecolor{currentstroke}{rgb}{0.000000,0.000000,0.000000}%
\pgfsetstrokecolor{currentstroke}%
\pgfsetdash{}{0pt}%
\pgfpathmoveto{\pgfqpoint{5.016000in}{1.267200in}}%
\pgfpathlineto{\pgfqpoint{5.200800in}{1.267200in}}%
\pgfpathlineto{\pgfqpoint{5.200800in}{1.359600in}}%
\pgfpathlineto{\pgfqpoint{5.016000in}{1.359600in}}%
\pgfpathlineto{\pgfqpoint{5.016000in}{1.267200in}}%
\pgfusepath{fill}%
\end{pgfscope}%
\begin{pgfscope}%
\pgfpathrectangle{\pgfqpoint{5.016000in}{0.528000in}}{\pgfqpoint{0.184800in}{3.696000in}}%
\pgfusepath{clip}%
\pgfsetbuttcap%
\pgfsetroundjoin%
\definecolor{currentfill}{rgb}{0.000000,0.441176,1.000000}%
\pgfsetfillcolor{currentfill}%
\pgfsetlinewidth{0.000000pt}%
\definecolor{currentstroke}{rgb}{0.000000,0.000000,0.000000}%
\pgfsetstrokecolor{currentstroke}%
\pgfsetdash{}{0pt}%
\pgfpathmoveto{\pgfqpoint{5.016000in}{1.359600in}}%
\pgfpathlineto{\pgfqpoint{5.200800in}{1.359600in}}%
\pgfpathlineto{\pgfqpoint{5.200800in}{1.452000in}}%
\pgfpathlineto{\pgfqpoint{5.016000in}{1.452000in}}%
\pgfpathlineto{\pgfqpoint{5.016000in}{1.359600in}}%
\pgfusepath{fill}%
\end{pgfscope}%
\begin{pgfscope}%
\pgfpathrectangle{\pgfqpoint{5.016000in}{0.528000in}}{\pgfqpoint{0.184800in}{3.696000in}}%
\pgfusepath{clip}%
\pgfsetbuttcap%
\pgfsetroundjoin%
\definecolor{currentfill}{rgb}{0.000000,0.550980,1.000000}%
\pgfsetfillcolor{currentfill}%
\pgfsetlinewidth{0.000000pt}%
\definecolor{currentstroke}{rgb}{0.000000,0.000000,0.000000}%
\pgfsetstrokecolor{currentstroke}%
\pgfsetdash{}{0pt}%
\pgfpathmoveto{\pgfqpoint{5.016000in}{1.452000in}}%
\pgfpathlineto{\pgfqpoint{5.200800in}{1.452000in}}%
\pgfpathlineto{\pgfqpoint{5.200800in}{1.544400in}}%
\pgfpathlineto{\pgfqpoint{5.016000in}{1.544400in}}%
\pgfpathlineto{\pgfqpoint{5.016000in}{1.452000in}}%
\pgfusepath{fill}%
\end{pgfscope}%
\begin{pgfscope}%
\pgfpathrectangle{\pgfqpoint{5.016000in}{0.528000in}}{\pgfqpoint{0.184800in}{3.696000in}}%
\pgfusepath{clip}%
\pgfsetbuttcap%
\pgfsetroundjoin%
\definecolor{currentfill}{rgb}{0.000000,0.645098,1.000000}%
\pgfsetfillcolor{currentfill}%
\pgfsetlinewidth{0.000000pt}%
\definecolor{currentstroke}{rgb}{0.000000,0.000000,0.000000}%
\pgfsetstrokecolor{currentstroke}%
\pgfsetdash{}{0pt}%
\pgfpathmoveto{\pgfqpoint{5.016000in}{1.544400in}}%
\pgfpathlineto{\pgfqpoint{5.200800in}{1.544400in}}%
\pgfpathlineto{\pgfqpoint{5.200800in}{1.636800in}}%
\pgfpathlineto{\pgfqpoint{5.016000in}{1.636800in}}%
\pgfpathlineto{\pgfqpoint{5.016000in}{1.544400in}}%
\pgfusepath{fill}%
\end{pgfscope}%
\begin{pgfscope}%
\pgfpathrectangle{\pgfqpoint{5.016000in}{0.528000in}}{\pgfqpoint{0.184800in}{3.696000in}}%
\pgfusepath{clip}%
\pgfsetbuttcap%
\pgfsetroundjoin%
\definecolor{currentfill}{rgb}{0.000000,0.754902,1.000000}%
\pgfsetfillcolor{currentfill}%
\pgfsetlinewidth{0.000000pt}%
\definecolor{currentstroke}{rgb}{0.000000,0.000000,0.000000}%
\pgfsetstrokecolor{currentstroke}%
\pgfsetdash{}{0pt}%
\pgfpathmoveto{\pgfqpoint{5.016000in}{1.636800in}}%
\pgfpathlineto{\pgfqpoint{5.200800in}{1.636800in}}%
\pgfpathlineto{\pgfqpoint{5.200800in}{1.729200in}}%
\pgfpathlineto{\pgfqpoint{5.016000in}{1.729200in}}%
\pgfpathlineto{\pgfqpoint{5.016000in}{1.636800in}}%
\pgfusepath{fill}%
\end{pgfscope}%
\begin{pgfscope}%
\pgfpathrectangle{\pgfqpoint{5.016000in}{0.528000in}}{\pgfqpoint{0.184800in}{3.696000in}}%
\pgfusepath{clip}%
\pgfsetbuttcap%
\pgfsetroundjoin%
\definecolor{currentfill}{rgb}{0.000000,0.849020,1.000000}%
\pgfsetfillcolor{currentfill}%
\pgfsetlinewidth{0.000000pt}%
\definecolor{currentstroke}{rgb}{0.000000,0.000000,0.000000}%
\pgfsetstrokecolor{currentstroke}%
\pgfsetdash{}{0pt}%
\pgfpathmoveto{\pgfqpoint{5.016000in}{1.729200in}}%
\pgfpathlineto{\pgfqpoint{5.200800in}{1.729200in}}%
\pgfpathlineto{\pgfqpoint{5.200800in}{1.821600in}}%
\pgfpathlineto{\pgfqpoint{5.016000in}{1.821600in}}%
\pgfpathlineto{\pgfqpoint{5.016000in}{1.729200in}}%
\pgfusepath{fill}%
\end{pgfscope}%
\begin{pgfscope}%
\pgfpathrectangle{\pgfqpoint{5.016000in}{0.528000in}}{\pgfqpoint{0.184800in}{3.696000in}}%
\pgfusepath{clip}%
\pgfsetbuttcap%
\pgfsetroundjoin%
\definecolor{currentfill}{rgb}{0.034788,0.943137,0.932954}%
\pgfsetfillcolor{currentfill}%
\pgfsetlinewidth{0.000000pt}%
\definecolor{currentstroke}{rgb}{0.000000,0.000000,0.000000}%
\pgfsetstrokecolor{currentstroke}%
\pgfsetdash{}{0pt}%
\pgfpathmoveto{\pgfqpoint{5.016000in}{1.821600in}}%
\pgfpathlineto{\pgfqpoint{5.200800in}{1.821600in}}%
\pgfpathlineto{\pgfqpoint{5.200800in}{1.914000in}}%
\pgfpathlineto{\pgfqpoint{5.016000in}{1.914000in}}%
\pgfpathlineto{\pgfqpoint{5.016000in}{1.821600in}}%
\pgfusepath{fill}%
\end{pgfscope}%
\begin{pgfscope}%
\pgfpathrectangle{\pgfqpoint{5.016000in}{0.528000in}}{\pgfqpoint{0.184800in}{3.696000in}}%
\pgfusepath{clip}%
\pgfsetbuttcap%
\pgfsetroundjoin%
\definecolor{currentfill}{rgb}{0.123340,1.000000,0.844402}%
\pgfsetfillcolor{currentfill}%
\pgfsetlinewidth{0.000000pt}%
\definecolor{currentstroke}{rgb}{0.000000,0.000000,0.000000}%
\pgfsetstrokecolor{currentstroke}%
\pgfsetdash{}{0pt}%
\pgfpathmoveto{\pgfqpoint{5.016000in}{1.914000in}}%
\pgfpathlineto{\pgfqpoint{5.200800in}{1.914000in}}%
\pgfpathlineto{\pgfqpoint{5.200800in}{2.006400in}}%
\pgfpathlineto{\pgfqpoint{5.016000in}{2.006400in}}%
\pgfpathlineto{\pgfqpoint{5.016000in}{1.914000in}}%
\pgfusepath{fill}%
\end{pgfscope}%
\begin{pgfscope}%
\pgfpathrectangle{\pgfqpoint{5.016000in}{0.528000in}}{\pgfqpoint{0.184800in}{3.696000in}}%
\pgfusepath{clip}%
\pgfsetbuttcap%
\pgfsetroundjoin%
\definecolor{currentfill}{rgb}{0.199241,1.000000,0.768501}%
\pgfsetfillcolor{currentfill}%
\pgfsetlinewidth{0.000000pt}%
\definecolor{currentstroke}{rgb}{0.000000,0.000000,0.000000}%
\pgfsetstrokecolor{currentstroke}%
\pgfsetdash{}{0pt}%
\pgfpathmoveto{\pgfqpoint{5.016000in}{2.006400in}}%
\pgfpathlineto{\pgfqpoint{5.200800in}{2.006400in}}%
\pgfpathlineto{\pgfqpoint{5.200800in}{2.098800in}}%
\pgfpathlineto{\pgfqpoint{5.016000in}{2.098800in}}%
\pgfpathlineto{\pgfqpoint{5.016000in}{2.006400in}}%
\pgfusepath{fill}%
\end{pgfscope}%
\begin{pgfscope}%
\pgfpathrectangle{\pgfqpoint{5.016000in}{0.528000in}}{\pgfqpoint{0.184800in}{3.696000in}}%
\pgfusepath{clip}%
\pgfsetbuttcap%
\pgfsetroundjoin%
\definecolor{currentfill}{rgb}{0.287793,1.000000,0.679949}%
\pgfsetfillcolor{currentfill}%
\pgfsetlinewidth{0.000000pt}%
\definecolor{currentstroke}{rgb}{0.000000,0.000000,0.000000}%
\pgfsetstrokecolor{currentstroke}%
\pgfsetdash{}{0pt}%
\pgfpathmoveto{\pgfqpoint{5.016000in}{2.098800in}}%
\pgfpathlineto{\pgfqpoint{5.200800in}{2.098800in}}%
\pgfpathlineto{\pgfqpoint{5.200800in}{2.191200in}}%
\pgfpathlineto{\pgfqpoint{5.016000in}{2.191200in}}%
\pgfpathlineto{\pgfqpoint{5.016000in}{2.098800in}}%
\pgfusepath{fill}%
\end{pgfscope}%
\begin{pgfscope}%
\pgfpathrectangle{\pgfqpoint{5.016000in}{0.528000in}}{\pgfqpoint{0.184800in}{3.696000in}}%
\pgfusepath{clip}%
\pgfsetbuttcap%
\pgfsetroundjoin%
\definecolor{currentfill}{rgb}{0.363694,1.000000,0.604048}%
\pgfsetfillcolor{currentfill}%
\pgfsetlinewidth{0.000000pt}%
\definecolor{currentstroke}{rgb}{0.000000,0.000000,0.000000}%
\pgfsetstrokecolor{currentstroke}%
\pgfsetdash{}{0pt}%
\pgfpathmoveto{\pgfqpoint{5.016000in}{2.191200in}}%
\pgfpathlineto{\pgfqpoint{5.200800in}{2.191200in}}%
\pgfpathlineto{\pgfqpoint{5.200800in}{2.283600in}}%
\pgfpathlineto{\pgfqpoint{5.016000in}{2.283600in}}%
\pgfpathlineto{\pgfqpoint{5.016000in}{2.191200in}}%
\pgfusepath{fill}%
\end{pgfscope}%
\begin{pgfscope}%
\pgfpathrectangle{\pgfqpoint{5.016000in}{0.528000in}}{\pgfqpoint{0.184800in}{3.696000in}}%
\pgfusepath{clip}%
\pgfsetbuttcap%
\pgfsetroundjoin%
\definecolor{currentfill}{rgb}{0.439595,1.000000,0.528147}%
\pgfsetfillcolor{currentfill}%
\pgfsetlinewidth{0.000000pt}%
\definecolor{currentstroke}{rgb}{0.000000,0.000000,0.000000}%
\pgfsetstrokecolor{currentstroke}%
\pgfsetdash{}{0pt}%
\pgfpathmoveto{\pgfqpoint{5.016000in}{2.283600in}}%
\pgfpathlineto{\pgfqpoint{5.200800in}{2.283600in}}%
\pgfpathlineto{\pgfqpoint{5.200800in}{2.376000in}}%
\pgfpathlineto{\pgfqpoint{5.016000in}{2.376000in}}%
\pgfpathlineto{\pgfqpoint{5.016000in}{2.283600in}}%
\pgfusepath{fill}%
\end{pgfscope}%
\begin{pgfscope}%
\pgfpathrectangle{\pgfqpoint{5.016000in}{0.528000in}}{\pgfqpoint{0.184800in}{3.696000in}}%
\pgfusepath{clip}%
\pgfsetbuttcap%
\pgfsetroundjoin%
\definecolor{currentfill}{rgb}{0.528147,1.000000,0.439595}%
\pgfsetfillcolor{currentfill}%
\pgfsetlinewidth{0.000000pt}%
\definecolor{currentstroke}{rgb}{0.000000,0.000000,0.000000}%
\pgfsetstrokecolor{currentstroke}%
\pgfsetdash{}{0pt}%
\pgfpathmoveto{\pgfqpoint{5.016000in}{2.376000in}}%
\pgfpathlineto{\pgfqpoint{5.200800in}{2.376000in}}%
\pgfpathlineto{\pgfqpoint{5.200800in}{2.468400in}}%
\pgfpathlineto{\pgfqpoint{5.016000in}{2.468400in}}%
\pgfpathlineto{\pgfqpoint{5.016000in}{2.376000in}}%
\pgfusepath{fill}%
\end{pgfscope}%
\begin{pgfscope}%
\pgfpathrectangle{\pgfqpoint{5.016000in}{0.528000in}}{\pgfqpoint{0.184800in}{3.696000in}}%
\pgfusepath{clip}%
\pgfsetbuttcap%
\pgfsetroundjoin%
\definecolor{currentfill}{rgb}{0.604048,1.000000,0.363694}%
\pgfsetfillcolor{currentfill}%
\pgfsetlinewidth{0.000000pt}%
\definecolor{currentstroke}{rgb}{0.000000,0.000000,0.000000}%
\pgfsetstrokecolor{currentstroke}%
\pgfsetdash{}{0pt}%
\pgfpathmoveto{\pgfqpoint{5.016000in}{2.468400in}}%
\pgfpathlineto{\pgfqpoint{5.200800in}{2.468400in}}%
\pgfpathlineto{\pgfqpoint{5.200800in}{2.560800in}}%
\pgfpathlineto{\pgfqpoint{5.016000in}{2.560800in}}%
\pgfpathlineto{\pgfqpoint{5.016000in}{2.468400in}}%
\pgfusepath{fill}%
\end{pgfscope}%
\begin{pgfscope}%
\pgfpathrectangle{\pgfqpoint{5.016000in}{0.528000in}}{\pgfqpoint{0.184800in}{3.696000in}}%
\pgfusepath{clip}%
\pgfsetbuttcap%
\pgfsetroundjoin%
\definecolor{currentfill}{rgb}{0.692600,1.000000,0.275142}%
\pgfsetfillcolor{currentfill}%
\pgfsetlinewidth{0.000000pt}%
\definecolor{currentstroke}{rgb}{0.000000,0.000000,0.000000}%
\pgfsetstrokecolor{currentstroke}%
\pgfsetdash{}{0pt}%
\pgfpathmoveto{\pgfqpoint{5.016000in}{2.560800in}}%
\pgfpathlineto{\pgfqpoint{5.200800in}{2.560800in}}%
\pgfpathlineto{\pgfqpoint{5.200800in}{2.653200in}}%
\pgfpathlineto{\pgfqpoint{5.016000in}{2.653200in}}%
\pgfpathlineto{\pgfqpoint{5.016000in}{2.560800in}}%
\pgfusepath{fill}%
\end{pgfscope}%
\begin{pgfscope}%
\pgfpathrectangle{\pgfqpoint{5.016000in}{0.528000in}}{\pgfqpoint{0.184800in}{3.696000in}}%
\pgfusepath{clip}%
\pgfsetbuttcap%
\pgfsetroundjoin%
\definecolor{currentfill}{rgb}{0.768501,1.000000,0.199241}%
\pgfsetfillcolor{currentfill}%
\pgfsetlinewidth{0.000000pt}%
\definecolor{currentstroke}{rgb}{0.000000,0.000000,0.000000}%
\pgfsetstrokecolor{currentstroke}%
\pgfsetdash{}{0pt}%
\pgfpathmoveto{\pgfqpoint{5.016000in}{2.653200in}}%
\pgfpathlineto{\pgfqpoint{5.200800in}{2.653200in}}%
\pgfpathlineto{\pgfqpoint{5.200800in}{2.745600in}}%
\pgfpathlineto{\pgfqpoint{5.016000in}{2.745600in}}%
\pgfpathlineto{\pgfqpoint{5.016000in}{2.653200in}}%
\pgfusepath{fill}%
\end{pgfscope}%
\begin{pgfscope}%
\pgfpathrectangle{\pgfqpoint{5.016000in}{0.528000in}}{\pgfqpoint{0.184800in}{3.696000in}}%
\pgfusepath{clip}%
\pgfsetbuttcap%
\pgfsetroundjoin%
\definecolor{currentfill}{rgb}{0.844402,1.000000,0.123340}%
\pgfsetfillcolor{currentfill}%
\pgfsetlinewidth{0.000000pt}%
\definecolor{currentstroke}{rgb}{0.000000,0.000000,0.000000}%
\pgfsetstrokecolor{currentstroke}%
\pgfsetdash{}{0pt}%
\pgfpathmoveto{\pgfqpoint{5.016000in}{2.745600in}}%
\pgfpathlineto{\pgfqpoint{5.200800in}{2.745600in}}%
\pgfpathlineto{\pgfqpoint{5.200800in}{2.838000in}}%
\pgfpathlineto{\pgfqpoint{5.016000in}{2.838000in}}%
\pgfpathlineto{\pgfqpoint{5.016000in}{2.745600in}}%
\pgfusepath{fill}%
\end{pgfscope}%
\begin{pgfscope}%
\pgfpathrectangle{\pgfqpoint{5.016000in}{0.528000in}}{\pgfqpoint{0.184800in}{3.696000in}}%
\pgfusepath{clip}%
\pgfsetbuttcap%
\pgfsetroundjoin%
\definecolor{currentfill}{rgb}{0.932954,1.000000,0.034788}%
\pgfsetfillcolor{currentfill}%
\pgfsetlinewidth{0.000000pt}%
\definecolor{currentstroke}{rgb}{0.000000,0.000000,0.000000}%
\pgfsetstrokecolor{currentstroke}%
\pgfsetdash{}{0pt}%
\pgfpathmoveto{\pgfqpoint{5.016000in}{2.838000in}}%
\pgfpathlineto{\pgfqpoint{5.200800in}{2.838000in}}%
\pgfpathlineto{\pgfqpoint{5.200800in}{2.930400in}}%
\pgfpathlineto{\pgfqpoint{5.016000in}{2.930400in}}%
\pgfpathlineto{\pgfqpoint{5.016000in}{2.838000in}}%
\pgfusepath{fill}%
\end{pgfscope}%
\begin{pgfscope}%
\pgfpathrectangle{\pgfqpoint{5.016000in}{0.528000in}}{\pgfqpoint{0.184800in}{3.696000in}}%
\pgfusepath{clip}%
\pgfsetbuttcap%
\pgfsetroundjoin%
\definecolor{currentfill}{rgb}{1.000000,0.915759,0.000000}%
\pgfsetfillcolor{currentfill}%
\pgfsetlinewidth{0.000000pt}%
\definecolor{currentstroke}{rgb}{0.000000,0.000000,0.000000}%
\pgfsetstrokecolor{currentstroke}%
\pgfsetdash{}{0pt}%
\pgfpathmoveto{\pgfqpoint{5.016000in}{2.930400in}}%
\pgfpathlineto{\pgfqpoint{5.200800in}{2.930400in}}%
\pgfpathlineto{\pgfqpoint{5.200800in}{3.022800in}}%
\pgfpathlineto{\pgfqpoint{5.016000in}{3.022800in}}%
\pgfpathlineto{\pgfqpoint{5.016000in}{2.930400in}}%
\pgfusepath{fill}%
\end{pgfscope}%
\begin{pgfscope}%
\pgfpathrectangle{\pgfqpoint{5.016000in}{0.528000in}}{\pgfqpoint{0.184800in}{3.696000in}}%
\pgfusepath{clip}%
\pgfsetbuttcap%
\pgfsetroundjoin%
\definecolor{currentfill}{rgb}{1.000000,0.814089,0.000000}%
\pgfsetfillcolor{currentfill}%
\pgfsetlinewidth{0.000000pt}%
\definecolor{currentstroke}{rgb}{0.000000,0.000000,0.000000}%
\pgfsetstrokecolor{currentstroke}%
\pgfsetdash{}{0pt}%
\pgfpathmoveto{\pgfqpoint{5.016000in}{3.022800in}}%
\pgfpathlineto{\pgfqpoint{5.200800in}{3.022800in}}%
\pgfpathlineto{\pgfqpoint{5.200800in}{3.115200in}}%
\pgfpathlineto{\pgfqpoint{5.016000in}{3.115200in}}%
\pgfpathlineto{\pgfqpoint{5.016000in}{3.022800in}}%
\pgfusepath{fill}%
\end{pgfscope}%
\begin{pgfscope}%
\pgfpathrectangle{\pgfqpoint{5.016000in}{0.528000in}}{\pgfqpoint{0.184800in}{3.696000in}}%
\pgfusepath{clip}%
\pgfsetbuttcap%
\pgfsetroundjoin%
\definecolor{currentfill}{rgb}{1.000000,0.726943,0.000000}%
\pgfsetfillcolor{currentfill}%
\pgfsetlinewidth{0.000000pt}%
\definecolor{currentstroke}{rgb}{0.000000,0.000000,0.000000}%
\pgfsetstrokecolor{currentstroke}%
\pgfsetdash{}{0pt}%
\pgfpathmoveto{\pgfqpoint{5.016000in}{3.115200in}}%
\pgfpathlineto{\pgfqpoint{5.200800in}{3.115200in}}%
\pgfpathlineto{\pgfqpoint{5.200800in}{3.207600in}}%
\pgfpathlineto{\pgfqpoint{5.016000in}{3.207600in}}%
\pgfpathlineto{\pgfqpoint{5.016000in}{3.115200in}}%
\pgfusepath{fill}%
\end{pgfscope}%
\begin{pgfscope}%
\pgfpathrectangle{\pgfqpoint{5.016000in}{0.528000in}}{\pgfqpoint{0.184800in}{3.696000in}}%
\pgfusepath{clip}%
\pgfsetbuttcap%
\pgfsetroundjoin%
\definecolor{currentfill}{rgb}{1.000000,0.639797,0.000000}%
\pgfsetfillcolor{currentfill}%
\pgfsetlinewidth{0.000000pt}%
\definecolor{currentstroke}{rgb}{0.000000,0.000000,0.000000}%
\pgfsetstrokecolor{currentstroke}%
\pgfsetdash{}{0pt}%
\pgfpathmoveto{\pgfqpoint{5.016000in}{3.207600in}}%
\pgfpathlineto{\pgfqpoint{5.200800in}{3.207600in}}%
\pgfpathlineto{\pgfqpoint{5.200800in}{3.300000in}}%
\pgfpathlineto{\pgfqpoint{5.016000in}{3.300000in}}%
\pgfpathlineto{\pgfqpoint{5.016000in}{3.207600in}}%
\pgfusepath{fill}%
\end{pgfscope}%
\begin{pgfscope}%
\pgfpathrectangle{\pgfqpoint{5.016000in}{0.528000in}}{\pgfqpoint{0.184800in}{3.696000in}}%
\pgfusepath{clip}%
\pgfsetbuttcap%
\pgfsetroundjoin%
\definecolor{currentfill}{rgb}{1.000000,0.538126,0.000000}%
\pgfsetfillcolor{currentfill}%
\pgfsetlinewidth{0.000000pt}%
\definecolor{currentstroke}{rgb}{0.000000,0.000000,0.000000}%
\pgfsetstrokecolor{currentstroke}%
\pgfsetdash{}{0pt}%
\pgfpathmoveto{\pgfqpoint{5.016000in}{3.300000in}}%
\pgfpathlineto{\pgfqpoint{5.200800in}{3.300000in}}%
\pgfpathlineto{\pgfqpoint{5.200800in}{3.392400in}}%
\pgfpathlineto{\pgfqpoint{5.016000in}{3.392400in}}%
\pgfpathlineto{\pgfqpoint{5.016000in}{3.300000in}}%
\pgfusepath{fill}%
\end{pgfscope}%
\begin{pgfscope}%
\pgfpathrectangle{\pgfqpoint{5.016000in}{0.528000in}}{\pgfqpoint{0.184800in}{3.696000in}}%
\pgfusepath{clip}%
\pgfsetbuttcap%
\pgfsetroundjoin%
\definecolor{currentfill}{rgb}{1.000000,0.450980,0.000000}%
\pgfsetfillcolor{currentfill}%
\pgfsetlinewidth{0.000000pt}%
\definecolor{currentstroke}{rgb}{0.000000,0.000000,0.000000}%
\pgfsetstrokecolor{currentstroke}%
\pgfsetdash{}{0pt}%
\pgfpathmoveto{\pgfqpoint{5.016000in}{3.392400in}}%
\pgfpathlineto{\pgfqpoint{5.200800in}{3.392400in}}%
\pgfpathlineto{\pgfqpoint{5.200800in}{3.484800in}}%
\pgfpathlineto{\pgfqpoint{5.016000in}{3.484800in}}%
\pgfpathlineto{\pgfqpoint{5.016000in}{3.392400in}}%
\pgfusepath{fill}%
\end{pgfscope}%
\begin{pgfscope}%
\pgfpathrectangle{\pgfqpoint{5.016000in}{0.528000in}}{\pgfqpoint{0.184800in}{3.696000in}}%
\pgfusepath{clip}%
\pgfsetbuttcap%
\pgfsetroundjoin%
\definecolor{currentfill}{rgb}{1.000000,0.349310,0.000000}%
\pgfsetfillcolor{currentfill}%
\pgfsetlinewidth{0.000000pt}%
\definecolor{currentstroke}{rgb}{0.000000,0.000000,0.000000}%
\pgfsetstrokecolor{currentstroke}%
\pgfsetdash{}{0pt}%
\pgfpathmoveto{\pgfqpoint{5.016000in}{3.484800in}}%
\pgfpathlineto{\pgfqpoint{5.200800in}{3.484800in}}%
\pgfpathlineto{\pgfqpoint{5.200800in}{3.577200in}}%
\pgfpathlineto{\pgfqpoint{5.016000in}{3.577200in}}%
\pgfpathlineto{\pgfqpoint{5.016000in}{3.484800in}}%
\pgfusepath{fill}%
\end{pgfscope}%
\begin{pgfscope}%
\pgfpathrectangle{\pgfqpoint{5.016000in}{0.528000in}}{\pgfqpoint{0.184800in}{3.696000in}}%
\pgfusepath{clip}%
\pgfsetbuttcap%
\pgfsetroundjoin%
\definecolor{currentfill}{rgb}{1.000000,0.262164,0.000000}%
\pgfsetfillcolor{currentfill}%
\pgfsetlinewidth{0.000000pt}%
\definecolor{currentstroke}{rgb}{0.000000,0.000000,0.000000}%
\pgfsetstrokecolor{currentstroke}%
\pgfsetdash{}{0pt}%
\pgfpathmoveto{\pgfqpoint{5.016000in}{3.577200in}}%
\pgfpathlineto{\pgfqpoint{5.200800in}{3.577200in}}%
\pgfpathlineto{\pgfqpoint{5.200800in}{3.669600in}}%
\pgfpathlineto{\pgfqpoint{5.016000in}{3.669600in}}%
\pgfpathlineto{\pgfqpoint{5.016000in}{3.577200in}}%
\pgfusepath{fill}%
\end{pgfscope}%
\begin{pgfscope}%
\pgfpathrectangle{\pgfqpoint{5.016000in}{0.528000in}}{\pgfqpoint{0.184800in}{3.696000in}}%
\pgfusepath{clip}%
\pgfsetbuttcap%
\pgfsetroundjoin%
\definecolor{currentfill}{rgb}{1.000000,0.175018,0.000000}%
\pgfsetfillcolor{currentfill}%
\pgfsetlinewidth{0.000000pt}%
\definecolor{currentstroke}{rgb}{0.000000,0.000000,0.000000}%
\pgfsetstrokecolor{currentstroke}%
\pgfsetdash{}{0pt}%
\pgfpathmoveto{\pgfqpoint{5.016000in}{3.669600in}}%
\pgfpathlineto{\pgfqpoint{5.200800in}{3.669600in}}%
\pgfpathlineto{\pgfqpoint{5.200800in}{3.762000in}}%
\pgfpathlineto{\pgfqpoint{5.016000in}{3.762000in}}%
\pgfpathlineto{\pgfqpoint{5.016000in}{3.669600in}}%
\pgfusepath{fill}%
\end{pgfscope}%
\begin{pgfscope}%
\pgfpathrectangle{\pgfqpoint{5.016000in}{0.528000in}}{\pgfqpoint{0.184800in}{3.696000in}}%
\pgfusepath{clip}%
\pgfsetbuttcap%
\pgfsetroundjoin%
\definecolor{currentfill}{rgb}{0.999109,0.073348,0.000000}%
\pgfsetfillcolor{currentfill}%
\pgfsetlinewidth{0.000000pt}%
\definecolor{currentstroke}{rgb}{0.000000,0.000000,0.000000}%
\pgfsetstrokecolor{currentstroke}%
\pgfsetdash{}{0pt}%
\pgfpathmoveto{\pgfqpoint{5.016000in}{3.762000in}}%
\pgfpathlineto{\pgfqpoint{5.200800in}{3.762000in}}%
\pgfpathlineto{\pgfqpoint{5.200800in}{3.854400in}}%
\pgfpathlineto{\pgfqpoint{5.016000in}{3.854400in}}%
\pgfpathlineto{\pgfqpoint{5.016000in}{3.762000in}}%
\pgfusepath{fill}%
\end{pgfscope}%
\begin{pgfscope}%
\pgfpathrectangle{\pgfqpoint{5.016000in}{0.528000in}}{\pgfqpoint{0.184800in}{3.696000in}}%
\pgfusepath{clip}%
\pgfsetbuttcap%
\pgfsetroundjoin%
\definecolor{currentfill}{rgb}{0.892157,0.000000,0.000000}%
\pgfsetfillcolor{currentfill}%
\pgfsetlinewidth{0.000000pt}%
\definecolor{currentstroke}{rgb}{0.000000,0.000000,0.000000}%
\pgfsetstrokecolor{currentstroke}%
\pgfsetdash{}{0pt}%
\pgfpathmoveto{\pgfqpoint{5.016000in}{3.854400in}}%
\pgfpathlineto{\pgfqpoint{5.200800in}{3.854400in}}%
\pgfpathlineto{\pgfqpoint{5.200800in}{3.946800in}}%
\pgfpathlineto{\pgfqpoint{5.016000in}{3.946800in}}%
\pgfpathlineto{\pgfqpoint{5.016000in}{3.854400in}}%
\pgfusepath{fill}%
\end{pgfscope}%
\begin{pgfscope}%
\pgfpathrectangle{\pgfqpoint{5.016000in}{0.528000in}}{\pgfqpoint{0.184800in}{3.696000in}}%
\pgfusepath{clip}%
\pgfsetbuttcap%
\pgfsetroundjoin%
\definecolor{currentfill}{rgb}{0.767380,0.000000,0.000000}%
\pgfsetfillcolor{currentfill}%
\pgfsetlinewidth{0.000000pt}%
\definecolor{currentstroke}{rgb}{0.000000,0.000000,0.000000}%
\pgfsetstrokecolor{currentstroke}%
\pgfsetdash{}{0pt}%
\pgfpathmoveto{\pgfqpoint{5.016000in}{3.946800in}}%
\pgfpathlineto{\pgfqpoint{5.200800in}{3.946800in}}%
\pgfpathlineto{\pgfqpoint{5.200800in}{4.039200in}}%
\pgfpathlineto{\pgfqpoint{5.016000in}{4.039200in}}%
\pgfpathlineto{\pgfqpoint{5.016000in}{3.946800in}}%
\pgfusepath{fill}%
\end{pgfscope}%
\begin{pgfscope}%
\pgfpathrectangle{\pgfqpoint{5.016000in}{0.528000in}}{\pgfqpoint{0.184800in}{3.696000in}}%
\pgfusepath{clip}%
\pgfsetbuttcap%
\pgfsetroundjoin%
\definecolor{currentfill}{rgb}{0.660428,0.000000,0.000000}%
\pgfsetfillcolor{currentfill}%
\pgfsetlinewidth{0.000000pt}%
\definecolor{currentstroke}{rgb}{0.000000,0.000000,0.000000}%
\pgfsetstrokecolor{currentstroke}%
\pgfsetdash{}{0pt}%
\pgfpathmoveto{\pgfqpoint{5.016000in}{4.039200in}}%
\pgfpathlineto{\pgfqpoint{5.200800in}{4.039200in}}%
\pgfpathlineto{\pgfqpoint{5.200800in}{4.131600in}}%
\pgfpathlineto{\pgfqpoint{5.016000in}{4.131600in}}%
\pgfpathlineto{\pgfqpoint{5.016000in}{4.039200in}}%
\pgfusepath{fill}%
\end{pgfscope}%
\begin{pgfscope}%
\pgfpathrectangle{\pgfqpoint{5.016000in}{0.528000in}}{\pgfqpoint{0.184800in}{3.696000in}}%
\pgfusepath{clip}%
\pgfsetbuttcap%
\pgfsetroundjoin%
\definecolor{currentfill}{rgb}{0.553476,0.000000,0.000000}%
\pgfsetfillcolor{currentfill}%
\pgfsetlinewidth{0.000000pt}%
\definecolor{currentstroke}{rgb}{0.000000,0.000000,0.000000}%
\pgfsetstrokecolor{currentstroke}%
\pgfsetdash{}{0pt}%
\pgfpathmoveto{\pgfqpoint{5.016000in}{4.131600in}}%
\pgfpathlineto{\pgfqpoint{5.200800in}{4.131600in}}%
\pgfpathlineto{\pgfqpoint{5.200800in}{4.224000in}}%
\pgfpathlineto{\pgfqpoint{5.016000in}{4.224000in}}%
\pgfpathlineto{\pgfqpoint{5.016000in}{4.131600in}}%
\pgfusepath{fill}%
\end{pgfscope}%
\begin{pgfscope}%
\pgfsetbuttcap%
\pgfsetroundjoin%
\definecolor{currentfill}{rgb}{0.000000,0.000000,0.000000}%
\pgfsetfillcolor{currentfill}%
\pgfsetlinewidth{0.803000pt}%
\definecolor{currentstroke}{rgb}{0.000000,0.000000,0.000000}%
\pgfsetstrokecolor{currentstroke}%
\pgfsetdash{}{0pt}%
\pgfsys@defobject{currentmarker}{\pgfqpoint{0.000000in}{0.000000in}}{\pgfqpoint{0.048611in}{0.000000in}}{%
\pgfpathmoveto{\pgfqpoint{0.000000in}{0.000000in}}%
\pgfpathlineto{\pgfqpoint{0.048611in}{0.000000in}}%
\pgfusepath{stroke,fill}%
}%
\begin{pgfscope}%
\pgfsys@transformshift{5.200800in}{0.528000in}%
\pgfsys@useobject{currentmarker}{}%
\end{pgfscope}%
\end{pgfscope}%
\begin{pgfscope}%
\definecolor{textcolor}{rgb}{0.000000,0.000000,0.000000}%
\pgfsetstrokecolor{textcolor}%
\pgfsetfillcolor{textcolor}%
\pgftext[x=5.298022in,y=0.479775in,left,base]{\color{textcolor}\rmfamily\fontsize{10.000000}{12.000000}\selectfont \(\displaystyle 0\)}%
\end{pgfscope}%
\begin{pgfscope}%
\pgfsetbuttcap%
\pgfsetroundjoin%
\definecolor{currentfill}{rgb}{0.000000,0.000000,0.000000}%
\pgfsetfillcolor{currentfill}%
\pgfsetlinewidth{0.803000pt}%
\definecolor{currentstroke}{rgb}{0.000000,0.000000,0.000000}%
\pgfsetstrokecolor{currentstroke}%
\pgfsetdash{}{0pt}%
\pgfsys@defobject{currentmarker}{\pgfqpoint{0.000000in}{0.000000in}}{\pgfqpoint{0.048611in}{0.000000in}}{%
\pgfpathmoveto{\pgfqpoint{0.000000in}{0.000000in}}%
\pgfpathlineto{\pgfqpoint{0.048611in}{0.000000in}}%
\pgfusepath{stroke,fill}%
}%
\begin{pgfscope}%
\pgfsys@transformshift{5.200800in}{0.990000in}%
\pgfsys@useobject{currentmarker}{}%
\end{pgfscope}%
\end{pgfscope}%
\begin{pgfscope}%
\definecolor{textcolor}{rgb}{0.000000,0.000000,0.000000}%
\pgfsetstrokecolor{textcolor}%
\pgfsetfillcolor{textcolor}%
\pgftext[x=5.298022in,y=0.941775in,left,base]{\color{textcolor}\rmfamily\fontsize{10.000000}{12.000000}\selectfont \(\displaystyle 5\)}%
\end{pgfscope}%
\begin{pgfscope}%
\pgfsetbuttcap%
\pgfsetroundjoin%
\definecolor{currentfill}{rgb}{0.000000,0.000000,0.000000}%
\pgfsetfillcolor{currentfill}%
\pgfsetlinewidth{0.803000pt}%
\definecolor{currentstroke}{rgb}{0.000000,0.000000,0.000000}%
\pgfsetstrokecolor{currentstroke}%
\pgfsetdash{}{0pt}%
\pgfsys@defobject{currentmarker}{\pgfqpoint{0.000000in}{0.000000in}}{\pgfqpoint{0.048611in}{0.000000in}}{%
\pgfpathmoveto{\pgfqpoint{0.000000in}{0.000000in}}%
\pgfpathlineto{\pgfqpoint{0.048611in}{0.000000in}}%
\pgfusepath{stroke,fill}%
}%
\begin{pgfscope}%
\pgfsys@transformshift{5.200800in}{1.452000in}%
\pgfsys@useobject{currentmarker}{}%
\end{pgfscope}%
\end{pgfscope}%
\begin{pgfscope}%
\definecolor{textcolor}{rgb}{0.000000,0.000000,0.000000}%
\pgfsetstrokecolor{textcolor}%
\pgfsetfillcolor{textcolor}%
\pgftext[x=5.298022in,y=1.403775in,left,base]{\color{textcolor}\rmfamily\fontsize{10.000000}{12.000000}\selectfont \(\displaystyle 10\)}%
\end{pgfscope}%
\begin{pgfscope}%
\pgfsetbuttcap%
\pgfsetroundjoin%
\definecolor{currentfill}{rgb}{0.000000,0.000000,0.000000}%
\pgfsetfillcolor{currentfill}%
\pgfsetlinewidth{0.803000pt}%
\definecolor{currentstroke}{rgb}{0.000000,0.000000,0.000000}%
\pgfsetstrokecolor{currentstroke}%
\pgfsetdash{}{0pt}%
\pgfsys@defobject{currentmarker}{\pgfqpoint{0.000000in}{0.000000in}}{\pgfqpoint{0.048611in}{0.000000in}}{%
\pgfpathmoveto{\pgfqpoint{0.000000in}{0.000000in}}%
\pgfpathlineto{\pgfqpoint{0.048611in}{0.000000in}}%
\pgfusepath{stroke,fill}%
}%
\begin{pgfscope}%
\pgfsys@transformshift{5.200800in}{1.914000in}%
\pgfsys@useobject{currentmarker}{}%
\end{pgfscope}%
\end{pgfscope}%
\begin{pgfscope}%
\definecolor{textcolor}{rgb}{0.000000,0.000000,0.000000}%
\pgfsetstrokecolor{textcolor}%
\pgfsetfillcolor{textcolor}%
\pgftext[x=5.298022in,y=1.865775in,left,base]{\color{textcolor}\rmfamily\fontsize{10.000000}{12.000000}\selectfont \(\displaystyle 15\)}%
\end{pgfscope}%
\begin{pgfscope}%
\pgfsetbuttcap%
\pgfsetroundjoin%
\definecolor{currentfill}{rgb}{0.000000,0.000000,0.000000}%
\pgfsetfillcolor{currentfill}%
\pgfsetlinewidth{0.803000pt}%
\definecolor{currentstroke}{rgb}{0.000000,0.000000,0.000000}%
\pgfsetstrokecolor{currentstroke}%
\pgfsetdash{}{0pt}%
\pgfsys@defobject{currentmarker}{\pgfqpoint{0.000000in}{0.000000in}}{\pgfqpoint{0.048611in}{0.000000in}}{%
\pgfpathmoveto{\pgfqpoint{0.000000in}{0.000000in}}%
\pgfpathlineto{\pgfqpoint{0.048611in}{0.000000in}}%
\pgfusepath{stroke,fill}%
}%
\begin{pgfscope}%
\pgfsys@transformshift{5.200800in}{2.376000in}%
\pgfsys@useobject{currentmarker}{}%
\end{pgfscope}%
\end{pgfscope}%
\begin{pgfscope}%
\definecolor{textcolor}{rgb}{0.000000,0.000000,0.000000}%
\pgfsetstrokecolor{textcolor}%
\pgfsetfillcolor{textcolor}%
\pgftext[x=5.298022in,y=2.327775in,left,base]{\color{textcolor}\rmfamily\fontsize{10.000000}{12.000000}\selectfont \(\displaystyle 20\)}%
\end{pgfscope}%
\begin{pgfscope}%
\pgfsetbuttcap%
\pgfsetroundjoin%
\definecolor{currentfill}{rgb}{0.000000,0.000000,0.000000}%
\pgfsetfillcolor{currentfill}%
\pgfsetlinewidth{0.803000pt}%
\definecolor{currentstroke}{rgb}{0.000000,0.000000,0.000000}%
\pgfsetstrokecolor{currentstroke}%
\pgfsetdash{}{0pt}%
\pgfsys@defobject{currentmarker}{\pgfqpoint{0.000000in}{0.000000in}}{\pgfqpoint{0.048611in}{0.000000in}}{%
\pgfpathmoveto{\pgfqpoint{0.000000in}{0.000000in}}%
\pgfpathlineto{\pgfqpoint{0.048611in}{0.000000in}}%
\pgfusepath{stroke,fill}%
}%
\begin{pgfscope}%
\pgfsys@transformshift{5.200800in}{2.838000in}%
\pgfsys@useobject{currentmarker}{}%
\end{pgfscope}%
\end{pgfscope}%
\begin{pgfscope}%
\definecolor{textcolor}{rgb}{0.000000,0.000000,0.000000}%
\pgfsetstrokecolor{textcolor}%
\pgfsetfillcolor{textcolor}%
\pgftext[x=5.298022in,y=2.789775in,left,base]{\color{textcolor}\rmfamily\fontsize{10.000000}{12.000000}\selectfont \(\displaystyle 25\)}%
\end{pgfscope}%
\begin{pgfscope}%
\pgfsetbuttcap%
\pgfsetroundjoin%
\definecolor{currentfill}{rgb}{0.000000,0.000000,0.000000}%
\pgfsetfillcolor{currentfill}%
\pgfsetlinewidth{0.803000pt}%
\definecolor{currentstroke}{rgb}{0.000000,0.000000,0.000000}%
\pgfsetstrokecolor{currentstroke}%
\pgfsetdash{}{0pt}%
\pgfsys@defobject{currentmarker}{\pgfqpoint{0.000000in}{0.000000in}}{\pgfqpoint{0.048611in}{0.000000in}}{%
\pgfpathmoveto{\pgfqpoint{0.000000in}{0.000000in}}%
\pgfpathlineto{\pgfqpoint{0.048611in}{0.000000in}}%
\pgfusepath{stroke,fill}%
}%
\begin{pgfscope}%
\pgfsys@transformshift{5.200800in}{3.300000in}%
\pgfsys@useobject{currentmarker}{}%
\end{pgfscope}%
\end{pgfscope}%
\begin{pgfscope}%
\definecolor{textcolor}{rgb}{0.000000,0.000000,0.000000}%
\pgfsetstrokecolor{textcolor}%
\pgfsetfillcolor{textcolor}%
\pgftext[x=5.298022in,y=3.251775in,left,base]{\color{textcolor}\rmfamily\fontsize{10.000000}{12.000000}\selectfont \(\displaystyle 30\)}%
\end{pgfscope}%
\begin{pgfscope}%
\pgfsetbuttcap%
\pgfsetroundjoin%
\definecolor{currentfill}{rgb}{0.000000,0.000000,0.000000}%
\pgfsetfillcolor{currentfill}%
\pgfsetlinewidth{0.803000pt}%
\definecolor{currentstroke}{rgb}{0.000000,0.000000,0.000000}%
\pgfsetstrokecolor{currentstroke}%
\pgfsetdash{}{0pt}%
\pgfsys@defobject{currentmarker}{\pgfqpoint{0.000000in}{0.000000in}}{\pgfqpoint{0.048611in}{0.000000in}}{%
\pgfpathmoveto{\pgfqpoint{0.000000in}{0.000000in}}%
\pgfpathlineto{\pgfqpoint{0.048611in}{0.000000in}}%
\pgfusepath{stroke,fill}%
}%
\begin{pgfscope}%
\pgfsys@transformshift{5.200800in}{3.762000in}%
\pgfsys@useobject{currentmarker}{}%
\end{pgfscope}%
\end{pgfscope}%
\begin{pgfscope}%
\definecolor{textcolor}{rgb}{0.000000,0.000000,0.000000}%
\pgfsetstrokecolor{textcolor}%
\pgfsetfillcolor{textcolor}%
\pgftext[x=5.298022in,y=3.713775in,left,base]{\color{textcolor}\rmfamily\fontsize{10.000000}{12.000000}\selectfont \(\displaystyle 35\)}%
\end{pgfscope}%
\begin{pgfscope}%
\pgfsetbuttcap%
\pgfsetroundjoin%
\definecolor{currentfill}{rgb}{0.000000,0.000000,0.000000}%
\pgfsetfillcolor{currentfill}%
\pgfsetlinewidth{0.803000pt}%
\definecolor{currentstroke}{rgb}{0.000000,0.000000,0.000000}%
\pgfsetstrokecolor{currentstroke}%
\pgfsetdash{}{0pt}%
\pgfsys@defobject{currentmarker}{\pgfqpoint{0.000000in}{0.000000in}}{\pgfqpoint{0.048611in}{0.000000in}}{%
\pgfpathmoveto{\pgfqpoint{0.000000in}{0.000000in}}%
\pgfpathlineto{\pgfqpoint{0.048611in}{0.000000in}}%
\pgfusepath{stroke,fill}%
}%
\begin{pgfscope}%
\pgfsys@transformshift{5.200800in}{4.224000in}%
\pgfsys@useobject{currentmarker}{}%
\end{pgfscope}%
\end{pgfscope}%
\begin{pgfscope}%
\definecolor{textcolor}{rgb}{0.000000,0.000000,0.000000}%
\pgfsetstrokecolor{textcolor}%
\pgfsetfillcolor{textcolor}%
\pgftext[x=5.298022in,y=4.175775in,left,base]{\color{textcolor}\rmfamily\fontsize{10.000000}{12.000000}\selectfont \(\displaystyle 40\)}%
\end{pgfscope}%
\begin{pgfscope}%
\definecolor{textcolor}{rgb}{0.000000,0.000000,0.000000}%
\pgfsetstrokecolor{textcolor}%
\pgfsetfillcolor{textcolor}%
\pgftext[x=5.492467in,y=2.376000in,,top]{\color{textcolor}\rmfamily\fontsize{10.000000}{12.000000}\selectfont \(\displaystyle \frac{\chi^2}{\texttt{dof}}\)}%
\end{pgfscope}%
\begin{pgfscope}%
\pgfsetbuttcap%
\pgfsetmiterjoin%
\pgfsetlinewidth{0.803000pt}%
\definecolor{currentstroke}{rgb}{0.000000,0.000000,0.000000}%
\pgfsetstrokecolor{currentstroke}%
\pgfsetdash{}{0pt}%
\pgfpathmoveto{\pgfqpoint{5.016000in}{0.528000in}}%
\pgfpathlineto{\pgfqpoint{5.016000in}{0.620400in}}%
\pgfpathlineto{\pgfqpoint{5.016000in}{4.131600in}}%
\pgfpathlineto{\pgfqpoint{5.016000in}{4.224000in}}%
\pgfpathlineto{\pgfqpoint{5.200800in}{4.224000in}}%
\pgfpathlineto{\pgfqpoint{5.200800in}{4.131600in}}%
\pgfpathlineto{\pgfqpoint{5.200800in}{0.620400in}}%
\pgfpathlineto{\pgfqpoint{5.200800in}{0.528000in}}%
\pgfpathclose%
\pgfusepath{stroke}%
\end{pgfscope}%
\end{pgfpicture}%
\makeatother%
\endgroup%
}
           \caption{CP1c Unweighted Contour for LinearNoErrors.txt}
           \label{fig:CP1c_Contour_Plot}
        \end{center}
    \end{figure}
    
    \noindent
    The values for the unweighted fit are $\sim0$ for most values considered and $\sim1$ for 
    two small sections in the top right and bottom left. The ideal value for 
    $\frac{\chi^2}{\texttt{dof}}$ is 1 and so we can see that an unweighted fit gives us a 
    much more accurate fit, but in reality data without errors is impossible and so all fits 
    should be weighted. Thus, when fitting a line to some data, it is vital to properly consider 
    the uncertainties of everything, including all the parameters and the correlations between 
    them, in order to be confident that the results are the actual results, even if the so-called 
    "goodness" factor is less than ideal. 
    
    \newpage
    \section{Appendix}
    \lstinputlisting[caption=CP2a\_Nonlinear\_Fitting.py]{CP2a_Nonlinear_Fitting.py}
    \lstinputlisting[caption=CP2b\_Visualising\_Uncertainties.py]{CP2b_Visualising_Uncertainties.py}
    \lstinputlisting[caption=CP1c.py]{CP1c.py}

\end{document}