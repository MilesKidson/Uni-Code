\documentclass[12pt]{article}
\usepackage[margin=1.2in]{geometry}
\usepackage[all]{nowidow}
\usepackage[hyperfigures=true, hidelinks, pdfhighlight=/N]{hyperref}
\usepackage[separate-uncertainty=true,group-digits=false]{siunitx}
\usepackage{graphicx,amsmath,physics,tabto,float,amssymb,pgfplots,verbatim,tcolorbox}
\usepackage{listings,xcolor,subfig,keyval2e,caption,import}
\numberwithin{equation}{section}
\numberwithin{figure}{section}
\definecolor{stringcolor}{HTML}{C792EA}
\definecolor{codeblue}{HTML}{2162DB}
\definecolor{commentcolor}{HTML}{4A6E46}
\lstdefinestyle{appendix}{
    basicstyle=\ttfamily\footnotesize,commentstyle=\color{commentcolor},keywordstyle=\color{codeblue},
    stringstyle=\color{stringcolor},showstringspaces=false,numbers=left,upquote=true,captionpos=t,
    abovecaptionskip=12pt,belowcaptionskip=12pt,language=Python,breaklines=true,frame=single}
\lstdefinestyle{inline}{
    basicstyle=\ttfamily\footnotesize,commentstyle=\color{commentcolor},keywordstyle=\color{codeblue},
    stringstyle=\color{stringcolor},showstringspaces=false,numbers=left,upquote=true,frame=tb,
    captionpos=b,language=Python}
\renewcommand{\lstlistingname}{Appendix}
\pgfplotsset{compat=1.17}

\title{Skin Depth in Conductors}
\author{KDSMIL001 \; PHY2004W: PHYLAB 2}
\date{\textbf{13 October 2020}}

\begin{document}
    \begin{titlepage}
        \maketitle
        \center
        \tableofcontents
    \end{titlepage}
    
    \section{Introduction}\label{sec:Introduction}
    When an alternating current is travelling on a conductor a phenomenon known as 
    Skin Effect is observed, where it appears that the only part of the conductor that is 
    actually conducting the current is the outermost part, or the ``skin". This extends to 
    conductors present in time-varying magnetic fields, where we find this relation
    \begin{equation}
        B(z)=B(0)e^{-\frac{z}{\delta}}
        \label{eqn:SkinDepthSheet}
    \end{equation}
    for a uniform alternating magnetic field at the surface of an infinite sheet of conductor. 
    In this case $z$ is the distance from the conductor surface and $\delta$ is a length known 
    as the skin depth, given by $\delta=\sqrt{\frac{2}{\mu\sigma\omega}}$ where $\mu$ is the 
    permeability of the conductor, $\sigma$ is the conductivity, $\omega$ is the angular 
    frequency of the field $B$, and $B(0)$ is the amplitude of the field without any conductor 
    in place. In this experiment we are using a copper tube so we have cylindrical symmetry, 
    resulting in the relation
    \begin{equation}
        B(d)=B(0)\frac{1}{\sqrt{1+(Rd/\delta^2)^2}}
        \label{eqn:SkinDepthCylinder}
    \end{equation}
    where $R$ is the inner radius of the tube and $R+d$ is the outer radius. This $B(d)$ represents 
    the magnetic field present within a tube of wall thickness $d$. It's important to note here 
    that the original field $B(0)$ is taken as being uniform, as well as the resultant field $B(d)$. 
    This means that measuring the field at any point within the tube will return the same result. 

    \section{Aim}\label{sec:Aim}
    The aim of this experiment is to investigate the shielding effect of a copper tube in a 
    time-varying magnetic field and to find the conductivity of the copper tube using 
    \autoref{eqn:SkinDepthCylinder} by varying the frequency of the magnetic field and fitting 
    the equation to the experimental data. 

    \section{Apparatus}\label{sec:Apparatus}
    All measurement devices used have an uncertainty rating of 2\%.
    \begin{itemize}
        \item Signal generator producing a sinusoidal AC current
        \item Audio amplifier
        \item Ammeter
        \item Helmholtz coil with 80 winds on each coil
        \item Axial search coil with 200 winds
        \item Oscilloscope 
        \item Copper tube with inner radius $R=\SI{20\pm0.1e-3}{\metre}$ and wall thickness 
        $d=\SI{1\pm0.1e-3}{\metre}.$
    \end{itemize}
    In order to produce the time-varying magnetic field we used a signal generator feeding into 
    an audio amplifier, to control the amplitude of the current, which in turn fed into a 
    Helmholtz coil, which is two identical coils connected in series, separated by a gap. This 
    is used in order to create a more stable magnetic field than if we were to use just one. 
    On this circuit is an ammeter connected in series in order to monitor the amplitude of the 
    current (in rms) to make sure it doesn't exceed 1 A or else the coils would heat up and 
    distort the sinusoidal current. \newline
    Along with this we had an axial search coil, aligned in the same way as the two larger coils, 
    in the middle of the two and connected to an oscilloscope to monitor the emf induced in the 
    axial coil. Lastly we had a copper tube that would fit over the axial coil.

    \section{Method}\label{sec:Method}
    With a sinusoidal AC current being supplied by the signal generator and the amplitude of this 
    current being kept below 1 A by adjusting the audio amplifier, we varied the frequency 
    of the supplied signal from 100 Hz to 5 kHz, measuring the induced voltage at each frequency 
    with and without the copper tube around the axial coil. It is vital that the source is AC 
    as a DC current in the coil will produce a magnetic field, but it won't be time-varying. The 
    time-varying nature of the magnetic field is crucial as, by Faraday's Law, the emf induced 
    in a conductor by a magnetic field is proportional to the time derivative of the magnetic 
    field. With no time-variance, no emf is induced and so no field can be set up in the shielding 
    material to oppose the original field. Looking at \autoref{eqn:SkinDepthCylinder}, 
    we can see that the ratio between the reduced magnetic field $B(d)$ and the magnetic field 
    without any shielding $B(0)$ should give us
    \begin{equation}
        \frac{B(d)}{B(0)}=\frac{1}{\sqrt{1+(Rd/\frac{2}{\mu\sigma\omega})^2}}
        \label{eqn:TheoreticalModel}
    \end{equation}
    Taking our measurements of the induced voltage with and without shielding and finding their 
    ratio should give us the same result. This is a consequence of Faraday's Law which says 
    \begin{equation}
        \epsilon = -N_a \frac{d}{dt}\int \vec{B}\cdot d\vec{A}\approx-N_a A \frac{dB}{dt}
        \label{eqn:FaradaysLaw}
    \end{equation}
    where $N_a$ is the number of winds in a search coil and $A$ is the cross-sectional area of 
    that coil. Since we're interested in amplitudes and the amplitude of our supplied $B$ does 
    not change with respect to time, we find the relation 
    \begin{equation}
        |\epsilon|\approx N_a A |B|
        \label{eqn:emfToB}
    \end{equation}
    for the amplitudes of induced emf and magnetic field. From this we find 
    \begin{equation*}
        \frac{B(d)}{B(0)}\approx \frac{N_a A \epsilon(d)}{N_a A \epsilon(0)}=\frac{\epsilon(d)}{\epsilon(0)}
    \end{equation*}
    so we did not need to do any conversions, we only needed to compare induced emfs. Plotting the 
    ratio of emfs above against angular frequency $\omega$, we then used the value for the 
    permeability of copper, $\mu=\SI{1.256629e-6}{\henry/\metre}$, and used 
    \texttt{scipy.optimize.curve\_fit} to fit \autoref{eqn:TheoreticalModel} to the plotted data, 
    thus finding the only unknown parameter $\sigma$, the conductivity. To determine uncertainties 
    we used the Jackknife Monte Carlo method of determining uncertainties of fitted parameters. 
    The uncertainty on our data points is found from the 2\% uncertainty of the measurement device, 
    propagated using the following formula. 
    \begin{equation*}
        u\left(\frac{x}{y}\right)=\left|\frac{x}{y}\right|\sqrt{\left(\frac{u(x)}{x}\right)^2+\left(\frac{u(y)}{y}\right)^2}
    \end{equation*}

    \section{Results}\label{sec:Results}
    \begin{figure}[H]
        \begin{center}
           \scalebox{.7}{\subimport{Data}{SkinEffect.pgf}}
           \caption{Plot of the ratio between shielded emf and non-shielded emf induced in an axial 
           search coil placed within a Helmholtz coil, against the angular frequency of the magnetic 
           field inducing the emf. The theoretical model is fitted to the data with 
           \texttt{curve\_fit} to find the conductivity of the copper tube shielding the axial coil.}
           \label{fig:SkinEffect}
        \end{center}
    \end{figure}
    The fitting to the data shown in \autoref{fig:SkinEffect} gave us a value for the 
    conductivity of copper:
    \begin{equation*}
        \sigma=\num{1958.9\pm1.2e4}
    \end{equation*}
    
    \section{Discussion and Recommendations}\label{sec:DiscussionRecommendations}
    Thanks to the magic of Google, we are able to find the agreed upon value for the conductivity 
    of copper:
    \begin{equation*}
        \sigma=\num{5.96e7}
    \end{equation*}
    Comparing this value to ours, we see that they are within an order of magnitude of each other, 
    but our value does not agree with the actual value within experimental uncertainty. There 
    are a few possible reasons for this discrepancy. We only took 23 data points and this isn't 
    really an appropriate amount of data to have when trying to accurately determine a property 
    of a material such as conductivity. A possible source of uncertainty could be the measurement 
    of the thickness of the copper tube. It could be that our measurement was off by a significant 
    amount, which would likely affect the result. The uncertainty on the data points in \autoref{fig:SkinEffect} 
    is also quite large at some points. This could be reduced with more accurate equipment, although 
    it would likely not solve the problem of our $\sigma$ being roughly a third of the expected 
    value. 

    \section{Conclusion}\label{sec:Conclusion}
    To conclude, we found a value for $\sigma$, the conductivity of copper, by examining the 
    shielding effect of a copper tube on a roughly uniform time-varying magnetic field and 
    fitting an appropriate equation to data. The value we found does not agree with the 
    expected value but we have outlined some possible reasons for this. 
    
    \newpage
    \section{Appendix}\label{sec:Appendix}
    \setcounter{figure}{0} \renewcommand{\thefigure}{A.\arabic{figure}}
    \lstinputlisting[caption=Code for determining $\sigma$ and plotting the data,label=app:SkinEffect,style=appendix]{Data/SkinEffect.py}
    

\end{document}