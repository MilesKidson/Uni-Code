\documentclass[12pt]{article}
\usepackage[margin=1.2in]{geometry}
\usepackage[all]{nowidow}
\usepackage[hyperfigures=true, hidelinks, pdfhighlight=/N]{hyperref}
\usepackage{graphicx,amsmath,physics,tabto,float,amssymb,pgfplots,verbatim,tcolorbox}
\usepackage{listings,xcolor,siunitx,subfig,keyval2e,caption}
\numberwithin{equation}{section}
\definecolor{stringcolor}{HTML}{C792EA}
\definecolor{codeblue}{HTML}{2162DB}
\definecolor{commentcolor}{HTML}{4A6E46}
\lstdefinestyle{appendix}{
    basicstyle=\ttfamily\footnotesize,commentstyle=\color{commentcolor},keywordstyle=\color{codeblue},
    stringstyle=\color{stringcolor},showstringspaces=false,numbers=left,upquote=true,captionpos=t,
    abovecaptionskip=12pt,belowcaptionskip=12pt,language=Python,breaklines=true,frame=single}
\lstdefinestyle{inline}{
    basicstyle=\ttfamily\footnotesize,commentstyle=\color{commentcolor},keywordstyle=\color{codeblue},
    stringstyle=\color{stringcolor},showstringspaces=false,numbers=left,upquote=true,frame=tb,
    captionpos=b,language=Python}
\renewcommand{\lstlistingname}{Appendix}
\pgfplotsset{compat=1.17}
\title{Assignment 1}

\begin{document}
    \maketitle
    \begin{center}
    \textbf{\large{MAM2000W 2IA}}\hspace{25pt}
    \textbf{\large{KDSMIL001}}
    \end{center}
    
    \begin{enumerate}
        \item Let $P_n$ be the statement $3|n^3-4n$ for all $n\in\mathbb{N}$. Another way of saying this is $n^3-4n=3t$ where $t\in\mathbb{Z}$.
        \newline 
        \newline
        \textbf{Base Case:} 
        \begin{equation*}
            P_n: 1^3-4(1)=1-4=-3
        \end{equation*}
        It's simple to see that $3|-3$, so our base case is true. 
        \newline
        \newline
        \textbf{Inductive Hypothesis:}\newline
        We assume that the statement $P_k$ is true, that is 
        \begin{align*}
            &3|k^3-4k \\
            \implies &k^3-4k = 3p, p\in\mathbb{Z}
        \end{align*}
        We now look at the case $P_{k+1}$:
        \begin{align*}
            (k+1)^3-4(k+1)&=k^3+3k^2+3k+1-4k-4 \\
            &=(k^3-4k)+(3k^2+3k-3)
        \end{align*}
        By the Inductive Hypothesis, we know that $k^3-4k=3p$. We can also easily see that 
        \begin{equation*}
            3k^2+3k-3=3(k^2+k-1)
        \end{equation*}
        We know that $k$ is a natural number, so $k^2+k-1$ is an integer and therefore we can say $3k^2+3k-3=3q, q\in\mathbb{Z}$. Finally we can say
        \begin{equation*}
            (k^3-4k)+(3k^2+3k-3)=3p+3q=3(p+q)
        \end{equation*}
        The sum of two integers is an integer and so we can say that $P_{k+1}$ is true and by the principle of mathematical induction $P_n$ is true for all $n\in\mathbb{N}$.
        \newline 
        \begin{flushright}$\square$\end{flushright}

    \end{enumerate}
\end{document}