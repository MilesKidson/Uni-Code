\documentclass[12pt]{article}
\usepackage[margin=1.2in]{geometry}
\usepackage[all]{nowidow}
\usepackage[hyperfigures=true, hidelinks, pdfhighlight=/N]{hyperref}
\usepackage{graphicx,amsmath,physics,tabto,float,amssymb,pgfplots,verbatim,tcolorbox}
\usepackage{listings,xcolor,siunitx,subfig,keyval2e,caption,import}
\numberwithin{equation}{section}
\numberwithin{figure}{section}
\definecolor{stringcolor}{HTML}{C792EA}
\definecolor{codeblue}{HTML}{2162DB}
\definecolor{commentcolor}{HTML}{4A6E46}
\lstdefinestyle{appendix}{
    basicstyle=\ttfamily\footnotesize,commentstyle=\color{commentcolor},keywordstyle=\color{codeblue},
    stringstyle=\color{stringcolor},showstringspaces=false,numbers=left,upquote=true,captionpos=t,
    abovecaptionskip=12pt,belowcaptionskip=12pt,language=Python,breaklines=true,frame=single}
\lstdefinestyle{inline}{
    basicstyle=\ttfamily\footnotesize,commentstyle=\color{commentcolor},keywordstyle=\color{codeblue},
    stringstyle=\color{stringcolor},showstringspaces=false,numbers=left,upquote=true,frame=tb,
    captionpos=b,language=Python}
\renewcommand{\lstlistingname}{Appendix}
\pgfplotsset{compat=1.17}

\title{IA Assignment 2}
\date{\textbf{24 August 2020}}
\author{}

\begin{document}

    \maketitle
    \begin{center}
    \textbf{\large{MAM2000W 2IA}}\hspace{25pt}
    \textbf{\large{KDSMIL001}}
    \end{center}

    \begin{enumerate}
        \item By the fact that gcd($a,b$)=1, we know that $a$ and $b$ are coprime. That means we can 
        write them as their prime factorisations
        \begin{equation*}
            a=p_1^{a_1}p_2^{a_2}\dots p_r^{a_r},\;b=p_1^{b_1}p_2^{b_2}\dots p_r^{b_r}
        \end{equation*}
        where $p_i$ are the prime numbers, $a_i,b_i\geq 0$, and 
        \begin{align*}
            a_i\neq0&\implies b_i=0 \\
            b_i\neq0&\implies a_i=0
        \end{align*}
        What we mean by this is that any prime factor $p_i$ of $a$ with $a_i\geq1$ is not a prime factor 
        of $b$, and vice versa. \newline
        Now we can write $c$ as its prime factorisation, as well as $c^2$:
        \begin{equation*}
            c=p_1^{c_1}p_2^{c_2}\dots p_r^{c_r}\implies c^2=p_1^{2c_1}p_2^{2c_2}\dots p_r^{2c_r}
        \end{equation*}
        But we are given that $c^2=ab$, so we can write 
        \begin{equation*}
            c^2=p_1^{a_1+b_1}p_2^{a_2+b_2}\dots p_r^{a_r+b_r}
        \end{equation*}
        and since $a$ and $b$ share no prime factors, all exponents in the equation above are either 
        $a_i$ or $b_i$ (or 0) but never $a_i+b_i$. Thus we can write 
        \begin{equation*}
            a_i=2c_i=2m_i,\; b_i=2c_i=2n_i
        \end{equation*}
        for $m_i,n_i\in\mathbb{Z},\; m_i,n_i\geq0$. Finally we can write 
        \begin{alignat*}{3}
            a&=p_1^{2m_1}p_2^{2m_2}\dots p_r^{2m_r}&&=(p_1^{m_1}p_2^{m_2}\dots p_r^{m_r})^2&&=m^2 \\
            b&=p_1^{2n_1}p_2^{2n_2}\dots p_r^{2n_r}&&=(p_1^{n_1}p_2^{n_2}\dots p_r^{n_r})^2&&=n^2
        \end{alignat*}
        for $m,n\in\mathbb{Z}^+$. Thus $a$ and $b$ are each squares.
        \newline
        \begin{flushright}$\square$\end{flushright}

    \end{enumerate}
\end{document}