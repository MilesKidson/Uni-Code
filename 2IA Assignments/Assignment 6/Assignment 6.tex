\documentclass[12pt]{article}
\usepackage[margin=1.2in]{geometry}
\usepackage[all]{nowidow}
\usepackage[hyperfigures=true, hidelinks, pdfhighlight=/N]{hyperref}
\usepackage[separate-uncertainty=true,group-digits=false]{siunitx}
\usepackage{graphicx,amsmath,physics,tabto,float,amssymb,pgfplots,verbatim,tcolorbox}
\usepackage{listings,xcolor,subfig,keyval2e,caption,import}
\numberwithin{equation}{section}
\numberwithin{figure}{section}
\definecolor{stringcolor}{HTML}{C792EA}
\definecolor{codeblue}{HTML}{2162DB}
\definecolor{commentcolor}{HTML}{4A6E46}
\lstdefinestyle{appendix}{
    basicstyle=\ttfamily\footnotesize,commentstyle=\color{commentcolor},keywordstyle=\color{codeblue},
    stringstyle=\color{stringcolor},showstringspaces=false,numbers=left,upquote=true,captionpos=t,
    abovecaptionskip=12pt,belowcaptionskip=12pt,language=Python,breaklines=true,frame=single}
\lstdefinestyle{inline}{
    basicstyle=\ttfamily\footnotesize,commentstyle=\color{commentcolor},keywordstyle=\color{codeblue},
    stringstyle=\color{stringcolor},showstringspaces=false,numbers=left,upquote=true,frame=tb,
    captionpos=b,language=Python}
\renewcommand{\lstlistingname}{Appendix}
\pgfplotsset{compat=1.17}

\title{Assignment 6}
\author{KDSMIL001 \; MAM2000W 2IA}
\date{\textbf{30 October 2020}}

\begin{document}
    \maketitle
    \begin{enumerate}
        \setcounter{enumi}{2}
        \item We can, in fact, drop the first condition. Provided $H$ is non-empty, we can choose some $h\in H$. 
        Since $H$ is closed under inversion we have that $h^{-1}\in H$. Since $H$ is also closed under the group operation, 
        we have that $e=hh^{-1}=h^{-1}h\in H$.

        \item For $H$ to be a subgroup of $G$, it must pass the subgroup test.
        \begin{itemize}
            \item Take the identity $e\in G$. $e^2=ee=e$ and thus $e$ satisfies the condition allowing $e\in H$. 
            \item If we take some $h_1, h_2\in H$, given that $G$ is abelian and $h_1,h_2\in G$, we have 
            \begin{equation}
                h_1h_2h_1h_2=h_1h_2h_2h_1=h_1eh_1=h_1h_1=e
            \end{equation}
            so $h_1h_2\in H$, i.e. $H$ is closed under the group operation of $G$.
            \item Take some $h\in H$. We have that $h^2=hh=e=hh^{-1}$ under $G$, and by left cancellation law, we have that $h=h^{-1}$, 
            so $h^{-1}\in H$.
        \end{itemize}
        Thus $H$ passes the subgroup test. 

        \item For $H\cap K$ to be a subgroup of $G$, it must pass the subgroup test.
        \begin{itemize}
            \item $H$ and $K$ are subgroups of $G$, so we know that if $e$ is the identity of $G$ then $e\in H$ and $e\in K$. 
            It is simple to see then that $e\in H\cap K$.
            \item We know that $H$ and $K$ are closed under the group operation, which we will call $*$. Thus, given some 
            $q_1,q_2\in H\cap K$, we know that $q_1,q_2\in H$ and $q_1,q_2\in K$. Since $H$ and $K$ are subgroups of $G$, they must be closed under $*$ and so it 
            must be that $q_1*q_2\in H$ and $q_1*q_2\in K$, and so clearly $q_1*q_2\in H\cap K$. So $*$ is closed under $H\cap K$. 
            \item Given some $q\in H\cap K$, we know then that $q\in H$ and $q\in K$. Since $H$ and $K$ are subgroups of $G$, they 
            must be closed under inversion, so $q^{-1}\in H$ and $q^{-1}\in K$. Clearly this means $q^{-1}\in H\cap K$, so $H\cap K$ is closed under inversion.
        \end{itemize}
        Thus $H\cap K$ passes the subgroup test. 
    \end{enumerate}

\end{document}