\documentclass[12pt]{article}
\usepackage[margin=1.2in]{geometry}
\usepackage[all]{nowidow}
\usepackage[hyperfigures=true, hidelinks, pdfhighlight=/N]{hyperref}
\usepackage{graphicx,amsmath,physics,tabto,float,amssymb,pgfplots,verbatim,tcolorbox}
\usepackage{listings,xcolor,siunitx,subfig,keyval2e,caption}
\definecolor{stringcolor}{HTML}{C792EA}
\definecolor{codeblue}{HTML}{2162DB}
\definecolor{commentcolor}{HTML}{4A6E46}
\lstdefinestyle{appendix}{
    basicstyle=\ttfamily\footnotesize,commentstyle=\color{commentcolor},keywordstyle=\color{codeblue},
    stringstyle=\color{stringcolor},showstringspaces=false,numbers=left,upquote=true,captionpos=t,
    abovecaptionskip=12pt,belowcaptionskip=12pt,language=Python,breaklines=true,frame=single}
\lstdefinestyle{inline}{
    basicstyle=\ttfamily\footnotesize,commentstyle=\color{commentcolor},keywordstyle=\color{codeblue},
    stringstyle=\color{stringcolor},showstringspaces=false,numbers=left,upquote=true,frame=tb,
    captionpos=b,language=Python}
\renewcommand{\lstlistingname}{Appendix}
\pgfplotsset{compat=1.17}

\title{Assignment 3}
\date{\textbf{5 June 2020}}
\author{}

\begin{document}

    \maketitle
    \begin{center}
    \textbf{\large{MAM2046W 2NA}}
    \textbf{\large{KDSMIL001}}
    \end{center}
    \vspace*{15pt}
    
    \begin{enumerate}
        \item \textbf{Deriving a second derivative approximation formula} \newline
        We are looking for an equation that approximates the second derivative and is of the form 
        \begin{equation}
            f''(x) = \lambda_1f(x)+\lambda_2f(x+h)+\lambda_3f(x-h)+\lambda_4f(x+2h)+\lambda_5f(x-2h)
            \label{eqn:1ToSolve}
        \end{equation}
        We can start by finding the Taylor expansions of each function
        \begin{align*}
            f(x) &= f(x) \\
            f(x+h) &= f(x)+hf'(x)+\frac{h^2}{2!}f''(x)+\frac{h^3}{3!}f'''(x)+\frac{h^4}{4!}f^{(4)}(x)+\mathcal{O}(h^5) \\
            f(x-h) &= f(x)-hf'(x)+\frac{h^2}{2!}f''(x)-\frac{h^3}{3!}f'''(x)+\frac{h^4}{4!}f^{(4)}(x)-\mathcal{O}(h^5) \\
            f(x+2h) &= f(x)+2hf'(x)+2hf''(x)+\frac{8h^3}{3!}f'''(x)+\frac{16h^4}{4!}f^{(4)}(x)+\mathcal{O}(h^5) \\
            f(x-2h) &= f(x)-2hf'(x)+2hf''(x)-\frac{8h^3}{3!}f'''(x)+\frac{16h^4}{4!}f^{(4)}(x)-\mathcal{O}(h^5) 
        \end{align*}
        and then, after collecting coefficients of derivatives of f, \autoref{eqn:1ToSolve} looks like 
        this, ignoring the error term
        \begin{equation}
            \begin{gathered}
                f''(x)=(\lambda_1+\lambda_2+\lambda_3+\lambda_4+\lambda_5)f(x)+(\lambda_2-\lambda_3+2\lambda_4-2\lambda_5)hf'(x) \\
                +(\lambda_2+\lambda_3+4\lambda_4+4\lambda_5)h^2f''(x)+(\lambda_2-\lambda_3+8\lambda_4-8\lambda_5)h^3f'''(x) \\
                +(\lambda_2+\lambda_3+16\lambda_4+16\lambda_5)h^4f^{(4)}(x)
            \end{gathered}
            \label{eqn:1ToSolveExpanded}
        \end{equation}
        Now we can equate coefficients of derivatives on either side of the equation, which we can 
        turn into a matrix equation:
        \begin{equation*}
            \left(
            \begin{array}{ccccc}
                1 & 1 & 1 & 1 & 1 \\
                0 & 1 & -1 & 2 & -2 \\
                0 & 1 & 1 & 4 & 4 \\
                0 & 1 & -1 & 8 & -8 \\
                0 & 1 & 1 & 16 & 16 \\
            \end{array}
            \right)
            \left(
            \begin{array}{c}
                \lambda_1 \\
                \lambda_2 \\
                \lambda_3 \\
                \lambda_4 \\
                \lambda_5
            \end{array}
            \right)
            =
            \left(
            \begin{array}{c}
                0 \\
                0 \\
                \frac{2}{h^2} \\
                0 \\
                0 \\
            \end{array}
            \right)
        \end{equation*}
        Solving this equation (finding the inverse of the $5\times5$ matrix) gives us values for 
        our $\lambda$'s, which are:
        \begin{equation*}
            \lambda_1=\frac{-30}{12h^2};\hspace{5pt} \lambda_2=\frac{16}{12h^2};\hspace{5pt} 
            \lambda_3=\frac{16}{12h^2};\hspace{5pt} \lambda_4=\frac{-1}{12h^2};\hspace{5pt}
            \lambda_5=\frac{-1}{12h^2}
        \end{equation*}
        Finally, we have \autoref{eqn:1ToSolve} becoming
        \begin{align}
            f''(x) &= \frac{-30f(x)+16f(x+h)+16f(x-h)-f(x+2h)-f(x-2h)}{12h^2} \nonumber \\ 
            &= \frac{-f(x-2h)+16f(x-h)-30f(x)+16f(x+h)-f(x+2h)}{12h^2}
        \end{align}
        \label{eqn:1Solved}

        \item \textbf{Evaluating an integral with the Composite Trapezoid Method} \newline
        We aim to evaluate the following integral using the composite trapezoid method, using n 
        equal subintervals
        \begin{equation}
            I = \int_a^b e^xdx
            \label{eqn:2ToEvaluate}
        \end{equation}
        Firstly, we can look at the general form of the trapezoid method for approximating an 
        integral:
        \begin{equation}
            \int_a^b f(x)dx = \underbrace{\frac{h}{2}\sum_{i=0}^{n-1}[f(x_i)+f(x_{i+1})]}_\text{approximation}
            -\underbrace{\frac{h^2}{12}(b-a)f''(c)}_\text{error}
            \label{eqn:TrapezoidRuleGeneralForm}
        \end{equation}
        where $n$ is the number of equal subintervals between $a$ and $b$, $h=(a-b)/n$, and $c\in[a,b]$.
        $c$ is arbitrary but we choose it to be the point which maximises $f''(x)$ on the given 
        interval in order to look at the "worst case scenario", if you will. Applying \autoref{eqn:TrapezoidRuleGeneralForm} 
        to \autoref{eqn:2ToEvaluate}, and ignoring the error term for now, we get 
        \begin{align*}
            I &\approx \frac{h}{2}\sum_{i=0}^{n-1}[e^{x_i}+e^{x_{i+1}}] \\
            &= \frac{h}{2}\left[\sum_{i=0}^{n-1}e^{x_i}+\sum_{i=0}^{n-1}e^{x_{i+1}}\right]
        \end{align*}
        These sums can be converted into 2 geometric series as each has a common ratio of $e^h$, therefore 
        we get
        \begin{equation}
            I \approx \frac{h(1-e^{hn})}{2(1-e^h)}[e^a+e^{a+h}]
            \label{eqn:2SolvedNoError}
        \end{equation}
        Of course, this is only an approximation as we haven't included the error term yet. In this 
        error term though, there is $f''(c)$, where $c$ is the the point which maximises $f''$. In this 
        case, $c=b$ as $f''(x)=e^x$ is strictly increasing everywhere. So finally we have 
        \begin{equation}
            I = \frac{h(1-e^{hn})}{2(1-e^h)}[e^a+e^{a+h}]-\frac{h^2}{12}(b-a)f''(b)
            \label{eqn:2Solved}
        \end{equation}

        \item \textbf{Implementing the Composite Midpoint Method} \newline
        The composite midpoint method has the general form of 
        \begin{equation}
            \int_a^b f(x)dx = \underbrace{h\sum_{i=0}^{n-1}f(w_i)}_\text{approximation}
            +\underbrace{\frac{h^2(b-a)}{24}f''(c)}_\text{error}
            \label{eqn:3MidpointGenForm}
        \end{equation}
        where $h=(b-a)/n$, $w_i=a+(i+\frac{1}{2})h$, and $c\in[a,b]$. Again, we choose c to be 
        the value which maximises $f''$, to get a worst case scenario for the error. We were to use 
        this method to estimate 
        \begin{equation}
            \int_0^1 \frac{4}{1+x^2}dx = \pi
            \label{eqn:3ToEvaluate}
        \end{equation}
        with $n=1000$. Below is our implementation of \autoref{eqn:3MidpointGenForm} applied to 
        \autoref{eqn:3ToEvaluate} in Python.
        \newline
        \lstinputlisting[title=Composite Midpoint Method, style=inline, linerange=1-15, firstnumber=1]{CompositeMidpoint.py}
        
        When we compute the value of the error term, using $c=1$ as that maximises the $f''$, 
        we get a value of $\pi = 3.1415928202531225$. The difference between this and the known 
        value of $\pi$ is $\approx\num{1.667e-07}$. This is quite good. 
    \end{enumerate}

\end{document}