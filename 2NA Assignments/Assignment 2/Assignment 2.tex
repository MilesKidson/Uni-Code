\documentclass[12pt]{article}
\usepackage[margin=1.2in]{geometry}
\usepackage[hyperfigures=true, hidelinks, pdfhighlight=/N]{hyperref}
\usepackage[all]{nowidow}
\usepackage{graphicx,amsmath,physics,tabto,float,amssymb,pgfplots,verbatim,tcolorbox}
\usepackage{listings,xcolor,siunitx,subfig,keyval2e,caption}
\definecolor{stringcolor}{HTML}{C792EA}
\definecolor{codeblue}{HTML}{2162DB}
\definecolor{commentcolor}{HTML}{4A6E46}
\lstdefinestyle{appendix}{
    basicstyle=\ttfamily\footnotesize,commentstyle=\color{commentcolor},keywordstyle=\color{codeblue},
    stringstyle=\color{stringcolor},showstringspaces=false,numbers=left,upquote=true,captionpos=t,
    abovecaptionskip=12pt,belowcaptionskip=12pt,language=Python,breaklines=true,frame=single}
\lstdefinestyle{inline}{
    basicstyle=\ttfamily\footnotesize,commentstyle=\color{commentcolor},keywordstyle=\color{codeblue},
    stringstyle=\color{stringcolor},showstringspaces=false,numbers=left,upquote=true,frame=tb,
    captionpos=b,language=Python}
\renewcommand{\lstlistingname}{Code}
\pgfplotsset{compat=1.17}

\title{Assignment 2}
\date{\textbf{10 May 2020}}
\author{}

\begin{document}

    \maketitle
    \begin{center}
    \textbf{\large{MAM2046W 2NA}}\\
    \textbf{\large{KDSMIL001}}\\
    \end{center}

    \begin{enumerate}
        \section*{Analytical Problems}
        \item \textbf{Interpolating $\sin(x)$}\newline
        \begin{enumerate}
            \item \textbf{Lagrange Method}
            We are trying to approximate the function $f(x) = \sin x$ using a Lagrange interpolating 
            polynomial. We have 4 nodes $0, \pi/6, \pi/3, \pi/2$ and their function values $0, 1/2, \sqrt{3}/2, 1$ 
            which we will use with the form of the Lagrange interpolating polynomial, which is given by

            \begin{equation}
                P_{n}(x) = \sum_{i=0}^{n} y_i L_i(x)
                \label{eqn:Lagrange Polynomial Form}
            \end{equation}
            where
            \begin{equation}
                L_{i,n} = \prod_{j=0, j\neq i}^{n} \frac{x-x_j}{x_i-x_j}
                \label{eqn:Li}
            \end{equation}

            From \autoref{eqn:Li} we can find

            \begin{equation*}
                \begin{split}
                    L_0(x) &= -\frac{36x^3}{\pi^3}+\frac{36x^2}{\pi^2}-\frac{11x}{\pi}+1 \\
                    L_1(x) &= \frac{108x^3}{\pi^3}-\frac{90x^2}{\pi^2}-\frac{18x}{\pi} \\
                    L_2(x) &= -\frac{108x^3}{\pi^3}+\frac{72x^2}{\pi^2}-\frac{9x}{\pi} \\
                    L_3(x) &= \frac{36x^3}{\pi^3}-\frac{18x^2}{\pi^2}+\frac{2x}{\pi} 
                \end{split}
            \end{equation*}
            
            Which, when combined with \autoref{eqn:Lagrange Polynomial Form} gives us 

            \begin{equation}
                P_3 = x^3 \left(\frac{90-54\sqrt{3}}{\pi^3}\right) + x^2 \left(\frac{-63+36\sqrt{3}}{\pi^2}\right) 
                + x \left( \frac{22-9\sqrt{3}}{2\pi}\right)
                \label{eqn:Lagrange Polynomial}
            \end{equation}

            \item \textbf{Plotting} \newline

            \begin{figure}[H]
                \begin{center}
                   \scalebox{.7}{%% Creator: Matplotlib, PGF backend
%%
%% To include the figure in your LaTeX document, write
%%   \input{<filename>.pgf}
%%
%% Make sure the required packages are loaded in your preamble
%%   \usepackage{pgf}
%%
%% Figures using additional raster images can only be included by \input if
%% they are in the same directory as the main LaTeX file. For loading figures
%% from other directories you can use the `import` package
%%   \usepackage{import}
%% and then include the figures with
%%   \import{<path to file>}{<filename>.pgf}
%%
%% Matplotlib used the following preamble
%%
\begingroup%
\makeatletter%
\begin{pgfpicture}%
\pgfpathrectangle{\pgfpointorigin}{\pgfqpoint{6.400000in}{4.800000in}}%
\pgfusepath{use as bounding box, clip}%
\begin{pgfscope}%
\pgfsetbuttcap%
\pgfsetmiterjoin%
\definecolor{currentfill}{rgb}{1.000000,1.000000,1.000000}%
\pgfsetfillcolor{currentfill}%
\pgfsetlinewidth{0.000000pt}%
\definecolor{currentstroke}{rgb}{1.000000,1.000000,1.000000}%
\pgfsetstrokecolor{currentstroke}%
\pgfsetdash{}{0pt}%
\pgfpathmoveto{\pgfqpoint{0.000000in}{0.000000in}}%
\pgfpathlineto{\pgfqpoint{6.400000in}{0.000000in}}%
\pgfpathlineto{\pgfqpoint{6.400000in}{4.800000in}}%
\pgfpathlineto{\pgfqpoint{0.000000in}{4.800000in}}%
\pgfpathclose%
\pgfusepath{fill}%
\end{pgfscope}%
\begin{pgfscope}%
\pgfsetbuttcap%
\pgfsetmiterjoin%
\definecolor{currentfill}{rgb}{1.000000,1.000000,1.000000}%
\pgfsetfillcolor{currentfill}%
\pgfsetlinewidth{0.000000pt}%
\definecolor{currentstroke}{rgb}{0.000000,0.000000,0.000000}%
\pgfsetstrokecolor{currentstroke}%
\pgfsetstrokeopacity{0.000000}%
\pgfsetdash{}{0pt}%
\pgfpathmoveto{\pgfqpoint{0.800000in}{0.528000in}}%
\pgfpathlineto{\pgfqpoint{5.760000in}{0.528000in}}%
\pgfpathlineto{\pgfqpoint{5.760000in}{4.224000in}}%
\pgfpathlineto{\pgfqpoint{0.800000in}{4.224000in}}%
\pgfpathclose%
\pgfusepath{fill}%
\end{pgfscope}%
\begin{pgfscope}%
\pgfpathrectangle{\pgfqpoint{0.800000in}{0.528000in}}{\pgfqpoint{4.960000in}{3.696000in}}%
\pgfusepath{clip}%
\pgfsetbuttcap%
\pgfsetroundjoin%
\pgfsetlinewidth{0.803000pt}%
\definecolor{currentstroke}{rgb}{0.800000,0.800000,0.800000}%
\pgfsetstrokecolor{currentstroke}%
\pgfsetdash{{0.800000pt}{1.320000pt}}{0.000000pt}%
\pgfpathmoveto{\pgfqpoint{1.236851in}{0.528000in}}%
\pgfpathlineto{\pgfqpoint{1.236851in}{4.224000in}}%
\pgfusepath{stroke}%
\end{pgfscope}%
\begin{pgfscope}%
\pgfsetbuttcap%
\pgfsetroundjoin%
\definecolor{currentfill}{rgb}{0.000000,0.000000,0.000000}%
\pgfsetfillcolor{currentfill}%
\pgfsetlinewidth{0.803000pt}%
\definecolor{currentstroke}{rgb}{0.000000,0.000000,0.000000}%
\pgfsetstrokecolor{currentstroke}%
\pgfsetdash{}{0pt}%
\pgfsys@defobject{currentmarker}{\pgfqpoint{0.000000in}{-0.048611in}}{\pgfqpoint{0.000000in}{0.000000in}}{%
\pgfpathmoveto{\pgfqpoint{0.000000in}{0.000000in}}%
\pgfpathlineto{\pgfqpoint{0.000000in}{-0.048611in}}%
\pgfusepath{stroke,fill}%
}%
\begin{pgfscope}%
\pgfsys@transformshift{1.236851in}{0.528000in}%
\pgfsys@useobject{currentmarker}{}%
\end{pgfscope}%
\end{pgfscope}%
\begin{pgfscope}%
\definecolor{textcolor}{rgb}{0.000000,0.000000,0.000000}%
\pgfsetstrokecolor{textcolor}%
\pgfsetfillcolor{textcolor}%
\pgftext[x=1.236851in,y=0.430778in,,top]{\color{textcolor}\rmfamily\fontsize{10.000000}{12.000000}\selectfont \(\displaystyle 0.0\)}%
\end{pgfscope}%
\begin{pgfscope}%
\pgfpathrectangle{\pgfqpoint{0.800000in}{0.528000in}}{\pgfqpoint{4.960000in}{3.696000in}}%
\pgfusepath{clip}%
\pgfsetbuttcap%
\pgfsetroundjoin%
\pgfsetlinewidth{0.803000pt}%
\definecolor{currentstroke}{rgb}{0.800000,0.800000,0.800000}%
\pgfsetstrokecolor{currentstroke}%
\pgfsetdash{{0.800000pt}{1.320000pt}}{0.000000pt}%
\pgfpathmoveto{\pgfqpoint{2.328979in}{0.528000in}}%
\pgfpathlineto{\pgfqpoint{2.328979in}{4.224000in}}%
\pgfusepath{stroke}%
\end{pgfscope}%
\begin{pgfscope}%
\pgfsetbuttcap%
\pgfsetroundjoin%
\definecolor{currentfill}{rgb}{0.000000,0.000000,0.000000}%
\pgfsetfillcolor{currentfill}%
\pgfsetlinewidth{0.803000pt}%
\definecolor{currentstroke}{rgb}{0.000000,0.000000,0.000000}%
\pgfsetstrokecolor{currentstroke}%
\pgfsetdash{}{0pt}%
\pgfsys@defobject{currentmarker}{\pgfqpoint{0.000000in}{-0.048611in}}{\pgfqpoint{0.000000in}{0.000000in}}{%
\pgfpathmoveto{\pgfqpoint{0.000000in}{0.000000in}}%
\pgfpathlineto{\pgfqpoint{0.000000in}{-0.048611in}}%
\pgfusepath{stroke,fill}%
}%
\begin{pgfscope}%
\pgfsys@transformshift{2.328979in}{0.528000in}%
\pgfsys@useobject{currentmarker}{}%
\end{pgfscope}%
\end{pgfscope}%
\begin{pgfscope}%
\definecolor{textcolor}{rgb}{0.000000,0.000000,0.000000}%
\pgfsetstrokecolor{textcolor}%
\pgfsetfillcolor{textcolor}%
\pgftext[x=2.328979in,y=0.430778in,,top]{\color{textcolor}\rmfamily\fontsize{10.000000}{12.000000}\selectfont \(\displaystyle 0.5\)}%
\end{pgfscope}%
\begin{pgfscope}%
\pgfpathrectangle{\pgfqpoint{0.800000in}{0.528000in}}{\pgfqpoint{4.960000in}{3.696000in}}%
\pgfusepath{clip}%
\pgfsetbuttcap%
\pgfsetroundjoin%
\pgfsetlinewidth{0.803000pt}%
\definecolor{currentstroke}{rgb}{0.800000,0.800000,0.800000}%
\pgfsetstrokecolor{currentstroke}%
\pgfsetdash{{0.800000pt}{1.320000pt}}{0.000000pt}%
\pgfpathmoveto{\pgfqpoint{3.421107in}{0.528000in}}%
\pgfpathlineto{\pgfqpoint{3.421107in}{4.224000in}}%
\pgfusepath{stroke}%
\end{pgfscope}%
\begin{pgfscope}%
\pgfsetbuttcap%
\pgfsetroundjoin%
\definecolor{currentfill}{rgb}{0.000000,0.000000,0.000000}%
\pgfsetfillcolor{currentfill}%
\pgfsetlinewidth{0.803000pt}%
\definecolor{currentstroke}{rgb}{0.000000,0.000000,0.000000}%
\pgfsetstrokecolor{currentstroke}%
\pgfsetdash{}{0pt}%
\pgfsys@defobject{currentmarker}{\pgfqpoint{0.000000in}{-0.048611in}}{\pgfqpoint{0.000000in}{0.000000in}}{%
\pgfpathmoveto{\pgfqpoint{0.000000in}{0.000000in}}%
\pgfpathlineto{\pgfqpoint{0.000000in}{-0.048611in}}%
\pgfusepath{stroke,fill}%
}%
\begin{pgfscope}%
\pgfsys@transformshift{3.421107in}{0.528000in}%
\pgfsys@useobject{currentmarker}{}%
\end{pgfscope}%
\end{pgfscope}%
\begin{pgfscope}%
\definecolor{textcolor}{rgb}{0.000000,0.000000,0.000000}%
\pgfsetstrokecolor{textcolor}%
\pgfsetfillcolor{textcolor}%
\pgftext[x=3.421107in,y=0.430778in,,top]{\color{textcolor}\rmfamily\fontsize{10.000000}{12.000000}\selectfont \(\displaystyle 1.0\)}%
\end{pgfscope}%
\begin{pgfscope}%
\pgfpathrectangle{\pgfqpoint{0.800000in}{0.528000in}}{\pgfqpoint{4.960000in}{3.696000in}}%
\pgfusepath{clip}%
\pgfsetbuttcap%
\pgfsetroundjoin%
\pgfsetlinewidth{0.803000pt}%
\definecolor{currentstroke}{rgb}{0.800000,0.800000,0.800000}%
\pgfsetstrokecolor{currentstroke}%
\pgfsetdash{{0.800000pt}{1.320000pt}}{0.000000pt}%
\pgfpathmoveto{\pgfqpoint{4.513235in}{0.528000in}}%
\pgfpathlineto{\pgfqpoint{4.513235in}{4.224000in}}%
\pgfusepath{stroke}%
\end{pgfscope}%
\begin{pgfscope}%
\pgfsetbuttcap%
\pgfsetroundjoin%
\definecolor{currentfill}{rgb}{0.000000,0.000000,0.000000}%
\pgfsetfillcolor{currentfill}%
\pgfsetlinewidth{0.803000pt}%
\definecolor{currentstroke}{rgb}{0.000000,0.000000,0.000000}%
\pgfsetstrokecolor{currentstroke}%
\pgfsetdash{}{0pt}%
\pgfsys@defobject{currentmarker}{\pgfqpoint{0.000000in}{-0.048611in}}{\pgfqpoint{0.000000in}{0.000000in}}{%
\pgfpathmoveto{\pgfqpoint{0.000000in}{0.000000in}}%
\pgfpathlineto{\pgfqpoint{0.000000in}{-0.048611in}}%
\pgfusepath{stroke,fill}%
}%
\begin{pgfscope}%
\pgfsys@transformshift{4.513235in}{0.528000in}%
\pgfsys@useobject{currentmarker}{}%
\end{pgfscope}%
\end{pgfscope}%
\begin{pgfscope}%
\definecolor{textcolor}{rgb}{0.000000,0.000000,0.000000}%
\pgfsetstrokecolor{textcolor}%
\pgfsetfillcolor{textcolor}%
\pgftext[x=4.513235in,y=0.430778in,,top]{\color{textcolor}\rmfamily\fontsize{10.000000}{12.000000}\selectfont \(\displaystyle 1.5\)}%
\end{pgfscope}%
\begin{pgfscope}%
\pgfpathrectangle{\pgfqpoint{0.800000in}{0.528000in}}{\pgfqpoint{4.960000in}{3.696000in}}%
\pgfusepath{clip}%
\pgfsetbuttcap%
\pgfsetroundjoin%
\pgfsetlinewidth{0.803000pt}%
\definecolor{currentstroke}{rgb}{0.800000,0.800000,0.800000}%
\pgfsetstrokecolor{currentstroke}%
\pgfsetdash{{0.800000pt}{1.320000pt}}{0.000000pt}%
\pgfpathmoveto{\pgfqpoint{5.605363in}{0.528000in}}%
\pgfpathlineto{\pgfqpoint{5.605363in}{4.224000in}}%
\pgfusepath{stroke}%
\end{pgfscope}%
\begin{pgfscope}%
\pgfsetbuttcap%
\pgfsetroundjoin%
\definecolor{currentfill}{rgb}{0.000000,0.000000,0.000000}%
\pgfsetfillcolor{currentfill}%
\pgfsetlinewidth{0.803000pt}%
\definecolor{currentstroke}{rgb}{0.000000,0.000000,0.000000}%
\pgfsetstrokecolor{currentstroke}%
\pgfsetdash{}{0pt}%
\pgfsys@defobject{currentmarker}{\pgfqpoint{0.000000in}{-0.048611in}}{\pgfqpoint{0.000000in}{0.000000in}}{%
\pgfpathmoveto{\pgfqpoint{0.000000in}{0.000000in}}%
\pgfpathlineto{\pgfqpoint{0.000000in}{-0.048611in}}%
\pgfusepath{stroke,fill}%
}%
\begin{pgfscope}%
\pgfsys@transformshift{5.605363in}{0.528000in}%
\pgfsys@useobject{currentmarker}{}%
\end{pgfscope}%
\end{pgfscope}%
\begin{pgfscope}%
\definecolor{textcolor}{rgb}{0.000000,0.000000,0.000000}%
\pgfsetstrokecolor{textcolor}%
\pgfsetfillcolor{textcolor}%
\pgftext[x=5.605363in,y=0.430778in,,top]{\color{textcolor}\rmfamily\fontsize{10.000000}{12.000000}\selectfont \(\displaystyle 2.0\)}%
\end{pgfscope}%
\begin{pgfscope}%
\definecolor{textcolor}{rgb}{0.000000,0.000000,0.000000}%
\pgfsetstrokecolor{textcolor}%
\pgfsetfillcolor{textcolor}%
\pgftext[x=3.280000in,y=0.251766in,,top]{\color{textcolor}\rmfamily\fontsize{10.000000}{12.000000}\selectfont x}%
\end{pgfscope}%
\begin{pgfscope}%
\pgfpathrectangle{\pgfqpoint{0.800000in}{0.528000in}}{\pgfqpoint{4.960000in}{3.696000in}}%
\pgfusepath{clip}%
\pgfsetbuttcap%
\pgfsetroundjoin%
\pgfsetlinewidth{0.803000pt}%
\definecolor{currentstroke}{rgb}{0.800000,0.800000,0.800000}%
\pgfsetstrokecolor{currentstroke}%
\pgfsetdash{{0.800000pt}{1.320000pt}}{0.000000pt}%
\pgfpathmoveto{\pgfqpoint{0.800000in}{0.528000in}}%
\pgfpathlineto{\pgfqpoint{5.760000in}{0.528000in}}%
\pgfusepath{stroke}%
\end{pgfscope}%
\begin{pgfscope}%
\pgfsetbuttcap%
\pgfsetroundjoin%
\definecolor{currentfill}{rgb}{0.000000,0.000000,0.000000}%
\pgfsetfillcolor{currentfill}%
\pgfsetlinewidth{0.803000pt}%
\definecolor{currentstroke}{rgb}{0.000000,0.000000,0.000000}%
\pgfsetstrokecolor{currentstroke}%
\pgfsetdash{}{0pt}%
\pgfsys@defobject{currentmarker}{\pgfqpoint{-0.048611in}{0.000000in}}{\pgfqpoint{0.000000in}{0.000000in}}{%
\pgfpathmoveto{\pgfqpoint{0.000000in}{0.000000in}}%
\pgfpathlineto{\pgfqpoint{-0.048611in}{0.000000in}}%
\pgfusepath{stroke,fill}%
}%
\begin{pgfscope}%
\pgfsys@transformshift{0.800000in}{0.528000in}%
\pgfsys@useobject{currentmarker}{}%
\end{pgfscope}%
\end{pgfscope}%
\begin{pgfscope}%
\definecolor{textcolor}{rgb}{0.000000,0.000000,0.000000}%
\pgfsetstrokecolor{textcolor}%
\pgfsetfillcolor{textcolor}%
\pgftext[x=0.347838in,y=0.479775in,left,base]{\color{textcolor}\rmfamily\fontsize{10.000000}{12.000000}\selectfont \(\displaystyle -0.50\)}%
\end{pgfscope}%
\begin{pgfscope}%
\pgfpathrectangle{\pgfqpoint{0.800000in}{0.528000in}}{\pgfqpoint{4.960000in}{3.696000in}}%
\pgfusepath{clip}%
\pgfsetbuttcap%
\pgfsetroundjoin%
\pgfsetlinewidth{0.803000pt}%
\definecolor{currentstroke}{rgb}{0.800000,0.800000,0.800000}%
\pgfsetstrokecolor{currentstroke}%
\pgfsetdash{{0.800000pt}{1.320000pt}}{0.000000pt}%
\pgfpathmoveto{\pgfqpoint{0.800000in}{0.990000in}}%
\pgfpathlineto{\pgfqpoint{5.760000in}{0.990000in}}%
\pgfusepath{stroke}%
\end{pgfscope}%
\begin{pgfscope}%
\pgfsetbuttcap%
\pgfsetroundjoin%
\definecolor{currentfill}{rgb}{0.000000,0.000000,0.000000}%
\pgfsetfillcolor{currentfill}%
\pgfsetlinewidth{0.803000pt}%
\definecolor{currentstroke}{rgb}{0.000000,0.000000,0.000000}%
\pgfsetstrokecolor{currentstroke}%
\pgfsetdash{}{0pt}%
\pgfsys@defobject{currentmarker}{\pgfqpoint{-0.048611in}{0.000000in}}{\pgfqpoint{0.000000in}{0.000000in}}{%
\pgfpathmoveto{\pgfqpoint{0.000000in}{0.000000in}}%
\pgfpathlineto{\pgfqpoint{-0.048611in}{0.000000in}}%
\pgfusepath{stroke,fill}%
}%
\begin{pgfscope}%
\pgfsys@transformshift{0.800000in}{0.990000in}%
\pgfsys@useobject{currentmarker}{}%
\end{pgfscope}%
\end{pgfscope}%
\begin{pgfscope}%
\definecolor{textcolor}{rgb}{0.000000,0.000000,0.000000}%
\pgfsetstrokecolor{textcolor}%
\pgfsetfillcolor{textcolor}%
\pgftext[x=0.347838in,y=0.941775in,left,base]{\color{textcolor}\rmfamily\fontsize{10.000000}{12.000000}\selectfont \(\displaystyle -0.25\)}%
\end{pgfscope}%
\begin{pgfscope}%
\pgfpathrectangle{\pgfqpoint{0.800000in}{0.528000in}}{\pgfqpoint{4.960000in}{3.696000in}}%
\pgfusepath{clip}%
\pgfsetbuttcap%
\pgfsetroundjoin%
\pgfsetlinewidth{0.803000pt}%
\definecolor{currentstroke}{rgb}{0.800000,0.800000,0.800000}%
\pgfsetstrokecolor{currentstroke}%
\pgfsetdash{{0.800000pt}{1.320000pt}}{0.000000pt}%
\pgfpathmoveto{\pgfqpoint{0.800000in}{1.452000in}}%
\pgfpathlineto{\pgfqpoint{5.760000in}{1.452000in}}%
\pgfusepath{stroke}%
\end{pgfscope}%
\begin{pgfscope}%
\pgfsetbuttcap%
\pgfsetroundjoin%
\definecolor{currentfill}{rgb}{0.000000,0.000000,0.000000}%
\pgfsetfillcolor{currentfill}%
\pgfsetlinewidth{0.803000pt}%
\definecolor{currentstroke}{rgb}{0.000000,0.000000,0.000000}%
\pgfsetstrokecolor{currentstroke}%
\pgfsetdash{}{0pt}%
\pgfsys@defobject{currentmarker}{\pgfqpoint{-0.048611in}{0.000000in}}{\pgfqpoint{0.000000in}{0.000000in}}{%
\pgfpathmoveto{\pgfqpoint{0.000000in}{0.000000in}}%
\pgfpathlineto{\pgfqpoint{-0.048611in}{0.000000in}}%
\pgfusepath{stroke,fill}%
}%
\begin{pgfscope}%
\pgfsys@transformshift{0.800000in}{1.452000in}%
\pgfsys@useobject{currentmarker}{}%
\end{pgfscope}%
\end{pgfscope}%
\begin{pgfscope}%
\definecolor{textcolor}{rgb}{0.000000,0.000000,0.000000}%
\pgfsetstrokecolor{textcolor}%
\pgfsetfillcolor{textcolor}%
\pgftext[x=0.455863in,y=1.403775in,left,base]{\color{textcolor}\rmfamily\fontsize{10.000000}{12.000000}\selectfont \(\displaystyle 0.00\)}%
\end{pgfscope}%
\begin{pgfscope}%
\pgfpathrectangle{\pgfqpoint{0.800000in}{0.528000in}}{\pgfqpoint{4.960000in}{3.696000in}}%
\pgfusepath{clip}%
\pgfsetbuttcap%
\pgfsetroundjoin%
\pgfsetlinewidth{0.803000pt}%
\definecolor{currentstroke}{rgb}{0.800000,0.800000,0.800000}%
\pgfsetstrokecolor{currentstroke}%
\pgfsetdash{{0.800000pt}{1.320000pt}}{0.000000pt}%
\pgfpathmoveto{\pgfqpoint{0.800000in}{1.914000in}}%
\pgfpathlineto{\pgfqpoint{5.760000in}{1.914000in}}%
\pgfusepath{stroke}%
\end{pgfscope}%
\begin{pgfscope}%
\pgfsetbuttcap%
\pgfsetroundjoin%
\definecolor{currentfill}{rgb}{0.000000,0.000000,0.000000}%
\pgfsetfillcolor{currentfill}%
\pgfsetlinewidth{0.803000pt}%
\definecolor{currentstroke}{rgb}{0.000000,0.000000,0.000000}%
\pgfsetstrokecolor{currentstroke}%
\pgfsetdash{}{0pt}%
\pgfsys@defobject{currentmarker}{\pgfqpoint{-0.048611in}{0.000000in}}{\pgfqpoint{0.000000in}{0.000000in}}{%
\pgfpathmoveto{\pgfqpoint{0.000000in}{0.000000in}}%
\pgfpathlineto{\pgfqpoint{-0.048611in}{0.000000in}}%
\pgfusepath{stroke,fill}%
}%
\begin{pgfscope}%
\pgfsys@transformshift{0.800000in}{1.914000in}%
\pgfsys@useobject{currentmarker}{}%
\end{pgfscope}%
\end{pgfscope}%
\begin{pgfscope}%
\definecolor{textcolor}{rgb}{0.000000,0.000000,0.000000}%
\pgfsetstrokecolor{textcolor}%
\pgfsetfillcolor{textcolor}%
\pgftext[x=0.455863in,y=1.865775in,left,base]{\color{textcolor}\rmfamily\fontsize{10.000000}{12.000000}\selectfont \(\displaystyle 0.25\)}%
\end{pgfscope}%
\begin{pgfscope}%
\pgfpathrectangle{\pgfqpoint{0.800000in}{0.528000in}}{\pgfqpoint{4.960000in}{3.696000in}}%
\pgfusepath{clip}%
\pgfsetbuttcap%
\pgfsetroundjoin%
\pgfsetlinewidth{0.803000pt}%
\definecolor{currentstroke}{rgb}{0.800000,0.800000,0.800000}%
\pgfsetstrokecolor{currentstroke}%
\pgfsetdash{{0.800000pt}{1.320000pt}}{0.000000pt}%
\pgfpathmoveto{\pgfqpoint{0.800000in}{2.376000in}}%
\pgfpathlineto{\pgfqpoint{5.760000in}{2.376000in}}%
\pgfusepath{stroke}%
\end{pgfscope}%
\begin{pgfscope}%
\pgfsetbuttcap%
\pgfsetroundjoin%
\definecolor{currentfill}{rgb}{0.000000,0.000000,0.000000}%
\pgfsetfillcolor{currentfill}%
\pgfsetlinewidth{0.803000pt}%
\definecolor{currentstroke}{rgb}{0.000000,0.000000,0.000000}%
\pgfsetstrokecolor{currentstroke}%
\pgfsetdash{}{0pt}%
\pgfsys@defobject{currentmarker}{\pgfqpoint{-0.048611in}{0.000000in}}{\pgfqpoint{0.000000in}{0.000000in}}{%
\pgfpathmoveto{\pgfqpoint{0.000000in}{0.000000in}}%
\pgfpathlineto{\pgfqpoint{-0.048611in}{0.000000in}}%
\pgfusepath{stroke,fill}%
}%
\begin{pgfscope}%
\pgfsys@transformshift{0.800000in}{2.376000in}%
\pgfsys@useobject{currentmarker}{}%
\end{pgfscope}%
\end{pgfscope}%
\begin{pgfscope}%
\definecolor{textcolor}{rgb}{0.000000,0.000000,0.000000}%
\pgfsetstrokecolor{textcolor}%
\pgfsetfillcolor{textcolor}%
\pgftext[x=0.455863in,y=2.327775in,left,base]{\color{textcolor}\rmfamily\fontsize{10.000000}{12.000000}\selectfont \(\displaystyle 0.50\)}%
\end{pgfscope}%
\begin{pgfscope}%
\pgfpathrectangle{\pgfqpoint{0.800000in}{0.528000in}}{\pgfqpoint{4.960000in}{3.696000in}}%
\pgfusepath{clip}%
\pgfsetbuttcap%
\pgfsetroundjoin%
\pgfsetlinewidth{0.803000pt}%
\definecolor{currentstroke}{rgb}{0.800000,0.800000,0.800000}%
\pgfsetstrokecolor{currentstroke}%
\pgfsetdash{{0.800000pt}{1.320000pt}}{0.000000pt}%
\pgfpathmoveto{\pgfqpoint{0.800000in}{2.838000in}}%
\pgfpathlineto{\pgfqpoint{5.760000in}{2.838000in}}%
\pgfusepath{stroke}%
\end{pgfscope}%
\begin{pgfscope}%
\pgfsetbuttcap%
\pgfsetroundjoin%
\definecolor{currentfill}{rgb}{0.000000,0.000000,0.000000}%
\pgfsetfillcolor{currentfill}%
\pgfsetlinewidth{0.803000pt}%
\definecolor{currentstroke}{rgb}{0.000000,0.000000,0.000000}%
\pgfsetstrokecolor{currentstroke}%
\pgfsetdash{}{0pt}%
\pgfsys@defobject{currentmarker}{\pgfqpoint{-0.048611in}{0.000000in}}{\pgfqpoint{0.000000in}{0.000000in}}{%
\pgfpathmoveto{\pgfqpoint{0.000000in}{0.000000in}}%
\pgfpathlineto{\pgfqpoint{-0.048611in}{0.000000in}}%
\pgfusepath{stroke,fill}%
}%
\begin{pgfscope}%
\pgfsys@transformshift{0.800000in}{2.838000in}%
\pgfsys@useobject{currentmarker}{}%
\end{pgfscope}%
\end{pgfscope}%
\begin{pgfscope}%
\definecolor{textcolor}{rgb}{0.000000,0.000000,0.000000}%
\pgfsetstrokecolor{textcolor}%
\pgfsetfillcolor{textcolor}%
\pgftext[x=0.455863in,y=2.789775in,left,base]{\color{textcolor}\rmfamily\fontsize{10.000000}{12.000000}\selectfont \(\displaystyle 0.75\)}%
\end{pgfscope}%
\begin{pgfscope}%
\pgfpathrectangle{\pgfqpoint{0.800000in}{0.528000in}}{\pgfqpoint{4.960000in}{3.696000in}}%
\pgfusepath{clip}%
\pgfsetbuttcap%
\pgfsetroundjoin%
\pgfsetlinewidth{0.803000pt}%
\definecolor{currentstroke}{rgb}{0.800000,0.800000,0.800000}%
\pgfsetstrokecolor{currentstroke}%
\pgfsetdash{{0.800000pt}{1.320000pt}}{0.000000pt}%
\pgfpathmoveto{\pgfqpoint{0.800000in}{3.300000in}}%
\pgfpathlineto{\pgfqpoint{5.760000in}{3.300000in}}%
\pgfusepath{stroke}%
\end{pgfscope}%
\begin{pgfscope}%
\pgfsetbuttcap%
\pgfsetroundjoin%
\definecolor{currentfill}{rgb}{0.000000,0.000000,0.000000}%
\pgfsetfillcolor{currentfill}%
\pgfsetlinewidth{0.803000pt}%
\definecolor{currentstroke}{rgb}{0.000000,0.000000,0.000000}%
\pgfsetstrokecolor{currentstroke}%
\pgfsetdash{}{0pt}%
\pgfsys@defobject{currentmarker}{\pgfqpoint{-0.048611in}{0.000000in}}{\pgfqpoint{0.000000in}{0.000000in}}{%
\pgfpathmoveto{\pgfqpoint{0.000000in}{0.000000in}}%
\pgfpathlineto{\pgfqpoint{-0.048611in}{0.000000in}}%
\pgfusepath{stroke,fill}%
}%
\begin{pgfscope}%
\pgfsys@transformshift{0.800000in}{3.300000in}%
\pgfsys@useobject{currentmarker}{}%
\end{pgfscope}%
\end{pgfscope}%
\begin{pgfscope}%
\definecolor{textcolor}{rgb}{0.000000,0.000000,0.000000}%
\pgfsetstrokecolor{textcolor}%
\pgfsetfillcolor{textcolor}%
\pgftext[x=0.455863in,y=3.251775in,left,base]{\color{textcolor}\rmfamily\fontsize{10.000000}{12.000000}\selectfont \(\displaystyle 1.00\)}%
\end{pgfscope}%
\begin{pgfscope}%
\pgfpathrectangle{\pgfqpoint{0.800000in}{0.528000in}}{\pgfqpoint{4.960000in}{3.696000in}}%
\pgfusepath{clip}%
\pgfsetbuttcap%
\pgfsetroundjoin%
\pgfsetlinewidth{0.803000pt}%
\definecolor{currentstroke}{rgb}{0.800000,0.800000,0.800000}%
\pgfsetstrokecolor{currentstroke}%
\pgfsetdash{{0.800000pt}{1.320000pt}}{0.000000pt}%
\pgfpathmoveto{\pgfqpoint{0.800000in}{3.762000in}}%
\pgfpathlineto{\pgfqpoint{5.760000in}{3.762000in}}%
\pgfusepath{stroke}%
\end{pgfscope}%
\begin{pgfscope}%
\pgfsetbuttcap%
\pgfsetroundjoin%
\definecolor{currentfill}{rgb}{0.000000,0.000000,0.000000}%
\pgfsetfillcolor{currentfill}%
\pgfsetlinewidth{0.803000pt}%
\definecolor{currentstroke}{rgb}{0.000000,0.000000,0.000000}%
\pgfsetstrokecolor{currentstroke}%
\pgfsetdash{}{0pt}%
\pgfsys@defobject{currentmarker}{\pgfqpoint{-0.048611in}{0.000000in}}{\pgfqpoint{0.000000in}{0.000000in}}{%
\pgfpathmoveto{\pgfqpoint{0.000000in}{0.000000in}}%
\pgfpathlineto{\pgfqpoint{-0.048611in}{0.000000in}}%
\pgfusepath{stroke,fill}%
}%
\begin{pgfscope}%
\pgfsys@transformshift{0.800000in}{3.762000in}%
\pgfsys@useobject{currentmarker}{}%
\end{pgfscope}%
\end{pgfscope}%
\begin{pgfscope}%
\definecolor{textcolor}{rgb}{0.000000,0.000000,0.000000}%
\pgfsetstrokecolor{textcolor}%
\pgfsetfillcolor{textcolor}%
\pgftext[x=0.455863in,y=3.713775in,left,base]{\color{textcolor}\rmfamily\fontsize{10.000000}{12.000000}\selectfont \(\displaystyle 1.25\)}%
\end{pgfscope}%
\begin{pgfscope}%
\pgfpathrectangle{\pgfqpoint{0.800000in}{0.528000in}}{\pgfqpoint{4.960000in}{3.696000in}}%
\pgfusepath{clip}%
\pgfsetbuttcap%
\pgfsetroundjoin%
\pgfsetlinewidth{0.803000pt}%
\definecolor{currentstroke}{rgb}{0.800000,0.800000,0.800000}%
\pgfsetstrokecolor{currentstroke}%
\pgfsetdash{{0.800000pt}{1.320000pt}}{0.000000pt}%
\pgfpathmoveto{\pgfqpoint{0.800000in}{4.224000in}}%
\pgfpathlineto{\pgfqpoint{5.760000in}{4.224000in}}%
\pgfusepath{stroke}%
\end{pgfscope}%
\begin{pgfscope}%
\pgfsetbuttcap%
\pgfsetroundjoin%
\definecolor{currentfill}{rgb}{0.000000,0.000000,0.000000}%
\pgfsetfillcolor{currentfill}%
\pgfsetlinewidth{0.803000pt}%
\definecolor{currentstroke}{rgb}{0.000000,0.000000,0.000000}%
\pgfsetstrokecolor{currentstroke}%
\pgfsetdash{}{0pt}%
\pgfsys@defobject{currentmarker}{\pgfqpoint{-0.048611in}{0.000000in}}{\pgfqpoint{0.000000in}{0.000000in}}{%
\pgfpathmoveto{\pgfqpoint{0.000000in}{0.000000in}}%
\pgfpathlineto{\pgfqpoint{-0.048611in}{0.000000in}}%
\pgfusepath{stroke,fill}%
}%
\begin{pgfscope}%
\pgfsys@transformshift{0.800000in}{4.224000in}%
\pgfsys@useobject{currentmarker}{}%
\end{pgfscope}%
\end{pgfscope}%
\begin{pgfscope}%
\definecolor{textcolor}{rgb}{0.000000,0.000000,0.000000}%
\pgfsetstrokecolor{textcolor}%
\pgfsetfillcolor{textcolor}%
\pgftext[x=0.455863in,y=4.175775in,left,base]{\color{textcolor}\rmfamily\fontsize{10.000000}{12.000000}\selectfont \(\displaystyle 1.50\)}%
\end{pgfscope}%
\begin{pgfscope}%
\definecolor{textcolor}{rgb}{0.000000,0.000000,0.000000}%
\pgfsetstrokecolor{textcolor}%
\pgfsetfillcolor{textcolor}%
\pgftext[x=0.292283in,y=2.376000in,,bottom]{\color{textcolor}\rmfamily\fontsize{10.000000}{12.000000}\selectfont y}%
\end{pgfscope}%
\begin{pgfscope}%
\pgfpathrectangle{\pgfqpoint{0.800000in}{0.528000in}}{\pgfqpoint{4.960000in}{3.696000in}}%
\pgfusepath{clip}%
\pgfsetrectcap%
\pgfsetroundjoin%
\pgfsetlinewidth{1.505625pt}%
\definecolor{currentstroke}{rgb}{0.000000,0.000000,0.000000}%
\pgfsetstrokecolor{currentstroke}%
\pgfsetdash{}{0pt}%
\pgfpathmoveto{\pgfqpoint{0.790000in}{1.452000in}}%
\pgfpathlineto{\pgfqpoint{5.770000in}{1.452000in}}%
\pgfpathlineto{\pgfqpoint{5.770000in}{1.452000in}}%
\pgfusepath{stroke}%
\end{pgfscope}%
\begin{pgfscope}%
\pgfpathrectangle{\pgfqpoint{0.800000in}{0.528000in}}{\pgfqpoint{4.960000in}{3.696000in}}%
\pgfusepath{clip}%
\pgfsetrectcap%
\pgfsetroundjoin%
\pgfsetlinewidth{1.505625pt}%
\definecolor{currentstroke}{rgb}{0.000000,0.000000,0.000000}%
\pgfsetstrokecolor{currentstroke}%
\pgfsetdash{}{0pt}%
\pgfpathmoveto{\pgfqpoint{1.236851in}{0.518000in}}%
\pgfpathlineto{\pgfqpoint{1.236851in}{4.234000in}}%
\pgfpathlineto{\pgfqpoint{1.236851in}{4.234000in}}%
\pgfusepath{stroke}%
\end{pgfscope}%
\begin{pgfscope}%
\pgfpathrectangle{\pgfqpoint{0.800000in}{0.528000in}}{\pgfqpoint{4.960000in}{3.696000in}}%
\pgfusepath{clip}%
\pgfsetrectcap%
\pgfsetroundjoin%
\pgfsetlinewidth{1.505625pt}%
\definecolor{currentstroke}{rgb}{0.000000,0.000000,1.000000}%
\pgfsetstrokecolor{currentstroke}%
\pgfsetdash{}{0pt}%
\pgfpathmoveto{\pgfqpoint{1.236851in}{1.452000in}}%
\pgfpathlineto{\pgfqpoint{1.271508in}{1.481889in}}%
\pgfpathlineto{\pgfqpoint{1.306165in}{1.511713in}}%
\pgfpathlineto{\pgfqpoint{1.340821in}{1.541465in}}%
\pgfpathlineto{\pgfqpoint{1.375478in}{1.571141in}}%
\pgfpathlineto{\pgfqpoint{1.410135in}{1.600736in}}%
\pgfpathlineto{\pgfqpoint{1.444792in}{1.630245in}}%
\pgfpathlineto{\pgfqpoint{1.479449in}{1.659663in}}%
\pgfpathlineto{\pgfqpoint{1.514105in}{1.688985in}}%
\pgfpathlineto{\pgfqpoint{1.548762in}{1.718205in}}%
\pgfpathlineto{\pgfqpoint{1.583419in}{1.747319in}}%
\pgfpathlineto{\pgfqpoint{1.618076in}{1.776321in}}%
\pgfpathlineto{\pgfqpoint{1.652732in}{1.805208in}}%
\pgfpathlineto{\pgfqpoint{1.687389in}{1.833973in}}%
\pgfpathlineto{\pgfqpoint{1.722046in}{1.862611in}}%
\pgfpathlineto{\pgfqpoint{1.756703in}{1.891118in}}%
\pgfpathlineto{\pgfqpoint{1.791360in}{1.919488in}}%
\pgfpathlineto{\pgfqpoint{1.826016in}{1.947717in}}%
\pgfpathlineto{\pgfqpoint{1.860673in}{1.975799in}}%
\pgfpathlineto{\pgfqpoint{1.895330in}{2.003729in}}%
\pgfpathlineto{\pgfqpoint{1.929987in}{2.031502in}}%
\pgfpathlineto{\pgfqpoint{1.964643in}{2.059114in}}%
\pgfpathlineto{\pgfqpoint{1.999300in}{2.086559in}}%
\pgfpathlineto{\pgfqpoint{2.033957in}{2.113832in}}%
\pgfpathlineto{\pgfqpoint{2.068614in}{2.140928in}}%
\pgfpathlineto{\pgfqpoint{2.103271in}{2.167842in}}%
\pgfpathlineto{\pgfqpoint{2.137927in}{2.194570in}}%
\pgfpathlineto{\pgfqpoint{2.172584in}{2.221105in}}%
\pgfpathlineto{\pgfqpoint{2.207241in}{2.247443in}}%
\pgfpathlineto{\pgfqpoint{2.241898in}{2.273579in}}%
\pgfpathlineto{\pgfqpoint{2.276554in}{2.299507in}}%
\pgfpathlineto{\pgfqpoint{2.311211in}{2.325224in}}%
\pgfpathlineto{\pgfqpoint{2.345868in}{2.350723in}}%
\pgfpathlineto{\pgfqpoint{2.380525in}{2.376000in}}%
\pgfpathlineto{\pgfqpoint{2.415182in}{2.401050in}}%
\pgfpathlineto{\pgfqpoint{2.449838in}{2.425867in}}%
\pgfpathlineto{\pgfqpoint{2.484495in}{2.450446in}}%
\pgfpathlineto{\pgfqpoint{2.519152in}{2.474784in}}%
\pgfpathlineto{\pgfqpoint{2.553809in}{2.498873in}}%
\pgfpathlineto{\pgfqpoint{2.588465in}{2.522711in}}%
\pgfpathlineto{\pgfqpoint{2.623122in}{2.546290in}}%
\pgfpathlineto{\pgfqpoint{2.657779in}{2.569607in}}%
\pgfpathlineto{\pgfqpoint{2.692436in}{2.592656in}}%
\pgfpathlineto{\pgfqpoint{2.727093in}{2.615433in}}%
\pgfpathlineto{\pgfqpoint{2.761749in}{2.637931in}}%
\pgfpathlineto{\pgfqpoint{2.796406in}{2.660147in}}%
\pgfpathlineto{\pgfqpoint{2.831063in}{2.682075in}}%
\pgfpathlineto{\pgfqpoint{2.865720in}{2.703710in}}%
\pgfpathlineto{\pgfqpoint{2.900376in}{2.725047in}}%
\pgfpathlineto{\pgfqpoint{2.935033in}{2.746082in}}%
\pgfpathlineto{\pgfqpoint{2.969690in}{2.766808in}}%
\pgfpathlineto{\pgfqpoint{3.004347in}{2.787221in}}%
\pgfpathlineto{\pgfqpoint{3.039004in}{2.807315in}}%
\pgfpathlineto{\pgfqpoint{3.073660in}{2.827087in}}%
\pgfpathlineto{\pgfqpoint{3.108317in}{2.846531in}}%
\pgfpathlineto{\pgfqpoint{3.142974in}{2.865641in}}%
\pgfpathlineto{\pgfqpoint{3.177631in}{2.884413in}}%
\pgfpathlineto{\pgfqpoint{3.212287in}{2.902841in}}%
\pgfpathlineto{\pgfqpoint{3.246944in}{2.920921in}}%
\pgfpathlineto{\pgfqpoint{3.281601in}{2.938648in}}%
\pgfpathlineto{\pgfqpoint{3.316258in}{2.956016in}}%
\pgfpathlineto{\pgfqpoint{3.350915in}{2.973021in}}%
\pgfpathlineto{\pgfqpoint{3.385571in}{2.989657in}}%
\pgfpathlineto{\pgfqpoint{3.420228in}{3.005919in}}%
\pgfpathlineto{\pgfqpoint{3.454885in}{3.021803in}}%
\pgfpathlineto{\pgfqpoint{3.489542in}{3.037304in}}%
\pgfpathlineto{\pgfqpoint{3.524198in}{3.052415in}}%
\pgfpathlineto{\pgfqpoint{3.558855in}{3.067133in}}%
\pgfpathlineto{\pgfqpoint{3.593512in}{3.081451in}}%
\pgfpathlineto{\pgfqpoint{3.628169in}{3.095366in}}%
\pgfpathlineto{\pgfqpoint{3.662826in}{3.108872in}}%
\pgfpathlineto{\pgfqpoint{3.697482in}{3.121964in}}%
\pgfpathlineto{\pgfqpoint{3.732139in}{3.134638in}}%
\pgfpathlineto{\pgfqpoint{3.766796in}{3.146887in}}%
\pgfpathlineto{\pgfqpoint{3.801453in}{3.158707in}}%
\pgfpathlineto{\pgfqpoint{3.836109in}{3.170092in}}%
\pgfpathlineto{\pgfqpoint{3.870766in}{3.181039in}}%
\pgfpathlineto{\pgfqpoint{3.905423in}{3.191541in}}%
\pgfpathlineto{\pgfqpoint{3.940080in}{3.201595in}}%
\pgfpathlineto{\pgfqpoint{3.974737in}{3.211193in}}%
\pgfpathlineto{\pgfqpoint{4.009393in}{3.220333in}}%
\pgfpathlineto{\pgfqpoint{4.044050in}{3.229008in}}%
\pgfpathlineto{\pgfqpoint{4.078707in}{3.237214in}}%
\pgfpathlineto{\pgfqpoint{4.113364in}{3.244945in}}%
\pgfpathlineto{\pgfqpoint{4.148020in}{3.252197in}}%
\pgfpathlineto{\pgfqpoint{4.182677in}{3.258964in}}%
\pgfpathlineto{\pgfqpoint{4.217334in}{3.265241in}}%
\pgfpathlineto{\pgfqpoint{4.251991in}{3.271024in}}%
\pgfpathlineto{\pgfqpoint{4.286648in}{3.276307in}}%
\pgfpathlineto{\pgfqpoint{4.321304in}{3.281086in}}%
\pgfpathlineto{\pgfqpoint{4.355961in}{3.285354in}}%
\pgfpathlineto{\pgfqpoint{4.390618in}{3.289108in}}%
\pgfpathlineto{\pgfqpoint{4.425275in}{3.292342in}}%
\pgfpathlineto{\pgfqpoint{4.459931in}{3.295051in}}%
\pgfpathlineto{\pgfqpoint{4.494588in}{3.297230in}}%
\pgfpathlineto{\pgfqpoint{4.529245in}{3.298874in}}%
\pgfpathlineto{\pgfqpoint{4.563902in}{3.299979in}}%
\pgfpathlineto{\pgfqpoint{4.598559in}{3.300537in}}%
\pgfpathlineto{\pgfqpoint{4.633215in}{3.300546in}}%
\pgfpathlineto{\pgfqpoint{4.667872in}{3.300000in}}%
\pgfusepath{stroke}%
\end{pgfscope}%
\begin{pgfscope}%
\pgfpathrectangle{\pgfqpoint{0.800000in}{0.528000in}}{\pgfqpoint{4.960000in}{3.696000in}}%
\pgfusepath{clip}%
\pgfsetrectcap%
\pgfsetroundjoin%
\pgfsetlinewidth{1.505625pt}%
\definecolor{currentstroke}{rgb}{1.000000,0.000000,0.000000}%
\pgfsetstrokecolor{currentstroke}%
\pgfsetdash{}{0pt}%
\pgfpathmoveto{\pgfqpoint{1.236851in}{3.300000in}}%
\pgfpathlineto{\pgfqpoint{1.271508in}{3.199022in}}%
\pgfpathlineto{\pgfqpoint{1.306165in}{3.101386in}}%
\pgfpathlineto{\pgfqpoint{1.340821in}{3.007041in}}%
\pgfpathlineto{\pgfqpoint{1.375478in}{2.915936in}}%
\pgfpathlineto{\pgfqpoint{1.410135in}{2.828020in}}%
\pgfpathlineto{\pgfqpoint{1.444792in}{2.743240in}}%
\pgfpathlineto{\pgfqpoint{1.479449in}{2.661545in}}%
\pgfpathlineto{\pgfqpoint{1.514105in}{2.582885in}}%
\pgfpathlineto{\pgfqpoint{1.548762in}{2.507207in}}%
\pgfpathlineto{\pgfqpoint{1.583419in}{2.434460in}}%
\pgfpathlineto{\pgfqpoint{1.618076in}{2.364593in}}%
\pgfpathlineto{\pgfqpoint{1.652732in}{2.297554in}}%
\pgfpathlineto{\pgfqpoint{1.687389in}{2.233292in}}%
\pgfpathlineto{\pgfqpoint{1.722046in}{2.171755in}}%
\pgfpathlineto{\pgfqpoint{1.756703in}{2.112893in}}%
\pgfpathlineto{\pgfqpoint{1.791360in}{2.056653in}}%
\pgfpathlineto{\pgfqpoint{1.826016in}{2.002984in}}%
\pgfpathlineto{\pgfqpoint{1.860673in}{1.951835in}}%
\pgfpathlineto{\pgfqpoint{1.895330in}{1.903154in}}%
\pgfpathlineto{\pgfqpoint{1.929987in}{1.856890in}}%
\pgfpathlineto{\pgfqpoint{1.964643in}{1.812992in}}%
\pgfpathlineto{\pgfqpoint{1.999300in}{1.771407in}}%
\pgfpathlineto{\pgfqpoint{2.033957in}{1.732086in}}%
\pgfpathlineto{\pgfqpoint{2.068614in}{1.694975in}}%
\pgfpathlineto{\pgfqpoint{2.103271in}{1.660024in}}%
\pgfpathlineto{\pgfqpoint{2.137927in}{1.627182in}}%
\pgfpathlineto{\pgfqpoint{2.172584in}{1.596397in}}%
\pgfpathlineto{\pgfqpoint{2.207241in}{1.567617in}}%
\pgfpathlineto{\pgfqpoint{2.241898in}{1.540791in}}%
\pgfpathlineto{\pgfqpoint{2.276554in}{1.515868in}}%
\pgfpathlineto{\pgfqpoint{2.311211in}{1.492796in}}%
\pgfpathlineto{\pgfqpoint{2.345868in}{1.471524in}}%
\pgfpathlineto{\pgfqpoint{2.380525in}{1.452000in}}%
\pgfpathlineto{\pgfqpoint{2.415182in}{1.434173in}}%
\pgfpathlineto{\pgfqpoint{2.449838in}{1.417992in}}%
\pgfpathlineto{\pgfqpoint{2.484495in}{1.403405in}}%
\pgfpathlineto{\pgfqpoint{2.519152in}{1.390361in}}%
\pgfpathlineto{\pgfqpoint{2.553809in}{1.378807in}}%
\pgfpathlineto{\pgfqpoint{2.588465in}{1.368694in}}%
\pgfpathlineto{\pgfqpoint{2.623122in}{1.359969in}}%
\pgfpathlineto{\pgfqpoint{2.657779in}{1.352582in}}%
\pgfpathlineto{\pgfqpoint{2.692436in}{1.346479in}}%
\pgfpathlineto{\pgfqpoint{2.727093in}{1.341611in}}%
\pgfpathlineto{\pgfqpoint{2.761749in}{1.337926in}}%
\pgfpathlineto{\pgfqpoint{2.796406in}{1.335372in}}%
\pgfpathlineto{\pgfqpoint{2.831063in}{1.333898in}}%
\pgfpathlineto{\pgfqpoint{2.865720in}{1.333452in}}%
\pgfpathlineto{\pgfqpoint{2.900376in}{1.333983in}}%
\pgfpathlineto{\pgfqpoint{2.935033in}{1.335440in}}%
\pgfpathlineto{\pgfqpoint{2.969690in}{1.337772in}}%
\pgfpathlineto{\pgfqpoint{3.004347in}{1.340926in}}%
\pgfpathlineto{\pgfqpoint{3.039004in}{1.344851in}}%
\pgfpathlineto{\pgfqpoint{3.073660in}{1.349496in}}%
\pgfpathlineto{\pgfqpoint{3.108317in}{1.354810in}}%
\pgfpathlineto{\pgfqpoint{3.142974in}{1.360741in}}%
\pgfpathlineto{\pgfqpoint{3.177631in}{1.367237in}}%
\pgfpathlineto{\pgfqpoint{3.212287in}{1.374248in}}%
\pgfpathlineto{\pgfqpoint{3.246944in}{1.381721in}}%
\pgfpathlineto{\pgfqpoint{3.281601in}{1.389606in}}%
\pgfpathlineto{\pgfqpoint{3.316258in}{1.397851in}}%
\pgfpathlineto{\pgfqpoint{3.350915in}{1.406405in}}%
\pgfpathlineto{\pgfqpoint{3.385571in}{1.415215in}}%
\pgfpathlineto{\pgfqpoint{3.420228in}{1.424231in}}%
\pgfpathlineto{\pgfqpoint{3.454885in}{1.433402in}}%
\pgfpathlineto{\pgfqpoint{3.489542in}{1.442675in}}%
\pgfpathlineto{\pgfqpoint{3.524198in}{1.452000in}}%
\pgfpathlineto{\pgfqpoint{3.558855in}{1.461325in}}%
\pgfpathlineto{\pgfqpoint{3.593512in}{1.470598in}}%
\pgfpathlineto{\pgfqpoint{3.628169in}{1.479769in}}%
\pgfpathlineto{\pgfqpoint{3.662826in}{1.488785in}}%
\pgfpathlineto{\pgfqpoint{3.697482in}{1.497595in}}%
\pgfpathlineto{\pgfqpoint{3.732139in}{1.506149in}}%
\pgfpathlineto{\pgfqpoint{3.766796in}{1.514394in}}%
\pgfpathlineto{\pgfqpoint{3.801453in}{1.522279in}}%
\pgfpathlineto{\pgfqpoint{3.836109in}{1.529752in}}%
\pgfpathlineto{\pgfqpoint{3.870766in}{1.536763in}}%
\pgfpathlineto{\pgfqpoint{3.905423in}{1.543259in}}%
\pgfpathlineto{\pgfqpoint{3.940080in}{1.549190in}}%
\pgfpathlineto{\pgfqpoint{3.974737in}{1.554504in}}%
\pgfpathlineto{\pgfqpoint{4.009393in}{1.559149in}}%
\pgfpathlineto{\pgfqpoint{4.044050in}{1.563074in}}%
\pgfpathlineto{\pgfqpoint{4.078707in}{1.566228in}}%
\pgfpathlineto{\pgfqpoint{4.113364in}{1.568560in}}%
\pgfpathlineto{\pgfqpoint{4.148020in}{1.570017in}}%
\pgfpathlineto{\pgfqpoint{4.182677in}{1.570548in}}%
\pgfpathlineto{\pgfqpoint{4.217334in}{1.570102in}}%
\pgfpathlineto{\pgfqpoint{4.251991in}{1.568628in}}%
\pgfpathlineto{\pgfqpoint{4.286648in}{1.566074in}}%
\pgfpathlineto{\pgfqpoint{4.321304in}{1.562389in}}%
\pgfpathlineto{\pgfqpoint{4.355961in}{1.557521in}}%
\pgfpathlineto{\pgfqpoint{4.390618in}{1.551418in}}%
\pgfpathlineto{\pgfqpoint{4.425275in}{1.544031in}}%
\pgfpathlineto{\pgfqpoint{4.459931in}{1.535306in}}%
\pgfpathlineto{\pgfqpoint{4.494588in}{1.525193in}}%
\pgfpathlineto{\pgfqpoint{4.529245in}{1.513639in}}%
\pgfpathlineto{\pgfqpoint{4.563902in}{1.500595in}}%
\pgfpathlineto{\pgfqpoint{4.598559in}{1.486008in}}%
\pgfpathlineto{\pgfqpoint{4.633215in}{1.469827in}}%
\pgfpathlineto{\pgfqpoint{4.667872in}{1.452000in}}%
\pgfusepath{stroke}%
\end{pgfscope}%
\begin{pgfscope}%
\pgfpathrectangle{\pgfqpoint{0.800000in}{0.528000in}}{\pgfqpoint{4.960000in}{3.696000in}}%
\pgfusepath{clip}%
\pgfsetrectcap%
\pgfsetroundjoin%
\pgfsetlinewidth{1.505625pt}%
\definecolor{currentstroke}{rgb}{0.000000,0.500000,0.000000}%
\pgfsetstrokecolor{currentstroke}%
\pgfsetdash{}{0pt}%
\pgfpathmoveto{\pgfqpoint{1.236851in}{1.452000in}}%
\pgfpathlineto{\pgfqpoint{1.271508in}{1.615783in}}%
\pgfpathlineto{\pgfqpoint{1.306165in}{1.771236in}}%
\pgfpathlineto{\pgfqpoint{1.340821in}{1.918512in}}%
\pgfpathlineto{\pgfqpoint{1.375478in}{2.057767in}}%
\pgfpathlineto{\pgfqpoint{1.410135in}{2.189153in}}%
\pgfpathlineto{\pgfqpoint{1.444792in}{2.312826in}}%
\pgfpathlineto{\pgfqpoint{1.479449in}{2.428940in}}%
\pgfpathlineto{\pgfqpoint{1.514105in}{2.537649in}}%
\pgfpathlineto{\pgfqpoint{1.548762in}{2.639107in}}%
\pgfpathlineto{\pgfqpoint{1.583419in}{2.733469in}}%
\pgfpathlineto{\pgfqpoint{1.618076in}{2.820889in}}%
\pgfpathlineto{\pgfqpoint{1.652732in}{2.901521in}}%
\pgfpathlineto{\pgfqpoint{1.687389in}{2.975519in}}%
\pgfpathlineto{\pgfqpoint{1.722046in}{3.043038in}}%
\pgfpathlineto{\pgfqpoint{1.756703in}{3.104231in}}%
\pgfpathlineto{\pgfqpoint{1.791360in}{3.159254in}}%
\pgfpathlineto{\pgfqpoint{1.826016in}{3.208261in}}%
\pgfpathlineto{\pgfqpoint{1.860673in}{3.251405in}}%
\pgfpathlineto{\pgfqpoint{1.895330in}{3.288841in}}%
\pgfpathlineto{\pgfqpoint{1.929987in}{3.320724in}}%
\pgfpathlineto{\pgfqpoint{1.964643in}{3.347207in}}%
\pgfpathlineto{\pgfqpoint{1.999300in}{3.368444in}}%
\pgfpathlineto{\pgfqpoint{2.033957in}{3.384591in}}%
\pgfpathlineto{\pgfqpoint{2.068614in}{3.395802in}}%
\pgfpathlineto{\pgfqpoint{2.103271in}{3.402230in}}%
\pgfpathlineto{\pgfqpoint{2.137927in}{3.404029in}}%
\pgfpathlineto{\pgfqpoint{2.172584in}{3.401355in}}%
\pgfpathlineto{\pgfqpoint{2.207241in}{3.394362in}}%
\pgfpathlineto{\pgfqpoint{2.241898in}{3.383203in}}%
\pgfpathlineto{\pgfqpoint{2.276554in}{3.368033in}}%
\pgfpathlineto{\pgfqpoint{2.311211in}{3.349006in}}%
\pgfpathlineto{\pgfqpoint{2.345868in}{3.326277in}}%
\pgfpathlineto{\pgfqpoint{2.380525in}{3.300000in}}%
\pgfpathlineto{\pgfqpoint{2.415182in}{3.270329in}}%
\pgfpathlineto{\pgfqpoint{2.449838in}{3.237418in}}%
\pgfpathlineto{\pgfqpoint{2.484495in}{3.201421in}}%
\pgfpathlineto{\pgfqpoint{2.519152in}{3.162494in}}%
\pgfpathlineto{\pgfqpoint{2.553809in}{3.120790in}}%
\pgfpathlineto{\pgfqpoint{2.588465in}{3.076463in}}%
\pgfpathlineto{\pgfqpoint{2.623122in}{3.029668in}}%
\pgfpathlineto{\pgfqpoint{2.657779in}{2.980558in}}%
\pgfpathlineto{\pgfqpoint{2.692436in}{2.929289in}}%
\pgfpathlineto{\pgfqpoint{2.727093in}{2.876015in}}%
\pgfpathlineto{\pgfqpoint{2.761749in}{2.820889in}}%
\pgfpathlineto{\pgfqpoint{2.796406in}{2.764066in}}%
\pgfpathlineto{\pgfqpoint{2.831063in}{2.705701in}}%
\pgfpathlineto{\pgfqpoint{2.865720in}{2.645947in}}%
\pgfpathlineto{\pgfqpoint{2.900376in}{2.584959in}}%
\pgfpathlineto{\pgfqpoint{2.935033in}{2.522891in}}%
\pgfpathlineto{\pgfqpoint{2.969690in}{2.459897in}}%
\pgfpathlineto{\pgfqpoint{3.004347in}{2.396132in}}%
\pgfpathlineto{\pgfqpoint{3.039004in}{2.331750in}}%
\pgfpathlineto{\pgfqpoint{3.073660in}{2.266905in}}%
\pgfpathlineto{\pgfqpoint{3.108317in}{2.201752in}}%
\pgfpathlineto{\pgfqpoint{3.142974in}{2.136444in}}%
\pgfpathlineto{\pgfqpoint{3.177631in}{2.071137in}}%
\pgfpathlineto{\pgfqpoint{3.212287in}{2.005983in}}%
\pgfpathlineto{\pgfqpoint{3.246944in}{1.941139in}}%
\pgfpathlineto{\pgfqpoint{3.281601in}{1.876757in}}%
\pgfpathlineto{\pgfqpoint{3.316258in}{1.812992in}}%
\pgfpathlineto{\pgfqpoint{3.350915in}{1.749998in}}%
\pgfpathlineto{\pgfqpoint{3.385571in}{1.687930in}}%
\pgfpathlineto{\pgfqpoint{3.420228in}{1.626942in}}%
\pgfpathlineto{\pgfqpoint{3.454885in}{1.567188in}}%
\pgfpathlineto{\pgfqpoint{3.489542in}{1.508823in}}%
\pgfpathlineto{\pgfqpoint{3.524198in}{1.452000in}}%
\pgfpathlineto{\pgfqpoint{3.558855in}{1.396874in}}%
\pgfpathlineto{\pgfqpoint{3.593512in}{1.343600in}}%
\pgfpathlineto{\pgfqpoint{3.628169in}{1.292331in}}%
\pgfpathlineto{\pgfqpoint{3.662826in}{1.243221in}}%
\pgfpathlineto{\pgfqpoint{3.697482in}{1.196426in}}%
\pgfpathlineto{\pgfqpoint{3.732139in}{1.152099in}}%
\pgfpathlineto{\pgfqpoint{3.766796in}{1.110395in}}%
\pgfpathlineto{\pgfqpoint{3.801453in}{1.071467in}}%
\pgfpathlineto{\pgfqpoint{3.836109in}{1.035471in}}%
\pgfpathlineto{\pgfqpoint{3.870766in}{1.002560in}}%
\pgfpathlineto{\pgfqpoint{3.905423in}{0.972889in}}%
\pgfpathlineto{\pgfqpoint{3.940080in}{0.946612in}}%
\pgfpathlineto{\pgfqpoint{3.974737in}{0.923882in}}%
\pgfpathlineto{\pgfqpoint{4.009393in}{0.904856in}}%
\pgfpathlineto{\pgfqpoint{4.044050in}{0.889686in}}%
\pgfpathlineto{\pgfqpoint{4.078707in}{0.878527in}}%
\pgfpathlineto{\pgfqpoint{4.113364in}{0.871534in}}%
\pgfpathlineto{\pgfqpoint{4.148020in}{0.868860in}}%
\pgfpathlineto{\pgfqpoint{4.182677in}{0.870659in}}%
\pgfpathlineto{\pgfqpoint{4.217334in}{0.877087in}}%
\pgfpathlineto{\pgfqpoint{4.251991in}{0.888298in}}%
\pgfpathlineto{\pgfqpoint{4.286648in}{0.904444in}}%
\pgfpathlineto{\pgfqpoint{4.321304in}{0.925682in}}%
\pgfpathlineto{\pgfqpoint{4.355961in}{0.952165in}}%
\pgfpathlineto{\pgfqpoint{4.390618in}{0.984048in}}%
\pgfpathlineto{\pgfqpoint{4.425275in}{1.021484in}}%
\pgfpathlineto{\pgfqpoint{4.459931in}{1.064628in}}%
\pgfpathlineto{\pgfqpoint{4.494588in}{1.113635in}}%
\pgfpathlineto{\pgfqpoint{4.529245in}{1.168657in}}%
\pgfpathlineto{\pgfqpoint{4.563902in}{1.229851in}}%
\pgfpathlineto{\pgfqpoint{4.598559in}{1.297370in}}%
\pgfpathlineto{\pgfqpoint{4.633215in}{1.371368in}}%
\pgfpathlineto{\pgfqpoint{4.667872in}{1.452000in}}%
\pgfusepath{stroke}%
\end{pgfscope}%
\begin{pgfscope}%
\pgfpathrectangle{\pgfqpoint{0.800000in}{0.528000in}}{\pgfqpoint{4.960000in}{3.696000in}}%
\pgfusepath{clip}%
\pgfsetrectcap%
\pgfsetroundjoin%
\pgfsetlinewidth{1.505625pt}%
\definecolor{currentstroke}{rgb}{0.750000,0.750000,0.000000}%
\pgfsetstrokecolor{currentstroke}%
\pgfsetdash{}{0pt}%
\pgfpathmoveto{\pgfqpoint{1.236851in}{1.452000in}}%
\pgfpathlineto{\pgfqpoint{1.271508in}{1.371368in}}%
\pgfpathlineto{\pgfqpoint{1.306165in}{1.297370in}}%
\pgfpathlineto{\pgfqpoint{1.340821in}{1.229851in}}%
\pgfpathlineto{\pgfqpoint{1.375478in}{1.168657in}}%
\pgfpathlineto{\pgfqpoint{1.410135in}{1.113635in}}%
\pgfpathlineto{\pgfqpoint{1.444792in}{1.064628in}}%
\pgfpathlineto{\pgfqpoint{1.479449in}{1.021484in}}%
\pgfpathlineto{\pgfqpoint{1.514105in}{0.984048in}}%
\pgfpathlineto{\pgfqpoint{1.548762in}{0.952165in}}%
\pgfpathlineto{\pgfqpoint{1.583419in}{0.925682in}}%
\pgfpathlineto{\pgfqpoint{1.618076in}{0.904444in}}%
\pgfpathlineto{\pgfqpoint{1.652732in}{0.888298in}}%
\pgfpathlineto{\pgfqpoint{1.687389in}{0.877087in}}%
\pgfpathlineto{\pgfqpoint{1.722046in}{0.870659in}}%
\pgfpathlineto{\pgfqpoint{1.756703in}{0.868860in}}%
\pgfpathlineto{\pgfqpoint{1.791360in}{0.871534in}}%
\pgfpathlineto{\pgfqpoint{1.826016in}{0.878527in}}%
\pgfpathlineto{\pgfqpoint{1.860673in}{0.889686in}}%
\pgfpathlineto{\pgfqpoint{1.895330in}{0.904856in}}%
\pgfpathlineto{\pgfqpoint{1.929987in}{0.923882in}}%
\pgfpathlineto{\pgfqpoint{1.964643in}{0.946612in}}%
\pgfpathlineto{\pgfqpoint{1.999300in}{0.972889in}}%
\pgfpathlineto{\pgfqpoint{2.033957in}{1.002560in}}%
\pgfpathlineto{\pgfqpoint{2.068614in}{1.035471in}}%
\pgfpathlineto{\pgfqpoint{2.103271in}{1.071467in}}%
\pgfpathlineto{\pgfqpoint{2.137927in}{1.110395in}}%
\pgfpathlineto{\pgfqpoint{2.172584in}{1.152099in}}%
\pgfpathlineto{\pgfqpoint{2.207241in}{1.196426in}}%
\pgfpathlineto{\pgfqpoint{2.241898in}{1.243221in}}%
\pgfpathlineto{\pgfqpoint{2.276554in}{1.292331in}}%
\pgfpathlineto{\pgfqpoint{2.311211in}{1.343600in}}%
\pgfpathlineto{\pgfqpoint{2.345868in}{1.396874in}}%
\pgfpathlineto{\pgfqpoint{2.380525in}{1.452000in}}%
\pgfpathlineto{\pgfqpoint{2.415182in}{1.508823in}}%
\pgfpathlineto{\pgfqpoint{2.449838in}{1.567188in}}%
\pgfpathlineto{\pgfqpoint{2.484495in}{1.626942in}}%
\pgfpathlineto{\pgfqpoint{2.519152in}{1.687930in}}%
\pgfpathlineto{\pgfqpoint{2.553809in}{1.749998in}}%
\pgfpathlineto{\pgfqpoint{2.588465in}{1.812992in}}%
\pgfpathlineto{\pgfqpoint{2.623122in}{1.876757in}}%
\pgfpathlineto{\pgfqpoint{2.657779in}{1.941139in}}%
\pgfpathlineto{\pgfqpoint{2.692436in}{2.005983in}}%
\pgfpathlineto{\pgfqpoint{2.727093in}{2.071137in}}%
\pgfpathlineto{\pgfqpoint{2.761749in}{2.136444in}}%
\pgfpathlineto{\pgfqpoint{2.796406in}{2.201752in}}%
\pgfpathlineto{\pgfqpoint{2.831063in}{2.266905in}}%
\pgfpathlineto{\pgfqpoint{2.865720in}{2.331750in}}%
\pgfpathlineto{\pgfqpoint{2.900376in}{2.396132in}}%
\pgfpathlineto{\pgfqpoint{2.935033in}{2.459897in}}%
\pgfpathlineto{\pgfqpoint{2.969690in}{2.522891in}}%
\pgfpathlineto{\pgfqpoint{3.004347in}{2.584959in}}%
\pgfpathlineto{\pgfqpoint{3.039004in}{2.645947in}}%
\pgfpathlineto{\pgfqpoint{3.073660in}{2.705701in}}%
\pgfpathlineto{\pgfqpoint{3.108317in}{2.764066in}}%
\pgfpathlineto{\pgfqpoint{3.142974in}{2.820889in}}%
\pgfpathlineto{\pgfqpoint{3.177631in}{2.876015in}}%
\pgfpathlineto{\pgfqpoint{3.212287in}{2.929289in}}%
\pgfpathlineto{\pgfqpoint{3.246944in}{2.980558in}}%
\pgfpathlineto{\pgfqpoint{3.281601in}{3.029668in}}%
\pgfpathlineto{\pgfqpoint{3.316258in}{3.076463in}}%
\pgfpathlineto{\pgfqpoint{3.350915in}{3.120790in}}%
\pgfpathlineto{\pgfqpoint{3.385571in}{3.162494in}}%
\pgfpathlineto{\pgfqpoint{3.420228in}{3.201421in}}%
\pgfpathlineto{\pgfqpoint{3.454885in}{3.237418in}}%
\pgfpathlineto{\pgfqpoint{3.489542in}{3.270329in}}%
\pgfpathlineto{\pgfqpoint{3.524198in}{3.300000in}}%
\pgfpathlineto{\pgfqpoint{3.558855in}{3.326277in}}%
\pgfpathlineto{\pgfqpoint{3.593512in}{3.349006in}}%
\pgfpathlineto{\pgfqpoint{3.628169in}{3.368033in}}%
\pgfpathlineto{\pgfqpoint{3.662826in}{3.383203in}}%
\pgfpathlineto{\pgfqpoint{3.697482in}{3.394362in}}%
\pgfpathlineto{\pgfqpoint{3.732139in}{3.401355in}}%
\pgfpathlineto{\pgfqpoint{3.766796in}{3.404029in}}%
\pgfpathlineto{\pgfqpoint{3.801453in}{3.402230in}}%
\pgfpathlineto{\pgfqpoint{3.836109in}{3.395802in}}%
\pgfpathlineto{\pgfqpoint{3.870766in}{3.384591in}}%
\pgfpathlineto{\pgfqpoint{3.905423in}{3.368444in}}%
\pgfpathlineto{\pgfqpoint{3.940080in}{3.347207in}}%
\pgfpathlineto{\pgfqpoint{3.974737in}{3.320724in}}%
\pgfpathlineto{\pgfqpoint{4.009393in}{3.288841in}}%
\pgfpathlineto{\pgfqpoint{4.044050in}{3.251405in}}%
\pgfpathlineto{\pgfqpoint{4.078707in}{3.208261in}}%
\pgfpathlineto{\pgfqpoint{4.113364in}{3.159254in}}%
\pgfpathlineto{\pgfqpoint{4.148020in}{3.104231in}}%
\pgfpathlineto{\pgfqpoint{4.182677in}{3.043038in}}%
\pgfpathlineto{\pgfqpoint{4.217334in}{2.975519in}}%
\pgfpathlineto{\pgfqpoint{4.251991in}{2.901521in}}%
\pgfpathlineto{\pgfqpoint{4.286648in}{2.820889in}}%
\pgfpathlineto{\pgfqpoint{4.321304in}{2.733469in}}%
\pgfpathlineto{\pgfqpoint{4.355961in}{2.639107in}}%
\pgfpathlineto{\pgfqpoint{4.390618in}{2.537649in}}%
\pgfpathlineto{\pgfqpoint{4.425275in}{2.428940in}}%
\pgfpathlineto{\pgfqpoint{4.459931in}{2.312826in}}%
\pgfpathlineto{\pgfqpoint{4.494588in}{2.189153in}}%
\pgfpathlineto{\pgfqpoint{4.529245in}{2.057767in}}%
\pgfpathlineto{\pgfqpoint{4.563902in}{1.918512in}}%
\pgfpathlineto{\pgfqpoint{4.598559in}{1.771236in}}%
\pgfpathlineto{\pgfqpoint{4.633215in}{1.615783in}}%
\pgfpathlineto{\pgfqpoint{4.667872in}{1.452000in}}%
\pgfusepath{stroke}%
\end{pgfscope}%
\begin{pgfscope}%
\pgfpathrectangle{\pgfqpoint{0.800000in}{0.528000in}}{\pgfqpoint{4.960000in}{3.696000in}}%
\pgfusepath{clip}%
\pgfsetrectcap%
\pgfsetroundjoin%
\pgfsetlinewidth{1.505625pt}%
\definecolor{currentstroke}{rgb}{0.121569,0.466667,0.705882}%
\pgfsetstrokecolor{currentstroke}%
\pgfsetdash{}{0pt}%
\pgfpathmoveto{\pgfqpoint{1.236851in}{1.452000in}}%
\pgfpathlineto{\pgfqpoint{1.271508in}{1.469827in}}%
\pgfpathlineto{\pgfqpoint{1.306165in}{1.486008in}}%
\pgfpathlineto{\pgfqpoint{1.340821in}{1.500595in}}%
\pgfpathlineto{\pgfqpoint{1.375478in}{1.513639in}}%
\pgfpathlineto{\pgfqpoint{1.410135in}{1.525193in}}%
\pgfpathlineto{\pgfqpoint{1.444792in}{1.535306in}}%
\pgfpathlineto{\pgfqpoint{1.479449in}{1.544031in}}%
\pgfpathlineto{\pgfqpoint{1.514105in}{1.551418in}}%
\pgfpathlineto{\pgfqpoint{1.548762in}{1.557521in}}%
\pgfpathlineto{\pgfqpoint{1.583419in}{1.562389in}}%
\pgfpathlineto{\pgfqpoint{1.618076in}{1.566074in}}%
\pgfpathlineto{\pgfqpoint{1.652732in}{1.568628in}}%
\pgfpathlineto{\pgfqpoint{1.687389in}{1.570102in}}%
\pgfpathlineto{\pgfqpoint{1.722046in}{1.570548in}}%
\pgfpathlineto{\pgfqpoint{1.756703in}{1.570017in}}%
\pgfpathlineto{\pgfqpoint{1.791360in}{1.568560in}}%
\pgfpathlineto{\pgfqpoint{1.826016in}{1.566228in}}%
\pgfpathlineto{\pgfqpoint{1.860673in}{1.563074in}}%
\pgfpathlineto{\pgfqpoint{1.895330in}{1.559149in}}%
\pgfpathlineto{\pgfqpoint{1.929987in}{1.554504in}}%
\pgfpathlineto{\pgfqpoint{1.964643in}{1.549190in}}%
\pgfpathlineto{\pgfqpoint{1.999300in}{1.543259in}}%
\pgfpathlineto{\pgfqpoint{2.033957in}{1.536763in}}%
\pgfpathlineto{\pgfqpoint{2.068614in}{1.529752in}}%
\pgfpathlineto{\pgfqpoint{2.103271in}{1.522279in}}%
\pgfpathlineto{\pgfqpoint{2.137927in}{1.514394in}}%
\pgfpathlineto{\pgfqpoint{2.172584in}{1.506149in}}%
\pgfpathlineto{\pgfqpoint{2.207241in}{1.497595in}}%
\pgfpathlineto{\pgfqpoint{2.241898in}{1.488785in}}%
\pgfpathlineto{\pgfqpoint{2.276554in}{1.479769in}}%
\pgfpathlineto{\pgfqpoint{2.311211in}{1.470598in}}%
\pgfpathlineto{\pgfqpoint{2.345868in}{1.461325in}}%
\pgfpathlineto{\pgfqpoint{2.380525in}{1.452000in}}%
\pgfpathlineto{\pgfqpoint{2.415182in}{1.442675in}}%
\pgfpathlineto{\pgfqpoint{2.449838in}{1.433402in}}%
\pgfpathlineto{\pgfqpoint{2.484495in}{1.424231in}}%
\pgfpathlineto{\pgfqpoint{2.519152in}{1.415215in}}%
\pgfpathlineto{\pgfqpoint{2.553809in}{1.406405in}}%
\pgfpathlineto{\pgfqpoint{2.588465in}{1.397851in}}%
\pgfpathlineto{\pgfqpoint{2.623122in}{1.389606in}}%
\pgfpathlineto{\pgfqpoint{2.657779in}{1.381721in}}%
\pgfpathlineto{\pgfqpoint{2.692436in}{1.374248in}}%
\pgfpathlineto{\pgfqpoint{2.727093in}{1.367237in}}%
\pgfpathlineto{\pgfqpoint{2.761749in}{1.360741in}}%
\pgfpathlineto{\pgfqpoint{2.796406in}{1.354810in}}%
\pgfpathlineto{\pgfqpoint{2.831063in}{1.349496in}}%
\pgfpathlineto{\pgfqpoint{2.865720in}{1.344851in}}%
\pgfpathlineto{\pgfqpoint{2.900376in}{1.340926in}}%
\pgfpathlineto{\pgfqpoint{2.935033in}{1.337772in}}%
\pgfpathlineto{\pgfqpoint{2.969690in}{1.335440in}}%
\pgfpathlineto{\pgfqpoint{3.004347in}{1.333983in}}%
\pgfpathlineto{\pgfqpoint{3.039004in}{1.333452in}}%
\pgfpathlineto{\pgfqpoint{3.073660in}{1.333898in}}%
\pgfpathlineto{\pgfqpoint{3.108317in}{1.335372in}}%
\pgfpathlineto{\pgfqpoint{3.142974in}{1.337926in}}%
\pgfpathlineto{\pgfqpoint{3.177631in}{1.341611in}}%
\pgfpathlineto{\pgfqpoint{3.212287in}{1.346479in}}%
\pgfpathlineto{\pgfqpoint{3.246944in}{1.352582in}}%
\pgfpathlineto{\pgfqpoint{3.281601in}{1.359969in}}%
\pgfpathlineto{\pgfqpoint{3.316258in}{1.368694in}}%
\pgfpathlineto{\pgfqpoint{3.350915in}{1.378807in}}%
\pgfpathlineto{\pgfqpoint{3.385571in}{1.390361in}}%
\pgfpathlineto{\pgfqpoint{3.420228in}{1.403405in}}%
\pgfpathlineto{\pgfqpoint{3.454885in}{1.417992in}}%
\pgfpathlineto{\pgfqpoint{3.489542in}{1.434173in}}%
\pgfpathlineto{\pgfqpoint{3.524198in}{1.452000in}}%
\pgfpathlineto{\pgfqpoint{3.558855in}{1.471524in}}%
\pgfpathlineto{\pgfqpoint{3.593512in}{1.492796in}}%
\pgfpathlineto{\pgfqpoint{3.628169in}{1.515868in}}%
\pgfpathlineto{\pgfqpoint{3.662826in}{1.540791in}}%
\pgfpathlineto{\pgfqpoint{3.697482in}{1.567617in}}%
\pgfpathlineto{\pgfqpoint{3.732139in}{1.596397in}}%
\pgfpathlineto{\pgfqpoint{3.766796in}{1.627182in}}%
\pgfpathlineto{\pgfqpoint{3.801453in}{1.660024in}}%
\pgfpathlineto{\pgfqpoint{3.836109in}{1.694975in}}%
\pgfpathlineto{\pgfqpoint{3.870766in}{1.732086in}}%
\pgfpathlineto{\pgfqpoint{3.905423in}{1.771407in}}%
\pgfpathlineto{\pgfqpoint{3.940080in}{1.812992in}}%
\pgfpathlineto{\pgfqpoint{3.974737in}{1.856890in}}%
\pgfpathlineto{\pgfqpoint{4.009393in}{1.903154in}}%
\pgfpathlineto{\pgfqpoint{4.044050in}{1.951835in}}%
\pgfpathlineto{\pgfqpoint{4.078707in}{2.002984in}}%
\pgfpathlineto{\pgfqpoint{4.113364in}{2.056653in}}%
\pgfpathlineto{\pgfqpoint{4.148020in}{2.112893in}}%
\pgfpathlineto{\pgfqpoint{4.182677in}{2.171755in}}%
\pgfpathlineto{\pgfqpoint{4.217334in}{2.233292in}}%
\pgfpathlineto{\pgfqpoint{4.251991in}{2.297554in}}%
\pgfpathlineto{\pgfqpoint{4.286648in}{2.364593in}}%
\pgfpathlineto{\pgfqpoint{4.321304in}{2.434460in}}%
\pgfpathlineto{\pgfqpoint{4.355961in}{2.507207in}}%
\pgfpathlineto{\pgfqpoint{4.390618in}{2.582885in}}%
\pgfpathlineto{\pgfqpoint{4.425275in}{2.661545in}}%
\pgfpathlineto{\pgfqpoint{4.459931in}{2.743240in}}%
\pgfpathlineto{\pgfqpoint{4.494588in}{2.828020in}}%
\pgfpathlineto{\pgfqpoint{4.529245in}{2.915936in}}%
\pgfpathlineto{\pgfqpoint{4.563902in}{3.007041in}}%
\pgfpathlineto{\pgfqpoint{4.598559in}{3.101386in}}%
\pgfpathlineto{\pgfqpoint{4.633215in}{3.199022in}}%
\pgfpathlineto{\pgfqpoint{4.667872in}{3.300000in}}%
\pgfusepath{stroke}%
\end{pgfscope}%
\begin{pgfscope}%
\pgfsetrectcap%
\pgfsetmiterjoin%
\pgfsetlinewidth{0.803000pt}%
\definecolor{currentstroke}{rgb}{0.000000,0.000000,0.000000}%
\pgfsetstrokecolor{currentstroke}%
\pgfsetdash{}{0pt}%
\pgfpathmoveto{\pgfqpoint{0.800000in}{0.528000in}}%
\pgfpathlineto{\pgfqpoint{0.800000in}{4.224000in}}%
\pgfusepath{stroke}%
\end{pgfscope}%
\begin{pgfscope}%
\pgfsetrectcap%
\pgfsetmiterjoin%
\pgfsetlinewidth{0.803000pt}%
\definecolor{currentstroke}{rgb}{0.000000,0.000000,0.000000}%
\pgfsetstrokecolor{currentstroke}%
\pgfsetdash{}{0pt}%
\pgfpathmoveto{\pgfqpoint{5.760000in}{0.528000in}}%
\pgfpathlineto{\pgfqpoint{5.760000in}{4.224000in}}%
\pgfusepath{stroke}%
\end{pgfscope}%
\begin{pgfscope}%
\pgfsetrectcap%
\pgfsetmiterjoin%
\pgfsetlinewidth{0.803000pt}%
\definecolor{currentstroke}{rgb}{0.000000,0.000000,0.000000}%
\pgfsetstrokecolor{currentstroke}%
\pgfsetdash{}{0pt}%
\pgfpathmoveto{\pgfqpoint{0.800000in}{0.528000in}}%
\pgfpathlineto{\pgfqpoint{5.760000in}{0.528000in}}%
\pgfusepath{stroke}%
\end{pgfscope}%
\begin{pgfscope}%
\pgfsetrectcap%
\pgfsetmiterjoin%
\pgfsetlinewidth{0.803000pt}%
\definecolor{currentstroke}{rgb}{0.000000,0.000000,0.000000}%
\pgfsetstrokecolor{currentstroke}%
\pgfsetdash{}{0pt}%
\pgfpathmoveto{\pgfqpoint{0.800000in}{4.224000in}}%
\pgfpathlineto{\pgfqpoint{5.760000in}{4.224000in}}%
\pgfusepath{stroke}%
\end{pgfscope}%
\begin{pgfscope}%
\pgfsetbuttcap%
\pgfsetmiterjoin%
\definecolor{currentfill}{rgb}{1.000000,1.000000,1.000000}%
\pgfsetfillcolor{currentfill}%
\pgfsetfillopacity{0.800000}%
\pgfsetlinewidth{1.003750pt}%
\definecolor{currentstroke}{rgb}{0.800000,0.800000,0.800000}%
\pgfsetstrokecolor{currentstroke}%
\pgfsetstrokeopacity{0.800000}%
\pgfsetdash{}{0pt}%
\pgfpathmoveto{\pgfqpoint{4.479134in}{1.834333in}}%
\pgfpathlineto{\pgfqpoint{5.662778in}{1.834333in}}%
\pgfpathquadraticcurveto{\pgfqpoint{5.690556in}{1.834333in}}{\pgfqpoint{5.690556in}{1.862111in}}%
\pgfpathlineto{\pgfqpoint{5.690556in}{2.889889in}}%
\pgfpathquadraticcurveto{\pgfqpoint{5.690556in}{2.917667in}}{\pgfqpoint{5.662778in}{2.917667in}}%
\pgfpathlineto{\pgfqpoint{4.479134in}{2.917667in}}%
\pgfpathquadraticcurveto{\pgfqpoint{4.451356in}{2.917667in}}{\pgfqpoint{4.451356in}{2.889889in}}%
\pgfpathlineto{\pgfqpoint{4.451356in}{1.862111in}}%
\pgfpathquadraticcurveto{\pgfqpoint{4.451356in}{1.834333in}}{\pgfqpoint{4.479134in}{1.834333in}}%
\pgfpathclose%
\pgfusepath{stroke,fill}%
\end{pgfscope}%
\begin{pgfscope}%
\pgfsetrectcap%
\pgfsetroundjoin%
\pgfsetlinewidth{1.505625pt}%
\definecolor{currentstroke}{rgb}{0.000000,0.000000,1.000000}%
\pgfsetstrokecolor{currentstroke}%
\pgfsetdash{}{0pt}%
\pgfpathmoveto{\pgfqpoint{4.506912in}{2.806556in}}%
\pgfpathlineto{\pgfqpoint{4.784689in}{2.806556in}}%
\pgfusepath{stroke}%
\end{pgfscope}%
\begin{pgfscope}%
\definecolor{textcolor}{rgb}{0.000000,0.000000,0.000000}%
\pgfsetstrokecolor{textcolor}%
\pgfsetfillcolor{textcolor}%
\pgftext[x=4.895800in,y=2.757944in,left,base]{\color{textcolor}\rmfamily\fontsize{10.000000}{12.000000}\selectfont Lagrange(x)}%
\end{pgfscope}%
\begin{pgfscope}%
\pgfsetrectcap%
\pgfsetroundjoin%
\pgfsetlinewidth{1.505625pt}%
\definecolor{currentstroke}{rgb}{1.000000,0.000000,0.000000}%
\pgfsetstrokecolor{currentstroke}%
\pgfsetdash{}{0pt}%
\pgfpathmoveto{\pgfqpoint{4.506912in}{2.598222in}}%
\pgfpathlineto{\pgfqpoint{4.784689in}{2.598222in}}%
\pgfusepath{stroke}%
\end{pgfscope}%
\begin{pgfscope}%
\definecolor{textcolor}{rgb}{0.000000,0.000000,0.000000}%
\pgfsetstrokecolor{textcolor}%
\pgfsetfillcolor{textcolor}%
\pgftext[x=4.895800in,y=2.549611in,left,base]{\color{textcolor}\rmfamily\fontsize{10.000000}{12.000000}\selectfont \(\displaystyle L_0(x)\)}%
\end{pgfscope}%
\begin{pgfscope}%
\pgfsetrectcap%
\pgfsetroundjoin%
\pgfsetlinewidth{1.505625pt}%
\definecolor{currentstroke}{rgb}{0.000000,0.500000,0.000000}%
\pgfsetstrokecolor{currentstroke}%
\pgfsetdash{}{0pt}%
\pgfpathmoveto{\pgfqpoint{4.506912in}{2.389889in}}%
\pgfpathlineto{\pgfqpoint{4.784689in}{2.389889in}}%
\pgfusepath{stroke}%
\end{pgfscope}%
\begin{pgfscope}%
\definecolor{textcolor}{rgb}{0.000000,0.000000,0.000000}%
\pgfsetstrokecolor{textcolor}%
\pgfsetfillcolor{textcolor}%
\pgftext[x=4.895800in,y=2.341278in,left,base]{\color{textcolor}\rmfamily\fontsize{10.000000}{12.000000}\selectfont \(\displaystyle L_1(x)\)}%
\end{pgfscope}%
\begin{pgfscope}%
\pgfsetrectcap%
\pgfsetroundjoin%
\pgfsetlinewidth{1.505625pt}%
\definecolor{currentstroke}{rgb}{0.750000,0.750000,0.000000}%
\pgfsetstrokecolor{currentstroke}%
\pgfsetdash{}{0pt}%
\pgfpathmoveto{\pgfqpoint{4.506912in}{2.181556in}}%
\pgfpathlineto{\pgfqpoint{4.784689in}{2.181556in}}%
\pgfusepath{stroke}%
\end{pgfscope}%
\begin{pgfscope}%
\definecolor{textcolor}{rgb}{0.000000,0.000000,0.000000}%
\pgfsetstrokecolor{textcolor}%
\pgfsetfillcolor{textcolor}%
\pgftext[x=4.895800in,y=2.132944in,left,base]{\color{textcolor}\rmfamily\fontsize{10.000000}{12.000000}\selectfont \(\displaystyle L_2(x)\)}%
\end{pgfscope}%
\begin{pgfscope}%
\pgfsetrectcap%
\pgfsetroundjoin%
\pgfsetlinewidth{1.505625pt}%
\definecolor{currentstroke}{rgb}{0.121569,0.466667,0.705882}%
\pgfsetstrokecolor{currentstroke}%
\pgfsetdash{}{0pt}%
\pgfpathmoveto{\pgfqpoint{4.506912in}{1.973222in}}%
\pgfpathlineto{\pgfqpoint{4.784689in}{1.973222in}}%
\pgfusepath{stroke}%
\end{pgfscope}%
\begin{pgfscope}%
\definecolor{textcolor}{rgb}{0.000000,0.000000,0.000000}%
\pgfsetstrokecolor{textcolor}%
\pgfsetfillcolor{textcolor}%
\pgftext[x=4.895800in,y=1.924611in,left,base]{\color{textcolor}\rmfamily\fontsize{10.000000}{12.000000}\selectfont \(\displaystyle L_3(x)\)}%
\end{pgfscope}%
\end{pgfpicture}%
\makeatother%
\endgroup%
}
                   \caption{Lagrange Interpolating Polynomial on $[0, \pi/2]$}
                   \label{fig:LagrangePlot}
                \end{center}
            \end{figure}
            
            \item \textbf{Newton Method} \newline
            Newton's method of finding the interpolating polynomial is slightly different. The polynomial is 
            of the form 

            \begin{equation}
                P_n(x) = f(x_0) + \sum_{i=1}^n f[x_0, \dots, x_i] (x-x_0)  \dots (x-x_{-1})
                \label{eqn:Newton Polynomial Form}
            \end{equation}
            where
            \begin{equation*}
                f[x_0, x_1, x_2] = \frac{f[x_0, x_1]-f[x_1, x_2]}{x_0-x_2} = \frac{\frac{f(x_0)-f(x_1)}{x_0-x_1}-\frac{f(x_1)-f(x_2)}{x_1-x_2}}{x_0-x_2}
            \end{equation*}

            Finding $P_3(x)$ in this case takes a lot of calculation but is relatively trivial and gives us 

            \begin{equation}
                P_3(x) = \frac{3x}{\pi}\frac{9(\sqrt{3}-2)}{\pi^2}(x^2-\frac{x\pi}{6})+\frac{18(5-3\sqrt{3})}{\pi^3}(x^3-\frac{x^2\pi}{2}+\frac{x\pi^2}{18})
                \label{eqn:Newton Polynomial}
            \end{equation}

            and plotting this gives us 

            \begin{figure}[H]
                \begin{center}
                   \scalebox{.7}{%% Creator: Matplotlib, PGF backend
%%
%% To include the figure in your LaTeX document, write
%%   \input{<filename>.pgf}
%%
%% Make sure the required packages are loaded in your preamble
%%   \usepackage{pgf}
%%
%% Figures using additional raster images can only be included by \input if
%% they are in the same directory as the main LaTeX file. For loading figures
%% from other directories you can use the `import` package
%%   \usepackage{import}
%% and then include the figures with
%%   \import{<path to file>}{<filename>.pgf}
%%
%% Matplotlib used the following preamble
%%
\begingroup%
\makeatletter%
\begin{pgfpicture}%
\pgfpathrectangle{\pgfpointorigin}{\pgfqpoint{6.400000in}{4.800000in}}%
\pgfusepath{use as bounding box, clip}%
\begin{pgfscope}%
\pgfsetbuttcap%
\pgfsetmiterjoin%
\definecolor{currentfill}{rgb}{1.000000,1.000000,1.000000}%
\pgfsetfillcolor{currentfill}%
\pgfsetlinewidth{0.000000pt}%
\definecolor{currentstroke}{rgb}{1.000000,1.000000,1.000000}%
\pgfsetstrokecolor{currentstroke}%
\pgfsetdash{}{0pt}%
\pgfpathmoveto{\pgfqpoint{0.000000in}{0.000000in}}%
\pgfpathlineto{\pgfqpoint{6.400000in}{0.000000in}}%
\pgfpathlineto{\pgfqpoint{6.400000in}{4.800000in}}%
\pgfpathlineto{\pgfqpoint{0.000000in}{4.800000in}}%
\pgfpathclose%
\pgfusepath{fill}%
\end{pgfscope}%
\begin{pgfscope}%
\pgfsetbuttcap%
\pgfsetmiterjoin%
\definecolor{currentfill}{rgb}{1.000000,1.000000,1.000000}%
\pgfsetfillcolor{currentfill}%
\pgfsetlinewidth{0.000000pt}%
\definecolor{currentstroke}{rgb}{0.000000,0.000000,0.000000}%
\pgfsetstrokecolor{currentstroke}%
\pgfsetstrokeopacity{0.000000}%
\pgfsetdash{}{0pt}%
\pgfpathmoveto{\pgfqpoint{0.800000in}{0.528000in}}%
\pgfpathlineto{\pgfqpoint{5.760000in}{0.528000in}}%
\pgfpathlineto{\pgfqpoint{5.760000in}{4.224000in}}%
\pgfpathlineto{\pgfqpoint{0.800000in}{4.224000in}}%
\pgfpathclose%
\pgfusepath{fill}%
\end{pgfscope}%
\begin{pgfscope}%
\pgfpathrectangle{\pgfqpoint{0.800000in}{0.528000in}}{\pgfqpoint{4.960000in}{3.696000in}}%
\pgfusepath{clip}%
\pgfsetbuttcap%
\pgfsetroundjoin%
\pgfsetlinewidth{0.803000pt}%
\definecolor{currentstroke}{rgb}{0.800000,0.800000,0.800000}%
\pgfsetstrokecolor{currentstroke}%
\pgfsetdash{{0.800000pt}{1.320000pt}}{0.000000pt}%
\pgfpathmoveto{\pgfqpoint{1.236851in}{0.528000in}}%
\pgfpathlineto{\pgfqpoint{1.236851in}{4.224000in}}%
\pgfusepath{stroke}%
\end{pgfscope}%
\begin{pgfscope}%
\pgfsetbuttcap%
\pgfsetroundjoin%
\definecolor{currentfill}{rgb}{0.000000,0.000000,0.000000}%
\pgfsetfillcolor{currentfill}%
\pgfsetlinewidth{0.803000pt}%
\definecolor{currentstroke}{rgb}{0.000000,0.000000,0.000000}%
\pgfsetstrokecolor{currentstroke}%
\pgfsetdash{}{0pt}%
\pgfsys@defobject{currentmarker}{\pgfqpoint{0.000000in}{-0.048611in}}{\pgfqpoint{0.000000in}{0.000000in}}{%
\pgfpathmoveto{\pgfqpoint{0.000000in}{0.000000in}}%
\pgfpathlineto{\pgfqpoint{0.000000in}{-0.048611in}}%
\pgfusepath{stroke,fill}%
}%
\begin{pgfscope}%
\pgfsys@transformshift{1.236851in}{0.528000in}%
\pgfsys@useobject{currentmarker}{}%
\end{pgfscope}%
\end{pgfscope}%
\begin{pgfscope}%
\definecolor{textcolor}{rgb}{0.000000,0.000000,0.000000}%
\pgfsetstrokecolor{textcolor}%
\pgfsetfillcolor{textcolor}%
\pgftext[x=1.236851in,y=0.430778in,,top]{\color{textcolor}\rmfamily\fontsize{10.000000}{12.000000}\selectfont \(\displaystyle 0.0\)}%
\end{pgfscope}%
\begin{pgfscope}%
\pgfpathrectangle{\pgfqpoint{0.800000in}{0.528000in}}{\pgfqpoint{4.960000in}{3.696000in}}%
\pgfusepath{clip}%
\pgfsetbuttcap%
\pgfsetroundjoin%
\pgfsetlinewidth{0.803000pt}%
\definecolor{currentstroke}{rgb}{0.800000,0.800000,0.800000}%
\pgfsetstrokecolor{currentstroke}%
\pgfsetdash{{0.800000pt}{1.320000pt}}{0.000000pt}%
\pgfpathmoveto{\pgfqpoint{2.328979in}{0.528000in}}%
\pgfpathlineto{\pgfqpoint{2.328979in}{4.224000in}}%
\pgfusepath{stroke}%
\end{pgfscope}%
\begin{pgfscope}%
\pgfsetbuttcap%
\pgfsetroundjoin%
\definecolor{currentfill}{rgb}{0.000000,0.000000,0.000000}%
\pgfsetfillcolor{currentfill}%
\pgfsetlinewidth{0.803000pt}%
\definecolor{currentstroke}{rgb}{0.000000,0.000000,0.000000}%
\pgfsetstrokecolor{currentstroke}%
\pgfsetdash{}{0pt}%
\pgfsys@defobject{currentmarker}{\pgfqpoint{0.000000in}{-0.048611in}}{\pgfqpoint{0.000000in}{0.000000in}}{%
\pgfpathmoveto{\pgfqpoint{0.000000in}{0.000000in}}%
\pgfpathlineto{\pgfqpoint{0.000000in}{-0.048611in}}%
\pgfusepath{stroke,fill}%
}%
\begin{pgfscope}%
\pgfsys@transformshift{2.328979in}{0.528000in}%
\pgfsys@useobject{currentmarker}{}%
\end{pgfscope}%
\end{pgfscope}%
\begin{pgfscope}%
\definecolor{textcolor}{rgb}{0.000000,0.000000,0.000000}%
\pgfsetstrokecolor{textcolor}%
\pgfsetfillcolor{textcolor}%
\pgftext[x=2.328979in,y=0.430778in,,top]{\color{textcolor}\rmfamily\fontsize{10.000000}{12.000000}\selectfont \(\displaystyle 0.5\)}%
\end{pgfscope}%
\begin{pgfscope}%
\pgfpathrectangle{\pgfqpoint{0.800000in}{0.528000in}}{\pgfqpoint{4.960000in}{3.696000in}}%
\pgfusepath{clip}%
\pgfsetbuttcap%
\pgfsetroundjoin%
\pgfsetlinewidth{0.803000pt}%
\definecolor{currentstroke}{rgb}{0.800000,0.800000,0.800000}%
\pgfsetstrokecolor{currentstroke}%
\pgfsetdash{{0.800000pt}{1.320000pt}}{0.000000pt}%
\pgfpathmoveto{\pgfqpoint{3.421107in}{0.528000in}}%
\pgfpathlineto{\pgfqpoint{3.421107in}{4.224000in}}%
\pgfusepath{stroke}%
\end{pgfscope}%
\begin{pgfscope}%
\pgfsetbuttcap%
\pgfsetroundjoin%
\definecolor{currentfill}{rgb}{0.000000,0.000000,0.000000}%
\pgfsetfillcolor{currentfill}%
\pgfsetlinewidth{0.803000pt}%
\definecolor{currentstroke}{rgb}{0.000000,0.000000,0.000000}%
\pgfsetstrokecolor{currentstroke}%
\pgfsetdash{}{0pt}%
\pgfsys@defobject{currentmarker}{\pgfqpoint{0.000000in}{-0.048611in}}{\pgfqpoint{0.000000in}{0.000000in}}{%
\pgfpathmoveto{\pgfqpoint{0.000000in}{0.000000in}}%
\pgfpathlineto{\pgfqpoint{0.000000in}{-0.048611in}}%
\pgfusepath{stroke,fill}%
}%
\begin{pgfscope}%
\pgfsys@transformshift{3.421107in}{0.528000in}%
\pgfsys@useobject{currentmarker}{}%
\end{pgfscope}%
\end{pgfscope}%
\begin{pgfscope}%
\definecolor{textcolor}{rgb}{0.000000,0.000000,0.000000}%
\pgfsetstrokecolor{textcolor}%
\pgfsetfillcolor{textcolor}%
\pgftext[x=3.421107in,y=0.430778in,,top]{\color{textcolor}\rmfamily\fontsize{10.000000}{12.000000}\selectfont \(\displaystyle 1.0\)}%
\end{pgfscope}%
\begin{pgfscope}%
\pgfpathrectangle{\pgfqpoint{0.800000in}{0.528000in}}{\pgfqpoint{4.960000in}{3.696000in}}%
\pgfusepath{clip}%
\pgfsetbuttcap%
\pgfsetroundjoin%
\pgfsetlinewidth{0.803000pt}%
\definecolor{currentstroke}{rgb}{0.800000,0.800000,0.800000}%
\pgfsetstrokecolor{currentstroke}%
\pgfsetdash{{0.800000pt}{1.320000pt}}{0.000000pt}%
\pgfpathmoveto{\pgfqpoint{4.513235in}{0.528000in}}%
\pgfpathlineto{\pgfqpoint{4.513235in}{4.224000in}}%
\pgfusepath{stroke}%
\end{pgfscope}%
\begin{pgfscope}%
\pgfsetbuttcap%
\pgfsetroundjoin%
\definecolor{currentfill}{rgb}{0.000000,0.000000,0.000000}%
\pgfsetfillcolor{currentfill}%
\pgfsetlinewidth{0.803000pt}%
\definecolor{currentstroke}{rgb}{0.000000,0.000000,0.000000}%
\pgfsetstrokecolor{currentstroke}%
\pgfsetdash{}{0pt}%
\pgfsys@defobject{currentmarker}{\pgfqpoint{0.000000in}{-0.048611in}}{\pgfqpoint{0.000000in}{0.000000in}}{%
\pgfpathmoveto{\pgfqpoint{0.000000in}{0.000000in}}%
\pgfpathlineto{\pgfqpoint{0.000000in}{-0.048611in}}%
\pgfusepath{stroke,fill}%
}%
\begin{pgfscope}%
\pgfsys@transformshift{4.513235in}{0.528000in}%
\pgfsys@useobject{currentmarker}{}%
\end{pgfscope}%
\end{pgfscope}%
\begin{pgfscope}%
\definecolor{textcolor}{rgb}{0.000000,0.000000,0.000000}%
\pgfsetstrokecolor{textcolor}%
\pgfsetfillcolor{textcolor}%
\pgftext[x=4.513235in,y=0.430778in,,top]{\color{textcolor}\rmfamily\fontsize{10.000000}{12.000000}\selectfont \(\displaystyle 1.5\)}%
\end{pgfscope}%
\begin{pgfscope}%
\pgfpathrectangle{\pgfqpoint{0.800000in}{0.528000in}}{\pgfqpoint{4.960000in}{3.696000in}}%
\pgfusepath{clip}%
\pgfsetbuttcap%
\pgfsetroundjoin%
\pgfsetlinewidth{0.803000pt}%
\definecolor{currentstroke}{rgb}{0.800000,0.800000,0.800000}%
\pgfsetstrokecolor{currentstroke}%
\pgfsetdash{{0.800000pt}{1.320000pt}}{0.000000pt}%
\pgfpathmoveto{\pgfqpoint{5.605363in}{0.528000in}}%
\pgfpathlineto{\pgfqpoint{5.605363in}{4.224000in}}%
\pgfusepath{stroke}%
\end{pgfscope}%
\begin{pgfscope}%
\pgfsetbuttcap%
\pgfsetroundjoin%
\definecolor{currentfill}{rgb}{0.000000,0.000000,0.000000}%
\pgfsetfillcolor{currentfill}%
\pgfsetlinewidth{0.803000pt}%
\definecolor{currentstroke}{rgb}{0.000000,0.000000,0.000000}%
\pgfsetstrokecolor{currentstroke}%
\pgfsetdash{}{0pt}%
\pgfsys@defobject{currentmarker}{\pgfqpoint{0.000000in}{-0.048611in}}{\pgfqpoint{0.000000in}{0.000000in}}{%
\pgfpathmoveto{\pgfqpoint{0.000000in}{0.000000in}}%
\pgfpathlineto{\pgfqpoint{0.000000in}{-0.048611in}}%
\pgfusepath{stroke,fill}%
}%
\begin{pgfscope}%
\pgfsys@transformshift{5.605363in}{0.528000in}%
\pgfsys@useobject{currentmarker}{}%
\end{pgfscope}%
\end{pgfscope}%
\begin{pgfscope}%
\definecolor{textcolor}{rgb}{0.000000,0.000000,0.000000}%
\pgfsetstrokecolor{textcolor}%
\pgfsetfillcolor{textcolor}%
\pgftext[x=5.605363in,y=0.430778in,,top]{\color{textcolor}\rmfamily\fontsize{10.000000}{12.000000}\selectfont \(\displaystyle 2.0\)}%
\end{pgfscope}%
\begin{pgfscope}%
\definecolor{textcolor}{rgb}{0.000000,0.000000,0.000000}%
\pgfsetstrokecolor{textcolor}%
\pgfsetfillcolor{textcolor}%
\pgftext[x=3.280000in,y=0.251766in,,top]{\color{textcolor}\rmfamily\fontsize{10.000000}{12.000000}\selectfont x}%
\end{pgfscope}%
\begin{pgfscope}%
\pgfpathrectangle{\pgfqpoint{0.800000in}{0.528000in}}{\pgfqpoint{4.960000in}{3.696000in}}%
\pgfusepath{clip}%
\pgfsetbuttcap%
\pgfsetroundjoin%
\pgfsetlinewidth{0.803000pt}%
\definecolor{currentstroke}{rgb}{0.800000,0.800000,0.800000}%
\pgfsetstrokecolor{currentstroke}%
\pgfsetdash{{0.800000pt}{1.320000pt}}{0.000000pt}%
\pgfpathmoveto{\pgfqpoint{0.800000in}{0.528000in}}%
\pgfpathlineto{\pgfqpoint{5.760000in}{0.528000in}}%
\pgfusepath{stroke}%
\end{pgfscope}%
\begin{pgfscope}%
\pgfsetbuttcap%
\pgfsetroundjoin%
\definecolor{currentfill}{rgb}{0.000000,0.000000,0.000000}%
\pgfsetfillcolor{currentfill}%
\pgfsetlinewidth{0.803000pt}%
\definecolor{currentstroke}{rgb}{0.000000,0.000000,0.000000}%
\pgfsetstrokecolor{currentstroke}%
\pgfsetdash{}{0pt}%
\pgfsys@defobject{currentmarker}{\pgfqpoint{-0.048611in}{0.000000in}}{\pgfqpoint{0.000000in}{0.000000in}}{%
\pgfpathmoveto{\pgfqpoint{0.000000in}{0.000000in}}%
\pgfpathlineto{\pgfqpoint{-0.048611in}{0.000000in}}%
\pgfusepath{stroke,fill}%
}%
\begin{pgfscope}%
\pgfsys@transformshift{0.800000in}{0.528000in}%
\pgfsys@useobject{currentmarker}{}%
\end{pgfscope}%
\end{pgfscope}%
\begin{pgfscope}%
\definecolor{textcolor}{rgb}{0.000000,0.000000,0.000000}%
\pgfsetstrokecolor{textcolor}%
\pgfsetfillcolor{textcolor}%
\pgftext[x=0.347838in,y=0.479775in,left,base]{\color{textcolor}\rmfamily\fontsize{10.000000}{12.000000}\selectfont \(\displaystyle -0.50\)}%
\end{pgfscope}%
\begin{pgfscope}%
\pgfpathrectangle{\pgfqpoint{0.800000in}{0.528000in}}{\pgfqpoint{4.960000in}{3.696000in}}%
\pgfusepath{clip}%
\pgfsetbuttcap%
\pgfsetroundjoin%
\pgfsetlinewidth{0.803000pt}%
\definecolor{currentstroke}{rgb}{0.800000,0.800000,0.800000}%
\pgfsetstrokecolor{currentstroke}%
\pgfsetdash{{0.800000pt}{1.320000pt}}{0.000000pt}%
\pgfpathmoveto{\pgfqpoint{0.800000in}{0.990000in}}%
\pgfpathlineto{\pgfqpoint{5.760000in}{0.990000in}}%
\pgfusepath{stroke}%
\end{pgfscope}%
\begin{pgfscope}%
\pgfsetbuttcap%
\pgfsetroundjoin%
\definecolor{currentfill}{rgb}{0.000000,0.000000,0.000000}%
\pgfsetfillcolor{currentfill}%
\pgfsetlinewidth{0.803000pt}%
\definecolor{currentstroke}{rgb}{0.000000,0.000000,0.000000}%
\pgfsetstrokecolor{currentstroke}%
\pgfsetdash{}{0pt}%
\pgfsys@defobject{currentmarker}{\pgfqpoint{-0.048611in}{0.000000in}}{\pgfqpoint{0.000000in}{0.000000in}}{%
\pgfpathmoveto{\pgfqpoint{0.000000in}{0.000000in}}%
\pgfpathlineto{\pgfqpoint{-0.048611in}{0.000000in}}%
\pgfusepath{stroke,fill}%
}%
\begin{pgfscope}%
\pgfsys@transformshift{0.800000in}{0.990000in}%
\pgfsys@useobject{currentmarker}{}%
\end{pgfscope}%
\end{pgfscope}%
\begin{pgfscope}%
\definecolor{textcolor}{rgb}{0.000000,0.000000,0.000000}%
\pgfsetstrokecolor{textcolor}%
\pgfsetfillcolor{textcolor}%
\pgftext[x=0.347838in,y=0.941775in,left,base]{\color{textcolor}\rmfamily\fontsize{10.000000}{12.000000}\selectfont \(\displaystyle -0.25\)}%
\end{pgfscope}%
\begin{pgfscope}%
\pgfpathrectangle{\pgfqpoint{0.800000in}{0.528000in}}{\pgfqpoint{4.960000in}{3.696000in}}%
\pgfusepath{clip}%
\pgfsetbuttcap%
\pgfsetroundjoin%
\pgfsetlinewidth{0.803000pt}%
\definecolor{currentstroke}{rgb}{0.800000,0.800000,0.800000}%
\pgfsetstrokecolor{currentstroke}%
\pgfsetdash{{0.800000pt}{1.320000pt}}{0.000000pt}%
\pgfpathmoveto{\pgfqpoint{0.800000in}{1.452000in}}%
\pgfpathlineto{\pgfqpoint{5.760000in}{1.452000in}}%
\pgfusepath{stroke}%
\end{pgfscope}%
\begin{pgfscope}%
\pgfsetbuttcap%
\pgfsetroundjoin%
\definecolor{currentfill}{rgb}{0.000000,0.000000,0.000000}%
\pgfsetfillcolor{currentfill}%
\pgfsetlinewidth{0.803000pt}%
\definecolor{currentstroke}{rgb}{0.000000,0.000000,0.000000}%
\pgfsetstrokecolor{currentstroke}%
\pgfsetdash{}{0pt}%
\pgfsys@defobject{currentmarker}{\pgfqpoint{-0.048611in}{0.000000in}}{\pgfqpoint{0.000000in}{0.000000in}}{%
\pgfpathmoveto{\pgfqpoint{0.000000in}{0.000000in}}%
\pgfpathlineto{\pgfqpoint{-0.048611in}{0.000000in}}%
\pgfusepath{stroke,fill}%
}%
\begin{pgfscope}%
\pgfsys@transformshift{0.800000in}{1.452000in}%
\pgfsys@useobject{currentmarker}{}%
\end{pgfscope}%
\end{pgfscope}%
\begin{pgfscope}%
\definecolor{textcolor}{rgb}{0.000000,0.000000,0.000000}%
\pgfsetstrokecolor{textcolor}%
\pgfsetfillcolor{textcolor}%
\pgftext[x=0.455863in,y=1.403775in,left,base]{\color{textcolor}\rmfamily\fontsize{10.000000}{12.000000}\selectfont \(\displaystyle 0.00\)}%
\end{pgfscope}%
\begin{pgfscope}%
\pgfpathrectangle{\pgfqpoint{0.800000in}{0.528000in}}{\pgfqpoint{4.960000in}{3.696000in}}%
\pgfusepath{clip}%
\pgfsetbuttcap%
\pgfsetroundjoin%
\pgfsetlinewidth{0.803000pt}%
\definecolor{currentstroke}{rgb}{0.800000,0.800000,0.800000}%
\pgfsetstrokecolor{currentstroke}%
\pgfsetdash{{0.800000pt}{1.320000pt}}{0.000000pt}%
\pgfpathmoveto{\pgfqpoint{0.800000in}{1.914000in}}%
\pgfpathlineto{\pgfqpoint{5.760000in}{1.914000in}}%
\pgfusepath{stroke}%
\end{pgfscope}%
\begin{pgfscope}%
\pgfsetbuttcap%
\pgfsetroundjoin%
\definecolor{currentfill}{rgb}{0.000000,0.000000,0.000000}%
\pgfsetfillcolor{currentfill}%
\pgfsetlinewidth{0.803000pt}%
\definecolor{currentstroke}{rgb}{0.000000,0.000000,0.000000}%
\pgfsetstrokecolor{currentstroke}%
\pgfsetdash{}{0pt}%
\pgfsys@defobject{currentmarker}{\pgfqpoint{-0.048611in}{0.000000in}}{\pgfqpoint{0.000000in}{0.000000in}}{%
\pgfpathmoveto{\pgfqpoint{0.000000in}{0.000000in}}%
\pgfpathlineto{\pgfqpoint{-0.048611in}{0.000000in}}%
\pgfusepath{stroke,fill}%
}%
\begin{pgfscope}%
\pgfsys@transformshift{0.800000in}{1.914000in}%
\pgfsys@useobject{currentmarker}{}%
\end{pgfscope}%
\end{pgfscope}%
\begin{pgfscope}%
\definecolor{textcolor}{rgb}{0.000000,0.000000,0.000000}%
\pgfsetstrokecolor{textcolor}%
\pgfsetfillcolor{textcolor}%
\pgftext[x=0.455863in,y=1.865775in,left,base]{\color{textcolor}\rmfamily\fontsize{10.000000}{12.000000}\selectfont \(\displaystyle 0.25\)}%
\end{pgfscope}%
\begin{pgfscope}%
\pgfpathrectangle{\pgfqpoint{0.800000in}{0.528000in}}{\pgfqpoint{4.960000in}{3.696000in}}%
\pgfusepath{clip}%
\pgfsetbuttcap%
\pgfsetroundjoin%
\pgfsetlinewidth{0.803000pt}%
\definecolor{currentstroke}{rgb}{0.800000,0.800000,0.800000}%
\pgfsetstrokecolor{currentstroke}%
\pgfsetdash{{0.800000pt}{1.320000pt}}{0.000000pt}%
\pgfpathmoveto{\pgfqpoint{0.800000in}{2.376000in}}%
\pgfpathlineto{\pgfqpoint{5.760000in}{2.376000in}}%
\pgfusepath{stroke}%
\end{pgfscope}%
\begin{pgfscope}%
\pgfsetbuttcap%
\pgfsetroundjoin%
\definecolor{currentfill}{rgb}{0.000000,0.000000,0.000000}%
\pgfsetfillcolor{currentfill}%
\pgfsetlinewidth{0.803000pt}%
\definecolor{currentstroke}{rgb}{0.000000,0.000000,0.000000}%
\pgfsetstrokecolor{currentstroke}%
\pgfsetdash{}{0pt}%
\pgfsys@defobject{currentmarker}{\pgfqpoint{-0.048611in}{0.000000in}}{\pgfqpoint{0.000000in}{0.000000in}}{%
\pgfpathmoveto{\pgfqpoint{0.000000in}{0.000000in}}%
\pgfpathlineto{\pgfqpoint{-0.048611in}{0.000000in}}%
\pgfusepath{stroke,fill}%
}%
\begin{pgfscope}%
\pgfsys@transformshift{0.800000in}{2.376000in}%
\pgfsys@useobject{currentmarker}{}%
\end{pgfscope}%
\end{pgfscope}%
\begin{pgfscope}%
\definecolor{textcolor}{rgb}{0.000000,0.000000,0.000000}%
\pgfsetstrokecolor{textcolor}%
\pgfsetfillcolor{textcolor}%
\pgftext[x=0.455863in,y=2.327775in,left,base]{\color{textcolor}\rmfamily\fontsize{10.000000}{12.000000}\selectfont \(\displaystyle 0.50\)}%
\end{pgfscope}%
\begin{pgfscope}%
\pgfpathrectangle{\pgfqpoint{0.800000in}{0.528000in}}{\pgfqpoint{4.960000in}{3.696000in}}%
\pgfusepath{clip}%
\pgfsetbuttcap%
\pgfsetroundjoin%
\pgfsetlinewidth{0.803000pt}%
\definecolor{currentstroke}{rgb}{0.800000,0.800000,0.800000}%
\pgfsetstrokecolor{currentstroke}%
\pgfsetdash{{0.800000pt}{1.320000pt}}{0.000000pt}%
\pgfpathmoveto{\pgfqpoint{0.800000in}{2.838000in}}%
\pgfpathlineto{\pgfqpoint{5.760000in}{2.838000in}}%
\pgfusepath{stroke}%
\end{pgfscope}%
\begin{pgfscope}%
\pgfsetbuttcap%
\pgfsetroundjoin%
\definecolor{currentfill}{rgb}{0.000000,0.000000,0.000000}%
\pgfsetfillcolor{currentfill}%
\pgfsetlinewidth{0.803000pt}%
\definecolor{currentstroke}{rgb}{0.000000,0.000000,0.000000}%
\pgfsetstrokecolor{currentstroke}%
\pgfsetdash{}{0pt}%
\pgfsys@defobject{currentmarker}{\pgfqpoint{-0.048611in}{0.000000in}}{\pgfqpoint{0.000000in}{0.000000in}}{%
\pgfpathmoveto{\pgfqpoint{0.000000in}{0.000000in}}%
\pgfpathlineto{\pgfqpoint{-0.048611in}{0.000000in}}%
\pgfusepath{stroke,fill}%
}%
\begin{pgfscope}%
\pgfsys@transformshift{0.800000in}{2.838000in}%
\pgfsys@useobject{currentmarker}{}%
\end{pgfscope}%
\end{pgfscope}%
\begin{pgfscope}%
\definecolor{textcolor}{rgb}{0.000000,0.000000,0.000000}%
\pgfsetstrokecolor{textcolor}%
\pgfsetfillcolor{textcolor}%
\pgftext[x=0.455863in,y=2.789775in,left,base]{\color{textcolor}\rmfamily\fontsize{10.000000}{12.000000}\selectfont \(\displaystyle 0.75\)}%
\end{pgfscope}%
\begin{pgfscope}%
\pgfpathrectangle{\pgfqpoint{0.800000in}{0.528000in}}{\pgfqpoint{4.960000in}{3.696000in}}%
\pgfusepath{clip}%
\pgfsetbuttcap%
\pgfsetroundjoin%
\pgfsetlinewidth{0.803000pt}%
\definecolor{currentstroke}{rgb}{0.800000,0.800000,0.800000}%
\pgfsetstrokecolor{currentstroke}%
\pgfsetdash{{0.800000pt}{1.320000pt}}{0.000000pt}%
\pgfpathmoveto{\pgfqpoint{0.800000in}{3.300000in}}%
\pgfpathlineto{\pgfqpoint{5.760000in}{3.300000in}}%
\pgfusepath{stroke}%
\end{pgfscope}%
\begin{pgfscope}%
\pgfsetbuttcap%
\pgfsetroundjoin%
\definecolor{currentfill}{rgb}{0.000000,0.000000,0.000000}%
\pgfsetfillcolor{currentfill}%
\pgfsetlinewidth{0.803000pt}%
\definecolor{currentstroke}{rgb}{0.000000,0.000000,0.000000}%
\pgfsetstrokecolor{currentstroke}%
\pgfsetdash{}{0pt}%
\pgfsys@defobject{currentmarker}{\pgfqpoint{-0.048611in}{0.000000in}}{\pgfqpoint{0.000000in}{0.000000in}}{%
\pgfpathmoveto{\pgfqpoint{0.000000in}{0.000000in}}%
\pgfpathlineto{\pgfqpoint{-0.048611in}{0.000000in}}%
\pgfusepath{stroke,fill}%
}%
\begin{pgfscope}%
\pgfsys@transformshift{0.800000in}{3.300000in}%
\pgfsys@useobject{currentmarker}{}%
\end{pgfscope}%
\end{pgfscope}%
\begin{pgfscope}%
\definecolor{textcolor}{rgb}{0.000000,0.000000,0.000000}%
\pgfsetstrokecolor{textcolor}%
\pgfsetfillcolor{textcolor}%
\pgftext[x=0.455863in,y=3.251775in,left,base]{\color{textcolor}\rmfamily\fontsize{10.000000}{12.000000}\selectfont \(\displaystyle 1.00\)}%
\end{pgfscope}%
\begin{pgfscope}%
\pgfpathrectangle{\pgfqpoint{0.800000in}{0.528000in}}{\pgfqpoint{4.960000in}{3.696000in}}%
\pgfusepath{clip}%
\pgfsetbuttcap%
\pgfsetroundjoin%
\pgfsetlinewidth{0.803000pt}%
\definecolor{currentstroke}{rgb}{0.800000,0.800000,0.800000}%
\pgfsetstrokecolor{currentstroke}%
\pgfsetdash{{0.800000pt}{1.320000pt}}{0.000000pt}%
\pgfpathmoveto{\pgfqpoint{0.800000in}{3.762000in}}%
\pgfpathlineto{\pgfqpoint{5.760000in}{3.762000in}}%
\pgfusepath{stroke}%
\end{pgfscope}%
\begin{pgfscope}%
\pgfsetbuttcap%
\pgfsetroundjoin%
\definecolor{currentfill}{rgb}{0.000000,0.000000,0.000000}%
\pgfsetfillcolor{currentfill}%
\pgfsetlinewidth{0.803000pt}%
\definecolor{currentstroke}{rgb}{0.000000,0.000000,0.000000}%
\pgfsetstrokecolor{currentstroke}%
\pgfsetdash{}{0pt}%
\pgfsys@defobject{currentmarker}{\pgfqpoint{-0.048611in}{0.000000in}}{\pgfqpoint{0.000000in}{0.000000in}}{%
\pgfpathmoveto{\pgfqpoint{0.000000in}{0.000000in}}%
\pgfpathlineto{\pgfqpoint{-0.048611in}{0.000000in}}%
\pgfusepath{stroke,fill}%
}%
\begin{pgfscope}%
\pgfsys@transformshift{0.800000in}{3.762000in}%
\pgfsys@useobject{currentmarker}{}%
\end{pgfscope}%
\end{pgfscope}%
\begin{pgfscope}%
\definecolor{textcolor}{rgb}{0.000000,0.000000,0.000000}%
\pgfsetstrokecolor{textcolor}%
\pgfsetfillcolor{textcolor}%
\pgftext[x=0.455863in,y=3.713775in,left,base]{\color{textcolor}\rmfamily\fontsize{10.000000}{12.000000}\selectfont \(\displaystyle 1.25\)}%
\end{pgfscope}%
\begin{pgfscope}%
\pgfpathrectangle{\pgfqpoint{0.800000in}{0.528000in}}{\pgfqpoint{4.960000in}{3.696000in}}%
\pgfusepath{clip}%
\pgfsetbuttcap%
\pgfsetroundjoin%
\pgfsetlinewidth{0.803000pt}%
\definecolor{currentstroke}{rgb}{0.800000,0.800000,0.800000}%
\pgfsetstrokecolor{currentstroke}%
\pgfsetdash{{0.800000pt}{1.320000pt}}{0.000000pt}%
\pgfpathmoveto{\pgfqpoint{0.800000in}{4.224000in}}%
\pgfpathlineto{\pgfqpoint{5.760000in}{4.224000in}}%
\pgfusepath{stroke}%
\end{pgfscope}%
\begin{pgfscope}%
\pgfsetbuttcap%
\pgfsetroundjoin%
\definecolor{currentfill}{rgb}{0.000000,0.000000,0.000000}%
\pgfsetfillcolor{currentfill}%
\pgfsetlinewidth{0.803000pt}%
\definecolor{currentstroke}{rgb}{0.000000,0.000000,0.000000}%
\pgfsetstrokecolor{currentstroke}%
\pgfsetdash{}{0pt}%
\pgfsys@defobject{currentmarker}{\pgfqpoint{-0.048611in}{0.000000in}}{\pgfqpoint{0.000000in}{0.000000in}}{%
\pgfpathmoveto{\pgfqpoint{0.000000in}{0.000000in}}%
\pgfpathlineto{\pgfqpoint{-0.048611in}{0.000000in}}%
\pgfusepath{stroke,fill}%
}%
\begin{pgfscope}%
\pgfsys@transformshift{0.800000in}{4.224000in}%
\pgfsys@useobject{currentmarker}{}%
\end{pgfscope}%
\end{pgfscope}%
\begin{pgfscope}%
\definecolor{textcolor}{rgb}{0.000000,0.000000,0.000000}%
\pgfsetstrokecolor{textcolor}%
\pgfsetfillcolor{textcolor}%
\pgftext[x=0.455863in,y=4.175775in,left,base]{\color{textcolor}\rmfamily\fontsize{10.000000}{12.000000}\selectfont \(\displaystyle 1.50\)}%
\end{pgfscope}%
\begin{pgfscope}%
\definecolor{textcolor}{rgb}{0.000000,0.000000,0.000000}%
\pgfsetstrokecolor{textcolor}%
\pgfsetfillcolor{textcolor}%
\pgftext[x=0.292283in,y=2.376000in,,bottom]{\color{textcolor}\rmfamily\fontsize{10.000000}{12.000000}\selectfont y}%
\end{pgfscope}%
\begin{pgfscope}%
\pgfpathrectangle{\pgfqpoint{0.800000in}{0.528000in}}{\pgfqpoint{4.960000in}{3.696000in}}%
\pgfusepath{clip}%
\pgfsetrectcap%
\pgfsetroundjoin%
\pgfsetlinewidth{1.505625pt}%
\definecolor{currentstroke}{rgb}{0.000000,0.000000,0.000000}%
\pgfsetstrokecolor{currentstroke}%
\pgfsetdash{}{0pt}%
\pgfpathmoveto{\pgfqpoint{0.790000in}{1.452000in}}%
\pgfpathlineto{\pgfqpoint{5.770000in}{1.452000in}}%
\pgfpathlineto{\pgfqpoint{5.770000in}{1.452000in}}%
\pgfusepath{stroke}%
\end{pgfscope}%
\begin{pgfscope}%
\pgfpathrectangle{\pgfqpoint{0.800000in}{0.528000in}}{\pgfqpoint{4.960000in}{3.696000in}}%
\pgfusepath{clip}%
\pgfsetrectcap%
\pgfsetroundjoin%
\pgfsetlinewidth{1.505625pt}%
\definecolor{currentstroke}{rgb}{0.000000,0.000000,0.000000}%
\pgfsetstrokecolor{currentstroke}%
\pgfsetdash{}{0pt}%
\pgfpathmoveto{\pgfqpoint{1.236851in}{0.518000in}}%
\pgfpathlineto{\pgfqpoint{1.236851in}{4.234000in}}%
\pgfpathlineto{\pgfqpoint{1.236851in}{4.234000in}}%
\pgfusepath{stroke}%
\end{pgfscope}%
\begin{pgfscope}%
\pgfpathrectangle{\pgfqpoint{0.800000in}{0.528000in}}{\pgfqpoint{4.960000in}{3.696000in}}%
\pgfusepath{clip}%
\pgfsetrectcap%
\pgfsetroundjoin%
\pgfsetlinewidth{1.505625pt}%
\definecolor{currentstroke}{rgb}{0.000000,0.000000,1.000000}%
\pgfsetstrokecolor{currentstroke}%
\pgfsetdash{}{0pt}%
\pgfpathmoveto{\pgfqpoint{1.236851in}{1.452000in}}%
\pgfpathlineto{\pgfqpoint{1.271508in}{1.481889in}}%
\pgfpathlineto{\pgfqpoint{1.306165in}{1.511713in}}%
\pgfpathlineto{\pgfqpoint{1.340821in}{1.541465in}}%
\pgfpathlineto{\pgfqpoint{1.375478in}{1.571141in}}%
\pgfpathlineto{\pgfqpoint{1.410135in}{1.600736in}}%
\pgfpathlineto{\pgfqpoint{1.444792in}{1.630245in}}%
\pgfpathlineto{\pgfqpoint{1.479449in}{1.659663in}}%
\pgfpathlineto{\pgfqpoint{1.514105in}{1.688985in}}%
\pgfpathlineto{\pgfqpoint{1.548762in}{1.718205in}}%
\pgfpathlineto{\pgfqpoint{1.583419in}{1.747319in}}%
\pgfpathlineto{\pgfqpoint{1.618076in}{1.776321in}}%
\pgfpathlineto{\pgfqpoint{1.652732in}{1.805208in}}%
\pgfpathlineto{\pgfqpoint{1.687389in}{1.833973in}}%
\pgfpathlineto{\pgfqpoint{1.722046in}{1.862611in}}%
\pgfpathlineto{\pgfqpoint{1.756703in}{1.891118in}}%
\pgfpathlineto{\pgfqpoint{1.791360in}{1.919488in}}%
\pgfpathlineto{\pgfqpoint{1.826016in}{1.947717in}}%
\pgfpathlineto{\pgfqpoint{1.860673in}{1.975799in}}%
\pgfpathlineto{\pgfqpoint{1.895330in}{2.003729in}}%
\pgfpathlineto{\pgfqpoint{1.929987in}{2.031502in}}%
\pgfpathlineto{\pgfqpoint{1.964643in}{2.059114in}}%
\pgfpathlineto{\pgfqpoint{1.999300in}{2.086559in}}%
\pgfpathlineto{\pgfqpoint{2.033957in}{2.113832in}}%
\pgfpathlineto{\pgfqpoint{2.068614in}{2.140928in}}%
\pgfpathlineto{\pgfqpoint{2.103271in}{2.167842in}}%
\pgfpathlineto{\pgfqpoint{2.137927in}{2.194570in}}%
\pgfpathlineto{\pgfqpoint{2.172584in}{2.221105in}}%
\pgfpathlineto{\pgfqpoint{2.207241in}{2.247443in}}%
\pgfpathlineto{\pgfqpoint{2.241898in}{2.273579in}}%
\pgfpathlineto{\pgfqpoint{2.276554in}{2.299507in}}%
\pgfpathlineto{\pgfqpoint{2.311211in}{2.325224in}}%
\pgfpathlineto{\pgfqpoint{2.345868in}{2.350723in}}%
\pgfpathlineto{\pgfqpoint{2.380525in}{2.376000in}}%
\pgfpathlineto{\pgfqpoint{2.415182in}{2.401050in}}%
\pgfpathlineto{\pgfqpoint{2.449838in}{2.425867in}}%
\pgfpathlineto{\pgfqpoint{2.484495in}{2.450446in}}%
\pgfpathlineto{\pgfqpoint{2.519152in}{2.474784in}}%
\pgfpathlineto{\pgfqpoint{2.553809in}{2.498873in}}%
\pgfpathlineto{\pgfqpoint{2.588465in}{2.522711in}}%
\pgfpathlineto{\pgfqpoint{2.623122in}{2.546290in}}%
\pgfpathlineto{\pgfqpoint{2.657779in}{2.569607in}}%
\pgfpathlineto{\pgfqpoint{2.692436in}{2.592656in}}%
\pgfpathlineto{\pgfqpoint{2.727093in}{2.615433in}}%
\pgfpathlineto{\pgfqpoint{2.761749in}{2.637931in}}%
\pgfpathlineto{\pgfqpoint{2.796406in}{2.660147in}}%
\pgfpathlineto{\pgfqpoint{2.831063in}{2.682075in}}%
\pgfpathlineto{\pgfqpoint{2.865720in}{2.703710in}}%
\pgfpathlineto{\pgfqpoint{2.900376in}{2.725047in}}%
\pgfpathlineto{\pgfqpoint{2.935033in}{2.746082in}}%
\pgfpathlineto{\pgfqpoint{2.969690in}{2.766808in}}%
\pgfpathlineto{\pgfqpoint{3.004347in}{2.787221in}}%
\pgfpathlineto{\pgfqpoint{3.039004in}{2.807315in}}%
\pgfpathlineto{\pgfqpoint{3.073660in}{2.827087in}}%
\pgfpathlineto{\pgfqpoint{3.108317in}{2.846531in}}%
\pgfpathlineto{\pgfqpoint{3.142974in}{2.865641in}}%
\pgfpathlineto{\pgfqpoint{3.177631in}{2.884413in}}%
\pgfpathlineto{\pgfqpoint{3.212287in}{2.902841in}}%
\pgfpathlineto{\pgfqpoint{3.246944in}{2.920921in}}%
\pgfpathlineto{\pgfqpoint{3.281601in}{2.938648in}}%
\pgfpathlineto{\pgfqpoint{3.316258in}{2.956016in}}%
\pgfpathlineto{\pgfqpoint{3.350915in}{2.973021in}}%
\pgfpathlineto{\pgfqpoint{3.385571in}{2.989657in}}%
\pgfpathlineto{\pgfqpoint{3.420228in}{3.005919in}}%
\pgfpathlineto{\pgfqpoint{3.454885in}{3.021803in}}%
\pgfpathlineto{\pgfqpoint{3.489542in}{3.037304in}}%
\pgfpathlineto{\pgfqpoint{3.524198in}{3.052415in}}%
\pgfpathlineto{\pgfqpoint{3.558855in}{3.067133in}}%
\pgfpathlineto{\pgfqpoint{3.593512in}{3.081451in}}%
\pgfpathlineto{\pgfqpoint{3.628169in}{3.095366in}}%
\pgfpathlineto{\pgfqpoint{3.662826in}{3.108872in}}%
\pgfpathlineto{\pgfqpoint{3.697482in}{3.121964in}}%
\pgfpathlineto{\pgfqpoint{3.732139in}{3.134638in}}%
\pgfpathlineto{\pgfqpoint{3.766796in}{3.146887in}}%
\pgfpathlineto{\pgfqpoint{3.801453in}{3.158707in}}%
\pgfpathlineto{\pgfqpoint{3.836109in}{3.170092in}}%
\pgfpathlineto{\pgfqpoint{3.870766in}{3.181039in}}%
\pgfpathlineto{\pgfqpoint{3.905423in}{3.191541in}}%
\pgfpathlineto{\pgfqpoint{3.940080in}{3.201595in}}%
\pgfpathlineto{\pgfqpoint{3.974737in}{3.211193in}}%
\pgfpathlineto{\pgfqpoint{4.009393in}{3.220333in}}%
\pgfpathlineto{\pgfqpoint{4.044050in}{3.229008in}}%
\pgfpathlineto{\pgfqpoint{4.078707in}{3.237214in}}%
\pgfpathlineto{\pgfqpoint{4.113364in}{3.244945in}}%
\pgfpathlineto{\pgfqpoint{4.148020in}{3.252197in}}%
\pgfpathlineto{\pgfqpoint{4.182677in}{3.258964in}}%
\pgfpathlineto{\pgfqpoint{4.217334in}{3.265241in}}%
\pgfpathlineto{\pgfqpoint{4.251991in}{3.271024in}}%
\pgfpathlineto{\pgfqpoint{4.286648in}{3.276307in}}%
\pgfpathlineto{\pgfqpoint{4.321304in}{3.281086in}}%
\pgfpathlineto{\pgfqpoint{4.355961in}{3.285354in}}%
\pgfpathlineto{\pgfqpoint{4.390618in}{3.289108in}}%
\pgfpathlineto{\pgfqpoint{4.425275in}{3.292342in}}%
\pgfpathlineto{\pgfqpoint{4.459931in}{3.295051in}}%
\pgfpathlineto{\pgfqpoint{4.494588in}{3.297230in}}%
\pgfpathlineto{\pgfqpoint{4.529245in}{3.298874in}}%
\pgfpathlineto{\pgfqpoint{4.563902in}{3.299979in}}%
\pgfpathlineto{\pgfqpoint{4.598559in}{3.300537in}}%
\pgfpathlineto{\pgfqpoint{4.633215in}{3.300546in}}%
\pgfpathlineto{\pgfqpoint{4.667872in}{3.300000in}}%
\pgfusepath{stroke}%
\end{pgfscope}%
\begin{pgfscope}%
\pgfsetrectcap%
\pgfsetmiterjoin%
\pgfsetlinewidth{0.803000pt}%
\definecolor{currentstroke}{rgb}{0.000000,0.000000,0.000000}%
\pgfsetstrokecolor{currentstroke}%
\pgfsetdash{}{0pt}%
\pgfpathmoveto{\pgfqpoint{0.800000in}{0.528000in}}%
\pgfpathlineto{\pgfqpoint{0.800000in}{4.224000in}}%
\pgfusepath{stroke}%
\end{pgfscope}%
\begin{pgfscope}%
\pgfsetrectcap%
\pgfsetmiterjoin%
\pgfsetlinewidth{0.803000pt}%
\definecolor{currentstroke}{rgb}{0.000000,0.000000,0.000000}%
\pgfsetstrokecolor{currentstroke}%
\pgfsetdash{}{0pt}%
\pgfpathmoveto{\pgfqpoint{5.760000in}{0.528000in}}%
\pgfpathlineto{\pgfqpoint{5.760000in}{4.224000in}}%
\pgfusepath{stroke}%
\end{pgfscope}%
\begin{pgfscope}%
\pgfsetrectcap%
\pgfsetmiterjoin%
\pgfsetlinewidth{0.803000pt}%
\definecolor{currentstroke}{rgb}{0.000000,0.000000,0.000000}%
\pgfsetstrokecolor{currentstroke}%
\pgfsetdash{}{0pt}%
\pgfpathmoveto{\pgfqpoint{0.800000in}{0.528000in}}%
\pgfpathlineto{\pgfqpoint{5.760000in}{0.528000in}}%
\pgfusepath{stroke}%
\end{pgfscope}%
\begin{pgfscope}%
\pgfsetrectcap%
\pgfsetmiterjoin%
\pgfsetlinewidth{0.803000pt}%
\definecolor{currentstroke}{rgb}{0.000000,0.000000,0.000000}%
\pgfsetstrokecolor{currentstroke}%
\pgfsetdash{}{0pt}%
\pgfpathmoveto{\pgfqpoint{0.800000in}{4.224000in}}%
\pgfpathlineto{\pgfqpoint{5.760000in}{4.224000in}}%
\pgfusepath{stroke}%
\end{pgfscope}%
\begin{pgfscope}%
\pgfsetbuttcap%
\pgfsetmiterjoin%
\definecolor{currentfill}{rgb}{1.000000,1.000000,1.000000}%
\pgfsetfillcolor{currentfill}%
\pgfsetfillopacity{0.800000}%
\pgfsetlinewidth{1.003750pt}%
\definecolor{currentstroke}{rgb}{0.800000,0.800000,0.800000}%
\pgfsetstrokecolor{currentstroke}%
\pgfsetstrokeopacity{0.800000}%
\pgfsetdash{}{0pt}%
\pgfpathmoveto{\pgfqpoint{4.570184in}{3.904556in}}%
\pgfpathlineto{\pgfqpoint{5.662778in}{3.904556in}}%
\pgfpathquadraticcurveto{\pgfqpoint{5.690556in}{3.904556in}}{\pgfqpoint{5.690556in}{3.932333in}}%
\pgfpathlineto{\pgfqpoint{5.690556in}{4.126778in}}%
\pgfpathquadraticcurveto{\pgfqpoint{5.690556in}{4.154556in}}{\pgfqpoint{5.662778in}{4.154556in}}%
\pgfpathlineto{\pgfqpoint{4.570184in}{4.154556in}}%
\pgfpathquadraticcurveto{\pgfqpoint{4.542406in}{4.154556in}}{\pgfqpoint{4.542406in}{4.126778in}}%
\pgfpathlineto{\pgfqpoint{4.542406in}{3.932333in}}%
\pgfpathquadraticcurveto{\pgfqpoint{4.542406in}{3.904556in}}{\pgfqpoint{4.570184in}{3.904556in}}%
\pgfpathclose%
\pgfusepath{stroke,fill}%
\end{pgfscope}%
\begin{pgfscope}%
\pgfsetrectcap%
\pgfsetroundjoin%
\pgfsetlinewidth{1.505625pt}%
\definecolor{currentstroke}{rgb}{0.000000,0.000000,1.000000}%
\pgfsetstrokecolor{currentstroke}%
\pgfsetdash{}{0pt}%
\pgfpathmoveto{\pgfqpoint{4.597961in}{4.043444in}}%
\pgfpathlineto{\pgfqpoint{4.875739in}{4.043444in}}%
\pgfusepath{stroke}%
\end{pgfscope}%
\begin{pgfscope}%
\definecolor{textcolor}{rgb}{0.000000,0.000000,0.000000}%
\pgfsetstrokecolor{textcolor}%
\pgfsetfillcolor{textcolor}%
\pgftext[x=4.986850in,y=3.994833in,left,base]{\color{textcolor}\rmfamily\fontsize{10.000000}{12.000000}\selectfont Newton(x)}%
\end{pgfscope}%
\end{pgfpicture}%
\makeatother%
\endgroup%
}
                   \caption{Newton Interpolating Polynomial on $[0, \pi/2]$}
                   \label{fig:NewtonPlot}
                \end{center}
            \end{figure}
            \noindent
            which is identical to \autoref{fig:LagrangePlot}. They also simplify to the same thing if we check using 
            Wolfram Alpha. 
            
        \end{enumerate}

        \item \textbf{Error Analysis} \newline
        \begin{enumerate}
            \item \textbf{Polynomial Approximation}
            The error bound formula for polynomial approximation $p_n(x)$ to a function $f(x)$ is given by

            \begin{equation}
                f(x)-p_n(x) = \frac{f^{(n+1)}(\xi)}{(n+1)!}(x-x_0)\dots(x-x_n)
                \label{eqn:Polynomial Approximation Error}
            \end{equation}

            and we are to show that it satisfies

            \begin{equation}
                |f(x)-p_3(x)| \leq \frac{h^4}{24} \max_{\xi \in [x_0, x_3]} |f^4(\xi)|
                \label{eqn:To Solve 2a}
            \end{equation}
            
            where $x_0, x_1, x_2, x_3$ are equally spaced nodes with step-size $h$. In terms of interpreting 
            \autoref{eqn:To Solve 2a}, we can see that \autoref{eqn:Polynomial Approximation Error} gives us an 
            upper and lower bound of the error of the polynomial. This means that taking the absolute value of 
            that error bound will give us the maximum absolute value of the error. This simplifies things nicely 
            and means that we can effectively ignore everything in \autoref{eqn:To Solve 2a} apart from $\frac{h^4}{24}$, 
            leaving us with having to show that 

            \begin{equation*}
                \frac{1}{4!}(x-x_0)\dots(x-x_n) = \frac{h^4}{24}
            \end{equation*}

            and we know that $4! = 24$. Now we can use a substitution to simplify the rest, which we'll call $w(x)$. 
            If we let $x = t+x_1+h/2$ and substitute in, we get 

            \begin{equation*}
                \begin{split}
                    w(t) = &(t+x_1+\frac{h}{2}-x_0)(t+x_1+\frac{h}{2}-x_1)(t+x_1+\frac{h}{2}-x_2)(t+x_1+\frac{h}{2}-x_3) \\
                    = &(t+\frac{3h}{2})(t+\frac{h}{2})(t-\frac{h}{2})(t-\frac{3h}{2}) \\
                    = &(t^2-\frac{9h^2}{4})(t^2-\frac{h^2}{4})
                \end{split}
            \end{equation*}
            We want to find the absolute maximum of this function, so we must find its stationary points and 
            take the absolute values to find the max. Finding these stationary points is relatively trivial and we 
            find them to be at 
            
            $t_{root} = 0, \pm \sqrt{5/4}h^2$ with the function being at its absolute maximum at $t=0$ with 
            $w(0) = -h^4 \implies |w(0)| = h^4$, which is what we aimed to show.

            \item \textbf{Hermite Approximation} \newline
            Now we aim to show that, for a cubic Hermite Interpolating Polynomial, the error bound satisfies 

            \begin{equation}
                |f(x)-H_3(x)| \leq \frac{(b-a)^4}{384} \max_{\xi \in [a, b]} |f^4(\xi)|
                \label{eqn:To Solve 2b}
            \end{equation}

            The error bound formula in question is

            \begin{equation}
                f(x)-H_{2n+1}(x) = \frac{f^{(2n+2)}(\xi)}{(2n+2)!}(x-x_0)^2\dots(x-x_n)^2
                \label{eqn:Hermite Approximation Error}
            \end{equation}

            which, similarly to Question 2a above, we can ignore most of. We need only focus on showing that 

            \begin{equation*}
                \begin{split}
                    \frac{1}{4!}(x-a)^2(x-b)^2 &= \frac{(b-a)^4}{384} \\
                    \implies (x-a)^2(x-b)^2 &= \frac{(b-a)^4}{16}
                \end{split}
            \end{equation*}

            For simplicity, we'll call the left hand side of this $w(x)$, as before. We can also make the same 
            substitution as before, letting $x = t+b+\frac{b-a}{2}$, which gives us 

            \begin{equation*}
                \begin{split}
                    w(t) &= (t+b-a+\frac{b-a}{2})^2(t+\frac{b-a}{2})^2 \\
                    &= (t+\frac{3(b-a)}{2})^2(t+\frac{b-a}{2})
                \end{split}
            \end{equation*}

            Once again we find the absolute maximum by taking the derivative and setting it to 0, which is 
            relatively trivial and gives us stationary points at $t_{root} = \frac{3(a-b)}{2}, \frac{(a-b)}{2}, (a-b)$, 
            with $w(t)$ being at its maximum when $t=(a-b)$, with $w(a-b) = \frac{(a-b)^4}{16} \implies |w(a-b)| = \frac{(b-a)^4}{16}$.
            This is what we aimed to show.
            
        \end{enumerate}
        
        \section*{Numerical Problems}
        \item \textbf{Expanding the Lagrange Approximation to any $x \in [0, \infty)$} \newline
        
        \begin{figure}[H]
            \begin{center}
                \scalebox{.7}{%% Creator: Matplotlib, PGF backend
%%
%% To include the figure in your LaTeX document, write
%%   \input{<filename>.pgf}
%%
%% Make sure the required packages are loaded in your preamble
%%   \usepackage{pgf}
%%
%% Figures using additional raster images can only be included by \input if
%% they are in the same directory as the main LaTeX file. For loading figures
%% from other directories you can use the `import` package
%%   \usepackage{import}
%% and then include the figures with
%%   \import{<path to file>}{<filename>.pgf}
%%
%% Matplotlib used the following preamble
%%
\begingroup%
\makeatletter%
\begin{pgfpicture}%
\pgfpathrectangle{\pgfpointorigin}{\pgfqpoint{6.400000in}{4.800000in}}%
\pgfusepath{use as bounding box, clip}%
\begin{pgfscope}%
\pgfsetbuttcap%
\pgfsetmiterjoin%
\definecolor{currentfill}{rgb}{1.000000,1.000000,1.000000}%
\pgfsetfillcolor{currentfill}%
\pgfsetlinewidth{0.000000pt}%
\definecolor{currentstroke}{rgb}{1.000000,1.000000,1.000000}%
\pgfsetstrokecolor{currentstroke}%
\pgfsetdash{}{0pt}%
\pgfpathmoveto{\pgfqpoint{0.000000in}{0.000000in}}%
\pgfpathlineto{\pgfqpoint{6.400000in}{0.000000in}}%
\pgfpathlineto{\pgfqpoint{6.400000in}{4.800000in}}%
\pgfpathlineto{\pgfqpoint{0.000000in}{4.800000in}}%
\pgfpathclose%
\pgfusepath{fill}%
\end{pgfscope}%
\begin{pgfscope}%
\pgfsetbuttcap%
\pgfsetmiterjoin%
\definecolor{currentfill}{rgb}{1.000000,1.000000,1.000000}%
\pgfsetfillcolor{currentfill}%
\pgfsetlinewidth{0.000000pt}%
\definecolor{currentstroke}{rgb}{0.000000,0.000000,0.000000}%
\pgfsetstrokecolor{currentstroke}%
\pgfsetstrokeopacity{0.000000}%
\pgfsetdash{}{0pt}%
\pgfpathmoveto{\pgfqpoint{0.800000in}{0.528000in}}%
\pgfpathlineto{\pgfqpoint{5.760000in}{0.528000in}}%
\pgfpathlineto{\pgfqpoint{5.760000in}{4.224000in}}%
\pgfpathlineto{\pgfqpoint{0.800000in}{4.224000in}}%
\pgfpathclose%
\pgfusepath{fill}%
\end{pgfscope}%
\begin{pgfscope}%
\pgfpathrectangle{\pgfqpoint{0.800000in}{0.528000in}}{\pgfqpoint{4.960000in}{3.696000in}}%
\pgfusepath{clip}%
\pgfsetbuttcap%
\pgfsetroundjoin%
\pgfsetlinewidth{0.803000pt}%
\definecolor{currentstroke}{rgb}{0.800000,0.800000,0.800000}%
\pgfsetstrokecolor{currentstroke}%
\pgfsetdash{{0.800000pt}{1.320000pt}}{0.000000pt}%
\pgfpathmoveto{\pgfqpoint{0.942056in}{0.528000in}}%
\pgfpathlineto{\pgfqpoint{0.942056in}{4.224000in}}%
\pgfusepath{stroke}%
\end{pgfscope}%
\begin{pgfscope}%
\pgfsetbuttcap%
\pgfsetroundjoin%
\definecolor{currentfill}{rgb}{0.000000,0.000000,0.000000}%
\pgfsetfillcolor{currentfill}%
\pgfsetlinewidth{0.803000pt}%
\definecolor{currentstroke}{rgb}{0.000000,0.000000,0.000000}%
\pgfsetstrokecolor{currentstroke}%
\pgfsetdash{}{0pt}%
\pgfsys@defobject{currentmarker}{\pgfqpoint{0.000000in}{-0.048611in}}{\pgfqpoint{0.000000in}{0.000000in}}{%
\pgfpathmoveto{\pgfqpoint{0.000000in}{0.000000in}}%
\pgfpathlineto{\pgfqpoint{0.000000in}{-0.048611in}}%
\pgfusepath{stroke,fill}%
}%
\begin{pgfscope}%
\pgfsys@transformshift{0.942056in}{0.528000in}%
\pgfsys@useobject{currentmarker}{}%
\end{pgfscope}%
\end{pgfscope}%
\begin{pgfscope}%
\definecolor{textcolor}{rgb}{0.000000,0.000000,0.000000}%
\pgfsetstrokecolor{textcolor}%
\pgfsetfillcolor{textcolor}%
\pgftext[x=0.942056in,y=0.430778in,,top]{\color{textcolor}\rmfamily\fontsize{10.000000}{12.000000}\selectfont \(\displaystyle 0\)}%
\end{pgfscope}%
\begin{pgfscope}%
\pgfpathrectangle{\pgfqpoint{0.800000in}{0.528000in}}{\pgfqpoint{4.960000in}{3.696000in}}%
\pgfusepath{clip}%
\pgfsetbuttcap%
\pgfsetroundjoin%
\pgfsetlinewidth{0.803000pt}%
\definecolor{currentstroke}{rgb}{0.800000,0.800000,0.800000}%
\pgfsetstrokecolor{currentstroke}%
\pgfsetdash{{0.800000pt}{1.320000pt}}{0.000000pt}%
\pgfpathmoveto{\pgfqpoint{1.652333in}{0.528000in}}%
\pgfpathlineto{\pgfqpoint{1.652333in}{4.224000in}}%
\pgfusepath{stroke}%
\end{pgfscope}%
\begin{pgfscope}%
\pgfsetbuttcap%
\pgfsetroundjoin%
\definecolor{currentfill}{rgb}{0.000000,0.000000,0.000000}%
\pgfsetfillcolor{currentfill}%
\pgfsetlinewidth{0.803000pt}%
\definecolor{currentstroke}{rgb}{0.000000,0.000000,0.000000}%
\pgfsetstrokecolor{currentstroke}%
\pgfsetdash{}{0pt}%
\pgfsys@defobject{currentmarker}{\pgfqpoint{0.000000in}{-0.048611in}}{\pgfqpoint{0.000000in}{0.000000in}}{%
\pgfpathmoveto{\pgfqpoint{0.000000in}{0.000000in}}%
\pgfpathlineto{\pgfqpoint{0.000000in}{-0.048611in}}%
\pgfusepath{stroke,fill}%
}%
\begin{pgfscope}%
\pgfsys@transformshift{1.652333in}{0.528000in}%
\pgfsys@useobject{currentmarker}{}%
\end{pgfscope}%
\end{pgfscope}%
\begin{pgfscope}%
\definecolor{textcolor}{rgb}{0.000000,0.000000,0.000000}%
\pgfsetstrokecolor{textcolor}%
\pgfsetfillcolor{textcolor}%
\pgftext[x=1.652333in,y=0.430778in,,top]{\color{textcolor}\rmfamily\fontsize{10.000000}{12.000000}\selectfont \(\displaystyle 1\)}%
\end{pgfscope}%
\begin{pgfscope}%
\pgfpathrectangle{\pgfqpoint{0.800000in}{0.528000in}}{\pgfqpoint{4.960000in}{3.696000in}}%
\pgfusepath{clip}%
\pgfsetbuttcap%
\pgfsetroundjoin%
\pgfsetlinewidth{0.803000pt}%
\definecolor{currentstroke}{rgb}{0.800000,0.800000,0.800000}%
\pgfsetstrokecolor{currentstroke}%
\pgfsetdash{{0.800000pt}{1.320000pt}}{0.000000pt}%
\pgfpathmoveto{\pgfqpoint{2.362611in}{0.528000in}}%
\pgfpathlineto{\pgfqpoint{2.362611in}{4.224000in}}%
\pgfusepath{stroke}%
\end{pgfscope}%
\begin{pgfscope}%
\pgfsetbuttcap%
\pgfsetroundjoin%
\definecolor{currentfill}{rgb}{0.000000,0.000000,0.000000}%
\pgfsetfillcolor{currentfill}%
\pgfsetlinewidth{0.803000pt}%
\definecolor{currentstroke}{rgb}{0.000000,0.000000,0.000000}%
\pgfsetstrokecolor{currentstroke}%
\pgfsetdash{}{0pt}%
\pgfsys@defobject{currentmarker}{\pgfqpoint{0.000000in}{-0.048611in}}{\pgfqpoint{0.000000in}{0.000000in}}{%
\pgfpathmoveto{\pgfqpoint{0.000000in}{0.000000in}}%
\pgfpathlineto{\pgfqpoint{0.000000in}{-0.048611in}}%
\pgfusepath{stroke,fill}%
}%
\begin{pgfscope}%
\pgfsys@transformshift{2.362611in}{0.528000in}%
\pgfsys@useobject{currentmarker}{}%
\end{pgfscope}%
\end{pgfscope}%
\begin{pgfscope}%
\definecolor{textcolor}{rgb}{0.000000,0.000000,0.000000}%
\pgfsetstrokecolor{textcolor}%
\pgfsetfillcolor{textcolor}%
\pgftext[x=2.362611in,y=0.430778in,,top]{\color{textcolor}\rmfamily\fontsize{10.000000}{12.000000}\selectfont \(\displaystyle 2\)}%
\end{pgfscope}%
\begin{pgfscope}%
\pgfpathrectangle{\pgfqpoint{0.800000in}{0.528000in}}{\pgfqpoint{4.960000in}{3.696000in}}%
\pgfusepath{clip}%
\pgfsetbuttcap%
\pgfsetroundjoin%
\pgfsetlinewidth{0.803000pt}%
\definecolor{currentstroke}{rgb}{0.800000,0.800000,0.800000}%
\pgfsetstrokecolor{currentstroke}%
\pgfsetdash{{0.800000pt}{1.320000pt}}{0.000000pt}%
\pgfpathmoveto{\pgfqpoint{3.072888in}{0.528000in}}%
\pgfpathlineto{\pgfqpoint{3.072888in}{4.224000in}}%
\pgfusepath{stroke}%
\end{pgfscope}%
\begin{pgfscope}%
\pgfsetbuttcap%
\pgfsetroundjoin%
\definecolor{currentfill}{rgb}{0.000000,0.000000,0.000000}%
\pgfsetfillcolor{currentfill}%
\pgfsetlinewidth{0.803000pt}%
\definecolor{currentstroke}{rgb}{0.000000,0.000000,0.000000}%
\pgfsetstrokecolor{currentstroke}%
\pgfsetdash{}{0pt}%
\pgfsys@defobject{currentmarker}{\pgfqpoint{0.000000in}{-0.048611in}}{\pgfqpoint{0.000000in}{0.000000in}}{%
\pgfpathmoveto{\pgfqpoint{0.000000in}{0.000000in}}%
\pgfpathlineto{\pgfqpoint{0.000000in}{-0.048611in}}%
\pgfusepath{stroke,fill}%
}%
\begin{pgfscope}%
\pgfsys@transformshift{3.072888in}{0.528000in}%
\pgfsys@useobject{currentmarker}{}%
\end{pgfscope}%
\end{pgfscope}%
\begin{pgfscope}%
\definecolor{textcolor}{rgb}{0.000000,0.000000,0.000000}%
\pgfsetstrokecolor{textcolor}%
\pgfsetfillcolor{textcolor}%
\pgftext[x=3.072888in,y=0.430778in,,top]{\color{textcolor}\rmfamily\fontsize{10.000000}{12.000000}\selectfont \(\displaystyle 3\)}%
\end{pgfscope}%
\begin{pgfscope}%
\pgfpathrectangle{\pgfqpoint{0.800000in}{0.528000in}}{\pgfqpoint{4.960000in}{3.696000in}}%
\pgfusepath{clip}%
\pgfsetbuttcap%
\pgfsetroundjoin%
\pgfsetlinewidth{0.803000pt}%
\definecolor{currentstroke}{rgb}{0.800000,0.800000,0.800000}%
\pgfsetstrokecolor{currentstroke}%
\pgfsetdash{{0.800000pt}{1.320000pt}}{0.000000pt}%
\pgfpathmoveto{\pgfqpoint{3.783166in}{0.528000in}}%
\pgfpathlineto{\pgfqpoint{3.783166in}{4.224000in}}%
\pgfusepath{stroke}%
\end{pgfscope}%
\begin{pgfscope}%
\pgfsetbuttcap%
\pgfsetroundjoin%
\definecolor{currentfill}{rgb}{0.000000,0.000000,0.000000}%
\pgfsetfillcolor{currentfill}%
\pgfsetlinewidth{0.803000pt}%
\definecolor{currentstroke}{rgb}{0.000000,0.000000,0.000000}%
\pgfsetstrokecolor{currentstroke}%
\pgfsetdash{}{0pt}%
\pgfsys@defobject{currentmarker}{\pgfqpoint{0.000000in}{-0.048611in}}{\pgfqpoint{0.000000in}{0.000000in}}{%
\pgfpathmoveto{\pgfqpoint{0.000000in}{0.000000in}}%
\pgfpathlineto{\pgfqpoint{0.000000in}{-0.048611in}}%
\pgfusepath{stroke,fill}%
}%
\begin{pgfscope}%
\pgfsys@transformshift{3.783166in}{0.528000in}%
\pgfsys@useobject{currentmarker}{}%
\end{pgfscope}%
\end{pgfscope}%
\begin{pgfscope}%
\definecolor{textcolor}{rgb}{0.000000,0.000000,0.000000}%
\pgfsetstrokecolor{textcolor}%
\pgfsetfillcolor{textcolor}%
\pgftext[x=3.783166in,y=0.430778in,,top]{\color{textcolor}\rmfamily\fontsize{10.000000}{12.000000}\selectfont \(\displaystyle 4\)}%
\end{pgfscope}%
\begin{pgfscope}%
\pgfpathrectangle{\pgfqpoint{0.800000in}{0.528000in}}{\pgfqpoint{4.960000in}{3.696000in}}%
\pgfusepath{clip}%
\pgfsetbuttcap%
\pgfsetroundjoin%
\pgfsetlinewidth{0.803000pt}%
\definecolor{currentstroke}{rgb}{0.800000,0.800000,0.800000}%
\pgfsetstrokecolor{currentstroke}%
\pgfsetdash{{0.800000pt}{1.320000pt}}{0.000000pt}%
\pgfpathmoveto{\pgfqpoint{4.493443in}{0.528000in}}%
\pgfpathlineto{\pgfqpoint{4.493443in}{4.224000in}}%
\pgfusepath{stroke}%
\end{pgfscope}%
\begin{pgfscope}%
\pgfsetbuttcap%
\pgfsetroundjoin%
\definecolor{currentfill}{rgb}{0.000000,0.000000,0.000000}%
\pgfsetfillcolor{currentfill}%
\pgfsetlinewidth{0.803000pt}%
\definecolor{currentstroke}{rgb}{0.000000,0.000000,0.000000}%
\pgfsetstrokecolor{currentstroke}%
\pgfsetdash{}{0pt}%
\pgfsys@defobject{currentmarker}{\pgfqpoint{0.000000in}{-0.048611in}}{\pgfqpoint{0.000000in}{0.000000in}}{%
\pgfpathmoveto{\pgfqpoint{0.000000in}{0.000000in}}%
\pgfpathlineto{\pgfqpoint{0.000000in}{-0.048611in}}%
\pgfusepath{stroke,fill}%
}%
\begin{pgfscope}%
\pgfsys@transformshift{4.493443in}{0.528000in}%
\pgfsys@useobject{currentmarker}{}%
\end{pgfscope}%
\end{pgfscope}%
\begin{pgfscope}%
\definecolor{textcolor}{rgb}{0.000000,0.000000,0.000000}%
\pgfsetstrokecolor{textcolor}%
\pgfsetfillcolor{textcolor}%
\pgftext[x=4.493443in,y=0.430778in,,top]{\color{textcolor}\rmfamily\fontsize{10.000000}{12.000000}\selectfont \(\displaystyle 5\)}%
\end{pgfscope}%
\begin{pgfscope}%
\pgfpathrectangle{\pgfqpoint{0.800000in}{0.528000in}}{\pgfqpoint{4.960000in}{3.696000in}}%
\pgfusepath{clip}%
\pgfsetbuttcap%
\pgfsetroundjoin%
\pgfsetlinewidth{0.803000pt}%
\definecolor{currentstroke}{rgb}{0.800000,0.800000,0.800000}%
\pgfsetstrokecolor{currentstroke}%
\pgfsetdash{{0.800000pt}{1.320000pt}}{0.000000pt}%
\pgfpathmoveto{\pgfqpoint{5.203721in}{0.528000in}}%
\pgfpathlineto{\pgfqpoint{5.203721in}{4.224000in}}%
\pgfusepath{stroke}%
\end{pgfscope}%
\begin{pgfscope}%
\pgfsetbuttcap%
\pgfsetroundjoin%
\definecolor{currentfill}{rgb}{0.000000,0.000000,0.000000}%
\pgfsetfillcolor{currentfill}%
\pgfsetlinewidth{0.803000pt}%
\definecolor{currentstroke}{rgb}{0.000000,0.000000,0.000000}%
\pgfsetstrokecolor{currentstroke}%
\pgfsetdash{}{0pt}%
\pgfsys@defobject{currentmarker}{\pgfqpoint{0.000000in}{-0.048611in}}{\pgfqpoint{0.000000in}{0.000000in}}{%
\pgfpathmoveto{\pgfqpoint{0.000000in}{0.000000in}}%
\pgfpathlineto{\pgfqpoint{0.000000in}{-0.048611in}}%
\pgfusepath{stroke,fill}%
}%
\begin{pgfscope}%
\pgfsys@transformshift{5.203721in}{0.528000in}%
\pgfsys@useobject{currentmarker}{}%
\end{pgfscope}%
\end{pgfscope}%
\begin{pgfscope}%
\definecolor{textcolor}{rgb}{0.000000,0.000000,0.000000}%
\pgfsetstrokecolor{textcolor}%
\pgfsetfillcolor{textcolor}%
\pgftext[x=5.203721in,y=0.430778in,,top]{\color{textcolor}\rmfamily\fontsize{10.000000}{12.000000}\selectfont \(\displaystyle 6\)}%
\end{pgfscope}%
\begin{pgfscope}%
\definecolor{textcolor}{rgb}{0.000000,0.000000,0.000000}%
\pgfsetstrokecolor{textcolor}%
\pgfsetfillcolor{textcolor}%
\pgftext[x=3.280000in,y=0.251766in,,top]{\color{textcolor}\rmfamily\fontsize{10.000000}{12.000000}\selectfont x}%
\end{pgfscope}%
\begin{pgfscope}%
\pgfpathrectangle{\pgfqpoint{0.800000in}{0.528000in}}{\pgfqpoint{4.960000in}{3.696000in}}%
\pgfusepath{clip}%
\pgfsetbuttcap%
\pgfsetroundjoin%
\pgfsetlinewidth{0.803000pt}%
\definecolor{currentstroke}{rgb}{0.800000,0.800000,0.800000}%
\pgfsetstrokecolor{currentstroke}%
\pgfsetdash{{0.800000pt}{1.320000pt}}{0.000000pt}%
\pgfpathmoveto{\pgfqpoint{0.800000in}{0.528000in}}%
\pgfpathlineto{\pgfqpoint{5.760000in}{0.528000in}}%
\pgfusepath{stroke}%
\end{pgfscope}%
\begin{pgfscope}%
\pgfsetbuttcap%
\pgfsetroundjoin%
\definecolor{currentfill}{rgb}{0.000000,0.000000,0.000000}%
\pgfsetfillcolor{currentfill}%
\pgfsetlinewidth{0.803000pt}%
\definecolor{currentstroke}{rgb}{0.000000,0.000000,0.000000}%
\pgfsetstrokecolor{currentstroke}%
\pgfsetdash{}{0pt}%
\pgfsys@defobject{currentmarker}{\pgfqpoint{-0.048611in}{0.000000in}}{\pgfqpoint{0.000000in}{0.000000in}}{%
\pgfpathmoveto{\pgfqpoint{0.000000in}{0.000000in}}%
\pgfpathlineto{\pgfqpoint{-0.048611in}{0.000000in}}%
\pgfusepath{stroke,fill}%
}%
\begin{pgfscope}%
\pgfsys@transformshift{0.800000in}{0.528000in}%
\pgfsys@useobject{currentmarker}{}%
\end{pgfscope}%
\end{pgfscope}%
\begin{pgfscope}%
\definecolor{textcolor}{rgb}{0.000000,0.000000,0.000000}%
\pgfsetstrokecolor{textcolor}%
\pgfsetfillcolor{textcolor}%
\pgftext[x=0.417283in,y=0.479775in,left,base]{\color{textcolor}\rmfamily\fontsize{10.000000}{12.000000}\selectfont \(\displaystyle -1.5\)}%
\end{pgfscope}%
\begin{pgfscope}%
\pgfpathrectangle{\pgfqpoint{0.800000in}{0.528000in}}{\pgfqpoint{4.960000in}{3.696000in}}%
\pgfusepath{clip}%
\pgfsetbuttcap%
\pgfsetroundjoin%
\pgfsetlinewidth{0.803000pt}%
\definecolor{currentstroke}{rgb}{0.800000,0.800000,0.800000}%
\pgfsetstrokecolor{currentstroke}%
\pgfsetdash{{0.800000pt}{1.320000pt}}{0.000000pt}%
\pgfpathmoveto{\pgfqpoint{0.800000in}{1.144000in}}%
\pgfpathlineto{\pgfqpoint{5.760000in}{1.144000in}}%
\pgfusepath{stroke}%
\end{pgfscope}%
\begin{pgfscope}%
\pgfsetbuttcap%
\pgfsetroundjoin%
\definecolor{currentfill}{rgb}{0.000000,0.000000,0.000000}%
\pgfsetfillcolor{currentfill}%
\pgfsetlinewidth{0.803000pt}%
\definecolor{currentstroke}{rgb}{0.000000,0.000000,0.000000}%
\pgfsetstrokecolor{currentstroke}%
\pgfsetdash{}{0pt}%
\pgfsys@defobject{currentmarker}{\pgfqpoint{-0.048611in}{0.000000in}}{\pgfqpoint{0.000000in}{0.000000in}}{%
\pgfpathmoveto{\pgfqpoint{0.000000in}{0.000000in}}%
\pgfpathlineto{\pgfqpoint{-0.048611in}{0.000000in}}%
\pgfusepath{stroke,fill}%
}%
\begin{pgfscope}%
\pgfsys@transformshift{0.800000in}{1.144000in}%
\pgfsys@useobject{currentmarker}{}%
\end{pgfscope}%
\end{pgfscope}%
\begin{pgfscope}%
\definecolor{textcolor}{rgb}{0.000000,0.000000,0.000000}%
\pgfsetstrokecolor{textcolor}%
\pgfsetfillcolor{textcolor}%
\pgftext[x=0.417283in,y=1.095775in,left,base]{\color{textcolor}\rmfamily\fontsize{10.000000}{12.000000}\selectfont \(\displaystyle -1.0\)}%
\end{pgfscope}%
\begin{pgfscope}%
\pgfpathrectangle{\pgfqpoint{0.800000in}{0.528000in}}{\pgfqpoint{4.960000in}{3.696000in}}%
\pgfusepath{clip}%
\pgfsetbuttcap%
\pgfsetroundjoin%
\pgfsetlinewidth{0.803000pt}%
\definecolor{currentstroke}{rgb}{0.800000,0.800000,0.800000}%
\pgfsetstrokecolor{currentstroke}%
\pgfsetdash{{0.800000pt}{1.320000pt}}{0.000000pt}%
\pgfpathmoveto{\pgfqpoint{0.800000in}{1.760000in}}%
\pgfpathlineto{\pgfqpoint{5.760000in}{1.760000in}}%
\pgfusepath{stroke}%
\end{pgfscope}%
\begin{pgfscope}%
\pgfsetbuttcap%
\pgfsetroundjoin%
\definecolor{currentfill}{rgb}{0.000000,0.000000,0.000000}%
\pgfsetfillcolor{currentfill}%
\pgfsetlinewidth{0.803000pt}%
\definecolor{currentstroke}{rgb}{0.000000,0.000000,0.000000}%
\pgfsetstrokecolor{currentstroke}%
\pgfsetdash{}{0pt}%
\pgfsys@defobject{currentmarker}{\pgfqpoint{-0.048611in}{0.000000in}}{\pgfqpoint{0.000000in}{0.000000in}}{%
\pgfpathmoveto{\pgfqpoint{0.000000in}{0.000000in}}%
\pgfpathlineto{\pgfqpoint{-0.048611in}{0.000000in}}%
\pgfusepath{stroke,fill}%
}%
\begin{pgfscope}%
\pgfsys@transformshift{0.800000in}{1.760000in}%
\pgfsys@useobject{currentmarker}{}%
\end{pgfscope}%
\end{pgfscope}%
\begin{pgfscope}%
\definecolor{textcolor}{rgb}{0.000000,0.000000,0.000000}%
\pgfsetstrokecolor{textcolor}%
\pgfsetfillcolor{textcolor}%
\pgftext[x=0.417283in,y=1.711775in,left,base]{\color{textcolor}\rmfamily\fontsize{10.000000}{12.000000}\selectfont \(\displaystyle -0.5\)}%
\end{pgfscope}%
\begin{pgfscope}%
\pgfpathrectangle{\pgfqpoint{0.800000in}{0.528000in}}{\pgfqpoint{4.960000in}{3.696000in}}%
\pgfusepath{clip}%
\pgfsetbuttcap%
\pgfsetroundjoin%
\pgfsetlinewidth{0.803000pt}%
\definecolor{currentstroke}{rgb}{0.800000,0.800000,0.800000}%
\pgfsetstrokecolor{currentstroke}%
\pgfsetdash{{0.800000pt}{1.320000pt}}{0.000000pt}%
\pgfpathmoveto{\pgfqpoint{0.800000in}{2.376000in}}%
\pgfpathlineto{\pgfqpoint{5.760000in}{2.376000in}}%
\pgfusepath{stroke}%
\end{pgfscope}%
\begin{pgfscope}%
\pgfsetbuttcap%
\pgfsetroundjoin%
\definecolor{currentfill}{rgb}{0.000000,0.000000,0.000000}%
\pgfsetfillcolor{currentfill}%
\pgfsetlinewidth{0.803000pt}%
\definecolor{currentstroke}{rgb}{0.000000,0.000000,0.000000}%
\pgfsetstrokecolor{currentstroke}%
\pgfsetdash{}{0pt}%
\pgfsys@defobject{currentmarker}{\pgfqpoint{-0.048611in}{0.000000in}}{\pgfqpoint{0.000000in}{0.000000in}}{%
\pgfpathmoveto{\pgfqpoint{0.000000in}{0.000000in}}%
\pgfpathlineto{\pgfqpoint{-0.048611in}{0.000000in}}%
\pgfusepath{stroke,fill}%
}%
\begin{pgfscope}%
\pgfsys@transformshift{0.800000in}{2.376000in}%
\pgfsys@useobject{currentmarker}{}%
\end{pgfscope}%
\end{pgfscope}%
\begin{pgfscope}%
\definecolor{textcolor}{rgb}{0.000000,0.000000,0.000000}%
\pgfsetstrokecolor{textcolor}%
\pgfsetfillcolor{textcolor}%
\pgftext[x=0.525308in,y=2.327775in,left,base]{\color{textcolor}\rmfamily\fontsize{10.000000}{12.000000}\selectfont \(\displaystyle 0.0\)}%
\end{pgfscope}%
\begin{pgfscope}%
\pgfpathrectangle{\pgfqpoint{0.800000in}{0.528000in}}{\pgfqpoint{4.960000in}{3.696000in}}%
\pgfusepath{clip}%
\pgfsetbuttcap%
\pgfsetroundjoin%
\pgfsetlinewidth{0.803000pt}%
\definecolor{currentstroke}{rgb}{0.800000,0.800000,0.800000}%
\pgfsetstrokecolor{currentstroke}%
\pgfsetdash{{0.800000pt}{1.320000pt}}{0.000000pt}%
\pgfpathmoveto{\pgfqpoint{0.800000in}{2.992000in}}%
\pgfpathlineto{\pgfqpoint{5.760000in}{2.992000in}}%
\pgfusepath{stroke}%
\end{pgfscope}%
\begin{pgfscope}%
\pgfsetbuttcap%
\pgfsetroundjoin%
\definecolor{currentfill}{rgb}{0.000000,0.000000,0.000000}%
\pgfsetfillcolor{currentfill}%
\pgfsetlinewidth{0.803000pt}%
\definecolor{currentstroke}{rgb}{0.000000,0.000000,0.000000}%
\pgfsetstrokecolor{currentstroke}%
\pgfsetdash{}{0pt}%
\pgfsys@defobject{currentmarker}{\pgfqpoint{-0.048611in}{0.000000in}}{\pgfqpoint{0.000000in}{0.000000in}}{%
\pgfpathmoveto{\pgfqpoint{0.000000in}{0.000000in}}%
\pgfpathlineto{\pgfqpoint{-0.048611in}{0.000000in}}%
\pgfusepath{stroke,fill}%
}%
\begin{pgfscope}%
\pgfsys@transformshift{0.800000in}{2.992000in}%
\pgfsys@useobject{currentmarker}{}%
\end{pgfscope}%
\end{pgfscope}%
\begin{pgfscope}%
\definecolor{textcolor}{rgb}{0.000000,0.000000,0.000000}%
\pgfsetstrokecolor{textcolor}%
\pgfsetfillcolor{textcolor}%
\pgftext[x=0.525308in,y=2.943775in,left,base]{\color{textcolor}\rmfamily\fontsize{10.000000}{12.000000}\selectfont \(\displaystyle 0.5\)}%
\end{pgfscope}%
\begin{pgfscope}%
\pgfpathrectangle{\pgfqpoint{0.800000in}{0.528000in}}{\pgfqpoint{4.960000in}{3.696000in}}%
\pgfusepath{clip}%
\pgfsetbuttcap%
\pgfsetroundjoin%
\pgfsetlinewidth{0.803000pt}%
\definecolor{currentstroke}{rgb}{0.800000,0.800000,0.800000}%
\pgfsetstrokecolor{currentstroke}%
\pgfsetdash{{0.800000pt}{1.320000pt}}{0.000000pt}%
\pgfpathmoveto{\pgfqpoint{0.800000in}{3.608000in}}%
\pgfpathlineto{\pgfqpoint{5.760000in}{3.608000in}}%
\pgfusepath{stroke}%
\end{pgfscope}%
\begin{pgfscope}%
\pgfsetbuttcap%
\pgfsetroundjoin%
\definecolor{currentfill}{rgb}{0.000000,0.000000,0.000000}%
\pgfsetfillcolor{currentfill}%
\pgfsetlinewidth{0.803000pt}%
\definecolor{currentstroke}{rgb}{0.000000,0.000000,0.000000}%
\pgfsetstrokecolor{currentstroke}%
\pgfsetdash{}{0pt}%
\pgfsys@defobject{currentmarker}{\pgfqpoint{-0.048611in}{0.000000in}}{\pgfqpoint{0.000000in}{0.000000in}}{%
\pgfpathmoveto{\pgfqpoint{0.000000in}{0.000000in}}%
\pgfpathlineto{\pgfqpoint{-0.048611in}{0.000000in}}%
\pgfusepath{stroke,fill}%
}%
\begin{pgfscope}%
\pgfsys@transformshift{0.800000in}{3.608000in}%
\pgfsys@useobject{currentmarker}{}%
\end{pgfscope}%
\end{pgfscope}%
\begin{pgfscope}%
\definecolor{textcolor}{rgb}{0.000000,0.000000,0.000000}%
\pgfsetstrokecolor{textcolor}%
\pgfsetfillcolor{textcolor}%
\pgftext[x=0.525308in,y=3.559775in,left,base]{\color{textcolor}\rmfamily\fontsize{10.000000}{12.000000}\selectfont \(\displaystyle 1.0\)}%
\end{pgfscope}%
\begin{pgfscope}%
\pgfpathrectangle{\pgfqpoint{0.800000in}{0.528000in}}{\pgfqpoint{4.960000in}{3.696000in}}%
\pgfusepath{clip}%
\pgfsetbuttcap%
\pgfsetroundjoin%
\pgfsetlinewidth{0.803000pt}%
\definecolor{currentstroke}{rgb}{0.800000,0.800000,0.800000}%
\pgfsetstrokecolor{currentstroke}%
\pgfsetdash{{0.800000pt}{1.320000pt}}{0.000000pt}%
\pgfpathmoveto{\pgfqpoint{0.800000in}{4.224000in}}%
\pgfpathlineto{\pgfqpoint{5.760000in}{4.224000in}}%
\pgfusepath{stroke}%
\end{pgfscope}%
\begin{pgfscope}%
\pgfsetbuttcap%
\pgfsetroundjoin%
\definecolor{currentfill}{rgb}{0.000000,0.000000,0.000000}%
\pgfsetfillcolor{currentfill}%
\pgfsetlinewidth{0.803000pt}%
\definecolor{currentstroke}{rgb}{0.000000,0.000000,0.000000}%
\pgfsetstrokecolor{currentstroke}%
\pgfsetdash{}{0pt}%
\pgfsys@defobject{currentmarker}{\pgfqpoint{-0.048611in}{0.000000in}}{\pgfqpoint{0.000000in}{0.000000in}}{%
\pgfpathmoveto{\pgfqpoint{0.000000in}{0.000000in}}%
\pgfpathlineto{\pgfqpoint{-0.048611in}{0.000000in}}%
\pgfusepath{stroke,fill}%
}%
\begin{pgfscope}%
\pgfsys@transformshift{0.800000in}{4.224000in}%
\pgfsys@useobject{currentmarker}{}%
\end{pgfscope}%
\end{pgfscope}%
\begin{pgfscope}%
\definecolor{textcolor}{rgb}{0.000000,0.000000,0.000000}%
\pgfsetstrokecolor{textcolor}%
\pgfsetfillcolor{textcolor}%
\pgftext[x=0.525308in,y=4.175775in,left,base]{\color{textcolor}\rmfamily\fontsize{10.000000}{12.000000}\selectfont \(\displaystyle 1.5\)}%
\end{pgfscope}%
\begin{pgfscope}%
\definecolor{textcolor}{rgb}{0.000000,0.000000,0.000000}%
\pgfsetstrokecolor{textcolor}%
\pgfsetfillcolor{textcolor}%
\pgftext[x=0.361727in,y=2.376000in,,bottom]{\color{textcolor}\rmfamily\fontsize{10.000000}{12.000000}\selectfont y}%
\end{pgfscope}%
\begin{pgfscope}%
\pgfpathrectangle{\pgfqpoint{0.800000in}{0.528000in}}{\pgfqpoint{4.960000in}{3.696000in}}%
\pgfusepath{clip}%
\pgfsetrectcap%
\pgfsetroundjoin%
\pgfsetlinewidth{1.505625pt}%
\definecolor{currentstroke}{rgb}{0.000000,0.000000,0.000000}%
\pgfsetstrokecolor{currentstroke}%
\pgfsetdash{}{0pt}%
\pgfpathmoveto{\pgfqpoint{0.790000in}{2.376000in}}%
\pgfpathlineto{\pgfqpoint{5.770000in}{2.376000in}}%
\pgfpathlineto{\pgfqpoint{5.770000in}{2.376000in}}%
\pgfusepath{stroke}%
\end{pgfscope}%
\begin{pgfscope}%
\pgfpathrectangle{\pgfqpoint{0.800000in}{0.528000in}}{\pgfqpoint{4.960000in}{3.696000in}}%
\pgfusepath{clip}%
\pgfsetrectcap%
\pgfsetroundjoin%
\pgfsetlinewidth{1.505625pt}%
\definecolor{currentstroke}{rgb}{0.000000,0.000000,0.000000}%
\pgfsetstrokecolor{currentstroke}%
\pgfsetdash{}{0pt}%
\pgfpathmoveto{\pgfqpoint{0.942056in}{0.518000in}}%
\pgfpathlineto{\pgfqpoint{0.942056in}{4.234000in}}%
\pgfpathlineto{\pgfqpoint{0.942056in}{4.234000in}}%
\pgfusepath{stroke}%
\end{pgfscope}%
\begin{pgfscope}%
\pgfpathrectangle{\pgfqpoint{0.800000in}{0.528000in}}{\pgfqpoint{4.960000in}{3.696000in}}%
\pgfusepath{clip}%
\pgfsetrectcap%
\pgfsetroundjoin%
\pgfsetlinewidth{1.505625pt}%
\definecolor{currentstroke}{rgb}{0.121569,0.466667,0.705882}%
\pgfsetstrokecolor{currentstroke}%
\pgfsetdash{}{0pt}%
\pgfpathmoveto{\pgfqpoint{0.942056in}{2.376000in}}%
\pgfpathlineto{\pgfqpoint{1.046984in}{2.559507in}}%
\pgfpathlineto{\pgfqpoint{1.129587in}{2.699720in}}%
\pgfpathlineto{\pgfqpoint{1.203260in}{2.820438in}}%
\pgfpathlineto{\pgfqpoint{1.268003in}{2.922373in}}%
\pgfpathlineto{\pgfqpoint{1.328281in}{3.013201in}}%
\pgfpathlineto{\pgfqpoint{1.384094in}{3.093337in}}%
\pgfpathlineto{\pgfqpoint{1.435442in}{3.163336in}}%
\pgfpathlineto{\pgfqpoint{1.484558in}{3.226647in}}%
\pgfpathlineto{\pgfqpoint{1.531441in}{3.283496in}}%
\pgfpathlineto{\pgfqpoint{1.576091in}{3.334160in}}%
\pgfpathlineto{\pgfqpoint{1.618509in}{3.378952in}}%
\pgfpathlineto{\pgfqpoint{1.658694in}{3.418223in}}%
\pgfpathlineto{\pgfqpoint{1.696647in}{3.452343in}}%
\pgfpathlineto{\pgfqpoint{1.734600in}{3.483450in}}%
\pgfpathlineto{\pgfqpoint{1.770320in}{3.509859in}}%
\pgfpathlineto{\pgfqpoint{1.803808in}{3.531994in}}%
\pgfpathlineto{\pgfqpoint{1.837296in}{3.551501in}}%
\pgfpathlineto{\pgfqpoint{1.868551in}{3.567257in}}%
\pgfpathlineto{\pgfqpoint{1.899806in}{3.580575in}}%
\pgfpathlineto{\pgfqpoint{1.931061in}{3.591383in}}%
\pgfpathlineto{\pgfqpoint{1.960084in}{3.599108in}}%
\pgfpathlineto{\pgfqpoint{1.989107in}{3.604550in}}%
\pgfpathlineto{\pgfqpoint{2.018130in}{3.607651in}}%
\pgfpathlineto{\pgfqpoint{2.044920in}{3.608385in}}%
\pgfpathlineto{\pgfqpoint{2.111896in}{3.606397in}}%
\pgfpathlineto{\pgfqpoint{2.140918in}{3.602118in}}%
\pgfpathlineto{\pgfqpoint{2.169941in}{3.595527in}}%
\pgfpathlineto{\pgfqpoint{2.198964in}{3.586682in}}%
\pgfpathlineto{\pgfqpoint{2.230219in}{3.574700in}}%
\pgfpathlineto{\pgfqpoint{2.261474in}{3.560240in}}%
\pgfpathlineto{\pgfqpoint{2.292730in}{3.543376in}}%
\pgfpathlineto{\pgfqpoint{2.326217in}{3.522720in}}%
\pgfpathlineto{\pgfqpoint{2.359705in}{3.499473in}}%
\pgfpathlineto{\pgfqpoint{2.395425in}{3.471920in}}%
\pgfpathlineto{\pgfqpoint{2.431146in}{3.441628in}}%
\pgfpathlineto{\pgfqpoint{2.469099in}{3.406560in}}%
\pgfpathlineto{\pgfqpoint{2.509284in}{3.366331in}}%
\pgfpathlineto{\pgfqpoint{2.551702in}{3.320578in}}%
\pgfpathlineto{\pgfqpoint{2.596352in}{3.268959in}}%
\pgfpathlineto{\pgfqpoint{2.643235in}{3.211168in}}%
\pgfpathlineto{\pgfqpoint{2.692350in}{3.146940in}}%
\pgfpathlineto{\pgfqpoint{2.743698in}{3.076056in}}%
\pgfpathlineto{\pgfqpoint{2.799511in}{2.995044in}}%
\pgfpathlineto{\pgfqpoint{2.857557in}{2.906838in}}%
\pgfpathlineto{\pgfqpoint{2.922300in}{2.804254in}}%
\pgfpathlineto{\pgfqpoint{2.993741in}{2.686658in}}%
\pgfpathlineto{\pgfqpoint{3.074111in}{2.549879in}}%
\pgfpathlineto{\pgfqpoint{3.170110in}{2.381925in}}%
\pgfpathlineto{\pgfqpoint{3.279503in}{2.190569in}}%
\pgfpathlineto{\pgfqpoint{3.362106in}{2.050418in}}%
\pgfpathlineto{\pgfqpoint{3.433547in}{1.933356in}}%
\pgfpathlineto{\pgfqpoint{3.498290in}{1.831348in}}%
\pgfpathlineto{\pgfqpoint{3.558568in}{1.740442in}}%
\pgfpathlineto{\pgfqpoint{3.614381in}{1.660225in}}%
\pgfpathlineto{\pgfqpoint{3.665729in}{1.590146in}}%
\pgfpathlineto{\pgfqpoint{3.714844in}{1.526751in}}%
\pgfpathlineto{\pgfqpoint{3.761727in}{1.469815in}}%
\pgfpathlineto{\pgfqpoint{3.806378in}{1.419064in}}%
\pgfpathlineto{\pgfqpoint{3.848795in}{1.374183in}}%
\pgfpathlineto{\pgfqpoint{3.888981in}{1.334825in}}%
\pgfpathlineto{\pgfqpoint{3.926934in}{1.300618in}}%
\pgfpathlineto{\pgfqpoint{3.964886in}{1.269421in}}%
\pgfpathlineto{\pgfqpoint{4.000607in}{1.242923in}}%
\pgfpathlineto{\pgfqpoint{4.034094in}{1.220702in}}%
\pgfpathlineto{\pgfqpoint{4.067582in}{1.201106in}}%
\pgfpathlineto{\pgfqpoint{4.098838in}{1.185264in}}%
\pgfpathlineto{\pgfqpoint{4.130093in}{1.171858in}}%
\pgfpathlineto{\pgfqpoint{4.161348in}{1.160959in}}%
\pgfpathlineto{\pgfqpoint{4.190371in}{1.153147in}}%
\pgfpathlineto{\pgfqpoint{4.219394in}{1.147616in}}%
\pgfpathlineto{\pgfqpoint{4.248416in}{1.144424in}}%
\pgfpathlineto{\pgfqpoint{4.275207in}{1.143604in}}%
\pgfpathlineto{\pgfqpoint{4.339950in}{1.145261in}}%
\pgfpathlineto{\pgfqpoint{4.368972in}{1.149269in}}%
\pgfpathlineto{\pgfqpoint{4.397995in}{1.155596in}}%
\pgfpathlineto{\pgfqpoint{4.427018in}{1.164184in}}%
\pgfpathlineto{\pgfqpoint{4.456041in}{1.174977in}}%
\pgfpathlineto{\pgfqpoint{4.487296in}{1.188999in}}%
\pgfpathlineto{\pgfqpoint{4.518551in}{1.205439in}}%
\pgfpathlineto{\pgfqpoint{4.552039in}{1.225655in}}%
\pgfpathlineto{\pgfqpoint{4.585527in}{1.248476in}}%
\pgfpathlineto{\pgfqpoint{4.621247in}{1.275591in}}%
\pgfpathlineto{\pgfqpoint{4.656967in}{1.305463in}}%
\pgfpathlineto{\pgfqpoint{4.694920in}{1.340102in}}%
\pgfpathlineto{\pgfqpoint{4.735105in}{1.379896in}}%
\pgfpathlineto{\pgfqpoint{4.777523in}{1.425215in}}%
\pgfpathlineto{\pgfqpoint{4.819941in}{1.473760in}}%
\pgfpathlineto{\pgfqpoint{4.864592in}{1.528150in}}%
\pgfpathlineto{\pgfqpoint{4.913707in}{1.591629in}}%
\pgfpathlineto{\pgfqpoint{4.965055in}{1.661790in}}%
\pgfpathlineto{\pgfqpoint{5.018635in}{1.738799in}}%
\pgfpathlineto{\pgfqpoint{5.076681in}{1.826190in}}%
\pgfpathlineto{\pgfqpoint{5.139191in}{1.924399in}}%
\pgfpathlineto{\pgfqpoint{5.208400in}{2.037409in}}%
\pgfpathlineto{\pgfqpoint{5.286538in}{2.169458in}}%
\pgfpathlineto{\pgfqpoint{5.378071in}{2.328704in}}%
\pgfpathlineto{\pgfqpoint{5.404861in}{2.376000in}}%
\pgfpathlineto{\pgfqpoint{5.404861in}{2.376000in}}%
\pgfusepath{stroke}%
\end{pgfscope}%
\begin{pgfscope}%
\pgfsetrectcap%
\pgfsetmiterjoin%
\pgfsetlinewidth{0.803000pt}%
\definecolor{currentstroke}{rgb}{0.000000,0.000000,0.000000}%
\pgfsetstrokecolor{currentstroke}%
\pgfsetdash{}{0pt}%
\pgfpathmoveto{\pgfqpoint{0.800000in}{0.528000in}}%
\pgfpathlineto{\pgfqpoint{0.800000in}{4.224000in}}%
\pgfusepath{stroke}%
\end{pgfscope}%
\begin{pgfscope}%
\pgfsetrectcap%
\pgfsetmiterjoin%
\pgfsetlinewidth{0.803000pt}%
\definecolor{currentstroke}{rgb}{0.000000,0.000000,0.000000}%
\pgfsetstrokecolor{currentstroke}%
\pgfsetdash{}{0pt}%
\pgfpathmoveto{\pgfqpoint{5.760000in}{0.528000in}}%
\pgfpathlineto{\pgfqpoint{5.760000in}{4.224000in}}%
\pgfusepath{stroke}%
\end{pgfscope}%
\begin{pgfscope}%
\pgfsetrectcap%
\pgfsetmiterjoin%
\pgfsetlinewidth{0.803000pt}%
\definecolor{currentstroke}{rgb}{0.000000,0.000000,0.000000}%
\pgfsetstrokecolor{currentstroke}%
\pgfsetdash{}{0pt}%
\pgfpathmoveto{\pgfqpoint{0.800000in}{0.528000in}}%
\pgfpathlineto{\pgfqpoint{5.760000in}{0.528000in}}%
\pgfusepath{stroke}%
\end{pgfscope}%
\begin{pgfscope}%
\pgfsetrectcap%
\pgfsetmiterjoin%
\pgfsetlinewidth{0.803000pt}%
\definecolor{currentstroke}{rgb}{0.000000,0.000000,0.000000}%
\pgfsetstrokecolor{currentstroke}%
\pgfsetdash{}{0pt}%
\pgfpathmoveto{\pgfqpoint{0.800000in}{4.224000in}}%
\pgfpathlineto{\pgfqpoint{5.760000in}{4.224000in}}%
\pgfusepath{stroke}%
\end{pgfscope}%
\begin{pgfscope}%
\pgfsetbuttcap%
\pgfsetmiterjoin%
\definecolor{currentfill}{rgb}{1.000000,1.000000,1.000000}%
\pgfsetfillcolor{currentfill}%
\pgfsetfillopacity{0.800000}%
\pgfsetlinewidth{1.003750pt}%
\definecolor{currentstroke}{rgb}{0.800000,0.800000,0.800000}%
\pgfsetstrokecolor{currentstroke}%
\pgfsetstrokeopacity{0.800000}%
\pgfsetdash{}{0pt}%
\pgfpathmoveto{\pgfqpoint{3.137021in}{3.904556in}}%
\pgfpathlineto{\pgfqpoint{5.662778in}{3.904556in}}%
\pgfpathquadraticcurveto{\pgfqpoint{5.690556in}{3.904556in}}{\pgfqpoint{5.690556in}{3.932333in}}%
\pgfpathlineto{\pgfqpoint{5.690556in}{4.126778in}}%
\pgfpathquadraticcurveto{\pgfqpoint{5.690556in}{4.154556in}}{\pgfqpoint{5.662778in}{4.154556in}}%
\pgfpathlineto{\pgfqpoint{3.137021in}{4.154556in}}%
\pgfpathquadraticcurveto{\pgfqpoint{3.109243in}{4.154556in}}{\pgfqpoint{3.109243in}{4.126778in}}%
\pgfpathlineto{\pgfqpoint{3.109243in}{3.932333in}}%
\pgfpathquadraticcurveto{\pgfqpoint{3.109243in}{3.904556in}}{\pgfqpoint{3.137021in}{3.904556in}}%
\pgfpathclose%
\pgfusepath{stroke,fill}%
\end{pgfscope}%
\begin{pgfscope}%
\pgfsetrectcap%
\pgfsetroundjoin%
\pgfsetlinewidth{1.505625pt}%
\definecolor{currentstroke}{rgb}{0.121569,0.466667,0.705882}%
\pgfsetstrokecolor{currentstroke}%
\pgfsetdash{}{0pt}%
\pgfpathmoveto{\pgfqpoint{3.164799in}{4.043444in}}%
\pgfpathlineto{\pgfqpoint{3.442576in}{4.043444in}}%
\pgfusepath{stroke}%
\end{pgfscope}%
\begin{pgfscope}%
\definecolor{textcolor}{rgb}{0.000000,0.000000,0.000000}%
\pgfsetstrokecolor{textcolor}%
\pgfsetfillcolor{textcolor}%
\pgftext[x=3.553687in,y=3.994833in,left,base]{\color{textcolor}\rmfamily\fontsize{10.000000}{12.000000}\selectfont Lagrange Approximation of \(\displaystyle \sin(x)\)}%
\end{pgfscope}%
\end{pgfpicture}%
\makeatother%
\endgroup%
}
                \caption{Lagrange Approximation Expanded onto $[0, 2\pi]$}
                \label{fig:LagrangeExpanded}
            \end{center}
        \end{figure}
        
        \begin{figure}[H]
            \begin{center}
                \scalebox{.7}{%% Creator: Matplotlib, PGF backend
%%
%% To include the figure in your LaTeX document, write
%%   \input{<filename>.pgf}
%%
%% Make sure the required packages are loaded in your preamble
%%   \usepackage{pgf}
%%
%% Figures using additional raster images can only be included by \input if
%% they are in the same directory as the main LaTeX file. For loading figures
%% from other directories you can use the `import` package
%%   \usepackage{import}
%% and then include the figures with
%%   \import{<path to file>}{<filename>.pgf}
%%
%% Matplotlib used the following preamble
%%
\begingroup%
\makeatletter%
\begin{pgfpicture}%
\pgfpathrectangle{\pgfpointorigin}{\pgfqpoint{6.400000in}{4.800000in}}%
\pgfusepath{use as bounding box, clip}%
\begin{pgfscope}%
\pgfsetbuttcap%
\pgfsetmiterjoin%
\definecolor{currentfill}{rgb}{1.000000,1.000000,1.000000}%
\pgfsetfillcolor{currentfill}%
\pgfsetlinewidth{0.000000pt}%
\definecolor{currentstroke}{rgb}{1.000000,1.000000,1.000000}%
\pgfsetstrokecolor{currentstroke}%
\pgfsetdash{}{0pt}%
\pgfpathmoveto{\pgfqpoint{0.000000in}{0.000000in}}%
\pgfpathlineto{\pgfqpoint{6.400000in}{0.000000in}}%
\pgfpathlineto{\pgfqpoint{6.400000in}{4.800000in}}%
\pgfpathlineto{\pgfqpoint{0.000000in}{4.800000in}}%
\pgfpathclose%
\pgfusepath{fill}%
\end{pgfscope}%
\begin{pgfscope}%
\pgfsetbuttcap%
\pgfsetmiterjoin%
\definecolor{currentfill}{rgb}{1.000000,1.000000,1.000000}%
\pgfsetfillcolor{currentfill}%
\pgfsetlinewidth{0.000000pt}%
\definecolor{currentstroke}{rgb}{0.000000,0.000000,0.000000}%
\pgfsetstrokecolor{currentstroke}%
\pgfsetstrokeopacity{0.000000}%
\pgfsetdash{}{0pt}%
\pgfpathmoveto{\pgfqpoint{0.800000in}{0.528000in}}%
\pgfpathlineto{\pgfqpoint{5.760000in}{0.528000in}}%
\pgfpathlineto{\pgfqpoint{5.760000in}{4.224000in}}%
\pgfpathlineto{\pgfqpoint{0.800000in}{4.224000in}}%
\pgfpathclose%
\pgfusepath{fill}%
\end{pgfscope}%
\begin{pgfscope}%
\pgfpathrectangle{\pgfqpoint{0.800000in}{0.528000in}}{\pgfqpoint{4.960000in}{3.696000in}}%
\pgfusepath{clip}%
\pgfsetbuttcap%
\pgfsetroundjoin%
\pgfsetlinewidth{0.803000pt}%
\definecolor{currentstroke}{rgb}{0.800000,0.800000,0.800000}%
\pgfsetstrokecolor{currentstroke}%
\pgfsetdash{{0.800000pt}{1.320000pt}}{0.000000pt}%
\pgfpathmoveto{\pgfqpoint{0.942056in}{0.528000in}}%
\pgfpathlineto{\pgfqpoint{0.942056in}{4.224000in}}%
\pgfusepath{stroke}%
\end{pgfscope}%
\begin{pgfscope}%
\pgfsetbuttcap%
\pgfsetroundjoin%
\definecolor{currentfill}{rgb}{0.000000,0.000000,0.000000}%
\pgfsetfillcolor{currentfill}%
\pgfsetlinewidth{0.803000pt}%
\definecolor{currentstroke}{rgb}{0.000000,0.000000,0.000000}%
\pgfsetstrokecolor{currentstroke}%
\pgfsetdash{}{0pt}%
\pgfsys@defobject{currentmarker}{\pgfqpoint{0.000000in}{-0.048611in}}{\pgfqpoint{0.000000in}{0.000000in}}{%
\pgfpathmoveto{\pgfqpoint{0.000000in}{0.000000in}}%
\pgfpathlineto{\pgfqpoint{0.000000in}{-0.048611in}}%
\pgfusepath{stroke,fill}%
}%
\begin{pgfscope}%
\pgfsys@transformshift{0.942056in}{0.528000in}%
\pgfsys@useobject{currentmarker}{}%
\end{pgfscope}%
\end{pgfscope}%
\begin{pgfscope}%
\definecolor{textcolor}{rgb}{0.000000,0.000000,0.000000}%
\pgfsetstrokecolor{textcolor}%
\pgfsetfillcolor{textcolor}%
\pgftext[x=0.942056in,y=0.430778in,,top]{\color{textcolor}\rmfamily\fontsize{10.000000}{12.000000}\selectfont \(\displaystyle 0\)}%
\end{pgfscope}%
\begin{pgfscope}%
\pgfpathrectangle{\pgfqpoint{0.800000in}{0.528000in}}{\pgfqpoint{4.960000in}{3.696000in}}%
\pgfusepath{clip}%
\pgfsetbuttcap%
\pgfsetroundjoin%
\pgfsetlinewidth{0.803000pt}%
\definecolor{currentstroke}{rgb}{0.800000,0.800000,0.800000}%
\pgfsetstrokecolor{currentstroke}%
\pgfsetdash{{0.800000pt}{1.320000pt}}{0.000000pt}%
\pgfpathmoveto{\pgfqpoint{1.652333in}{0.528000in}}%
\pgfpathlineto{\pgfqpoint{1.652333in}{4.224000in}}%
\pgfusepath{stroke}%
\end{pgfscope}%
\begin{pgfscope}%
\pgfsetbuttcap%
\pgfsetroundjoin%
\definecolor{currentfill}{rgb}{0.000000,0.000000,0.000000}%
\pgfsetfillcolor{currentfill}%
\pgfsetlinewidth{0.803000pt}%
\definecolor{currentstroke}{rgb}{0.000000,0.000000,0.000000}%
\pgfsetstrokecolor{currentstroke}%
\pgfsetdash{}{0pt}%
\pgfsys@defobject{currentmarker}{\pgfqpoint{0.000000in}{-0.048611in}}{\pgfqpoint{0.000000in}{0.000000in}}{%
\pgfpathmoveto{\pgfqpoint{0.000000in}{0.000000in}}%
\pgfpathlineto{\pgfqpoint{0.000000in}{-0.048611in}}%
\pgfusepath{stroke,fill}%
}%
\begin{pgfscope}%
\pgfsys@transformshift{1.652333in}{0.528000in}%
\pgfsys@useobject{currentmarker}{}%
\end{pgfscope}%
\end{pgfscope}%
\begin{pgfscope}%
\definecolor{textcolor}{rgb}{0.000000,0.000000,0.000000}%
\pgfsetstrokecolor{textcolor}%
\pgfsetfillcolor{textcolor}%
\pgftext[x=1.652333in,y=0.430778in,,top]{\color{textcolor}\rmfamily\fontsize{10.000000}{12.000000}\selectfont \(\displaystyle 1\)}%
\end{pgfscope}%
\begin{pgfscope}%
\pgfpathrectangle{\pgfqpoint{0.800000in}{0.528000in}}{\pgfqpoint{4.960000in}{3.696000in}}%
\pgfusepath{clip}%
\pgfsetbuttcap%
\pgfsetroundjoin%
\pgfsetlinewidth{0.803000pt}%
\definecolor{currentstroke}{rgb}{0.800000,0.800000,0.800000}%
\pgfsetstrokecolor{currentstroke}%
\pgfsetdash{{0.800000pt}{1.320000pt}}{0.000000pt}%
\pgfpathmoveto{\pgfqpoint{2.362611in}{0.528000in}}%
\pgfpathlineto{\pgfqpoint{2.362611in}{4.224000in}}%
\pgfusepath{stroke}%
\end{pgfscope}%
\begin{pgfscope}%
\pgfsetbuttcap%
\pgfsetroundjoin%
\definecolor{currentfill}{rgb}{0.000000,0.000000,0.000000}%
\pgfsetfillcolor{currentfill}%
\pgfsetlinewidth{0.803000pt}%
\definecolor{currentstroke}{rgb}{0.000000,0.000000,0.000000}%
\pgfsetstrokecolor{currentstroke}%
\pgfsetdash{}{0pt}%
\pgfsys@defobject{currentmarker}{\pgfqpoint{0.000000in}{-0.048611in}}{\pgfqpoint{0.000000in}{0.000000in}}{%
\pgfpathmoveto{\pgfqpoint{0.000000in}{0.000000in}}%
\pgfpathlineto{\pgfqpoint{0.000000in}{-0.048611in}}%
\pgfusepath{stroke,fill}%
}%
\begin{pgfscope}%
\pgfsys@transformshift{2.362611in}{0.528000in}%
\pgfsys@useobject{currentmarker}{}%
\end{pgfscope}%
\end{pgfscope}%
\begin{pgfscope}%
\definecolor{textcolor}{rgb}{0.000000,0.000000,0.000000}%
\pgfsetstrokecolor{textcolor}%
\pgfsetfillcolor{textcolor}%
\pgftext[x=2.362611in,y=0.430778in,,top]{\color{textcolor}\rmfamily\fontsize{10.000000}{12.000000}\selectfont \(\displaystyle 2\)}%
\end{pgfscope}%
\begin{pgfscope}%
\pgfpathrectangle{\pgfqpoint{0.800000in}{0.528000in}}{\pgfqpoint{4.960000in}{3.696000in}}%
\pgfusepath{clip}%
\pgfsetbuttcap%
\pgfsetroundjoin%
\pgfsetlinewidth{0.803000pt}%
\definecolor{currentstroke}{rgb}{0.800000,0.800000,0.800000}%
\pgfsetstrokecolor{currentstroke}%
\pgfsetdash{{0.800000pt}{1.320000pt}}{0.000000pt}%
\pgfpathmoveto{\pgfqpoint{3.072888in}{0.528000in}}%
\pgfpathlineto{\pgfqpoint{3.072888in}{4.224000in}}%
\pgfusepath{stroke}%
\end{pgfscope}%
\begin{pgfscope}%
\pgfsetbuttcap%
\pgfsetroundjoin%
\definecolor{currentfill}{rgb}{0.000000,0.000000,0.000000}%
\pgfsetfillcolor{currentfill}%
\pgfsetlinewidth{0.803000pt}%
\definecolor{currentstroke}{rgb}{0.000000,0.000000,0.000000}%
\pgfsetstrokecolor{currentstroke}%
\pgfsetdash{}{0pt}%
\pgfsys@defobject{currentmarker}{\pgfqpoint{0.000000in}{-0.048611in}}{\pgfqpoint{0.000000in}{0.000000in}}{%
\pgfpathmoveto{\pgfqpoint{0.000000in}{0.000000in}}%
\pgfpathlineto{\pgfqpoint{0.000000in}{-0.048611in}}%
\pgfusepath{stroke,fill}%
}%
\begin{pgfscope}%
\pgfsys@transformshift{3.072888in}{0.528000in}%
\pgfsys@useobject{currentmarker}{}%
\end{pgfscope}%
\end{pgfscope}%
\begin{pgfscope}%
\definecolor{textcolor}{rgb}{0.000000,0.000000,0.000000}%
\pgfsetstrokecolor{textcolor}%
\pgfsetfillcolor{textcolor}%
\pgftext[x=3.072888in,y=0.430778in,,top]{\color{textcolor}\rmfamily\fontsize{10.000000}{12.000000}\selectfont \(\displaystyle 3\)}%
\end{pgfscope}%
\begin{pgfscope}%
\pgfpathrectangle{\pgfqpoint{0.800000in}{0.528000in}}{\pgfqpoint{4.960000in}{3.696000in}}%
\pgfusepath{clip}%
\pgfsetbuttcap%
\pgfsetroundjoin%
\pgfsetlinewidth{0.803000pt}%
\definecolor{currentstroke}{rgb}{0.800000,0.800000,0.800000}%
\pgfsetstrokecolor{currentstroke}%
\pgfsetdash{{0.800000pt}{1.320000pt}}{0.000000pt}%
\pgfpathmoveto{\pgfqpoint{3.783166in}{0.528000in}}%
\pgfpathlineto{\pgfqpoint{3.783166in}{4.224000in}}%
\pgfusepath{stroke}%
\end{pgfscope}%
\begin{pgfscope}%
\pgfsetbuttcap%
\pgfsetroundjoin%
\definecolor{currentfill}{rgb}{0.000000,0.000000,0.000000}%
\pgfsetfillcolor{currentfill}%
\pgfsetlinewidth{0.803000pt}%
\definecolor{currentstroke}{rgb}{0.000000,0.000000,0.000000}%
\pgfsetstrokecolor{currentstroke}%
\pgfsetdash{}{0pt}%
\pgfsys@defobject{currentmarker}{\pgfqpoint{0.000000in}{-0.048611in}}{\pgfqpoint{0.000000in}{0.000000in}}{%
\pgfpathmoveto{\pgfqpoint{0.000000in}{0.000000in}}%
\pgfpathlineto{\pgfqpoint{0.000000in}{-0.048611in}}%
\pgfusepath{stroke,fill}%
}%
\begin{pgfscope}%
\pgfsys@transformshift{3.783166in}{0.528000in}%
\pgfsys@useobject{currentmarker}{}%
\end{pgfscope}%
\end{pgfscope}%
\begin{pgfscope}%
\definecolor{textcolor}{rgb}{0.000000,0.000000,0.000000}%
\pgfsetstrokecolor{textcolor}%
\pgfsetfillcolor{textcolor}%
\pgftext[x=3.783166in,y=0.430778in,,top]{\color{textcolor}\rmfamily\fontsize{10.000000}{12.000000}\selectfont \(\displaystyle 4\)}%
\end{pgfscope}%
\begin{pgfscope}%
\pgfpathrectangle{\pgfqpoint{0.800000in}{0.528000in}}{\pgfqpoint{4.960000in}{3.696000in}}%
\pgfusepath{clip}%
\pgfsetbuttcap%
\pgfsetroundjoin%
\pgfsetlinewidth{0.803000pt}%
\definecolor{currentstroke}{rgb}{0.800000,0.800000,0.800000}%
\pgfsetstrokecolor{currentstroke}%
\pgfsetdash{{0.800000pt}{1.320000pt}}{0.000000pt}%
\pgfpathmoveto{\pgfqpoint{4.493443in}{0.528000in}}%
\pgfpathlineto{\pgfqpoint{4.493443in}{4.224000in}}%
\pgfusepath{stroke}%
\end{pgfscope}%
\begin{pgfscope}%
\pgfsetbuttcap%
\pgfsetroundjoin%
\definecolor{currentfill}{rgb}{0.000000,0.000000,0.000000}%
\pgfsetfillcolor{currentfill}%
\pgfsetlinewidth{0.803000pt}%
\definecolor{currentstroke}{rgb}{0.000000,0.000000,0.000000}%
\pgfsetstrokecolor{currentstroke}%
\pgfsetdash{}{0pt}%
\pgfsys@defobject{currentmarker}{\pgfqpoint{0.000000in}{-0.048611in}}{\pgfqpoint{0.000000in}{0.000000in}}{%
\pgfpathmoveto{\pgfqpoint{0.000000in}{0.000000in}}%
\pgfpathlineto{\pgfqpoint{0.000000in}{-0.048611in}}%
\pgfusepath{stroke,fill}%
}%
\begin{pgfscope}%
\pgfsys@transformshift{4.493443in}{0.528000in}%
\pgfsys@useobject{currentmarker}{}%
\end{pgfscope}%
\end{pgfscope}%
\begin{pgfscope}%
\definecolor{textcolor}{rgb}{0.000000,0.000000,0.000000}%
\pgfsetstrokecolor{textcolor}%
\pgfsetfillcolor{textcolor}%
\pgftext[x=4.493443in,y=0.430778in,,top]{\color{textcolor}\rmfamily\fontsize{10.000000}{12.000000}\selectfont \(\displaystyle 5\)}%
\end{pgfscope}%
\begin{pgfscope}%
\pgfpathrectangle{\pgfqpoint{0.800000in}{0.528000in}}{\pgfqpoint{4.960000in}{3.696000in}}%
\pgfusepath{clip}%
\pgfsetbuttcap%
\pgfsetroundjoin%
\pgfsetlinewidth{0.803000pt}%
\definecolor{currentstroke}{rgb}{0.800000,0.800000,0.800000}%
\pgfsetstrokecolor{currentstroke}%
\pgfsetdash{{0.800000pt}{1.320000pt}}{0.000000pt}%
\pgfpathmoveto{\pgfqpoint{5.203721in}{0.528000in}}%
\pgfpathlineto{\pgfqpoint{5.203721in}{4.224000in}}%
\pgfusepath{stroke}%
\end{pgfscope}%
\begin{pgfscope}%
\pgfsetbuttcap%
\pgfsetroundjoin%
\definecolor{currentfill}{rgb}{0.000000,0.000000,0.000000}%
\pgfsetfillcolor{currentfill}%
\pgfsetlinewidth{0.803000pt}%
\definecolor{currentstroke}{rgb}{0.000000,0.000000,0.000000}%
\pgfsetstrokecolor{currentstroke}%
\pgfsetdash{}{0pt}%
\pgfsys@defobject{currentmarker}{\pgfqpoint{0.000000in}{-0.048611in}}{\pgfqpoint{0.000000in}{0.000000in}}{%
\pgfpathmoveto{\pgfqpoint{0.000000in}{0.000000in}}%
\pgfpathlineto{\pgfqpoint{0.000000in}{-0.048611in}}%
\pgfusepath{stroke,fill}%
}%
\begin{pgfscope}%
\pgfsys@transformshift{5.203721in}{0.528000in}%
\pgfsys@useobject{currentmarker}{}%
\end{pgfscope}%
\end{pgfscope}%
\begin{pgfscope}%
\definecolor{textcolor}{rgb}{0.000000,0.000000,0.000000}%
\pgfsetstrokecolor{textcolor}%
\pgfsetfillcolor{textcolor}%
\pgftext[x=5.203721in,y=0.430778in,,top]{\color{textcolor}\rmfamily\fontsize{10.000000}{12.000000}\selectfont \(\displaystyle 6\)}%
\end{pgfscope}%
\begin{pgfscope}%
\definecolor{textcolor}{rgb}{0.000000,0.000000,0.000000}%
\pgfsetstrokecolor{textcolor}%
\pgfsetfillcolor{textcolor}%
\pgftext[x=3.280000in,y=0.251766in,,top]{\color{textcolor}\rmfamily\fontsize{10.000000}{12.000000}\selectfont x}%
\end{pgfscope}%
\begin{pgfscope}%
\pgfpathrectangle{\pgfqpoint{0.800000in}{0.528000in}}{\pgfqpoint{4.960000in}{3.696000in}}%
\pgfusepath{clip}%
\pgfsetbuttcap%
\pgfsetroundjoin%
\pgfsetlinewidth{0.803000pt}%
\definecolor{currentstroke}{rgb}{0.800000,0.800000,0.800000}%
\pgfsetstrokecolor{currentstroke}%
\pgfsetdash{{0.800000pt}{1.320000pt}}{0.000000pt}%
\pgfpathmoveto{\pgfqpoint{0.800000in}{0.528000in}}%
\pgfpathlineto{\pgfqpoint{5.760000in}{0.528000in}}%
\pgfusepath{stroke}%
\end{pgfscope}%
\begin{pgfscope}%
\pgfsetbuttcap%
\pgfsetroundjoin%
\definecolor{currentfill}{rgb}{0.000000,0.000000,0.000000}%
\pgfsetfillcolor{currentfill}%
\pgfsetlinewidth{0.803000pt}%
\definecolor{currentstroke}{rgb}{0.000000,0.000000,0.000000}%
\pgfsetstrokecolor{currentstroke}%
\pgfsetdash{}{0pt}%
\pgfsys@defobject{currentmarker}{\pgfqpoint{-0.048611in}{0.000000in}}{\pgfqpoint{0.000000in}{0.000000in}}{%
\pgfpathmoveto{\pgfqpoint{0.000000in}{0.000000in}}%
\pgfpathlineto{\pgfqpoint{-0.048611in}{0.000000in}}%
\pgfusepath{stroke,fill}%
}%
\begin{pgfscope}%
\pgfsys@transformshift{0.800000in}{0.528000in}%
\pgfsys@useobject{currentmarker}{}%
\end{pgfscope}%
\end{pgfscope}%
\begin{pgfscope}%
\definecolor{textcolor}{rgb}{0.000000,0.000000,0.000000}%
\pgfsetstrokecolor{textcolor}%
\pgfsetfillcolor{textcolor}%
\pgftext[x=0.278394in,y=0.479775in,left,base]{\color{textcolor}\rmfamily\fontsize{10.000000}{12.000000}\selectfont \(\displaystyle -0.003\)}%
\end{pgfscope}%
\begin{pgfscope}%
\pgfpathrectangle{\pgfqpoint{0.800000in}{0.528000in}}{\pgfqpoint{4.960000in}{3.696000in}}%
\pgfusepath{clip}%
\pgfsetbuttcap%
\pgfsetroundjoin%
\pgfsetlinewidth{0.803000pt}%
\definecolor{currentstroke}{rgb}{0.800000,0.800000,0.800000}%
\pgfsetstrokecolor{currentstroke}%
\pgfsetdash{{0.800000pt}{1.320000pt}}{0.000000pt}%
\pgfpathmoveto{\pgfqpoint{0.800000in}{1.144000in}}%
\pgfpathlineto{\pgfqpoint{5.760000in}{1.144000in}}%
\pgfusepath{stroke}%
\end{pgfscope}%
\begin{pgfscope}%
\pgfsetbuttcap%
\pgfsetroundjoin%
\definecolor{currentfill}{rgb}{0.000000,0.000000,0.000000}%
\pgfsetfillcolor{currentfill}%
\pgfsetlinewidth{0.803000pt}%
\definecolor{currentstroke}{rgb}{0.000000,0.000000,0.000000}%
\pgfsetstrokecolor{currentstroke}%
\pgfsetdash{}{0pt}%
\pgfsys@defobject{currentmarker}{\pgfqpoint{-0.048611in}{0.000000in}}{\pgfqpoint{0.000000in}{0.000000in}}{%
\pgfpathmoveto{\pgfqpoint{0.000000in}{0.000000in}}%
\pgfpathlineto{\pgfqpoint{-0.048611in}{0.000000in}}%
\pgfusepath{stroke,fill}%
}%
\begin{pgfscope}%
\pgfsys@transformshift{0.800000in}{1.144000in}%
\pgfsys@useobject{currentmarker}{}%
\end{pgfscope}%
\end{pgfscope}%
\begin{pgfscope}%
\definecolor{textcolor}{rgb}{0.000000,0.000000,0.000000}%
\pgfsetstrokecolor{textcolor}%
\pgfsetfillcolor{textcolor}%
\pgftext[x=0.278394in,y=1.095775in,left,base]{\color{textcolor}\rmfamily\fontsize{10.000000}{12.000000}\selectfont \(\displaystyle -0.002\)}%
\end{pgfscope}%
\begin{pgfscope}%
\pgfpathrectangle{\pgfqpoint{0.800000in}{0.528000in}}{\pgfqpoint{4.960000in}{3.696000in}}%
\pgfusepath{clip}%
\pgfsetbuttcap%
\pgfsetroundjoin%
\pgfsetlinewidth{0.803000pt}%
\definecolor{currentstroke}{rgb}{0.800000,0.800000,0.800000}%
\pgfsetstrokecolor{currentstroke}%
\pgfsetdash{{0.800000pt}{1.320000pt}}{0.000000pt}%
\pgfpathmoveto{\pgfqpoint{0.800000in}{1.760000in}}%
\pgfpathlineto{\pgfqpoint{5.760000in}{1.760000in}}%
\pgfusepath{stroke}%
\end{pgfscope}%
\begin{pgfscope}%
\pgfsetbuttcap%
\pgfsetroundjoin%
\definecolor{currentfill}{rgb}{0.000000,0.000000,0.000000}%
\pgfsetfillcolor{currentfill}%
\pgfsetlinewidth{0.803000pt}%
\definecolor{currentstroke}{rgb}{0.000000,0.000000,0.000000}%
\pgfsetstrokecolor{currentstroke}%
\pgfsetdash{}{0pt}%
\pgfsys@defobject{currentmarker}{\pgfqpoint{-0.048611in}{0.000000in}}{\pgfqpoint{0.000000in}{0.000000in}}{%
\pgfpathmoveto{\pgfqpoint{0.000000in}{0.000000in}}%
\pgfpathlineto{\pgfqpoint{-0.048611in}{0.000000in}}%
\pgfusepath{stroke,fill}%
}%
\begin{pgfscope}%
\pgfsys@transformshift{0.800000in}{1.760000in}%
\pgfsys@useobject{currentmarker}{}%
\end{pgfscope}%
\end{pgfscope}%
\begin{pgfscope}%
\definecolor{textcolor}{rgb}{0.000000,0.000000,0.000000}%
\pgfsetstrokecolor{textcolor}%
\pgfsetfillcolor{textcolor}%
\pgftext[x=0.278394in,y=1.711775in,left,base]{\color{textcolor}\rmfamily\fontsize{10.000000}{12.000000}\selectfont \(\displaystyle -0.001\)}%
\end{pgfscope}%
\begin{pgfscope}%
\pgfpathrectangle{\pgfqpoint{0.800000in}{0.528000in}}{\pgfqpoint{4.960000in}{3.696000in}}%
\pgfusepath{clip}%
\pgfsetbuttcap%
\pgfsetroundjoin%
\pgfsetlinewidth{0.803000pt}%
\definecolor{currentstroke}{rgb}{0.800000,0.800000,0.800000}%
\pgfsetstrokecolor{currentstroke}%
\pgfsetdash{{0.800000pt}{1.320000pt}}{0.000000pt}%
\pgfpathmoveto{\pgfqpoint{0.800000in}{2.376000in}}%
\pgfpathlineto{\pgfqpoint{5.760000in}{2.376000in}}%
\pgfusepath{stroke}%
\end{pgfscope}%
\begin{pgfscope}%
\pgfsetbuttcap%
\pgfsetroundjoin%
\definecolor{currentfill}{rgb}{0.000000,0.000000,0.000000}%
\pgfsetfillcolor{currentfill}%
\pgfsetlinewidth{0.803000pt}%
\definecolor{currentstroke}{rgb}{0.000000,0.000000,0.000000}%
\pgfsetstrokecolor{currentstroke}%
\pgfsetdash{}{0pt}%
\pgfsys@defobject{currentmarker}{\pgfqpoint{-0.048611in}{0.000000in}}{\pgfqpoint{0.000000in}{0.000000in}}{%
\pgfpathmoveto{\pgfqpoint{0.000000in}{0.000000in}}%
\pgfpathlineto{\pgfqpoint{-0.048611in}{0.000000in}}%
\pgfusepath{stroke,fill}%
}%
\begin{pgfscope}%
\pgfsys@transformshift{0.800000in}{2.376000in}%
\pgfsys@useobject{currentmarker}{}%
\end{pgfscope}%
\end{pgfscope}%
\begin{pgfscope}%
\definecolor{textcolor}{rgb}{0.000000,0.000000,0.000000}%
\pgfsetstrokecolor{textcolor}%
\pgfsetfillcolor{textcolor}%
\pgftext[x=0.386419in,y=2.327775in,left,base]{\color{textcolor}\rmfamily\fontsize{10.000000}{12.000000}\selectfont \(\displaystyle 0.000\)}%
\end{pgfscope}%
\begin{pgfscope}%
\pgfpathrectangle{\pgfqpoint{0.800000in}{0.528000in}}{\pgfqpoint{4.960000in}{3.696000in}}%
\pgfusepath{clip}%
\pgfsetbuttcap%
\pgfsetroundjoin%
\pgfsetlinewidth{0.803000pt}%
\definecolor{currentstroke}{rgb}{0.800000,0.800000,0.800000}%
\pgfsetstrokecolor{currentstroke}%
\pgfsetdash{{0.800000pt}{1.320000pt}}{0.000000pt}%
\pgfpathmoveto{\pgfqpoint{0.800000in}{2.992000in}}%
\pgfpathlineto{\pgfqpoint{5.760000in}{2.992000in}}%
\pgfusepath{stroke}%
\end{pgfscope}%
\begin{pgfscope}%
\pgfsetbuttcap%
\pgfsetroundjoin%
\definecolor{currentfill}{rgb}{0.000000,0.000000,0.000000}%
\pgfsetfillcolor{currentfill}%
\pgfsetlinewidth{0.803000pt}%
\definecolor{currentstroke}{rgb}{0.000000,0.000000,0.000000}%
\pgfsetstrokecolor{currentstroke}%
\pgfsetdash{}{0pt}%
\pgfsys@defobject{currentmarker}{\pgfqpoint{-0.048611in}{0.000000in}}{\pgfqpoint{0.000000in}{0.000000in}}{%
\pgfpathmoveto{\pgfqpoint{0.000000in}{0.000000in}}%
\pgfpathlineto{\pgfqpoint{-0.048611in}{0.000000in}}%
\pgfusepath{stroke,fill}%
}%
\begin{pgfscope}%
\pgfsys@transformshift{0.800000in}{2.992000in}%
\pgfsys@useobject{currentmarker}{}%
\end{pgfscope}%
\end{pgfscope}%
\begin{pgfscope}%
\definecolor{textcolor}{rgb}{0.000000,0.000000,0.000000}%
\pgfsetstrokecolor{textcolor}%
\pgfsetfillcolor{textcolor}%
\pgftext[x=0.386419in,y=2.943775in,left,base]{\color{textcolor}\rmfamily\fontsize{10.000000}{12.000000}\selectfont \(\displaystyle 0.001\)}%
\end{pgfscope}%
\begin{pgfscope}%
\pgfpathrectangle{\pgfqpoint{0.800000in}{0.528000in}}{\pgfqpoint{4.960000in}{3.696000in}}%
\pgfusepath{clip}%
\pgfsetbuttcap%
\pgfsetroundjoin%
\pgfsetlinewidth{0.803000pt}%
\definecolor{currentstroke}{rgb}{0.800000,0.800000,0.800000}%
\pgfsetstrokecolor{currentstroke}%
\pgfsetdash{{0.800000pt}{1.320000pt}}{0.000000pt}%
\pgfpathmoveto{\pgfqpoint{0.800000in}{3.608000in}}%
\pgfpathlineto{\pgfqpoint{5.760000in}{3.608000in}}%
\pgfusepath{stroke}%
\end{pgfscope}%
\begin{pgfscope}%
\pgfsetbuttcap%
\pgfsetroundjoin%
\definecolor{currentfill}{rgb}{0.000000,0.000000,0.000000}%
\pgfsetfillcolor{currentfill}%
\pgfsetlinewidth{0.803000pt}%
\definecolor{currentstroke}{rgb}{0.000000,0.000000,0.000000}%
\pgfsetstrokecolor{currentstroke}%
\pgfsetdash{}{0pt}%
\pgfsys@defobject{currentmarker}{\pgfqpoint{-0.048611in}{0.000000in}}{\pgfqpoint{0.000000in}{0.000000in}}{%
\pgfpathmoveto{\pgfqpoint{0.000000in}{0.000000in}}%
\pgfpathlineto{\pgfqpoint{-0.048611in}{0.000000in}}%
\pgfusepath{stroke,fill}%
}%
\begin{pgfscope}%
\pgfsys@transformshift{0.800000in}{3.608000in}%
\pgfsys@useobject{currentmarker}{}%
\end{pgfscope}%
\end{pgfscope}%
\begin{pgfscope}%
\definecolor{textcolor}{rgb}{0.000000,0.000000,0.000000}%
\pgfsetstrokecolor{textcolor}%
\pgfsetfillcolor{textcolor}%
\pgftext[x=0.386419in,y=3.559775in,left,base]{\color{textcolor}\rmfamily\fontsize{10.000000}{12.000000}\selectfont \(\displaystyle 0.002\)}%
\end{pgfscope}%
\begin{pgfscope}%
\pgfpathrectangle{\pgfqpoint{0.800000in}{0.528000in}}{\pgfqpoint{4.960000in}{3.696000in}}%
\pgfusepath{clip}%
\pgfsetbuttcap%
\pgfsetroundjoin%
\pgfsetlinewidth{0.803000pt}%
\definecolor{currentstroke}{rgb}{0.800000,0.800000,0.800000}%
\pgfsetstrokecolor{currentstroke}%
\pgfsetdash{{0.800000pt}{1.320000pt}}{0.000000pt}%
\pgfpathmoveto{\pgfqpoint{0.800000in}{4.224000in}}%
\pgfpathlineto{\pgfqpoint{5.760000in}{4.224000in}}%
\pgfusepath{stroke}%
\end{pgfscope}%
\begin{pgfscope}%
\pgfsetbuttcap%
\pgfsetroundjoin%
\definecolor{currentfill}{rgb}{0.000000,0.000000,0.000000}%
\pgfsetfillcolor{currentfill}%
\pgfsetlinewidth{0.803000pt}%
\definecolor{currentstroke}{rgb}{0.000000,0.000000,0.000000}%
\pgfsetstrokecolor{currentstroke}%
\pgfsetdash{}{0pt}%
\pgfsys@defobject{currentmarker}{\pgfqpoint{-0.048611in}{0.000000in}}{\pgfqpoint{0.000000in}{0.000000in}}{%
\pgfpathmoveto{\pgfqpoint{0.000000in}{0.000000in}}%
\pgfpathlineto{\pgfqpoint{-0.048611in}{0.000000in}}%
\pgfusepath{stroke,fill}%
}%
\begin{pgfscope}%
\pgfsys@transformshift{0.800000in}{4.224000in}%
\pgfsys@useobject{currentmarker}{}%
\end{pgfscope}%
\end{pgfscope}%
\begin{pgfscope}%
\definecolor{textcolor}{rgb}{0.000000,0.000000,0.000000}%
\pgfsetstrokecolor{textcolor}%
\pgfsetfillcolor{textcolor}%
\pgftext[x=0.386419in,y=4.175775in,left,base]{\color{textcolor}\rmfamily\fontsize{10.000000}{12.000000}\selectfont \(\displaystyle 0.003\)}%
\end{pgfscope}%
\begin{pgfscope}%
\definecolor{textcolor}{rgb}{0.000000,0.000000,0.000000}%
\pgfsetstrokecolor{textcolor}%
\pgfsetfillcolor{textcolor}%
\pgftext[x=0.222838in,y=2.376000in,,bottom]{\color{textcolor}\rmfamily\fontsize{10.000000}{12.000000}\selectfont y}%
\end{pgfscope}%
\begin{pgfscope}%
\pgfpathrectangle{\pgfqpoint{0.800000in}{0.528000in}}{\pgfqpoint{4.960000in}{3.696000in}}%
\pgfusepath{clip}%
\pgfsetrectcap%
\pgfsetroundjoin%
\pgfsetlinewidth{1.505625pt}%
\definecolor{currentstroke}{rgb}{0.000000,0.000000,0.000000}%
\pgfsetstrokecolor{currentstroke}%
\pgfsetdash{}{0pt}%
\pgfpathmoveto{\pgfqpoint{0.790000in}{2.376000in}}%
\pgfpathlineto{\pgfqpoint{5.770000in}{2.376000in}}%
\pgfpathlineto{\pgfqpoint{5.770000in}{2.376000in}}%
\pgfusepath{stroke}%
\end{pgfscope}%
\begin{pgfscope}%
\pgfpathrectangle{\pgfqpoint{0.800000in}{0.528000in}}{\pgfqpoint{4.960000in}{3.696000in}}%
\pgfusepath{clip}%
\pgfsetrectcap%
\pgfsetroundjoin%
\pgfsetlinewidth{1.505625pt}%
\definecolor{currentstroke}{rgb}{0.000000,0.000000,0.000000}%
\pgfsetstrokecolor{currentstroke}%
\pgfsetdash{}{0pt}%
\pgfpathmoveto{\pgfqpoint{0.942056in}{0.518000in}}%
\pgfpathlineto{\pgfqpoint{0.942056in}{4.234000in}}%
\pgfpathlineto{\pgfqpoint{0.942056in}{4.234000in}}%
\pgfusepath{stroke}%
\end{pgfscope}%
\begin{pgfscope}%
\pgfpathrectangle{\pgfqpoint{0.800000in}{0.528000in}}{\pgfqpoint{4.960000in}{3.696000in}}%
\pgfusepath{clip}%
\pgfsetrectcap%
\pgfsetroundjoin%
\pgfsetlinewidth{1.505625pt}%
\definecolor{currentstroke}{rgb}{0.121569,0.466667,0.705882}%
\pgfsetstrokecolor{currentstroke}%
\pgfsetdash{}{0pt}%
\pgfpathmoveto{\pgfqpoint{0.942056in}{2.376000in}}%
\pgfpathlineto{\pgfqpoint{0.957683in}{2.633700in}}%
\pgfpathlineto{\pgfqpoint{0.973311in}{2.854430in}}%
\pgfpathlineto{\pgfqpoint{0.986706in}{3.015775in}}%
\pgfpathlineto{\pgfqpoint{1.000101in}{3.152786in}}%
\pgfpathlineto{\pgfqpoint{1.013496in}{3.266761in}}%
\pgfpathlineto{\pgfqpoint{1.024659in}{3.345074in}}%
\pgfpathlineto{\pgfqpoint{1.035821in}{3.409030in}}%
\pgfpathlineto{\pgfqpoint{1.046984in}{3.459367in}}%
\pgfpathlineto{\pgfqpoint{1.055914in}{3.490325in}}%
\pgfpathlineto{\pgfqpoint{1.064844in}{3.513411in}}%
\pgfpathlineto{\pgfqpoint{1.071542in}{3.525781in}}%
\pgfpathlineto{\pgfqpoint{1.078239in}{3.534088in}}%
\pgfpathlineto{\pgfqpoint{1.084937in}{3.538485in}}%
\pgfpathlineto{\pgfqpoint{1.089402in}{3.539320in}}%
\pgfpathlineto{\pgfqpoint{1.093867in}{3.538531in}}%
\pgfpathlineto{\pgfqpoint{1.100564in}{3.534402in}}%
\pgfpathlineto{\pgfqpoint{1.107262in}{3.526871in}}%
\pgfpathlineto{\pgfqpoint{1.113959in}{3.516089in}}%
\pgfpathlineto{\pgfqpoint{1.122890in}{3.496913in}}%
\pgfpathlineto{\pgfqpoint{1.131820in}{3.472573in}}%
\pgfpathlineto{\pgfqpoint{1.142982in}{3.435413in}}%
\pgfpathlineto{\pgfqpoint{1.156377in}{3.381822in}}%
\pgfpathlineto{\pgfqpoint{1.172005in}{3.308322in}}%
\pgfpathlineto{\pgfqpoint{1.189865in}{3.211997in}}%
\pgfpathlineto{\pgfqpoint{1.209958in}{3.090963in}}%
\pgfpathlineto{\pgfqpoint{1.236748in}{2.914553in}}%
\pgfpathlineto{\pgfqpoint{1.281398in}{2.602083in}}%
\pgfpathlineto{\pgfqpoint{1.326049in}{2.295157in}}%
\pgfpathlineto{\pgfqpoint{1.352839in}{2.125925in}}%
\pgfpathlineto{\pgfqpoint{1.375164in}{1.998404in}}%
\pgfpathlineto{\pgfqpoint{1.395257in}{1.896611in}}%
\pgfpathlineto{\pgfqpoint{1.413117in}{1.817961in}}%
\pgfpathlineto{\pgfqpoint{1.428745in}{1.759189in}}%
\pgfpathlineto{\pgfqpoint{1.442140in}{1.716791in}}%
\pgfpathlineto{\pgfqpoint{1.455535in}{1.682133in}}%
\pgfpathlineto{\pgfqpoint{1.466698in}{1.659381in}}%
\pgfpathlineto{\pgfqpoint{1.475628in}{1.645299in}}%
\pgfpathlineto{\pgfqpoint{1.484558in}{1.634948in}}%
\pgfpathlineto{\pgfqpoint{1.493488in}{1.628383in}}%
\pgfpathlineto{\pgfqpoint{1.500185in}{1.625968in}}%
\pgfpathlineto{\pgfqpoint{1.506883in}{1.625720in}}%
\pgfpathlineto{\pgfqpoint{1.513580in}{1.627649in}}%
\pgfpathlineto{\pgfqpoint{1.520278in}{1.631763in}}%
\pgfpathlineto{\pgfqpoint{1.526976in}{1.638064in}}%
\pgfpathlineto{\pgfqpoint{1.535906in}{1.649868in}}%
\pgfpathlineto{\pgfqpoint{1.544836in}{1.665551in}}%
\pgfpathlineto{\pgfqpoint{1.555998in}{1.690575in}}%
\pgfpathlineto{\pgfqpoint{1.567161in}{1.721558in}}%
\pgfpathlineto{\pgfqpoint{1.580556in}{1.766465in}}%
\pgfpathlineto{\pgfqpoint{1.593951in}{1.819593in}}%
\pgfpathlineto{\pgfqpoint{1.609579in}{1.891584in}}%
\pgfpathlineto{\pgfqpoint{1.625206in}{1.973823in}}%
\pgfpathlineto{\pgfqpoint{1.643067in}{2.079493in}}%
\pgfpathlineto{\pgfqpoint{1.663159in}{2.211827in}}%
\pgfpathlineto{\pgfqpoint{1.687717in}{2.390052in}}%
\pgfpathlineto{\pgfqpoint{1.716740in}{2.618367in}}%
\pgfpathlineto{\pgfqpoint{1.817203in}{3.426171in}}%
\pgfpathlineto{\pgfqpoint{1.837296in}{3.561649in}}%
\pgfpathlineto{\pgfqpoint{1.852923in}{3.653682in}}%
\pgfpathlineto{\pgfqpoint{1.866318in}{3.721166in}}%
\pgfpathlineto{\pgfqpoint{1.877481in}{3.768081in}}%
\pgfpathlineto{\pgfqpoint{1.888644in}{3.805440in}}%
\pgfpathlineto{\pgfqpoint{1.897574in}{3.827742in}}%
\pgfpathlineto{\pgfqpoint{1.904271in}{3.839681in}}%
\pgfpathlineto{\pgfqpoint{1.910969in}{3.847250in}}%
\pgfpathlineto{\pgfqpoint{1.915434in}{3.849748in}}%
\pgfpathlineto{\pgfqpoint{1.919899in}{3.850121in}}%
\pgfpathlineto{\pgfqpoint{1.924364in}{3.848296in}}%
\pgfpathlineto{\pgfqpoint{1.928829in}{3.844196in}}%
\pgfpathlineto{\pgfqpoint{1.935527in}{3.833614in}}%
\pgfpathlineto{\pgfqpoint{1.942224in}{3.817481in}}%
\pgfpathlineto{\pgfqpoint{1.948922in}{3.795528in}}%
\pgfpathlineto{\pgfqpoint{1.957852in}{3.756738in}}%
\pgfpathlineto{\pgfqpoint{1.966782in}{3.706462in}}%
\pgfpathlineto{\pgfqpoint{1.975712in}{3.644029in}}%
\pgfpathlineto{\pgfqpoint{1.986874in}{3.547843in}}%
\pgfpathlineto{\pgfqpoint{1.998037in}{3.430209in}}%
\pgfpathlineto{\pgfqpoint{2.009200in}{3.289713in}}%
\pgfpathlineto{\pgfqpoint{2.022595in}{3.088894in}}%
\pgfpathlineto{\pgfqpoint{2.035990in}{2.850481in}}%
\pgfpathlineto{\pgfqpoint{2.049385in}{2.571817in}}%
\pgfpathlineto{\pgfqpoint{2.058315in}{2.389590in}}%
\pgfpathlineto{\pgfqpoint{2.073943in}{2.739417in}}%
\pgfpathlineto{\pgfqpoint{2.087338in}{2.994274in}}%
\pgfpathlineto{\pgfqpoint{2.100733in}{3.210440in}}%
\pgfpathlineto{\pgfqpoint{2.114128in}{3.390526in}}%
\pgfpathlineto{\pgfqpoint{2.125291in}{3.514869in}}%
\pgfpathlineto{\pgfqpoint{2.136453in}{3.617343in}}%
\pgfpathlineto{\pgfqpoint{2.147616in}{3.699337in}}%
\pgfpathlineto{\pgfqpoint{2.156546in}{3.751095in}}%
\pgfpathlineto{\pgfqpoint{2.165476in}{3.791285in}}%
\pgfpathlineto{\pgfqpoint{2.174406in}{3.820566in}}%
\pgfpathlineto{\pgfqpoint{2.181104in}{3.835756in}}%
\pgfpathlineto{\pgfqpoint{2.187801in}{3.845438in}}%
\pgfpathlineto{\pgfqpoint{2.192266in}{3.848962in}}%
\pgfpathlineto{\pgfqpoint{2.196731in}{3.850231in}}%
\pgfpathlineto{\pgfqpoint{2.201196in}{3.849320in}}%
\pgfpathlineto{\pgfqpoint{2.205661in}{3.846302in}}%
\pgfpathlineto{\pgfqpoint{2.212359in}{3.837988in}}%
\pgfpathlineto{\pgfqpoint{2.219056in}{3.825344in}}%
\pgfpathlineto{\pgfqpoint{2.227987in}{3.802164in}}%
\pgfpathlineto{\pgfqpoint{2.236917in}{3.772264in}}%
\pgfpathlineto{\pgfqpoint{2.248079in}{3.726258in}}%
\pgfpathlineto{\pgfqpoint{2.261474in}{3.659734in}}%
\pgfpathlineto{\pgfqpoint{2.277102in}{3.568647in}}%
\pgfpathlineto{\pgfqpoint{2.294962in}{3.449898in}}%
\pgfpathlineto{\pgfqpoint{2.317287in}{3.284726in}}%
\pgfpathlineto{\pgfqpoint{2.348543in}{3.033772in}}%
\pgfpathlineto{\pgfqpoint{2.424448in}{2.415547in}}%
\pgfpathlineto{\pgfqpoint{2.451238in}{2.219560in}}%
\pgfpathlineto{\pgfqpoint{2.473564in}{2.072543in}}%
\pgfpathlineto{\pgfqpoint{2.493656in}{1.955370in}}%
\pgfpathlineto{\pgfqpoint{2.511516in}{1.864665in}}%
\pgfpathlineto{\pgfqpoint{2.527144in}{1.796474in}}%
\pgfpathlineto{\pgfqpoint{2.540539in}{1.746741in}}%
\pgfpathlineto{\pgfqpoint{2.553934in}{1.705327in}}%
\pgfpathlineto{\pgfqpoint{2.565097in}{1.677313in}}%
\pgfpathlineto{\pgfqpoint{2.576260in}{1.655296in}}%
\pgfpathlineto{\pgfqpoint{2.585190in}{1.642035in}}%
\pgfpathlineto{\pgfqpoint{2.594120in}{1.632661in}}%
\pgfpathlineto{\pgfqpoint{2.600817in}{1.628183in}}%
\pgfpathlineto{\pgfqpoint{2.607515in}{1.625890in}}%
\pgfpathlineto{\pgfqpoint{2.614212in}{1.625776in}}%
\pgfpathlineto{\pgfqpoint{2.620910in}{1.627830in}}%
\pgfpathlineto{\pgfqpoint{2.627607in}{1.632040in}}%
\pgfpathlineto{\pgfqpoint{2.636538in}{1.640977in}}%
\pgfpathlineto{\pgfqpoint{2.645468in}{1.653666in}}%
\pgfpathlineto{\pgfqpoint{2.654398in}{1.670049in}}%
\pgfpathlineto{\pgfqpoint{2.665560in}{1.695613in}}%
\pgfpathlineto{\pgfqpoint{2.676723in}{1.726678in}}%
\pgfpathlineto{\pgfqpoint{2.690118in}{1.770957in}}%
\pgfpathlineto{\pgfqpoint{2.705746in}{1.831803in}}%
\pgfpathlineto{\pgfqpoint{2.721373in}{1.901910in}}%
\pgfpathlineto{\pgfqpoint{2.739233in}{1.992407in}}%
\pgfpathlineto{\pgfqpoint{2.759326in}{2.105930in}}%
\pgfpathlineto{\pgfqpoint{2.783884in}{2.258677in}}%
\pgfpathlineto{\pgfqpoint{2.815139in}{2.469172in}}%
\pgfpathlineto{\pgfqpoint{2.906672in}{3.097983in}}%
\pgfpathlineto{\pgfqpoint{2.928998in}{3.230949in}}%
\pgfpathlineto{\pgfqpoint{2.946858in}{3.324985in}}%
\pgfpathlineto{\pgfqpoint{2.962485in}{3.396083in}}%
\pgfpathlineto{\pgfqpoint{2.975880in}{3.447312in}}%
\pgfpathlineto{\pgfqpoint{2.987043in}{3.482284in}}%
\pgfpathlineto{\pgfqpoint{2.998206in}{3.509525in}}%
\pgfpathlineto{\pgfqpoint{3.007136in}{3.525296in}}%
\pgfpathlineto{\pgfqpoint{3.013833in}{3.533379in}}%
\pgfpathlineto{\pgfqpoint{3.020531in}{3.538086in}}%
\pgfpathlineto{\pgfqpoint{3.024996in}{3.539274in}}%
\pgfpathlineto{\pgfqpoint{3.029461in}{3.538849in}}%
\pgfpathlineto{\pgfqpoint{3.033926in}{3.536766in}}%
\pgfpathlineto{\pgfqpoint{3.040623in}{3.530433in}}%
\pgfpathlineto{\pgfqpoint{3.047321in}{3.520114in}}%
\pgfpathlineto{\pgfqpoint{3.054019in}{3.505652in}}%
\pgfpathlineto{\pgfqpoint{3.062949in}{3.479658in}}%
\pgfpathlineto{\pgfqpoint{3.071879in}{3.445652in}}%
\pgfpathlineto{\pgfqpoint{3.080809in}{3.403259in}}%
\pgfpathlineto{\pgfqpoint{3.091971in}{3.337901in}}%
\pgfpathlineto{\pgfqpoint{3.103134in}{3.258111in}}%
\pgfpathlineto{\pgfqpoint{3.114297in}{3.163146in}}%
\pgfpathlineto{\pgfqpoint{3.127692in}{3.028105in}}%
\pgfpathlineto{\pgfqpoint{3.141087in}{2.868837in}}%
\pgfpathlineto{\pgfqpoint{3.154482in}{2.684041in}}%
\pgfpathlineto{\pgfqpoint{3.170110in}{2.434438in}}%
\pgfpathlineto{\pgfqpoint{3.190202in}{2.101332in}}%
\pgfpathlineto{\pgfqpoint{3.205830in}{1.883163in}}%
\pgfpathlineto{\pgfqpoint{3.219225in}{1.723895in}}%
\pgfpathlineto{\pgfqpoint{3.232620in}{1.588854in}}%
\pgfpathlineto{\pgfqpoint{3.246015in}{1.476741in}}%
\pgfpathlineto{\pgfqpoint{3.257178in}{1.399897in}}%
\pgfpathlineto{\pgfqpoint{3.268340in}{1.337336in}}%
\pgfpathlineto{\pgfqpoint{3.279503in}{1.288320in}}%
\pgfpathlineto{\pgfqpoint{3.288433in}{1.258366in}}%
\pgfpathlineto{\pgfqpoint{3.297363in}{1.236239in}}%
\pgfpathlineto{\pgfqpoint{3.304061in}{1.224556in}}%
\pgfpathlineto{\pgfqpoint{3.310758in}{1.216912in}}%
\pgfpathlineto{\pgfqpoint{3.317456in}{1.213151in}}%
\pgfpathlineto{\pgfqpoint{3.321921in}{1.212726in}}%
\pgfpathlineto{\pgfqpoint{3.326386in}{1.213914in}}%
\pgfpathlineto{\pgfqpoint{3.333083in}{1.218621in}}%
\pgfpathlineto{\pgfqpoint{3.339781in}{1.226704in}}%
\pgfpathlineto{\pgfqpoint{3.346479in}{1.238014in}}%
\pgfpathlineto{\pgfqpoint{3.355409in}{1.257854in}}%
\pgfpathlineto{\pgfqpoint{3.364339in}{1.282815in}}%
\pgfpathlineto{\pgfqpoint{3.375501in}{1.320691in}}%
\pgfpathlineto{\pgfqpoint{3.388896in}{1.375054in}}%
\pgfpathlineto{\pgfqpoint{3.404524in}{1.449337in}}%
\pgfpathlineto{\pgfqpoint{3.422384in}{1.546405in}}%
\pgfpathlineto{\pgfqpoint{3.442477in}{1.668087in}}%
\pgfpathlineto{\pgfqpoint{3.469267in}{1.845067in}}%
\pgfpathlineto{\pgfqpoint{3.516150in}{2.173517in}}%
\pgfpathlineto{\pgfqpoint{3.558568in}{2.464184in}}%
\pgfpathlineto{\pgfqpoint{3.585358in}{2.632772in}}%
\pgfpathlineto{\pgfqpoint{3.607683in}{2.759593in}}%
\pgfpathlineto{\pgfqpoint{3.627776in}{2.860644in}}%
\pgfpathlineto{\pgfqpoint{3.645636in}{2.938557in}}%
\pgfpathlineto{\pgfqpoint{3.661264in}{2.996632in}}%
\pgfpathlineto{\pgfqpoint{3.674659in}{3.038397in}}%
\pgfpathlineto{\pgfqpoint{3.685821in}{3.067282in}}%
\pgfpathlineto{\pgfqpoint{3.696984in}{3.090600in}}%
\pgfpathlineto{\pgfqpoint{3.705914in}{3.105144in}}%
\pgfpathlineto{\pgfqpoint{3.714844in}{3.115964in}}%
\pgfpathlineto{\pgfqpoint{3.723774in}{3.123005in}}%
\pgfpathlineto{\pgfqpoint{3.730472in}{3.125780in}}%
\pgfpathlineto{\pgfqpoint{3.737169in}{3.126390in}}%
\pgfpathlineto{\pgfqpoint{3.743867in}{3.124824in}}%
\pgfpathlineto{\pgfqpoint{3.750565in}{3.121075in}}%
\pgfpathlineto{\pgfqpoint{3.757262in}{3.115138in}}%
\pgfpathlineto{\pgfqpoint{3.766192in}{3.103820in}}%
\pgfpathlineto{\pgfqpoint{3.775122in}{3.088621in}}%
\pgfpathlineto{\pgfqpoint{3.786285in}{3.064197in}}%
\pgfpathlineto{\pgfqpoint{3.797447in}{3.033806in}}%
\pgfpathlineto{\pgfqpoint{3.810843in}{2.989594in}}%
\pgfpathlineto{\pgfqpoint{3.824238in}{2.937142in}}%
\pgfpathlineto{\pgfqpoint{3.839865in}{2.865905in}}%
\pgfpathlineto{\pgfqpoint{3.855493in}{2.784377in}}%
\pgfpathlineto{\pgfqpoint{3.873353in}{2.679457in}}%
\pgfpathlineto{\pgfqpoint{3.893446in}{2.547868in}}%
\pgfpathlineto{\pgfqpoint{3.915771in}{2.387182in}}%
\pgfpathlineto{\pgfqpoint{3.944794in}{2.160714in}}%
\pgfpathlineto{\pgfqpoint{3.993909in}{1.755130in}}%
\pgfpathlineto{\pgfqpoint{4.031862in}{1.450089in}}%
\pgfpathlineto{\pgfqpoint{4.054187in}{1.286521in}}%
\pgfpathlineto{\pgfqpoint{4.072047in}{1.169543in}}%
\pgfpathlineto{\pgfqpoint{4.087675in}{1.080388in}}%
\pgfpathlineto{\pgfqpoint{4.101070in}{1.015818in}}%
\pgfpathlineto{\pgfqpoint{4.112233in}{0.971661in}}%
\pgfpathlineto{\pgfqpoint{4.121163in}{0.943390in}}%
\pgfpathlineto{\pgfqpoint{4.130093in}{0.921975in}}%
\pgfpathlineto{\pgfqpoint{4.136790in}{0.910747in}}%
\pgfpathlineto{\pgfqpoint{4.143488in}{0.903930in}}%
\pgfpathlineto{\pgfqpoint{4.147953in}{0.901957in}}%
\pgfpathlineto{\pgfqpoint{4.152418in}{0.902126in}}%
\pgfpathlineto{\pgfqpoint{4.156883in}{0.904513in}}%
\pgfpathlineto{\pgfqpoint{4.161348in}{0.909193in}}%
\pgfpathlineto{\pgfqpoint{4.168046in}{0.920682in}}%
\pgfpathlineto{\pgfqpoint{4.174743in}{0.937767in}}%
\pgfpathlineto{\pgfqpoint{4.181441in}{0.960715in}}%
\pgfpathlineto{\pgfqpoint{4.190371in}{1.000905in}}%
\pgfpathlineto{\pgfqpoint{4.199301in}{1.052663in}}%
\pgfpathlineto{\pgfqpoint{4.208231in}{1.116664in}}%
\pgfpathlineto{\pgfqpoint{4.219394in}{1.214932in}}%
\pgfpathlineto{\pgfqpoint{4.230556in}{1.334788in}}%
\pgfpathlineto{\pgfqpoint{4.241719in}{1.477649in}}%
\pgfpathlineto{\pgfqpoint{4.255114in}{1.681502in}}%
\pgfpathlineto{\pgfqpoint{4.268509in}{1.923167in}}%
\pgfpathlineto{\pgfqpoint{4.281904in}{2.205307in}}%
\pgfpathlineto{\pgfqpoint{4.288602in}{2.362410in}}%
\pgfpathlineto{\pgfqpoint{4.290834in}{2.335461in}}%
\pgfpathlineto{\pgfqpoint{4.306462in}{1.989802in}}%
\pgfpathlineto{\pgfqpoint{4.319857in}{1.738271in}}%
\pgfpathlineto{\pgfqpoint{4.333252in}{1.525210in}}%
\pgfpathlineto{\pgfqpoint{4.346647in}{1.348015in}}%
\pgfpathlineto{\pgfqpoint{4.357810in}{1.225923in}}%
\pgfpathlineto{\pgfqpoint{4.368972in}{1.125559in}}%
\pgfpathlineto{\pgfqpoint{4.380135in}{1.045538in}}%
\pgfpathlineto{\pgfqpoint{4.389065in}{0.995262in}}%
\pgfpathlineto{\pgfqpoint{4.397995in}{0.956472in}}%
\pgfpathlineto{\pgfqpoint{4.406925in}{0.928508in}}%
\pgfpathlineto{\pgfqpoint{4.413623in}{0.914255in}}%
\pgfpathlineto{\pgfqpoint{4.420320in}{0.905465in}}%
\pgfpathlineto{\pgfqpoint{4.424785in}{0.902512in}}%
\pgfpathlineto{\pgfqpoint{4.429250in}{0.901795in}}%
\pgfpathlineto{\pgfqpoint{4.433715in}{0.903240in}}%
\pgfpathlineto{\pgfqpoint{4.438180in}{0.906772in}}%
\pgfpathlineto{\pgfqpoint{4.444878in}{0.915825in}}%
\pgfpathlineto{\pgfqpoint{4.451576in}{0.929167in}}%
\pgfpathlineto{\pgfqpoint{4.460506in}{0.953218in}}%
\pgfpathlineto{\pgfqpoint{4.469436in}{0.983919in}}%
\pgfpathlineto{\pgfqpoint{4.480598in}{1.030834in}}%
\pgfpathlineto{\pgfqpoint{4.493993in}{1.098318in}}%
\pgfpathlineto{\pgfqpoint{4.509621in}{1.190351in}}%
\pgfpathlineto{\pgfqpoint{4.527481in}{1.309965in}}%
\pgfpathlineto{\pgfqpoint{4.549806in}{1.475913in}}%
\pgfpathlineto{\pgfqpoint{4.581062in}{1.727439in}}%
\pgfpathlineto{\pgfqpoint{4.654735in}{2.327898in}}%
\pgfpathlineto{\pgfqpoint{4.681525in}{2.524670in}}%
\pgfpathlineto{\pgfqpoint{4.703850in}{2.672507in}}%
\pgfpathlineto{\pgfqpoint{4.723943in}{2.790529in}}%
\pgfpathlineto{\pgfqpoint{4.741803in}{2.882057in}}%
\pgfpathlineto{\pgfqpoint{4.757431in}{2.951013in}}%
\pgfpathlineto{\pgfqpoint{4.770826in}{3.001429in}}%
\pgfpathlineto{\pgfqpoint{4.784221in}{3.043545in}}%
\pgfpathlineto{\pgfqpoint{4.795383in}{3.072154in}}%
\pgfpathlineto{\pgfqpoint{4.806546in}{3.094774in}}%
\pgfpathlineto{\pgfqpoint{4.815476in}{3.108520in}}%
\pgfpathlineto{\pgfqpoint{4.824406in}{3.118380in}}%
\pgfpathlineto{\pgfqpoint{4.831104in}{3.123223in}}%
\pgfpathlineto{\pgfqpoint{4.837801in}{3.125879in}}%
\pgfpathlineto{\pgfqpoint{4.844499in}{3.126356in}}%
\pgfpathlineto{\pgfqpoint{4.851196in}{3.124662in}}%
\pgfpathlineto{\pgfqpoint{4.857894in}{3.120811in}}%
\pgfpathlineto{\pgfqpoint{4.864592in}{3.114817in}}%
\pgfpathlineto{\pgfqpoint{4.873522in}{3.103528in}}%
\pgfpathlineto{\pgfqpoint{4.882452in}{3.088523in}}%
\pgfpathlineto{\pgfqpoint{4.893614in}{3.064642in}}%
\pgfpathlineto{\pgfqpoint{4.904777in}{3.035209in}}%
\pgfpathlineto{\pgfqpoint{4.918172in}{2.992811in}}%
\pgfpathlineto{\pgfqpoint{4.931567in}{2.943027in}}%
\pgfpathlineto{\pgfqpoint{4.947195in}{2.876146in}}%
\pgfpathlineto{\pgfqpoint{4.965055in}{2.788994in}}%
\pgfpathlineto{\pgfqpoint{4.985148in}{2.678744in}}%
\pgfpathlineto{\pgfqpoint{5.007473in}{2.543380in}}%
\pgfpathlineto{\pgfqpoint{5.034263in}{2.367164in}}%
\pgfpathlineto{\pgfqpoint{5.076681in}{2.071086in}}%
\pgfpathlineto{\pgfqpoint{5.123564in}{1.747514in}}%
\pgfpathlineto{\pgfqpoint{5.148122in}{1.592335in}}%
\pgfpathlineto{\pgfqpoint{5.168214in}{1.478386in}}%
\pgfpathlineto{\pgfqpoint{5.186074in}{1.390041in}}%
\pgfpathlineto{\pgfqpoint{5.199469in}{1.333409in}}%
\pgfpathlineto{\pgfqpoint{5.212865in}{1.286278in}}%
\pgfpathlineto{\pgfqpoint{5.224027in}{1.255087in}}%
\pgfpathlineto{\pgfqpoint{5.232957in}{1.235911in}}%
\pgfpathlineto{\pgfqpoint{5.241887in}{1.222250in}}%
\pgfpathlineto{\pgfqpoint{5.248585in}{1.215836in}}%
\pgfpathlineto{\pgfqpoint{5.255282in}{1.212874in}}%
\pgfpathlineto{\pgfqpoint{5.259747in}{1.212892in}}%
\pgfpathlineto{\pgfqpoint{5.264213in}{1.214556in}}%
\pgfpathlineto{\pgfqpoint{5.270910in}{1.220239in}}%
\pgfpathlineto{\pgfqpoint{5.277608in}{1.229883in}}%
\pgfpathlineto{\pgfqpoint{5.284305in}{1.243643in}}%
\pgfpathlineto{\pgfqpoint{5.293235in}{1.268662in}}%
\pgfpathlineto{\pgfqpoint{5.302165in}{1.301646in}}%
\pgfpathlineto{\pgfqpoint{5.311095in}{1.342970in}}%
\pgfpathlineto{\pgfqpoint{5.322258in}{1.406926in}}%
\pgfpathlineto{\pgfqpoint{5.333421in}{1.485239in}}%
\pgfpathlineto{\pgfqpoint{5.344583in}{1.578653in}}%
\pgfpathlineto{\pgfqpoint{5.357978in}{1.711735in}}%
\pgfpathlineto{\pgfqpoint{5.371373in}{1.868934in}}%
\pgfpathlineto{\pgfqpoint{5.384769in}{2.051553in}}%
\pgfpathlineto{\pgfqpoint{5.400396in}{2.298478in}}%
\pgfpathlineto{\pgfqpoint{5.404861in}{2.376000in}}%
\pgfpathlineto{\pgfqpoint{5.404861in}{2.376000in}}%
\pgfusepath{stroke}%
\end{pgfscope}%
\begin{pgfscope}%
\pgfsetrectcap%
\pgfsetmiterjoin%
\pgfsetlinewidth{0.803000pt}%
\definecolor{currentstroke}{rgb}{0.000000,0.000000,0.000000}%
\pgfsetstrokecolor{currentstroke}%
\pgfsetdash{}{0pt}%
\pgfpathmoveto{\pgfqpoint{0.800000in}{0.528000in}}%
\pgfpathlineto{\pgfqpoint{0.800000in}{4.224000in}}%
\pgfusepath{stroke}%
\end{pgfscope}%
\begin{pgfscope}%
\pgfsetrectcap%
\pgfsetmiterjoin%
\pgfsetlinewidth{0.803000pt}%
\definecolor{currentstroke}{rgb}{0.000000,0.000000,0.000000}%
\pgfsetstrokecolor{currentstroke}%
\pgfsetdash{}{0pt}%
\pgfpathmoveto{\pgfqpoint{5.760000in}{0.528000in}}%
\pgfpathlineto{\pgfqpoint{5.760000in}{4.224000in}}%
\pgfusepath{stroke}%
\end{pgfscope}%
\begin{pgfscope}%
\pgfsetrectcap%
\pgfsetmiterjoin%
\pgfsetlinewidth{0.803000pt}%
\definecolor{currentstroke}{rgb}{0.000000,0.000000,0.000000}%
\pgfsetstrokecolor{currentstroke}%
\pgfsetdash{}{0pt}%
\pgfpathmoveto{\pgfqpoint{0.800000in}{0.528000in}}%
\pgfpathlineto{\pgfqpoint{5.760000in}{0.528000in}}%
\pgfusepath{stroke}%
\end{pgfscope}%
\begin{pgfscope}%
\pgfsetrectcap%
\pgfsetmiterjoin%
\pgfsetlinewidth{0.803000pt}%
\definecolor{currentstroke}{rgb}{0.000000,0.000000,0.000000}%
\pgfsetstrokecolor{currentstroke}%
\pgfsetdash{}{0pt}%
\pgfpathmoveto{\pgfqpoint{0.800000in}{4.224000in}}%
\pgfpathlineto{\pgfqpoint{5.760000in}{4.224000in}}%
\pgfusepath{stroke}%
\end{pgfscope}%
\begin{pgfscope}%
\pgfsetbuttcap%
\pgfsetmiterjoin%
\definecolor{currentfill}{rgb}{1.000000,1.000000,1.000000}%
\pgfsetfillcolor{currentfill}%
\pgfsetfillopacity{0.800000}%
\pgfsetlinewidth{1.003750pt}%
\definecolor{currentstroke}{rgb}{0.800000,0.800000,0.800000}%
\pgfsetstrokecolor{currentstroke}%
\pgfsetstrokeopacity{0.800000}%
\pgfsetdash{}{0pt}%
\pgfpathmoveto{\pgfqpoint{4.351773in}{3.904556in}}%
\pgfpathlineto{\pgfqpoint{5.662778in}{3.904556in}}%
\pgfpathquadraticcurveto{\pgfqpoint{5.690556in}{3.904556in}}{\pgfqpoint{5.690556in}{3.932333in}}%
\pgfpathlineto{\pgfqpoint{5.690556in}{4.126778in}}%
\pgfpathquadraticcurveto{\pgfqpoint{5.690556in}{4.154556in}}{\pgfqpoint{5.662778in}{4.154556in}}%
\pgfpathlineto{\pgfqpoint{4.351773in}{4.154556in}}%
\pgfpathquadraticcurveto{\pgfqpoint{4.323995in}{4.154556in}}{\pgfqpoint{4.323995in}{4.126778in}}%
\pgfpathlineto{\pgfqpoint{4.323995in}{3.932333in}}%
\pgfpathquadraticcurveto{\pgfqpoint{4.323995in}{3.904556in}}{\pgfqpoint{4.351773in}{3.904556in}}%
\pgfpathclose%
\pgfusepath{stroke,fill}%
\end{pgfscope}%
\begin{pgfscope}%
\pgfsetrectcap%
\pgfsetroundjoin%
\pgfsetlinewidth{1.505625pt}%
\definecolor{currentstroke}{rgb}{0.121569,0.466667,0.705882}%
\pgfsetstrokecolor{currentstroke}%
\pgfsetdash{}{0pt}%
\pgfpathmoveto{\pgfqpoint{4.379551in}{4.043444in}}%
\pgfpathlineto{\pgfqpoint{4.657328in}{4.043444in}}%
\pgfusepath{stroke}%
\end{pgfscope}%
\begin{pgfscope}%
\definecolor{textcolor}{rgb}{0.000000,0.000000,0.000000}%
\pgfsetstrokecolor{textcolor}%
\pgfsetfillcolor{textcolor}%
\pgftext[x=4.768440in,y=3.994833in,left,base]{\color{textcolor}\rmfamily\fontsize{10.000000}{12.000000}\selectfont \(\displaystyle P_3(x)-\sin(x)\)}%
\end{pgfscope}%
\end{pgfpicture}%
\makeatother%
\endgroup%
}
                \caption{Expanded Lagrange Error}
                \label{fig:LagrangeError}
            \end{center}
        \end{figure}

        \lstinputlisting[caption=Expanding Lagrange Approximation, style=appendix]{2_3.py}
        
    \end{enumerate}

\end{document}