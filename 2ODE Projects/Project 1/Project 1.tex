\documentclass[12pt]{article}
\usepackage[margin=1.2in]{geometry}
\usepackage{graphicx}
\usepackage{amsmath}
\usepackage{physics}
\usepackage{tabto}
\usepackage{float}
\usepackage{amssymb}
\usepackage{pgfplots}
\usepackage{verbatim}
\usepackage{tcolorbox}
\usepackage{listings}
\usepackage{xcolor}
\usepackage{siunitx}
\usepackage[all]{nowidow}
\definecolor{stringcolor}{HTML}{C792EA}
\definecolor{codeblue}{HTML}{2162DB}
\definecolor{commentcolor}{HTML}{4A6E46}
\lstdefinestyle{appendix}{
    basicstyle=\ttfamily\footnotesize,
    commentstyle=\color{commentcolor},
    keywordstyle=\color{codeblue},
    stringstyle=\color{stringcolor},
    showstringspaces=false,
    numbers=left,
    upquote=true,
    captionpos=t,
    abovecaptionskip=12pt,
    belowcaptionskip=12pt,
    language=Python,
    breaklines=true,
    frame=single,
}
\renewcommand{\lstlistingname}{Appendix}
\pgfplotsset{compat=1.16}

\title{Car on a Washboard Road Surface}
\date{\textbf{25 April 2020}}
\author{}

\begin{document}

    \maketitle
    \begin{center}
    \textbf{\large{MAM2046W 2OD}}\\
    \textbf{\large{EmplID: 1669971}}\\
    \end{center}

    \section*{Abstract}
    In this project we aim to analyse a system described by a second order differential equation 
    both analytically and numerically. This equation describes a "car on a washboard surface". As 
    the car drives along it oscillates up and down on a damped spring, but the washboard surface 
    makes it much more complicated than that. We will first determine the differential equation, 
    then solve it analytically using the method of Undetermined Coefficients in order to find the 
    velocity at which the car oscillates with greatest amplitude. Then we will simulate the equation 
    numerically to verify our result.


    \begin{enumerate}
        \item \textbf{Modelling} \newline
        In order to model this system, we must consider all of the forces acting on the car. 
        It's relatively safe to assume that the car is in an inertial reference frame, therefore 
        we know that $\vec{F}_{net}=m\vec{a}$. On the other hand, we know that the only forces 
        acting on the car are the force of gravity $\vec{F}_G = m\vec{g}$ and the force of the 
        "spring", which can be modelled as $\vec{F}_S = k\Delta \vec{y}$ where $\Delta \vec{y}$ 
        is the distance from the equilibrium position to the current position of the mass.
        Now if we consider the system when it's at rest, in other words when the spring is at a 
        relative equilibrium position, there is a force being applied on the car by the spring in 
        order to perfectly balance the force of gravity, in which case we can effectively ignore 
        the force of gravity and choose the new position from which to measure $\Delta \vec{y}$, 
        leaving us with

        \begin{equation*}
            \vec{F}_{net} = k\Delta \vec{y}
        \end{equation*}

        Now we must consider the dashpot, which applies a force on the mass in proportion to 
        the velocity of the mass in the form $\vec{F}_D = c\vec{v}$. Adding this into our 
        equation for $\vec{F_{net}}$, we find 

        \begin{equation*}
            \vec{F}_{net} = k\Delta \vec{y} + c\vec{v}
        \end{equation*}

        With a usual mass on a spring system, this is as far as it goes as the only thing that 
        moves is the mass, but in this case both the mass and the connection point of the 
        "spring" are moving and they're not necessarily moving in sync with each other. In 
        order to account for this we need to modify the $\Delta\vec{y}$ and $\vec{v}$ terms 
        as they will not be changing in a simple manner. For the $\Delta\vec{y}$ term, this 
        isn't too hard to do. We just need to consider the effect that different values of 
        $\Delta\vec{y}$ will have on $\vec{F}_S$. From this we find 

        \begin{equation*}
            \vec{F}_S = k(y(t) - Y(t))
        \end{equation*}

        where $Y(t)$ is the upward displacement of the car and $y(t)$ is the upward displacement 
        of the connection point of the "spring", given by 

        \begin{equation*}
            \begin{split}
                y(t) &= a\sin\frac{2\pi x}{\lambda} \\
                &= a\sin\frac{2\pi v t}{\lambda}
            \end{split}
        \end{equation*}

        as $x = vt$. Now to consider the $\vec{F}_D$ term. Through some careful consideration 
        of the force depending on which way the mass and the connection point are moving, we 
        can find that 

        \begin{equation*}
            \vec{F}_D = c(\dot{y}(t) - \dot{Y}(t))
        \end{equation*}

        Putting this all together, we can see that 

        \begin{equation*}
            \begin{split}
                \vec{F}_{net} &= k(y - Y) + c(\dot{y} - \dot{Y}) \\
                m\vec{a} &= k(y - Y) + c(\dot{y} - \dot{Y}) \\
                m\ddot{Y} &= ky - kY + c\dot{y} - c\dot{Y}
            \end{split}
        \end{equation*}

        and finally 

        \begin{equation}
            m\ddot{Y} + c\dot{Y} + kY = ky + c\dot{y}
            \label{eqn:differential equation}
        \end{equation}

        which is the final form of the equation of motion we are looking for.

        \item \textbf{Analysis} \newline
        In order to analyse this differential equation we first need to recognise 
        that it's a non-homogeneous second order linear ODE. This means that we can 
        find the general solution to the homogeneous equation and a particular solution 
        to the non-homogeneous equation and thus we will have found the general 
        solution to the non-homogeneous equation. 
        \newline
        Firstly, the general solution to the homogeneous equation 

        \begin{equation}
            m\ddot{Y}+c\dot{Y}+kY=0
            \label{eqn:homogeneous equation}
        \end{equation}

        \noindent
        can be found by firstly finding the roots of the characteristic equation 
        
        \begin{equation}
            mr^2+cr+k=0
            \label{eqn:homogeneous characteristic}
        \end{equation}
        
        which are $r_{1,2}=-\frac{5}{4}\pm i\sqrt{\frac{1375}{16}}$. Using the general formula for 
        the solution to a second order linear homogeneous equation
        
        \begin{equation}
            \begin{gathered}
                r_{1,2} = a+bi \\
                y_g = e^{at}(C1\cos(bt)+C2\sin(bt))
            \end{gathered}
            \label{eqn:gen formula homo soln}
        \end{equation}

        we find 

        \begin{equation}
            Y_{hg} = e^{-\frac{5t}{4}}(C1\cos(pt)+C2\sin(pt))
            \label{eqn:homogeneous gen sol 1}
        \end{equation}

        where $p = \sqrt{\frac{1375}{16}}$. $C1$ and $C2$ are determined from initial conditions, which 
        for this problem are relatively insignificant, so we've chosen $Y(0)=0.2$ and $\dot{Y}(0)=1$. By 
        differentiating Equation (\ref{eqn:homogeneous gen sol 1}) and substituting those initial conditions, 
        we find that $C1=0.2$ and $C2=\frac{\sqrt{55}}{55}$, resulting in
        
        \begin{equation}
            Y_{hg} = e^{-\frac{5t}{4}}(0.2\cos(pt)+\frac{\sqrt{55}}{55}\sin(pt))
            \label{eqn:homogeneous gen sol 2}
        \end{equation}

        Now, in order to find a general solution to the non-homogeneous equation we must find any 
        particular solution to it and sum it with Equation (\ref{eqn:homogeneous gen sol 2}). To do 
        this, we can use the method of Undetermined Coefficients, starting with a guessed solution. 
        In our case we guess that $Y = A\sin(\frac{2\pi v}{\lambda}t)$ and substitute it into Equation 
        (\ref{eqn:differential equation}) resulting in 

        \begin{equation*}
            \begin{gathered}
                -mz^2[A\sin(zt)+B\cos(zt)]+cz[A\cos(zt)-B\sin(zt)] \\
                +k[A\sin(zt)+B\cos(zt)]=caz\cos(zt)+ka\sin(zt)
            \end{gathered}
        \end{equation*}

        where $z=\frac{\pi v}{5}$. Simplifying this we find 

        \begin{equation*}
            \begin{gathered}
                \sin(zt)[A(k-mz^2)-B(cz)]+\cos(zt)[B(k-mz^2)+A(cz)] \\
                =\sin(zt)(ka)+\cos(zt)(caz)
            \end{gathered}
        \end{equation*}

        From the coefficients of the $\cos$'s and $\sin$'s we can extract a system of equations, which we 
        can solve to find $A$ and $B$:

        \begin{equation*}
            \begin{split}
                ka &= A(k-mz^2)-B(cz) \\
                caz &= A(cz)+B(k-mz^2) \\
                \implies A &= \frac{k^2a-mkaz^2+ac^2z^2}{(k-mz^2)^2+c^2z^2}; \\
                B &= \frac{camz^3}{(k-mz^2)^2+c^2z^2}
            \end{split}
        \end{equation*}

        So we have found a particular solution to the non-homogeneous equation, and thus the general 
        solution to the differential equation, which is 

        \begin{equation}
            \begin{gathered}
                Y_g = e^{-\frac{5t}{4}}(0.2\cos(pt)+\frac{\sqrt{55}}{55}\sin(pt)) \\
                +\frac{k^2a-mkaz^2+ac^2z^2}{(k-mz^2)^2+c^2z^2}\sin(zt)+\frac{camz^3}{(k-mz^2)^2+c^2z^2}\cos(zt)
            \end{gathered}
            \label{eqn:non-homogeneous gen sol}
        \end{equation}
        
        where, as before, $a=0.05, c=\num{2e3}, k=\num{7e4}, m=800, p=\sqrt{\frac{1375}{16}}$, and 
        $z=\frac{\pi v}{5}$.
        \newline
        Now, in order to find out the velocity that maximises the amplitude of oscillations we need to 
        substitute some things in and rearrange the equation to give us a term we can maximise. 
        The two terms from $y_{hg}$ don't have any impact past the transient stage as the 
        $e^{\frac{-5t}{4}}$ term goes to 0 as $t$ increases. So we have to maximise the second and 
        third terms. Firstly, we can define two things:

        \begin{equation*}
            \sin(\alpha) = \frac{k^2-mka^2+a^2z^2}{\sqrt{(k-mz^2)^2+c^2z^2}};\hspace{5pt} \cos(\alpha) = \frac{cmz^3}{\sqrt{(k-mz^2)^2+c^2z^2}}
        \end{equation*}
        
        Now we can simplify Equation (\ref{eqn:non-homogeneous gen sol}), ignoring $y_{hg}$ to find

        \begin{equation}
            \begin{split}
                Y_g &= \frac{a}{\sqrt{(k-mz^2)^2+c^2z^2}}\left[\cos(zt)\cos(\alpha)+\sin(zt)\sin(\alpha)\right] \\
                &= \frac{a}{\sqrt{(k-mz^2)^2+c^2z^2}}\cdot\cos(zt-\alpha)
            \end{split}
            \label{eqn:max amplitude soln}
        \end{equation}

        This gives us a nice term to maximise, namely by minimising the denominator. The $(cz)^2$ 
        term can never be zero, apart from the trivial case where $v=0$ as $z=\frac{\pi v}{5}$, so 
        we minimise $(k-mz^2)$:

        \begin{equation*}
            \begin{split}
                k-mz^2=&0 \\
                \implies z=&\sqrt{\frac{k}{m}} \\
                \implies v=&\sqrt{\frac{k}{m}}\cdot\frac{5}{\pi} \\
                v\approx&14.88758171
            \end{split}
        \end{equation*}

        And that is the value of $v$ for which the oscillations have a maximum amplitude. To find 
        that amplitude we can plot Equation (\ref{eqn:max amplitude soln}) using Python and extract the 
        maximum amplitude. Below in Figure (\ref{fig:analytical}) we have plotted the amplitude of 
        oscillations for various values of $v$. This has a shape very similar to a resonance curve 
        and peaks at $v \approx 14.6573$ with a value of $A \approx 0.1952$.

        \begin{figure}[H]
            \begin{center}
                \scalebox{.7}{%% Creator: Matplotlib, PGF backend
%%
%% To include the figure in your LaTeX document, write
%%   \input{<filename>.pgf}
%%
%% Make sure the required packages are loaded in your preamble
%%   \usepackage{pgf}
%%
%% Figures using additional raster images can only be included by \input if
%% they are in the same directory as the main LaTeX file. For loading figures
%% from other directories you can use the `import` package
%%   \usepackage{import}
%% and then include the figures with
%%   \import{<path to file>}{<filename>.pgf}
%%
%% Matplotlib used the following preamble
%%
\begingroup%
\makeatletter%
\begin{pgfpicture}%
\pgfpathrectangle{\pgfpointorigin}{\pgfqpoint{6.400000in}{4.800000in}}%
\pgfusepath{use as bounding box, clip}%
\begin{pgfscope}%
\pgfsetbuttcap%
\pgfsetmiterjoin%
\definecolor{currentfill}{rgb}{1.000000,1.000000,1.000000}%
\pgfsetfillcolor{currentfill}%
\pgfsetlinewidth{0.000000pt}%
\definecolor{currentstroke}{rgb}{1.000000,1.000000,1.000000}%
\pgfsetstrokecolor{currentstroke}%
\pgfsetdash{}{0pt}%
\pgfpathmoveto{\pgfqpoint{0.000000in}{0.000000in}}%
\pgfpathlineto{\pgfqpoint{6.400000in}{0.000000in}}%
\pgfpathlineto{\pgfqpoint{6.400000in}{4.800000in}}%
\pgfpathlineto{\pgfqpoint{0.000000in}{4.800000in}}%
\pgfpathclose%
\pgfusepath{fill}%
\end{pgfscope}%
\begin{pgfscope}%
\pgfsetbuttcap%
\pgfsetmiterjoin%
\definecolor{currentfill}{rgb}{1.000000,1.000000,1.000000}%
\pgfsetfillcolor{currentfill}%
\pgfsetlinewidth{0.000000pt}%
\definecolor{currentstroke}{rgb}{0.000000,0.000000,0.000000}%
\pgfsetstrokecolor{currentstroke}%
\pgfsetstrokeopacity{0.000000}%
\pgfsetdash{}{0pt}%
\pgfpathmoveto{\pgfqpoint{0.800000in}{0.528000in}}%
\pgfpathlineto{\pgfqpoint{5.760000in}{0.528000in}}%
\pgfpathlineto{\pgfqpoint{5.760000in}{4.224000in}}%
\pgfpathlineto{\pgfqpoint{0.800000in}{4.224000in}}%
\pgfpathclose%
\pgfusepath{fill}%
\end{pgfscope}%
\begin{pgfscope}%
\pgfsetbuttcap%
\pgfsetroundjoin%
\definecolor{currentfill}{rgb}{0.000000,0.000000,0.000000}%
\pgfsetfillcolor{currentfill}%
\pgfsetlinewidth{0.803000pt}%
\definecolor{currentstroke}{rgb}{0.000000,0.000000,0.000000}%
\pgfsetstrokecolor{currentstroke}%
\pgfsetdash{}{0pt}%
\pgfsys@defobject{currentmarker}{\pgfqpoint{0.000000in}{-0.048611in}}{\pgfqpoint{0.000000in}{0.000000in}}{%
\pgfpathmoveto{\pgfqpoint{0.000000in}{0.000000in}}%
\pgfpathlineto{\pgfqpoint{0.000000in}{-0.048611in}}%
\pgfusepath{stroke,fill}%
}%
\begin{pgfscope}%
\pgfsys@transformshift{0.869969in}{0.528000in}%
\pgfsys@useobject{currentmarker}{}%
\end{pgfscope}%
\end{pgfscope}%
\begin{pgfscope}%
\definecolor{textcolor}{rgb}{0.000000,0.000000,0.000000}%
\pgfsetstrokecolor{textcolor}%
\pgfsetfillcolor{textcolor}%
\pgftext[x=0.869969in,y=0.430778in,,top]{\color{textcolor}\rmfamily\fontsize{10.000000}{12.000000}\selectfont \(\displaystyle 0\)}%
\end{pgfscope}%
\begin{pgfscope}%
\pgfsetbuttcap%
\pgfsetroundjoin%
\definecolor{currentfill}{rgb}{0.000000,0.000000,0.000000}%
\pgfsetfillcolor{currentfill}%
\pgfsetlinewidth{0.803000pt}%
\definecolor{currentstroke}{rgb}{0.000000,0.000000,0.000000}%
\pgfsetstrokecolor{currentstroke}%
\pgfsetdash{}{0pt}%
\pgfsys@defobject{currentmarker}{\pgfqpoint{0.000000in}{-0.048611in}}{\pgfqpoint{0.000000in}{0.000000in}}{%
\pgfpathmoveto{\pgfqpoint{0.000000in}{0.000000in}}%
\pgfpathlineto{\pgfqpoint{0.000000in}{-0.048611in}}%
\pgfusepath{stroke,fill}%
}%
\begin{pgfscope}%
\pgfsys@transformshift{1.647398in}{0.528000in}%
\pgfsys@useobject{currentmarker}{}%
\end{pgfscope}%
\end{pgfscope}%
\begin{pgfscope}%
\definecolor{textcolor}{rgb}{0.000000,0.000000,0.000000}%
\pgfsetstrokecolor{textcolor}%
\pgfsetfillcolor{textcolor}%
\pgftext[x=1.647398in,y=0.430778in,,top]{\color{textcolor}\rmfamily\fontsize{10.000000}{12.000000}\selectfont \(\displaystyle 5\)}%
\end{pgfscope}%
\begin{pgfscope}%
\pgfsetbuttcap%
\pgfsetroundjoin%
\definecolor{currentfill}{rgb}{0.000000,0.000000,0.000000}%
\pgfsetfillcolor{currentfill}%
\pgfsetlinewidth{0.803000pt}%
\definecolor{currentstroke}{rgb}{0.000000,0.000000,0.000000}%
\pgfsetstrokecolor{currentstroke}%
\pgfsetdash{}{0pt}%
\pgfsys@defobject{currentmarker}{\pgfqpoint{0.000000in}{-0.048611in}}{\pgfqpoint{0.000000in}{0.000000in}}{%
\pgfpathmoveto{\pgfqpoint{0.000000in}{0.000000in}}%
\pgfpathlineto{\pgfqpoint{0.000000in}{-0.048611in}}%
\pgfusepath{stroke,fill}%
}%
\begin{pgfscope}%
\pgfsys@transformshift{2.424828in}{0.528000in}%
\pgfsys@useobject{currentmarker}{}%
\end{pgfscope}%
\end{pgfscope}%
\begin{pgfscope}%
\definecolor{textcolor}{rgb}{0.000000,0.000000,0.000000}%
\pgfsetstrokecolor{textcolor}%
\pgfsetfillcolor{textcolor}%
\pgftext[x=2.424828in,y=0.430778in,,top]{\color{textcolor}\rmfamily\fontsize{10.000000}{12.000000}\selectfont \(\displaystyle 10\)}%
\end{pgfscope}%
\begin{pgfscope}%
\pgfsetbuttcap%
\pgfsetroundjoin%
\definecolor{currentfill}{rgb}{0.000000,0.000000,0.000000}%
\pgfsetfillcolor{currentfill}%
\pgfsetlinewidth{0.803000pt}%
\definecolor{currentstroke}{rgb}{0.000000,0.000000,0.000000}%
\pgfsetstrokecolor{currentstroke}%
\pgfsetdash{}{0pt}%
\pgfsys@defobject{currentmarker}{\pgfqpoint{0.000000in}{-0.048611in}}{\pgfqpoint{0.000000in}{0.000000in}}{%
\pgfpathmoveto{\pgfqpoint{0.000000in}{0.000000in}}%
\pgfpathlineto{\pgfqpoint{0.000000in}{-0.048611in}}%
\pgfusepath{stroke,fill}%
}%
\begin{pgfscope}%
\pgfsys@transformshift{3.202257in}{0.528000in}%
\pgfsys@useobject{currentmarker}{}%
\end{pgfscope}%
\end{pgfscope}%
\begin{pgfscope}%
\definecolor{textcolor}{rgb}{0.000000,0.000000,0.000000}%
\pgfsetstrokecolor{textcolor}%
\pgfsetfillcolor{textcolor}%
\pgftext[x=3.202257in,y=0.430778in,,top]{\color{textcolor}\rmfamily\fontsize{10.000000}{12.000000}\selectfont \(\displaystyle 15\)}%
\end{pgfscope}%
\begin{pgfscope}%
\pgfsetbuttcap%
\pgfsetroundjoin%
\definecolor{currentfill}{rgb}{0.000000,0.000000,0.000000}%
\pgfsetfillcolor{currentfill}%
\pgfsetlinewidth{0.803000pt}%
\definecolor{currentstroke}{rgb}{0.000000,0.000000,0.000000}%
\pgfsetstrokecolor{currentstroke}%
\pgfsetdash{}{0pt}%
\pgfsys@defobject{currentmarker}{\pgfqpoint{0.000000in}{-0.048611in}}{\pgfqpoint{0.000000in}{0.000000in}}{%
\pgfpathmoveto{\pgfqpoint{0.000000in}{0.000000in}}%
\pgfpathlineto{\pgfqpoint{0.000000in}{-0.048611in}}%
\pgfusepath{stroke,fill}%
}%
\begin{pgfscope}%
\pgfsys@transformshift{3.979687in}{0.528000in}%
\pgfsys@useobject{currentmarker}{}%
\end{pgfscope}%
\end{pgfscope}%
\begin{pgfscope}%
\definecolor{textcolor}{rgb}{0.000000,0.000000,0.000000}%
\pgfsetstrokecolor{textcolor}%
\pgfsetfillcolor{textcolor}%
\pgftext[x=3.979687in,y=0.430778in,,top]{\color{textcolor}\rmfamily\fontsize{10.000000}{12.000000}\selectfont \(\displaystyle 20\)}%
\end{pgfscope}%
\begin{pgfscope}%
\pgfsetbuttcap%
\pgfsetroundjoin%
\definecolor{currentfill}{rgb}{0.000000,0.000000,0.000000}%
\pgfsetfillcolor{currentfill}%
\pgfsetlinewidth{0.803000pt}%
\definecolor{currentstroke}{rgb}{0.000000,0.000000,0.000000}%
\pgfsetstrokecolor{currentstroke}%
\pgfsetdash{}{0pt}%
\pgfsys@defobject{currentmarker}{\pgfqpoint{0.000000in}{-0.048611in}}{\pgfqpoint{0.000000in}{0.000000in}}{%
\pgfpathmoveto{\pgfqpoint{0.000000in}{0.000000in}}%
\pgfpathlineto{\pgfqpoint{0.000000in}{-0.048611in}}%
\pgfusepath{stroke,fill}%
}%
\begin{pgfscope}%
\pgfsys@transformshift{4.757116in}{0.528000in}%
\pgfsys@useobject{currentmarker}{}%
\end{pgfscope}%
\end{pgfscope}%
\begin{pgfscope}%
\definecolor{textcolor}{rgb}{0.000000,0.000000,0.000000}%
\pgfsetstrokecolor{textcolor}%
\pgfsetfillcolor{textcolor}%
\pgftext[x=4.757116in,y=0.430778in,,top]{\color{textcolor}\rmfamily\fontsize{10.000000}{12.000000}\selectfont \(\displaystyle 25\)}%
\end{pgfscope}%
\begin{pgfscope}%
\pgfsetbuttcap%
\pgfsetroundjoin%
\definecolor{currentfill}{rgb}{0.000000,0.000000,0.000000}%
\pgfsetfillcolor{currentfill}%
\pgfsetlinewidth{0.803000pt}%
\definecolor{currentstroke}{rgb}{0.000000,0.000000,0.000000}%
\pgfsetstrokecolor{currentstroke}%
\pgfsetdash{}{0pt}%
\pgfsys@defobject{currentmarker}{\pgfqpoint{0.000000in}{-0.048611in}}{\pgfqpoint{0.000000in}{0.000000in}}{%
\pgfpathmoveto{\pgfqpoint{0.000000in}{0.000000in}}%
\pgfpathlineto{\pgfqpoint{0.000000in}{-0.048611in}}%
\pgfusepath{stroke,fill}%
}%
\begin{pgfscope}%
\pgfsys@transformshift{5.534545in}{0.528000in}%
\pgfsys@useobject{currentmarker}{}%
\end{pgfscope}%
\end{pgfscope}%
\begin{pgfscope}%
\definecolor{textcolor}{rgb}{0.000000,0.000000,0.000000}%
\pgfsetstrokecolor{textcolor}%
\pgfsetfillcolor{textcolor}%
\pgftext[x=5.534545in,y=0.430778in,,top]{\color{textcolor}\rmfamily\fontsize{10.000000}{12.000000}\selectfont \(\displaystyle 30\)}%
\end{pgfscope}%
\begin{pgfscope}%
\definecolor{textcolor}{rgb}{0.000000,0.000000,0.000000}%
\pgfsetstrokecolor{textcolor}%
\pgfsetfillcolor{textcolor}%
\pgftext[x=3.280000in,y=0.251766in,,top]{\color{textcolor}\rmfamily\fontsize{10.000000}{12.000000}\selectfont Velocity (\(\displaystyle \frac{m}{s}\))}%
\end{pgfscope}%
\begin{pgfscope}%
\pgfsetbuttcap%
\pgfsetroundjoin%
\definecolor{currentfill}{rgb}{0.000000,0.000000,0.000000}%
\pgfsetfillcolor{currentfill}%
\pgfsetlinewidth{0.803000pt}%
\definecolor{currentstroke}{rgb}{0.000000,0.000000,0.000000}%
\pgfsetstrokecolor{currentstroke}%
\pgfsetdash{}{0pt}%
\pgfsys@defobject{currentmarker}{\pgfqpoint{-0.048611in}{0.000000in}}{\pgfqpoint{0.000000in}{0.000000in}}{%
\pgfpathmoveto{\pgfqpoint{0.000000in}{0.000000in}}%
\pgfpathlineto{\pgfqpoint{-0.048611in}{0.000000in}}%
\pgfusepath{stroke,fill}%
}%
\begin{pgfscope}%
\pgfsys@transformshift{0.800000in}{0.823755in}%
\pgfsys@useobject{currentmarker}{}%
\end{pgfscope}%
\end{pgfscope}%
\begin{pgfscope}%
\definecolor{textcolor}{rgb}{0.000000,0.000000,0.000000}%
\pgfsetstrokecolor{textcolor}%
\pgfsetfillcolor{textcolor}%
\pgftext[x=0.386419in,y=0.775529in,left,base]{\color{textcolor}\rmfamily\fontsize{10.000000}{12.000000}\selectfont \(\displaystyle 0.025\)}%
\end{pgfscope}%
\begin{pgfscope}%
\pgfsetbuttcap%
\pgfsetroundjoin%
\definecolor{currentfill}{rgb}{0.000000,0.000000,0.000000}%
\pgfsetfillcolor{currentfill}%
\pgfsetlinewidth{0.803000pt}%
\definecolor{currentstroke}{rgb}{0.000000,0.000000,0.000000}%
\pgfsetstrokecolor{currentstroke}%
\pgfsetdash{}{0pt}%
\pgfsys@defobject{currentmarker}{\pgfqpoint{-0.048611in}{0.000000in}}{\pgfqpoint{0.000000in}{0.000000in}}{%
\pgfpathmoveto{\pgfqpoint{0.000000in}{0.000000in}}%
\pgfpathlineto{\pgfqpoint{-0.048611in}{0.000000in}}%
\pgfusepath{stroke,fill}%
}%
\begin{pgfscope}%
\pgfsys@transformshift{0.800000in}{1.298621in}%
\pgfsys@useobject{currentmarker}{}%
\end{pgfscope}%
\end{pgfscope}%
\begin{pgfscope}%
\definecolor{textcolor}{rgb}{0.000000,0.000000,0.000000}%
\pgfsetstrokecolor{textcolor}%
\pgfsetfillcolor{textcolor}%
\pgftext[x=0.386419in,y=1.250395in,left,base]{\color{textcolor}\rmfamily\fontsize{10.000000}{12.000000}\selectfont \(\displaystyle 0.050\)}%
\end{pgfscope}%
\begin{pgfscope}%
\pgfsetbuttcap%
\pgfsetroundjoin%
\definecolor{currentfill}{rgb}{0.000000,0.000000,0.000000}%
\pgfsetfillcolor{currentfill}%
\pgfsetlinewidth{0.803000pt}%
\definecolor{currentstroke}{rgb}{0.000000,0.000000,0.000000}%
\pgfsetstrokecolor{currentstroke}%
\pgfsetdash{}{0pt}%
\pgfsys@defobject{currentmarker}{\pgfqpoint{-0.048611in}{0.000000in}}{\pgfqpoint{0.000000in}{0.000000in}}{%
\pgfpathmoveto{\pgfqpoint{0.000000in}{0.000000in}}%
\pgfpathlineto{\pgfqpoint{-0.048611in}{0.000000in}}%
\pgfusepath{stroke,fill}%
}%
\begin{pgfscope}%
\pgfsys@transformshift{0.800000in}{1.773487in}%
\pgfsys@useobject{currentmarker}{}%
\end{pgfscope}%
\end{pgfscope}%
\begin{pgfscope}%
\definecolor{textcolor}{rgb}{0.000000,0.000000,0.000000}%
\pgfsetstrokecolor{textcolor}%
\pgfsetfillcolor{textcolor}%
\pgftext[x=0.386419in,y=1.725262in,left,base]{\color{textcolor}\rmfamily\fontsize{10.000000}{12.000000}\selectfont \(\displaystyle 0.075\)}%
\end{pgfscope}%
\begin{pgfscope}%
\pgfsetbuttcap%
\pgfsetroundjoin%
\definecolor{currentfill}{rgb}{0.000000,0.000000,0.000000}%
\pgfsetfillcolor{currentfill}%
\pgfsetlinewidth{0.803000pt}%
\definecolor{currentstroke}{rgb}{0.000000,0.000000,0.000000}%
\pgfsetstrokecolor{currentstroke}%
\pgfsetdash{}{0pt}%
\pgfsys@defobject{currentmarker}{\pgfqpoint{-0.048611in}{0.000000in}}{\pgfqpoint{0.000000in}{0.000000in}}{%
\pgfpathmoveto{\pgfqpoint{0.000000in}{0.000000in}}%
\pgfpathlineto{\pgfqpoint{-0.048611in}{0.000000in}}%
\pgfusepath{stroke,fill}%
}%
\begin{pgfscope}%
\pgfsys@transformshift{0.800000in}{2.248353in}%
\pgfsys@useobject{currentmarker}{}%
\end{pgfscope}%
\end{pgfscope}%
\begin{pgfscope}%
\definecolor{textcolor}{rgb}{0.000000,0.000000,0.000000}%
\pgfsetstrokecolor{textcolor}%
\pgfsetfillcolor{textcolor}%
\pgftext[x=0.386419in,y=2.200128in,left,base]{\color{textcolor}\rmfamily\fontsize{10.000000}{12.000000}\selectfont \(\displaystyle 0.100\)}%
\end{pgfscope}%
\begin{pgfscope}%
\pgfsetbuttcap%
\pgfsetroundjoin%
\definecolor{currentfill}{rgb}{0.000000,0.000000,0.000000}%
\pgfsetfillcolor{currentfill}%
\pgfsetlinewidth{0.803000pt}%
\definecolor{currentstroke}{rgb}{0.000000,0.000000,0.000000}%
\pgfsetstrokecolor{currentstroke}%
\pgfsetdash{}{0pt}%
\pgfsys@defobject{currentmarker}{\pgfqpoint{-0.048611in}{0.000000in}}{\pgfqpoint{0.000000in}{0.000000in}}{%
\pgfpathmoveto{\pgfqpoint{0.000000in}{0.000000in}}%
\pgfpathlineto{\pgfqpoint{-0.048611in}{0.000000in}}%
\pgfusepath{stroke,fill}%
}%
\begin{pgfscope}%
\pgfsys@transformshift{0.800000in}{2.723219in}%
\pgfsys@useobject{currentmarker}{}%
\end{pgfscope}%
\end{pgfscope}%
\begin{pgfscope}%
\definecolor{textcolor}{rgb}{0.000000,0.000000,0.000000}%
\pgfsetstrokecolor{textcolor}%
\pgfsetfillcolor{textcolor}%
\pgftext[x=0.386419in,y=2.674994in,left,base]{\color{textcolor}\rmfamily\fontsize{10.000000}{12.000000}\selectfont \(\displaystyle 0.125\)}%
\end{pgfscope}%
\begin{pgfscope}%
\pgfsetbuttcap%
\pgfsetroundjoin%
\definecolor{currentfill}{rgb}{0.000000,0.000000,0.000000}%
\pgfsetfillcolor{currentfill}%
\pgfsetlinewidth{0.803000pt}%
\definecolor{currentstroke}{rgb}{0.000000,0.000000,0.000000}%
\pgfsetstrokecolor{currentstroke}%
\pgfsetdash{}{0pt}%
\pgfsys@defobject{currentmarker}{\pgfqpoint{-0.048611in}{0.000000in}}{\pgfqpoint{0.000000in}{0.000000in}}{%
\pgfpathmoveto{\pgfqpoint{0.000000in}{0.000000in}}%
\pgfpathlineto{\pgfqpoint{-0.048611in}{0.000000in}}%
\pgfusepath{stroke,fill}%
}%
\begin{pgfscope}%
\pgfsys@transformshift{0.800000in}{3.198085in}%
\pgfsys@useobject{currentmarker}{}%
\end{pgfscope}%
\end{pgfscope}%
\begin{pgfscope}%
\definecolor{textcolor}{rgb}{0.000000,0.000000,0.000000}%
\pgfsetstrokecolor{textcolor}%
\pgfsetfillcolor{textcolor}%
\pgftext[x=0.386419in,y=3.149860in,left,base]{\color{textcolor}\rmfamily\fontsize{10.000000}{12.000000}\selectfont \(\displaystyle 0.150\)}%
\end{pgfscope}%
\begin{pgfscope}%
\pgfsetbuttcap%
\pgfsetroundjoin%
\definecolor{currentfill}{rgb}{0.000000,0.000000,0.000000}%
\pgfsetfillcolor{currentfill}%
\pgfsetlinewidth{0.803000pt}%
\definecolor{currentstroke}{rgb}{0.000000,0.000000,0.000000}%
\pgfsetstrokecolor{currentstroke}%
\pgfsetdash{}{0pt}%
\pgfsys@defobject{currentmarker}{\pgfqpoint{-0.048611in}{0.000000in}}{\pgfqpoint{0.000000in}{0.000000in}}{%
\pgfpathmoveto{\pgfqpoint{0.000000in}{0.000000in}}%
\pgfpathlineto{\pgfqpoint{-0.048611in}{0.000000in}}%
\pgfusepath{stroke,fill}%
}%
\begin{pgfscope}%
\pgfsys@transformshift{0.800000in}{3.672951in}%
\pgfsys@useobject{currentmarker}{}%
\end{pgfscope}%
\end{pgfscope}%
\begin{pgfscope}%
\definecolor{textcolor}{rgb}{0.000000,0.000000,0.000000}%
\pgfsetstrokecolor{textcolor}%
\pgfsetfillcolor{textcolor}%
\pgftext[x=0.386419in,y=3.624726in,left,base]{\color{textcolor}\rmfamily\fontsize{10.000000}{12.000000}\selectfont \(\displaystyle 0.175\)}%
\end{pgfscope}%
\begin{pgfscope}%
\pgfsetbuttcap%
\pgfsetroundjoin%
\definecolor{currentfill}{rgb}{0.000000,0.000000,0.000000}%
\pgfsetfillcolor{currentfill}%
\pgfsetlinewidth{0.803000pt}%
\definecolor{currentstroke}{rgb}{0.000000,0.000000,0.000000}%
\pgfsetstrokecolor{currentstroke}%
\pgfsetdash{}{0pt}%
\pgfsys@defobject{currentmarker}{\pgfqpoint{-0.048611in}{0.000000in}}{\pgfqpoint{0.000000in}{0.000000in}}{%
\pgfpathmoveto{\pgfqpoint{0.000000in}{0.000000in}}%
\pgfpathlineto{\pgfqpoint{-0.048611in}{0.000000in}}%
\pgfusepath{stroke,fill}%
}%
\begin{pgfscope}%
\pgfsys@transformshift{0.800000in}{4.147817in}%
\pgfsys@useobject{currentmarker}{}%
\end{pgfscope}%
\end{pgfscope}%
\begin{pgfscope}%
\definecolor{textcolor}{rgb}{0.000000,0.000000,0.000000}%
\pgfsetstrokecolor{textcolor}%
\pgfsetfillcolor{textcolor}%
\pgftext[x=0.386419in,y=4.099592in,left,base]{\color{textcolor}\rmfamily\fontsize{10.000000}{12.000000}\selectfont \(\displaystyle 0.200\)}%
\end{pgfscope}%
\begin{pgfscope}%
\definecolor{textcolor}{rgb}{0.000000,0.000000,0.000000}%
\pgfsetstrokecolor{textcolor}%
\pgfsetfillcolor{textcolor}%
\pgftext[x=0.330863in,y=2.376000in,,bottom,rotate=90.000000]{\color{textcolor}\rmfamily\fontsize{10.000000}{12.000000}\selectfont Amplitude (\(\displaystyle m\))}%
\end{pgfscope}%
\begin{pgfscope}%
\pgfpathrectangle{\pgfqpoint{0.800000in}{0.528000in}}{\pgfqpoint{4.960000in}{3.696000in}}%
\pgfusepath{clip}%
\pgfsetrectcap%
\pgfsetroundjoin%
\pgfsetlinewidth{1.505625pt}%
\definecolor{currentstroke}{rgb}{0.121569,0.466667,0.705882}%
\pgfsetstrokecolor{currentstroke}%
\pgfsetdash{}{0pt}%
\pgfpathmoveto{\pgfqpoint{1.025455in}{1.302919in}}%
\pgfpathlineto{\pgfqpoint{1.106781in}{1.308656in}}%
\pgfpathlineto{\pgfqpoint{1.179071in}{1.315831in}}%
\pgfpathlineto{\pgfqpoint{1.260397in}{1.326371in}}%
\pgfpathlineto{\pgfqpoint{1.269433in}{1.327669in}}%
\pgfpathlineto{\pgfqpoint{1.278470in}{1.325272in}}%
\pgfpathlineto{\pgfqpoint{1.287506in}{1.330482in}}%
\pgfpathlineto{\pgfqpoint{1.314615in}{1.334905in}}%
\pgfpathlineto{\pgfqpoint{1.323651in}{1.333157in}}%
\pgfpathlineto{\pgfqpoint{1.332687in}{1.338030in}}%
\pgfpathlineto{\pgfqpoint{1.404977in}{1.351993in}}%
\pgfpathlineto{\pgfqpoint{1.477267in}{1.368413in}}%
\pgfpathlineto{\pgfqpoint{1.513412in}{1.377639in}}%
\pgfpathlineto{\pgfqpoint{1.576666in}{1.395483in}}%
\pgfpathlineto{\pgfqpoint{1.639920in}{1.415661in}}%
\pgfpathlineto{\pgfqpoint{1.703174in}{1.438386in}}%
\pgfpathlineto{\pgfqpoint{1.757391in}{1.460075in}}%
\pgfpathlineto{\pgfqpoint{1.802572in}{1.479892in}}%
\pgfpathlineto{\pgfqpoint{1.856790in}{1.505900in}}%
\pgfpathlineto{\pgfqpoint{1.911007in}{1.534597in}}%
\pgfpathlineto{\pgfqpoint{1.938116in}{1.550038in}}%
\pgfpathlineto{\pgfqpoint{1.983298in}{1.577533in}}%
\pgfpathlineto{\pgfqpoint{2.037515in}{1.610977in}}%
\pgfpathlineto{\pgfqpoint{2.046551in}{1.620014in}}%
\pgfpathlineto{\pgfqpoint{2.055588in}{1.625755in}}%
\pgfpathlineto{\pgfqpoint{2.082696in}{1.646715in}}%
\pgfpathlineto{\pgfqpoint{2.127878in}{1.682710in}}%
\pgfpathlineto{\pgfqpoint{2.136914in}{1.690284in}}%
\pgfpathlineto{\pgfqpoint{2.145950in}{1.694434in}}%
\pgfpathlineto{\pgfqpoint{2.154986in}{1.705842in}}%
\pgfpathlineto{\pgfqpoint{2.191131in}{1.738179in}}%
\pgfpathlineto{\pgfqpoint{2.263421in}{1.811655in}}%
\pgfpathlineto{\pgfqpoint{2.299566in}{1.852384in}}%
\pgfpathlineto{\pgfqpoint{2.335711in}{1.896257in}}%
\pgfpathlineto{\pgfqpoint{2.371856in}{1.943587in}}%
\pgfpathlineto{\pgfqpoint{2.408001in}{1.994722in}}%
\pgfpathlineto{\pgfqpoint{2.444146in}{2.050029in}}%
\pgfpathlineto{\pgfqpoint{2.480291in}{2.109986in}}%
\pgfpathlineto{\pgfqpoint{2.516437in}{2.175010in}}%
\pgfpathlineto{\pgfqpoint{2.534509in}{2.209595in}}%
\pgfpathlineto{\pgfqpoint{2.561618in}{2.264247in}}%
\pgfpathlineto{\pgfqpoint{2.597763in}{2.342688in}}%
\pgfpathlineto{\pgfqpoint{2.633908in}{2.428048in}}%
\pgfpathlineto{\pgfqpoint{2.670053in}{2.520948in}}%
\pgfpathlineto{\pgfqpoint{2.706198in}{2.622062in}}%
\pgfpathlineto{\pgfqpoint{2.742343in}{2.731831in}}%
\pgfpathlineto{\pgfqpoint{2.778488in}{2.850743in}}%
\pgfpathlineto{\pgfqpoint{2.814633in}{2.978771in}}%
\pgfpathlineto{\pgfqpoint{2.850778in}{3.115488in}}%
\pgfpathlineto{\pgfqpoint{2.895959in}{3.296953in}}%
\pgfpathlineto{\pgfqpoint{3.004394in}{3.740730in}}%
\pgfpathlineto{\pgfqpoint{3.031503in}{3.838634in}}%
\pgfpathlineto{\pgfqpoint{3.049576in}{3.896938in}}%
\pgfpathlineto{\pgfqpoint{3.067648in}{3.948045in}}%
\pgfpathlineto{\pgfqpoint{3.085721in}{3.990670in}}%
\pgfpathlineto{\pgfqpoint{3.094757in}{4.008381in}}%
\pgfpathlineto{\pgfqpoint{3.103793in}{4.023487in}}%
\pgfpathlineto{\pgfqpoint{3.112829in}{4.035834in}}%
\pgfpathlineto{\pgfqpoint{3.121866in}{4.045384in}}%
\pgfpathlineto{\pgfqpoint{3.130902in}{4.051946in}}%
\pgfpathlineto{\pgfqpoint{3.139938in}{4.055534in}}%
\pgfpathlineto{\pgfqpoint{3.148974in}{4.056000in}}%
\pgfpathlineto{\pgfqpoint{3.158011in}{4.053322in}}%
\pgfpathlineto{\pgfqpoint{3.167047in}{4.047119in}}%
\pgfpathlineto{\pgfqpoint{3.176083in}{4.038651in}}%
\pgfpathlineto{\pgfqpoint{3.185119in}{4.026671in}}%
\pgfpathlineto{\pgfqpoint{3.194156in}{4.011659in}}%
\pgfpathlineto{\pgfqpoint{3.203192in}{3.938767in}}%
\pgfpathlineto{\pgfqpoint{3.212228in}{3.972891in}}%
\pgfpathlineto{\pgfqpoint{3.230301in}{3.923241in}}%
\pgfpathlineto{\pgfqpoint{3.248373in}{3.863880in}}%
\pgfpathlineto{\pgfqpoint{3.266446in}{3.796133in}}%
\pgfpathlineto{\pgfqpoint{3.293554in}{3.681805in}}%
\pgfpathlineto{\pgfqpoint{3.320663in}{3.556745in}}%
\pgfpathlineto{\pgfqpoint{3.365844in}{3.336216in}}%
\pgfpathlineto{\pgfqpoint{3.429098in}{3.027487in}}%
\pgfpathlineto{\pgfqpoint{3.465243in}{2.860408in}}%
\pgfpathlineto{\pgfqpoint{3.501388in}{2.703419in}}%
\pgfpathlineto{\pgfqpoint{3.537533in}{2.557538in}}%
\pgfpathlineto{\pgfqpoint{3.564642in}{2.455549in}}%
\pgfpathlineto{\pgfqpoint{3.600787in}{2.329222in}}%
\pgfpathlineto{\pgfqpoint{3.627896in}{2.241408in}}%
\pgfpathlineto{\pgfqpoint{3.664041in}{2.130370in}}%
\pgfpathlineto{\pgfqpoint{3.682113in}{2.082239in}}%
\pgfpathlineto{\pgfqpoint{3.718258in}{1.987226in}}%
\pgfpathlineto{\pgfqpoint{3.754403in}{1.900150in}}%
\pgfpathlineto{\pgfqpoint{3.790548in}{1.820244in}}%
\pgfpathlineto{\pgfqpoint{3.826693in}{1.746815in}}%
\pgfpathlineto{\pgfqpoint{3.862838in}{1.679221in}}%
\pgfpathlineto{\pgfqpoint{3.898983in}{1.616857in}}%
\pgfpathlineto{\pgfqpoint{3.935128in}{1.559244in}}%
\pgfpathlineto{\pgfqpoint{3.971273in}{1.505887in}}%
\pgfpathlineto{\pgfqpoint{4.007418in}{1.456384in}}%
\pgfpathlineto{\pgfqpoint{4.043563in}{1.410367in}}%
\pgfpathlineto{\pgfqpoint{4.079709in}{1.367514in}}%
\pgfpathlineto{\pgfqpoint{4.115854in}{1.327038in}}%
\pgfpathlineto{\pgfqpoint{4.197180in}{1.246773in}}%
\pgfpathlineto{\pgfqpoint{4.242361in}{1.206713in}}%
\pgfpathlineto{\pgfqpoint{4.287542in}{1.169634in}}%
\pgfpathlineto{\pgfqpoint{4.332724in}{1.135236in}}%
\pgfpathlineto{\pgfqpoint{4.377905in}{1.103250in}}%
\pgfpathlineto{\pgfqpoint{4.432122in}{1.067739in}}%
\pgfpathlineto{\pgfqpoint{4.486340in}{1.035020in}}%
\pgfpathlineto{\pgfqpoint{4.540557in}{1.004768in}}%
\pgfpathlineto{\pgfqpoint{4.603811in}{0.972349in}}%
\pgfpathlineto{\pgfqpoint{4.667065in}{0.942588in}}%
\pgfpathlineto{\pgfqpoint{4.730319in}{0.915206in}}%
\pgfpathlineto{\pgfqpoint{4.802609in}{0.886497in}}%
\pgfpathlineto{\pgfqpoint{4.874899in}{0.860227in}}%
\pgfpathlineto{\pgfqpoint{4.947189in}{0.836110in}}%
\pgfpathlineto{\pgfqpoint{4.956225in}{0.810035in}}%
\pgfpathlineto{\pgfqpoint{4.965261in}{0.830389in}}%
\pgfpathlineto{\pgfqpoint{5.055624in}{0.803236in}}%
\pgfpathlineto{\pgfqpoint{5.263458in}{0.750543in}}%
\pgfpathlineto{\pgfqpoint{5.353820in}{0.730767in}}%
\pgfpathlineto{\pgfqpoint{5.480328in}{0.705837in}}%
\pgfpathlineto{\pgfqpoint{5.534545in}{0.696000in}}%
\pgfpathlineto{\pgfqpoint{5.534545in}{0.696000in}}%
\pgfusepath{stroke}%
\end{pgfscope}%
\begin{pgfscope}%
\pgfsetrectcap%
\pgfsetmiterjoin%
\pgfsetlinewidth{0.803000pt}%
\definecolor{currentstroke}{rgb}{0.000000,0.000000,0.000000}%
\pgfsetstrokecolor{currentstroke}%
\pgfsetdash{}{0pt}%
\pgfpathmoveto{\pgfqpoint{0.800000in}{0.528000in}}%
\pgfpathlineto{\pgfqpoint{0.800000in}{4.224000in}}%
\pgfusepath{stroke}%
\end{pgfscope}%
\begin{pgfscope}%
\pgfsetrectcap%
\pgfsetmiterjoin%
\pgfsetlinewidth{0.803000pt}%
\definecolor{currentstroke}{rgb}{0.000000,0.000000,0.000000}%
\pgfsetstrokecolor{currentstroke}%
\pgfsetdash{}{0pt}%
\pgfpathmoveto{\pgfqpoint{5.760000in}{0.528000in}}%
\pgfpathlineto{\pgfqpoint{5.760000in}{4.224000in}}%
\pgfusepath{stroke}%
\end{pgfscope}%
\begin{pgfscope}%
\pgfsetrectcap%
\pgfsetmiterjoin%
\pgfsetlinewidth{0.803000pt}%
\definecolor{currentstroke}{rgb}{0.000000,0.000000,0.000000}%
\pgfsetstrokecolor{currentstroke}%
\pgfsetdash{}{0pt}%
\pgfpathmoveto{\pgfqpoint{0.800000in}{0.528000in}}%
\pgfpathlineto{\pgfqpoint{5.760000in}{0.528000in}}%
\pgfusepath{stroke}%
\end{pgfscope}%
\begin{pgfscope}%
\pgfsetrectcap%
\pgfsetmiterjoin%
\pgfsetlinewidth{0.803000pt}%
\definecolor{currentstroke}{rgb}{0.000000,0.000000,0.000000}%
\pgfsetstrokecolor{currentstroke}%
\pgfsetdash{}{0pt}%
\pgfpathmoveto{\pgfqpoint{0.800000in}{4.224000in}}%
\pgfpathlineto{\pgfqpoint{5.760000in}{4.224000in}}%
\pgfusepath{stroke}%
\end{pgfscope}%
\end{pgfpicture}%
\makeatother%
\endgroup%
}
                \caption{Analytical Plot}
                \label{fig:analytical}
            \end{center}
        \end{figure}

        \item \textbf{Numerical Simulation} \newline
        To verify the result above we can simulate the original differential equation in Equation 
        (\ref{eqn:differential equation}) using \texttt{scipy.integrate.odeint}. Following the 
        documentation and extracting the amplitude for each value of $v \in [1, 30]$ with $dv \approx 0.5$ 
        we get the plot in Figure (\ref{fig:numerical}), with its peak at $v \approx 14.6573$ with 
        $A \approx 0.1952$.

        \begin{figure}[H]
            \begin{center}
               \scalebox{.7}{%% Creator: Matplotlib, PGF backend
%%
%% To include the figure in your LaTeX document, write
%%   \input{<filename>.pgf}
%%
%% Make sure the required packages are loaded in your preamble
%%   \usepackage{pgf}
%%
%% Figures using additional raster images can only be included by \input if
%% they are in the same directory as the main LaTeX file. For loading figures
%% from other directories you can use the `import` package
%%   \usepackage{import}
%% and then include the figures with
%%   \import{<path to file>}{<filename>.pgf}
%%
%% Matplotlib used the following preamble
%%
\begingroup%
\makeatletter%
\begin{pgfpicture}%
\pgfpathrectangle{\pgfpointorigin}{\pgfqpoint{6.400000in}{4.800000in}}%
\pgfusepath{use as bounding box, clip}%
\begin{pgfscope}%
\pgfsetbuttcap%
\pgfsetmiterjoin%
\definecolor{currentfill}{rgb}{1.000000,1.000000,1.000000}%
\pgfsetfillcolor{currentfill}%
\pgfsetlinewidth{0.000000pt}%
\definecolor{currentstroke}{rgb}{1.000000,1.000000,1.000000}%
\pgfsetstrokecolor{currentstroke}%
\pgfsetdash{}{0pt}%
\pgfpathmoveto{\pgfqpoint{0.000000in}{0.000000in}}%
\pgfpathlineto{\pgfqpoint{6.400000in}{0.000000in}}%
\pgfpathlineto{\pgfqpoint{6.400000in}{4.800000in}}%
\pgfpathlineto{\pgfqpoint{0.000000in}{4.800000in}}%
\pgfpathclose%
\pgfusepath{fill}%
\end{pgfscope}%
\begin{pgfscope}%
\pgfsetbuttcap%
\pgfsetmiterjoin%
\definecolor{currentfill}{rgb}{1.000000,1.000000,1.000000}%
\pgfsetfillcolor{currentfill}%
\pgfsetlinewidth{0.000000pt}%
\definecolor{currentstroke}{rgb}{0.000000,0.000000,0.000000}%
\pgfsetstrokecolor{currentstroke}%
\pgfsetstrokeopacity{0.000000}%
\pgfsetdash{}{0pt}%
\pgfpathmoveto{\pgfqpoint{0.800000in}{0.528000in}}%
\pgfpathlineto{\pgfqpoint{5.760000in}{0.528000in}}%
\pgfpathlineto{\pgfqpoint{5.760000in}{4.224000in}}%
\pgfpathlineto{\pgfqpoint{0.800000in}{4.224000in}}%
\pgfpathclose%
\pgfusepath{fill}%
\end{pgfscope}%
\begin{pgfscope}%
\pgfsetbuttcap%
\pgfsetroundjoin%
\definecolor{currentfill}{rgb}{0.000000,0.000000,0.000000}%
\pgfsetfillcolor{currentfill}%
\pgfsetlinewidth{0.803000pt}%
\definecolor{currentstroke}{rgb}{0.000000,0.000000,0.000000}%
\pgfsetstrokecolor{currentstroke}%
\pgfsetdash{}{0pt}%
\pgfsys@defobject{currentmarker}{\pgfqpoint{0.000000in}{-0.048611in}}{\pgfqpoint{0.000000in}{0.000000in}}{%
\pgfpathmoveto{\pgfqpoint{0.000000in}{0.000000in}}%
\pgfpathlineto{\pgfqpoint{0.000000in}{-0.048611in}}%
\pgfusepath{stroke,fill}%
}%
\begin{pgfscope}%
\pgfsys@transformshift{0.979908in}{0.528000in}%
\pgfsys@useobject{currentmarker}{}%
\end{pgfscope}%
\end{pgfscope}%
\begin{pgfscope}%
\definecolor{textcolor}{rgb}{0.000000,0.000000,0.000000}%
\pgfsetstrokecolor{textcolor}%
\pgfsetfillcolor{textcolor}%
\pgftext[x=0.979908in,y=0.430778in,,top]{\color{textcolor}\rmfamily\fontsize{10.000000}{12.000000}\selectfont \(\displaystyle 0\)}%
\end{pgfscope}%
\begin{pgfscope}%
\pgfsetbuttcap%
\pgfsetroundjoin%
\definecolor{currentfill}{rgb}{0.000000,0.000000,0.000000}%
\pgfsetfillcolor{currentfill}%
\pgfsetlinewidth{0.803000pt}%
\definecolor{currentstroke}{rgb}{0.000000,0.000000,0.000000}%
\pgfsetstrokecolor{currentstroke}%
\pgfsetdash{}{0pt}%
\pgfsys@defobject{currentmarker}{\pgfqpoint{0.000000in}{-0.048611in}}{\pgfqpoint{0.000000in}{0.000000in}}{%
\pgfpathmoveto{\pgfqpoint{0.000000in}{0.000000in}}%
\pgfpathlineto{\pgfqpoint{0.000000in}{-0.048611in}}%
\pgfusepath{stroke,fill}%
}%
\begin{pgfscope}%
\pgfsys@transformshift{1.890836in}{0.528000in}%
\pgfsys@useobject{currentmarker}{}%
\end{pgfscope}%
\end{pgfscope}%
\begin{pgfscope}%
\definecolor{textcolor}{rgb}{0.000000,0.000000,0.000000}%
\pgfsetstrokecolor{textcolor}%
\pgfsetfillcolor{textcolor}%
\pgftext[x=1.890836in,y=0.430778in,,top]{\color{textcolor}\rmfamily\fontsize{10.000000}{12.000000}\selectfont \(\displaystyle 20\)}%
\end{pgfscope}%
\begin{pgfscope}%
\pgfsetbuttcap%
\pgfsetroundjoin%
\definecolor{currentfill}{rgb}{0.000000,0.000000,0.000000}%
\pgfsetfillcolor{currentfill}%
\pgfsetlinewidth{0.803000pt}%
\definecolor{currentstroke}{rgb}{0.000000,0.000000,0.000000}%
\pgfsetstrokecolor{currentstroke}%
\pgfsetdash{}{0pt}%
\pgfsys@defobject{currentmarker}{\pgfqpoint{0.000000in}{-0.048611in}}{\pgfqpoint{0.000000in}{0.000000in}}{%
\pgfpathmoveto{\pgfqpoint{0.000000in}{0.000000in}}%
\pgfpathlineto{\pgfqpoint{0.000000in}{-0.048611in}}%
\pgfusepath{stroke,fill}%
}%
\begin{pgfscope}%
\pgfsys@transformshift{2.801763in}{0.528000in}%
\pgfsys@useobject{currentmarker}{}%
\end{pgfscope}%
\end{pgfscope}%
\begin{pgfscope}%
\definecolor{textcolor}{rgb}{0.000000,0.000000,0.000000}%
\pgfsetstrokecolor{textcolor}%
\pgfsetfillcolor{textcolor}%
\pgftext[x=2.801763in,y=0.430778in,,top]{\color{textcolor}\rmfamily\fontsize{10.000000}{12.000000}\selectfont \(\displaystyle 40\)}%
\end{pgfscope}%
\begin{pgfscope}%
\pgfsetbuttcap%
\pgfsetroundjoin%
\definecolor{currentfill}{rgb}{0.000000,0.000000,0.000000}%
\pgfsetfillcolor{currentfill}%
\pgfsetlinewidth{0.803000pt}%
\definecolor{currentstroke}{rgb}{0.000000,0.000000,0.000000}%
\pgfsetstrokecolor{currentstroke}%
\pgfsetdash{}{0pt}%
\pgfsys@defobject{currentmarker}{\pgfqpoint{0.000000in}{-0.048611in}}{\pgfqpoint{0.000000in}{0.000000in}}{%
\pgfpathmoveto{\pgfqpoint{0.000000in}{0.000000in}}%
\pgfpathlineto{\pgfqpoint{0.000000in}{-0.048611in}}%
\pgfusepath{stroke,fill}%
}%
\begin{pgfscope}%
\pgfsys@transformshift{3.712691in}{0.528000in}%
\pgfsys@useobject{currentmarker}{}%
\end{pgfscope}%
\end{pgfscope}%
\begin{pgfscope}%
\definecolor{textcolor}{rgb}{0.000000,0.000000,0.000000}%
\pgfsetstrokecolor{textcolor}%
\pgfsetfillcolor{textcolor}%
\pgftext[x=3.712691in,y=0.430778in,,top]{\color{textcolor}\rmfamily\fontsize{10.000000}{12.000000}\selectfont \(\displaystyle 60\)}%
\end{pgfscope}%
\begin{pgfscope}%
\pgfsetbuttcap%
\pgfsetroundjoin%
\definecolor{currentfill}{rgb}{0.000000,0.000000,0.000000}%
\pgfsetfillcolor{currentfill}%
\pgfsetlinewidth{0.803000pt}%
\definecolor{currentstroke}{rgb}{0.000000,0.000000,0.000000}%
\pgfsetstrokecolor{currentstroke}%
\pgfsetdash{}{0pt}%
\pgfsys@defobject{currentmarker}{\pgfqpoint{0.000000in}{-0.048611in}}{\pgfqpoint{0.000000in}{0.000000in}}{%
\pgfpathmoveto{\pgfqpoint{0.000000in}{0.000000in}}%
\pgfpathlineto{\pgfqpoint{0.000000in}{-0.048611in}}%
\pgfusepath{stroke,fill}%
}%
\begin{pgfscope}%
\pgfsys@transformshift{4.623618in}{0.528000in}%
\pgfsys@useobject{currentmarker}{}%
\end{pgfscope}%
\end{pgfscope}%
\begin{pgfscope}%
\definecolor{textcolor}{rgb}{0.000000,0.000000,0.000000}%
\pgfsetstrokecolor{textcolor}%
\pgfsetfillcolor{textcolor}%
\pgftext[x=4.623618in,y=0.430778in,,top]{\color{textcolor}\rmfamily\fontsize{10.000000}{12.000000}\selectfont \(\displaystyle 80\)}%
\end{pgfscope}%
\begin{pgfscope}%
\pgfsetbuttcap%
\pgfsetroundjoin%
\definecolor{currentfill}{rgb}{0.000000,0.000000,0.000000}%
\pgfsetfillcolor{currentfill}%
\pgfsetlinewidth{0.803000pt}%
\definecolor{currentstroke}{rgb}{0.000000,0.000000,0.000000}%
\pgfsetstrokecolor{currentstroke}%
\pgfsetdash{}{0pt}%
\pgfsys@defobject{currentmarker}{\pgfqpoint{0.000000in}{-0.048611in}}{\pgfqpoint{0.000000in}{0.000000in}}{%
\pgfpathmoveto{\pgfqpoint{0.000000in}{0.000000in}}%
\pgfpathlineto{\pgfqpoint{0.000000in}{-0.048611in}}%
\pgfusepath{stroke,fill}%
}%
\begin{pgfscope}%
\pgfsys@transformshift{5.534545in}{0.528000in}%
\pgfsys@useobject{currentmarker}{}%
\end{pgfscope}%
\end{pgfscope}%
\begin{pgfscope}%
\definecolor{textcolor}{rgb}{0.000000,0.000000,0.000000}%
\pgfsetstrokecolor{textcolor}%
\pgfsetfillcolor{textcolor}%
\pgftext[x=5.534545in,y=0.430778in,,top]{\color{textcolor}\rmfamily\fontsize{10.000000}{12.000000}\selectfont \(\displaystyle 100\)}%
\end{pgfscope}%
\begin{pgfscope}%
\pgfsetbuttcap%
\pgfsetroundjoin%
\definecolor{currentfill}{rgb}{0.000000,0.000000,0.000000}%
\pgfsetfillcolor{currentfill}%
\pgfsetlinewidth{0.803000pt}%
\definecolor{currentstroke}{rgb}{0.000000,0.000000,0.000000}%
\pgfsetstrokecolor{currentstroke}%
\pgfsetdash{}{0pt}%
\pgfsys@defobject{currentmarker}{\pgfqpoint{-0.048611in}{0.000000in}}{\pgfqpoint{0.000000in}{0.000000in}}{%
\pgfpathmoveto{\pgfqpoint{0.000000in}{0.000000in}}%
\pgfpathlineto{\pgfqpoint{-0.048611in}{0.000000in}}%
\pgfusepath{stroke,fill}%
}%
\begin{pgfscope}%
\pgfsys@transformshift{0.800000in}{0.655339in}%
\pgfsys@useobject{currentmarker}{}%
\end{pgfscope}%
\end{pgfscope}%
\begin{pgfscope}%
\definecolor{textcolor}{rgb}{0.000000,0.000000,0.000000}%
\pgfsetstrokecolor{textcolor}%
\pgfsetfillcolor{textcolor}%
\pgftext[x=0.386419in,y=0.607113in,left,base]{\color{textcolor}\rmfamily\fontsize{10.000000}{12.000000}\selectfont \(\displaystyle 0.000\)}%
\end{pgfscope}%
\begin{pgfscope}%
\pgfsetbuttcap%
\pgfsetroundjoin%
\definecolor{currentfill}{rgb}{0.000000,0.000000,0.000000}%
\pgfsetfillcolor{currentfill}%
\pgfsetlinewidth{0.803000pt}%
\definecolor{currentstroke}{rgb}{0.000000,0.000000,0.000000}%
\pgfsetstrokecolor{currentstroke}%
\pgfsetdash{}{0pt}%
\pgfsys@defobject{currentmarker}{\pgfqpoint{-0.048611in}{0.000000in}}{\pgfqpoint{0.000000in}{0.000000in}}{%
\pgfpathmoveto{\pgfqpoint{0.000000in}{0.000000in}}%
\pgfpathlineto{\pgfqpoint{-0.048611in}{0.000000in}}%
\pgfusepath{stroke,fill}%
}%
\begin{pgfscope}%
\pgfsys@transformshift{0.800000in}{1.091078in}%
\pgfsys@useobject{currentmarker}{}%
\end{pgfscope}%
\end{pgfscope}%
\begin{pgfscope}%
\definecolor{textcolor}{rgb}{0.000000,0.000000,0.000000}%
\pgfsetstrokecolor{textcolor}%
\pgfsetfillcolor{textcolor}%
\pgftext[x=0.386419in,y=1.042853in,left,base]{\color{textcolor}\rmfamily\fontsize{10.000000}{12.000000}\selectfont \(\displaystyle 0.025\)}%
\end{pgfscope}%
\begin{pgfscope}%
\pgfsetbuttcap%
\pgfsetroundjoin%
\definecolor{currentfill}{rgb}{0.000000,0.000000,0.000000}%
\pgfsetfillcolor{currentfill}%
\pgfsetlinewidth{0.803000pt}%
\definecolor{currentstroke}{rgb}{0.000000,0.000000,0.000000}%
\pgfsetstrokecolor{currentstroke}%
\pgfsetdash{}{0pt}%
\pgfsys@defobject{currentmarker}{\pgfqpoint{-0.048611in}{0.000000in}}{\pgfqpoint{0.000000in}{0.000000in}}{%
\pgfpathmoveto{\pgfqpoint{0.000000in}{0.000000in}}%
\pgfpathlineto{\pgfqpoint{-0.048611in}{0.000000in}}%
\pgfusepath{stroke,fill}%
}%
\begin{pgfscope}%
\pgfsys@transformshift{0.800000in}{1.526818in}%
\pgfsys@useobject{currentmarker}{}%
\end{pgfscope}%
\end{pgfscope}%
\begin{pgfscope}%
\definecolor{textcolor}{rgb}{0.000000,0.000000,0.000000}%
\pgfsetstrokecolor{textcolor}%
\pgfsetfillcolor{textcolor}%
\pgftext[x=0.386419in,y=1.478592in,left,base]{\color{textcolor}\rmfamily\fontsize{10.000000}{12.000000}\selectfont \(\displaystyle 0.050\)}%
\end{pgfscope}%
\begin{pgfscope}%
\pgfsetbuttcap%
\pgfsetroundjoin%
\definecolor{currentfill}{rgb}{0.000000,0.000000,0.000000}%
\pgfsetfillcolor{currentfill}%
\pgfsetlinewidth{0.803000pt}%
\definecolor{currentstroke}{rgb}{0.000000,0.000000,0.000000}%
\pgfsetstrokecolor{currentstroke}%
\pgfsetdash{}{0pt}%
\pgfsys@defobject{currentmarker}{\pgfqpoint{-0.048611in}{0.000000in}}{\pgfqpoint{0.000000in}{0.000000in}}{%
\pgfpathmoveto{\pgfqpoint{0.000000in}{0.000000in}}%
\pgfpathlineto{\pgfqpoint{-0.048611in}{0.000000in}}%
\pgfusepath{stroke,fill}%
}%
\begin{pgfscope}%
\pgfsys@transformshift{0.800000in}{1.962557in}%
\pgfsys@useobject{currentmarker}{}%
\end{pgfscope}%
\end{pgfscope}%
\begin{pgfscope}%
\definecolor{textcolor}{rgb}{0.000000,0.000000,0.000000}%
\pgfsetstrokecolor{textcolor}%
\pgfsetfillcolor{textcolor}%
\pgftext[x=0.386419in,y=1.914332in,left,base]{\color{textcolor}\rmfamily\fontsize{10.000000}{12.000000}\selectfont \(\displaystyle 0.075\)}%
\end{pgfscope}%
\begin{pgfscope}%
\pgfsetbuttcap%
\pgfsetroundjoin%
\definecolor{currentfill}{rgb}{0.000000,0.000000,0.000000}%
\pgfsetfillcolor{currentfill}%
\pgfsetlinewidth{0.803000pt}%
\definecolor{currentstroke}{rgb}{0.000000,0.000000,0.000000}%
\pgfsetstrokecolor{currentstroke}%
\pgfsetdash{}{0pt}%
\pgfsys@defobject{currentmarker}{\pgfqpoint{-0.048611in}{0.000000in}}{\pgfqpoint{0.000000in}{0.000000in}}{%
\pgfpathmoveto{\pgfqpoint{0.000000in}{0.000000in}}%
\pgfpathlineto{\pgfqpoint{-0.048611in}{0.000000in}}%
\pgfusepath{stroke,fill}%
}%
\begin{pgfscope}%
\pgfsys@transformshift{0.800000in}{2.398297in}%
\pgfsys@useobject{currentmarker}{}%
\end{pgfscope}%
\end{pgfscope}%
\begin{pgfscope}%
\definecolor{textcolor}{rgb}{0.000000,0.000000,0.000000}%
\pgfsetstrokecolor{textcolor}%
\pgfsetfillcolor{textcolor}%
\pgftext[x=0.386419in,y=2.350071in,left,base]{\color{textcolor}\rmfamily\fontsize{10.000000}{12.000000}\selectfont \(\displaystyle 0.100\)}%
\end{pgfscope}%
\begin{pgfscope}%
\pgfsetbuttcap%
\pgfsetroundjoin%
\definecolor{currentfill}{rgb}{0.000000,0.000000,0.000000}%
\pgfsetfillcolor{currentfill}%
\pgfsetlinewidth{0.803000pt}%
\definecolor{currentstroke}{rgb}{0.000000,0.000000,0.000000}%
\pgfsetstrokecolor{currentstroke}%
\pgfsetdash{}{0pt}%
\pgfsys@defobject{currentmarker}{\pgfqpoint{-0.048611in}{0.000000in}}{\pgfqpoint{0.000000in}{0.000000in}}{%
\pgfpathmoveto{\pgfqpoint{0.000000in}{0.000000in}}%
\pgfpathlineto{\pgfqpoint{-0.048611in}{0.000000in}}%
\pgfusepath{stroke,fill}%
}%
\begin{pgfscope}%
\pgfsys@transformshift{0.800000in}{2.834036in}%
\pgfsys@useobject{currentmarker}{}%
\end{pgfscope}%
\end{pgfscope}%
\begin{pgfscope}%
\definecolor{textcolor}{rgb}{0.000000,0.000000,0.000000}%
\pgfsetstrokecolor{textcolor}%
\pgfsetfillcolor{textcolor}%
\pgftext[x=0.386419in,y=2.785811in,left,base]{\color{textcolor}\rmfamily\fontsize{10.000000}{12.000000}\selectfont \(\displaystyle 0.125\)}%
\end{pgfscope}%
\begin{pgfscope}%
\pgfsetbuttcap%
\pgfsetroundjoin%
\definecolor{currentfill}{rgb}{0.000000,0.000000,0.000000}%
\pgfsetfillcolor{currentfill}%
\pgfsetlinewidth{0.803000pt}%
\definecolor{currentstroke}{rgb}{0.000000,0.000000,0.000000}%
\pgfsetstrokecolor{currentstroke}%
\pgfsetdash{}{0pt}%
\pgfsys@defobject{currentmarker}{\pgfqpoint{-0.048611in}{0.000000in}}{\pgfqpoint{0.000000in}{0.000000in}}{%
\pgfpathmoveto{\pgfqpoint{0.000000in}{0.000000in}}%
\pgfpathlineto{\pgfqpoint{-0.048611in}{0.000000in}}%
\pgfusepath{stroke,fill}%
}%
\begin{pgfscope}%
\pgfsys@transformshift{0.800000in}{3.269776in}%
\pgfsys@useobject{currentmarker}{}%
\end{pgfscope}%
\end{pgfscope}%
\begin{pgfscope}%
\definecolor{textcolor}{rgb}{0.000000,0.000000,0.000000}%
\pgfsetstrokecolor{textcolor}%
\pgfsetfillcolor{textcolor}%
\pgftext[x=0.386419in,y=3.221550in,left,base]{\color{textcolor}\rmfamily\fontsize{10.000000}{12.000000}\selectfont \(\displaystyle 0.150\)}%
\end{pgfscope}%
\begin{pgfscope}%
\pgfsetbuttcap%
\pgfsetroundjoin%
\definecolor{currentfill}{rgb}{0.000000,0.000000,0.000000}%
\pgfsetfillcolor{currentfill}%
\pgfsetlinewidth{0.803000pt}%
\definecolor{currentstroke}{rgb}{0.000000,0.000000,0.000000}%
\pgfsetstrokecolor{currentstroke}%
\pgfsetdash{}{0pt}%
\pgfsys@defobject{currentmarker}{\pgfqpoint{-0.048611in}{0.000000in}}{\pgfqpoint{0.000000in}{0.000000in}}{%
\pgfpathmoveto{\pgfqpoint{0.000000in}{0.000000in}}%
\pgfpathlineto{\pgfqpoint{-0.048611in}{0.000000in}}%
\pgfusepath{stroke,fill}%
}%
\begin{pgfscope}%
\pgfsys@transformshift{0.800000in}{3.705515in}%
\pgfsys@useobject{currentmarker}{}%
\end{pgfscope}%
\end{pgfscope}%
\begin{pgfscope}%
\definecolor{textcolor}{rgb}{0.000000,0.000000,0.000000}%
\pgfsetstrokecolor{textcolor}%
\pgfsetfillcolor{textcolor}%
\pgftext[x=0.386419in,y=3.657290in,left,base]{\color{textcolor}\rmfamily\fontsize{10.000000}{12.000000}\selectfont \(\displaystyle 0.175\)}%
\end{pgfscope}%
\begin{pgfscope}%
\pgfsetbuttcap%
\pgfsetroundjoin%
\definecolor{currentfill}{rgb}{0.000000,0.000000,0.000000}%
\pgfsetfillcolor{currentfill}%
\pgfsetlinewidth{0.803000pt}%
\definecolor{currentstroke}{rgb}{0.000000,0.000000,0.000000}%
\pgfsetstrokecolor{currentstroke}%
\pgfsetdash{}{0pt}%
\pgfsys@defobject{currentmarker}{\pgfqpoint{-0.048611in}{0.000000in}}{\pgfqpoint{0.000000in}{0.000000in}}{%
\pgfpathmoveto{\pgfqpoint{0.000000in}{0.000000in}}%
\pgfpathlineto{\pgfqpoint{-0.048611in}{0.000000in}}%
\pgfusepath{stroke,fill}%
}%
\begin{pgfscope}%
\pgfsys@transformshift{0.800000in}{4.141255in}%
\pgfsys@useobject{currentmarker}{}%
\end{pgfscope}%
\end{pgfscope}%
\begin{pgfscope}%
\definecolor{textcolor}{rgb}{0.000000,0.000000,0.000000}%
\pgfsetstrokecolor{textcolor}%
\pgfsetfillcolor{textcolor}%
\pgftext[x=0.386419in,y=4.093029in,left,base]{\color{textcolor}\rmfamily\fontsize{10.000000}{12.000000}\selectfont \(\displaystyle 0.200\)}%
\end{pgfscope}%
\begin{pgfscope}%
\pgfpathrectangle{\pgfqpoint{0.800000in}{0.528000in}}{\pgfqpoint{4.960000in}{3.696000in}}%
\pgfusepath{clip}%
\pgfsetrectcap%
\pgfsetroundjoin%
\pgfsetlinewidth{1.505625pt}%
\definecolor{currentstroke}{rgb}{0.121569,0.466667,0.705882}%
\pgfsetstrokecolor{currentstroke}%
\pgfsetdash{}{0pt}%
\pgfpathmoveto{\pgfqpoint{1.025455in}{1.531356in}}%
\pgfpathlineto{\pgfqpoint{1.034491in}{1.535762in}}%
\pgfpathlineto{\pgfqpoint{1.043527in}{1.534551in}}%
\pgfpathlineto{\pgfqpoint{1.070636in}{1.542673in}}%
\pgfpathlineto{\pgfqpoint{1.088708in}{1.549804in}}%
\pgfpathlineto{\pgfqpoint{1.097745in}{1.555562in}}%
\pgfpathlineto{\pgfqpoint{1.106781in}{1.558351in}}%
\pgfpathlineto{\pgfqpoint{1.124853in}{1.568399in}}%
\pgfpathlineto{\pgfqpoint{1.142926in}{1.580028in}}%
\pgfpathlineto{\pgfqpoint{1.151962in}{1.588591in}}%
\pgfpathlineto{\pgfqpoint{1.160998in}{1.593363in}}%
\pgfpathlineto{\pgfqpoint{1.188107in}{1.616824in}}%
\pgfpathlineto{\pgfqpoint{1.197143in}{1.625641in}}%
\pgfpathlineto{\pgfqpoint{1.206180in}{1.638093in}}%
\pgfpathlineto{\pgfqpoint{1.215216in}{1.644894in}}%
\pgfpathlineto{\pgfqpoint{1.224252in}{1.655389in}}%
\pgfpathlineto{\pgfqpoint{1.233288in}{1.668820in}}%
\pgfpathlineto{\pgfqpoint{1.242325in}{1.678257in}}%
\pgfpathlineto{\pgfqpoint{1.278470in}{1.732447in}}%
\pgfpathlineto{\pgfqpoint{1.296542in}{1.763866in}}%
\pgfpathlineto{\pgfqpoint{1.323651in}{1.819753in}}%
\pgfpathlineto{\pgfqpoint{1.332687in}{1.845232in}}%
\pgfpathlineto{\pgfqpoint{1.341723in}{1.862302in}}%
\pgfpathlineto{\pgfqpoint{1.359796in}{1.910109in}}%
\pgfpathlineto{\pgfqpoint{1.395941in}{2.025024in}}%
\pgfpathlineto{\pgfqpoint{1.414013in}{2.094268in}}%
\pgfpathlineto{\pgfqpoint{1.432086in}{2.172103in}}%
\pgfpathlineto{\pgfqpoint{1.450158in}{2.263476in}}%
\pgfpathlineto{\pgfqpoint{1.468231in}{2.367213in}}%
\pgfpathlineto{\pgfqpoint{1.486304in}{2.488425in}}%
\pgfpathlineto{\pgfqpoint{1.504376in}{2.624896in}}%
\pgfpathlineto{\pgfqpoint{1.522449in}{2.784504in}}%
\pgfpathlineto{\pgfqpoint{1.540521in}{2.967723in}}%
\pgfpathlineto{\pgfqpoint{1.558594in}{3.177845in}}%
\pgfpathlineto{\pgfqpoint{1.576666in}{3.403054in}}%
\pgfpathlineto{\pgfqpoint{1.603775in}{3.751313in}}%
\pgfpathlineto{\pgfqpoint{1.621847in}{3.946227in}}%
\pgfpathlineto{\pgfqpoint{1.630884in}{4.009060in}}%
\pgfpathlineto{\pgfqpoint{1.639920in}{4.049853in}}%
\pgfpathlineto{\pgfqpoint{1.648956in}{4.056000in}}%
\pgfpathlineto{\pgfqpoint{1.657992in}{4.030520in}}%
\pgfpathlineto{\pgfqpoint{1.667029in}{3.972828in}}%
\pgfpathlineto{\pgfqpoint{1.676065in}{3.886692in}}%
\pgfpathlineto{\pgfqpoint{1.685101in}{3.777846in}}%
\pgfpathlineto{\pgfqpoint{1.703174in}{3.517610in}}%
\pgfpathlineto{\pgfqpoint{1.748355in}{2.847498in}}%
\pgfpathlineto{\pgfqpoint{1.766427in}{2.620839in}}%
\pgfpathlineto{\pgfqpoint{1.784500in}{2.424622in}}%
\pgfpathlineto{\pgfqpoint{1.802572in}{2.255334in}}%
\pgfpathlineto{\pgfqpoint{1.820645in}{2.109641in}}%
\pgfpathlineto{\pgfqpoint{1.838717in}{1.983991in}}%
\pgfpathlineto{\pgfqpoint{1.865826in}{1.825984in}}%
\pgfpathlineto{\pgfqpoint{1.883899in}{1.737234in}}%
\pgfpathlineto{\pgfqpoint{1.911007in}{1.623751in}}%
\pgfpathlineto{\pgfqpoint{1.929080in}{1.558777in}}%
\pgfpathlineto{\pgfqpoint{1.947152in}{1.501084in}}%
\pgfpathlineto{\pgfqpoint{1.974261in}{1.425629in}}%
\pgfpathlineto{\pgfqpoint{1.992334in}{1.381324in}}%
\pgfpathlineto{\pgfqpoint{2.001370in}{1.363326in}}%
\pgfpathlineto{\pgfqpoint{2.010406in}{1.341346in}}%
\pgfpathlineto{\pgfqpoint{2.019443in}{1.322739in}}%
\pgfpathlineto{\pgfqpoint{2.028479in}{1.307168in}}%
\pgfpathlineto{\pgfqpoint{2.046551in}{1.271925in}}%
\pgfpathlineto{\pgfqpoint{2.073660in}{1.227430in}}%
\pgfpathlineto{\pgfqpoint{2.091733in}{1.200764in}}%
\pgfpathlineto{\pgfqpoint{2.100769in}{1.192445in}}%
\pgfpathlineto{\pgfqpoint{2.109805in}{1.176180in}}%
\pgfpathlineto{\pgfqpoint{2.118841in}{1.164602in}}%
\pgfpathlineto{\pgfqpoint{2.127878in}{1.057544in}}%
\pgfpathlineto{\pgfqpoint{2.136914in}{1.144868in}}%
\pgfpathlineto{\pgfqpoint{2.145950in}{1.132417in}}%
\pgfpathlineto{\pgfqpoint{2.164023in}{1.112836in}}%
\pgfpathlineto{\pgfqpoint{2.173059in}{1.108158in}}%
\pgfpathlineto{\pgfqpoint{2.182095in}{1.094670in}}%
\pgfpathlineto{\pgfqpoint{2.209204in}{1.069631in}}%
\pgfpathlineto{\pgfqpoint{2.245349in}{1.039897in}}%
\pgfpathlineto{\pgfqpoint{2.281494in}{1.013691in}}%
\pgfpathlineto{\pgfqpoint{2.290530in}{1.007703in}}%
\pgfpathlineto{\pgfqpoint{2.299566in}{1.003063in}}%
\pgfpathlineto{\pgfqpoint{2.317639in}{0.990468in}}%
\pgfpathlineto{\pgfqpoint{2.335711in}{0.980273in}}%
\pgfpathlineto{\pgfqpoint{2.344748in}{0.978523in}}%
\pgfpathlineto{\pgfqpoint{2.353784in}{0.969770in}}%
\pgfpathlineto{\pgfqpoint{2.389929in}{0.951163in}}%
\pgfpathlineto{\pgfqpoint{2.435110in}{0.930511in}}%
\pgfpathlineto{\pgfqpoint{2.462219in}{0.919299in}}%
\pgfpathlineto{\pgfqpoint{2.471255in}{0.919392in}}%
\pgfpathlineto{\pgfqpoint{2.480291in}{0.912402in}}%
\pgfpathlineto{\pgfqpoint{2.516437in}{0.899142in}}%
\pgfpathlineto{\pgfqpoint{2.534509in}{0.893017in}}%
\pgfpathlineto{\pgfqpoint{2.543545in}{0.896036in}}%
\pgfpathlineto{\pgfqpoint{2.552582in}{0.889802in}}%
\pgfpathlineto{\pgfqpoint{2.561618in}{0.884971in}}%
\pgfpathlineto{\pgfqpoint{2.579690in}{0.878841in}}%
\pgfpathlineto{\pgfqpoint{2.597763in}{0.873582in}}%
\pgfpathlineto{\pgfqpoint{2.606799in}{0.872465in}}%
\pgfpathlineto{\pgfqpoint{2.615835in}{0.873326in}}%
\pgfpathlineto{\pgfqpoint{2.624872in}{0.866092in}}%
\pgfpathlineto{\pgfqpoint{2.633908in}{0.865060in}}%
\pgfpathlineto{\pgfqpoint{2.642944in}{0.861351in}}%
\pgfpathlineto{\pgfqpoint{2.679089in}{0.852889in}}%
\pgfpathlineto{\pgfqpoint{2.688125in}{0.851243in}}%
\pgfpathlineto{\pgfqpoint{2.706198in}{0.846174in}}%
\pgfpathlineto{\pgfqpoint{2.733307in}{0.840264in}}%
\pgfpathlineto{\pgfqpoint{2.742343in}{0.840817in}}%
\pgfpathlineto{\pgfqpoint{2.751379in}{0.836507in}}%
\pgfpathlineto{\pgfqpoint{2.760415in}{0.838125in}}%
\pgfpathlineto{\pgfqpoint{2.769452in}{0.832883in}}%
\pgfpathlineto{\pgfqpoint{2.814633in}{0.824372in}}%
\pgfpathlineto{\pgfqpoint{2.823669in}{0.823506in}}%
\pgfpathlineto{\pgfqpoint{2.832705in}{0.821429in}}%
\pgfpathlineto{\pgfqpoint{2.841742in}{0.820791in}}%
\pgfpathlineto{\pgfqpoint{2.859814in}{0.816570in}}%
\pgfpathlineto{\pgfqpoint{2.868850in}{0.815289in}}%
\pgfpathlineto{\pgfqpoint{2.877887in}{0.815767in}}%
\pgfpathlineto{\pgfqpoint{2.904995in}{0.809397in}}%
\pgfpathlineto{\pgfqpoint{2.914032in}{0.813074in}}%
\pgfpathlineto{\pgfqpoint{2.923068in}{0.806733in}}%
\pgfpathlineto{\pgfqpoint{2.932104in}{0.806223in}}%
\pgfpathlineto{\pgfqpoint{2.950177in}{0.802786in}}%
\pgfpathlineto{\pgfqpoint{2.959213in}{0.803688in}}%
\pgfpathlineto{\pgfqpoint{2.968249in}{0.802790in}}%
\pgfpathlineto{\pgfqpoint{2.977285in}{0.799068in}}%
\pgfpathlineto{\pgfqpoint{2.986322in}{0.804020in}}%
\pgfpathlineto{\pgfqpoint{2.995358in}{0.797986in}}%
\pgfpathlineto{\pgfqpoint{3.004394in}{0.795507in}}%
\pgfpathlineto{\pgfqpoint{3.031503in}{0.793963in}}%
\pgfpathlineto{\pgfqpoint{3.049576in}{0.789928in}}%
\pgfpathlineto{\pgfqpoint{3.058612in}{0.794194in}}%
\pgfpathlineto{\pgfqpoint{3.067648in}{0.787810in}}%
\pgfpathlineto{\pgfqpoint{3.112829in}{0.783186in}}%
\pgfpathlineto{\pgfqpoint{3.130902in}{0.783700in}}%
\pgfpathlineto{\pgfqpoint{3.139938in}{0.779919in}}%
\pgfpathlineto{\pgfqpoint{3.185119in}{0.777386in}}%
\pgfpathlineto{\pgfqpoint{3.203192in}{0.773699in}}%
\pgfpathlineto{\pgfqpoint{3.212228in}{0.774358in}}%
\pgfpathlineto{\pgfqpoint{3.230301in}{0.771206in}}%
\pgfpathlineto{\pgfqpoint{3.239337in}{0.770398in}}%
\pgfpathlineto{\pgfqpoint{3.248373in}{0.771479in}}%
\pgfpathlineto{\pgfqpoint{3.257409in}{0.768840in}}%
\pgfpathlineto{\pgfqpoint{3.266446in}{0.768033in}}%
\pgfpathlineto{\pgfqpoint{3.275482in}{0.760089in}}%
\pgfpathlineto{\pgfqpoint{3.284518in}{0.769633in}}%
\pgfpathlineto{\pgfqpoint{3.293554in}{0.765759in}}%
\pgfpathlineto{\pgfqpoint{3.302591in}{0.765027in}}%
\pgfpathlineto{\pgfqpoint{3.311627in}{0.765901in}}%
\pgfpathlineto{\pgfqpoint{3.338736in}{0.762152in}}%
\pgfpathlineto{\pgfqpoint{3.356808in}{0.766368in}}%
\pgfpathlineto{\pgfqpoint{3.365844in}{0.760453in}}%
\pgfpathlineto{\pgfqpoint{3.383917in}{0.761140in}}%
\pgfpathlineto{\pgfqpoint{3.392953in}{0.758108in}}%
\pgfpathlineto{\pgfqpoint{3.420062in}{0.756193in}}%
\pgfpathlineto{\pgfqpoint{3.429098in}{0.761814in}}%
\pgfpathlineto{\pgfqpoint{3.438134in}{0.755719in}}%
\pgfpathlineto{\pgfqpoint{3.447171in}{0.754740in}}%
\pgfpathlineto{\pgfqpoint{3.456207in}{0.755477in}}%
\pgfpathlineto{\pgfqpoint{3.465243in}{0.753147in}}%
\pgfpathlineto{\pgfqpoint{3.474279in}{0.752541in}}%
\pgfpathlineto{\pgfqpoint{3.483316in}{0.754354in}}%
\pgfpathlineto{\pgfqpoint{3.492352in}{0.751402in}}%
\pgfpathlineto{\pgfqpoint{3.501388in}{0.756398in}}%
\pgfpathlineto{\pgfqpoint{3.510424in}{0.750271in}}%
\pgfpathlineto{\pgfqpoint{3.546570in}{0.748389in}}%
\pgfpathlineto{\pgfqpoint{3.564642in}{0.747026in}}%
\pgfpathlineto{\pgfqpoint{3.573678in}{0.750192in}}%
\pgfpathlineto{\pgfqpoint{3.582715in}{0.746542in}}%
\pgfpathlineto{\pgfqpoint{3.591751in}{0.748165in}}%
\pgfpathlineto{\pgfqpoint{3.600787in}{0.744978in}}%
\pgfpathlineto{\pgfqpoint{3.609823in}{0.744480in}}%
\pgfpathlineto{\pgfqpoint{3.618860in}{0.745768in}}%
\pgfpathlineto{\pgfqpoint{3.627896in}{0.743499in}}%
\pgfpathlineto{\pgfqpoint{3.655005in}{0.742830in}}%
\pgfpathlineto{\pgfqpoint{3.664041in}{0.744389in}}%
\pgfpathlineto{\pgfqpoint{3.682113in}{0.740679in}}%
\pgfpathlineto{\pgfqpoint{3.691150in}{0.740223in}}%
\pgfpathlineto{\pgfqpoint{3.700186in}{0.741382in}}%
\pgfpathlineto{\pgfqpoint{3.718258in}{0.738890in}}%
\pgfpathlineto{\pgfqpoint{3.727295in}{0.742746in}}%
\pgfpathlineto{\pgfqpoint{3.736331in}{0.738023in}}%
\pgfpathlineto{\pgfqpoint{3.745367in}{0.739312in}}%
\pgfpathlineto{\pgfqpoint{3.772476in}{0.736331in}}%
\pgfpathlineto{\pgfqpoint{3.790548in}{0.735526in}}%
\pgfpathlineto{\pgfqpoint{3.799585in}{0.741100in}}%
\pgfpathlineto{\pgfqpoint{3.808621in}{0.734727in}}%
\pgfpathlineto{\pgfqpoint{3.817657in}{0.736430in}}%
\pgfpathlineto{\pgfqpoint{3.826693in}{0.733942in}}%
\pgfpathlineto{\pgfqpoint{3.862838in}{0.732430in}}%
\pgfpathlineto{\pgfqpoint{3.871875in}{0.738332in}}%
\pgfpathlineto{\pgfqpoint{3.880911in}{0.731973in}}%
\pgfpathlineto{\pgfqpoint{3.889947in}{0.733235in}}%
\pgfpathlineto{\pgfqpoint{3.898983in}{0.731191in}}%
\pgfpathlineto{\pgfqpoint{3.926092in}{0.729882in}}%
\pgfpathlineto{\pgfqpoint{3.935128in}{0.730339in}}%
\pgfpathlineto{\pgfqpoint{3.944165in}{0.734791in}}%
\pgfpathlineto{\pgfqpoint{3.953201in}{0.730972in}}%
\pgfpathlineto{\pgfqpoint{3.980310in}{0.727835in}}%
\pgfpathlineto{\pgfqpoint{3.998382in}{0.727352in}}%
\pgfpathlineto{\pgfqpoint{4.016455in}{0.730532in}}%
\pgfpathlineto{\pgfqpoint{4.025491in}{0.726211in}}%
\pgfpathlineto{\pgfqpoint{4.061636in}{0.725403in}}%
\pgfpathlineto{\pgfqpoint{4.070672in}{0.724671in}}%
\pgfpathlineto{\pgfqpoint{4.079709in}{0.727637in}}%
\pgfpathlineto{\pgfqpoint{4.115854in}{0.723172in}}%
\pgfpathlineto{\pgfqpoint{4.124890in}{0.723759in}}%
\pgfpathlineto{\pgfqpoint{4.142962in}{0.722309in}}%
\pgfpathlineto{\pgfqpoint{4.161035in}{0.722441in}}%
\pgfpathlineto{\pgfqpoint{4.170071in}{0.726596in}}%
\pgfpathlineto{\pgfqpoint{4.179107in}{0.721191in}}%
\pgfpathlineto{\pgfqpoint{4.206216in}{0.720378in}}%
\pgfpathlineto{\pgfqpoint{4.215252in}{0.722004in}}%
\pgfpathlineto{\pgfqpoint{4.224289in}{0.722248in}}%
\pgfpathlineto{\pgfqpoint{4.233325in}{0.720071in}}%
\pgfpathlineto{\pgfqpoint{4.242361in}{0.725592in}}%
\pgfpathlineto{\pgfqpoint{4.251397in}{0.719066in}}%
\pgfpathlineto{\pgfqpoint{4.278506in}{0.718303in}}%
\pgfpathlineto{\pgfqpoint{4.287542in}{0.720980in}}%
\pgfpathlineto{\pgfqpoint{4.296579in}{0.717804in}}%
\pgfpathlineto{\pgfqpoint{4.305615in}{0.718370in}}%
\pgfpathlineto{\pgfqpoint{4.314651in}{0.723605in}}%
\pgfpathlineto{\pgfqpoint{4.323687in}{0.716987in}}%
\pgfpathlineto{\pgfqpoint{4.359832in}{0.717967in}}%
\pgfpathlineto{\pgfqpoint{4.368869in}{0.715885in}}%
\pgfpathlineto{\pgfqpoint{4.377905in}{0.715656in}}%
\pgfpathlineto{\pgfqpoint{4.386941in}{0.720948in}}%
\pgfpathlineto{\pgfqpoint{4.395977in}{0.715545in}}%
\pgfpathlineto{\pgfqpoint{4.414050in}{0.714747in}}%
\pgfpathlineto{\pgfqpoint{4.423086in}{0.716057in}}%
\pgfpathlineto{\pgfqpoint{4.432122in}{0.714302in}}%
\pgfpathlineto{\pgfqpoint{4.450195in}{0.713977in}}%
\pgfpathlineto{\pgfqpoint{4.459231in}{0.717677in}}%
\pgfpathlineto{\pgfqpoint{4.468267in}{0.713856in}}%
\pgfpathlineto{\pgfqpoint{4.477304in}{0.713326in}}%
\pgfpathlineto{\pgfqpoint{4.486340in}{0.713988in}}%
\pgfpathlineto{\pgfqpoint{4.495376in}{0.716214in}}%
\pgfpathlineto{\pgfqpoint{4.504412in}{0.712591in}}%
\pgfpathlineto{\pgfqpoint{4.513449in}{0.712385in}}%
\pgfpathlineto{\pgfqpoint{4.540557in}{0.715547in}}%
\pgfpathlineto{\pgfqpoint{4.549594in}{0.711706in}}%
\pgfpathlineto{\pgfqpoint{4.558630in}{0.713143in}}%
\pgfpathlineto{\pgfqpoint{4.576702in}{0.710977in}}%
\pgfpathlineto{\pgfqpoint{4.585739in}{0.712387in}}%
\pgfpathlineto{\pgfqpoint{4.594775in}{0.710587in}}%
\pgfpathlineto{\pgfqpoint{4.603811in}{0.712005in}}%
\pgfpathlineto{\pgfqpoint{4.612848in}{0.715968in}}%
\pgfpathlineto{\pgfqpoint{4.621884in}{0.710173in}}%
\pgfpathlineto{\pgfqpoint{4.630920in}{0.711289in}}%
\pgfpathlineto{\pgfqpoint{4.639956in}{0.709637in}}%
\pgfpathlineto{\pgfqpoint{4.648993in}{0.711229in}}%
\pgfpathlineto{\pgfqpoint{4.658029in}{0.709266in}}%
\pgfpathlineto{\pgfqpoint{4.676101in}{0.710863in}}%
\pgfpathlineto{\pgfqpoint{4.685138in}{0.715202in}}%
\pgfpathlineto{\pgfqpoint{4.694174in}{0.710894in}}%
\pgfpathlineto{\pgfqpoint{4.703210in}{0.711497in}}%
\pgfpathlineto{\pgfqpoint{4.730319in}{0.707833in}}%
\pgfpathlineto{\pgfqpoint{4.739355in}{0.707741in}}%
\pgfpathlineto{\pgfqpoint{4.748391in}{0.709497in}}%
\pgfpathlineto{\pgfqpoint{4.757428in}{0.713559in}}%
\pgfpathlineto{\pgfqpoint{4.766464in}{0.707365in}}%
\pgfpathlineto{\pgfqpoint{4.775500in}{0.708826in}}%
\pgfpathlineto{\pgfqpoint{4.793573in}{0.706639in}}%
\pgfpathlineto{\pgfqpoint{4.811645in}{0.706309in}}%
\pgfpathlineto{\pgfqpoint{4.820681in}{0.707810in}}%
\pgfpathlineto{\pgfqpoint{4.829718in}{0.711335in}}%
\pgfpathlineto{\pgfqpoint{4.838754in}{0.707652in}}%
\pgfpathlineto{\pgfqpoint{4.865863in}{0.705344in}}%
\pgfpathlineto{\pgfqpoint{4.892971in}{0.705878in}}%
\pgfpathlineto{\pgfqpoint{4.902008in}{0.708596in}}%
\pgfpathlineto{\pgfqpoint{4.911044in}{0.707840in}}%
\pgfpathlineto{\pgfqpoint{4.920080in}{0.705476in}}%
\pgfpathlineto{\pgfqpoint{4.938153in}{0.704110in}}%
\pgfpathlineto{\pgfqpoint{4.956225in}{0.704335in}}%
\pgfpathlineto{\pgfqpoint{4.965261in}{0.706007in}}%
\pgfpathlineto{\pgfqpoint{4.974298in}{0.705192in}}%
\pgfpathlineto{\pgfqpoint{4.983334in}{0.708111in}}%
\pgfpathlineto{\pgfqpoint{4.992370in}{0.703225in}}%
\pgfpathlineto{\pgfqpoint{5.010443in}{0.703074in}}%
\pgfpathlineto{\pgfqpoint{5.019479in}{0.702792in}}%
\pgfpathlineto{\pgfqpoint{5.028515in}{0.704137in}}%
\pgfpathlineto{\pgfqpoint{5.037551in}{0.702512in}}%
\pgfpathlineto{\pgfqpoint{5.046588in}{0.703358in}}%
\pgfpathlineto{\pgfqpoint{5.055624in}{0.708473in}}%
\pgfpathlineto{\pgfqpoint{5.064660in}{0.702442in}}%
\pgfpathlineto{\pgfqpoint{5.073696in}{0.703501in}}%
\pgfpathlineto{\pgfqpoint{5.082733in}{0.701820in}}%
\pgfpathlineto{\pgfqpoint{5.100805in}{0.703590in}}%
\pgfpathlineto{\pgfqpoint{5.109842in}{0.701411in}}%
\pgfpathlineto{\pgfqpoint{5.127914in}{0.707765in}}%
\pgfpathlineto{\pgfqpoint{5.136950in}{0.701201in}}%
\pgfpathlineto{\pgfqpoint{5.155023in}{0.701566in}}%
\pgfpathlineto{\pgfqpoint{5.164059in}{0.701864in}}%
\pgfpathlineto{\pgfqpoint{5.173095in}{0.700497in}}%
\pgfpathlineto{\pgfqpoint{5.182132in}{0.700377in}}%
\pgfpathlineto{\pgfqpoint{5.191168in}{0.702498in}}%
\pgfpathlineto{\pgfqpoint{5.200204in}{0.706258in}}%
\pgfpathlineto{\pgfqpoint{5.209240in}{0.701003in}}%
\pgfpathlineto{\pgfqpoint{5.218277in}{0.700257in}}%
\pgfpathlineto{\pgfqpoint{5.236349in}{0.701062in}}%
\pgfpathlineto{\pgfqpoint{5.245385in}{0.699654in}}%
\pgfpathlineto{\pgfqpoint{5.263458in}{0.699250in}}%
\pgfpathlineto{\pgfqpoint{5.272494in}{0.704238in}}%
\pgfpathlineto{\pgfqpoint{5.281530in}{0.699411in}}%
\pgfpathlineto{\pgfqpoint{5.290567in}{0.699167in}}%
\pgfpathlineto{\pgfqpoint{5.299603in}{0.700653in}}%
\pgfpathlineto{\pgfqpoint{5.317675in}{0.699277in}}%
\pgfpathlineto{\pgfqpoint{5.326712in}{0.701346in}}%
\pgfpathlineto{\pgfqpoint{5.335748in}{0.698301in}}%
\pgfpathlineto{\pgfqpoint{5.344784in}{0.701792in}}%
\pgfpathlineto{\pgfqpoint{5.353820in}{0.701413in}}%
\pgfpathlineto{\pgfqpoint{5.362857in}{0.698051in}}%
\pgfpathlineto{\pgfqpoint{5.399002in}{0.700447in}}%
\pgfpathlineto{\pgfqpoint{5.408038in}{0.697394in}}%
\pgfpathlineto{\pgfqpoint{5.417074in}{0.698783in}}%
\pgfpathlineto{\pgfqpoint{5.426110in}{0.702645in}}%
\pgfpathlineto{\pgfqpoint{5.444183in}{0.696955in}}%
\pgfpathlineto{\pgfqpoint{5.462255in}{0.698187in}}%
\pgfpathlineto{\pgfqpoint{5.471292in}{0.696691in}}%
\pgfpathlineto{\pgfqpoint{5.489364in}{0.696403in}}%
\pgfpathlineto{\pgfqpoint{5.498400in}{0.702891in}}%
\pgfpathlineto{\pgfqpoint{5.507437in}{0.696529in}}%
\pgfpathlineto{\pgfqpoint{5.525509in}{0.696000in}}%
\pgfpathlineto{\pgfqpoint{5.534545in}{0.698847in}}%
\pgfpathlineto{\pgfqpoint{5.534545in}{0.698847in}}%
\pgfusepath{stroke}%
\end{pgfscope}%
\begin{pgfscope}%
\pgfsetrectcap%
\pgfsetmiterjoin%
\pgfsetlinewidth{0.803000pt}%
\definecolor{currentstroke}{rgb}{0.000000,0.000000,0.000000}%
\pgfsetstrokecolor{currentstroke}%
\pgfsetdash{}{0pt}%
\pgfpathmoveto{\pgfqpoint{0.800000in}{0.528000in}}%
\pgfpathlineto{\pgfqpoint{0.800000in}{4.224000in}}%
\pgfusepath{stroke}%
\end{pgfscope}%
\begin{pgfscope}%
\pgfsetrectcap%
\pgfsetmiterjoin%
\pgfsetlinewidth{0.803000pt}%
\definecolor{currentstroke}{rgb}{0.000000,0.000000,0.000000}%
\pgfsetstrokecolor{currentstroke}%
\pgfsetdash{}{0pt}%
\pgfpathmoveto{\pgfqpoint{5.760000in}{0.528000in}}%
\pgfpathlineto{\pgfqpoint{5.760000in}{4.224000in}}%
\pgfusepath{stroke}%
\end{pgfscope}%
\begin{pgfscope}%
\pgfsetrectcap%
\pgfsetmiterjoin%
\pgfsetlinewidth{0.803000pt}%
\definecolor{currentstroke}{rgb}{0.000000,0.000000,0.000000}%
\pgfsetstrokecolor{currentstroke}%
\pgfsetdash{}{0pt}%
\pgfpathmoveto{\pgfqpoint{0.800000in}{0.528000in}}%
\pgfpathlineto{\pgfqpoint{5.760000in}{0.528000in}}%
\pgfusepath{stroke}%
\end{pgfscope}%
\begin{pgfscope}%
\pgfsetrectcap%
\pgfsetmiterjoin%
\pgfsetlinewidth{0.803000pt}%
\definecolor{currentstroke}{rgb}{0.000000,0.000000,0.000000}%
\pgfsetstrokecolor{currentstroke}%
\pgfsetdash{}{0pt}%
\pgfpathmoveto{\pgfqpoint{0.800000in}{4.224000in}}%
\pgfpathlineto{\pgfqpoint{5.760000in}{4.224000in}}%
\pgfusepath{stroke}%
\end{pgfscope}%
\end{pgfpicture}%
\makeatother%
\endgroup%
}
               \caption{Numerical Plot}
               \label{fig:numerical}
            \end{center}
        \end{figure}

        This agrees with our simulation of the solution but not with the analytical solution for $v$, 
        which is strange but most likely a rounding error in the simulation. 
        
    \end{enumerate}

    \section*{Conclusion}
    To conclude, we found that the system we were analysing acts in some ways like we would expect, 
    having a resonance curve that's very regular when looking at the amplitude of oscillations with 
    respect to the velocity of the car. We also found that the velocity at which the amplitude was 
    greatest was $\approx 14.89$.

\end{document}