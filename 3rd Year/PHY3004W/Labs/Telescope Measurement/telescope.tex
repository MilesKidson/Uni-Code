\documentclass[12pt]{article}
\usepackage[margin=1.2in]{geometry}
\usepackage[all]{nowidow}
\usepackage[hyperfigures=true, hidelinks, pdfhighlight=/N]{hyperref}
\usepackage[separate-uncertainty=true,group-digits=false]{siunitx}
\usepackage{graphicx,amsmath,physics,tabto,float,amssymb,pgfplots,verbatim,tcolorbox}
\usepackage{listings,xcolor,subfig,keyval2e,caption,import}
\numberwithin{equation}{section}
\numberwithin{figure}{section}
\numberwithin{table}{section}
\definecolor{stringcolor}{HTML}{C792EA}
\definecolor{codeblue}{HTML}{2162DB}
\definecolor{commentcolor}{HTML}{4A6E46}
\lstdefinestyle{appendix}{
    basicstyle=\ttfamily\footnotesize,commentstyle=\color{commentcolor},keywordstyle=\color{codeblue},
    stringstyle=\color{stringcolor},showstringspaces=false,numbers=left,upquote=true,captionpos=t,
    abovecaptionskip=12pt,belowcaptionskip=12pt,language=Python,breaklines=true,frame=single}
\lstdefinestyle{inline}{
    basicstyle=\ttfamily\footnotesize,commentstyle=\color{commentcolor},keywordstyle=\color{codeblue},
    stringstyle=\color{stringcolor},showstringspaces=false,numbers=left,upquote=true,frame=tb,
    captionpos=b,language=Python}
\renewcommand{\lstlistingname}{Appendix}
\pgfplotsset{compat=1.17}

\title{Telescope Measurement Challenge}
\author{KDSMIL001 \; PHY3004W PHYLAB3}
\date{\textbf{date}}

\begin{document}
    \textbf{Measurement Challenge}\hspace{25pt}\textbf{KDSMIL001 \; PHY3004W PHYLAB3}

    \section{Introduction}\label{sec:Introduction}
    In this report we measure the distance between the two telescopes on the roof of the 
    R.W. James building without going onto the roof of the building ourselves. 

    \section{Apparatus and Method}\label{sec:Apparatus and Method}
    To make this measurement we used a pair of shoes and a ruler. We went to the south side 
    of the building, onto the balcony that runs almost the full width of the building, and 
    approximated that the box the first telescope sits on starts at the same place the wall 
    starts lower down and the box for the second telescope starts at the same place as a line 
    that conveniently runs down the side of the building. Assuming both of these features 
    are constructed with sufficient precision, measuring the distance between them will 
    give us the same result as measuring the distance between the two boxes. Assuming that 
    the telescopes are both situated directly in the centre of their respective boxes, 
    measuring the distance between the start of the boxes is equivalent to measuring the 
    distance between the peaks of the domes of the telescopes. \newline
    We placed one shoe in front of the other for the whole distance we were measuring and 
    then measured the length of the shoe. 

    \section{Results}\label{sec:Results}
    We found the distance to be 75 shoes long and each shoe to be $\SI{30.5}{\cm}$ long. 
    Using this we find the distance between the peaks of the domes of the telescopes to be 
    \begin{equation*}
        \SI{2287.5}{\cm}=\SI{22.875}{\metre}
    \end{equation*}
    The uncertainty for this measurement comes from the uncertainty on the length of the 
    shoe and the uncertainty on the number of shoe-lengths it took to cover the side of 
    the building. On the measurement of the shoe we can use an analogue pdf with our 
    interval $a=\SI{0.5}{\cm}$ to get $u(\text{shoe})=\frac{0.5}{2\sqrt{6}}=0.10$. Since 
    our shoe was the smallest unit we had when measuring the side of the building, we must 
    take that as our interval. So $a=1$ gives us 
    $u(\text{shoe-lengths})=\frac{1}{2\sqrt{6}}=0.20$. This means that our final uncertainty 
    is $\sqrt{(75\cdot 0.1)^2 + 0.2^2}=\SI{7.5}{\cm}$.

    \section{Conclusion}\label{sec:Conclusion}
    We found the distance between the telescopes to be $\SI{22.875\pm0.075}{\metre}$.
    
    

\end{document}