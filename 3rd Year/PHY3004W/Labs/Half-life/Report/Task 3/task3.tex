\documentclass[11pt]{article}
\usepackage[margin=1in, top=0.3in]{geometry}
\usepackage[all]{nowidow}
\usepackage[hyperfigures=true, hidelinks, pdfhighlight=/N]{hyperref}
\usepackage[group-digits=false]{siunitx}
\usepackage{graphicx,amsmath,physics,tabto,float,amssymb,pgfplots,verbatim,tcolorbox}
\usepackage{listings,xcolor,subfig,caption,import,wrapfig,pgf}
\usepackage[version=4]{mhchem}
\numberwithin{equation}{section}
\numberwithin{table}{section}
\definecolor{stringcolor}{HTML}{C792EA}
\definecolor{codeblue}{HTML}{2162DB}
\definecolor{commentcolor}{HTML}{4A6E46}
\lstdefinestyle{appendix}{
    basicstyle=\ttfamily\footnotesize,commentstyle=\color{commentcolor},keywordstyle=\color{codeblue},
    stringstyle=\color{stringcolor},showstringspaces=false,numbers=left,upquote=true,captionpos=t,
    abovecaptionskip=12pt,belowcaptionskip=12pt,language=Python,breaklines=true,frame=single}
\lstdefinestyle{inline}{
    basicstyle=\ttfamily\footnotesize,commentstyle=\color{commentcolor},keywordstyle=\color{codeblue},
    stringstyle=\color{stringcolor},showstringspaces=false,numbers=left,upquote=true,frame=tb,
    captionpos=b,language=Python}
\renewcommand{\lstlistingname}{Appendix}
\pgfplotsset{compat=1.17}

\begin{document}

\begin{center}
    {\textbf{\LARGE Half Life Lab Task 3}}\\
    \vspace{0.2in}
    KDSMIL001 | June 2021
    
\end{center}

\begin{figure}[h]
    \begin{center}
        \subimport{Plots}{task4.pgf}
        \caption{Gamma ray spectrum for \ce{^{28}Al} from NaI scintillator detector at experimental station 1 in PHYLAB3. Energy calibration done using \ce{^{137}Cs}, \ce{^{22}Na}, and \ce{^{60}Co} and data from the Nudat2 database \cite{nudat}. Identification of features done with the help of the Nudat2 database \cite{nudat} and Gilmore (2008)\cite{gilmore}.}
        \label{fig:28Al Gamma Ray Spectrum}
    \end{center}
\end{figure}
% Going from left to right in \autoref{fig:28Al Gamma Ray Spectrum}:
% \begin{itemize}
%     \item Bremsstrahlung: The $\beta^{-}$ particles emitted by \ce{^{28}Al} being detected.
%     \item $\sim 647$ keV Compton edge: The Compton edge for the 843.76 keV photopeak.
%     \item 843.76 keV photopeak: One of the two primary gamma ray emissions from \ce{^{27}Mg} $\beta^{-}$ decay. The \ce{^{27}Mg} comes from a neutron knocking out and replacing a proton in \ce{^{27}Al}.
%     \item 1014.52 keV photopeak: The second primary gamma ray emission from \ce{^{27}Mg}.
%     \item $\sim 1555$ keV Compton edge: The Compton edge for the 1778.987 keV photopeak.
%     \item 1778.987 keV photopeak: The primary gamma ray emission from \ce{^{28}Al}.
% \end{itemize}

\begin{thebibliography}{9}
    \bibitem{nudat}
    National Nuclear Data Center, Brookhaven National Laboratory. (2008, March 18). NuDat (Nuclear Structure and Decay Data). https://www.nndc.bnl.gov/nudat2/
    \bibitem{gilmore}
    Gordon Gilmore. (2008). \textit{Practical Gamma-ray Spectrometry}. Chichester: Wiley
\end{thebibliography}

\end{document}