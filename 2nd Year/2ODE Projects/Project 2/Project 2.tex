\documentclass[12pt]{article}
\usepackage[margin=1.2in]{geometry}
\usepackage[all]{nowidow}
\usepackage[hyperfigures=true, hidelinks, pdfhighlight=/N]{hyperref}
\usepackage{graphicx,amsmath,physics,tabto,float,amssymb,pgfplots,verbatim,tcolorbox}
\usepackage{listings,xcolor,siunitx,subfig,keyval2e,caption,cancel}
\definecolor{stringcolor}{HTML}{C792EA}
\definecolor{codeblue}{HTML}{2162DB}
\definecolor{commentcolor}{HTML}{4A6E46}
\lstdefinestyle{appendix}{
    basicstyle=\ttfamily\footnotesize,commentstyle=\color{commentcolor},keywordstyle=\color{codeblue},
    stringstyle=\color{stringcolor},showstringspaces=false,numbers=left,upquote=true,captionpos=t,
    abovecaptionskip=12pt,belowcaptionskip=12pt,language=Python,breaklines=true,frame=single}
\lstdefinestyle{inline}{
    basicstyle=\ttfamily\footnotesize,commentstyle=\color{commentcolor},keywordstyle=\color{codeblue},
    stringstyle=\color{stringcolor},showstringspaces=false,numbers=left,upquote=true,frame=tb,
    captionpos=b,language=Python}
\renewcommand{\lstlistingname}{Appendix}
\pgfplotsset{compat=1.17}

\title{The Man Who Flew Into Space}
\date{\textbf{14 May 2020}}
\author{}

\begin{document}

    \maketitle
    \begin{center}
    \textbf{\large{MAM2046W 2OD}}\\
    \textbf{\large{EmplID: 1669971}}\\
    \end{center}

    \section{Abstract}
    In this project our aim is to compare the approaches of two people swinging on a swing set, regarding 
    their aim of achieving growth in amplitudes, one aiming for exponential growth, the other for 
    linear. The motion of a person on a swing driving themselves can be modelled with a differential 
    equation, which we can analyse and work out how each person needs to tune their approach in order 
    to obtain the kind of growth they are looking for.

    \begin{enumerate}
        \item \textbf{Analysis} \newline
        \begin{enumerate}
            \item We aim to analyse the behaviour of the differential equation 
            
            \begin{equation}
                \frac{d^2\theta}{d\tau^2}+\nu \theta +\epsilon\cos(2\tau)\theta = 0
                \label{eqn:Main Differential Equation}
            \end{equation}

            for $\nu \approx 1$ and determine if this $\theta$ will grow exponentially or linearly. 
            To do this we use an expansion of $\nu = \nu_0 +\epsilon\nu_1+\epsilon^2\nu_2\dots$ 
            where in this case $\nu_0 = 1$, as well as the method of multiple time scales, to find 
            the leading order solution to \autoref{eqn:Main Differential Equation}. The amplitude of 
            this solution will show us the behaviour of this system. \newline
            \newline
            First, we define an operator
            \begin{equation*}
                \begin{split}
                    \frac{d}{d\tau} &= (D_0+\epsilon D_1+\epsilon^2 D_2+\dots) \\
                    \implies \frac{d^2}{d\tau^2} &= (D_0+\epsilon D_1+\epsilon^2 D_2+\dots) \\
                    &= D_0^2 + 2\epsilon D_0D_1 +\epsilon^2(D_1^2+2D_0D_2)+\dots
                \end{split}
            \end{equation*}
            where $D_n = \frac{\partial}{\partial T_n}$ and $T_0=\tau, T_1=\epsilon\tau, T_2=\epsilon^2\tau$ etc. 
            We can also expand $\theta = \theta_0+\epsilon\theta_1+\epsilon^2\theta_2+\dots$ and then we 
            can rewrite \autoref{eqn:Main Differential Equation} as 
            \begin{equation}
                \begin{gathered}
                    (D_0^2 + 2\epsilon D_0D_1 +\dots)(\theta_0+\epsilon\theta_1+\dots)+(\nu_0 +\epsilon\nu_1+\dots)(\theta_0+\epsilon\theta_1+\dots) \\
                    +\epsilon\cos(2\tau)(\theta_0+\epsilon\theta_1+\dots) = 0
                \end{gathered}
                \label{eqn:Expanded Main Differential Equation A}
            \end{equation}

            We can then multiply these brackets out and set coefficients of powers of $\epsilon$ to 0, 
            starting with $\epsilon^0$, which gives us
            \begin{equation*}
                D_0^2\theta_0 + \theta_0\cancelto{1}{\nu_1} = 0
            \end{equation*}
            which has the solution $\theta_0 = Ae^{iT_0}+A^*e^{-iT_0}$ where $A^*$ is the 
            complex conjugate of $A$. Now the coefficients of $\epsilon^1$:
            \begin{equation*}
                \begin{split}    
                    D_0^2 \theta_1 +\theta_1 =& -2D_0D_1\theta_0-\cos(2T_0)\theta_0-\nu_1\theta_0 \\
                    =& -2D_0D_1(Ae^{iT_0}+A^*e^{-iT_0})-\cos(2T_0)(Ae^{iT_0}+A^*e^{-iT_0}) \\
                    &-\nu_1(Ae^{iT_0}+A^*e^{-iT_0}) \\
                    =& -2D_0D_1(Ae^{iT_0}+A^*e^{-iT_0})-\left(\frac{e^{2iT_0}+e^{-2iT_0}}{2}\right)(Ae^{iT_0}+A^*e^{-iT_0}) \\
                    &-\nu_1(Ae^{iT_0}+A^*e^{-iT_0}) \\
                    =& -2D_1(iAe^{iT_0}-iA^*e^{-iT_0})-\nu_1(Ae^{iT_0}+A^*e^{-iT_0}) \\
                    &-\frac{1}{2}(Ae^{3iT_0}+Ae^{-iT_0}+A^*e^{iT_0}+A^*e^{-3iT_0})
                \end{split}
            \end{equation*}
            At this point, we kill the secular terms, those terms with frequency 1, by setting each 
            of them to 0:
            \begin{equation}
                \begin{split}
                    -2iD_1A-\nu_1A-\frac{1}{2}A^* = 0 \\
                    2iD_1A^*-\nu_1A^*-\frac{1}{2}A = 0
                \end{split}
                \label{eqn:SecularA}
            \end{equation}

            We can solve this with an Ansatz. If we guess $A=a+ib; A^*=a-ib$ and substitute in, 
            we end up with 
            \begin{equation*}
                \begin{split}
                    -2iD_1a+2D_1b-\nu_1(a+ib)-\frac{1}{2}(a-ib)=0 \\
                    2iD_1a+2D_1b-\nu_1(a-ib)-\frac{1}{2}(a+ib)=0 \\
                    \implies 2D_1b-i2D_1a=a(\nu_1+\frac{1}{2})+ib(\nu_1-\frac{1}{2}) \\
                    2D_1b+i2D_1a=a(\nu_1+\frac{1}{2})+ib(\frac{1}{2}-\nu_1)
                \end{split}
            \end{equation*}
            Now we can equate the real and imaginary parts of each equation, leaving us with 
            \begin{equation*}
                \begin{split}
                    D_1b&=\frac{a}{2}(\nu_1+\frac{1}{2}) \\
                    D_1b&=\frac{a}{2}(\nu_1+\frac{1}{2}) \\
                    D_1a&=\frac{b}{2}(\frac{1}{2}-\nu_1) \\
                    D_1a&=\frac{b}{2}(\frac{1}{2}-\nu_1) \\
                    \implies a&=(\frac{1}{4}-\frac{\nu_1}{2})\int b dT_1 \\
                    b&=(\frac{\nu_1}{2}+\frac{1}{4})\int a dT_1
                \end{split}
            \end{equation*}
            we can solve for $a$ and $b$
            \begin{equation*}
                \begin{split}
                    D_1a&=(\frac{1}{16}-\frac{\nu_1^2}{4})\int a dT_1 \hspace{15pt} | \frac{\partial}{\partial T_1} \\
                    \implies D_1^2a&=(\frac{1}{16}-\frac{\nu_1^2}{4})a \\
                    \implies 0&= D_1^2a + (\frac{\nu_1^2}{4}-\frac{1}{16})a
                \end{split}
            \end{equation*}
            We can solve this with an Ansatz of $a = e^{kT_1}$
            \begin{equation*}
                \begin{split}
                    \implies k^2 &= (\frac{\nu_1^2}{4}-\frac{1}{16}) \\
                    \implies k &= \pm i\sqrt{(\frac{\nu_1^2}{4}-\frac{1}{16})} \\
                    \implies a &= C e^{\frac{i}{4}\sqrt{4\nu_1^2-1}T_1} + C^* e^{\frac{-i}{4}\sqrt{4\nu_1^2-1}T_1} \\
                    \implies b &= \frac{i(2\nu_1+1)}{\sqrt{4\nu_1^2-1}}\left[C^*e^{\frac{-i}{4}\sqrt{4\nu_1^2-1}T_1}-Ce^{\frac{i}{4}\sqrt{1-4\nu_1^2}T_1}\right]
                \end{split}
            \end{equation*}
            Importantly, both $a$ and $b$ need to be real in order for $A$ and $A^*$ to be complex 
            conjugates. Luckily for us, given a complex number $z=x+iy$ and its conjugate $z^*x-iy$, 
            $z+z^*=2x$ and $z-z^*=2iy$ and so $a$ is entirely real and the bracket in $b$ is entirely 
            imaginary, which means $b$ is also entirely real.
            so we have found
            \begin{equation*}
                \begin{split}
                    A = C e^{\frac{i}{4}\sqrt{4\nu_1^2-1}T_1} + C^* e^{\frac{-i}{4}\sqrt{4\nu_1^2-1}T_1} - \frac{(2\nu_1+1)}{\sqrt{4\nu_1^2-1}}\left[C^*e^{\frac{-i}{4}\sqrt{4\nu_1^2-1}T_1}-Ce^{\frac{i}{4}\sqrt{4\nu_1^2-1}T_1}\right] \\
                    A^* = C e^{\frac{i}{4}\sqrt{4\nu_1^2-1}T_1} + C^* e^{\frac{-i}{4}\sqrt{4\nu_1^2-1}T_1} + \frac{(2\nu_1+1)}{\sqrt{4\nu_1^2-1}}\left[C^*e^{\frac{-i}{4}\sqrt{4\nu_1^2-1}T_1}-Ce^{\frac{i}{4}\sqrt{4\nu_1^2-1}T_1}\right]
                \end{split}
            \end{equation*}
            and finally
            \begin{equation}
                \begin{gathered}
                    \theta_0 = e^{iT_0}\left(C e^{\frac{i}{4}\sqrt{4\nu_1^2-1}T_1} + C^* e^{\frac{-i}{4}\sqrt{4\nu_1^2-1}T_1} - \frac{(2\nu_1+1)}{\sqrt{4\nu_1^2-1}}\left[C^*e^{\frac{-i}{4}\sqrt{4\nu_1^2-1}T_1}-Ce^{\frac{i}{4}\sqrt{4\nu_1^2-1}T_1}\right]\right) \\
                    + e^{iT_0}\left(C e^{\frac{i}{4}\sqrt{4\nu_1^2-1}T_1} + C^* e^{\frac{-i}{4}\sqrt{4\nu_1^2-1}T_1} + \frac{(2\nu_1+1)}{\sqrt{4\nu_1^2-1}}\left[C^*e^{\frac{-i}{4}\sqrt{4\nu_1^2-1}T_1}-Ce^{\frac{i}{4}\sqrt{4\nu_1^2-1}T_1}\right]\right)
                \end{gathered}
                \label{eqn:Leading Term A}
            \end{equation}

            From our values of $A$ and $A^*$, our amplitudes, we can tell that $\theta$' growth will be governed 
            by the $\sqrt{4\nu_1^2-1}$ term. If $\nu_1^2>\frac{1}{4}$, the motion will be purely oscillatory 
            as the exponential term will be entirely imaginary. If $\nu_1^2<\frac{1}{4}$, the exponential 
            term will be entirely real and thus the amplitude will grow exponentially. Thus we can return to 
            our expansion of $\nu$ and find, in order to have exponential growth,
            \begin{equation*}
                \begin{gathered}
                    \nu < 1+\frac{\epsilon}{2}; \nu > 1-\frac{\epsilon}{2} \\
                    \implies \epsilon > |2\nu -2|
                \end{gathered}
            \end{equation*}
            This gives us \autoref{fig:Nu1Analytical} below
            \begin{figure}[H]
                \begin{center}
                   \scalebox{.7}{%% Creator: Matplotlib, PGF backend
%%
%% To include the figure in your LaTeX document, write
%%   \input{<filename>.pgf}
%%
%% Make sure the required packages are loaded in your preamble
%%   \usepackage{pgf}
%%
%% Figures using additional raster images can only be included by \input if
%% they are in the same directory as the main LaTeX file. For loading figures
%% from other directories you can use the `import` package
%%   \usepackage{import}
%% and then include the figures with
%%   \import{<path to file>}{<filename>.pgf}
%%
%% Matplotlib used the following preamble
%%
\begingroup%
\makeatletter%
\begin{pgfpicture}%
\pgfpathrectangle{\pgfpointorigin}{\pgfqpoint{6.400000in}{4.800000in}}%
\pgfusepath{use as bounding box, clip}%
\begin{pgfscope}%
\pgfsetbuttcap%
\pgfsetmiterjoin%
\definecolor{currentfill}{rgb}{1.000000,1.000000,1.000000}%
\pgfsetfillcolor{currentfill}%
\pgfsetlinewidth{0.000000pt}%
\definecolor{currentstroke}{rgb}{1.000000,1.000000,1.000000}%
\pgfsetstrokecolor{currentstroke}%
\pgfsetdash{}{0pt}%
\pgfpathmoveto{\pgfqpoint{0.000000in}{0.000000in}}%
\pgfpathlineto{\pgfqpoint{6.400000in}{0.000000in}}%
\pgfpathlineto{\pgfqpoint{6.400000in}{4.800000in}}%
\pgfpathlineto{\pgfqpoint{0.000000in}{4.800000in}}%
\pgfpathclose%
\pgfusepath{fill}%
\end{pgfscope}%
\begin{pgfscope}%
\pgfsetbuttcap%
\pgfsetmiterjoin%
\definecolor{currentfill}{rgb}{1.000000,1.000000,1.000000}%
\pgfsetfillcolor{currentfill}%
\pgfsetlinewidth{0.000000pt}%
\definecolor{currentstroke}{rgb}{0.000000,0.000000,0.000000}%
\pgfsetstrokecolor{currentstroke}%
\pgfsetstrokeopacity{0.000000}%
\pgfsetdash{}{0pt}%
\pgfpathmoveto{\pgfqpoint{0.800000in}{0.528000in}}%
\pgfpathlineto{\pgfqpoint{5.760000in}{0.528000in}}%
\pgfpathlineto{\pgfqpoint{5.760000in}{4.224000in}}%
\pgfpathlineto{\pgfqpoint{0.800000in}{4.224000in}}%
\pgfpathclose%
\pgfusepath{fill}%
\end{pgfscope}%
\begin{pgfscope}%
\pgfsetbuttcap%
\pgfsetroundjoin%
\definecolor{currentfill}{rgb}{0.000000,0.000000,0.000000}%
\pgfsetfillcolor{currentfill}%
\pgfsetlinewidth{0.803000pt}%
\definecolor{currentstroke}{rgb}{0.000000,0.000000,0.000000}%
\pgfsetstrokecolor{currentstroke}%
\pgfsetdash{}{0pt}%
\pgfsys@defobject{currentmarker}{\pgfqpoint{0.000000in}{-0.048611in}}{\pgfqpoint{0.000000in}{0.000000in}}{%
\pgfpathmoveto{\pgfqpoint{0.000000in}{0.000000in}}%
\pgfpathlineto{\pgfqpoint{0.000000in}{-0.048611in}}%
\pgfusepath{stroke,fill}%
}%
\begin{pgfscope}%
\pgfsys@transformshift{1.025455in}{0.528000in}%
\pgfsys@useobject{currentmarker}{}%
\end{pgfscope}%
\end{pgfscope}%
\begin{pgfscope}%
\definecolor{textcolor}{rgb}{0.000000,0.000000,0.000000}%
\pgfsetstrokecolor{textcolor}%
\pgfsetfillcolor{textcolor}%
\pgftext[x=1.025455in,y=0.430778in,,top]{\color{textcolor}\rmfamily\fontsize{10.000000}{12.000000}\selectfont \(\displaystyle 0.7\)}%
\end{pgfscope}%
\begin{pgfscope}%
\pgfsetbuttcap%
\pgfsetroundjoin%
\definecolor{currentfill}{rgb}{0.000000,0.000000,0.000000}%
\pgfsetfillcolor{currentfill}%
\pgfsetlinewidth{0.803000pt}%
\definecolor{currentstroke}{rgb}{0.000000,0.000000,0.000000}%
\pgfsetstrokecolor{currentstroke}%
\pgfsetdash{}{0pt}%
\pgfsys@defobject{currentmarker}{\pgfqpoint{0.000000in}{-0.048611in}}{\pgfqpoint{0.000000in}{0.000000in}}{%
\pgfpathmoveto{\pgfqpoint{0.000000in}{0.000000in}}%
\pgfpathlineto{\pgfqpoint{0.000000in}{-0.048611in}}%
\pgfusepath{stroke,fill}%
}%
\begin{pgfscope}%
\pgfsys@transformshift{1.776970in}{0.528000in}%
\pgfsys@useobject{currentmarker}{}%
\end{pgfscope}%
\end{pgfscope}%
\begin{pgfscope}%
\definecolor{textcolor}{rgb}{0.000000,0.000000,0.000000}%
\pgfsetstrokecolor{textcolor}%
\pgfsetfillcolor{textcolor}%
\pgftext[x=1.776970in,y=0.430778in,,top]{\color{textcolor}\rmfamily\fontsize{10.000000}{12.000000}\selectfont \(\displaystyle 0.8\)}%
\end{pgfscope}%
\begin{pgfscope}%
\pgfsetbuttcap%
\pgfsetroundjoin%
\definecolor{currentfill}{rgb}{0.000000,0.000000,0.000000}%
\pgfsetfillcolor{currentfill}%
\pgfsetlinewidth{0.803000pt}%
\definecolor{currentstroke}{rgb}{0.000000,0.000000,0.000000}%
\pgfsetstrokecolor{currentstroke}%
\pgfsetdash{}{0pt}%
\pgfsys@defobject{currentmarker}{\pgfqpoint{0.000000in}{-0.048611in}}{\pgfqpoint{0.000000in}{0.000000in}}{%
\pgfpathmoveto{\pgfqpoint{0.000000in}{0.000000in}}%
\pgfpathlineto{\pgfqpoint{0.000000in}{-0.048611in}}%
\pgfusepath{stroke,fill}%
}%
\begin{pgfscope}%
\pgfsys@transformshift{2.528485in}{0.528000in}%
\pgfsys@useobject{currentmarker}{}%
\end{pgfscope}%
\end{pgfscope}%
\begin{pgfscope}%
\definecolor{textcolor}{rgb}{0.000000,0.000000,0.000000}%
\pgfsetstrokecolor{textcolor}%
\pgfsetfillcolor{textcolor}%
\pgftext[x=2.528485in,y=0.430778in,,top]{\color{textcolor}\rmfamily\fontsize{10.000000}{12.000000}\selectfont \(\displaystyle 0.9\)}%
\end{pgfscope}%
\begin{pgfscope}%
\pgfsetbuttcap%
\pgfsetroundjoin%
\definecolor{currentfill}{rgb}{0.000000,0.000000,0.000000}%
\pgfsetfillcolor{currentfill}%
\pgfsetlinewidth{0.803000pt}%
\definecolor{currentstroke}{rgb}{0.000000,0.000000,0.000000}%
\pgfsetstrokecolor{currentstroke}%
\pgfsetdash{}{0pt}%
\pgfsys@defobject{currentmarker}{\pgfqpoint{0.000000in}{-0.048611in}}{\pgfqpoint{0.000000in}{0.000000in}}{%
\pgfpathmoveto{\pgfqpoint{0.000000in}{0.000000in}}%
\pgfpathlineto{\pgfqpoint{0.000000in}{-0.048611in}}%
\pgfusepath{stroke,fill}%
}%
\begin{pgfscope}%
\pgfsys@transformshift{3.280000in}{0.528000in}%
\pgfsys@useobject{currentmarker}{}%
\end{pgfscope}%
\end{pgfscope}%
\begin{pgfscope}%
\definecolor{textcolor}{rgb}{0.000000,0.000000,0.000000}%
\pgfsetstrokecolor{textcolor}%
\pgfsetfillcolor{textcolor}%
\pgftext[x=3.280000in,y=0.430778in,,top]{\color{textcolor}\rmfamily\fontsize{10.000000}{12.000000}\selectfont \(\displaystyle 1.0\)}%
\end{pgfscope}%
\begin{pgfscope}%
\pgfsetbuttcap%
\pgfsetroundjoin%
\definecolor{currentfill}{rgb}{0.000000,0.000000,0.000000}%
\pgfsetfillcolor{currentfill}%
\pgfsetlinewidth{0.803000pt}%
\definecolor{currentstroke}{rgb}{0.000000,0.000000,0.000000}%
\pgfsetstrokecolor{currentstroke}%
\pgfsetdash{}{0pt}%
\pgfsys@defobject{currentmarker}{\pgfqpoint{0.000000in}{-0.048611in}}{\pgfqpoint{0.000000in}{0.000000in}}{%
\pgfpathmoveto{\pgfqpoint{0.000000in}{0.000000in}}%
\pgfpathlineto{\pgfqpoint{0.000000in}{-0.048611in}}%
\pgfusepath{stroke,fill}%
}%
\begin{pgfscope}%
\pgfsys@transformshift{4.031515in}{0.528000in}%
\pgfsys@useobject{currentmarker}{}%
\end{pgfscope}%
\end{pgfscope}%
\begin{pgfscope}%
\definecolor{textcolor}{rgb}{0.000000,0.000000,0.000000}%
\pgfsetstrokecolor{textcolor}%
\pgfsetfillcolor{textcolor}%
\pgftext[x=4.031515in,y=0.430778in,,top]{\color{textcolor}\rmfamily\fontsize{10.000000}{12.000000}\selectfont \(\displaystyle 1.1\)}%
\end{pgfscope}%
\begin{pgfscope}%
\pgfsetbuttcap%
\pgfsetroundjoin%
\definecolor{currentfill}{rgb}{0.000000,0.000000,0.000000}%
\pgfsetfillcolor{currentfill}%
\pgfsetlinewidth{0.803000pt}%
\definecolor{currentstroke}{rgb}{0.000000,0.000000,0.000000}%
\pgfsetstrokecolor{currentstroke}%
\pgfsetdash{}{0pt}%
\pgfsys@defobject{currentmarker}{\pgfqpoint{0.000000in}{-0.048611in}}{\pgfqpoint{0.000000in}{0.000000in}}{%
\pgfpathmoveto{\pgfqpoint{0.000000in}{0.000000in}}%
\pgfpathlineto{\pgfqpoint{0.000000in}{-0.048611in}}%
\pgfusepath{stroke,fill}%
}%
\begin{pgfscope}%
\pgfsys@transformshift{4.783030in}{0.528000in}%
\pgfsys@useobject{currentmarker}{}%
\end{pgfscope}%
\end{pgfscope}%
\begin{pgfscope}%
\definecolor{textcolor}{rgb}{0.000000,0.000000,0.000000}%
\pgfsetstrokecolor{textcolor}%
\pgfsetfillcolor{textcolor}%
\pgftext[x=4.783030in,y=0.430778in,,top]{\color{textcolor}\rmfamily\fontsize{10.000000}{12.000000}\selectfont \(\displaystyle 1.2\)}%
\end{pgfscope}%
\begin{pgfscope}%
\pgfsetbuttcap%
\pgfsetroundjoin%
\definecolor{currentfill}{rgb}{0.000000,0.000000,0.000000}%
\pgfsetfillcolor{currentfill}%
\pgfsetlinewidth{0.803000pt}%
\definecolor{currentstroke}{rgb}{0.000000,0.000000,0.000000}%
\pgfsetstrokecolor{currentstroke}%
\pgfsetdash{}{0pt}%
\pgfsys@defobject{currentmarker}{\pgfqpoint{0.000000in}{-0.048611in}}{\pgfqpoint{0.000000in}{0.000000in}}{%
\pgfpathmoveto{\pgfqpoint{0.000000in}{0.000000in}}%
\pgfpathlineto{\pgfqpoint{0.000000in}{-0.048611in}}%
\pgfusepath{stroke,fill}%
}%
\begin{pgfscope}%
\pgfsys@transformshift{5.534545in}{0.528000in}%
\pgfsys@useobject{currentmarker}{}%
\end{pgfscope}%
\end{pgfscope}%
\begin{pgfscope}%
\definecolor{textcolor}{rgb}{0.000000,0.000000,0.000000}%
\pgfsetstrokecolor{textcolor}%
\pgfsetfillcolor{textcolor}%
\pgftext[x=5.534545in,y=0.430778in,,top]{\color{textcolor}\rmfamily\fontsize{10.000000}{12.000000}\selectfont \(\displaystyle 1.3\)}%
\end{pgfscope}%
\begin{pgfscope}%
\definecolor{textcolor}{rgb}{0.000000,0.000000,0.000000}%
\pgfsetstrokecolor{textcolor}%
\pgfsetfillcolor{textcolor}%
\pgftext[x=3.280000in,y=0.251766in,,top]{\color{textcolor}\rmfamily\fontsize{10.000000}{12.000000}\selectfont \(\displaystyle \nu\)}%
\end{pgfscope}%
\begin{pgfscope}%
\pgfsetbuttcap%
\pgfsetroundjoin%
\definecolor{currentfill}{rgb}{0.000000,0.000000,0.000000}%
\pgfsetfillcolor{currentfill}%
\pgfsetlinewidth{0.803000pt}%
\definecolor{currentstroke}{rgb}{0.000000,0.000000,0.000000}%
\pgfsetstrokecolor{currentstroke}%
\pgfsetdash{}{0pt}%
\pgfsys@defobject{currentmarker}{\pgfqpoint{-0.048611in}{0.000000in}}{\pgfqpoint{0.000000in}{0.000000in}}{%
\pgfpathmoveto{\pgfqpoint{0.000000in}{0.000000in}}%
\pgfpathlineto{\pgfqpoint{-0.048611in}{0.000000in}}%
\pgfusepath{stroke,fill}%
}%
\begin{pgfscope}%
\pgfsys@transformshift{0.800000in}{0.692633in}%
\pgfsys@useobject{currentmarker}{}%
\end{pgfscope}%
\end{pgfscope}%
\begin{pgfscope}%
\definecolor{textcolor}{rgb}{0.000000,0.000000,0.000000}%
\pgfsetstrokecolor{textcolor}%
\pgfsetfillcolor{textcolor}%
\pgftext[x=0.525308in,y=0.644408in,left,base]{\color{textcolor}\rmfamily\fontsize{10.000000}{12.000000}\selectfont \(\displaystyle 0.0\)}%
\end{pgfscope}%
\begin{pgfscope}%
\pgfsetbuttcap%
\pgfsetroundjoin%
\definecolor{currentfill}{rgb}{0.000000,0.000000,0.000000}%
\pgfsetfillcolor{currentfill}%
\pgfsetlinewidth{0.803000pt}%
\definecolor{currentstroke}{rgb}{0.000000,0.000000,0.000000}%
\pgfsetstrokecolor{currentstroke}%
\pgfsetdash{}{0pt}%
\pgfsys@defobject{currentmarker}{\pgfqpoint{-0.048611in}{0.000000in}}{\pgfqpoint{0.000000in}{0.000000in}}{%
\pgfpathmoveto{\pgfqpoint{0.000000in}{0.000000in}}%
\pgfpathlineto{\pgfqpoint{-0.048611in}{0.000000in}}%
\pgfusepath{stroke,fill}%
}%
\begin{pgfscope}%
\pgfsys@transformshift{0.800000in}{1.253194in}%
\pgfsys@useobject{currentmarker}{}%
\end{pgfscope}%
\end{pgfscope}%
\begin{pgfscope}%
\definecolor{textcolor}{rgb}{0.000000,0.000000,0.000000}%
\pgfsetstrokecolor{textcolor}%
\pgfsetfillcolor{textcolor}%
\pgftext[x=0.525308in,y=1.204969in,left,base]{\color{textcolor}\rmfamily\fontsize{10.000000}{12.000000}\selectfont \(\displaystyle 0.1\)}%
\end{pgfscope}%
\begin{pgfscope}%
\pgfsetbuttcap%
\pgfsetroundjoin%
\definecolor{currentfill}{rgb}{0.000000,0.000000,0.000000}%
\pgfsetfillcolor{currentfill}%
\pgfsetlinewidth{0.803000pt}%
\definecolor{currentstroke}{rgb}{0.000000,0.000000,0.000000}%
\pgfsetstrokecolor{currentstroke}%
\pgfsetdash{}{0pt}%
\pgfsys@defobject{currentmarker}{\pgfqpoint{-0.048611in}{0.000000in}}{\pgfqpoint{0.000000in}{0.000000in}}{%
\pgfpathmoveto{\pgfqpoint{0.000000in}{0.000000in}}%
\pgfpathlineto{\pgfqpoint{-0.048611in}{0.000000in}}%
\pgfusepath{stroke,fill}%
}%
\begin{pgfscope}%
\pgfsys@transformshift{0.800000in}{1.813756in}%
\pgfsys@useobject{currentmarker}{}%
\end{pgfscope}%
\end{pgfscope}%
\begin{pgfscope}%
\definecolor{textcolor}{rgb}{0.000000,0.000000,0.000000}%
\pgfsetstrokecolor{textcolor}%
\pgfsetfillcolor{textcolor}%
\pgftext[x=0.525308in,y=1.765530in,left,base]{\color{textcolor}\rmfamily\fontsize{10.000000}{12.000000}\selectfont \(\displaystyle 0.2\)}%
\end{pgfscope}%
\begin{pgfscope}%
\pgfsetbuttcap%
\pgfsetroundjoin%
\definecolor{currentfill}{rgb}{0.000000,0.000000,0.000000}%
\pgfsetfillcolor{currentfill}%
\pgfsetlinewidth{0.803000pt}%
\definecolor{currentstroke}{rgb}{0.000000,0.000000,0.000000}%
\pgfsetstrokecolor{currentstroke}%
\pgfsetdash{}{0pt}%
\pgfsys@defobject{currentmarker}{\pgfqpoint{-0.048611in}{0.000000in}}{\pgfqpoint{0.000000in}{0.000000in}}{%
\pgfpathmoveto{\pgfqpoint{0.000000in}{0.000000in}}%
\pgfpathlineto{\pgfqpoint{-0.048611in}{0.000000in}}%
\pgfusepath{stroke,fill}%
}%
\begin{pgfscope}%
\pgfsys@transformshift{0.800000in}{2.374317in}%
\pgfsys@useobject{currentmarker}{}%
\end{pgfscope}%
\end{pgfscope}%
\begin{pgfscope}%
\definecolor{textcolor}{rgb}{0.000000,0.000000,0.000000}%
\pgfsetstrokecolor{textcolor}%
\pgfsetfillcolor{textcolor}%
\pgftext[x=0.525308in,y=2.326091in,left,base]{\color{textcolor}\rmfamily\fontsize{10.000000}{12.000000}\selectfont \(\displaystyle 0.3\)}%
\end{pgfscope}%
\begin{pgfscope}%
\pgfsetbuttcap%
\pgfsetroundjoin%
\definecolor{currentfill}{rgb}{0.000000,0.000000,0.000000}%
\pgfsetfillcolor{currentfill}%
\pgfsetlinewidth{0.803000pt}%
\definecolor{currentstroke}{rgb}{0.000000,0.000000,0.000000}%
\pgfsetstrokecolor{currentstroke}%
\pgfsetdash{}{0pt}%
\pgfsys@defobject{currentmarker}{\pgfqpoint{-0.048611in}{0.000000in}}{\pgfqpoint{0.000000in}{0.000000in}}{%
\pgfpathmoveto{\pgfqpoint{0.000000in}{0.000000in}}%
\pgfpathlineto{\pgfqpoint{-0.048611in}{0.000000in}}%
\pgfusepath{stroke,fill}%
}%
\begin{pgfscope}%
\pgfsys@transformshift{0.800000in}{2.934878in}%
\pgfsys@useobject{currentmarker}{}%
\end{pgfscope}%
\end{pgfscope}%
\begin{pgfscope}%
\definecolor{textcolor}{rgb}{0.000000,0.000000,0.000000}%
\pgfsetstrokecolor{textcolor}%
\pgfsetfillcolor{textcolor}%
\pgftext[x=0.525308in,y=2.886652in,left,base]{\color{textcolor}\rmfamily\fontsize{10.000000}{12.000000}\selectfont \(\displaystyle 0.4\)}%
\end{pgfscope}%
\begin{pgfscope}%
\pgfsetbuttcap%
\pgfsetroundjoin%
\definecolor{currentfill}{rgb}{0.000000,0.000000,0.000000}%
\pgfsetfillcolor{currentfill}%
\pgfsetlinewidth{0.803000pt}%
\definecolor{currentstroke}{rgb}{0.000000,0.000000,0.000000}%
\pgfsetstrokecolor{currentstroke}%
\pgfsetdash{}{0pt}%
\pgfsys@defobject{currentmarker}{\pgfqpoint{-0.048611in}{0.000000in}}{\pgfqpoint{0.000000in}{0.000000in}}{%
\pgfpathmoveto{\pgfqpoint{0.000000in}{0.000000in}}%
\pgfpathlineto{\pgfqpoint{-0.048611in}{0.000000in}}%
\pgfusepath{stroke,fill}%
}%
\begin{pgfscope}%
\pgfsys@transformshift{0.800000in}{3.495439in}%
\pgfsys@useobject{currentmarker}{}%
\end{pgfscope}%
\end{pgfscope}%
\begin{pgfscope}%
\definecolor{textcolor}{rgb}{0.000000,0.000000,0.000000}%
\pgfsetstrokecolor{textcolor}%
\pgfsetfillcolor{textcolor}%
\pgftext[x=0.525308in,y=3.447214in,left,base]{\color{textcolor}\rmfamily\fontsize{10.000000}{12.000000}\selectfont \(\displaystyle 0.5\)}%
\end{pgfscope}%
\begin{pgfscope}%
\pgfsetbuttcap%
\pgfsetroundjoin%
\definecolor{currentfill}{rgb}{0.000000,0.000000,0.000000}%
\pgfsetfillcolor{currentfill}%
\pgfsetlinewidth{0.803000pt}%
\definecolor{currentstroke}{rgb}{0.000000,0.000000,0.000000}%
\pgfsetstrokecolor{currentstroke}%
\pgfsetdash{}{0pt}%
\pgfsys@defobject{currentmarker}{\pgfqpoint{-0.048611in}{0.000000in}}{\pgfqpoint{0.000000in}{0.000000in}}{%
\pgfpathmoveto{\pgfqpoint{0.000000in}{0.000000in}}%
\pgfpathlineto{\pgfqpoint{-0.048611in}{0.000000in}}%
\pgfusepath{stroke,fill}%
}%
\begin{pgfscope}%
\pgfsys@transformshift{0.800000in}{4.056000in}%
\pgfsys@useobject{currentmarker}{}%
\end{pgfscope}%
\end{pgfscope}%
\begin{pgfscope}%
\definecolor{textcolor}{rgb}{0.000000,0.000000,0.000000}%
\pgfsetstrokecolor{textcolor}%
\pgfsetfillcolor{textcolor}%
\pgftext[x=0.525308in,y=4.007775in,left,base]{\color{textcolor}\rmfamily\fontsize{10.000000}{12.000000}\selectfont \(\displaystyle 0.6\)}%
\end{pgfscope}%
\begin{pgfscope}%
\definecolor{textcolor}{rgb}{0.000000,0.000000,0.000000}%
\pgfsetstrokecolor{textcolor}%
\pgfsetfillcolor{textcolor}%
\pgftext[x=0.469752in,y=2.376000in,,bottom]{\color{textcolor}\rmfamily\fontsize{10.000000}{12.000000}\selectfont \(\displaystyle \epsilon\)}%
\end{pgfscope}%
\begin{pgfscope}%
\pgfpathrectangle{\pgfqpoint{0.800000in}{0.528000in}}{\pgfqpoint{4.960000in}{3.696000in}}%
\pgfusepath{clip}%
\pgfsetrectcap%
\pgfsetroundjoin%
\pgfsetlinewidth{1.505625pt}%
\definecolor{currentstroke}{rgb}{1.000000,0.000000,0.000000}%
\pgfsetstrokecolor{currentstroke}%
\pgfsetdash{}{0pt}%
\pgfpathmoveto{\pgfqpoint{1.025455in}{4.056000in}}%
\pgfpathlineto{\pgfqpoint{3.277743in}{0.696000in}}%
\pgfpathlineto{\pgfqpoint{3.282257in}{0.696000in}}%
\pgfpathlineto{\pgfqpoint{5.534545in}{4.056000in}}%
\pgfpathlineto{\pgfqpoint{5.534545in}{4.056000in}}%
\pgfusepath{stroke}%
\end{pgfscope}%
\begin{pgfscope}%
\pgfsetrectcap%
\pgfsetmiterjoin%
\pgfsetlinewidth{0.803000pt}%
\definecolor{currentstroke}{rgb}{0.000000,0.000000,0.000000}%
\pgfsetstrokecolor{currentstroke}%
\pgfsetdash{}{0pt}%
\pgfpathmoveto{\pgfqpoint{0.800000in}{0.528000in}}%
\pgfpathlineto{\pgfqpoint{0.800000in}{4.224000in}}%
\pgfusepath{stroke}%
\end{pgfscope}%
\begin{pgfscope}%
\pgfsetrectcap%
\pgfsetmiterjoin%
\pgfsetlinewidth{0.803000pt}%
\definecolor{currentstroke}{rgb}{0.000000,0.000000,0.000000}%
\pgfsetstrokecolor{currentstroke}%
\pgfsetdash{}{0pt}%
\pgfpathmoveto{\pgfqpoint{5.760000in}{0.528000in}}%
\pgfpathlineto{\pgfqpoint{5.760000in}{4.224000in}}%
\pgfusepath{stroke}%
\end{pgfscope}%
\begin{pgfscope}%
\pgfsetrectcap%
\pgfsetmiterjoin%
\pgfsetlinewidth{0.803000pt}%
\definecolor{currentstroke}{rgb}{0.000000,0.000000,0.000000}%
\pgfsetstrokecolor{currentstroke}%
\pgfsetdash{}{0pt}%
\pgfpathmoveto{\pgfqpoint{0.800000in}{0.528000in}}%
\pgfpathlineto{\pgfqpoint{5.760000in}{0.528000in}}%
\pgfusepath{stroke}%
\end{pgfscope}%
\begin{pgfscope}%
\pgfsetrectcap%
\pgfsetmiterjoin%
\pgfsetlinewidth{0.803000pt}%
\definecolor{currentstroke}{rgb}{0.000000,0.000000,0.000000}%
\pgfsetstrokecolor{currentstroke}%
\pgfsetdash{}{0pt}%
\pgfpathmoveto{\pgfqpoint{0.800000in}{4.224000in}}%
\pgfpathlineto{\pgfqpoint{5.760000in}{4.224000in}}%
\pgfusepath{stroke}%
\end{pgfscope}%
\end{pgfpicture}%
\makeatother%
\endgroup%
}
                   \caption{Analytical Determination of $\epsilon_1(\nu)$}
                   \label{fig:Nu1Analytical}
                \end{center}
            \end{figure}
            \noindent
            where any combination of $\epsilon, \nu$ above that line will produce exponential 
            growth.

            \item Now we aim to analyse the same equation but for $\nu \approx 4$. We do the same 
            analysis as before, but this time we expand $\nu$ as $\nu = 4 + \epsilon\nu_1 + 
            \epsilon^2 \nu_2 + \dots$, which gives us
            \begin{equation*}
                \begin{split}
                    D_0^2\theta_0 + 4\theta_0 &= 0 \\
                    \implies \theta_0 &= Ae^{2iT_0}+A^*e^{-2iT_0}
                \end{split}
            \end{equation*}
            and then 
            \begin{equation*}
                D_0^2 \theta_1 +\theta_1 = -4iD_1(Ae^{2iT_0}-A^*e^{-2iT_0})-\frac{1}{2}(Ae^{4iT_0}
                +A^*e^{-4iT_0}+A+A^*)-\nu_1(Ae^{2iT_0}-A^*e^{-2iT_0})
            \end{equation*}
            To kill the secular terms in this case, we kill terms with frequency 2, which gives us 
            \begin{equation*}
                \begin{split}
                    4iD_1A-\nu_1A&=0 \\
                    -4iD_1A^*-\nu_1A^*&=0 \\
                    \implies A&=Ce^{-4i\nu_1T_1} \\
                    \implies \theta_0 &= Ce^{i(T_0-4\nu_1T_1)}+C^*e^{i(4\nu_1T_1-T_0)} \\
                    &= Ce^{i\tau(1-4\nu_1\epsilon)}+C^*e^{i\tau(4\nu_1\epsilon-1)}
                \end{split}
            \end{equation*}
            This solution gives purely oscillatory motion, which might not be correct as, intuitively, 
            we would expect Miu Miu to have some kind of growth, whether it be exponential or linear.
            We aren't sure if finding the solution is possible. Maybe by finding $\theta_1$ by looking 
            at coefficients of $\epsilon^2$ terms we could uncover a solution for $\epsilon_4(\nu)$.
            

        \end{enumerate}

        \item \textbf{Numerical}
        \begin{enumerate}
            \item To simulate this differential equation we use \texttt{scipy.integrate.odeint}, 
            splitting up \autoref{eqn:Main Differential Equation} into two first order differential 
            equations. Running that program for a series of values of $\nu$ around 1 and reasonably 
            sized $\epsilon$'s, checking if $\theta$ goes to 100 before $\tau = 100$ and classifying 
            that $\epsilon$ as being above $\epsilon_1$. This gives us the plot in 
            \autoref{fig:nu1 Numerical} below.
            \begin{figure}[H]
                \begin{center}
                    \scalebox{.7}{%% Creator: Matplotlib, PGF backend
%%
%% To include the figure in your LaTeX document, write
%%   \input{<filename>.pgf}
%%
%% Make sure the required packages are loaded in your preamble
%%   \usepackage{pgf}
%%
%% Figures using additional raster images can only be included by \input if
%% they are in the same directory as the main LaTeX file. For loading figures
%% from other directories you can use the `import` package
%%   \usepackage{import}
%% and then include the figures with
%%   \import{<path to file>}{<filename>.pgf}
%%
%% Matplotlib used the following preamble
%%
\begingroup%
\makeatletter%
\begin{pgfpicture}%
\pgfpathrectangle{\pgfpointorigin}{\pgfqpoint{6.400000in}{4.800000in}}%
\pgfusepath{use as bounding box, clip}%
\begin{pgfscope}%
\pgfsetbuttcap%
\pgfsetmiterjoin%
\definecolor{currentfill}{rgb}{1.000000,1.000000,1.000000}%
\pgfsetfillcolor{currentfill}%
\pgfsetlinewidth{0.000000pt}%
\definecolor{currentstroke}{rgb}{1.000000,1.000000,1.000000}%
\pgfsetstrokecolor{currentstroke}%
\pgfsetdash{}{0pt}%
\pgfpathmoveto{\pgfqpoint{0.000000in}{0.000000in}}%
\pgfpathlineto{\pgfqpoint{6.400000in}{0.000000in}}%
\pgfpathlineto{\pgfqpoint{6.400000in}{4.800000in}}%
\pgfpathlineto{\pgfqpoint{0.000000in}{4.800000in}}%
\pgfpathclose%
\pgfusepath{fill}%
\end{pgfscope}%
\begin{pgfscope}%
\pgfsetbuttcap%
\pgfsetmiterjoin%
\definecolor{currentfill}{rgb}{1.000000,1.000000,1.000000}%
\pgfsetfillcolor{currentfill}%
\pgfsetlinewidth{0.000000pt}%
\definecolor{currentstroke}{rgb}{0.000000,0.000000,0.000000}%
\pgfsetstrokecolor{currentstroke}%
\pgfsetstrokeopacity{0.000000}%
\pgfsetdash{}{0pt}%
\pgfpathmoveto{\pgfqpoint{0.800000in}{0.528000in}}%
\pgfpathlineto{\pgfqpoint{5.760000in}{0.528000in}}%
\pgfpathlineto{\pgfqpoint{5.760000in}{4.224000in}}%
\pgfpathlineto{\pgfqpoint{0.800000in}{4.224000in}}%
\pgfpathclose%
\pgfusepath{fill}%
\end{pgfscope}%
\begin{pgfscope}%
\pgfsetbuttcap%
\pgfsetroundjoin%
\definecolor{currentfill}{rgb}{0.000000,0.000000,0.000000}%
\pgfsetfillcolor{currentfill}%
\pgfsetlinewidth{0.803000pt}%
\definecolor{currentstroke}{rgb}{0.000000,0.000000,0.000000}%
\pgfsetstrokecolor{currentstroke}%
\pgfsetdash{}{0pt}%
\pgfsys@defobject{currentmarker}{\pgfqpoint{0.000000in}{-0.048611in}}{\pgfqpoint{0.000000in}{0.000000in}}{%
\pgfpathmoveto{\pgfqpoint{0.000000in}{0.000000in}}%
\pgfpathlineto{\pgfqpoint{0.000000in}{-0.048611in}}%
\pgfusepath{stroke,fill}%
}%
\begin{pgfscope}%
\pgfsys@transformshift{1.476815in}{0.528000in}%
\pgfsys@useobject{currentmarker}{}%
\end{pgfscope}%
\end{pgfscope}%
\begin{pgfscope}%
\definecolor{textcolor}{rgb}{0.000000,0.000000,0.000000}%
\pgfsetstrokecolor{textcolor}%
\pgfsetfillcolor{textcolor}%
\pgftext[x=1.476815in,y=0.430778in,,top]{\color{textcolor}\rmfamily\fontsize{10.000000}{12.000000}\selectfont \(\displaystyle 0.8\)}%
\end{pgfscope}%
\begin{pgfscope}%
\pgfsetbuttcap%
\pgfsetroundjoin%
\definecolor{currentfill}{rgb}{0.000000,0.000000,0.000000}%
\pgfsetfillcolor{currentfill}%
\pgfsetlinewidth{0.803000pt}%
\definecolor{currentstroke}{rgb}{0.000000,0.000000,0.000000}%
\pgfsetstrokecolor{currentstroke}%
\pgfsetdash{}{0pt}%
\pgfsys@defobject{currentmarker}{\pgfqpoint{0.000000in}{-0.048611in}}{\pgfqpoint{0.000000in}{0.000000in}}{%
\pgfpathmoveto{\pgfqpoint{0.000000in}{0.000000in}}%
\pgfpathlineto{\pgfqpoint{0.000000in}{-0.048611in}}%
\pgfusepath{stroke,fill}%
}%
\begin{pgfscope}%
\pgfsys@transformshift{2.379536in}{0.528000in}%
\pgfsys@useobject{currentmarker}{}%
\end{pgfscope}%
\end{pgfscope}%
\begin{pgfscope}%
\definecolor{textcolor}{rgb}{0.000000,0.000000,0.000000}%
\pgfsetstrokecolor{textcolor}%
\pgfsetfillcolor{textcolor}%
\pgftext[x=2.379536in,y=0.430778in,,top]{\color{textcolor}\rmfamily\fontsize{10.000000}{12.000000}\selectfont \(\displaystyle 0.9\)}%
\end{pgfscope}%
\begin{pgfscope}%
\pgfsetbuttcap%
\pgfsetroundjoin%
\definecolor{currentfill}{rgb}{0.000000,0.000000,0.000000}%
\pgfsetfillcolor{currentfill}%
\pgfsetlinewidth{0.803000pt}%
\definecolor{currentstroke}{rgb}{0.000000,0.000000,0.000000}%
\pgfsetstrokecolor{currentstroke}%
\pgfsetdash{}{0pt}%
\pgfsys@defobject{currentmarker}{\pgfqpoint{0.000000in}{-0.048611in}}{\pgfqpoint{0.000000in}{0.000000in}}{%
\pgfpathmoveto{\pgfqpoint{0.000000in}{0.000000in}}%
\pgfpathlineto{\pgfqpoint{0.000000in}{-0.048611in}}%
\pgfusepath{stroke,fill}%
}%
\begin{pgfscope}%
\pgfsys@transformshift{3.282257in}{0.528000in}%
\pgfsys@useobject{currentmarker}{}%
\end{pgfscope}%
\end{pgfscope}%
\begin{pgfscope}%
\definecolor{textcolor}{rgb}{0.000000,0.000000,0.000000}%
\pgfsetstrokecolor{textcolor}%
\pgfsetfillcolor{textcolor}%
\pgftext[x=3.282257in,y=0.430778in,,top]{\color{textcolor}\rmfamily\fontsize{10.000000}{12.000000}\selectfont \(\displaystyle 1.0\)}%
\end{pgfscope}%
\begin{pgfscope}%
\pgfsetbuttcap%
\pgfsetroundjoin%
\definecolor{currentfill}{rgb}{0.000000,0.000000,0.000000}%
\pgfsetfillcolor{currentfill}%
\pgfsetlinewidth{0.803000pt}%
\definecolor{currentstroke}{rgb}{0.000000,0.000000,0.000000}%
\pgfsetstrokecolor{currentstroke}%
\pgfsetdash{}{0pt}%
\pgfsys@defobject{currentmarker}{\pgfqpoint{0.000000in}{-0.048611in}}{\pgfqpoint{0.000000in}{0.000000in}}{%
\pgfpathmoveto{\pgfqpoint{0.000000in}{0.000000in}}%
\pgfpathlineto{\pgfqpoint{0.000000in}{-0.048611in}}%
\pgfusepath{stroke,fill}%
}%
\begin{pgfscope}%
\pgfsys@transformshift{4.184978in}{0.528000in}%
\pgfsys@useobject{currentmarker}{}%
\end{pgfscope}%
\end{pgfscope}%
\begin{pgfscope}%
\definecolor{textcolor}{rgb}{0.000000,0.000000,0.000000}%
\pgfsetstrokecolor{textcolor}%
\pgfsetfillcolor{textcolor}%
\pgftext[x=4.184978in,y=0.430778in,,top]{\color{textcolor}\rmfamily\fontsize{10.000000}{12.000000}\selectfont \(\displaystyle 1.1\)}%
\end{pgfscope}%
\begin{pgfscope}%
\pgfsetbuttcap%
\pgfsetroundjoin%
\definecolor{currentfill}{rgb}{0.000000,0.000000,0.000000}%
\pgfsetfillcolor{currentfill}%
\pgfsetlinewidth{0.803000pt}%
\definecolor{currentstroke}{rgb}{0.000000,0.000000,0.000000}%
\pgfsetstrokecolor{currentstroke}%
\pgfsetdash{}{0pt}%
\pgfsys@defobject{currentmarker}{\pgfqpoint{0.000000in}{-0.048611in}}{\pgfqpoint{0.000000in}{0.000000in}}{%
\pgfpathmoveto{\pgfqpoint{0.000000in}{0.000000in}}%
\pgfpathlineto{\pgfqpoint{0.000000in}{-0.048611in}}%
\pgfusepath{stroke,fill}%
}%
\begin{pgfscope}%
\pgfsys@transformshift{5.087699in}{0.528000in}%
\pgfsys@useobject{currentmarker}{}%
\end{pgfscope}%
\end{pgfscope}%
\begin{pgfscope}%
\definecolor{textcolor}{rgb}{0.000000,0.000000,0.000000}%
\pgfsetstrokecolor{textcolor}%
\pgfsetfillcolor{textcolor}%
\pgftext[x=5.087699in,y=0.430778in,,top]{\color{textcolor}\rmfamily\fontsize{10.000000}{12.000000}\selectfont \(\displaystyle 1.2\)}%
\end{pgfscope}%
\begin{pgfscope}%
\definecolor{textcolor}{rgb}{0.000000,0.000000,0.000000}%
\pgfsetstrokecolor{textcolor}%
\pgfsetfillcolor{textcolor}%
\pgftext[x=3.280000in,y=0.251766in,,top]{\color{textcolor}\rmfamily\fontsize{10.000000}{12.000000}\selectfont \(\displaystyle \nu\)}%
\end{pgfscope}%
\begin{pgfscope}%
\pgfsetbuttcap%
\pgfsetroundjoin%
\definecolor{currentfill}{rgb}{0.000000,0.000000,0.000000}%
\pgfsetfillcolor{currentfill}%
\pgfsetlinewidth{0.803000pt}%
\definecolor{currentstroke}{rgb}{0.000000,0.000000,0.000000}%
\pgfsetstrokecolor{currentstroke}%
\pgfsetdash{}{0pt}%
\pgfsys@defobject{currentmarker}{\pgfqpoint{-0.048611in}{0.000000in}}{\pgfqpoint{0.000000in}{0.000000in}}{%
\pgfpathmoveto{\pgfqpoint{0.000000in}{0.000000in}}%
\pgfpathlineto{\pgfqpoint{-0.048611in}{0.000000in}}%
\pgfusepath{stroke,fill}%
}%
\begin{pgfscope}%
\pgfsys@transformshift{0.800000in}{0.696000in}%
\pgfsys@useobject{currentmarker}{}%
\end{pgfscope}%
\end{pgfscope}%
\begin{pgfscope}%
\definecolor{textcolor}{rgb}{0.000000,0.000000,0.000000}%
\pgfsetstrokecolor{textcolor}%
\pgfsetfillcolor{textcolor}%
\pgftext[x=0.525308in,y=0.647775in,left,base]{\color{textcolor}\rmfamily\fontsize{10.000000}{12.000000}\selectfont \(\displaystyle 0.0\)}%
\end{pgfscope}%
\begin{pgfscope}%
\pgfsetbuttcap%
\pgfsetroundjoin%
\definecolor{currentfill}{rgb}{0.000000,0.000000,0.000000}%
\pgfsetfillcolor{currentfill}%
\pgfsetlinewidth{0.803000pt}%
\definecolor{currentstroke}{rgb}{0.000000,0.000000,0.000000}%
\pgfsetstrokecolor{currentstroke}%
\pgfsetdash{}{0pt}%
\pgfsys@defobject{currentmarker}{\pgfqpoint{-0.048611in}{0.000000in}}{\pgfqpoint{0.000000in}{0.000000in}}{%
\pgfpathmoveto{\pgfqpoint{0.000000in}{0.000000in}}%
\pgfpathlineto{\pgfqpoint{-0.048611in}{0.000000in}}%
\pgfusepath{stroke,fill}%
}%
\begin{pgfscope}%
\pgfsys@transformshift{0.800000in}{1.322866in}%
\pgfsys@useobject{currentmarker}{}%
\end{pgfscope}%
\end{pgfscope}%
\begin{pgfscope}%
\definecolor{textcolor}{rgb}{0.000000,0.000000,0.000000}%
\pgfsetstrokecolor{textcolor}%
\pgfsetfillcolor{textcolor}%
\pgftext[x=0.525308in,y=1.274640in,left,base]{\color{textcolor}\rmfamily\fontsize{10.000000}{12.000000}\selectfont \(\displaystyle 0.1\)}%
\end{pgfscope}%
\begin{pgfscope}%
\pgfsetbuttcap%
\pgfsetroundjoin%
\definecolor{currentfill}{rgb}{0.000000,0.000000,0.000000}%
\pgfsetfillcolor{currentfill}%
\pgfsetlinewidth{0.803000pt}%
\definecolor{currentstroke}{rgb}{0.000000,0.000000,0.000000}%
\pgfsetstrokecolor{currentstroke}%
\pgfsetdash{}{0pt}%
\pgfsys@defobject{currentmarker}{\pgfqpoint{-0.048611in}{0.000000in}}{\pgfqpoint{0.000000in}{0.000000in}}{%
\pgfpathmoveto{\pgfqpoint{0.000000in}{0.000000in}}%
\pgfpathlineto{\pgfqpoint{-0.048611in}{0.000000in}}%
\pgfusepath{stroke,fill}%
}%
\begin{pgfscope}%
\pgfsys@transformshift{0.800000in}{1.949731in}%
\pgfsys@useobject{currentmarker}{}%
\end{pgfscope}%
\end{pgfscope}%
\begin{pgfscope}%
\definecolor{textcolor}{rgb}{0.000000,0.000000,0.000000}%
\pgfsetstrokecolor{textcolor}%
\pgfsetfillcolor{textcolor}%
\pgftext[x=0.525308in,y=1.901506in,left,base]{\color{textcolor}\rmfamily\fontsize{10.000000}{12.000000}\selectfont \(\displaystyle 0.2\)}%
\end{pgfscope}%
\begin{pgfscope}%
\pgfsetbuttcap%
\pgfsetroundjoin%
\definecolor{currentfill}{rgb}{0.000000,0.000000,0.000000}%
\pgfsetfillcolor{currentfill}%
\pgfsetlinewidth{0.803000pt}%
\definecolor{currentstroke}{rgb}{0.000000,0.000000,0.000000}%
\pgfsetstrokecolor{currentstroke}%
\pgfsetdash{}{0pt}%
\pgfsys@defobject{currentmarker}{\pgfqpoint{-0.048611in}{0.000000in}}{\pgfqpoint{0.000000in}{0.000000in}}{%
\pgfpathmoveto{\pgfqpoint{0.000000in}{0.000000in}}%
\pgfpathlineto{\pgfqpoint{-0.048611in}{0.000000in}}%
\pgfusepath{stroke,fill}%
}%
\begin{pgfscope}%
\pgfsys@transformshift{0.800000in}{2.576597in}%
\pgfsys@useobject{currentmarker}{}%
\end{pgfscope}%
\end{pgfscope}%
\begin{pgfscope}%
\definecolor{textcolor}{rgb}{0.000000,0.000000,0.000000}%
\pgfsetstrokecolor{textcolor}%
\pgfsetfillcolor{textcolor}%
\pgftext[x=0.525308in,y=2.528372in,left,base]{\color{textcolor}\rmfamily\fontsize{10.000000}{12.000000}\selectfont \(\displaystyle 0.3\)}%
\end{pgfscope}%
\begin{pgfscope}%
\pgfsetbuttcap%
\pgfsetroundjoin%
\definecolor{currentfill}{rgb}{0.000000,0.000000,0.000000}%
\pgfsetfillcolor{currentfill}%
\pgfsetlinewidth{0.803000pt}%
\definecolor{currentstroke}{rgb}{0.000000,0.000000,0.000000}%
\pgfsetstrokecolor{currentstroke}%
\pgfsetdash{}{0pt}%
\pgfsys@defobject{currentmarker}{\pgfqpoint{-0.048611in}{0.000000in}}{\pgfqpoint{0.000000in}{0.000000in}}{%
\pgfpathmoveto{\pgfqpoint{0.000000in}{0.000000in}}%
\pgfpathlineto{\pgfqpoint{-0.048611in}{0.000000in}}%
\pgfusepath{stroke,fill}%
}%
\begin{pgfscope}%
\pgfsys@transformshift{0.800000in}{3.203463in}%
\pgfsys@useobject{currentmarker}{}%
\end{pgfscope}%
\end{pgfscope}%
\begin{pgfscope}%
\definecolor{textcolor}{rgb}{0.000000,0.000000,0.000000}%
\pgfsetstrokecolor{textcolor}%
\pgfsetfillcolor{textcolor}%
\pgftext[x=0.525308in,y=3.155237in,left,base]{\color{textcolor}\rmfamily\fontsize{10.000000}{12.000000}\selectfont \(\displaystyle 0.4\)}%
\end{pgfscope}%
\begin{pgfscope}%
\pgfsetbuttcap%
\pgfsetroundjoin%
\definecolor{currentfill}{rgb}{0.000000,0.000000,0.000000}%
\pgfsetfillcolor{currentfill}%
\pgfsetlinewidth{0.803000pt}%
\definecolor{currentstroke}{rgb}{0.000000,0.000000,0.000000}%
\pgfsetstrokecolor{currentstroke}%
\pgfsetdash{}{0pt}%
\pgfsys@defobject{currentmarker}{\pgfqpoint{-0.048611in}{0.000000in}}{\pgfqpoint{0.000000in}{0.000000in}}{%
\pgfpathmoveto{\pgfqpoint{0.000000in}{0.000000in}}%
\pgfpathlineto{\pgfqpoint{-0.048611in}{0.000000in}}%
\pgfusepath{stroke,fill}%
}%
\begin{pgfscope}%
\pgfsys@transformshift{0.800000in}{3.830328in}%
\pgfsys@useobject{currentmarker}{}%
\end{pgfscope}%
\end{pgfscope}%
\begin{pgfscope}%
\definecolor{textcolor}{rgb}{0.000000,0.000000,0.000000}%
\pgfsetstrokecolor{textcolor}%
\pgfsetfillcolor{textcolor}%
\pgftext[x=0.525308in,y=3.782103in,left,base]{\color{textcolor}\rmfamily\fontsize{10.000000}{12.000000}\selectfont \(\displaystyle 0.5\)}%
\end{pgfscope}%
\begin{pgfscope}%
\definecolor{textcolor}{rgb}{0.000000,0.000000,0.000000}%
\pgfsetstrokecolor{textcolor}%
\pgfsetfillcolor{textcolor}%
\pgftext[x=0.469752in,y=2.376000in,,bottom]{\color{textcolor}\rmfamily\fontsize{10.000000}{12.000000}\selectfont \(\displaystyle \epsilon\)}%
\end{pgfscope}%
\begin{pgfscope}%
\pgfpathrectangle{\pgfqpoint{0.800000in}{0.528000in}}{\pgfqpoint{4.960000in}{3.696000in}}%
\pgfusepath{clip}%
\pgfsetrectcap%
\pgfsetroundjoin%
\pgfsetlinewidth{1.505625pt}%
\definecolor{currentstroke}{rgb}{0.000000,0.000000,1.000000}%
\pgfsetstrokecolor{currentstroke}%
\pgfsetdash{}{0pt}%
\pgfpathmoveto{\pgfqpoint{1.025455in}{3.974507in}}%
\pgfpathlineto{\pgfqpoint{1.034482in}{3.961970in}}%
\pgfpathlineto{\pgfqpoint{1.038995in}{3.961970in}}%
\pgfpathlineto{\pgfqpoint{1.066077in}{3.924358in}}%
\pgfpathlineto{\pgfqpoint{1.070591in}{3.924358in}}%
\pgfpathlineto{\pgfqpoint{1.102186in}{3.880478in}}%
\pgfpathlineto{\pgfqpoint{1.106699in}{3.880478in}}%
\pgfpathlineto{\pgfqpoint{1.133781in}{3.842866in}}%
\pgfpathlineto{\pgfqpoint{1.138295in}{3.842866in}}%
\pgfpathlineto{\pgfqpoint{1.165376in}{3.805254in}}%
\pgfpathlineto{\pgfqpoint{1.169890in}{3.805254in}}%
\pgfpathlineto{\pgfqpoint{1.196972in}{3.767642in}}%
\pgfpathlineto{\pgfqpoint{1.201485in}{3.767642in}}%
\pgfpathlineto{\pgfqpoint{1.228567in}{3.730030in}}%
\pgfpathlineto{\pgfqpoint{1.233080in}{3.730030in}}%
\pgfpathlineto{\pgfqpoint{1.260162in}{3.692418in}}%
\pgfpathlineto{\pgfqpoint{1.264676in}{3.692418in}}%
\pgfpathlineto{\pgfqpoint{1.291757in}{3.654806in}}%
\pgfpathlineto{\pgfqpoint{1.296271in}{3.654806in}}%
\pgfpathlineto{\pgfqpoint{1.323352in}{3.617194in}}%
\pgfpathlineto{\pgfqpoint{1.327866in}{3.617194in}}%
\pgfpathlineto{\pgfqpoint{1.354948in}{3.579582in}}%
\pgfpathlineto{\pgfqpoint{1.359461in}{3.579582in}}%
\pgfpathlineto{\pgfqpoint{1.382029in}{3.548239in}}%
\pgfpathlineto{\pgfqpoint{1.386543in}{3.548239in}}%
\pgfpathlineto{\pgfqpoint{1.413625in}{3.510627in}}%
\pgfpathlineto{\pgfqpoint{1.418138in}{3.510627in}}%
\pgfpathlineto{\pgfqpoint{1.440706in}{3.479284in}}%
\pgfpathlineto{\pgfqpoint{1.445220in}{3.479284in}}%
\pgfpathlineto{\pgfqpoint{1.472301in}{3.441672in}}%
\pgfpathlineto{\pgfqpoint{1.476815in}{3.441672in}}%
\pgfpathlineto{\pgfqpoint{1.499383in}{3.410328in}}%
\pgfpathlineto{\pgfqpoint{1.503897in}{3.410328in}}%
\pgfpathlineto{\pgfqpoint{1.526465in}{3.378985in}}%
\pgfpathlineto{\pgfqpoint{1.530978in}{3.378985in}}%
\pgfpathlineto{\pgfqpoint{1.553546in}{3.347642in}}%
\pgfpathlineto{\pgfqpoint{1.558060in}{3.347642in}}%
\pgfpathlineto{\pgfqpoint{1.580628in}{3.316299in}}%
\pgfpathlineto{\pgfqpoint{1.585142in}{3.316299in}}%
\pgfpathlineto{\pgfqpoint{1.607710in}{3.284955in}}%
\pgfpathlineto{\pgfqpoint{1.612223in}{3.284955in}}%
\pgfpathlineto{\pgfqpoint{1.634791in}{3.253612in}}%
\pgfpathlineto{\pgfqpoint{1.639305in}{3.253612in}}%
\pgfpathlineto{\pgfqpoint{1.661873in}{3.222269in}}%
\pgfpathlineto{\pgfqpoint{1.666386in}{3.222269in}}%
\pgfpathlineto{\pgfqpoint{1.684441in}{3.197194in}}%
\pgfpathlineto{\pgfqpoint{1.688954in}{3.197194in}}%
\pgfpathlineto{\pgfqpoint{1.711522in}{3.165851in}}%
\pgfpathlineto{\pgfqpoint{1.716036in}{3.165851in}}%
\pgfpathlineto{\pgfqpoint{1.734090in}{3.140776in}}%
\pgfpathlineto{\pgfqpoint{1.738604in}{3.140776in}}%
\pgfpathlineto{\pgfqpoint{1.761172in}{3.109433in}}%
\pgfpathlineto{\pgfqpoint{1.765686in}{3.109433in}}%
\pgfpathlineto{\pgfqpoint{1.783740in}{3.084358in}}%
\pgfpathlineto{\pgfqpoint{1.788254in}{3.084358in}}%
\pgfpathlineto{\pgfqpoint{1.806308in}{3.059284in}}%
\pgfpathlineto{\pgfqpoint{1.810822in}{3.059284in}}%
\pgfpathlineto{\pgfqpoint{1.828876in}{3.034209in}}%
\pgfpathlineto{\pgfqpoint{1.833390in}{3.034209in}}%
\pgfpathlineto{\pgfqpoint{1.851444in}{3.009134in}}%
\pgfpathlineto{\pgfqpoint{1.855958in}{3.009134in}}%
\pgfpathlineto{\pgfqpoint{1.874012in}{2.984060in}}%
\pgfpathlineto{\pgfqpoint{1.878526in}{2.984060in}}%
\pgfpathlineto{\pgfqpoint{1.896580in}{2.958985in}}%
\pgfpathlineto{\pgfqpoint{1.901094in}{2.958985in}}%
\pgfpathlineto{\pgfqpoint{1.919148in}{2.933910in}}%
\pgfpathlineto{\pgfqpoint{1.923662in}{2.933910in}}%
\pgfpathlineto{\pgfqpoint{1.941716in}{2.908836in}}%
\pgfpathlineto{\pgfqpoint{1.946230in}{2.908836in}}%
\pgfpathlineto{\pgfqpoint{1.959771in}{2.890030in}}%
\pgfpathlineto{\pgfqpoint{1.964284in}{2.890030in}}%
\pgfpathlineto{\pgfqpoint{1.982339in}{2.864955in}}%
\pgfpathlineto{\pgfqpoint{1.986852in}{2.864955in}}%
\pgfpathlineto{\pgfqpoint{2.000393in}{2.846149in}}%
\pgfpathlineto{\pgfqpoint{2.004907in}{2.846149in}}%
\pgfpathlineto{\pgfqpoint{2.022961in}{2.821075in}}%
\pgfpathlineto{\pgfqpoint{2.027475in}{2.821075in}}%
\pgfpathlineto{\pgfqpoint{2.041016in}{2.802269in}}%
\pgfpathlineto{\pgfqpoint{2.045529in}{2.802269in}}%
\pgfpathlineto{\pgfqpoint{2.059070in}{2.783463in}}%
\pgfpathlineto{\pgfqpoint{2.063584in}{2.783463in}}%
\pgfpathlineto{\pgfqpoint{2.081638in}{2.758388in}}%
\pgfpathlineto{\pgfqpoint{2.086152in}{2.758388in}}%
\pgfpathlineto{\pgfqpoint{2.099692in}{2.739582in}}%
\pgfpathlineto{\pgfqpoint{2.104206in}{2.739582in}}%
\pgfpathlineto{\pgfqpoint{2.117747in}{2.720776in}}%
\pgfpathlineto{\pgfqpoint{2.122260in}{2.720776in}}%
\pgfpathlineto{\pgfqpoint{2.135801in}{2.701970in}}%
\pgfpathlineto{\pgfqpoint{2.140315in}{2.701970in}}%
\pgfpathlineto{\pgfqpoint{2.153856in}{2.683164in}}%
\pgfpathlineto{\pgfqpoint{2.158369in}{2.683164in}}%
\pgfpathlineto{\pgfqpoint{2.171910in}{2.664358in}}%
\pgfpathlineto{\pgfqpoint{2.176424in}{2.664358in}}%
\pgfpathlineto{\pgfqpoint{2.185451in}{2.651821in}}%
\pgfpathlineto{\pgfqpoint{2.189965in}{2.651821in}}%
\pgfpathlineto{\pgfqpoint{2.203505in}{2.633015in}}%
\pgfpathlineto{\pgfqpoint{2.208019in}{2.633015in}}%
\pgfpathlineto{\pgfqpoint{2.221560in}{2.614209in}}%
\pgfpathlineto{\pgfqpoint{2.226073in}{2.614209in}}%
\pgfpathlineto{\pgfqpoint{2.235101in}{2.601672in}}%
\pgfpathlineto{\pgfqpoint{2.239614in}{2.601672in}}%
\pgfpathlineto{\pgfqpoint{2.253155in}{2.582866in}}%
\pgfpathlineto{\pgfqpoint{2.257669in}{2.582866in}}%
\pgfpathlineto{\pgfqpoint{2.266696in}{2.570328in}}%
\pgfpathlineto{\pgfqpoint{2.271209in}{2.570328in}}%
\pgfpathlineto{\pgfqpoint{2.284750in}{2.551522in}}%
\pgfpathlineto{\pgfqpoint{2.289264in}{2.551522in}}%
\pgfpathlineto{\pgfqpoint{2.298291in}{2.538985in}}%
\pgfpathlineto{\pgfqpoint{2.302805in}{2.538985in}}%
\pgfpathlineto{\pgfqpoint{2.316345in}{2.520179in}}%
\pgfpathlineto{\pgfqpoint{2.320859in}{2.520179in}}%
\pgfpathlineto{\pgfqpoint{2.329886in}{2.507642in}}%
\pgfpathlineto{\pgfqpoint{2.334400in}{2.507642in}}%
\pgfpathlineto{\pgfqpoint{2.343427in}{2.495104in}}%
\pgfpathlineto{\pgfqpoint{2.347941in}{2.495104in}}%
\pgfpathlineto{\pgfqpoint{2.356968in}{2.482567in}}%
\pgfpathlineto{\pgfqpoint{2.361481in}{2.482567in}}%
\pgfpathlineto{\pgfqpoint{2.370509in}{2.470030in}}%
\pgfpathlineto{\pgfqpoint{2.375022in}{2.470030in}}%
\pgfpathlineto{\pgfqpoint{2.388563in}{2.451224in}}%
\pgfpathlineto{\pgfqpoint{2.393077in}{2.451224in}}%
\pgfpathlineto{\pgfqpoint{2.402104in}{2.438687in}}%
\pgfpathlineto{\pgfqpoint{2.406618in}{2.438687in}}%
\pgfpathlineto{\pgfqpoint{2.415645in}{2.426149in}}%
\pgfpathlineto{\pgfqpoint{2.420158in}{2.426149in}}%
\pgfpathlineto{\pgfqpoint{2.429186in}{2.413612in}}%
\pgfpathlineto{\pgfqpoint{2.433699in}{2.413612in}}%
\pgfpathlineto{\pgfqpoint{2.438213in}{2.407343in}}%
\pgfpathlineto{\pgfqpoint{2.442726in}{2.407343in}}%
\pgfpathlineto{\pgfqpoint{2.451754in}{2.394806in}}%
\pgfpathlineto{\pgfqpoint{2.456267in}{2.394806in}}%
\pgfpathlineto{\pgfqpoint{2.465294in}{2.382269in}}%
\pgfpathlineto{\pgfqpoint{2.469808in}{2.382269in}}%
\pgfpathlineto{\pgfqpoint{2.478835in}{2.369731in}}%
\pgfpathlineto{\pgfqpoint{2.483349in}{2.369731in}}%
\pgfpathlineto{\pgfqpoint{2.492376in}{2.357194in}}%
\pgfpathlineto{\pgfqpoint{2.496890in}{2.357194in}}%
\pgfpathlineto{\pgfqpoint{2.501403in}{2.350925in}}%
\pgfpathlineto{\pgfqpoint{2.505917in}{2.350925in}}%
\pgfpathlineto{\pgfqpoint{2.514944in}{2.338388in}}%
\pgfpathlineto{\pgfqpoint{2.519458in}{2.338388in}}%
\pgfpathlineto{\pgfqpoint{2.528485in}{2.325851in}}%
\pgfpathlineto{\pgfqpoint{2.532998in}{2.325851in}}%
\pgfpathlineto{\pgfqpoint{2.537512in}{2.319582in}}%
\pgfpathlineto{\pgfqpoint{2.542026in}{2.319582in}}%
\pgfpathlineto{\pgfqpoint{2.551053in}{2.307045in}}%
\pgfpathlineto{\pgfqpoint{2.555566in}{2.307045in}}%
\pgfpathlineto{\pgfqpoint{2.560080in}{2.300776in}}%
\pgfpathlineto{\pgfqpoint{2.564594in}{2.300776in}}%
\pgfpathlineto{\pgfqpoint{2.573621in}{2.288239in}}%
\pgfpathlineto{\pgfqpoint{2.578134in}{2.288239in}}%
\pgfpathlineto{\pgfqpoint{2.582648in}{2.281970in}}%
\pgfpathlineto{\pgfqpoint{2.587162in}{2.281970in}}%
\pgfpathlineto{\pgfqpoint{2.596189in}{2.269433in}}%
\pgfpathlineto{\pgfqpoint{2.600703in}{2.269433in}}%
\pgfpathlineto{\pgfqpoint{2.605216in}{2.263164in}}%
\pgfpathlineto{\pgfqpoint{2.609730in}{2.263164in}}%
\pgfpathlineto{\pgfqpoint{2.618757in}{2.250627in}}%
\pgfpathlineto{\pgfqpoint{2.623271in}{2.250627in}}%
\pgfpathlineto{\pgfqpoint{2.627784in}{2.244358in}}%
\pgfpathlineto{\pgfqpoint{2.632298in}{2.244358in}}%
\pgfpathlineto{\pgfqpoint{2.636811in}{2.238090in}}%
\pgfpathlineto{\pgfqpoint{2.641325in}{2.238090in}}%
\pgfpathlineto{\pgfqpoint{2.645839in}{2.231821in}}%
\pgfpathlineto{\pgfqpoint{2.650352in}{2.231821in}}%
\pgfpathlineto{\pgfqpoint{2.659379in}{2.219284in}}%
\pgfpathlineto{\pgfqpoint{2.663893in}{2.219284in}}%
\pgfpathlineto{\pgfqpoint{2.668407in}{2.213015in}}%
\pgfpathlineto{\pgfqpoint{2.672920in}{2.213015in}}%
\pgfpathlineto{\pgfqpoint{2.677434in}{2.206746in}}%
\pgfpathlineto{\pgfqpoint{2.681947in}{2.206746in}}%
\pgfpathlineto{\pgfqpoint{2.686461in}{2.200478in}}%
\pgfpathlineto{\pgfqpoint{2.690975in}{2.200478in}}%
\pgfpathlineto{\pgfqpoint{2.695488in}{2.194209in}}%
\pgfpathlineto{\pgfqpoint{2.700002in}{2.194209in}}%
\pgfpathlineto{\pgfqpoint{2.709029in}{2.181672in}}%
\pgfpathlineto{\pgfqpoint{2.713543in}{2.181672in}}%
\pgfpathlineto{\pgfqpoint{2.718056in}{2.175403in}}%
\pgfpathlineto{\pgfqpoint{2.722570in}{2.175403in}}%
\pgfpathlineto{\pgfqpoint{2.727083in}{2.169134in}}%
\pgfpathlineto{\pgfqpoint{2.731597in}{2.169134in}}%
\pgfpathlineto{\pgfqpoint{2.736111in}{2.162866in}}%
\pgfpathlineto{\pgfqpoint{2.740624in}{2.162866in}}%
\pgfpathlineto{\pgfqpoint{2.745138in}{2.156597in}}%
\pgfpathlineto{\pgfqpoint{2.749651in}{2.156597in}}%
\pgfpathlineto{\pgfqpoint{2.754165in}{2.150328in}}%
\pgfpathlineto{\pgfqpoint{2.758679in}{2.150328in}}%
\pgfpathlineto{\pgfqpoint{2.763192in}{2.144060in}}%
\pgfpathlineto{\pgfqpoint{2.767706in}{2.144060in}}%
\pgfpathlineto{\pgfqpoint{2.772219in}{2.137791in}}%
\pgfpathlineto{\pgfqpoint{2.776733in}{2.137791in}}%
\pgfpathlineto{\pgfqpoint{2.781247in}{2.131522in}}%
\pgfpathlineto{\pgfqpoint{2.785760in}{2.131522in}}%
\pgfpathlineto{\pgfqpoint{2.790274in}{2.125254in}}%
\pgfpathlineto{\pgfqpoint{2.799301in}{2.125254in}}%
\pgfpathlineto{\pgfqpoint{2.803815in}{2.118985in}}%
\pgfpathlineto{\pgfqpoint{2.808328in}{2.118985in}}%
\pgfpathlineto{\pgfqpoint{2.812842in}{2.112716in}}%
\pgfpathlineto{\pgfqpoint{2.817356in}{2.112716in}}%
\pgfpathlineto{\pgfqpoint{2.821869in}{2.106448in}}%
\pgfpathlineto{\pgfqpoint{2.826383in}{2.106448in}}%
\pgfpathlineto{\pgfqpoint{2.830896in}{2.100179in}}%
\pgfpathlineto{\pgfqpoint{2.835410in}{2.100179in}}%
\pgfpathlineto{\pgfqpoint{2.839924in}{2.093910in}}%
\pgfpathlineto{\pgfqpoint{2.848951in}{2.093910in}}%
\pgfpathlineto{\pgfqpoint{2.853464in}{2.087642in}}%
\pgfpathlineto{\pgfqpoint{2.857978in}{2.087642in}}%
\pgfpathlineto{\pgfqpoint{2.862492in}{2.081373in}}%
\pgfpathlineto{\pgfqpoint{2.867005in}{2.081373in}}%
\pgfpathlineto{\pgfqpoint{2.871519in}{2.075104in}}%
\pgfpathlineto{\pgfqpoint{2.880546in}{2.075104in}}%
\pgfpathlineto{\pgfqpoint{2.885060in}{2.068836in}}%
\pgfpathlineto{\pgfqpoint{2.889573in}{2.068836in}}%
\pgfpathlineto{\pgfqpoint{2.894087in}{2.062567in}}%
\pgfpathlineto{\pgfqpoint{2.903114in}{2.062567in}}%
\pgfpathlineto{\pgfqpoint{2.907628in}{2.056299in}}%
\pgfpathlineto{\pgfqpoint{2.912141in}{2.056299in}}%
\pgfpathlineto{\pgfqpoint{2.916655in}{2.050030in}}%
\pgfpathlineto{\pgfqpoint{2.925682in}{2.050030in}}%
\pgfpathlineto{\pgfqpoint{2.930196in}{2.043761in}}%
\pgfpathlineto{\pgfqpoint{2.939223in}{2.043761in}}%
\pgfpathlineto{\pgfqpoint{2.943736in}{2.037493in}}%
\pgfpathlineto{\pgfqpoint{2.948250in}{2.037493in}}%
\pgfpathlineto{\pgfqpoint{2.952764in}{2.031224in}}%
\pgfpathlineto{\pgfqpoint{2.961791in}{2.031224in}}%
\pgfpathlineto{\pgfqpoint{2.966304in}{2.024955in}}%
\pgfpathlineto{\pgfqpoint{2.975332in}{2.024955in}}%
\pgfpathlineto{\pgfqpoint{2.979845in}{2.018687in}}%
\pgfpathlineto{\pgfqpoint{2.988873in}{2.018687in}}%
\pgfpathlineto{\pgfqpoint{2.993386in}{2.012418in}}%
\pgfpathlineto{\pgfqpoint{3.006927in}{2.012418in}}%
\pgfpathlineto{\pgfqpoint{3.011441in}{2.006149in}}%
\pgfpathlineto{\pgfqpoint{3.020468in}{2.006149in}}%
\pgfpathlineto{\pgfqpoint{3.024981in}{1.999881in}}%
\pgfpathlineto{\pgfqpoint{3.034009in}{1.999881in}}%
\pgfpathlineto{\pgfqpoint{3.038522in}{1.993612in}}%
\pgfpathlineto{\pgfqpoint{3.052063in}{1.993612in}}%
\pgfpathlineto{\pgfqpoint{3.056577in}{1.987343in}}%
\pgfpathlineto{\pgfqpoint{3.070117in}{1.987343in}}%
\pgfpathlineto{\pgfqpoint{3.074631in}{1.981075in}}%
\pgfpathlineto{\pgfqpoint{3.088172in}{1.981075in}}%
\pgfpathlineto{\pgfqpoint{3.092685in}{1.974806in}}%
\pgfpathlineto{\pgfqpoint{3.110740in}{1.974806in}}%
\pgfpathlineto{\pgfqpoint{3.115253in}{1.968537in}}%
\pgfpathlineto{\pgfqpoint{3.128794in}{1.968537in}}%
\pgfpathlineto{\pgfqpoint{3.133308in}{1.962269in}}%
\pgfpathlineto{\pgfqpoint{3.151362in}{1.962269in}}%
\pgfpathlineto{\pgfqpoint{3.155876in}{1.956000in}}%
\pgfpathlineto{\pgfqpoint{3.178444in}{1.956000in}}%
\pgfpathlineto{\pgfqpoint{3.182958in}{1.949731in}}%
\pgfpathlineto{\pgfqpoint{3.205526in}{1.949731in}}%
\pgfpathlineto{\pgfqpoint{3.210039in}{1.943463in}}%
\pgfpathlineto{\pgfqpoint{3.246148in}{1.943463in}}%
\pgfpathlineto{\pgfqpoint{3.250662in}{1.937194in}}%
\pgfpathlineto{\pgfqpoint{3.431206in}{1.937194in}}%
\pgfpathlineto{\pgfqpoint{3.435719in}{1.943463in}}%
\pgfpathlineto{\pgfqpoint{3.467315in}{1.943463in}}%
\pgfpathlineto{\pgfqpoint{3.471828in}{1.949731in}}%
\pgfpathlineto{\pgfqpoint{3.498910in}{1.949731in}}%
\pgfpathlineto{\pgfqpoint{3.503423in}{1.956000in}}%
\pgfpathlineto{\pgfqpoint{3.530505in}{1.956000in}}%
\pgfpathlineto{\pgfqpoint{3.535019in}{1.962269in}}%
\pgfpathlineto{\pgfqpoint{3.553073in}{1.962269in}}%
\pgfpathlineto{\pgfqpoint{3.557587in}{1.968537in}}%
\pgfpathlineto{\pgfqpoint{3.571127in}{1.968537in}}%
\pgfpathlineto{\pgfqpoint{3.575641in}{1.974806in}}%
\pgfpathlineto{\pgfqpoint{3.593696in}{1.974806in}}%
\pgfpathlineto{\pgfqpoint{3.598209in}{1.981075in}}%
\pgfpathlineto{\pgfqpoint{3.611750in}{1.981075in}}%
\pgfpathlineto{\pgfqpoint{3.616264in}{1.987343in}}%
\pgfpathlineto{\pgfqpoint{3.625291in}{1.987343in}}%
\pgfpathlineto{\pgfqpoint{3.629804in}{1.993612in}}%
\pgfpathlineto{\pgfqpoint{3.643345in}{1.993612in}}%
\pgfpathlineto{\pgfqpoint{3.647859in}{1.999881in}}%
\pgfpathlineto{\pgfqpoint{3.656886in}{1.999881in}}%
\pgfpathlineto{\pgfqpoint{3.661400in}{2.006149in}}%
\pgfpathlineto{\pgfqpoint{3.670427in}{2.006149in}}%
\pgfpathlineto{\pgfqpoint{3.674940in}{2.012418in}}%
\pgfpathlineto{\pgfqpoint{3.683968in}{2.012418in}}%
\pgfpathlineto{\pgfqpoint{3.688481in}{2.018687in}}%
\pgfpathlineto{\pgfqpoint{3.697508in}{2.018687in}}%
\pgfpathlineto{\pgfqpoint{3.702022in}{2.024955in}}%
\pgfpathlineto{\pgfqpoint{3.711049in}{2.024955in}}%
\pgfpathlineto{\pgfqpoint{3.715563in}{2.031224in}}%
\pgfpathlineto{\pgfqpoint{3.724590in}{2.031224in}}%
\pgfpathlineto{\pgfqpoint{3.729104in}{2.037493in}}%
\pgfpathlineto{\pgfqpoint{3.733617in}{2.037493in}}%
\pgfpathlineto{\pgfqpoint{3.738131in}{2.043761in}}%
\pgfpathlineto{\pgfqpoint{3.747158in}{2.043761in}}%
\pgfpathlineto{\pgfqpoint{3.751672in}{2.050030in}}%
\pgfpathlineto{\pgfqpoint{3.760699in}{2.050030in}}%
\pgfpathlineto{\pgfqpoint{3.765212in}{2.056299in}}%
\pgfpathlineto{\pgfqpoint{3.769726in}{2.056299in}}%
\pgfpathlineto{\pgfqpoint{3.774240in}{2.062567in}}%
\pgfpathlineto{\pgfqpoint{3.778753in}{2.062567in}}%
\pgfpathlineto{\pgfqpoint{3.783267in}{2.068836in}}%
\pgfpathlineto{\pgfqpoint{3.792294in}{2.068836in}}%
\pgfpathlineto{\pgfqpoint{3.796808in}{2.075104in}}%
\pgfpathlineto{\pgfqpoint{3.801321in}{2.075104in}}%
\pgfpathlineto{\pgfqpoint{3.805835in}{2.081373in}}%
\pgfpathlineto{\pgfqpoint{3.810349in}{2.081373in}}%
\pgfpathlineto{\pgfqpoint{3.814862in}{2.087642in}}%
\pgfpathlineto{\pgfqpoint{3.823889in}{2.087642in}}%
\pgfpathlineto{\pgfqpoint{3.828403in}{2.093910in}}%
\pgfpathlineto{\pgfqpoint{3.832917in}{2.093910in}}%
\pgfpathlineto{\pgfqpoint{3.837430in}{2.100179in}}%
\pgfpathlineto{\pgfqpoint{3.841944in}{2.100179in}}%
\pgfpathlineto{\pgfqpoint{3.846457in}{2.106448in}}%
\pgfpathlineto{\pgfqpoint{3.850971in}{2.106448in}}%
\pgfpathlineto{\pgfqpoint{3.855485in}{2.112716in}}%
\pgfpathlineto{\pgfqpoint{3.859998in}{2.112716in}}%
\pgfpathlineto{\pgfqpoint{3.864512in}{2.118985in}}%
\pgfpathlineto{\pgfqpoint{3.869025in}{2.118985in}}%
\pgfpathlineto{\pgfqpoint{3.873539in}{2.125254in}}%
\pgfpathlineto{\pgfqpoint{3.878053in}{2.125254in}}%
\pgfpathlineto{\pgfqpoint{3.882566in}{2.131522in}}%
\pgfpathlineto{\pgfqpoint{3.887080in}{2.131522in}}%
\pgfpathlineto{\pgfqpoint{3.891593in}{2.137791in}}%
\pgfpathlineto{\pgfqpoint{3.896107in}{2.137791in}}%
\pgfpathlineto{\pgfqpoint{3.900621in}{2.144060in}}%
\pgfpathlineto{\pgfqpoint{3.905134in}{2.144060in}}%
\pgfpathlineto{\pgfqpoint{3.909648in}{2.150328in}}%
\pgfpathlineto{\pgfqpoint{3.914161in}{2.150328in}}%
\pgfpathlineto{\pgfqpoint{3.918675in}{2.156597in}}%
\pgfpathlineto{\pgfqpoint{3.923189in}{2.156597in}}%
\pgfpathlineto{\pgfqpoint{3.927702in}{2.162866in}}%
\pgfpathlineto{\pgfqpoint{3.932216in}{2.162866in}}%
\pgfpathlineto{\pgfqpoint{3.936729in}{2.169134in}}%
\pgfpathlineto{\pgfqpoint{3.941243in}{2.169134in}}%
\pgfpathlineto{\pgfqpoint{3.950270in}{2.181672in}}%
\pgfpathlineto{\pgfqpoint{3.954784in}{2.181672in}}%
\pgfpathlineto{\pgfqpoint{3.959297in}{2.187940in}}%
\pgfpathlineto{\pgfqpoint{3.963811in}{2.187940in}}%
\pgfpathlineto{\pgfqpoint{3.968325in}{2.194209in}}%
\pgfpathlineto{\pgfqpoint{3.972838in}{2.194209in}}%
\pgfpathlineto{\pgfqpoint{3.977352in}{2.200478in}}%
\pgfpathlineto{\pgfqpoint{3.981866in}{2.200478in}}%
\pgfpathlineto{\pgfqpoint{3.990893in}{2.213015in}}%
\pgfpathlineto{\pgfqpoint{3.995406in}{2.213015in}}%
\pgfpathlineto{\pgfqpoint{3.999920in}{2.219284in}}%
\pgfpathlineto{\pgfqpoint{4.004434in}{2.219284in}}%
\pgfpathlineto{\pgfqpoint{4.008947in}{2.225552in}}%
\pgfpathlineto{\pgfqpoint{4.013461in}{2.225552in}}%
\pgfpathlineto{\pgfqpoint{4.022488in}{2.238090in}}%
\pgfpathlineto{\pgfqpoint{4.027002in}{2.238090in}}%
\pgfpathlineto{\pgfqpoint{4.031515in}{2.244358in}}%
\pgfpathlineto{\pgfqpoint{4.036029in}{2.244358in}}%
\pgfpathlineto{\pgfqpoint{4.045056in}{2.256896in}}%
\pgfpathlineto{\pgfqpoint{4.049570in}{2.256896in}}%
\pgfpathlineto{\pgfqpoint{4.054083in}{2.263164in}}%
\pgfpathlineto{\pgfqpoint{4.058597in}{2.263164in}}%
\pgfpathlineto{\pgfqpoint{4.067624in}{2.275701in}}%
\pgfpathlineto{\pgfqpoint{4.072138in}{2.275701in}}%
\pgfpathlineto{\pgfqpoint{4.081165in}{2.288239in}}%
\pgfpathlineto{\pgfqpoint{4.085678in}{2.288239in}}%
\pgfpathlineto{\pgfqpoint{4.090192in}{2.294507in}}%
\pgfpathlineto{\pgfqpoint{4.094706in}{2.294507in}}%
\pgfpathlineto{\pgfqpoint{4.103733in}{2.307045in}}%
\pgfpathlineto{\pgfqpoint{4.108246in}{2.307045in}}%
\pgfpathlineto{\pgfqpoint{4.117274in}{2.319582in}}%
\pgfpathlineto{\pgfqpoint{4.121787in}{2.319582in}}%
\pgfpathlineto{\pgfqpoint{4.130814in}{2.332119in}}%
\pgfpathlineto{\pgfqpoint{4.135328in}{2.332119in}}%
\pgfpathlineto{\pgfqpoint{4.144355in}{2.344657in}}%
\pgfpathlineto{\pgfqpoint{4.148869in}{2.344657in}}%
\pgfpathlineto{\pgfqpoint{4.157896in}{2.357194in}}%
\pgfpathlineto{\pgfqpoint{4.162410in}{2.357194in}}%
\pgfpathlineto{\pgfqpoint{4.171437in}{2.369731in}}%
\pgfpathlineto{\pgfqpoint{4.175950in}{2.369731in}}%
\pgfpathlineto{\pgfqpoint{4.184978in}{2.382269in}}%
\pgfpathlineto{\pgfqpoint{4.189491in}{2.382269in}}%
\pgfpathlineto{\pgfqpoint{4.203032in}{2.401075in}}%
\pgfpathlineto{\pgfqpoint{4.207546in}{2.401075in}}%
\pgfpathlineto{\pgfqpoint{4.216573in}{2.413612in}}%
\pgfpathlineto{\pgfqpoint{4.221087in}{2.413612in}}%
\pgfpathlineto{\pgfqpoint{4.234627in}{2.432418in}}%
\pgfpathlineto{\pgfqpoint{4.239141in}{2.432418in}}%
\pgfpathlineto{\pgfqpoint{4.248168in}{2.444955in}}%
\pgfpathlineto{\pgfqpoint{4.252682in}{2.444955in}}%
\pgfpathlineto{\pgfqpoint{4.266223in}{2.463761in}}%
\pgfpathlineto{\pgfqpoint{4.270736in}{2.463761in}}%
\pgfpathlineto{\pgfqpoint{4.284277in}{2.482567in}}%
\pgfpathlineto{\pgfqpoint{4.288791in}{2.482567in}}%
\pgfpathlineto{\pgfqpoint{4.297818in}{2.495104in}}%
\pgfpathlineto{\pgfqpoint{4.302331in}{2.495104in}}%
\pgfpathlineto{\pgfqpoint{4.320386in}{2.520179in}}%
\pgfpathlineto{\pgfqpoint{4.324899in}{2.520179in}}%
\pgfpathlineto{\pgfqpoint{4.338440in}{2.538985in}}%
\pgfpathlineto{\pgfqpoint{4.342954in}{2.538985in}}%
\pgfpathlineto{\pgfqpoint{4.356495in}{2.557791in}}%
\pgfpathlineto{\pgfqpoint{4.361008in}{2.557791in}}%
\pgfpathlineto{\pgfqpoint{4.374549in}{2.576597in}}%
\pgfpathlineto{\pgfqpoint{4.379063in}{2.576597in}}%
\pgfpathlineto{\pgfqpoint{4.397117in}{2.601672in}}%
\pgfpathlineto{\pgfqpoint{4.401631in}{2.601672in}}%
\pgfpathlineto{\pgfqpoint{4.419685in}{2.626746in}}%
\pgfpathlineto{\pgfqpoint{4.424199in}{2.626746in}}%
\pgfpathlineto{\pgfqpoint{4.442253in}{2.651821in}}%
\pgfpathlineto{\pgfqpoint{4.446767in}{2.651821in}}%
\pgfpathlineto{\pgfqpoint{4.464821in}{2.676896in}}%
\pgfpathlineto{\pgfqpoint{4.469335in}{2.676896in}}%
\pgfpathlineto{\pgfqpoint{4.491903in}{2.708239in}}%
\pgfpathlineto{\pgfqpoint{4.496416in}{2.708239in}}%
\pgfpathlineto{\pgfqpoint{4.514471in}{2.733313in}}%
\pgfpathlineto{\pgfqpoint{4.518984in}{2.733313in}}%
\pgfpathlineto{\pgfqpoint{4.541552in}{2.764657in}}%
\pgfpathlineto{\pgfqpoint{4.546066in}{2.764657in}}%
\pgfpathlineto{\pgfqpoint{4.568634in}{2.796000in}}%
\pgfpathlineto{\pgfqpoint{4.573148in}{2.796000in}}%
\pgfpathlineto{\pgfqpoint{4.600229in}{2.833612in}}%
\pgfpathlineto{\pgfqpoint{4.604743in}{2.833612in}}%
\pgfpathlineto{\pgfqpoint{4.631825in}{2.871224in}}%
\pgfpathlineto{\pgfqpoint{4.636338in}{2.871224in}}%
\pgfpathlineto{\pgfqpoint{4.663420in}{2.908836in}}%
\pgfpathlineto{\pgfqpoint{4.667933in}{2.908836in}}%
\pgfpathlineto{\pgfqpoint{4.699529in}{2.952716in}}%
\pgfpathlineto{\pgfqpoint{4.704042in}{2.952716in}}%
\pgfpathlineto{\pgfqpoint{4.735637in}{2.996597in}}%
\pgfpathlineto{\pgfqpoint{4.740151in}{2.996597in}}%
\pgfpathlineto{\pgfqpoint{4.780774in}{3.053015in}}%
\pgfpathlineto{\pgfqpoint{4.785287in}{3.053015in}}%
\pgfpathlineto{\pgfqpoint{4.825910in}{3.109433in}}%
\pgfpathlineto{\pgfqpoint{4.830423in}{3.109433in}}%
\pgfpathlineto{\pgfqpoint{4.875559in}{3.172119in}}%
\pgfpathlineto{\pgfqpoint{4.880073in}{3.172119in}}%
\pgfpathlineto{\pgfqpoint{4.934236in}{3.247343in}}%
\pgfpathlineto{\pgfqpoint{4.938750in}{3.247343in}}%
\pgfpathlineto{\pgfqpoint{5.001940in}{3.335104in}}%
\pgfpathlineto{\pgfqpoint{5.006454in}{3.335104in}}%
\pgfpathlineto{\pgfqpoint{5.092212in}{3.454209in}}%
\pgfpathlineto{\pgfqpoint{5.096726in}{3.454209in}}%
\pgfpathlineto{\pgfqpoint{5.214080in}{3.617194in}}%
\pgfpathlineto{\pgfqpoint{5.218593in}{3.617194in}}%
\pgfpathlineto{\pgfqpoint{5.534545in}{4.056000in}}%
\pgfpathlineto{\pgfqpoint{5.534545in}{4.056000in}}%
\pgfusepath{stroke}%
\end{pgfscope}%
\begin{pgfscope}%
\pgfpathrectangle{\pgfqpoint{0.800000in}{0.528000in}}{\pgfqpoint{4.960000in}{3.696000in}}%
\pgfusepath{clip}%
\pgfsetrectcap%
\pgfsetroundjoin%
\pgfsetlinewidth{1.505625pt}%
\definecolor{currentstroke}{rgb}{1.000000,0.000000,0.000000}%
\pgfsetstrokecolor{currentstroke}%
\pgfsetdash{}{0pt}%
\pgfpathmoveto{\pgfqpoint{1.025455in}{3.830328in}}%
\pgfpathlineto{\pgfqpoint{3.282257in}{0.696000in}}%
\pgfpathlineto{\pgfqpoint{5.534545in}{3.824060in}}%
\pgfpathlineto{\pgfqpoint{5.534545in}{3.824060in}}%
\pgfusepath{stroke}%
\end{pgfscope}%
\begin{pgfscope}%
\pgfsetrectcap%
\pgfsetmiterjoin%
\pgfsetlinewidth{0.803000pt}%
\definecolor{currentstroke}{rgb}{0.000000,0.000000,0.000000}%
\pgfsetstrokecolor{currentstroke}%
\pgfsetdash{}{0pt}%
\pgfpathmoveto{\pgfqpoint{0.800000in}{0.528000in}}%
\pgfpathlineto{\pgfqpoint{0.800000in}{4.224000in}}%
\pgfusepath{stroke}%
\end{pgfscope}%
\begin{pgfscope}%
\pgfsetrectcap%
\pgfsetmiterjoin%
\pgfsetlinewidth{0.803000pt}%
\definecolor{currentstroke}{rgb}{0.000000,0.000000,0.000000}%
\pgfsetstrokecolor{currentstroke}%
\pgfsetdash{}{0pt}%
\pgfpathmoveto{\pgfqpoint{5.760000in}{0.528000in}}%
\pgfpathlineto{\pgfqpoint{5.760000in}{4.224000in}}%
\pgfusepath{stroke}%
\end{pgfscope}%
\begin{pgfscope}%
\pgfsetrectcap%
\pgfsetmiterjoin%
\pgfsetlinewidth{0.803000pt}%
\definecolor{currentstroke}{rgb}{0.000000,0.000000,0.000000}%
\pgfsetstrokecolor{currentstroke}%
\pgfsetdash{}{0pt}%
\pgfpathmoveto{\pgfqpoint{0.800000in}{0.528000in}}%
\pgfpathlineto{\pgfqpoint{5.760000in}{0.528000in}}%
\pgfusepath{stroke}%
\end{pgfscope}%
\begin{pgfscope}%
\pgfsetrectcap%
\pgfsetmiterjoin%
\pgfsetlinewidth{0.803000pt}%
\definecolor{currentstroke}{rgb}{0.000000,0.000000,0.000000}%
\pgfsetstrokecolor{currentstroke}%
\pgfsetdash{}{0pt}%
\pgfpathmoveto{\pgfqpoint{0.800000in}{4.224000in}}%
\pgfpathlineto{\pgfqpoint{5.760000in}{4.224000in}}%
\pgfusepath{stroke}%
\end{pgfscope}%
\begin{pgfscope}%
\pgfsetbuttcap%
\pgfsetmiterjoin%
\definecolor{currentfill}{rgb}{1.000000,1.000000,1.000000}%
\pgfsetfillcolor{currentfill}%
\pgfsetfillopacity{0.800000}%
\pgfsetlinewidth{1.003750pt}%
\definecolor{currentstroke}{rgb}{0.800000,0.800000,0.800000}%
\pgfsetstrokecolor{currentstroke}%
\pgfsetstrokeopacity{0.800000}%
\pgfsetdash{}{0pt}%
\pgfpathmoveto{\pgfqpoint{0.897222in}{0.597444in}}%
\pgfpathlineto{\pgfqpoint{2.514509in}{0.597444in}}%
\pgfpathquadraticcurveto{\pgfqpoint{2.542287in}{0.597444in}}{\pgfqpoint{2.542287in}{0.625222in}}%
\pgfpathlineto{\pgfqpoint{2.542287in}{0.998679in}}%
\pgfpathquadraticcurveto{\pgfqpoint{2.542287in}{1.026457in}}{\pgfqpoint{2.514509in}{1.026457in}}%
\pgfpathlineto{\pgfqpoint{0.897222in}{1.026457in}}%
\pgfpathquadraticcurveto{\pgfqpoint{0.869444in}{1.026457in}}{\pgfqpoint{0.869444in}{0.998679in}}%
\pgfpathlineto{\pgfqpoint{0.869444in}{0.625222in}}%
\pgfpathquadraticcurveto{\pgfqpoint{0.869444in}{0.597444in}}{\pgfqpoint{0.897222in}{0.597444in}}%
\pgfpathclose%
\pgfusepath{stroke,fill}%
\end{pgfscope}%
\begin{pgfscope}%
\pgfsetrectcap%
\pgfsetroundjoin%
\pgfsetlinewidth{1.505625pt}%
\definecolor{currentstroke}{rgb}{0.000000,0.000000,1.000000}%
\pgfsetstrokecolor{currentstroke}%
\pgfsetdash{}{0pt}%
\pgfpathmoveto{\pgfqpoint{0.925000in}{0.922290in}}%
\pgfpathlineto{\pgfqpoint{1.202778in}{0.922290in}}%
\pgfusepath{stroke}%
\end{pgfscope}%
\begin{pgfscope}%
\definecolor{textcolor}{rgb}{0.000000,0.000000,0.000000}%
\pgfsetstrokecolor{textcolor}%
\pgfsetfillcolor{textcolor}%
\pgftext[x=1.313889in,y=0.873679in,left,base]{\color{textcolor}\rmfamily\fontsize{10.000000}{12.000000}\selectfont Numerical Solution}%
\end{pgfscope}%
\begin{pgfscope}%
\pgfsetrectcap%
\pgfsetroundjoin%
\pgfsetlinewidth{1.505625pt}%
\definecolor{currentstroke}{rgb}{1.000000,0.000000,0.000000}%
\pgfsetstrokecolor{currentstroke}%
\pgfsetdash{}{0pt}%
\pgfpathmoveto{\pgfqpoint{0.925000in}{0.728617in}}%
\pgfpathlineto{\pgfqpoint{1.202778in}{0.728617in}}%
\pgfusepath{stroke}%
\end{pgfscope}%
\begin{pgfscope}%
\definecolor{textcolor}{rgb}{0.000000,0.000000,0.000000}%
\pgfsetstrokecolor{textcolor}%
\pgfsetfillcolor{textcolor}%
\pgftext[x=1.313889in,y=0.680006in,left,base]{\color{textcolor}\rmfamily\fontsize{10.000000}{12.000000}\selectfont Analytical Solution}%
\end{pgfscope}%
\end{pgfpicture}%
\makeatother%
\endgroup%
}
                    \caption{Numerical Determination of $\epsilon_1(\nu)$}
                    \label{fig:nu1 Numerical}
                \end{center}
            \end{figure}
            \noindent
            So our numerical result agrees somewhat with our analytical solution, with some 
            exceptions for numerical methods being imperfect.

            \item For $\nu \approx 4$, we do exactly the same thing but with values around 4, producing 
            \autoref{fig:nu4 Numerical} below.
            \begin{figure}[H]
                \begin{center}
                    \scalebox{.7}{%% Creator: Matplotlib, PGF backend
%%
%% To include the figure in your LaTeX document, write
%%   \input{<filename>.pgf}
%%
%% Make sure the required packages are loaded in your preamble
%%   \usepackage{pgf}
%%
%% Figures using additional raster images can only be included by \input if
%% they are in the same directory as the main LaTeX file. For loading figures
%% from other directories you can use the `import` package
%%   \usepackage{import}
%% and then include the figures with
%%   \import{<path to file>}{<filename>.pgf}
%%
%% Matplotlib used the following preamble
%%
\begingroup%
\makeatletter%
\begin{pgfpicture}%
\pgfpathrectangle{\pgfpointorigin}{\pgfqpoint{6.400000in}{4.800000in}}%
\pgfusepath{use as bounding box, clip}%
\begin{pgfscope}%
\pgfsetbuttcap%
\pgfsetmiterjoin%
\definecolor{currentfill}{rgb}{1.000000,1.000000,1.000000}%
\pgfsetfillcolor{currentfill}%
\pgfsetlinewidth{0.000000pt}%
\definecolor{currentstroke}{rgb}{1.000000,1.000000,1.000000}%
\pgfsetstrokecolor{currentstroke}%
\pgfsetdash{}{0pt}%
\pgfpathmoveto{\pgfqpoint{0.000000in}{0.000000in}}%
\pgfpathlineto{\pgfqpoint{6.400000in}{0.000000in}}%
\pgfpathlineto{\pgfqpoint{6.400000in}{4.800000in}}%
\pgfpathlineto{\pgfqpoint{0.000000in}{4.800000in}}%
\pgfpathclose%
\pgfusepath{fill}%
\end{pgfscope}%
\begin{pgfscope}%
\pgfsetbuttcap%
\pgfsetmiterjoin%
\definecolor{currentfill}{rgb}{1.000000,1.000000,1.000000}%
\pgfsetfillcolor{currentfill}%
\pgfsetlinewidth{0.000000pt}%
\definecolor{currentstroke}{rgb}{0.000000,0.000000,0.000000}%
\pgfsetstrokecolor{currentstroke}%
\pgfsetstrokeopacity{0.000000}%
\pgfsetdash{}{0pt}%
\pgfpathmoveto{\pgfqpoint{0.800000in}{0.528000in}}%
\pgfpathlineto{\pgfqpoint{5.760000in}{0.528000in}}%
\pgfpathlineto{\pgfqpoint{5.760000in}{4.224000in}}%
\pgfpathlineto{\pgfqpoint{0.800000in}{4.224000in}}%
\pgfpathclose%
\pgfusepath{fill}%
\end{pgfscope}%
\begin{pgfscope}%
\pgfsetbuttcap%
\pgfsetroundjoin%
\definecolor{currentfill}{rgb}{0.000000,0.000000,0.000000}%
\pgfsetfillcolor{currentfill}%
\pgfsetlinewidth{0.803000pt}%
\definecolor{currentstroke}{rgb}{0.000000,0.000000,0.000000}%
\pgfsetstrokecolor{currentstroke}%
\pgfsetdash{}{0pt}%
\pgfsys@defobject{currentmarker}{\pgfqpoint{0.000000in}{-0.048611in}}{\pgfqpoint{0.000000in}{0.000000in}}{%
\pgfpathmoveto{\pgfqpoint{0.000000in}{0.000000in}}%
\pgfpathlineto{\pgfqpoint{0.000000in}{-0.048611in}}%
\pgfusepath{stroke,fill}%
}%
\begin{pgfscope}%
\pgfsys@transformshift{1.326362in}{0.528000in}%
\pgfsys@useobject{currentmarker}{}%
\end{pgfscope}%
\end{pgfscope}%
\begin{pgfscope}%
\definecolor{textcolor}{rgb}{0.000000,0.000000,0.000000}%
\pgfsetstrokecolor{textcolor}%
\pgfsetfillcolor{textcolor}%
\pgftext[x=1.326362in,y=0.430778in,,top]{\color{textcolor}\rmfamily\fontsize{10.000000}{12.000000}\selectfont \(\displaystyle 3.8\)}%
\end{pgfscope}%
\begin{pgfscope}%
\pgfsetbuttcap%
\pgfsetroundjoin%
\definecolor{currentfill}{rgb}{0.000000,0.000000,0.000000}%
\pgfsetfillcolor{currentfill}%
\pgfsetlinewidth{0.803000pt}%
\definecolor{currentstroke}{rgb}{0.000000,0.000000,0.000000}%
\pgfsetstrokecolor{currentstroke}%
\pgfsetdash{}{0pt}%
\pgfsys@defobject{currentmarker}{\pgfqpoint{0.000000in}{-0.048611in}}{\pgfqpoint{0.000000in}{0.000000in}}{%
\pgfpathmoveto{\pgfqpoint{0.000000in}{0.000000in}}%
\pgfpathlineto{\pgfqpoint{0.000000in}{-0.048611in}}%
\pgfusepath{stroke,fill}%
}%
\begin{pgfscope}%
\pgfsys@transformshift{1.928175in}{0.528000in}%
\pgfsys@useobject{currentmarker}{}%
\end{pgfscope}%
\end{pgfscope}%
\begin{pgfscope}%
\definecolor{textcolor}{rgb}{0.000000,0.000000,0.000000}%
\pgfsetstrokecolor{textcolor}%
\pgfsetfillcolor{textcolor}%
\pgftext[x=1.928175in,y=0.430778in,,top]{\color{textcolor}\rmfamily\fontsize{10.000000}{12.000000}\selectfont \(\displaystyle 3.9\)}%
\end{pgfscope}%
\begin{pgfscope}%
\pgfsetbuttcap%
\pgfsetroundjoin%
\definecolor{currentfill}{rgb}{0.000000,0.000000,0.000000}%
\pgfsetfillcolor{currentfill}%
\pgfsetlinewidth{0.803000pt}%
\definecolor{currentstroke}{rgb}{0.000000,0.000000,0.000000}%
\pgfsetstrokecolor{currentstroke}%
\pgfsetdash{}{0pt}%
\pgfsys@defobject{currentmarker}{\pgfqpoint{0.000000in}{-0.048611in}}{\pgfqpoint{0.000000in}{0.000000in}}{%
\pgfpathmoveto{\pgfqpoint{0.000000in}{0.000000in}}%
\pgfpathlineto{\pgfqpoint{0.000000in}{-0.048611in}}%
\pgfusepath{stroke,fill}%
}%
\begin{pgfscope}%
\pgfsys@transformshift{2.529989in}{0.528000in}%
\pgfsys@useobject{currentmarker}{}%
\end{pgfscope}%
\end{pgfscope}%
\begin{pgfscope}%
\definecolor{textcolor}{rgb}{0.000000,0.000000,0.000000}%
\pgfsetstrokecolor{textcolor}%
\pgfsetfillcolor{textcolor}%
\pgftext[x=2.529989in,y=0.430778in,,top]{\color{textcolor}\rmfamily\fontsize{10.000000}{12.000000}\selectfont \(\displaystyle 4.0\)}%
\end{pgfscope}%
\begin{pgfscope}%
\pgfsetbuttcap%
\pgfsetroundjoin%
\definecolor{currentfill}{rgb}{0.000000,0.000000,0.000000}%
\pgfsetfillcolor{currentfill}%
\pgfsetlinewidth{0.803000pt}%
\definecolor{currentstroke}{rgb}{0.000000,0.000000,0.000000}%
\pgfsetstrokecolor{currentstroke}%
\pgfsetdash{}{0pt}%
\pgfsys@defobject{currentmarker}{\pgfqpoint{0.000000in}{-0.048611in}}{\pgfqpoint{0.000000in}{0.000000in}}{%
\pgfpathmoveto{\pgfqpoint{0.000000in}{0.000000in}}%
\pgfpathlineto{\pgfqpoint{0.000000in}{-0.048611in}}%
\pgfusepath{stroke,fill}%
}%
\begin{pgfscope}%
\pgfsys@transformshift{3.131803in}{0.528000in}%
\pgfsys@useobject{currentmarker}{}%
\end{pgfscope}%
\end{pgfscope}%
\begin{pgfscope}%
\definecolor{textcolor}{rgb}{0.000000,0.000000,0.000000}%
\pgfsetstrokecolor{textcolor}%
\pgfsetfillcolor{textcolor}%
\pgftext[x=3.131803in,y=0.430778in,,top]{\color{textcolor}\rmfamily\fontsize{10.000000}{12.000000}\selectfont \(\displaystyle 4.1\)}%
\end{pgfscope}%
\begin{pgfscope}%
\pgfsetbuttcap%
\pgfsetroundjoin%
\definecolor{currentfill}{rgb}{0.000000,0.000000,0.000000}%
\pgfsetfillcolor{currentfill}%
\pgfsetlinewidth{0.803000pt}%
\definecolor{currentstroke}{rgb}{0.000000,0.000000,0.000000}%
\pgfsetstrokecolor{currentstroke}%
\pgfsetdash{}{0pt}%
\pgfsys@defobject{currentmarker}{\pgfqpoint{0.000000in}{-0.048611in}}{\pgfqpoint{0.000000in}{0.000000in}}{%
\pgfpathmoveto{\pgfqpoint{0.000000in}{0.000000in}}%
\pgfpathlineto{\pgfqpoint{0.000000in}{-0.048611in}}%
\pgfusepath{stroke,fill}%
}%
\begin{pgfscope}%
\pgfsys@transformshift{3.733617in}{0.528000in}%
\pgfsys@useobject{currentmarker}{}%
\end{pgfscope}%
\end{pgfscope}%
\begin{pgfscope}%
\definecolor{textcolor}{rgb}{0.000000,0.000000,0.000000}%
\pgfsetstrokecolor{textcolor}%
\pgfsetfillcolor{textcolor}%
\pgftext[x=3.733617in,y=0.430778in,,top]{\color{textcolor}\rmfamily\fontsize{10.000000}{12.000000}\selectfont \(\displaystyle 4.2\)}%
\end{pgfscope}%
\begin{pgfscope}%
\pgfsetbuttcap%
\pgfsetroundjoin%
\definecolor{currentfill}{rgb}{0.000000,0.000000,0.000000}%
\pgfsetfillcolor{currentfill}%
\pgfsetlinewidth{0.803000pt}%
\definecolor{currentstroke}{rgb}{0.000000,0.000000,0.000000}%
\pgfsetstrokecolor{currentstroke}%
\pgfsetdash{}{0pt}%
\pgfsys@defobject{currentmarker}{\pgfqpoint{0.000000in}{-0.048611in}}{\pgfqpoint{0.000000in}{0.000000in}}{%
\pgfpathmoveto{\pgfqpoint{0.000000in}{0.000000in}}%
\pgfpathlineto{\pgfqpoint{0.000000in}{-0.048611in}}%
\pgfusepath{stroke,fill}%
}%
\begin{pgfscope}%
\pgfsys@transformshift{4.335431in}{0.528000in}%
\pgfsys@useobject{currentmarker}{}%
\end{pgfscope}%
\end{pgfscope}%
\begin{pgfscope}%
\definecolor{textcolor}{rgb}{0.000000,0.000000,0.000000}%
\pgfsetstrokecolor{textcolor}%
\pgfsetfillcolor{textcolor}%
\pgftext[x=4.335431in,y=0.430778in,,top]{\color{textcolor}\rmfamily\fontsize{10.000000}{12.000000}\selectfont \(\displaystyle 4.3\)}%
\end{pgfscope}%
\begin{pgfscope}%
\pgfsetbuttcap%
\pgfsetroundjoin%
\definecolor{currentfill}{rgb}{0.000000,0.000000,0.000000}%
\pgfsetfillcolor{currentfill}%
\pgfsetlinewidth{0.803000pt}%
\definecolor{currentstroke}{rgb}{0.000000,0.000000,0.000000}%
\pgfsetstrokecolor{currentstroke}%
\pgfsetdash{}{0pt}%
\pgfsys@defobject{currentmarker}{\pgfqpoint{0.000000in}{-0.048611in}}{\pgfqpoint{0.000000in}{0.000000in}}{%
\pgfpathmoveto{\pgfqpoint{0.000000in}{0.000000in}}%
\pgfpathlineto{\pgfqpoint{0.000000in}{-0.048611in}}%
\pgfusepath{stroke,fill}%
}%
\begin{pgfscope}%
\pgfsys@transformshift{4.937245in}{0.528000in}%
\pgfsys@useobject{currentmarker}{}%
\end{pgfscope}%
\end{pgfscope}%
\begin{pgfscope}%
\definecolor{textcolor}{rgb}{0.000000,0.000000,0.000000}%
\pgfsetstrokecolor{textcolor}%
\pgfsetfillcolor{textcolor}%
\pgftext[x=4.937245in,y=0.430778in,,top]{\color{textcolor}\rmfamily\fontsize{10.000000}{12.000000}\selectfont \(\displaystyle 4.4\)}%
\end{pgfscope}%
\begin{pgfscope}%
\pgfsetbuttcap%
\pgfsetroundjoin%
\definecolor{currentfill}{rgb}{0.000000,0.000000,0.000000}%
\pgfsetfillcolor{currentfill}%
\pgfsetlinewidth{0.803000pt}%
\definecolor{currentstroke}{rgb}{0.000000,0.000000,0.000000}%
\pgfsetstrokecolor{currentstroke}%
\pgfsetdash{}{0pt}%
\pgfsys@defobject{currentmarker}{\pgfqpoint{0.000000in}{-0.048611in}}{\pgfqpoint{0.000000in}{0.000000in}}{%
\pgfpathmoveto{\pgfqpoint{0.000000in}{0.000000in}}%
\pgfpathlineto{\pgfqpoint{0.000000in}{-0.048611in}}%
\pgfusepath{stroke,fill}%
}%
\begin{pgfscope}%
\pgfsys@transformshift{5.539059in}{0.528000in}%
\pgfsys@useobject{currentmarker}{}%
\end{pgfscope}%
\end{pgfscope}%
\begin{pgfscope}%
\definecolor{textcolor}{rgb}{0.000000,0.000000,0.000000}%
\pgfsetstrokecolor{textcolor}%
\pgfsetfillcolor{textcolor}%
\pgftext[x=5.539059in,y=0.430778in,,top]{\color{textcolor}\rmfamily\fontsize{10.000000}{12.000000}\selectfont \(\displaystyle 4.5\)}%
\end{pgfscope}%
\begin{pgfscope}%
\definecolor{textcolor}{rgb}{0.000000,0.000000,0.000000}%
\pgfsetstrokecolor{textcolor}%
\pgfsetfillcolor{textcolor}%
\pgftext[x=3.280000in,y=0.251766in,,top]{\color{textcolor}\rmfamily\fontsize{10.000000}{12.000000}\selectfont \(\displaystyle \nu\)}%
\end{pgfscope}%
\begin{pgfscope}%
\pgfsetbuttcap%
\pgfsetroundjoin%
\definecolor{currentfill}{rgb}{0.000000,0.000000,0.000000}%
\pgfsetfillcolor{currentfill}%
\pgfsetlinewidth{0.803000pt}%
\definecolor{currentstroke}{rgb}{0.000000,0.000000,0.000000}%
\pgfsetstrokecolor{currentstroke}%
\pgfsetdash{}{0pt}%
\pgfsys@defobject{currentmarker}{\pgfqpoint{-0.048611in}{0.000000in}}{\pgfqpoint{0.000000in}{0.000000in}}{%
\pgfpathmoveto{\pgfqpoint{0.000000in}{0.000000in}}%
\pgfpathlineto{\pgfqpoint{-0.048611in}{0.000000in}}%
\pgfusepath{stroke,fill}%
}%
\begin{pgfscope}%
\pgfsys@transformshift{0.800000in}{0.565492in}%
\pgfsys@useobject{currentmarker}{}%
\end{pgfscope}%
\end{pgfscope}%
\begin{pgfscope}%
\definecolor{textcolor}{rgb}{0.000000,0.000000,0.000000}%
\pgfsetstrokecolor{textcolor}%
\pgfsetfillcolor{textcolor}%
\pgftext[x=0.455863in,y=0.517266in,left,base]{\color{textcolor}\rmfamily\fontsize{10.000000}{12.000000}\selectfont \(\displaystyle 1.75\)}%
\end{pgfscope}%
\begin{pgfscope}%
\pgfsetbuttcap%
\pgfsetroundjoin%
\definecolor{currentfill}{rgb}{0.000000,0.000000,0.000000}%
\pgfsetfillcolor{currentfill}%
\pgfsetlinewidth{0.803000pt}%
\definecolor{currentstroke}{rgb}{0.000000,0.000000,0.000000}%
\pgfsetstrokecolor{currentstroke}%
\pgfsetdash{}{0pt}%
\pgfsys@defobject{currentmarker}{\pgfqpoint{-0.048611in}{0.000000in}}{\pgfqpoint{0.000000in}{0.000000in}}{%
\pgfpathmoveto{\pgfqpoint{0.000000in}{0.000000in}}%
\pgfpathlineto{\pgfqpoint{-0.048611in}{0.000000in}}%
\pgfusepath{stroke,fill}%
}%
\begin{pgfscope}%
\pgfsys@transformshift{0.800000in}{1.059842in}%
\pgfsys@useobject{currentmarker}{}%
\end{pgfscope}%
\end{pgfscope}%
\begin{pgfscope}%
\definecolor{textcolor}{rgb}{0.000000,0.000000,0.000000}%
\pgfsetstrokecolor{textcolor}%
\pgfsetfillcolor{textcolor}%
\pgftext[x=0.455863in,y=1.011617in,left,base]{\color{textcolor}\rmfamily\fontsize{10.000000}{12.000000}\selectfont \(\displaystyle 2.00\)}%
\end{pgfscope}%
\begin{pgfscope}%
\pgfsetbuttcap%
\pgfsetroundjoin%
\definecolor{currentfill}{rgb}{0.000000,0.000000,0.000000}%
\pgfsetfillcolor{currentfill}%
\pgfsetlinewidth{0.803000pt}%
\definecolor{currentstroke}{rgb}{0.000000,0.000000,0.000000}%
\pgfsetstrokecolor{currentstroke}%
\pgfsetdash{}{0pt}%
\pgfsys@defobject{currentmarker}{\pgfqpoint{-0.048611in}{0.000000in}}{\pgfqpoint{0.000000in}{0.000000in}}{%
\pgfpathmoveto{\pgfqpoint{0.000000in}{0.000000in}}%
\pgfpathlineto{\pgfqpoint{-0.048611in}{0.000000in}}%
\pgfusepath{stroke,fill}%
}%
\begin{pgfscope}%
\pgfsys@transformshift{0.800000in}{1.554192in}%
\pgfsys@useobject{currentmarker}{}%
\end{pgfscope}%
\end{pgfscope}%
\begin{pgfscope}%
\definecolor{textcolor}{rgb}{0.000000,0.000000,0.000000}%
\pgfsetstrokecolor{textcolor}%
\pgfsetfillcolor{textcolor}%
\pgftext[x=0.455863in,y=1.505967in,left,base]{\color{textcolor}\rmfamily\fontsize{10.000000}{12.000000}\selectfont \(\displaystyle 2.25\)}%
\end{pgfscope}%
\begin{pgfscope}%
\pgfsetbuttcap%
\pgfsetroundjoin%
\definecolor{currentfill}{rgb}{0.000000,0.000000,0.000000}%
\pgfsetfillcolor{currentfill}%
\pgfsetlinewidth{0.803000pt}%
\definecolor{currentstroke}{rgb}{0.000000,0.000000,0.000000}%
\pgfsetstrokecolor{currentstroke}%
\pgfsetdash{}{0pt}%
\pgfsys@defobject{currentmarker}{\pgfqpoint{-0.048611in}{0.000000in}}{\pgfqpoint{0.000000in}{0.000000in}}{%
\pgfpathmoveto{\pgfqpoint{0.000000in}{0.000000in}}%
\pgfpathlineto{\pgfqpoint{-0.048611in}{0.000000in}}%
\pgfusepath{stroke,fill}%
}%
\begin{pgfscope}%
\pgfsys@transformshift{0.800000in}{2.048542in}%
\pgfsys@useobject{currentmarker}{}%
\end{pgfscope}%
\end{pgfscope}%
\begin{pgfscope}%
\definecolor{textcolor}{rgb}{0.000000,0.000000,0.000000}%
\pgfsetstrokecolor{textcolor}%
\pgfsetfillcolor{textcolor}%
\pgftext[x=0.455863in,y=2.000317in,left,base]{\color{textcolor}\rmfamily\fontsize{10.000000}{12.000000}\selectfont \(\displaystyle 2.50\)}%
\end{pgfscope}%
\begin{pgfscope}%
\pgfsetbuttcap%
\pgfsetroundjoin%
\definecolor{currentfill}{rgb}{0.000000,0.000000,0.000000}%
\pgfsetfillcolor{currentfill}%
\pgfsetlinewidth{0.803000pt}%
\definecolor{currentstroke}{rgb}{0.000000,0.000000,0.000000}%
\pgfsetstrokecolor{currentstroke}%
\pgfsetdash{}{0pt}%
\pgfsys@defobject{currentmarker}{\pgfqpoint{-0.048611in}{0.000000in}}{\pgfqpoint{0.000000in}{0.000000in}}{%
\pgfpathmoveto{\pgfqpoint{0.000000in}{0.000000in}}%
\pgfpathlineto{\pgfqpoint{-0.048611in}{0.000000in}}%
\pgfusepath{stroke,fill}%
}%
\begin{pgfscope}%
\pgfsys@transformshift{0.800000in}{2.542893in}%
\pgfsys@useobject{currentmarker}{}%
\end{pgfscope}%
\end{pgfscope}%
\begin{pgfscope}%
\definecolor{textcolor}{rgb}{0.000000,0.000000,0.000000}%
\pgfsetstrokecolor{textcolor}%
\pgfsetfillcolor{textcolor}%
\pgftext[x=0.455863in,y=2.494667in,left,base]{\color{textcolor}\rmfamily\fontsize{10.000000}{12.000000}\selectfont \(\displaystyle 2.75\)}%
\end{pgfscope}%
\begin{pgfscope}%
\pgfsetbuttcap%
\pgfsetroundjoin%
\definecolor{currentfill}{rgb}{0.000000,0.000000,0.000000}%
\pgfsetfillcolor{currentfill}%
\pgfsetlinewidth{0.803000pt}%
\definecolor{currentstroke}{rgb}{0.000000,0.000000,0.000000}%
\pgfsetstrokecolor{currentstroke}%
\pgfsetdash{}{0pt}%
\pgfsys@defobject{currentmarker}{\pgfqpoint{-0.048611in}{0.000000in}}{\pgfqpoint{0.000000in}{0.000000in}}{%
\pgfpathmoveto{\pgfqpoint{0.000000in}{0.000000in}}%
\pgfpathlineto{\pgfqpoint{-0.048611in}{0.000000in}}%
\pgfusepath{stroke,fill}%
}%
\begin{pgfscope}%
\pgfsys@transformshift{0.800000in}{3.037243in}%
\pgfsys@useobject{currentmarker}{}%
\end{pgfscope}%
\end{pgfscope}%
\begin{pgfscope}%
\definecolor{textcolor}{rgb}{0.000000,0.000000,0.000000}%
\pgfsetstrokecolor{textcolor}%
\pgfsetfillcolor{textcolor}%
\pgftext[x=0.455863in,y=2.989018in,left,base]{\color{textcolor}\rmfamily\fontsize{10.000000}{12.000000}\selectfont \(\displaystyle 3.00\)}%
\end{pgfscope}%
\begin{pgfscope}%
\pgfsetbuttcap%
\pgfsetroundjoin%
\definecolor{currentfill}{rgb}{0.000000,0.000000,0.000000}%
\pgfsetfillcolor{currentfill}%
\pgfsetlinewidth{0.803000pt}%
\definecolor{currentstroke}{rgb}{0.000000,0.000000,0.000000}%
\pgfsetstrokecolor{currentstroke}%
\pgfsetdash{}{0pt}%
\pgfsys@defobject{currentmarker}{\pgfqpoint{-0.048611in}{0.000000in}}{\pgfqpoint{0.000000in}{0.000000in}}{%
\pgfpathmoveto{\pgfqpoint{0.000000in}{0.000000in}}%
\pgfpathlineto{\pgfqpoint{-0.048611in}{0.000000in}}%
\pgfusepath{stroke,fill}%
}%
\begin{pgfscope}%
\pgfsys@transformshift{0.800000in}{3.531593in}%
\pgfsys@useobject{currentmarker}{}%
\end{pgfscope}%
\end{pgfscope}%
\begin{pgfscope}%
\definecolor{textcolor}{rgb}{0.000000,0.000000,0.000000}%
\pgfsetstrokecolor{textcolor}%
\pgfsetfillcolor{textcolor}%
\pgftext[x=0.455863in,y=3.483368in,left,base]{\color{textcolor}\rmfamily\fontsize{10.000000}{12.000000}\selectfont \(\displaystyle 3.25\)}%
\end{pgfscope}%
\begin{pgfscope}%
\pgfsetbuttcap%
\pgfsetroundjoin%
\definecolor{currentfill}{rgb}{0.000000,0.000000,0.000000}%
\pgfsetfillcolor{currentfill}%
\pgfsetlinewidth{0.803000pt}%
\definecolor{currentstroke}{rgb}{0.000000,0.000000,0.000000}%
\pgfsetstrokecolor{currentstroke}%
\pgfsetdash{}{0pt}%
\pgfsys@defobject{currentmarker}{\pgfqpoint{-0.048611in}{0.000000in}}{\pgfqpoint{0.000000in}{0.000000in}}{%
\pgfpathmoveto{\pgfqpoint{0.000000in}{0.000000in}}%
\pgfpathlineto{\pgfqpoint{-0.048611in}{0.000000in}}%
\pgfusepath{stroke,fill}%
}%
\begin{pgfscope}%
\pgfsys@transformshift{0.800000in}{4.025944in}%
\pgfsys@useobject{currentmarker}{}%
\end{pgfscope}%
\end{pgfscope}%
\begin{pgfscope}%
\definecolor{textcolor}{rgb}{0.000000,0.000000,0.000000}%
\pgfsetstrokecolor{textcolor}%
\pgfsetfillcolor{textcolor}%
\pgftext[x=0.455863in,y=3.977718in,left,base]{\color{textcolor}\rmfamily\fontsize{10.000000}{12.000000}\selectfont \(\displaystyle 3.50\)}%
\end{pgfscope}%
\begin{pgfscope}%
\definecolor{textcolor}{rgb}{0.000000,0.000000,0.000000}%
\pgfsetstrokecolor{textcolor}%
\pgfsetfillcolor{textcolor}%
\pgftext[x=0.400308in,y=2.376000in,,bottom]{\color{textcolor}\rmfamily\fontsize{10.000000}{12.000000}\selectfont \(\displaystyle \epsilon\)}%
\end{pgfscope}%
\begin{pgfscope}%
\pgfpathrectangle{\pgfqpoint{0.800000in}{0.528000in}}{\pgfqpoint{4.960000in}{3.696000in}}%
\pgfusepath{clip}%
\pgfsetrectcap%
\pgfsetroundjoin%
\pgfsetlinewidth{1.505625pt}%
\definecolor{currentstroke}{rgb}{0.000000,0.000000,1.000000}%
\pgfsetstrokecolor{currentstroke}%
\pgfsetdash{}{0pt}%
\pgfpathmoveto{\pgfqpoint{1.025455in}{4.056000in}}%
\pgfpathlineto{\pgfqpoint{1.029968in}{4.041763in}}%
\pgfpathlineto{\pgfqpoint{1.079618in}{3.937356in}}%
\pgfpathlineto{\pgfqpoint{1.084131in}{3.923119in}}%
\pgfpathlineto{\pgfqpoint{1.120240in}{3.847186in}}%
\pgfpathlineto{\pgfqpoint{1.124754in}{3.832949in}}%
\pgfpathlineto{\pgfqpoint{1.156349in}{3.766508in}}%
\pgfpathlineto{\pgfqpoint{1.160863in}{3.752271in}}%
\pgfpathlineto{\pgfqpoint{1.192458in}{3.685831in}}%
\pgfpathlineto{\pgfqpoint{1.196972in}{3.671593in}}%
\pgfpathlineto{\pgfqpoint{1.224053in}{3.614644in}}%
\pgfpathlineto{\pgfqpoint{1.228567in}{3.600407in}}%
\pgfpathlineto{\pgfqpoint{1.251135in}{3.552949in}}%
\pgfpathlineto{\pgfqpoint{1.255648in}{3.538712in}}%
\pgfpathlineto{\pgfqpoint{1.278216in}{3.491254in}}%
\pgfpathlineto{\pgfqpoint{1.282730in}{3.477017in}}%
\pgfpathlineto{\pgfqpoint{1.300784in}{3.439051in}}%
\pgfpathlineto{\pgfqpoint{1.305298in}{3.424814in}}%
\pgfpathlineto{\pgfqpoint{1.327866in}{3.377356in}}%
\pgfpathlineto{\pgfqpoint{1.332380in}{3.363119in}}%
\pgfpathlineto{\pgfqpoint{1.350434in}{3.325153in}}%
\pgfpathlineto{\pgfqpoint{1.354948in}{3.310915in}}%
\pgfpathlineto{\pgfqpoint{1.373002in}{3.272949in}}%
\pgfpathlineto{\pgfqpoint{1.377516in}{3.258712in}}%
\pgfpathlineto{\pgfqpoint{1.395570in}{3.220746in}}%
\pgfpathlineto{\pgfqpoint{1.400084in}{3.206508in}}%
\pgfpathlineto{\pgfqpoint{1.413625in}{3.178034in}}%
\pgfpathlineto{\pgfqpoint{1.418138in}{3.163797in}}%
\pgfpathlineto{\pgfqpoint{1.436193in}{3.125831in}}%
\pgfpathlineto{\pgfqpoint{1.440706in}{3.111593in}}%
\pgfpathlineto{\pgfqpoint{1.454247in}{3.083119in}}%
\pgfpathlineto{\pgfqpoint{1.458761in}{3.068881in}}%
\pgfpathlineto{\pgfqpoint{1.476815in}{3.030915in}}%
\pgfpathlineto{\pgfqpoint{1.481329in}{3.016678in}}%
\pgfpathlineto{\pgfqpoint{1.494869in}{2.988203in}}%
\pgfpathlineto{\pgfqpoint{1.499383in}{2.973966in}}%
\pgfpathlineto{\pgfqpoint{1.512924in}{2.945492in}}%
\pgfpathlineto{\pgfqpoint{1.517437in}{2.931254in}}%
\pgfpathlineto{\pgfqpoint{1.530978in}{2.902780in}}%
\pgfpathlineto{\pgfqpoint{1.535492in}{2.888542in}}%
\pgfpathlineto{\pgfqpoint{1.549033in}{2.860068in}}%
\pgfpathlineto{\pgfqpoint{1.553546in}{2.845831in}}%
\pgfpathlineto{\pgfqpoint{1.567087in}{2.817356in}}%
\pgfpathlineto{\pgfqpoint{1.571601in}{2.803119in}}%
\pgfpathlineto{\pgfqpoint{1.585142in}{2.774644in}}%
\pgfpathlineto{\pgfqpoint{1.589655in}{2.760407in}}%
\pgfpathlineto{\pgfqpoint{1.607710in}{2.722441in}}%
\pgfpathlineto{\pgfqpoint{1.612223in}{2.708203in}}%
\pgfpathlineto{\pgfqpoint{1.625764in}{2.679729in}}%
\pgfpathlineto{\pgfqpoint{1.630278in}{2.665492in}}%
\pgfpathlineto{\pgfqpoint{1.643818in}{2.637017in}}%
\pgfpathlineto{\pgfqpoint{1.648332in}{2.622780in}}%
\pgfpathlineto{\pgfqpoint{1.661873in}{2.594305in}}%
\pgfpathlineto{\pgfqpoint{1.666386in}{2.580068in}}%
\pgfpathlineto{\pgfqpoint{1.679927in}{2.551593in}}%
\pgfpathlineto{\pgfqpoint{1.684441in}{2.537356in}}%
\pgfpathlineto{\pgfqpoint{1.697982in}{2.508881in}}%
\pgfpathlineto{\pgfqpoint{1.702495in}{2.494644in}}%
\pgfpathlineto{\pgfqpoint{1.716036in}{2.466169in}}%
\pgfpathlineto{\pgfqpoint{1.720550in}{2.451932in}}%
\pgfpathlineto{\pgfqpoint{1.738604in}{2.413966in}}%
\pgfpathlineto{\pgfqpoint{1.743118in}{2.399729in}}%
\pgfpathlineto{\pgfqpoint{1.756658in}{2.371254in}}%
\pgfpathlineto{\pgfqpoint{1.761172in}{2.357017in}}%
\pgfpathlineto{\pgfqpoint{1.779226in}{2.319051in}}%
\pgfpathlineto{\pgfqpoint{1.783740in}{2.304814in}}%
\pgfpathlineto{\pgfqpoint{1.797281in}{2.276339in}}%
\pgfpathlineto{\pgfqpoint{1.801795in}{2.262102in}}%
\pgfpathlineto{\pgfqpoint{1.819849in}{2.224136in}}%
\pgfpathlineto{\pgfqpoint{1.824363in}{2.209898in}}%
\pgfpathlineto{\pgfqpoint{1.846931in}{2.162441in}}%
\pgfpathlineto{\pgfqpoint{1.851444in}{2.148203in}}%
\pgfpathlineto{\pgfqpoint{1.874012in}{2.100746in}}%
\pgfpathlineto{\pgfqpoint{1.878526in}{2.086508in}}%
\pgfpathlineto{\pgfqpoint{1.901094in}{2.039051in}}%
\pgfpathlineto{\pgfqpoint{1.905607in}{2.024814in}}%
\pgfpathlineto{\pgfqpoint{1.941716in}{1.948881in}}%
\pgfpathlineto{\pgfqpoint{1.946230in}{1.934644in}}%
\pgfpathlineto{\pgfqpoint{2.099692in}{1.611932in}}%
\pgfpathlineto{\pgfqpoint{2.104206in}{1.607186in}}%
\pgfpathlineto{\pgfqpoint{2.131288in}{1.550237in}}%
\pgfpathlineto{\pgfqpoint{2.135801in}{1.545492in}}%
\pgfpathlineto{\pgfqpoint{2.153856in}{1.507525in}}%
\pgfpathlineto{\pgfqpoint{2.158369in}{1.502780in}}%
\pgfpathlineto{\pgfqpoint{2.176424in}{1.464814in}}%
\pgfpathlineto{\pgfqpoint{2.180937in}{1.460068in}}%
\pgfpathlineto{\pgfqpoint{2.189965in}{1.441085in}}%
\pgfpathlineto{\pgfqpoint{2.194478in}{1.436339in}}%
\pgfpathlineto{\pgfqpoint{2.208019in}{1.407864in}}%
\pgfpathlineto{\pgfqpoint{2.212533in}{1.403119in}}%
\pgfpathlineto{\pgfqpoint{2.221560in}{1.384136in}}%
\pgfpathlineto{\pgfqpoint{2.226073in}{1.379390in}}%
\pgfpathlineto{\pgfqpoint{2.235101in}{1.360407in}}%
\pgfpathlineto{\pgfqpoint{2.239614in}{1.355661in}}%
\pgfpathlineto{\pgfqpoint{2.244128in}{1.346169in}}%
\pgfpathlineto{\pgfqpoint{2.248641in}{1.341424in}}%
\pgfpathlineto{\pgfqpoint{2.257669in}{1.322441in}}%
\pgfpathlineto{\pgfqpoint{2.262182in}{1.317695in}}%
\pgfpathlineto{\pgfqpoint{2.266696in}{1.308203in}}%
\pgfpathlineto{\pgfqpoint{2.271209in}{1.303458in}}%
\pgfpathlineto{\pgfqpoint{2.275723in}{1.293966in}}%
\pgfpathlineto{\pgfqpoint{2.280237in}{1.289220in}}%
\pgfpathlineto{\pgfqpoint{2.284750in}{1.279729in}}%
\pgfpathlineto{\pgfqpoint{2.289264in}{1.274983in}}%
\pgfpathlineto{\pgfqpoint{2.293777in}{1.265492in}}%
\pgfpathlineto{\pgfqpoint{2.298291in}{1.260746in}}%
\pgfpathlineto{\pgfqpoint{2.302805in}{1.251254in}}%
\pgfpathlineto{\pgfqpoint{2.307318in}{1.246508in}}%
\pgfpathlineto{\pgfqpoint{2.311832in}{1.237017in}}%
\pgfpathlineto{\pgfqpoint{2.316345in}{1.232271in}}%
\pgfpathlineto{\pgfqpoint{2.320859in}{1.222780in}}%
\pgfpathlineto{\pgfqpoint{2.325373in}{1.218034in}}%
\pgfpathlineto{\pgfqpoint{2.329886in}{1.208542in}}%
\pgfpathlineto{\pgfqpoint{2.338913in}{1.199051in}}%
\pgfpathlineto{\pgfqpoint{2.343427in}{1.189559in}}%
\pgfpathlineto{\pgfqpoint{2.352454in}{1.180068in}}%
\pgfpathlineto{\pgfqpoint{2.356968in}{1.170576in}}%
\pgfpathlineto{\pgfqpoint{2.361481in}{1.165831in}}%
\pgfpathlineto{\pgfqpoint{2.365995in}{1.156339in}}%
\pgfpathlineto{\pgfqpoint{2.379536in}{1.142102in}}%
\pgfpathlineto{\pgfqpoint{2.384050in}{1.132610in}}%
\pgfpathlineto{\pgfqpoint{2.393077in}{1.123119in}}%
\pgfpathlineto{\pgfqpoint{2.397590in}{1.113627in}}%
\pgfpathlineto{\pgfqpoint{2.415645in}{1.094644in}}%
\pgfpathlineto{\pgfqpoint{2.420158in}{1.085153in}}%
\pgfpathlineto{\pgfqpoint{2.438213in}{1.066169in}}%
\pgfpathlineto{\pgfqpoint{2.442726in}{1.056678in}}%
\pgfpathlineto{\pgfqpoint{2.487862in}{1.009220in}}%
\pgfpathlineto{\pgfqpoint{2.492376in}{0.999729in}}%
\pgfpathlineto{\pgfqpoint{2.514944in}{0.976000in}}%
\pgfpathlineto{\pgfqpoint{2.519458in}{0.976000in}}%
\pgfpathlineto{\pgfqpoint{2.564594in}{0.928542in}}%
\pgfpathlineto{\pgfqpoint{2.569107in}{0.928542in}}%
\pgfpathlineto{\pgfqpoint{2.605216in}{0.890576in}}%
\pgfpathlineto{\pgfqpoint{2.609730in}{0.890576in}}%
\pgfpathlineto{\pgfqpoint{2.632298in}{0.866847in}}%
\pgfpathlineto{\pgfqpoint{2.636811in}{0.866847in}}%
\pgfpathlineto{\pgfqpoint{2.650352in}{0.852610in}}%
\pgfpathlineto{\pgfqpoint{2.654866in}{0.852610in}}%
\pgfpathlineto{\pgfqpoint{2.668407in}{0.838373in}}%
\pgfpathlineto{\pgfqpoint{2.672920in}{0.838373in}}%
\pgfpathlineto{\pgfqpoint{2.681947in}{0.828881in}}%
\pgfpathlineto{\pgfqpoint{2.686461in}{0.828881in}}%
\pgfpathlineto{\pgfqpoint{2.695488in}{0.819390in}}%
\pgfpathlineto{\pgfqpoint{2.700002in}{0.819390in}}%
\pgfpathlineto{\pgfqpoint{2.709029in}{0.809898in}}%
\pgfpathlineto{\pgfqpoint{2.713543in}{0.809898in}}%
\pgfpathlineto{\pgfqpoint{2.718056in}{0.805153in}}%
\pgfpathlineto{\pgfqpoint{2.722570in}{0.805153in}}%
\pgfpathlineto{\pgfqpoint{2.731597in}{0.795661in}}%
\pgfpathlineto{\pgfqpoint{2.736111in}{0.795661in}}%
\pgfpathlineto{\pgfqpoint{2.740624in}{0.790915in}}%
\pgfpathlineto{\pgfqpoint{2.745138in}{0.790915in}}%
\pgfpathlineto{\pgfqpoint{2.749651in}{0.786169in}}%
\pgfpathlineto{\pgfqpoint{2.754165in}{0.786169in}}%
\pgfpathlineto{\pgfqpoint{2.763192in}{0.776678in}}%
\pgfpathlineto{\pgfqpoint{2.767706in}{0.776678in}}%
\pgfpathlineto{\pgfqpoint{2.772219in}{0.771932in}}%
\pgfpathlineto{\pgfqpoint{2.776733in}{0.771932in}}%
\pgfpathlineto{\pgfqpoint{2.781247in}{0.767186in}}%
\pgfpathlineto{\pgfqpoint{2.790274in}{0.767186in}}%
\pgfpathlineto{\pgfqpoint{2.794788in}{0.762441in}}%
\pgfpathlineto{\pgfqpoint{2.799301in}{0.762441in}}%
\pgfpathlineto{\pgfqpoint{2.803815in}{0.757695in}}%
\pgfpathlineto{\pgfqpoint{2.808328in}{0.757695in}}%
\pgfpathlineto{\pgfqpoint{2.812842in}{0.752949in}}%
\pgfpathlineto{\pgfqpoint{2.817356in}{0.752949in}}%
\pgfpathlineto{\pgfqpoint{2.821869in}{0.748203in}}%
\pgfpathlineto{\pgfqpoint{2.830896in}{0.748203in}}%
\pgfpathlineto{\pgfqpoint{2.835410in}{0.743458in}}%
\pgfpathlineto{\pgfqpoint{2.844437in}{0.743458in}}%
\pgfpathlineto{\pgfqpoint{2.848951in}{0.738712in}}%
\pgfpathlineto{\pgfqpoint{2.853464in}{0.738712in}}%
\pgfpathlineto{\pgfqpoint{2.857978in}{0.733966in}}%
\pgfpathlineto{\pgfqpoint{2.867005in}{0.733966in}}%
\pgfpathlineto{\pgfqpoint{2.871519in}{0.729220in}}%
\pgfpathlineto{\pgfqpoint{2.885060in}{0.729220in}}%
\pgfpathlineto{\pgfqpoint{2.889573in}{0.724475in}}%
\pgfpathlineto{\pgfqpoint{2.898600in}{0.724475in}}%
\pgfpathlineto{\pgfqpoint{2.903114in}{0.719729in}}%
\pgfpathlineto{\pgfqpoint{2.916655in}{0.719729in}}%
\pgfpathlineto{\pgfqpoint{2.921168in}{0.714983in}}%
\pgfpathlineto{\pgfqpoint{2.939223in}{0.714983in}}%
\pgfpathlineto{\pgfqpoint{2.943736in}{0.710237in}}%
\pgfpathlineto{\pgfqpoint{2.961791in}{0.710237in}}%
\pgfpathlineto{\pgfqpoint{2.966304in}{0.705492in}}%
\pgfpathlineto{\pgfqpoint{2.993386in}{0.705492in}}%
\pgfpathlineto{\pgfqpoint{2.997900in}{0.700746in}}%
\pgfpathlineto{\pgfqpoint{3.052063in}{0.700746in}}%
\pgfpathlineto{\pgfqpoint{3.056577in}{0.696000in}}%
\pgfpathlineto{\pgfqpoint{3.110740in}{0.696000in}}%
\pgfpathlineto{\pgfqpoint{3.115253in}{0.700746in}}%
\pgfpathlineto{\pgfqpoint{3.173930in}{0.700746in}}%
\pgfpathlineto{\pgfqpoint{3.178444in}{0.705492in}}%
\pgfpathlineto{\pgfqpoint{3.210039in}{0.705492in}}%
\pgfpathlineto{\pgfqpoint{3.214553in}{0.710237in}}%
\pgfpathlineto{\pgfqpoint{3.237121in}{0.710237in}}%
\pgfpathlineto{\pgfqpoint{3.241634in}{0.714983in}}%
\pgfpathlineto{\pgfqpoint{3.259689in}{0.714983in}}%
\pgfpathlineto{\pgfqpoint{3.264202in}{0.719729in}}%
\pgfpathlineto{\pgfqpoint{3.282257in}{0.719729in}}%
\pgfpathlineto{\pgfqpoint{3.286770in}{0.724475in}}%
\pgfpathlineto{\pgfqpoint{3.300311in}{0.724475in}}%
\pgfpathlineto{\pgfqpoint{3.304825in}{0.729220in}}%
\pgfpathlineto{\pgfqpoint{3.318366in}{0.729220in}}%
\pgfpathlineto{\pgfqpoint{3.322879in}{0.733966in}}%
\pgfpathlineto{\pgfqpoint{3.336420in}{0.733966in}}%
\pgfpathlineto{\pgfqpoint{3.340934in}{0.738712in}}%
\pgfpathlineto{\pgfqpoint{3.354474in}{0.738712in}}%
\pgfpathlineto{\pgfqpoint{3.358988in}{0.743458in}}%
\pgfpathlineto{\pgfqpoint{3.368015in}{0.743458in}}%
\pgfpathlineto{\pgfqpoint{3.372529in}{0.748203in}}%
\pgfpathlineto{\pgfqpoint{3.386070in}{0.748203in}}%
\pgfpathlineto{\pgfqpoint{3.390583in}{0.752949in}}%
\pgfpathlineto{\pgfqpoint{3.399611in}{0.752949in}}%
\pgfpathlineto{\pgfqpoint{3.404124in}{0.757695in}}%
\pgfpathlineto{\pgfqpoint{3.413151in}{0.757695in}}%
\pgfpathlineto{\pgfqpoint{3.417665in}{0.762441in}}%
\pgfpathlineto{\pgfqpoint{3.426692in}{0.762441in}}%
\pgfpathlineto{\pgfqpoint{3.431206in}{0.767186in}}%
\pgfpathlineto{\pgfqpoint{3.440233in}{0.767186in}}%
\pgfpathlineto{\pgfqpoint{3.444747in}{0.771932in}}%
\pgfpathlineto{\pgfqpoint{3.453774in}{0.771932in}}%
\pgfpathlineto{\pgfqpoint{3.458287in}{0.776678in}}%
\pgfpathlineto{\pgfqpoint{3.462801in}{0.776678in}}%
\pgfpathlineto{\pgfqpoint{3.467315in}{0.781424in}}%
\pgfpathlineto{\pgfqpoint{3.476342in}{0.781424in}}%
\pgfpathlineto{\pgfqpoint{3.480855in}{0.786169in}}%
\pgfpathlineto{\pgfqpoint{3.489883in}{0.786169in}}%
\pgfpathlineto{\pgfqpoint{3.494396in}{0.790915in}}%
\pgfpathlineto{\pgfqpoint{3.498910in}{0.790915in}}%
\pgfpathlineto{\pgfqpoint{3.503423in}{0.795661in}}%
\pgfpathlineto{\pgfqpoint{3.512451in}{0.795661in}}%
\pgfpathlineto{\pgfqpoint{3.516964in}{0.800407in}}%
\pgfpathlineto{\pgfqpoint{3.521478in}{0.800407in}}%
\pgfpathlineto{\pgfqpoint{3.525991in}{0.805153in}}%
\pgfpathlineto{\pgfqpoint{3.535019in}{0.805153in}}%
\pgfpathlineto{\pgfqpoint{3.539532in}{0.809898in}}%
\pgfpathlineto{\pgfqpoint{3.544046in}{0.809898in}}%
\pgfpathlineto{\pgfqpoint{3.548559in}{0.814644in}}%
\pgfpathlineto{\pgfqpoint{3.557587in}{0.814644in}}%
\pgfpathlineto{\pgfqpoint{3.562100in}{0.819390in}}%
\pgfpathlineto{\pgfqpoint{3.566614in}{0.819390in}}%
\pgfpathlineto{\pgfqpoint{3.571127in}{0.824136in}}%
\pgfpathlineto{\pgfqpoint{3.575641in}{0.824136in}}%
\pgfpathlineto{\pgfqpoint{3.580155in}{0.828881in}}%
\pgfpathlineto{\pgfqpoint{3.589182in}{0.828881in}}%
\pgfpathlineto{\pgfqpoint{3.593696in}{0.833627in}}%
\pgfpathlineto{\pgfqpoint{3.598209in}{0.833627in}}%
\pgfpathlineto{\pgfqpoint{3.602723in}{0.838373in}}%
\pgfpathlineto{\pgfqpoint{3.607236in}{0.838373in}}%
\pgfpathlineto{\pgfqpoint{3.611750in}{0.843119in}}%
\pgfpathlineto{\pgfqpoint{3.620777in}{0.843119in}}%
\pgfpathlineto{\pgfqpoint{3.625291in}{0.847864in}}%
\pgfpathlineto{\pgfqpoint{3.629804in}{0.847864in}}%
\pgfpathlineto{\pgfqpoint{3.634318in}{0.852610in}}%
\pgfpathlineto{\pgfqpoint{3.643345in}{0.852610in}}%
\pgfpathlineto{\pgfqpoint{3.647859in}{0.857356in}}%
\pgfpathlineto{\pgfqpoint{3.652372in}{0.857356in}}%
\pgfpathlineto{\pgfqpoint{3.656886in}{0.862102in}}%
\pgfpathlineto{\pgfqpoint{3.661400in}{0.862102in}}%
\pgfpathlineto{\pgfqpoint{3.665913in}{0.866847in}}%
\pgfpathlineto{\pgfqpoint{3.674940in}{0.866847in}}%
\pgfpathlineto{\pgfqpoint{3.679454in}{0.871593in}}%
\pgfpathlineto{\pgfqpoint{3.683968in}{0.871593in}}%
\pgfpathlineto{\pgfqpoint{3.688481in}{0.876339in}}%
\pgfpathlineto{\pgfqpoint{3.692995in}{0.876339in}}%
\pgfpathlineto{\pgfqpoint{3.697508in}{0.881085in}}%
\pgfpathlineto{\pgfqpoint{3.702022in}{0.881085in}}%
\pgfpathlineto{\pgfqpoint{3.706536in}{0.885831in}}%
\pgfpathlineto{\pgfqpoint{3.715563in}{0.885831in}}%
\pgfpathlineto{\pgfqpoint{3.720076in}{0.890576in}}%
\pgfpathlineto{\pgfqpoint{3.724590in}{0.890576in}}%
\pgfpathlineto{\pgfqpoint{3.729104in}{0.895322in}}%
\pgfpathlineto{\pgfqpoint{3.733617in}{0.895322in}}%
\pgfpathlineto{\pgfqpoint{3.738131in}{0.900068in}}%
\pgfpathlineto{\pgfqpoint{3.742644in}{0.900068in}}%
\pgfpathlineto{\pgfqpoint{3.747158in}{0.904814in}}%
\pgfpathlineto{\pgfqpoint{3.751672in}{0.904814in}}%
\pgfpathlineto{\pgfqpoint{3.756185in}{0.909559in}}%
\pgfpathlineto{\pgfqpoint{3.760699in}{0.909559in}}%
\pgfpathlineto{\pgfqpoint{3.765212in}{0.914305in}}%
\pgfpathlineto{\pgfqpoint{3.769726in}{0.914305in}}%
\pgfpathlineto{\pgfqpoint{3.774240in}{0.919051in}}%
\pgfpathlineto{\pgfqpoint{3.778753in}{0.919051in}}%
\pgfpathlineto{\pgfqpoint{3.783267in}{0.923797in}}%
\pgfpathlineto{\pgfqpoint{3.787781in}{0.923797in}}%
\pgfpathlineto{\pgfqpoint{3.792294in}{0.928542in}}%
\pgfpathlineto{\pgfqpoint{3.796808in}{0.928542in}}%
\pgfpathlineto{\pgfqpoint{3.801321in}{0.933288in}}%
\pgfpathlineto{\pgfqpoint{3.805835in}{0.933288in}}%
\pgfpathlineto{\pgfqpoint{3.810349in}{0.938034in}}%
\pgfpathlineto{\pgfqpoint{3.814862in}{0.938034in}}%
\pgfpathlineto{\pgfqpoint{3.819376in}{0.942780in}}%
\pgfpathlineto{\pgfqpoint{3.823889in}{0.942780in}}%
\pgfpathlineto{\pgfqpoint{3.828403in}{0.947525in}}%
\pgfpathlineto{\pgfqpoint{3.832917in}{0.947525in}}%
\pgfpathlineto{\pgfqpoint{3.837430in}{0.952271in}}%
\pgfpathlineto{\pgfqpoint{3.841944in}{0.952271in}}%
\pgfpathlineto{\pgfqpoint{3.846457in}{0.957017in}}%
\pgfpathlineto{\pgfqpoint{3.850971in}{0.957017in}}%
\pgfpathlineto{\pgfqpoint{3.855485in}{0.961763in}}%
\pgfpathlineto{\pgfqpoint{3.859998in}{0.961763in}}%
\pgfpathlineto{\pgfqpoint{3.864512in}{0.966508in}}%
\pgfpathlineto{\pgfqpoint{3.869025in}{0.966508in}}%
\pgfpathlineto{\pgfqpoint{3.873539in}{0.971254in}}%
\pgfpathlineto{\pgfqpoint{3.878053in}{0.971254in}}%
\pgfpathlineto{\pgfqpoint{3.882566in}{0.976000in}}%
\pgfpathlineto{\pgfqpoint{3.887080in}{0.976000in}}%
\pgfpathlineto{\pgfqpoint{3.891593in}{0.980746in}}%
\pgfpathlineto{\pgfqpoint{3.896107in}{0.980746in}}%
\pgfpathlineto{\pgfqpoint{3.900621in}{0.985492in}}%
\pgfpathlineto{\pgfqpoint{3.905134in}{0.985492in}}%
\pgfpathlineto{\pgfqpoint{3.914161in}{0.994983in}}%
\pgfpathlineto{\pgfqpoint{3.918675in}{0.994983in}}%
\pgfpathlineto{\pgfqpoint{3.923189in}{0.999729in}}%
\pgfpathlineto{\pgfqpoint{3.927702in}{0.999729in}}%
\pgfpathlineto{\pgfqpoint{3.932216in}{1.004475in}}%
\pgfpathlineto{\pgfqpoint{3.936729in}{1.004475in}}%
\pgfpathlineto{\pgfqpoint{3.941243in}{1.009220in}}%
\pgfpathlineto{\pgfqpoint{3.945757in}{1.009220in}}%
\pgfpathlineto{\pgfqpoint{3.950270in}{1.013966in}}%
\pgfpathlineto{\pgfqpoint{3.954784in}{1.013966in}}%
\pgfpathlineto{\pgfqpoint{3.963811in}{1.023458in}}%
\pgfpathlineto{\pgfqpoint{3.968325in}{1.023458in}}%
\pgfpathlineto{\pgfqpoint{3.972838in}{1.028203in}}%
\pgfpathlineto{\pgfqpoint{3.977352in}{1.028203in}}%
\pgfpathlineto{\pgfqpoint{3.981866in}{1.032949in}}%
\pgfpathlineto{\pgfqpoint{3.986379in}{1.032949in}}%
\pgfpathlineto{\pgfqpoint{3.995406in}{1.042441in}}%
\pgfpathlineto{\pgfqpoint{3.999920in}{1.042441in}}%
\pgfpathlineto{\pgfqpoint{4.004434in}{1.047186in}}%
\pgfpathlineto{\pgfqpoint{4.008947in}{1.047186in}}%
\pgfpathlineto{\pgfqpoint{4.013461in}{1.051932in}}%
\pgfpathlineto{\pgfqpoint{4.017974in}{1.051932in}}%
\pgfpathlineto{\pgfqpoint{4.027002in}{1.061424in}}%
\pgfpathlineto{\pgfqpoint{4.031515in}{1.061424in}}%
\pgfpathlineto{\pgfqpoint{4.036029in}{1.066169in}}%
\pgfpathlineto{\pgfqpoint{4.040542in}{1.066169in}}%
\pgfpathlineto{\pgfqpoint{4.045056in}{1.070915in}}%
\pgfpathlineto{\pgfqpoint{4.049570in}{1.070915in}}%
\pgfpathlineto{\pgfqpoint{4.058597in}{1.080407in}}%
\pgfpathlineto{\pgfqpoint{4.063110in}{1.080407in}}%
\pgfpathlineto{\pgfqpoint{4.067624in}{1.085153in}}%
\pgfpathlineto{\pgfqpoint{4.072138in}{1.085153in}}%
\pgfpathlineto{\pgfqpoint{4.081165in}{1.094644in}}%
\pgfpathlineto{\pgfqpoint{4.085678in}{1.094644in}}%
\pgfpathlineto{\pgfqpoint{4.090192in}{1.099390in}}%
\pgfpathlineto{\pgfqpoint{4.094706in}{1.099390in}}%
\pgfpathlineto{\pgfqpoint{4.103733in}{1.108881in}}%
\pgfpathlineto{\pgfqpoint{4.108246in}{1.108881in}}%
\pgfpathlineto{\pgfqpoint{4.112760in}{1.113627in}}%
\pgfpathlineto{\pgfqpoint{4.117274in}{1.113627in}}%
\pgfpathlineto{\pgfqpoint{4.121787in}{1.118373in}}%
\pgfpathlineto{\pgfqpoint{4.126301in}{1.118373in}}%
\pgfpathlineto{\pgfqpoint{4.135328in}{1.127864in}}%
\pgfpathlineto{\pgfqpoint{4.139842in}{1.127864in}}%
\pgfpathlineto{\pgfqpoint{4.148869in}{1.137356in}}%
\pgfpathlineto{\pgfqpoint{4.153382in}{1.137356in}}%
\pgfpathlineto{\pgfqpoint{4.157896in}{1.142102in}}%
\pgfpathlineto{\pgfqpoint{4.162410in}{1.142102in}}%
\pgfpathlineto{\pgfqpoint{4.171437in}{1.151593in}}%
\pgfpathlineto{\pgfqpoint{4.175950in}{1.151593in}}%
\pgfpathlineto{\pgfqpoint{4.180464in}{1.156339in}}%
\pgfpathlineto{\pgfqpoint{4.184978in}{1.156339in}}%
\pgfpathlineto{\pgfqpoint{4.194005in}{1.165831in}}%
\pgfpathlineto{\pgfqpoint{4.198519in}{1.165831in}}%
\pgfpathlineto{\pgfqpoint{4.207546in}{1.175322in}}%
\pgfpathlineto{\pgfqpoint{4.212059in}{1.175322in}}%
\pgfpathlineto{\pgfqpoint{4.216573in}{1.180068in}}%
\pgfpathlineto{\pgfqpoint{4.221087in}{1.180068in}}%
\pgfpathlineto{\pgfqpoint{4.230114in}{1.189559in}}%
\pgfpathlineto{\pgfqpoint{4.234627in}{1.189559in}}%
\pgfpathlineto{\pgfqpoint{4.239141in}{1.194305in}}%
\pgfpathlineto{\pgfqpoint{4.243655in}{1.194305in}}%
\pgfpathlineto{\pgfqpoint{4.252682in}{1.203797in}}%
\pgfpathlineto{\pgfqpoint{4.257195in}{1.203797in}}%
\pgfpathlineto{\pgfqpoint{4.266223in}{1.213288in}}%
\pgfpathlineto{\pgfqpoint{4.270736in}{1.213288in}}%
\pgfpathlineto{\pgfqpoint{4.279763in}{1.222780in}}%
\pgfpathlineto{\pgfqpoint{4.284277in}{1.222780in}}%
\pgfpathlineto{\pgfqpoint{4.288791in}{1.227525in}}%
\pgfpathlineto{\pgfqpoint{4.293304in}{1.227525in}}%
\pgfpathlineto{\pgfqpoint{4.302331in}{1.237017in}}%
\pgfpathlineto{\pgfqpoint{4.306845in}{1.237017in}}%
\pgfpathlineto{\pgfqpoint{4.315872in}{1.246508in}}%
\pgfpathlineto{\pgfqpoint{4.320386in}{1.246508in}}%
\pgfpathlineto{\pgfqpoint{4.329413in}{1.256000in}}%
\pgfpathlineto{\pgfqpoint{4.333927in}{1.256000in}}%
\pgfpathlineto{\pgfqpoint{4.338440in}{1.260746in}}%
\pgfpathlineto{\pgfqpoint{4.342954in}{1.260746in}}%
\pgfpathlineto{\pgfqpoint{4.351981in}{1.270237in}}%
\pgfpathlineto{\pgfqpoint{4.356495in}{1.270237in}}%
\pgfpathlineto{\pgfqpoint{4.365522in}{1.279729in}}%
\pgfpathlineto{\pgfqpoint{4.370035in}{1.279729in}}%
\pgfpathlineto{\pgfqpoint{4.379063in}{1.289220in}}%
\pgfpathlineto{\pgfqpoint{4.383576in}{1.289220in}}%
\pgfpathlineto{\pgfqpoint{4.392604in}{1.298712in}}%
\pgfpathlineto{\pgfqpoint{4.397117in}{1.298712in}}%
\pgfpathlineto{\pgfqpoint{4.406144in}{1.308203in}}%
\pgfpathlineto{\pgfqpoint{4.410658in}{1.308203in}}%
\pgfpathlineto{\pgfqpoint{4.419685in}{1.317695in}}%
\pgfpathlineto{\pgfqpoint{4.424199in}{1.317695in}}%
\pgfpathlineto{\pgfqpoint{4.433226in}{1.327186in}}%
\pgfpathlineto{\pgfqpoint{4.437740in}{1.327186in}}%
\pgfpathlineto{\pgfqpoint{4.442253in}{1.331932in}}%
\pgfpathlineto{\pgfqpoint{4.446767in}{1.331932in}}%
\pgfpathlineto{\pgfqpoint{4.455794in}{1.341424in}}%
\pgfpathlineto{\pgfqpoint{4.460308in}{1.341424in}}%
\pgfpathlineto{\pgfqpoint{4.469335in}{1.350915in}}%
\pgfpathlineto{\pgfqpoint{4.473848in}{1.350915in}}%
\pgfpathlineto{\pgfqpoint{4.482876in}{1.360407in}}%
\pgfpathlineto{\pgfqpoint{4.487389in}{1.360407in}}%
\pgfpathlineto{\pgfqpoint{4.496416in}{1.369898in}}%
\pgfpathlineto{\pgfqpoint{4.500930in}{1.369898in}}%
\pgfpathlineto{\pgfqpoint{4.509957in}{1.379390in}}%
\pgfpathlineto{\pgfqpoint{4.514471in}{1.379390in}}%
\pgfpathlineto{\pgfqpoint{4.523498in}{1.388881in}}%
\pgfpathlineto{\pgfqpoint{4.528012in}{1.388881in}}%
\pgfpathlineto{\pgfqpoint{4.537039in}{1.398373in}}%
\pgfpathlineto{\pgfqpoint{4.541552in}{1.398373in}}%
\pgfpathlineto{\pgfqpoint{4.550580in}{1.407864in}}%
\pgfpathlineto{\pgfqpoint{4.555093in}{1.407864in}}%
\pgfpathlineto{\pgfqpoint{4.568634in}{1.422102in}}%
\pgfpathlineto{\pgfqpoint{4.573148in}{1.422102in}}%
\pgfpathlineto{\pgfqpoint{4.582175in}{1.431593in}}%
\pgfpathlineto{\pgfqpoint{4.586689in}{1.431593in}}%
\pgfpathlineto{\pgfqpoint{4.595716in}{1.441085in}}%
\pgfpathlineto{\pgfqpoint{4.600229in}{1.441085in}}%
\pgfpathlineto{\pgfqpoint{4.609257in}{1.450576in}}%
\pgfpathlineto{\pgfqpoint{4.613770in}{1.450576in}}%
\pgfpathlineto{\pgfqpoint{4.622797in}{1.460068in}}%
\pgfpathlineto{\pgfqpoint{4.627311in}{1.460068in}}%
\pgfpathlineto{\pgfqpoint{4.636338in}{1.469559in}}%
\pgfpathlineto{\pgfqpoint{4.640852in}{1.469559in}}%
\pgfpathlineto{\pgfqpoint{4.649879in}{1.479051in}}%
\pgfpathlineto{\pgfqpoint{4.654393in}{1.479051in}}%
\pgfpathlineto{\pgfqpoint{4.663420in}{1.488542in}}%
\pgfpathlineto{\pgfqpoint{4.667933in}{1.488542in}}%
\pgfpathlineto{\pgfqpoint{4.681474in}{1.502780in}}%
\pgfpathlineto{\pgfqpoint{4.685988in}{1.502780in}}%
\pgfpathlineto{\pgfqpoint{4.695015in}{1.512271in}}%
\pgfpathlineto{\pgfqpoint{4.699529in}{1.512271in}}%
\pgfpathlineto{\pgfqpoint{4.708556in}{1.521763in}}%
\pgfpathlineto{\pgfqpoint{4.713069in}{1.521763in}}%
\pgfpathlineto{\pgfqpoint{4.722097in}{1.531254in}}%
\pgfpathlineto{\pgfqpoint{4.726610in}{1.531254in}}%
\pgfpathlineto{\pgfqpoint{4.735637in}{1.540746in}}%
\pgfpathlineto{\pgfqpoint{4.740151in}{1.540746in}}%
\pgfpathlineto{\pgfqpoint{4.753692in}{1.554983in}}%
\pgfpathlineto{\pgfqpoint{4.758205in}{1.554983in}}%
\pgfpathlineto{\pgfqpoint{4.767233in}{1.564475in}}%
\pgfpathlineto{\pgfqpoint{4.771746in}{1.564475in}}%
\pgfpathlineto{\pgfqpoint{4.780774in}{1.573966in}}%
\pgfpathlineto{\pgfqpoint{4.785287in}{1.573966in}}%
\pgfpathlineto{\pgfqpoint{4.794314in}{1.583458in}}%
\pgfpathlineto{\pgfqpoint{4.798828in}{1.583458in}}%
\pgfpathlineto{\pgfqpoint{4.812369in}{1.597695in}}%
\pgfpathlineto{\pgfqpoint{4.816882in}{1.597695in}}%
\pgfpathlineto{\pgfqpoint{4.825910in}{1.607186in}}%
\pgfpathlineto{\pgfqpoint{4.830423in}{1.607186in}}%
\pgfpathlineto{\pgfqpoint{4.839450in}{1.616678in}}%
\pgfpathlineto{\pgfqpoint{4.843964in}{1.616678in}}%
\pgfpathlineto{\pgfqpoint{4.852991in}{1.626169in}}%
\pgfpathlineto{\pgfqpoint{4.857505in}{1.626169in}}%
\pgfpathlineto{\pgfqpoint{4.871046in}{1.640407in}}%
\pgfpathlineto{\pgfqpoint{4.875559in}{1.640407in}}%
\pgfpathlineto{\pgfqpoint{4.884586in}{1.649898in}}%
\pgfpathlineto{\pgfqpoint{4.889100in}{1.649898in}}%
\pgfpathlineto{\pgfqpoint{4.898127in}{1.659390in}}%
\pgfpathlineto{\pgfqpoint{4.902641in}{1.659390in}}%
\pgfpathlineto{\pgfqpoint{4.916182in}{1.673627in}}%
\pgfpathlineto{\pgfqpoint{4.920695in}{1.673627in}}%
\pgfpathlineto{\pgfqpoint{4.929722in}{1.683119in}}%
\pgfpathlineto{\pgfqpoint{4.934236in}{1.683119in}}%
\pgfpathlineto{\pgfqpoint{4.943263in}{1.692610in}}%
\pgfpathlineto{\pgfqpoint{4.947777in}{1.692610in}}%
\pgfpathlineto{\pgfqpoint{4.961318in}{1.706847in}}%
\pgfpathlineto{\pgfqpoint{4.965831in}{1.706847in}}%
\pgfpathlineto{\pgfqpoint{4.974858in}{1.716339in}}%
\pgfpathlineto{\pgfqpoint{4.979372in}{1.716339in}}%
\pgfpathlineto{\pgfqpoint{4.988399in}{1.725831in}}%
\pgfpathlineto{\pgfqpoint{4.992913in}{1.725831in}}%
\pgfpathlineto{\pgfqpoint{5.006454in}{1.740068in}}%
\pgfpathlineto{\pgfqpoint{5.010967in}{1.740068in}}%
\pgfpathlineto{\pgfqpoint{5.019995in}{1.749559in}}%
\pgfpathlineto{\pgfqpoint{5.024508in}{1.749559in}}%
\pgfpathlineto{\pgfqpoint{5.038049in}{1.763797in}}%
\pgfpathlineto{\pgfqpoint{5.042563in}{1.763797in}}%
\pgfpathlineto{\pgfqpoint{5.051590in}{1.773288in}}%
\pgfpathlineto{\pgfqpoint{5.056103in}{1.773288in}}%
\pgfpathlineto{\pgfqpoint{5.065131in}{1.782780in}}%
\pgfpathlineto{\pgfqpoint{5.069644in}{1.782780in}}%
\pgfpathlineto{\pgfqpoint{5.083185in}{1.797017in}}%
\pgfpathlineto{\pgfqpoint{5.087699in}{1.797017in}}%
\pgfpathlineto{\pgfqpoint{5.096726in}{1.806508in}}%
\pgfpathlineto{\pgfqpoint{5.101239in}{1.806508in}}%
\pgfpathlineto{\pgfqpoint{5.114780in}{1.820746in}}%
\pgfpathlineto{\pgfqpoint{5.119294in}{1.820746in}}%
\pgfpathlineto{\pgfqpoint{5.128321in}{1.830237in}}%
\pgfpathlineto{\pgfqpoint{5.132835in}{1.830237in}}%
\pgfpathlineto{\pgfqpoint{5.146375in}{1.844475in}}%
\pgfpathlineto{\pgfqpoint{5.150889in}{1.844475in}}%
\pgfpathlineto{\pgfqpoint{5.159916in}{1.853966in}}%
\pgfpathlineto{\pgfqpoint{5.164430in}{1.853966in}}%
\pgfpathlineto{\pgfqpoint{5.177971in}{1.868203in}}%
\pgfpathlineto{\pgfqpoint{5.182484in}{1.868203in}}%
\pgfpathlineto{\pgfqpoint{5.191512in}{1.877695in}}%
\pgfpathlineto{\pgfqpoint{5.196025in}{1.877695in}}%
\pgfpathlineto{\pgfqpoint{5.205052in}{1.887186in}}%
\pgfpathlineto{\pgfqpoint{5.209566in}{1.887186in}}%
\pgfpathlineto{\pgfqpoint{5.223107in}{1.901424in}}%
\pgfpathlineto{\pgfqpoint{5.227620in}{1.901424in}}%
\pgfpathlineto{\pgfqpoint{5.236648in}{1.910915in}}%
\pgfpathlineto{\pgfqpoint{5.241161in}{1.910915in}}%
\pgfpathlineto{\pgfqpoint{5.254702in}{1.925153in}}%
\pgfpathlineto{\pgfqpoint{5.259216in}{1.925153in}}%
\pgfpathlineto{\pgfqpoint{5.268243in}{1.934644in}}%
\pgfpathlineto{\pgfqpoint{5.272756in}{1.934644in}}%
\pgfpathlineto{\pgfqpoint{5.286297in}{1.948881in}}%
\pgfpathlineto{\pgfqpoint{5.290811in}{1.948881in}}%
\pgfpathlineto{\pgfqpoint{5.299838in}{1.958373in}}%
\pgfpathlineto{\pgfqpoint{5.304352in}{1.958373in}}%
\pgfpathlineto{\pgfqpoint{5.317892in}{1.972610in}}%
\pgfpathlineto{\pgfqpoint{5.322406in}{1.972610in}}%
\pgfpathlineto{\pgfqpoint{5.331433in}{1.982102in}}%
\pgfpathlineto{\pgfqpoint{5.335947in}{1.982102in}}%
\pgfpathlineto{\pgfqpoint{5.349488in}{1.996339in}}%
\pgfpathlineto{\pgfqpoint{5.354001in}{1.996339in}}%
\pgfpathlineto{\pgfqpoint{5.363028in}{2.005831in}}%
\pgfpathlineto{\pgfqpoint{5.367542in}{2.005831in}}%
\pgfpathlineto{\pgfqpoint{5.381083in}{2.020068in}}%
\pgfpathlineto{\pgfqpoint{5.385597in}{2.020068in}}%
\pgfpathlineto{\pgfqpoint{5.394624in}{2.029559in}}%
\pgfpathlineto{\pgfqpoint{5.399137in}{2.029559in}}%
\pgfpathlineto{\pgfqpoint{5.412678in}{2.043797in}}%
\pgfpathlineto{\pgfqpoint{5.417192in}{2.043797in}}%
\pgfpathlineto{\pgfqpoint{5.426219in}{2.053288in}}%
\pgfpathlineto{\pgfqpoint{5.430733in}{2.053288in}}%
\pgfpathlineto{\pgfqpoint{5.444273in}{2.067525in}}%
\pgfpathlineto{\pgfqpoint{5.448787in}{2.067525in}}%
\pgfpathlineto{\pgfqpoint{5.457814in}{2.077017in}}%
\pgfpathlineto{\pgfqpoint{5.462328in}{2.077017in}}%
\pgfpathlineto{\pgfqpoint{5.475869in}{2.091254in}}%
\pgfpathlineto{\pgfqpoint{5.480382in}{2.091254in}}%
\pgfpathlineto{\pgfqpoint{5.493923in}{2.105492in}}%
\pgfpathlineto{\pgfqpoint{5.498437in}{2.105492in}}%
\pgfpathlineto{\pgfqpoint{5.507464in}{2.114983in}}%
\pgfpathlineto{\pgfqpoint{5.511977in}{2.114983in}}%
\pgfpathlineto{\pgfqpoint{5.525518in}{2.129220in}}%
\pgfpathlineto{\pgfqpoint{5.530032in}{2.129220in}}%
\pgfpathlineto{\pgfqpoint{5.534545in}{2.133966in}}%
\pgfpathlineto{\pgfqpoint{5.534545in}{2.133966in}}%
\pgfusepath{stroke}%
\end{pgfscope}%
\begin{pgfscope}%
\pgfsetrectcap%
\pgfsetmiterjoin%
\pgfsetlinewidth{0.803000pt}%
\definecolor{currentstroke}{rgb}{0.000000,0.000000,0.000000}%
\pgfsetstrokecolor{currentstroke}%
\pgfsetdash{}{0pt}%
\pgfpathmoveto{\pgfqpoint{0.800000in}{0.528000in}}%
\pgfpathlineto{\pgfqpoint{0.800000in}{4.224000in}}%
\pgfusepath{stroke}%
\end{pgfscope}%
\begin{pgfscope}%
\pgfsetrectcap%
\pgfsetmiterjoin%
\pgfsetlinewidth{0.803000pt}%
\definecolor{currentstroke}{rgb}{0.000000,0.000000,0.000000}%
\pgfsetstrokecolor{currentstroke}%
\pgfsetdash{}{0pt}%
\pgfpathmoveto{\pgfqpoint{5.760000in}{0.528000in}}%
\pgfpathlineto{\pgfqpoint{5.760000in}{4.224000in}}%
\pgfusepath{stroke}%
\end{pgfscope}%
\begin{pgfscope}%
\pgfsetrectcap%
\pgfsetmiterjoin%
\pgfsetlinewidth{0.803000pt}%
\definecolor{currentstroke}{rgb}{0.000000,0.000000,0.000000}%
\pgfsetstrokecolor{currentstroke}%
\pgfsetdash{}{0pt}%
\pgfpathmoveto{\pgfqpoint{0.800000in}{0.528000in}}%
\pgfpathlineto{\pgfqpoint{5.760000in}{0.528000in}}%
\pgfusepath{stroke}%
\end{pgfscope}%
\begin{pgfscope}%
\pgfsetrectcap%
\pgfsetmiterjoin%
\pgfsetlinewidth{0.803000pt}%
\definecolor{currentstroke}{rgb}{0.000000,0.000000,0.000000}%
\pgfsetstrokecolor{currentstroke}%
\pgfsetdash{}{0pt}%
\pgfpathmoveto{\pgfqpoint{0.800000in}{4.224000in}}%
\pgfpathlineto{\pgfqpoint{5.760000in}{4.224000in}}%
\pgfusepath{stroke}%
\end{pgfscope}%
\begin{pgfscope}%
\pgfsetbuttcap%
\pgfsetmiterjoin%
\definecolor{currentfill}{rgb}{1.000000,1.000000,1.000000}%
\pgfsetfillcolor{currentfill}%
\pgfsetfillopacity{0.800000}%
\pgfsetlinewidth{1.003750pt}%
\definecolor{currentstroke}{rgb}{0.800000,0.800000,0.800000}%
\pgfsetstrokecolor{currentstroke}%
\pgfsetstrokeopacity{0.800000}%
\pgfsetdash{}{0pt}%
\pgfpathmoveto{\pgfqpoint{4.048963in}{3.919216in}}%
\pgfpathlineto{\pgfqpoint{5.662778in}{3.919216in}}%
\pgfpathquadraticcurveto{\pgfqpoint{5.690556in}{3.919216in}}{\pgfqpoint{5.690556in}{3.946994in}}%
\pgfpathlineto{\pgfqpoint{5.690556in}{4.126778in}}%
\pgfpathquadraticcurveto{\pgfqpoint{5.690556in}{4.154556in}}{\pgfqpoint{5.662778in}{4.154556in}}%
\pgfpathlineto{\pgfqpoint{4.048963in}{4.154556in}}%
\pgfpathquadraticcurveto{\pgfqpoint{4.021186in}{4.154556in}}{\pgfqpoint{4.021186in}{4.126778in}}%
\pgfpathlineto{\pgfqpoint{4.021186in}{3.946994in}}%
\pgfpathquadraticcurveto{\pgfqpoint{4.021186in}{3.919216in}}{\pgfqpoint{4.048963in}{3.919216in}}%
\pgfpathclose%
\pgfusepath{stroke,fill}%
\end{pgfscope}%
\begin{pgfscope}%
\pgfsetrectcap%
\pgfsetroundjoin%
\pgfsetlinewidth{1.505625pt}%
\definecolor{currentstroke}{rgb}{0.000000,0.000000,1.000000}%
\pgfsetstrokecolor{currentstroke}%
\pgfsetdash{}{0pt}%
\pgfpathmoveto{\pgfqpoint{4.076741in}{4.050389in}}%
\pgfpathlineto{\pgfqpoint{4.354519in}{4.050389in}}%
\pgfusepath{stroke}%
\end{pgfscope}%
\begin{pgfscope}%
\definecolor{textcolor}{rgb}{0.000000,0.000000,0.000000}%
\pgfsetstrokecolor{textcolor}%
\pgfsetfillcolor{textcolor}%
\pgftext[x=4.465630in,y=4.001778in,left,base]{\color{textcolor}\rmfamily\fontsize{10.000000}{12.000000}\selectfont Numerical Solution}%
\end{pgfscope}%
\end{pgfpicture}%
\makeatother%
\endgroup%
}
                    \caption{Numerical Determination of $\epsilon_4(\nu)$}
                    \label{fig:nu4 Numerical}
                \end{center}
            \end{figure}
            \noindent
            This at least tells us that Miu Miu would obtain exponential growth with some combination 
            of $\nu$ and $\epsilon$ but that solution has a rather strange form.
            
        \end{enumerate}
    \end{enumerate}


\end{document}