\documentclass[12pt]{article}
\usepackage[margin=1.2in]{geometry}
\usepackage{graphicx}
\usepackage{amsmath}
\usepackage{physics}
\usepackage{tabto}
\usepackage{float}
\usepackage{amssymb}
\usepackage{pgfplots}
\usepackage{verbatim}
\usepackage{tcolorbox}
\usepackage{listings}
\usepackage{xcolor}
\usepackage{siunitx}
\usepackage{subfig}
\usepackage{keyval2e}
\usepackage{caption}
\usepackage[all]{nowidow}
\definecolor{stringcolor}{HTML}{C792EA}
\definecolor{codeblue}{HTML}{2162DB}
\definecolor{commentcolor}{HTML}{4A6E46}
\lstdefinestyle{appendix}{
    basicstyle=\ttfamily\footnotesize,
    commentstyle=\color{commentcolor},
    keywordstyle=\color{codeblue},
    stringstyle=\color{stringcolor},
    showstringspaces=false,
    numbers=left,
    upquote=true,
    captionpos=t,
    abovecaptionskip=12pt,
    belowcaptionskip=12pt,
    language=Python,
    breaklines=true,
    frame=single,
}
\lstdefinestyle{inline}{
    basicstyle=\ttfamily\footnotesize,
    commentstyle=\color{commentcolor},
    keywordstyle=\color{codeblue},
    stringstyle=\color{stringcolor},
    showstringspaces=false,
    numbers=left,
    upquote=true,
    frame=tb,
    captionpos=b,
    language={}
}
\renewcommand{\lstlistingname}{Appendix}
\pgfplotsset{compat=1.16}

\title{Introduction to Monte Carlo Methods}
\date{\textbf{5 May 2020}}
\author{}

\begin{document}

    \begin{titlepage}
        \maketitle
        \center
        \textbf{\large{PHY2004W}}\ \
        \textbf{\large{KDSMIL001}}\ \
        \tableofcontents
    \end{titlepage}

    \section{Introduction and Aim}
    In this assignment we investigated the Monte Carlo method, a way of simulating systems by 
    use of random numbers. We revisited some data from a previous assignment and created our 
    own version of the data to analyse.

    \section{Activity}
    \begin{itemize}
        \item \textbf{Histogramming Data}\newline
        Firstly, we had a look back at some data from CP1, namely Activity1Data.txt, a list 
        of 60 values generated with a certain mean ($\mu=40$) and standard deviation ($\sigma=2$) 
        using a Monte Carlo method. We plotted these values on a histogram (Appendix 1) and plotted 
        the gaussian based on a $\mu$ and $\sigma$ calculated from the data itself. We also plotted 
        the expected gaussian given the $\mu$ and $\sigma$ used to generate the data. 
        
        \begin{figure}[H]
            \begin{center}
                \scalebox{.7}{%% Creator: Matplotlib, PGF backend
%%
%% To include the figure in your LaTeX document, write
%%   \input{<filename>.pgf}
%%
%% Make sure the required packages are loaded in your preamble
%%   \usepackage{pgf}
%%
%% Figures using additional raster images can only be included by \input if
%% they are in the same directory as the main LaTeX file. For loading figures
%% from other directories you can use the `import` package
%%   \usepackage{import}
%% and then include the figures with
%%   \import{<path to file>}{<filename>.pgf}
%%
%% Matplotlib used the following preamble
%%
\begingroup%
\makeatletter%
\begin{pgfpicture}%
\pgfpathrectangle{\pgfpointorigin}{\pgfqpoint{6.400000in}{4.800000in}}%
\pgfusepath{use as bounding box, clip}%
\begin{pgfscope}%
\pgfsetbuttcap%
\pgfsetmiterjoin%
\definecolor{currentfill}{rgb}{1.000000,1.000000,1.000000}%
\pgfsetfillcolor{currentfill}%
\pgfsetlinewidth{0.000000pt}%
\definecolor{currentstroke}{rgb}{1.000000,1.000000,1.000000}%
\pgfsetstrokecolor{currentstroke}%
\pgfsetdash{}{0pt}%
\pgfpathmoveto{\pgfqpoint{0.000000in}{0.000000in}}%
\pgfpathlineto{\pgfqpoint{6.400000in}{0.000000in}}%
\pgfpathlineto{\pgfqpoint{6.400000in}{4.800000in}}%
\pgfpathlineto{\pgfqpoint{0.000000in}{4.800000in}}%
\pgfpathclose%
\pgfusepath{fill}%
\end{pgfscope}%
\begin{pgfscope}%
\pgfsetbuttcap%
\pgfsetmiterjoin%
\definecolor{currentfill}{rgb}{1.000000,1.000000,1.000000}%
\pgfsetfillcolor{currentfill}%
\pgfsetlinewidth{0.000000pt}%
\definecolor{currentstroke}{rgb}{0.000000,0.000000,0.000000}%
\pgfsetstrokecolor{currentstroke}%
\pgfsetstrokeopacity{0.000000}%
\pgfsetdash{}{0pt}%
\pgfpathmoveto{\pgfqpoint{0.800000in}{0.528000in}}%
\pgfpathlineto{\pgfqpoint{5.760000in}{0.528000in}}%
\pgfpathlineto{\pgfqpoint{5.760000in}{4.224000in}}%
\pgfpathlineto{\pgfqpoint{0.800000in}{4.224000in}}%
\pgfpathclose%
\pgfusepath{fill}%
\end{pgfscope}%
\begin{pgfscope}%
\pgfpathrectangle{\pgfqpoint{0.800000in}{0.528000in}}{\pgfqpoint{4.960000in}{3.696000in}}%
\pgfusepath{clip}%
\pgfsetbuttcap%
\pgfsetmiterjoin%
\definecolor{currentfill}{rgb}{0.121569,0.466667,0.705882}%
\pgfsetfillcolor{currentfill}%
\pgfsetlinewidth{0.000000pt}%
\definecolor{currentstroke}{rgb}{0.000000,0.000000,0.000000}%
\pgfsetstrokecolor{currentstroke}%
\pgfsetstrokeopacity{0.000000}%
\pgfsetdash{}{0pt}%
\pgfpathmoveto{\pgfqpoint{1.025455in}{0.528000in}}%
\pgfpathlineto{\pgfqpoint{1.262775in}{0.528000in}}%
\pgfpathlineto{\pgfqpoint{1.262775in}{0.880000in}}%
\pgfpathlineto{\pgfqpoint{1.025455in}{0.880000in}}%
\pgfpathclose%
\pgfusepath{fill}%
\end{pgfscope}%
\begin{pgfscope}%
\pgfpathrectangle{\pgfqpoint{0.800000in}{0.528000in}}{\pgfqpoint{4.960000in}{3.696000in}}%
\pgfusepath{clip}%
\pgfsetbuttcap%
\pgfsetmiterjoin%
\definecolor{currentfill}{rgb}{0.121569,0.466667,0.705882}%
\pgfsetfillcolor{currentfill}%
\pgfsetlinewidth{0.000000pt}%
\definecolor{currentstroke}{rgb}{0.000000,0.000000,0.000000}%
\pgfsetstrokecolor{currentstroke}%
\pgfsetstrokeopacity{0.000000}%
\pgfsetdash{}{0pt}%
\pgfpathmoveto{\pgfqpoint{1.262775in}{0.528000in}}%
\pgfpathlineto{\pgfqpoint{1.500096in}{0.528000in}}%
\pgfpathlineto{\pgfqpoint{1.500096in}{1.232000in}}%
\pgfpathlineto{\pgfqpoint{1.262775in}{1.232000in}}%
\pgfpathclose%
\pgfusepath{fill}%
\end{pgfscope}%
\begin{pgfscope}%
\pgfpathrectangle{\pgfqpoint{0.800000in}{0.528000in}}{\pgfqpoint{4.960000in}{3.696000in}}%
\pgfusepath{clip}%
\pgfsetbuttcap%
\pgfsetmiterjoin%
\definecolor{currentfill}{rgb}{0.121569,0.466667,0.705882}%
\pgfsetfillcolor{currentfill}%
\pgfsetlinewidth{0.000000pt}%
\definecolor{currentstroke}{rgb}{0.000000,0.000000,0.000000}%
\pgfsetstrokecolor{currentstroke}%
\pgfsetstrokeopacity{0.000000}%
\pgfsetdash{}{0pt}%
\pgfpathmoveto{\pgfqpoint{1.500096in}{0.528000in}}%
\pgfpathlineto{\pgfqpoint{1.737416in}{0.528000in}}%
\pgfpathlineto{\pgfqpoint{1.737416in}{0.528000in}}%
\pgfpathlineto{\pgfqpoint{1.500096in}{0.528000in}}%
\pgfpathclose%
\pgfusepath{fill}%
\end{pgfscope}%
\begin{pgfscope}%
\pgfpathrectangle{\pgfqpoint{0.800000in}{0.528000in}}{\pgfqpoint{4.960000in}{3.696000in}}%
\pgfusepath{clip}%
\pgfsetbuttcap%
\pgfsetmiterjoin%
\definecolor{currentfill}{rgb}{0.121569,0.466667,0.705882}%
\pgfsetfillcolor{currentfill}%
\pgfsetlinewidth{0.000000pt}%
\definecolor{currentstroke}{rgb}{0.000000,0.000000,0.000000}%
\pgfsetstrokecolor{currentstroke}%
\pgfsetstrokeopacity{0.000000}%
\pgfsetdash{}{0pt}%
\pgfpathmoveto{\pgfqpoint{1.737416in}{0.528000in}}%
\pgfpathlineto{\pgfqpoint{1.974737in}{0.528000in}}%
\pgfpathlineto{\pgfqpoint{1.974737in}{1.936000in}}%
\pgfpathlineto{\pgfqpoint{1.737416in}{1.936000in}}%
\pgfpathclose%
\pgfusepath{fill}%
\end{pgfscope}%
\begin{pgfscope}%
\pgfpathrectangle{\pgfqpoint{0.800000in}{0.528000in}}{\pgfqpoint{4.960000in}{3.696000in}}%
\pgfusepath{clip}%
\pgfsetbuttcap%
\pgfsetmiterjoin%
\definecolor{currentfill}{rgb}{0.121569,0.466667,0.705882}%
\pgfsetfillcolor{currentfill}%
\pgfsetlinewidth{0.000000pt}%
\definecolor{currentstroke}{rgb}{0.000000,0.000000,0.000000}%
\pgfsetstrokecolor{currentstroke}%
\pgfsetstrokeopacity{0.000000}%
\pgfsetdash{}{0pt}%
\pgfpathmoveto{\pgfqpoint{1.974737in}{0.528000in}}%
\pgfpathlineto{\pgfqpoint{2.212057in}{0.528000in}}%
\pgfpathlineto{\pgfqpoint{2.212057in}{2.288000in}}%
\pgfpathlineto{\pgfqpoint{1.974737in}{2.288000in}}%
\pgfpathclose%
\pgfusepath{fill}%
\end{pgfscope}%
\begin{pgfscope}%
\pgfpathrectangle{\pgfqpoint{0.800000in}{0.528000in}}{\pgfqpoint{4.960000in}{3.696000in}}%
\pgfusepath{clip}%
\pgfsetbuttcap%
\pgfsetmiterjoin%
\definecolor{currentfill}{rgb}{0.121569,0.466667,0.705882}%
\pgfsetfillcolor{currentfill}%
\pgfsetlinewidth{0.000000pt}%
\definecolor{currentstroke}{rgb}{0.000000,0.000000,0.000000}%
\pgfsetstrokecolor{currentstroke}%
\pgfsetstrokeopacity{0.000000}%
\pgfsetdash{}{0pt}%
\pgfpathmoveto{\pgfqpoint{2.212057in}{0.528000in}}%
\pgfpathlineto{\pgfqpoint{2.449378in}{0.528000in}}%
\pgfpathlineto{\pgfqpoint{2.449378in}{1.936000in}}%
\pgfpathlineto{\pgfqpoint{2.212057in}{1.936000in}}%
\pgfpathclose%
\pgfusepath{fill}%
\end{pgfscope}%
\begin{pgfscope}%
\pgfpathrectangle{\pgfqpoint{0.800000in}{0.528000in}}{\pgfqpoint{4.960000in}{3.696000in}}%
\pgfusepath{clip}%
\pgfsetbuttcap%
\pgfsetmiterjoin%
\definecolor{currentfill}{rgb}{0.121569,0.466667,0.705882}%
\pgfsetfillcolor{currentfill}%
\pgfsetlinewidth{0.000000pt}%
\definecolor{currentstroke}{rgb}{0.000000,0.000000,0.000000}%
\pgfsetstrokecolor{currentstroke}%
\pgfsetstrokeopacity{0.000000}%
\pgfsetdash{}{0pt}%
\pgfpathmoveto{\pgfqpoint{2.449378in}{0.528000in}}%
\pgfpathlineto{\pgfqpoint{2.686699in}{0.528000in}}%
\pgfpathlineto{\pgfqpoint{2.686699in}{4.048000in}}%
\pgfpathlineto{\pgfqpoint{2.449378in}{4.048000in}}%
\pgfpathclose%
\pgfusepath{fill}%
\end{pgfscope}%
\begin{pgfscope}%
\pgfpathrectangle{\pgfqpoint{0.800000in}{0.528000in}}{\pgfqpoint{4.960000in}{3.696000in}}%
\pgfusepath{clip}%
\pgfsetbuttcap%
\pgfsetmiterjoin%
\definecolor{currentfill}{rgb}{0.121569,0.466667,0.705882}%
\pgfsetfillcolor{currentfill}%
\pgfsetlinewidth{0.000000pt}%
\definecolor{currentstroke}{rgb}{0.000000,0.000000,0.000000}%
\pgfsetstrokecolor{currentstroke}%
\pgfsetstrokeopacity{0.000000}%
\pgfsetdash{}{0pt}%
\pgfpathmoveto{\pgfqpoint{2.686699in}{0.528000in}}%
\pgfpathlineto{\pgfqpoint{2.924019in}{0.528000in}}%
\pgfpathlineto{\pgfqpoint{2.924019in}{3.696000in}}%
\pgfpathlineto{\pgfqpoint{2.686699in}{3.696000in}}%
\pgfpathclose%
\pgfusepath{fill}%
\end{pgfscope}%
\begin{pgfscope}%
\pgfpathrectangle{\pgfqpoint{0.800000in}{0.528000in}}{\pgfqpoint{4.960000in}{3.696000in}}%
\pgfusepath{clip}%
\pgfsetbuttcap%
\pgfsetmiterjoin%
\definecolor{currentfill}{rgb}{0.121569,0.466667,0.705882}%
\pgfsetfillcolor{currentfill}%
\pgfsetlinewidth{0.000000pt}%
\definecolor{currentstroke}{rgb}{0.000000,0.000000,0.000000}%
\pgfsetstrokecolor{currentstroke}%
\pgfsetstrokeopacity{0.000000}%
\pgfsetdash{}{0pt}%
\pgfpathmoveto{\pgfqpoint{2.924019in}{0.528000in}}%
\pgfpathlineto{\pgfqpoint{3.161340in}{0.528000in}}%
\pgfpathlineto{\pgfqpoint{3.161340in}{1.584000in}}%
\pgfpathlineto{\pgfqpoint{2.924019in}{1.584000in}}%
\pgfpathclose%
\pgfusepath{fill}%
\end{pgfscope}%
\begin{pgfscope}%
\pgfpathrectangle{\pgfqpoint{0.800000in}{0.528000in}}{\pgfqpoint{4.960000in}{3.696000in}}%
\pgfusepath{clip}%
\pgfsetbuttcap%
\pgfsetmiterjoin%
\definecolor{currentfill}{rgb}{0.121569,0.466667,0.705882}%
\pgfsetfillcolor{currentfill}%
\pgfsetlinewidth{0.000000pt}%
\definecolor{currentstroke}{rgb}{0.000000,0.000000,0.000000}%
\pgfsetstrokecolor{currentstroke}%
\pgfsetstrokeopacity{0.000000}%
\pgfsetdash{}{0pt}%
\pgfpathmoveto{\pgfqpoint{3.161340in}{0.528000in}}%
\pgfpathlineto{\pgfqpoint{3.398660in}{0.528000in}}%
\pgfpathlineto{\pgfqpoint{3.398660in}{1.936000in}}%
\pgfpathlineto{\pgfqpoint{3.161340in}{1.936000in}}%
\pgfpathclose%
\pgfusepath{fill}%
\end{pgfscope}%
\begin{pgfscope}%
\pgfpathrectangle{\pgfqpoint{0.800000in}{0.528000in}}{\pgfqpoint{4.960000in}{3.696000in}}%
\pgfusepath{clip}%
\pgfsetbuttcap%
\pgfsetmiterjoin%
\definecolor{currentfill}{rgb}{0.121569,0.466667,0.705882}%
\pgfsetfillcolor{currentfill}%
\pgfsetlinewidth{0.000000pt}%
\definecolor{currentstroke}{rgb}{0.000000,0.000000,0.000000}%
\pgfsetstrokecolor{currentstroke}%
\pgfsetstrokeopacity{0.000000}%
\pgfsetdash{}{0pt}%
\pgfpathmoveto{\pgfqpoint{3.398660in}{0.528000in}}%
\pgfpathlineto{\pgfqpoint{3.635981in}{0.528000in}}%
\pgfpathlineto{\pgfqpoint{3.635981in}{2.288000in}}%
\pgfpathlineto{\pgfqpoint{3.398660in}{2.288000in}}%
\pgfpathclose%
\pgfusepath{fill}%
\end{pgfscope}%
\begin{pgfscope}%
\pgfpathrectangle{\pgfqpoint{0.800000in}{0.528000in}}{\pgfqpoint{4.960000in}{3.696000in}}%
\pgfusepath{clip}%
\pgfsetbuttcap%
\pgfsetmiterjoin%
\definecolor{currentfill}{rgb}{0.121569,0.466667,0.705882}%
\pgfsetfillcolor{currentfill}%
\pgfsetlinewidth{0.000000pt}%
\definecolor{currentstroke}{rgb}{0.000000,0.000000,0.000000}%
\pgfsetstrokecolor{currentstroke}%
\pgfsetstrokeopacity{0.000000}%
\pgfsetdash{}{0pt}%
\pgfpathmoveto{\pgfqpoint{3.635981in}{0.528000in}}%
\pgfpathlineto{\pgfqpoint{3.873301in}{0.528000in}}%
\pgfpathlineto{\pgfqpoint{3.873301in}{1.584000in}}%
\pgfpathlineto{\pgfqpoint{3.635981in}{1.584000in}}%
\pgfpathclose%
\pgfusepath{fill}%
\end{pgfscope}%
\begin{pgfscope}%
\pgfpathrectangle{\pgfqpoint{0.800000in}{0.528000in}}{\pgfqpoint{4.960000in}{3.696000in}}%
\pgfusepath{clip}%
\pgfsetbuttcap%
\pgfsetmiterjoin%
\definecolor{currentfill}{rgb}{0.121569,0.466667,0.705882}%
\pgfsetfillcolor{currentfill}%
\pgfsetlinewidth{0.000000pt}%
\definecolor{currentstroke}{rgb}{0.000000,0.000000,0.000000}%
\pgfsetstrokecolor{currentstroke}%
\pgfsetstrokeopacity{0.000000}%
\pgfsetdash{}{0pt}%
\pgfpathmoveto{\pgfqpoint{3.873301in}{0.528000in}}%
\pgfpathlineto{\pgfqpoint{4.110622in}{0.528000in}}%
\pgfpathlineto{\pgfqpoint{4.110622in}{1.936000in}}%
\pgfpathlineto{\pgfqpoint{3.873301in}{1.936000in}}%
\pgfpathclose%
\pgfusepath{fill}%
\end{pgfscope}%
\begin{pgfscope}%
\pgfpathrectangle{\pgfqpoint{0.800000in}{0.528000in}}{\pgfqpoint{4.960000in}{3.696000in}}%
\pgfusepath{clip}%
\pgfsetbuttcap%
\pgfsetmiterjoin%
\definecolor{currentfill}{rgb}{0.121569,0.466667,0.705882}%
\pgfsetfillcolor{currentfill}%
\pgfsetlinewidth{0.000000pt}%
\definecolor{currentstroke}{rgb}{0.000000,0.000000,0.000000}%
\pgfsetstrokecolor{currentstroke}%
\pgfsetstrokeopacity{0.000000}%
\pgfsetdash{}{0pt}%
\pgfpathmoveto{\pgfqpoint{4.110622in}{0.528000in}}%
\pgfpathlineto{\pgfqpoint{4.347943in}{0.528000in}}%
\pgfpathlineto{\pgfqpoint{4.347943in}{0.880000in}}%
\pgfpathlineto{\pgfqpoint{4.110622in}{0.880000in}}%
\pgfpathclose%
\pgfusepath{fill}%
\end{pgfscope}%
\begin{pgfscope}%
\pgfpathrectangle{\pgfqpoint{0.800000in}{0.528000in}}{\pgfqpoint{4.960000in}{3.696000in}}%
\pgfusepath{clip}%
\pgfsetbuttcap%
\pgfsetmiterjoin%
\definecolor{currentfill}{rgb}{0.121569,0.466667,0.705882}%
\pgfsetfillcolor{currentfill}%
\pgfsetlinewidth{0.000000pt}%
\definecolor{currentstroke}{rgb}{0.000000,0.000000,0.000000}%
\pgfsetstrokecolor{currentstroke}%
\pgfsetstrokeopacity{0.000000}%
\pgfsetdash{}{0pt}%
\pgfpathmoveto{\pgfqpoint{4.347943in}{0.528000in}}%
\pgfpathlineto{\pgfqpoint{4.585263in}{0.528000in}}%
\pgfpathlineto{\pgfqpoint{4.585263in}{1.232000in}}%
\pgfpathlineto{\pgfqpoint{4.347943in}{1.232000in}}%
\pgfpathclose%
\pgfusepath{fill}%
\end{pgfscope}%
\begin{pgfscope}%
\pgfpathrectangle{\pgfqpoint{0.800000in}{0.528000in}}{\pgfqpoint{4.960000in}{3.696000in}}%
\pgfusepath{clip}%
\pgfsetbuttcap%
\pgfsetmiterjoin%
\definecolor{currentfill}{rgb}{0.121569,0.466667,0.705882}%
\pgfsetfillcolor{currentfill}%
\pgfsetlinewidth{0.000000pt}%
\definecolor{currentstroke}{rgb}{0.000000,0.000000,0.000000}%
\pgfsetstrokecolor{currentstroke}%
\pgfsetstrokeopacity{0.000000}%
\pgfsetdash{}{0pt}%
\pgfpathmoveto{\pgfqpoint{4.585263in}{0.528000in}}%
\pgfpathlineto{\pgfqpoint{4.822584in}{0.528000in}}%
\pgfpathlineto{\pgfqpoint{4.822584in}{0.880000in}}%
\pgfpathlineto{\pgfqpoint{4.585263in}{0.880000in}}%
\pgfpathclose%
\pgfusepath{fill}%
\end{pgfscope}%
\begin{pgfscope}%
\pgfpathrectangle{\pgfqpoint{0.800000in}{0.528000in}}{\pgfqpoint{4.960000in}{3.696000in}}%
\pgfusepath{clip}%
\pgfsetbuttcap%
\pgfsetmiterjoin%
\definecolor{currentfill}{rgb}{0.121569,0.466667,0.705882}%
\pgfsetfillcolor{currentfill}%
\pgfsetlinewidth{0.000000pt}%
\definecolor{currentstroke}{rgb}{0.000000,0.000000,0.000000}%
\pgfsetstrokecolor{currentstroke}%
\pgfsetstrokeopacity{0.000000}%
\pgfsetdash{}{0pt}%
\pgfpathmoveto{\pgfqpoint{4.822584in}{0.528000in}}%
\pgfpathlineto{\pgfqpoint{5.059904in}{0.528000in}}%
\pgfpathlineto{\pgfqpoint{5.059904in}{0.528000in}}%
\pgfpathlineto{\pgfqpoint{4.822584in}{0.528000in}}%
\pgfpathclose%
\pgfusepath{fill}%
\end{pgfscope}%
\begin{pgfscope}%
\pgfpathrectangle{\pgfqpoint{0.800000in}{0.528000in}}{\pgfqpoint{4.960000in}{3.696000in}}%
\pgfusepath{clip}%
\pgfsetbuttcap%
\pgfsetmiterjoin%
\definecolor{currentfill}{rgb}{0.121569,0.466667,0.705882}%
\pgfsetfillcolor{currentfill}%
\pgfsetlinewidth{0.000000pt}%
\definecolor{currentstroke}{rgb}{0.000000,0.000000,0.000000}%
\pgfsetstrokecolor{currentstroke}%
\pgfsetstrokeopacity{0.000000}%
\pgfsetdash{}{0pt}%
\pgfpathmoveto{\pgfqpoint{5.059904in}{0.528000in}}%
\pgfpathlineto{\pgfqpoint{5.297225in}{0.528000in}}%
\pgfpathlineto{\pgfqpoint{5.297225in}{0.528000in}}%
\pgfpathlineto{\pgfqpoint{5.059904in}{0.528000in}}%
\pgfpathclose%
\pgfusepath{fill}%
\end{pgfscope}%
\begin{pgfscope}%
\pgfpathrectangle{\pgfqpoint{0.800000in}{0.528000in}}{\pgfqpoint{4.960000in}{3.696000in}}%
\pgfusepath{clip}%
\pgfsetbuttcap%
\pgfsetmiterjoin%
\definecolor{currentfill}{rgb}{0.121569,0.466667,0.705882}%
\pgfsetfillcolor{currentfill}%
\pgfsetlinewidth{0.000000pt}%
\definecolor{currentstroke}{rgb}{0.000000,0.000000,0.000000}%
\pgfsetstrokecolor{currentstroke}%
\pgfsetstrokeopacity{0.000000}%
\pgfsetdash{}{0pt}%
\pgfpathmoveto{\pgfqpoint{5.297225in}{0.528000in}}%
\pgfpathlineto{\pgfqpoint{5.534545in}{0.528000in}}%
\pgfpathlineto{\pgfqpoint{5.534545in}{1.232000in}}%
\pgfpathlineto{\pgfqpoint{5.297225in}{1.232000in}}%
\pgfpathclose%
\pgfusepath{fill}%
\end{pgfscope}%
\begin{pgfscope}%
\pgfsetbuttcap%
\pgfsetroundjoin%
\definecolor{currentfill}{rgb}{0.000000,0.000000,0.000000}%
\pgfsetfillcolor{currentfill}%
\pgfsetlinewidth{0.803000pt}%
\definecolor{currentstroke}{rgb}{0.000000,0.000000,0.000000}%
\pgfsetstrokecolor{currentstroke}%
\pgfsetdash{}{0pt}%
\pgfsys@defobject{currentmarker}{\pgfqpoint{0.000000in}{-0.048611in}}{\pgfqpoint{0.000000in}{0.000000in}}{%
\pgfpathmoveto{\pgfqpoint{0.000000in}{0.000000in}}%
\pgfpathlineto{\pgfqpoint{0.000000in}{-0.048611in}}%
\pgfusepath{stroke,fill}%
}%
\begin{pgfscope}%
\pgfsys@transformshift{1.025455in}{0.528000in}%
\pgfsys@useobject{currentmarker}{}%
\end{pgfscope}%
\end{pgfscope}%
\begin{pgfscope}%
\definecolor{textcolor}{rgb}{0.000000,0.000000,0.000000}%
\pgfsetstrokecolor{textcolor}%
\pgfsetfillcolor{textcolor}%
\pgftext[x=1.025455in,y=0.430778in,,top]{\color{textcolor}\rmfamily\fontsize{10.000000}{12.000000}\selectfont \(\displaystyle 36\)}%
\end{pgfscope}%
\begin{pgfscope}%
\pgfsetbuttcap%
\pgfsetroundjoin%
\definecolor{currentfill}{rgb}{0.000000,0.000000,0.000000}%
\pgfsetfillcolor{currentfill}%
\pgfsetlinewidth{0.803000pt}%
\definecolor{currentstroke}{rgb}{0.000000,0.000000,0.000000}%
\pgfsetstrokecolor{currentstroke}%
\pgfsetdash{}{0pt}%
\pgfsys@defobject{currentmarker}{\pgfqpoint{0.000000in}{-0.048611in}}{\pgfqpoint{0.000000in}{0.000000in}}{%
\pgfpathmoveto{\pgfqpoint{0.000000in}{0.000000in}}%
\pgfpathlineto{\pgfqpoint{0.000000in}{-0.048611in}}%
\pgfusepath{stroke,fill}%
}%
\begin{pgfscope}%
\pgfsys@transformshift{1.974737in}{0.528000in}%
\pgfsys@useobject{currentmarker}{}%
\end{pgfscope}%
\end{pgfscope}%
\begin{pgfscope}%
\definecolor{textcolor}{rgb}{0.000000,0.000000,0.000000}%
\pgfsetstrokecolor{textcolor}%
\pgfsetfillcolor{textcolor}%
\pgftext[x=1.974737in,y=0.430778in,,top]{\color{textcolor}\rmfamily\fontsize{10.000000}{12.000000}\selectfont \(\displaystyle 38\)}%
\end{pgfscope}%
\begin{pgfscope}%
\pgfsetbuttcap%
\pgfsetroundjoin%
\definecolor{currentfill}{rgb}{0.000000,0.000000,0.000000}%
\pgfsetfillcolor{currentfill}%
\pgfsetlinewidth{0.803000pt}%
\definecolor{currentstroke}{rgb}{0.000000,0.000000,0.000000}%
\pgfsetstrokecolor{currentstroke}%
\pgfsetdash{}{0pt}%
\pgfsys@defobject{currentmarker}{\pgfqpoint{0.000000in}{-0.048611in}}{\pgfqpoint{0.000000in}{0.000000in}}{%
\pgfpathmoveto{\pgfqpoint{0.000000in}{0.000000in}}%
\pgfpathlineto{\pgfqpoint{0.000000in}{-0.048611in}}%
\pgfusepath{stroke,fill}%
}%
\begin{pgfscope}%
\pgfsys@transformshift{2.924019in}{0.528000in}%
\pgfsys@useobject{currentmarker}{}%
\end{pgfscope}%
\end{pgfscope}%
\begin{pgfscope}%
\definecolor{textcolor}{rgb}{0.000000,0.000000,0.000000}%
\pgfsetstrokecolor{textcolor}%
\pgfsetfillcolor{textcolor}%
\pgftext[x=2.924019in,y=0.430778in,,top]{\color{textcolor}\rmfamily\fontsize{10.000000}{12.000000}\selectfont \(\displaystyle 40\)}%
\end{pgfscope}%
\begin{pgfscope}%
\pgfsetbuttcap%
\pgfsetroundjoin%
\definecolor{currentfill}{rgb}{0.000000,0.000000,0.000000}%
\pgfsetfillcolor{currentfill}%
\pgfsetlinewidth{0.803000pt}%
\definecolor{currentstroke}{rgb}{0.000000,0.000000,0.000000}%
\pgfsetstrokecolor{currentstroke}%
\pgfsetdash{}{0pt}%
\pgfsys@defobject{currentmarker}{\pgfqpoint{0.000000in}{-0.048611in}}{\pgfqpoint{0.000000in}{0.000000in}}{%
\pgfpathmoveto{\pgfqpoint{0.000000in}{0.000000in}}%
\pgfpathlineto{\pgfqpoint{0.000000in}{-0.048611in}}%
\pgfusepath{stroke,fill}%
}%
\begin{pgfscope}%
\pgfsys@transformshift{3.873301in}{0.528000in}%
\pgfsys@useobject{currentmarker}{}%
\end{pgfscope}%
\end{pgfscope}%
\begin{pgfscope}%
\definecolor{textcolor}{rgb}{0.000000,0.000000,0.000000}%
\pgfsetstrokecolor{textcolor}%
\pgfsetfillcolor{textcolor}%
\pgftext[x=3.873301in,y=0.430778in,,top]{\color{textcolor}\rmfamily\fontsize{10.000000}{12.000000}\selectfont \(\displaystyle 42\)}%
\end{pgfscope}%
\begin{pgfscope}%
\pgfsetbuttcap%
\pgfsetroundjoin%
\definecolor{currentfill}{rgb}{0.000000,0.000000,0.000000}%
\pgfsetfillcolor{currentfill}%
\pgfsetlinewidth{0.803000pt}%
\definecolor{currentstroke}{rgb}{0.000000,0.000000,0.000000}%
\pgfsetstrokecolor{currentstroke}%
\pgfsetdash{}{0pt}%
\pgfsys@defobject{currentmarker}{\pgfqpoint{0.000000in}{-0.048611in}}{\pgfqpoint{0.000000in}{0.000000in}}{%
\pgfpathmoveto{\pgfqpoint{0.000000in}{0.000000in}}%
\pgfpathlineto{\pgfqpoint{0.000000in}{-0.048611in}}%
\pgfusepath{stroke,fill}%
}%
\begin{pgfscope}%
\pgfsys@transformshift{4.822584in}{0.528000in}%
\pgfsys@useobject{currentmarker}{}%
\end{pgfscope}%
\end{pgfscope}%
\begin{pgfscope}%
\definecolor{textcolor}{rgb}{0.000000,0.000000,0.000000}%
\pgfsetstrokecolor{textcolor}%
\pgfsetfillcolor{textcolor}%
\pgftext[x=4.822584in,y=0.430778in,,top]{\color{textcolor}\rmfamily\fontsize{10.000000}{12.000000}\selectfont \(\displaystyle 44\)}%
\end{pgfscope}%
\begin{pgfscope}%
\definecolor{textcolor}{rgb}{0.000000,0.000000,0.000000}%
\pgfsetstrokecolor{textcolor}%
\pgfsetfillcolor{textcolor}%
\pgftext[x=3.280000in,y=0.251766in,,top]{\color{textcolor}\rmfamily\fontsize{10.000000}{12.000000}\selectfont x}%
\end{pgfscope}%
\begin{pgfscope}%
\pgfsetbuttcap%
\pgfsetroundjoin%
\definecolor{currentfill}{rgb}{0.000000,0.000000,0.000000}%
\pgfsetfillcolor{currentfill}%
\pgfsetlinewidth{0.803000pt}%
\definecolor{currentstroke}{rgb}{0.000000,0.000000,0.000000}%
\pgfsetstrokecolor{currentstroke}%
\pgfsetdash{}{0pt}%
\pgfsys@defobject{currentmarker}{\pgfqpoint{-0.048611in}{0.000000in}}{\pgfqpoint{0.000000in}{0.000000in}}{%
\pgfpathmoveto{\pgfqpoint{0.000000in}{0.000000in}}%
\pgfpathlineto{\pgfqpoint{-0.048611in}{0.000000in}}%
\pgfusepath{stroke,fill}%
}%
\begin{pgfscope}%
\pgfsys@transformshift{0.800000in}{0.528000in}%
\pgfsys@useobject{currentmarker}{}%
\end{pgfscope}%
\end{pgfscope}%
\begin{pgfscope}%
\definecolor{textcolor}{rgb}{0.000000,0.000000,0.000000}%
\pgfsetstrokecolor{textcolor}%
\pgfsetfillcolor{textcolor}%
\pgftext[x=0.455863in,y=0.479775in,left,base]{\color{textcolor}\rmfamily\fontsize{10.000000}{12.000000}\selectfont \(\displaystyle 0.00\)}%
\end{pgfscope}%
\begin{pgfscope}%
\pgfsetbuttcap%
\pgfsetroundjoin%
\definecolor{currentfill}{rgb}{0.000000,0.000000,0.000000}%
\pgfsetfillcolor{currentfill}%
\pgfsetlinewidth{0.803000pt}%
\definecolor{currentstroke}{rgb}{0.000000,0.000000,0.000000}%
\pgfsetstrokecolor{currentstroke}%
\pgfsetdash{}{0pt}%
\pgfsys@defobject{currentmarker}{\pgfqpoint{-0.048611in}{0.000000in}}{\pgfqpoint{0.000000in}{0.000000in}}{%
\pgfpathmoveto{\pgfqpoint{0.000000in}{0.000000in}}%
\pgfpathlineto{\pgfqpoint{-0.048611in}{0.000000in}}%
\pgfusepath{stroke,fill}%
}%
\begin{pgfscope}%
\pgfsys@transformshift{0.800000in}{1.056000in}%
\pgfsys@useobject{currentmarker}{}%
\end{pgfscope}%
\end{pgfscope}%
\begin{pgfscope}%
\definecolor{textcolor}{rgb}{0.000000,0.000000,0.000000}%
\pgfsetstrokecolor{textcolor}%
\pgfsetfillcolor{textcolor}%
\pgftext[x=0.455863in,y=1.007775in,left,base]{\color{textcolor}\rmfamily\fontsize{10.000000}{12.000000}\selectfont \(\displaystyle 0.05\)}%
\end{pgfscope}%
\begin{pgfscope}%
\pgfsetbuttcap%
\pgfsetroundjoin%
\definecolor{currentfill}{rgb}{0.000000,0.000000,0.000000}%
\pgfsetfillcolor{currentfill}%
\pgfsetlinewidth{0.803000pt}%
\definecolor{currentstroke}{rgb}{0.000000,0.000000,0.000000}%
\pgfsetstrokecolor{currentstroke}%
\pgfsetdash{}{0pt}%
\pgfsys@defobject{currentmarker}{\pgfqpoint{-0.048611in}{0.000000in}}{\pgfqpoint{0.000000in}{0.000000in}}{%
\pgfpathmoveto{\pgfqpoint{0.000000in}{0.000000in}}%
\pgfpathlineto{\pgfqpoint{-0.048611in}{0.000000in}}%
\pgfusepath{stroke,fill}%
}%
\begin{pgfscope}%
\pgfsys@transformshift{0.800000in}{1.584000in}%
\pgfsys@useobject{currentmarker}{}%
\end{pgfscope}%
\end{pgfscope}%
\begin{pgfscope}%
\definecolor{textcolor}{rgb}{0.000000,0.000000,0.000000}%
\pgfsetstrokecolor{textcolor}%
\pgfsetfillcolor{textcolor}%
\pgftext[x=0.455863in,y=1.535775in,left,base]{\color{textcolor}\rmfamily\fontsize{10.000000}{12.000000}\selectfont \(\displaystyle 0.10\)}%
\end{pgfscope}%
\begin{pgfscope}%
\pgfsetbuttcap%
\pgfsetroundjoin%
\definecolor{currentfill}{rgb}{0.000000,0.000000,0.000000}%
\pgfsetfillcolor{currentfill}%
\pgfsetlinewidth{0.803000pt}%
\definecolor{currentstroke}{rgb}{0.000000,0.000000,0.000000}%
\pgfsetstrokecolor{currentstroke}%
\pgfsetdash{}{0pt}%
\pgfsys@defobject{currentmarker}{\pgfqpoint{-0.048611in}{0.000000in}}{\pgfqpoint{0.000000in}{0.000000in}}{%
\pgfpathmoveto{\pgfqpoint{0.000000in}{0.000000in}}%
\pgfpathlineto{\pgfqpoint{-0.048611in}{0.000000in}}%
\pgfusepath{stroke,fill}%
}%
\begin{pgfscope}%
\pgfsys@transformshift{0.800000in}{2.112000in}%
\pgfsys@useobject{currentmarker}{}%
\end{pgfscope}%
\end{pgfscope}%
\begin{pgfscope}%
\definecolor{textcolor}{rgb}{0.000000,0.000000,0.000000}%
\pgfsetstrokecolor{textcolor}%
\pgfsetfillcolor{textcolor}%
\pgftext[x=0.455863in,y=2.063775in,left,base]{\color{textcolor}\rmfamily\fontsize{10.000000}{12.000000}\selectfont \(\displaystyle 0.15\)}%
\end{pgfscope}%
\begin{pgfscope}%
\pgfsetbuttcap%
\pgfsetroundjoin%
\definecolor{currentfill}{rgb}{0.000000,0.000000,0.000000}%
\pgfsetfillcolor{currentfill}%
\pgfsetlinewidth{0.803000pt}%
\definecolor{currentstroke}{rgb}{0.000000,0.000000,0.000000}%
\pgfsetstrokecolor{currentstroke}%
\pgfsetdash{}{0pt}%
\pgfsys@defobject{currentmarker}{\pgfqpoint{-0.048611in}{0.000000in}}{\pgfqpoint{0.000000in}{0.000000in}}{%
\pgfpathmoveto{\pgfqpoint{0.000000in}{0.000000in}}%
\pgfpathlineto{\pgfqpoint{-0.048611in}{0.000000in}}%
\pgfusepath{stroke,fill}%
}%
\begin{pgfscope}%
\pgfsys@transformshift{0.800000in}{2.640000in}%
\pgfsys@useobject{currentmarker}{}%
\end{pgfscope}%
\end{pgfscope}%
\begin{pgfscope}%
\definecolor{textcolor}{rgb}{0.000000,0.000000,0.000000}%
\pgfsetstrokecolor{textcolor}%
\pgfsetfillcolor{textcolor}%
\pgftext[x=0.455863in,y=2.591775in,left,base]{\color{textcolor}\rmfamily\fontsize{10.000000}{12.000000}\selectfont \(\displaystyle 0.20\)}%
\end{pgfscope}%
\begin{pgfscope}%
\pgfsetbuttcap%
\pgfsetroundjoin%
\definecolor{currentfill}{rgb}{0.000000,0.000000,0.000000}%
\pgfsetfillcolor{currentfill}%
\pgfsetlinewidth{0.803000pt}%
\definecolor{currentstroke}{rgb}{0.000000,0.000000,0.000000}%
\pgfsetstrokecolor{currentstroke}%
\pgfsetdash{}{0pt}%
\pgfsys@defobject{currentmarker}{\pgfqpoint{-0.048611in}{0.000000in}}{\pgfqpoint{0.000000in}{0.000000in}}{%
\pgfpathmoveto{\pgfqpoint{0.000000in}{0.000000in}}%
\pgfpathlineto{\pgfqpoint{-0.048611in}{0.000000in}}%
\pgfusepath{stroke,fill}%
}%
\begin{pgfscope}%
\pgfsys@transformshift{0.800000in}{3.168000in}%
\pgfsys@useobject{currentmarker}{}%
\end{pgfscope}%
\end{pgfscope}%
\begin{pgfscope}%
\definecolor{textcolor}{rgb}{0.000000,0.000000,0.000000}%
\pgfsetstrokecolor{textcolor}%
\pgfsetfillcolor{textcolor}%
\pgftext[x=0.455863in,y=3.119775in,left,base]{\color{textcolor}\rmfamily\fontsize{10.000000}{12.000000}\selectfont \(\displaystyle 0.25\)}%
\end{pgfscope}%
\begin{pgfscope}%
\pgfsetbuttcap%
\pgfsetroundjoin%
\definecolor{currentfill}{rgb}{0.000000,0.000000,0.000000}%
\pgfsetfillcolor{currentfill}%
\pgfsetlinewidth{0.803000pt}%
\definecolor{currentstroke}{rgb}{0.000000,0.000000,0.000000}%
\pgfsetstrokecolor{currentstroke}%
\pgfsetdash{}{0pt}%
\pgfsys@defobject{currentmarker}{\pgfqpoint{-0.048611in}{0.000000in}}{\pgfqpoint{0.000000in}{0.000000in}}{%
\pgfpathmoveto{\pgfqpoint{0.000000in}{0.000000in}}%
\pgfpathlineto{\pgfqpoint{-0.048611in}{0.000000in}}%
\pgfusepath{stroke,fill}%
}%
\begin{pgfscope}%
\pgfsys@transformshift{0.800000in}{3.696000in}%
\pgfsys@useobject{currentmarker}{}%
\end{pgfscope}%
\end{pgfscope}%
\begin{pgfscope}%
\definecolor{textcolor}{rgb}{0.000000,0.000000,0.000000}%
\pgfsetstrokecolor{textcolor}%
\pgfsetfillcolor{textcolor}%
\pgftext[x=0.455863in,y=3.647775in,left,base]{\color{textcolor}\rmfamily\fontsize{10.000000}{12.000000}\selectfont \(\displaystyle 0.30\)}%
\end{pgfscope}%
\begin{pgfscope}%
\pgfsetbuttcap%
\pgfsetroundjoin%
\definecolor{currentfill}{rgb}{0.000000,0.000000,0.000000}%
\pgfsetfillcolor{currentfill}%
\pgfsetlinewidth{0.803000pt}%
\definecolor{currentstroke}{rgb}{0.000000,0.000000,0.000000}%
\pgfsetstrokecolor{currentstroke}%
\pgfsetdash{}{0pt}%
\pgfsys@defobject{currentmarker}{\pgfqpoint{-0.048611in}{0.000000in}}{\pgfqpoint{0.000000in}{0.000000in}}{%
\pgfpathmoveto{\pgfqpoint{0.000000in}{0.000000in}}%
\pgfpathlineto{\pgfqpoint{-0.048611in}{0.000000in}}%
\pgfusepath{stroke,fill}%
}%
\begin{pgfscope}%
\pgfsys@transformshift{0.800000in}{4.224000in}%
\pgfsys@useobject{currentmarker}{}%
\end{pgfscope}%
\end{pgfscope}%
\begin{pgfscope}%
\definecolor{textcolor}{rgb}{0.000000,0.000000,0.000000}%
\pgfsetstrokecolor{textcolor}%
\pgfsetfillcolor{textcolor}%
\pgftext[x=0.455863in,y=4.175775in,left,base]{\color{textcolor}\rmfamily\fontsize{10.000000}{12.000000}\selectfont \(\displaystyle 0.35\)}%
\end{pgfscope}%
\begin{pgfscope}%
\definecolor{textcolor}{rgb}{0.000000,0.000000,0.000000}%
\pgfsetstrokecolor{textcolor}%
\pgfsetfillcolor{textcolor}%
\pgftext[x=0.400308in,y=2.376000in,,bottom,rotate=90.000000]{\color{textcolor}\rmfamily\fontsize{10.000000}{12.000000}\selectfont Occurrence}%
\end{pgfscope}%
\begin{pgfscope}%
\pgfpathrectangle{\pgfqpoint{0.800000in}{0.528000in}}{\pgfqpoint{4.960000in}{3.696000in}}%
\pgfusepath{clip}%
\pgfsetrectcap%
\pgfsetroundjoin%
\pgfsetlinewidth{1.505625pt}%
\definecolor{currentstroke}{rgb}{1.000000,0.000000,0.000000}%
\pgfsetstrokecolor{currentstroke}%
\pgfsetdash{}{0pt}%
\pgfpathmoveto{\pgfqpoint{1.152184in}{0.825394in}}%
\pgfpathlineto{\pgfqpoint{1.314986in}{0.947691in}}%
\pgfpathlineto{\pgfqpoint{1.388080in}{1.012748in}}%
\pgfpathlineto{\pgfqpoint{1.788203in}{1.478228in}}%
\pgfpathlineto{\pgfqpoint{1.868892in}{1.590848in}}%
\pgfpathlineto{\pgfqpoint{1.882656in}{1.610490in}}%
\pgfpathlineto{\pgfqpoint{1.936291in}{1.687984in}}%
\pgfpathlineto{\pgfqpoint{1.987078in}{1.762461in}}%
\pgfpathlineto{\pgfqpoint{2.130894in}{1.975350in}}%
\pgfpathlineto{\pgfqpoint{2.142760in}{1.992818in}}%
\pgfpathlineto{\pgfqpoint{2.158423in}{2.015808in}}%
\pgfpathlineto{\pgfqpoint{2.160322in}{2.018589in}}%
\pgfpathlineto{\pgfqpoint{2.219177in}{2.103965in}}%
\pgfpathlineto{\pgfqpoint{2.309833in}{2.230963in}}%
\pgfpathlineto{\pgfqpoint{2.371062in}{2.312301in}}%
\pgfpathlineto{\pgfqpoint{2.401439in}{2.350980in}}%
\pgfpathlineto{\pgfqpoint{2.468364in}{2.431583in}}%
\pgfpathlineto{\pgfqpoint{2.482603in}{2.447830in}}%
\pgfpathlineto{\pgfqpoint{2.522473in}{2.491470in}}%
\pgfpathlineto{\pgfqpoint{2.538611in}{2.508319in}}%
\pgfpathlineto{\pgfqpoint{2.559969in}{2.529862in}}%
\pgfpathlineto{\pgfqpoint{2.606010in}{2.573215in}}%
\pgfpathlineto{\pgfqpoint{2.610281in}{2.577015in}}%
\pgfpathlineto{\pgfqpoint{2.613604in}{2.579944in}}%
\pgfpathlineto{\pgfqpoint{2.634963in}{2.598207in}}%
\pgfpathlineto{\pgfqpoint{2.638285in}{2.600958in}}%
\pgfpathlineto{\pgfqpoint{2.750775in}{2.679096in}}%
\pgfpathlineto{\pgfqpoint{2.754572in}{2.681202in}}%
\pgfpathlineto{\pgfqpoint{2.792544in}{2.700276in}}%
\pgfpathlineto{\pgfqpoint{2.821971in}{2.712523in}}%
\pgfpathlineto{\pgfqpoint{2.829566in}{2.715319in}}%
\pgfpathlineto{\pgfqpoint{2.849026in}{2.721793in}}%
\pgfpathlineto{\pgfqpoint{2.858044in}{2.724454in}}%
\pgfpathlineto{\pgfqpoint{2.880352in}{2.730110in}}%
\pgfpathlineto{\pgfqpoint{2.899812in}{2.733959in}}%
\pgfpathlineto{\pgfqpoint{3.002810in}{2.737336in}}%
\pgfpathlineto{\pgfqpoint{3.021795in}{2.734832in}}%
\pgfpathlineto{\pgfqpoint{3.091567in}{2.717344in}}%
\pgfpathlineto{\pgfqpoint{3.220195in}{2.652057in}}%
\pgfpathlineto{\pgfqpoint{3.228264in}{2.646597in}}%
\pgfpathlineto{\pgfqpoint{3.293290in}{2.597086in}}%
\pgfpathlineto{\pgfqpoint{3.310377in}{2.582505in}}%
\pgfpathlineto{\pgfqpoint{3.488842in}{2.395927in}}%
\pgfpathlineto{\pgfqpoint{3.535357in}{2.338522in}}%
\pgfpathlineto{\pgfqpoint{3.541053in}{2.331287in}}%
\pgfpathlineto{\pgfqpoint{3.557190in}{2.310561in}}%
\pgfpathlineto{\pgfqpoint{3.593738in}{2.262455in}}%
\pgfpathlineto{\pgfqpoint{3.685344in}{2.135920in}}%
\pgfpathlineto{\pgfqpoint{3.747047in}{2.047121in}}%
\pgfpathlineto{\pgfqpoint{3.796410in}{1.974767in}}%
\pgfpathlineto{\pgfqpoint{3.919342in}{1.792714in}}%
\pgfpathlineto{\pgfqpoint{3.942599in}{1.758380in}}%
\pgfpathlineto{\pgfqpoint{3.951143in}{1.745801in}}%
\pgfpathlineto{\pgfqpoint{3.999556in}{1.674974in}}%
\pgfpathlineto{\pgfqpoint{4.131032in}{1.488732in}}%
\pgfpathlineto{\pgfqpoint{4.379269in}{1.177048in}}%
\pgfpathlineto{\pgfqpoint{4.442871in}{1.107922in}}%
\pgfpathlineto{\pgfqpoint{4.749489in}{0.842371in}}%
\pgfpathlineto{\pgfqpoint{5.367946in}{0.592433in}}%
\pgfpathlineto{\pgfqpoint{5.486607in}{0.573062in}}%
\pgfusepath{stroke}%
\end{pgfscope}%
\begin{pgfscope}%
\pgfpathrectangle{\pgfqpoint{0.800000in}{0.528000in}}{\pgfqpoint{4.960000in}{3.696000in}}%
\pgfusepath{clip}%
\pgfsetrectcap%
\pgfsetroundjoin%
\pgfsetlinewidth{1.505625pt}%
\definecolor{currentstroke}{rgb}{0.000000,0.500000,0.000000}%
\pgfsetstrokecolor{currentstroke}%
\pgfsetdash{}{0pt}%
\pgfpathmoveto{\pgfqpoint{1.152184in}{0.897013in}}%
\pgfpathlineto{\pgfqpoint{1.314986in}{1.028812in}}%
\pgfpathlineto{\pgfqpoint{1.388080in}{1.096943in}}%
\pgfpathlineto{\pgfqpoint{1.788203in}{1.557611in}}%
\pgfpathlineto{\pgfqpoint{1.868892in}{1.663725in}}%
\pgfpathlineto{\pgfqpoint{1.882656in}{1.682056in}}%
\pgfpathlineto{\pgfqpoint{1.936291in}{1.753890in}}%
\pgfpathlineto{\pgfqpoint{1.987078in}{1.822213in}}%
\pgfpathlineto{\pgfqpoint{2.130894in}{2.013806in}}%
\pgfpathlineto{\pgfqpoint{2.142760in}{2.029287in}}%
\pgfpathlineto{\pgfqpoint{2.158423in}{2.049606in}}%
\pgfpathlineto{\pgfqpoint{2.160322in}{2.052059in}}%
\pgfpathlineto{\pgfqpoint{2.219177in}{2.126930in}}%
\pgfpathlineto{\pgfqpoint{2.309833in}{2.236615in}}%
\pgfpathlineto{\pgfqpoint{2.371062in}{2.305725in}}%
\pgfpathlineto{\pgfqpoint{2.401439in}{2.338245in}}%
\pgfpathlineto{\pgfqpoint{2.468364in}{2.405212in}}%
\pgfpathlineto{\pgfqpoint{2.482603in}{2.418564in}}%
\pgfpathlineto{\pgfqpoint{2.522473in}{2.454150in}}%
\pgfpathlineto{\pgfqpoint{2.538611in}{2.467770in}}%
\pgfpathlineto{\pgfqpoint{2.559969in}{2.485076in}}%
\pgfpathlineto{\pgfqpoint{2.606010in}{2.519474in}}%
\pgfpathlineto{\pgfqpoint{2.610281in}{2.522458in}}%
\pgfpathlineto{\pgfqpoint{2.613604in}{2.524754in}}%
\pgfpathlineto{\pgfqpoint{2.634963in}{2.538991in}}%
\pgfpathlineto{\pgfqpoint{2.638285in}{2.541123in}}%
\pgfpathlineto{\pgfqpoint{2.750775in}{2.599627in}}%
\pgfpathlineto{\pgfqpoint{2.754572in}{2.601124in}}%
\pgfpathlineto{\pgfqpoint{2.792544in}{2.614309in}}%
\pgfpathlineto{\pgfqpoint{2.821971in}{2.622279in}}%
\pgfpathlineto{\pgfqpoint{2.829566in}{2.624014in}}%
\pgfpathlineto{\pgfqpoint{2.849026in}{2.627852in}}%
\pgfpathlineto{\pgfqpoint{2.858044in}{2.629334in}}%
\pgfpathlineto{\pgfqpoint{2.880352in}{2.632188in}}%
\pgfpathlineto{\pgfqpoint{2.899812in}{2.633731in}}%
\pgfpathlineto{\pgfqpoint{3.002810in}{2.627172in}}%
\pgfpathlineto{\pgfqpoint{3.021795in}{2.623271in}}%
\pgfpathlineto{\pgfqpoint{3.091567in}{2.601860in}}%
\pgfpathlineto{\pgfqpoint{3.220195in}{2.534347in}}%
\pgfpathlineto{\pgfqpoint{3.228264in}{2.528961in}}%
\pgfpathlineto{\pgfqpoint{3.293290in}{2.480923in}}%
\pgfpathlineto{\pgfqpoint{3.310377in}{2.466982in}}%
\pgfpathlineto{\pgfqpoint{3.488842in}{2.292690in}}%
\pgfpathlineto{\pgfqpoint{3.535357in}{2.239927in}}%
\pgfpathlineto{\pgfqpoint{3.541053in}{2.233294in}}%
\pgfpathlineto{\pgfqpoint{3.557190in}{2.214311in}}%
\pgfpathlineto{\pgfqpoint{3.593738in}{2.170341in}}%
\pgfpathlineto{\pgfqpoint{3.685344in}{2.055123in}}%
\pgfpathlineto{\pgfqpoint{3.747047in}{1.974495in}}%
\pgfpathlineto{\pgfqpoint{3.796410in}{1.908861in}}%
\pgfpathlineto{\pgfqpoint{3.919342in}{1.743690in}}%
\pgfpathlineto{\pgfqpoint{3.942599in}{1.712503in}}%
\pgfpathlineto{\pgfqpoint{3.951143in}{1.701072in}}%
\pgfpathlineto{\pgfqpoint{3.999556in}{1.636650in}}%
\pgfpathlineto{\pgfqpoint{4.131032in}{1.466597in}}%
\pgfpathlineto{\pgfqpoint{4.379269in}{1.178472in}}%
\pgfpathlineto{\pgfqpoint{4.442871in}{1.113662in}}%
\pgfpathlineto{\pgfqpoint{4.749489in}{0.859550in}}%
\pgfpathlineto{\pgfqpoint{5.367946in}{0.604609in}}%
\pgfpathlineto{\pgfqpoint{5.486607in}{0.583097in}}%
\pgfusepath{stroke}%
\end{pgfscope}%
\begin{pgfscope}%
\pgfsetrectcap%
\pgfsetmiterjoin%
\pgfsetlinewidth{0.803000pt}%
\definecolor{currentstroke}{rgb}{0.000000,0.000000,0.000000}%
\pgfsetstrokecolor{currentstroke}%
\pgfsetdash{}{0pt}%
\pgfpathmoveto{\pgfqpoint{0.800000in}{0.528000in}}%
\pgfpathlineto{\pgfqpoint{0.800000in}{4.224000in}}%
\pgfusepath{stroke}%
\end{pgfscope}%
\begin{pgfscope}%
\pgfsetrectcap%
\pgfsetmiterjoin%
\pgfsetlinewidth{0.803000pt}%
\definecolor{currentstroke}{rgb}{0.000000,0.000000,0.000000}%
\pgfsetstrokecolor{currentstroke}%
\pgfsetdash{}{0pt}%
\pgfpathmoveto{\pgfqpoint{5.760000in}{0.528000in}}%
\pgfpathlineto{\pgfqpoint{5.760000in}{4.224000in}}%
\pgfusepath{stroke}%
\end{pgfscope}%
\begin{pgfscope}%
\pgfsetrectcap%
\pgfsetmiterjoin%
\pgfsetlinewidth{0.803000pt}%
\definecolor{currentstroke}{rgb}{0.000000,0.000000,0.000000}%
\pgfsetstrokecolor{currentstroke}%
\pgfsetdash{}{0pt}%
\pgfpathmoveto{\pgfqpoint{0.800000in}{0.528000in}}%
\pgfpathlineto{\pgfqpoint{5.760000in}{0.528000in}}%
\pgfusepath{stroke}%
\end{pgfscope}%
\begin{pgfscope}%
\pgfsetrectcap%
\pgfsetmiterjoin%
\pgfsetlinewidth{0.803000pt}%
\definecolor{currentstroke}{rgb}{0.000000,0.000000,0.000000}%
\pgfsetstrokecolor{currentstroke}%
\pgfsetdash{}{0pt}%
\pgfpathmoveto{\pgfqpoint{0.800000in}{4.224000in}}%
\pgfpathlineto{\pgfqpoint{5.760000in}{4.224000in}}%
\pgfusepath{stroke}%
\end{pgfscope}%
\begin{pgfscope}%
\pgfsetbuttcap%
\pgfsetmiterjoin%
\definecolor{currentfill}{rgb}{1.000000,0.647059,0.000000}%
\pgfsetfillcolor{currentfill}%
\pgfsetlinewidth{1.003750pt}%
\definecolor{currentstroke}{rgb}{0.000000,0.000000,0.000000}%
\pgfsetstrokecolor{currentstroke}%
\pgfsetdash{}{0pt}%
\pgfpathmoveto{\pgfqpoint{3.520163in}{2.673666in}}%
\pgfpathlineto{\pgfqpoint{4.745296in}{2.673666in}}%
\pgfpathquadraticcurveto{\pgfqpoint{4.786962in}{2.673666in}}{\pgfqpoint{4.786962in}{2.715333in}}%
\pgfpathlineto{\pgfqpoint{4.786962in}{2.838789in}}%
\pgfpathquadraticcurveto{\pgfqpoint{4.786962in}{2.880456in}}{\pgfqpoint{4.745296in}{2.880456in}}%
\pgfpathlineto{\pgfqpoint{3.520163in}{2.880456in}}%
\pgfpathquadraticcurveto{\pgfqpoint{3.478497in}{2.880456in}}{\pgfqpoint{3.478497in}{2.838789in}}%
\pgfpathlineto{\pgfqpoint{3.478497in}{2.715333in}}%
\pgfpathquadraticcurveto{\pgfqpoint{3.478497in}{2.673666in}}{\pgfqpoint{3.520163in}{2.673666in}}%
\pgfpathclose%
\pgfusepath{stroke,fill}%
\end{pgfscope}%
\begin{pgfscope}%
\definecolor{textcolor}{rgb}{0.000000,0.000000,0.000000}%
\pgfsetstrokecolor{textcolor}%
\pgfsetfillcolor{textcolor}%
\pgftext[x=3.520163in,y=2.742339in,left,base]{\color{textcolor}\rmfamily\fontsize{10.000000}{12.000000}\selectfont \(\displaystyle \mu=\)40.083\(\displaystyle ; \sigma=\)1.905}%
\end{pgfscope}%
\begin{pgfscope}%
\pgfsetbuttcap%
\pgfsetmiterjoin%
\definecolor{currentfill}{rgb}{1.000000,1.000000,1.000000}%
\pgfsetfillcolor{currentfill}%
\pgfsetfillopacity{0.800000}%
\pgfsetlinewidth{1.003750pt}%
\definecolor{currentstroke}{rgb}{0.800000,0.800000,0.800000}%
\pgfsetstrokecolor{currentstroke}%
\pgfsetstrokeopacity{0.800000}%
\pgfsetdash{}{0pt}%
\pgfpathmoveto{\pgfqpoint{4.056487in}{3.531871in}}%
\pgfpathlineto{\pgfqpoint{5.662778in}{3.531871in}}%
\pgfpathquadraticcurveto{\pgfqpoint{5.690556in}{3.531871in}}{\pgfqpoint{5.690556in}{3.559648in}}%
\pgfpathlineto{\pgfqpoint{5.690556in}{4.126778in}}%
\pgfpathquadraticcurveto{\pgfqpoint{5.690556in}{4.154556in}}{\pgfqpoint{5.662778in}{4.154556in}}%
\pgfpathlineto{\pgfqpoint{4.056487in}{4.154556in}}%
\pgfpathquadraticcurveto{\pgfqpoint{4.028709in}{4.154556in}}{\pgfqpoint{4.028709in}{4.126778in}}%
\pgfpathlineto{\pgfqpoint{4.028709in}{3.559648in}}%
\pgfpathquadraticcurveto{\pgfqpoint{4.028709in}{3.531871in}}{\pgfqpoint{4.056487in}{3.531871in}}%
\pgfpathclose%
\pgfusepath{stroke,fill}%
\end{pgfscope}%
\begin{pgfscope}%
\pgfsetrectcap%
\pgfsetroundjoin%
\pgfsetlinewidth{1.505625pt}%
\definecolor{currentstroke}{rgb}{1.000000,0.000000,0.000000}%
\pgfsetstrokecolor{currentstroke}%
\pgfsetdash{}{0pt}%
\pgfpathmoveto{\pgfqpoint{4.084265in}{4.050389in}}%
\pgfpathlineto{\pgfqpoint{4.362042in}{4.050389in}}%
\pgfusepath{stroke}%
\end{pgfscope}%
\begin{pgfscope}%
\definecolor{textcolor}{rgb}{0.000000,0.000000,0.000000}%
\pgfsetstrokecolor{textcolor}%
\pgfsetfillcolor{textcolor}%
\pgftext[x=4.473154in,y=4.001778in,left,base]{\color{textcolor}\rmfamily\fontsize{10.000000}{12.000000}\selectfont Actual Gaussian}%
\end{pgfscope}%
\begin{pgfscope}%
\pgfsetrectcap%
\pgfsetroundjoin%
\pgfsetlinewidth{1.505625pt}%
\definecolor{currentstroke}{rgb}{0.000000,0.500000,0.000000}%
\pgfsetstrokecolor{currentstroke}%
\pgfsetdash{}{0pt}%
\pgfpathmoveto{\pgfqpoint{4.084265in}{3.856716in}}%
\pgfpathlineto{\pgfqpoint{4.362042in}{3.856716in}}%
\pgfusepath{stroke}%
\end{pgfscope}%
\begin{pgfscope}%
\definecolor{textcolor}{rgb}{0.000000,0.000000,0.000000}%
\pgfsetstrokecolor{textcolor}%
\pgfsetfillcolor{textcolor}%
\pgftext[x=4.473154in,y=3.808105in,left,base]{\color{textcolor}\rmfamily\fontsize{10.000000}{12.000000}\selectfont Expected Gaussian}%
\end{pgfscope}%
\begin{pgfscope}%
\pgfsetbuttcap%
\pgfsetmiterjoin%
\definecolor{currentfill}{rgb}{0.121569,0.466667,0.705882}%
\pgfsetfillcolor{currentfill}%
\pgfsetlinewidth{0.000000pt}%
\definecolor{currentstroke}{rgb}{0.000000,0.000000,0.000000}%
\pgfsetstrokecolor{currentstroke}%
\pgfsetstrokeopacity{0.000000}%
\pgfsetdash{}{0pt}%
\pgfpathmoveto{\pgfqpoint{4.084265in}{3.614432in}}%
\pgfpathlineto{\pgfqpoint{4.362042in}{3.614432in}}%
\pgfpathlineto{\pgfqpoint{4.362042in}{3.711654in}}%
\pgfpathlineto{\pgfqpoint{4.084265in}{3.711654in}}%
\pgfpathclose%
\pgfusepath{fill}%
\end{pgfscope}%
\begin{pgfscope}%
\definecolor{textcolor}{rgb}{0.000000,0.000000,0.000000}%
\pgfsetstrokecolor{textcolor}%
\pgfsetfillcolor{textcolor}%
\pgftext[x=4.473154in,y=3.614432in,left,base]{\color{textcolor}\rmfamily\fontsize{10.000000}{12.000000}\selectfont Activity1Data}%
\end{pgfscope}%
\end{pgfpicture}%
\makeatother%
\endgroup%
}
                \caption{Activity1Data Histogram and Gaussians}
                \label{fig:Activity1DataHist}
            \end{center}
        \end{figure}
        
        \noindent
        Before scaling, the histogram was much larger than the gaussian as the area under a gaussian 
        distribution is always 1, whereas the histogram doesn't necessarily have an area of 1. To 
        resolve this and make the graph more readable, we set the density argument to True when 
        creating the histogram, which normalises it, giving us the graph above. 
        \newline
        Now, looking at the Actual Gaussian compared to the Expected Gaussian in Figure 
        (\ref{fig:Activity1DataHist}), we see that they're a bit different, in fact the actual 
        gaussian peaks slightly higher than the expected one. This is probably because the method 
        of generating data isn't perfect so we can expect slight variations from dataset to dataset. 
        We will investigate this further in the next section.

        \item \textbf{Generating Random Numbers}\newline
        Next, we created some of our own data to analyse in the same way (Appendix 2). Below is an 
        example of the file we created to analyse.
        \newline
        \lstinputlisting[title=Randomised Data, style=inline, linerange=1-4, firstnumber=1]{RandomisedData.txt}
        
        Doing exactly the same analysis as before, we have the plots below, an example of 2 different 
        datasets created from the same initial conditions.

        \begin{figure}[H]%
            \centering
            \subfloat{\scalebox{0.45}{%% Creator: Matplotlib, PGF backend
%%
%% To include the figure in your LaTeX document, write
%%   \input{<filename>.pgf}
%%
%% Make sure the required packages are loaded in your preamble
%%   \usepackage{pgf}
%%
%% Figures using additional raster images can only be included by \input if
%% they are in the same directory as the main LaTeX file. For loading figures
%% from other directories you can use the `import` package
%%   \usepackage{import}
%% and then include the figures with
%%   \import{<path to file>}{<filename>.pgf}
%%
%% Matplotlib used the following preamble
%%
\begingroup%
\makeatletter%
\begin{pgfpicture}%
\pgfpathrectangle{\pgfpointorigin}{\pgfqpoint{6.400000in}{4.800000in}}%
\pgfusepath{use as bounding box, clip}%
\begin{pgfscope}%
\pgfsetbuttcap%
\pgfsetmiterjoin%
\definecolor{currentfill}{rgb}{1.000000,1.000000,1.000000}%
\pgfsetfillcolor{currentfill}%
\pgfsetlinewidth{0.000000pt}%
\definecolor{currentstroke}{rgb}{1.000000,1.000000,1.000000}%
\pgfsetstrokecolor{currentstroke}%
\pgfsetdash{}{0pt}%
\pgfpathmoveto{\pgfqpoint{0.000000in}{0.000000in}}%
\pgfpathlineto{\pgfqpoint{6.400000in}{0.000000in}}%
\pgfpathlineto{\pgfqpoint{6.400000in}{4.800000in}}%
\pgfpathlineto{\pgfqpoint{0.000000in}{4.800000in}}%
\pgfpathclose%
\pgfusepath{fill}%
\end{pgfscope}%
\begin{pgfscope}%
\pgfsetbuttcap%
\pgfsetmiterjoin%
\definecolor{currentfill}{rgb}{1.000000,1.000000,1.000000}%
\pgfsetfillcolor{currentfill}%
\pgfsetlinewidth{0.000000pt}%
\definecolor{currentstroke}{rgb}{0.000000,0.000000,0.000000}%
\pgfsetstrokecolor{currentstroke}%
\pgfsetstrokeopacity{0.000000}%
\pgfsetdash{}{0pt}%
\pgfpathmoveto{\pgfqpoint{0.800000in}{0.528000in}}%
\pgfpathlineto{\pgfqpoint{5.760000in}{0.528000in}}%
\pgfpathlineto{\pgfqpoint{5.760000in}{4.224000in}}%
\pgfpathlineto{\pgfqpoint{0.800000in}{4.224000in}}%
\pgfpathclose%
\pgfusepath{fill}%
\end{pgfscope}%
\begin{pgfscope}%
\pgfpathrectangle{\pgfqpoint{0.800000in}{0.528000in}}{\pgfqpoint{4.960000in}{3.696000in}}%
\pgfusepath{clip}%
\pgfsetbuttcap%
\pgfsetmiterjoin%
\definecolor{currentfill}{rgb}{0.121569,0.466667,0.705882}%
\pgfsetfillcolor{currentfill}%
\pgfsetlinewidth{0.000000pt}%
\definecolor{currentstroke}{rgb}{0.000000,0.000000,0.000000}%
\pgfsetstrokecolor{currentstroke}%
\pgfsetstrokeopacity{0.000000}%
\pgfsetdash{}{0pt}%
\pgfpathmoveto{\pgfqpoint{1.025455in}{0.528000in}}%
\pgfpathlineto{\pgfqpoint{1.290695in}{0.528000in}}%
\pgfpathlineto{\pgfqpoint{1.290695in}{0.968000in}}%
\pgfpathlineto{\pgfqpoint{1.025455in}{0.968000in}}%
\pgfpathclose%
\pgfusepath{fill}%
\end{pgfscope}%
\begin{pgfscope}%
\pgfpathrectangle{\pgfqpoint{0.800000in}{0.528000in}}{\pgfqpoint{4.960000in}{3.696000in}}%
\pgfusepath{clip}%
\pgfsetbuttcap%
\pgfsetmiterjoin%
\definecolor{currentfill}{rgb}{0.121569,0.466667,0.705882}%
\pgfsetfillcolor{currentfill}%
\pgfsetlinewidth{0.000000pt}%
\definecolor{currentstroke}{rgb}{0.000000,0.000000,0.000000}%
\pgfsetstrokecolor{currentstroke}%
\pgfsetstrokeopacity{0.000000}%
\pgfsetdash{}{0pt}%
\pgfpathmoveto{\pgfqpoint{1.290695in}{0.528000in}}%
\pgfpathlineto{\pgfqpoint{1.555936in}{0.528000in}}%
\pgfpathlineto{\pgfqpoint{1.555936in}{0.528000in}}%
\pgfpathlineto{\pgfqpoint{1.290695in}{0.528000in}}%
\pgfpathclose%
\pgfusepath{fill}%
\end{pgfscope}%
\begin{pgfscope}%
\pgfpathrectangle{\pgfqpoint{0.800000in}{0.528000in}}{\pgfqpoint{4.960000in}{3.696000in}}%
\pgfusepath{clip}%
\pgfsetbuttcap%
\pgfsetmiterjoin%
\definecolor{currentfill}{rgb}{0.121569,0.466667,0.705882}%
\pgfsetfillcolor{currentfill}%
\pgfsetlinewidth{0.000000pt}%
\definecolor{currentstroke}{rgb}{0.000000,0.000000,0.000000}%
\pgfsetstrokecolor{currentstroke}%
\pgfsetstrokeopacity{0.000000}%
\pgfsetdash{}{0pt}%
\pgfpathmoveto{\pgfqpoint{1.555936in}{0.528000in}}%
\pgfpathlineto{\pgfqpoint{1.821176in}{0.528000in}}%
\pgfpathlineto{\pgfqpoint{1.821176in}{0.968000in}}%
\pgfpathlineto{\pgfqpoint{1.555936in}{0.968000in}}%
\pgfpathclose%
\pgfusepath{fill}%
\end{pgfscope}%
\begin{pgfscope}%
\pgfpathrectangle{\pgfqpoint{0.800000in}{0.528000in}}{\pgfqpoint{4.960000in}{3.696000in}}%
\pgfusepath{clip}%
\pgfsetbuttcap%
\pgfsetmiterjoin%
\definecolor{currentfill}{rgb}{0.121569,0.466667,0.705882}%
\pgfsetfillcolor{currentfill}%
\pgfsetlinewidth{0.000000pt}%
\definecolor{currentstroke}{rgb}{0.000000,0.000000,0.000000}%
\pgfsetstrokecolor{currentstroke}%
\pgfsetstrokeopacity{0.000000}%
\pgfsetdash{}{0pt}%
\pgfpathmoveto{\pgfqpoint{1.821176in}{0.528000in}}%
\pgfpathlineto{\pgfqpoint{2.086417in}{0.528000in}}%
\pgfpathlineto{\pgfqpoint{2.086417in}{2.288000in}}%
\pgfpathlineto{\pgfqpoint{1.821176in}{2.288000in}}%
\pgfpathclose%
\pgfusepath{fill}%
\end{pgfscope}%
\begin{pgfscope}%
\pgfpathrectangle{\pgfqpoint{0.800000in}{0.528000in}}{\pgfqpoint{4.960000in}{3.696000in}}%
\pgfusepath{clip}%
\pgfsetbuttcap%
\pgfsetmiterjoin%
\definecolor{currentfill}{rgb}{0.121569,0.466667,0.705882}%
\pgfsetfillcolor{currentfill}%
\pgfsetlinewidth{0.000000pt}%
\definecolor{currentstroke}{rgb}{0.000000,0.000000,0.000000}%
\pgfsetstrokecolor{currentstroke}%
\pgfsetstrokeopacity{0.000000}%
\pgfsetdash{}{0pt}%
\pgfpathmoveto{\pgfqpoint{2.086417in}{0.528000in}}%
\pgfpathlineto{\pgfqpoint{2.351658in}{0.528000in}}%
\pgfpathlineto{\pgfqpoint{2.351658in}{3.168000in}}%
\pgfpathlineto{\pgfqpoint{2.086417in}{3.168000in}}%
\pgfpathclose%
\pgfusepath{fill}%
\end{pgfscope}%
\begin{pgfscope}%
\pgfpathrectangle{\pgfqpoint{0.800000in}{0.528000in}}{\pgfqpoint{4.960000in}{3.696000in}}%
\pgfusepath{clip}%
\pgfsetbuttcap%
\pgfsetmiterjoin%
\definecolor{currentfill}{rgb}{0.121569,0.466667,0.705882}%
\pgfsetfillcolor{currentfill}%
\pgfsetlinewidth{0.000000pt}%
\definecolor{currentstroke}{rgb}{0.000000,0.000000,0.000000}%
\pgfsetstrokecolor{currentstroke}%
\pgfsetstrokeopacity{0.000000}%
\pgfsetdash{}{0pt}%
\pgfpathmoveto{\pgfqpoint{2.351658in}{0.528000in}}%
\pgfpathlineto{\pgfqpoint{2.616898in}{0.528000in}}%
\pgfpathlineto{\pgfqpoint{2.616898in}{1.848000in}}%
\pgfpathlineto{\pgfqpoint{2.351658in}{1.848000in}}%
\pgfpathclose%
\pgfusepath{fill}%
\end{pgfscope}%
\begin{pgfscope}%
\pgfpathrectangle{\pgfqpoint{0.800000in}{0.528000in}}{\pgfqpoint{4.960000in}{3.696000in}}%
\pgfusepath{clip}%
\pgfsetbuttcap%
\pgfsetmiterjoin%
\definecolor{currentfill}{rgb}{0.121569,0.466667,0.705882}%
\pgfsetfillcolor{currentfill}%
\pgfsetlinewidth{0.000000pt}%
\definecolor{currentstroke}{rgb}{0.000000,0.000000,0.000000}%
\pgfsetstrokecolor{currentstroke}%
\pgfsetstrokeopacity{0.000000}%
\pgfsetdash{}{0pt}%
\pgfpathmoveto{\pgfqpoint{2.616898in}{0.528000in}}%
\pgfpathlineto{\pgfqpoint{2.882139in}{0.528000in}}%
\pgfpathlineto{\pgfqpoint{2.882139in}{4.048000in}}%
\pgfpathlineto{\pgfqpoint{2.616898in}{4.048000in}}%
\pgfpathclose%
\pgfusepath{fill}%
\end{pgfscope}%
\begin{pgfscope}%
\pgfpathrectangle{\pgfqpoint{0.800000in}{0.528000in}}{\pgfqpoint{4.960000in}{3.696000in}}%
\pgfusepath{clip}%
\pgfsetbuttcap%
\pgfsetmiterjoin%
\definecolor{currentfill}{rgb}{0.121569,0.466667,0.705882}%
\pgfsetfillcolor{currentfill}%
\pgfsetlinewidth{0.000000pt}%
\definecolor{currentstroke}{rgb}{0.000000,0.000000,0.000000}%
\pgfsetstrokecolor{currentstroke}%
\pgfsetstrokeopacity{0.000000}%
\pgfsetdash{}{0pt}%
\pgfpathmoveto{\pgfqpoint{2.882139in}{0.528000in}}%
\pgfpathlineto{\pgfqpoint{3.147380in}{0.528000in}}%
\pgfpathlineto{\pgfqpoint{3.147380in}{3.608000in}}%
\pgfpathlineto{\pgfqpoint{2.882139in}{3.608000in}}%
\pgfpathclose%
\pgfusepath{fill}%
\end{pgfscope}%
\begin{pgfscope}%
\pgfpathrectangle{\pgfqpoint{0.800000in}{0.528000in}}{\pgfqpoint{4.960000in}{3.696000in}}%
\pgfusepath{clip}%
\pgfsetbuttcap%
\pgfsetmiterjoin%
\definecolor{currentfill}{rgb}{0.121569,0.466667,0.705882}%
\pgfsetfillcolor{currentfill}%
\pgfsetlinewidth{0.000000pt}%
\definecolor{currentstroke}{rgb}{0.000000,0.000000,0.000000}%
\pgfsetstrokecolor{currentstroke}%
\pgfsetstrokeopacity{0.000000}%
\pgfsetdash{}{0pt}%
\pgfpathmoveto{\pgfqpoint{3.147380in}{0.528000in}}%
\pgfpathlineto{\pgfqpoint{3.412620in}{0.528000in}}%
\pgfpathlineto{\pgfqpoint{3.412620in}{3.168000in}}%
\pgfpathlineto{\pgfqpoint{3.147380in}{3.168000in}}%
\pgfpathclose%
\pgfusepath{fill}%
\end{pgfscope}%
\begin{pgfscope}%
\pgfpathrectangle{\pgfqpoint{0.800000in}{0.528000in}}{\pgfqpoint{4.960000in}{3.696000in}}%
\pgfusepath{clip}%
\pgfsetbuttcap%
\pgfsetmiterjoin%
\definecolor{currentfill}{rgb}{0.121569,0.466667,0.705882}%
\pgfsetfillcolor{currentfill}%
\pgfsetlinewidth{0.000000pt}%
\definecolor{currentstroke}{rgb}{0.000000,0.000000,0.000000}%
\pgfsetstrokecolor{currentstroke}%
\pgfsetstrokeopacity{0.000000}%
\pgfsetdash{}{0pt}%
\pgfpathmoveto{\pgfqpoint{3.412620in}{0.528000in}}%
\pgfpathlineto{\pgfqpoint{3.677861in}{0.528000in}}%
\pgfpathlineto{\pgfqpoint{3.677861in}{2.728000in}}%
\pgfpathlineto{\pgfqpoint{3.412620in}{2.728000in}}%
\pgfpathclose%
\pgfusepath{fill}%
\end{pgfscope}%
\begin{pgfscope}%
\pgfpathrectangle{\pgfqpoint{0.800000in}{0.528000in}}{\pgfqpoint{4.960000in}{3.696000in}}%
\pgfusepath{clip}%
\pgfsetbuttcap%
\pgfsetmiterjoin%
\definecolor{currentfill}{rgb}{0.121569,0.466667,0.705882}%
\pgfsetfillcolor{currentfill}%
\pgfsetlinewidth{0.000000pt}%
\definecolor{currentstroke}{rgb}{0.000000,0.000000,0.000000}%
\pgfsetstrokecolor{currentstroke}%
\pgfsetstrokeopacity{0.000000}%
\pgfsetdash{}{0pt}%
\pgfpathmoveto{\pgfqpoint{3.677861in}{0.528000in}}%
\pgfpathlineto{\pgfqpoint{3.943102in}{0.528000in}}%
\pgfpathlineto{\pgfqpoint{3.943102in}{2.288000in}}%
\pgfpathlineto{\pgfqpoint{3.677861in}{2.288000in}}%
\pgfpathclose%
\pgfusepath{fill}%
\end{pgfscope}%
\begin{pgfscope}%
\pgfpathrectangle{\pgfqpoint{0.800000in}{0.528000in}}{\pgfqpoint{4.960000in}{3.696000in}}%
\pgfusepath{clip}%
\pgfsetbuttcap%
\pgfsetmiterjoin%
\definecolor{currentfill}{rgb}{0.121569,0.466667,0.705882}%
\pgfsetfillcolor{currentfill}%
\pgfsetlinewidth{0.000000pt}%
\definecolor{currentstroke}{rgb}{0.000000,0.000000,0.000000}%
\pgfsetstrokecolor{currentstroke}%
\pgfsetstrokeopacity{0.000000}%
\pgfsetdash{}{0pt}%
\pgfpathmoveto{\pgfqpoint{3.943102in}{0.528000in}}%
\pgfpathlineto{\pgfqpoint{4.208342in}{0.528000in}}%
\pgfpathlineto{\pgfqpoint{4.208342in}{3.168000in}}%
\pgfpathlineto{\pgfqpoint{3.943102in}{3.168000in}}%
\pgfpathclose%
\pgfusepath{fill}%
\end{pgfscope}%
\begin{pgfscope}%
\pgfpathrectangle{\pgfqpoint{0.800000in}{0.528000in}}{\pgfqpoint{4.960000in}{3.696000in}}%
\pgfusepath{clip}%
\pgfsetbuttcap%
\pgfsetmiterjoin%
\definecolor{currentfill}{rgb}{0.121569,0.466667,0.705882}%
\pgfsetfillcolor{currentfill}%
\pgfsetlinewidth{0.000000pt}%
\definecolor{currentstroke}{rgb}{0.000000,0.000000,0.000000}%
\pgfsetstrokecolor{currentstroke}%
\pgfsetstrokeopacity{0.000000}%
\pgfsetdash{}{0pt}%
\pgfpathmoveto{\pgfqpoint{4.208342in}{0.528000in}}%
\pgfpathlineto{\pgfqpoint{4.473583in}{0.528000in}}%
\pgfpathlineto{\pgfqpoint{4.473583in}{0.528000in}}%
\pgfpathlineto{\pgfqpoint{4.208342in}{0.528000in}}%
\pgfpathclose%
\pgfusepath{fill}%
\end{pgfscope}%
\begin{pgfscope}%
\pgfpathrectangle{\pgfqpoint{0.800000in}{0.528000in}}{\pgfqpoint{4.960000in}{3.696000in}}%
\pgfusepath{clip}%
\pgfsetbuttcap%
\pgfsetmiterjoin%
\definecolor{currentfill}{rgb}{0.121569,0.466667,0.705882}%
\pgfsetfillcolor{currentfill}%
\pgfsetlinewidth{0.000000pt}%
\definecolor{currentstroke}{rgb}{0.000000,0.000000,0.000000}%
\pgfsetstrokecolor{currentstroke}%
\pgfsetstrokeopacity{0.000000}%
\pgfsetdash{}{0pt}%
\pgfpathmoveto{\pgfqpoint{4.473583in}{0.528000in}}%
\pgfpathlineto{\pgfqpoint{4.738824in}{0.528000in}}%
\pgfpathlineto{\pgfqpoint{4.738824in}{0.528000in}}%
\pgfpathlineto{\pgfqpoint{4.473583in}{0.528000in}}%
\pgfpathclose%
\pgfusepath{fill}%
\end{pgfscope}%
\begin{pgfscope}%
\pgfpathrectangle{\pgfqpoint{0.800000in}{0.528000in}}{\pgfqpoint{4.960000in}{3.696000in}}%
\pgfusepath{clip}%
\pgfsetbuttcap%
\pgfsetmiterjoin%
\definecolor{currentfill}{rgb}{0.121569,0.466667,0.705882}%
\pgfsetfillcolor{currentfill}%
\pgfsetlinewidth{0.000000pt}%
\definecolor{currentstroke}{rgb}{0.000000,0.000000,0.000000}%
\pgfsetstrokecolor{currentstroke}%
\pgfsetstrokeopacity{0.000000}%
\pgfsetdash{}{0pt}%
\pgfpathmoveto{\pgfqpoint{4.738824in}{0.528000in}}%
\pgfpathlineto{\pgfqpoint{5.004064in}{0.528000in}}%
\pgfpathlineto{\pgfqpoint{5.004064in}{2.288000in}}%
\pgfpathlineto{\pgfqpoint{4.738824in}{2.288000in}}%
\pgfpathclose%
\pgfusepath{fill}%
\end{pgfscope}%
\begin{pgfscope}%
\pgfpathrectangle{\pgfqpoint{0.800000in}{0.528000in}}{\pgfqpoint{4.960000in}{3.696000in}}%
\pgfusepath{clip}%
\pgfsetbuttcap%
\pgfsetmiterjoin%
\definecolor{currentfill}{rgb}{0.121569,0.466667,0.705882}%
\pgfsetfillcolor{currentfill}%
\pgfsetlinewidth{0.000000pt}%
\definecolor{currentstroke}{rgb}{0.000000,0.000000,0.000000}%
\pgfsetstrokecolor{currentstroke}%
\pgfsetstrokeopacity{0.000000}%
\pgfsetdash{}{0pt}%
\pgfpathmoveto{\pgfqpoint{5.004064in}{0.528000in}}%
\pgfpathlineto{\pgfqpoint{5.269305in}{0.528000in}}%
\pgfpathlineto{\pgfqpoint{5.269305in}{1.848000in}}%
\pgfpathlineto{\pgfqpoint{5.004064in}{1.848000in}}%
\pgfpathclose%
\pgfusepath{fill}%
\end{pgfscope}%
\begin{pgfscope}%
\pgfpathrectangle{\pgfqpoint{0.800000in}{0.528000in}}{\pgfqpoint{4.960000in}{3.696000in}}%
\pgfusepath{clip}%
\pgfsetbuttcap%
\pgfsetmiterjoin%
\definecolor{currentfill}{rgb}{0.121569,0.466667,0.705882}%
\pgfsetfillcolor{currentfill}%
\pgfsetlinewidth{0.000000pt}%
\definecolor{currentstroke}{rgb}{0.000000,0.000000,0.000000}%
\pgfsetstrokecolor{currentstroke}%
\pgfsetstrokeopacity{0.000000}%
\pgfsetdash{}{0pt}%
\pgfpathmoveto{\pgfqpoint{5.269305in}{0.528000in}}%
\pgfpathlineto{\pgfqpoint{5.534545in}{0.528000in}}%
\pgfpathlineto{\pgfqpoint{5.534545in}{1.408000in}}%
\pgfpathlineto{\pgfqpoint{5.269305in}{1.408000in}}%
\pgfpathclose%
\pgfusepath{fill}%
\end{pgfscope}%
\begin{pgfscope}%
\pgfsetbuttcap%
\pgfsetroundjoin%
\definecolor{currentfill}{rgb}{0.000000,0.000000,0.000000}%
\pgfsetfillcolor{currentfill}%
\pgfsetlinewidth{0.803000pt}%
\definecolor{currentstroke}{rgb}{0.000000,0.000000,0.000000}%
\pgfsetstrokecolor{currentstroke}%
\pgfsetdash{}{0pt}%
\pgfsys@defobject{currentmarker}{\pgfqpoint{0.000000in}{-0.048611in}}{\pgfqpoint{0.000000in}{0.000000in}}{%
\pgfpathmoveto{\pgfqpoint{0.000000in}{0.000000in}}%
\pgfpathlineto{\pgfqpoint{0.000000in}{-0.048611in}}%
\pgfusepath{stroke,fill}%
}%
\begin{pgfscope}%
\pgfsys@transformshift{1.025455in}{0.528000in}%
\pgfsys@useobject{currentmarker}{}%
\end{pgfscope}%
\end{pgfscope}%
\begin{pgfscope}%
\definecolor{textcolor}{rgb}{0.000000,0.000000,0.000000}%
\pgfsetstrokecolor{textcolor}%
\pgfsetfillcolor{textcolor}%
\pgftext[x=1.025455in,y=0.430778in,,top]{\color{textcolor}\rmfamily\fontsize{10.000000}{12.000000}\selectfont \(\displaystyle 36\)}%
\end{pgfscope}%
\begin{pgfscope}%
\pgfsetbuttcap%
\pgfsetroundjoin%
\definecolor{currentfill}{rgb}{0.000000,0.000000,0.000000}%
\pgfsetfillcolor{currentfill}%
\pgfsetlinewidth{0.803000pt}%
\definecolor{currentstroke}{rgb}{0.000000,0.000000,0.000000}%
\pgfsetstrokecolor{currentstroke}%
\pgfsetdash{}{0pt}%
\pgfsys@defobject{currentmarker}{\pgfqpoint{0.000000in}{-0.048611in}}{\pgfqpoint{0.000000in}{0.000000in}}{%
\pgfpathmoveto{\pgfqpoint{0.000000in}{0.000000in}}%
\pgfpathlineto{\pgfqpoint{0.000000in}{-0.048611in}}%
\pgfusepath{stroke,fill}%
}%
\begin{pgfscope}%
\pgfsys@transformshift{2.086417in}{0.528000in}%
\pgfsys@useobject{currentmarker}{}%
\end{pgfscope}%
\end{pgfscope}%
\begin{pgfscope}%
\definecolor{textcolor}{rgb}{0.000000,0.000000,0.000000}%
\pgfsetstrokecolor{textcolor}%
\pgfsetfillcolor{textcolor}%
\pgftext[x=2.086417in,y=0.430778in,,top]{\color{textcolor}\rmfamily\fontsize{10.000000}{12.000000}\selectfont \(\displaystyle 38\)}%
\end{pgfscope}%
\begin{pgfscope}%
\pgfsetbuttcap%
\pgfsetroundjoin%
\definecolor{currentfill}{rgb}{0.000000,0.000000,0.000000}%
\pgfsetfillcolor{currentfill}%
\pgfsetlinewidth{0.803000pt}%
\definecolor{currentstroke}{rgb}{0.000000,0.000000,0.000000}%
\pgfsetstrokecolor{currentstroke}%
\pgfsetdash{}{0pt}%
\pgfsys@defobject{currentmarker}{\pgfqpoint{0.000000in}{-0.048611in}}{\pgfqpoint{0.000000in}{0.000000in}}{%
\pgfpathmoveto{\pgfqpoint{0.000000in}{0.000000in}}%
\pgfpathlineto{\pgfqpoint{0.000000in}{-0.048611in}}%
\pgfusepath{stroke,fill}%
}%
\begin{pgfscope}%
\pgfsys@transformshift{3.147380in}{0.528000in}%
\pgfsys@useobject{currentmarker}{}%
\end{pgfscope}%
\end{pgfscope}%
\begin{pgfscope}%
\definecolor{textcolor}{rgb}{0.000000,0.000000,0.000000}%
\pgfsetstrokecolor{textcolor}%
\pgfsetfillcolor{textcolor}%
\pgftext[x=3.147380in,y=0.430778in,,top]{\color{textcolor}\rmfamily\fontsize{10.000000}{12.000000}\selectfont \(\displaystyle 40\)}%
\end{pgfscope}%
\begin{pgfscope}%
\pgfsetbuttcap%
\pgfsetroundjoin%
\definecolor{currentfill}{rgb}{0.000000,0.000000,0.000000}%
\pgfsetfillcolor{currentfill}%
\pgfsetlinewidth{0.803000pt}%
\definecolor{currentstroke}{rgb}{0.000000,0.000000,0.000000}%
\pgfsetstrokecolor{currentstroke}%
\pgfsetdash{}{0pt}%
\pgfsys@defobject{currentmarker}{\pgfqpoint{0.000000in}{-0.048611in}}{\pgfqpoint{0.000000in}{0.000000in}}{%
\pgfpathmoveto{\pgfqpoint{0.000000in}{0.000000in}}%
\pgfpathlineto{\pgfqpoint{0.000000in}{-0.048611in}}%
\pgfusepath{stroke,fill}%
}%
\begin{pgfscope}%
\pgfsys@transformshift{4.208342in}{0.528000in}%
\pgfsys@useobject{currentmarker}{}%
\end{pgfscope}%
\end{pgfscope}%
\begin{pgfscope}%
\definecolor{textcolor}{rgb}{0.000000,0.000000,0.000000}%
\pgfsetstrokecolor{textcolor}%
\pgfsetfillcolor{textcolor}%
\pgftext[x=4.208342in,y=0.430778in,,top]{\color{textcolor}\rmfamily\fontsize{10.000000}{12.000000}\selectfont \(\displaystyle 42\)}%
\end{pgfscope}%
\begin{pgfscope}%
\pgfsetbuttcap%
\pgfsetroundjoin%
\definecolor{currentfill}{rgb}{0.000000,0.000000,0.000000}%
\pgfsetfillcolor{currentfill}%
\pgfsetlinewidth{0.803000pt}%
\definecolor{currentstroke}{rgb}{0.000000,0.000000,0.000000}%
\pgfsetstrokecolor{currentstroke}%
\pgfsetdash{}{0pt}%
\pgfsys@defobject{currentmarker}{\pgfqpoint{0.000000in}{-0.048611in}}{\pgfqpoint{0.000000in}{0.000000in}}{%
\pgfpathmoveto{\pgfqpoint{0.000000in}{0.000000in}}%
\pgfpathlineto{\pgfqpoint{0.000000in}{-0.048611in}}%
\pgfusepath{stroke,fill}%
}%
\begin{pgfscope}%
\pgfsys@transformshift{5.269305in}{0.528000in}%
\pgfsys@useobject{currentmarker}{}%
\end{pgfscope}%
\end{pgfscope}%
\begin{pgfscope}%
\definecolor{textcolor}{rgb}{0.000000,0.000000,0.000000}%
\pgfsetstrokecolor{textcolor}%
\pgfsetfillcolor{textcolor}%
\pgftext[x=5.269305in,y=0.430778in,,top]{\color{textcolor}\rmfamily\fontsize{10.000000}{12.000000}\selectfont \(\displaystyle 44\)}%
\end{pgfscope}%
\begin{pgfscope}%
\definecolor{textcolor}{rgb}{0.000000,0.000000,0.000000}%
\pgfsetstrokecolor{textcolor}%
\pgfsetfillcolor{textcolor}%
\pgftext[x=3.280000in,y=0.251766in,,top]{\color{textcolor}\rmfamily\fontsize{10.000000}{12.000000}\selectfont x}%
\end{pgfscope}%
\begin{pgfscope}%
\pgfsetbuttcap%
\pgfsetroundjoin%
\definecolor{currentfill}{rgb}{0.000000,0.000000,0.000000}%
\pgfsetfillcolor{currentfill}%
\pgfsetlinewidth{0.803000pt}%
\definecolor{currentstroke}{rgb}{0.000000,0.000000,0.000000}%
\pgfsetstrokecolor{currentstroke}%
\pgfsetdash{}{0pt}%
\pgfsys@defobject{currentmarker}{\pgfqpoint{-0.048611in}{0.000000in}}{\pgfqpoint{0.000000in}{0.000000in}}{%
\pgfpathmoveto{\pgfqpoint{0.000000in}{0.000000in}}%
\pgfpathlineto{\pgfqpoint{-0.048611in}{0.000000in}}%
\pgfusepath{stroke,fill}%
}%
\begin{pgfscope}%
\pgfsys@transformshift{0.800000in}{0.528000in}%
\pgfsys@useobject{currentmarker}{}%
\end{pgfscope}%
\end{pgfscope}%
\begin{pgfscope}%
\definecolor{textcolor}{rgb}{0.000000,0.000000,0.000000}%
\pgfsetstrokecolor{textcolor}%
\pgfsetfillcolor{textcolor}%
\pgftext[x=0.455863in,y=0.479775in,left,base]{\color{textcolor}\rmfamily\fontsize{10.000000}{12.000000}\selectfont \(\displaystyle 0.00\)}%
\end{pgfscope}%
\begin{pgfscope}%
\pgfsetbuttcap%
\pgfsetroundjoin%
\definecolor{currentfill}{rgb}{0.000000,0.000000,0.000000}%
\pgfsetfillcolor{currentfill}%
\pgfsetlinewidth{0.803000pt}%
\definecolor{currentstroke}{rgb}{0.000000,0.000000,0.000000}%
\pgfsetstrokecolor{currentstroke}%
\pgfsetdash{}{0pt}%
\pgfsys@defobject{currentmarker}{\pgfqpoint{-0.048611in}{0.000000in}}{\pgfqpoint{0.000000in}{0.000000in}}{%
\pgfpathmoveto{\pgfqpoint{0.000000in}{0.000000in}}%
\pgfpathlineto{\pgfqpoint{-0.048611in}{0.000000in}}%
\pgfusepath{stroke,fill}%
}%
\begin{pgfscope}%
\pgfsys@transformshift{0.800000in}{1.188000in}%
\pgfsys@useobject{currentmarker}{}%
\end{pgfscope}%
\end{pgfscope}%
\begin{pgfscope}%
\definecolor{textcolor}{rgb}{0.000000,0.000000,0.000000}%
\pgfsetstrokecolor{textcolor}%
\pgfsetfillcolor{textcolor}%
\pgftext[x=0.455863in,y=1.139775in,left,base]{\color{textcolor}\rmfamily\fontsize{10.000000}{12.000000}\selectfont \(\displaystyle 0.05\)}%
\end{pgfscope}%
\begin{pgfscope}%
\pgfsetbuttcap%
\pgfsetroundjoin%
\definecolor{currentfill}{rgb}{0.000000,0.000000,0.000000}%
\pgfsetfillcolor{currentfill}%
\pgfsetlinewidth{0.803000pt}%
\definecolor{currentstroke}{rgb}{0.000000,0.000000,0.000000}%
\pgfsetstrokecolor{currentstroke}%
\pgfsetdash{}{0pt}%
\pgfsys@defobject{currentmarker}{\pgfqpoint{-0.048611in}{0.000000in}}{\pgfqpoint{0.000000in}{0.000000in}}{%
\pgfpathmoveto{\pgfqpoint{0.000000in}{0.000000in}}%
\pgfpathlineto{\pgfqpoint{-0.048611in}{0.000000in}}%
\pgfusepath{stroke,fill}%
}%
\begin{pgfscope}%
\pgfsys@transformshift{0.800000in}{1.848000in}%
\pgfsys@useobject{currentmarker}{}%
\end{pgfscope}%
\end{pgfscope}%
\begin{pgfscope}%
\definecolor{textcolor}{rgb}{0.000000,0.000000,0.000000}%
\pgfsetstrokecolor{textcolor}%
\pgfsetfillcolor{textcolor}%
\pgftext[x=0.455863in,y=1.799775in,left,base]{\color{textcolor}\rmfamily\fontsize{10.000000}{12.000000}\selectfont \(\displaystyle 0.10\)}%
\end{pgfscope}%
\begin{pgfscope}%
\pgfsetbuttcap%
\pgfsetroundjoin%
\definecolor{currentfill}{rgb}{0.000000,0.000000,0.000000}%
\pgfsetfillcolor{currentfill}%
\pgfsetlinewidth{0.803000pt}%
\definecolor{currentstroke}{rgb}{0.000000,0.000000,0.000000}%
\pgfsetstrokecolor{currentstroke}%
\pgfsetdash{}{0pt}%
\pgfsys@defobject{currentmarker}{\pgfqpoint{-0.048611in}{0.000000in}}{\pgfqpoint{0.000000in}{0.000000in}}{%
\pgfpathmoveto{\pgfqpoint{0.000000in}{0.000000in}}%
\pgfpathlineto{\pgfqpoint{-0.048611in}{0.000000in}}%
\pgfusepath{stroke,fill}%
}%
\begin{pgfscope}%
\pgfsys@transformshift{0.800000in}{2.508000in}%
\pgfsys@useobject{currentmarker}{}%
\end{pgfscope}%
\end{pgfscope}%
\begin{pgfscope}%
\definecolor{textcolor}{rgb}{0.000000,0.000000,0.000000}%
\pgfsetstrokecolor{textcolor}%
\pgfsetfillcolor{textcolor}%
\pgftext[x=0.455863in,y=2.459775in,left,base]{\color{textcolor}\rmfamily\fontsize{10.000000}{12.000000}\selectfont \(\displaystyle 0.15\)}%
\end{pgfscope}%
\begin{pgfscope}%
\pgfsetbuttcap%
\pgfsetroundjoin%
\definecolor{currentfill}{rgb}{0.000000,0.000000,0.000000}%
\pgfsetfillcolor{currentfill}%
\pgfsetlinewidth{0.803000pt}%
\definecolor{currentstroke}{rgb}{0.000000,0.000000,0.000000}%
\pgfsetstrokecolor{currentstroke}%
\pgfsetdash{}{0pt}%
\pgfsys@defobject{currentmarker}{\pgfqpoint{-0.048611in}{0.000000in}}{\pgfqpoint{0.000000in}{0.000000in}}{%
\pgfpathmoveto{\pgfqpoint{0.000000in}{0.000000in}}%
\pgfpathlineto{\pgfqpoint{-0.048611in}{0.000000in}}%
\pgfusepath{stroke,fill}%
}%
\begin{pgfscope}%
\pgfsys@transformshift{0.800000in}{3.168000in}%
\pgfsys@useobject{currentmarker}{}%
\end{pgfscope}%
\end{pgfscope}%
\begin{pgfscope}%
\definecolor{textcolor}{rgb}{0.000000,0.000000,0.000000}%
\pgfsetstrokecolor{textcolor}%
\pgfsetfillcolor{textcolor}%
\pgftext[x=0.455863in,y=3.119775in,left,base]{\color{textcolor}\rmfamily\fontsize{10.000000}{12.000000}\selectfont \(\displaystyle 0.20\)}%
\end{pgfscope}%
\begin{pgfscope}%
\pgfsetbuttcap%
\pgfsetroundjoin%
\definecolor{currentfill}{rgb}{0.000000,0.000000,0.000000}%
\pgfsetfillcolor{currentfill}%
\pgfsetlinewidth{0.803000pt}%
\definecolor{currentstroke}{rgb}{0.000000,0.000000,0.000000}%
\pgfsetstrokecolor{currentstroke}%
\pgfsetdash{}{0pt}%
\pgfsys@defobject{currentmarker}{\pgfqpoint{-0.048611in}{0.000000in}}{\pgfqpoint{0.000000in}{0.000000in}}{%
\pgfpathmoveto{\pgfqpoint{0.000000in}{0.000000in}}%
\pgfpathlineto{\pgfqpoint{-0.048611in}{0.000000in}}%
\pgfusepath{stroke,fill}%
}%
\begin{pgfscope}%
\pgfsys@transformshift{0.800000in}{3.828000in}%
\pgfsys@useobject{currentmarker}{}%
\end{pgfscope}%
\end{pgfscope}%
\begin{pgfscope}%
\definecolor{textcolor}{rgb}{0.000000,0.000000,0.000000}%
\pgfsetstrokecolor{textcolor}%
\pgfsetfillcolor{textcolor}%
\pgftext[x=0.455863in,y=3.779775in,left,base]{\color{textcolor}\rmfamily\fontsize{10.000000}{12.000000}\selectfont \(\displaystyle 0.25\)}%
\end{pgfscope}%
\begin{pgfscope}%
\definecolor{textcolor}{rgb}{0.000000,0.000000,0.000000}%
\pgfsetstrokecolor{textcolor}%
\pgfsetfillcolor{textcolor}%
\pgftext[x=0.400308in,y=2.376000in,,bottom,rotate=90.000000]{\color{textcolor}\rmfamily\fontsize{10.000000}{12.000000}\selectfont Occurrence}%
\end{pgfscope}%
\begin{pgfscope}%
\pgfpathrectangle{\pgfqpoint{0.800000in}{0.528000in}}{\pgfqpoint{4.960000in}{3.696000in}}%
\pgfusepath{clip}%
\pgfsetrectcap%
\pgfsetroundjoin%
\pgfsetlinewidth{1.505625pt}%
\definecolor{currentstroke}{rgb}{1.000000,0.000000,0.000000}%
\pgfsetstrokecolor{currentstroke}%
\pgfsetdash{}{0pt}%
\pgfpathmoveto{\pgfqpoint{1.230751in}{0.847036in}}%
\pgfpathlineto{\pgfqpoint{1.746909in}{1.354622in}}%
\pgfpathlineto{\pgfqpoint{1.923559in}{1.603797in}}%
\pgfpathlineto{\pgfqpoint{1.963345in}{1.664569in}}%
\pgfpathlineto{\pgfqpoint{1.996766in}{1.716779in}}%
\pgfpathlineto{\pgfqpoint{2.067320in}{1.830136in}}%
\pgfpathlineto{\pgfqpoint{2.095966in}{1.877235in}}%
\pgfpathlineto{\pgfqpoint{2.146892in}{1.962247in}}%
\pgfpathlineto{\pgfqpoint{2.169172in}{1.999881in}}%
\pgfpathlineto{\pgfqpoint{2.197818in}{2.048592in}}%
\pgfpathlineto{\pgfqpoint{2.255641in}{2.147750in}}%
\pgfpathlineto{\pgfqpoint{2.261476in}{2.157800in}}%
\pgfpathlineto{\pgfqpoint{2.406297in}{2.407232in}}%
\pgfpathlineto{\pgfqpoint{2.565972in}{2.672916in}}%
\pgfpathlineto{\pgfqpoint{2.609472in}{2.741604in}}%
\pgfpathlineto{\pgfqpoint{2.691696in}{2.865164in}}%
\pgfpathlineto{\pgfqpoint{2.697532in}{2.873577in}}%
\pgfpathlineto{\pgfqpoint{2.728830in}{2.917801in}}%
\pgfpathlineto{\pgfqpoint{2.730952in}{2.920743in}}%
\pgfpathlineto{\pgfqpoint{2.748458in}{2.944728in}}%
\pgfpathlineto{\pgfqpoint{2.838109in}{3.058963in}}%
\pgfpathlineto{\pgfqpoint{2.839170in}{3.060222in}}%
\pgfpathlineto{\pgfqpoint{2.863572in}{3.088555in}}%
\pgfpathlineto{\pgfqpoint{2.903889in}{3.132596in}}%
\pgfpathlineto{\pgfqpoint{2.961711in}{3.189380in}}%
\pgfpathlineto{\pgfqpoint{3.043405in}{3.255835in}}%
\pgfpathlineto{\pgfqpoint{3.072582in}{3.275437in}}%
\pgfpathlineto{\pgfqpoint{3.081600in}{3.281041in}}%
\pgfpathlineto{\pgfqpoint{3.093271in}{3.287971in}}%
\pgfpathlineto{\pgfqpoint{3.105472in}{3.294824in}}%
\pgfpathlineto{\pgfqpoint{3.148971in}{3.315950in}}%
\pgfpathlineto{\pgfqpoint{3.194593in}{3.332464in}}%
\pgfpathlineto{\pgfqpoint{3.236501in}{3.342454in}}%
\pgfpathlineto{\pgfqpoint{3.261433in}{3.346021in}}%
\pgfpathlineto{\pgfqpoint{3.310768in}{3.347823in}}%
\pgfpathlineto{\pgfqpoint{3.359572in}{3.342729in}}%
\pgfpathlineto{\pgfqpoint{3.528796in}{3.273044in}}%
\pgfpathlineto{\pgfqpoint{3.551606in}{3.257685in}}%
\pgfpathlineto{\pgfqpoint{3.635953in}{3.189348in}}%
\pgfpathlineto{\pgfqpoint{3.646563in}{3.179512in}}%
\pgfpathlineto{\pgfqpoint{3.671495in}{3.155356in}}%
\pgfpathlineto{\pgfqpoint{3.761677in}{3.056395in}}%
\pgfpathlineto{\pgfqpoint{3.787671in}{3.024724in}}%
\pgfpathlineto{\pgfqpoint{3.815786in}{2.989006in}}%
\pgfpathlineto{\pgfqpoint{3.855572in}{2.936017in}}%
\pgfpathlineto{\pgfqpoint{4.001985in}{2.719992in}}%
\pgfpathlineto{\pgfqpoint{4.117099in}{2.532854in}}%
\pgfpathlineto{\pgfqpoint{4.146806in}{2.482921in}}%
\pgfpathlineto{\pgfqpoint{4.161129in}{2.458671in}}%
\pgfpathlineto{\pgfqpoint{4.161660in}{2.457771in}}%
\pgfpathlineto{\pgfqpoint{4.182879in}{2.421665in}}%
\pgfpathlineto{\pgfqpoint{4.844920in}{1.362225in}}%
\pgfpathlineto{\pgfqpoint{4.868261in}{1.331880in}}%
\pgfpathlineto{\pgfqpoint{4.875688in}{1.322365in}}%
\pgfpathlineto{\pgfqpoint{4.916535in}{1.271267in}}%
\pgfpathlineto{\pgfqpoint{5.012552in}{1.159482in}}%
\pgfpathlineto{\pgfqpoint{5.030588in}{1.139799in}}%
\pgfpathlineto{\pgfqpoint{5.156312in}{1.014130in}}%
\pgfpathlineto{\pgfqpoint{5.285750in}{0.905257in}}%
\pgfpathlineto{\pgfqpoint{5.362139in}{0.850235in}}%
\pgfusepath{stroke}%
\end{pgfscope}%
\begin{pgfscope}%
\pgfpathrectangle{\pgfqpoint{0.800000in}{0.528000in}}{\pgfqpoint{4.960000in}{3.696000in}}%
\pgfusepath{clip}%
\pgfsetrectcap%
\pgfsetroundjoin%
\pgfsetlinewidth{1.505625pt}%
\definecolor{currentstroke}{rgb}{0.000000,0.500000,0.000000}%
\pgfsetstrokecolor{currentstroke}%
\pgfsetdash{}{0pt}%
\pgfpathmoveto{\pgfqpoint{1.230751in}{1.042999in}}%
\pgfpathlineto{\pgfqpoint{1.746909in}{1.629784in}}%
\pgfpathlineto{\pgfqpoint{1.923559in}{1.881711in}}%
\pgfpathlineto{\pgfqpoint{1.963345in}{1.940559in}}%
\pgfpathlineto{\pgfqpoint{1.996766in}{1.990374in}}%
\pgfpathlineto{\pgfqpoint{2.067320in}{2.096264in}}%
\pgfpathlineto{\pgfqpoint{2.095966in}{2.139379in}}%
\pgfpathlineto{\pgfqpoint{2.146892in}{2.215936in}}%
\pgfpathlineto{\pgfqpoint{2.169172in}{2.249316in}}%
\pgfpathlineto{\pgfqpoint{2.197818in}{2.292061in}}%
\pgfpathlineto{\pgfqpoint{2.255641in}{2.377492in}}%
\pgfpathlineto{\pgfqpoint{2.261476in}{2.386033in}}%
\pgfpathlineto{\pgfqpoint{2.406297in}{2.591039in}}%
\pgfpathlineto{\pgfqpoint{2.565972in}{2.793916in}}%
\pgfpathlineto{\pgfqpoint{2.609472in}{2.843456in}}%
\pgfpathlineto{\pgfqpoint{2.691696in}{2.929026in}}%
\pgfpathlineto{\pgfqpoint{2.697532in}{2.934668in}}%
\pgfpathlineto{\pgfqpoint{2.728830in}{2.963899in}}%
\pgfpathlineto{\pgfqpoint{2.730952in}{2.965817in}}%
\pgfpathlineto{\pgfqpoint{2.748458in}{2.981322in}}%
\pgfpathlineto{\pgfqpoint{2.838109in}{3.051495in}}%
\pgfpathlineto{\pgfqpoint{2.839170in}{3.052230in}}%
\pgfpathlineto{\pgfqpoint{2.863572in}{3.068480in}}%
\pgfpathlineto{\pgfqpoint{2.903889in}{3.092583in}}%
\pgfpathlineto{\pgfqpoint{2.961711in}{3.121008in}}%
\pgfpathlineto{\pgfqpoint{3.043405in}{3.148406in}}%
\pgfpathlineto{\pgfqpoint{3.072582in}{3.154484in}}%
\pgfpathlineto{\pgfqpoint{3.081600in}{3.155963in}}%
\pgfpathlineto{\pgfqpoint{3.093271in}{3.157597in}}%
\pgfpathlineto{\pgfqpoint{3.105472in}{3.158966in}}%
\pgfpathlineto{\pgfqpoint{3.148971in}{3.161016in}}%
\pgfpathlineto{\pgfqpoint{3.194593in}{3.158413in}}%
\pgfpathlineto{\pgfqpoint{3.236501in}{3.151746in}}%
\pgfpathlineto{\pgfqpoint{3.261433in}{3.145849in}}%
\pgfpathlineto{\pgfqpoint{3.310768in}{3.129981in}}%
\pgfpathlineto{\pgfqpoint{3.359572in}{3.108882in}}%
\pgfpathlineto{\pgfqpoint{3.528796in}{2.996254in}}%
\pgfpathlineto{\pgfqpoint{3.551606in}{2.976684in}}%
\pgfpathlineto{\pgfqpoint{3.635953in}{2.896131in}}%
\pgfpathlineto{\pgfqpoint{3.646563in}{2.885133in}}%
\pgfpathlineto{\pgfqpoint{3.671495in}{2.858571in}}%
\pgfpathlineto{\pgfqpoint{3.761677in}{2.754678in}}%
\pgfpathlineto{\pgfqpoint{3.787671in}{2.722655in}}%
\pgfpathlineto{\pgfqpoint{3.815786in}{2.687078in}}%
\pgfpathlineto{\pgfqpoint{3.855572in}{2.635185in}}%
\pgfpathlineto{\pgfqpoint{4.001985in}{2.431541in}}%
\pgfpathlineto{\pgfqpoint{4.117099in}{2.262004in}}%
\pgfpathlineto{\pgfqpoint{4.146806in}{2.217528in}}%
\pgfpathlineto{\pgfqpoint{4.161129in}{2.196026in}}%
\pgfpathlineto{\pgfqpoint{4.161660in}{2.195229in}}%
\pgfpathlineto{\pgfqpoint{4.182879in}{2.163328in}}%
\pgfpathlineto{\pgfqpoint{4.844920in}{1.260078in}}%
\pgfpathlineto{\pgfqpoint{4.868261in}{1.234586in}}%
\pgfpathlineto{\pgfqpoint{4.875688in}{1.226591in}}%
\pgfpathlineto{\pgfqpoint{4.916535in}{1.183638in}}%
\pgfpathlineto{\pgfqpoint{5.012552in}{1.089496in}}%
\pgfpathlineto{\pgfqpoint{5.030588in}{1.072885in}}%
\pgfpathlineto{\pgfqpoint{5.156312in}{0.966437in}}%
\pgfpathlineto{\pgfqpoint{5.285750in}{0.873422in}}%
\pgfpathlineto{\pgfqpoint{5.362139in}{0.825990in}}%
\pgfusepath{stroke}%
\end{pgfscope}%
\begin{pgfscope}%
\pgfsetrectcap%
\pgfsetmiterjoin%
\pgfsetlinewidth{0.803000pt}%
\definecolor{currentstroke}{rgb}{0.000000,0.000000,0.000000}%
\pgfsetstrokecolor{currentstroke}%
\pgfsetdash{}{0pt}%
\pgfpathmoveto{\pgfqpoint{0.800000in}{0.528000in}}%
\pgfpathlineto{\pgfqpoint{0.800000in}{4.224000in}}%
\pgfusepath{stroke}%
\end{pgfscope}%
\begin{pgfscope}%
\pgfsetrectcap%
\pgfsetmiterjoin%
\pgfsetlinewidth{0.803000pt}%
\definecolor{currentstroke}{rgb}{0.000000,0.000000,0.000000}%
\pgfsetstrokecolor{currentstroke}%
\pgfsetdash{}{0pt}%
\pgfpathmoveto{\pgfqpoint{5.760000in}{0.528000in}}%
\pgfpathlineto{\pgfqpoint{5.760000in}{4.224000in}}%
\pgfusepath{stroke}%
\end{pgfscope}%
\begin{pgfscope}%
\pgfsetrectcap%
\pgfsetmiterjoin%
\pgfsetlinewidth{0.803000pt}%
\definecolor{currentstroke}{rgb}{0.000000,0.000000,0.000000}%
\pgfsetstrokecolor{currentstroke}%
\pgfsetdash{}{0pt}%
\pgfpathmoveto{\pgfqpoint{0.800000in}{0.528000in}}%
\pgfpathlineto{\pgfqpoint{5.760000in}{0.528000in}}%
\pgfusepath{stroke}%
\end{pgfscope}%
\begin{pgfscope}%
\pgfsetrectcap%
\pgfsetmiterjoin%
\pgfsetlinewidth{0.803000pt}%
\definecolor{currentstroke}{rgb}{0.000000,0.000000,0.000000}%
\pgfsetstrokecolor{currentstroke}%
\pgfsetdash{}{0pt}%
\pgfpathmoveto{\pgfqpoint{0.800000in}{4.224000in}}%
\pgfpathlineto{\pgfqpoint{5.760000in}{4.224000in}}%
\pgfusepath{stroke}%
\end{pgfscope}%
\begin{pgfscope}%
\pgfsetbuttcap%
\pgfsetmiterjoin%
\definecolor{currentfill}{rgb}{1.000000,0.647059,0.000000}%
\pgfsetfillcolor{currentfill}%
\pgfsetlinewidth{1.003750pt}%
\definecolor{currentstroke}{rgb}{0.000000,0.000000,0.000000}%
\pgfsetstrokecolor{currentstroke}%
\pgfsetdash{}{0pt}%
\pgfpathmoveto{\pgfqpoint{3.858335in}{3.282323in}}%
\pgfpathlineto{\pgfqpoint{5.083467in}{3.282323in}}%
\pgfpathquadraticcurveto{\pgfqpoint{5.125134in}{3.282323in}}{\pgfqpoint{5.125134in}{3.323989in}}%
\pgfpathlineto{\pgfqpoint{5.125134in}{3.447446in}}%
\pgfpathquadraticcurveto{\pgfqpoint{5.125134in}{3.489113in}}{\pgfqpoint{5.083467in}{3.489113in}}%
\pgfpathlineto{\pgfqpoint{3.858335in}{3.489113in}}%
\pgfpathquadraticcurveto{\pgfqpoint{3.816668in}{3.489113in}}{\pgfqpoint{3.816668in}{3.447446in}}%
\pgfpathlineto{\pgfqpoint{3.816668in}{3.323989in}}%
\pgfpathquadraticcurveto{\pgfqpoint{3.816668in}{3.282323in}}{\pgfqpoint{3.858335in}{3.282323in}}%
\pgfpathclose%
\pgfusepath{stroke,fill}%
\end{pgfscope}%
\begin{pgfscope}%
\definecolor{textcolor}{rgb}{0.000000,0.000000,0.000000}%
\pgfsetstrokecolor{textcolor}%
\pgfsetfillcolor{textcolor}%
\pgftext[x=3.858335in,y=3.350995in,left,base]{\color{textcolor}\rmfamily\fontsize{10.000000}{12.000000}\selectfont \(\displaystyle \mu=\)40.285\(\displaystyle ; \sigma=\)1.867}%
\end{pgfscope}%
\begin{pgfscope}%
\pgfsetbuttcap%
\pgfsetmiterjoin%
\definecolor{currentfill}{rgb}{1.000000,1.000000,1.000000}%
\pgfsetfillcolor{currentfill}%
\pgfsetfillopacity{0.800000}%
\pgfsetlinewidth{1.003750pt}%
\definecolor{currentstroke}{rgb}{0.800000,0.800000,0.800000}%
\pgfsetstrokecolor{currentstroke}%
\pgfsetstrokeopacity{0.800000}%
\pgfsetdash{}{0pt}%
\pgfpathmoveto{\pgfqpoint{4.056487in}{3.531871in}}%
\pgfpathlineto{\pgfqpoint{5.662778in}{3.531871in}}%
\pgfpathquadraticcurveto{\pgfqpoint{5.690556in}{3.531871in}}{\pgfqpoint{5.690556in}{3.559648in}}%
\pgfpathlineto{\pgfqpoint{5.690556in}{4.126778in}}%
\pgfpathquadraticcurveto{\pgfqpoint{5.690556in}{4.154556in}}{\pgfqpoint{5.662778in}{4.154556in}}%
\pgfpathlineto{\pgfqpoint{4.056487in}{4.154556in}}%
\pgfpathquadraticcurveto{\pgfqpoint{4.028709in}{4.154556in}}{\pgfqpoint{4.028709in}{4.126778in}}%
\pgfpathlineto{\pgfqpoint{4.028709in}{3.559648in}}%
\pgfpathquadraticcurveto{\pgfqpoint{4.028709in}{3.531871in}}{\pgfqpoint{4.056487in}{3.531871in}}%
\pgfpathclose%
\pgfusepath{stroke,fill}%
\end{pgfscope}%
\begin{pgfscope}%
\pgfsetrectcap%
\pgfsetroundjoin%
\pgfsetlinewidth{1.505625pt}%
\definecolor{currentstroke}{rgb}{1.000000,0.000000,0.000000}%
\pgfsetstrokecolor{currentstroke}%
\pgfsetdash{}{0pt}%
\pgfpathmoveto{\pgfqpoint{4.084265in}{4.050389in}}%
\pgfpathlineto{\pgfqpoint{4.362042in}{4.050389in}}%
\pgfusepath{stroke}%
\end{pgfscope}%
\begin{pgfscope}%
\definecolor{textcolor}{rgb}{0.000000,0.000000,0.000000}%
\pgfsetstrokecolor{textcolor}%
\pgfsetfillcolor{textcolor}%
\pgftext[x=4.473154in,y=4.001778in,left,base]{\color{textcolor}\rmfamily\fontsize{10.000000}{12.000000}\selectfont Actual Gaussian}%
\end{pgfscope}%
\begin{pgfscope}%
\pgfsetrectcap%
\pgfsetroundjoin%
\pgfsetlinewidth{1.505625pt}%
\definecolor{currentstroke}{rgb}{0.000000,0.500000,0.000000}%
\pgfsetstrokecolor{currentstroke}%
\pgfsetdash{}{0pt}%
\pgfpathmoveto{\pgfqpoint{4.084265in}{3.856716in}}%
\pgfpathlineto{\pgfqpoint{4.362042in}{3.856716in}}%
\pgfusepath{stroke}%
\end{pgfscope}%
\begin{pgfscope}%
\definecolor{textcolor}{rgb}{0.000000,0.000000,0.000000}%
\pgfsetstrokecolor{textcolor}%
\pgfsetfillcolor{textcolor}%
\pgftext[x=4.473154in,y=3.808105in,left,base]{\color{textcolor}\rmfamily\fontsize{10.000000}{12.000000}\selectfont Expected Gaussian}%
\end{pgfscope}%
\begin{pgfscope}%
\pgfsetbuttcap%
\pgfsetmiterjoin%
\definecolor{currentfill}{rgb}{0.121569,0.466667,0.705882}%
\pgfsetfillcolor{currentfill}%
\pgfsetlinewidth{0.000000pt}%
\definecolor{currentstroke}{rgb}{0.000000,0.000000,0.000000}%
\pgfsetstrokecolor{currentstroke}%
\pgfsetstrokeopacity{0.000000}%
\pgfsetdash{}{0pt}%
\pgfpathmoveto{\pgfqpoint{4.084265in}{3.614432in}}%
\pgfpathlineto{\pgfqpoint{4.362042in}{3.614432in}}%
\pgfpathlineto{\pgfqpoint{4.362042in}{3.711654in}}%
\pgfpathlineto{\pgfqpoint{4.084265in}{3.711654in}}%
\pgfpathclose%
\pgfusepath{fill}%
\end{pgfscope}%
\begin{pgfscope}%
\definecolor{textcolor}{rgb}{0.000000,0.000000,0.000000}%
\pgfsetstrokecolor{textcolor}%
\pgfsetfillcolor{textcolor}%
\pgftext[x=4.473154in,y=3.614432in,left,base]{\color{textcolor}\rmfamily\fontsize{10.000000}{12.000000}\selectfont Randomised Data}%
\end{pgfscope}%
\end{pgfpicture}%
\makeatother%
\endgroup%
}}%
            \qquad
            \subfloat{\scalebox{0.45}{%% Creator: Matplotlib, PGF backend
%%
%% To include the figure in your LaTeX document, write
%%   \input{<filename>.pgf}
%%
%% Make sure the required packages are loaded in your preamble
%%   \usepackage{pgf}
%%
%% Figures using additional raster images can only be included by \input if
%% they are in the same directory as the main LaTeX file. For loading figures
%% from other directories you can use the `import` package
%%   \usepackage{import}
%% and then include the figures with
%%   \import{<path to file>}{<filename>.pgf}
%%
%% Matplotlib used the following preamble
%%
\begingroup%
\makeatletter%
\begin{pgfpicture}%
\pgfpathrectangle{\pgfpointorigin}{\pgfqpoint{6.400000in}{4.800000in}}%
\pgfusepath{use as bounding box, clip}%
\begin{pgfscope}%
\pgfsetbuttcap%
\pgfsetmiterjoin%
\definecolor{currentfill}{rgb}{1.000000,1.000000,1.000000}%
\pgfsetfillcolor{currentfill}%
\pgfsetlinewidth{0.000000pt}%
\definecolor{currentstroke}{rgb}{1.000000,1.000000,1.000000}%
\pgfsetstrokecolor{currentstroke}%
\pgfsetdash{}{0pt}%
\pgfpathmoveto{\pgfqpoint{0.000000in}{0.000000in}}%
\pgfpathlineto{\pgfqpoint{6.400000in}{0.000000in}}%
\pgfpathlineto{\pgfqpoint{6.400000in}{4.800000in}}%
\pgfpathlineto{\pgfqpoint{0.000000in}{4.800000in}}%
\pgfpathclose%
\pgfusepath{fill}%
\end{pgfscope}%
\begin{pgfscope}%
\pgfsetbuttcap%
\pgfsetmiterjoin%
\definecolor{currentfill}{rgb}{1.000000,1.000000,1.000000}%
\pgfsetfillcolor{currentfill}%
\pgfsetlinewidth{0.000000pt}%
\definecolor{currentstroke}{rgb}{0.000000,0.000000,0.000000}%
\pgfsetstrokecolor{currentstroke}%
\pgfsetstrokeopacity{0.000000}%
\pgfsetdash{}{0pt}%
\pgfpathmoveto{\pgfqpoint{0.800000in}{0.528000in}}%
\pgfpathlineto{\pgfqpoint{5.760000in}{0.528000in}}%
\pgfpathlineto{\pgfqpoint{5.760000in}{4.224000in}}%
\pgfpathlineto{\pgfqpoint{0.800000in}{4.224000in}}%
\pgfpathclose%
\pgfusepath{fill}%
\end{pgfscope}%
\begin{pgfscope}%
\pgfpathrectangle{\pgfqpoint{0.800000in}{0.528000in}}{\pgfqpoint{4.960000in}{3.696000in}}%
\pgfusepath{clip}%
\pgfsetbuttcap%
\pgfsetmiterjoin%
\definecolor{currentfill}{rgb}{0.121569,0.466667,0.705882}%
\pgfsetfillcolor{currentfill}%
\pgfsetlinewidth{0.000000pt}%
\definecolor{currentstroke}{rgb}{0.000000,0.000000,0.000000}%
\pgfsetstrokecolor{currentstroke}%
\pgfsetstrokeopacity{0.000000}%
\pgfsetdash{}{0pt}%
\pgfpathmoveto{\pgfqpoint{1.025455in}{0.528000in}}%
\pgfpathlineto{\pgfqpoint{1.282061in}{0.528000in}}%
\pgfpathlineto{\pgfqpoint{1.282061in}{0.968000in}}%
\pgfpathlineto{\pgfqpoint{1.025455in}{0.968000in}}%
\pgfpathclose%
\pgfusepath{fill}%
\end{pgfscope}%
\begin{pgfscope}%
\pgfpathrectangle{\pgfqpoint{0.800000in}{0.528000in}}{\pgfqpoint{4.960000in}{3.696000in}}%
\pgfusepath{clip}%
\pgfsetbuttcap%
\pgfsetmiterjoin%
\definecolor{currentfill}{rgb}{0.121569,0.466667,0.705882}%
\pgfsetfillcolor{currentfill}%
\pgfsetlinewidth{0.000000pt}%
\definecolor{currentstroke}{rgb}{0.000000,0.000000,0.000000}%
\pgfsetstrokecolor{currentstroke}%
\pgfsetstrokeopacity{0.000000}%
\pgfsetdash{}{0pt}%
\pgfpathmoveto{\pgfqpoint{1.282061in}{0.528000in}}%
\pgfpathlineto{\pgfqpoint{1.538668in}{0.528000in}}%
\pgfpathlineto{\pgfqpoint{1.538668in}{0.528000in}}%
\pgfpathlineto{\pgfqpoint{1.282061in}{0.528000in}}%
\pgfpathclose%
\pgfusepath{fill}%
\end{pgfscope}%
\begin{pgfscope}%
\pgfpathrectangle{\pgfqpoint{0.800000in}{0.528000in}}{\pgfqpoint{4.960000in}{3.696000in}}%
\pgfusepath{clip}%
\pgfsetbuttcap%
\pgfsetmiterjoin%
\definecolor{currentfill}{rgb}{0.121569,0.466667,0.705882}%
\pgfsetfillcolor{currentfill}%
\pgfsetlinewidth{0.000000pt}%
\definecolor{currentstroke}{rgb}{0.000000,0.000000,0.000000}%
\pgfsetstrokecolor{currentstroke}%
\pgfsetstrokeopacity{0.000000}%
\pgfsetdash{}{0pt}%
\pgfpathmoveto{\pgfqpoint{1.538668in}{0.528000in}}%
\pgfpathlineto{\pgfqpoint{1.795274in}{0.528000in}}%
\pgfpathlineto{\pgfqpoint{1.795274in}{0.528000in}}%
\pgfpathlineto{\pgfqpoint{1.538668in}{0.528000in}}%
\pgfpathclose%
\pgfusepath{fill}%
\end{pgfscope}%
\begin{pgfscope}%
\pgfpathrectangle{\pgfqpoint{0.800000in}{0.528000in}}{\pgfqpoint{4.960000in}{3.696000in}}%
\pgfusepath{clip}%
\pgfsetbuttcap%
\pgfsetmiterjoin%
\definecolor{currentfill}{rgb}{0.121569,0.466667,0.705882}%
\pgfsetfillcolor{currentfill}%
\pgfsetlinewidth{0.000000pt}%
\definecolor{currentstroke}{rgb}{0.000000,0.000000,0.000000}%
\pgfsetstrokecolor{currentstroke}%
\pgfsetstrokeopacity{0.000000}%
\pgfsetdash{}{0pt}%
\pgfpathmoveto{\pgfqpoint{1.795274in}{0.528000in}}%
\pgfpathlineto{\pgfqpoint{2.051881in}{0.528000in}}%
\pgfpathlineto{\pgfqpoint{2.051881in}{2.288000in}}%
\pgfpathlineto{\pgfqpoint{1.795274in}{2.288000in}}%
\pgfpathclose%
\pgfusepath{fill}%
\end{pgfscope}%
\begin{pgfscope}%
\pgfpathrectangle{\pgfqpoint{0.800000in}{0.528000in}}{\pgfqpoint{4.960000in}{3.696000in}}%
\pgfusepath{clip}%
\pgfsetbuttcap%
\pgfsetmiterjoin%
\definecolor{currentfill}{rgb}{0.121569,0.466667,0.705882}%
\pgfsetfillcolor{currentfill}%
\pgfsetlinewidth{0.000000pt}%
\definecolor{currentstroke}{rgb}{0.000000,0.000000,0.000000}%
\pgfsetstrokecolor{currentstroke}%
\pgfsetstrokeopacity{0.000000}%
\pgfsetdash{}{0pt}%
\pgfpathmoveto{\pgfqpoint{2.051881in}{0.528000in}}%
\pgfpathlineto{\pgfqpoint{2.308487in}{0.528000in}}%
\pgfpathlineto{\pgfqpoint{2.308487in}{2.728000in}}%
\pgfpathlineto{\pgfqpoint{2.051881in}{2.728000in}}%
\pgfpathclose%
\pgfusepath{fill}%
\end{pgfscope}%
\begin{pgfscope}%
\pgfpathrectangle{\pgfqpoint{0.800000in}{0.528000in}}{\pgfqpoint{4.960000in}{3.696000in}}%
\pgfusepath{clip}%
\pgfsetbuttcap%
\pgfsetmiterjoin%
\definecolor{currentfill}{rgb}{0.121569,0.466667,0.705882}%
\pgfsetfillcolor{currentfill}%
\pgfsetlinewidth{0.000000pt}%
\definecolor{currentstroke}{rgb}{0.000000,0.000000,0.000000}%
\pgfsetstrokecolor{currentstroke}%
\pgfsetstrokeopacity{0.000000}%
\pgfsetdash{}{0pt}%
\pgfpathmoveto{\pgfqpoint{2.308487in}{0.528000in}}%
\pgfpathlineto{\pgfqpoint{2.565094in}{0.528000in}}%
\pgfpathlineto{\pgfqpoint{2.565094in}{0.968000in}}%
\pgfpathlineto{\pgfqpoint{2.308487in}{0.968000in}}%
\pgfpathclose%
\pgfusepath{fill}%
\end{pgfscope}%
\begin{pgfscope}%
\pgfpathrectangle{\pgfqpoint{0.800000in}{0.528000in}}{\pgfqpoint{4.960000in}{3.696000in}}%
\pgfusepath{clip}%
\pgfsetbuttcap%
\pgfsetmiterjoin%
\definecolor{currentfill}{rgb}{0.121569,0.466667,0.705882}%
\pgfsetfillcolor{currentfill}%
\pgfsetlinewidth{0.000000pt}%
\definecolor{currentstroke}{rgb}{0.000000,0.000000,0.000000}%
\pgfsetstrokecolor{currentstroke}%
\pgfsetstrokeopacity{0.000000}%
\pgfsetdash{}{0pt}%
\pgfpathmoveto{\pgfqpoint{2.565094in}{0.528000in}}%
\pgfpathlineto{\pgfqpoint{2.821701in}{0.528000in}}%
\pgfpathlineto{\pgfqpoint{2.821701in}{1.408000in}}%
\pgfpathlineto{\pgfqpoint{2.565094in}{1.408000in}}%
\pgfpathclose%
\pgfusepath{fill}%
\end{pgfscope}%
\begin{pgfscope}%
\pgfpathrectangle{\pgfqpoint{0.800000in}{0.528000in}}{\pgfqpoint{4.960000in}{3.696000in}}%
\pgfusepath{clip}%
\pgfsetbuttcap%
\pgfsetmiterjoin%
\definecolor{currentfill}{rgb}{0.121569,0.466667,0.705882}%
\pgfsetfillcolor{currentfill}%
\pgfsetlinewidth{0.000000pt}%
\definecolor{currentstroke}{rgb}{0.000000,0.000000,0.000000}%
\pgfsetstrokecolor{currentstroke}%
\pgfsetstrokeopacity{0.000000}%
\pgfsetdash{}{0pt}%
\pgfpathmoveto{\pgfqpoint{2.821701in}{0.528000in}}%
\pgfpathlineto{\pgfqpoint{3.078307in}{0.528000in}}%
\pgfpathlineto{\pgfqpoint{3.078307in}{3.168000in}}%
\pgfpathlineto{\pgfqpoint{2.821701in}{3.168000in}}%
\pgfpathclose%
\pgfusepath{fill}%
\end{pgfscope}%
\begin{pgfscope}%
\pgfpathrectangle{\pgfqpoint{0.800000in}{0.528000in}}{\pgfqpoint{4.960000in}{3.696000in}}%
\pgfusepath{clip}%
\pgfsetbuttcap%
\pgfsetmiterjoin%
\definecolor{currentfill}{rgb}{0.121569,0.466667,0.705882}%
\pgfsetfillcolor{currentfill}%
\pgfsetlinewidth{0.000000pt}%
\definecolor{currentstroke}{rgb}{0.000000,0.000000,0.000000}%
\pgfsetstrokecolor{currentstroke}%
\pgfsetstrokeopacity{0.000000}%
\pgfsetdash{}{0pt}%
\pgfpathmoveto{\pgfqpoint{3.078307in}{0.528000in}}%
\pgfpathlineto{\pgfqpoint{3.334914in}{0.528000in}}%
\pgfpathlineto{\pgfqpoint{3.334914in}{2.288000in}}%
\pgfpathlineto{\pgfqpoint{3.078307in}{2.288000in}}%
\pgfpathclose%
\pgfusepath{fill}%
\end{pgfscope}%
\begin{pgfscope}%
\pgfpathrectangle{\pgfqpoint{0.800000in}{0.528000in}}{\pgfqpoint{4.960000in}{3.696000in}}%
\pgfusepath{clip}%
\pgfsetbuttcap%
\pgfsetmiterjoin%
\definecolor{currentfill}{rgb}{0.121569,0.466667,0.705882}%
\pgfsetfillcolor{currentfill}%
\pgfsetlinewidth{0.000000pt}%
\definecolor{currentstroke}{rgb}{0.000000,0.000000,0.000000}%
\pgfsetstrokecolor{currentstroke}%
\pgfsetstrokeopacity{0.000000}%
\pgfsetdash{}{0pt}%
\pgfpathmoveto{\pgfqpoint{3.334914in}{0.528000in}}%
\pgfpathlineto{\pgfqpoint{3.591520in}{0.528000in}}%
\pgfpathlineto{\pgfqpoint{3.591520in}{1.848000in}}%
\pgfpathlineto{\pgfqpoint{3.334914in}{1.848000in}}%
\pgfpathclose%
\pgfusepath{fill}%
\end{pgfscope}%
\begin{pgfscope}%
\pgfpathrectangle{\pgfqpoint{0.800000in}{0.528000in}}{\pgfqpoint{4.960000in}{3.696000in}}%
\pgfusepath{clip}%
\pgfsetbuttcap%
\pgfsetmiterjoin%
\definecolor{currentfill}{rgb}{0.121569,0.466667,0.705882}%
\pgfsetfillcolor{currentfill}%
\pgfsetlinewidth{0.000000pt}%
\definecolor{currentstroke}{rgb}{0.000000,0.000000,0.000000}%
\pgfsetstrokecolor{currentstroke}%
\pgfsetstrokeopacity{0.000000}%
\pgfsetdash{}{0pt}%
\pgfpathmoveto{\pgfqpoint{3.591520in}{0.528000in}}%
\pgfpathlineto{\pgfqpoint{3.848127in}{0.528000in}}%
\pgfpathlineto{\pgfqpoint{3.848127in}{4.048000in}}%
\pgfpathlineto{\pgfqpoint{3.591520in}{4.048000in}}%
\pgfpathclose%
\pgfusepath{fill}%
\end{pgfscope}%
\begin{pgfscope}%
\pgfpathrectangle{\pgfqpoint{0.800000in}{0.528000in}}{\pgfqpoint{4.960000in}{3.696000in}}%
\pgfusepath{clip}%
\pgfsetbuttcap%
\pgfsetmiterjoin%
\definecolor{currentfill}{rgb}{0.121569,0.466667,0.705882}%
\pgfsetfillcolor{currentfill}%
\pgfsetlinewidth{0.000000pt}%
\definecolor{currentstroke}{rgb}{0.000000,0.000000,0.000000}%
\pgfsetstrokecolor{currentstroke}%
\pgfsetstrokeopacity{0.000000}%
\pgfsetdash{}{0pt}%
\pgfpathmoveto{\pgfqpoint{3.848127in}{0.528000in}}%
\pgfpathlineto{\pgfqpoint{4.104734in}{0.528000in}}%
\pgfpathlineto{\pgfqpoint{4.104734in}{3.168000in}}%
\pgfpathlineto{\pgfqpoint{3.848127in}{3.168000in}}%
\pgfpathclose%
\pgfusepath{fill}%
\end{pgfscope}%
\begin{pgfscope}%
\pgfpathrectangle{\pgfqpoint{0.800000in}{0.528000in}}{\pgfqpoint{4.960000in}{3.696000in}}%
\pgfusepath{clip}%
\pgfsetbuttcap%
\pgfsetmiterjoin%
\definecolor{currentfill}{rgb}{0.121569,0.466667,0.705882}%
\pgfsetfillcolor{currentfill}%
\pgfsetlinewidth{0.000000pt}%
\definecolor{currentstroke}{rgb}{0.000000,0.000000,0.000000}%
\pgfsetstrokecolor{currentstroke}%
\pgfsetstrokeopacity{0.000000}%
\pgfsetdash{}{0pt}%
\pgfpathmoveto{\pgfqpoint{4.104734in}{0.528000in}}%
\pgfpathlineto{\pgfqpoint{4.361340in}{0.528000in}}%
\pgfpathlineto{\pgfqpoint{4.361340in}{1.408000in}}%
\pgfpathlineto{\pgfqpoint{4.104734in}{1.408000in}}%
\pgfpathclose%
\pgfusepath{fill}%
\end{pgfscope}%
\begin{pgfscope}%
\pgfpathrectangle{\pgfqpoint{0.800000in}{0.528000in}}{\pgfqpoint{4.960000in}{3.696000in}}%
\pgfusepath{clip}%
\pgfsetbuttcap%
\pgfsetmiterjoin%
\definecolor{currentfill}{rgb}{0.121569,0.466667,0.705882}%
\pgfsetfillcolor{currentfill}%
\pgfsetlinewidth{0.000000pt}%
\definecolor{currentstroke}{rgb}{0.000000,0.000000,0.000000}%
\pgfsetstrokecolor{currentstroke}%
\pgfsetstrokeopacity{0.000000}%
\pgfsetdash{}{0pt}%
\pgfpathmoveto{\pgfqpoint{4.361340in}{0.528000in}}%
\pgfpathlineto{\pgfqpoint{4.617947in}{0.528000in}}%
\pgfpathlineto{\pgfqpoint{4.617947in}{3.168000in}}%
\pgfpathlineto{\pgfqpoint{4.361340in}{3.168000in}}%
\pgfpathclose%
\pgfusepath{fill}%
\end{pgfscope}%
\begin{pgfscope}%
\pgfpathrectangle{\pgfqpoint{0.800000in}{0.528000in}}{\pgfqpoint{4.960000in}{3.696000in}}%
\pgfusepath{clip}%
\pgfsetbuttcap%
\pgfsetmiterjoin%
\definecolor{currentfill}{rgb}{0.121569,0.466667,0.705882}%
\pgfsetfillcolor{currentfill}%
\pgfsetlinewidth{0.000000pt}%
\definecolor{currentstroke}{rgb}{0.000000,0.000000,0.000000}%
\pgfsetstrokecolor{currentstroke}%
\pgfsetstrokeopacity{0.000000}%
\pgfsetdash{}{0pt}%
\pgfpathmoveto{\pgfqpoint{4.617947in}{0.528000in}}%
\pgfpathlineto{\pgfqpoint{4.874553in}{0.528000in}}%
\pgfpathlineto{\pgfqpoint{4.874553in}{1.848000in}}%
\pgfpathlineto{\pgfqpoint{4.617947in}{1.848000in}}%
\pgfpathclose%
\pgfusepath{fill}%
\end{pgfscope}%
\begin{pgfscope}%
\pgfpathrectangle{\pgfqpoint{0.800000in}{0.528000in}}{\pgfqpoint{4.960000in}{3.696000in}}%
\pgfusepath{clip}%
\pgfsetbuttcap%
\pgfsetmiterjoin%
\definecolor{currentfill}{rgb}{0.121569,0.466667,0.705882}%
\pgfsetfillcolor{currentfill}%
\pgfsetlinewidth{0.000000pt}%
\definecolor{currentstroke}{rgb}{0.000000,0.000000,0.000000}%
\pgfsetstrokecolor{currentstroke}%
\pgfsetstrokeopacity{0.000000}%
\pgfsetdash{}{0pt}%
\pgfpathmoveto{\pgfqpoint{4.874553in}{0.528000in}}%
\pgfpathlineto{\pgfqpoint{5.131160in}{0.528000in}}%
\pgfpathlineto{\pgfqpoint{5.131160in}{2.728000in}}%
\pgfpathlineto{\pgfqpoint{4.874553in}{2.728000in}}%
\pgfpathclose%
\pgfusepath{fill}%
\end{pgfscope}%
\begin{pgfscope}%
\pgfpathrectangle{\pgfqpoint{0.800000in}{0.528000in}}{\pgfqpoint{4.960000in}{3.696000in}}%
\pgfusepath{clip}%
\pgfsetbuttcap%
\pgfsetmiterjoin%
\definecolor{currentfill}{rgb}{0.121569,0.466667,0.705882}%
\pgfsetfillcolor{currentfill}%
\pgfsetlinewidth{0.000000pt}%
\definecolor{currentstroke}{rgb}{0.000000,0.000000,0.000000}%
\pgfsetstrokecolor{currentstroke}%
\pgfsetstrokeopacity{0.000000}%
\pgfsetdash{}{0pt}%
\pgfpathmoveto{\pgfqpoint{5.131160in}{0.528000in}}%
\pgfpathlineto{\pgfqpoint{5.387766in}{0.528000in}}%
\pgfpathlineto{\pgfqpoint{5.387766in}{0.968000in}}%
\pgfpathlineto{\pgfqpoint{5.131160in}{0.968000in}}%
\pgfpathclose%
\pgfusepath{fill}%
\end{pgfscope}%
\begin{pgfscope}%
\pgfsetbuttcap%
\pgfsetroundjoin%
\definecolor{currentfill}{rgb}{0.000000,0.000000,0.000000}%
\pgfsetfillcolor{currentfill}%
\pgfsetlinewidth{0.803000pt}%
\definecolor{currentstroke}{rgb}{0.000000,0.000000,0.000000}%
\pgfsetstrokecolor{currentstroke}%
\pgfsetdash{}{0pt}%
\pgfsys@defobject{currentmarker}{\pgfqpoint{0.000000in}{-0.048611in}}{\pgfqpoint{0.000000in}{0.000000in}}{%
\pgfpathmoveto{\pgfqpoint{0.000000in}{0.000000in}}%
\pgfpathlineto{\pgfqpoint{0.000000in}{-0.048611in}}%
\pgfusepath{stroke,fill}%
}%
\begin{pgfscope}%
\pgfsys@transformshift{1.538668in}{0.528000in}%
\pgfsys@useobject{currentmarker}{}%
\end{pgfscope}%
\end{pgfscope}%
\begin{pgfscope}%
\definecolor{textcolor}{rgb}{0.000000,0.000000,0.000000}%
\pgfsetstrokecolor{textcolor}%
\pgfsetfillcolor{textcolor}%
\pgftext[x=1.538668in,y=0.430778in,,top]{\color{textcolor}\rmfamily\fontsize{10.000000}{12.000000}\selectfont \(\displaystyle 36\)}%
\end{pgfscope}%
\begin{pgfscope}%
\pgfsetbuttcap%
\pgfsetroundjoin%
\definecolor{currentfill}{rgb}{0.000000,0.000000,0.000000}%
\pgfsetfillcolor{currentfill}%
\pgfsetlinewidth{0.803000pt}%
\definecolor{currentstroke}{rgb}{0.000000,0.000000,0.000000}%
\pgfsetstrokecolor{currentstroke}%
\pgfsetdash{}{0pt}%
\pgfsys@defobject{currentmarker}{\pgfqpoint{0.000000in}{-0.048611in}}{\pgfqpoint{0.000000in}{0.000000in}}{%
\pgfpathmoveto{\pgfqpoint{0.000000in}{0.000000in}}%
\pgfpathlineto{\pgfqpoint{0.000000in}{-0.048611in}}%
\pgfusepath{stroke,fill}%
}%
\begin{pgfscope}%
\pgfsys@transformshift{2.565094in}{0.528000in}%
\pgfsys@useobject{currentmarker}{}%
\end{pgfscope}%
\end{pgfscope}%
\begin{pgfscope}%
\definecolor{textcolor}{rgb}{0.000000,0.000000,0.000000}%
\pgfsetstrokecolor{textcolor}%
\pgfsetfillcolor{textcolor}%
\pgftext[x=2.565094in,y=0.430778in,,top]{\color{textcolor}\rmfamily\fontsize{10.000000}{12.000000}\selectfont \(\displaystyle 38\)}%
\end{pgfscope}%
\begin{pgfscope}%
\pgfsetbuttcap%
\pgfsetroundjoin%
\definecolor{currentfill}{rgb}{0.000000,0.000000,0.000000}%
\pgfsetfillcolor{currentfill}%
\pgfsetlinewidth{0.803000pt}%
\definecolor{currentstroke}{rgb}{0.000000,0.000000,0.000000}%
\pgfsetstrokecolor{currentstroke}%
\pgfsetdash{}{0pt}%
\pgfsys@defobject{currentmarker}{\pgfqpoint{0.000000in}{-0.048611in}}{\pgfqpoint{0.000000in}{0.000000in}}{%
\pgfpathmoveto{\pgfqpoint{0.000000in}{0.000000in}}%
\pgfpathlineto{\pgfqpoint{0.000000in}{-0.048611in}}%
\pgfusepath{stroke,fill}%
}%
\begin{pgfscope}%
\pgfsys@transformshift{3.591520in}{0.528000in}%
\pgfsys@useobject{currentmarker}{}%
\end{pgfscope}%
\end{pgfscope}%
\begin{pgfscope}%
\definecolor{textcolor}{rgb}{0.000000,0.000000,0.000000}%
\pgfsetstrokecolor{textcolor}%
\pgfsetfillcolor{textcolor}%
\pgftext[x=3.591520in,y=0.430778in,,top]{\color{textcolor}\rmfamily\fontsize{10.000000}{12.000000}\selectfont \(\displaystyle 40\)}%
\end{pgfscope}%
\begin{pgfscope}%
\pgfsetbuttcap%
\pgfsetroundjoin%
\definecolor{currentfill}{rgb}{0.000000,0.000000,0.000000}%
\pgfsetfillcolor{currentfill}%
\pgfsetlinewidth{0.803000pt}%
\definecolor{currentstroke}{rgb}{0.000000,0.000000,0.000000}%
\pgfsetstrokecolor{currentstroke}%
\pgfsetdash{}{0pt}%
\pgfsys@defobject{currentmarker}{\pgfqpoint{0.000000in}{-0.048611in}}{\pgfqpoint{0.000000in}{0.000000in}}{%
\pgfpathmoveto{\pgfqpoint{0.000000in}{0.000000in}}%
\pgfpathlineto{\pgfqpoint{0.000000in}{-0.048611in}}%
\pgfusepath{stroke,fill}%
}%
\begin{pgfscope}%
\pgfsys@transformshift{4.617947in}{0.528000in}%
\pgfsys@useobject{currentmarker}{}%
\end{pgfscope}%
\end{pgfscope}%
\begin{pgfscope}%
\definecolor{textcolor}{rgb}{0.000000,0.000000,0.000000}%
\pgfsetstrokecolor{textcolor}%
\pgfsetfillcolor{textcolor}%
\pgftext[x=4.617947in,y=0.430778in,,top]{\color{textcolor}\rmfamily\fontsize{10.000000}{12.000000}\selectfont \(\displaystyle 42\)}%
\end{pgfscope}%
\begin{pgfscope}%
\pgfsetbuttcap%
\pgfsetroundjoin%
\definecolor{currentfill}{rgb}{0.000000,0.000000,0.000000}%
\pgfsetfillcolor{currentfill}%
\pgfsetlinewidth{0.803000pt}%
\definecolor{currentstroke}{rgb}{0.000000,0.000000,0.000000}%
\pgfsetstrokecolor{currentstroke}%
\pgfsetdash{}{0pt}%
\pgfsys@defobject{currentmarker}{\pgfqpoint{0.000000in}{-0.048611in}}{\pgfqpoint{0.000000in}{0.000000in}}{%
\pgfpathmoveto{\pgfqpoint{0.000000in}{0.000000in}}%
\pgfpathlineto{\pgfqpoint{0.000000in}{-0.048611in}}%
\pgfusepath{stroke,fill}%
}%
\begin{pgfscope}%
\pgfsys@transformshift{5.644373in}{0.528000in}%
\pgfsys@useobject{currentmarker}{}%
\end{pgfscope}%
\end{pgfscope}%
\begin{pgfscope}%
\definecolor{textcolor}{rgb}{0.000000,0.000000,0.000000}%
\pgfsetstrokecolor{textcolor}%
\pgfsetfillcolor{textcolor}%
\pgftext[x=5.644373in,y=0.430778in,,top]{\color{textcolor}\rmfamily\fontsize{10.000000}{12.000000}\selectfont \(\displaystyle 44\)}%
\end{pgfscope}%
\begin{pgfscope}%
\definecolor{textcolor}{rgb}{0.000000,0.000000,0.000000}%
\pgfsetstrokecolor{textcolor}%
\pgfsetfillcolor{textcolor}%
\pgftext[x=3.280000in,y=0.251766in,,top]{\color{textcolor}\rmfamily\fontsize{10.000000}{12.000000}\selectfont x}%
\end{pgfscope}%
\begin{pgfscope}%
\pgfsetbuttcap%
\pgfsetroundjoin%
\definecolor{currentfill}{rgb}{0.000000,0.000000,0.000000}%
\pgfsetfillcolor{currentfill}%
\pgfsetlinewidth{0.803000pt}%
\definecolor{currentstroke}{rgb}{0.000000,0.000000,0.000000}%
\pgfsetstrokecolor{currentstroke}%
\pgfsetdash{}{0pt}%
\pgfsys@defobject{currentmarker}{\pgfqpoint{-0.048611in}{0.000000in}}{\pgfqpoint{0.000000in}{0.000000in}}{%
\pgfpathmoveto{\pgfqpoint{0.000000in}{0.000000in}}%
\pgfpathlineto{\pgfqpoint{-0.048611in}{0.000000in}}%
\pgfusepath{stroke,fill}%
}%
\begin{pgfscope}%
\pgfsys@transformshift{0.800000in}{0.528000in}%
\pgfsys@useobject{currentmarker}{}%
\end{pgfscope}%
\end{pgfscope}%
\begin{pgfscope}%
\definecolor{textcolor}{rgb}{0.000000,0.000000,0.000000}%
\pgfsetstrokecolor{textcolor}%
\pgfsetfillcolor{textcolor}%
\pgftext[x=0.455863in,y=0.479775in,left,base]{\color{textcolor}\rmfamily\fontsize{10.000000}{12.000000}\selectfont \(\displaystyle 0.00\)}%
\end{pgfscope}%
\begin{pgfscope}%
\pgfsetbuttcap%
\pgfsetroundjoin%
\definecolor{currentfill}{rgb}{0.000000,0.000000,0.000000}%
\pgfsetfillcolor{currentfill}%
\pgfsetlinewidth{0.803000pt}%
\definecolor{currentstroke}{rgb}{0.000000,0.000000,0.000000}%
\pgfsetstrokecolor{currentstroke}%
\pgfsetdash{}{0pt}%
\pgfsys@defobject{currentmarker}{\pgfqpoint{-0.048611in}{0.000000in}}{\pgfqpoint{0.000000in}{0.000000in}}{%
\pgfpathmoveto{\pgfqpoint{0.000000in}{0.000000in}}%
\pgfpathlineto{\pgfqpoint{-0.048611in}{0.000000in}}%
\pgfusepath{stroke,fill}%
}%
\begin{pgfscope}%
\pgfsys@transformshift{0.800000in}{1.155000in}%
\pgfsys@useobject{currentmarker}{}%
\end{pgfscope}%
\end{pgfscope}%
\begin{pgfscope}%
\definecolor{textcolor}{rgb}{0.000000,0.000000,0.000000}%
\pgfsetstrokecolor{textcolor}%
\pgfsetfillcolor{textcolor}%
\pgftext[x=0.455863in,y=1.106775in,left,base]{\color{textcolor}\rmfamily\fontsize{10.000000}{12.000000}\selectfont \(\displaystyle 0.05\)}%
\end{pgfscope}%
\begin{pgfscope}%
\pgfsetbuttcap%
\pgfsetroundjoin%
\definecolor{currentfill}{rgb}{0.000000,0.000000,0.000000}%
\pgfsetfillcolor{currentfill}%
\pgfsetlinewidth{0.803000pt}%
\definecolor{currentstroke}{rgb}{0.000000,0.000000,0.000000}%
\pgfsetstrokecolor{currentstroke}%
\pgfsetdash{}{0pt}%
\pgfsys@defobject{currentmarker}{\pgfqpoint{-0.048611in}{0.000000in}}{\pgfqpoint{0.000000in}{0.000000in}}{%
\pgfpathmoveto{\pgfqpoint{0.000000in}{0.000000in}}%
\pgfpathlineto{\pgfqpoint{-0.048611in}{0.000000in}}%
\pgfusepath{stroke,fill}%
}%
\begin{pgfscope}%
\pgfsys@transformshift{0.800000in}{1.782000in}%
\pgfsys@useobject{currentmarker}{}%
\end{pgfscope}%
\end{pgfscope}%
\begin{pgfscope}%
\definecolor{textcolor}{rgb}{0.000000,0.000000,0.000000}%
\pgfsetstrokecolor{textcolor}%
\pgfsetfillcolor{textcolor}%
\pgftext[x=0.455863in,y=1.733775in,left,base]{\color{textcolor}\rmfamily\fontsize{10.000000}{12.000000}\selectfont \(\displaystyle 0.10\)}%
\end{pgfscope}%
\begin{pgfscope}%
\pgfsetbuttcap%
\pgfsetroundjoin%
\definecolor{currentfill}{rgb}{0.000000,0.000000,0.000000}%
\pgfsetfillcolor{currentfill}%
\pgfsetlinewidth{0.803000pt}%
\definecolor{currentstroke}{rgb}{0.000000,0.000000,0.000000}%
\pgfsetstrokecolor{currentstroke}%
\pgfsetdash{}{0pt}%
\pgfsys@defobject{currentmarker}{\pgfqpoint{-0.048611in}{0.000000in}}{\pgfqpoint{0.000000in}{0.000000in}}{%
\pgfpathmoveto{\pgfqpoint{0.000000in}{0.000000in}}%
\pgfpathlineto{\pgfqpoint{-0.048611in}{0.000000in}}%
\pgfusepath{stroke,fill}%
}%
\begin{pgfscope}%
\pgfsys@transformshift{0.800000in}{2.409000in}%
\pgfsys@useobject{currentmarker}{}%
\end{pgfscope}%
\end{pgfscope}%
\begin{pgfscope}%
\definecolor{textcolor}{rgb}{0.000000,0.000000,0.000000}%
\pgfsetstrokecolor{textcolor}%
\pgfsetfillcolor{textcolor}%
\pgftext[x=0.455863in,y=2.360775in,left,base]{\color{textcolor}\rmfamily\fontsize{10.000000}{12.000000}\selectfont \(\displaystyle 0.15\)}%
\end{pgfscope}%
\begin{pgfscope}%
\pgfsetbuttcap%
\pgfsetroundjoin%
\definecolor{currentfill}{rgb}{0.000000,0.000000,0.000000}%
\pgfsetfillcolor{currentfill}%
\pgfsetlinewidth{0.803000pt}%
\definecolor{currentstroke}{rgb}{0.000000,0.000000,0.000000}%
\pgfsetstrokecolor{currentstroke}%
\pgfsetdash{}{0pt}%
\pgfsys@defobject{currentmarker}{\pgfqpoint{-0.048611in}{0.000000in}}{\pgfqpoint{0.000000in}{0.000000in}}{%
\pgfpathmoveto{\pgfqpoint{0.000000in}{0.000000in}}%
\pgfpathlineto{\pgfqpoint{-0.048611in}{0.000000in}}%
\pgfusepath{stroke,fill}%
}%
\begin{pgfscope}%
\pgfsys@transformshift{0.800000in}{3.036000in}%
\pgfsys@useobject{currentmarker}{}%
\end{pgfscope}%
\end{pgfscope}%
\begin{pgfscope}%
\definecolor{textcolor}{rgb}{0.000000,0.000000,0.000000}%
\pgfsetstrokecolor{textcolor}%
\pgfsetfillcolor{textcolor}%
\pgftext[x=0.455863in,y=2.987775in,left,base]{\color{textcolor}\rmfamily\fontsize{10.000000}{12.000000}\selectfont \(\displaystyle 0.20\)}%
\end{pgfscope}%
\begin{pgfscope}%
\pgfsetbuttcap%
\pgfsetroundjoin%
\definecolor{currentfill}{rgb}{0.000000,0.000000,0.000000}%
\pgfsetfillcolor{currentfill}%
\pgfsetlinewidth{0.803000pt}%
\definecolor{currentstroke}{rgb}{0.000000,0.000000,0.000000}%
\pgfsetstrokecolor{currentstroke}%
\pgfsetdash{}{0pt}%
\pgfsys@defobject{currentmarker}{\pgfqpoint{-0.048611in}{0.000000in}}{\pgfqpoint{0.000000in}{0.000000in}}{%
\pgfpathmoveto{\pgfqpoint{0.000000in}{0.000000in}}%
\pgfpathlineto{\pgfqpoint{-0.048611in}{0.000000in}}%
\pgfusepath{stroke,fill}%
}%
\begin{pgfscope}%
\pgfsys@transformshift{0.800000in}{3.663000in}%
\pgfsys@useobject{currentmarker}{}%
\end{pgfscope}%
\end{pgfscope}%
\begin{pgfscope}%
\definecolor{textcolor}{rgb}{0.000000,0.000000,0.000000}%
\pgfsetstrokecolor{textcolor}%
\pgfsetfillcolor{textcolor}%
\pgftext[x=0.455863in,y=3.614775in,left,base]{\color{textcolor}\rmfamily\fontsize{10.000000}{12.000000}\selectfont \(\displaystyle 0.25\)}%
\end{pgfscope}%
\begin{pgfscope}%
\definecolor{textcolor}{rgb}{0.000000,0.000000,0.000000}%
\pgfsetstrokecolor{textcolor}%
\pgfsetfillcolor{textcolor}%
\pgftext[x=0.400308in,y=2.376000in,,bottom,rotate=90.000000]{\color{textcolor}\rmfamily\fontsize{10.000000}{12.000000}\selectfont Occurrence}%
\end{pgfscope}%
\begin{pgfscope}%
\pgfpathrectangle{\pgfqpoint{0.800000in}{0.528000in}}{\pgfqpoint{4.960000in}{3.696000in}}%
\pgfusepath{clip}%
\pgfsetrectcap%
\pgfsetroundjoin%
\pgfsetlinewidth{1.505625pt}%
\definecolor{currentstroke}{rgb}{1.000000,0.000000,0.000000}%
\pgfsetstrokecolor{currentstroke}%
\pgfsetdash{}{0pt}%
\pgfpathmoveto{\pgfqpoint{1.250242in}{0.707799in}}%
\pgfpathlineto{\pgfqpoint{1.797840in}{1.047142in}}%
\pgfpathlineto{\pgfqpoint{1.846596in}{1.091004in}}%
\pgfpathlineto{\pgfqpoint{1.926144in}{1.167674in}}%
\pgfpathlineto{\pgfqpoint{2.027247in}{1.274109in}}%
\pgfpathlineto{\pgfqpoint{2.213543in}{1.494802in}}%
\pgfpathlineto{\pgfqpoint{2.221241in}{1.504543in}}%
\pgfpathlineto{\pgfqpoint{2.246389in}{1.536670in}}%
\pgfpathlineto{\pgfqpoint{2.274615in}{1.573271in}}%
\pgfpathlineto{\pgfqpoint{2.301816in}{1.609052in}}%
\pgfpathlineto{\pgfqpoint{2.435764in}{1.791411in}}%
\pgfpathlineto{\pgfqpoint{2.571766in}{1.983421in}}%
\pgfpathlineto{\pgfqpoint{2.584083in}{2.000958in}}%
\pgfpathlineto{\pgfqpoint{2.877641in}{2.408774in}}%
\pgfpathlineto{\pgfqpoint{2.982850in}{2.541939in}}%
\pgfpathlineto{\pgfqpoint{2.984902in}{2.544426in}}%
\pgfpathlineto{\pgfqpoint{3.015695in}{2.581151in}}%
\pgfpathlineto{\pgfqpoint{3.026473in}{2.593739in}}%
\pgfpathlineto{\pgfqpoint{3.048541in}{2.619063in}}%
\pgfpathlineto{\pgfqpoint{3.129115in}{2.705935in}}%
\pgfpathlineto{\pgfqpoint{3.173765in}{2.749931in}}%
\pgfpathlineto{\pgfqpoint{3.199426in}{2.773763in}}%
\pgfpathlineto{\pgfqpoint{3.204044in}{2.777936in}}%
\pgfpathlineto{\pgfqpoint{3.410356in}{2.924291in}}%
\pgfpathlineto{\pgfqpoint{3.579203in}{2.979085in}}%
\pgfpathlineto{\pgfqpoint{3.583822in}{2.979703in}}%
\pgfpathlineto{\pgfqpoint{3.647974in}{2.983342in}}%
\pgfpathlineto{\pgfqpoint{3.669016in}{2.982524in}}%
\pgfpathlineto{\pgfqpoint{3.672095in}{2.982321in}}%
\pgfpathlineto{\pgfqpoint{3.720337in}{2.976369in}}%
\pgfpathlineto{\pgfqpoint{3.727009in}{2.975137in}}%
\pgfpathlineto{\pgfqpoint{3.769605in}{2.964945in}}%
\pgfpathlineto{\pgfqpoint{3.796292in}{2.956527in}}%
\pgfpathlineto{\pgfqpoint{3.846587in}{2.936474in}}%
\pgfpathlineto{\pgfqpoint{3.856852in}{2.931718in}}%
\pgfpathlineto{\pgfqpoint{3.874301in}{2.923126in}}%
\pgfpathlineto{\pgfqpoint{3.889184in}{2.915295in}}%
\pgfpathlineto{\pgfqpoint{3.901501in}{2.908470in}}%
\pgfpathlineto{\pgfqpoint{3.945124in}{2.881825in}}%
\pgfpathlineto{\pgfqpoint{4.053925in}{2.799370in}}%
\pgfpathlineto{\pgfqpoint{4.264343in}{2.583881in}}%
\pgfpathlineto{\pgfqpoint{4.322849in}{2.513347in}}%
\pgfpathlineto{\pgfqpoint{4.384948in}{2.434587in}}%
\pgfpathlineto{\pgfqpoint{4.433703in}{2.370387in}}%
\pgfpathlineto{\pgfqpoint{4.433703in}{2.370387in}}%
\pgfpathlineto{\pgfqpoint{4.436782in}{2.366271in}}%
\pgfpathlineto{\pgfqpoint{4.446534in}{2.353194in}}%
\pgfpathlineto{\pgfqpoint{4.560467in}{2.196341in}}%
\pgfpathlineto{\pgfqpoint{4.771911in}{1.896352in}}%
\pgfpathlineto{\pgfqpoint{4.843761in}{1.795368in}}%
\pgfpathlineto{\pgfqpoint{4.862749in}{1.768968in}}%
\pgfpathlineto{\pgfqpoint{4.898674in}{1.719442in}}%
\pgfpathlineto{\pgfqpoint{4.907912in}{1.706805in}}%
\pgfpathlineto{\pgfqpoint{4.914584in}{1.697704in}}%
\pgfpathlineto{\pgfqpoint{4.945377in}{1.656009in}}%
\pgfpathlineto{\pgfqpoint{5.094209in}{1.463004in}}%
\pgfpathlineto{\pgfqpoint{5.296415in}{1.229439in}}%
\pgfpathlineto{\pgfqpoint{5.398031in}{1.126590in}}%
\pgfpathlineto{\pgfqpoint{5.436522in}{1.090309in}}%
\pgfpathlineto{\pgfqpoint{5.534545in}{1.004611in}}%
\pgfusepath{stroke}%
\end{pgfscope}%
\begin{pgfscope}%
\pgfpathrectangle{\pgfqpoint{0.800000in}{0.528000in}}{\pgfqpoint{4.960000in}{3.696000in}}%
\pgfusepath{clip}%
\pgfsetrectcap%
\pgfsetroundjoin%
\pgfsetlinewidth{1.505625pt}%
\definecolor{currentstroke}{rgb}{0.000000,0.500000,0.000000}%
\pgfsetstrokecolor{currentstroke}%
\pgfsetdash{}{0pt}%
\pgfpathmoveto{\pgfqpoint{1.250242in}{0.713511in}}%
\pgfpathlineto{\pgfqpoint{1.797840in}{1.071329in}}%
\pgfpathlineto{\pgfqpoint{1.846596in}{1.117688in}}%
\pgfpathlineto{\pgfqpoint{1.926144in}{1.198712in}}%
\pgfpathlineto{\pgfqpoint{2.027247in}{1.311135in}}%
\pgfpathlineto{\pgfqpoint{2.213543in}{1.543808in}}%
\pgfpathlineto{\pgfqpoint{2.221241in}{1.554059in}}%
\pgfpathlineto{\pgfqpoint{2.246389in}{1.587856in}}%
\pgfpathlineto{\pgfqpoint{2.274615in}{1.626333in}}%
\pgfpathlineto{\pgfqpoint{2.301816in}{1.663919in}}%
\pgfpathlineto{\pgfqpoint{2.435764in}{1.854972in}}%
\pgfpathlineto{\pgfqpoint{2.571766in}{2.055018in}}%
\pgfpathlineto{\pgfqpoint{2.584083in}{2.073221in}}%
\pgfpathlineto{\pgfqpoint{2.877641in}{2.491989in}}%
\pgfpathlineto{\pgfqpoint{2.982850in}{2.626060in}}%
\pgfpathlineto{\pgfqpoint{2.984902in}{2.628546in}}%
\pgfpathlineto{\pgfqpoint{3.015695in}{2.665159in}}%
\pgfpathlineto{\pgfqpoint{3.026473in}{2.677666in}}%
\pgfpathlineto{\pgfqpoint{3.048541in}{2.702758in}}%
\pgfpathlineto{\pgfqpoint{3.129115in}{2.787995in}}%
\pgfpathlineto{\pgfqpoint{3.173765in}{2.830541in}}%
\pgfpathlineto{\pgfqpoint{3.199426in}{2.853362in}}%
\pgfpathlineto{\pgfqpoint{3.204044in}{2.857339in}}%
\pgfpathlineto{\pgfqpoint{3.410356in}{2.990708in}}%
\pgfpathlineto{\pgfqpoint{3.579203in}{3.029188in}}%
\pgfpathlineto{\pgfqpoint{3.583822in}{3.029298in}}%
\pgfpathlineto{\pgfqpoint{3.647974in}{3.025588in}}%
\pgfpathlineto{\pgfqpoint{3.669016in}{3.022249in}}%
\pgfpathlineto{\pgfqpoint{3.672095in}{3.021673in}}%
\pgfpathlineto{\pgfqpoint{3.720337in}{3.009747in}}%
\pgfpathlineto{\pgfqpoint{3.727009in}{3.007671in}}%
\pgfpathlineto{\pgfqpoint{3.769605in}{2.992002in}}%
\pgfpathlineto{\pgfqpoint{3.796292in}{2.980083in}}%
\pgfpathlineto{\pgfqpoint{3.846587in}{2.953316in}}%
\pgfpathlineto{\pgfqpoint{3.856852in}{2.947175in}}%
\pgfpathlineto{\pgfqpoint{3.874301in}{2.936220in}}%
\pgfpathlineto{\pgfqpoint{3.889184in}{2.926366in}}%
\pgfpathlineto{\pgfqpoint{3.901501in}{2.917862in}}%
\pgfpathlineto{\pgfqpoint{3.945124in}{2.885255in}}%
\pgfpathlineto{\pgfqpoint{4.053925in}{2.787995in}}%
\pgfpathlineto{\pgfqpoint{4.264343in}{2.545780in}}%
\pgfpathlineto{\pgfqpoint{4.322849in}{2.468624in}}%
\pgfpathlineto{\pgfqpoint{4.384948in}{2.383350in}}%
\pgfpathlineto{\pgfqpoint{4.433703in}{2.314445in}}%
\pgfpathlineto{\pgfqpoint{4.433703in}{2.314445in}}%
\pgfpathlineto{\pgfqpoint{4.436782in}{2.310045in}}%
\pgfpathlineto{\pgfqpoint{4.446534in}{2.296078in}}%
\pgfpathlineto{\pgfqpoint{4.560467in}{2.130027in}}%
\pgfpathlineto{\pgfqpoint{4.771911in}{1.819220in}}%
\pgfpathlineto{\pgfqpoint{4.843761in}{1.716436in}}%
\pgfpathlineto{\pgfqpoint{4.862749in}{1.689714in}}%
\pgfpathlineto{\pgfqpoint{4.898674in}{1.639751in}}%
\pgfpathlineto{\pgfqpoint{4.907912in}{1.627037in}}%
\pgfpathlineto{\pgfqpoint{4.914584in}{1.617891in}}%
\pgfpathlineto{\pgfqpoint{4.945377in}{1.576077in}}%
\pgfpathlineto{\pgfqpoint{5.094209in}{1.384578in}}%
\pgfpathlineto{\pgfqpoint{5.296415in}{1.157630in}}%
\pgfpathlineto{\pgfqpoint{5.398031in}{1.059548in}}%
\pgfpathlineto{\pgfqpoint{5.436522in}{1.025249in}}%
\pgfpathlineto{\pgfqpoint{5.534545in}{0.944909in}}%
\pgfusepath{stroke}%
\end{pgfscope}%
\begin{pgfscope}%
\pgfsetrectcap%
\pgfsetmiterjoin%
\pgfsetlinewidth{0.803000pt}%
\definecolor{currentstroke}{rgb}{0.000000,0.000000,0.000000}%
\pgfsetstrokecolor{currentstroke}%
\pgfsetdash{}{0pt}%
\pgfpathmoveto{\pgfqpoint{0.800000in}{0.528000in}}%
\pgfpathlineto{\pgfqpoint{0.800000in}{4.224000in}}%
\pgfusepath{stroke}%
\end{pgfscope}%
\begin{pgfscope}%
\pgfsetrectcap%
\pgfsetmiterjoin%
\pgfsetlinewidth{0.803000pt}%
\definecolor{currentstroke}{rgb}{0.000000,0.000000,0.000000}%
\pgfsetstrokecolor{currentstroke}%
\pgfsetdash{}{0pt}%
\pgfpathmoveto{\pgfqpoint{5.760000in}{0.528000in}}%
\pgfpathlineto{\pgfqpoint{5.760000in}{4.224000in}}%
\pgfusepath{stroke}%
\end{pgfscope}%
\begin{pgfscope}%
\pgfsetrectcap%
\pgfsetmiterjoin%
\pgfsetlinewidth{0.803000pt}%
\definecolor{currentstroke}{rgb}{0.000000,0.000000,0.000000}%
\pgfsetstrokecolor{currentstroke}%
\pgfsetdash{}{0pt}%
\pgfpathmoveto{\pgfqpoint{0.800000in}{0.528000in}}%
\pgfpathlineto{\pgfqpoint{5.760000in}{0.528000in}}%
\pgfusepath{stroke}%
\end{pgfscope}%
\begin{pgfscope}%
\pgfsetrectcap%
\pgfsetmiterjoin%
\pgfsetlinewidth{0.803000pt}%
\definecolor{currentstroke}{rgb}{0.000000,0.000000,0.000000}%
\pgfsetstrokecolor{currentstroke}%
\pgfsetdash{}{0pt}%
\pgfpathmoveto{\pgfqpoint{0.800000in}{4.224000in}}%
\pgfpathlineto{\pgfqpoint{5.760000in}{4.224000in}}%
\pgfusepath{stroke}%
\end{pgfscope}%
\begin{pgfscope}%
\pgfsetbuttcap%
\pgfsetmiterjoin%
\definecolor{currentfill}{rgb}{1.000000,0.647059,0.000000}%
\pgfsetfillcolor{currentfill}%
\pgfsetlinewidth{1.003750pt}%
\definecolor{currentstroke}{rgb}{0.000000,0.000000,0.000000}%
\pgfsetstrokecolor{currentstroke}%
\pgfsetdash{}{0pt}%
\pgfpathmoveto{\pgfqpoint{4.198082in}{2.917441in}}%
\pgfpathlineto{\pgfqpoint{5.423214in}{2.917441in}}%
\pgfpathquadraticcurveto{\pgfqpoint{5.464881in}{2.917441in}}{\pgfqpoint{5.464881in}{2.959108in}}%
\pgfpathlineto{\pgfqpoint{5.464881in}{3.082564in}}%
\pgfpathquadraticcurveto{\pgfqpoint{5.464881in}{3.124231in}}{\pgfqpoint{5.423214in}{3.124231in}}%
\pgfpathlineto{\pgfqpoint{4.198082in}{3.124231in}}%
\pgfpathquadraticcurveto{\pgfqpoint{4.156415in}{3.124231in}}{\pgfqpoint{4.156415in}{3.082564in}}%
\pgfpathlineto{\pgfqpoint{4.156415in}{2.959108in}}%
\pgfpathquadraticcurveto{\pgfqpoint{4.156415in}{2.917441in}}{\pgfqpoint{4.198082in}{2.917441in}}%
\pgfpathclose%
\pgfusepath{stroke,fill}%
\end{pgfscope}%
\begin{pgfscope}%
\definecolor{textcolor}{rgb}{0.000000,0.000000,0.000000}%
\pgfsetstrokecolor{textcolor}%
\pgfsetfillcolor{textcolor}%
\pgftext[x=4.198082in,y=2.986114in,left,base]{\color{textcolor}\rmfamily\fontsize{10.000000}{12.000000}\selectfont \(\displaystyle \mu=\)40.097\(\displaystyle ; \sigma=\)2.037}%
\end{pgfscope}%
\begin{pgfscope}%
\pgfsetbuttcap%
\pgfsetmiterjoin%
\definecolor{currentfill}{rgb}{1.000000,1.000000,1.000000}%
\pgfsetfillcolor{currentfill}%
\pgfsetfillopacity{0.800000}%
\pgfsetlinewidth{1.003750pt}%
\definecolor{currentstroke}{rgb}{0.800000,0.800000,0.800000}%
\pgfsetstrokecolor{currentstroke}%
\pgfsetstrokeopacity{0.800000}%
\pgfsetdash{}{0pt}%
\pgfpathmoveto{\pgfqpoint{4.056487in}{3.531871in}}%
\pgfpathlineto{\pgfqpoint{5.662778in}{3.531871in}}%
\pgfpathquadraticcurveto{\pgfqpoint{5.690556in}{3.531871in}}{\pgfqpoint{5.690556in}{3.559648in}}%
\pgfpathlineto{\pgfqpoint{5.690556in}{4.126778in}}%
\pgfpathquadraticcurveto{\pgfqpoint{5.690556in}{4.154556in}}{\pgfqpoint{5.662778in}{4.154556in}}%
\pgfpathlineto{\pgfqpoint{4.056487in}{4.154556in}}%
\pgfpathquadraticcurveto{\pgfqpoint{4.028709in}{4.154556in}}{\pgfqpoint{4.028709in}{4.126778in}}%
\pgfpathlineto{\pgfqpoint{4.028709in}{3.559648in}}%
\pgfpathquadraticcurveto{\pgfqpoint{4.028709in}{3.531871in}}{\pgfqpoint{4.056487in}{3.531871in}}%
\pgfpathclose%
\pgfusepath{stroke,fill}%
\end{pgfscope}%
\begin{pgfscope}%
\pgfsetrectcap%
\pgfsetroundjoin%
\pgfsetlinewidth{1.505625pt}%
\definecolor{currentstroke}{rgb}{1.000000,0.000000,0.000000}%
\pgfsetstrokecolor{currentstroke}%
\pgfsetdash{}{0pt}%
\pgfpathmoveto{\pgfqpoint{4.084265in}{4.050389in}}%
\pgfpathlineto{\pgfqpoint{4.362042in}{4.050389in}}%
\pgfusepath{stroke}%
\end{pgfscope}%
\begin{pgfscope}%
\definecolor{textcolor}{rgb}{0.000000,0.000000,0.000000}%
\pgfsetstrokecolor{textcolor}%
\pgfsetfillcolor{textcolor}%
\pgftext[x=4.473154in,y=4.001778in,left,base]{\color{textcolor}\rmfamily\fontsize{10.000000}{12.000000}\selectfont Actual Gaussian}%
\end{pgfscope}%
\begin{pgfscope}%
\pgfsetrectcap%
\pgfsetroundjoin%
\pgfsetlinewidth{1.505625pt}%
\definecolor{currentstroke}{rgb}{0.000000,0.500000,0.000000}%
\pgfsetstrokecolor{currentstroke}%
\pgfsetdash{}{0pt}%
\pgfpathmoveto{\pgfqpoint{4.084265in}{3.856716in}}%
\pgfpathlineto{\pgfqpoint{4.362042in}{3.856716in}}%
\pgfusepath{stroke}%
\end{pgfscope}%
\begin{pgfscope}%
\definecolor{textcolor}{rgb}{0.000000,0.000000,0.000000}%
\pgfsetstrokecolor{textcolor}%
\pgfsetfillcolor{textcolor}%
\pgftext[x=4.473154in,y=3.808105in,left,base]{\color{textcolor}\rmfamily\fontsize{10.000000}{12.000000}\selectfont Expected Gaussian}%
\end{pgfscope}%
\begin{pgfscope}%
\pgfsetbuttcap%
\pgfsetmiterjoin%
\definecolor{currentfill}{rgb}{0.121569,0.466667,0.705882}%
\pgfsetfillcolor{currentfill}%
\pgfsetlinewidth{0.000000pt}%
\definecolor{currentstroke}{rgb}{0.000000,0.000000,0.000000}%
\pgfsetstrokecolor{currentstroke}%
\pgfsetstrokeopacity{0.000000}%
\pgfsetdash{}{0pt}%
\pgfpathmoveto{\pgfqpoint{4.084265in}{3.614432in}}%
\pgfpathlineto{\pgfqpoint{4.362042in}{3.614432in}}%
\pgfpathlineto{\pgfqpoint{4.362042in}{3.711654in}}%
\pgfpathlineto{\pgfqpoint{4.084265in}{3.711654in}}%
\pgfpathclose%
\pgfusepath{fill}%
\end{pgfscope}%
\begin{pgfscope}%
\definecolor{textcolor}{rgb}{0.000000,0.000000,0.000000}%
\pgfsetstrokecolor{textcolor}%
\pgfsetfillcolor{textcolor}%
\pgftext[x=4.473154in,y=3.614432in,left,base]{\color{textcolor}\rmfamily\fontsize{10.000000}{12.000000}\selectfont Randomised Data}%
\end{pgfscope}%
\end{pgfpicture}%
\makeatother%
\endgroup%
}}%
            \caption{Randomised Data Histogram and Gaussians}
            \label{fig:RandomDataHistograms}
        \end{figure}
        
        \noindent
        As can be seen, the mean still sits at around 40 but due to the different distributions of 
        the random numbers used to create the data it's never exactly 40.


    \end{itemize}
    \newpage
    \section{Appendix}
    \setcounter{figure}{0} \renewcommand{\thefigure}{A.\arabic{figure}}
    \lstinputlisting[caption=CP4\_2a, style=appendix]{CP4_2a.py}
    \lstinputlisting[caption=CP4\_2b, style=appendix]{CP4_2b.py}
    \lstinputlisting[caption=CP4\_3, style=appendix]{CP4_3.py}
    
\end{document}