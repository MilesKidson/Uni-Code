\documentclass[12pt]{article}
\usepackage[margin=1.2in]{geometry}
\usepackage{graphicx}
\usepackage{amsmath}
\usepackage{physics}
\usepackage{tabto}
\usepackage{float}
\usepackage{amssymb}
\usepackage{pgfplots}
\usepackage{verbatim}
\usepackage{tcolorbox}
\usepackage{listings}
\usepackage{xcolor}
\usepackage{siunitx}
\usepackage[all]{nowidow}
\definecolor{stringcolor}{HTML}{C792EA}
\definecolor{codeblue}{HTML}{2162DB}
\definecolor{commentcolor}{HTML}{4A6E46}
\lstdefinestyle{appendix}{
    basicstyle=\ttfamily\footnotesize,
    commentstyle=\color{commentcolor},
    keywordstyle=\color{codeblue},
    stringstyle=\color{stringcolor},
    showstringspaces=false,
    numbers=left,
    upquote=true,
    captionpos=t,
    abovecaptionskip=12pt,
    belowcaptionskip=12pt,
    language=Python,
    breaklines=true,
    frame=single,
}
\lstdefinestyle{inline}{
    basicstyle=\ttfamily\footnotesize,
    commentstyle=\color{commentcolor},
    keywordstyle=\color{codeblue},
    stringstyle=\color{stringcolor},
    showstringspaces=false,
    numbers=left,
    upquote=true,
    frame=tb,
    captionpos=b,
    language=Python
}
\renewcommand{\lstlistingname}{Appendix}
\pgfplotsset{compat=1.16}

\title{Integrating Equations of Motion}
\date{\textbf{9 March 2020}}
\author{}

\begin{document}

    \begin{titlepage}
        \maketitle
        \center
        \textbf{\large{PHY2004W}}\
        \textbf{\large{KDSMIL001}}\
        \tableofcontents
    \end{titlepage}

    \section{Introduction}
    This computational activity was an introduction into the world of numerical integration, 
    which is one of the methods that we can use to solve differential equations, something 
    that we have to do quite often in physics as we are often solving second order ordinary 
    differential equations that come out of applying Newton's second law. This is often 
    reduced to two coupled equations:

    \begin{equation}
        \frac{d\vec{p}}{dt} = \vec{F}
        \hspace{15pt}
        \textrm{and}
        \hspace{15pt}
        \frac{d\vec{q}}{dt} = \frac{\vec{p}}{m}
        \label{eq:newton}
    \end{equation}

    \noindent
    where $\vec{p}$ is a momentum, $\vec{F}$ is a net force, $\vec{q}$ is a coordinate, 
    and $m$ is a mass. 
    \newline
    In order to numerically solve these equations, we approximate the derivatives, which 
    results in 2 equations, which we call update equations:

    \begin{equation}
        \begin{split}
            &\vec{p}(t+\Delta t) = \vec{p}(t)+\vec{F}(\vec{q}(t))\Delta t \\
            &\vec{q}(t+\Delta t) = \vec{q}(t)+\left(\frac{\vec{p}(t)}{m}\right)\Delta t 
        \end{split}
        \label{eq:update}
    \end{equation}

    \noindent
    These are what we will use in this Activity in order to approximate the motion of, in 
    this case, a simple pendulum. These equations are very tedious to evaluate for all t, 
    but that's where the computer comes in. 

    \section{Activity}
    The first thing that we need to do is check the accuracy of these equations (\ref{eq:update}). 
    In order to do this, we perform a Taylor expansion of $y(t+\Delta t)$, truncating after 
    only 2 terms, a first order approximation, making the error small, but still significant:
    
    \begin{equation*}
        \begin{split}
            y(t+\Delta t) &= \sum_{n=0}^\infty \frac{y^{(n)}(t)\Delta t^n}{n!} \\
            &= y(t) + \dot{y}(t) \Delta t + \ddot{y}(t) \Delta t^2 + \dots
        \end{split}
    \end{equation*}

    \noindent
    When truncating this at the second term, this gives us something of the form of the 
    equation for $\vec{p}$ in Equation \ref{eq:update}, which confirms that we are using 
    first order approximations in this activity. 
    \newline
    For the sake of interest, we can try to get a second order approximation of the first 
    derivative. For the sake of computation, we have approximated the derivative to be 

    \begin{equation}
        \frac{dy}{dt} \approx \frac{y(t+\Delta t) - y(t)}{\Delta t}
    \end{equation}

    \noindent
    From this, we can substitute in a second order approximation for $y(t+\Delta t)$, 
    which returns

    \begin{equation}
        \begin{split}
            \frac{dy}{dt} &\approx \frac{y(t) + \dot{y}(t) \Delta t + \ddot{y}(t) \Delta t^2 - y(t)}{\Delta t} \\
            &= \dot{y}(t) + \ddot{y}(t) \Delta t
        \end{split}
    \end{equation}

    \noindent
    We see that this gives us something of the form of the equation for $\vec{q}$ in 
    Equation \ref{eq:update}.
    \newline
    Next, we need to create the model that we will be using to simulate the system of 
    interest. In its most complete form, the equation of motion for a pendulum is:

    \begin{equation}
        \frac{d^2\theta}{dt^2}+\Omega_0^2\sin\theta = 0 
        \label{eq:pendulum}
    \end{equation}

    \noindent
    where $\Omega_0^2 = \frac{g}{L}$, $\theta$ is the angle from the vertical, and $L$ is the 
    length of the pendulum. This equation comes from an analysis of the forces on the system. 
    We know that the torque $\tau = I\alpha$, where $I = mL^2$ is the moment of inertia for a pendulum 
    with a massless string and $\alpha = \ddot{\theta}$ is the angular acceleration. Then, by 
    equating all the forces, we get

    \begin{equation*}
        \begin{split}
            &\tau = I\alpha = mL^2\ddot{\theta} \\
            \textrm{Equating forces:}\hspace{5pt} &mL^2\ddot{\theta} = -mg\sin\theta L \\
            \implies &\ddot{\theta} + \frac{g}{L}\sin\theta = 0 \\
        \end{split}
        \label{eq:pendulum derivation}
    \end{equation*}

    \noindent
    In order to simplify this, we can take what we call a small angle approximation, where 
    $\sin\theta = \theta$ for angles $<\approx\ang{10}$. This leads us to the equation 

    \begin{equation}
        \ddot{\theta} = -\Omega_0^2 \theta
        \label{eq:SHM}
    \end{equation}

    \noindent
    which we know is the general form for a simple harmonic oscillator.
    \newline
    \newline
    Now, going back to Equation \ref{eq:pendulum derivation}, if we let 
    $\Omega_0 = \sqrt{\frac{g}{L}}$, then we get the equation as in (\ref{eq:pendulum}). More 
    manipulation will lead to a coupled form of this equation:

    \begin{equation}
        \frac{d\omega}{dt}=-\Omega_0^2 \sin\theta 
        \hspace{15pt}
        \textrm{and}
        \hspace{15pt}
        \frac{d\theta}{dt}=\omega
        \label{eq:coupled pendulum}
    \end{equation}

    \noindent
    Given the initial conditions of $\theta(0)=\ang{90}, \omega(0)=0$, we interpret this as the 
    pendulum being held at \ang{90} to the vertical and at rest (angular velocity $\omega = 0$), 
    then let go. In the small angle approximation, the system should oscillate with an angular 
    frequency of $\Omega_0^2$, but these initial conditions do not satisfy the small angle 
    approximation, so we will need another method in order to figure out what the motion of the 
    system will look like, for example numerical integration. 
    \newline
    \newline
    In order to perform this numerical integration, we need to have some equations in the form 
    of Equation \ref{eq:update}, which we can adapt to have the following:

    \begin{equation}
        \begin{split}
            &p_{i+1} = p_i - \Omega_0^2 \sin(q_i) \Delta t \\
            &q_{i+1} = q_i + p_i \Delta t
        \end{split}
        \label{eq:pendulum update}
    \end{equation}

    \noindent
    where $i=0,1,2\dots$ and $q \equiv \theta, p \equiv \omega$.
    \newline
    We can do this because we got the equations for the motion of the pendulum into a similar 
    form in Equation \ref{eq:coupled pendulum} as in Equation \ref{eq:update}. This is known 
    as Euler's Method and is a commonly used method for numerically solving systems that are 
    hard or impossible to solve analytically. Now we can begin to implement this in Python in 
    order for the computer to do the heavy lifting. 
    \newline
    \newline
    The entire code for this calculation is in Appendix 1 but the most important part is the 
    loop we use a loop to continuously update the values of $\theta$ and $\omega$, appending 
    them to an array so that we can plot them later on, which can be seen below.
    \newline
        
    \lstinputlisting[title=Euler's Method Loop, style=inline, linerange=25-29, firstnumber=25]{CP3a.py}
    
    \noindent
    We also calculate a value for the total energy of the system at each step, the equation 
    for which is 

    \begin{equation}
        E = E_K + E_P = \frac{1}{2}mL^2 \omega^2 + mgL(1-\cos\theta)
    \end{equation}

    \noindent
    In this case, we are not given a value for the mass or the length of the pendulum, so 
    we assume them to be 1 as this shouldn't affect the shape of the plot, just the scale 
    of it. Below is the first plot [Figure \ref{fig:CP390}] which has initial conditions 
    of $\theta = \ang{90}$ and $\omega = 0$. This clearly does not satisfy the conditions 
    for a small angle approximation, and thus at around 6.5s, the plots go haywire, with 
    $\theta$ and $\omega$ becoming only negative, which is not a natural way for this 
    system to behave. 

    \begin{figure}[H]
        \begin{center}
           \scalebox{.7}{%% Creator: Matplotlib, PGF backend
%%
%% To include the figure in your LaTeX document, write
%%   \input{<filename>.pgf}
%%
%% Make sure the required packages are loaded in your preamble
%%   \usepackage{pgf}
%%
%% Figures using additional raster images can only be included by \input if
%% they are in the same directory as the main LaTeX file. For loading figures
%% from other directories you can use the `import` package
%%   \usepackage{import}
%% and then include the figures with
%%   \import{<path to file>}{<filename>.pgf}
%%
%% Matplotlib used the following preamble
%%
\begingroup%
\makeatletter%
\begin{pgfpicture}%
\pgfpathrectangle{\pgfpointorigin}{\pgfqpoint{6.400000in}{4.800000in}}%
\pgfusepath{use as bounding box, clip}%
\begin{pgfscope}%
\pgfsetbuttcap%
\pgfsetmiterjoin%
\definecolor{currentfill}{rgb}{1.000000,1.000000,1.000000}%
\pgfsetfillcolor{currentfill}%
\pgfsetlinewidth{0.000000pt}%
\definecolor{currentstroke}{rgb}{1.000000,1.000000,1.000000}%
\pgfsetstrokecolor{currentstroke}%
\pgfsetdash{}{0pt}%
\pgfpathmoveto{\pgfqpoint{0.000000in}{0.000000in}}%
\pgfpathlineto{\pgfqpoint{6.400000in}{0.000000in}}%
\pgfpathlineto{\pgfqpoint{6.400000in}{4.800000in}}%
\pgfpathlineto{\pgfqpoint{0.000000in}{4.800000in}}%
\pgfpathclose%
\pgfusepath{fill}%
\end{pgfscope}%
\begin{pgfscope}%
\pgfsetbuttcap%
\pgfsetmiterjoin%
\definecolor{currentfill}{rgb}{1.000000,1.000000,1.000000}%
\pgfsetfillcolor{currentfill}%
\pgfsetlinewidth{0.000000pt}%
\definecolor{currentstroke}{rgb}{0.000000,0.000000,0.000000}%
\pgfsetstrokecolor{currentstroke}%
\pgfsetstrokeopacity{0.000000}%
\pgfsetdash{}{0pt}%
\pgfpathmoveto{\pgfqpoint{0.796484in}{2.870679in}}%
\pgfpathlineto{\pgfqpoint{3.196863in}{2.870679in}}%
\pgfpathlineto{\pgfqpoint{3.196863in}{4.489815in}}%
\pgfpathlineto{\pgfqpoint{0.796484in}{4.489815in}}%
\pgfpathclose%
\pgfusepath{fill}%
\end{pgfscope}%
\begin{pgfscope}%
\pgfsetbuttcap%
\pgfsetroundjoin%
\definecolor{currentfill}{rgb}{0.000000,0.000000,0.000000}%
\pgfsetfillcolor{currentfill}%
\pgfsetlinewidth{0.803000pt}%
\definecolor{currentstroke}{rgb}{0.000000,0.000000,0.000000}%
\pgfsetstrokecolor{currentstroke}%
\pgfsetdash{}{0pt}%
\pgfsys@defobject{currentmarker}{\pgfqpoint{0.000000in}{-0.048611in}}{\pgfqpoint{0.000000in}{0.000000in}}{%
\pgfpathmoveto{\pgfqpoint{0.000000in}{0.000000in}}%
\pgfpathlineto{\pgfqpoint{0.000000in}{-0.048611in}}%
\pgfusepath{stroke,fill}%
}%
\begin{pgfscope}%
\pgfsys@transformshift{0.905592in}{2.870679in}%
\pgfsys@useobject{currentmarker}{}%
\end{pgfscope}%
\end{pgfscope}%
\begin{pgfscope}%
\definecolor{textcolor}{rgb}{0.000000,0.000000,0.000000}%
\pgfsetstrokecolor{textcolor}%
\pgfsetfillcolor{textcolor}%
\pgftext[x=0.905592in,y=2.773457in,,top]{\color{textcolor}\rmfamily\fontsize{10.000000}{12.000000}\selectfont \(\displaystyle 0.0\)}%
\end{pgfscope}%
\begin{pgfscope}%
\pgfsetbuttcap%
\pgfsetroundjoin%
\definecolor{currentfill}{rgb}{0.000000,0.000000,0.000000}%
\pgfsetfillcolor{currentfill}%
\pgfsetlinewidth{0.803000pt}%
\definecolor{currentstroke}{rgb}{0.000000,0.000000,0.000000}%
\pgfsetstrokecolor{currentstroke}%
\pgfsetdash{}{0pt}%
\pgfsys@defobject{currentmarker}{\pgfqpoint{0.000000in}{-0.048611in}}{\pgfqpoint{0.000000in}{0.000000in}}{%
\pgfpathmoveto{\pgfqpoint{0.000000in}{0.000000in}}%
\pgfpathlineto{\pgfqpoint{0.000000in}{-0.048611in}}%
\pgfusepath{stroke,fill}%
}%
\begin{pgfscope}%
\pgfsys@transformshift{1.451133in}{2.870679in}%
\pgfsys@useobject{currentmarker}{}%
\end{pgfscope}%
\end{pgfscope}%
\begin{pgfscope}%
\definecolor{textcolor}{rgb}{0.000000,0.000000,0.000000}%
\pgfsetstrokecolor{textcolor}%
\pgfsetfillcolor{textcolor}%
\pgftext[x=1.451133in,y=2.773457in,,top]{\color{textcolor}\rmfamily\fontsize{10.000000}{12.000000}\selectfont \(\displaystyle 2.5\)}%
\end{pgfscope}%
\begin{pgfscope}%
\pgfsetbuttcap%
\pgfsetroundjoin%
\definecolor{currentfill}{rgb}{0.000000,0.000000,0.000000}%
\pgfsetfillcolor{currentfill}%
\pgfsetlinewidth{0.803000pt}%
\definecolor{currentstroke}{rgb}{0.000000,0.000000,0.000000}%
\pgfsetstrokecolor{currentstroke}%
\pgfsetdash{}{0pt}%
\pgfsys@defobject{currentmarker}{\pgfqpoint{0.000000in}{-0.048611in}}{\pgfqpoint{0.000000in}{0.000000in}}{%
\pgfpathmoveto{\pgfqpoint{0.000000in}{0.000000in}}%
\pgfpathlineto{\pgfqpoint{0.000000in}{-0.048611in}}%
\pgfusepath{stroke,fill}%
}%
\begin{pgfscope}%
\pgfsys@transformshift{1.996673in}{2.870679in}%
\pgfsys@useobject{currentmarker}{}%
\end{pgfscope}%
\end{pgfscope}%
\begin{pgfscope}%
\definecolor{textcolor}{rgb}{0.000000,0.000000,0.000000}%
\pgfsetstrokecolor{textcolor}%
\pgfsetfillcolor{textcolor}%
\pgftext[x=1.996673in,y=2.773457in,,top]{\color{textcolor}\rmfamily\fontsize{10.000000}{12.000000}\selectfont \(\displaystyle 5.0\)}%
\end{pgfscope}%
\begin{pgfscope}%
\pgfsetbuttcap%
\pgfsetroundjoin%
\definecolor{currentfill}{rgb}{0.000000,0.000000,0.000000}%
\pgfsetfillcolor{currentfill}%
\pgfsetlinewidth{0.803000pt}%
\definecolor{currentstroke}{rgb}{0.000000,0.000000,0.000000}%
\pgfsetstrokecolor{currentstroke}%
\pgfsetdash{}{0pt}%
\pgfsys@defobject{currentmarker}{\pgfqpoint{0.000000in}{-0.048611in}}{\pgfqpoint{0.000000in}{0.000000in}}{%
\pgfpathmoveto{\pgfqpoint{0.000000in}{0.000000in}}%
\pgfpathlineto{\pgfqpoint{0.000000in}{-0.048611in}}%
\pgfusepath{stroke,fill}%
}%
\begin{pgfscope}%
\pgfsys@transformshift{2.542214in}{2.870679in}%
\pgfsys@useobject{currentmarker}{}%
\end{pgfscope}%
\end{pgfscope}%
\begin{pgfscope}%
\definecolor{textcolor}{rgb}{0.000000,0.000000,0.000000}%
\pgfsetstrokecolor{textcolor}%
\pgfsetfillcolor{textcolor}%
\pgftext[x=2.542214in,y=2.773457in,,top]{\color{textcolor}\rmfamily\fontsize{10.000000}{12.000000}\selectfont \(\displaystyle 7.5\)}%
\end{pgfscope}%
\begin{pgfscope}%
\pgfsetbuttcap%
\pgfsetroundjoin%
\definecolor{currentfill}{rgb}{0.000000,0.000000,0.000000}%
\pgfsetfillcolor{currentfill}%
\pgfsetlinewidth{0.803000pt}%
\definecolor{currentstroke}{rgb}{0.000000,0.000000,0.000000}%
\pgfsetstrokecolor{currentstroke}%
\pgfsetdash{}{0pt}%
\pgfsys@defobject{currentmarker}{\pgfqpoint{0.000000in}{-0.048611in}}{\pgfqpoint{0.000000in}{0.000000in}}{%
\pgfpathmoveto{\pgfqpoint{0.000000in}{0.000000in}}%
\pgfpathlineto{\pgfqpoint{0.000000in}{-0.048611in}}%
\pgfusepath{stroke,fill}%
}%
\begin{pgfscope}%
\pgfsys@transformshift{3.087755in}{2.870679in}%
\pgfsys@useobject{currentmarker}{}%
\end{pgfscope}%
\end{pgfscope}%
\begin{pgfscope}%
\definecolor{textcolor}{rgb}{0.000000,0.000000,0.000000}%
\pgfsetstrokecolor{textcolor}%
\pgfsetfillcolor{textcolor}%
\pgftext[x=3.087755in,y=2.773457in,,top]{\color{textcolor}\rmfamily\fontsize{10.000000}{12.000000}\selectfont \(\displaystyle 10.0\)}%
\end{pgfscope}%
\begin{pgfscope}%
\definecolor{textcolor}{rgb}{0.000000,0.000000,0.000000}%
\pgfsetstrokecolor{textcolor}%
\pgfsetfillcolor{textcolor}%
\pgftext[x=1.996673in,y=2.594444in,,top]{\color{textcolor}\rmfamily\fontsize{10.000000}{12.000000}\selectfont time (s)}%
\end{pgfscope}%
\begin{pgfscope}%
\pgfsetbuttcap%
\pgfsetroundjoin%
\definecolor{currentfill}{rgb}{0.000000,0.000000,0.000000}%
\pgfsetfillcolor{currentfill}%
\pgfsetlinewidth{0.803000pt}%
\definecolor{currentstroke}{rgb}{0.000000,0.000000,0.000000}%
\pgfsetstrokecolor{currentstroke}%
\pgfsetdash{}{0pt}%
\pgfsys@defobject{currentmarker}{\pgfqpoint{-0.048611in}{0.000000in}}{\pgfqpoint{0.000000in}{0.000000in}}{%
\pgfpathmoveto{\pgfqpoint{0.000000in}{0.000000in}}%
\pgfpathlineto{\pgfqpoint{-0.048611in}{0.000000in}}%
\pgfusepath{stroke,fill}%
}%
\begin{pgfscope}%
\pgfsys@transformshift{0.796484in}{3.111319in}%
\pgfsys@useobject{currentmarker}{}%
\end{pgfscope}%
\end{pgfscope}%
\begin{pgfscope}%
\definecolor{textcolor}{rgb}{0.000000,0.000000,0.000000}%
\pgfsetstrokecolor{textcolor}%
\pgfsetfillcolor{textcolor}%
\pgftext[x=0.452348in,y=3.063094in,left,base]{\color{textcolor}\rmfamily\fontsize{10.000000}{12.000000}\selectfont \(\displaystyle -20\)}%
\end{pgfscope}%
\begin{pgfscope}%
\pgfsetbuttcap%
\pgfsetroundjoin%
\definecolor{currentfill}{rgb}{0.000000,0.000000,0.000000}%
\pgfsetfillcolor{currentfill}%
\pgfsetlinewidth{0.803000pt}%
\definecolor{currentstroke}{rgb}{0.000000,0.000000,0.000000}%
\pgfsetstrokecolor{currentstroke}%
\pgfsetdash{}{0pt}%
\pgfsys@defobject{currentmarker}{\pgfqpoint{-0.048611in}{0.000000in}}{\pgfqpoint{0.000000in}{0.000000in}}{%
\pgfpathmoveto{\pgfqpoint{0.000000in}{0.000000in}}%
\pgfpathlineto{\pgfqpoint{-0.048611in}{0.000000in}}%
\pgfusepath{stroke,fill}%
}%
\begin{pgfscope}%
\pgfsys@transformshift{0.796484in}{3.687969in}%
\pgfsys@useobject{currentmarker}{}%
\end{pgfscope}%
\end{pgfscope}%
\begin{pgfscope}%
\definecolor{textcolor}{rgb}{0.000000,0.000000,0.000000}%
\pgfsetstrokecolor{textcolor}%
\pgfsetfillcolor{textcolor}%
\pgftext[x=0.452348in,y=3.639744in,left,base]{\color{textcolor}\rmfamily\fontsize{10.000000}{12.000000}\selectfont \(\displaystyle -10\)}%
\end{pgfscope}%
\begin{pgfscope}%
\pgfsetbuttcap%
\pgfsetroundjoin%
\definecolor{currentfill}{rgb}{0.000000,0.000000,0.000000}%
\pgfsetfillcolor{currentfill}%
\pgfsetlinewidth{0.803000pt}%
\definecolor{currentstroke}{rgb}{0.000000,0.000000,0.000000}%
\pgfsetstrokecolor{currentstroke}%
\pgfsetdash{}{0pt}%
\pgfsys@defobject{currentmarker}{\pgfqpoint{-0.048611in}{0.000000in}}{\pgfqpoint{0.000000in}{0.000000in}}{%
\pgfpathmoveto{\pgfqpoint{0.000000in}{0.000000in}}%
\pgfpathlineto{\pgfqpoint{-0.048611in}{0.000000in}}%
\pgfusepath{stroke,fill}%
}%
\begin{pgfscope}%
\pgfsys@transformshift{0.796484in}{4.264618in}%
\pgfsys@useobject{currentmarker}{}%
\end{pgfscope}%
\end{pgfscope}%
\begin{pgfscope}%
\definecolor{textcolor}{rgb}{0.000000,0.000000,0.000000}%
\pgfsetstrokecolor{textcolor}%
\pgfsetfillcolor{textcolor}%
\pgftext[x=0.629817in,y=4.216393in,left,base]{\color{textcolor}\rmfamily\fontsize{10.000000}{12.000000}\selectfont \(\displaystyle 0\)}%
\end{pgfscope}%
\begin{pgfscope}%
\definecolor{textcolor}{rgb}{0.000000,0.000000,0.000000}%
\pgfsetstrokecolor{textcolor}%
\pgfsetfillcolor{textcolor}%
\pgftext[x=0.396792in,y=3.680247in,,bottom,rotate=90.000000]{\color{textcolor}\rmfamily\fontsize{10.000000}{12.000000}\selectfont angle (rad)}%
\end{pgfscope}%
\begin{pgfscope}%
\pgfpathrectangle{\pgfqpoint{0.796484in}{2.870679in}}{\pgfqpoint{2.400379in}{1.619136in}}%
\pgfusepath{clip}%
\pgfsetrectcap%
\pgfsetroundjoin%
\pgfsetlinewidth{1.505625pt}%
\definecolor{currentstroke}{rgb}{0.000000,0.000000,1.000000}%
\pgfsetstrokecolor{currentstroke}%
\pgfsetdash{}{0pt}%
\pgfpathmoveto{\pgfqpoint{0.905592in}{4.355198in}}%
\pgfpathlineto{\pgfqpoint{0.914321in}{4.354333in}}%
\pgfpathlineto{\pgfqpoint{0.923050in}{4.351162in}}%
\pgfpathlineto{\pgfqpoint{0.931778in}{4.345688in}}%
\pgfpathlineto{\pgfqpoint{0.942689in}{4.335643in}}%
\pgfpathlineto{\pgfqpoint{0.953600in}{4.322170in}}%
\pgfpathlineto{\pgfqpoint{0.966693in}{4.301956in}}%
\pgfpathlineto{\pgfqpoint{0.984150in}{4.270119in}}%
\pgfpathlineto{\pgfqpoint{1.012518in}{4.217520in}}%
\pgfpathlineto{\pgfqpoint{1.025611in}{4.197651in}}%
\pgfpathlineto{\pgfqpoint{1.036522in}{4.184577in}}%
\pgfpathlineto{\pgfqpoint{1.047433in}{4.175001in}}%
\pgfpathlineto{\pgfqpoint{1.056161in}{4.169927in}}%
\pgfpathlineto{\pgfqpoint{1.064890in}{4.167155in}}%
\pgfpathlineto{\pgfqpoint{1.073619in}{4.166676in}}%
\pgfpathlineto{\pgfqpoint{1.082347in}{4.168486in}}%
\pgfpathlineto{\pgfqpoint{1.091076in}{4.172589in}}%
\pgfpathlineto{\pgfqpoint{1.099805in}{4.178995in}}%
\pgfpathlineto{\pgfqpoint{1.110716in}{4.190234in}}%
\pgfpathlineto{\pgfqpoint{1.121626in}{4.204959in}}%
\pgfpathlineto{\pgfqpoint{1.134719in}{4.226763in}}%
\pgfpathlineto{\pgfqpoint{1.152177in}{4.260774in}}%
\pgfpathlineto{\pgfqpoint{1.180545in}{4.316324in}}%
\pgfpathlineto{\pgfqpoint{1.193638in}{4.337135in}}%
\pgfpathlineto{\pgfqpoint{1.204549in}{4.350844in}}%
\pgfpathlineto{\pgfqpoint{1.215459in}{4.360989in}}%
\pgfpathlineto{\pgfqpoint{1.224188in}{4.366518in}}%
\pgfpathlineto{\pgfqpoint{1.232917in}{4.369777in}}%
\pgfpathlineto{\pgfqpoint{1.241645in}{4.370796in}}%
\pgfpathlineto{\pgfqpoint{1.250374in}{4.369589in}}%
\pgfpathlineto{\pgfqpoint{1.259103in}{4.366151in}}%
\pgfpathlineto{\pgfqpoint{1.267831in}{4.360455in}}%
\pgfpathlineto{\pgfqpoint{1.276560in}{4.352471in}}%
\pgfpathlineto{\pgfqpoint{1.287471in}{4.339255in}}%
\pgfpathlineto{\pgfqpoint{1.298381in}{4.322548in}}%
\pgfpathlineto{\pgfqpoint{1.311474in}{4.298433in}}%
\pgfpathlineto{\pgfqpoint{1.333296in}{4.252352in}}%
\pgfpathlineto{\pgfqpoint{1.352936in}{4.212094in}}%
\pgfpathlineto{\pgfqpoint{1.366029in}{4.189601in}}%
\pgfpathlineto{\pgfqpoint{1.376939in}{4.174540in}}%
\pgfpathlineto{\pgfqpoint{1.387850in}{4.163060in}}%
\pgfpathlineto{\pgfqpoint{1.398761in}{4.155133in}}%
\pgfpathlineto{\pgfqpoint{1.407490in}{4.151274in}}%
\pgfpathlineto{\pgfqpoint{1.416218in}{4.149557in}}%
\pgfpathlineto{\pgfqpoint{1.424947in}{4.149948in}}%
\pgfpathlineto{\pgfqpoint{1.433676in}{4.152445in}}%
\pgfpathlineto{\pgfqpoint{1.442404in}{4.157079in}}%
\pgfpathlineto{\pgfqpoint{1.451133in}{4.163905in}}%
\pgfpathlineto{\pgfqpoint{1.462044in}{4.175614in}}%
\pgfpathlineto{\pgfqpoint{1.472954in}{4.190913in}}%
\pgfpathlineto{\pgfqpoint{1.486047in}{4.213832in}}%
\pgfpathlineto{\pgfqpoint{1.501323in}{4.245706in}}%
\pgfpathlineto{\pgfqpoint{1.540602in}{4.330698in}}%
\pgfpathlineto{\pgfqpoint{1.553694in}{4.352447in}}%
\pgfpathlineto{\pgfqpoint{1.564605in}{4.366673in}}%
\pgfpathlineto{\pgfqpoint{1.575516in}{4.377350in}}%
\pgfpathlineto{\pgfqpoint{1.586427in}{4.384646in}}%
\pgfpathlineto{\pgfqpoint{1.595156in}{4.388182in}}%
\pgfpathlineto{\pgfqpoint{1.603884in}{4.389761in}}%
\pgfpathlineto{\pgfqpoint{1.612613in}{4.389427in}}%
\pgfpathlineto{\pgfqpoint{1.621342in}{4.387183in}}%
\pgfpathlineto{\pgfqpoint{1.630070in}{4.382992in}}%
\pgfpathlineto{\pgfqpoint{1.640981in}{4.374897in}}%
\pgfpathlineto{\pgfqpoint{1.651892in}{4.363448in}}%
\pgfpathlineto{\pgfqpoint{1.662803in}{4.348469in}}%
\pgfpathlineto{\pgfqpoint{1.673713in}{4.329916in}}%
\pgfpathlineto{\pgfqpoint{1.686806in}{4.303316in}}%
\pgfpathlineto{\pgfqpoint{1.706446in}{4.257608in}}%
\pgfpathlineto{\pgfqpoint{1.728267in}{4.207608in}}%
\pgfpathlineto{\pgfqpoint{1.741360in}{4.182346in}}%
\pgfpathlineto{\pgfqpoint{1.754453in}{4.162082in}}%
\pgfpathlineto{\pgfqpoint{1.765364in}{4.149097in}}%
\pgfpathlineto{\pgfqpoint{1.776275in}{4.139438in}}%
\pgfpathlineto{\pgfqpoint{1.787186in}{4.132813in}}%
\pgfpathlineto{\pgfqpoint{1.798097in}{4.128961in}}%
\pgfpathlineto{\pgfqpoint{1.809007in}{4.127700in}}%
\pgfpathlineto{\pgfqpoint{1.819918in}{4.128952in}}%
\pgfpathlineto{\pgfqpoint{1.830829in}{4.132750in}}%
\pgfpathlineto{\pgfqpoint{1.841740in}{4.139235in}}%
\pgfpathlineto{\pgfqpoint{1.852651in}{4.148636in}}%
\pgfpathlineto{\pgfqpoint{1.863562in}{4.161237in}}%
\pgfpathlineto{\pgfqpoint{1.874472in}{4.177301in}}%
\pgfpathlineto{\pgfqpoint{1.887565in}{4.201291in}}%
\pgfpathlineto{\pgfqpoint{1.902840in}{4.235126in}}%
\pgfpathlineto{\pgfqpoint{1.948666in}{4.343117in}}%
\pgfpathlineto{\pgfqpoint{1.961759in}{4.366144in}}%
\pgfpathlineto{\pgfqpoint{1.972670in}{4.381427in}}%
\pgfpathlineto{\pgfqpoint{1.983580in}{4.393410in}}%
\pgfpathlineto{\pgfqpoint{1.994491in}{4.402470in}}%
\pgfpathlineto{\pgfqpoint{2.005402in}{4.408995in}}%
\pgfpathlineto{\pgfqpoint{2.016313in}{4.413310in}}%
\pgfpathlineto{\pgfqpoint{2.029406in}{4.415905in}}%
\pgfpathlineto{\pgfqpoint{2.042499in}{4.415894in}}%
\pgfpathlineto{\pgfqpoint{2.055592in}{4.413310in}}%
\pgfpathlineto{\pgfqpoint{2.066503in}{4.409077in}}%
\pgfpathlineto{\pgfqpoint{2.077413in}{4.402746in}}%
\pgfpathlineto{\pgfqpoint{2.088324in}{4.394030in}}%
\pgfpathlineto{\pgfqpoint{2.099235in}{4.382571in}}%
\pgfpathlineto{\pgfqpoint{2.110146in}{4.367986in}}%
\pgfpathlineto{\pgfqpoint{2.121057in}{4.349950in}}%
\pgfpathlineto{\pgfqpoint{2.134150in}{4.323609in}}%
\pgfpathlineto{\pgfqpoint{2.149425in}{4.287253in}}%
\pgfpathlineto{\pgfqpoint{2.188704in}{4.189846in}}%
\pgfpathlineto{\pgfqpoint{2.201797in}{4.164402in}}%
\pgfpathlineto{\pgfqpoint{2.214890in}{4.144054in}}%
\pgfpathlineto{\pgfqpoint{2.227983in}{4.128316in}}%
\pgfpathlineto{\pgfqpoint{2.241076in}{4.116368in}}%
\pgfpathlineto{\pgfqpoint{2.254169in}{4.107364in}}%
\pgfpathlineto{\pgfqpoint{2.269444in}{4.099611in}}%
\pgfpathlineto{\pgfqpoint{2.286901in}{4.093259in}}%
\pgfpathlineto{\pgfqpoint{2.310905in}{4.087121in}}%
\pgfpathlineto{\pgfqpoint{2.380734in}{4.071064in}}%
\pgfpathlineto{\pgfqpoint{2.398191in}{4.064134in}}%
\pgfpathlineto{\pgfqpoint{2.413466in}{4.055725in}}%
\pgfpathlineto{\pgfqpoint{2.426559in}{4.046061in}}%
\pgfpathlineto{\pgfqpoint{2.439652in}{4.033383in}}%
\pgfpathlineto{\pgfqpoint{2.450563in}{4.019935in}}%
\pgfpathlineto{\pgfqpoint{2.461474in}{4.003367in}}%
\pgfpathlineto{\pgfqpoint{2.474567in}{3.978885in}}%
\pgfpathlineto{\pgfqpoint{2.487660in}{3.949374in}}%
\pgfpathlineto{\pgfqpoint{2.507299in}{3.898343in}}%
\pgfpathlineto{\pgfqpoint{2.531303in}{3.836474in}}%
\pgfpathlineto{\pgfqpoint{2.544396in}{3.808132in}}%
\pgfpathlineto{\pgfqpoint{2.557489in}{3.784890in}}%
\pgfpathlineto{\pgfqpoint{2.570582in}{3.766379in}}%
\pgfpathlineto{\pgfqpoint{2.583675in}{3.751699in}}%
\pgfpathlineto{\pgfqpoint{2.598950in}{3.738041in}}%
\pgfpathlineto{\pgfqpoint{2.620772in}{3.722105in}}%
\pgfpathlineto{\pgfqpoint{2.651322in}{3.700010in}}%
\pgfpathlineto{\pgfqpoint{2.666597in}{3.686129in}}%
\pgfpathlineto{\pgfqpoint{2.679690in}{3.671253in}}%
\pgfpathlineto{\pgfqpoint{2.692783in}{3.652595in}}%
\pgfpathlineto{\pgfqpoint{2.705876in}{3.629277in}}%
\pgfpathlineto{\pgfqpoint{2.718969in}{3.600853in}}%
\pgfpathlineto{\pgfqpoint{2.734244in}{3.561979in}}%
\pgfpathlineto{\pgfqpoint{2.771341in}{3.463849in}}%
\pgfpathlineto{\pgfqpoint{2.784434in}{3.436053in}}%
\pgfpathlineto{\pgfqpoint{2.797527in}{3.413304in}}%
\pgfpathlineto{\pgfqpoint{2.810620in}{3.394904in}}%
\pgfpathlineto{\pgfqpoint{2.825895in}{3.377445in}}%
\pgfpathlineto{\pgfqpoint{2.847717in}{3.356417in}}%
\pgfpathlineto{\pgfqpoint{2.873903in}{3.330621in}}%
\pgfpathlineto{\pgfqpoint{2.889178in}{3.311990in}}%
\pgfpathlineto{\pgfqpoint{2.902271in}{3.292218in}}%
\pgfpathlineto{\pgfqpoint{2.915364in}{3.267875in}}%
\pgfpathlineto{\pgfqpoint{2.928457in}{3.238437in}}%
\pgfpathlineto{\pgfqpoint{2.943732in}{3.198390in}}%
\pgfpathlineto{\pgfqpoint{2.980829in}{3.098121in}}%
\pgfpathlineto{\pgfqpoint{2.993922in}{3.069818in}}%
\pgfpathlineto{\pgfqpoint{3.007015in}{3.046497in}}%
\pgfpathlineto{\pgfqpoint{3.020108in}{3.027302in}}%
\pgfpathlineto{\pgfqpoint{3.037565in}{3.005931in}}%
\pgfpathlineto{\pgfqpoint{3.076844in}{2.960066in}}%
\pgfpathlineto{\pgfqpoint{3.087755in}{2.944276in}}%
\pgfpathlineto{\pgfqpoint{3.087755in}{2.944276in}}%
\pgfusepath{stroke}%
\end{pgfscope}%
\begin{pgfscope}%
\pgfsetrectcap%
\pgfsetmiterjoin%
\pgfsetlinewidth{0.803000pt}%
\definecolor{currentstroke}{rgb}{0.000000,0.000000,0.000000}%
\pgfsetstrokecolor{currentstroke}%
\pgfsetdash{}{0pt}%
\pgfpathmoveto{\pgfqpoint{0.796484in}{2.870679in}}%
\pgfpathlineto{\pgfqpoint{0.796484in}{4.489815in}}%
\pgfusepath{stroke}%
\end{pgfscope}%
\begin{pgfscope}%
\pgfsetrectcap%
\pgfsetmiterjoin%
\pgfsetlinewidth{0.803000pt}%
\definecolor{currentstroke}{rgb}{0.000000,0.000000,0.000000}%
\pgfsetstrokecolor{currentstroke}%
\pgfsetdash{}{0pt}%
\pgfpathmoveto{\pgfqpoint{3.196863in}{2.870679in}}%
\pgfpathlineto{\pgfqpoint{3.196863in}{4.489815in}}%
\pgfusepath{stroke}%
\end{pgfscope}%
\begin{pgfscope}%
\pgfsetrectcap%
\pgfsetmiterjoin%
\pgfsetlinewidth{0.803000pt}%
\definecolor{currentstroke}{rgb}{0.000000,0.000000,0.000000}%
\pgfsetstrokecolor{currentstroke}%
\pgfsetdash{}{0pt}%
\pgfpathmoveto{\pgfqpoint{0.796484in}{2.870679in}}%
\pgfpathlineto{\pgfqpoint{3.196863in}{2.870679in}}%
\pgfusepath{stroke}%
\end{pgfscope}%
\begin{pgfscope}%
\pgfsetrectcap%
\pgfsetmiterjoin%
\pgfsetlinewidth{0.803000pt}%
\definecolor{currentstroke}{rgb}{0.000000,0.000000,0.000000}%
\pgfsetstrokecolor{currentstroke}%
\pgfsetdash{}{0pt}%
\pgfpathmoveto{\pgfqpoint{0.796484in}{4.489815in}}%
\pgfpathlineto{\pgfqpoint{3.196863in}{4.489815in}}%
\pgfusepath{stroke}%
\end{pgfscope}%
\begin{pgfscope}%
\definecolor{textcolor}{rgb}{0.000000,0.000000,0.000000}%
\pgfsetstrokecolor{textcolor}%
\pgfsetfillcolor{textcolor}%
\pgftext[x=1.996673in,y=4.573148in,,base]{\color{textcolor}\rmfamily\fontsize{12.000000}{14.400000}\selectfont \(\displaystyle \theta\)}%
\end{pgfscope}%
\begin{pgfscope}%
\pgfsetbuttcap%
\pgfsetmiterjoin%
\definecolor{currentfill}{rgb}{1.000000,1.000000,1.000000}%
\pgfsetfillcolor{currentfill}%
\pgfsetlinewidth{0.000000pt}%
\definecolor{currentstroke}{rgb}{0.000000,0.000000,0.000000}%
\pgfsetstrokecolor{currentstroke}%
\pgfsetstrokeopacity{0.000000}%
\pgfsetdash{}{0pt}%
\pgfpathmoveto{\pgfqpoint{3.867533in}{2.870679in}}%
\pgfpathlineto{\pgfqpoint{6.267911in}{2.870679in}}%
\pgfpathlineto{\pgfqpoint{6.267911in}{4.489815in}}%
\pgfpathlineto{\pgfqpoint{3.867533in}{4.489815in}}%
\pgfpathclose%
\pgfusepath{fill}%
\end{pgfscope}%
\begin{pgfscope}%
\pgfsetbuttcap%
\pgfsetroundjoin%
\definecolor{currentfill}{rgb}{0.000000,0.000000,0.000000}%
\pgfsetfillcolor{currentfill}%
\pgfsetlinewidth{0.803000pt}%
\definecolor{currentstroke}{rgb}{0.000000,0.000000,0.000000}%
\pgfsetstrokecolor{currentstroke}%
\pgfsetdash{}{0pt}%
\pgfsys@defobject{currentmarker}{\pgfqpoint{0.000000in}{-0.048611in}}{\pgfqpoint{0.000000in}{0.000000in}}{%
\pgfpathmoveto{\pgfqpoint{0.000000in}{0.000000in}}%
\pgfpathlineto{\pgfqpoint{0.000000in}{-0.048611in}}%
\pgfusepath{stroke,fill}%
}%
\begin{pgfscope}%
\pgfsys@transformshift{3.976641in}{2.870679in}%
\pgfsys@useobject{currentmarker}{}%
\end{pgfscope}%
\end{pgfscope}%
\begin{pgfscope}%
\definecolor{textcolor}{rgb}{0.000000,0.000000,0.000000}%
\pgfsetstrokecolor{textcolor}%
\pgfsetfillcolor{textcolor}%
\pgftext[x=3.976641in,y=2.773457in,,top]{\color{textcolor}\rmfamily\fontsize{10.000000}{12.000000}\selectfont \(\displaystyle 0.0\)}%
\end{pgfscope}%
\begin{pgfscope}%
\pgfsetbuttcap%
\pgfsetroundjoin%
\definecolor{currentfill}{rgb}{0.000000,0.000000,0.000000}%
\pgfsetfillcolor{currentfill}%
\pgfsetlinewidth{0.803000pt}%
\definecolor{currentstroke}{rgb}{0.000000,0.000000,0.000000}%
\pgfsetstrokecolor{currentstroke}%
\pgfsetdash{}{0pt}%
\pgfsys@defobject{currentmarker}{\pgfqpoint{0.000000in}{-0.048611in}}{\pgfqpoint{0.000000in}{0.000000in}}{%
\pgfpathmoveto{\pgfqpoint{0.000000in}{0.000000in}}%
\pgfpathlineto{\pgfqpoint{0.000000in}{-0.048611in}}%
\pgfusepath{stroke,fill}%
}%
\begin{pgfscope}%
\pgfsys@transformshift{4.522181in}{2.870679in}%
\pgfsys@useobject{currentmarker}{}%
\end{pgfscope}%
\end{pgfscope}%
\begin{pgfscope}%
\definecolor{textcolor}{rgb}{0.000000,0.000000,0.000000}%
\pgfsetstrokecolor{textcolor}%
\pgfsetfillcolor{textcolor}%
\pgftext[x=4.522181in,y=2.773457in,,top]{\color{textcolor}\rmfamily\fontsize{10.000000}{12.000000}\selectfont \(\displaystyle 2.5\)}%
\end{pgfscope}%
\begin{pgfscope}%
\pgfsetbuttcap%
\pgfsetroundjoin%
\definecolor{currentfill}{rgb}{0.000000,0.000000,0.000000}%
\pgfsetfillcolor{currentfill}%
\pgfsetlinewidth{0.803000pt}%
\definecolor{currentstroke}{rgb}{0.000000,0.000000,0.000000}%
\pgfsetstrokecolor{currentstroke}%
\pgfsetdash{}{0pt}%
\pgfsys@defobject{currentmarker}{\pgfqpoint{0.000000in}{-0.048611in}}{\pgfqpoint{0.000000in}{0.000000in}}{%
\pgfpathmoveto{\pgfqpoint{0.000000in}{0.000000in}}%
\pgfpathlineto{\pgfqpoint{0.000000in}{-0.048611in}}%
\pgfusepath{stroke,fill}%
}%
\begin{pgfscope}%
\pgfsys@transformshift{5.067722in}{2.870679in}%
\pgfsys@useobject{currentmarker}{}%
\end{pgfscope}%
\end{pgfscope}%
\begin{pgfscope}%
\definecolor{textcolor}{rgb}{0.000000,0.000000,0.000000}%
\pgfsetstrokecolor{textcolor}%
\pgfsetfillcolor{textcolor}%
\pgftext[x=5.067722in,y=2.773457in,,top]{\color{textcolor}\rmfamily\fontsize{10.000000}{12.000000}\selectfont \(\displaystyle 5.0\)}%
\end{pgfscope}%
\begin{pgfscope}%
\pgfsetbuttcap%
\pgfsetroundjoin%
\definecolor{currentfill}{rgb}{0.000000,0.000000,0.000000}%
\pgfsetfillcolor{currentfill}%
\pgfsetlinewidth{0.803000pt}%
\definecolor{currentstroke}{rgb}{0.000000,0.000000,0.000000}%
\pgfsetstrokecolor{currentstroke}%
\pgfsetdash{}{0pt}%
\pgfsys@defobject{currentmarker}{\pgfqpoint{0.000000in}{-0.048611in}}{\pgfqpoint{0.000000in}{0.000000in}}{%
\pgfpathmoveto{\pgfqpoint{0.000000in}{0.000000in}}%
\pgfpathlineto{\pgfqpoint{0.000000in}{-0.048611in}}%
\pgfusepath{stroke,fill}%
}%
\begin{pgfscope}%
\pgfsys@transformshift{5.613262in}{2.870679in}%
\pgfsys@useobject{currentmarker}{}%
\end{pgfscope}%
\end{pgfscope}%
\begin{pgfscope}%
\definecolor{textcolor}{rgb}{0.000000,0.000000,0.000000}%
\pgfsetstrokecolor{textcolor}%
\pgfsetfillcolor{textcolor}%
\pgftext[x=5.613262in,y=2.773457in,,top]{\color{textcolor}\rmfamily\fontsize{10.000000}{12.000000}\selectfont \(\displaystyle 7.5\)}%
\end{pgfscope}%
\begin{pgfscope}%
\pgfsetbuttcap%
\pgfsetroundjoin%
\definecolor{currentfill}{rgb}{0.000000,0.000000,0.000000}%
\pgfsetfillcolor{currentfill}%
\pgfsetlinewidth{0.803000pt}%
\definecolor{currentstroke}{rgb}{0.000000,0.000000,0.000000}%
\pgfsetstrokecolor{currentstroke}%
\pgfsetdash{}{0pt}%
\pgfsys@defobject{currentmarker}{\pgfqpoint{0.000000in}{-0.048611in}}{\pgfqpoint{0.000000in}{0.000000in}}{%
\pgfpathmoveto{\pgfqpoint{0.000000in}{0.000000in}}%
\pgfpathlineto{\pgfqpoint{0.000000in}{-0.048611in}}%
\pgfusepath{stroke,fill}%
}%
\begin{pgfscope}%
\pgfsys@transformshift{6.158803in}{2.870679in}%
\pgfsys@useobject{currentmarker}{}%
\end{pgfscope}%
\end{pgfscope}%
\begin{pgfscope}%
\definecolor{textcolor}{rgb}{0.000000,0.000000,0.000000}%
\pgfsetstrokecolor{textcolor}%
\pgfsetfillcolor{textcolor}%
\pgftext[x=6.158803in,y=2.773457in,,top]{\color{textcolor}\rmfamily\fontsize{10.000000}{12.000000}\selectfont \(\displaystyle 10.0\)}%
\end{pgfscope}%
\begin{pgfscope}%
\definecolor{textcolor}{rgb}{0.000000,0.000000,0.000000}%
\pgfsetstrokecolor{textcolor}%
\pgfsetfillcolor{textcolor}%
\pgftext[x=5.067722in,y=2.594444in,,top]{\color{textcolor}\rmfamily\fontsize{10.000000}{12.000000}\selectfont time (s)}%
\end{pgfscope}%
\begin{pgfscope}%
\pgfsetbuttcap%
\pgfsetroundjoin%
\definecolor{currentfill}{rgb}{0.000000,0.000000,0.000000}%
\pgfsetfillcolor{currentfill}%
\pgfsetlinewidth{0.803000pt}%
\definecolor{currentstroke}{rgb}{0.000000,0.000000,0.000000}%
\pgfsetstrokecolor{currentstroke}%
\pgfsetdash{}{0pt}%
\pgfsys@defobject{currentmarker}{\pgfqpoint{-0.048611in}{0.000000in}}{\pgfqpoint{0.000000in}{0.000000in}}{%
\pgfpathmoveto{\pgfqpoint{0.000000in}{0.000000in}}%
\pgfpathlineto{\pgfqpoint{-0.048611in}{0.000000in}}%
\pgfusepath{stroke,fill}%
}%
\begin{pgfscope}%
\pgfsys@transformshift{3.867533in}{2.997217in}%
\pgfsys@useobject{currentmarker}{}%
\end{pgfscope}%
\end{pgfscope}%
\begin{pgfscope}%
\definecolor{textcolor}{rgb}{0.000000,0.000000,0.000000}%
\pgfsetstrokecolor{textcolor}%
\pgfsetfillcolor{textcolor}%
\pgftext[x=3.523396in,y=2.948992in,left,base]{\color{textcolor}\rmfamily\fontsize{10.000000}{12.000000}\selectfont \(\displaystyle -10\)}%
\end{pgfscope}%
\begin{pgfscope}%
\pgfsetbuttcap%
\pgfsetroundjoin%
\definecolor{currentfill}{rgb}{0.000000,0.000000,0.000000}%
\pgfsetfillcolor{currentfill}%
\pgfsetlinewidth{0.803000pt}%
\definecolor{currentstroke}{rgb}{0.000000,0.000000,0.000000}%
\pgfsetstrokecolor{currentstroke}%
\pgfsetdash{}{0pt}%
\pgfsys@defobject{currentmarker}{\pgfqpoint{-0.048611in}{0.000000in}}{\pgfqpoint{0.000000in}{0.000000in}}{%
\pgfpathmoveto{\pgfqpoint{0.000000in}{0.000000in}}%
\pgfpathlineto{\pgfqpoint{-0.048611in}{0.000000in}}%
\pgfusepath{stroke,fill}%
}%
\begin{pgfscope}%
\pgfsys@transformshift{3.867533in}{3.361382in}%
\pgfsys@useobject{currentmarker}{}%
\end{pgfscope}%
\end{pgfscope}%
\begin{pgfscope}%
\definecolor{textcolor}{rgb}{0.000000,0.000000,0.000000}%
\pgfsetstrokecolor{textcolor}%
\pgfsetfillcolor{textcolor}%
\pgftext[x=3.592841in,y=3.313156in,left,base]{\color{textcolor}\rmfamily\fontsize{10.000000}{12.000000}\selectfont \(\displaystyle -5\)}%
\end{pgfscope}%
\begin{pgfscope}%
\pgfsetbuttcap%
\pgfsetroundjoin%
\definecolor{currentfill}{rgb}{0.000000,0.000000,0.000000}%
\pgfsetfillcolor{currentfill}%
\pgfsetlinewidth{0.803000pt}%
\definecolor{currentstroke}{rgb}{0.000000,0.000000,0.000000}%
\pgfsetstrokecolor{currentstroke}%
\pgfsetdash{}{0pt}%
\pgfsys@defobject{currentmarker}{\pgfqpoint{-0.048611in}{0.000000in}}{\pgfqpoint{0.000000in}{0.000000in}}{%
\pgfpathmoveto{\pgfqpoint{0.000000in}{0.000000in}}%
\pgfpathlineto{\pgfqpoint{-0.048611in}{0.000000in}}%
\pgfusepath{stroke,fill}%
}%
\begin{pgfscope}%
\pgfsys@transformshift{3.867533in}{3.725546in}%
\pgfsys@useobject{currentmarker}{}%
\end{pgfscope}%
\end{pgfscope}%
\begin{pgfscope}%
\definecolor{textcolor}{rgb}{0.000000,0.000000,0.000000}%
\pgfsetstrokecolor{textcolor}%
\pgfsetfillcolor{textcolor}%
\pgftext[x=3.700866in,y=3.677321in,left,base]{\color{textcolor}\rmfamily\fontsize{10.000000}{12.000000}\selectfont \(\displaystyle 0\)}%
\end{pgfscope}%
\begin{pgfscope}%
\pgfsetbuttcap%
\pgfsetroundjoin%
\definecolor{currentfill}{rgb}{0.000000,0.000000,0.000000}%
\pgfsetfillcolor{currentfill}%
\pgfsetlinewidth{0.803000pt}%
\definecolor{currentstroke}{rgb}{0.000000,0.000000,0.000000}%
\pgfsetstrokecolor{currentstroke}%
\pgfsetdash{}{0pt}%
\pgfsys@defobject{currentmarker}{\pgfqpoint{-0.048611in}{0.000000in}}{\pgfqpoint{0.000000in}{0.000000in}}{%
\pgfpathmoveto{\pgfqpoint{0.000000in}{0.000000in}}%
\pgfpathlineto{\pgfqpoint{-0.048611in}{0.000000in}}%
\pgfusepath{stroke,fill}%
}%
\begin{pgfscope}%
\pgfsys@transformshift{3.867533in}{4.089711in}%
\pgfsys@useobject{currentmarker}{}%
\end{pgfscope}%
\end{pgfscope}%
\begin{pgfscope}%
\definecolor{textcolor}{rgb}{0.000000,0.000000,0.000000}%
\pgfsetstrokecolor{textcolor}%
\pgfsetfillcolor{textcolor}%
\pgftext[x=3.700866in,y=4.041485in,left,base]{\color{textcolor}\rmfamily\fontsize{10.000000}{12.000000}\selectfont \(\displaystyle 5\)}%
\end{pgfscope}%
\begin{pgfscope}%
\pgfsetbuttcap%
\pgfsetroundjoin%
\definecolor{currentfill}{rgb}{0.000000,0.000000,0.000000}%
\pgfsetfillcolor{currentfill}%
\pgfsetlinewidth{0.803000pt}%
\definecolor{currentstroke}{rgb}{0.000000,0.000000,0.000000}%
\pgfsetstrokecolor{currentstroke}%
\pgfsetdash{}{0pt}%
\pgfsys@defobject{currentmarker}{\pgfqpoint{-0.048611in}{0.000000in}}{\pgfqpoint{0.000000in}{0.000000in}}{%
\pgfpathmoveto{\pgfqpoint{0.000000in}{0.000000in}}%
\pgfpathlineto{\pgfqpoint{-0.048611in}{0.000000in}}%
\pgfusepath{stroke,fill}%
}%
\begin{pgfscope}%
\pgfsys@transformshift{3.867533in}{4.453875in}%
\pgfsys@useobject{currentmarker}{}%
\end{pgfscope}%
\end{pgfscope}%
\begin{pgfscope}%
\definecolor{textcolor}{rgb}{0.000000,0.000000,0.000000}%
\pgfsetstrokecolor{textcolor}%
\pgfsetfillcolor{textcolor}%
\pgftext[x=3.631421in,y=4.405650in,left,base]{\color{textcolor}\rmfamily\fontsize{10.000000}{12.000000}\selectfont \(\displaystyle 10\)}%
\end{pgfscope}%
\begin{pgfscope}%
\definecolor{textcolor}{rgb}{0.000000,0.000000,0.000000}%
\pgfsetstrokecolor{textcolor}%
\pgfsetfillcolor{textcolor}%
\pgftext[x=3.467840in,y=3.680247in,,bottom,rotate=90.000000]{\color{textcolor}\rmfamily\fontsize{10.000000}{12.000000}\selectfont angle (rad)}%
\end{pgfscope}%
\begin{pgfscope}%
\pgfpathrectangle{\pgfqpoint{3.867533in}{2.870679in}}{\pgfqpoint{2.400379in}{1.619136in}}%
\pgfusepath{clip}%
\pgfsetrectcap%
\pgfsetroundjoin%
\pgfsetlinewidth{1.505625pt}%
\definecolor{currentstroke}{rgb}{0.000000,0.000000,1.000000}%
\pgfsetstrokecolor{currentstroke}%
\pgfsetdash{}{0pt}%
\pgfpathmoveto{\pgfqpoint{3.976641in}{3.725546in}}%
\pgfpathlineto{\pgfqpoint{4.022466in}{3.352080in}}%
\pgfpathlineto{\pgfqpoint{4.033377in}{3.278644in}}%
\pgfpathlineto{\pgfqpoint{4.042106in}{3.233062in}}%
\pgfpathlineto{\pgfqpoint{4.048652in}{3.209414in}}%
\pgfpathlineto{\pgfqpoint{4.053016in}{3.199543in}}%
\pgfpathlineto{\pgfqpoint{4.057381in}{3.194771in}}%
\pgfpathlineto{\pgfqpoint{4.059563in}{3.194356in}}%
\pgfpathlineto{\pgfqpoint{4.061745in}{3.195269in}}%
\pgfpathlineto{\pgfqpoint{4.066109in}{3.201047in}}%
\pgfpathlineto{\pgfqpoint{4.070474in}{3.211954in}}%
\pgfpathlineto{\pgfqpoint{4.077020in}{3.237230in}}%
\pgfpathlineto{\pgfqpoint{4.083567in}{3.271809in}}%
\pgfpathlineto{\pgfqpoint{4.094477in}{3.344943in}}%
\pgfpathlineto{\pgfqpoint{4.111935in}{3.483567in}}%
\pgfpathlineto{\pgfqpoint{4.194857in}{4.163129in}}%
\pgfpathlineto{\pgfqpoint{4.205768in}{4.231400in}}%
\pgfpathlineto{\pgfqpoint{4.212314in}{4.261583in}}%
\pgfpathlineto{\pgfqpoint{4.218861in}{4.281285in}}%
\pgfpathlineto{\pgfqpoint{4.223225in}{4.287859in}}%
\pgfpathlineto{\pgfqpoint{4.225407in}{4.289072in}}%
\pgfpathlineto{\pgfqpoint{4.227589in}{4.288881in}}%
\pgfpathlineto{\pgfqpoint{4.229772in}{4.287282in}}%
\pgfpathlineto{\pgfqpoint{4.234136in}{4.279916in}}%
\pgfpathlineto{\pgfqpoint{4.238500in}{4.267194in}}%
\pgfpathlineto{\pgfqpoint{4.245047in}{4.238948in}}%
\pgfpathlineto{\pgfqpoint{4.253775in}{4.187150in}}%
\pgfpathlineto{\pgfqpoint{4.264686in}{4.106953in}}%
\pgfpathlineto{\pgfqpoint{4.293054in}{3.873096in}}%
\pgfpathlineto{\pgfqpoint{4.369430in}{3.253895in}}%
\pgfpathlineto{\pgfqpoint{4.380341in}{3.184573in}}%
\pgfpathlineto{\pgfqpoint{4.386887in}{3.154425in}}%
\pgfpathlineto{\pgfqpoint{4.393434in}{3.135496in}}%
\pgfpathlineto{\pgfqpoint{4.397798in}{3.129890in}}%
\pgfpathlineto{\pgfqpoint{4.399980in}{3.129295in}}%
\pgfpathlineto{\pgfqpoint{4.402162in}{3.130189in}}%
\pgfpathlineto{\pgfqpoint{4.406527in}{3.136411in}}%
\pgfpathlineto{\pgfqpoint{4.410891in}{3.148359in}}%
\pgfpathlineto{\pgfqpoint{4.417437in}{3.176108in}}%
\pgfpathlineto{\pgfqpoint{4.426166in}{3.228227in}}%
\pgfpathlineto{\pgfqpoint{4.437077in}{3.309566in}}%
\pgfpathlineto{\pgfqpoint{4.513453in}{3.914805in}}%
\pgfpathlineto{\pgfqpoint{4.561460in}{4.290669in}}%
\pgfpathlineto{\pgfqpoint{4.568007in}{4.323786in}}%
\pgfpathlineto{\pgfqpoint{4.574553in}{4.345555in}}%
\pgfpathlineto{\pgfqpoint{4.578917in}{4.352822in}}%
\pgfpathlineto{\pgfqpoint{4.581100in}{4.354146in}}%
\pgfpathlineto{\pgfqpoint{4.583282in}{4.353903in}}%
\pgfpathlineto{\pgfqpoint{4.585464in}{4.352092in}}%
\pgfpathlineto{\pgfqpoint{4.589828in}{4.343836in}}%
\pgfpathlineto{\pgfqpoint{4.594193in}{4.329669in}}%
\pgfpathlineto{\pgfqpoint{4.600739in}{4.298494in}}%
\pgfpathlineto{\pgfqpoint{4.609468in}{4.242220in}}%
\pgfpathlineto{\pgfqpoint{4.622561in}{4.139388in}}%
\pgfpathlineto{\pgfqpoint{4.655293in}{3.876921in}}%
\pgfpathlineto{\pgfqpoint{4.679297in}{3.707484in}}%
\pgfpathlineto{\pgfqpoint{4.707665in}{3.504740in}}%
\pgfpathlineto{\pgfqpoint{4.729487in}{3.329033in}}%
\pgfpathlineto{\pgfqpoint{4.749126in}{3.172204in}}%
\pgfpathlineto{\pgfqpoint{4.757855in}{3.116699in}}%
\pgfpathlineto{\pgfqpoint{4.764401in}{3.086518in}}%
\pgfpathlineto{\pgfqpoint{4.768766in}{3.073296in}}%
\pgfpathlineto{\pgfqpoint{4.773130in}{3.066264in}}%
\pgfpathlineto{\pgfqpoint{4.775312in}{3.065179in}}%
\pgfpathlineto{\pgfqpoint{4.777494in}{3.065741in}}%
\pgfpathlineto{\pgfqpoint{4.779676in}{3.067950in}}%
\pgfpathlineto{\pgfqpoint{4.784041in}{3.077208in}}%
\pgfpathlineto{\pgfqpoint{4.788405in}{3.092596in}}%
\pgfpathlineto{\pgfqpoint{4.794952in}{3.125821in}}%
\pgfpathlineto{\pgfqpoint{4.803680in}{3.184716in}}%
\pgfpathlineto{\pgfqpoint{4.818955in}{3.308048in}}%
\pgfpathlineto{\pgfqpoint{4.840777in}{3.481837in}}%
\pgfpathlineto{\pgfqpoint{4.858234in}{3.601139in}}%
\pgfpathlineto{\pgfqpoint{4.880056in}{3.731848in}}%
\pgfpathlineto{\pgfqpoint{4.906242in}{3.889112in}}%
\pgfpathlineto{\pgfqpoint{4.921517in}{3.994183in}}%
\pgfpathlineto{\pgfqpoint{4.938974in}{4.131182in}}%
\pgfpathlineto{\pgfqpoint{4.962978in}{4.323206in}}%
\pgfpathlineto{\pgfqpoint{4.971707in}{4.376000in}}%
\pgfpathlineto{\pgfqpoint{4.978253in}{4.402640in}}%
\pgfpathlineto{\pgfqpoint{4.982617in}{4.412773in}}%
\pgfpathlineto{\pgfqpoint{4.986982in}{4.416218in}}%
\pgfpathlineto{\pgfqpoint{4.989164in}{4.415362in}}%
\pgfpathlineto{\pgfqpoint{4.993528in}{4.408520in}}%
\pgfpathlineto{\pgfqpoint{4.997893in}{4.395101in}}%
\pgfpathlineto{\pgfqpoint{5.004439in}{4.363931in}}%
\pgfpathlineto{\pgfqpoint{5.013168in}{4.306286in}}%
\pgfpathlineto{\pgfqpoint{5.028443in}{4.183099in}}%
\pgfpathlineto{\pgfqpoint{5.048082in}{4.027844in}}%
\pgfpathlineto{\pgfqpoint{5.061175in}{3.940476in}}%
\pgfpathlineto{\pgfqpoint{5.074268in}{3.866855in}}%
\pgfpathlineto{\pgfqpoint{5.089543in}{3.794238in}}%
\pgfpathlineto{\pgfqpoint{5.113547in}{3.693978in}}%
\pgfpathlineto{\pgfqpoint{5.133187in}{3.608722in}}%
\pgfpathlineto{\pgfqpoint{5.146280in}{3.542827in}}%
\pgfpathlineto{\pgfqpoint{5.159373in}{3.465352in}}%
\pgfpathlineto{\pgfqpoint{5.172466in}{3.374027in}}%
\pgfpathlineto{\pgfqpoint{5.189923in}{3.233790in}}%
\pgfpathlineto{\pgfqpoint{5.207380in}{3.095737in}}%
\pgfpathlineto{\pgfqpoint{5.216109in}{3.043395in}}%
\pgfpathlineto{\pgfqpoint{5.222655in}{3.017903in}}%
\pgfpathlineto{\pgfqpoint{5.227020in}{3.008984in}}%
\pgfpathlineto{\pgfqpoint{5.229202in}{3.007143in}}%
\pgfpathlineto{\pgfqpoint{5.231384in}{3.007091in}}%
\pgfpathlineto{\pgfqpoint{5.233566in}{3.008831in}}%
\pgfpathlineto{\pgfqpoint{5.237930in}{3.017604in}}%
\pgfpathlineto{\pgfqpoint{5.242295in}{3.033075in}}%
\pgfpathlineto{\pgfqpoint{5.248841in}{3.067270in}}%
\pgfpathlineto{\pgfqpoint{5.257570in}{3.128207in}}%
\pgfpathlineto{\pgfqpoint{5.299031in}{3.443757in}}%
\pgfpathlineto{\pgfqpoint{5.309942in}{3.502606in}}%
\pgfpathlineto{\pgfqpoint{5.320853in}{3.549597in}}%
\pgfpathlineto{\pgfqpoint{5.331763in}{3.586338in}}%
\pgfpathlineto{\pgfqpoint{5.342674in}{3.614538in}}%
\pgfpathlineto{\pgfqpoint{5.353585in}{3.635729in}}%
\pgfpathlineto{\pgfqpoint{5.362314in}{3.648489in}}%
\pgfpathlineto{\pgfqpoint{5.371042in}{3.658080in}}%
\pgfpathlineto{\pgfqpoint{5.379771in}{3.664897in}}%
\pgfpathlineto{\pgfqpoint{5.388500in}{3.669230in}}%
\pgfpathlineto{\pgfqpoint{5.397228in}{3.671270in}}%
\pgfpathlineto{\pgfqpoint{5.405957in}{3.671120in}}%
\pgfpathlineto{\pgfqpoint{5.414686in}{3.668797in}}%
\pgfpathlineto{\pgfqpoint{5.423414in}{3.664231in}}%
\pgfpathlineto{\pgfqpoint{5.432143in}{3.657268in}}%
\pgfpathlineto{\pgfqpoint{5.440872in}{3.647662in}}%
\pgfpathlineto{\pgfqpoint{5.449600in}{3.635074in}}%
\pgfpathlineto{\pgfqpoint{5.458329in}{3.619064in}}%
\pgfpathlineto{\pgfqpoint{5.467058in}{3.599086in}}%
\pgfpathlineto{\pgfqpoint{5.475786in}{3.574487in}}%
\pgfpathlineto{\pgfqpoint{5.486697in}{3.536100in}}%
\pgfpathlineto{\pgfqpoint{5.497608in}{3.487755in}}%
\pgfpathlineto{\pgfqpoint{5.508519in}{3.428009in}}%
\pgfpathlineto{\pgfqpoint{5.519429in}{3.356103in}}%
\pgfpathlineto{\pgfqpoint{5.534705in}{3.237423in}}%
\pgfpathlineto{\pgfqpoint{5.556526in}{3.063807in}}%
\pgfpathlineto{\pgfqpoint{5.565255in}{3.013430in}}%
\pgfpathlineto{\pgfqpoint{5.571801in}{2.990515in}}%
\pgfpathlineto{\pgfqpoint{5.576166in}{2.983817in}}%
\pgfpathlineto{\pgfqpoint{5.578348in}{2.983212in}}%
\pgfpathlineto{\pgfqpoint{5.580530in}{2.984460in}}%
\pgfpathlineto{\pgfqpoint{5.584894in}{2.992446in}}%
\pgfpathlineto{\pgfqpoint{5.589259in}{3.007406in}}%
\pgfpathlineto{\pgfqpoint{5.595805in}{3.041310in}}%
\pgfpathlineto{\pgfqpoint{5.604534in}{3.102449in}}%
\pgfpathlineto{\pgfqpoint{5.641631in}{3.386424in}}%
\pgfpathlineto{\pgfqpoint{5.652541in}{3.444602in}}%
\pgfpathlineto{\pgfqpoint{5.661270in}{3.480254in}}%
\pgfpathlineto{\pgfqpoint{5.669999in}{3.506742in}}%
\pgfpathlineto{\pgfqpoint{5.676545in}{3.520955in}}%
\pgfpathlineto{\pgfqpoint{5.683092in}{3.530579in}}%
\pgfpathlineto{\pgfqpoint{5.689638in}{3.535792in}}%
\pgfpathlineto{\pgfqpoint{5.694002in}{3.536876in}}%
\pgfpathlineto{\pgfqpoint{5.698367in}{3.536070in}}%
\pgfpathlineto{\pgfqpoint{5.702731in}{3.533377in}}%
\pgfpathlineto{\pgfqpoint{5.709278in}{3.525773in}}%
\pgfpathlineto{\pgfqpoint{5.715824in}{3.513810in}}%
\pgfpathlineto{\pgfqpoint{5.722371in}{3.497340in}}%
\pgfpathlineto{\pgfqpoint{5.731099in}{3.468039in}}%
\pgfpathlineto{\pgfqpoint{5.739828in}{3.429900in}}%
\pgfpathlineto{\pgfqpoint{5.748556in}{3.382549in}}%
\pgfpathlineto{\pgfqpoint{5.759467in}{3.310522in}}%
\pgfpathlineto{\pgfqpoint{5.772560in}{3.208401in}}%
\pgfpathlineto{\pgfqpoint{5.794382in}{3.035740in}}%
\pgfpathlineto{\pgfqpoint{5.800928in}{2.997551in}}%
\pgfpathlineto{\pgfqpoint{5.807475in}{2.972297in}}%
\pgfpathlineto{\pgfqpoint{5.811839in}{2.964102in}}%
\pgfpathlineto{\pgfqpoint{5.814021in}{2.962801in}}%
\pgfpathlineto{\pgfqpoint{5.816203in}{2.963403in}}%
\pgfpathlineto{\pgfqpoint{5.818386in}{2.965904in}}%
\pgfpathlineto{\pgfqpoint{5.822750in}{2.976463in}}%
\pgfpathlineto{\pgfqpoint{5.827114in}{2.993958in}}%
\pgfpathlineto{\pgfqpoint{5.833661in}{3.031267in}}%
\pgfpathlineto{\pgfqpoint{5.842389in}{3.095577in}}%
\pgfpathlineto{\pgfqpoint{5.870758in}{3.318286in}}%
\pgfpathlineto{\pgfqpoint{5.881668in}{3.382058in}}%
\pgfpathlineto{\pgfqpoint{5.890397in}{3.420477in}}%
\pgfpathlineto{\pgfqpoint{5.899126in}{3.447717in}}%
\pgfpathlineto{\pgfqpoint{5.905672in}{3.461021in}}%
\pgfpathlineto{\pgfqpoint{5.910036in}{3.466587in}}%
\pgfpathlineto{\pgfqpoint{5.914401in}{3.469556in}}%
\pgfpathlineto{\pgfqpoint{5.918765in}{3.469956in}}%
\pgfpathlineto{\pgfqpoint{5.923129in}{3.467801in}}%
\pgfpathlineto{\pgfqpoint{5.927494in}{3.463094in}}%
\pgfpathlineto{\pgfqpoint{5.934040in}{3.451221in}}%
\pgfpathlineto{\pgfqpoint{5.940587in}{3.433504in}}%
\pgfpathlineto{\pgfqpoint{5.947133in}{3.409843in}}%
\pgfpathlineto{\pgfqpoint{5.955862in}{3.368921in}}%
\pgfpathlineto{\pgfqpoint{5.964591in}{3.317388in}}%
\pgfpathlineto{\pgfqpoint{5.975501in}{3.239393in}}%
\pgfpathlineto{\pgfqpoint{6.008234in}{2.987887in}}%
\pgfpathlineto{\pgfqpoint{6.014780in}{2.958697in}}%
\pgfpathlineto{\pgfqpoint{6.019145in}{2.947712in}}%
\pgfpathlineto{\pgfqpoint{6.021327in}{2.945027in}}%
\pgfpathlineto{\pgfqpoint{6.023509in}{2.944276in}}%
\pgfpathlineto{\pgfqpoint{6.025691in}{2.945476in}}%
\pgfpathlineto{\pgfqpoint{6.030055in}{2.953651in}}%
\pgfpathlineto{\pgfqpoint{6.034420in}{2.969139in}}%
\pgfpathlineto{\pgfqpoint{6.040966in}{3.004262in}}%
\pgfpathlineto{\pgfqpoint{6.049695in}{3.067128in}}%
\pgfpathlineto{\pgfqpoint{6.080245in}{3.303150in}}%
\pgfpathlineto{\pgfqpoint{6.088974in}{3.351184in}}%
\pgfpathlineto{\pgfqpoint{6.097702in}{3.386672in}}%
\pgfpathlineto{\pgfqpoint{6.104249in}{3.404958in}}%
\pgfpathlineto{\pgfqpoint{6.110795in}{3.416186in}}%
\pgfpathlineto{\pgfqpoint{6.115160in}{3.419800in}}%
\pgfpathlineto{\pgfqpoint{6.119524in}{3.420347in}}%
\pgfpathlineto{\pgfqpoint{6.123888in}{3.417843in}}%
\pgfpathlineto{\pgfqpoint{6.128253in}{3.412298in}}%
\pgfpathlineto{\pgfqpoint{6.132617in}{3.403711in}}%
\pgfpathlineto{\pgfqpoint{6.139164in}{3.385115in}}%
\pgfpathlineto{\pgfqpoint{6.145710in}{3.359660in}}%
\pgfpathlineto{\pgfqpoint{6.154439in}{3.315175in}}%
\pgfpathlineto{\pgfqpoint{6.158803in}{3.288576in}}%
\pgfpathlineto{\pgfqpoint{6.158803in}{3.288576in}}%
\pgfusepath{stroke}%
\end{pgfscope}%
\begin{pgfscope}%
\pgfsetrectcap%
\pgfsetmiterjoin%
\pgfsetlinewidth{0.803000pt}%
\definecolor{currentstroke}{rgb}{0.000000,0.000000,0.000000}%
\pgfsetstrokecolor{currentstroke}%
\pgfsetdash{}{0pt}%
\pgfpathmoveto{\pgfqpoint{3.867533in}{2.870679in}}%
\pgfpathlineto{\pgfqpoint{3.867533in}{4.489815in}}%
\pgfusepath{stroke}%
\end{pgfscope}%
\begin{pgfscope}%
\pgfsetrectcap%
\pgfsetmiterjoin%
\pgfsetlinewidth{0.803000pt}%
\definecolor{currentstroke}{rgb}{0.000000,0.000000,0.000000}%
\pgfsetstrokecolor{currentstroke}%
\pgfsetdash{}{0pt}%
\pgfpathmoveto{\pgfqpoint{6.267911in}{2.870679in}}%
\pgfpathlineto{\pgfqpoint{6.267911in}{4.489815in}}%
\pgfusepath{stroke}%
\end{pgfscope}%
\begin{pgfscope}%
\pgfsetrectcap%
\pgfsetmiterjoin%
\pgfsetlinewidth{0.803000pt}%
\definecolor{currentstroke}{rgb}{0.000000,0.000000,0.000000}%
\pgfsetstrokecolor{currentstroke}%
\pgfsetdash{}{0pt}%
\pgfpathmoveto{\pgfqpoint{3.867533in}{2.870679in}}%
\pgfpathlineto{\pgfqpoint{6.267911in}{2.870679in}}%
\pgfusepath{stroke}%
\end{pgfscope}%
\begin{pgfscope}%
\pgfsetrectcap%
\pgfsetmiterjoin%
\pgfsetlinewidth{0.803000pt}%
\definecolor{currentstroke}{rgb}{0.000000,0.000000,0.000000}%
\pgfsetstrokecolor{currentstroke}%
\pgfsetdash{}{0pt}%
\pgfpathmoveto{\pgfqpoint{3.867533in}{4.489815in}}%
\pgfpathlineto{\pgfqpoint{6.267911in}{4.489815in}}%
\pgfusepath{stroke}%
\end{pgfscope}%
\begin{pgfscope}%
\definecolor{textcolor}{rgb}{0.000000,0.000000,0.000000}%
\pgfsetstrokecolor{textcolor}%
\pgfsetfillcolor{textcolor}%
\pgftext[x=5.067722in,y=4.573148in,,base]{\color{textcolor}\rmfamily\fontsize{12.000000}{14.400000}\selectfont \(\displaystyle \omega\)}%
\end{pgfscope}%
\begin{pgfscope}%
\pgfsetbuttcap%
\pgfsetmiterjoin%
\definecolor{currentfill}{rgb}{1.000000,1.000000,1.000000}%
\pgfsetfillcolor{currentfill}%
\pgfsetlinewidth{0.000000pt}%
\definecolor{currentstroke}{rgb}{0.000000,0.000000,0.000000}%
\pgfsetstrokecolor{currentstroke}%
\pgfsetstrokeopacity{0.000000}%
\pgfsetdash{}{0pt}%
\pgfpathmoveto{\pgfqpoint{0.796484in}{0.526234in}}%
\pgfpathlineto{\pgfqpoint{3.196863in}{0.526234in}}%
\pgfpathlineto{\pgfqpoint{3.196863in}{2.145371in}}%
\pgfpathlineto{\pgfqpoint{0.796484in}{2.145371in}}%
\pgfpathclose%
\pgfusepath{fill}%
\end{pgfscope}%
\begin{pgfscope}%
\pgfsetbuttcap%
\pgfsetroundjoin%
\definecolor{currentfill}{rgb}{0.000000,0.000000,0.000000}%
\pgfsetfillcolor{currentfill}%
\pgfsetlinewidth{0.803000pt}%
\definecolor{currentstroke}{rgb}{0.000000,0.000000,0.000000}%
\pgfsetstrokecolor{currentstroke}%
\pgfsetdash{}{0pt}%
\pgfsys@defobject{currentmarker}{\pgfqpoint{0.000000in}{-0.048611in}}{\pgfqpoint{0.000000in}{0.000000in}}{%
\pgfpathmoveto{\pgfqpoint{0.000000in}{0.000000in}}%
\pgfpathlineto{\pgfqpoint{0.000000in}{-0.048611in}}%
\pgfusepath{stroke,fill}%
}%
\begin{pgfscope}%
\pgfsys@transformshift{1.153235in}{0.526234in}%
\pgfsys@useobject{currentmarker}{}%
\end{pgfscope}%
\end{pgfscope}%
\begin{pgfscope}%
\definecolor{textcolor}{rgb}{0.000000,0.000000,0.000000}%
\pgfsetstrokecolor{textcolor}%
\pgfsetfillcolor{textcolor}%
\pgftext[x=1.153235in,y=0.429012in,,top]{\color{textcolor}\rmfamily\fontsize{10.000000}{12.000000}\selectfont \(\displaystyle -20\)}%
\end{pgfscope}%
\begin{pgfscope}%
\pgfsetbuttcap%
\pgfsetroundjoin%
\definecolor{currentfill}{rgb}{0.000000,0.000000,0.000000}%
\pgfsetfillcolor{currentfill}%
\pgfsetlinewidth{0.803000pt}%
\definecolor{currentstroke}{rgb}{0.000000,0.000000,0.000000}%
\pgfsetstrokecolor{currentstroke}%
\pgfsetdash{}{0pt}%
\pgfsys@defobject{currentmarker}{\pgfqpoint{0.000000in}{-0.048611in}}{\pgfqpoint{0.000000in}{0.000000in}}{%
\pgfpathmoveto{\pgfqpoint{0.000000in}{0.000000in}}%
\pgfpathlineto{\pgfqpoint{0.000000in}{-0.048611in}}%
\pgfusepath{stroke,fill}%
}%
\begin{pgfscope}%
\pgfsys@transformshift{2.008121in}{0.526234in}%
\pgfsys@useobject{currentmarker}{}%
\end{pgfscope}%
\end{pgfscope}%
\begin{pgfscope}%
\definecolor{textcolor}{rgb}{0.000000,0.000000,0.000000}%
\pgfsetstrokecolor{textcolor}%
\pgfsetfillcolor{textcolor}%
\pgftext[x=2.008121in,y=0.429012in,,top]{\color{textcolor}\rmfamily\fontsize{10.000000}{12.000000}\selectfont \(\displaystyle -10\)}%
\end{pgfscope}%
\begin{pgfscope}%
\pgfsetbuttcap%
\pgfsetroundjoin%
\definecolor{currentfill}{rgb}{0.000000,0.000000,0.000000}%
\pgfsetfillcolor{currentfill}%
\pgfsetlinewidth{0.803000pt}%
\definecolor{currentstroke}{rgb}{0.000000,0.000000,0.000000}%
\pgfsetstrokecolor{currentstroke}%
\pgfsetdash{}{0pt}%
\pgfsys@defobject{currentmarker}{\pgfqpoint{0.000000in}{-0.048611in}}{\pgfqpoint{0.000000in}{0.000000in}}{%
\pgfpathmoveto{\pgfqpoint{0.000000in}{0.000000in}}%
\pgfpathlineto{\pgfqpoint{0.000000in}{-0.048611in}}%
\pgfusepath{stroke,fill}%
}%
\begin{pgfscope}%
\pgfsys@transformshift{2.863007in}{0.526234in}%
\pgfsys@useobject{currentmarker}{}%
\end{pgfscope}%
\end{pgfscope}%
\begin{pgfscope}%
\definecolor{textcolor}{rgb}{0.000000,0.000000,0.000000}%
\pgfsetstrokecolor{textcolor}%
\pgfsetfillcolor{textcolor}%
\pgftext[x=2.863007in,y=0.429012in,,top]{\color{textcolor}\rmfamily\fontsize{10.000000}{12.000000}\selectfont \(\displaystyle 0\)}%
\end{pgfscope}%
\begin{pgfscope}%
\definecolor{textcolor}{rgb}{0.000000,0.000000,0.000000}%
\pgfsetstrokecolor{textcolor}%
\pgfsetfillcolor{textcolor}%
\pgftext[x=1.996673in,y=0.250000in,,top]{\color{textcolor}\rmfamily\fontsize{10.000000}{12.000000}\selectfont angle (rad)}%
\end{pgfscope}%
\begin{pgfscope}%
\pgfsetbuttcap%
\pgfsetroundjoin%
\definecolor{currentfill}{rgb}{0.000000,0.000000,0.000000}%
\pgfsetfillcolor{currentfill}%
\pgfsetlinewidth{0.803000pt}%
\definecolor{currentstroke}{rgb}{0.000000,0.000000,0.000000}%
\pgfsetstrokecolor{currentstroke}%
\pgfsetdash{}{0pt}%
\pgfsys@defobject{currentmarker}{\pgfqpoint{-0.048611in}{0.000000in}}{\pgfqpoint{0.000000in}{0.000000in}}{%
\pgfpathmoveto{\pgfqpoint{0.000000in}{0.000000in}}%
\pgfpathlineto{\pgfqpoint{-0.048611in}{0.000000in}}%
\pgfusepath{stroke,fill}%
}%
\begin{pgfscope}%
\pgfsys@transformshift{0.796484in}{0.652773in}%
\pgfsys@useobject{currentmarker}{}%
\end{pgfscope}%
\end{pgfscope}%
\begin{pgfscope}%
\definecolor{textcolor}{rgb}{0.000000,0.000000,0.000000}%
\pgfsetstrokecolor{textcolor}%
\pgfsetfillcolor{textcolor}%
\pgftext[x=0.452348in,y=0.604547in,left,base]{\color{textcolor}\rmfamily\fontsize{10.000000}{12.000000}\selectfont \(\displaystyle -10\)}%
\end{pgfscope}%
\begin{pgfscope}%
\pgfsetbuttcap%
\pgfsetroundjoin%
\definecolor{currentfill}{rgb}{0.000000,0.000000,0.000000}%
\pgfsetfillcolor{currentfill}%
\pgfsetlinewidth{0.803000pt}%
\definecolor{currentstroke}{rgb}{0.000000,0.000000,0.000000}%
\pgfsetstrokecolor{currentstroke}%
\pgfsetdash{}{0pt}%
\pgfsys@defobject{currentmarker}{\pgfqpoint{-0.048611in}{0.000000in}}{\pgfqpoint{0.000000in}{0.000000in}}{%
\pgfpathmoveto{\pgfqpoint{0.000000in}{0.000000in}}%
\pgfpathlineto{\pgfqpoint{-0.048611in}{0.000000in}}%
\pgfusepath{stroke,fill}%
}%
\begin{pgfscope}%
\pgfsys@transformshift{0.796484in}{1.016937in}%
\pgfsys@useobject{currentmarker}{}%
\end{pgfscope}%
\end{pgfscope}%
\begin{pgfscope}%
\definecolor{textcolor}{rgb}{0.000000,0.000000,0.000000}%
\pgfsetstrokecolor{textcolor}%
\pgfsetfillcolor{textcolor}%
\pgftext[x=0.521792in,y=0.968712in,left,base]{\color{textcolor}\rmfamily\fontsize{10.000000}{12.000000}\selectfont \(\displaystyle -5\)}%
\end{pgfscope}%
\begin{pgfscope}%
\pgfsetbuttcap%
\pgfsetroundjoin%
\definecolor{currentfill}{rgb}{0.000000,0.000000,0.000000}%
\pgfsetfillcolor{currentfill}%
\pgfsetlinewidth{0.803000pt}%
\definecolor{currentstroke}{rgb}{0.000000,0.000000,0.000000}%
\pgfsetstrokecolor{currentstroke}%
\pgfsetdash{}{0pt}%
\pgfsys@defobject{currentmarker}{\pgfqpoint{-0.048611in}{0.000000in}}{\pgfqpoint{0.000000in}{0.000000in}}{%
\pgfpathmoveto{\pgfqpoint{0.000000in}{0.000000in}}%
\pgfpathlineto{\pgfqpoint{-0.048611in}{0.000000in}}%
\pgfusepath{stroke,fill}%
}%
\begin{pgfscope}%
\pgfsys@transformshift{0.796484in}{1.381102in}%
\pgfsys@useobject{currentmarker}{}%
\end{pgfscope}%
\end{pgfscope}%
\begin{pgfscope}%
\definecolor{textcolor}{rgb}{0.000000,0.000000,0.000000}%
\pgfsetstrokecolor{textcolor}%
\pgfsetfillcolor{textcolor}%
\pgftext[x=0.629817in,y=1.332876in,left,base]{\color{textcolor}\rmfamily\fontsize{10.000000}{12.000000}\selectfont \(\displaystyle 0\)}%
\end{pgfscope}%
\begin{pgfscope}%
\pgfsetbuttcap%
\pgfsetroundjoin%
\definecolor{currentfill}{rgb}{0.000000,0.000000,0.000000}%
\pgfsetfillcolor{currentfill}%
\pgfsetlinewidth{0.803000pt}%
\definecolor{currentstroke}{rgb}{0.000000,0.000000,0.000000}%
\pgfsetstrokecolor{currentstroke}%
\pgfsetdash{}{0pt}%
\pgfsys@defobject{currentmarker}{\pgfqpoint{-0.048611in}{0.000000in}}{\pgfqpoint{0.000000in}{0.000000in}}{%
\pgfpathmoveto{\pgfqpoint{0.000000in}{0.000000in}}%
\pgfpathlineto{\pgfqpoint{-0.048611in}{0.000000in}}%
\pgfusepath{stroke,fill}%
}%
\begin{pgfscope}%
\pgfsys@transformshift{0.796484in}{1.745266in}%
\pgfsys@useobject{currentmarker}{}%
\end{pgfscope}%
\end{pgfscope}%
\begin{pgfscope}%
\definecolor{textcolor}{rgb}{0.000000,0.000000,0.000000}%
\pgfsetstrokecolor{textcolor}%
\pgfsetfillcolor{textcolor}%
\pgftext[x=0.629817in,y=1.697041in,left,base]{\color{textcolor}\rmfamily\fontsize{10.000000}{12.000000}\selectfont \(\displaystyle 5\)}%
\end{pgfscope}%
\begin{pgfscope}%
\pgfsetbuttcap%
\pgfsetroundjoin%
\definecolor{currentfill}{rgb}{0.000000,0.000000,0.000000}%
\pgfsetfillcolor{currentfill}%
\pgfsetlinewidth{0.803000pt}%
\definecolor{currentstroke}{rgb}{0.000000,0.000000,0.000000}%
\pgfsetstrokecolor{currentstroke}%
\pgfsetdash{}{0pt}%
\pgfsys@defobject{currentmarker}{\pgfqpoint{-0.048611in}{0.000000in}}{\pgfqpoint{0.000000in}{0.000000in}}{%
\pgfpathmoveto{\pgfqpoint{0.000000in}{0.000000in}}%
\pgfpathlineto{\pgfqpoint{-0.048611in}{0.000000in}}%
\pgfusepath{stroke,fill}%
}%
\begin{pgfscope}%
\pgfsys@transformshift{0.796484in}{2.109431in}%
\pgfsys@useobject{currentmarker}{}%
\end{pgfscope}%
\end{pgfscope}%
\begin{pgfscope}%
\definecolor{textcolor}{rgb}{0.000000,0.000000,0.000000}%
\pgfsetstrokecolor{textcolor}%
\pgfsetfillcolor{textcolor}%
\pgftext[x=0.560373in,y=2.061205in,left,base]{\color{textcolor}\rmfamily\fontsize{10.000000}{12.000000}\selectfont \(\displaystyle 10\)}%
\end{pgfscope}%
\begin{pgfscope}%
\definecolor{textcolor}{rgb}{0.000000,0.000000,0.000000}%
\pgfsetstrokecolor{textcolor}%
\pgfsetfillcolor{textcolor}%
\pgftext[x=0.396792in,y=1.335803in,,bottom,rotate=90.000000]{\color{textcolor}\rmfamily\fontsize{10.000000}{12.000000}\selectfont velocity (\(\displaystyle \frac{rad}{s}\))}%
\end{pgfscope}%
\begin{pgfscope}%
\pgfpathrectangle{\pgfqpoint{0.796484in}{0.526234in}}{\pgfqpoint{2.400379in}{1.619136in}}%
\pgfusepath{clip}%
\pgfsetrectcap%
\pgfsetroundjoin%
\pgfsetlinewidth{1.505625pt}%
\definecolor{currentstroke}{rgb}{0.000000,0.000000,1.000000}%
\pgfsetstrokecolor{currentstroke}%
\pgfsetdash{}{0pt}%
\pgfpathmoveto{\pgfqpoint{2.997292in}{1.381102in}}%
\pgfpathlineto{\pgfqpoint{2.996651in}{1.326477in}}%
\pgfpathlineto{\pgfqpoint{2.994086in}{1.271861in}}%
\pgfpathlineto{\pgfqpoint{2.989599in}{1.217318in}}%
\pgfpathlineto{\pgfqpoint{2.983194in}{1.163054in}}%
\pgfpathlineto{\pgfqpoint{2.974884in}{1.109489in}}%
\pgfpathlineto{\pgfqpoint{2.964703in}{1.057334in}}%
\pgfpathlineto{\pgfqpoint{2.952711in}{1.007635in}}%
\pgfpathlineto{\pgfqpoint{2.939010in}{0.961792in}}%
\pgfpathlineto{\pgfqpoint{2.929000in}{0.934199in}}%
\pgfpathlineto{\pgfqpoint{2.918360in}{0.909621in}}%
\pgfpathlineto{\pgfqpoint{2.907163in}{0.888618in}}%
\pgfpathlineto{\pgfqpoint{2.895496in}{0.871724in}}%
\pgfpathlineto{\pgfqpoint{2.883459in}{0.859413in}}%
\pgfpathlineto{\pgfqpoint{2.871162in}{0.852061in}}%
\pgfpathlineto{\pgfqpoint{2.864952in}{0.850326in}}%
\pgfpathlineto{\pgfqpoint{2.858722in}{0.849912in}}%
\pgfpathlineto{\pgfqpoint{2.852487in}{0.850824in}}%
\pgfpathlineto{\pgfqpoint{2.846263in}{0.853059in}}%
\pgfpathlineto{\pgfqpoint{2.840065in}{0.856603in}}%
\pgfpathlineto{\pgfqpoint{2.827809in}{0.867509in}}%
\pgfpathlineto{\pgfqpoint{2.815838in}{0.883241in}}%
\pgfpathlineto{\pgfqpoint{2.804262in}{0.903366in}}%
\pgfpathlineto{\pgfqpoint{2.793183in}{0.927365in}}%
\pgfpathlineto{\pgfqpoint{2.782687in}{0.954660in}}%
\pgfpathlineto{\pgfqpoint{2.768196in}{1.000498in}}%
\pgfpathlineto{\pgfqpoint{2.755372in}{1.050547in}}%
\pgfpathlineto{\pgfqpoint{2.744345in}{1.103200in}}%
\pgfpathlineto{\pgfqpoint{2.735192in}{1.157223in}}%
\pgfpathlineto{\pgfqpoint{2.727948in}{1.211775in}}%
\pgfpathlineto{\pgfqpoint{2.722626in}{1.266364in}}%
\pgfpathlineto{\pgfqpoint{2.719225in}{1.320782in}}%
\pgfpathlineto{\pgfqpoint{2.717738in}{1.375022in}}%
\pgfpathlineto{\pgfqpoint{2.718160in}{1.429197in}}%
\pgfpathlineto{\pgfqpoint{2.720490in}{1.483452in}}%
\pgfpathlineto{\pgfqpoint{2.724733in}{1.537892in}}%
\pgfpathlineto{\pgfqpoint{2.730894in}{1.592491in}}%
\pgfpathlineto{\pgfqpoint{2.738979in}{1.647015in}}%
\pgfpathlineto{\pgfqpoint{2.748978in}{1.700940in}}%
\pgfpathlineto{\pgfqpoint{2.760861in}{1.753376in}}%
\pgfpathlineto{\pgfqpoint{2.774562in}{1.803032in}}%
\pgfpathlineto{\pgfqpoint{2.789964in}{1.848221in}}%
\pgfpathlineto{\pgfqpoint{2.801091in}{1.874892in}}%
\pgfpathlineto{\pgfqpoint{2.812824in}{1.898069in}}%
\pgfpathlineto{\pgfqpoint{2.825079in}{1.917138in}}%
\pgfpathlineto{\pgfqpoint{2.837754in}{1.931540in}}%
\pgfpathlineto{\pgfqpoint{2.850738in}{1.940811in}}%
\pgfpathlineto{\pgfqpoint{2.857308in}{1.943415in}}%
\pgfpathlineto{\pgfqpoint{2.863908in}{1.944628in}}%
\pgfpathlineto{\pgfqpoint{2.870522in}{1.944436in}}%
\pgfpathlineto{\pgfqpoint{2.877135in}{1.942837in}}%
\pgfpathlineto{\pgfqpoint{2.883728in}{1.939842in}}%
\pgfpathlineto{\pgfqpoint{2.890286in}{1.935472in}}%
\pgfpathlineto{\pgfqpoint{2.903233in}{1.922749in}}%
\pgfpathlineto{\pgfqpoint{2.915852in}{1.905056in}}%
\pgfpathlineto{\pgfqpoint{2.928028in}{1.882912in}}%
\pgfpathlineto{\pgfqpoint{2.939661in}{1.856931in}}%
\pgfpathlineto{\pgfqpoint{2.950664in}{1.827769in}}%
\pgfpathlineto{\pgfqpoint{2.965838in}{1.779498in}}%
\pgfpathlineto{\pgfqpoint{2.979264in}{1.727594in}}%
\pgfpathlineto{\pgfqpoint{2.990836in}{1.673742in}}%
\pgfpathlineto{\pgfqpoint{3.000501in}{1.619180in}}%
\pgfpathlineto{\pgfqpoint{3.008245in}{1.564705in}}%
\pgfpathlineto{\pgfqpoint{3.014075in}{1.510725in}}%
\pgfpathlineto{\pgfqpoint{3.018011in}{1.457343in}}%
\pgfpathlineto{\pgfqpoint{3.020074in}{1.404445in}}%
\pgfpathlineto{\pgfqpoint{3.020277in}{1.351783in}}%
\pgfpathlineto{\pgfqpoint{3.018627in}{1.299052in}}%
\pgfpathlineto{\pgfqpoint{3.015116in}{1.245964in}}%
\pgfpathlineto{\pgfqpoint{3.009729in}{1.192320in}}%
\pgfpathlineto{\pgfqpoint{3.002446in}{1.138093in}}%
\pgfpathlineto{\pgfqpoint{2.993249in}{1.083509in}}%
\pgfpathlineto{\pgfqpoint{2.982130in}{1.029132in}}%
\pgfpathlineto{\pgfqpoint{2.969107in}{0.975947in}}%
\pgfpathlineto{\pgfqpoint{2.954238in}{0.925398in}}%
\pgfpathlineto{\pgfqpoint{2.937636in}{0.879381in}}%
\pgfpathlineto{\pgfqpoint{2.925695in}{0.852316in}}%
\pgfpathlineto{\pgfqpoint{2.913138in}{0.828955in}}%
\pgfpathlineto{\pgfqpoint{2.900058in}{0.809981in}}%
\pgfpathlineto{\pgfqpoint{2.886561in}{0.796005in}}%
\pgfpathlineto{\pgfqpoint{2.879693in}{0.791052in}}%
\pgfpathlineto{\pgfqpoint{2.872768in}{0.787520in}}%
\pgfpathlineto{\pgfqpoint{2.865800in}{0.785446in}}%
\pgfpathlineto{\pgfqpoint{2.858809in}{0.784851in}}%
\pgfpathlineto{\pgfqpoint{2.851810in}{0.785745in}}%
\pgfpathlineto{\pgfqpoint{2.844822in}{0.788122in}}%
\pgfpathlineto{\pgfqpoint{2.837862in}{0.791967in}}%
\pgfpathlineto{\pgfqpoint{2.830947in}{0.797245in}}%
\pgfpathlineto{\pgfqpoint{2.817319in}{0.811920in}}%
\pgfpathlineto{\pgfqpoint{2.804066in}{0.831663in}}%
\pgfpathlineto{\pgfqpoint{2.791304in}{0.855855in}}%
\pgfpathlineto{\pgfqpoint{2.779133in}{0.883783in}}%
\pgfpathlineto{\pgfqpoint{2.767636in}{0.914705in}}%
\pgfpathlineto{\pgfqpoint{2.751794in}{0.965122in}}%
\pgfpathlineto{\pgfqpoint{2.737767in}{1.018450in}}%
\pgfpathlineto{\pgfqpoint{2.725635in}{1.072943in}}%
\pgfpathlineto{\pgfqpoint{2.715424in}{1.127373in}}%
\pgfpathlineto{\pgfqpoint{2.707122in}{1.180999in}}%
\pgfpathlineto{\pgfqpoint{2.700695in}{1.233494in}}%
\pgfpathlineto{\pgfqpoint{2.696103in}{1.284851in}}%
\pgfpathlineto{\pgfqpoint{2.693308in}{1.335288in}}%
\pgfpathlineto{\pgfqpoint{2.692281in}{1.385169in}}%
\pgfpathlineto{\pgfqpoint{2.693008in}{1.434936in}}%
\pgfpathlineto{\pgfqpoint{2.695490in}{1.485040in}}%
\pgfpathlineto{\pgfqpoint{2.699745in}{1.535891in}}%
\pgfpathlineto{\pgfqpoint{2.705801in}{1.587787in}}%
\pgfpathlineto{\pgfqpoint{2.713699in}{1.640844in}}%
\pgfpathlineto{\pgfqpoint{2.723478in}{1.694916in}}%
\pgfpathlineto{\pgfqpoint{2.735169in}{1.749496in}}%
\pgfpathlineto{\pgfqpoint{2.748779in}{1.803634in}}%
\pgfpathlineto{\pgfqpoint{2.764278in}{1.855862in}}%
\pgfpathlineto{\pgfqpoint{2.781576in}{1.904196in}}%
\pgfpathlineto{\pgfqpoint{2.794030in}{1.933076in}}%
\pgfpathlineto{\pgfqpoint{2.807142in}{1.958381in}}%
\pgfpathlineto{\pgfqpoint{2.820824in}{1.979341in}}%
\pgfpathlineto{\pgfqpoint{2.834969in}{1.995245in}}%
\pgfpathlineto{\pgfqpoint{2.842178in}{2.001110in}}%
\pgfpathlineto{\pgfqpoint{2.849455in}{2.005503in}}%
\pgfpathlineto{\pgfqpoint{2.856784in}{2.008377in}}%
\pgfpathlineto{\pgfqpoint{2.864147in}{2.009702in}}%
\pgfpathlineto{\pgfqpoint{2.871525in}{2.009459in}}%
\pgfpathlineto{\pgfqpoint{2.878901in}{2.007648in}}%
\pgfpathlineto{\pgfqpoint{2.886255in}{2.004282in}}%
\pgfpathlineto{\pgfqpoint{2.893569in}{1.999391in}}%
\pgfpathlineto{\pgfqpoint{2.900827in}{1.993019in}}%
\pgfpathlineto{\pgfqpoint{2.915100in}{1.976076in}}%
\pgfpathlineto{\pgfqpoint{2.928945in}{1.954050in}}%
\pgfpathlineto{\pgfqpoint{2.942246in}{1.927678in}}%
\pgfpathlineto{\pgfqpoint{2.954906in}{1.897775in}}%
\pgfpathlineto{\pgfqpoint{2.972529in}{1.848104in}}%
\pgfpathlineto{\pgfqpoint{2.988356in}{1.794943in}}%
\pgfpathlineto{\pgfqpoint{3.002290in}{1.740439in}}%
\pgfpathlineto{\pgfqpoint{3.017885in}{1.668320in}}%
\pgfpathlineto{\pgfqpoint{3.030133in}{1.598748in}}%
\pgfpathlineto{\pgfqpoint{3.039173in}{1.532476in}}%
\pgfpathlineto{\pgfqpoint{3.045157in}{1.469244in}}%
\pgfpathlineto{\pgfqpoint{3.048214in}{1.408164in}}%
\pgfpathlineto{\pgfqpoint{3.048424in}{1.348018in}}%
\pgfpathlineto{\pgfqpoint{3.045809in}{1.287469in}}%
\pgfpathlineto{\pgfqpoint{3.040326in}{1.225239in}}%
\pgfpathlineto{\pgfqpoint{3.031876in}{1.160296in}}%
\pgfpathlineto{\pgfqpoint{3.020320in}{1.092094in}}%
\pgfpathlineto{\pgfqpoint{3.005507in}{1.020901in}}%
\pgfpathlineto{\pgfqpoint{2.992184in}{0.966381in}}%
\pgfpathlineto{\pgfqpoint{2.976942in}{0.912194in}}%
\pgfpathlineto{\pgfqpoint{2.959811in}{0.860137in}}%
\pgfpathlineto{\pgfqpoint{2.947387in}{0.827759in}}%
\pgfpathlineto{\pgfqpoint{2.934219in}{0.798179in}}%
\pgfpathlineto{\pgfqpoint{2.920377in}{0.772255in}}%
\pgfpathlineto{\pgfqpoint{2.905951in}{0.750840in}}%
\pgfpathlineto{\pgfqpoint{2.891053in}{0.734719in}}%
\pgfpathlineto{\pgfqpoint{2.883466in}{0.728852in}}%
\pgfpathlineto{\pgfqpoint{2.875810in}{0.724536in}}%
\pgfpathlineto{\pgfqpoint{2.868103in}{0.721819in}}%
\pgfpathlineto{\pgfqpoint{2.860365in}{0.720734in}}%
\pgfpathlineto{\pgfqpoint{2.852614in}{0.721297in}}%
\pgfpathlineto{\pgfqpoint{2.844869in}{0.723505in}}%
\pgfpathlineto{\pgfqpoint{2.837150in}{0.727340in}}%
\pgfpathlineto{\pgfqpoint{2.829477in}{0.732763in}}%
\pgfpathlineto{\pgfqpoint{2.821867in}{0.739723in}}%
\pgfpathlineto{\pgfqpoint{2.806909in}{0.757966in}}%
\pgfpathlineto{\pgfqpoint{2.792411in}{0.781376in}}%
\pgfpathlineto{\pgfqpoint{2.778490in}{0.809116in}}%
\pgfpathlineto{\pgfqpoint{2.765241in}{0.840272in}}%
\pgfpathlineto{\pgfqpoint{2.746786in}{0.891424in}}%
\pgfpathlineto{\pgfqpoint{2.730173in}{0.945398in}}%
\pgfpathlineto{\pgfqpoint{2.715471in}{0.999912in}}%
\pgfpathlineto{\pgfqpoint{2.698833in}{1.070700in}}%
\pgfpathlineto{\pgfqpoint{2.685450in}{1.137393in}}%
\pgfpathlineto{\pgfqpoint{2.675115in}{1.199287in}}%
\pgfpathlineto{\pgfqpoint{2.667605in}{1.256694in}}%
\pgfpathlineto{\pgfqpoint{2.662723in}{1.310537in}}%
\pgfpathlineto{\pgfqpoint{2.660323in}{1.362069in}}%
\pgfpathlineto{\pgfqpoint{2.660321in}{1.412693in}}%
\pgfpathlineto{\pgfqpoint{2.662701in}{1.463858in}}%
\pgfpathlineto{\pgfqpoint{2.667512in}{1.516973in}}%
\pgfpathlineto{\pgfqpoint{2.674869in}{1.573309in}}%
\pgfpathlineto{\pgfqpoint{2.684943in}{1.633866in}}%
\pgfpathlineto{\pgfqpoint{2.697942in}{1.699142in}}%
\pgfpathlineto{\pgfqpoint{2.714086in}{1.768795in}}%
\pgfpathlineto{\pgfqpoint{2.733558in}{1.841200in}}%
\pgfpathlineto{\pgfqpoint{2.750399in}{1.895353in}}%
\pgfpathlineto{\pgfqpoint{2.769123in}{1.947029in}}%
\pgfpathlineto{\pgfqpoint{2.782599in}{1.978762in}}%
\pgfpathlineto{\pgfqpoint{2.796802in}{2.007253in}}%
\pgfpathlineto{\pgfqpoint{2.811650in}{2.031555in}}%
\pgfpathlineto{\pgfqpoint{2.827041in}{2.050759in}}%
\pgfpathlineto{\pgfqpoint{2.834901in}{2.058196in}}%
\pgfpathlineto{\pgfqpoint{2.842849in}{2.064075in}}%
\pgfpathlineto{\pgfqpoint{2.850865in}{2.068328in}}%
\pgfpathlineto{\pgfqpoint{2.858931in}{2.070906in}}%
\pgfpathlineto{\pgfqpoint{2.867028in}{2.071774in}}%
\pgfpathlineto{\pgfqpoint{2.875135in}{2.070917in}}%
\pgfpathlineto{\pgfqpoint{2.883232in}{2.068343in}}%
\pgfpathlineto{\pgfqpoint{2.891298in}{2.064075in}}%
\pgfpathlineto{\pgfqpoint{2.899315in}{2.058159in}}%
\pgfpathlineto{\pgfqpoint{2.907262in}{2.050656in}}%
\pgfpathlineto{\pgfqpoint{2.922874in}{2.031221in}}%
\pgfpathlineto{\pgfqpoint{2.937998in}{2.006558in}}%
\pgfpathlineto{\pgfqpoint{2.952517in}{1.977609in}}%
\pgfpathlineto{\pgfqpoint{2.966335in}{1.945382in}}%
\pgfpathlineto{\pgfqpoint{2.985600in}{1.893069in}}%
\pgfpathlineto{\pgfqpoint{3.002991in}{1.838654in}}%
\pgfpathlineto{\pgfqpoint{3.023200in}{1.766897in}}%
\pgfpathlineto{\pgfqpoint{3.043840in}{1.683399in}}%
\pgfpathlineto{\pgfqpoint{3.059801in}{1.609586in}}%
\pgfpathlineto{\pgfqpoint{3.071664in}{1.545577in}}%
\pgfpathlineto{\pgfqpoint{3.079975in}{1.489863in}}%
\pgfpathlineto{\pgfqpoint{3.085170in}{1.440241in}}%
\pgfpathlineto{\pgfqpoint{3.087550in}{1.394283in}}%
\pgfpathlineto{\pgfqpoint{3.087274in}{1.349534in}}%
\pgfpathlineto{\pgfqpoint{3.084353in}{1.303559in}}%
\pgfpathlineto{\pgfqpoint{3.078660in}{1.253953in}}%
\pgfpathlineto{\pgfqpoint{3.069927in}{1.198382in}}%
\pgfpathlineto{\pgfqpoint{3.057752in}{1.134749in}}%
\pgfpathlineto{\pgfqpoint{3.041623in}{1.061592in}}%
\pgfpathlineto{\pgfqpoint{3.020971in}{0.978888in}}%
\pgfpathlineto{\pgfqpoint{3.000843in}{0.907538in}}%
\pgfpathlineto{\pgfqpoint{2.983526in}{0.853010in}}%
\pgfpathlineto{\pgfqpoint{2.964301in}{0.799975in}}%
\pgfpathlineto{\pgfqpoint{2.950461in}{0.766836in}}%
\pgfpathlineto{\pgfqpoint{2.935858in}{0.736603in}}%
\pgfpathlineto{\pgfqpoint{2.920568in}{0.710305in}}%
\pgfpathlineto{\pgfqpoint{2.904687in}{0.688953in}}%
\pgfpathlineto{\pgfqpoint{2.896563in}{0.680424in}}%
\pgfpathlineto{\pgfqpoint{2.888339in}{0.673459in}}%
\pgfpathlineto{\pgfqpoint{2.880033in}{0.668142in}}%
\pgfpathlineto{\pgfqpoint{2.871664in}{0.664539in}}%
\pgfpathlineto{\pgfqpoint{2.863254in}{0.662699in}}%
\pgfpathlineto{\pgfqpoint{2.854821in}{0.662646in}}%
\pgfpathlineto{\pgfqpoint{2.846388in}{0.664387in}}%
\pgfpathlineto{\pgfqpoint{2.837976in}{0.667904in}}%
\pgfpathlineto{\pgfqpoint{2.829604in}{0.673160in}}%
\pgfpathlineto{\pgfqpoint{2.821295in}{0.680095in}}%
\pgfpathlineto{\pgfqpoint{2.813067in}{0.688631in}}%
\pgfpathlineto{\pgfqpoint{2.796929in}{0.710111in}}%
\pgfpathlineto{\pgfqpoint{2.781326in}{0.736684in}}%
\pgfpathlineto{\pgfqpoint{2.766373in}{0.767292in}}%
\pgfpathlineto{\pgfqpoint{2.752157in}{0.800828in}}%
\pgfpathlineto{\pgfqpoint{2.732345in}{0.854305in}}%
\pgfpathlineto{\pgfqpoint{2.714435in}{0.908814in}}%
\pgfpathlineto{\pgfqpoint{2.693514in}{0.978992in}}%
\pgfpathlineto{\pgfqpoint{2.668028in}{1.072210in}}%
\pgfpathlineto{\pgfqpoint{2.603616in}{1.313636in}}%
\pgfpathlineto{\pgfqpoint{2.598486in}{1.323923in}}%
\pgfpathlineto{\pgfqpoint{2.595215in}{1.326825in}}%
\pgfpathlineto{\pgfqpoint{2.592672in}{1.326676in}}%
\pgfpathlineto{\pgfqpoint{2.590084in}{1.324353in}}%
\pgfpathlineto{\pgfqpoint{2.585891in}{1.316618in}}%
\pgfpathlineto{\pgfqpoint{2.580077in}{1.300365in}}%
\pgfpathlineto{\pgfqpoint{2.570055in}{1.265165in}}%
\pgfpathlineto{\pgfqpoint{2.544211in}{1.163944in}}%
\pgfpathlineto{\pgfqpoint{2.508756in}{1.027002in}}%
\pgfpathlineto{\pgfqpoint{2.481013in}{0.928555in}}%
\pgfpathlineto{\pgfqpoint{2.458511in}{0.856710in}}%
\pgfpathlineto{\pgfqpoint{2.439405in}{0.802414in}}%
\pgfpathlineto{\pgfqpoint{2.418410in}{0.750725in}}%
\pgfpathlineto{\pgfqpoint{2.403423in}{0.719363in}}%
\pgfpathlineto{\pgfqpoint{2.387721in}{0.691751in}}%
\pgfpathlineto{\pgfqpoint{2.371396in}{0.668985in}}%
\pgfpathlineto{\pgfqpoint{2.363038in}{0.659740in}}%
\pgfpathlineto{\pgfqpoint{2.354571in}{0.652070in}}%
\pgfpathlineto{\pgfqpoint{2.346014in}{0.646070in}}%
\pgfpathlineto{\pgfqpoint{2.337386in}{0.641819in}}%
\pgfpathlineto{\pgfqpoint{2.328709in}{0.639372in}}%
\pgfpathlineto{\pgfqpoint{2.320003in}{0.638767in}}%
\pgfpathlineto{\pgfqpoint{2.311289in}{0.640015in}}%
\pgfpathlineto{\pgfqpoint{2.302591in}{0.643105in}}%
\pgfpathlineto{\pgfqpoint{2.293928in}{0.648001in}}%
\pgfpathlineto{\pgfqpoint{2.285323in}{0.654647in}}%
\pgfpathlineto{\pgfqpoint{2.276797in}{0.662962in}}%
\pgfpathlineto{\pgfqpoint{2.260054in}{0.684193in}}%
\pgfpathlineto{\pgfqpoint{2.243843in}{0.710730in}}%
\pgfpathlineto{\pgfqpoint{2.228281in}{0.741449in}}%
\pgfpathlineto{\pgfqpoint{2.213459in}{0.775162in}}%
\pgfpathlineto{\pgfqpoint{2.192746in}{0.828845in}}%
\pgfpathlineto{\pgfqpoint{2.168097in}{0.901088in}}%
\pgfpathlineto{\pgfqpoint{2.132807in}{1.014435in}}%
\pgfpathlineto{\pgfqpoint{2.102602in}{1.109961in}}%
\pgfpathlineto{\pgfqpoint{2.087701in}{1.150161in}}%
\pgfpathlineto{\pgfqpoint{2.077279in}{1.172293in}}%
\pgfpathlineto{\pgfqpoint{2.070069in}{1.183424in}}%
\pgfpathlineto{\pgfqpoint{2.063198in}{1.190092in}}%
\pgfpathlineto{\pgfqpoint{2.058728in}{1.192127in}}%
\pgfpathlineto{\pgfqpoint{2.054296in}{1.192265in}}%
\pgfpathlineto{\pgfqpoint{2.049855in}{1.190515in}}%
\pgfpathlineto{\pgfqpoint{2.045362in}{1.186876in}}%
\pgfpathlineto{\pgfqpoint{2.038428in}{1.177831in}}%
\pgfpathlineto{\pgfqpoint{2.031124in}{1.164384in}}%
\pgfpathlineto{\pgfqpoint{2.020539in}{1.139319in}}%
\pgfpathlineto{\pgfqpoint{2.005394in}{1.095843in}}%
\pgfpathlineto{\pgfqpoint{1.979157in}{1.010957in}}%
\pgfpathlineto{\pgfqpoint{1.939322in}{0.881834in}}%
\pgfpathlineto{\pgfqpoint{1.914614in}{0.809496in}}%
\pgfpathlineto{\pgfqpoint{1.893850in}{0.755844in}}%
\pgfpathlineto{\pgfqpoint{1.878971in}{0.722111in}}%
\pgfpathlineto{\pgfqpoint{1.863315in}{0.691296in}}%
\pgfpathlineto{\pgfqpoint{1.846958in}{0.664573in}}%
\pgfpathlineto{\pgfqpoint{1.830002in}{0.643086in}}%
\pgfpathlineto{\pgfqpoint{1.821340in}{0.634632in}}%
\pgfpathlineto{\pgfqpoint{1.812578in}{0.627853in}}%
\pgfpathlineto{\pgfqpoint{1.803737in}{0.622838in}}%
\pgfpathlineto{\pgfqpoint{1.794836in}{0.619657in}}%
\pgfpathlineto{\pgfqpoint{1.785899in}{0.618357in}}%
\pgfpathlineto{\pgfqpoint{1.776946in}{0.618959in}}%
\pgfpathlineto{\pgfqpoint{1.768000in}{0.621460in}}%
\pgfpathlineto{\pgfqpoint{1.759084in}{0.625831in}}%
\pgfpathlineto{\pgfqpoint{1.750219in}{0.632018in}}%
\pgfpathlineto{\pgfqpoint{1.741426in}{0.639945in}}%
\pgfpathlineto{\pgfqpoint{1.724140in}{0.660606in}}%
\pgfpathlineto{\pgfqpoint{1.707373in}{0.686823in}}%
\pgfpathlineto{\pgfqpoint{1.691248in}{0.717417in}}%
\pgfpathlineto{\pgfqpoint{1.675862in}{0.751133in}}%
\pgfpathlineto{\pgfqpoint{1.654304in}{0.804886in}}%
\pgfpathlineto{\pgfqpoint{1.628529in}{0.876949in}}%
\pgfpathlineto{\pgfqpoint{1.565713in}{1.058236in}}%
\pgfpathlineto{\pgfqpoint{1.551172in}{1.091031in}}%
\pgfpathlineto{\pgfqpoint{1.541178in}{1.108376in}}%
\pgfpathlineto{\pgfqpoint{1.531723in}{1.119686in}}%
\pgfpathlineto{\pgfqpoint{1.525615in}{1.123950in}}%
\pgfpathlineto{\pgfqpoint{1.519592in}{1.125632in}}%
\pgfpathlineto{\pgfqpoint{1.513593in}{1.124753in}}%
\pgfpathlineto{\pgfqpoint{1.507559in}{1.121322in}}%
\pgfpathlineto{\pgfqpoint{1.501429in}{1.115335in}}%
\pgfpathlineto{\pgfqpoint{1.491924in}{1.101525in}}%
\pgfpathlineto{\pgfqpoint{1.481863in}{1.081838in}}%
\pgfpathlineto{\pgfqpoint{1.467226in}{1.046280in}}%
\pgfpathlineto{\pgfqpoint{1.446257in}{0.986795in}}%
\pgfpathlineto{\pgfqpoint{1.385330in}{0.806407in}}%
\pgfpathlineto{\pgfqpoint{1.364453in}{0.752100in}}%
\pgfpathlineto{\pgfqpoint{1.349480in}{0.717276in}}%
\pgfpathlineto{\pgfqpoint{1.333701in}{0.684863in}}%
\pgfpathlineto{\pgfqpoint{1.317182in}{0.656083in}}%
\pgfpathlineto{\pgfqpoint{1.300013in}{0.632167in}}%
\pgfpathlineto{\pgfqpoint{1.291223in}{0.622396in}}%
\pgfpathlineto{\pgfqpoint{1.282317in}{0.614252in}}%
\pgfpathlineto{\pgfqpoint{1.273316in}{0.607847in}}%
\pgfpathlineto{\pgfqpoint{1.264240in}{0.603268in}}%
\pgfpathlineto{\pgfqpoint{1.255110in}{0.600582in}}%
\pgfpathlineto{\pgfqpoint{1.245949in}{0.599832in}}%
\pgfpathlineto{\pgfqpoint{1.236778in}{0.601031in}}%
\pgfpathlineto{\pgfqpoint{1.227622in}{0.604169in}}%
\pgfpathlineto{\pgfqpoint{1.218503in}{0.609206in}}%
\pgfpathlineto{\pgfqpoint{1.209443in}{0.616078in}}%
\pgfpathlineto{\pgfqpoint{1.200463in}{0.624694in}}%
\pgfpathlineto{\pgfqpoint{1.182827in}{0.646701in}}%
\pgfpathlineto{\pgfqpoint{1.165740in}{0.674138in}}%
\pgfpathlineto{\pgfqpoint{1.149324in}{0.705738in}}%
\pgfpathlineto{\pgfqpoint{1.133669in}{0.740173in}}%
\pgfpathlineto{\pgfqpoint{1.111732in}{0.794365in}}%
\pgfpathlineto{\pgfqpoint{1.079406in}{0.882662in}}%
\pgfpathlineto{\pgfqpoint{1.042253in}{0.984257in}}%
\pgfpathlineto{\pgfqpoint{1.024403in}{1.026073in}}%
\pgfpathlineto{\pgfqpoint{1.012191in}{1.049113in}}%
\pgfpathlineto{\pgfqpoint{1.000706in}{1.065035in}}%
\pgfpathlineto{\pgfqpoint{0.993330in}{1.071741in}}%
\pgfpathlineto{\pgfqpoint{0.986093in}{1.075355in}}%
\pgfpathlineto{\pgfqpoint{0.978923in}{1.075902in}}%
\pgfpathlineto{\pgfqpoint{0.971749in}{1.073399in}}%
\pgfpathlineto{\pgfqpoint{0.964497in}{1.067854in}}%
\pgfpathlineto{\pgfqpoint{0.957098in}{1.059266in}}%
\pgfpathlineto{\pgfqpoint{0.949479in}{1.047632in}}%
\pgfpathlineto{\pgfqpoint{0.937482in}{1.024461in}}%
\pgfpathlineto{\pgfqpoint{0.924589in}{0.994459in}}%
\pgfpathlineto{\pgfqpoint{0.905592in}{0.944132in}}%
\pgfpathlineto{\pgfqpoint{0.905592in}{0.944132in}}%
\pgfusepath{stroke}%
\end{pgfscope}%
\begin{pgfscope}%
\pgfsetrectcap%
\pgfsetmiterjoin%
\pgfsetlinewidth{0.803000pt}%
\definecolor{currentstroke}{rgb}{0.000000,0.000000,0.000000}%
\pgfsetstrokecolor{currentstroke}%
\pgfsetdash{}{0pt}%
\pgfpathmoveto{\pgfqpoint{0.796484in}{0.526234in}}%
\pgfpathlineto{\pgfqpoint{0.796484in}{2.145371in}}%
\pgfusepath{stroke}%
\end{pgfscope}%
\begin{pgfscope}%
\pgfsetrectcap%
\pgfsetmiterjoin%
\pgfsetlinewidth{0.803000pt}%
\definecolor{currentstroke}{rgb}{0.000000,0.000000,0.000000}%
\pgfsetstrokecolor{currentstroke}%
\pgfsetdash{}{0pt}%
\pgfpathmoveto{\pgfqpoint{3.196863in}{0.526234in}}%
\pgfpathlineto{\pgfqpoint{3.196863in}{2.145371in}}%
\pgfusepath{stroke}%
\end{pgfscope}%
\begin{pgfscope}%
\pgfsetrectcap%
\pgfsetmiterjoin%
\pgfsetlinewidth{0.803000pt}%
\definecolor{currentstroke}{rgb}{0.000000,0.000000,0.000000}%
\pgfsetstrokecolor{currentstroke}%
\pgfsetdash{}{0pt}%
\pgfpathmoveto{\pgfqpoint{0.796484in}{0.526234in}}%
\pgfpathlineto{\pgfqpoint{3.196863in}{0.526234in}}%
\pgfusepath{stroke}%
\end{pgfscope}%
\begin{pgfscope}%
\pgfsetrectcap%
\pgfsetmiterjoin%
\pgfsetlinewidth{0.803000pt}%
\definecolor{currentstroke}{rgb}{0.000000,0.000000,0.000000}%
\pgfsetstrokecolor{currentstroke}%
\pgfsetdash{}{0pt}%
\pgfpathmoveto{\pgfqpoint{0.796484in}{2.145371in}}%
\pgfpathlineto{\pgfqpoint{3.196863in}{2.145371in}}%
\pgfusepath{stroke}%
\end{pgfscope}%
\begin{pgfscope}%
\definecolor{textcolor}{rgb}{0.000000,0.000000,0.000000}%
\pgfsetstrokecolor{textcolor}%
\pgfsetfillcolor{textcolor}%
\pgftext[x=1.996673in,y=2.228704in,,base]{\color{textcolor}\rmfamily\fontsize{12.000000}{14.400000}\selectfont phase plot}%
\end{pgfscope}%
\begin{pgfscope}%
\pgfsetbuttcap%
\pgfsetmiterjoin%
\definecolor{currentfill}{rgb}{1.000000,1.000000,1.000000}%
\pgfsetfillcolor{currentfill}%
\pgfsetlinewidth{0.000000pt}%
\definecolor{currentstroke}{rgb}{0.000000,0.000000,0.000000}%
\pgfsetstrokecolor{currentstroke}%
\pgfsetstrokeopacity{0.000000}%
\pgfsetdash{}{0pt}%
\pgfpathmoveto{\pgfqpoint{3.867533in}{0.526234in}}%
\pgfpathlineto{\pgfqpoint{6.267911in}{0.526234in}}%
\pgfpathlineto{\pgfqpoint{6.267911in}{2.145371in}}%
\pgfpathlineto{\pgfqpoint{3.867533in}{2.145371in}}%
\pgfpathclose%
\pgfusepath{fill}%
\end{pgfscope}%
\begin{pgfscope}%
\pgfsetbuttcap%
\pgfsetroundjoin%
\definecolor{currentfill}{rgb}{0.000000,0.000000,0.000000}%
\pgfsetfillcolor{currentfill}%
\pgfsetlinewidth{0.803000pt}%
\definecolor{currentstroke}{rgb}{0.000000,0.000000,0.000000}%
\pgfsetstrokecolor{currentstroke}%
\pgfsetdash{}{0pt}%
\pgfsys@defobject{currentmarker}{\pgfqpoint{0.000000in}{-0.048611in}}{\pgfqpoint{0.000000in}{0.000000in}}{%
\pgfpathmoveto{\pgfqpoint{0.000000in}{0.000000in}}%
\pgfpathlineto{\pgfqpoint{0.000000in}{-0.048611in}}%
\pgfusepath{stroke,fill}%
}%
\begin{pgfscope}%
\pgfsys@transformshift{3.976641in}{0.526234in}%
\pgfsys@useobject{currentmarker}{}%
\end{pgfscope}%
\end{pgfscope}%
\begin{pgfscope}%
\definecolor{textcolor}{rgb}{0.000000,0.000000,0.000000}%
\pgfsetstrokecolor{textcolor}%
\pgfsetfillcolor{textcolor}%
\pgftext[x=3.976641in,y=0.429012in,,top]{\color{textcolor}\rmfamily\fontsize{10.000000}{12.000000}\selectfont \(\displaystyle 0.0\)}%
\end{pgfscope}%
\begin{pgfscope}%
\pgfsetbuttcap%
\pgfsetroundjoin%
\definecolor{currentfill}{rgb}{0.000000,0.000000,0.000000}%
\pgfsetfillcolor{currentfill}%
\pgfsetlinewidth{0.803000pt}%
\definecolor{currentstroke}{rgb}{0.000000,0.000000,0.000000}%
\pgfsetstrokecolor{currentstroke}%
\pgfsetdash{}{0pt}%
\pgfsys@defobject{currentmarker}{\pgfqpoint{0.000000in}{-0.048611in}}{\pgfqpoint{0.000000in}{0.000000in}}{%
\pgfpathmoveto{\pgfqpoint{0.000000in}{0.000000in}}%
\pgfpathlineto{\pgfqpoint{0.000000in}{-0.048611in}}%
\pgfusepath{stroke,fill}%
}%
\begin{pgfscope}%
\pgfsys@transformshift{4.522181in}{0.526234in}%
\pgfsys@useobject{currentmarker}{}%
\end{pgfscope}%
\end{pgfscope}%
\begin{pgfscope}%
\definecolor{textcolor}{rgb}{0.000000,0.000000,0.000000}%
\pgfsetstrokecolor{textcolor}%
\pgfsetfillcolor{textcolor}%
\pgftext[x=4.522181in,y=0.429012in,,top]{\color{textcolor}\rmfamily\fontsize{10.000000}{12.000000}\selectfont \(\displaystyle 2.5\)}%
\end{pgfscope}%
\begin{pgfscope}%
\pgfsetbuttcap%
\pgfsetroundjoin%
\definecolor{currentfill}{rgb}{0.000000,0.000000,0.000000}%
\pgfsetfillcolor{currentfill}%
\pgfsetlinewidth{0.803000pt}%
\definecolor{currentstroke}{rgb}{0.000000,0.000000,0.000000}%
\pgfsetstrokecolor{currentstroke}%
\pgfsetdash{}{0pt}%
\pgfsys@defobject{currentmarker}{\pgfqpoint{0.000000in}{-0.048611in}}{\pgfqpoint{0.000000in}{0.000000in}}{%
\pgfpathmoveto{\pgfqpoint{0.000000in}{0.000000in}}%
\pgfpathlineto{\pgfqpoint{0.000000in}{-0.048611in}}%
\pgfusepath{stroke,fill}%
}%
\begin{pgfscope}%
\pgfsys@transformshift{5.067722in}{0.526234in}%
\pgfsys@useobject{currentmarker}{}%
\end{pgfscope}%
\end{pgfscope}%
\begin{pgfscope}%
\definecolor{textcolor}{rgb}{0.000000,0.000000,0.000000}%
\pgfsetstrokecolor{textcolor}%
\pgfsetfillcolor{textcolor}%
\pgftext[x=5.067722in,y=0.429012in,,top]{\color{textcolor}\rmfamily\fontsize{10.000000}{12.000000}\selectfont \(\displaystyle 5.0\)}%
\end{pgfscope}%
\begin{pgfscope}%
\pgfsetbuttcap%
\pgfsetroundjoin%
\definecolor{currentfill}{rgb}{0.000000,0.000000,0.000000}%
\pgfsetfillcolor{currentfill}%
\pgfsetlinewidth{0.803000pt}%
\definecolor{currentstroke}{rgb}{0.000000,0.000000,0.000000}%
\pgfsetstrokecolor{currentstroke}%
\pgfsetdash{}{0pt}%
\pgfsys@defobject{currentmarker}{\pgfqpoint{0.000000in}{-0.048611in}}{\pgfqpoint{0.000000in}{0.000000in}}{%
\pgfpathmoveto{\pgfqpoint{0.000000in}{0.000000in}}%
\pgfpathlineto{\pgfqpoint{0.000000in}{-0.048611in}}%
\pgfusepath{stroke,fill}%
}%
\begin{pgfscope}%
\pgfsys@transformshift{5.613262in}{0.526234in}%
\pgfsys@useobject{currentmarker}{}%
\end{pgfscope}%
\end{pgfscope}%
\begin{pgfscope}%
\definecolor{textcolor}{rgb}{0.000000,0.000000,0.000000}%
\pgfsetstrokecolor{textcolor}%
\pgfsetfillcolor{textcolor}%
\pgftext[x=5.613262in,y=0.429012in,,top]{\color{textcolor}\rmfamily\fontsize{10.000000}{12.000000}\selectfont \(\displaystyle 7.5\)}%
\end{pgfscope}%
\begin{pgfscope}%
\pgfsetbuttcap%
\pgfsetroundjoin%
\definecolor{currentfill}{rgb}{0.000000,0.000000,0.000000}%
\pgfsetfillcolor{currentfill}%
\pgfsetlinewidth{0.803000pt}%
\definecolor{currentstroke}{rgb}{0.000000,0.000000,0.000000}%
\pgfsetstrokecolor{currentstroke}%
\pgfsetdash{}{0pt}%
\pgfsys@defobject{currentmarker}{\pgfqpoint{0.000000in}{-0.048611in}}{\pgfqpoint{0.000000in}{0.000000in}}{%
\pgfpathmoveto{\pgfqpoint{0.000000in}{0.000000in}}%
\pgfpathlineto{\pgfqpoint{0.000000in}{-0.048611in}}%
\pgfusepath{stroke,fill}%
}%
\begin{pgfscope}%
\pgfsys@transformshift{6.158803in}{0.526234in}%
\pgfsys@useobject{currentmarker}{}%
\end{pgfscope}%
\end{pgfscope}%
\begin{pgfscope}%
\definecolor{textcolor}{rgb}{0.000000,0.000000,0.000000}%
\pgfsetstrokecolor{textcolor}%
\pgfsetfillcolor{textcolor}%
\pgftext[x=6.158803in,y=0.429012in,,top]{\color{textcolor}\rmfamily\fontsize{10.000000}{12.000000}\selectfont \(\displaystyle 10.0\)}%
\end{pgfscope}%
\begin{pgfscope}%
\definecolor{textcolor}{rgb}{0.000000,0.000000,0.000000}%
\pgfsetstrokecolor{textcolor}%
\pgfsetfillcolor{textcolor}%
\pgftext[x=5.067722in,y=0.250000in,,top]{\color{textcolor}\rmfamily\fontsize{10.000000}{12.000000}\selectfont time (s)}%
\end{pgfscope}%
\begin{pgfscope}%
\pgfsetbuttcap%
\pgfsetroundjoin%
\definecolor{currentfill}{rgb}{0.000000,0.000000,0.000000}%
\pgfsetfillcolor{currentfill}%
\pgfsetlinewidth{0.803000pt}%
\definecolor{currentstroke}{rgb}{0.000000,0.000000,0.000000}%
\pgfsetstrokecolor{currentstroke}%
\pgfsetdash{}{0pt}%
\pgfsys@defobject{currentmarker}{\pgfqpoint{-0.048611in}{0.000000in}}{\pgfqpoint{0.000000in}{0.000000in}}{%
\pgfpathmoveto{\pgfqpoint{0.000000in}{0.000000in}}%
\pgfpathlineto{\pgfqpoint{-0.048611in}{0.000000in}}%
\pgfusepath{stroke,fill}%
}%
\begin{pgfscope}%
\pgfsys@transformshift{3.867533in}{0.914228in}%
\pgfsys@useobject{currentmarker}{}%
\end{pgfscope}%
\end{pgfscope}%
\begin{pgfscope}%
\definecolor{textcolor}{rgb}{0.000000,0.000000,0.000000}%
\pgfsetstrokecolor{textcolor}%
\pgfsetfillcolor{textcolor}%
\pgftext[x=3.631421in,y=0.866003in,left,base]{\color{textcolor}\rmfamily\fontsize{10.000000}{12.000000}\selectfont \(\displaystyle 20\)}%
\end{pgfscope}%
\begin{pgfscope}%
\pgfsetbuttcap%
\pgfsetroundjoin%
\definecolor{currentfill}{rgb}{0.000000,0.000000,0.000000}%
\pgfsetfillcolor{currentfill}%
\pgfsetlinewidth{0.803000pt}%
\definecolor{currentstroke}{rgb}{0.000000,0.000000,0.000000}%
\pgfsetstrokecolor{currentstroke}%
\pgfsetdash{}{0pt}%
\pgfsys@defobject{currentmarker}{\pgfqpoint{-0.048611in}{0.000000in}}{\pgfqpoint{0.000000in}{0.000000in}}{%
\pgfpathmoveto{\pgfqpoint{0.000000in}{0.000000in}}%
\pgfpathlineto{\pgfqpoint{-0.048611in}{0.000000in}}%
\pgfusepath{stroke,fill}%
}%
\begin{pgfscope}%
\pgfsys@transformshift{3.867533in}{1.530693in}%
\pgfsys@useobject{currentmarker}{}%
\end{pgfscope}%
\end{pgfscope}%
\begin{pgfscope}%
\definecolor{textcolor}{rgb}{0.000000,0.000000,0.000000}%
\pgfsetstrokecolor{textcolor}%
\pgfsetfillcolor{textcolor}%
\pgftext[x=3.631421in,y=1.482468in,left,base]{\color{textcolor}\rmfamily\fontsize{10.000000}{12.000000}\selectfont \(\displaystyle 40\)}%
\end{pgfscope}%
\begin{pgfscope}%
\definecolor{textcolor}{rgb}{0.000000,0.000000,0.000000}%
\pgfsetstrokecolor{textcolor}%
\pgfsetfillcolor{textcolor}%
\pgftext[x=3.575865in,y=1.335803in,,bottom,rotate=90.000000]{\color{textcolor}\rmfamily\fontsize{10.000000}{12.000000}\selectfont energy (J)}%
\end{pgfscope}%
\begin{pgfscope}%
\pgfpathrectangle{\pgfqpoint{3.867533in}{0.526234in}}{\pgfqpoint{2.400379in}{1.619136in}}%
\pgfusepath{clip}%
\pgfsetrectcap%
\pgfsetroundjoin%
\pgfsetlinewidth{1.505625pt}%
\definecolor{currentstroke}{rgb}{0.000000,0.000000,1.000000}%
\pgfsetstrokecolor{currentstroke}%
\pgfsetdash{}{0pt}%
\pgfpathmoveto{\pgfqpoint{3.976641in}{0.599832in}}%
\pgfpathlineto{\pgfqpoint{3.981005in}{0.600795in}}%
\pgfpathlineto{\pgfqpoint{3.985369in}{0.606235in}}%
\pgfpathlineto{\pgfqpoint{3.989734in}{0.616360in}}%
\pgfpathlineto{\pgfqpoint{3.996280in}{0.640313in}}%
\pgfpathlineto{\pgfqpoint{4.002827in}{0.674667in}}%
\pgfpathlineto{\pgfqpoint{4.011555in}{0.735893in}}%
\pgfpathlineto{\pgfqpoint{4.022466in}{0.833177in}}%
\pgfpathlineto{\pgfqpoint{4.046470in}{1.058313in}}%
\pgfpathlineto{\pgfqpoint{4.053016in}{1.097075in}}%
\pgfpathlineto{\pgfqpoint{4.057381in}{1.112289in}}%
\pgfpathlineto{\pgfqpoint{4.059563in}{1.116334in}}%
\pgfpathlineto{\pgfqpoint{4.061745in}{1.117913in}}%
\pgfpathlineto{\pgfqpoint{4.063927in}{1.117005in}}%
\pgfpathlineto{\pgfqpoint{4.066109in}{1.113624in}}%
\pgfpathlineto{\pgfqpoint{4.070474in}{1.099696in}}%
\pgfpathlineto{\pgfqpoint{4.074838in}{1.076976in}}%
\pgfpathlineto{\pgfqpoint{4.081384in}{1.029401in}}%
\pgfpathlineto{\pgfqpoint{4.092295in}{0.928680in}}%
\pgfpathlineto{\pgfqpoint{4.107570in}{0.786888in}}%
\pgfpathlineto{\pgfqpoint{4.116299in}{0.722612in}}%
\pgfpathlineto{\pgfqpoint{4.125028in}{0.675936in}}%
\pgfpathlineto{\pgfqpoint{4.131574in}{0.653192in}}%
\pgfpathlineto{\pgfqpoint{4.135939in}{0.643888in}}%
\pgfpathlineto{\pgfqpoint{4.140303in}{0.639234in}}%
\pgfpathlineto{\pgfqpoint{4.142485in}{0.638641in}}%
\pgfpathlineto{\pgfqpoint{4.144667in}{0.639201in}}%
\pgfpathlineto{\pgfqpoint{4.149032in}{0.643776in}}%
\pgfpathlineto{\pgfqpoint{4.153396in}{0.652970in}}%
\pgfpathlineto{\pgfqpoint{4.157760in}{0.666812in}}%
\pgfpathlineto{\pgfqpoint{4.164307in}{0.696377in}}%
\pgfpathlineto{\pgfqpoint{4.170853in}{0.736542in}}%
\pgfpathlineto{\pgfqpoint{4.179582in}{0.806094in}}%
\pgfpathlineto{\pgfqpoint{4.190493in}{0.914828in}}%
\pgfpathlineto{\pgfqpoint{4.212314in}{1.144147in}}%
\pgfpathlineto{\pgfqpoint{4.218861in}{1.191105in}}%
\pgfpathlineto{\pgfqpoint{4.223225in}{1.211030in}}%
\pgfpathlineto{\pgfqpoint{4.227589in}{1.220398in}}%
\pgfpathlineto{\pgfqpoint{4.229772in}{1.220919in}}%
\pgfpathlineto{\pgfqpoint{4.231954in}{1.218643in}}%
\pgfpathlineto{\pgfqpoint{4.236318in}{1.205892in}}%
\pgfpathlineto{\pgfqpoint{4.240682in}{1.182960in}}%
\pgfpathlineto{\pgfqpoint{4.247229in}{1.132668in}}%
\pgfpathlineto{\pgfqpoint{4.258140in}{1.022767in}}%
\pgfpathlineto{\pgfqpoint{4.275597in}{0.844719in}}%
\pgfpathlineto{\pgfqpoint{4.284326in}{0.775846in}}%
\pgfpathlineto{\pgfqpoint{4.293054in}{0.725584in}}%
\pgfpathlineto{\pgfqpoint{4.299601in}{0.700305in}}%
\pgfpathlineto{\pgfqpoint{4.303965in}{0.689213in}}%
\pgfpathlineto{\pgfqpoint{4.308329in}{0.682608in}}%
\pgfpathlineto{\pgfqpoint{4.312694in}{0.680410in}}%
\pgfpathlineto{\pgfqpoint{4.317058in}{0.682570in}}%
\pgfpathlineto{\pgfqpoint{4.321422in}{0.689089in}}%
\pgfpathlineto{\pgfqpoint{4.325787in}{0.700008in}}%
\pgfpathlineto{\pgfqpoint{4.332333in}{0.724828in}}%
\pgfpathlineto{\pgfqpoint{4.338880in}{0.760107in}}%
\pgfpathlineto{\pgfqpoint{4.347608in}{0.823909in}}%
\pgfpathlineto{\pgfqpoint{4.356337in}{0.906454in}}%
\pgfpathlineto{\pgfqpoint{4.369430in}{1.057387in}}%
\pgfpathlineto{\pgfqpoint{4.384705in}{1.233960in}}%
\pgfpathlineto{\pgfqpoint{4.391252in}{1.289796in}}%
\pgfpathlineto{\pgfqpoint{4.395616in}{1.315011in}}%
\pgfpathlineto{\pgfqpoint{4.399980in}{1.328748in}}%
\pgfpathlineto{\pgfqpoint{4.402162in}{1.331011in}}%
\pgfpathlineto{\pgfqpoint{4.404344in}{1.330139in}}%
\pgfpathlineto{\pgfqpoint{4.406527in}{1.326151in}}%
\pgfpathlineto{\pgfqpoint{4.410891in}{1.309145in}}%
\pgfpathlineto{\pgfqpoint{4.415255in}{1.281114in}}%
\pgfpathlineto{\pgfqpoint{4.421802in}{1.222377in}}%
\pgfpathlineto{\pgfqpoint{4.432713in}{1.099013in}}%
\pgfpathlineto{\pgfqpoint{4.447988in}{0.927816in}}%
\pgfpathlineto{\pgfqpoint{4.456716in}{0.850107in}}%
\pgfpathlineto{\pgfqpoint{4.465445in}{0.791675in}}%
\pgfpathlineto{\pgfqpoint{4.471992in}{0.760398in}}%
\pgfpathlineto{\pgfqpoint{4.478538in}{0.739227in}}%
\pgfpathlineto{\pgfqpoint{4.482902in}{0.730402in}}%
\pgfpathlineto{\pgfqpoint{4.487267in}{0.725621in}}%
\pgfpathlineto{\pgfqpoint{4.491631in}{0.724777in}}%
\pgfpathlineto{\pgfqpoint{4.495995in}{0.727822in}}%
\pgfpathlineto{\pgfqpoint{4.500360in}{0.734772in}}%
\pgfpathlineto{\pgfqpoint{4.504724in}{0.745705in}}%
\pgfpathlineto{\pgfqpoint{4.511270in}{0.769882in}}%
\pgfpathlineto{\pgfqpoint{4.517817in}{0.803964in}}%
\pgfpathlineto{\pgfqpoint{4.524363in}{0.848651in}}%
\pgfpathlineto{\pgfqpoint{4.533092in}{0.925557in}}%
\pgfpathlineto{\pgfqpoint{4.544003in}{1.048305in}}%
\pgfpathlineto{\pgfqpoint{4.572371in}{1.392935in}}%
\pgfpathlineto{\pgfqpoint{4.576735in}{1.423520in}}%
\pgfpathlineto{\pgfqpoint{4.581100in}{1.441732in}}%
\pgfpathlineto{\pgfqpoint{4.583282in}{1.445791in}}%
\pgfpathlineto{\pgfqpoint{4.585464in}{1.446376in}}%
\pgfpathlineto{\pgfqpoint{4.587646in}{1.443479in}}%
\pgfpathlineto{\pgfqpoint{4.592010in}{1.427512in}}%
\pgfpathlineto{\pgfqpoint{4.596375in}{1.398995in}}%
\pgfpathlineto{\pgfqpoint{4.602921in}{1.336975in}}%
\pgfpathlineto{\pgfqpoint{4.613832in}{1.203560in}}%
\pgfpathlineto{\pgfqpoint{4.629107in}{1.015956in}}%
\pgfpathlineto{\pgfqpoint{4.637836in}{0.929930in}}%
\pgfpathlineto{\pgfqpoint{4.646565in}{0.864046in}}%
\pgfpathlineto{\pgfqpoint{4.653111in}{0.827397in}}%
\pgfpathlineto{\pgfqpoint{4.659657in}{0.800665in}}%
\pgfpathlineto{\pgfqpoint{4.666204in}{0.782830in}}%
\pgfpathlineto{\pgfqpoint{4.670568in}{0.775445in}}%
\pgfpathlineto{\pgfqpoint{4.674933in}{0.771435in}}%
\pgfpathlineto{\pgfqpoint{4.679297in}{0.770672in}}%
\pgfpathlineto{\pgfqpoint{4.683661in}{0.773102in}}%
\pgfpathlineto{\pgfqpoint{4.688026in}{0.778742in}}%
\pgfpathlineto{\pgfqpoint{4.692390in}{0.787678in}}%
\pgfpathlineto{\pgfqpoint{4.698936in}{0.807622in}}%
\pgfpathlineto{\pgfqpoint{4.705483in}{0.836125in}}%
\pgfpathlineto{\pgfqpoint{4.712029in}{0.874191in}}%
\pgfpathlineto{\pgfqpoint{4.720758in}{0.941641in}}%
\pgfpathlineto{\pgfqpoint{4.729487in}{1.029749in}}%
\pgfpathlineto{\pgfqpoint{4.740397in}{1.167405in}}%
\pgfpathlineto{\pgfqpoint{4.764401in}{1.489157in}}%
\pgfpathlineto{\pgfqpoint{4.770948in}{1.542379in}}%
\pgfpathlineto{\pgfqpoint{4.775312in}{1.561105in}}%
\pgfpathlineto{\pgfqpoint{4.777494in}{1.564871in}}%
\pgfpathlineto{\pgfqpoint{4.779676in}{1.564799in}}%
\pgfpathlineto{\pgfqpoint{4.781859in}{1.560890in}}%
\pgfpathlineto{\pgfqpoint{4.786223in}{1.541929in}}%
\pgfpathlineto{\pgfqpoint{4.790587in}{1.509351in}}%
\pgfpathlineto{\pgfqpoint{4.797134in}{1.439992in}}%
\pgfpathlineto{\pgfqpoint{4.810227in}{1.263394in}}%
\pgfpathlineto{\pgfqpoint{4.823320in}{1.093280in}}%
\pgfpathlineto{\pgfqpoint{4.832048in}{1.002563in}}%
\pgfpathlineto{\pgfqpoint{4.840777in}{0.933068in}}%
\pgfpathlineto{\pgfqpoint{4.849506in}{0.882971in}}%
\pgfpathlineto{\pgfqpoint{4.856052in}{0.856236in}}%
\pgfpathlineto{\pgfqpoint{4.862599in}{0.837295in}}%
\pgfpathlineto{\pgfqpoint{4.869145in}{0.825016in}}%
\pgfpathlineto{\pgfqpoint{4.873509in}{0.820100in}}%
\pgfpathlineto{\pgfqpoint{4.877874in}{0.817593in}}%
\pgfpathlineto{\pgfqpoint{4.882238in}{0.817388in}}%
\pgfpathlineto{\pgfqpoint{4.886602in}{0.819444in}}%
\pgfpathlineto{\pgfqpoint{4.890967in}{0.823786in}}%
\pgfpathlineto{\pgfqpoint{4.897513in}{0.834796in}}%
\pgfpathlineto{\pgfqpoint{4.904060in}{0.851735in}}%
\pgfpathlineto{\pgfqpoint{4.910606in}{0.875486in}}%
\pgfpathlineto{\pgfqpoint{4.917153in}{0.907214in}}%
\pgfpathlineto{\pgfqpoint{4.923699in}{0.948287in}}%
\pgfpathlineto{\pgfqpoint{4.932428in}{1.020019in}}%
\pgfpathlineto{\pgfqpoint{4.941156in}{1.113423in}}%
\pgfpathlineto{\pgfqpoint{4.952067in}{1.259921in}}%
\pgfpathlineto{\pgfqpoint{4.976071in}{1.604832in}}%
\pgfpathlineto{\pgfqpoint{4.982617in}{1.661312in}}%
\pgfpathlineto{\pgfqpoint{4.986982in}{1.680542in}}%
\pgfpathlineto{\pgfqpoint{4.989164in}{1.684013in}}%
\pgfpathlineto{\pgfqpoint{4.991346in}{1.683280in}}%
\pgfpathlineto{\pgfqpoint{4.993528in}{1.678359in}}%
\pgfpathlineto{\pgfqpoint{4.997893in}{1.656415in}}%
\pgfpathlineto{\pgfqpoint{5.002257in}{1.619815in}}%
\pgfpathlineto{\pgfqpoint{5.008803in}{1.543265in}}%
\pgfpathlineto{\pgfqpoint{5.021896in}{1.352424in}}%
\pgfpathlineto{\pgfqpoint{5.034989in}{1.172570in}}%
\pgfpathlineto{\pgfqpoint{5.043718in}{1.077951in}}%
\pgfpathlineto{\pgfqpoint{5.052447in}{1.005729in}}%
\pgfpathlineto{\pgfqpoint{5.061175in}{0.953211in}}%
\pgfpathlineto{\pgfqpoint{5.069904in}{0.916473in}}%
\pgfpathlineto{\pgfqpoint{5.076450in}{0.896979in}}%
\pgfpathlineto{\pgfqpoint{5.082997in}{0.882887in}}%
\pgfpathlineto{\pgfqpoint{5.089543in}{0.873132in}}%
\pgfpathlineto{\pgfqpoint{5.096090in}{0.866918in}}%
\pgfpathlineto{\pgfqpoint{5.102636in}{0.863695in}}%
\pgfpathlineto{\pgfqpoint{5.109183in}{0.863138in}}%
\pgfpathlineto{\pgfqpoint{5.115729in}{0.865130in}}%
\pgfpathlineto{\pgfqpoint{5.122276in}{0.869758in}}%
\pgfpathlineto{\pgfqpoint{5.128822in}{0.877309in}}%
\pgfpathlineto{\pgfqpoint{5.135369in}{0.888284in}}%
\pgfpathlineto{\pgfqpoint{5.141915in}{0.903419in}}%
\pgfpathlineto{\pgfqpoint{5.148462in}{0.923695in}}%
\pgfpathlineto{\pgfqpoint{5.155008in}{0.950360in}}%
\pgfpathlineto{\pgfqpoint{5.161555in}{0.984907in}}%
\pgfpathlineto{\pgfqpoint{5.168101in}{1.029013in}}%
\pgfpathlineto{\pgfqpoint{5.176830in}{1.105602in}}%
\pgfpathlineto{\pgfqpoint{5.185559in}{1.205258in}}%
\pgfpathlineto{\pgfqpoint{5.196469in}{1.361637in}}%
\pgfpathlineto{\pgfqpoint{5.220473in}{1.724806in}}%
\pgfpathlineto{\pgfqpoint{5.227020in}{1.780540in}}%
\pgfpathlineto{\pgfqpoint{5.231384in}{1.797206in}}%
\pgfpathlineto{\pgfqpoint{5.233566in}{1.798807in}}%
\pgfpathlineto{\pgfqpoint{5.235748in}{1.795855in}}%
\pgfpathlineto{\pgfqpoint{5.237930in}{1.788409in}}%
\pgfpathlineto{\pgfqpoint{5.242295in}{1.760715in}}%
\pgfpathlineto{\pgfqpoint{5.248841in}{1.691649in}}%
\pgfpathlineto{\pgfqpoint{5.257570in}{1.565667in}}%
\pgfpathlineto{\pgfqpoint{5.277209in}{1.268982in}}%
\pgfpathlineto{\pgfqpoint{5.285938in}{1.167428in}}%
\pgfpathlineto{\pgfqpoint{5.294667in}{1.089873in}}%
\pgfpathlineto{\pgfqpoint{5.303395in}{1.033307in}}%
\pgfpathlineto{\pgfqpoint{5.312124in}{0.993317in}}%
\pgfpathlineto{\pgfqpoint{5.320853in}{0.965634in}}%
\pgfpathlineto{\pgfqpoint{5.329581in}{0.946739in}}%
\pgfpathlineto{\pgfqpoint{5.338310in}{0.933966in}}%
\pgfpathlineto{\pgfqpoint{5.347039in}{0.925388in}}%
\pgfpathlineto{\pgfqpoint{5.355767in}{0.919662in}}%
\pgfpathlineto{\pgfqpoint{5.366678in}{0.915141in}}%
\pgfpathlineto{\pgfqpoint{5.379771in}{0.912132in}}%
\pgfpathlineto{\pgfqpoint{5.397228in}{0.910512in}}%
\pgfpathlineto{\pgfqpoint{5.416868in}{0.910864in}}%
\pgfpathlineto{\pgfqpoint{5.434325in}{0.913346in}}%
\pgfpathlineto{\pgfqpoint{5.447418in}{0.917462in}}%
\pgfpathlineto{\pgfqpoint{5.458329in}{0.923451in}}%
\pgfpathlineto{\pgfqpoint{5.467058in}{0.930886in}}%
\pgfpathlineto{\pgfqpoint{5.475786in}{0.941825in}}%
\pgfpathlineto{\pgfqpoint{5.482333in}{0.953237in}}%
\pgfpathlineto{\pgfqpoint{5.488879in}{0.968348in}}%
\pgfpathlineto{\pgfqpoint{5.495426in}{0.988256in}}%
\pgfpathlineto{\pgfqpoint{5.501972in}{1.014313in}}%
\pgfpathlineto{\pgfqpoint{5.508519in}{1.048133in}}%
\pgfpathlineto{\pgfqpoint{5.515065in}{1.091543in}}%
\pgfpathlineto{\pgfqpoint{5.523794in}{1.167608in}}%
\pgfpathlineto{\pgfqpoint{5.532522in}{1.267818in}}%
\pgfpathlineto{\pgfqpoint{5.543433in}{1.427679in}}%
\pgfpathlineto{\pgfqpoint{5.567437in}{1.812914in}}%
\pgfpathlineto{\pgfqpoint{5.573983in}{1.875771in}}%
\pgfpathlineto{\pgfqpoint{5.578348in}{1.896324in}}%
\pgfpathlineto{\pgfqpoint{5.580530in}{1.899477in}}%
\pgfpathlineto{\pgfqpoint{5.582712in}{1.897769in}}%
\pgfpathlineto{\pgfqpoint{5.584894in}{1.891238in}}%
\pgfpathlineto{\pgfqpoint{5.589259in}{1.864342in}}%
\pgfpathlineto{\pgfqpoint{5.593623in}{1.820921in}}%
\pgfpathlineto{\pgfqpoint{5.600169in}{1.732099in}}%
\pgfpathlineto{\pgfqpoint{5.630720in}{1.269015in}}%
\pgfpathlineto{\pgfqpoint{5.639448in}{1.182132in}}%
\pgfpathlineto{\pgfqpoint{5.648177in}{1.118656in}}%
\pgfpathlineto{\pgfqpoint{5.656906in}{1.074054in}}%
\pgfpathlineto{\pgfqpoint{5.663452in}{1.050208in}}%
\pgfpathlineto{\pgfqpoint{5.669999in}{1.032747in}}%
\pgfpathlineto{\pgfqpoint{5.676545in}{1.020359in}}%
\pgfpathlineto{\pgfqpoint{5.683092in}{1.012063in}}%
\pgfpathlineto{\pgfqpoint{5.689638in}{1.007179in}}%
\pgfpathlineto{\pgfqpoint{5.696185in}{1.005305in}}%
\pgfpathlineto{\pgfqpoint{5.702731in}{1.006286in}}%
\pgfpathlineto{\pgfqpoint{5.709278in}{1.010211in}}%
\pgfpathlineto{\pgfqpoint{5.715824in}{1.017410in}}%
\pgfpathlineto{\pgfqpoint{5.722371in}{1.028468in}}%
\pgfpathlineto{\pgfqpoint{5.728917in}{1.044246in}}%
\pgfpathlineto{\pgfqpoint{5.735463in}{1.065906in}}%
\pgfpathlineto{\pgfqpoint{5.742010in}{1.094923in}}%
\pgfpathlineto{\pgfqpoint{5.748556in}{1.133067in}}%
\pgfpathlineto{\pgfqpoint{5.755103in}{1.182312in}}%
\pgfpathlineto{\pgfqpoint{5.763832in}{1.268592in}}%
\pgfpathlineto{\pgfqpoint{5.772560in}{1.381319in}}%
\pgfpathlineto{\pgfqpoint{5.783471in}{1.557134in}}%
\pgfpathlineto{\pgfqpoint{5.803111in}{1.887373in}}%
\pgfpathlineto{\pgfqpoint{5.809657in}{1.957874in}}%
\pgfpathlineto{\pgfqpoint{5.814021in}{1.983029in}}%
\pgfpathlineto{\pgfqpoint{5.816203in}{1.988181in}}%
\pgfpathlineto{\pgfqpoint{5.818386in}{1.988212in}}%
\pgfpathlineto{\pgfqpoint{5.820568in}{1.983125in}}%
\pgfpathlineto{\pgfqpoint{5.824932in}{1.958138in}}%
\pgfpathlineto{\pgfqpoint{5.829296in}{1.915270in}}%
\pgfpathlineto{\pgfqpoint{5.835843in}{1.824952in}}%
\pgfpathlineto{\pgfqpoint{5.868575in}{1.317598in}}%
\pgfpathlineto{\pgfqpoint{5.877304in}{1.234073in}}%
\pgfpathlineto{\pgfqpoint{5.886033in}{1.174555in}}%
\pgfpathlineto{\pgfqpoint{5.892579in}{1.142802in}}%
\pgfpathlineto{\pgfqpoint{5.899126in}{1.119964in}}%
\pgfpathlineto{\pgfqpoint{5.905672in}{1.104488in}}%
\pgfpathlineto{\pgfqpoint{5.912219in}{1.095237in}}%
\pgfpathlineto{\pgfqpoint{5.916583in}{1.092164in}}%
\pgfpathlineto{\pgfqpoint{5.920947in}{1.091430in}}%
\pgfpathlineto{\pgfqpoint{5.925312in}{1.093015in}}%
\pgfpathlineto{\pgfqpoint{5.929676in}{1.096976in}}%
\pgfpathlineto{\pgfqpoint{5.934040in}{1.103453in}}%
\pgfpathlineto{\pgfqpoint{5.940587in}{1.118389in}}%
\pgfpathlineto{\pgfqpoint{5.947133in}{1.140587in}}%
\pgfpathlineto{\pgfqpoint{5.953680in}{1.171530in}}%
\pgfpathlineto{\pgfqpoint{5.960226in}{1.213050in}}%
\pgfpathlineto{\pgfqpoint{5.966773in}{1.267182in}}%
\pgfpathlineto{\pgfqpoint{5.975501in}{1.362263in}}%
\pgfpathlineto{\pgfqpoint{5.984230in}{1.485714in}}%
\pgfpathlineto{\pgfqpoint{5.995141in}{1.674409in}}%
\pgfpathlineto{\pgfqpoint{6.010416in}{1.943546in}}%
\pgfpathlineto{\pgfqpoint{6.016962in}{2.025565in}}%
\pgfpathlineto{\pgfqpoint{6.021327in}{2.058856in}}%
\pgfpathlineto{\pgfqpoint{6.023509in}{2.067968in}}%
\pgfpathlineto{\pgfqpoint{6.025691in}{2.071774in}}%
\pgfpathlineto{\pgfqpoint{6.027873in}{2.070195in}}%
\pgfpathlineto{\pgfqpoint{6.030055in}{2.063271in}}%
\pgfpathlineto{\pgfqpoint{6.034420in}{2.034097in}}%
\pgfpathlineto{\pgfqpoint{6.038784in}{1.986688in}}%
\pgfpathlineto{\pgfqpoint{6.045331in}{1.889751in}}%
\pgfpathlineto{\pgfqpoint{6.073699in}{1.422688in}}%
\pgfpathlineto{\pgfqpoint{6.082427in}{1.327098in}}%
\pgfpathlineto{\pgfqpoint{6.091156in}{1.258746in}}%
\pgfpathlineto{\pgfqpoint{6.097702in}{1.222644in}}%
\pgfpathlineto{\pgfqpoint{6.104249in}{1.197353in}}%
\pgfpathlineto{\pgfqpoint{6.110795in}{1.181273in}}%
\pgfpathlineto{\pgfqpoint{6.115160in}{1.175100in}}%
\pgfpathlineto{\pgfqpoint{6.119524in}{1.172322in}}%
\pgfpathlineto{\pgfqpoint{6.123888in}{1.172851in}}%
\pgfpathlineto{\pgfqpoint{6.128253in}{1.176709in}}%
\pgfpathlineto{\pgfqpoint{6.132617in}{1.184026in}}%
\pgfpathlineto{\pgfqpoint{6.136981in}{1.195037in}}%
\pgfpathlineto{\pgfqpoint{6.143528in}{1.219247in}}%
\pgfpathlineto{\pgfqpoint{6.150074in}{1.254085in}}%
\pgfpathlineto{\pgfqpoint{6.156621in}{1.301455in}}%
\pgfpathlineto{\pgfqpoint{6.158803in}{1.320385in}}%
\pgfpathlineto{\pgfqpoint{6.158803in}{1.320385in}}%
\pgfusepath{stroke}%
\end{pgfscope}%
\begin{pgfscope}%
\pgfsetrectcap%
\pgfsetmiterjoin%
\pgfsetlinewidth{0.803000pt}%
\definecolor{currentstroke}{rgb}{0.000000,0.000000,0.000000}%
\pgfsetstrokecolor{currentstroke}%
\pgfsetdash{}{0pt}%
\pgfpathmoveto{\pgfqpoint{3.867533in}{0.526234in}}%
\pgfpathlineto{\pgfqpoint{3.867533in}{2.145371in}}%
\pgfusepath{stroke}%
\end{pgfscope}%
\begin{pgfscope}%
\pgfsetrectcap%
\pgfsetmiterjoin%
\pgfsetlinewidth{0.803000pt}%
\definecolor{currentstroke}{rgb}{0.000000,0.000000,0.000000}%
\pgfsetstrokecolor{currentstroke}%
\pgfsetdash{}{0pt}%
\pgfpathmoveto{\pgfqpoint{6.267911in}{0.526234in}}%
\pgfpathlineto{\pgfqpoint{6.267911in}{2.145371in}}%
\pgfusepath{stroke}%
\end{pgfscope}%
\begin{pgfscope}%
\pgfsetrectcap%
\pgfsetmiterjoin%
\pgfsetlinewidth{0.803000pt}%
\definecolor{currentstroke}{rgb}{0.000000,0.000000,0.000000}%
\pgfsetstrokecolor{currentstroke}%
\pgfsetdash{}{0pt}%
\pgfpathmoveto{\pgfqpoint{3.867533in}{0.526234in}}%
\pgfpathlineto{\pgfqpoint{6.267911in}{0.526234in}}%
\pgfusepath{stroke}%
\end{pgfscope}%
\begin{pgfscope}%
\pgfsetrectcap%
\pgfsetmiterjoin%
\pgfsetlinewidth{0.803000pt}%
\definecolor{currentstroke}{rgb}{0.000000,0.000000,0.000000}%
\pgfsetstrokecolor{currentstroke}%
\pgfsetdash{}{0pt}%
\pgfpathmoveto{\pgfqpoint{3.867533in}{2.145371in}}%
\pgfpathlineto{\pgfqpoint{6.267911in}{2.145371in}}%
\pgfusepath{stroke}%
\end{pgfscope}%
\begin{pgfscope}%
\definecolor{textcolor}{rgb}{0.000000,0.000000,0.000000}%
\pgfsetstrokecolor{textcolor}%
\pgfsetfillcolor{textcolor}%
\pgftext[x=5.067722in,y=2.228704in,,base]{\color{textcolor}\rmfamily\fontsize{12.000000}{14.400000}\selectfont energy}%
\end{pgfscope}%
\end{pgfpicture}%
\makeatother%
\endgroup%
}
           \caption{Plotting Values for Large $\theta$}
           \label{fig:CP390}
        \end{center}
    \end{figure}

    \begin{figure}[H]
        \begin{center}
           \scalebox{.7}{%% Creator: Matplotlib, PGF backend
%%
%% To include the figure in your LaTeX document, write
%%   \input{<filename>.pgf}
%%
%% Make sure the required packages are loaded in your preamble
%%   \usepackage{pgf}
%%
%% Figures using additional raster images can only be included by \input if
%% they are in the same directory as the main LaTeX file. For loading figures
%% from other directories you can use the `import` package
%%   \usepackage{import}
%% and then include the figures with
%%   \import{<path to file>}{<filename>.pgf}
%%
%% Matplotlib used the following preamble
%%
\begingroup%
\makeatletter%
\begin{pgfpicture}%
\pgfpathrectangle{\pgfpointorigin}{\pgfqpoint{6.400000in}{4.800000in}}%
\pgfusepath{use as bounding box, clip}%
\begin{pgfscope}%
\pgfsetbuttcap%
\pgfsetmiterjoin%
\definecolor{currentfill}{rgb}{1.000000,1.000000,1.000000}%
\pgfsetfillcolor{currentfill}%
\pgfsetlinewidth{0.000000pt}%
\definecolor{currentstroke}{rgb}{1.000000,1.000000,1.000000}%
\pgfsetstrokecolor{currentstroke}%
\pgfsetdash{}{0pt}%
\pgfpathmoveto{\pgfqpoint{0.000000in}{0.000000in}}%
\pgfpathlineto{\pgfqpoint{6.400000in}{0.000000in}}%
\pgfpathlineto{\pgfqpoint{6.400000in}{4.800000in}}%
\pgfpathlineto{\pgfqpoint{0.000000in}{4.800000in}}%
\pgfpathclose%
\pgfusepath{fill}%
\end{pgfscope}%
\begin{pgfscope}%
\pgfsetbuttcap%
\pgfsetmiterjoin%
\definecolor{currentfill}{rgb}{1.000000,1.000000,1.000000}%
\pgfsetfillcolor{currentfill}%
\pgfsetlinewidth{0.000000pt}%
\definecolor{currentstroke}{rgb}{0.000000,0.000000,0.000000}%
\pgfsetstrokecolor{currentstroke}%
\pgfsetstrokeopacity{0.000000}%
\pgfsetdash{}{0pt}%
\pgfpathmoveto{\pgfqpoint{0.727040in}{2.870679in}}%
\pgfpathlineto{\pgfqpoint{3.196863in}{2.870679in}}%
\pgfpathlineto{\pgfqpoint{3.196863in}{4.489815in}}%
\pgfpathlineto{\pgfqpoint{0.727040in}{4.489815in}}%
\pgfpathclose%
\pgfusepath{fill}%
\end{pgfscope}%
\begin{pgfscope}%
\pgfsetbuttcap%
\pgfsetroundjoin%
\definecolor{currentfill}{rgb}{0.000000,0.000000,0.000000}%
\pgfsetfillcolor{currentfill}%
\pgfsetlinewidth{0.803000pt}%
\definecolor{currentstroke}{rgb}{0.000000,0.000000,0.000000}%
\pgfsetstrokecolor{currentstroke}%
\pgfsetdash{}{0pt}%
\pgfsys@defobject{currentmarker}{\pgfqpoint{0.000000in}{-0.048611in}}{\pgfqpoint{0.000000in}{0.000000in}}{%
\pgfpathmoveto{\pgfqpoint{0.000000in}{0.000000in}}%
\pgfpathlineto{\pgfqpoint{0.000000in}{-0.048611in}}%
\pgfusepath{stroke,fill}%
}%
\begin{pgfscope}%
\pgfsys@transformshift{0.839304in}{2.870679in}%
\pgfsys@useobject{currentmarker}{}%
\end{pgfscope}%
\end{pgfscope}%
\begin{pgfscope}%
\definecolor{textcolor}{rgb}{0.000000,0.000000,0.000000}%
\pgfsetstrokecolor{textcolor}%
\pgfsetfillcolor{textcolor}%
\pgftext[x=0.839304in,y=2.773457in,,top]{\color{textcolor}\rmfamily\fontsize{10.000000}{12.000000}\selectfont \(\displaystyle 0.0\)}%
\end{pgfscope}%
\begin{pgfscope}%
\pgfsetbuttcap%
\pgfsetroundjoin%
\definecolor{currentfill}{rgb}{0.000000,0.000000,0.000000}%
\pgfsetfillcolor{currentfill}%
\pgfsetlinewidth{0.803000pt}%
\definecolor{currentstroke}{rgb}{0.000000,0.000000,0.000000}%
\pgfsetstrokecolor{currentstroke}%
\pgfsetdash{}{0pt}%
\pgfsys@defobject{currentmarker}{\pgfqpoint{0.000000in}{-0.048611in}}{\pgfqpoint{0.000000in}{0.000000in}}{%
\pgfpathmoveto{\pgfqpoint{0.000000in}{0.000000in}}%
\pgfpathlineto{\pgfqpoint{0.000000in}{-0.048611in}}%
\pgfusepath{stroke,fill}%
}%
\begin{pgfscope}%
\pgfsys@transformshift{1.400628in}{2.870679in}%
\pgfsys@useobject{currentmarker}{}%
\end{pgfscope}%
\end{pgfscope}%
\begin{pgfscope}%
\definecolor{textcolor}{rgb}{0.000000,0.000000,0.000000}%
\pgfsetstrokecolor{textcolor}%
\pgfsetfillcolor{textcolor}%
\pgftext[x=1.400628in,y=2.773457in,,top]{\color{textcolor}\rmfamily\fontsize{10.000000}{12.000000}\selectfont \(\displaystyle 2.5\)}%
\end{pgfscope}%
\begin{pgfscope}%
\pgfsetbuttcap%
\pgfsetroundjoin%
\definecolor{currentfill}{rgb}{0.000000,0.000000,0.000000}%
\pgfsetfillcolor{currentfill}%
\pgfsetlinewidth{0.803000pt}%
\definecolor{currentstroke}{rgb}{0.000000,0.000000,0.000000}%
\pgfsetstrokecolor{currentstroke}%
\pgfsetdash{}{0pt}%
\pgfsys@defobject{currentmarker}{\pgfqpoint{0.000000in}{-0.048611in}}{\pgfqpoint{0.000000in}{0.000000in}}{%
\pgfpathmoveto{\pgfqpoint{0.000000in}{0.000000in}}%
\pgfpathlineto{\pgfqpoint{0.000000in}{-0.048611in}}%
\pgfusepath{stroke,fill}%
}%
\begin{pgfscope}%
\pgfsys@transformshift{1.961951in}{2.870679in}%
\pgfsys@useobject{currentmarker}{}%
\end{pgfscope}%
\end{pgfscope}%
\begin{pgfscope}%
\definecolor{textcolor}{rgb}{0.000000,0.000000,0.000000}%
\pgfsetstrokecolor{textcolor}%
\pgfsetfillcolor{textcolor}%
\pgftext[x=1.961951in,y=2.773457in,,top]{\color{textcolor}\rmfamily\fontsize{10.000000}{12.000000}\selectfont \(\displaystyle 5.0\)}%
\end{pgfscope}%
\begin{pgfscope}%
\pgfsetbuttcap%
\pgfsetroundjoin%
\definecolor{currentfill}{rgb}{0.000000,0.000000,0.000000}%
\pgfsetfillcolor{currentfill}%
\pgfsetlinewidth{0.803000pt}%
\definecolor{currentstroke}{rgb}{0.000000,0.000000,0.000000}%
\pgfsetstrokecolor{currentstroke}%
\pgfsetdash{}{0pt}%
\pgfsys@defobject{currentmarker}{\pgfqpoint{0.000000in}{-0.048611in}}{\pgfqpoint{0.000000in}{0.000000in}}{%
\pgfpathmoveto{\pgfqpoint{0.000000in}{0.000000in}}%
\pgfpathlineto{\pgfqpoint{0.000000in}{-0.048611in}}%
\pgfusepath{stroke,fill}%
}%
\begin{pgfscope}%
\pgfsys@transformshift{2.523275in}{2.870679in}%
\pgfsys@useobject{currentmarker}{}%
\end{pgfscope}%
\end{pgfscope}%
\begin{pgfscope}%
\definecolor{textcolor}{rgb}{0.000000,0.000000,0.000000}%
\pgfsetstrokecolor{textcolor}%
\pgfsetfillcolor{textcolor}%
\pgftext[x=2.523275in,y=2.773457in,,top]{\color{textcolor}\rmfamily\fontsize{10.000000}{12.000000}\selectfont \(\displaystyle 7.5\)}%
\end{pgfscope}%
\begin{pgfscope}%
\pgfsetbuttcap%
\pgfsetroundjoin%
\definecolor{currentfill}{rgb}{0.000000,0.000000,0.000000}%
\pgfsetfillcolor{currentfill}%
\pgfsetlinewidth{0.803000pt}%
\definecolor{currentstroke}{rgb}{0.000000,0.000000,0.000000}%
\pgfsetstrokecolor{currentstroke}%
\pgfsetdash{}{0pt}%
\pgfsys@defobject{currentmarker}{\pgfqpoint{0.000000in}{-0.048611in}}{\pgfqpoint{0.000000in}{0.000000in}}{%
\pgfpathmoveto{\pgfqpoint{0.000000in}{0.000000in}}%
\pgfpathlineto{\pgfqpoint{0.000000in}{-0.048611in}}%
\pgfusepath{stroke,fill}%
}%
\begin{pgfscope}%
\pgfsys@transformshift{3.084598in}{2.870679in}%
\pgfsys@useobject{currentmarker}{}%
\end{pgfscope}%
\end{pgfscope}%
\begin{pgfscope}%
\definecolor{textcolor}{rgb}{0.000000,0.000000,0.000000}%
\pgfsetstrokecolor{textcolor}%
\pgfsetfillcolor{textcolor}%
\pgftext[x=3.084598in,y=2.773457in,,top]{\color{textcolor}\rmfamily\fontsize{10.000000}{12.000000}\selectfont \(\displaystyle 10.0\)}%
\end{pgfscope}%
\begin{pgfscope}%
\definecolor{textcolor}{rgb}{0.000000,0.000000,0.000000}%
\pgfsetstrokecolor{textcolor}%
\pgfsetfillcolor{textcolor}%
\pgftext[x=1.961951in,y=2.594444in,,top]{\color{textcolor}\rmfamily\fontsize{10.000000}{12.000000}\selectfont time (s)}%
\end{pgfscope}%
\begin{pgfscope}%
\pgfsetbuttcap%
\pgfsetroundjoin%
\definecolor{currentfill}{rgb}{0.000000,0.000000,0.000000}%
\pgfsetfillcolor{currentfill}%
\pgfsetlinewidth{0.803000pt}%
\definecolor{currentstroke}{rgb}{0.000000,0.000000,0.000000}%
\pgfsetstrokecolor{currentstroke}%
\pgfsetdash{}{0pt}%
\pgfsys@defobject{currentmarker}{\pgfqpoint{-0.048611in}{0.000000in}}{\pgfqpoint{0.000000in}{0.000000in}}{%
\pgfpathmoveto{\pgfqpoint{0.000000in}{0.000000in}}%
\pgfpathlineto{\pgfqpoint{-0.048611in}{0.000000in}}%
\pgfusepath{stroke,fill}%
}%
\begin{pgfscope}%
\pgfsys@transformshift{0.727040in}{3.033457in}%
\pgfsys@useobject{currentmarker}{}%
\end{pgfscope}%
\end{pgfscope}%
\begin{pgfscope}%
\definecolor{textcolor}{rgb}{0.000000,0.000000,0.000000}%
\pgfsetstrokecolor{textcolor}%
\pgfsetfillcolor{textcolor}%
\pgftext[x=0.274878in,y=2.985231in,left,base]{\color{textcolor}\rmfamily\fontsize{10.000000}{12.000000}\selectfont \(\displaystyle -0.50\)}%
\end{pgfscope}%
\begin{pgfscope}%
\pgfsetbuttcap%
\pgfsetroundjoin%
\definecolor{currentfill}{rgb}{0.000000,0.000000,0.000000}%
\pgfsetfillcolor{currentfill}%
\pgfsetlinewidth{0.803000pt}%
\definecolor{currentstroke}{rgb}{0.000000,0.000000,0.000000}%
\pgfsetstrokecolor{currentstroke}%
\pgfsetdash{}{0pt}%
\pgfsys@defobject{currentmarker}{\pgfqpoint{-0.048611in}{0.000000in}}{\pgfqpoint{0.000000in}{0.000000in}}{%
\pgfpathmoveto{\pgfqpoint{0.000000in}{0.000000in}}%
\pgfpathlineto{\pgfqpoint{-0.048611in}{0.000000in}}%
\pgfusepath{stroke,fill}%
}%
\begin{pgfscope}%
\pgfsys@transformshift{0.727040in}{3.371248in}%
\pgfsys@useobject{currentmarker}{}%
\end{pgfscope}%
\end{pgfscope}%
\begin{pgfscope}%
\definecolor{textcolor}{rgb}{0.000000,0.000000,0.000000}%
\pgfsetstrokecolor{textcolor}%
\pgfsetfillcolor{textcolor}%
\pgftext[x=0.274878in,y=3.323022in,left,base]{\color{textcolor}\rmfamily\fontsize{10.000000}{12.000000}\selectfont \(\displaystyle -0.25\)}%
\end{pgfscope}%
\begin{pgfscope}%
\pgfsetbuttcap%
\pgfsetroundjoin%
\definecolor{currentfill}{rgb}{0.000000,0.000000,0.000000}%
\pgfsetfillcolor{currentfill}%
\pgfsetlinewidth{0.803000pt}%
\definecolor{currentstroke}{rgb}{0.000000,0.000000,0.000000}%
\pgfsetstrokecolor{currentstroke}%
\pgfsetdash{}{0pt}%
\pgfsys@defobject{currentmarker}{\pgfqpoint{-0.048611in}{0.000000in}}{\pgfqpoint{0.000000in}{0.000000in}}{%
\pgfpathmoveto{\pgfqpoint{0.000000in}{0.000000in}}%
\pgfpathlineto{\pgfqpoint{-0.048611in}{0.000000in}}%
\pgfusepath{stroke,fill}%
}%
\begin{pgfscope}%
\pgfsys@transformshift{0.727040in}{3.709039in}%
\pgfsys@useobject{currentmarker}{}%
\end{pgfscope}%
\end{pgfscope}%
\begin{pgfscope}%
\definecolor{textcolor}{rgb}{0.000000,0.000000,0.000000}%
\pgfsetstrokecolor{textcolor}%
\pgfsetfillcolor{textcolor}%
\pgftext[x=0.382903in,y=3.660814in,left,base]{\color{textcolor}\rmfamily\fontsize{10.000000}{12.000000}\selectfont \(\displaystyle 0.00\)}%
\end{pgfscope}%
\begin{pgfscope}%
\pgfsetbuttcap%
\pgfsetroundjoin%
\definecolor{currentfill}{rgb}{0.000000,0.000000,0.000000}%
\pgfsetfillcolor{currentfill}%
\pgfsetlinewidth{0.803000pt}%
\definecolor{currentstroke}{rgb}{0.000000,0.000000,0.000000}%
\pgfsetstrokecolor{currentstroke}%
\pgfsetdash{}{0pt}%
\pgfsys@defobject{currentmarker}{\pgfqpoint{-0.048611in}{0.000000in}}{\pgfqpoint{0.000000in}{0.000000in}}{%
\pgfpathmoveto{\pgfqpoint{0.000000in}{0.000000in}}%
\pgfpathlineto{\pgfqpoint{-0.048611in}{0.000000in}}%
\pgfusepath{stroke,fill}%
}%
\begin{pgfscope}%
\pgfsys@transformshift{0.727040in}{4.046830in}%
\pgfsys@useobject{currentmarker}{}%
\end{pgfscope}%
\end{pgfscope}%
\begin{pgfscope}%
\definecolor{textcolor}{rgb}{0.000000,0.000000,0.000000}%
\pgfsetstrokecolor{textcolor}%
\pgfsetfillcolor{textcolor}%
\pgftext[x=0.382903in,y=3.998605in,left,base]{\color{textcolor}\rmfamily\fontsize{10.000000}{12.000000}\selectfont \(\displaystyle 0.25\)}%
\end{pgfscope}%
\begin{pgfscope}%
\pgfsetbuttcap%
\pgfsetroundjoin%
\definecolor{currentfill}{rgb}{0.000000,0.000000,0.000000}%
\pgfsetfillcolor{currentfill}%
\pgfsetlinewidth{0.803000pt}%
\definecolor{currentstroke}{rgb}{0.000000,0.000000,0.000000}%
\pgfsetstrokecolor{currentstroke}%
\pgfsetdash{}{0pt}%
\pgfsys@defobject{currentmarker}{\pgfqpoint{-0.048611in}{0.000000in}}{\pgfqpoint{0.000000in}{0.000000in}}{%
\pgfpathmoveto{\pgfqpoint{0.000000in}{0.000000in}}%
\pgfpathlineto{\pgfqpoint{-0.048611in}{0.000000in}}%
\pgfusepath{stroke,fill}%
}%
\begin{pgfscope}%
\pgfsys@transformshift{0.727040in}{4.384621in}%
\pgfsys@useobject{currentmarker}{}%
\end{pgfscope}%
\end{pgfscope}%
\begin{pgfscope}%
\definecolor{textcolor}{rgb}{0.000000,0.000000,0.000000}%
\pgfsetstrokecolor{textcolor}%
\pgfsetfillcolor{textcolor}%
\pgftext[x=0.382903in,y=4.336396in,left,base]{\color{textcolor}\rmfamily\fontsize{10.000000}{12.000000}\selectfont \(\displaystyle 0.50\)}%
\end{pgfscope}%
\begin{pgfscope}%
\definecolor{textcolor}{rgb}{0.000000,0.000000,0.000000}%
\pgfsetstrokecolor{textcolor}%
\pgfsetfillcolor{textcolor}%
\pgftext[x=0.219323in,y=3.680247in,,bottom,rotate=90.000000]{\color{textcolor}\rmfamily\fontsize{10.000000}{12.000000}\selectfont angle (rad)}%
\end{pgfscope}%
\begin{pgfscope}%
\pgfpathrectangle{\pgfqpoint{0.727040in}{2.870679in}}{\pgfqpoint{2.469823in}{1.619136in}}%
\pgfusepath{clip}%
\pgfsetrectcap%
\pgfsetroundjoin%
\pgfsetlinewidth{1.505625pt}%
\definecolor{currentstroke}{rgb}{0.000000,0.000000,1.000000}%
\pgfsetstrokecolor{currentstroke}%
\pgfsetdash{}{0pt}%
\pgfpathmoveto{\pgfqpoint{0.839304in}{3.944862in}}%
\pgfpathlineto{\pgfqpoint{0.843795in}{3.944275in}}%
\pgfpathlineto{\pgfqpoint{0.848285in}{3.941344in}}%
\pgfpathlineto{\pgfqpoint{0.852776in}{3.936085in}}%
\pgfpathlineto{\pgfqpoint{0.859512in}{3.923927in}}%
\pgfpathlineto{\pgfqpoint{0.866248in}{3.906860in}}%
\pgfpathlineto{\pgfqpoint{0.875229in}{3.877077in}}%
\pgfpathlineto{\pgfqpoint{0.886455in}{3.830156in}}%
\pgfpathlineto{\pgfqpoint{0.899927in}{3.763383in}}%
\pgfpathlineto{\pgfqpoint{0.942588in}{3.543799in}}%
\pgfpathlineto{\pgfqpoint{0.953814in}{3.501426in}}%
\pgfpathlineto{\pgfqpoint{0.962795in}{3.476347in}}%
\pgfpathlineto{\pgfqpoint{0.969531in}{3.463410in}}%
\pgfpathlineto{\pgfqpoint{0.976267in}{3.455879in}}%
\pgfpathlineto{\pgfqpoint{0.980758in}{3.453967in}}%
\pgfpathlineto{\pgfqpoint{0.985248in}{3.454584in}}%
\pgfpathlineto{\pgfqpoint{0.989739in}{3.457735in}}%
\pgfpathlineto{\pgfqpoint{0.994230in}{3.463401in}}%
\pgfpathlineto{\pgfqpoint{1.000965in}{3.476516in}}%
\pgfpathlineto{\pgfqpoint{1.007701in}{3.494938in}}%
\pgfpathlineto{\pgfqpoint{1.016682in}{3.527101in}}%
\pgfpathlineto{\pgfqpoint{1.027909in}{3.577794in}}%
\pgfpathlineto{\pgfqpoint{1.041381in}{3.649963in}}%
\pgfpathlineto{\pgfqpoint{1.084041in}{3.887478in}}%
\pgfpathlineto{\pgfqpoint{1.095268in}{3.933363in}}%
\pgfpathlineto{\pgfqpoint{1.104249in}{3.960550in}}%
\pgfpathlineto{\pgfqpoint{1.110985in}{3.974597in}}%
\pgfpathlineto{\pgfqpoint{1.117721in}{3.982807in}}%
\pgfpathlineto{\pgfqpoint{1.122211in}{3.984921in}}%
\pgfpathlineto{\pgfqpoint{1.126702in}{3.984303in}}%
\pgfpathlineto{\pgfqpoint{1.131192in}{3.980948in}}%
\pgfpathlineto{\pgfqpoint{1.135683in}{3.974873in}}%
\pgfpathlineto{\pgfqpoint{1.142419in}{3.960772in}}%
\pgfpathlineto{\pgfqpoint{1.149155in}{3.940930in}}%
\pgfpathlineto{\pgfqpoint{1.158136in}{3.906250in}}%
\pgfpathlineto{\pgfqpoint{1.169362in}{3.851539in}}%
\pgfpathlineto{\pgfqpoint{1.182834in}{3.773583in}}%
\pgfpathlineto{\pgfqpoint{1.225495in}{3.516621in}}%
\pgfpathlineto{\pgfqpoint{1.236721in}{3.466872in}}%
\pgfpathlineto{\pgfqpoint{1.245702in}{3.437342in}}%
\pgfpathlineto{\pgfqpoint{1.252438in}{3.422038in}}%
\pgfpathlineto{\pgfqpoint{1.259174in}{3.413036in}}%
\pgfpathlineto{\pgfqpoint{1.263665in}{3.410662in}}%
\pgfpathlineto{\pgfqpoint{1.268155in}{3.411238in}}%
\pgfpathlineto{\pgfqpoint{1.272646in}{3.414773in}}%
\pgfpathlineto{\pgfqpoint{1.277137in}{3.421246in}}%
\pgfpathlineto{\pgfqpoint{1.283872in}{3.436350in}}%
\pgfpathlineto{\pgfqpoint{1.290608in}{3.457665in}}%
\pgfpathlineto{\pgfqpoint{1.299589in}{3.494991in}}%
\pgfpathlineto{\pgfqpoint{1.310816in}{3.553968in}}%
\pgfpathlineto{\pgfqpoint{1.324288in}{3.638116in}}%
\pgfpathlineto{\pgfqpoint{1.366948in}{3.916178in}}%
\pgfpathlineto{\pgfqpoint{1.378175in}{3.970191in}}%
\pgfpathlineto{\pgfqpoint{1.387156in}{4.002341in}}%
\pgfpathlineto{\pgfqpoint{1.393892in}{4.019076in}}%
\pgfpathlineto{\pgfqpoint{1.398382in}{4.026475in}}%
\pgfpathlineto{\pgfqpoint{1.402873in}{4.030764in}}%
\pgfpathlineto{\pgfqpoint{1.407364in}{4.031884in}}%
\pgfpathlineto{\pgfqpoint{1.411854in}{4.029810in}}%
\pgfpathlineto{\pgfqpoint{1.416345in}{4.024547in}}%
\pgfpathlineto{\pgfqpoint{1.420835in}{4.016129in}}%
\pgfpathlineto{\pgfqpoint{1.427571in}{3.997743in}}%
\pgfpathlineto{\pgfqpoint{1.434307in}{3.972770in}}%
\pgfpathlineto{\pgfqpoint{1.443288in}{3.930114in}}%
\pgfpathlineto{\pgfqpoint{1.454515in}{3.864036in}}%
\pgfpathlineto{\pgfqpoint{1.470232in}{3.754926in}}%
\pgfpathlineto{\pgfqpoint{1.503911in}{3.513866in}}%
\pgfpathlineto{\pgfqpoint{1.515138in}{3.449445in}}%
\pgfpathlineto{\pgfqpoint{1.524119in}{3.408775in}}%
\pgfpathlineto{\pgfqpoint{1.530855in}{3.385757in}}%
\pgfpathlineto{\pgfqpoint{1.537591in}{3.369780in}}%
\pgfpathlineto{\pgfqpoint{1.542081in}{3.363246in}}%
\pgfpathlineto{\pgfqpoint{1.546572in}{3.360102in}}%
\pgfpathlineto{\pgfqpoint{1.551062in}{3.360396in}}%
\pgfpathlineto{\pgfqpoint{1.555553in}{3.364142in}}%
\pgfpathlineto{\pgfqpoint{1.560044in}{3.371319in}}%
\pgfpathlineto{\pgfqpoint{1.566779in}{3.388395in}}%
\pgfpathlineto{\pgfqpoint{1.573515in}{3.412748in}}%
\pgfpathlineto{\pgfqpoint{1.582496in}{3.455687in}}%
\pgfpathlineto{\pgfqpoint{1.593723in}{3.523904in}}%
\pgfpathlineto{\pgfqpoint{1.607195in}{3.621692in}}%
\pgfpathlineto{\pgfqpoint{1.649855in}{3.947532in}}%
\pgfpathlineto{\pgfqpoint{1.661082in}{4.011523in}}%
\pgfpathlineto{\pgfqpoint{1.670063in}{4.049947in}}%
\pgfpathlineto{\pgfqpoint{1.676799in}{4.070220in}}%
\pgfpathlineto{\pgfqpoint{1.683535in}{4.082604in}}%
\pgfpathlineto{\pgfqpoint{1.688025in}{4.086309in}}%
\pgfpathlineto{\pgfqpoint{1.692516in}{4.086305in}}%
\pgfpathlineto{\pgfqpoint{1.697006in}{4.082576in}}%
\pgfpathlineto{\pgfqpoint{1.701497in}{4.075138in}}%
\pgfpathlineto{\pgfqpoint{1.708233in}{4.057156in}}%
\pgfpathlineto{\pgfqpoint{1.714969in}{4.031291in}}%
\pgfpathlineto{\pgfqpoint{1.723950in}{3.985441in}}%
\pgfpathlineto{\pgfqpoint{1.735176in}{3.912290in}}%
\pgfpathlineto{\pgfqpoint{1.748648in}{3.807055in}}%
\pgfpathlineto{\pgfqpoint{1.793554in}{3.438886in}}%
\pgfpathlineto{\pgfqpoint{1.804781in}{3.372555in}}%
\pgfpathlineto{\pgfqpoint{1.813762in}{3.333726in}}%
\pgfpathlineto{\pgfqpoint{1.820498in}{3.314021in}}%
\pgfpathlineto{\pgfqpoint{1.824988in}{3.305654in}}%
\pgfpathlineto{\pgfqpoint{1.829479in}{3.301224in}}%
\pgfpathlineto{\pgfqpoint{1.833969in}{3.300792in}}%
\pgfpathlineto{\pgfqpoint{1.838460in}{3.304381in}}%
\pgfpathlineto{\pgfqpoint{1.842951in}{3.311975in}}%
\pgfpathlineto{\pgfqpoint{1.847441in}{3.323522in}}%
\pgfpathlineto{\pgfqpoint{1.854177in}{3.348042in}}%
\pgfpathlineto{\pgfqpoint{1.860913in}{3.380773in}}%
\pgfpathlineto{\pgfqpoint{1.869894in}{3.436032in}}%
\pgfpathlineto{\pgfqpoint{1.881121in}{3.520843in}}%
\pgfpathlineto{\pgfqpoint{1.896838in}{3.659742in}}%
\pgfpathlineto{\pgfqpoint{1.928272in}{3.945091in}}%
\pgfpathlineto{\pgfqpoint{1.939498in}{4.028448in}}%
\pgfpathlineto{\pgfqpoint{1.948479in}{4.081848in}}%
\pgfpathlineto{\pgfqpoint{1.957461in}{4.121034in}}%
\pgfpathlineto{\pgfqpoint{1.964196in}{4.140083in}}%
\pgfpathlineto{\pgfqpoint{1.968687in}{4.147578in}}%
\pgfpathlineto{\pgfqpoint{1.973178in}{4.150798in}}%
\pgfpathlineto{\pgfqpoint{1.977668in}{4.149690in}}%
\pgfpathlineto{\pgfqpoint{1.982159in}{4.144246in}}%
\pgfpathlineto{\pgfqpoint{1.986649in}{4.134497in}}%
\pgfpathlineto{\pgfqpoint{1.993385in}{4.111968in}}%
\pgfpathlineto{\pgfqpoint{2.000121in}{4.080339in}}%
\pgfpathlineto{\pgfqpoint{2.009102in}{4.025104in}}%
\pgfpathlineto{\pgfqpoint{2.020329in}{3.937963in}}%
\pgfpathlineto{\pgfqpoint{2.033801in}{3.813710in}}%
\pgfpathlineto{\pgfqpoint{2.076461in}{3.403160in}}%
\pgfpathlineto{\pgfqpoint{2.087688in}{3.323323in}}%
\pgfpathlineto{\pgfqpoint{2.096669in}{3.275679in}}%
\pgfpathlineto{\pgfqpoint{2.103405in}{3.250751in}}%
\pgfpathlineto{\pgfqpoint{2.107895in}{3.239629in}}%
\pgfpathlineto{\pgfqpoint{2.112386in}{3.233058in}}%
\pgfpathlineto{\pgfqpoint{2.116876in}{3.231121in}}%
\pgfpathlineto{\pgfqpoint{2.121367in}{3.233858in}}%
\pgfpathlineto{\pgfqpoint{2.125858in}{3.241264in}}%
\pgfpathlineto{\pgfqpoint{2.130348in}{3.253293in}}%
\pgfpathlineto{\pgfqpoint{2.137084in}{3.279792in}}%
\pgfpathlineto{\pgfqpoint{2.143820in}{3.315983in}}%
\pgfpathlineto{\pgfqpoint{2.152801in}{3.378064in}}%
\pgfpathlineto{\pgfqpoint{2.164028in}{3.474635in}}%
\pgfpathlineto{\pgfqpoint{2.179745in}{3.634835in}}%
\pgfpathlineto{\pgfqpoint{2.215669in}{4.012184in}}%
\pgfpathlineto{\pgfqpoint{2.226896in}{4.104377in}}%
\pgfpathlineto{\pgfqpoint{2.235877in}{4.161633in}}%
\pgfpathlineto{\pgfqpoint{2.242613in}{4.193369in}}%
\pgfpathlineto{\pgfqpoint{2.249349in}{4.214656in}}%
\pgfpathlineto{\pgfqpoint{2.253839in}{4.222774in}}%
\pgfpathlineto{\pgfqpoint{2.258330in}{4.225913in}}%
\pgfpathlineto{\pgfqpoint{2.260575in}{4.225597in}}%
\pgfpathlineto{\pgfqpoint{2.265066in}{4.221183in}}%
\pgfpathlineto{\pgfqpoint{2.269556in}{4.211745in}}%
\pgfpathlineto{\pgfqpoint{2.274047in}{4.197349in}}%
\pgfpathlineto{\pgfqpoint{2.280783in}{4.166713in}}%
\pgfpathlineto{\pgfqpoint{2.287519in}{4.125751in}}%
\pgfpathlineto{\pgfqpoint{2.296500in}{4.056484in}}%
\pgfpathlineto{\pgfqpoint{2.307726in}{3.949973in}}%
\pgfpathlineto{\pgfqpoint{2.323443in}{3.775122in}}%
\pgfpathlineto{\pgfqpoint{2.357123in}{3.391759in}}%
\pgfpathlineto{\pgfqpoint{2.368349in}{3.290088in}}%
\pgfpathlineto{\pgfqpoint{2.377330in}{3.226162in}}%
\pgfpathlineto{\pgfqpoint{2.384066in}{3.190119in}}%
\pgfpathlineto{\pgfqpoint{2.390802in}{3.165224in}}%
\pgfpathlineto{\pgfqpoint{2.395293in}{3.155125in}}%
\pgfpathlineto{\pgfqpoint{2.399783in}{3.150368in}}%
\pgfpathlineto{\pgfqpoint{2.402029in}{3.150017in}}%
\pgfpathlineto{\pgfqpoint{2.404274in}{3.151022in}}%
\pgfpathlineto{\pgfqpoint{2.408765in}{3.157104in}}%
\pgfpathlineto{\pgfqpoint{2.413255in}{3.168585in}}%
\pgfpathlineto{\pgfqpoint{2.419991in}{3.195737in}}%
\pgfpathlineto{\pgfqpoint{2.426727in}{3.234360in}}%
\pgfpathlineto{\pgfqpoint{2.435708in}{3.302405in}}%
\pgfpathlineto{\pgfqpoint{2.446935in}{3.410567in}}%
\pgfpathlineto{\pgfqpoint{2.460406in}{3.565868in}}%
\pgfpathlineto{\pgfqpoint{2.505312in}{4.108291in}}%
\pgfpathlineto{\pgfqpoint{2.516539in}{4.206028in}}%
\pgfpathlineto{\pgfqpoint{2.525520in}{4.263423in}}%
\pgfpathlineto{\pgfqpoint{2.532256in}{4.292760in}}%
\pgfpathlineto{\pgfqpoint{2.536746in}{4.305384in}}%
\pgfpathlineto{\pgfqpoint{2.541237in}{4.312290in}}%
\pgfpathlineto{\pgfqpoint{2.543482in}{4.313569in}}%
\pgfpathlineto{\pgfqpoint{2.545727in}{4.313390in}}%
\pgfpathlineto{\pgfqpoint{2.547973in}{4.311750in}}%
\pgfpathlineto{\pgfqpoint{2.552463in}{4.304089in}}%
\pgfpathlineto{\pgfqpoint{2.556954in}{4.290631in}}%
\pgfpathlineto{\pgfqpoint{2.563690in}{4.259796in}}%
\pgfpathlineto{\pgfqpoint{2.570426in}{4.216688in}}%
\pgfpathlineto{\pgfqpoint{2.579407in}{4.141556in}}%
\pgfpathlineto{\pgfqpoint{2.590633in}{4.023094in}}%
\pgfpathlineto{\pgfqpoint{2.604105in}{3.854080in}}%
\pgfpathlineto{\pgfqpoint{2.646766in}{3.294202in}}%
\pgfpathlineto{\pgfqpoint{2.657992in}{3.184760in}}%
\pgfpathlineto{\pgfqpoint{2.666973in}{3.119019in}}%
\pgfpathlineto{\pgfqpoint{2.673709in}{3.084215in}}%
\pgfpathlineto{\pgfqpoint{2.678200in}{3.068384in}}%
\pgfpathlineto{\pgfqpoint{2.682690in}{3.058655in}}%
\pgfpathlineto{\pgfqpoint{2.687181in}{3.055142in}}%
\pgfpathlineto{\pgfqpoint{2.689426in}{3.055737in}}%
\pgfpathlineto{\pgfqpoint{2.691672in}{3.057904in}}%
\pgfpathlineto{\pgfqpoint{2.696162in}{3.066943in}}%
\pgfpathlineto{\pgfqpoint{2.700653in}{3.082208in}}%
\pgfpathlineto{\pgfqpoint{2.707389in}{3.116527in}}%
\pgfpathlineto{\pgfqpoint{2.714125in}{3.163996in}}%
\pgfpathlineto{\pgfqpoint{2.723106in}{3.246181in}}%
\pgfpathlineto{\pgfqpoint{2.734332in}{3.375126in}}%
\pgfpathlineto{\pgfqpoint{2.747804in}{3.558406in}}%
\pgfpathlineto{\pgfqpoint{2.790465in}{4.161902in}}%
\pgfpathlineto{\pgfqpoint{2.801691in}{4.279058in}}%
\pgfpathlineto{\pgfqpoint{2.810672in}{4.349146in}}%
\pgfpathlineto{\pgfqpoint{2.817408in}{4.386053in}}%
\pgfpathlineto{\pgfqpoint{2.821899in}{4.402713in}}%
\pgfpathlineto{\pgfqpoint{2.826389in}{4.412807in}}%
\pgfpathlineto{\pgfqpoint{2.828635in}{4.415353in}}%
\pgfpathlineto{\pgfqpoint{2.830880in}{4.416218in}}%
\pgfpathlineto{\pgfqpoint{2.833125in}{4.415396in}}%
\pgfpathlineto{\pgfqpoint{2.835370in}{4.412886in}}%
\pgfpathlineto{\pgfqpoint{2.839861in}{4.402812in}}%
\pgfpathlineto{\pgfqpoint{2.844352in}{4.386054in}}%
\pgfpathlineto{\pgfqpoint{2.851087in}{4.348655in}}%
\pgfpathlineto{\pgfqpoint{2.857823in}{4.297137in}}%
\pgfpathlineto{\pgfqpoint{2.866804in}{4.208156in}}%
\pgfpathlineto{\pgfqpoint{2.878031in}{4.068757in}}%
\pgfpathlineto{\pgfqpoint{2.891503in}{3.870780in}}%
\pgfpathlineto{\pgfqpoint{2.934163in}{3.219430in}}%
\pgfpathlineto{\pgfqpoint{2.945390in}{3.093012in}}%
\pgfpathlineto{\pgfqpoint{2.954371in}{3.017308in}}%
\pgfpathlineto{\pgfqpoint{2.961107in}{2.977351in}}%
\pgfpathlineto{\pgfqpoint{2.965597in}{2.959239in}}%
\pgfpathlineto{\pgfqpoint{2.970088in}{2.948172in}}%
\pgfpathlineto{\pgfqpoint{2.972333in}{2.945323in}}%
\pgfpathlineto{\pgfqpoint{2.974579in}{2.944276in}}%
\pgfpathlineto{\pgfqpoint{2.976824in}{2.945039in}}%
\pgfpathlineto{\pgfqpoint{2.979069in}{2.947613in}}%
\pgfpathlineto{\pgfqpoint{2.983560in}{2.958185in}}%
\pgfpathlineto{\pgfqpoint{2.988050in}{2.975934in}}%
\pgfpathlineto{\pgfqpoint{2.994786in}{3.015733in}}%
\pgfpathlineto{\pgfqpoint{3.001522in}{3.070725in}}%
\pgfpathlineto{\pgfqpoint{3.010503in}{3.165935in}}%
\pgfpathlineto{\pgfqpoint{3.021730in}{3.315450in}}%
\pgfpathlineto{\pgfqpoint{3.035202in}{3.528332in}}%
\pgfpathlineto{\pgfqpoint{3.077862in}{4.232413in}}%
\pgfpathlineto{\pgfqpoint{3.084598in}{4.319191in}}%
\pgfpathlineto{\pgfqpoint{3.084598in}{4.319191in}}%
\pgfusepath{stroke}%
\end{pgfscope}%
\begin{pgfscope}%
\pgfsetrectcap%
\pgfsetmiterjoin%
\pgfsetlinewidth{0.803000pt}%
\definecolor{currentstroke}{rgb}{0.000000,0.000000,0.000000}%
\pgfsetstrokecolor{currentstroke}%
\pgfsetdash{}{0pt}%
\pgfpathmoveto{\pgfqpoint{0.727040in}{2.870679in}}%
\pgfpathlineto{\pgfqpoint{0.727040in}{4.489815in}}%
\pgfusepath{stroke}%
\end{pgfscope}%
\begin{pgfscope}%
\pgfsetrectcap%
\pgfsetmiterjoin%
\pgfsetlinewidth{0.803000pt}%
\definecolor{currentstroke}{rgb}{0.000000,0.000000,0.000000}%
\pgfsetstrokecolor{currentstroke}%
\pgfsetdash{}{0pt}%
\pgfpathmoveto{\pgfqpoint{3.196863in}{2.870679in}}%
\pgfpathlineto{\pgfqpoint{3.196863in}{4.489815in}}%
\pgfusepath{stroke}%
\end{pgfscope}%
\begin{pgfscope}%
\pgfsetrectcap%
\pgfsetmiterjoin%
\pgfsetlinewidth{0.803000pt}%
\definecolor{currentstroke}{rgb}{0.000000,0.000000,0.000000}%
\pgfsetstrokecolor{currentstroke}%
\pgfsetdash{}{0pt}%
\pgfpathmoveto{\pgfqpoint{0.727040in}{2.870679in}}%
\pgfpathlineto{\pgfqpoint{3.196863in}{2.870679in}}%
\pgfusepath{stroke}%
\end{pgfscope}%
\begin{pgfscope}%
\pgfsetrectcap%
\pgfsetmiterjoin%
\pgfsetlinewidth{0.803000pt}%
\definecolor{currentstroke}{rgb}{0.000000,0.000000,0.000000}%
\pgfsetstrokecolor{currentstroke}%
\pgfsetdash{}{0pt}%
\pgfpathmoveto{\pgfqpoint{0.727040in}{4.489815in}}%
\pgfpathlineto{\pgfqpoint{3.196863in}{4.489815in}}%
\pgfusepath{stroke}%
\end{pgfscope}%
\begin{pgfscope}%
\definecolor{textcolor}{rgb}{0.000000,0.000000,0.000000}%
\pgfsetstrokecolor{textcolor}%
\pgfsetfillcolor{textcolor}%
\pgftext[x=1.961951in,y=4.573148in,,base]{\color{textcolor}\rmfamily\fontsize{12.000000}{14.400000}\selectfont \(\displaystyle \theta\)}%
\end{pgfscope}%
\begin{pgfscope}%
\pgfsetbuttcap%
\pgfsetmiterjoin%
\definecolor{currentfill}{rgb}{1.000000,1.000000,1.000000}%
\pgfsetfillcolor{currentfill}%
\pgfsetlinewidth{0.000000pt}%
\definecolor{currentstroke}{rgb}{0.000000,0.000000,0.000000}%
\pgfsetstrokecolor{currentstroke}%
\pgfsetstrokeopacity{0.000000}%
\pgfsetdash{}{0pt}%
\pgfpathmoveto{\pgfqpoint{3.798088in}{2.870679in}}%
\pgfpathlineto{\pgfqpoint{6.267911in}{2.870679in}}%
\pgfpathlineto{\pgfqpoint{6.267911in}{4.489815in}}%
\pgfpathlineto{\pgfqpoint{3.798088in}{4.489815in}}%
\pgfpathclose%
\pgfusepath{fill}%
\end{pgfscope}%
\begin{pgfscope}%
\pgfsetbuttcap%
\pgfsetroundjoin%
\definecolor{currentfill}{rgb}{0.000000,0.000000,0.000000}%
\pgfsetfillcolor{currentfill}%
\pgfsetlinewidth{0.803000pt}%
\definecolor{currentstroke}{rgb}{0.000000,0.000000,0.000000}%
\pgfsetstrokecolor{currentstroke}%
\pgfsetdash{}{0pt}%
\pgfsys@defobject{currentmarker}{\pgfqpoint{0.000000in}{-0.048611in}}{\pgfqpoint{0.000000in}{0.000000in}}{%
\pgfpathmoveto{\pgfqpoint{0.000000in}{0.000000in}}%
\pgfpathlineto{\pgfqpoint{0.000000in}{-0.048611in}}%
\pgfusepath{stroke,fill}%
}%
\begin{pgfscope}%
\pgfsys@transformshift{3.910353in}{2.870679in}%
\pgfsys@useobject{currentmarker}{}%
\end{pgfscope}%
\end{pgfscope}%
\begin{pgfscope}%
\definecolor{textcolor}{rgb}{0.000000,0.000000,0.000000}%
\pgfsetstrokecolor{textcolor}%
\pgfsetfillcolor{textcolor}%
\pgftext[x=3.910353in,y=2.773457in,,top]{\color{textcolor}\rmfamily\fontsize{10.000000}{12.000000}\selectfont \(\displaystyle 0.0\)}%
\end{pgfscope}%
\begin{pgfscope}%
\pgfsetbuttcap%
\pgfsetroundjoin%
\definecolor{currentfill}{rgb}{0.000000,0.000000,0.000000}%
\pgfsetfillcolor{currentfill}%
\pgfsetlinewidth{0.803000pt}%
\definecolor{currentstroke}{rgb}{0.000000,0.000000,0.000000}%
\pgfsetstrokecolor{currentstroke}%
\pgfsetdash{}{0pt}%
\pgfsys@defobject{currentmarker}{\pgfqpoint{0.000000in}{-0.048611in}}{\pgfqpoint{0.000000in}{0.000000in}}{%
\pgfpathmoveto{\pgfqpoint{0.000000in}{0.000000in}}%
\pgfpathlineto{\pgfqpoint{0.000000in}{-0.048611in}}%
\pgfusepath{stroke,fill}%
}%
\begin{pgfscope}%
\pgfsys@transformshift{4.471676in}{2.870679in}%
\pgfsys@useobject{currentmarker}{}%
\end{pgfscope}%
\end{pgfscope}%
\begin{pgfscope}%
\definecolor{textcolor}{rgb}{0.000000,0.000000,0.000000}%
\pgfsetstrokecolor{textcolor}%
\pgfsetfillcolor{textcolor}%
\pgftext[x=4.471676in,y=2.773457in,,top]{\color{textcolor}\rmfamily\fontsize{10.000000}{12.000000}\selectfont \(\displaystyle 2.5\)}%
\end{pgfscope}%
\begin{pgfscope}%
\pgfsetbuttcap%
\pgfsetroundjoin%
\definecolor{currentfill}{rgb}{0.000000,0.000000,0.000000}%
\pgfsetfillcolor{currentfill}%
\pgfsetlinewidth{0.803000pt}%
\definecolor{currentstroke}{rgb}{0.000000,0.000000,0.000000}%
\pgfsetstrokecolor{currentstroke}%
\pgfsetdash{}{0pt}%
\pgfsys@defobject{currentmarker}{\pgfqpoint{0.000000in}{-0.048611in}}{\pgfqpoint{0.000000in}{0.000000in}}{%
\pgfpathmoveto{\pgfqpoint{0.000000in}{0.000000in}}%
\pgfpathlineto{\pgfqpoint{0.000000in}{-0.048611in}}%
\pgfusepath{stroke,fill}%
}%
\begin{pgfscope}%
\pgfsys@transformshift{5.033000in}{2.870679in}%
\pgfsys@useobject{currentmarker}{}%
\end{pgfscope}%
\end{pgfscope}%
\begin{pgfscope}%
\definecolor{textcolor}{rgb}{0.000000,0.000000,0.000000}%
\pgfsetstrokecolor{textcolor}%
\pgfsetfillcolor{textcolor}%
\pgftext[x=5.033000in,y=2.773457in,,top]{\color{textcolor}\rmfamily\fontsize{10.000000}{12.000000}\selectfont \(\displaystyle 5.0\)}%
\end{pgfscope}%
\begin{pgfscope}%
\pgfsetbuttcap%
\pgfsetroundjoin%
\definecolor{currentfill}{rgb}{0.000000,0.000000,0.000000}%
\pgfsetfillcolor{currentfill}%
\pgfsetlinewidth{0.803000pt}%
\definecolor{currentstroke}{rgb}{0.000000,0.000000,0.000000}%
\pgfsetstrokecolor{currentstroke}%
\pgfsetdash{}{0pt}%
\pgfsys@defobject{currentmarker}{\pgfqpoint{0.000000in}{-0.048611in}}{\pgfqpoint{0.000000in}{0.000000in}}{%
\pgfpathmoveto{\pgfqpoint{0.000000in}{0.000000in}}%
\pgfpathlineto{\pgfqpoint{0.000000in}{-0.048611in}}%
\pgfusepath{stroke,fill}%
}%
\begin{pgfscope}%
\pgfsys@transformshift{5.594323in}{2.870679in}%
\pgfsys@useobject{currentmarker}{}%
\end{pgfscope}%
\end{pgfscope}%
\begin{pgfscope}%
\definecolor{textcolor}{rgb}{0.000000,0.000000,0.000000}%
\pgfsetstrokecolor{textcolor}%
\pgfsetfillcolor{textcolor}%
\pgftext[x=5.594323in,y=2.773457in,,top]{\color{textcolor}\rmfamily\fontsize{10.000000}{12.000000}\selectfont \(\displaystyle 7.5\)}%
\end{pgfscope}%
\begin{pgfscope}%
\pgfsetbuttcap%
\pgfsetroundjoin%
\definecolor{currentfill}{rgb}{0.000000,0.000000,0.000000}%
\pgfsetfillcolor{currentfill}%
\pgfsetlinewidth{0.803000pt}%
\definecolor{currentstroke}{rgb}{0.000000,0.000000,0.000000}%
\pgfsetstrokecolor{currentstroke}%
\pgfsetdash{}{0pt}%
\pgfsys@defobject{currentmarker}{\pgfqpoint{0.000000in}{-0.048611in}}{\pgfqpoint{0.000000in}{0.000000in}}{%
\pgfpathmoveto{\pgfqpoint{0.000000in}{0.000000in}}%
\pgfpathlineto{\pgfqpoint{0.000000in}{-0.048611in}}%
\pgfusepath{stroke,fill}%
}%
\begin{pgfscope}%
\pgfsys@transformshift{6.155646in}{2.870679in}%
\pgfsys@useobject{currentmarker}{}%
\end{pgfscope}%
\end{pgfscope}%
\begin{pgfscope}%
\definecolor{textcolor}{rgb}{0.000000,0.000000,0.000000}%
\pgfsetstrokecolor{textcolor}%
\pgfsetfillcolor{textcolor}%
\pgftext[x=6.155646in,y=2.773457in,,top]{\color{textcolor}\rmfamily\fontsize{10.000000}{12.000000}\selectfont \(\displaystyle 10.0\)}%
\end{pgfscope}%
\begin{pgfscope}%
\definecolor{textcolor}{rgb}{0.000000,0.000000,0.000000}%
\pgfsetstrokecolor{textcolor}%
\pgfsetfillcolor{textcolor}%
\pgftext[x=5.033000in,y=2.594444in,,top]{\color{textcolor}\rmfamily\fontsize{10.000000}{12.000000}\selectfont time (s)}%
\end{pgfscope}%
\begin{pgfscope}%
\pgfsetbuttcap%
\pgfsetroundjoin%
\definecolor{currentfill}{rgb}{0.000000,0.000000,0.000000}%
\pgfsetfillcolor{currentfill}%
\pgfsetlinewidth{0.803000pt}%
\definecolor{currentstroke}{rgb}{0.000000,0.000000,0.000000}%
\pgfsetstrokecolor{currentstroke}%
\pgfsetdash{}{0pt}%
\pgfsys@defobject{currentmarker}{\pgfqpoint{-0.048611in}{0.000000in}}{\pgfqpoint{0.000000in}{0.000000in}}{%
\pgfpathmoveto{\pgfqpoint{0.000000in}{0.000000in}}%
\pgfpathlineto{\pgfqpoint{-0.048611in}{0.000000in}}%
\pgfusepath{stroke,fill}%
}%
\begin{pgfscope}%
\pgfsys@transformshift{3.798088in}{3.125486in}%
\pgfsys@useobject{currentmarker}{}%
\end{pgfscope}%
\end{pgfscope}%
\begin{pgfscope}%
\definecolor{textcolor}{rgb}{0.000000,0.000000,0.000000}%
\pgfsetstrokecolor{textcolor}%
\pgfsetfillcolor{textcolor}%
\pgftext[x=3.523396in,y=3.077261in,left,base]{\color{textcolor}\rmfamily\fontsize{10.000000}{12.000000}\selectfont \(\displaystyle -2\)}%
\end{pgfscope}%
\begin{pgfscope}%
\pgfsetbuttcap%
\pgfsetroundjoin%
\definecolor{currentfill}{rgb}{0.000000,0.000000,0.000000}%
\pgfsetfillcolor{currentfill}%
\pgfsetlinewidth{0.803000pt}%
\definecolor{currentstroke}{rgb}{0.000000,0.000000,0.000000}%
\pgfsetstrokecolor{currentstroke}%
\pgfsetdash{}{0pt}%
\pgfsys@defobject{currentmarker}{\pgfqpoint{-0.048611in}{0.000000in}}{\pgfqpoint{0.000000in}{0.000000in}}{%
\pgfpathmoveto{\pgfqpoint{0.000000in}{0.000000in}}%
\pgfpathlineto{\pgfqpoint{-0.048611in}{0.000000in}}%
\pgfusepath{stroke,fill}%
}%
\begin{pgfscope}%
\pgfsys@transformshift{3.798088in}{3.652245in}%
\pgfsys@useobject{currentmarker}{}%
\end{pgfscope}%
\end{pgfscope}%
\begin{pgfscope}%
\definecolor{textcolor}{rgb}{0.000000,0.000000,0.000000}%
\pgfsetstrokecolor{textcolor}%
\pgfsetfillcolor{textcolor}%
\pgftext[x=3.631421in,y=3.604019in,left,base]{\color{textcolor}\rmfamily\fontsize{10.000000}{12.000000}\selectfont \(\displaystyle 0\)}%
\end{pgfscope}%
\begin{pgfscope}%
\pgfsetbuttcap%
\pgfsetroundjoin%
\definecolor{currentfill}{rgb}{0.000000,0.000000,0.000000}%
\pgfsetfillcolor{currentfill}%
\pgfsetlinewidth{0.803000pt}%
\definecolor{currentstroke}{rgb}{0.000000,0.000000,0.000000}%
\pgfsetstrokecolor{currentstroke}%
\pgfsetdash{}{0pt}%
\pgfsys@defobject{currentmarker}{\pgfqpoint{-0.048611in}{0.000000in}}{\pgfqpoint{0.000000in}{0.000000in}}{%
\pgfpathmoveto{\pgfqpoint{0.000000in}{0.000000in}}%
\pgfpathlineto{\pgfqpoint{-0.048611in}{0.000000in}}%
\pgfusepath{stroke,fill}%
}%
\begin{pgfscope}%
\pgfsys@transformshift{3.798088in}{4.179004in}%
\pgfsys@useobject{currentmarker}{}%
\end{pgfscope}%
\end{pgfscope}%
\begin{pgfscope}%
\definecolor{textcolor}{rgb}{0.000000,0.000000,0.000000}%
\pgfsetstrokecolor{textcolor}%
\pgfsetfillcolor{textcolor}%
\pgftext[x=3.631421in,y=4.130778in,left,base]{\color{textcolor}\rmfamily\fontsize{10.000000}{12.000000}\selectfont \(\displaystyle 2\)}%
\end{pgfscope}%
\begin{pgfscope}%
\definecolor{textcolor}{rgb}{0.000000,0.000000,0.000000}%
\pgfsetstrokecolor{textcolor}%
\pgfsetfillcolor{textcolor}%
\pgftext[x=3.467840in,y=3.680247in,,bottom,rotate=90.000000]{\color{textcolor}\rmfamily\fontsize{10.000000}{12.000000}\selectfont angle (rad)}%
\end{pgfscope}%
\begin{pgfscope}%
\pgfpathrectangle{\pgfqpoint{3.798088in}{2.870679in}}{\pgfqpoint{2.469823in}{1.619136in}}%
\pgfusepath{clip}%
\pgfsetrectcap%
\pgfsetroundjoin%
\pgfsetlinewidth{1.505625pt}%
\definecolor{currentstroke}{rgb}{0.000000,0.000000,1.000000}%
\pgfsetstrokecolor{currentstroke}%
\pgfsetdash{}{0pt}%
\pgfpathmoveto{\pgfqpoint{3.910353in}{3.652245in}}%
\pgfpathlineto{\pgfqpoint{3.935051in}{3.531090in}}%
\pgfpathlineto{\pgfqpoint{3.948523in}{3.476635in}}%
\pgfpathlineto{\pgfqpoint{3.957504in}{3.448320in}}%
\pgfpathlineto{\pgfqpoint{3.966485in}{3.427849in}}%
\pgfpathlineto{\pgfqpoint{3.973221in}{3.418207in}}%
\pgfpathlineto{\pgfqpoint{3.977711in}{3.414652in}}%
\pgfpathlineto{\pgfqpoint{3.982202in}{3.413458in}}%
\pgfpathlineto{\pgfqpoint{3.986693in}{3.414646in}}%
\pgfpathlineto{\pgfqpoint{3.991183in}{3.418217in}}%
\pgfpathlineto{\pgfqpoint{3.995674in}{3.424147in}}%
\pgfpathlineto{\pgfqpoint{4.002410in}{3.437348in}}%
\pgfpathlineto{\pgfqpoint{4.009146in}{3.455472in}}%
\pgfpathlineto{\pgfqpoint{4.018127in}{3.486622in}}%
\pgfpathlineto{\pgfqpoint{4.029353in}{3.535072in}}%
\pgfpathlineto{\pgfqpoint{4.045070in}{3.615311in}}%
\pgfpathlineto{\pgfqpoint{4.080995in}{3.804144in}}%
\pgfpathlineto{\pgfqpoint{4.092221in}{3.850344in}}%
\pgfpathlineto{\pgfqpoint{4.101203in}{3.878995in}}%
\pgfpathlineto{\pgfqpoint{4.107939in}{3.894798in}}%
\pgfpathlineto{\pgfqpoint{4.114674in}{3.905269in}}%
\pgfpathlineto{\pgfqpoint{4.119165in}{3.909143in}}%
\pgfpathlineto{\pgfqpoint{4.123656in}{3.910466in}}%
\pgfpathlineto{\pgfqpoint{4.128146in}{3.909212in}}%
\pgfpathlineto{\pgfqpoint{4.132637in}{3.905381in}}%
\pgfpathlineto{\pgfqpoint{4.137127in}{3.899000in}}%
\pgfpathlineto{\pgfqpoint{4.143863in}{3.884771in}}%
\pgfpathlineto{\pgfqpoint{4.150599in}{3.865219in}}%
\pgfpathlineto{\pgfqpoint{4.159580in}{3.831595in}}%
\pgfpathlineto{\pgfqpoint{4.170807in}{3.779283in}}%
\pgfpathlineto{\pgfqpoint{4.186524in}{3.692622in}}%
\pgfpathlineto{\pgfqpoint{4.222448in}{3.488481in}}%
\pgfpathlineto{\pgfqpoint{4.233675in}{3.438443in}}%
\pgfpathlineto{\pgfqpoint{4.242656in}{3.407365in}}%
\pgfpathlineto{\pgfqpoint{4.249392in}{3.390191in}}%
\pgfpathlineto{\pgfqpoint{4.256128in}{3.378774in}}%
\pgfpathlineto{\pgfqpoint{4.260618in}{3.374519in}}%
\pgfpathlineto{\pgfqpoint{4.265109in}{3.373022in}}%
\pgfpathlineto{\pgfqpoint{4.269600in}{3.374310in}}%
\pgfpathlineto{\pgfqpoint{4.274090in}{3.378386in}}%
\pgfpathlineto{\pgfqpoint{4.278581in}{3.385220in}}%
\pgfpathlineto{\pgfqpoint{4.285317in}{3.400509in}}%
\pgfpathlineto{\pgfqpoint{4.292053in}{3.421557in}}%
\pgfpathlineto{\pgfqpoint{4.301034in}{3.457792in}}%
\pgfpathlineto{\pgfqpoint{4.312260in}{3.514215in}}%
\pgfpathlineto{\pgfqpoint{4.327977in}{3.607752in}}%
\pgfpathlineto{\pgfqpoint{4.363902in}{3.828462in}}%
\pgfpathlineto{\pgfqpoint{4.375128in}{3.882715in}}%
\pgfpathlineto{\pgfqpoint{4.384110in}{3.916487in}}%
\pgfpathlineto{\pgfqpoint{4.390846in}{3.935205in}}%
\pgfpathlineto{\pgfqpoint{4.397581in}{3.947711in}}%
\pgfpathlineto{\pgfqpoint{4.402072in}{3.952424in}}%
\pgfpathlineto{\pgfqpoint{4.406563in}{3.954158in}}%
\pgfpathlineto{\pgfqpoint{4.411053in}{3.952879in}}%
\pgfpathlineto{\pgfqpoint{4.415544in}{3.948586in}}%
\pgfpathlineto{\pgfqpoint{4.420034in}{3.941310in}}%
\pgfpathlineto{\pgfqpoint{4.426770in}{3.924944in}}%
\pgfpathlineto{\pgfqpoint{4.433506in}{3.902345in}}%
\pgfpathlineto{\pgfqpoint{4.442487in}{3.863366in}}%
\pgfpathlineto{\pgfqpoint{4.453714in}{3.802586in}}%
\pgfpathlineto{\pgfqpoint{4.469431in}{3.701704in}}%
\pgfpathlineto{\pgfqpoint{4.505356in}{3.463059in}}%
\pgfpathlineto{\pgfqpoint{4.516582in}{3.404164in}}%
\pgfpathlineto{\pgfqpoint{4.525563in}{3.367387in}}%
\pgfpathlineto{\pgfqpoint{4.532299in}{3.346920in}}%
\pgfpathlineto{\pgfqpoint{4.539035in}{3.333149in}}%
\pgfpathlineto{\pgfqpoint{4.543526in}{3.327880in}}%
\pgfpathlineto{\pgfqpoint{4.548016in}{3.325830in}}%
\pgfpathlineto{\pgfqpoint{4.552507in}{3.327035in}}%
\pgfpathlineto{\pgfqpoint{4.556997in}{3.331499in}}%
\pgfpathlineto{\pgfqpoint{4.561488in}{3.339192in}}%
\pgfpathlineto{\pgfqpoint{4.568224in}{3.356631in}}%
\pgfpathlineto{\pgfqpoint{4.574960in}{3.380821in}}%
\pgfpathlineto{\pgfqpoint{4.583941in}{3.422660in}}%
\pgfpathlineto{\pgfqpoint{4.595167in}{3.488033in}}%
\pgfpathlineto{\pgfqpoint{4.610884in}{3.596739in}}%
\pgfpathlineto{\pgfqpoint{4.646809in}{3.854800in}}%
\pgfpathlineto{\pgfqpoint{4.658035in}{3.918821in}}%
\pgfpathlineto{\pgfqpoint{4.667017in}{3.958964in}}%
\pgfpathlineto{\pgfqpoint{4.673753in}{3.981426in}}%
\pgfpathlineto{\pgfqpoint{4.680488in}{3.996678in}}%
\pgfpathlineto{\pgfqpoint{4.684979in}{4.002627in}}%
\pgfpathlineto{\pgfqpoint{4.689470in}{4.005102in}}%
\pgfpathlineto{\pgfqpoint{4.693960in}{4.004059in}}%
\pgfpathlineto{\pgfqpoint{4.698451in}{3.999491in}}%
\pgfpathlineto{\pgfqpoint{4.702941in}{3.991429in}}%
\pgfpathlineto{\pgfqpoint{4.709677in}{3.972948in}}%
\pgfpathlineto{\pgfqpoint{4.716413in}{3.947152in}}%
\pgfpathlineto{\pgfqpoint{4.725394in}{3.902362in}}%
\pgfpathlineto{\pgfqpoint{4.736621in}{3.832172in}}%
\pgfpathlineto{\pgfqpoint{4.752338in}{3.715162in}}%
\pgfpathlineto{\pgfqpoint{4.790508in}{3.421024in}}%
\pgfpathlineto{\pgfqpoint{4.801734in}{3.354322in}}%
\pgfpathlineto{\pgfqpoint{4.810715in}{3.313381in}}%
\pgfpathlineto{\pgfqpoint{4.817451in}{3.291150in}}%
\pgfpathlineto{\pgfqpoint{4.824187in}{3.276848in}}%
\pgfpathlineto{\pgfqpoint{4.828678in}{3.271926in}}%
\pgfpathlineto{\pgfqpoint{4.833168in}{3.270784in}}%
\pgfpathlineto{\pgfqpoint{4.837659in}{3.273453in}}%
\pgfpathlineto{\pgfqpoint{4.842150in}{3.279923in}}%
\pgfpathlineto{\pgfqpoint{4.846640in}{3.290144in}}%
\pgfpathlineto{\pgfqpoint{4.853376in}{3.312295in}}%
\pgfpathlineto{\pgfqpoint{4.860112in}{3.342199in}}%
\pgfpathlineto{\pgfqpoint{4.869093in}{3.392983in}}%
\pgfpathlineto{\pgfqpoint{4.880320in}{3.471150in}}%
\pgfpathlineto{\pgfqpoint{4.896037in}{3.599384in}}%
\pgfpathlineto{\pgfqpoint{4.929716in}{3.881982in}}%
\pgfpathlineto{\pgfqpoint{4.940942in}{3.957994in}}%
\pgfpathlineto{\pgfqpoint{4.949924in}{4.006216in}}%
\pgfpathlineto{\pgfqpoint{4.956660in}{4.033614in}}%
\pgfpathlineto{\pgfqpoint{4.963395in}{4.052689in}}%
\pgfpathlineto{\pgfqpoint{4.967886in}{4.060514in}}%
\pgfpathlineto{\pgfqpoint{4.972377in}{4.064298in}}%
\pgfpathlineto{\pgfqpoint{4.976867in}{4.063978in}}%
\pgfpathlineto{\pgfqpoint{4.981358in}{4.059537in}}%
\pgfpathlineto{\pgfqpoint{4.985848in}{4.051004in}}%
\pgfpathlineto{\pgfqpoint{4.992584in}{4.030702in}}%
\pgfpathlineto{\pgfqpoint{4.999320in}{4.001794in}}%
\pgfpathlineto{\pgfqpoint{5.008301in}{3.950980in}}%
\pgfpathlineto{\pgfqpoint{5.019528in}{3.870624in}}%
\pgfpathlineto{\pgfqpoint{5.033000in}{3.756024in}}%
\pgfpathlineto{\pgfqpoint{5.077905in}{3.357474in}}%
\pgfpathlineto{\pgfqpoint{5.089132in}{3.285143in}}%
\pgfpathlineto{\pgfqpoint{5.098113in}{3.242664in}}%
\pgfpathlineto{\pgfqpoint{5.104849in}{3.221107in}}%
\pgfpathlineto{\pgfqpoint{5.109340in}{3.211990in}}%
\pgfpathlineto{\pgfqpoint{5.113830in}{3.207227in}}%
\pgfpathlineto{\pgfqpoint{5.118321in}{3.206892in}}%
\pgfpathlineto{\pgfqpoint{5.122811in}{3.211011in}}%
\pgfpathlineto{\pgfqpoint{5.127302in}{3.219560in}}%
\pgfpathlineto{\pgfqpoint{5.131792in}{3.232467in}}%
\pgfpathlineto{\pgfqpoint{5.138528in}{3.259720in}}%
\pgfpathlineto{\pgfqpoint{5.145264in}{3.295899in}}%
\pgfpathlineto{\pgfqpoint{5.154245in}{3.356607in}}%
\pgfpathlineto{\pgfqpoint{5.165472in}{3.449095in}}%
\pgfpathlineto{\pgfqpoint{5.181189in}{3.599409in}}%
\pgfpathlineto{\pgfqpoint{5.214868in}{3.927123in}}%
\pgfpathlineto{\pgfqpoint{5.226095in}{4.014488in}}%
\pgfpathlineto{\pgfqpoint{5.235076in}{4.069543in}}%
\pgfpathlineto{\pgfqpoint{5.241812in}{4.100522in}}%
\pgfpathlineto{\pgfqpoint{5.248548in}{4.121728in}}%
\pgfpathlineto{\pgfqpoint{5.253038in}{4.130128in}}%
\pgfpathlineto{\pgfqpoint{5.257529in}{4.133794in}}%
\pgfpathlineto{\pgfqpoint{5.259774in}{4.133829in}}%
\pgfpathlineto{\pgfqpoint{5.264265in}{4.130287in}}%
\pgfpathlineto{\pgfqpoint{5.268755in}{4.121950in}}%
\pgfpathlineto{\pgfqpoint{5.273246in}{4.108888in}}%
\pgfpathlineto{\pgfqpoint{5.279982in}{4.080716in}}%
\pgfpathlineto{\pgfqpoint{5.286718in}{4.042825in}}%
\pgfpathlineto{\pgfqpoint{5.295699in}{3.978683in}}%
\pgfpathlineto{\pgfqpoint{5.306925in}{3.880280in}}%
\pgfpathlineto{\pgfqpoint{5.322642in}{3.719335in}}%
\pgfpathlineto{\pgfqpoint{5.356322in}{3.364595in}}%
\pgfpathlineto{\pgfqpoint{5.367548in}{3.268593in}}%
\pgfpathlineto{\pgfqpoint{5.376529in}{3.207360in}}%
\pgfpathlineto{\pgfqpoint{5.383265in}{3.172358in}}%
\pgfpathlineto{\pgfqpoint{5.390001in}{3.147776in}}%
\pgfpathlineto{\pgfqpoint{5.394492in}{3.137534in}}%
\pgfpathlineto{\pgfqpoint{5.398982in}{3.132382in}}%
\pgfpathlineto{\pgfqpoint{5.401228in}{3.131743in}}%
\pgfpathlineto{\pgfqpoint{5.403473in}{3.132404in}}%
\pgfpathlineto{\pgfqpoint{5.407964in}{3.137626in}}%
\pgfpathlineto{\pgfqpoint{5.412454in}{3.148014in}}%
\pgfpathlineto{\pgfqpoint{5.416945in}{3.163475in}}%
\pgfpathlineto{\pgfqpoint{5.423681in}{3.195831in}}%
\pgfpathlineto{\pgfqpoint{5.432662in}{3.254876in}}%
\pgfpathlineto{\pgfqpoint{5.441643in}{3.329873in}}%
\pgfpathlineto{\pgfqpoint{5.455115in}{3.465672in}}%
\pgfpathlineto{\pgfqpoint{5.475322in}{3.698312in}}%
\pgfpathlineto{\pgfqpoint{5.497775in}{3.950571in}}%
\pgfpathlineto{\pgfqpoint{5.509002in}{4.056291in}}%
\pgfpathlineto{\pgfqpoint{5.517983in}{4.124662in}}%
\pgfpathlineto{\pgfqpoint{5.526964in}{4.175287in}}%
\pgfpathlineto{\pgfqpoint{5.533700in}{4.200120in}}%
\pgfpathlineto{\pgfqpoint{5.538191in}{4.209990in}}%
\pgfpathlineto{\pgfqpoint{5.542681in}{4.214337in}}%
\pgfpathlineto{\pgfqpoint{5.544926in}{4.214412in}}%
\pgfpathlineto{\pgfqpoint{5.547172in}{4.213083in}}%
\pgfpathlineto{\pgfqpoint{5.551662in}{4.206213in}}%
\pgfpathlineto{\pgfqpoint{5.556153in}{4.193783in}}%
\pgfpathlineto{\pgfqpoint{5.562889in}{4.164983in}}%
\pgfpathlineto{\pgfqpoint{5.569625in}{4.124605in}}%
\pgfpathlineto{\pgfqpoint{5.578606in}{4.054411in}}%
\pgfpathlineto{\pgfqpoint{5.589832in}{3.944500in}}%
\pgfpathlineto{\pgfqpoint{5.605549in}{3.761500in}}%
\pgfpathlineto{\pgfqpoint{5.643719in}{3.297404in}}%
\pgfpathlineto{\pgfqpoint{5.654946in}{3.189354in}}%
\pgfpathlineto{\pgfqpoint{5.663927in}{3.121613in}}%
\pgfpathlineto{\pgfqpoint{5.670663in}{3.083820in}}%
\pgfpathlineto{\pgfqpoint{5.677399in}{3.058377in}}%
\pgfpathlineto{\pgfqpoint{5.681889in}{3.048669in}}%
\pgfpathlineto{\pgfqpoint{5.686380in}{3.044943in}}%
\pgfpathlineto{\pgfqpoint{5.688625in}{3.045350in}}%
\pgfpathlineto{\pgfqpoint{5.690871in}{3.047274in}}%
\pgfpathlineto{\pgfqpoint{5.695361in}{3.055665in}}%
\pgfpathlineto{\pgfqpoint{5.699852in}{3.070043in}}%
\pgfpathlineto{\pgfqpoint{5.706588in}{3.102509in}}%
\pgfpathlineto{\pgfqpoint{5.713324in}{3.147350in}}%
\pgfpathlineto{\pgfqpoint{5.722305in}{3.224519in}}%
\pgfpathlineto{\pgfqpoint{5.733531in}{3.344348in}}%
\pgfpathlineto{\pgfqpoint{5.749248in}{3.542392in}}%
\pgfpathlineto{\pgfqpoint{5.787418in}{4.040677in}}%
\pgfpathlineto{\pgfqpoint{5.798645in}{4.156088in}}%
\pgfpathlineto{\pgfqpoint{5.807626in}{4.228173in}}%
\pgfpathlineto{\pgfqpoint{5.814362in}{4.268150in}}%
\pgfpathlineto{\pgfqpoint{5.821098in}{4.294757in}}%
\pgfpathlineto{\pgfqpoint{5.825588in}{4.304646in}}%
\pgfpathlineto{\pgfqpoint{5.830079in}{4.308066in}}%
\pgfpathlineto{\pgfqpoint{5.832324in}{4.307323in}}%
\pgfpathlineto{\pgfqpoint{5.834569in}{4.304941in}}%
\pgfpathlineto{\pgfqpoint{5.839060in}{4.295279in}}%
\pgfpathlineto{\pgfqpoint{5.843551in}{4.279167in}}%
\pgfpathlineto{\pgfqpoint{5.850286in}{4.243287in}}%
\pgfpathlineto{\pgfqpoint{5.857022in}{4.194151in}}%
\pgfpathlineto{\pgfqpoint{5.866003in}{4.110106in}}%
\pgfpathlineto{\pgfqpoint{5.877230in}{3.980298in}}%
\pgfpathlineto{\pgfqpoint{5.892947in}{3.766795in}}%
\pgfpathlineto{\pgfqpoint{5.931117in}{3.231591in}}%
\pgfpathlineto{\pgfqpoint{5.942343in}{3.107561in}}%
\pgfpathlineto{\pgfqpoint{5.951325in}{3.030043in}}%
\pgfpathlineto{\pgfqpoint{5.958061in}{2.987052in}}%
\pgfpathlineto{\pgfqpoint{5.964796in}{2.958468in}}%
\pgfpathlineto{\pgfqpoint{5.969287in}{2.947880in}}%
\pgfpathlineto{\pgfqpoint{5.971532in}{2.945198in}}%
\pgfpathlineto{\pgfqpoint{5.973778in}{2.944276in}}%
\pgfpathlineto{\pgfqpoint{5.976023in}{2.945122in}}%
\pgfpathlineto{\pgfqpoint{5.978268in}{2.947736in}}%
\pgfpathlineto{\pgfqpoint{5.982759in}{2.958252in}}%
\pgfpathlineto{\pgfqpoint{5.987249in}{2.975724in}}%
\pgfpathlineto{\pgfqpoint{5.993985in}{3.014542in}}%
\pgfpathlineto{\pgfqpoint{6.000721in}{3.067605in}}%
\pgfpathlineto{\pgfqpoint{6.009702in}{3.158201in}}%
\pgfpathlineto{\pgfqpoint{6.020929in}{3.297836in}}%
\pgfpathlineto{\pgfqpoint{6.036646in}{3.527043in}}%
\pgfpathlineto{\pgfqpoint{6.074816in}{4.102199in}}%
\pgfpathlineto{\pgfqpoint{6.086042in}{4.236439in}}%
\pgfpathlineto{\pgfqpoint{6.095023in}{4.320832in}}%
\pgfpathlineto{\pgfqpoint{6.101759in}{4.367973in}}%
\pgfpathlineto{\pgfqpoint{6.108495in}{4.399675in}}%
\pgfpathlineto{\pgfqpoint{6.112986in}{4.411708in}}%
\pgfpathlineto{\pgfqpoint{6.117476in}{4.416218in}}%
\pgfpathlineto{\pgfqpoint{6.119722in}{4.415618in}}%
\pgfpathlineto{\pgfqpoint{6.121967in}{4.413108in}}%
\pgfpathlineto{\pgfqpoint{6.126458in}{4.402382in}}%
\pgfpathlineto{\pgfqpoint{6.130948in}{4.384143in}}%
\pgfpathlineto{\pgfqpoint{6.137684in}{4.343168in}}%
\pgfpathlineto{\pgfqpoint{6.144420in}{4.286822in}}%
\pgfpathlineto{\pgfqpoint{6.153401in}{4.190315in}}%
\pgfpathlineto{\pgfqpoint{6.155646in}{4.162797in}}%
\pgfpathlineto{\pgfqpoint{6.155646in}{4.162797in}}%
\pgfusepath{stroke}%
\end{pgfscope}%
\begin{pgfscope}%
\pgfsetrectcap%
\pgfsetmiterjoin%
\pgfsetlinewidth{0.803000pt}%
\definecolor{currentstroke}{rgb}{0.000000,0.000000,0.000000}%
\pgfsetstrokecolor{currentstroke}%
\pgfsetdash{}{0pt}%
\pgfpathmoveto{\pgfqpoint{3.798088in}{2.870679in}}%
\pgfpathlineto{\pgfqpoint{3.798088in}{4.489815in}}%
\pgfusepath{stroke}%
\end{pgfscope}%
\begin{pgfscope}%
\pgfsetrectcap%
\pgfsetmiterjoin%
\pgfsetlinewidth{0.803000pt}%
\definecolor{currentstroke}{rgb}{0.000000,0.000000,0.000000}%
\pgfsetstrokecolor{currentstroke}%
\pgfsetdash{}{0pt}%
\pgfpathmoveto{\pgfqpoint{6.267911in}{2.870679in}}%
\pgfpathlineto{\pgfqpoint{6.267911in}{4.489815in}}%
\pgfusepath{stroke}%
\end{pgfscope}%
\begin{pgfscope}%
\pgfsetrectcap%
\pgfsetmiterjoin%
\pgfsetlinewidth{0.803000pt}%
\definecolor{currentstroke}{rgb}{0.000000,0.000000,0.000000}%
\pgfsetstrokecolor{currentstroke}%
\pgfsetdash{}{0pt}%
\pgfpathmoveto{\pgfqpoint{3.798088in}{2.870679in}}%
\pgfpathlineto{\pgfqpoint{6.267911in}{2.870679in}}%
\pgfusepath{stroke}%
\end{pgfscope}%
\begin{pgfscope}%
\pgfsetrectcap%
\pgfsetmiterjoin%
\pgfsetlinewidth{0.803000pt}%
\definecolor{currentstroke}{rgb}{0.000000,0.000000,0.000000}%
\pgfsetstrokecolor{currentstroke}%
\pgfsetdash{}{0pt}%
\pgfpathmoveto{\pgfqpoint{3.798088in}{4.489815in}}%
\pgfpathlineto{\pgfqpoint{6.267911in}{4.489815in}}%
\pgfusepath{stroke}%
\end{pgfscope}%
\begin{pgfscope}%
\definecolor{textcolor}{rgb}{0.000000,0.000000,0.000000}%
\pgfsetstrokecolor{textcolor}%
\pgfsetfillcolor{textcolor}%
\pgftext[x=5.033000in,y=4.573148in,,base]{\color{textcolor}\rmfamily\fontsize{12.000000}{14.400000}\selectfont \(\displaystyle \omega\)}%
\end{pgfscope}%
\begin{pgfscope}%
\pgfsetbuttcap%
\pgfsetmiterjoin%
\definecolor{currentfill}{rgb}{1.000000,1.000000,1.000000}%
\pgfsetfillcolor{currentfill}%
\pgfsetlinewidth{0.000000pt}%
\definecolor{currentstroke}{rgb}{0.000000,0.000000,0.000000}%
\pgfsetstrokecolor{currentstroke}%
\pgfsetstrokeopacity{0.000000}%
\pgfsetdash{}{0pt}%
\pgfpathmoveto{\pgfqpoint{0.727040in}{0.526234in}}%
\pgfpathlineto{\pgfqpoint{3.196863in}{0.526234in}}%
\pgfpathlineto{\pgfqpoint{3.196863in}{2.145371in}}%
\pgfpathlineto{\pgfqpoint{0.727040in}{2.145371in}}%
\pgfpathclose%
\pgfusepath{fill}%
\end{pgfscope}%
\begin{pgfscope}%
\pgfsetbuttcap%
\pgfsetroundjoin%
\definecolor{currentfill}{rgb}{0.000000,0.000000,0.000000}%
\pgfsetfillcolor{currentfill}%
\pgfsetlinewidth{0.803000pt}%
\definecolor{currentstroke}{rgb}{0.000000,0.000000,0.000000}%
\pgfsetstrokecolor{currentstroke}%
\pgfsetdash{}{0pt}%
\pgfsys@defobject{currentmarker}{\pgfqpoint{0.000000in}{-0.048611in}}{\pgfqpoint{0.000000in}{0.000000in}}{%
\pgfpathmoveto{\pgfqpoint{0.000000in}{0.000000in}}%
\pgfpathlineto{\pgfqpoint{0.000000in}{-0.048611in}}%
\pgfusepath{stroke,fill}%
}%
\begin{pgfscope}%
\pgfsys@transformshift{0.975340in}{0.526234in}%
\pgfsys@useobject{currentmarker}{}%
\end{pgfscope}%
\end{pgfscope}%
\begin{pgfscope}%
\definecolor{textcolor}{rgb}{0.000000,0.000000,0.000000}%
\pgfsetstrokecolor{textcolor}%
\pgfsetfillcolor{textcolor}%
\pgftext[x=0.975340in,y=0.429012in,,top]{\color{textcolor}\rmfamily\fontsize{10.000000}{12.000000}\selectfont \(\displaystyle -0.50\)}%
\end{pgfscope}%
\begin{pgfscope}%
\pgfsetbuttcap%
\pgfsetroundjoin%
\definecolor{currentfill}{rgb}{0.000000,0.000000,0.000000}%
\pgfsetfillcolor{currentfill}%
\pgfsetlinewidth{0.803000pt}%
\definecolor{currentstroke}{rgb}{0.000000,0.000000,0.000000}%
\pgfsetstrokecolor{currentstroke}%
\pgfsetdash{}{0pt}%
\pgfsys@defobject{currentmarker}{\pgfqpoint{0.000000in}{-0.048611in}}{\pgfqpoint{0.000000in}{0.000000in}}{%
\pgfpathmoveto{\pgfqpoint{0.000000in}{0.000000in}}%
\pgfpathlineto{\pgfqpoint{0.000000in}{-0.048611in}}%
\pgfusepath{stroke,fill}%
}%
\begin{pgfscope}%
\pgfsys@transformshift{1.490605in}{0.526234in}%
\pgfsys@useobject{currentmarker}{}%
\end{pgfscope}%
\end{pgfscope}%
\begin{pgfscope}%
\definecolor{textcolor}{rgb}{0.000000,0.000000,0.000000}%
\pgfsetstrokecolor{textcolor}%
\pgfsetfillcolor{textcolor}%
\pgftext[x=1.490605in,y=0.429012in,,top]{\color{textcolor}\rmfamily\fontsize{10.000000}{12.000000}\selectfont \(\displaystyle -0.25\)}%
\end{pgfscope}%
\begin{pgfscope}%
\pgfsetbuttcap%
\pgfsetroundjoin%
\definecolor{currentfill}{rgb}{0.000000,0.000000,0.000000}%
\pgfsetfillcolor{currentfill}%
\pgfsetlinewidth{0.803000pt}%
\definecolor{currentstroke}{rgb}{0.000000,0.000000,0.000000}%
\pgfsetstrokecolor{currentstroke}%
\pgfsetdash{}{0pt}%
\pgfsys@defobject{currentmarker}{\pgfqpoint{0.000000in}{-0.048611in}}{\pgfqpoint{0.000000in}{0.000000in}}{%
\pgfpathmoveto{\pgfqpoint{0.000000in}{0.000000in}}%
\pgfpathlineto{\pgfqpoint{0.000000in}{-0.048611in}}%
\pgfusepath{stroke,fill}%
}%
\begin{pgfscope}%
\pgfsys@transformshift{2.005870in}{0.526234in}%
\pgfsys@useobject{currentmarker}{}%
\end{pgfscope}%
\end{pgfscope}%
\begin{pgfscope}%
\definecolor{textcolor}{rgb}{0.000000,0.000000,0.000000}%
\pgfsetstrokecolor{textcolor}%
\pgfsetfillcolor{textcolor}%
\pgftext[x=2.005870in,y=0.429012in,,top]{\color{textcolor}\rmfamily\fontsize{10.000000}{12.000000}\selectfont \(\displaystyle 0.00\)}%
\end{pgfscope}%
\begin{pgfscope}%
\pgfsetbuttcap%
\pgfsetroundjoin%
\definecolor{currentfill}{rgb}{0.000000,0.000000,0.000000}%
\pgfsetfillcolor{currentfill}%
\pgfsetlinewidth{0.803000pt}%
\definecolor{currentstroke}{rgb}{0.000000,0.000000,0.000000}%
\pgfsetstrokecolor{currentstroke}%
\pgfsetdash{}{0pt}%
\pgfsys@defobject{currentmarker}{\pgfqpoint{0.000000in}{-0.048611in}}{\pgfqpoint{0.000000in}{0.000000in}}{%
\pgfpathmoveto{\pgfqpoint{0.000000in}{0.000000in}}%
\pgfpathlineto{\pgfqpoint{0.000000in}{-0.048611in}}%
\pgfusepath{stroke,fill}%
}%
\begin{pgfscope}%
\pgfsys@transformshift{2.521135in}{0.526234in}%
\pgfsys@useobject{currentmarker}{}%
\end{pgfscope}%
\end{pgfscope}%
\begin{pgfscope}%
\definecolor{textcolor}{rgb}{0.000000,0.000000,0.000000}%
\pgfsetstrokecolor{textcolor}%
\pgfsetfillcolor{textcolor}%
\pgftext[x=2.521135in,y=0.429012in,,top]{\color{textcolor}\rmfamily\fontsize{10.000000}{12.000000}\selectfont \(\displaystyle 0.25\)}%
\end{pgfscope}%
\begin{pgfscope}%
\pgfsetbuttcap%
\pgfsetroundjoin%
\definecolor{currentfill}{rgb}{0.000000,0.000000,0.000000}%
\pgfsetfillcolor{currentfill}%
\pgfsetlinewidth{0.803000pt}%
\definecolor{currentstroke}{rgb}{0.000000,0.000000,0.000000}%
\pgfsetstrokecolor{currentstroke}%
\pgfsetdash{}{0pt}%
\pgfsys@defobject{currentmarker}{\pgfqpoint{0.000000in}{-0.048611in}}{\pgfqpoint{0.000000in}{0.000000in}}{%
\pgfpathmoveto{\pgfqpoint{0.000000in}{0.000000in}}%
\pgfpathlineto{\pgfqpoint{0.000000in}{-0.048611in}}%
\pgfusepath{stroke,fill}%
}%
\begin{pgfscope}%
\pgfsys@transformshift{3.036400in}{0.526234in}%
\pgfsys@useobject{currentmarker}{}%
\end{pgfscope}%
\end{pgfscope}%
\begin{pgfscope}%
\definecolor{textcolor}{rgb}{0.000000,0.000000,0.000000}%
\pgfsetstrokecolor{textcolor}%
\pgfsetfillcolor{textcolor}%
\pgftext[x=3.036400in,y=0.429012in,,top]{\color{textcolor}\rmfamily\fontsize{10.000000}{12.000000}\selectfont \(\displaystyle 0.50\)}%
\end{pgfscope}%
\begin{pgfscope}%
\definecolor{textcolor}{rgb}{0.000000,0.000000,0.000000}%
\pgfsetstrokecolor{textcolor}%
\pgfsetfillcolor{textcolor}%
\pgftext[x=1.961951in,y=0.250000in,,top]{\color{textcolor}\rmfamily\fontsize{10.000000}{12.000000}\selectfont angle (rad)}%
\end{pgfscope}%
\begin{pgfscope}%
\pgfsetbuttcap%
\pgfsetroundjoin%
\definecolor{currentfill}{rgb}{0.000000,0.000000,0.000000}%
\pgfsetfillcolor{currentfill}%
\pgfsetlinewidth{0.803000pt}%
\definecolor{currentstroke}{rgb}{0.000000,0.000000,0.000000}%
\pgfsetstrokecolor{currentstroke}%
\pgfsetdash{}{0pt}%
\pgfsys@defobject{currentmarker}{\pgfqpoint{-0.048611in}{0.000000in}}{\pgfqpoint{0.000000in}{0.000000in}}{%
\pgfpathmoveto{\pgfqpoint{0.000000in}{0.000000in}}%
\pgfpathlineto{\pgfqpoint{-0.048611in}{0.000000in}}%
\pgfusepath{stroke,fill}%
}%
\begin{pgfscope}%
\pgfsys@transformshift{0.727040in}{0.781041in}%
\pgfsys@useobject{currentmarker}{}%
\end{pgfscope}%
\end{pgfscope}%
\begin{pgfscope}%
\definecolor{textcolor}{rgb}{0.000000,0.000000,0.000000}%
\pgfsetstrokecolor{textcolor}%
\pgfsetfillcolor{textcolor}%
\pgftext[x=0.452348in,y=0.732816in,left,base]{\color{textcolor}\rmfamily\fontsize{10.000000}{12.000000}\selectfont \(\displaystyle -2\)}%
\end{pgfscope}%
\begin{pgfscope}%
\pgfsetbuttcap%
\pgfsetroundjoin%
\definecolor{currentfill}{rgb}{0.000000,0.000000,0.000000}%
\pgfsetfillcolor{currentfill}%
\pgfsetlinewidth{0.803000pt}%
\definecolor{currentstroke}{rgb}{0.000000,0.000000,0.000000}%
\pgfsetstrokecolor{currentstroke}%
\pgfsetdash{}{0pt}%
\pgfsys@defobject{currentmarker}{\pgfqpoint{-0.048611in}{0.000000in}}{\pgfqpoint{0.000000in}{0.000000in}}{%
\pgfpathmoveto{\pgfqpoint{0.000000in}{0.000000in}}%
\pgfpathlineto{\pgfqpoint{-0.048611in}{0.000000in}}%
\pgfusepath{stroke,fill}%
}%
\begin{pgfscope}%
\pgfsys@transformshift{0.727040in}{1.307800in}%
\pgfsys@useobject{currentmarker}{}%
\end{pgfscope}%
\end{pgfscope}%
\begin{pgfscope}%
\definecolor{textcolor}{rgb}{0.000000,0.000000,0.000000}%
\pgfsetstrokecolor{textcolor}%
\pgfsetfillcolor{textcolor}%
\pgftext[x=0.560373in,y=1.259575in,left,base]{\color{textcolor}\rmfamily\fontsize{10.000000}{12.000000}\selectfont \(\displaystyle 0\)}%
\end{pgfscope}%
\begin{pgfscope}%
\pgfsetbuttcap%
\pgfsetroundjoin%
\definecolor{currentfill}{rgb}{0.000000,0.000000,0.000000}%
\pgfsetfillcolor{currentfill}%
\pgfsetlinewidth{0.803000pt}%
\definecolor{currentstroke}{rgb}{0.000000,0.000000,0.000000}%
\pgfsetstrokecolor{currentstroke}%
\pgfsetdash{}{0pt}%
\pgfsys@defobject{currentmarker}{\pgfqpoint{-0.048611in}{0.000000in}}{\pgfqpoint{0.000000in}{0.000000in}}{%
\pgfpathmoveto{\pgfqpoint{0.000000in}{0.000000in}}%
\pgfpathlineto{\pgfqpoint{-0.048611in}{0.000000in}}%
\pgfusepath{stroke,fill}%
}%
\begin{pgfscope}%
\pgfsys@transformshift{0.727040in}{1.834559in}%
\pgfsys@useobject{currentmarker}{}%
\end{pgfscope}%
\end{pgfscope}%
\begin{pgfscope}%
\definecolor{textcolor}{rgb}{0.000000,0.000000,0.000000}%
\pgfsetstrokecolor{textcolor}%
\pgfsetfillcolor{textcolor}%
\pgftext[x=0.560373in,y=1.786334in,left,base]{\color{textcolor}\rmfamily\fontsize{10.000000}{12.000000}\selectfont \(\displaystyle 2\)}%
\end{pgfscope}%
\begin{pgfscope}%
\definecolor{textcolor}{rgb}{0.000000,0.000000,0.000000}%
\pgfsetstrokecolor{textcolor}%
\pgfsetfillcolor{textcolor}%
\pgftext[x=0.396792in,y=1.335803in,,bottom,rotate=90.000000]{\color{textcolor}\rmfamily\fontsize{10.000000}{12.000000}\selectfont angular velocity (\(\displaystyle \frac{rad}{s}\))}%
\end{pgfscope}%
\begin{pgfscope}%
\pgfpathrectangle{\pgfqpoint{0.727040in}{0.526234in}}{\pgfqpoint{2.469823in}{1.619136in}}%
\pgfusepath{clip}%
\pgfsetrectcap%
\pgfsetroundjoin%
\pgfsetlinewidth{1.505625pt}%
\definecolor{currentstroke}{rgb}{0.000000,0.000000,1.000000}%
\pgfsetstrokecolor{currentstroke}%
\pgfsetdash{}{0pt}%
\pgfpathmoveto{\pgfqpoint{2.365593in}{1.307800in}}%
\pgfpathlineto{\pgfqpoint{2.364698in}{1.284933in}}%
\pgfpathlineto{\pgfqpoint{2.360227in}{1.262178in}}%
\pgfpathlineto{\pgfqpoint{2.352205in}{1.239760in}}%
\pgfpathlineto{\pgfqpoint{2.340694in}{1.217903in}}%
\pgfpathlineto{\pgfqpoint{2.325791in}{1.196825in}}%
\pgfpathlineto{\pgfqpoint{2.307626in}{1.176741in}}%
\pgfpathlineto{\pgfqpoint{2.286362in}{1.157853in}}%
\pgfpathlineto{\pgfqpoint{2.262195in}{1.140358in}}%
\pgfpathlineto{\pgfqpoint{2.235350in}{1.124438in}}%
\pgfpathlineto{\pgfqpoint{2.206079in}{1.110262in}}%
\pgfpathlineto{\pgfqpoint{2.174663in}{1.097981in}}%
\pgfpathlineto{\pgfqpoint{2.141403in}{1.087732in}}%
\pgfpathlineto{\pgfqpoint{2.106622in}{1.079627in}}%
\pgfpathlineto{\pgfqpoint{2.070659in}{1.073762in}}%
\pgfpathlineto{\pgfqpoint{2.033868in}{1.070208in}}%
\pgfpathlineto{\pgfqpoint{1.996613in}{1.069013in}}%
\pgfpathlineto{\pgfqpoint{1.959264in}{1.070202in}}%
\pgfpathlineto{\pgfqpoint{1.922194in}{1.073773in}}%
\pgfpathlineto{\pgfqpoint{1.885776in}{1.079702in}}%
\pgfpathlineto{\pgfqpoint{1.850376in}{1.087941in}}%
\pgfpathlineto{\pgfqpoint{1.816355in}{1.098414in}}%
\pgfpathlineto{\pgfqpoint{1.784057in}{1.111027in}}%
\pgfpathlineto{\pgfqpoint{1.753814in}{1.125661in}}%
\pgfpathlineto{\pgfqpoint{1.725936in}{1.142178in}}%
\pgfpathlineto{\pgfqpoint{1.700712in}{1.160419in}}%
\pgfpathlineto{\pgfqpoint{1.678406in}{1.180210in}}%
\pgfpathlineto{\pgfqpoint{1.659252in}{1.201360in}}%
\pgfpathlineto{\pgfqpoint{1.643456in}{1.223665in}}%
\pgfpathlineto{\pgfqpoint{1.631189in}{1.246909in}}%
\pgfpathlineto{\pgfqpoint{1.622591in}{1.270867in}}%
\pgfpathlineto{\pgfqpoint{1.617763in}{1.295306in}}%
\pgfpathlineto{\pgfqpoint{1.616772in}{1.319988in}}%
\pgfpathlineto{\pgfqpoint{1.619646in}{1.344672in}}%
\pgfpathlineto{\pgfqpoint{1.626377in}{1.369115in}}%
\pgfpathlineto{\pgfqpoint{1.636917in}{1.393074in}}%
\pgfpathlineto{\pgfqpoint{1.651180in}{1.416313in}}%
\pgfpathlineto{\pgfqpoint{1.669046in}{1.438596in}}%
\pgfpathlineto{\pgfqpoint{1.690355in}{1.459699in}}%
\pgfpathlineto{\pgfqpoint{1.714914in}{1.479405in}}%
\pgfpathlineto{\pgfqpoint{1.742497in}{1.497508in}}%
\pgfpathlineto{\pgfqpoint{1.772844in}{1.513820in}}%
\pgfpathlineto{\pgfqpoint{1.805669in}{1.528165in}}%
\pgfpathlineto{\pgfqpoint{1.840658in}{1.540388in}}%
\pgfpathlineto{\pgfqpoint{1.877473in}{1.550353in}}%
\pgfpathlineto{\pgfqpoint{1.915755in}{1.557946in}}%
\pgfpathlineto{\pgfqpoint{1.955131in}{1.563077in}}%
\pgfpathlineto{\pgfqpoint{1.995211in}{1.565681in}}%
\pgfpathlineto{\pgfqpoint{2.035598in}{1.565717in}}%
\pgfpathlineto{\pgfqpoint{2.075890in}{1.563173in}}%
\pgfpathlineto{\pgfqpoint{2.115683in}{1.558062in}}%
\pgfpathlineto{\pgfqpoint{2.154577in}{1.550425in}}%
\pgfpathlineto{\pgfqpoint{2.192178in}{1.540327in}}%
\pgfpathlineto{\pgfqpoint{2.228105in}{1.527860in}}%
\pgfpathlineto{\pgfqpoint{2.261992in}{1.513141in}}%
\pgfpathlineto{\pgfqpoint{2.293491in}{1.496310in}}%
\pgfpathlineto{\pgfqpoint{2.322278in}{1.477525in}}%
\pgfpathlineto{\pgfqpoint{2.348054in}{1.456969in}}%
\pgfpathlineto{\pgfqpoint{2.370548in}{1.434838in}}%
\pgfpathlineto{\pgfqpoint{2.389524in}{1.411346in}}%
\pgfpathlineto{\pgfqpoint{2.404776in}{1.386720in}}%
\pgfpathlineto{\pgfqpoint{2.416137in}{1.361198in}}%
\pgfpathlineto{\pgfqpoint{2.423475in}{1.335026in}}%
\pgfpathlineto{\pgfqpoint{2.426699in}{1.308459in}}%
\pgfpathlineto{\pgfqpoint{2.425757in}{1.281755in}}%
\pgfpathlineto{\pgfqpoint{2.420639in}{1.255176in}}%
\pgfpathlineto{\pgfqpoint{2.411372in}{1.228982in}}%
\pgfpathlineto{\pgfqpoint{2.398029in}{1.203433in}}%
\pgfpathlineto{\pgfqpoint{2.380721in}{1.178783in}}%
\pgfpathlineto{\pgfqpoint{2.359596in}{1.155282in}}%
\pgfpathlineto{\pgfqpoint{2.334846in}{1.133167in}}%
\pgfpathlineto{\pgfqpoint{2.306695in}{1.112668in}}%
\pgfpathlineto{\pgfqpoint{2.275406in}{1.093998in}}%
\pgfpathlineto{\pgfqpoint{2.241272in}{1.077356in}}%
\pgfpathlineto{\pgfqpoint{2.204618in}{1.062920in}}%
\pgfpathlineto{\pgfqpoint{2.165796in}{1.050851in}}%
\pgfpathlineto{\pgfqpoint{2.125182in}{1.041283in}}%
\pgfpathlineto{\pgfqpoint{2.083172in}{1.034330in}}%
\pgfpathlineto{\pgfqpoint{2.040178in}{1.030075in}}%
\pgfpathlineto{\pgfqpoint{1.996626in}{1.028577in}}%
\pgfpathlineto{\pgfqpoint{1.952948in}{1.029866in}}%
\pgfpathlineto{\pgfqpoint{1.909581in}{1.033941in}}%
\pgfpathlineto{\pgfqpoint{1.866960in}{1.040775in}}%
\pgfpathlineto{\pgfqpoint{1.825516in}{1.050310in}}%
\pgfpathlineto{\pgfqpoint{1.785666in}{1.062460in}}%
\pgfpathlineto{\pgfqpoint{1.747818in}{1.077112in}}%
\pgfpathlineto{\pgfqpoint{1.712357in}{1.094129in}}%
\pgfpathlineto{\pgfqpoint{1.679646in}{1.113348in}}%
\pgfpathlineto{\pgfqpoint{1.650025in}{1.134585in}}%
\pgfpathlineto{\pgfqpoint{1.623800in}{1.157635in}}%
\pgfpathlineto{\pgfqpoint{1.601248in}{1.182276in}}%
\pgfpathlineto{\pgfqpoint{1.582607in}{1.208271in}}%
\pgfpathlineto{\pgfqpoint{1.568081in}{1.235369in}}%
\pgfpathlineto{\pgfqpoint{1.557831in}{1.263308in}}%
\pgfpathlineto{\pgfqpoint{1.551979in}{1.291819in}}%
\pgfpathlineto{\pgfqpoint{1.550728in}{1.306202in}}%
\pgfpathlineto{\pgfqpoint{1.550603in}{1.320625in}}%
\pgfpathlineto{\pgfqpoint{1.551607in}{1.335051in}}%
\pgfpathlineto{\pgfqpoint{1.553739in}{1.349446in}}%
\pgfpathlineto{\pgfqpoint{1.556998in}{1.363775in}}%
\pgfpathlineto{\pgfqpoint{1.566872in}{1.392093in}}%
\pgfpathlineto{\pgfqpoint{1.581154in}{1.419724in}}%
\pgfpathlineto{\pgfqpoint{1.599725in}{1.446396in}}%
\pgfpathlineto{\pgfqpoint{1.622425in}{1.471838in}}%
\pgfpathlineto{\pgfqpoint{1.649052in}{1.495794in}}%
\pgfpathlineto{\pgfqpoint{1.679362in}{1.518015in}}%
\pgfpathlineto{\pgfqpoint{1.713075in}{1.538271in}}%
\pgfpathlineto{\pgfqpoint{1.749875in}{1.556345in}}%
\pgfpathlineto{\pgfqpoint{1.789413in}{1.572043in}}%
\pgfpathlineto{\pgfqpoint{1.831310in}{1.585191in}}%
\pgfpathlineto{\pgfqpoint{1.875160in}{1.595639in}}%
\pgfpathlineto{\pgfqpoint{1.920536in}{1.603267in}}%
\pgfpathlineto{\pgfqpoint{1.966992in}{1.607980in}}%
\pgfpathlineto{\pgfqpoint{2.014070in}{1.609713in}}%
\pgfpathlineto{\pgfqpoint{2.061301in}{1.608434in}}%
\pgfpathlineto{\pgfqpoint{2.108215in}{1.604142in}}%
\pgfpathlineto{\pgfqpoint{2.154339in}{1.596866in}}%
\pgfpathlineto{\pgfqpoint{2.199209in}{1.586667in}}%
\pgfpathlineto{\pgfqpoint{2.242372in}{1.573638in}}%
\pgfpathlineto{\pgfqpoint{2.283388in}{1.557900in}}%
\pgfpathlineto{\pgfqpoint{2.321839in}{1.539603in}}%
\pgfpathlineto{\pgfqpoint{2.357331in}{1.518921in}}%
\pgfpathlineto{\pgfqpoint{2.389499in}{1.496054in}}%
\pgfpathlineto{\pgfqpoint{2.418009in}{1.471221in}}%
\pgfpathlineto{\pgfqpoint{2.442562in}{1.444663in}}%
\pgfpathlineto{\pgfqpoint{2.462899in}{1.416635in}}%
\pgfpathlineto{\pgfqpoint{2.471416in}{1.402153in}}%
\pgfpathlineto{\pgfqpoint{2.478799in}{1.387407in}}%
\pgfpathlineto{\pgfqpoint{2.485029in}{1.372430in}}%
\pgfpathlineto{\pgfqpoint{2.490086in}{1.357260in}}%
\pgfpathlineto{\pgfqpoint{2.493957in}{1.341933in}}%
\pgfpathlineto{\pgfqpoint{2.496628in}{1.326485in}}%
\pgfpathlineto{\pgfqpoint{2.498090in}{1.310954in}}%
\pgfpathlineto{\pgfqpoint{2.498337in}{1.295378in}}%
\pgfpathlineto{\pgfqpoint{2.497365in}{1.279795in}}%
\pgfpathlineto{\pgfqpoint{2.495173in}{1.264241in}}%
\pgfpathlineto{\pgfqpoint{2.491764in}{1.248756in}}%
\pgfpathlineto{\pgfqpoint{2.487144in}{1.233377in}}%
\pgfpathlineto{\pgfqpoint{2.481320in}{1.218141in}}%
\pgfpathlineto{\pgfqpoint{2.474304in}{1.203086in}}%
\pgfpathlineto{\pgfqpoint{2.466109in}{1.188249in}}%
\pgfpathlineto{\pgfqpoint{2.456754in}{1.173668in}}%
\pgfpathlineto{\pgfqpoint{2.434643in}{1.145416in}}%
\pgfpathlineto{\pgfqpoint{2.408164in}{1.118614in}}%
\pgfpathlineto{\pgfqpoint{2.377555in}{1.093537in}}%
\pgfpathlineto{\pgfqpoint{2.343097in}{1.070444in}}%
\pgfpathlineto{\pgfqpoint{2.305109in}{1.049580in}}%
\pgfpathlineto{\pgfqpoint{2.263950in}{1.031166in}}%
\pgfpathlineto{\pgfqpoint{2.220011in}{1.015406in}}%
\pgfpathlineto{\pgfqpoint{2.173714in}{1.002476in}}%
\pgfpathlineto{\pgfqpoint{2.125509in}{0.992524in}}%
\pgfpathlineto{\pgfqpoint{2.075867in}{0.985672in}}%
\pgfpathlineto{\pgfqpoint{2.025276in}{0.982005in}}%
\pgfpathlineto{\pgfqpoint{1.974238in}{0.981580in}}%
\pgfpathlineto{\pgfqpoint{1.923260in}{0.984416in}}%
\pgfpathlineto{\pgfqpoint{1.872854in}{0.990501in}}%
\pgfpathlineto{\pgfqpoint{1.823526in}{0.999785in}}%
\pgfpathlineto{\pgfqpoint{1.775774in}{1.012187in}}%
\pgfpathlineto{\pgfqpoint{1.730082in}{1.027592in}}%
\pgfpathlineto{\pgfqpoint{1.686915in}{1.045855in}}%
\pgfpathlineto{\pgfqpoint{1.646712in}{1.066799in}}%
\pgfpathlineto{\pgfqpoint{1.609887in}{1.090224in}}%
\pgfpathlineto{\pgfqpoint{1.576818in}{1.115903in}}%
\pgfpathlineto{\pgfqpoint{1.547849in}{1.143589in}}%
\pgfpathlineto{\pgfqpoint{1.534999in}{1.158101in}}%
\pgfpathlineto{\pgfqpoint{1.523285in}{1.173014in}}%
\pgfpathlineto{\pgfqpoint{1.512737in}{1.188290in}}%
\pgfpathlineto{\pgfqpoint{1.503385in}{1.203895in}}%
\pgfpathlineto{\pgfqpoint{1.495254in}{1.219789in}}%
\pgfpathlineto{\pgfqpoint{1.488366in}{1.235935in}}%
\pgfpathlineto{\pgfqpoint{1.482743in}{1.252295in}}%
\pgfpathlineto{\pgfqpoint{1.478399in}{1.268828in}}%
\pgfpathlineto{\pgfqpoint{1.475349in}{1.285496in}}%
\pgfpathlineto{\pgfqpoint{1.473604in}{1.302258in}}%
\pgfpathlineto{\pgfqpoint{1.473170in}{1.319074in}}%
\pgfpathlineto{\pgfqpoint{1.474052in}{1.335904in}}%
\pgfpathlineto{\pgfqpoint{1.476252in}{1.352706in}}%
\pgfpathlineto{\pgfqpoint{1.479766in}{1.369440in}}%
\pgfpathlineto{\pgfqpoint{1.484589in}{1.386065in}}%
\pgfpathlineto{\pgfqpoint{1.490714in}{1.402542in}}%
\pgfpathlineto{\pgfqpoint{1.498128in}{1.418829in}}%
\pgfpathlineto{\pgfqpoint{1.506816in}{1.434886in}}%
\pgfpathlineto{\pgfqpoint{1.516761in}{1.450674in}}%
\pgfpathlineto{\pgfqpoint{1.527942in}{1.466153in}}%
\pgfpathlineto{\pgfqpoint{1.540334in}{1.481285in}}%
\pgfpathlineto{\pgfqpoint{1.553910in}{1.496032in}}%
\pgfpathlineto{\pgfqpoint{1.584490in}{1.524219in}}%
\pgfpathlineto{\pgfqpoint{1.619408in}{1.550425in}}%
\pgfpathlineto{\pgfqpoint{1.658341in}{1.574377in}}%
\pgfpathlineto{\pgfqpoint{1.700928in}{1.595819in}}%
\pgfpathlineto{\pgfqpoint{1.746765in}{1.614520in}}%
\pgfpathlineto{\pgfqpoint{1.795415in}{1.630270in}}%
\pgfpathlineto{\pgfqpoint{1.846410in}{1.642891in}}%
\pgfpathlineto{\pgfqpoint{1.899253in}{1.652234in}}%
\pgfpathlineto{\pgfqpoint{1.953426in}{1.658183in}}%
\pgfpathlineto{\pgfqpoint{2.008395in}{1.660658in}}%
\pgfpathlineto{\pgfqpoint{2.063614in}{1.659614in}}%
\pgfpathlineto{\pgfqpoint{2.118532in}{1.655046in}}%
\pgfpathlineto{\pgfqpoint{2.172597in}{1.646985in}}%
\pgfpathlineto{\pgfqpoint{2.225266in}{1.635499in}}%
\pgfpathlineto{\pgfqpoint{2.276006in}{1.620693in}}%
\pgfpathlineto{\pgfqpoint{2.324303in}{1.602708in}}%
\pgfpathlineto{\pgfqpoint{2.369666in}{1.581715in}}%
\pgfpathlineto{\pgfqpoint{2.411631in}{1.557917in}}%
\pgfpathlineto{\pgfqpoint{2.449769in}{1.531546in}}%
\pgfpathlineto{\pgfqpoint{2.467278in}{1.517474in}}%
\pgfpathlineto{\pgfqpoint{2.483686in}{1.502856in}}%
\pgfpathlineto{\pgfqpoint{2.498950in}{1.487728in}}%
\pgfpathlineto{\pgfqpoint{2.513030in}{1.472125in}}%
\pgfpathlineto{\pgfqpoint{2.525889in}{1.456086in}}%
\pgfpathlineto{\pgfqpoint{2.537493in}{1.439648in}}%
\pgfpathlineto{\pgfqpoint{2.547811in}{1.422852in}}%
\pgfpathlineto{\pgfqpoint{2.556814in}{1.405738in}}%
\pgfpathlineto{\pgfqpoint{2.564478in}{1.388345in}}%
\pgfpathlineto{\pgfqpoint{2.570781in}{1.370717in}}%
\pgfpathlineto{\pgfqpoint{2.575705in}{1.352895in}}%
\pgfpathlineto{\pgfqpoint{2.579234in}{1.334922in}}%
\pgfpathlineto{\pgfqpoint{2.581356in}{1.316840in}}%
\pgfpathlineto{\pgfqpoint{2.582063in}{1.298692in}}%
\pgfpathlineto{\pgfqpoint{2.581351in}{1.280524in}}%
\pgfpathlineto{\pgfqpoint{2.579216in}{1.262377in}}%
\pgfpathlineto{\pgfqpoint{2.575661in}{1.244295in}}%
\pgfpathlineto{\pgfqpoint{2.570692in}{1.226323in}}%
\pgfpathlineto{\pgfqpoint{2.564316in}{1.208504in}}%
\pgfpathlineto{\pgfqpoint{2.556546in}{1.190880in}}%
\pgfpathlineto{\pgfqpoint{2.547396in}{1.173497in}}%
\pgfpathlineto{\pgfqpoint{2.536886in}{1.156395in}}%
\pgfpathlineto{\pgfqpoint{2.525038in}{1.139617in}}%
\pgfpathlineto{\pgfqpoint{2.511877in}{1.123206in}}%
\pgfpathlineto{\pgfqpoint{2.497432in}{1.107203in}}%
\pgfpathlineto{\pgfqpoint{2.481734in}{1.091647in}}%
\pgfpathlineto{\pgfqpoint{2.464819in}{1.076579in}}%
\pgfpathlineto{\pgfqpoint{2.446725in}{1.062038in}}%
\pgfpathlineto{\pgfqpoint{2.427493in}{1.048061in}}%
\pgfpathlineto{\pgfqpoint{2.385795in}{1.021946in}}%
\pgfpathlineto{\pgfqpoint{2.340112in}{0.998512in}}%
\pgfpathlineto{\pgfqpoint{2.290873in}{0.978013in}}%
\pgfpathlineto{\pgfqpoint{2.238549in}{0.960678in}}%
\pgfpathlineto{\pgfqpoint{2.183640in}{0.946706in}}%
\pgfpathlineto{\pgfqpoint{2.126682in}{0.936261in}}%
\pgfpathlineto{\pgfqpoint{2.068231in}{0.929474in}}%
\pgfpathlineto{\pgfqpoint{2.008864in}{0.926436in}}%
\pgfpathlineto{\pgfqpoint{1.949169in}{0.927198in}}%
\pgfpathlineto{\pgfqpoint{1.889743in}{0.931771in}}%
\pgfpathlineto{\pgfqpoint{1.831182in}{0.940125in}}%
\pgfpathlineto{\pgfqpoint{1.774073in}{0.952189in}}%
\pgfpathlineto{\pgfqpoint{1.718995in}{0.967851in}}%
\pgfpathlineto{\pgfqpoint{1.666505in}{0.986962in}}%
\pgfpathlineto{\pgfqpoint{1.617136in}{1.009338in}}%
\pgfpathlineto{\pgfqpoint{1.571390in}{1.034761in}}%
\pgfpathlineto{\pgfqpoint{1.550023in}{1.048539in}}%
\pgfpathlineto{\pgfqpoint{1.529735in}{1.062983in}}%
\pgfpathlineto{\pgfqpoint{1.510577in}{1.078059in}}%
\pgfpathlineto{\pgfqpoint{1.492599in}{1.093731in}}%
\pgfpathlineto{\pgfqpoint{1.475847in}{1.109959in}}%
\pgfpathlineto{\pgfqpoint{1.460365in}{1.126706in}}%
\pgfpathlineto{\pgfqpoint{1.446193in}{1.143931in}}%
\pgfpathlineto{\pgfqpoint{1.433370in}{1.161592in}}%
\pgfpathlineto{\pgfqpoint{1.421928in}{1.179647in}}%
\pgfpathlineto{\pgfqpoint{1.411900in}{1.198054in}}%
\pgfpathlineto{\pgfqpoint{1.403312in}{1.216768in}}%
\pgfpathlineto{\pgfqpoint{1.396188in}{1.235745in}}%
\pgfpathlineto{\pgfqpoint{1.390549in}{1.254939in}}%
\pgfpathlineto{\pgfqpoint{1.386413in}{1.274306in}}%
\pgfpathlineto{\pgfqpoint{1.383792in}{1.293800in}}%
\pgfpathlineto{\pgfqpoint{1.382696in}{1.313373in}}%
\pgfpathlineto{\pgfqpoint{1.383132in}{1.332980in}}%
\pgfpathlineto{\pgfqpoint{1.385103in}{1.352573in}}%
\pgfpathlineto{\pgfqpoint{1.388606in}{1.372106in}}%
\pgfpathlineto{\pgfqpoint{1.393638in}{1.391532in}}%
\pgfpathlineto{\pgfqpoint{1.400191in}{1.410805in}}%
\pgfpathlineto{\pgfqpoint{1.408251in}{1.429878in}}%
\pgfpathlineto{\pgfqpoint{1.417804in}{1.448703in}}%
\pgfpathlineto{\pgfqpoint{1.428831in}{1.467236in}}%
\pgfpathlineto{\pgfqpoint{1.441307in}{1.485431in}}%
\pgfpathlineto{\pgfqpoint{1.455208in}{1.503243in}}%
\pgfpathlineto{\pgfqpoint{1.470502in}{1.520626in}}%
\pgfpathlineto{\pgfqpoint{1.487156in}{1.537538in}}%
\pgfpathlineto{\pgfqpoint{1.505134in}{1.553935in}}%
\pgfpathlineto{\pgfqpoint{1.524396in}{1.569775in}}%
\pgfpathlineto{\pgfqpoint{1.544896in}{1.585017in}}%
\pgfpathlineto{\pgfqpoint{1.566590in}{1.599622in}}%
\pgfpathlineto{\pgfqpoint{1.589426in}{1.613549in}}%
\pgfpathlineto{\pgfqpoint{1.613352in}{1.626763in}}%
\pgfpathlineto{\pgfqpoint{1.638312in}{1.639227in}}%
\pgfpathlineto{\pgfqpoint{1.664248in}{1.650907in}}%
\pgfpathlineto{\pgfqpoint{1.718798in}{1.671788in}}%
\pgfpathlineto{\pgfqpoint{1.776480in}{1.689169in}}%
\pgfpathlineto{\pgfqpoint{1.836740in}{1.702848in}}%
\pgfpathlineto{\pgfqpoint{1.898991in}{1.712657in}}%
\pgfpathlineto{\pgfqpoint{1.962622in}{1.718472in}}%
\pgfpathlineto{\pgfqpoint{2.027003in}{1.720208in}}%
\pgfpathlineto{\pgfqpoint{2.091496in}{1.717827in}}%
\pgfpathlineto{\pgfqpoint{2.155455in}{1.711334in}}%
\pgfpathlineto{\pgfqpoint{2.218238in}{1.700779in}}%
\pgfpathlineto{\pgfqpoint{2.279212in}{1.686258in}}%
\pgfpathlineto{\pgfqpoint{2.337763in}{1.667907in}}%
\pgfpathlineto{\pgfqpoint{2.365943in}{1.657350in}}%
\pgfpathlineto{\pgfqpoint{2.393297in}{1.645905in}}%
\pgfpathlineto{\pgfqpoint{2.419755in}{1.633600in}}%
\pgfpathlineto{\pgfqpoint{2.445250in}{1.620467in}}%
\pgfpathlineto{\pgfqpoint{2.469718in}{1.606536in}}%
\pgfpathlineto{\pgfqpoint{2.493095in}{1.591842in}}%
\pgfpathlineto{\pgfqpoint{2.515323in}{1.576421in}}%
\pgfpathlineto{\pgfqpoint{2.536343in}{1.560311in}}%
\pgfpathlineto{\pgfqpoint{2.556103in}{1.543550in}}%
\pgfpathlineto{\pgfqpoint{2.574552in}{1.526180in}}%
\pgfpathlineto{\pgfqpoint{2.591641in}{1.508242in}}%
\pgfpathlineto{\pgfqpoint{2.607326in}{1.489779in}}%
\pgfpathlineto{\pgfqpoint{2.621567in}{1.470836in}}%
\pgfpathlineto{\pgfqpoint{2.634325in}{1.451457in}}%
\pgfpathlineto{\pgfqpoint{2.645567in}{1.431690in}}%
\pgfpathlineto{\pgfqpoint{2.655262in}{1.411580in}}%
\pgfpathlineto{\pgfqpoint{2.663383in}{1.391175in}}%
\pgfpathlineto{\pgfqpoint{2.669908in}{1.370524in}}%
\pgfpathlineto{\pgfqpoint{2.674816in}{1.349675in}}%
\pgfpathlineto{\pgfqpoint{2.678093in}{1.328678in}}%
\pgfpathlineto{\pgfqpoint{2.679727in}{1.307581in}}%
\pgfpathlineto{\pgfqpoint{2.679710in}{1.286434in}}%
\pgfpathlineto{\pgfqpoint{2.678038in}{1.265289in}}%
\pgfpathlineto{\pgfqpoint{2.674711in}{1.244194in}}%
\pgfpathlineto{\pgfqpoint{2.669733in}{1.223199in}}%
\pgfpathlineto{\pgfqpoint{2.663113in}{1.202355in}}%
\pgfpathlineto{\pgfqpoint{2.654861in}{1.181712in}}%
\pgfpathlineto{\pgfqpoint{2.644995in}{1.161320in}}%
\pgfpathlineto{\pgfqpoint{2.633532in}{1.141227in}}%
\pgfpathlineto{\pgfqpoint{2.620497in}{1.121484in}}%
\pgfpathlineto{\pgfqpoint{2.605917in}{1.102138in}}%
\pgfpathlineto{\pgfqpoint{2.589823in}{1.083238in}}%
\pgfpathlineto{\pgfqpoint{2.572250in}{1.064831in}}%
\pgfpathlineto{\pgfqpoint{2.553236in}{1.046964in}}%
\pgfpathlineto{\pgfqpoint{2.532825in}{1.029682in}}%
\pgfpathlineto{\pgfqpoint{2.511061in}{1.013030in}}%
\pgfpathlineto{\pgfqpoint{2.487994in}{0.997052in}}%
\pgfpathlineto{\pgfqpoint{2.463676in}{0.981789in}}%
\pgfpathlineto{\pgfqpoint{2.438164in}{0.967284in}}%
\pgfpathlineto{\pgfqpoint{2.411518in}{0.953574in}}%
\pgfpathlineto{\pgfqpoint{2.383798in}{0.940699in}}%
\pgfpathlineto{\pgfqpoint{2.355070in}{0.928692in}}%
\pgfpathlineto{\pgfqpoint{2.325404in}{0.917590in}}%
\pgfpathlineto{\pgfqpoint{2.294868in}{0.907422in}}%
\pgfpathlineto{\pgfqpoint{2.263537in}{0.898220in}}%
\pgfpathlineto{\pgfqpoint{2.231485in}{0.890010in}}%
\pgfpathlineto{\pgfqpoint{2.198791in}{0.882816in}}%
\pgfpathlineto{\pgfqpoint{2.165534in}{0.876662in}}%
\pgfpathlineto{\pgfqpoint{2.131796in}{0.871566in}}%
\pgfpathlineto{\pgfqpoint{2.097659in}{0.867546in}}%
\pgfpathlineto{\pgfqpoint{2.063207in}{0.864615in}}%
\pgfpathlineto{\pgfqpoint{2.028526in}{0.862783in}}%
\pgfpathlineto{\pgfqpoint{1.993701in}{0.862059in}}%
\pgfpathlineto{\pgfqpoint{1.958820in}{0.862448in}}%
\pgfpathlineto{\pgfqpoint{1.923969in}{0.863951in}}%
\pgfpathlineto{\pgfqpoint{1.889236in}{0.866567in}}%
\pgfpathlineto{\pgfqpoint{1.854707in}{0.870291in}}%
\pgfpathlineto{\pgfqpoint{1.820470in}{0.875116in}}%
\pgfpathlineto{\pgfqpoint{1.786611in}{0.881031in}}%
\pgfpathlineto{\pgfqpoint{1.753214in}{0.888022in}}%
\pgfpathlineto{\pgfqpoint{1.720365in}{0.896074in}}%
\pgfpathlineto{\pgfqpoint{1.688145in}{0.905166in}}%
\pgfpathlineto{\pgfqpoint{1.656638in}{0.915276in}}%
\pgfpathlineto{\pgfqpoint{1.625921in}{0.926380in}}%
\pgfpathlineto{\pgfqpoint{1.596073in}{0.938449in}}%
\pgfpathlineto{\pgfqpoint{1.567170in}{0.951455in}}%
\pgfpathlineto{\pgfqpoint{1.539284in}{0.965364in}}%
\pgfpathlineto{\pgfqpoint{1.512487in}{0.980144in}}%
\pgfpathlineto{\pgfqpoint{1.486846in}{0.995756in}}%
\pgfpathlineto{\pgfqpoint{1.462427in}{1.012162in}}%
\pgfpathlineto{\pgfqpoint{1.439292in}{1.029323in}}%
\pgfpathlineto{\pgfqpoint{1.417500in}{1.047197in}}%
\pgfpathlineto{\pgfqpoint{1.397107in}{1.065739in}}%
\pgfpathlineto{\pgfqpoint{1.378165in}{1.084906in}}%
\pgfpathlineto{\pgfqpoint{1.360722in}{1.104650in}}%
\pgfpathlineto{\pgfqpoint{1.344825in}{1.124926in}}%
\pgfpathlineto{\pgfqpoint{1.330514in}{1.145684in}}%
\pgfpathlineto{\pgfqpoint{1.317828in}{1.166876in}}%
\pgfpathlineto{\pgfqpoint{1.306800in}{1.188451in}}%
\pgfpathlineto{\pgfqpoint{1.297460in}{1.210358in}}%
\pgfpathlineto{\pgfqpoint{1.289835in}{1.232547in}}%
\pgfpathlineto{\pgfqpoint{1.283946in}{1.254965in}}%
\pgfpathlineto{\pgfqpoint{1.279811in}{1.277560in}}%
\pgfpathlineto{\pgfqpoint{1.277445in}{1.300278in}}%
\pgfpathlineto{\pgfqpoint{1.276856in}{1.323068in}}%
\pgfpathlineto{\pgfqpoint{1.278051in}{1.345875in}}%
\pgfpathlineto{\pgfqpoint{1.281031in}{1.368647in}}%
\pgfpathlineto{\pgfqpoint{1.285792in}{1.391329in}}%
\pgfpathlineto{\pgfqpoint{1.292329in}{1.413868in}}%
\pgfpathlineto{\pgfqpoint{1.300629in}{1.436211in}}%
\pgfpathlineto{\pgfqpoint{1.310678in}{1.458304in}}%
\pgfpathlineto{\pgfqpoint{1.322455in}{1.480095in}}%
\pgfpathlineto{\pgfqpoint{1.335938in}{1.501530in}}%
\pgfpathlineto{\pgfqpoint{1.351098in}{1.522558in}}%
\pgfpathlineto{\pgfqpoint{1.367904in}{1.543126in}}%
\pgfpathlineto{\pgfqpoint{1.386319in}{1.563183in}}%
\pgfpathlineto{\pgfqpoint{1.406304in}{1.582679in}}%
\pgfpathlineto{\pgfqpoint{1.427814in}{1.601564in}}%
\pgfpathlineto{\pgfqpoint{1.450803in}{1.619790in}}%
\pgfpathlineto{\pgfqpoint{1.475217in}{1.637310in}}%
\pgfpathlineto{\pgfqpoint{1.501003in}{1.654076in}}%
\pgfpathlineto{\pgfqpoint{1.528100in}{1.670044in}}%
\pgfpathlineto{\pgfqpoint{1.556448in}{1.685171in}}%
\pgfpathlineto{\pgfqpoint{1.585978in}{1.699415in}}%
\pgfpathlineto{\pgfqpoint{1.616624in}{1.712737in}}%
\pgfpathlineto{\pgfqpoint{1.648312in}{1.725098in}}%
\pgfpathlineto{\pgfqpoint{1.680967in}{1.736464in}}%
\pgfpathlineto{\pgfqpoint{1.714512in}{1.746801in}}%
\pgfpathlineto{\pgfqpoint{1.748866in}{1.756078in}}%
\pgfpathlineto{\pgfqpoint{1.783946in}{1.764267in}}%
\pgfpathlineto{\pgfqpoint{1.819666in}{1.771343in}}%
\pgfpathlineto{\pgfqpoint{1.855941in}{1.777284in}}%
\pgfpathlineto{\pgfqpoint{1.892680in}{1.782069in}}%
\pgfpathlineto{\pgfqpoint{1.929793in}{1.785684in}}%
\pgfpathlineto{\pgfqpoint{1.967190in}{1.788114in}}%
\pgfpathlineto{\pgfqpoint{2.004776in}{1.789349in}}%
\pgfpathlineto{\pgfqpoint{2.042460in}{1.789384in}}%
\pgfpathlineto{\pgfqpoint{2.080146in}{1.788215in}}%
\pgfpathlineto{\pgfqpoint{2.117740in}{1.785843in}}%
\pgfpathlineto{\pgfqpoint{2.155149in}{1.782271in}}%
\pgfpathlineto{\pgfqpoint{2.192279in}{1.777506in}}%
\pgfpathlineto{\pgfqpoint{2.229035in}{1.771559in}}%
\pgfpathlineto{\pgfqpoint{2.265326in}{1.764443in}}%
\pgfpathlineto{\pgfqpoint{2.301061in}{1.756176in}}%
\pgfpathlineto{\pgfqpoint{2.336148in}{1.746778in}}%
\pgfpathlineto{\pgfqpoint{2.370500in}{1.736272in}}%
\pgfpathlineto{\pgfqpoint{2.404030in}{1.724683in}}%
\pgfpathlineto{\pgfqpoint{2.436653in}{1.712042in}}%
\pgfpathlineto{\pgfqpoint{2.468286in}{1.698380in}}%
\pgfpathlineto{\pgfqpoint{2.498851in}{1.683731in}}%
\pgfpathlineto{\pgfqpoint{2.528269in}{1.668131in}}%
\pgfpathlineto{\pgfqpoint{2.556467in}{1.651620in}}%
\pgfpathlineto{\pgfqpoint{2.583372in}{1.634239in}}%
\pgfpathlineto{\pgfqpoint{2.608917in}{1.616030in}}%
\pgfpathlineto{\pgfqpoint{2.633038in}{1.597038in}}%
\pgfpathlineto{\pgfqpoint{2.655672in}{1.577309in}}%
\pgfpathlineto{\pgfqpoint{2.676762in}{1.556892in}}%
\pgfpathlineto{\pgfqpoint{2.696255in}{1.535836in}}%
\pgfpathlineto{\pgfqpoint{2.714099in}{1.514190in}}%
\pgfpathlineto{\pgfqpoint{2.730250in}{1.492007in}}%
\pgfpathlineto{\pgfqpoint{2.744665in}{1.469339in}}%
\pgfpathlineto{\pgfqpoint{2.757306in}{1.446238in}}%
\pgfpathlineto{\pgfqpoint{2.768140in}{1.422761in}}%
\pgfpathlineto{\pgfqpoint{2.777136in}{1.398960in}}%
\pgfpathlineto{\pgfqpoint{2.784270in}{1.374891in}}%
\pgfpathlineto{\pgfqpoint{2.789520in}{1.350610in}}%
\pgfpathlineto{\pgfqpoint{2.792870in}{1.326174in}}%
\pgfpathlineto{\pgfqpoint{2.794308in}{1.301638in}}%
\pgfpathlineto{\pgfqpoint{2.793825in}{1.277060in}}%
\pgfpathlineto{\pgfqpoint{2.791420in}{1.252496in}}%
\pgfpathlineto{\pgfqpoint{2.787092in}{1.228003in}}%
\pgfpathlineto{\pgfqpoint{2.780847in}{1.203638in}}%
\pgfpathlineto{\pgfqpoint{2.772696in}{1.179459in}}%
\pgfpathlineto{\pgfqpoint{2.762653in}{1.155523in}}%
\pgfpathlineto{\pgfqpoint{2.750737in}{1.131885in}}%
\pgfpathlineto{\pgfqpoint{2.736971in}{1.108604in}}%
\pgfpathlineto{\pgfqpoint{2.721383in}{1.085734in}}%
\pgfpathlineto{\pgfqpoint{2.704005in}{1.063332in}}%
\pgfpathlineto{\pgfqpoint{2.684874in}{1.041452in}}%
\pgfpathlineto{\pgfqpoint{2.664031in}{1.020150in}}%
\pgfpathlineto{\pgfqpoint{2.641521in}{0.999480in}}%
\pgfpathlineto{\pgfqpoint{2.617394in}{0.979493in}}%
\pgfpathlineto{\pgfqpoint{2.591702in}{0.960242in}}%
\pgfpathlineto{\pgfqpoint{2.564504in}{0.941777in}}%
\pgfpathlineto{\pgfqpoint{2.535861in}{0.924148in}}%
\pgfpathlineto{\pgfqpoint{2.505839in}{0.907402in}}%
\pgfpathlineto{\pgfqpoint{2.474506in}{0.891586in}}%
\pgfpathlineto{\pgfqpoint{2.441936in}{0.876743in}}%
\pgfpathlineto{\pgfqpoint{2.408203in}{0.862916in}}%
\pgfpathlineto{\pgfqpoint{2.373389in}{0.850144in}}%
\pgfpathlineto{\pgfqpoint{2.337576in}{0.838465in}}%
\pgfpathlineto{\pgfqpoint{2.300848in}{0.827914in}}%
\pgfpathlineto{\pgfqpoint{2.263295in}{0.818522in}}%
\pgfpathlineto{\pgfqpoint{2.225007in}{0.810319in}}%
\pgfpathlineto{\pgfqpoint{2.186077in}{0.803332in}}%
\pgfpathlineto{\pgfqpoint{2.146600in}{0.797582in}}%
\pgfpathlineto{\pgfqpoint{2.106673in}{0.793090in}}%
\pgfpathlineto{\pgfqpoint{2.066394in}{0.789871in}}%
\pgfpathlineto{\pgfqpoint{2.025864in}{0.787937in}}%
\pgfpathlineto{\pgfqpoint{1.985183in}{0.787299in}}%
\pgfpathlineto{\pgfqpoint{1.944451in}{0.787959in}}%
\pgfpathlineto{\pgfqpoint{1.903771in}{0.789921in}}%
\pgfpathlineto{\pgfqpoint{1.863245in}{0.793182in}}%
\pgfpathlineto{\pgfqpoint{1.822974in}{0.797735in}}%
\pgfpathlineto{\pgfqpoint{1.783059in}{0.803570in}}%
\pgfpathlineto{\pgfqpoint{1.743601in}{0.810674in}}%
\pgfpathlineto{\pgfqpoint{1.704698in}{0.819030in}}%
\pgfpathlineto{\pgfqpoint{1.666450in}{0.828618in}}%
\pgfpathlineto{\pgfqpoint{1.628952in}{0.839412in}}%
\pgfpathlineto{\pgfqpoint{1.592298in}{0.851387in}}%
\pgfpathlineto{\pgfqpoint{1.556582in}{0.864511in}}%
\pgfpathlineto{\pgfqpoint{1.521893in}{0.878751in}}%
\pgfpathlineto{\pgfqpoint{1.488318in}{0.894071in}}%
\pgfpathlineto{\pgfqpoint{1.455941in}{0.910432in}}%
\pgfpathlineto{\pgfqpoint{1.424846in}{0.927793in}}%
\pgfpathlineto{\pgfqpoint{1.395108in}{0.946110in}}%
\pgfpathlineto{\pgfqpoint{1.366804in}{0.965337in}}%
\pgfpathlineto{\pgfqpoint{1.340005in}{0.985428in}}%
\pgfpathlineto{\pgfqpoint{1.314778in}{1.006332in}}%
\pgfpathlineto{\pgfqpoint{1.291187in}{1.027999in}}%
\pgfpathlineto{\pgfqpoint{1.269291in}{1.050377in}}%
\pgfpathlineto{\pgfqpoint{1.249147in}{1.073411in}}%
\pgfpathlineto{\pgfqpoint{1.230805in}{1.097046in}}%
\pgfpathlineto{\pgfqpoint{1.214312in}{1.121228in}}%
\pgfpathlineto{\pgfqpoint{1.199712in}{1.145899in}}%
\pgfpathlineto{\pgfqpoint{1.187043in}{1.171001in}}%
\pgfpathlineto{\pgfqpoint{1.176338in}{1.196478in}}%
\pgfpathlineto{\pgfqpoint{1.167626in}{1.222269in}}%
\pgfpathlineto{\pgfqpoint{1.160933in}{1.248316in}}%
\pgfpathlineto{\pgfqpoint{1.156278in}{1.274560in}}%
\pgfpathlineto{\pgfqpoint{1.153677in}{1.300940in}}%
\pgfpathlineto{\pgfqpoint{1.153140in}{1.327396in}}%
\pgfpathlineto{\pgfqpoint{1.154673in}{1.353867in}}%
\pgfpathlineto{\pgfqpoint{1.158278in}{1.380294in}}%
\pgfpathlineto{\pgfqpoint{1.163951in}{1.406616in}}%
\pgfpathlineto{\pgfqpoint{1.171684in}{1.432771in}}%
\pgfpathlineto{\pgfqpoint{1.181463in}{1.458699in}}%
\pgfpathlineto{\pgfqpoint{1.193272in}{1.484339in}}%
\pgfpathlineto{\pgfqpoint{1.207087in}{1.509632in}}%
\pgfpathlineto{\pgfqpoint{1.222881in}{1.534517in}}%
\pgfpathlineto{\pgfqpoint{1.240623in}{1.558934in}}%
\pgfpathlineto{\pgfqpoint{1.260275in}{1.582823in}}%
\pgfpathlineto{\pgfqpoint{1.281797in}{1.606127in}}%
\pgfpathlineto{\pgfqpoint{1.305142in}{1.628786in}}%
\pgfpathlineto{\pgfqpoint{1.330261in}{1.650743in}}%
\pgfpathlineto{\pgfqpoint{1.357097in}{1.671943in}}%
\pgfpathlineto{\pgfqpoint{1.385593in}{1.692329in}}%
\pgfpathlineto{\pgfqpoint{1.415684in}{1.711847in}}%
\pgfpathlineto{\pgfqpoint{1.447303in}{1.730445in}}%
\pgfpathlineto{\pgfqpoint{1.480376in}{1.748072in}}%
\pgfpathlineto{\pgfqpoint{1.514830in}{1.764679in}}%
\pgfpathlineto{\pgfqpoint{1.550582in}{1.780218in}}%
\pgfpathlineto{\pgfqpoint{1.587551in}{1.794645in}}%
\pgfpathlineto{\pgfqpoint{1.625649in}{1.807917in}}%
\pgfpathlineto{\pgfqpoint{1.664785in}{1.819996in}}%
\pgfpathlineto{\pgfqpoint{1.704867in}{1.830843in}}%
\pgfpathlineto{\pgfqpoint{1.745797in}{1.840425in}}%
\pgfpathlineto{\pgfqpoint{1.787477in}{1.848711in}}%
\pgfpathlineto{\pgfqpoint{1.829806in}{1.855675in}}%
\pgfpathlineto{\pgfqpoint{1.872680in}{1.861293in}}%
\pgfpathlineto{\pgfqpoint{1.915993in}{1.865545in}}%
\pgfpathlineto{\pgfqpoint{1.959639in}{1.868416in}}%
\pgfpathlineto{\pgfqpoint{2.003509in}{1.869892in}}%
\pgfpathlineto{\pgfqpoint{2.047496in}{1.869968in}}%
\pgfpathlineto{\pgfqpoint{2.091488in}{1.868638in}}%
\pgfpathlineto{\pgfqpoint{2.135376in}{1.865904in}}%
\pgfpathlineto{\pgfqpoint{2.179050in}{1.861769in}}%
\pgfpathlineto{\pgfqpoint{2.222400in}{1.856243in}}%
\pgfpathlineto{\pgfqpoint{2.265318in}{1.849338in}}%
\pgfpathlineto{\pgfqpoint{2.307696in}{1.841071in}}%
\pgfpathlineto{\pgfqpoint{2.349427in}{1.831463in}}%
\pgfpathlineto{\pgfqpoint{2.390406in}{1.820538in}}%
\pgfpathlineto{\pgfqpoint{2.430530in}{1.808325in}}%
\pgfpathlineto{\pgfqpoint{2.469698in}{1.794854in}}%
\pgfpathlineto{\pgfqpoint{2.507812in}{1.780161in}}%
\pgfpathlineto{\pgfqpoint{2.544776in}{1.764283in}}%
\pgfpathlineto{\pgfqpoint{2.580498in}{1.747262in}}%
\pgfpathlineto{\pgfqpoint{2.614888in}{1.729141in}}%
\pgfpathlineto{\pgfqpoint{2.647860in}{1.709967in}}%
\pgfpathlineto{\pgfqpoint{2.679331in}{1.689787in}}%
\pgfpathlineto{\pgfqpoint{2.709223in}{1.668653in}}%
\pgfpathlineto{\pgfqpoint{2.737462in}{1.646617in}}%
\pgfpathlineto{\pgfqpoint{2.763975in}{1.623732in}}%
\pgfpathlineto{\pgfqpoint{2.788699in}{1.600055in}}%
\pgfpathlineto{\pgfqpoint{2.811569in}{1.575643in}}%
\pgfpathlineto{\pgfqpoint{2.832529in}{1.550554in}}%
\pgfpathlineto{\pgfqpoint{2.851525in}{1.524847in}}%
\pgfpathlineto{\pgfqpoint{2.868510in}{1.498582in}}%
\pgfpathlineto{\pgfqpoint{2.883440in}{1.471821in}}%
\pgfpathlineto{\pgfqpoint{2.896275in}{1.444625in}}%
\pgfpathlineto{\pgfqpoint{2.906982in}{1.417056in}}%
\pgfpathlineto{\pgfqpoint{2.915532in}{1.389176in}}%
\pgfpathlineto{\pgfqpoint{2.921900in}{1.361050in}}%
\pgfpathlineto{\pgfqpoint{2.926067in}{1.332739in}}%
\pgfpathlineto{\pgfqpoint{2.928018in}{1.304308in}}%
\pgfpathlineto{\pgfqpoint{2.927745in}{1.275821in}}%
\pgfpathlineto{\pgfqpoint{2.925243in}{1.247343in}}%
\pgfpathlineto{\pgfqpoint{2.920512in}{1.218936in}}%
\pgfpathlineto{\pgfqpoint{2.913557in}{1.190665in}}%
\pgfpathlineto{\pgfqpoint{2.904391in}{1.162596in}}%
\pgfpathlineto{\pgfqpoint{2.893028in}{1.134791in}}%
\pgfpathlineto{\pgfqpoint{2.879489in}{1.107316in}}%
\pgfpathlineto{\pgfqpoint{2.863801in}{1.080235in}}%
\pgfpathlineto{\pgfqpoint{2.845993in}{1.053611in}}%
\pgfpathlineto{\pgfqpoint{2.826101in}{1.027509in}}%
\pgfpathlineto{\pgfqpoint{2.804167in}{1.001991in}}%
\pgfpathlineto{\pgfqpoint{2.780236in}{0.977120in}}%
\pgfpathlineto{\pgfqpoint{2.754359in}{0.952960in}}%
\pgfpathlineto{\pgfqpoint{2.726591in}{0.929570in}}%
\pgfpathlineto{\pgfqpoint{2.696993in}{0.907011in}}%
\pgfpathlineto{\pgfqpoint{2.665630in}{0.885343in}}%
\pgfpathlineto{\pgfqpoint{2.632570in}{0.864624in}}%
\pgfpathlineto{\pgfqpoint{2.597890in}{0.844910in}}%
\pgfpathlineto{\pgfqpoint{2.561667in}{0.826256in}}%
\pgfpathlineto{\pgfqpoint{2.523984in}{0.808714in}}%
\pgfpathlineto{\pgfqpoint{2.484928in}{0.792335in}}%
\pgfpathlineto{\pgfqpoint{2.444591in}{0.777168in}}%
\pgfpathlineto{\pgfqpoint{2.403066in}{0.763258in}}%
\pgfpathlineto{\pgfqpoint{2.360454in}{0.750647in}}%
\pgfpathlineto{\pgfqpoint{2.316854in}{0.739375in}}%
\pgfpathlineto{\pgfqpoint{2.272372in}{0.729478in}}%
\pgfpathlineto{\pgfqpoint{2.227116in}{0.720987in}}%
\pgfpathlineto{\pgfqpoint{2.181195in}{0.713933in}}%
\pgfpathlineto{\pgfqpoint{2.134722in}{0.708339in}}%
\pgfpathlineto{\pgfqpoint{2.087812in}{0.704225in}}%
\pgfpathlineto{\pgfqpoint{2.040579in}{0.701608in}}%
\pgfpathlineto{\pgfqpoint{1.993142in}{0.700499in}}%
\pgfpathlineto{\pgfqpoint{1.945618in}{0.700905in}}%
\pgfpathlineto{\pgfqpoint{1.898126in}{0.702830in}}%
\pgfpathlineto{\pgfqpoint{1.850784in}{0.706271in}}%
\pgfpathlineto{\pgfqpoint{1.803712in}{0.711220in}}%
\pgfpathlineto{\pgfqpoint{1.757027in}{0.717668in}}%
\pgfpathlineto{\pgfqpoint{1.710847in}{0.725599in}}%
\pgfpathlineto{\pgfqpoint{1.665287in}{0.734992in}}%
\pgfpathlineto{\pgfqpoint{1.620462in}{0.745823in}}%
\pgfpathlineto{\pgfqpoint{1.576485in}{0.758064in}}%
\pgfpathlineto{\pgfqpoint{1.533466in}{0.771683in}}%
\pgfpathlineto{\pgfqpoint{1.491512in}{0.786643in}}%
\pgfpathlineto{\pgfqpoint{1.450729in}{0.802905in}}%
\pgfpathlineto{\pgfqpoint{1.411219in}{0.820427in}}%
\pgfpathlineto{\pgfqpoint{1.373080in}{0.839161in}}%
\pgfpathlineto{\pgfqpoint{1.336407in}{0.859061in}}%
\pgfpathlineto{\pgfqpoint{1.301291in}{0.880074in}}%
\pgfpathlineto{\pgfqpoint{1.267819in}{0.902148in}}%
\pgfpathlineto{\pgfqpoint{1.236075in}{0.925226in}}%
\pgfpathlineto{\pgfqpoint{1.206137in}{0.949251in}}%
\pgfpathlineto{\pgfqpoint{1.178079in}{0.974163in}}%
\pgfpathlineto{\pgfqpoint{1.151971in}{0.999904in}}%
\pgfpathlineto{\pgfqpoint{1.127876in}{1.026410in}}%
\pgfpathlineto{\pgfqpoint{1.105856in}{1.053618in}}%
\pgfpathlineto{\pgfqpoint{1.085965in}{1.081466in}}%
\pgfpathlineto{\pgfqpoint{1.068254in}{1.109888in}}%
\pgfpathlineto{\pgfqpoint{1.052766in}{1.138820in}}%
\pgfpathlineto{\pgfqpoint{1.039543in}{1.168195in}}%
\pgfpathlineto{\pgfqpoint{1.028618in}{1.197948in}}%
\pgfpathlineto{\pgfqpoint{1.020022in}{1.228011in}}%
\pgfpathlineto{\pgfqpoint{1.013778in}{1.258319in}}%
\pgfpathlineto{\pgfqpoint{1.009906in}{1.288804in}}%
\pgfpathlineto{\pgfqpoint{1.008419in}{1.319398in}}%
\pgfpathlineto{\pgfqpoint{1.009327in}{1.350034in}}%
\pgfpathlineto{\pgfqpoint{1.012632in}{1.380645in}}%
\pgfpathlineto{\pgfqpoint{1.018332in}{1.411162in}}%
\pgfpathlineto{\pgfqpoint{1.026420in}{1.441517in}}%
\pgfpathlineto{\pgfqpoint{1.036884in}{1.471643in}}%
\pgfpathlineto{\pgfqpoint{1.049706in}{1.501472in}}%
\pgfpathlineto{\pgfqpoint{1.064861in}{1.530935in}}%
\pgfpathlineto{\pgfqpoint{1.082323in}{1.559963in}}%
\pgfpathlineto{\pgfqpoint{1.102056in}{1.588491in}}%
\pgfpathlineto{\pgfqpoint{1.124021in}{1.616448in}}%
\pgfpathlineto{\pgfqpoint{1.148174in}{1.643769in}}%
\pgfpathlineto{\pgfqpoint{1.174465in}{1.670386in}}%
\pgfpathlineto{\pgfqpoint{1.202839in}{1.696232in}}%
\pgfpathlineto{\pgfqpoint{1.233235in}{1.721243in}}%
\pgfpathlineto{\pgfqpoint{1.265589in}{1.745352in}}%
\pgfpathlineto{\pgfqpoint{1.299829in}{1.768497in}}%
\pgfpathlineto{\pgfqpoint{1.335881in}{1.790614in}}%
\pgfpathlineto{\pgfqpoint{1.373663in}{1.811643in}}%
\pgfpathlineto{\pgfqpoint{1.413091in}{1.831525in}}%
\pgfpathlineto{\pgfqpoint{1.454075in}{1.850203in}}%
\pgfpathlineto{\pgfqpoint{1.496520in}{1.867621in}}%
\pgfpathlineto{\pgfqpoint{1.540329in}{1.883728in}}%
\pgfpathlineto{\pgfqpoint{1.585398in}{1.898475in}}%
\pgfpathlineto{\pgfqpoint{1.631620in}{1.911815in}}%
\pgfpathlineto{\pgfqpoint{1.678887in}{1.923705in}}%
\pgfpathlineto{\pgfqpoint{1.727084in}{1.934108in}}%
\pgfpathlineto{\pgfqpoint{1.776096in}{1.942987in}}%
\pgfpathlineto{\pgfqpoint{1.825802in}{1.950312in}}%
\pgfpathlineto{\pgfqpoint{1.876081in}{1.956058in}}%
\pgfpathlineto{\pgfqpoint{1.926810in}{1.960201in}}%
\pgfpathlineto{\pgfqpoint{1.977864in}{1.962726in}}%
\pgfpathlineto{\pgfqpoint{2.029114in}{1.963621in}}%
\pgfpathlineto{\pgfqpoint{2.080435in}{1.962878in}}%
\pgfpathlineto{\pgfqpoint{2.131698in}{1.960497in}}%
\pgfpathlineto{\pgfqpoint{2.182774in}{1.956480in}}%
\pgfpathlineto{\pgfqpoint{2.233536in}{1.950835in}}%
\pgfpathlineto{\pgfqpoint{2.283857in}{1.943576in}}%
\pgfpathlineto{\pgfqpoint{2.333609in}{1.934722in}}%
\pgfpathlineto{\pgfqpoint{2.382668in}{1.924296in}}%
\pgfpathlineto{\pgfqpoint{2.430912in}{1.912326in}}%
\pgfpathlineto{\pgfqpoint{2.478219in}{1.898843in}}%
\pgfpathlineto{\pgfqpoint{2.524470in}{1.883884in}}%
\pgfpathlineto{\pgfqpoint{2.569551in}{1.867491in}}%
\pgfpathlineto{\pgfqpoint{2.613350in}{1.849706in}}%
\pgfpathlineto{\pgfqpoint{2.655756in}{1.830579in}}%
\pgfpathlineto{\pgfqpoint{2.696666in}{1.810159in}}%
\pgfpathlineto{\pgfqpoint{2.735978in}{1.788501in}}%
\pgfpathlineto{\pgfqpoint{2.773595in}{1.765661in}}%
\pgfpathlineto{\pgfqpoint{2.809424in}{1.741698in}}%
\pgfpathlineto{\pgfqpoint{2.843379in}{1.716672in}}%
\pgfpathlineto{\pgfqpoint{2.875375in}{1.690647in}}%
\pgfpathlineto{\pgfqpoint{2.905334in}{1.663685in}}%
\pgfpathlineto{\pgfqpoint{2.933184in}{1.635853in}}%
\pgfpathlineto{\pgfqpoint{2.958855in}{1.607218in}}%
\pgfpathlineto{\pgfqpoint{2.982286in}{1.577846in}}%
\pgfpathlineto{\pgfqpoint{3.003418in}{1.547806in}}%
\pgfpathlineto{\pgfqpoint{3.022200in}{1.517167in}}%
\pgfpathlineto{\pgfqpoint{3.038584in}{1.485998in}}%
\pgfpathlineto{\pgfqpoint{3.052528in}{1.454369in}}%
\pgfpathlineto{\pgfqpoint{3.063998in}{1.422350in}}%
\pgfpathlineto{\pgfqpoint{3.072962in}{1.390012in}}%
\pgfpathlineto{\pgfqpoint{3.079396in}{1.357424in}}%
\pgfpathlineto{\pgfqpoint{3.083279in}{1.324658in}}%
\pgfpathlineto{\pgfqpoint{3.084598in}{1.291784in}}%
\pgfpathlineto{\pgfqpoint{3.083345in}{1.258874in}}%
\pgfpathlineto{\pgfqpoint{3.079516in}{1.225998in}}%
\pgfpathlineto{\pgfqpoint{3.073115in}{1.193229in}}%
\pgfpathlineto{\pgfqpoint{3.064149in}{1.160637in}}%
\pgfpathlineto{\pgfqpoint{3.052633in}{1.128294in}}%
\pgfpathlineto{\pgfqpoint{3.038585in}{1.096272in}}%
\pgfpathlineto{\pgfqpoint{3.022032in}{1.064643in}}%
\pgfpathlineto{\pgfqpoint{3.003004in}{1.033479in}}%
\pgfpathlineto{\pgfqpoint{2.981538in}{1.002852in}}%
\pgfpathlineto{\pgfqpoint{2.957674in}{0.972833in}}%
\pgfpathlineto{\pgfqpoint{2.931461in}{0.943495in}}%
\pgfpathlineto{\pgfqpoint{2.902953in}{0.914909in}}%
\pgfpathlineto{\pgfqpoint{2.872207in}{0.887146in}}%
\pgfpathlineto{\pgfqpoint{2.839289in}{0.860277in}}%
\pgfpathlineto{\pgfqpoint{2.804269in}{0.834372in}}%
\pgfpathlineto{\pgfqpoint{2.767221in}{0.809498in}}%
\pgfpathlineto{\pgfqpoint{2.728226in}{0.785725in}}%
\pgfpathlineto{\pgfqpoint{2.687372in}{0.763117in}}%
\pgfpathlineto{\pgfqpoint{2.644748in}{0.741740in}}%
\pgfpathlineto{\pgfqpoint{2.600451in}{0.721655in}}%
\pgfpathlineto{\pgfqpoint{2.554583in}{0.702922in}}%
\pgfpathlineto{\pgfqpoint{2.507248in}{0.685598in}}%
\pgfpathlineto{\pgfqpoint{2.458558in}{0.669738in}}%
\pgfpathlineto{\pgfqpoint{2.408627in}{0.655392in}}%
\pgfpathlineto{\pgfqpoint{2.357573in}{0.642607in}}%
\pgfpathlineto{\pgfqpoint{2.305519in}{0.631426in}}%
\pgfpathlineto{\pgfqpoint{2.252590in}{0.621886in}}%
\pgfpathlineto{\pgfqpoint{2.198914in}{0.614023in}}%
\pgfpathlineto{\pgfqpoint{2.144623in}{0.607865in}}%
\pgfpathlineto{\pgfqpoint{2.089850in}{0.603436in}}%
\pgfpathlineto{\pgfqpoint{2.034730in}{0.600754in}}%
\pgfpathlineto{\pgfqpoint{1.979401in}{0.599832in}}%
\pgfpathlineto{\pgfqpoint{1.923999in}{0.600677in}}%
\pgfpathlineto{\pgfqpoint{1.868663in}{0.603292in}}%
\pgfpathlineto{\pgfqpoint{1.813533in}{0.607672in}}%
\pgfpathlineto{\pgfqpoint{1.758745in}{0.613808in}}%
\pgfpathlineto{\pgfqpoint{1.704437in}{0.621684in}}%
\pgfpathlineto{\pgfqpoint{1.650745in}{0.631279in}}%
\pgfpathlineto{\pgfqpoint{1.597804in}{0.642569in}}%
\pgfpathlineto{\pgfqpoint{1.545747in}{0.655520in}}%
\pgfpathlineto{\pgfqpoint{1.494703in}{0.670098in}}%
\pgfpathlineto{\pgfqpoint{1.444800in}{0.686261in}}%
\pgfpathlineto{\pgfqpoint{1.396162in}{0.703965in}}%
\pgfpathlineto{\pgfqpoint{1.348909in}{0.723161in}}%
\pgfpathlineto{\pgfqpoint{1.303159in}{0.743795in}}%
\pgfpathlineto{\pgfqpoint{1.259023in}{0.765812in}}%
\pgfpathlineto{\pgfqpoint{1.216610in}{0.789153in}}%
\pgfpathlineto{\pgfqpoint{1.176023in}{0.813756in}}%
\pgfpathlineto{\pgfqpoint{1.137362in}{0.839557in}}%
\pgfpathlineto{\pgfqpoint{1.100720in}{0.866489in}}%
\pgfpathlineto{\pgfqpoint{1.066186in}{0.894485in}}%
\pgfpathlineto{\pgfqpoint{1.033842in}{0.923476in}}%
\pgfpathlineto{\pgfqpoint{1.003767in}{0.953391in}}%
\pgfpathlineto{\pgfqpoint{0.976033in}{0.984159in}}%
\pgfpathlineto{\pgfqpoint{0.950706in}{1.015707in}}%
\pgfpathlineto{\pgfqpoint{0.927849in}{1.047964in}}%
\pgfpathlineto{\pgfqpoint{0.907516in}{1.080854in}}%
\pgfpathlineto{\pgfqpoint{0.889756in}{1.114306in}}%
\pgfpathlineto{\pgfqpoint{0.874614in}{1.148245in}}%
\pgfpathlineto{\pgfqpoint{0.862128in}{1.182598in}}%
\pgfpathlineto{\pgfqpoint{0.852331in}{1.217291in}}%
\pgfpathlineto{\pgfqpoint{0.845248in}{1.252249in}}%
\pgfpathlineto{\pgfqpoint{0.840901in}{1.287399in}}%
\pgfpathlineto{\pgfqpoint{0.839304in}{1.322666in}}%
\pgfpathlineto{\pgfqpoint{0.840468in}{1.357976in}}%
\pgfpathlineto{\pgfqpoint{0.844394in}{1.393255in}}%
\pgfpathlineto{\pgfqpoint{0.851081in}{1.428427in}}%
\pgfpathlineto{\pgfqpoint{0.860521in}{1.463419in}}%
\pgfpathlineto{\pgfqpoint{0.872699in}{1.498155in}}%
\pgfpathlineto{\pgfqpoint{0.887595in}{1.532561in}}%
\pgfpathlineto{\pgfqpoint{0.905183in}{1.566559in}}%
\pgfpathlineto{\pgfqpoint{0.925432in}{1.600075in}}%
\pgfpathlineto{\pgfqpoint{0.948304in}{1.633032in}}%
\pgfpathlineto{\pgfqpoint{0.973755in}{1.665355in}}%
\pgfpathlineto{\pgfqpoint{1.001735in}{1.696968in}}%
\pgfpathlineto{\pgfqpoint{1.032189in}{1.727793in}}%
\pgfpathlineto{\pgfqpoint{1.065055in}{1.757755in}}%
\pgfpathlineto{\pgfqpoint{1.100266in}{1.786778in}}%
\pgfpathlineto{\pgfqpoint{1.137748in}{1.814787in}}%
\pgfpathlineto{\pgfqpoint{1.177422in}{1.841709in}}%
\pgfpathlineto{\pgfqpoint{1.219203in}{1.867468in}}%
\pgfpathlineto{\pgfqpoint{1.263000in}{1.891994in}}%
\pgfpathlineto{\pgfqpoint{1.308715in}{1.915216in}}%
\pgfpathlineto{\pgfqpoint{1.356248in}{1.937066in}}%
\pgfpathlineto{\pgfqpoint{1.405491in}{1.957478in}}%
\pgfpathlineto{\pgfqpoint{1.456331in}{1.976388in}}%
\pgfpathlineto{\pgfqpoint{1.508651in}{1.993737in}}%
\pgfpathlineto{\pgfqpoint{1.562329in}{2.009468in}}%
\pgfpathlineto{\pgfqpoint{1.617237in}{2.023528in}}%
\pgfpathlineto{\pgfqpoint{1.673246in}{2.035871in}}%
\pgfpathlineto{\pgfqpoint{1.730221in}{2.046451in}}%
\pgfpathlineto{\pgfqpoint{1.788023in}{2.055231in}}%
\pgfpathlineto{\pgfqpoint{1.846513in}{2.062178in}}%
\pgfpathlineto{\pgfqpoint{1.905547in}{2.067264in}}%
\pgfpathlineto{\pgfqpoint{1.964978in}{2.070467in}}%
\pgfpathlineto{\pgfqpoint{2.024660in}{2.071774in}}%
\pgfpathlineto{\pgfqpoint{2.084444in}{2.071173in}}%
\pgfpathlineto{\pgfqpoint{2.144181in}{2.068664in}}%
\pgfpathlineto{\pgfqpoint{2.203722in}{2.064248in}}%
\pgfpathlineto{\pgfqpoint{2.262918in}{2.057937in}}%
\pgfpathlineto{\pgfqpoint{2.321619in}{2.049747in}}%
\pgfpathlineto{\pgfqpoint{2.379680in}{2.039699in}}%
\pgfpathlineto{\pgfqpoint{2.436954in}{2.027822in}}%
\pgfpathlineto{\pgfqpoint{2.493299in}{2.014150in}}%
\pgfpathlineto{\pgfqpoint{2.548574in}{1.998723in}}%
\pgfpathlineto{\pgfqpoint{2.602642in}{1.981585in}}%
\pgfpathlineto{\pgfqpoint{2.655368in}{1.962785in}}%
\pgfpathlineto{\pgfqpoint{2.706624in}{1.942377in}}%
\pgfpathlineto{\pgfqpoint{2.756282in}{1.920419in}}%
\pgfpathlineto{\pgfqpoint{2.804222in}{1.896972in}}%
\pgfpathlineto{\pgfqpoint{2.850328in}{1.872100in}}%
\pgfpathlineto{\pgfqpoint{2.894487in}{1.845870in}}%
\pgfpathlineto{\pgfqpoint{2.936593in}{1.818353in}}%
\pgfpathlineto{\pgfqpoint{2.936593in}{1.818353in}}%
\pgfusepath{stroke}%
\end{pgfscope}%
\begin{pgfscope}%
\pgfsetrectcap%
\pgfsetmiterjoin%
\pgfsetlinewidth{0.803000pt}%
\definecolor{currentstroke}{rgb}{0.000000,0.000000,0.000000}%
\pgfsetstrokecolor{currentstroke}%
\pgfsetdash{}{0pt}%
\pgfpathmoveto{\pgfqpoint{0.727040in}{0.526234in}}%
\pgfpathlineto{\pgfqpoint{0.727040in}{2.145371in}}%
\pgfusepath{stroke}%
\end{pgfscope}%
\begin{pgfscope}%
\pgfsetrectcap%
\pgfsetmiterjoin%
\pgfsetlinewidth{0.803000pt}%
\definecolor{currentstroke}{rgb}{0.000000,0.000000,0.000000}%
\pgfsetstrokecolor{currentstroke}%
\pgfsetdash{}{0pt}%
\pgfpathmoveto{\pgfqpoint{3.196863in}{0.526234in}}%
\pgfpathlineto{\pgfqpoint{3.196863in}{2.145371in}}%
\pgfusepath{stroke}%
\end{pgfscope}%
\begin{pgfscope}%
\pgfsetrectcap%
\pgfsetmiterjoin%
\pgfsetlinewidth{0.803000pt}%
\definecolor{currentstroke}{rgb}{0.000000,0.000000,0.000000}%
\pgfsetstrokecolor{currentstroke}%
\pgfsetdash{}{0pt}%
\pgfpathmoveto{\pgfqpoint{0.727040in}{0.526234in}}%
\pgfpathlineto{\pgfqpoint{3.196863in}{0.526234in}}%
\pgfusepath{stroke}%
\end{pgfscope}%
\begin{pgfscope}%
\pgfsetrectcap%
\pgfsetmiterjoin%
\pgfsetlinewidth{0.803000pt}%
\definecolor{currentstroke}{rgb}{0.000000,0.000000,0.000000}%
\pgfsetstrokecolor{currentstroke}%
\pgfsetdash{}{0pt}%
\pgfpathmoveto{\pgfqpoint{0.727040in}{2.145371in}}%
\pgfpathlineto{\pgfqpoint{3.196863in}{2.145371in}}%
\pgfusepath{stroke}%
\end{pgfscope}%
\begin{pgfscope}%
\definecolor{textcolor}{rgb}{0.000000,0.000000,0.000000}%
\pgfsetstrokecolor{textcolor}%
\pgfsetfillcolor{textcolor}%
\pgftext[x=1.961951in,y=2.228704in,,base]{\color{textcolor}\rmfamily\fontsize{12.000000}{14.400000}\selectfont phase plot}%
\end{pgfscope}%
\begin{pgfscope}%
\pgfsetbuttcap%
\pgfsetmiterjoin%
\definecolor{currentfill}{rgb}{1.000000,1.000000,1.000000}%
\pgfsetfillcolor{currentfill}%
\pgfsetlinewidth{0.000000pt}%
\definecolor{currentstroke}{rgb}{0.000000,0.000000,0.000000}%
\pgfsetstrokecolor{currentstroke}%
\pgfsetstrokeopacity{0.000000}%
\pgfsetdash{}{0pt}%
\pgfpathmoveto{\pgfqpoint{3.798088in}{0.526234in}}%
\pgfpathlineto{\pgfqpoint{6.267911in}{0.526234in}}%
\pgfpathlineto{\pgfqpoint{6.267911in}{2.145371in}}%
\pgfpathlineto{\pgfqpoint{3.798088in}{2.145371in}}%
\pgfpathclose%
\pgfusepath{fill}%
\end{pgfscope}%
\begin{pgfscope}%
\pgfsetbuttcap%
\pgfsetroundjoin%
\definecolor{currentfill}{rgb}{0.000000,0.000000,0.000000}%
\pgfsetfillcolor{currentfill}%
\pgfsetlinewidth{0.803000pt}%
\definecolor{currentstroke}{rgb}{0.000000,0.000000,0.000000}%
\pgfsetstrokecolor{currentstroke}%
\pgfsetdash{}{0pt}%
\pgfsys@defobject{currentmarker}{\pgfqpoint{0.000000in}{-0.048611in}}{\pgfqpoint{0.000000in}{0.000000in}}{%
\pgfpathmoveto{\pgfqpoint{0.000000in}{0.000000in}}%
\pgfpathlineto{\pgfqpoint{0.000000in}{-0.048611in}}%
\pgfusepath{stroke,fill}%
}%
\begin{pgfscope}%
\pgfsys@transformshift{3.910353in}{0.526234in}%
\pgfsys@useobject{currentmarker}{}%
\end{pgfscope}%
\end{pgfscope}%
\begin{pgfscope}%
\definecolor{textcolor}{rgb}{0.000000,0.000000,0.000000}%
\pgfsetstrokecolor{textcolor}%
\pgfsetfillcolor{textcolor}%
\pgftext[x=3.910353in,y=0.429012in,,top]{\color{textcolor}\rmfamily\fontsize{10.000000}{12.000000}\selectfont \(\displaystyle 0.0\)}%
\end{pgfscope}%
\begin{pgfscope}%
\pgfsetbuttcap%
\pgfsetroundjoin%
\definecolor{currentfill}{rgb}{0.000000,0.000000,0.000000}%
\pgfsetfillcolor{currentfill}%
\pgfsetlinewidth{0.803000pt}%
\definecolor{currentstroke}{rgb}{0.000000,0.000000,0.000000}%
\pgfsetstrokecolor{currentstroke}%
\pgfsetdash{}{0pt}%
\pgfsys@defobject{currentmarker}{\pgfqpoint{0.000000in}{-0.048611in}}{\pgfqpoint{0.000000in}{0.000000in}}{%
\pgfpathmoveto{\pgfqpoint{0.000000in}{0.000000in}}%
\pgfpathlineto{\pgfqpoint{0.000000in}{-0.048611in}}%
\pgfusepath{stroke,fill}%
}%
\begin{pgfscope}%
\pgfsys@transformshift{4.471676in}{0.526234in}%
\pgfsys@useobject{currentmarker}{}%
\end{pgfscope}%
\end{pgfscope}%
\begin{pgfscope}%
\definecolor{textcolor}{rgb}{0.000000,0.000000,0.000000}%
\pgfsetstrokecolor{textcolor}%
\pgfsetfillcolor{textcolor}%
\pgftext[x=4.471676in,y=0.429012in,,top]{\color{textcolor}\rmfamily\fontsize{10.000000}{12.000000}\selectfont \(\displaystyle 2.5\)}%
\end{pgfscope}%
\begin{pgfscope}%
\pgfsetbuttcap%
\pgfsetroundjoin%
\definecolor{currentfill}{rgb}{0.000000,0.000000,0.000000}%
\pgfsetfillcolor{currentfill}%
\pgfsetlinewidth{0.803000pt}%
\definecolor{currentstroke}{rgb}{0.000000,0.000000,0.000000}%
\pgfsetstrokecolor{currentstroke}%
\pgfsetdash{}{0pt}%
\pgfsys@defobject{currentmarker}{\pgfqpoint{0.000000in}{-0.048611in}}{\pgfqpoint{0.000000in}{0.000000in}}{%
\pgfpathmoveto{\pgfqpoint{0.000000in}{0.000000in}}%
\pgfpathlineto{\pgfqpoint{0.000000in}{-0.048611in}}%
\pgfusepath{stroke,fill}%
}%
\begin{pgfscope}%
\pgfsys@transformshift{5.033000in}{0.526234in}%
\pgfsys@useobject{currentmarker}{}%
\end{pgfscope}%
\end{pgfscope}%
\begin{pgfscope}%
\definecolor{textcolor}{rgb}{0.000000,0.000000,0.000000}%
\pgfsetstrokecolor{textcolor}%
\pgfsetfillcolor{textcolor}%
\pgftext[x=5.033000in,y=0.429012in,,top]{\color{textcolor}\rmfamily\fontsize{10.000000}{12.000000}\selectfont \(\displaystyle 5.0\)}%
\end{pgfscope}%
\begin{pgfscope}%
\pgfsetbuttcap%
\pgfsetroundjoin%
\definecolor{currentfill}{rgb}{0.000000,0.000000,0.000000}%
\pgfsetfillcolor{currentfill}%
\pgfsetlinewidth{0.803000pt}%
\definecolor{currentstroke}{rgb}{0.000000,0.000000,0.000000}%
\pgfsetstrokecolor{currentstroke}%
\pgfsetdash{}{0pt}%
\pgfsys@defobject{currentmarker}{\pgfqpoint{0.000000in}{-0.048611in}}{\pgfqpoint{0.000000in}{0.000000in}}{%
\pgfpathmoveto{\pgfqpoint{0.000000in}{0.000000in}}%
\pgfpathlineto{\pgfqpoint{0.000000in}{-0.048611in}}%
\pgfusepath{stroke,fill}%
}%
\begin{pgfscope}%
\pgfsys@transformshift{5.594323in}{0.526234in}%
\pgfsys@useobject{currentmarker}{}%
\end{pgfscope}%
\end{pgfscope}%
\begin{pgfscope}%
\definecolor{textcolor}{rgb}{0.000000,0.000000,0.000000}%
\pgfsetstrokecolor{textcolor}%
\pgfsetfillcolor{textcolor}%
\pgftext[x=5.594323in,y=0.429012in,,top]{\color{textcolor}\rmfamily\fontsize{10.000000}{12.000000}\selectfont \(\displaystyle 7.5\)}%
\end{pgfscope}%
\begin{pgfscope}%
\pgfsetbuttcap%
\pgfsetroundjoin%
\definecolor{currentfill}{rgb}{0.000000,0.000000,0.000000}%
\pgfsetfillcolor{currentfill}%
\pgfsetlinewidth{0.803000pt}%
\definecolor{currentstroke}{rgb}{0.000000,0.000000,0.000000}%
\pgfsetstrokecolor{currentstroke}%
\pgfsetdash{}{0pt}%
\pgfsys@defobject{currentmarker}{\pgfqpoint{0.000000in}{-0.048611in}}{\pgfqpoint{0.000000in}{0.000000in}}{%
\pgfpathmoveto{\pgfqpoint{0.000000in}{0.000000in}}%
\pgfpathlineto{\pgfqpoint{0.000000in}{-0.048611in}}%
\pgfusepath{stroke,fill}%
}%
\begin{pgfscope}%
\pgfsys@transformshift{6.155646in}{0.526234in}%
\pgfsys@useobject{currentmarker}{}%
\end{pgfscope}%
\end{pgfscope}%
\begin{pgfscope}%
\definecolor{textcolor}{rgb}{0.000000,0.000000,0.000000}%
\pgfsetstrokecolor{textcolor}%
\pgfsetfillcolor{textcolor}%
\pgftext[x=6.155646in,y=0.429012in,,top]{\color{textcolor}\rmfamily\fontsize{10.000000}{12.000000}\selectfont \(\displaystyle 10.0\)}%
\end{pgfscope}%
\begin{pgfscope}%
\definecolor{textcolor}{rgb}{0.000000,0.000000,0.000000}%
\pgfsetstrokecolor{textcolor}%
\pgfsetfillcolor{textcolor}%
\pgftext[x=5.033000in,y=0.250000in,,top]{\color{textcolor}\rmfamily\fontsize{10.000000}{12.000000}\selectfont time (s)}%
\end{pgfscope}%
\begin{pgfscope}%
\pgfsetbuttcap%
\pgfsetroundjoin%
\definecolor{currentfill}{rgb}{0.000000,0.000000,0.000000}%
\pgfsetfillcolor{currentfill}%
\pgfsetlinewidth{0.803000pt}%
\definecolor{currentstroke}{rgb}{0.000000,0.000000,0.000000}%
\pgfsetstrokecolor{currentstroke}%
\pgfsetdash{}{0pt}%
\pgfsys@defobject{currentmarker}{\pgfqpoint{-0.048611in}{0.000000in}}{\pgfqpoint{0.000000in}{0.000000in}}{%
\pgfpathmoveto{\pgfqpoint{0.000000in}{0.000000in}}%
\pgfpathlineto{\pgfqpoint{-0.048611in}{0.000000in}}%
\pgfusepath{stroke,fill}%
}%
\begin{pgfscope}%
\pgfsys@transformshift{3.798088in}{0.545835in}%
\pgfsys@useobject{currentmarker}{}%
\end{pgfscope}%
\end{pgfscope}%
\begin{pgfscope}%
\definecolor{textcolor}{rgb}{0.000000,0.000000,0.000000}%
\pgfsetstrokecolor{textcolor}%
\pgfsetfillcolor{textcolor}%
\pgftext[x=3.631421in,y=0.497609in,left,base]{\color{textcolor}\rmfamily\fontsize{10.000000}{12.000000}\selectfont \(\displaystyle 0\)}%
\end{pgfscope}%
\begin{pgfscope}%
\pgfsetbuttcap%
\pgfsetroundjoin%
\definecolor{currentfill}{rgb}{0.000000,0.000000,0.000000}%
\pgfsetfillcolor{currentfill}%
\pgfsetlinewidth{0.803000pt}%
\definecolor{currentstroke}{rgb}{0.000000,0.000000,0.000000}%
\pgfsetstrokecolor{currentstroke}%
\pgfsetdash{}{0pt}%
\pgfsys@defobject{currentmarker}{\pgfqpoint{-0.048611in}{0.000000in}}{\pgfqpoint{0.000000in}{0.000000in}}{%
\pgfpathmoveto{\pgfqpoint{0.000000in}{0.000000in}}%
\pgfpathlineto{\pgfqpoint{-0.048611in}{0.000000in}}%
\pgfusepath{stroke,fill}%
}%
\begin{pgfscope}%
\pgfsys@transformshift{3.798088in}{0.908513in}%
\pgfsys@useobject{currentmarker}{}%
\end{pgfscope}%
\end{pgfscope}%
\begin{pgfscope}%
\definecolor{textcolor}{rgb}{0.000000,0.000000,0.000000}%
\pgfsetstrokecolor{textcolor}%
\pgfsetfillcolor{textcolor}%
\pgftext[x=3.631421in,y=0.860288in,left,base]{\color{textcolor}\rmfamily\fontsize{10.000000}{12.000000}\selectfont \(\displaystyle 1\)}%
\end{pgfscope}%
\begin{pgfscope}%
\pgfsetbuttcap%
\pgfsetroundjoin%
\definecolor{currentfill}{rgb}{0.000000,0.000000,0.000000}%
\pgfsetfillcolor{currentfill}%
\pgfsetlinewidth{0.803000pt}%
\definecolor{currentstroke}{rgb}{0.000000,0.000000,0.000000}%
\pgfsetstrokecolor{currentstroke}%
\pgfsetdash{}{0pt}%
\pgfsys@defobject{currentmarker}{\pgfqpoint{-0.048611in}{0.000000in}}{\pgfqpoint{0.000000in}{0.000000in}}{%
\pgfpathmoveto{\pgfqpoint{0.000000in}{0.000000in}}%
\pgfpathlineto{\pgfqpoint{-0.048611in}{0.000000in}}%
\pgfusepath{stroke,fill}%
}%
\begin{pgfscope}%
\pgfsys@transformshift{3.798088in}{1.271192in}%
\pgfsys@useobject{currentmarker}{}%
\end{pgfscope}%
\end{pgfscope}%
\begin{pgfscope}%
\definecolor{textcolor}{rgb}{0.000000,0.000000,0.000000}%
\pgfsetstrokecolor{textcolor}%
\pgfsetfillcolor{textcolor}%
\pgftext[x=3.631421in,y=1.222966in,left,base]{\color{textcolor}\rmfamily\fontsize{10.000000}{12.000000}\selectfont \(\displaystyle 2\)}%
\end{pgfscope}%
\begin{pgfscope}%
\pgfsetbuttcap%
\pgfsetroundjoin%
\definecolor{currentfill}{rgb}{0.000000,0.000000,0.000000}%
\pgfsetfillcolor{currentfill}%
\pgfsetlinewidth{0.803000pt}%
\definecolor{currentstroke}{rgb}{0.000000,0.000000,0.000000}%
\pgfsetstrokecolor{currentstroke}%
\pgfsetdash{}{0pt}%
\pgfsys@defobject{currentmarker}{\pgfqpoint{-0.048611in}{0.000000in}}{\pgfqpoint{0.000000in}{0.000000in}}{%
\pgfpathmoveto{\pgfqpoint{0.000000in}{0.000000in}}%
\pgfpathlineto{\pgfqpoint{-0.048611in}{0.000000in}}%
\pgfusepath{stroke,fill}%
}%
\begin{pgfscope}%
\pgfsys@transformshift{3.798088in}{1.633870in}%
\pgfsys@useobject{currentmarker}{}%
\end{pgfscope}%
\end{pgfscope}%
\begin{pgfscope}%
\definecolor{textcolor}{rgb}{0.000000,0.000000,0.000000}%
\pgfsetstrokecolor{textcolor}%
\pgfsetfillcolor{textcolor}%
\pgftext[x=3.631421in,y=1.585645in,left,base]{\color{textcolor}\rmfamily\fontsize{10.000000}{12.000000}\selectfont \(\displaystyle 3\)}%
\end{pgfscope}%
\begin{pgfscope}%
\pgfsetbuttcap%
\pgfsetroundjoin%
\definecolor{currentfill}{rgb}{0.000000,0.000000,0.000000}%
\pgfsetfillcolor{currentfill}%
\pgfsetlinewidth{0.803000pt}%
\definecolor{currentstroke}{rgb}{0.000000,0.000000,0.000000}%
\pgfsetstrokecolor{currentstroke}%
\pgfsetdash{}{0pt}%
\pgfsys@defobject{currentmarker}{\pgfqpoint{-0.048611in}{0.000000in}}{\pgfqpoint{0.000000in}{0.000000in}}{%
\pgfpathmoveto{\pgfqpoint{0.000000in}{0.000000in}}%
\pgfpathlineto{\pgfqpoint{-0.048611in}{0.000000in}}%
\pgfusepath{stroke,fill}%
}%
\begin{pgfscope}%
\pgfsys@transformshift{3.798088in}{1.996549in}%
\pgfsys@useobject{currentmarker}{}%
\end{pgfscope}%
\end{pgfscope}%
\begin{pgfscope}%
\definecolor{textcolor}{rgb}{0.000000,0.000000,0.000000}%
\pgfsetstrokecolor{textcolor}%
\pgfsetfillcolor{textcolor}%
\pgftext[x=3.631421in,y=1.948323in,left,base]{\color{textcolor}\rmfamily\fontsize{10.000000}{12.000000}\selectfont \(\displaystyle 4\)}%
\end{pgfscope}%
\begin{pgfscope}%
\definecolor{textcolor}{rgb}{0.000000,0.000000,0.000000}%
\pgfsetstrokecolor{textcolor}%
\pgfsetfillcolor{textcolor}%
\pgftext[x=3.575865in,y=1.335803in,,bottom,rotate=90.000000]{\color{textcolor}\rmfamily\fontsize{10.000000}{12.000000}\selectfont energy (J)}%
\end{pgfscope}%
\begin{pgfscope}%
\pgfpathrectangle{\pgfqpoint{3.798088in}{0.526234in}}{\pgfqpoint{2.469823in}{1.619136in}}%
\pgfusepath{clip}%
\pgfsetrectcap%
\pgfsetroundjoin%
\pgfsetlinewidth{1.505625pt}%
\definecolor{currentstroke}{rgb}{0.000000,0.000000,1.000000}%
\pgfsetstrokecolor{currentstroke}%
\pgfsetdash{}{0pt}%
\pgfpathmoveto{\pgfqpoint{3.910353in}{0.599832in}}%
\pgfpathlineto{\pgfqpoint{3.917089in}{0.600931in}}%
\pgfpathlineto{\pgfqpoint{3.923824in}{0.605648in}}%
\pgfpathlineto{\pgfqpoint{3.930560in}{0.613757in}}%
\pgfpathlineto{\pgfqpoint{3.939541in}{0.628762in}}%
\pgfpathlineto{\pgfqpoint{3.968730in}{0.683319in}}%
\pgfpathlineto{\pgfqpoint{3.975466in}{0.690777in}}%
\pgfpathlineto{\pgfqpoint{3.982202in}{0.694553in}}%
\pgfpathlineto{\pgfqpoint{3.986693in}{0.694848in}}%
\pgfpathlineto{\pgfqpoint{3.991183in}{0.693345in}}%
\pgfpathlineto{\pgfqpoint{3.997919in}{0.687876in}}%
\pgfpathlineto{\pgfqpoint{4.004655in}{0.679027in}}%
\pgfpathlineto{\pgfqpoint{4.013636in}{0.663422in}}%
\pgfpathlineto{\pgfqpoint{4.036089in}{0.622212in}}%
\pgfpathlineto{\pgfqpoint{4.042825in}{0.614096in}}%
\pgfpathlineto{\pgfqpoint{4.049561in}{0.609642in}}%
\pgfpathlineto{\pgfqpoint{4.054051in}{0.608979in}}%
\pgfpathlineto{\pgfqpoint{4.058542in}{0.610250in}}%
\pgfpathlineto{\pgfqpoint{4.063033in}{0.613445in}}%
\pgfpathlineto{\pgfqpoint{4.069769in}{0.621640in}}%
\pgfpathlineto{\pgfqpoint{4.078750in}{0.637912in}}%
\pgfpathlineto{\pgfqpoint{4.092221in}{0.668894in}}%
\pgfpathlineto{\pgfqpoint{4.105693in}{0.698664in}}%
\pgfpathlineto{\pgfqpoint{4.114674in}{0.712806in}}%
\pgfpathlineto{\pgfqpoint{4.121410in}{0.718754in}}%
\pgfpathlineto{\pgfqpoint{4.125901in}{0.720178in}}%
\pgfpathlineto{\pgfqpoint{4.130391in}{0.719494in}}%
\pgfpathlineto{\pgfqpoint{4.134882in}{0.716732in}}%
\pgfpathlineto{\pgfqpoint{4.141618in}{0.708967in}}%
\pgfpathlineto{\pgfqpoint{4.148354in}{0.697525in}}%
\pgfpathlineto{\pgfqpoint{4.159580in}{0.673282in}}%
\pgfpathlineto{\pgfqpoint{4.175297in}{0.639226in}}%
\pgfpathlineto{\pgfqpoint{4.182033in}{0.628479in}}%
\pgfpathlineto{\pgfqpoint{4.188769in}{0.621712in}}%
\pgfpathlineto{\pgfqpoint{4.193260in}{0.619794in}}%
\pgfpathlineto{\pgfqpoint{4.197750in}{0.620105in}}%
\pgfpathlineto{\pgfqpoint{4.202241in}{0.622678in}}%
\pgfpathlineto{\pgfqpoint{4.206731in}{0.627462in}}%
\pgfpathlineto{\pgfqpoint{4.213467in}{0.638452in}}%
\pgfpathlineto{\pgfqpoint{4.222448in}{0.658860in}}%
\pgfpathlineto{\pgfqpoint{4.251637in}{0.733528in}}%
\pgfpathlineto{\pgfqpoint{4.258373in}{0.743835in}}%
\pgfpathlineto{\pgfqpoint{4.262864in}{0.747959in}}%
\pgfpathlineto{\pgfqpoint{4.267354in}{0.749683in}}%
\pgfpathlineto{\pgfqpoint{4.271845in}{0.748942in}}%
\pgfpathlineto{\pgfqpoint{4.276336in}{0.745769in}}%
\pgfpathlineto{\pgfqpoint{4.283071in}{0.736768in}}%
\pgfpathlineto{\pgfqpoint{4.289807in}{0.723451in}}%
\pgfpathlineto{\pgfqpoint{4.301034in}{0.695160in}}%
\pgfpathlineto{\pgfqpoint{4.316751in}{0.655295in}}%
\pgfpathlineto{\pgfqpoint{4.323487in}{0.642663in}}%
\pgfpathlineto{\pgfqpoint{4.330223in}{0.634658in}}%
\pgfpathlineto{\pgfqpoint{4.334713in}{0.632341in}}%
\pgfpathlineto{\pgfqpoint{4.339204in}{0.632622in}}%
\pgfpathlineto{\pgfqpoint{4.343694in}{0.635546in}}%
\pgfpathlineto{\pgfqpoint{4.348185in}{0.641052in}}%
\pgfpathlineto{\pgfqpoint{4.354921in}{0.653778in}}%
\pgfpathlineto{\pgfqpoint{4.363902in}{0.677509in}}%
\pgfpathlineto{\pgfqpoint{4.395336in}{0.769564in}}%
\pgfpathlineto{\pgfqpoint{4.402072in}{0.779898in}}%
\pgfpathlineto{\pgfqpoint{4.406563in}{0.783440in}}%
\pgfpathlineto{\pgfqpoint{4.411053in}{0.784127in}}%
\pgfpathlineto{\pgfqpoint{4.415544in}{0.781935in}}%
\pgfpathlineto{\pgfqpoint{4.420034in}{0.776958in}}%
\pgfpathlineto{\pgfqpoint{4.426770in}{0.764754in}}%
\pgfpathlineto{\pgfqpoint{4.435751in}{0.741520in}}%
\pgfpathlineto{\pgfqpoint{4.462695in}{0.663824in}}%
\pgfpathlineto{\pgfqpoint{4.469431in}{0.652372in}}%
\pgfpathlineto{\pgfqpoint{4.473921in}{0.648077in}}%
\pgfpathlineto{\pgfqpoint{4.478412in}{0.646737in}}%
\pgfpathlineto{\pgfqpoint{4.482903in}{0.648466in}}%
\pgfpathlineto{\pgfqpoint{4.487393in}{0.653260in}}%
\pgfpathlineto{\pgfqpoint{4.491884in}{0.660999in}}%
\pgfpathlineto{\pgfqpoint{4.498620in}{0.677577in}}%
\pgfpathlineto{\pgfqpoint{4.507601in}{0.706889in}}%
\pgfpathlineto{\pgfqpoint{4.534544in}{0.801311in}}%
\pgfpathlineto{\pgfqpoint{4.541280in}{0.815784in}}%
\pgfpathlineto{\pgfqpoint{4.545771in}{0.821713in}}%
\pgfpathlineto{\pgfqpoint{4.550261in}{0.824378in}}%
\pgfpathlineto{\pgfqpoint{4.554752in}{0.823676in}}%
\pgfpathlineto{\pgfqpoint{4.559243in}{0.819642in}}%
\pgfpathlineto{\pgfqpoint{4.563733in}{0.812444in}}%
\pgfpathlineto{\pgfqpoint{4.570469in}{0.796402in}}%
\pgfpathlineto{\pgfqpoint{4.579450in}{0.767679in}}%
\pgfpathlineto{\pgfqpoint{4.601903in}{0.690311in}}%
\pgfpathlineto{\pgfqpoint{4.608639in}{0.674602in}}%
\pgfpathlineto{\pgfqpoint{4.613130in}{0.667761in}}%
\pgfpathlineto{\pgfqpoint{4.617620in}{0.664231in}}%
\pgfpathlineto{\pgfqpoint{4.622111in}{0.664221in}}%
\pgfpathlineto{\pgfqpoint{4.626601in}{0.667806in}}%
\pgfpathlineto{\pgfqpoint{4.631092in}{0.674921in}}%
\pgfpathlineto{\pgfqpoint{4.637828in}{0.691738in}}%
\pgfpathlineto{\pgfqpoint{4.646809in}{0.723569in}}%
\pgfpathlineto{\pgfqpoint{4.678243in}{0.849997in}}%
\pgfpathlineto{\pgfqpoint{4.684979in}{0.864777in}}%
\pgfpathlineto{\pgfqpoint{4.689470in}{0.870105in}}%
\pgfpathlineto{\pgfqpoint{4.693960in}{0.871547in}}%
\pgfpathlineto{\pgfqpoint{4.698451in}{0.869052in}}%
\pgfpathlineto{\pgfqpoint{4.702941in}{0.862726in}}%
\pgfpathlineto{\pgfqpoint{4.709677in}{0.846675in}}%
\pgfpathlineto{\pgfqpoint{4.716413in}{0.824131in}}%
\pgfpathlineto{\pgfqpoint{4.729885in}{0.768377in}}%
\pgfpathlineto{\pgfqpoint{4.743357in}{0.715846in}}%
\pgfpathlineto{\pgfqpoint{4.750093in}{0.697139in}}%
\pgfpathlineto{\pgfqpoint{4.754583in}{0.688854in}}%
\pgfpathlineto{\pgfqpoint{4.759074in}{0.684402in}}%
\pgfpathlineto{\pgfqpoint{4.763564in}{0.684040in}}%
\pgfpathlineto{\pgfqpoint{4.768055in}{0.687865in}}%
\pgfpathlineto{\pgfqpoint{4.772545in}{0.695818in}}%
\pgfpathlineto{\pgfqpoint{4.779281in}{0.714963in}}%
\pgfpathlineto{\pgfqpoint{4.786017in}{0.741489in}}%
\pgfpathlineto{\pgfqpoint{4.797244in}{0.796304in}}%
\pgfpathlineto{\pgfqpoint{4.812961in}{0.874080in}}%
\pgfpathlineto{\pgfqpoint{4.821942in}{0.906830in}}%
\pgfpathlineto{\pgfqpoint{4.828678in}{0.921625in}}%
\pgfpathlineto{\pgfqpoint{4.833168in}{0.926036in}}%
\pgfpathlineto{\pgfqpoint{4.835414in}{0.926524in}}%
\pgfpathlineto{\pgfqpoint{4.839904in}{0.924047in}}%
\pgfpathlineto{\pgfqpoint{4.844395in}{0.917073in}}%
\pgfpathlineto{\pgfqpoint{4.848885in}{0.905897in}}%
\pgfpathlineto{\pgfqpoint{4.855621in}{0.882312in}}%
\pgfpathlineto{\pgfqpoint{4.864603in}{0.841732in}}%
\pgfpathlineto{\pgfqpoint{4.884810in}{0.746272in}}%
\pgfpathlineto{\pgfqpoint{4.891546in}{0.723894in}}%
\pgfpathlineto{\pgfqpoint{4.896037in}{0.713781in}}%
\pgfpathlineto{\pgfqpoint{4.900527in}{0.708094in}}%
\pgfpathlineto{\pgfqpoint{4.905018in}{0.707145in}}%
\pgfpathlineto{\pgfqpoint{4.909508in}{0.711071in}}%
\pgfpathlineto{\pgfqpoint{4.913999in}{0.719821in}}%
\pgfpathlineto{\pgfqpoint{4.918490in}{0.733160in}}%
\pgfpathlineto{\pgfqpoint{4.925225in}{0.760826in}}%
\pgfpathlineto{\pgfqpoint{4.934207in}{0.808584in}}%
\pgfpathlineto{\pgfqpoint{4.958905in}{0.948031in}}%
\pgfpathlineto{\pgfqpoint{4.965641in}{0.973015in}}%
\pgfpathlineto{\pgfqpoint{4.970131in}{0.983935in}}%
\pgfpathlineto{\pgfqpoint{4.974622in}{0.989734in}}%
\pgfpathlineto{\pgfqpoint{4.976867in}{0.990634in}}%
\pgfpathlineto{\pgfqpoint{4.979112in}{0.990186in}}%
\pgfpathlineto{\pgfqpoint{4.983603in}{0.985280in}}%
\pgfpathlineto{\pgfqpoint{4.988094in}{0.975226in}}%
\pgfpathlineto{\pgfqpoint{4.994830in}{0.951468in}}%
\pgfpathlineto{\pgfqpoint{5.003811in}{0.907277in}}%
\pgfpathlineto{\pgfqpoint{5.028509in}{0.772732in}}%
\pgfpathlineto{\pgfqpoint{5.035245in}{0.748956in}}%
\pgfpathlineto{\pgfqpoint{5.039735in}{0.738998in}}%
\pgfpathlineto{\pgfqpoint{5.044226in}{0.734348in}}%
\pgfpathlineto{\pgfqpoint{5.046471in}{0.734113in}}%
\pgfpathlineto{\pgfqpoint{5.048717in}{0.735296in}}%
\pgfpathlineto{\pgfqpoint{5.053207in}{0.741921in}}%
\pgfpathlineto{\pgfqpoint{5.057698in}{0.754087in}}%
\pgfpathlineto{\pgfqpoint{5.064434in}{0.781909in}}%
\pgfpathlineto{\pgfqpoint{5.073415in}{0.833515in}}%
\pgfpathlineto{\pgfqpoint{5.104849in}{1.033536in}}%
\pgfpathlineto{\pgfqpoint{5.111585in}{1.056040in}}%
\pgfpathlineto{\pgfqpoint{5.116075in}{1.063753in}}%
\pgfpathlineto{\pgfqpoint{5.118321in}{1.065286in}}%
\pgfpathlineto{\pgfqpoint{5.120566in}{1.065245in}}%
\pgfpathlineto{\pgfqpoint{5.122811in}{1.063631in}}%
\pgfpathlineto{\pgfqpoint{5.127302in}{1.055773in}}%
\pgfpathlineto{\pgfqpoint{5.131792in}{1.042045in}}%
\pgfpathlineto{\pgfqpoint{5.138528in}{1.011772in}}%
\pgfpathlineto{\pgfqpoint{5.147510in}{0.957993in}}%
\pgfpathlineto{\pgfqpoint{5.169962in}{0.814507in}}%
\pgfpathlineto{\pgfqpoint{5.176698in}{0.785564in}}%
\pgfpathlineto{\pgfqpoint{5.181189in}{0.772979in}}%
\pgfpathlineto{\pgfqpoint{5.185680in}{0.766490in}}%
\pgfpathlineto{\pgfqpoint{5.187925in}{0.765658in}}%
\pgfpathlineto{\pgfqpoint{5.190170in}{0.766471in}}%
\pgfpathlineto{\pgfqpoint{5.194661in}{0.773053in}}%
\pgfpathlineto{\pgfqpoint{5.199151in}{0.786123in}}%
\pgfpathlineto{\pgfqpoint{5.205887in}{0.817044in}}%
\pgfpathlineto{\pgfqpoint{5.212623in}{0.859512in}}%
\pgfpathlineto{\pgfqpoint{5.223849in}{0.946822in}}%
\pgfpathlineto{\pgfqpoint{5.239567in}{1.070259in}}%
\pgfpathlineto{\pgfqpoint{5.246302in}{1.111105in}}%
\pgfpathlineto{\pgfqpoint{5.253038in}{1.139194in}}%
\pgfpathlineto{\pgfqpoint{5.257529in}{1.149544in}}%
\pgfpathlineto{\pgfqpoint{5.259774in}{1.152025in}}%
\pgfpathlineto{\pgfqpoint{5.262019in}{1.152673in}}%
\pgfpathlineto{\pgfqpoint{5.264265in}{1.151481in}}%
\pgfpathlineto{\pgfqpoint{5.268755in}{1.143653in}}%
\pgfpathlineto{\pgfqpoint{5.273246in}{1.128876in}}%
\pgfpathlineto{\pgfqpoint{5.279982in}{1.095119in}}%
\pgfpathlineto{\pgfqpoint{5.288963in}{1.033708in}}%
\pgfpathlineto{\pgfqpoint{5.313661in}{0.851805in}}%
\pgfpathlineto{\pgfqpoint{5.320397in}{0.820707in}}%
\pgfpathlineto{\pgfqpoint{5.324888in}{0.808081in}}%
\pgfpathlineto{\pgfqpoint{5.329378in}{0.802695in}}%
\pgfpathlineto{\pgfqpoint{5.331624in}{0.802836in}}%
\pgfpathlineto{\pgfqpoint{5.333869in}{0.804897in}}%
\pgfpathlineto{\pgfqpoint{5.338359in}{0.814759in}}%
\pgfpathlineto{\pgfqpoint{5.342850in}{0.832066in}}%
\pgfpathlineto{\pgfqpoint{5.349586in}{0.870836in}}%
\pgfpathlineto{\pgfqpoint{5.358567in}{0.941818in}}%
\pgfpathlineto{\pgfqpoint{5.387756in}{1.199328in}}%
\pgfpathlineto{\pgfqpoint{5.394492in}{1.234635in}}%
\pgfpathlineto{\pgfqpoint{5.398982in}{1.248612in}}%
\pgfpathlineto{\pgfqpoint{5.403473in}{1.254240in}}%
\pgfpathlineto{\pgfqpoint{5.405718in}{1.253842in}}%
\pgfpathlineto{\pgfqpoint{5.407964in}{1.251302in}}%
\pgfpathlineto{\pgfqpoint{5.412454in}{1.239931in}}%
\pgfpathlineto{\pgfqpoint{5.416945in}{1.220614in}}%
\pgfpathlineto{\pgfqpoint{5.423681in}{1.178611in}}%
\pgfpathlineto{\pgfqpoint{5.432662in}{1.104773in}}%
\pgfpathlineto{\pgfqpoint{5.455115in}{0.910312in}}%
\pgfpathlineto{\pgfqpoint{5.461851in}{0.871578in}}%
\pgfpathlineto{\pgfqpoint{5.466341in}{0.854881in}}%
\pgfpathlineto{\pgfqpoint{5.470832in}{0.846437in}}%
\pgfpathlineto{\pgfqpoint{5.473077in}{0.845471in}}%
\pgfpathlineto{\pgfqpoint{5.475322in}{0.846724in}}%
\pgfpathlineto{\pgfqpoint{5.477568in}{0.850206in}}%
\pgfpathlineto{\pgfqpoint{5.482058in}{0.863795in}}%
\pgfpathlineto{\pgfqpoint{5.486549in}{0.885916in}}%
\pgfpathlineto{\pgfqpoint{5.493285in}{0.933609in}}%
\pgfpathlineto{\pgfqpoint{5.502266in}{1.018619in}}%
\pgfpathlineto{\pgfqpoint{5.529209in}{1.298828in}}%
\pgfpathlineto{\pgfqpoint{5.535945in}{1.343471in}}%
\pgfpathlineto{\pgfqpoint{5.540436in}{1.362417in}}%
\pgfpathlineto{\pgfqpoint{5.544926in}{1.371766in}}%
\pgfpathlineto{\pgfqpoint{5.547172in}{1.372710in}}%
\pgfpathlineto{\pgfqpoint{5.549417in}{1.371148in}}%
\pgfpathlineto{\pgfqpoint{5.551662in}{1.367097in}}%
\pgfpathlineto{\pgfqpoint{5.556153in}{1.351735in}}%
\pgfpathlineto{\pgfqpoint{5.560644in}{1.327280in}}%
\pgfpathlineto{\pgfqpoint{5.567379in}{1.275917in}}%
\pgfpathlineto{\pgfqpoint{5.576361in}{1.187939in}}%
\pgfpathlineto{\pgfqpoint{5.596568in}{0.981955in}}%
\pgfpathlineto{\pgfqpoint{5.603304in}{0.933505in}}%
\pgfpathlineto{\pgfqpoint{5.607795in}{0.911361in}}%
\pgfpathlineto{\pgfqpoint{5.612285in}{0.898546in}}%
\pgfpathlineto{\pgfqpoint{5.614531in}{0.895855in}}%
\pgfpathlineto{\pgfqpoint{5.616776in}{0.895716in}}%
\pgfpathlineto{\pgfqpoint{5.619021in}{0.898153in}}%
\pgfpathlineto{\pgfqpoint{5.623512in}{0.910748in}}%
\pgfpathlineto{\pgfqpoint{5.628002in}{0.933385in}}%
\pgfpathlineto{\pgfqpoint{5.634738in}{0.984712in}}%
\pgfpathlineto{\pgfqpoint{5.643719in}{1.079610in}}%
\pgfpathlineto{\pgfqpoint{5.675154in}{1.449821in}}%
\pgfpathlineto{\pgfqpoint{5.681889in}{1.492179in}}%
\pgfpathlineto{\pgfqpoint{5.686380in}{1.506953in}}%
\pgfpathlineto{\pgfqpoint{5.688625in}{1.510035in}}%
\pgfpathlineto{\pgfqpoint{5.690871in}{1.510195in}}%
\pgfpathlineto{\pgfqpoint{5.693116in}{1.507434in}}%
\pgfpathlineto{\pgfqpoint{5.697606in}{1.493308in}}%
\pgfpathlineto{\pgfqpoint{5.702097in}{1.468268in}}%
\pgfpathlineto{\pgfqpoint{5.708833in}{1.412704in}}%
\pgfpathlineto{\pgfqpoint{5.717814in}{1.313686in}}%
\pgfpathlineto{\pgfqpoint{5.740267in}{1.048251in}}%
\pgfpathlineto{\pgfqpoint{5.747003in}{0.993785in}}%
\pgfpathlineto{\pgfqpoint{5.751494in}{0.969482in}}%
\pgfpathlineto{\pgfqpoint{5.755984in}{0.956103in}}%
\pgfpathlineto{\pgfqpoint{5.758229in}{0.953742in}}%
\pgfpathlineto{\pgfqpoint{5.760475in}{0.954343in}}%
\pgfpathlineto{\pgfqpoint{5.762720in}{0.957926in}}%
\pgfpathlineto{\pgfqpoint{5.767211in}{0.974002in}}%
\pgfpathlineto{\pgfqpoint{5.771701in}{1.001628in}}%
\pgfpathlineto{\pgfqpoint{5.778437in}{1.062951in}}%
\pgfpathlineto{\pgfqpoint{5.787418in}{1.174786in}}%
\pgfpathlineto{\pgfqpoint{5.816607in}{1.582112in}}%
\pgfpathlineto{\pgfqpoint{5.823343in}{1.638562in}}%
\pgfpathlineto{\pgfqpoint{5.827834in}{1.661095in}}%
\pgfpathlineto{\pgfqpoint{5.830079in}{1.667439in}}%
\pgfpathlineto{\pgfqpoint{5.832324in}{1.670402in}}%
\pgfpathlineto{\pgfqpoint{5.834569in}{1.669957in}}%
\pgfpathlineto{\pgfqpoint{5.836815in}{1.666111in}}%
\pgfpathlineto{\pgfqpoint{5.841305in}{1.648424in}}%
\pgfpathlineto{\pgfqpoint{5.845796in}{1.618113in}}%
\pgfpathlineto{\pgfqpoint{5.852532in}{1.551964in}}%
\pgfpathlineto{\pgfqpoint{5.861513in}{1.435502in}}%
\pgfpathlineto{\pgfqpoint{5.883966in}{1.127912in}}%
\pgfpathlineto{\pgfqpoint{5.890702in}{1.065754in}}%
\pgfpathlineto{\pgfqpoint{5.895192in}{1.038334in}}%
\pgfpathlineto{\pgfqpoint{5.899683in}{1.023597in}}%
\pgfpathlineto{\pgfqpoint{5.901928in}{1.021240in}}%
\pgfpathlineto{\pgfqpoint{5.904173in}{1.022304in}}%
\pgfpathlineto{\pgfqpoint{5.906419in}{1.026812in}}%
\pgfpathlineto{\pgfqpoint{5.910909in}{1.046099in}}%
\pgfpathlineto{\pgfqpoint{5.915400in}{1.078688in}}%
\pgfpathlineto{\pgfqpoint{5.922136in}{1.150456in}}%
\pgfpathlineto{\pgfqpoint{5.931117in}{1.280748in}}%
\pgfpathlineto{\pgfqpoint{5.960306in}{1.754164in}}%
\pgfpathlineto{\pgfqpoint{5.967042in}{1.819666in}}%
\pgfpathlineto{\pgfqpoint{5.971532in}{1.845734in}}%
\pgfpathlineto{\pgfqpoint{5.973778in}{1.853029in}}%
\pgfpathlineto{\pgfqpoint{5.976023in}{1.856384in}}%
\pgfpathlineto{\pgfqpoint{5.978268in}{1.855766in}}%
\pgfpathlineto{\pgfqpoint{5.980513in}{1.851186in}}%
\pgfpathlineto{\pgfqpoint{5.985004in}{1.830387in}}%
\pgfpathlineto{\pgfqpoint{5.989495in}{1.794902in}}%
\pgfpathlineto{\pgfqpoint{5.996231in}{1.717661in}}%
\pgfpathlineto{\pgfqpoint{6.005212in}{1.581983in}}%
\pgfpathlineto{\pgfqpoint{6.027665in}{1.224558in}}%
\pgfpathlineto{\pgfqpoint{6.034401in}{1.152235in}}%
\pgfpathlineto{\pgfqpoint{6.038891in}{1.120171in}}%
\pgfpathlineto{\pgfqpoint{6.043382in}{1.102698in}}%
\pgfpathlineto{\pgfqpoint{6.045627in}{1.099728in}}%
\pgfpathlineto{\pgfqpoint{6.047872in}{1.100693in}}%
\pgfpathlineto{\pgfqpoint{6.050118in}{1.105621in}}%
\pgfpathlineto{\pgfqpoint{6.054608in}{1.127311in}}%
\pgfpathlineto{\pgfqpoint{6.059099in}{1.164351in}}%
\pgfpathlineto{\pgfqpoint{6.065835in}{1.246420in}}%
\pgfpathlineto{\pgfqpoint{6.074816in}{1.396245in}}%
\pgfpathlineto{\pgfqpoint{6.106250in}{1.977737in}}%
\pgfpathlineto{\pgfqpoint{6.112986in}{2.044118in}}%
\pgfpathlineto{\pgfqpoint{6.117476in}{2.067074in}}%
\pgfpathlineto{\pgfqpoint{6.119722in}{2.071735in}}%
\pgfpathlineto{\pgfqpoint{6.121967in}{2.071774in}}%
\pgfpathlineto{\pgfqpoint{6.124212in}{2.067191in}}%
\pgfpathlineto{\pgfqpoint{6.128703in}{2.044428in}}%
\pgfpathlineto{\pgfqpoint{6.133193in}{2.004459in}}%
\pgfpathlineto{\pgfqpoint{6.139929in}{1.916260in}}%
\pgfpathlineto{\pgfqpoint{6.148911in}{1.759912in}}%
\pgfpathlineto{\pgfqpoint{6.155646in}{1.627934in}}%
\pgfpathlineto{\pgfqpoint{6.155646in}{1.627934in}}%
\pgfusepath{stroke}%
\end{pgfscope}%
\begin{pgfscope}%
\pgfsetrectcap%
\pgfsetmiterjoin%
\pgfsetlinewidth{0.803000pt}%
\definecolor{currentstroke}{rgb}{0.000000,0.000000,0.000000}%
\pgfsetstrokecolor{currentstroke}%
\pgfsetdash{}{0pt}%
\pgfpathmoveto{\pgfqpoint{3.798088in}{0.526234in}}%
\pgfpathlineto{\pgfqpoint{3.798088in}{2.145371in}}%
\pgfusepath{stroke}%
\end{pgfscope}%
\begin{pgfscope}%
\pgfsetrectcap%
\pgfsetmiterjoin%
\pgfsetlinewidth{0.803000pt}%
\definecolor{currentstroke}{rgb}{0.000000,0.000000,0.000000}%
\pgfsetstrokecolor{currentstroke}%
\pgfsetdash{}{0pt}%
\pgfpathmoveto{\pgfqpoint{6.267911in}{0.526234in}}%
\pgfpathlineto{\pgfqpoint{6.267911in}{2.145371in}}%
\pgfusepath{stroke}%
\end{pgfscope}%
\begin{pgfscope}%
\pgfsetrectcap%
\pgfsetmiterjoin%
\pgfsetlinewidth{0.803000pt}%
\definecolor{currentstroke}{rgb}{0.000000,0.000000,0.000000}%
\pgfsetstrokecolor{currentstroke}%
\pgfsetdash{}{0pt}%
\pgfpathmoveto{\pgfqpoint{3.798088in}{0.526234in}}%
\pgfpathlineto{\pgfqpoint{6.267911in}{0.526234in}}%
\pgfusepath{stroke}%
\end{pgfscope}%
\begin{pgfscope}%
\pgfsetrectcap%
\pgfsetmiterjoin%
\pgfsetlinewidth{0.803000pt}%
\definecolor{currentstroke}{rgb}{0.000000,0.000000,0.000000}%
\pgfsetstrokecolor{currentstroke}%
\pgfsetdash{}{0pt}%
\pgfpathmoveto{\pgfqpoint{3.798088in}{2.145371in}}%
\pgfpathlineto{\pgfqpoint{6.267911in}{2.145371in}}%
\pgfusepath{stroke}%
\end{pgfscope}%
\begin{pgfscope}%
\definecolor{textcolor}{rgb}{0.000000,0.000000,0.000000}%
\pgfsetstrokecolor{textcolor}%
\pgfsetfillcolor{textcolor}%
\pgftext[x=5.033000in,y=2.228704in,,base]{\color{textcolor}\rmfamily\fontsize{12.000000}{14.400000}\selectfont energy}%
\end{pgfscope}%
\end{pgfpicture}%
\makeatother%
\endgroup%
}
           \caption{Plotting Values for Small $\theta$}
           \label{fig:CP310}
        \end{center}
    \end{figure}

    \newpage\noindent
    Figure \ref{fig:CP310} is another plot of the same thing, except with a initial 
    angle of \ang{10}. This is well within the small angle approximation conditions and thus 
    it seems to behave somewhat naturally, oscillating around $\theta, \omega = 0$.
    \newline
    \newline
    The plot on the bottom left of the Figure is what we call a phase plot, effectively plotting 
    the position of the bob against the velocity of the bob. It reveals information about the 
    solution to the differential equation that describes the motion we're considering. In this 
    case, for the two plots above, we notice that the solution is periodic in some sense but 
    seems to be spiralling out of control, which we can attribute to the increasing amplitude 
    of the oscillations. It's also clear that energy is not being conserved, which we will 
    discuss next.
    \newline
    Firstly, $\theta$ seems to increase in amplitude as time goes on, as well as $\omega$. 
    This obviously has an effect on the energy of the system, which seems to be increasing 
    at some kind of exponential rate. It becomes obvious that the reason for this is rounding 
    errors in the code when we change $\Delta t$ to be 0.001, rather than its initial value of 
    0.01, producing the plots below in Figure \ref{fig:CP310b}.

    \begin{figure}[H]
        \begin{center}
           \scalebox{.7}{%% Creator: Matplotlib, PGF backend
%%
%% To include the figure in your LaTeX document, write
%%   \input{<filename>.pgf}
%%
%% Make sure the required packages are loaded in your preamble
%%   \usepackage{pgf}
%%
%% Figures using additional raster images can only be included by \input if
%% they are in the same directory as the main LaTeX file. For loading figures
%% from other directories you can use the `import` package
%%   \usepackage{import}
%% and then include the figures with
%%   \import{<path to file>}{<filename>.pgf}
%%
%% Matplotlib used the following preamble
%%
\begingroup%
\makeatletter%
\begin{pgfpicture}%
\pgfpathrectangle{\pgfpointorigin}{\pgfqpoint{6.400000in}{4.800000in}}%
\pgfusepath{use as bounding box, clip}%
\begin{pgfscope}%
\pgfsetbuttcap%
\pgfsetmiterjoin%
\definecolor{currentfill}{rgb}{1.000000,1.000000,1.000000}%
\pgfsetfillcolor{currentfill}%
\pgfsetlinewidth{0.000000pt}%
\definecolor{currentstroke}{rgb}{1.000000,1.000000,1.000000}%
\pgfsetstrokecolor{currentstroke}%
\pgfsetdash{}{0pt}%
\pgfpathmoveto{\pgfqpoint{0.000000in}{0.000000in}}%
\pgfpathlineto{\pgfqpoint{6.400000in}{0.000000in}}%
\pgfpathlineto{\pgfqpoint{6.400000in}{4.800000in}}%
\pgfpathlineto{\pgfqpoint{0.000000in}{4.800000in}}%
\pgfpathclose%
\pgfusepath{fill}%
\end{pgfscope}%
\begin{pgfscope}%
\pgfsetbuttcap%
\pgfsetmiterjoin%
\definecolor{currentfill}{rgb}{1.000000,1.000000,1.000000}%
\pgfsetfillcolor{currentfill}%
\pgfsetlinewidth{0.000000pt}%
\definecolor{currentstroke}{rgb}{0.000000,0.000000,0.000000}%
\pgfsetstrokecolor{currentstroke}%
\pgfsetstrokeopacity{0.000000}%
\pgfsetdash{}{0pt}%
\pgfpathmoveto{\pgfqpoint{0.835065in}{2.870679in}}%
\pgfpathlineto{\pgfqpoint{3.196863in}{2.870679in}}%
\pgfpathlineto{\pgfqpoint{3.196863in}{4.489815in}}%
\pgfpathlineto{\pgfqpoint{0.835065in}{4.489815in}}%
\pgfpathclose%
\pgfusepath{fill}%
\end{pgfscope}%
\begin{pgfscope}%
\pgfsetbuttcap%
\pgfsetroundjoin%
\definecolor{currentfill}{rgb}{0.000000,0.000000,0.000000}%
\pgfsetfillcolor{currentfill}%
\pgfsetlinewidth{0.803000pt}%
\definecolor{currentstroke}{rgb}{0.000000,0.000000,0.000000}%
\pgfsetstrokecolor{currentstroke}%
\pgfsetdash{}{0pt}%
\pgfsys@defobject{currentmarker}{\pgfqpoint{0.000000in}{-0.048611in}}{\pgfqpoint{0.000000in}{0.000000in}}{%
\pgfpathmoveto{\pgfqpoint{0.000000in}{0.000000in}}%
\pgfpathlineto{\pgfqpoint{0.000000in}{-0.048611in}}%
\pgfusepath{stroke,fill}%
}%
\begin{pgfscope}%
\pgfsys@transformshift{0.942419in}{2.870679in}%
\pgfsys@useobject{currentmarker}{}%
\end{pgfscope}%
\end{pgfscope}%
\begin{pgfscope}%
\definecolor{textcolor}{rgb}{0.000000,0.000000,0.000000}%
\pgfsetstrokecolor{textcolor}%
\pgfsetfillcolor{textcolor}%
\pgftext[x=0.942419in,y=2.773457in,,top]{\color{textcolor}\rmfamily\fontsize{10.000000}{12.000000}\selectfont \(\displaystyle 0.0\)}%
\end{pgfscope}%
\begin{pgfscope}%
\pgfsetbuttcap%
\pgfsetroundjoin%
\definecolor{currentfill}{rgb}{0.000000,0.000000,0.000000}%
\pgfsetfillcolor{currentfill}%
\pgfsetlinewidth{0.803000pt}%
\definecolor{currentstroke}{rgb}{0.000000,0.000000,0.000000}%
\pgfsetstrokecolor{currentstroke}%
\pgfsetdash{}{0pt}%
\pgfsys@defobject{currentmarker}{\pgfqpoint{0.000000in}{-0.048611in}}{\pgfqpoint{0.000000in}{0.000000in}}{%
\pgfpathmoveto{\pgfqpoint{0.000000in}{0.000000in}}%
\pgfpathlineto{\pgfqpoint{0.000000in}{-0.048611in}}%
\pgfusepath{stroke,fill}%
}%
\begin{pgfscope}%
\pgfsys@transformshift{1.479191in}{2.870679in}%
\pgfsys@useobject{currentmarker}{}%
\end{pgfscope}%
\end{pgfscope}%
\begin{pgfscope}%
\definecolor{textcolor}{rgb}{0.000000,0.000000,0.000000}%
\pgfsetstrokecolor{textcolor}%
\pgfsetfillcolor{textcolor}%
\pgftext[x=1.479191in,y=2.773457in,,top]{\color{textcolor}\rmfamily\fontsize{10.000000}{12.000000}\selectfont \(\displaystyle 2.5\)}%
\end{pgfscope}%
\begin{pgfscope}%
\pgfsetbuttcap%
\pgfsetroundjoin%
\definecolor{currentfill}{rgb}{0.000000,0.000000,0.000000}%
\pgfsetfillcolor{currentfill}%
\pgfsetlinewidth{0.803000pt}%
\definecolor{currentstroke}{rgb}{0.000000,0.000000,0.000000}%
\pgfsetstrokecolor{currentstroke}%
\pgfsetdash{}{0pt}%
\pgfsys@defobject{currentmarker}{\pgfqpoint{0.000000in}{-0.048611in}}{\pgfqpoint{0.000000in}{0.000000in}}{%
\pgfpathmoveto{\pgfqpoint{0.000000in}{0.000000in}}%
\pgfpathlineto{\pgfqpoint{0.000000in}{-0.048611in}}%
\pgfusepath{stroke,fill}%
}%
\begin{pgfscope}%
\pgfsys@transformshift{2.015964in}{2.870679in}%
\pgfsys@useobject{currentmarker}{}%
\end{pgfscope}%
\end{pgfscope}%
\begin{pgfscope}%
\definecolor{textcolor}{rgb}{0.000000,0.000000,0.000000}%
\pgfsetstrokecolor{textcolor}%
\pgfsetfillcolor{textcolor}%
\pgftext[x=2.015964in,y=2.773457in,,top]{\color{textcolor}\rmfamily\fontsize{10.000000}{12.000000}\selectfont \(\displaystyle 5.0\)}%
\end{pgfscope}%
\begin{pgfscope}%
\pgfsetbuttcap%
\pgfsetroundjoin%
\definecolor{currentfill}{rgb}{0.000000,0.000000,0.000000}%
\pgfsetfillcolor{currentfill}%
\pgfsetlinewidth{0.803000pt}%
\definecolor{currentstroke}{rgb}{0.000000,0.000000,0.000000}%
\pgfsetstrokecolor{currentstroke}%
\pgfsetdash{}{0pt}%
\pgfsys@defobject{currentmarker}{\pgfqpoint{0.000000in}{-0.048611in}}{\pgfqpoint{0.000000in}{0.000000in}}{%
\pgfpathmoveto{\pgfqpoint{0.000000in}{0.000000in}}%
\pgfpathlineto{\pgfqpoint{0.000000in}{-0.048611in}}%
\pgfusepath{stroke,fill}%
}%
\begin{pgfscope}%
\pgfsys@transformshift{2.552736in}{2.870679in}%
\pgfsys@useobject{currentmarker}{}%
\end{pgfscope}%
\end{pgfscope}%
\begin{pgfscope}%
\definecolor{textcolor}{rgb}{0.000000,0.000000,0.000000}%
\pgfsetstrokecolor{textcolor}%
\pgfsetfillcolor{textcolor}%
\pgftext[x=2.552736in,y=2.773457in,,top]{\color{textcolor}\rmfamily\fontsize{10.000000}{12.000000}\selectfont \(\displaystyle 7.5\)}%
\end{pgfscope}%
\begin{pgfscope}%
\pgfsetbuttcap%
\pgfsetroundjoin%
\definecolor{currentfill}{rgb}{0.000000,0.000000,0.000000}%
\pgfsetfillcolor{currentfill}%
\pgfsetlinewidth{0.803000pt}%
\definecolor{currentstroke}{rgb}{0.000000,0.000000,0.000000}%
\pgfsetstrokecolor{currentstroke}%
\pgfsetdash{}{0pt}%
\pgfsys@defobject{currentmarker}{\pgfqpoint{0.000000in}{-0.048611in}}{\pgfqpoint{0.000000in}{0.000000in}}{%
\pgfpathmoveto{\pgfqpoint{0.000000in}{0.000000in}}%
\pgfpathlineto{\pgfqpoint{0.000000in}{-0.048611in}}%
\pgfusepath{stroke,fill}%
}%
\begin{pgfscope}%
\pgfsys@transformshift{3.089508in}{2.870679in}%
\pgfsys@useobject{currentmarker}{}%
\end{pgfscope}%
\end{pgfscope}%
\begin{pgfscope}%
\definecolor{textcolor}{rgb}{0.000000,0.000000,0.000000}%
\pgfsetstrokecolor{textcolor}%
\pgfsetfillcolor{textcolor}%
\pgftext[x=3.089508in,y=2.773457in,,top]{\color{textcolor}\rmfamily\fontsize{10.000000}{12.000000}\selectfont \(\displaystyle 10.0\)}%
\end{pgfscope}%
\begin{pgfscope}%
\definecolor{textcolor}{rgb}{0.000000,0.000000,0.000000}%
\pgfsetstrokecolor{textcolor}%
\pgfsetfillcolor{textcolor}%
\pgftext[x=2.015964in,y=2.594444in,,top]{\color{textcolor}\rmfamily\fontsize{10.000000}{12.000000}\selectfont time (s)}%
\end{pgfscope}%
\begin{pgfscope}%
\pgfsetbuttcap%
\pgfsetroundjoin%
\definecolor{currentfill}{rgb}{0.000000,0.000000,0.000000}%
\pgfsetfillcolor{currentfill}%
\pgfsetlinewidth{0.803000pt}%
\definecolor{currentstroke}{rgb}{0.000000,0.000000,0.000000}%
\pgfsetstrokecolor{currentstroke}%
\pgfsetdash{}{0pt}%
\pgfsys@defobject{currentmarker}{\pgfqpoint{-0.048611in}{0.000000in}}{\pgfqpoint{0.000000in}{0.000000in}}{%
\pgfpathmoveto{\pgfqpoint{0.000000in}{0.000000in}}%
\pgfpathlineto{\pgfqpoint{-0.048611in}{0.000000in}}%
\pgfusepath{stroke,fill}%
}%
\begin{pgfscope}%
\pgfsys@transformshift{0.835065in}{2.930529in}%
\pgfsys@useobject{currentmarker}{}%
\end{pgfscope}%
\end{pgfscope}%
\begin{pgfscope}%
\definecolor{textcolor}{rgb}{0.000000,0.000000,0.000000}%
\pgfsetstrokecolor{textcolor}%
\pgfsetfillcolor{textcolor}%
\pgftext[x=0.452348in,y=2.882304in,left,base]{\color{textcolor}\rmfamily\fontsize{10.000000}{12.000000}\selectfont \(\displaystyle -0.2\)}%
\end{pgfscope}%
\begin{pgfscope}%
\pgfsetbuttcap%
\pgfsetroundjoin%
\definecolor{currentfill}{rgb}{0.000000,0.000000,0.000000}%
\pgfsetfillcolor{currentfill}%
\pgfsetlinewidth{0.803000pt}%
\definecolor{currentstroke}{rgb}{0.000000,0.000000,0.000000}%
\pgfsetstrokecolor{currentstroke}%
\pgfsetdash{}{0pt}%
\pgfsys@defobject{currentmarker}{\pgfqpoint{-0.048611in}{0.000000in}}{\pgfqpoint{0.000000in}{0.000000in}}{%
\pgfpathmoveto{\pgfqpoint{0.000000in}{0.000000in}}%
\pgfpathlineto{\pgfqpoint{-0.048611in}{0.000000in}}%
\pgfusepath{stroke,fill}%
}%
\begin{pgfscope}%
\pgfsys@transformshift{0.835065in}{3.306832in}%
\pgfsys@useobject{currentmarker}{}%
\end{pgfscope}%
\end{pgfscope}%
\begin{pgfscope}%
\definecolor{textcolor}{rgb}{0.000000,0.000000,0.000000}%
\pgfsetstrokecolor{textcolor}%
\pgfsetfillcolor{textcolor}%
\pgftext[x=0.452348in,y=3.258607in,left,base]{\color{textcolor}\rmfamily\fontsize{10.000000}{12.000000}\selectfont \(\displaystyle -0.1\)}%
\end{pgfscope}%
\begin{pgfscope}%
\pgfsetbuttcap%
\pgfsetroundjoin%
\definecolor{currentfill}{rgb}{0.000000,0.000000,0.000000}%
\pgfsetfillcolor{currentfill}%
\pgfsetlinewidth{0.803000pt}%
\definecolor{currentstroke}{rgb}{0.000000,0.000000,0.000000}%
\pgfsetstrokecolor{currentstroke}%
\pgfsetdash{}{0pt}%
\pgfsys@defobject{currentmarker}{\pgfqpoint{-0.048611in}{0.000000in}}{\pgfqpoint{0.000000in}{0.000000in}}{%
\pgfpathmoveto{\pgfqpoint{0.000000in}{0.000000in}}%
\pgfpathlineto{\pgfqpoint{-0.048611in}{0.000000in}}%
\pgfusepath{stroke,fill}%
}%
\begin{pgfscope}%
\pgfsys@transformshift{0.835065in}{3.683136in}%
\pgfsys@useobject{currentmarker}{}%
\end{pgfscope}%
\end{pgfscope}%
\begin{pgfscope}%
\definecolor{textcolor}{rgb}{0.000000,0.000000,0.000000}%
\pgfsetstrokecolor{textcolor}%
\pgfsetfillcolor{textcolor}%
\pgftext[x=0.560373in,y=3.634910in,left,base]{\color{textcolor}\rmfamily\fontsize{10.000000}{12.000000}\selectfont \(\displaystyle 0.0\)}%
\end{pgfscope}%
\begin{pgfscope}%
\pgfsetbuttcap%
\pgfsetroundjoin%
\definecolor{currentfill}{rgb}{0.000000,0.000000,0.000000}%
\pgfsetfillcolor{currentfill}%
\pgfsetlinewidth{0.803000pt}%
\definecolor{currentstroke}{rgb}{0.000000,0.000000,0.000000}%
\pgfsetstrokecolor{currentstroke}%
\pgfsetdash{}{0pt}%
\pgfsys@defobject{currentmarker}{\pgfqpoint{-0.048611in}{0.000000in}}{\pgfqpoint{0.000000in}{0.000000in}}{%
\pgfpathmoveto{\pgfqpoint{0.000000in}{0.000000in}}%
\pgfpathlineto{\pgfqpoint{-0.048611in}{0.000000in}}%
\pgfusepath{stroke,fill}%
}%
\begin{pgfscope}%
\pgfsys@transformshift{0.835065in}{4.059439in}%
\pgfsys@useobject{currentmarker}{}%
\end{pgfscope}%
\end{pgfscope}%
\begin{pgfscope}%
\definecolor{textcolor}{rgb}{0.000000,0.000000,0.000000}%
\pgfsetstrokecolor{textcolor}%
\pgfsetfillcolor{textcolor}%
\pgftext[x=0.560373in,y=4.011214in,left,base]{\color{textcolor}\rmfamily\fontsize{10.000000}{12.000000}\selectfont \(\displaystyle 0.1\)}%
\end{pgfscope}%
\begin{pgfscope}%
\pgfsetbuttcap%
\pgfsetroundjoin%
\definecolor{currentfill}{rgb}{0.000000,0.000000,0.000000}%
\pgfsetfillcolor{currentfill}%
\pgfsetlinewidth{0.803000pt}%
\definecolor{currentstroke}{rgb}{0.000000,0.000000,0.000000}%
\pgfsetstrokecolor{currentstroke}%
\pgfsetdash{}{0pt}%
\pgfsys@defobject{currentmarker}{\pgfqpoint{-0.048611in}{0.000000in}}{\pgfqpoint{0.000000in}{0.000000in}}{%
\pgfpathmoveto{\pgfqpoint{0.000000in}{0.000000in}}%
\pgfpathlineto{\pgfqpoint{-0.048611in}{0.000000in}}%
\pgfusepath{stroke,fill}%
}%
\begin{pgfscope}%
\pgfsys@transformshift{0.835065in}{4.435742in}%
\pgfsys@useobject{currentmarker}{}%
\end{pgfscope}%
\end{pgfscope}%
\begin{pgfscope}%
\definecolor{textcolor}{rgb}{0.000000,0.000000,0.000000}%
\pgfsetstrokecolor{textcolor}%
\pgfsetfillcolor{textcolor}%
\pgftext[x=0.560373in,y=4.387517in,left,base]{\color{textcolor}\rmfamily\fontsize{10.000000}{12.000000}\selectfont \(\displaystyle 0.2\)}%
\end{pgfscope}%
\begin{pgfscope}%
\definecolor{textcolor}{rgb}{0.000000,0.000000,0.000000}%
\pgfsetstrokecolor{textcolor}%
\pgfsetfillcolor{textcolor}%
\pgftext[x=0.396792in,y=3.680247in,,bottom,rotate=90.000000]{\color{textcolor}\rmfamily\fontsize{10.000000}{12.000000}\selectfont angle (rad)}%
\end{pgfscope}%
\begin{pgfscope}%
\pgfpathrectangle{\pgfqpoint{0.835065in}{2.870679in}}{\pgfqpoint{2.361798in}{1.619136in}}%
\pgfusepath{clip}%
\pgfsetrectcap%
\pgfsetroundjoin%
\pgfsetlinewidth{1.505625pt}%
\definecolor{currentstroke}{rgb}{0.000000,0.000000,1.000000}%
\pgfsetstrokecolor{currentstroke}%
\pgfsetdash{}{0pt}%
\pgfpathmoveto{\pgfqpoint{0.942419in}{4.339909in}}%
\pgfpathlineto{\pgfqpoint{0.944996in}{4.338831in}}%
\pgfpathlineto{\pgfqpoint{0.948001in}{4.334606in}}%
\pgfpathlineto{\pgfqpoint{0.951652in}{4.325207in}}%
\pgfpathlineto{\pgfqpoint{0.956375in}{4.306201in}}%
\pgfpathlineto{\pgfqpoint{0.962172in}{4.272643in}}%
\pgfpathlineto{\pgfqpoint{0.969258in}{4.217160in}}%
\pgfpathlineto{\pgfqpoint{0.977846in}{4.130627in}}%
\pgfpathlineto{\pgfqpoint{0.988581in}{3.997770in}}%
\pgfpathlineto{\pgfqpoint{1.003611in}{3.780735in}}%
\pgfpathlineto{\pgfqpoint{1.035388in}{3.315566in}}%
\pgfpathlineto{\pgfqpoint{1.046768in}{3.184604in}}%
\pgfpathlineto{\pgfqpoint{1.055571in}{3.106607in}}%
\pgfpathlineto{\pgfqpoint{1.062871in}{3.060006in}}%
\pgfpathlineto{\pgfqpoint{1.068668in}{3.035668in}}%
\pgfpathlineto{\pgfqpoint{1.073177in}{3.024820in}}%
\pgfpathlineto{\pgfqpoint{1.076612in}{3.021392in}}%
\pgfpathlineto{\pgfqpoint{1.079189in}{3.021583in}}%
\pgfpathlineto{\pgfqpoint{1.081765in}{3.024143in}}%
\pgfpathlineto{\pgfqpoint{1.085200in}{3.031227in}}%
\pgfpathlineto{\pgfqpoint{1.089495in}{3.045913in}}%
\pgfpathlineto{\pgfqpoint{1.094862in}{3.073172in}}%
\pgfpathlineto{\pgfqpoint{1.101304in}{3.118370in}}%
\pgfpathlineto{\pgfqpoint{1.109248in}{3.191439in}}%
\pgfpathlineto{\pgfqpoint{1.118910in}{3.302672in}}%
\pgfpathlineto{\pgfqpoint{1.131578in}{3.476717in}}%
\pgfpathlineto{\pgfqpoint{1.178384in}{4.147742in}}%
\pgfpathlineto{\pgfqpoint{1.188046in}{4.242586in}}%
\pgfpathlineto{\pgfqpoint{1.195776in}{4.298478in}}%
\pgfpathlineto{\pgfqpoint{1.202002in}{4.329200in}}%
\pgfpathlineto{\pgfqpoint{1.206940in}{4.344005in}}%
\pgfpathlineto{\pgfqpoint{1.210805in}{4.349544in}}%
\pgfpathlineto{\pgfqpoint{1.213596in}{4.350209in}}%
\pgfpathlineto{\pgfqpoint{1.216173in}{4.348334in}}%
\pgfpathlineto{\pgfqpoint{1.219394in}{4.342642in}}%
\pgfpathlineto{\pgfqpoint{1.223473in}{4.330139in}}%
\pgfpathlineto{\pgfqpoint{1.228411in}{4.307240in}}%
\pgfpathlineto{\pgfqpoint{1.234638in}{4.266694in}}%
\pgfpathlineto{\pgfqpoint{1.242153in}{4.201565in}}%
\pgfpathlineto{\pgfqpoint{1.251170in}{4.102696in}}%
\pgfpathlineto{\pgfqpoint{1.262765in}{3.949082in}}%
\pgfpathlineto{\pgfqpoint{1.281015in}{3.672650in}}%
\pgfpathlineto{\pgfqpoint{1.303774in}{3.335496in}}%
\pgfpathlineto{\pgfqpoint{1.315583in}{3.192702in}}%
\pgfpathlineto{\pgfqpoint{1.324816in}{3.106091in}}%
\pgfpathlineto{\pgfqpoint{1.332330in}{3.054931in}}%
\pgfpathlineto{\pgfqpoint{1.338342in}{3.027674in}}%
\pgfpathlineto{\pgfqpoint{1.343066in}{3.015183in}}%
\pgfpathlineto{\pgfqpoint{1.346501in}{3.011134in}}%
\pgfpathlineto{\pgfqpoint{1.349078in}{3.010903in}}%
\pgfpathlineto{\pgfqpoint{1.351654in}{3.013079in}}%
\pgfpathlineto{\pgfqpoint{1.354875in}{3.019171in}}%
\pgfpathlineto{\pgfqpoint{1.358954in}{3.032218in}}%
\pgfpathlineto{\pgfqpoint{1.364107in}{3.057037in}}%
\pgfpathlineto{\pgfqpoint{1.370334in}{3.098968in}}%
\pgfpathlineto{\pgfqpoint{1.377849in}{3.165784in}}%
\pgfpathlineto{\pgfqpoint{1.387081in}{3.269277in}}%
\pgfpathlineto{\pgfqpoint{1.398890in}{3.428804in}}%
\pgfpathlineto{\pgfqpoint{1.418214in}{3.725435in}}%
\pgfpathlineto{\pgfqpoint{1.439041in}{4.034721in}}%
\pgfpathlineto{\pgfqpoint{1.450850in}{4.178395in}}%
\pgfpathlineto{\pgfqpoint{1.460082in}{4.265446in}}%
\pgfpathlineto{\pgfqpoint{1.467597in}{4.316788in}}%
\pgfpathlineto{\pgfqpoint{1.473609in}{4.344073in}}%
\pgfpathlineto{\pgfqpoint{1.478332in}{4.356512in}}%
\pgfpathlineto{\pgfqpoint{1.481768in}{4.360482in}}%
\pgfpathlineto{\pgfqpoint{1.484344in}{4.360632in}}%
\pgfpathlineto{\pgfqpoint{1.486921in}{4.358356in}}%
\pgfpathlineto{\pgfqpoint{1.490141in}{4.352113in}}%
\pgfpathlineto{\pgfqpoint{1.494221in}{4.338837in}}%
\pgfpathlineto{\pgfqpoint{1.499374in}{4.313666in}}%
\pgfpathlineto{\pgfqpoint{1.505601in}{4.271227in}}%
\pgfpathlineto{\pgfqpoint{1.513115in}{4.203689in}}%
\pgfpathlineto{\pgfqpoint{1.522348in}{4.099179in}}%
\pgfpathlineto{\pgfqpoint{1.534157in}{3.938210in}}%
\pgfpathlineto{\pgfqpoint{1.553695in}{3.635755in}}%
\pgfpathlineto{\pgfqpoint{1.574307in}{3.327623in}}%
\pgfpathlineto{\pgfqpoint{1.586116in}{3.183054in}}%
\pgfpathlineto{\pgfqpoint{1.595349in}{3.095550in}}%
\pgfpathlineto{\pgfqpoint{1.602864in}{3.044015in}}%
\pgfpathlineto{\pgfqpoint{1.608876in}{3.016695in}}%
\pgfpathlineto{\pgfqpoint{1.613599in}{3.004302in}}%
\pgfpathlineto{\pgfqpoint{1.617034in}{3.000407in}}%
\pgfpathlineto{\pgfqpoint{1.619611in}{3.000335in}}%
\pgfpathlineto{\pgfqpoint{1.622187in}{3.002708in}}%
\pgfpathlineto{\pgfqpoint{1.625408in}{3.009098in}}%
\pgfpathlineto{\pgfqpoint{1.629488in}{3.022601in}}%
\pgfpathlineto{\pgfqpoint{1.634641in}{3.048120in}}%
\pgfpathlineto{\pgfqpoint{1.640867in}{3.091064in}}%
\pgfpathlineto{\pgfqpoint{1.648382in}{3.159320in}}%
\pgfpathlineto{\pgfqpoint{1.657614in}{3.264846in}}%
\pgfpathlineto{\pgfqpoint{1.669423in}{3.427260in}}%
\pgfpathlineto{\pgfqpoint{1.688962in}{3.732186in}}%
\pgfpathlineto{\pgfqpoint{1.709574in}{4.042555in}}%
\pgfpathlineto{\pgfqpoint{1.721383in}{4.188036in}}%
\pgfpathlineto{\pgfqpoint{1.730615in}{4.276006in}}%
\pgfpathlineto{\pgfqpoint{1.738130in}{4.327744in}}%
\pgfpathlineto{\pgfqpoint{1.744142in}{4.355108in}}%
\pgfpathlineto{\pgfqpoint{1.748866in}{4.367460in}}%
\pgfpathlineto{\pgfqpoint{1.752301in}{4.371284in}}%
\pgfpathlineto{\pgfqpoint{1.754878in}{4.371281in}}%
\pgfpathlineto{\pgfqpoint{1.757454in}{4.368814in}}%
\pgfpathlineto{\pgfqpoint{1.760675in}{4.362280in}}%
\pgfpathlineto{\pgfqpoint{1.764754in}{4.348554in}}%
\pgfpathlineto{\pgfqpoint{1.769907in}{4.322691in}}%
\pgfpathlineto{\pgfqpoint{1.776134in}{4.279247in}}%
\pgfpathlineto{\pgfqpoint{1.783649in}{4.210275in}}%
\pgfpathlineto{\pgfqpoint{1.792881in}{4.103735in}}%
\pgfpathlineto{\pgfqpoint{1.804690in}{3.939875in}}%
\pgfpathlineto{\pgfqpoint{1.824443in}{3.629020in}}%
\pgfpathlineto{\pgfqpoint{1.844841in}{3.319833in}}%
\pgfpathlineto{\pgfqpoint{1.856435in}{3.175774in}}%
\pgfpathlineto{\pgfqpoint{1.865667in}{3.086731in}}%
\pgfpathlineto{\pgfqpoint{1.873182in}{3.034237in}}%
\pgfpathlineto{\pgfqpoint{1.879194in}{3.006359in}}%
\pgfpathlineto{\pgfqpoint{1.883918in}{2.993667in}}%
\pgfpathlineto{\pgfqpoint{1.887353in}{2.989633in}}%
\pgfpathlineto{\pgfqpoint{1.889929in}{2.989502in}}%
\pgfpathlineto{\pgfqpoint{1.892506in}{2.991854in}}%
\pgfpathlineto{\pgfqpoint{1.895727in}{2.998272in}}%
\pgfpathlineto{\pgfqpoint{1.899806in}{3.011896in}}%
\pgfpathlineto{\pgfqpoint{1.904959in}{3.037703in}}%
\pgfpathlineto{\pgfqpoint{1.911186in}{3.081193in}}%
\pgfpathlineto{\pgfqpoint{1.918700in}{3.150379in}}%
\pgfpathlineto{\pgfqpoint{1.927933in}{3.257416in}}%
\pgfpathlineto{\pgfqpoint{1.939742in}{3.422249in}}%
\pgfpathlineto{\pgfqpoint{1.959280in}{3.731911in}}%
\pgfpathlineto{\pgfqpoint{1.979893in}{4.047329in}}%
\pgfpathlineto{\pgfqpoint{1.991702in}{4.195289in}}%
\pgfpathlineto{\pgfqpoint{2.000934in}{4.284832in}}%
\pgfpathlineto{\pgfqpoint{2.008449in}{4.337557in}}%
\pgfpathlineto{\pgfqpoint{2.014461in}{4.365499in}}%
\pgfpathlineto{\pgfqpoint{2.019184in}{4.378167in}}%
\pgfpathlineto{\pgfqpoint{2.022620in}{4.382141in}}%
\pgfpathlineto{\pgfqpoint{2.025196in}{4.382205in}}%
\pgfpathlineto{\pgfqpoint{2.027773in}{4.379767in}}%
\pgfpathlineto{\pgfqpoint{2.030993in}{4.373214in}}%
\pgfpathlineto{\pgfqpoint{2.035073in}{4.359377in}}%
\pgfpathlineto{\pgfqpoint{2.040226in}{4.333238in}}%
\pgfpathlineto{\pgfqpoint{2.046452in}{4.289261in}}%
\pgfpathlineto{\pgfqpoint{2.053967in}{4.219370in}}%
\pgfpathlineto{\pgfqpoint{2.063200in}{4.111325in}}%
\pgfpathlineto{\pgfqpoint{2.075009in}{3.945044in}}%
\pgfpathlineto{\pgfqpoint{2.094547in}{3.632867in}}%
\pgfpathlineto{\pgfqpoint{2.115159in}{3.315130in}}%
\pgfpathlineto{\pgfqpoint{2.126968in}{3.166198in}}%
\pgfpathlineto{\pgfqpoint{2.136201in}{3.076140in}}%
\pgfpathlineto{\pgfqpoint{2.143715in}{3.023173in}}%
\pgfpathlineto{\pgfqpoint{2.149727in}{2.995157in}}%
\pgfpathlineto{\pgfqpoint{2.154451in}{2.982507in}}%
\pgfpathlineto{\pgfqpoint{2.157886in}{2.978588in}}%
\pgfpathlineto{\pgfqpoint{2.160463in}{2.978588in}}%
\pgfpathlineto{\pgfqpoint{2.163039in}{2.981109in}}%
\pgfpathlineto{\pgfqpoint{2.166260in}{2.987793in}}%
\pgfpathlineto{\pgfqpoint{2.170339in}{3.001837in}}%
\pgfpathlineto{\pgfqpoint{2.175492in}{3.028302in}}%
\pgfpathlineto{\pgfqpoint{2.181719in}{3.072764in}}%
\pgfpathlineto{\pgfqpoint{2.189234in}{3.143355in}}%
\pgfpathlineto{\pgfqpoint{2.198466in}{3.252405in}}%
\pgfpathlineto{\pgfqpoint{2.210275in}{3.420136in}}%
\pgfpathlineto{\pgfqpoint{2.230028in}{3.738365in}}%
\pgfpathlineto{\pgfqpoint{2.250426in}{4.054925in}}%
\pgfpathlineto{\pgfqpoint{2.262020in}{4.202439in}}%
\pgfpathlineto{\pgfqpoint{2.271253in}{4.293633in}}%
\pgfpathlineto{\pgfqpoint{2.278767in}{4.347410in}}%
\pgfpathlineto{\pgfqpoint{2.284779in}{4.375983in}}%
\pgfpathlineto{\pgfqpoint{2.289503in}{4.389006in}}%
\pgfpathlineto{\pgfqpoint{2.292938in}{4.393157in}}%
\pgfpathlineto{\pgfqpoint{2.295515in}{4.393308in}}%
\pgfpathlineto{\pgfqpoint{2.298091in}{4.390918in}}%
\pgfpathlineto{\pgfqpoint{2.301312in}{4.384370in}}%
\pgfpathlineto{\pgfqpoint{2.305391in}{4.370452in}}%
\pgfpathlineto{\pgfqpoint{2.310544in}{4.344069in}}%
\pgfpathlineto{\pgfqpoint{2.316771in}{4.299591in}}%
\pgfpathlineto{\pgfqpoint{2.324286in}{4.228813in}}%
\pgfpathlineto{\pgfqpoint{2.333518in}{4.119290in}}%
\pgfpathlineto{\pgfqpoint{2.345327in}{3.950595in}}%
\pgfpathlineto{\pgfqpoint{2.364866in}{3.633604in}}%
\pgfpathlineto{\pgfqpoint{2.385478in}{3.310633in}}%
\pgfpathlineto{\pgfqpoint{2.397287in}{3.159084in}}%
\pgfpathlineto{\pgfqpoint{2.406519in}{3.067339in}}%
\pgfpathlineto{\pgfqpoint{2.414034in}{3.013289in}}%
\pgfpathlineto{\pgfqpoint{2.420046in}{2.984619in}}%
\pgfpathlineto{\pgfqpoint{2.424769in}{2.971596in}}%
\pgfpathlineto{\pgfqpoint{2.428205in}{2.967487in}}%
\pgfpathlineto{\pgfqpoint{2.430781in}{2.967391in}}%
\pgfpathlineto{\pgfqpoint{2.433358in}{2.969855in}}%
\pgfpathlineto{\pgfqpoint{2.436578in}{2.976524in}}%
\pgfpathlineto{\pgfqpoint{2.440658in}{2.990638in}}%
\pgfpathlineto{\pgfqpoint{2.445811in}{3.017333in}}%
\pgfpathlineto{\pgfqpoint{2.452037in}{3.062280in}}%
\pgfpathlineto{\pgfqpoint{2.459552in}{3.133745in}}%
\pgfpathlineto{\pgfqpoint{2.468785in}{3.244264in}}%
\pgfpathlineto{\pgfqpoint{2.480594in}{3.414409in}}%
\pgfpathlineto{\pgfqpoint{2.500132in}{3.733957in}}%
\pgfpathlineto{\pgfqpoint{2.520744in}{4.059337in}}%
\pgfpathlineto{\pgfqpoint{2.532553in}{4.211922in}}%
\pgfpathlineto{\pgfqpoint{2.541786in}{4.304235in}}%
\pgfpathlineto{\pgfqpoint{2.549301in}{4.358571in}}%
\pgfpathlineto{\pgfqpoint{2.555312in}{4.387349in}}%
\pgfpathlineto{\pgfqpoint{2.560036in}{4.400380in}}%
\pgfpathlineto{\pgfqpoint{2.563471in}{4.404452in}}%
\pgfpathlineto{\pgfqpoint{2.566048in}{4.404497in}}%
\pgfpathlineto{\pgfqpoint{2.568624in}{4.401962in}}%
\pgfpathlineto{\pgfqpoint{2.571845in}{4.395178in}}%
\pgfpathlineto{\pgfqpoint{2.575924in}{4.380873in}}%
\pgfpathlineto{\pgfqpoint{2.581077in}{4.353872in}}%
\pgfpathlineto{\pgfqpoint{2.587304in}{4.308462in}}%
\pgfpathlineto{\pgfqpoint{2.594819in}{4.236315in}}%
\pgfpathlineto{\pgfqpoint{2.604051in}{4.124803in}}%
\pgfpathlineto{\pgfqpoint{2.615860in}{3.953207in}}%
\pgfpathlineto{\pgfqpoint{2.635614in}{3.627480in}}%
\pgfpathlineto{\pgfqpoint{2.656011in}{3.303271in}}%
\pgfpathlineto{\pgfqpoint{2.667605in}{3.152099in}}%
\pgfpathlineto{\pgfqpoint{2.676838in}{3.058580in}}%
\pgfpathlineto{\pgfqpoint{2.684352in}{3.003376in}}%
\pgfpathlineto{\pgfqpoint{2.690364in}{2.973995in}}%
\pgfpathlineto{\pgfqpoint{2.695088in}{2.960557in}}%
\pgfpathlineto{\pgfqpoint{2.698523in}{2.956227in}}%
\pgfpathlineto{\pgfqpoint{2.701100in}{2.956012in}}%
\pgfpathlineto{\pgfqpoint{2.703676in}{2.958397in}}%
\pgfpathlineto{\pgfqpoint{2.706897in}{2.965024in}}%
\pgfpathlineto{\pgfqpoint{2.710976in}{2.979177in}}%
\pgfpathlineto{\pgfqpoint{2.716129in}{3.006065in}}%
\pgfpathlineto{\pgfqpoint{2.722356in}{3.051459in}}%
\pgfpathlineto{\pgfqpoint{2.729871in}{3.123760in}}%
\pgfpathlineto{\pgfqpoint{2.739103in}{3.235718in}}%
\pgfpathlineto{\pgfqpoint{2.750912in}{3.408265in}}%
\pgfpathlineto{\pgfqpoint{2.770236in}{3.729065in}}%
\pgfpathlineto{\pgfqpoint{2.791063in}{4.063510in}}%
\pgfpathlineto{\pgfqpoint{2.802872in}{4.218860in}}%
\pgfpathlineto{\pgfqpoint{2.812104in}{4.312987in}}%
\pgfpathlineto{\pgfqpoint{2.819619in}{4.368510in}}%
\pgfpathlineto{\pgfqpoint{2.825631in}{4.398025in}}%
\pgfpathlineto{\pgfqpoint{2.830355in}{4.411491in}}%
\pgfpathlineto{\pgfqpoint{2.833790in}{4.415797in}}%
\pgfpathlineto{\pgfqpoint{2.836366in}{4.415972in}}%
\pgfpathlineto{\pgfqpoint{2.838943in}{4.413526in}}%
\pgfpathlineto{\pgfqpoint{2.842164in}{4.406795in}}%
\pgfpathlineto{\pgfqpoint{2.846243in}{4.392466in}}%
\pgfpathlineto{\pgfqpoint{2.851396in}{4.365287in}}%
\pgfpathlineto{\pgfqpoint{2.857623in}{4.319447in}}%
\pgfpathlineto{\pgfqpoint{2.865137in}{4.246480in}}%
\pgfpathlineto{\pgfqpoint{2.874370in}{4.133541in}}%
\pgfpathlineto{\pgfqpoint{2.886179in}{3.959546in}}%
\pgfpathlineto{\pgfqpoint{2.905717in}{3.632509in}}%
\pgfpathlineto{\pgfqpoint{2.926329in}{3.299198in}}%
\pgfpathlineto{\pgfqpoint{2.938138in}{3.142743in}}%
\pgfpathlineto{\pgfqpoint{2.947371in}{3.047988in}}%
\pgfpathlineto{\pgfqpoint{2.954886in}{2.992132in}}%
\pgfpathlineto{\pgfqpoint{2.960898in}{2.962471in}}%
\pgfpathlineto{\pgfqpoint{2.965621in}{2.948968in}}%
\pgfpathlineto{\pgfqpoint{2.969057in}{2.944678in}}%
\pgfpathlineto{\pgfqpoint{2.971633in}{2.944539in}}%
\pgfpathlineto{\pgfqpoint{2.974210in}{2.947042in}}%
\pgfpathlineto{\pgfqpoint{2.977430in}{2.953873in}}%
\pgfpathlineto{\pgfqpoint{2.981510in}{2.968371in}}%
\pgfpathlineto{\pgfqpoint{2.986663in}{2.995833in}}%
\pgfpathlineto{\pgfqpoint{2.992889in}{3.042112in}}%
\pgfpathlineto{\pgfqpoint{3.000404in}{3.115739in}}%
\pgfpathlineto{\pgfqpoint{3.009636in}{3.229655in}}%
\pgfpathlineto{\pgfqpoint{3.021445in}{3.405098in}}%
\pgfpathlineto{\pgfqpoint{3.040984in}{3.734748in}}%
\pgfpathlineto{\pgfqpoint{3.061596in}{4.070593in}}%
\pgfpathlineto{\pgfqpoint{3.073405in}{4.228177in}}%
\pgfpathlineto{\pgfqpoint{3.082638in}{4.323577in}}%
\pgfpathlineto{\pgfqpoint{3.089508in}{4.375773in}}%
\pgfpathlineto{\pgfqpoint{3.089508in}{4.375773in}}%
\pgfusepath{stroke}%
\end{pgfscope}%
\begin{pgfscope}%
\pgfsetrectcap%
\pgfsetmiterjoin%
\pgfsetlinewidth{0.803000pt}%
\definecolor{currentstroke}{rgb}{0.000000,0.000000,0.000000}%
\pgfsetstrokecolor{currentstroke}%
\pgfsetdash{}{0pt}%
\pgfpathmoveto{\pgfqpoint{0.835065in}{2.870679in}}%
\pgfpathlineto{\pgfqpoint{0.835065in}{4.489815in}}%
\pgfusepath{stroke}%
\end{pgfscope}%
\begin{pgfscope}%
\pgfsetrectcap%
\pgfsetmiterjoin%
\pgfsetlinewidth{0.803000pt}%
\definecolor{currentstroke}{rgb}{0.000000,0.000000,0.000000}%
\pgfsetstrokecolor{currentstroke}%
\pgfsetdash{}{0pt}%
\pgfpathmoveto{\pgfqpoint{3.196863in}{2.870679in}}%
\pgfpathlineto{\pgfqpoint{3.196863in}{4.489815in}}%
\pgfusepath{stroke}%
\end{pgfscope}%
\begin{pgfscope}%
\pgfsetrectcap%
\pgfsetmiterjoin%
\pgfsetlinewidth{0.803000pt}%
\definecolor{currentstroke}{rgb}{0.000000,0.000000,0.000000}%
\pgfsetstrokecolor{currentstroke}%
\pgfsetdash{}{0pt}%
\pgfpathmoveto{\pgfqpoint{0.835065in}{2.870679in}}%
\pgfpathlineto{\pgfqpoint{3.196863in}{2.870679in}}%
\pgfusepath{stroke}%
\end{pgfscope}%
\begin{pgfscope}%
\pgfsetrectcap%
\pgfsetmiterjoin%
\pgfsetlinewidth{0.803000pt}%
\definecolor{currentstroke}{rgb}{0.000000,0.000000,0.000000}%
\pgfsetstrokecolor{currentstroke}%
\pgfsetdash{}{0pt}%
\pgfpathmoveto{\pgfqpoint{0.835065in}{4.489815in}}%
\pgfpathlineto{\pgfqpoint{3.196863in}{4.489815in}}%
\pgfusepath{stroke}%
\end{pgfscope}%
\begin{pgfscope}%
\definecolor{textcolor}{rgb}{0.000000,0.000000,0.000000}%
\pgfsetstrokecolor{textcolor}%
\pgfsetfillcolor{textcolor}%
\pgftext[x=2.015964in,y=4.573148in,,base]{\color{textcolor}\rmfamily\fontsize{12.000000}{14.400000}\selectfont \(\displaystyle \theta\)}%
\end{pgfscope}%
\begin{pgfscope}%
\pgfsetbuttcap%
\pgfsetmiterjoin%
\definecolor{currentfill}{rgb}{1.000000,1.000000,1.000000}%
\pgfsetfillcolor{currentfill}%
\pgfsetlinewidth{0.000000pt}%
\definecolor{currentstroke}{rgb}{0.000000,0.000000,0.000000}%
\pgfsetstrokecolor{currentstroke}%
\pgfsetstrokeopacity{0.000000}%
\pgfsetdash{}{0pt}%
\pgfpathmoveto{\pgfqpoint{3.906113in}{2.870679in}}%
\pgfpathlineto{\pgfqpoint{6.267911in}{2.870679in}}%
\pgfpathlineto{\pgfqpoint{6.267911in}{4.489815in}}%
\pgfpathlineto{\pgfqpoint{3.906113in}{4.489815in}}%
\pgfpathclose%
\pgfusepath{fill}%
\end{pgfscope}%
\begin{pgfscope}%
\pgfsetbuttcap%
\pgfsetroundjoin%
\definecolor{currentfill}{rgb}{0.000000,0.000000,0.000000}%
\pgfsetfillcolor{currentfill}%
\pgfsetlinewidth{0.803000pt}%
\definecolor{currentstroke}{rgb}{0.000000,0.000000,0.000000}%
\pgfsetstrokecolor{currentstroke}%
\pgfsetdash{}{0pt}%
\pgfsys@defobject{currentmarker}{\pgfqpoint{0.000000in}{-0.048611in}}{\pgfqpoint{0.000000in}{0.000000in}}{%
\pgfpathmoveto{\pgfqpoint{0.000000in}{0.000000in}}%
\pgfpathlineto{\pgfqpoint{0.000000in}{-0.048611in}}%
\pgfusepath{stroke,fill}%
}%
\begin{pgfscope}%
\pgfsys@transformshift{4.013467in}{2.870679in}%
\pgfsys@useobject{currentmarker}{}%
\end{pgfscope}%
\end{pgfscope}%
\begin{pgfscope}%
\definecolor{textcolor}{rgb}{0.000000,0.000000,0.000000}%
\pgfsetstrokecolor{textcolor}%
\pgfsetfillcolor{textcolor}%
\pgftext[x=4.013467in,y=2.773457in,,top]{\color{textcolor}\rmfamily\fontsize{10.000000}{12.000000}\selectfont \(\displaystyle 0.0\)}%
\end{pgfscope}%
\begin{pgfscope}%
\pgfsetbuttcap%
\pgfsetroundjoin%
\definecolor{currentfill}{rgb}{0.000000,0.000000,0.000000}%
\pgfsetfillcolor{currentfill}%
\pgfsetlinewidth{0.803000pt}%
\definecolor{currentstroke}{rgb}{0.000000,0.000000,0.000000}%
\pgfsetstrokecolor{currentstroke}%
\pgfsetdash{}{0pt}%
\pgfsys@defobject{currentmarker}{\pgfqpoint{0.000000in}{-0.048611in}}{\pgfqpoint{0.000000in}{0.000000in}}{%
\pgfpathmoveto{\pgfqpoint{0.000000in}{0.000000in}}%
\pgfpathlineto{\pgfqpoint{0.000000in}{-0.048611in}}%
\pgfusepath{stroke,fill}%
}%
\begin{pgfscope}%
\pgfsys@transformshift{4.550240in}{2.870679in}%
\pgfsys@useobject{currentmarker}{}%
\end{pgfscope}%
\end{pgfscope}%
\begin{pgfscope}%
\definecolor{textcolor}{rgb}{0.000000,0.000000,0.000000}%
\pgfsetstrokecolor{textcolor}%
\pgfsetfillcolor{textcolor}%
\pgftext[x=4.550240in,y=2.773457in,,top]{\color{textcolor}\rmfamily\fontsize{10.000000}{12.000000}\selectfont \(\displaystyle 2.5\)}%
\end{pgfscope}%
\begin{pgfscope}%
\pgfsetbuttcap%
\pgfsetroundjoin%
\definecolor{currentfill}{rgb}{0.000000,0.000000,0.000000}%
\pgfsetfillcolor{currentfill}%
\pgfsetlinewidth{0.803000pt}%
\definecolor{currentstroke}{rgb}{0.000000,0.000000,0.000000}%
\pgfsetstrokecolor{currentstroke}%
\pgfsetdash{}{0pt}%
\pgfsys@defobject{currentmarker}{\pgfqpoint{0.000000in}{-0.048611in}}{\pgfqpoint{0.000000in}{0.000000in}}{%
\pgfpathmoveto{\pgfqpoint{0.000000in}{0.000000in}}%
\pgfpathlineto{\pgfqpoint{0.000000in}{-0.048611in}}%
\pgfusepath{stroke,fill}%
}%
\begin{pgfscope}%
\pgfsys@transformshift{5.087012in}{2.870679in}%
\pgfsys@useobject{currentmarker}{}%
\end{pgfscope}%
\end{pgfscope}%
\begin{pgfscope}%
\definecolor{textcolor}{rgb}{0.000000,0.000000,0.000000}%
\pgfsetstrokecolor{textcolor}%
\pgfsetfillcolor{textcolor}%
\pgftext[x=5.087012in,y=2.773457in,,top]{\color{textcolor}\rmfamily\fontsize{10.000000}{12.000000}\selectfont \(\displaystyle 5.0\)}%
\end{pgfscope}%
\begin{pgfscope}%
\pgfsetbuttcap%
\pgfsetroundjoin%
\definecolor{currentfill}{rgb}{0.000000,0.000000,0.000000}%
\pgfsetfillcolor{currentfill}%
\pgfsetlinewidth{0.803000pt}%
\definecolor{currentstroke}{rgb}{0.000000,0.000000,0.000000}%
\pgfsetstrokecolor{currentstroke}%
\pgfsetdash{}{0pt}%
\pgfsys@defobject{currentmarker}{\pgfqpoint{0.000000in}{-0.048611in}}{\pgfqpoint{0.000000in}{0.000000in}}{%
\pgfpathmoveto{\pgfqpoint{0.000000in}{0.000000in}}%
\pgfpathlineto{\pgfqpoint{0.000000in}{-0.048611in}}%
\pgfusepath{stroke,fill}%
}%
\begin{pgfscope}%
\pgfsys@transformshift{5.623784in}{2.870679in}%
\pgfsys@useobject{currentmarker}{}%
\end{pgfscope}%
\end{pgfscope}%
\begin{pgfscope}%
\definecolor{textcolor}{rgb}{0.000000,0.000000,0.000000}%
\pgfsetstrokecolor{textcolor}%
\pgfsetfillcolor{textcolor}%
\pgftext[x=5.623784in,y=2.773457in,,top]{\color{textcolor}\rmfamily\fontsize{10.000000}{12.000000}\selectfont \(\displaystyle 7.5\)}%
\end{pgfscope}%
\begin{pgfscope}%
\pgfsetbuttcap%
\pgfsetroundjoin%
\definecolor{currentfill}{rgb}{0.000000,0.000000,0.000000}%
\pgfsetfillcolor{currentfill}%
\pgfsetlinewidth{0.803000pt}%
\definecolor{currentstroke}{rgb}{0.000000,0.000000,0.000000}%
\pgfsetstrokecolor{currentstroke}%
\pgfsetdash{}{0pt}%
\pgfsys@defobject{currentmarker}{\pgfqpoint{0.000000in}{-0.048611in}}{\pgfqpoint{0.000000in}{0.000000in}}{%
\pgfpathmoveto{\pgfqpoint{0.000000in}{0.000000in}}%
\pgfpathlineto{\pgfqpoint{0.000000in}{-0.048611in}}%
\pgfusepath{stroke,fill}%
}%
\begin{pgfscope}%
\pgfsys@transformshift{6.160557in}{2.870679in}%
\pgfsys@useobject{currentmarker}{}%
\end{pgfscope}%
\end{pgfscope}%
\begin{pgfscope}%
\definecolor{textcolor}{rgb}{0.000000,0.000000,0.000000}%
\pgfsetstrokecolor{textcolor}%
\pgfsetfillcolor{textcolor}%
\pgftext[x=6.160557in,y=2.773457in,,top]{\color{textcolor}\rmfamily\fontsize{10.000000}{12.000000}\selectfont \(\displaystyle 10.0\)}%
\end{pgfscope}%
\begin{pgfscope}%
\definecolor{textcolor}{rgb}{0.000000,0.000000,0.000000}%
\pgfsetstrokecolor{textcolor}%
\pgfsetfillcolor{textcolor}%
\pgftext[x=5.087012in,y=2.594444in,,top]{\color{textcolor}\rmfamily\fontsize{10.000000}{12.000000}\selectfont time (s)}%
\end{pgfscope}%
\begin{pgfscope}%
\pgfsetbuttcap%
\pgfsetroundjoin%
\definecolor{currentfill}{rgb}{0.000000,0.000000,0.000000}%
\pgfsetfillcolor{currentfill}%
\pgfsetlinewidth{0.803000pt}%
\definecolor{currentstroke}{rgb}{0.000000,0.000000,0.000000}%
\pgfsetstrokecolor{currentstroke}%
\pgfsetdash{}{0pt}%
\pgfsys@defobject{currentmarker}{\pgfqpoint{-0.048611in}{0.000000in}}{\pgfqpoint{0.000000in}{0.000000in}}{%
\pgfpathmoveto{\pgfqpoint{0.000000in}{0.000000in}}%
\pgfpathlineto{\pgfqpoint{-0.048611in}{0.000000in}}%
\pgfusepath{stroke,fill}%
}%
\begin{pgfscope}%
\pgfsys@transformshift{3.906113in}{2.926503in}%
\pgfsys@useobject{currentmarker}{}%
\end{pgfscope}%
\end{pgfscope}%
\begin{pgfscope}%
\definecolor{textcolor}{rgb}{0.000000,0.000000,0.000000}%
\pgfsetstrokecolor{textcolor}%
\pgfsetfillcolor{textcolor}%
\pgftext[x=3.523396in,y=2.878278in,left,base]{\color{textcolor}\rmfamily\fontsize{10.000000}{12.000000}\selectfont \(\displaystyle -1.0\)}%
\end{pgfscope}%
\begin{pgfscope}%
\pgfsetbuttcap%
\pgfsetroundjoin%
\definecolor{currentfill}{rgb}{0.000000,0.000000,0.000000}%
\pgfsetfillcolor{currentfill}%
\pgfsetlinewidth{0.803000pt}%
\definecolor{currentstroke}{rgb}{0.000000,0.000000,0.000000}%
\pgfsetstrokecolor{currentstroke}%
\pgfsetdash{}{0pt}%
\pgfsys@defobject{currentmarker}{\pgfqpoint{-0.048611in}{0.000000in}}{\pgfqpoint{0.000000in}{0.000000in}}{%
\pgfpathmoveto{\pgfqpoint{0.000000in}{0.000000in}}%
\pgfpathlineto{\pgfqpoint{-0.048611in}{0.000000in}}%
\pgfusepath{stroke,fill}%
}%
\begin{pgfscope}%
\pgfsys@transformshift{3.906113in}{3.301936in}%
\pgfsys@useobject{currentmarker}{}%
\end{pgfscope}%
\end{pgfscope}%
\begin{pgfscope}%
\definecolor{textcolor}{rgb}{0.000000,0.000000,0.000000}%
\pgfsetstrokecolor{textcolor}%
\pgfsetfillcolor{textcolor}%
\pgftext[x=3.523396in,y=3.253710in,left,base]{\color{textcolor}\rmfamily\fontsize{10.000000}{12.000000}\selectfont \(\displaystyle -0.5\)}%
\end{pgfscope}%
\begin{pgfscope}%
\pgfsetbuttcap%
\pgfsetroundjoin%
\definecolor{currentfill}{rgb}{0.000000,0.000000,0.000000}%
\pgfsetfillcolor{currentfill}%
\pgfsetlinewidth{0.803000pt}%
\definecolor{currentstroke}{rgb}{0.000000,0.000000,0.000000}%
\pgfsetstrokecolor{currentstroke}%
\pgfsetdash{}{0pt}%
\pgfsys@defobject{currentmarker}{\pgfqpoint{-0.048611in}{0.000000in}}{\pgfqpoint{0.000000in}{0.000000in}}{%
\pgfpathmoveto{\pgfqpoint{0.000000in}{0.000000in}}%
\pgfpathlineto{\pgfqpoint{-0.048611in}{0.000000in}}%
\pgfusepath{stroke,fill}%
}%
\begin{pgfscope}%
\pgfsys@transformshift{3.906113in}{3.677368in}%
\pgfsys@useobject{currentmarker}{}%
\end{pgfscope}%
\end{pgfscope}%
\begin{pgfscope}%
\definecolor{textcolor}{rgb}{0.000000,0.000000,0.000000}%
\pgfsetstrokecolor{textcolor}%
\pgfsetfillcolor{textcolor}%
\pgftext[x=3.631421in,y=3.629143in,left,base]{\color{textcolor}\rmfamily\fontsize{10.000000}{12.000000}\selectfont \(\displaystyle 0.0\)}%
\end{pgfscope}%
\begin{pgfscope}%
\pgfsetbuttcap%
\pgfsetroundjoin%
\definecolor{currentfill}{rgb}{0.000000,0.000000,0.000000}%
\pgfsetfillcolor{currentfill}%
\pgfsetlinewidth{0.803000pt}%
\definecolor{currentstroke}{rgb}{0.000000,0.000000,0.000000}%
\pgfsetstrokecolor{currentstroke}%
\pgfsetdash{}{0pt}%
\pgfsys@defobject{currentmarker}{\pgfqpoint{-0.048611in}{0.000000in}}{\pgfqpoint{0.000000in}{0.000000in}}{%
\pgfpathmoveto{\pgfqpoint{0.000000in}{0.000000in}}%
\pgfpathlineto{\pgfqpoint{-0.048611in}{0.000000in}}%
\pgfusepath{stroke,fill}%
}%
\begin{pgfscope}%
\pgfsys@transformshift{3.906113in}{4.052800in}%
\pgfsys@useobject{currentmarker}{}%
\end{pgfscope}%
\end{pgfscope}%
\begin{pgfscope}%
\definecolor{textcolor}{rgb}{0.000000,0.000000,0.000000}%
\pgfsetstrokecolor{textcolor}%
\pgfsetfillcolor{textcolor}%
\pgftext[x=3.631421in,y=4.004575in,left,base]{\color{textcolor}\rmfamily\fontsize{10.000000}{12.000000}\selectfont \(\displaystyle 0.5\)}%
\end{pgfscope}%
\begin{pgfscope}%
\pgfsetbuttcap%
\pgfsetroundjoin%
\definecolor{currentfill}{rgb}{0.000000,0.000000,0.000000}%
\pgfsetfillcolor{currentfill}%
\pgfsetlinewidth{0.803000pt}%
\definecolor{currentstroke}{rgb}{0.000000,0.000000,0.000000}%
\pgfsetstrokecolor{currentstroke}%
\pgfsetdash{}{0pt}%
\pgfsys@defobject{currentmarker}{\pgfqpoint{-0.048611in}{0.000000in}}{\pgfqpoint{0.000000in}{0.000000in}}{%
\pgfpathmoveto{\pgfqpoint{0.000000in}{0.000000in}}%
\pgfpathlineto{\pgfqpoint{-0.048611in}{0.000000in}}%
\pgfusepath{stroke,fill}%
}%
\begin{pgfscope}%
\pgfsys@transformshift{3.906113in}{4.428232in}%
\pgfsys@useobject{currentmarker}{}%
\end{pgfscope}%
\end{pgfscope}%
\begin{pgfscope}%
\definecolor{textcolor}{rgb}{0.000000,0.000000,0.000000}%
\pgfsetstrokecolor{textcolor}%
\pgfsetfillcolor{textcolor}%
\pgftext[x=3.631421in,y=4.380007in,left,base]{\color{textcolor}\rmfamily\fontsize{10.000000}{12.000000}\selectfont \(\displaystyle 1.0\)}%
\end{pgfscope}%
\begin{pgfscope}%
\definecolor{textcolor}{rgb}{0.000000,0.000000,0.000000}%
\pgfsetstrokecolor{textcolor}%
\pgfsetfillcolor{textcolor}%
\pgftext[x=3.467840in,y=3.680247in,,bottom,rotate=90.000000]{\color{textcolor}\rmfamily\fontsize{10.000000}{12.000000}\selectfont angle (rad)}%
\end{pgfscope}%
\begin{pgfscope}%
\pgfpathrectangle{\pgfqpoint{3.906113in}{2.870679in}}{\pgfqpoint{2.361798in}{1.619136in}}%
\pgfusepath{clip}%
\pgfsetrectcap%
\pgfsetroundjoin%
\pgfsetlinewidth{1.505625pt}%
\definecolor{currentstroke}{rgb}{0.000000,0.000000,1.000000}%
\pgfsetstrokecolor{currentstroke}%
\pgfsetdash{}{0pt}%
\pgfpathmoveto{\pgfqpoint{4.013467in}{3.677368in}}%
\pgfpathlineto{\pgfqpoint{4.036871in}{3.338705in}}%
\pgfpathlineto{\pgfqpoint{4.048680in}{3.199048in}}%
\pgfpathlineto{\pgfqpoint{4.057912in}{3.114157in}}%
\pgfpathlineto{\pgfqpoint{4.065427in}{3.063925in}}%
\pgfpathlineto{\pgfqpoint{4.071439in}{3.037122in}}%
\pgfpathlineto{\pgfqpoint{4.076162in}{3.024815in}}%
\pgfpathlineto{\pgfqpoint{4.079813in}{3.020699in}}%
\pgfpathlineto{\pgfqpoint{4.082389in}{3.020647in}}%
\pgfpathlineto{\pgfqpoint{4.084966in}{3.022959in}}%
\pgfpathlineto{\pgfqpoint{4.088186in}{3.029160in}}%
\pgfpathlineto{\pgfqpoint{4.092266in}{3.042242in}}%
\pgfpathlineto{\pgfqpoint{4.097419in}{3.066936in}}%
\pgfpathlineto{\pgfqpoint{4.103645in}{3.108443in}}%
\pgfpathlineto{\pgfqpoint{4.111375in}{3.176444in}}%
\pgfpathlineto{\pgfqpoint{4.120822in}{3.281224in}}%
\pgfpathlineto{\pgfqpoint{4.132846in}{3.441352in}}%
\pgfpathlineto{\pgfqpoint{4.153887in}{3.757575in}}%
\pgfpathlineto{\pgfqpoint{4.172996in}{4.031208in}}%
\pgfpathlineto{\pgfqpoint{4.184590in}{4.167206in}}%
\pgfpathlineto{\pgfqpoint{4.193608in}{4.249149in}}%
\pgfpathlineto{\pgfqpoint{4.200908in}{4.297345in}}%
\pgfpathlineto{\pgfqpoint{4.206920in}{4.323713in}}%
\pgfpathlineto{\pgfqpoint{4.211644in}{4.335594in}}%
\pgfpathlineto{\pgfqpoint{4.215079in}{4.339245in}}%
\pgfpathlineto{\pgfqpoint{4.217656in}{4.339206in}}%
\pgfpathlineto{\pgfqpoint{4.220232in}{4.336785in}}%
\pgfpathlineto{\pgfqpoint{4.223453in}{4.330423in}}%
\pgfpathlineto{\pgfqpoint{4.227532in}{4.317097in}}%
\pgfpathlineto{\pgfqpoint{4.232685in}{4.292036in}}%
\pgfpathlineto{\pgfqpoint{4.238912in}{4.250005in}}%
\pgfpathlineto{\pgfqpoint{4.246641in}{4.181249in}}%
\pgfpathlineto{\pgfqpoint{4.256089in}{4.075421in}}%
\pgfpathlineto{\pgfqpoint{4.268327in}{3.910746in}}%
\pgfpathlineto{\pgfqpoint{4.290227in}{3.578647in}}%
\pgfpathlineto{\pgfqpoint{4.308692in}{3.313915in}}%
\pgfpathlineto{\pgfqpoint{4.320072in}{3.180466in}}%
\pgfpathlineto{\pgfqpoint{4.329090in}{3.098688in}}%
\pgfpathlineto{\pgfqpoint{4.336390in}{3.050848in}}%
\pgfpathlineto{\pgfqpoint{4.342402in}{3.024915in}}%
\pgfpathlineto{\pgfqpoint{4.346910in}{3.013809in}}%
\pgfpathlineto{\pgfqpoint{4.350346in}{3.010246in}}%
\pgfpathlineto{\pgfqpoint{4.352922in}{3.010373in}}%
\pgfpathlineto{\pgfqpoint{4.355499in}{3.012901in}}%
\pgfpathlineto{\pgfqpoint{4.358934in}{3.019990in}}%
\pgfpathlineto{\pgfqpoint{4.363228in}{3.034759in}}%
\pgfpathlineto{\pgfqpoint{4.368596in}{3.062227in}}%
\pgfpathlineto{\pgfqpoint{4.375037in}{3.107807in}}%
\pgfpathlineto{\pgfqpoint{4.382981in}{3.181480in}}%
\pgfpathlineto{\pgfqpoint{4.392643in}{3.293540in}}%
\pgfpathlineto{\pgfqpoint{4.405311in}{3.468668in}}%
\pgfpathlineto{\pgfqpoint{4.452332in}{4.146671in}}%
\pgfpathlineto{\pgfqpoint{4.461994in}{4.242107in}}%
\pgfpathlineto{\pgfqpoint{4.469724in}{4.298309in}}%
\pgfpathlineto{\pgfqpoint{4.475950in}{4.329124in}}%
\pgfpathlineto{\pgfqpoint{4.480889in}{4.343883in}}%
\pgfpathlineto{\pgfqpoint{4.484539in}{4.349145in}}%
\pgfpathlineto{\pgfqpoint{4.487330in}{4.349897in}}%
\pgfpathlineto{\pgfqpoint{4.489907in}{4.348071in}}%
\pgfpathlineto{\pgfqpoint{4.492912in}{4.342889in}}%
\pgfpathlineto{\pgfqpoint{4.496777in}{4.331441in}}%
\pgfpathlineto{\pgfqpoint{4.501716in}{4.309123in}}%
\pgfpathlineto{\pgfqpoint{4.507727in}{4.270729in}}%
\pgfpathlineto{\pgfqpoint{4.515027in}{4.208602in}}%
\pgfpathlineto{\pgfqpoint{4.524045in}{4.110993in}}%
\pgfpathlineto{\pgfqpoint{4.535425in}{3.961341in}}%
\pgfpathlineto{\pgfqpoint{4.552602in}{3.701713in}}%
\pgfpathlineto{\pgfqpoint{4.577722in}{3.325812in}}%
\pgfpathlineto{\pgfqpoint{4.589531in}{3.182152in}}%
\pgfpathlineto{\pgfqpoint{4.598764in}{3.094984in}}%
\pgfpathlineto{\pgfqpoint{4.606279in}{3.043538in}}%
\pgfpathlineto{\pgfqpoint{4.612291in}{3.016208in}}%
\pgfpathlineto{\pgfqpoint{4.617014in}{3.003774in}}%
\pgfpathlineto{\pgfqpoint{4.620450in}{2.999835in}}%
\pgfpathlineto{\pgfqpoint{4.623026in}{2.999725in}}%
\pgfpathlineto{\pgfqpoint{4.625603in}{3.002053in}}%
\pgfpathlineto{\pgfqpoint{4.628823in}{3.008380in}}%
\pgfpathlineto{\pgfqpoint{4.632903in}{3.021790in}}%
\pgfpathlineto{\pgfqpoint{4.638056in}{3.047160in}}%
\pgfpathlineto{\pgfqpoint{4.644282in}{3.089861in}}%
\pgfpathlineto{\pgfqpoint{4.651797in}{3.157696in}}%
\pgfpathlineto{\pgfqpoint{4.661030in}{3.262453in}}%
\pgfpathlineto{\pgfqpoint{4.673053in}{3.426582in}}%
\pgfpathlineto{\pgfqpoint{4.693236in}{3.738796in}}%
\pgfpathlineto{\pgfqpoint{4.713204in}{4.035700in}}%
\pgfpathlineto{\pgfqpoint{4.725013in}{4.179676in}}%
\pgfpathlineto{\pgfqpoint{4.734245in}{4.266724in}}%
\pgfpathlineto{\pgfqpoint{4.741545in}{4.316638in}}%
\pgfpathlineto{\pgfqpoint{4.747557in}{4.344010in}}%
\pgfpathlineto{\pgfqpoint{4.752281in}{4.356401in}}%
\pgfpathlineto{\pgfqpoint{4.755716in}{4.360267in}}%
\pgfpathlineto{\pgfqpoint{4.758293in}{4.360301in}}%
\pgfpathlineto{\pgfqpoint{4.760869in}{4.357877in}}%
\pgfpathlineto{\pgfqpoint{4.764090in}{4.351403in}}%
\pgfpathlineto{\pgfqpoint{4.768169in}{4.337768in}}%
\pgfpathlineto{\pgfqpoint{4.773322in}{4.312051in}}%
\pgfpathlineto{\pgfqpoint{4.779549in}{4.268846in}}%
\pgfpathlineto{\pgfqpoint{4.787278in}{4.198090in}}%
\pgfpathlineto{\pgfqpoint{4.796726in}{4.089099in}}%
\pgfpathlineto{\pgfqpoint{4.808749in}{3.922581in}}%
\pgfpathlineto{\pgfqpoint{4.829791in}{3.593818in}}%
\pgfpathlineto{\pgfqpoint{4.848900in}{3.309352in}}%
\pgfpathlineto{\pgfqpoint{4.860494in}{3.167957in}}%
\pgfpathlineto{\pgfqpoint{4.869512in}{3.082754in}}%
\pgfpathlineto{\pgfqpoint{4.876812in}{3.032637in}}%
\pgfpathlineto{\pgfqpoint{4.882824in}{3.005217in}}%
\pgfpathlineto{\pgfqpoint{4.887547in}{2.992861in}}%
\pgfpathlineto{\pgfqpoint{4.890983in}{2.989064in}}%
\pgfpathlineto{\pgfqpoint{4.893559in}{2.989104in}}%
\pgfpathlineto{\pgfqpoint{4.896136in}{2.991622in}}%
\pgfpathlineto{\pgfqpoint{4.899356in}{2.998238in}}%
\pgfpathlineto{\pgfqpoint{4.903436in}{3.012095in}}%
\pgfpathlineto{\pgfqpoint{4.908589in}{3.038155in}}%
\pgfpathlineto{\pgfqpoint{4.914815in}{3.081860in}}%
\pgfpathlineto{\pgfqpoint{4.922545in}{3.153351in}}%
\pgfpathlineto{\pgfqpoint{4.931992in}{3.263380in}}%
\pgfpathlineto{\pgfqpoint{4.944231in}{3.434578in}}%
\pgfpathlineto{\pgfqpoint{4.965916in}{3.776415in}}%
\pgfpathlineto{\pgfqpoint{4.984381in}{4.052185in}}%
\pgfpathlineto{\pgfqpoint{4.995975in}{4.193890in}}%
\pgfpathlineto{\pgfqpoint{5.004993in}{4.278980in}}%
\pgfpathlineto{\pgfqpoint{5.012293in}{4.328781in}}%
\pgfpathlineto{\pgfqpoint{5.018305in}{4.355795in}}%
\pgfpathlineto{\pgfqpoint{5.022814in}{4.367381in}}%
\pgfpathlineto{\pgfqpoint{5.026249in}{4.371115in}}%
\pgfpathlineto{\pgfqpoint{5.028826in}{4.371004in}}%
\pgfpathlineto{\pgfqpoint{5.031402in}{4.368396in}}%
\pgfpathlineto{\pgfqpoint{5.034623in}{4.361640in}}%
\pgfpathlineto{\pgfqpoint{5.038917in}{4.346656in}}%
\pgfpathlineto{\pgfqpoint{5.044070in}{4.319863in}}%
\pgfpathlineto{\pgfqpoint{5.050512in}{4.273443in}}%
\pgfpathlineto{\pgfqpoint{5.058241in}{4.200209in}}%
\pgfpathlineto{\pgfqpoint{5.067688in}{4.088112in}}%
\pgfpathlineto{\pgfqpoint{5.079927in}{3.914485in}}%
\pgfpathlineto{\pgfqpoint{5.103115in}{3.545617in}}%
\pgfpathlineto{\pgfqpoint{5.120507in}{3.287038in}}%
\pgfpathlineto{\pgfqpoint{5.131886in}{3.149131in}}%
\pgfpathlineto{\pgfqpoint{5.140689in}{3.067035in}}%
\pgfpathlineto{\pgfqpoint{5.147989in}{3.018087in}}%
\pgfpathlineto{\pgfqpoint{5.153787in}{2.992653in}}%
\pgfpathlineto{\pgfqpoint{5.158295in}{2.981453in}}%
\pgfpathlineto{\pgfqpoint{5.161731in}{2.978058in}}%
\pgfpathlineto{\pgfqpoint{5.164307in}{2.978446in}}%
\pgfpathlineto{\pgfqpoint{5.166884in}{2.981350in}}%
\pgfpathlineto{\pgfqpoint{5.170319in}{2.989116in}}%
\pgfpathlineto{\pgfqpoint{5.174613in}{3.005009in}}%
\pgfpathlineto{\pgfqpoint{5.179981in}{3.034297in}}%
\pgfpathlineto{\pgfqpoint{5.186422in}{3.082630in}}%
\pgfpathlineto{\pgfqpoint{5.194366in}{3.160458in}}%
\pgfpathlineto{\pgfqpoint{5.204028in}{3.278503in}}%
\pgfpathlineto{\pgfqpoint{5.216911in}{3.465866in}}%
\pgfpathlineto{\pgfqpoint{5.262859in}{4.161902in}}%
\pgfpathlineto{\pgfqpoint{5.272521in}{4.263658in}}%
\pgfpathlineto{\pgfqpoint{5.280465in}{4.325563in}}%
\pgfpathlineto{\pgfqpoint{5.286906in}{4.359673in}}%
\pgfpathlineto{\pgfqpoint{5.291844in}{4.375502in}}%
\pgfpathlineto{\pgfqpoint{5.295709in}{4.381470in}}%
\pgfpathlineto{\pgfqpoint{5.298500in}{4.382238in}}%
\pgfpathlineto{\pgfqpoint{5.300862in}{4.380563in}}%
\pgfpathlineto{\pgfqpoint{5.303868in}{4.375355in}}%
\pgfpathlineto{\pgfqpoint{5.307733in}{4.363640in}}%
\pgfpathlineto{\pgfqpoint{5.312671in}{4.340603in}}%
\pgfpathlineto{\pgfqpoint{5.318683in}{4.300775in}}%
\pgfpathlineto{\pgfqpoint{5.325983in}{4.236127in}}%
\pgfpathlineto{\pgfqpoint{5.334786in}{4.137008in}}%
\pgfpathlineto{\pgfqpoint{5.345951in}{3.984345in}}%
\pgfpathlineto{\pgfqpoint{5.362484in}{3.723880in}}%
\pgfpathlineto{\pgfqpoint{5.389107in}{3.305925in}}%
\pgfpathlineto{\pgfqpoint{5.400916in}{3.155986in}}%
\pgfpathlineto{\pgfqpoint{5.410149in}{3.065226in}}%
\pgfpathlineto{\pgfqpoint{5.417449in}{3.013099in}}%
\pgfpathlineto{\pgfqpoint{5.423461in}{2.984439in}}%
\pgfpathlineto{\pgfqpoint{5.428184in}{2.971393in}}%
\pgfpathlineto{\pgfqpoint{5.431620in}{2.967255in}}%
\pgfpathlineto{\pgfqpoint{5.434196in}{2.967131in}}%
\pgfpathlineto{\pgfqpoint{5.436773in}{2.969564in}}%
\pgfpathlineto{\pgfqpoint{5.439993in}{2.976186in}}%
\pgfpathlineto{\pgfqpoint{5.444073in}{2.990228in}}%
\pgfpathlineto{\pgfqpoint{5.449226in}{3.016802in}}%
\pgfpathlineto{\pgfqpoint{5.455453in}{3.061536in}}%
\pgfpathlineto{\pgfqpoint{5.462967in}{3.132603in}}%
\pgfpathlineto{\pgfqpoint{5.472200in}{3.242350in}}%
\pgfpathlineto{\pgfqpoint{5.484224in}{3.414289in}}%
\pgfpathlineto{\pgfqpoint{5.504406in}{3.741372in}}%
\pgfpathlineto{\pgfqpoint{5.524589in}{4.055534in}}%
\pgfpathlineto{\pgfqpoint{5.536183in}{4.203460in}}%
\pgfpathlineto{\pgfqpoint{5.545416in}{4.294777in}}%
\pgfpathlineto{\pgfqpoint{5.552716in}{4.347176in}}%
\pgfpathlineto{\pgfqpoint{5.558728in}{4.375940in}}%
\pgfpathlineto{\pgfqpoint{5.563451in}{4.388991in}}%
\pgfpathlineto{\pgfqpoint{5.566886in}{4.393090in}}%
\pgfpathlineto{\pgfqpoint{5.569463in}{4.393161in}}%
\pgfpathlineto{\pgfqpoint{5.572039in}{4.390655in}}%
\pgfpathlineto{\pgfqpoint{5.575260in}{4.383914in}}%
\pgfpathlineto{\pgfqpoint{5.579340in}{4.369678in}}%
\pgfpathlineto{\pgfqpoint{5.584493in}{4.342792in}}%
\pgfpathlineto{\pgfqpoint{5.590719in}{4.297591in}}%
\pgfpathlineto{\pgfqpoint{5.598234in}{4.225839in}}%
\pgfpathlineto{\pgfqpoint{5.607466in}{4.115101in}}%
\pgfpathlineto{\pgfqpoint{5.619490in}{3.941698in}}%
\pgfpathlineto{\pgfqpoint{5.639673in}{3.612009in}}%
\pgfpathlineto{\pgfqpoint{5.659641in}{3.298571in}}%
\pgfpathlineto{\pgfqpoint{5.671450in}{3.146579in}}%
\pgfpathlineto{\pgfqpoint{5.680682in}{3.054689in}}%
\pgfpathlineto{\pgfqpoint{5.687982in}{3.002005in}}%
\pgfpathlineto{\pgfqpoint{5.693994in}{2.973125in}}%
\pgfpathlineto{\pgfqpoint{5.698718in}{2.960061in}}%
\pgfpathlineto{\pgfqpoint{5.702153in}{2.955996in}}%
\pgfpathlineto{\pgfqpoint{5.704730in}{2.955975in}}%
\pgfpathlineto{\pgfqpoint{5.707306in}{2.958550in}}%
\pgfpathlineto{\pgfqpoint{5.710527in}{2.965405in}}%
\pgfpathlineto{\pgfqpoint{5.714606in}{2.979830in}}%
\pgfpathlineto{\pgfqpoint{5.719759in}{3.007020in}}%
\pgfpathlineto{\pgfqpoint{5.725986in}{3.052684in}}%
\pgfpathlineto{\pgfqpoint{5.733715in}{3.127443in}}%
\pgfpathlineto{\pgfqpoint{5.743162in}{3.242571in}}%
\pgfpathlineto{\pgfqpoint{5.755401in}{3.421786in}}%
\pgfpathlineto{\pgfqpoint{5.777086in}{3.779847in}}%
\pgfpathlineto{\pgfqpoint{5.795551in}{4.068977in}}%
\pgfpathlineto{\pgfqpoint{5.807146in}{4.217716in}}%
\pgfpathlineto{\pgfqpoint{5.816164in}{4.307142in}}%
\pgfpathlineto{\pgfqpoint{5.823464in}{4.359569in}}%
\pgfpathlineto{\pgfqpoint{5.829475in}{4.388089in}}%
\pgfpathlineto{\pgfqpoint{5.833984in}{4.400393in}}%
\pgfpathlineto{\pgfqpoint{5.837420in}{4.404430in}}%
\pgfpathlineto{\pgfqpoint{5.839996in}{4.404407in}}%
\pgfpathlineto{\pgfqpoint{5.842573in}{4.401767in}}%
\pgfpathlineto{\pgfqpoint{5.845793in}{4.394802in}}%
\pgfpathlineto{\pgfqpoint{5.849873in}{4.380194in}}%
\pgfpathlineto{\pgfqpoint{5.855026in}{4.352705in}}%
\pgfpathlineto{\pgfqpoint{5.861252in}{4.306586in}}%
\pgfpathlineto{\pgfqpoint{5.868982in}{4.231132in}}%
\pgfpathlineto{\pgfqpoint{5.878429in}{4.114993in}}%
\pgfpathlineto{\pgfqpoint{5.890668in}{3.934281in}}%
\pgfpathlineto{\pgfqpoint{5.912353in}{3.573391in}}%
\pgfpathlineto{\pgfqpoint{5.930818in}{3.282108in}}%
\pgfpathlineto{\pgfqpoint{5.942412in}{3.132324in}}%
\pgfpathlineto{\pgfqpoint{5.951430in}{3.042311in}}%
\pgfpathlineto{\pgfqpoint{5.958730in}{2.989575in}}%
\pgfpathlineto{\pgfqpoint{5.964742in}{2.960920in}}%
\pgfpathlineto{\pgfqpoint{5.969251in}{2.948590in}}%
\pgfpathlineto{\pgfqpoint{5.972686in}{2.944574in}}%
\pgfpathlineto{\pgfqpoint{5.975263in}{2.944638in}}%
\pgfpathlineto{\pgfqpoint{5.977839in}{2.947338in}}%
\pgfpathlineto{\pgfqpoint{5.981060in}{2.954408in}}%
\pgfpathlineto{\pgfqpoint{5.985139in}{2.969192in}}%
\pgfpathlineto{\pgfqpoint{5.990292in}{2.996973in}}%
\pgfpathlineto{\pgfqpoint{5.996519in}{3.043541in}}%
\pgfpathlineto{\pgfqpoint{6.004249in}{3.119683in}}%
\pgfpathlineto{\pgfqpoint{6.013696in}{3.236829in}}%
\pgfpathlineto{\pgfqpoint{6.025934in}{3.419039in}}%
\pgfpathlineto{\pgfqpoint{6.047834in}{3.786399in}}%
\pgfpathlineto{\pgfqpoint{6.066299in}{4.079326in}}%
\pgfpathlineto{\pgfqpoint{6.077679in}{4.227090in}}%
\pgfpathlineto{\pgfqpoint{6.086697in}{4.317708in}}%
\pgfpathlineto{\pgfqpoint{6.093997in}{4.370768in}}%
\pgfpathlineto{\pgfqpoint{6.100009in}{4.399568in}}%
\pgfpathlineto{\pgfqpoint{6.104518in}{4.411935in}}%
\pgfpathlineto{\pgfqpoint{6.107953in}{4.415935in}}%
\pgfpathlineto{\pgfqpoint{6.110315in}{4.415946in}}%
\pgfpathlineto{\pgfqpoint{6.112891in}{4.413410in}}%
\pgfpathlineto{\pgfqpoint{6.116112in}{4.406516in}}%
\pgfpathlineto{\pgfqpoint{6.120191in}{4.391904in}}%
\pgfpathlineto{\pgfqpoint{6.125344in}{4.364259in}}%
\pgfpathlineto{\pgfqpoint{6.131571in}{4.317731in}}%
\pgfpathlineto{\pgfqpoint{6.139086in}{4.243825in}}%
\pgfpathlineto{\pgfqpoint{6.148318in}{4.129710in}}%
\pgfpathlineto{\pgfqpoint{6.160342in}{3.950956in}}%
\pgfpathlineto{\pgfqpoint{6.160557in}{3.947527in}}%
\pgfpathlineto{\pgfqpoint{6.160557in}{3.947527in}}%
\pgfusepath{stroke}%
\end{pgfscope}%
\begin{pgfscope}%
\pgfsetrectcap%
\pgfsetmiterjoin%
\pgfsetlinewidth{0.803000pt}%
\definecolor{currentstroke}{rgb}{0.000000,0.000000,0.000000}%
\pgfsetstrokecolor{currentstroke}%
\pgfsetdash{}{0pt}%
\pgfpathmoveto{\pgfqpoint{3.906113in}{2.870679in}}%
\pgfpathlineto{\pgfqpoint{3.906113in}{4.489815in}}%
\pgfusepath{stroke}%
\end{pgfscope}%
\begin{pgfscope}%
\pgfsetrectcap%
\pgfsetmiterjoin%
\pgfsetlinewidth{0.803000pt}%
\definecolor{currentstroke}{rgb}{0.000000,0.000000,0.000000}%
\pgfsetstrokecolor{currentstroke}%
\pgfsetdash{}{0pt}%
\pgfpathmoveto{\pgfqpoint{6.267911in}{2.870679in}}%
\pgfpathlineto{\pgfqpoint{6.267911in}{4.489815in}}%
\pgfusepath{stroke}%
\end{pgfscope}%
\begin{pgfscope}%
\pgfsetrectcap%
\pgfsetmiterjoin%
\pgfsetlinewidth{0.803000pt}%
\definecolor{currentstroke}{rgb}{0.000000,0.000000,0.000000}%
\pgfsetstrokecolor{currentstroke}%
\pgfsetdash{}{0pt}%
\pgfpathmoveto{\pgfqpoint{3.906113in}{2.870679in}}%
\pgfpathlineto{\pgfqpoint{6.267911in}{2.870679in}}%
\pgfusepath{stroke}%
\end{pgfscope}%
\begin{pgfscope}%
\pgfsetrectcap%
\pgfsetmiterjoin%
\pgfsetlinewidth{0.803000pt}%
\definecolor{currentstroke}{rgb}{0.000000,0.000000,0.000000}%
\pgfsetstrokecolor{currentstroke}%
\pgfsetdash{}{0pt}%
\pgfpathmoveto{\pgfqpoint{3.906113in}{4.489815in}}%
\pgfpathlineto{\pgfqpoint{6.267911in}{4.489815in}}%
\pgfusepath{stroke}%
\end{pgfscope}%
\begin{pgfscope}%
\definecolor{textcolor}{rgb}{0.000000,0.000000,0.000000}%
\pgfsetstrokecolor{textcolor}%
\pgfsetfillcolor{textcolor}%
\pgftext[x=5.087012in,y=4.573148in,,base]{\color{textcolor}\rmfamily\fontsize{12.000000}{14.400000}\selectfont \(\displaystyle \omega\)}%
\end{pgfscope}%
\begin{pgfscope}%
\pgfsetbuttcap%
\pgfsetmiterjoin%
\definecolor{currentfill}{rgb}{1.000000,1.000000,1.000000}%
\pgfsetfillcolor{currentfill}%
\pgfsetlinewidth{0.000000pt}%
\definecolor{currentstroke}{rgb}{0.000000,0.000000,0.000000}%
\pgfsetstrokecolor{currentstroke}%
\pgfsetstrokeopacity{0.000000}%
\pgfsetdash{}{0pt}%
\pgfpathmoveto{\pgfqpoint{0.835065in}{0.526234in}}%
\pgfpathlineto{\pgfqpoint{3.196863in}{0.526234in}}%
\pgfpathlineto{\pgfqpoint{3.196863in}{2.145371in}}%
\pgfpathlineto{\pgfqpoint{0.835065in}{2.145371in}}%
\pgfpathclose%
\pgfusepath{fill}%
\end{pgfscope}%
\begin{pgfscope}%
\pgfsetbuttcap%
\pgfsetroundjoin%
\definecolor{currentfill}{rgb}{0.000000,0.000000,0.000000}%
\pgfsetfillcolor{currentfill}%
\pgfsetlinewidth{0.803000pt}%
\definecolor{currentstroke}{rgb}{0.000000,0.000000,0.000000}%
\pgfsetstrokecolor{currentstroke}%
\pgfsetdash{}{0pt}%
\pgfsys@defobject{currentmarker}{\pgfqpoint{0.000000in}{-0.048611in}}{\pgfqpoint{0.000000in}{0.000000in}}{%
\pgfpathmoveto{\pgfqpoint{0.000000in}{0.000000in}}%
\pgfpathlineto{\pgfqpoint{0.000000in}{-0.048611in}}%
\pgfusepath{stroke,fill}%
}%
\begin{pgfscope}%
\pgfsys@transformshift{0.922366in}{0.526234in}%
\pgfsys@useobject{currentmarker}{}%
\end{pgfscope}%
\end{pgfscope}%
\begin{pgfscope}%
\definecolor{textcolor}{rgb}{0.000000,0.000000,0.000000}%
\pgfsetstrokecolor{textcolor}%
\pgfsetfillcolor{textcolor}%
\pgftext[x=0.922366in,y=0.429012in,,top]{\color{textcolor}\rmfamily\fontsize{10.000000}{12.000000}\selectfont \(\displaystyle -0.2\)}%
\end{pgfscope}%
\begin{pgfscope}%
\pgfsetbuttcap%
\pgfsetroundjoin%
\definecolor{currentfill}{rgb}{0.000000,0.000000,0.000000}%
\pgfsetfillcolor{currentfill}%
\pgfsetlinewidth{0.803000pt}%
\definecolor{currentstroke}{rgb}{0.000000,0.000000,0.000000}%
\pgfsetstrokecolor{currentstroke}%
\pgfsetdash{}{0pt}%
\pgfsys@defobject{currentmarker}{\pgfqpoint{0.000000in}{-0.048611in}}{\pgfqpoint{0.000000in}{0.000000in}}{%
\pgfpathmoveto{\pgfqpoint{0.000000in}{0.000000in}}%
\pgfpathlineto{\pgfqpoint{0.000000in}{-0.048611in}}%
\pgfusepath{stroke,fill}%
}%
\begin{pgfscope}%
\pgfsys@transformshift{1.471272in}{0.526234in}%
\pgfsys@useobject{currentmarker}{}%
\end{pgfscope}%
\end{pgfscope}%
\begin{pgfscope}%
\definecolor{textcolor}{rgb}{0.000000,0.000000,0.000000}%
\pgfsetstrokecolor{textcolor}%
\pgfsetfillcolor{textcolor}%
\pgftext[x=1.471272in,y=0.429012in,,top]{\color{textcolor}\rmfamily\fontsize{10.000000}{12.000000}\selectfont \(\displaystyle -0.1\)}%
\end{pgfscope}%
\begin{pgfscope}%
\pgfsetbuttcap%
\pgfsetroundjoin%
\definecolor{currentfill}{rgb}{0.000000,0.000000,0.000000}%
\pgfsetfillcolor{currentfill}%
\pgfsetlinewidth{0.803000pt}%
\definecolor{currentstroke}{rgb}{0.000000,0.000000,0.000000}%
\pgfsetstrokecolor{currentstroke}%
\pgfsetdash{}{0pt}%
\pgfsys@defobject{currentmarker}{\pgfqpoint{0.000000in}{-0.048611in}}{\pgfqpoint{0.000000in}{0.000000in}}{%
\pgfpathmoveto{\pgfqpoint{0.000000in}{0.000000in}}%
\pgfpathlineto{\pgfqpoint{0.000000in}{-0.048611in}}%
\pgfusepath{stroke,fill}%
}%
\begin{pgfscope}%
\pgfsys@transformshift{2.020177in}{0.526234in}%
\pgfsys@useobject{currentmarker}{}%
\end{pgfscope}%
\end{pgfscope}%
\begin{pgfscope}%
\definecolor{textcolor}{rgb}{0.000000,0.000000,0.000000}%
\pgfsetstrokecolor{textcolor}%
\pgfsetfillcolor{textcolor}%
\pgftext[x=2.020177in,y=0.429012in,,top]{\color{textcolor}\rmfamily\fontsize{10.000000}{12.000000}\selectfont \(\displaystyle 0.0\)}%
\end{pgfscope}%
\begin{pgfscope}%
\pgfsetbuttcap%
\pgfsetroundjoin%
\definecolor{currentfill}{rgb}{0.000000,0.000000,0.000000}%
\pgfsetfillcolor{currentfill}%
\pgfsetlinewidth{0.803000pt}%
\definecolor{currentstroke}{rgb}{0.000000,0.000000,0.000000}%
\pgfsetstrokecolor{currentstroke}%
\pgfsetdash{}{0pt}%
\pgfsys@defobject{currentmarker}{\pgfqpoint{0.000000in}{-0.048611in}}{\pgfqpoint{0.000000in}{0.000000in}}{%
\pgfpathmoveto{\pgfqpoint{0.000000in}{0.000000in}}%
\pgfpathlineto{\pgfqpoint{0.000000in}{-0.048611in}}%
\pgfusepath{stroke,fill}%
}%
\begin{pgfscope}%
\pgfsys@transformshift{2.569082in}{0.526234in}%
\pgfsys@useobject{currentmarker}{}%
\end{pgfscope}%
\end{pgfscope}%
\begin{pgfscope}%
\definecolor{textcolor}{rgb}{0.000000,0.000000,0.000000}%
\pgfsetstrokecolor{textcolor}%
\pgfsetfillcolor{textcolor}%
\pgftext[x=2.569082in,y=0.429012in,,top]{\color{textcolor}\rmfamily\fontsize{10.000000}{12.000000}\selectfont \(\displaystyle 0.1\)}%
\end{pgfscope}%
\begin{pgfscope}%
\pgfsetbuttcap%
\pgfsetroundjoin%
\definecolor{currentfill}{rgb}{0.000000,0.000000,0.000000}%
\pgfsetfillcolor{currentfill}%
\pgfsetlinewidth{0.803000pt}%
\definecolor{currentstroke}{rgb}{0.000000,0.000000,0.000000}%
\pgfsetstrokecolor{currentstroke}%
\pgfsetdash{}{0pt}%
\pgfsys@defobject{currentmarker}{\pgfqpoint{0.000000in}{-0.048611in}}{\pgfqpoint{0.000000in}{0.000000in}}{%
\pgfpathmoveto{\pgfqpoint{0.000000in}{0.000000in}}%
\pgfpathlineto{\pgfqpoint{0.000000in}{-0.048611in}}%
\pgfusepath{stroke,fill}%
}%
\begin{pgfscope}%
\pgfsys@transformshift{3.117988in}{0.526234in}%
\pgfsys@useobject{currentmarker}{}%
\end{pgfscope}%
\end{pgfscope}%
\begin{pgfscope}%
\definecolor{textcolor}{rgb}{0.000000,0.000000,0.000000}%
\pgfsetstrokecolor{textcolor}%
\pgfsetfillcolor{textcolor}%
\pgftext[x=3.117988in,y=0.429012in,,top]{\color{textcolor}\rmfamily\fontsize{10.000000}{12.000000}\selectfont \(\displaystyle 0.2\)}%
\end{pgfscope}%
\begin{pgfscope}%
\definecolor{textcolor}{rgb}{0.000000,0.000000,0.000000}%
\pgfsetstrokecolor{textcolor}%
\pgfsetfillcolor{textcolor}%
\pgftext[x=2.015964in,y=0.250000in,,top]{\color{textcolor}\rmfamily\fontsize{10.000000}{12.000000}\selectfont angle (rad)}%
\end{pgfscope}%
\begin{pgfscope}%
\pgfsetbuttcap%
\pgfsetroundjoin%
\definecolor{currentfill}{rgb}{0.000000,0.000000,0.000000}%
\pgfsetfillcolor{currentfill}%
\pgfsetlinewidth{0.803000pt}%
\definecolor{currentstroke}{rgb}{0.000000,0.000000,0.000000}%
\pgfsetstrokecolor{currentstroke}%
\pgfsetdash{}{0pt}%
\pgfsys@defobject{currentmarker}{\pgfqpoint{-0.048611in}{0.000000in}}{\pgfqpoint{0.000000in}{0.000000in}}{%
\pgfpathmoveto{\pgfqpoint{0.000000in}{0.000000in}}%
\pgfpathlineto{\pgfqpoint{-0.048611in}{0.000000in}}%
\pgfusepath{stroke,fill}%
}%
\begin{pgfscope}%
\pgfsys@transformshift{0.835065in}{0.582059in}%
\pgfsys@useobject{currentmarker}{}%
\end{pgfscope}%
\end{pgfscope}%
\begin{pgfscope}%
\definecolor{textcolor}{rgb}{0.000000,0.000000,0.000000}%
\pgfsetstrokecolor{textcolor}%
\pgfsetfillcolor{textcolor}%
\pgftext[x=0.452348in,y=0.533834in,left,base]{\color{textcolor}\rmfamily\fontsize{10.000000}{12.000000}\selectfont \(\displaystyle -1.0\)}%
\end{pgfscope}%
\begin{pgfscope}%
\pgfsetbuttcap%
\pgfsetroundjoin%
\definecolor{currentfill}{rgb}{0.000000,0.000000,0.000000}%
\pgfsetfillcolor{currentfill}%
\pgfsetlinewidth{0.803000pt}%
\definecolor{currentstroke}{rgb}{0.000000,0.000000,0.000000}%
\pgfsetstrokecolor{currentstroke}%
\pgfsetdash{}{0pt}%
\pgfsys@defobject{currentmarker}{\pgfqpoint{-0.048611in}{0.000000in}}{\pgfqpoint{0.000000in}{0.000000in}}{%
\pgfpathmoveto{\pgfqpoint{0.000000in}{0.000000in}}%
\pgfpathlineto{\pgfqpoint{-0.048611in}{0.000000in}}%
\pgfusepath{stroke,fill}%
}%
\begin{pgfscope}%
\pgfsys@transformshift{0.835065in}{0.957491in}%
\pgfsys@useobject{currentmarker}{}%
\end{pgfscope}%
\end{pgfscope}%
\begin{pgfscope}%
\definecolor{textcolor}{rgb}{0.000000,0.000000,0.000000}%
\pgfsetstrokecolor{textcolor}%
\pgfsetfillcolor{textcolor}%
\pgftext[x=0.452348in,y=0.909266in,left,base]{\color{textcolor}\rmfamily\fontsize{10.000000}{12.000000}\selectfont \(\displaystyle -0.5\)}%
\end{pgfscope}%
\begin{pgfscope}%
\pgfsetbuttcap%
\pgfsetroundjoin%
\definecolor{currentfill}{rgb}{0.000000,0.000000,0.000000}%
\pgfsetfillcolor{currentfill}%
\pgfsetlinewidth{0.803000pt}%
\definecolor{currentstroke}{rgb}{0.000000,0.000000,0.000000}%
\pgfsetstrokecolor{currentstroke}%
\pgfsetdash{}{0pt}%
\pgfsys@defobject{currentmarker}{\pgfqpoint{-0.048611in}{0.000000in}}{\pgfqpoint{0.000000in}{0.000000in}}{%
\pgfpathmoveto{\pgfqpoint{0.000000in}{0.000000in}}%
\pgfpathlineto{\pgfqpoint{-0.048611in}{0.000000in}}%
\pgfusepath{stroke,fill}%
}%
\begin{pgfscope}%
\pgfsys@transformshift{0.835065in}{1.332923in}%
\pgfsys@useobject{currentmarker}{}%
\end{pgfscope}%
\end{pgfscope}%
\begin{pgfscope}%
\definecolor{textcolor}{rgb}{0.000000,0.000000,0.000000}%
\pgfsetstrokecolor{textcolor}%
\pgfsetfillcolor{textcolor}%
\pgftext[x=0.560373in,y=1.284698in,left,base]{\color{textcolor}\rmfamily\fontsize{10.000000}{12.000000}\selectfont \(\displaystyle 0.0\)}%
\end{pgfscope}%
\begin{pgfscope}%
\pgfsetbuttcap%
\pgfsetroundjoin%
\definecolor{currentfill}{rgb}{0.000000,0.000000,0.000000}%
\pgfsetfillcolor{currentfill}%
\pgfsetlinewidth{0.803000pt}%
\definecolor{currentstroke}{rgb}{0.000000,0.000000,0.000000}%
\pgfsetstrokecolor{currentstroke}%
\pgfsetdash{}{0pt}%
\pgfsys@defobject{currentmarker}{\pgfqpoint{-0.048611in}{0.000000in}}{\pgfqpoint{0.000000in}{0.000000in}}{%
\pgfpathmoveto{\pgfqpoint{0.000000in}{0.000000in}}%
\pgfpathlineto{\pgfqpoint{-0.048611in}{0.000000in}}%
\pgfusepath{stroke,fill}%
}%
\begin{pgfscope}%
\pgfsys@transformshift{0.835065in}{1.708356in}%
\pgfsys@useobject{currentmarker}{}%
\end{pgfscope}%
\end{pgfscope}%
\begin{pgfscope}%
\definecolor{textcolor}{rgb}{0.000000,0.000000,0.000000}%
\pgfsetstrokecolor{textcolor}%
\pgfsetfillcolor{textcolor}%
\pgftext[x=0.560373in,y=1.660130in,left,base]{\color{textcolor}\rmfamily\fontsize{10.000000}{12.000000}\selectfont \(\displaystyle 0.5\)}%
\end{pgfscope}%
\begin{pgfscope}%
\pgfsetbuttcap%
\pgfsetroundjoin%
\definecolor{currentfill}{rgb}{0.000000,0.000000,0.000000}%
\pgfsetfillcolor{currentfill}%
\pgfsetlinewidth{0.803000pt}%
\definecolor{currentstroke}{rgb}{0.000000,0.000000,0.000000}%
\pgfsetstrokecolor{currentstroke}%
\pgfsetdash{}{0pt}%
\pgfsys@defobject{currentmarker}{\pgfqpoint{-0.048611in}{0.000000in}}{\pgfqpoint{0.000000in}{0.000000in}}{%
\pgfpathmoveto{\pgfqpoint{0.000000in}{0.000000in}}%
\pgfpathlineto{\pgfqpoint{-0.048611in}{0.000000in}}%
\pgfusepath{stroke,fill}%
}%
\begin{pgfscope}%
\pgfsys@transformshift{0.835065in}{2.083788in}%
\pgfsys@useobject{currentmarker}{}%
\end{pgfscope}%
\end{pgfscope}%
\begin{pgfscope}%
\definecolor{textcolor}{rgb}{0.000000,0.000000,0.000000}%
\pgfsetstrokecolor{textcolor}%
\pgfsetfillcolor{textcolor}%
\pgftext[x=0.560373in,y=2.035563in,left,base]{\color{textcolor}\rmfamily\fontsize{10.000000}{12.000000}\selectfont \(\displaystyle 1.0\)}%
\end{pgfscope}%
\begin{pgfscope}%
\definecolor{textcolor}{rgb}{0.000000,0.000000,0.000000}%
\pgfsetstrokecolor{textcolor}%
\pgfsetfillcolor{textcolor}%
\pgftext[x=0.396792in,y=1.335803in,,bottom,rotate=90.000000]{\color{textcolor}\rmfamily\fontsize{10.000000}{12.000000}\selectfont angular velocity (\(\displaystyle \frac{rad}{s}\))}%
\end{pgfscope}%
\begin{pgfscope}%
\pgfpathrectangle{\pgfqpoint{0.835065in}{0.526234in}}{\pgfqpoint{2.361798in}{1.619136in}}%
\pgfusepath{clip}%
\pgfsetrectcap%
\pgfsetroundjoin%
\pgfsetlinewidth{1.505625pt}%
\definecolor{currentstroke}{rgb}{0.000000,0.000000,1.000000}%
\pgfsetstrokecolor{currentstroke}%
\pgfsetdash{}{0pt}%
\pgfpathmoveto{\pgfqpoint{2.978198in}{1.332923in}}%
\pgfpathlineto{\pgfqpoint{2.977126in}{1.300336in}}%
\pgfpathlineto{\pgfqpoint{2.973673in}{1.267822in}}%
\pgfpathlineto{\pgfqpoint{2.967848in}{1.235459in}}%
\pgfpathlineto{\pgfqpoint{2.959665in}{1.203329in}}%
\pgfpathlineto{\pgfqpoint{2.949142in}{1.171510in}}%
\pgfpathlineto{\pgfqpoint{2.936306in}{1.140081in}}%
\pgfpathlineto{\pgfqpoint{2.921187in}{1.109119in}}%
\pgfpathlineto{\pgfqpoint{2.903822in}{1.078702in}}%
\pgfpathlineto{\pgfqpoint{2.884253in}{1.048904in}}%
\pgfpathlineto{\pgfqpoint{2.862529in}{1.019801in}}%
\pgfpathlineto{\pgfqpoint{2.838701in}{0.991463in}}%
\pgfpathlineto{\pgfqpoint{2.812830in}{0.963963in}}%
\pgfpathlineto{\pgfqpoint{2.784977in}{0.937369in}}%
\pgfpathlineto{\pgfqpoint{2.752133in}{0.909241in}}%
\pgfpathlineto{\pgfqpoint{2.717072in}{0.882375in}}%
\pgfpathlineto{\pgfqpoint{2.679900in}{0.856854in}}%
\pgfpathlineto{\pgfqpoint{2.640727in}{0.832755in}}%
\pgfpathlineto{\pgfqpoint{2.599671in}{0.810153in}}%
\pgfpathlineto{\pgfqpoint{2.556853in}{0.789117in}}%
\pgfpathlineto{\pgfqpoint{2.512404in}{0.769712in}}%
\pgfpathlineto{\pgfqpoint{2.466455in}{0.751999in}}%
\pgfpathlineto{\pgfqpoint{2.419145in}{0.736032in}}%
\pgfpathlineto{\pgfqpoint{2.370617in}{0.721861in}}%
\pgfpathlineto{\pgfqpoint{2.321016in}{0.709530in}}%
\pgfpathlineto{\pgfqpoint{2.265859in}{0.698223in}}%
\pgfpathlineto{\pgfqpoint{2.209801in}{0.689195in}}%
\pgfpathlineto{\pgfqpoint{2.153044in}{0.682480in}}%
\pgfpathlineto{\pgfqpoint{2.095792in}{0.678103in}}%
\pgfpathlineto{\pgfqpoint{2.038250in}{0.676082in}}%
\pgfpathlineto{\pgfqpoint{1.980626in}{0.676424in}}%
\pgfpathlineto{\pgfqpoint{1.923127in}{0.679129in}}%
\pgfpathlineto{\pgfqpoint{1.865960in}{0.684188in}}%
\pgfpathlineto{\pgfqpoint{1.809331in}{0.691583in}}%
\pgfpathlineto{\pgfqpoint{1.753444in}{0.701287in}}%
\pgfpathlineto{\pgfqpoint{1.698499in}{0.713264in}}%
\pgfpathlineto{\pgfqpoint{1.649131in}{0.726204in}}%
\pgfpathlineto{\pgfqpoint{1.600870in}{0.740978in}}%
\pgfpathlineto{\pgfqpoint{1.553864in}{0.757542in}}%
\pgfpathlineto{\pgfqpoint{1.508253in}{0.775844in}}%
\pgfpathlineto{\pgfqpoint{1.464176in}{0.795829in}}%
\pgfpathlineto{\pgfqpoint{1.421765in}{0.817435in}}%
\pgfpathlineto{\pgfqpoint{1.381150in}{0.840598in}}%
\pgfpathlineto{\pgfqpoint{1.342451in}{0.865246in}}%
\pgfpathlineto{\pgfqpoint{1.305787in}{0.891304in}}%
\pgfpathlineto{\pgfqpoint{1.271268in}{0.918694in}}%
\pgfpathlineto{\pgfqpoint{1.238997in}{0.947333in}}%
\pgfpathlineto{\pgfqpoint{1.211693in}{0.974379in}}%
\pgfpathlineto{\pgfqpoint{1.186396in}{1.002319in}}%
\pgfpathlineto{\pgfqpoint{1.163169in}{1.031082in}}%
\pgfpathlineto{\pgfqpoint{1.142071in}{1.060598in}}%
\pgfpathlineto{\pgfqpoint{1.123152in}{1.090793in}}%
\pgfpathlineto{\pgfqpoint{1.106462in}{1.121593in}}%
\pgfpathlineto{\pgfqpoint{1.092040in}{1.152922in}}%
\pgfpathlineto{\pgfqpoint{1.079925in}{1.184702in}}%
\pgfpathlineto{\pgfqpoint{1.070145in}{1.216855in}}%
\pgfpathlineto{\pgfqpoint{1.062726in}{1.249301in}}%
\pgfpathlineto{\pgfqpoint{1.057686in}{1.281962in}}%
\pgfpathlineto{\pgfqpoint{1.055039in}{1.314756in}}%
\pgfpathlineto{\pgfqpoint{1.054791in}{1.347602in}}%
\pgfpathlineto{\pgfqpoint{1.056944in}{1.380421in}}%
\pgfpathlineto{\pgfqpoint{1.061493in}{1.413131in}}%
\pgfpathlineto{\pgfqpoint{1.068428in}{1.445651in}}%
\pgfpathlineto{\pgfqpoint{1.077731in}{1.477902in}}%
\pgfpathlineto{\pgfqpoint{1.089382in}{1.509804in}}%
\pgfpathlineto{\pgfqpoint{1.103350in}{1.541277in}}%
\pgfpathlineto{\pgfqpoint{1.119604in}{1.572245in}}%
\pgfpathlineto{\pgfqpoint{1.138102in}{1.602631in}}%
\pgfpathlineto{\pgfqpoint{1.158801in}{1.632358in}}%
\pgfpathlineto{\pgfqpoint{1.181649in}{1.661353in}}%
\pgfpathlineto{\pgfqpoint{1.206590in}{1.689544in}}%
\pgfpathlineto{\pgfqpoint{1.233565in}{1.716861in}}%
\pgfpathlineto{\pgfqpoint{1.262505in}{1.743234in}}%
\pgfpathlineto{\pgfqpoint{1.296526in}{1.771076in}}%
\pgfpathlineto{\pgfqpoint{1.332738in}{1.797613in}}%
\pgfpathlineto{\pgfqpoint{1.371034in}{1.822762in}}%
\pgfpathlineto{\pgfqpoint{1.411299in}{1.846446in}}%
\pgfpathlineto{\pgfqpoint{1.453413in}{1.868594in}}%
\pgfpathlineto{\pgfqpoint{1.497250in}{1.889136in}}%
\pgfpathlineto{\pgfqpoint{1.542678in}{1.908009in}}%
\pgfpathlineto{\pgfqpoint{1.589561in}{1.925154in}}%
\pgfpathlineto{\pgfqpoint{1.637758in}{1.940519in}}%
\pgfpathlineto{\pgfqpoint{1.687124in}{1.954054in}}%
\pgfpathlineto{\pgfqpoint{1.737510in}{1.965718in}}%
\pgfpathlineto{\pgfqpoint{1.788765in}{1.975474in}}%
\pgfpathlineto{\pgfqpoint{1.845488in}{1.983904in}}%
\pgfpathlineto{\pgfqpoint{1.902856in}{1.989995in}}%
\pgfpathlineto{\pgfqpoint{1.960664in}{1.993723in}}%
\pgfpathlineto{\pgfqpoint{2.018703in}{1.995074in}}%
\pgfpathlineto{\pgfqpoint{2.076766in}{1.994042in}}%
\pgfpathlineto{\pgfqpoint{2.134641in}{1.990631in}}%
\pgfpathlineto{\pgfqpoint{2.192123in}{1.984851in}}%
\pgfpathlineto{\pgfqpoint{2.249002in}{1.976724in}}%
\pgfpathlineto{\pgfqpoint{2.305076in}{1.966280in}}%
\pgfpathlineto{\pgfqpoint{2.355596in}{1.954702in}}%
\pgfpathlineto{\pgfqpoint{2.405115in}{1.941244in}}%
\pgfpathlineto{\pgfqpoint{2.453486in}{1.925946in}}%
\pgfpathlineto{\pgfqpoint{2.500560in}{1.908856in}}%
\pgfpathlineto{\pgfqpoint{2.546196in}{1.890027in}}%
\pgfpathlineto{\pgfqpoint{2.590255in}{1.869514in}}%
\pgfpathlineto{\pgfqpoint{2.632606in}{1.847382in}}%
\pgfpathlineto{\pgfqpoint{2.673120in}{1.823697in}}%
\pgfpathlineto{\pgfqpoint{2.711674in}{1.798532in}}%
\pgfpathlineto{\pgfqpoint{2.748153in}{1.771963in}}%
\pgfpathlineto{\pgfqpoint{2.782447in}{1.744070in}}%
\pgfpathlineto{\pgfqpoint{2.814452in}{1.714938in}}%
\pgfpathlineto{\pgfqpoint{2.841480in}{1.687453in}}%
\pgfpathlineto{\pgfqpoint{2.866469in}{1.659084in}}%
\pgfpathlineto{\pgfqpoint{2.889357in}{1.629903in}}%
\pgfpathlineto{\pgfqpoint{2.910087in}{1.599982in}}%
\pgfpathlineto{\pgfqpoint{2.928607in}{1.569394in}}%
\pgfpathlineto{\pgfqpoint{2.944872in}{1.538216in}}%
\pgfpathlineto{\pgfqpoint{2.958840in}{1.506524in}}%
\pgfpathlineto{\pgfqpoint{2.970476in}{1.474396in}}%
\pgfpathlineto{\pgfqpoint{2.979751in}{1.441913in}}%
\pgfpathlineto{\pgfqpoint{2.986642in}{1.409153in}}%
\pgfpathlineto{\pgfqpoint{2.991132in}{1.376197in}}%
\pgfpathlineto{\pgfqpoint{2.993208in}{1.343127in}}%
\pgfpathlineto{\pgfqpoint{2.992865in}{1.310023in}}%
\pgfpathlineto{\pgfqpoint{2.990103in}{1.276968in}}%
\pgfpathlineto{\pgfqpoint{2.984928in}{1.244042in}}%
\pgfpathlineto{\pgfqpoint{2.977353in}{1.211327in}}%
\pgfpathlineto{\pgfqpoint{2.967395in}{1.178904in}}%
\pgfpathlineto{\pgfqpoint{2.955079in}{1.146851in}}%
\pgfpathlineto{\pgfqpoint{2.940434in}{1.115249in}}%
\pgfpathlineto{\pgfqpoint{2.923496in}{1.084176in}}%
\pgfpathlineto{\pgfqpoint{2.904305in}{1.053708in}}%
\pgfpathlineto{\pgfqpoint{2.882910in}{1.023922in}}%
\pgfpathlineto{\pgfqpoint{2.859361in}{0.994892in}}%
\pgfpathlineto{\pgfqpoint{2.833717in}{0.966689in}}%
\pgfpathlineto{\pgfqpoint{2.806040in}{0.939385in}}%
\pgfpathlineto{\pgfqpoint{2.776399in}{0.913048in}}%
\pgfpathlineto{\pgfqpoint{2.741611in}{0.885273in}}%
\pgfpathlineto{\pgfqpoint{2.704639in}{0.858832in}}%
\pgfpathlineto{\pgfqpoint{2.665591in}{0.833807in}}%
\pgfpathlineto{\pgfqpoint{2.624585in}{0.810274in}}%
\pgfpathlineto{\pgfqpoint{2.581743in}{0.788307in}}%
\pgfpathlineto{\pgfqpoint{2.537194in}{0.767971in}}%
\pgfpathlineto{\pgfqpoint{2.491072in}{0.749331in}}%
\pgfpathlineto{\pgfqpoint{2.443514in}{0.732445in}}%
\pgfpathlineto{\pgfqpoint{2.394664in}{0.717365in}}%
\pgfpathlineto{\pgfqpoint{2.344669in}{0.704138in}}%
\pgfpathlineto{\pgfqpoint{2.293679in}{0.692806in}}%
\pgfpathlineto{\pgfqpoint{2.241849in}{0.683405in}}%
\pgfpathlineto{\pgfqpoint{2.189334in}{0.675964in}}%
\pgfpathlineto{\pgfqpoint{2.131450in}{0.670109in}}%
\pgfpathlineto{\pgfqpoint{2.073149in}{0.666639in}}%
\pgfpathlineto{\pgfqpoint{2.014639in}{0.665565in}}%
\pgfpathlineto{\pgfqpoint{1.956132in}{0.666893in}}%
\pgfpathlineto{\pgfqpoint{1.897838in}{0.670619in}}%
\pgfpathlineto{\pgfqpoint{1.839966in}{0.676729in}}%
\pgfpathlineto{\pgfqpoint{1.782726in}{0.685201in}}%
\pgfpathlineto{\pgfqpoint{1.730986in}{0.695017in}}%
\pgfpathlineto{\pgfqpoint{1.680107in}{0.706763in}}%
\pgfpathlineto{\pgfqpoint{1.630241in}{0.720401in}}%
\pgfpathlineto{\pgfqpoint{1.581540in}{0.735892in}}%
\pgfpathlineto{\pgfqpoint{1.534151in}{0.753187in}}%
\pgfpathlineto{\pgfqpoint{1.488217in}{0.772233in}}%
\pgfpathlineto{\pgfqpoint{1.443877in}{0.792973in}}%
\pgfpathlineto{\pgfqpoint{1.401265in}{0.815342in}}%
\pgfpathlineto{\pgfqpoint{1.360509in}{0.839274in}}%
\pgfpathlineto{\pgfqpoint{1.321732in}{0.864694in}}%
\pgfpathlineto{\pgfqpoint{1.285052in}{0.891526in}}%
\pgfpathlineto{\pgfqpoint{1.250578in}{0.919688in}}%
\pgfpathlineto{\pgfqpoint{1.221241in}{0.946373in}}%
\pgfpathlineto{\pgfqpoint{1.193887in}{0.974021in}}%
\pgfpathlineto{\pgfqpoint{1.168584in}{1.002563in}}%
\pgfpathlineto{\pgfqpoint{1.145395in}{1.031928in}}%
\pgfpathlineto{\pgfqpoint{1.124378in}{1.062043in}}%
\pgfpathlineto{\pgfqpoint{1.105585in}{1.092833in}}%
\pgfpathlineto{\pgfqpoint{1.089062in}{1.124223in}}%
\pgfpathlineto{\pgfqpoint{1.074853in}{1.156135in}}%
\pgfpathlineto{\pgfqpoint{1.062991in}{1.188490in}}%
\pgfpathlineto{\pgfqpoint{1.053507in}{1.221209in}}%
\pgfpathlineto{\pgfqpoint{1.046424in}{1.254211in}}%
\pgfpathlineto{\pgfqpoint{1.041761in}{1.287415in}}%
\pgfpathlineto{\pgfqpoint{1.039530in}{1.320739in}}%
\pgfpathlineto{\pgfqpoint{1.039737in}{1.354100in}}%
\pgfpathlineto{\pgfqpoint{1.042382in}{1.387419in}}%
\pgfpathlineto{\pgfqpoint{1.047458in}{1.420611in}}%
\pgfpathlineto{\pgfqpoint{1.054955in}{1.453595in}}%
\pgfpathlineto{\pgfqpoint{1.064854in}{1.486291in}}%
\pgfpathlineto{\pgfqpoint{1.077132in}{1.518617in}}%
\pgfpathlineto{\pgfqpoint{1.091758in}{1.550494in}}%
\pgfpathlineto{\pgfqpoint{1.108698in}{1.581842in}}%
\pgfpathlineto{\pgfqpoint{1.127910in}{1.612585in}}%
\pgfpathlineto{\pgfqpoint{1.149347in}{1.642645in}}%
\pgfpathlineto{\pgfqpoint{1.172958in}{1.671948in}}%
\pgfpathlineto{\pgfqpoint{1.198683in}{1.700422in}}%
\pgfpathlineto{\pgfqpoint{1.226461in}{1.727993in}}%
\pgfpathlineto{\pgfqpoint{1.256223in}{1.754595in}}%
\pgfpathlineto{\pgfqpoint{1.287896in}{1.780159in}}%
\pgfpathlineto{\pgfqpoint{1.324850in}{1.807004in}}%
\pgfpathlineto{\pgfqpoint{1.363912in}{1.832435in}}%
\pgfpathlineto{\pgfqpoint{1.404964in}{1.856372in}}%
\pgfpathlineto{\pgfqpoint{1.447885in}{1.878743in}}%
\pgfpathlineto{\pgfqpoint{1.492545in}{1.899478in}}%
\pgfpathlineto{\pgfqpoint{1.538811in}{1.918513in}}%
\pgfpathlineto{\pgfqpoint{1.586544in}{1.935789in}}%
\pgfpathlineto{\pgfqpoint{1.635600in}{1.951252in}}%
\pgfpathlineto{\pgfqpoint{1.685831in}{1.964854in}}%
\pgfpathlineto{\pgfqpoint{1.737087in}{1.976551in}}%
\pgfpathlineto{\pgfqpoint{1.789213in}{1.986308in}}%
\pgfpathlineto{\pgfqpoint{1.842052in}{1.994093in}}%
\pgfpathlineto{\pgfqpoint{1.895444in}{1.999882in}}%
\pgfpathlineto{\pgfqpoint{1.949228in}{2.003655in}}%
\pgfpathlineto{\pgfqpoint{2.008157in}{2.005459in}}%
\pgfpathlineto{\pgfqpoint{2.067147in}{2.004842in}}%
\pgfpathlineto{\pgfqpoint{2.125986in}{2.001807in}}%
\pgfpathlineto{\pgfqpoint{2.179607in}{1.996909in}}%
\pgfpathlineto{\pgfqpoint{2.232761in}{1.990003in}}%
\pgfpathlineto{\pgfqpoint{2.285288in}{1.981108in}}%
\pgfpathlineto{\pgfqpoint{2.337026in}{1.970253in}}%
\pgfpathlineto{\pgfqpoint{2.387822in}{1.957470in}}%
\pgfpathlineto{\pgfqpoint{2.437520in}{1.942799in}}%
\pgfpathlineto{\pgfqpoint{2.485970in}{1.926284in}}%
\pgfpathlineto{\pgfqpoint{2.533027in}{1.907977in}}%
\pgfpathlineto{\pgfqpoint{2.578548in}{1.887933in}}%
\pgfpathlineto{\pgfqpoint{2.622395in}{1.866213in}}%
\pgfpathlineto{\pgfqpoint{2.664437in}{1.842884in}}%
\pgfpathlineto{\pgfqpoint{2.704545in}{1.818017in}}%
\pgfpathlineto{\pgfqpoint{2.742600in}{1.791687in}}%
\pgfpathlineto{\pgfqpoint{2.778487in}{1.763974in}}%
\pgfpathlineto{\pgfqpoint{2.809139in}{1.737651in}}%
\pgfpathlineto{\pgfqpoint{2.837832in}{1.710320in}}%
\pgfpathlineto{\pgfqpoint{2.864497in}{1.682050in}}%
\pgfpathlineto{\pgfqpoint{2.889065in}{1.652910in}}%
\pgfpathlineto{\pgfqpoint{2.911477in}{1.622973in}}%
\pgfpathlineto{\pgfqpoint{2.931676in}{1.592313in}}%
\pgfpathlineto{\pgfqpoint{2.949612in}{1.561005in}}%
\pgfpathlineto{\pgfqpoint{2.965240in}{1.529129in}}%
\pgfpathlineto{\pgfqpoint{2.978521in}{1.496761in}}%
\pgfpathlineto{\pgfqpoint{2.989422in}{1.463981in}}%
\pgfpathlineto{\pgfqpoint{2.997916in}{1.430871in}}%
\pgfpathlineto{\pgfqpoint{3.003980in}{1.397512in}}%
\pgfpathlineto{\pgfqpoint{3.007599in}{1.363986in}}%
\pgfpathlineto{\pgfqpoint{3.008765in}{1.330375in}}%
\pgfpathlineto{\pgfqpoint{3.007472in}{1.296763in}}%
\pgfpathlineto{\pgfqpoint{3.003725in}{1.263231in}}%
\pgfpathlineto{\pgfqpoint{2.997531in}{1.229862in}}%
\pgfpathlineto{\pgfqpoint{2.988906in}{1.196738in}}%
\pgfpathlineto{\pgfqpoint{2.977869in}{1.163942in}}%
\pgfpathlineto{\pgfqpoint{2.964448in}{1.131553in}}%
\pgfpathlineto{\pgfqpoint{2.948675in}{1.099652in}}%
\pgfpathlineto{\pgfqpoint{2.930587in}{1.068319in}}%
\pgfpathlineto{\pgfqpoint{2.910230in}{1.037629in}}%
\pgfpathlineto{\pgfqpoint{2.887653in}{1.007661in}}%
\pgfpathlineto{\pgfqpoint{2.862910in}{0.978489in}}%
\pgfpathlineto{\pgfqpoint{2.836064in}{0.950184in}}%
\pgfpathlineto{\pgfqpoint{2.807178in}{0.922819in}}%
\pgfpathlineto{\pgfqpoint{2.776325in}{0.896462in}}%
\pgfpathlineto{\pgfqpoint{2.743580in}{0.871179in}}%
\pgfpathlineto{\pgfqpoint{2.705471in}{0.844684in}}%
\pgfpathlineto{\pgfqpoint{2.665285in}{0.819647in}}%
\pgfpathlineto{\pgfqpoint{2.623141in}{0.796145in}}%
\pgfpathlineto{\pgfqpoint{2.579166in}{0.774251in}}%
\pgfpathlineto{\pgfqpoint{2.533491in}{0.754031in}}%
\pgfpathlineto{\pgfqpoint{2.486253in}{0.735549in}}%
\pgfpathlineto{\pgfqpoint{2.437594in}{0.718862in}}%
\pgfpathlineto{\pgfqpoint{2.387660in}{0.704023in}}%
\pgfpathlineto{\pgfqpoint{2.336602in}{0.691077in}}%
\pgfpathlineto{\pgfqpoint{2.284574in}{0.680066in}}%
\pgfpathlineto{\pgfqpoint{2.231732in}{0.671024in}}%
\pgfpathlineto{\pgfqpoint{2.178235in}{0.663980in}}%
\pgfpathlineto{\pgfqpoint{2.124246in}{0.658957in}}%
\pgfpathlineto{\pgfqpoint{2.069927in}{0.655970in}}%
\pgfpathlineto{\pgfqpoint{2.015443in}{0.655030in}}%
\pgfpathlineto{\pgfqpoint{1.960958in}{0.656140in}}%
\pgfpathlineto{\pgfqpoint{1.906638in}{0.659297in}}%
\pgfpathlineto{\pgfqpoint{1.852645in}{0.664492in}}%
\pgfpathlineto{\pgfqpoint{1.799145in}{0.671709in}}%
\pgfpathlineto{\pgfqpoint{1.746298in}{0.680927in}}%
\pgfpathlineto{\pgfqpoint{1.694265in}{0.692117in}}%
\pgfpathlineto{\pgfqpoint{1.643202in}{0.705244in}}%
\pgfpathlineto{\pgfqpoint{1.593265in}{0.720270in}}%
\pgfpathlineto{\pgfqpoint{1.544604in}{0.737147in}}%
\pgfpathlineto{\pgfqpoint{1.497367in}{0.755824in}}%
\pgfpathlineto{\pgfqpoint{1.451695in}{0.776243in}}%
\pgfpathlineto{\pgfqpoint{1.407728in}{0.798343in}}%
\pgfpathlineto{\pgfqpoint{1.365596in}{0.822056in}}%
\pgfpathlineto{\pgfqpoint{1.325429in}{0.847310in}}%
\pgfpathlineto{\pgfqpoint{1.287346in}{0.874028in}}%
\pgfpathlineto{\pgfqpoint{1.251463in}{0.902129in}}%
\pgfpathlineto{\pgfqpoint{1.220842in}{0.928803in}}%
\pgfpathlineto{\pgfqpoint{1.192204in}{0.956484in}}%
\pgfpathlineto{\pgfqpoint{1.165621in}{0.985102in}}%
\pgfpathlineto{\pgfqpoint{1.141159in}{1.014587in}}%
\pgfpathlineto{\pgfqpoint{1.118879in}{1.044864in}}%
\pgfpathlineto{\pgfqpoint{1.098837in}{1.075859in}}%
\pgfpathlineto{\pgfqpoint{1.081082in}{1.107495in}}%
\pgfpathlineto{\pgfqpoint{1.065658in}{1.139694in}}%
\pgfpathlineto{\pgfqpoint{1.052605in}{1.172378in}}%
\pgfpathlineto{\pgfqpoint{1.041955in}{1.205464in}}%
\pgfpathlineto{\pgfqpoint{1.033735in}{1.238872in}}%
\pgfpathlineto{\pgfqpoint{1.027965in}{1.272519in}}%
\pgfpathlineto{\pgfqpoint{1.024660in}{1.306324in}}%
\pgfpathlineto{\pgfqpoint{1.023802in}{1.336813in}}%
\pgfpathlineto{\pgfqpoint{1.024949in}{1.367301in}}%
\pgfpathlineto{\pgfqpoint{1.028101in}{1.397728in}}%
\pgfpathlineto{\pgfqpoint{1.033947in}{1.431389in}}%
\pgfpathlineto{\pgfqpoint{1.042247in}{1.464816in}}%
\pgfpathlineto{\pgfqpoint{1.052980in}{1.497927in}}%
\pgfpathlineto{\pgfqpoint{1.066121in}{1.530640in}}%
\pgfpathlineto{\pgfqpoint{1.081638in}{1.562875in}}%
\pgfpathlineto{\pgfqpoint{1.099494in}{1.594551in}}%
\pgfpathlineto{\pgfqpoint{1.119645in}{1.625589in}}%
\pgfpathlineto{\pgfqpoint{1.142042in}{1.655914in}}%
\pgfpathlineto{\pgfqpoint{1.166630in}{1.685449in}}%
\pgfpathlineto{\pgfqpoint{1.193349in}{1.714121in}}%
\pgfpathlineto{\pgfqpoint{1.222134in}{1.741858in}}%
\pgfpathlineto{\pgfqpoint{1.252914in}{1.768590in}}%
\pgfpathlineto{\pgfqpoint{1.285614in}{1.794251in}}%
\pgfpathlineto{\pgfqpoint{1.323704in}{1.821162in}}%
\pgfpathlineto{\pgfqpoint{1.363905in}{1.846616in}}%
\pgfpathlineto{\pgfqpoint{1.406098in}{1.870535in}}%
\pgfpathlineto{\pgfqpoint{1.450155in}{1.892844in}}%
\pgfpathlineto{\pgfqpoint{1.495946in}{1.913474in}}%
\pgfpathlineto{\pgfqpoint{1.543332in}{1.932363in}}%
\pgfpathlineto{\pgfqpoint{1.592172in}{1.949451in}}%
\pgfpathlineto{\pgfqpoint{1.642318in}{1.964684in}}%
\pgfpathlineto{\pgfqpoint{1.693620in}{1.978015in}}%
\pgfpathlineto{\pgfqpoint{1.745924in}{1.989402in}}%
\pgfpathlineto{\pgfqpoint{1.799070in}{1.998809in}}%
\pgfpathlineto{\pgfqpoint{1.852900in}{2.006207in}}%
\pgfpathlineto{\pgfqpoint{1.907251in}{2.011571in}}%
\pgfpathlineto{\pgfqpoint{1.961958in}{2.014884in}}%
\pgfpathlineto{\pgfqpoint{2.016857in}{2.016136in}}%
\pgfpathlineto{\pgfqpoint{2.071780in}{2.015321in}}%
\pgfpathlineto{\pgfqpoint{2.126562in}{2.012443in}}%
\pgfpathlineto{\pgfqpoint{2.181038in}{2.007509in}}%
\pgfpathlineto{\pgfqpoint{2.235042in}{2.000534in}}%
\pgfpathlineto{\pgfqpoint{2.288412in}{1.991539in}}%
\pgfpathlineto{\pgfqpoint{2.340985in}{1.980552in}}%
\pgfpathlineto{\pgfqpoint{2.392603in}{1.967606in}}%
\pgfpathlineto{\pgfqpoint{2.443109in}{1.952742in}}%
\pgfpathlineto{\pgfqpoint{2.492351in}{1.936004in}}%
\pgfpathlineto{\pgfqpoint{2.540181in}{1.917445in}}%
\pgfpathlineto{\pgfqpoint{2.586453in}{1.897120in}}%
\pgfpathlineto{\pgfqpoint{2.631028in}{1.875092in}}%
\pgfpathlineto{\pgfqpoint{2.673771in}{1.851429in}}%
\pgfpathlineto{\pgfqpoint{2.714554in}{1.826201in}}%
\pgfpathlineto{\pgfqpoint{2.753253in}{1.799487in}}%
\pgfpathlineto{\pgfqpoint{2.789752in}{1.771367in}}%
\pgfpathlineto{\pgfqpoint{2.820931in}{1.744654in}}%
\pgfpathlineto{\pgfqpoint{2.850124in}{1.716917in}}%
\pgfpathlineto{\pgfqpoint{2.877257in}{1.688223in}}%
\pgfpathlineto{\pgfqpoint{2.902262in}{1.658646in}}%
\pgfpathlineto{\pgfqpoint{2.925079in}{1.628257in}}%
\pgfpathlineto{\pgfqpoint{2.945649in}{1.597132in}}%
\pgfpathlineto{\pgfqpoint{2.963921in}{1.565349in}}%
\pgfpathlineto{\pgfqpoint{2.979851in}{1.532985in}}%
\pgfpathlineto{\pgfqpoint{2.993398in}{1.500121in}}%
\pgfpathlineto{\pgfqpoint{3.004528in}{1.466837in}}%
\pgfpathlineto{\pgfqpoint{3.013213in}{1.433216in}}%
\pgfpathlineto{\pgfqpoint{3.019432in}{1.399340in}}%
\pgfpathlineto{\pgfqpoint{3.022906in}{1.368702in}}%
\pgfpathlineto{\pgfqpoint{3.024362in}{1.337987in}}%
\pgfpathlineto{\pgfqpoint{3.023797in}{1.307254in}}%
\pgfpathlineto{\pgfqpoint{3.021210in}{1.276566in}}%
\pgfpathlineto{\pgfqpoint{3.016607in}{1.245983in}}%
\pgfpathlineto{\pgfqpoint{3.009997in}{1.215567in}}%
\pgfpathlineto{\pgfqpoint{3.000313in}{1.182039in}}%
\pgfpathlineto{\pgfqpoint{2.988189in}{1.148875in}}%
\pgfpathlineto{\pgfqpoint{2.973655in}{1.116156in}}%
\pgfpathlineto{\pgfqpoint{2.956746in}{1.083962in}}%
\pgfpathlineto{\pgfqpoint{2.937503in}{1.052374in}}%
\pgfpathlineto{\pgfqpoint{2.915973in}{1.021469in}}%
\pgfpathlineto{\pgfqpoint{2.892209in}{0.991325in}}%
\pgfpathlineto{\pgfqpoint{2.866267in}{0.962017in}}%
\pgfpathlineto{\pgfqpoint{2.838213in}{0.933617in}}%
\pgfpathlineto{\pgfqpoint{2.808114in}{0.906198in}}%
\pgfpathlineto{\pgfqpoint{2.776045in}{0.879827in}}%
\pgfpathlineto{\pgfqpoint{2.742085in}{0.854572in}}%
\pgfpathlineto{\pgfqpoint{2.702644in}{0.828156in}}%
\pgfpathlineto{\pgfqpoint{2.661133in}{0.803247in}}%
\pgfpathlineto{\pgfqpoint{2.617676in}{0.779923in}}%
\pgfpathlineto{\pgfqpoint{2.572404in}{0.758255in}}%
\pgfpathlineto{\pgfqpoint{2.525453in}{0.738310in}}%
\pgfpathlineto{\pgfqpoint{2.476962in}{0.720151in}}%
\pgfpathlineto{\pgfqpoint{2.427078in}{0.703833in}}%
\pgfpathlineto{\pgfqpoint{2.375951in}{0.689409in}}%
\pgfpathlineto{\pgfqpoint{2.323734in}{0.676923in}}%
\pgfpathlineto{\pgfqpoint{2.270585in}{0.666414in}}%
\pgfpathlineto{\pgfqpoint{2.216665in}{0.657917in}}%
\pgfpathlineto{\pgfqpoint{2.162136in}{0.651457in}}%
\pgfpathlineto{\pgfqpoint{2.107162in}{0.647056in}}%
\pgfpathlineto{\pgfqpoint{2.051910in}{0.644728in}}%
\pgfpathlineto{\pgfqpoint{1.996547in}{0.644481in}}%
\pgfpathlineto{\pgfqpoint{1.941240in}{0.646316in}}%
\pgfpathlineto{\pgfqpoint{1.886157in}{0.650228in}}%
\pgfpathlineto{\pgfqpoint{1.831464in}{0.656205in}}%
\pgfpathlineto{\pgfqpoint{1.777326in}{0.664230in}}%
\pgfpathlineto{\pgfqpoint{1.723908in}{0.674277in}}%
\pgfpathlineto{\pgfqpoint{1.671371in}{0.686317in}}%
\pgfpathlineto{\pgfqpoint{1.619874in}{0.700312in}}%
\pgfpathlineto{\pgfqpoint{1.569572in}{0.716218in}}%
\pgfpathlineto{\pgfqpoint{1.520618in}{0.733989in}}%
\pgfpathlineto{\pgfqpoint{1.473160in}{0.753568in}}%
\pgfpathlineto{\pgfqpoint{1.427340in}{0.774896in}}%
\pgfpathlineto{\pgfqpoint{1.383297in}{0.797909in}}%
\pgfpathlineto{\pgfqpoint{1.341165in}{0.822535in}}%
\pgfpathlineto{\pgfqpoint{1.301069in}{0.848699in}}%
\pgfpathlineto{\pgfqpoint{1.263131in}{0.876323in}}%
\pgfpathlineto{\pgfqpoint{1.230611in}{0.902632in}}%
\pgfpathlineto{\pgfqpoint{1.200050in}{0.930011in}}%
\pgfpathlineto{\pgfqpoint{1.171523in}{0.958394in}}%
\pgfpathlineto{\pgfqpoint{1.145103in}{0.987709in}}%
\pgfpathlineto{\pgfqpoint{1.120854in}{1.017882in}}%
\pgfpathlineto{\pgfqpoint{1.098838in}{1.048841in}}%
\pgfpathlineto{\pgfqpoint{1.079108in}{1.080507in}}%
\pgfpathlineto{\pgfqpoint{1.061715in}{1.112803in}}%
\pgfpathlineto{\pgfqpoint{1.046700in}{1.145648in}}%
\pgfpathlineto{\pgfqpoint{1.034103in}{1.178963in}}%
\pgfpathlineto{\pgfqpoint{1.023955in}{1.212664in}}%
\pgfpathlineto{\pgfqpoint{1.016936in}{1.243258in}}%
\pgfpathlineto{\pgfqpoint{1.011936in}{1.274037in}}%
\pgfpathlineto{\pgfqpoint{1.008965in}{1.304940in}}%
\pgfpathlineto{\pgfqpoint{1.008029in}{1.335905in}}%
\pgfpathlineto{\pgfqpoint{1.009131in}{1.366871in}}%
\pgfpathlineto{\pgfqpoint{1.012269in}{1.397777in}}%
\pgfpathlineto{\pgfqpoint{1.017436in}{1.428560in}}%
\pgfpathlineto{\pgfqpoint{1.024625in}{1.459160in}}%
\pgfpathlineto{\pgfqpoint{1.033819in}{1.489515in}}%
\pgfpathlineto{\pgfqpoint{1.046367in}{1.522882in}}%
\pgfpathlineto{\pgfqpoint{1.061339in}{1.555790in}}%
\pgfpathlineto{\pgfqpoint{1.078699in}{1.588158in}}%
\pgfpathlineto{\pgfqpoint{1.098406in}{1.619905in}}%
\pgfpathlineto{\pgfqpoint{1.120411in}{1.650952in}}%
\pgfpathlineto{\pgfqpoint{1.144661in}{1.681223in}}%
\pgfpathlineto{\pgfqpoint{1.171096in}{1.710642in}}%
\pgfpathlineto{\pgfqpoint{1.199651in}{1.739134in}}%
\pgfpathlineto{\pgfqpoint{1.230257in}{1.766630in}}%
\pgfpathlineto{\pgfqpoint{1.262839in}{1.793060in}}%
\pgfpathlineto{\pgfqpoint{1.297315in}{1.818357in}}%
\pgfpathlineto{\pgfqpoint{1.337327in}{1.844800in}}%
\pgfpathlineto{\pgfqpoint{1.379410in}{1.869713in}}%
\pgfpathlineto{\pgfqpoint{1.423438in}{1.893020in}}%
\pgfpathlineto{\pgfqpoint{1.469279in}{1.914650in}}%
\pgfpathlineto{\pgfqpoint{1.516797in}{1.934536in}}%
\pgfpathlineto{\pgfqpoint{1.565847in}{1.952614in}}%
\pgfpathlineto{\pgfqpoint{1.616284in}{1.968830in}}%
\pgfpathlineto{\pgfqpoint{1.667955in}{1.983133in}}%
\pgfpathlineto{\pgfqpoint{1.720705in}{1.995476in}}%
\pgfpathlineto{\pgfqpoint{1.774375in}{2.005822in}}%
\pgfpathlineto{\pgfqpoint{1.828802in}{2.014138in}}%
\pgfpathlineto{\pgfqpoint{1.883824in}{2.020397in}}%
\pgfpathlineto{\pgfqpoint{1.939273in}{2.024579in}}%
\pgfpathlineto{\pgfqpoint{1.994981in}{2.026670in}}%
\pgfpathlineto{\pgfqpoint{2.050782in}{2.026664in}}%
\pgfpathlineto{\pgfqpoint{2.106505in}{2.024560in}}%
\pgfpathlineto{\pgfqpoint{2.161982in}{2.020363in}}%
\pgfpathlineto{\pgfqpoint{2.217046in}{2.014086in}}%
\pgfpathlineto{\pgfqpoint{2.271529in}{2.005749in}}%
\pgfpathlineto{\pgfqpoint{2.325268in}{1.995377in}}%
\pgfpathlineto{\pgfqpoint{2.378099in}{1.983001in}}%
\pgfpathlineto{\pgfqpoint{2.429862in}{1.968659in}}%
\pgfpathlineto{\pgfqpoint{2.480402in}{1.952396in}}%
\pgfpathlineto{\pgfqpoint{2.529566in}{1.934262in}}%
\pgfpathlineto{\pgfqpoint{2.577204in}{1.914311in}}%
\pgfpathlineto{\pgfqpoint{2.623174in}{1.892606in}}%
\pgfpathlineto{\pgfqpoint{2.667335in}{1.869212in}}%
\pgfpathlineto{\pgfqpoint{2.709557in}{1.844200in}}%
\pgfpathlineto{\pgfqpoint{2.749709in}{1.817648in}}%
\pgfpathlineto{\pgfqpoint{2.784316in}{1.792239in}}%
\pgfpathlineto{\pgfqpoint{2.817026in}{1.765688in}}%
\pgfpathlineto{\pgfqpoint{2.847760in}{1.738059in}}%
\pgfpathlineto{\pgfqpoint{2.876441in}{1.709422in}}%
\pgfpathlineto{\pgfqpoint{2.902996in}{1.679848in}}%
\pgfpathlineto{\pgfqpoint{2.927361in}{1.649411in}}%
\pgfpathlineto{\pgfqpoint{2.949475in}{1.618186in}}%
\pgfpathlineto{\pgfqpoint{2.969282in}{1.586251in}}%
\pgfpathlineto{\pgfqpoint{2.986733in}{1.553683in}}%
\pgfpathlineto{\pgfqpoint{3.001785in}{1.520565in}}%
\pgfpathlineto{\pgfqpoint{3.014400in}{1.486976in}}%
\pgfpathlineto{\pgfqpoint{3.024546in}{1.453001in}}%
\pgfpathlineto{\pgfqpoint{3.031546in}{1.422161in}}%
\pgfpathlineto{\pgfqpoint{3.036511in}{1.391136in}}%
\pgfpathlineto{\pgfqpoint{3.039430in}{1.359989in}}%
\pgfpathlineto{\pgfqpoint{3.040299in}{1.328781in}}%
\pgfpathlineto{\pgfqpoint{3.039113in}{1.297575in}}%
\pgfpathlineto{\pgfqpoint{3.035877in}{1.266432in}}%
\pgfpathlineto{\pgfqpoint{3.030594in}{1.235414in}}%
\pgfpathlineto{\pgfqpoint{3.023276in}{1.204584in}}%
\pgfpathlineto{\pgfqpoint{3.013936in}{1.174002in}}%
\pgfpathlineto{\pgfqpoint{3.001210in}{1.140389in}}%
\pgfpathlineto{\pgfqpoint{2.986042in}{1.107240in}}%
\pgfpathlineto{\pgfqpoint{2.968468in}{1.074639in}}%
\pgfpathlineto{\pgfqpoint{2.948531in}{1.042666in}}%
\pgfpathlineto{\pgfqpoint{2.926279in}{1.011401in}}%
\pgfpathlineto{\pgfqpoint{2.901767in}{0.980920in}}%
\pgfpathlineto{\pgfqpoint{2.875055in}{0.951300in}}%
\pgfpathlineto{\pgfqpoint{2.846207in}{0.922616in}}%
\pgfpathlineto{\pgfqpoint{2.815296in}{0.894939in}}%
\pgfpathlineto{\pgfqpoint{2.782396in}{0.868338in}}%
\pgfpathlineto{\pgfqpoint{2.747589in}{0.842880in}}%
\pgfpathlineto{\pgfqpoint{2.710960in}{0.818631in}}%
\pgfpathlineto{\pgfqpoint{2.668672in}{0.793425in}}%
\pgfpathlineto{\pgfqpoint{2.624416in}{0.769832in}}%
\pgfpathlineto{\pgfqpoint{2.578323in}{0.747926in}}%
\pgfpathlineto{\pgfqpoint{2.530533in}{0.727774in}}%
\pgfpathlineto{\pgfqpoint{2.481188in}{0.709438in}}%
\pgfpathlineto{\pgfqpoint{2.430438in}{0.692977in}}%
\pgfpathlineto{\pgfqpoint{2.378433in}{0.678442in}}%
\pgfpathlineto{\pgfqpoint{2.325331in}{0.665877in}}%
\pgfpathlineto{\pgfqpoint{2.271293in}{0.655324in}}%
\pgfpathlineto{\pgfqpoint{2.216480in}{0.646815in}}%
\pgfpathlineto{\pgfqpoint{2.161058in}{0.640378in}}%
\pgfpathlineto{\pgfqpoint{2.105196in}{0.636032in}}%
\pgfpathlineto{\pgfqpoint{2.049060in}{0.633793in}}%
\pgfpathlineto{\pgfqpoint{1.992822in}{0.633668in}}%
\pgfpathlineto{\pgfqpoint{1.936652in}{0.635658in}}%
\pgfpathlineto{\pgfqpoint{1.880718in}{0.639758in}}%
\pgfpathlineto{\pgfqpoint{1.825191in}{0.645954in}}%
\pgfpathlineto{\pgfqpoint{1.770238in}{0.654228in}}%
\pgfpathlineto{\pgfqpoint{1.716025in}{0.664556in}}%
\pgfpathlineto{\pgfqpoint{1.662718in}{0.676905in}}%
\pgfpathlineto{\pgfqpoint{1.610476in}{0.691237in}}%
\pgfpathlineto{\pgfqpoint{1.559458in}{0.707508in}}%
\pgfpathlineto{\pgfqpoint{1.509817in}{0.725670in}}%
\pgfpathlineto{\pgfqpoint{1.461705in}{0.745664in}}%
\pgfpathlineto{\pgfqpoint{1.415265in}{0.767432in}}%
\pgfpathlineto{\pgfqpoint{1.370639in}{0.790906in}}%
\pgfpathlineto{\pgfqpoint{1.327961in}{0.816014in}}%
\pgfpathlineto{\pgfqpoint{1.287359in}{0.842680in}}%
\pgfpathlineto{\pgfqpoint{1.252353in}{0.868205in}}%
\pgfpathlineto{\pgfqpoint{1.219252in}{0.894888in}}%
\pgfpathlineto{\pgfqpoint{1.188137in}{0.922660in}}%
\pgfpathlineto{\pgfqpoint{1.159087in}{0.951453in}}%
\pgfpathlineto{\pgfqpoint{1.132173in}{0.981196in}}%
\pgfpathlineto{\pgfqpoint{1.107463in}{1.011813in}}%
\pgfpathlineto{\pgfqpoint{1.085018in}{1.043230in}}%
\pgfpathlineto{\pgfqpoint{1.064893in}{1.075368in}}%
\pgfpathlineto{\pgfqpoint{1.047140in}{1.108148in}}%
\pgfpathlineto{\pgfqpoint{1.031802in}{1.141489in}}%
\pgfpathlineto{\pgfqpoint{1.018917in}{1.175309in}}%
\pgfpathlineto{\pgfqpoint{1.009446in}{1.206088in}}%
\pgfpathlineto{\pgfqpoint{1.002007in}{1.237127in}}%
\pgfpathlineto{\pgfqpoint{0.996617in}{1.268362in}}%
\pgfpathlineto{\pgfqpoint{0.993286in}{1.299734in}}%
\pgfpathlineto{\pgfqpoint{0.992022in}{1.331178in}}%
\pgfpathlineto{\pgfqpoint{0.992827in}{1.362632in}}%
\pgfpathlineto{\pgfqpoint{0.995700in}{1.394035in}}%
\pgfpathlineto{\pgfqpoint{1.000636in}{1.425323in}}%
\pgfpathlineto{\pgfqpoint{1.007626in}{1.456434in}}%
\pgfpathlineto{\pgfqpoint{1.016657in}{1.487306in}}%
\pgfpathlineto{\pgfqpoint{1.027710in}{1.517877in}}%
\pgfpathlineto{\pgfqpoint{1.042337in}{1.551418in}}%
\pgfpathlineto{\pgfqpoint{1.059399in}{1.584430in}}%
\pgfpathlineto{\pgfqpoint{1.078854in}{1.616831in}}%
\pgfpathlineto{\pgfqpoint{1.100657in}{1.648541in}}%
\pgfpathlineto{\pgfqpoint{1.124752in}{1.679481in}}%
\pgfpathlineto{\pgfqpoint{1.151082in}{1.709574in}}%
\pgfpathlineto{\pgfqpoint{1.179581in}{1.738744in}}%
\pgfpathlineto{\pgfqpoint{1.210181in}{1.766919in}}%
\pgfpathlineto{\pgfqpoint{1.242806in}{1.794028in}}%
\pgfpathlineto{\pgfqpoint{1.277376in}{1.820002in}}%
\pgfpathlineto{\pgfqpoint{1.313805in}{1.844777in}}%
\pgfpathlineto{\pgfqpoint{1.352004in}{1.868288in}}%
\pgfpathlineto{\pgfqpoint{1.395955in}{1.892621in}}%
\pgfpathlineto{\pgfqpoint{1.441802in}{1.915279in}}%
\pgfpathlineto{\pgfqpoint{1.489408in}{1.936192in}}%
\pgfpathlineto{\pgfqpoint{1.538629in}{1.955294in}}%
\pgfpathlineto{\pgfqpoint{1.589318in}{1.972527in}}%
\pgfpathlineto{\pgfqpoint{1.641324in}{1.987835in}}%
\pgfpathlineto{\pgfqpoint{1.694488in}{2.001172in}}%
\pgfpathlineto{\pgfqpoint{1.748652in}{2.012495in}}%
\pgfpathlineto{\pgfqpoint{1.803651in}{2.021768in}}%
\pgfpathlineto{\pgfqpoint{1.859320in}{2.028962in}}%
\pgfpathlineto{\pgfqpoint{1.915491in}{2.034053in}}%
\pgfpathlineto{\pgfqpoint{1.971994in}{2.037026in}}%
\pgfpathlineto{\pgfqpoint{2.028658in}{2.037869in}}%
\pgfpathlineto{\pgfqpoint{2.085313in}{2.036581in}}%
\pgfpathlineto{\pgfqpoint{2.141785in}{2.033165in}}%
\pgfpathlineto{\pgfqpoint{2.197906in}{2.027630in}}%
\pgfpathlineto{\pgfqpoint{2.253504in}{2.019993in}}%
\pgfpathlineto{\pgfqpoint{2.308413in}{2.010278in}}%
\pgfpathlineto{\pgfqpoint{2.362465in}{1.998515in}}%
\pgfpathlineto{\pgfqpoint{2.415497in}{1.984738in}}%
\pgfpathlineto{\pgfqpoint{2.467349in}{1.968992in}}%
\pgfpathlineto{\pgfqpoint{2.517864in}{1.951324in}}%
\pgfpathlineto{\pgfqpoint{2.566890in}{1.931788in}}%
\pgfpathlineto{\pgfqpoint{2.614279in}{1.910444in}}%
\pgfpathlineto{\pgfqpoint{2.659886in}{1.887358in}}%
\pgfpathlineto{\pgfqpoint{2.703576in}{1.862600in}}%
\pgfpathlineto{\pgfqpoint{2.745217in}{1.836246in}}%
\pgfpathlineto{\pgfqpoint{2.781187in}{1.810969in}}%
\pgfpathlineto{\pgfqpoint{2.815270in}{1.784502in}}%
\pgfpathlineto{\pgfqpoint{2.847381in}{1.756912in}}%
\pgfpathlineto{\pgfqpoint{2.877439in}{1.728267in}}%
\pgfpathlineto{\pgfqpoint{2.905371in}{1.698638in}}%
\pgfpathlineto{\pgfqpoint{2.931107in}{1.668100in}}%
\pgfpathlineto{\pgfqpoint{2.954582in}{1.636728in}}%
\pgfpathlineto{\pgfqpoint{2.975739in}{1.604600in}}%
\pgfpathlineto{\pgfqpoint{2.994524in}{1.571795in}}%
\pgfpathlineto{\pgfqpoint{3.010890in}{1.538394in}}%
\pgfpathlineto{\pgfqpoint{3.024798in}{1.504480in}}%
\pgfpathlineto{\pgfqpoint{3.035184in}{1.473587in}}%
\pgfpathlineto{\pgfqpoint{3.043528in}{1.442407in}}%
\pgfpathlineto{\pgfqpoint{3.049814in}{1.411002in}}%
\pgfpathlineto{\pgfqpoint{3.054029in}{1.379435in}}%
\pgfpathlineto{\pgfqpoint{3.056163in}{1.347769in}}%
\pgfpathlineto{\pgfqpoint{3.056213in}{1.316066in}}%
\pgfpathlineto{\pgfqpoint{3.054177in}{1.284390in}}%
\pgfpathlineto{\pgfqpoint{3.050060in}{1.252803in}}%
\pgfpathlineto{\pgfqpoint{3.043868in}{1.221368in}}%
\pgfpathlineto{\pgfqpoint{3.035614in}{1.190149in}}%
\pgfpathlineto{\pgfqpoint{3.025314in}{1.159207in}}%
\pgfpathlineto{\pgfqpoint{3.012988in}{1.128603in}}%
\pgfpathlineto{\pgfqpoint{2.996946in}{1.095072in}}%
\pgfpathlineto{\pgfqpoint{2.978470in}{1.062117in}}%
\pgfpathlineto{\pgfqpoint{2.957607in}{1.029821in}}%
\pgfpathlineto{\pgfqpoint{2.934406in}{0.998264in}}%
\pgfpathlineto{\pgfqpoint{2.908925in}{0.967524in}}%
\pgfpathlineto{\pgfqpoint{2.881226in}{0.937678in}}%
\pgfpathlineto{\pgfqpoint{2.851376in}{0.908800in}}%
\pgfpathlineto{\pgfqpoint{2.819449in}{0.880963in}}%
\pgfpathlineto{\pgfqpoint{2.785523in}{0.854237in}}%
\pgfpathlineto{\pgfqpoint{2.749682in}{0.828689in}}%
\pgfpathlineto{\pgfqpoint{2.712014in}{0.804385in}}%
\pgfpathlineto{\pgfqpoint{2.668579in}{0.779159in}}%
\pgfpathlineto{\pgfqpoint{2.623176in}{0.755589in}}%
\pgfpathlineto{\pgfqpoint{2.575941in}{0.733748in}}%
\pgfpathlineto{\pgfqpoint{2.527014in}{0.713705in}}%
\pgfpathlineto{\pgfqpoint{2.476544in}{0.695521in}}%
\pgfpathlineto{\pgfqpoint{2.424680in}{0.679253in}}%
\pgfpathlineto{\pgfqpoint{2.371580in}{0.664952in}}%
\pgfpathlineto{\pgfqpoint{2.317404in}{0.652664in}}%
\pgfpathlineto{\pgfqpoint{2.262314in}{0.642426in}}%
\pgfpathlineto{\pgfqpoint{2.206477in}{0.634272in}}%
\pgfpathlineto{\pgfqpoint{2.150061in}{0.628228in}}%
\pgfpathlineto{\pgfqpoint{2.093236in}{0.624313in}}%
\pgfpathlineto{\pgfqpoint{2.036176in}{0.622540in}}%
\pgfpathlineto{\pgfqpoint{1.979051in}{0.622915in}}%
\pgfpathlineto{\pgfqpoint{1.922034in}{0.625437in}}%
\pgfpathlineto{\pgfqpoint{1.865299in}{0.630100in}}%
\pgfpathlineto{\pgfqpoint{1.809017in}{0.636889in}}%
\pgfpathlineto{\pgfqpoint{1.753357in}{0.645784in}}%
\pgfpathlineto{\pgfqpoint{1.698490in}{0.656757in}}%
\pgfpathlineto{\pgfqpoint{1.644579in}{0.669774in}}%
\pgfpathlineto{\pgfqpoint{1.591789in}{0.684797in}}%
\pgfpathlineto{\pgfqpoint{1.540279in}{0.701778in}}%
\pgfpathlineto{\pgfqpoint{1.490204in}{0.720665in}}%
\pgfpathlineto{\pgfqpoint{1.441716in}{0.741401in}}%
\pgfpathlineto{\pgfqpoint{1.394961in}{0.763921in}}%
\pgfpathlineto{\pgfqpoint{1.350080in}{0.788158in}}%
\pgfpathlineto{\pgfqpoint{1.307209in}{0.814037in}}%
\pgfpathlineto{\pgfqpoint{1.270088in}{0.838922in}}%
\pgfpathlineto{\pgfqpoint{1.234827in}{0.865037in}}%
\pgfpathlineto{\pgfqpoint{1.201513in}{0.892316in}}%
\pgfpathlineto{\pgfqpoint{1.170231in}{0.920691in}}%
\pgfpathlineto{\pgfqpoint{1.141056in}{0.950091in}}%
\pgfpathlineto{\pgfqpoint{1.114063in}{0.980443in}}%
\pgfpathlineto{\pgfqpoint{1.089318in}{1.011672in}}%
\pgfpathlineto{\pgfqpoint{1.066882in}{1.043700in}}%
\pgfpathlineto{\pgfqpoint{1.046813in}{1.076448in}}%
\pgfpathlineto{\pgfqpoint{1.029158in}{1.109836in}}%
\pgfpathlineto{\pgfqpoint{1.013963in}{1.143780in}}%
\pgfpathlineto{\pgfqpoint{1.002422in}{1.174737in}}%
\pgfpathlineto{\pgfqpoint{0.992928in}{1.206015in}}%
\pgfpathlineto{\pgfqpoint{0.985499in}{1.237553in}}%
\pgfpathlineto{\pgfqpoint{0.980151in}{1.269288in}}%
\pgfpathlineto{\pgfqpoint{0.976896in}{1.301156in}}%
\pgfpathlineto{\pgfqpoint{0.975740in}{1.333094in}}%
\pgfpathlineto{\pgfqpoint{0.976685in}{1.365039in}}%
\pgfpathlineto{\pgfqpoint{0.979731in}{1.396927in}}%
\pgfpathlineto{\pgfqpoint{0.984872in}{1.428695in}}%
\pgfpathlineto{\pgfqpoint{0.992098in}{1.460280in}}%
\pgfpathlineto{\pgfqpoint{1.001395in}{1.491618in}}%
\pgfpathlineto{\pgfqpoint{1.012745in}{1.522647in}}%
\pgfpathlineto{\pgfqpoint{1.026126in}{1.553306in}}%
\pgfpathlineto{\pgfqpoint{1.043344in}{1.586861in}}%
\pgfpathlineto{\pgfqpoint{1.062995in}{1.619800in}}%
\pgfpathlineto{\pgfqpoint{1.085032in}{1.652041in}}%
\pgfpathlineto{\pgfqpoint{1.109400in}{1.683504in}}%
\pgfpathlineto{\pgfqpoint{1.136041in}{1.714110in}}%
\pgfpathlineto{\pgfqpoint{1.164889in}{1.743783in}}%
\pgfpathlineto{\pgfqpoint{1.195873in}{1.772449in}}%
\pgfpathlineto{\pgfqpoint{1.228918in}{1.800035in}}%
\pgfpathlineto{\pgfqpoint{1.263942in}{1.826473in}}%
\pgfpathlineto{\pgfqpoint{1.300860in}{1.851696in}}%
\pgfpathlineto{\pgfqpoint{1.339579in}{1.875639in}}%
\pgfpathlineto{\pgfqpoint{1.384138in}{1.900426in}}%
\pgfpathlineto{\pgfqpoint{1.430628in}{1.923515in}}%
\pgfpathlineto{\pgfqpoint{1.478911in}{1.944834in}}%
\pgfpathlineto{\pgfqpoint{1.528841in}{1.964317in}}%
\pgfpathlineto{\pgfqpoint{1.580269in}{1.981904in}}%
\pgfpathlineto{\pgfqpoint{1.633041in}{1.997539in}}%
\pgfpathlineto{\pgfqpoint{1.686997in}{2.011173in}}%
\pgfpathlineto{\pgfqpoint{1.741974in}{2.022764in}}%
\pgfpathlineto{\pgfqpoint{1.797808in}{2.032274in}}%
\pgfpathlineto{\pgfqpoint{1.854330in}{2.039672in}}%
\pgfpathlineto{\pgfqpoint{1.911369in}{2.044937in}}%
\pgfpathlineto{\pgfqpoint{1.968752in}{2.048049in}}%
\pgfpathlineto{\pgfqpoint{2.026307in}{2.049000in}}%
\pgfpathlineto{\pgfqpoint{2.083859in}{2.047785in}}%
\pgfpathlineto{\pgfqpoint{2.141235in}{2.044408in}}%
\pgfpathlineto{\pgfqpoint{2.198260in}{2.038878in}}%
\pgfpathlineto{\pgfqpoint{2.254762in}{2.031213in}}%
\pgfpathlineto{\pgfqpoint{2.310570in}{2.021435in}}%
\pgfpathlineto{\pgfqpoint{2.365516in}{2.009575in}}%
\pgfpathlineto{\pgfqpoint{2.419433in}{1.995669in}}%
\pgfpathlineto{\pgfqpoint{2.472159in}{1.979760in}}%
\pgfpathlineto{\pgfqpoint{2.523533in}{1.961897in}}%
\pgfpathlineto{\pgfqpoint{2.573401in}{1.942135in}}%
\pgfpathlineto{\pgfqpoint{2.621612in}{1.920533in}}%
\pgfpathlineto{\pgfqpoint{2.668021in}{1.897159in}}%
\pgfpathlineto{\pgfqpoint{2.712488in}{1.872084in}}%
\pgfpathlineto{\pgfqpoint{2.751114in}{1.847877in}}%
\pgfpathlineto{\pgfqpoint{2.787927in}{1.822387in}}%
\pgfpathlineto{\pgfqpoint{2.822837in}{1.795680in}}%
\pgfpathlineto{\pgfqpoint{2.855756in}{1.767821in}}%
\pgfpathlineto{\pgfqpoint{2.886603in}{1.738881in}}%
\pgfpathlineto{\pgfqpoint{2.915301in}{1.708932in}}%
\pgfpathlineto{\pgfqpoint{2.941777in}{1.678047in}}%
\pgfpathlineto{\pgfqpoint{2.965968in}{1.646304in}}%
\pgfpathlineto{\pgfqpoint{2.987811in}{1.613782in}}%
\pgfpathlineto{\pgfqpoint{3.007254in}{1.580561in}}%
\pgfpathlineto{\pgfqpoint{3.024248in}{1.546722in}}%
\pgfpathlineto{\pgfqpoint{3.037412in}{1.515808in}}%
\pgfpathlineto{\pgfqpoint{3.048532in}{1.484523in}}%
\pgfpathlineto{\pgfqpoint{3.057584in}{1.452930in}}%
\pgfpathlineto{\pgfqpoint{3.064549in}{1.421092in}}%
\pgfpathlineto{\pgfqpoint{3.069415in}{1.389070in}}%
\pgfpathlineto{\pgfqpoint{3.072169in}{1.356931in}}%
\pgfpathlineto{\pgfqpoint{3.072808in}{1.324736in}}%
\pgfpathlineto{\pgfqpoint{3.071328in}{1.292550in}}%
\pgfpathlineto{\pgfqpoint{3.067732in}{1.260438in}}%
\pgfpathlineto{\pgfqpoint{3.062027in}{1.228463in}}%
\pgfpathlineto{\pgfqpoint{3.054224in}{1.196688in}}%
\pgfpathlineto{\pgfqpoint{3.044337in}{1.165177in}}%
\pgfpathlineto{\pgfqpoint{3.032387in}{1.133993in}}%
\pgfpathlineto{\pgfqpoint{3.018396in}{1.103199in}}%
\pgfpathlineto{\pgfqpoint{3.002392in}{1.072854in}}%
\pgfpathlineto{\pgfqpoint{2.982287in}{1.039741in}}%
\pgfpathlineto{\pgfqpoint{2.959784in}{1.007341in}}%
\pgfpathlineto{\pgfqpoint{2.934938in}{0.975736in}}%
\pgfpathlineto{\pgfqpoint{2.907810in}{0.945002in}}%
\pgfpathlineto{\pgfqpoint{2.878466in}{0.915219in}}%
\pgfpathlineto{\pgfqpoint{2.846978in}{0.886459in}}%
\pgfpathlineto{\pgfqpoint{2.813423in}{0.858796in}}%
\pgfpathlineto{\pgfqpoint{2.777884in}{0.832299in}}%
\pgfpathlineto{\pgfqpoint{2.740448in}{0.807034in}}%
\pgfpathlineto{\pgfqpoint{2.701207in}{0.783067in}}%
\pgfpathlineto{\pgfqpoint{2.660259in}{0.760457in}}%
\pgfpathlineto{\pgfqpoint{2.613364in}{0.737224in}}%
\pgfpathlineto{\pgfqpoint{2.564666in}{0.715776in}}%
\pgfpathlineto{\pgfqpoint{2.514310in}{0.696179in}}%
\pgfpathlineto{\pgfqpoint{2.462448in}{0.678494in}}%
\pgfpathlineto{\pgfqpoint{2.409235in}{0.662778in}}%
\pgfpathlineto{\pgfqpoint{2.354832in}{0.649080in}}%
\pgfpathlineto{\pgfqpoint{2.299403in}{0.637441in}}%
\pgfpathlineto{\pgfqpoint{2.243114in}{0.627901in}}%
\pgfpathlineto{\pgfqpoint{2.186136in}{0.620489in}}%
\pgfpathlineto{\pgfqpoint{2.128640in}{0.615228in}}%
\pgfpathlineto{\pgfqpoint{2.070800in}{0.612136in}}%
\pgfpathlineto{\pgfqpoint{2.012791in}{0.611223in}}%
\pgfpathlineto{\pgfqpoint{1.954788in}{0.612493in}}%
\pgfpathlineto{\pgfqpoint{1.896968in}{0.615943in}}%
\pgfpathlineto{\pgfqpoint{1.839504in}{0.621561in}}%
\pgfpathlineto{\pgfqpoint{1.782571in}{0.629331in}}%
\pgfpathlineto{\pgfqpoint{1.726340in}{0.639229in}}%
\pgfpathlineto{\pgfqpoint{1.670982in}{0.651226in}}%
\pgfpathlineto{\pgfqpoint{1.616665in}{0.665283in}}%
\pgfpathlineto{\pgfqpoint{1.563552in}{0.681358in}}%
\pgfpathlineto{\pgfqpoint{1.511804in}{0.699402in}}%
\pgfpathlineto{\pgfqpoint{1.461577in}{0.719358in}}%
\pgfpathlineto{\pgfqpoint{1.413023in}{0.741167in}}%
\pgfpathlineto{\pgfqpoint{1.366289in}{0.764760in}}%
\pgfpathlineto{\pgfqpoint{1.321515in}{0.790066in}}%
\pgfpathlineto{\pgfqpoint{1.282626in}{0.814493in}}%
\pgfpathlineto{\pgfqpoint{1.245565in}{0.840210in}}%
\pgfpathlineto{\pgfqpoint{1.210426in}{0.867154in}}%
\pgfpathlineto{\pgfqpoint{1.177294in}{0.895255in}}%
\pgfpathlineto{\pgfqpoint{1.146253in}{0.924445in}}%
\pgfpathlineto{\pgfqpoint{1.117380in}{0.954650in}}%
\pgfpathlineto{\pgfqpoint{1.090746in}{0.985796in}}%
\pgfpathlineto{\pgfqpoint{1.066418in}{1.017805in}}%
\pgfpathlineto{\pgfqpoint{1.044456in}{1.050597in}}%
\pgfpathlineto{\pgfqpoint{1.024915in}{1.084093in}}%
\pgfpathlineto{\pgfqpoint{1.007843in}{1.118208in}}%
\pgfpathlineto{\pgfqpoint{0.994626in}{1.149372in}}%
\pgfpathlineto{\pgfqpoint{0.983470in}{1.180909in}}%
\pgfpathlineto{\pgfqpoint{0.974398in}{1.212755in}}%
\pgfpathlineto{\pgfqpoint{0.967429in}{1.244847in}}%
\pgfpathlineto{\pgfqpoint{0.962577in}{1.277121in}}%
\pgfpathlineto{\pgfqpoint{0.959852in}{1.309513in}}%
\pgfpathlineto{\pgfqpoint{0.959261in}{1.341959in}}%
\pgfpathlineto{\pgfqpoint{0.960804in}{1.374394in}}%
\pgfpathlineto{\pgfqpoint{0.964479in}{1.406754in}}%
\pgfpathlineto{\pgfqpoint{0.970280in}{1.438974in}}%
\pgfpathlineto{\pgfqpoint{0.978195in}{1.470990in}}%
\pgfpathlineto{\pgfqpoint{0.988208in}{1.502739in}}%
\pgfpathlineto{\pgfqpoint{1.000302in}{1.534157in}}%
\pgfpathlineto{\pgfqpoint{1.014451in}{1.565182in}}%
\pgfpathlineto{\pgfqpoint{1.030629in}{1.595751in}}%
\pgfpathlineto{\pgfqpoint{1.050943in}{1.629108in}}%
\pgfpathlineto{\pgfqpoint{1.073674in}{1.661745in}}%
\pgfpathlineto{\pgfqpoint{1.098764in}{1.693579in}}%
\pgfpathlineto{\pgfqpoint{1.126153in}{1.724533in}}%
\pgfpathlineto{\pgfqpoint{1.155773in}{1.754528in}}%
\pgfpathlineto{\pgfqpoint{1.187553in}{1.783490in}}%
\pgfpathlineto{\pgfqpoint{1.221414in}{1.811346in}}%
\pgfpathlineto{\pgfqpoint{1.257273in}{1.838025in}}%
\pgfpathlineto{\pgfqpoint{1.295042in}{1.863460in}}%
\pgfpathlineto{\pgfqpoint{1.334628in}{1.887587in}}%
\pgfpathlineto{\pgfqpoint{1.375933in}{1.910345in}}%
\pgfpathlineto{\pgfqpoint{1.423232in}{1.933727in}}%
\pgfpathlineto{\pgfqpoint{1.472345in}{1.955309in}}%
\pgfpathlineto{\pgfqpoint{1.523127in}{1.975024in}}%
\pgfpathlineto{\pgfqpoint{1.575423in}{1.992810in}}%
\pgfpathlineto{\pgfqpoint{1.629078in}{2.008612in}}%
\pgfpathlineto{\pgfqpoint{1.683929in}{2.022379in}}%
\pgfpathlineto{\pgfqpoint{1.739812in}{2.034070in}}%
\pgfpathlineto{\pgfqpoint{1.796558in}{2.043645in}}%
\pgfpathlineto{\pgfqpoint{1.853995in}{2.051075in}}%
\pgfpathlineto{\pgfqpoint{1.911951in}{2.056336in}}%
\pgfpathlineto{\pgfqpoint{1.970250in}{2.059411in}}%
\pgfpathlineto{\pgfqpoint{2.028716in}{2.060290in}}%
\pgfpathlineto{\pgfqpoint{2.087173in}{2.058969in}}%
\pgfpathlineto{\pgfqpoint{2.145443in}{2.055452in}}%
\pgfpathlineto{\pgfqpoint{2.203350in}{2.049749in}}%
\pgfpathlineto{\pgfqpoint{2.260719in}{2.041877in}}%
\pgfpathlineto{\pgfqpoint{2.317376in}{2.031861in}}%
\pgfpathlineto{\pgfqpoint{2.373151in}{2.019731in}}%
\pgfpathlineto{\pgfqpoint{2.427874in}{2.005525in}}%
\pgfpathlineto{\pgfqpoint{2.481380in}{1.989286in}}%
\pgfpathlineto{\pgfqpoint{2.533507in}{1.971065in}}%
\pgfpathlineto{\pgfqpoint{2.584098in}{1.950916in}}%
\pgfpathlineto{\pgfqpoint{2.633001in}{1.928903in}}%
\pgfpathlineto{\pgfqpoint{2.680067in}{1.905092in}}%
\pgfpathlineto{\pgfqpoint{2.721141in}{1.881947in}}%
\pgfpathlineto{\pgfqpoint{2.760479in}{1.857435in}}%
\pgfpathlineto{\pgfqpoint{2.797980in}{1.831618in}}%
\pgfpathlineto{\pgfqpoint{2.833554in}{1.804560in}}%
\pgfpathlineto{\pgfqpoint{2.867110in}{1.776329in}}%
\pgfpathlineto{\pgfqpoint{2.898566in}{1.746996in}}%
\pgfpathlineto{\pgfqpoint{2.927843in}{1.716634in}}%
\pgfpathlineto{\pgfqpoint{2.954869in}{1.685318in}}%
\pgfpathlineto{\pgfqpoint{2.979576in}{1.653127in}}%
\pgfpathlineto{\pgfqpoint{3.001903in}{1.620139in}}%
\pgfpathlineto{\pgfqpoint{3.021795in}{1.586436in}}%
\pgfpathlineto{\pgfqpoint{3.037574in}{1.555561in}}%
\pgfpathlineto{\pgfqpoint{3.051308in}{1.524237in}}%
\pgfpathlineto{\pgfqpoint{3.062970in}{1.492525in}}%
\pgfpathlineto{\pgfqpoint{3.072536in}{1.460488in}}%
\pgfpathlineto{\pgfqpoint{3.079985in}{1.428191in}}%
\pgfpathlineto{\pgfqpoint{3.085304in}{1.395697in}}%
\pgfpathlineto{\pgfqpoint{3.088480in}{1.363072in}}%
\pgfpathlineto{\pgfqpoint{3.089508in}{1.330379in}}%
\pgfpathlineto{\pgfqpoint{3.088385in}{1.297684in}}%
\pgfpathlineto{\pgfqpoint{3.085111in}{1.265052in}}%
\pgfpathlineto{\pgfqpoint{3.079695in}{1.232548in}}%
\pgfpathlineto{\pgfqpoint{3.072145in}{1.200236in}}%
\pgfpathlineto{\pgfqpoint{3.062476in}{1.168180in}}%
\pgfpathlineto{\pgfqpoint{3.050707in}{1.136445in}}%
\pgfpathlineto{\pgfqpoint{3.036861in}{1.105094in}}%
\pgfpathlineto{\pgfqpoint{3.020965in}{1.074189in}}%
\pgfpathlineto{\pgfqpoint{3.003051in}{1.043793in}}%
\pgfpathlineto{\pgfqpoint{2.980821in}{1.010690in}}%
\pgfpathlineto{\pgfqpoint{2.956197in}{0.978372in}}%
\pgfpathlineto{\pgfqpoint{2.929238in}{0.946919in}}%
\pgfpathlineto{\pgfqpoint{2.900010in}{0.916410in}}%
\pgfpathlineto{\pgfqpoint{2.868585in}{0.886922in}}%
\pgfpathlineto{\pgfqpoint{2.835040in}{0.858527in}}%
\pgfpathlineto{\pgfqpoint{2.799457in}{0.831299in}}%
\pgfpathlineto{\pgfqpoint{2.761924in}{0.805305in}}%
\pgfpathlineto{\pgfqpoint{2.722533in}{0.780611in}}%
\pgfpathlineto{\pgfqpoint{2.681381in}{0.757281in}}%
\pgfpathlineto{\pgfqpoint{2.638571in}{0.735373in}}%
\pgfpathlineto{\pgfqpoint{2.589689in}{0.712985in}}%
\pgfpathlineto{\pgfqpoint{2.539075in}{0.692455in}}%
\pgfpathlineto{\pgfqpoint{2.486880in}{0.673849in}}%
\pgfpathlineto{\pgfqpoint{2.433260in}{0.657223in}}%
\pgfpathlineto{\pgfqpoint{2.378378in}{0.642630in}}%
\pgfpathlineto{\pgfqpoint{2.322397in}{0.630117in}}%
\pgfpathlineto{\pgfqpoint{2.265488in}{0.619723in}}%
\pgfpathlineto{\pgfqpoint{2.207822in}{0.611481in}}%
\pgfpathlineto{\pgfqpoint{2.149572in}{0.605418in}}%
\pgfpathlineto{\pgfqpoint{2.090915in}{0.601553in}}%
\pgfpathlineto{\pgfqpoint{2.032028in}{0.599898in}}%
\pgfpathlineto{\pgfqpoint{1.973088in}{0.600460in}}%
\pgfpathlineto{\pgfqpoint{1.914275in}{0.603238in}}%
\pgfpathlineto{\pgfqpoint{1.855766in}{0.608223in}}%
\pgfpathlineto{\pgfqpoint{1.797738in}{0.615400in}}%
\pgfpathlineto{\pgfqpoint{1.740367in}{0.624748in}}%
\pgfpathlineto{\pgfqpoint{1.683826in}{0.636238in}}%
\pgfpathlineto{\pgfqpoint{1.628286in}{0.649834in}}%
\pgfpathlineto{\pgfqpoint{1.573916in}{0.665494in}}%
\pgfpathlineto{\pgfqpoint{1.520878in}{0.683171in}}%
\pgfpathlineto{\pgfqpoint{1.469334in}{0.702810in}}%
\pgfpathlineto{\pgfqpoint{1.419440in}{0.724351in}}%
\pgfpathlineto{\pgfqpoint{1.371345in}{0.747728in}}%
\pgfpathlineto{\pgfqpoint{1.329306in}{0.770512in}}%
\pgfpathlineto{\pgfqpoint{1.288979in}{0.794697in}}%
\pgfpathlineto{\pgfqpoint{1.250465in}{0.820221in}}%
\pgfpathlineto{\pgfqpoint{1.213859in}{0.847021in}}%
\pgfpathlineto{\pgfqpoint{1.179252in}{0.875029in}}%
\pgfpathlineto{\pgfqpoint{1.146731in}{0.904175in}}%
\pgfpathlineto{\pgfqpoint{1.116376in}{0.934386in}}%
\pgfpathlineto{\pgfqpoint{1.088262in}{0.965588in}}%
\pgfpathlineto{\pgfqpoint{1.062460in}{0.997703in}}%
\pgfpathlineto{\pgfqpoint{1.039033in}{1.030651in}}%
\pgfpathlineto{\pgfqpoint{1.018041in}{1.064351in}}%
\pgfpathlineto{\pgfqpoint{1.001272in}{1.095255in}}%
\pgfpathlineto{\pgfqpoint{0.986550in}{1.126638in}}%
\pgfpathlineto{\pgfqpoint{0.973906in}{1.158439in}}%
\pgfpathlineto{\pgfqpoint{0.963364in}{1.190594in}}%
\pgfpathlineto{\pgfqpoint{0.954947in}{1.223039in}}%
\pgfpathlineto{\pgfqpoint{0.948672in}{1.255709in}}%
\pgfpathlineto{\pgfqpoint{0.944551in}{1.288540in}}%
\pgfpathlineto{\pgfqpoint{0.942593in}{1.321466in}}%
\pgfpathlineto{\pgfqpoint{0.942803in}{1.354422in}}%
\pgfpathlineto{\pgfqpoint{0.945181in}{1.387343in}}%
\pgfpathlineto{\pgfqpoint{0.949722in}{1.420163in}}%
\pgfpathlineto{\pgfqpoint{0.956417in}{1.452817in}}%
\pgfpathlineto{\pgfqpoint{0.965255in}{1.485240in}}%
\pgfpathlineto{\pgfqpoint{0.976218in}{1.517367in}}%
\pgfpathlineto{\pgfqpoint{0.989284in}{1.549135in}}%
\pgfpathlineto{\pgfqpoint{1.004428in}{1.580480in}}%
\pgfpathlineto{\pgfqpoint{1.021620in}{1.611339in}}%
\pgfpathlineto{\pgfqpoint{1.040828in}{1.641650in}}%
\pgfpathlineto{\pgfqpoint{1.064486in}{1.674613in}}%
\pgfpathlineto{\pgfqpoint{1.090527in}{1.706743in}}%
\pgfpathlineto{\pgfqpoint{1.118887in}{1.737960in}}%
\pgfpathlineto{\pgfqpoint{1.149496in}{1.768185in}}%
\pgfpathlineto{\pgfqpoint{1.182281in}{1.797344in}}%
\pgfpathlineto{\pgfqpoint{1.217161in}{1.825362in}}%
\pgfpathlineto{\pgfqpoint{1.254049in}{1.852170in}}%
\pgfpathlineto{\pgfqpoint{1.292855in}{1.877699in}}%
\pgfpathlineto{\pgfqpoint{1.333484in}{1.901884in}}%
\pgfpathlineto{\pgfqpoint{1.375835in}{1.924665in}}%
\pgfpathlineto{\pgfqpoint{1.419803in}{1.945982in}}%
\pgfpathlineto{\pgfqpoint{1.469907in}{1.967675in}}%
\pgfpathlineto{\pgfqpoint{1.521685in}{1.987464in}}%
\pgfpathlineto{\pgfqpoint{1.574984in}{2.005287in}}%
\pgfpathlineto{\pgfqpoint{1.629642in}{2.021089in}}%
\pgfpathlineto{\pgfqpoint{1.685496in}{2.034818in}}%
\pgfpathlineto{\pgfqpoint{1.742376in}{2.046432in}}%
\pgfpathlineto{\pgfqpoint{1.800112in}{2.055894in}}%
\pgfpathlineto{\pgfqpoint{1.858529in}{2.063174in}}%
\pgfpathlineto{\pgfqpoint{1.917452in}{2.068248in}}%
\pgfpathlineto{\pgfqpoint{1.976701in}{2.071100in}}%
\pgfpathlineto{\pgfqpoint{2.036097in}{2.071720in}}%
\pgfpathlineto{\pgfqpoint{2.095463in}{2.070106in}}%
\pgfpathlineto{\pgfqpoint{2.154616in}{2.066262in}}%
\pgfpathlineto{\pgfqpoint{2.213379in}{2.060199in}}%
\pgfpathlineto{\pgfqpoint{2.271575in}{2.051936in}}%
\pgfpathlineto{\pgfqpoint{2.329026in}{2.041498in}}%
\pgfpathlineto{\pgfqpoint{2.385559in}{2.028918in}}%
\pgfpathlineto{\pgfqpoint{2.441004in}{2.014234in}}%
\pgfpathlineto{\pgfqpoint{2.495192in}{1.997491in}}%
\pgfpathlineto{\pgfqpoint{2.547960in}{1.978741in}}%
\pgfpathlineto{\pgfqpoint{2.599148in}{1.958041in}}%
\pgfpathlineto{\pgfqpoint{2.648603in}{1.935456in}}%
\pgfpathlineto{\pgfqpoint{2.691931in}{1.913345in}}%
\pgfpathlineto{\pgfqpoint{2.733596in}{1.889788in}}%
\pgfpathlineto{\pgfqpoint{2.773492in}{1.864845in}}%
\pgfpathlineto{\pgfqpoint{2.811521in}{1.838579in}}%
\pgfpathlineto{\pgfqpoint{2.847588in}{1.811056in}}%
\pgfpathlineto{\pgfqpoint{2.881603in}{1.782343in}}%
\pgfpathlineto{\pgfqpoint{2.913482in}{1.752514in}}%
\pgfpathlineto{\pgfqpoint{2.943146in}{1.721643in}}%
\pgfpathlineto{\pgfqpoint{2.970521in}{1.689806in}}%
\pgfpathlineto{\pgfqpoint{2.995539in}{1.657082in}}%
\pgfpathlineto{\pgfqpoint{3.018138in}{1.623553in}}%
\pgfpathlineto{\pgfqpoint{3.030512in}{1.603083in}}%
\pgfpathlineto{\pgfqpoint{3.030512in}{1.603083in}}%
\pgfusepath{stroke}%
\end{pgfscope}%
\begin{pgfscope}%
\pgfsetrectcap%
\pgfsetmiterjoin%
\pgfsetlinewidth{0.803000pt}%
\definecolor{currentstroke}{rgb}{0.000000,0.000000,0.000000}%
\pgfsetstrokecolor{currentstroke}%
\pgfsetdash{}{0pt}%
\pgfpathmoveto{\pgfqpoint{0.835065in}{0.526234in}}%
\pgfpathlineto{\pgfqpoint{0.835065in}{2.145371in}}%
\pgfusepath{stroke}%
\end{pgfscope}%
\begin{pgfscope}%
\pgfsetrectcap%
\pgfsetmiterjoin%
\pgfsetlinewidth{0.803000pt}%
\definecolor{currentstroke}{rgb}{0.000000,0.000000,0.000000}%
\pgfsetstrokecolor{currentstroke}%
\pgfsetdash{}{0pt}%
\pgfpathmoveto{\pgfqpoint{3.196863in}{0.526234in}}%
\pgfpathlineto{\pgfqpoint{3.196863in}{2.145371in}}%
\pgfusepath{stroke}%
\end{pgfscope}%
\begin{pgfscope}%
\pgfsetrectcap%
\pgfsetmiterjoin%
\pgfsetlinewidth{0.803000pt}%
\definecolor{currentstroke}{rgb}{0.000000,0.000000,0.000000}%
\pgfsetstrokecolor{currentstroke}%
\pgfsetdash{}{0pt}%
\pgfpathmoveto{\pgfqpoint{0.835065in}{0.526234in}}%
\pgfpathlineto{\pgfqpoint{3.196863in}{0.526234in}}%
\pgfusepath{stroke}%
\end{pgfscope}%
\begin{pgfscope}%
\pgfsetrectcap%
\pgfsetmiterjoin%
\pgfsetlinewidth{0.803000pt}%
\definecolor{currentstroke}{rgb}{0.000000,0.000000,0.000000}%
\pgfsetstrokecolor{currentstroke}%
\pgfsetdash{}{0pt}%
\pgfpathmoveto{\pgfqpoint{0.835065in}{2.145371in}}%
\pgfpathlineto{\pgfqpoint{3.196863in}{2.145371in}}%
\pgfusepath{stroke}%
\end{pgfscope}%
\begin{pgfscope}%
\definecolor{textcolor}{rgb}{0.000000,0.000000,0.000000}%
\pgfsetstrokecolor{textcolor}%
\pgfsetfillcolor{textcolor}%
\pgftext[x=2.015964in,y=2.228704in,,base]{\color{textcolor}\rmfamily\fontsize{12.000000}{14.400000}\selectfont phase plot}%
\end{pgfscope}%
\begin{pgfscope}%
\pgfsetbuttcap%
\pgfsetmiterjoin%
\definecolor{currentfill}{rgb}{1.000000,1.000000,1.000000}%
\pgfsetfillcolor{currentfill}%
\pgfsetlinewidth{0.000000pt}%
\definecolor{currentstroke}{rgb}{0.000000,0.000000,0.000000}%
\pgfsetstrokecolor{currentstroke}%
\pgfsetstrokeopacity{0.000000}%
\pgfsetdash{}{0pt}%
\pgfpathmoveto{\pgfqpoint{3.906113in}{0.526234in}}%
\pgfpathlineto{\pgfqpoint{6.267911in}{0.526234in}}%
\pgfpathlineto{\pgfqpoint{6.267911in}{2.145371in}}%
\pgfpathlineto{\pgfqpoint{3.906113in}{2.145371in}}%
\pgfpathclose%
\pgfusepath{fill}%
\end{pgfscope}%
\begin{pgfscope}%
\pgfsetbuttcap%
\pgfsetroundjoin%
\definecolor{currentfill}{rgb}{0.000000,0.000000,0.000000}%
\pgfsetfillcolor{currentfill}%
\pgfsetlinewidth{0.803000pt}%
\definecolor{currentstroke}{rgb}{0.000000,0.000000,0.000000}%
\pgfsetstrokecolor{currentstroke}%
\pgfsetdash{}{0pt}%
\pgfsys@defobject{currentmarker}{\pgfqpoint{0.000000in}{-0.048611in}}{\pgfqpoint{0.000000in}{0.000000in}}{%
\pgfpathmoveto{\pgfqpoint{0.000000in}{0.000000in}}%
\pgfpathlineto{\pgfqpoint{0.000000in}{-0.048611in}}%
\pgfusepath{stroke,fill}%
}%
\begin{pgfscope}%
\pgfsys@transformshift{4.013467in}{0.526234in}%
\pgfsys@useobject{currentmarker}{}%
\end{pgfscope}%
\end{pgfscope}%
\begin{pgfscope}%
\definecolor{textcolor}{rgb}{0.000000,0.000000,0.000000}%
\pgfsetstrokecolor{textcolor}%
\pgfsetfillcolor{textcolor}%
\pgftext[x=4.013467in,y=0.429012in,,top]{\color{textcolor}\rmfamily\fontsize{10.000000}{12.000000}\selectfont \(\displaystyle 0.0\)}%
\end{pgfscope}%
\begin{pgfscope}%
\pgfsetbuttcap%
\pgfsetroundjoin%
\definecolor{currentfill}{rgb}{0.000000,0.000000,0.000000}%
\pgfsetfillcolor{currentfill}%
\pgfsetlinewidth{0.803000pt}%
\definecolor{currentstroke}{rgb}{0.000000,0.000000,0.000000}%
\pgfsetstrokecolor{currentstroke}%
\pgfsetdash{}{0pt}%
\pgfsys@defobject{currentmarker}{\pgfqpoint{0.000000in}{-0.048611in}}{\pgfqpoint{0.000000in}{0.000000in}}{%
\pgfpathmoveto{\pgfqpoint{0.000000in}{0.000000in}}%
\pgfpathlineto{\pgfqpoint{0.000000in}{-0.048611in}}%
\pgfusepath{stroke,fill}%
}%
\begin{pgfscope}%
\pgfsys@transformshift{4.550240in}{0.526234in}%
\pgfsys@useobject{currentmarker}{}%
\end{pgfscope}%
\end{pgfscope}%
\begin{pgfscope}%
\definecolor{textcolor}{rgb}{0.000000,0.000000,0.000000}%
\pgfsetstrokecolor{textcolor}%
\pgfsetfillcolor{textcolor}%
\pgftext[x=4.550240in,y=0.429012in,,top]{\color{textcolor}\rmfamily\fontsize{10.000000}{12.000000}\selectfont \(\displaystyle 2.5\)}%
\end{pgfscope}%
\begin{pgfscope}%
\pgfsetbuttcap%
\pgfsetroundjoin%
\definecolor{currentfill}{rgb}{0.000000,0.000000,0.000000}%
\pgfsetfillcolor{currentfill}%
\pgfsetlinewidth{0.803000pt}%
\definecolor{currentstroke}{rgb}{0.000000,0.000000,0.000000}%
\pgfsetstrokecolor{currentstroke}%
\pgfsetdash{}{0pt}%
\pgfsys@defobject{currentmarker}{\pgfqpoint{0.000000in}{-0.048611in}}{\pgfqpoint{0.000000in}{0.000000in}}{%
\pgfpathmoveto{\pgfqpoint{0.000000in}{0.000000in}}%
\pgfpathlineto{\pgfqpoint{0.000000in}{-0.048611in}}%
\pgfusepath{stroke,fill}%
}%
\begin{pgfscope}%
\pgfsys@transformshift{5.087012in}{0.526234in}%
\pgfsys@useobject{currentmarker}{}%
\end{pgfscope}%
\end{pgfscope}%
\begin{pgfscope}%
\definecolor{textcolor}{rgb}{0.000000,0.000000,0.000000}%
\pgfsetstrokecolor{textcolor}%
\pgfsetfillcolor{textcolor}%
\pgftext[x=5.087012in,y=0.429012in,,top]{\color{textcolor}\rmfamily\fontsize{10.000000}{12.000000}\selectfont \(\displaystyle 5.0\)}%
\end{pgfscope}%
\begin{pgfscope}%
\pgfsetbuttcap%
\pgfsetroundjoin%
\definecolor{currentfill}{rgb}{0.000000,0.000000,0.000000}%
\pgfsetfillcolor{currentfill}%
\pgfsetlinewidth{0.803000pt}%
\definecolor{currentstroke}{rgb}{0.000000,0.000000,0.000000}%
\pgfsetstrokecolor{currentstroke}%
\pgfsetdash{}{0pt}%
\pgfsys@defobject{currentmarker}{\pgfqpoint{0.000000in}{-0.048611in}}{\pgfqpoint{0.000000in}{0.000000in}}{%
\pgfpathmoveto{\pgfqpoint{0.000000in}{0.000000in}}%
\pgfpathlineto{\pgfqpoint{0.000000in}{-0.048611in}}%
\pgfusepath{stroke,fill}%
}%
\begin{pgfscope}%
\pgfsys@transformshift{5.623784in}{0.526234in}%
\pgfsys@useobject{currentmarker}{}%
\end{pgfscope}%
\end{pgfscope}%
\begin{pgfscope}%
\definecolor{textcolor}{rgb}{0.000000,0.000000,0.000000}%
\pgfsetstrokecolor{textcolor}%
\pgfsetfillcolor{textcolor}%
\pgftext[x=5.623784in,y=0.429012in,,top]{\color{textcolor}\rmfamily\fontsize{10.000000}{12.000000}\selectfont \(\displaystyle 7.5\)}%
\end{pgfscope}%
\begin{pgfscope}%
\pgfsetbuttcap%
\pgfsetroundjoin%
\definecolor{currentfill}{rgb}{0.000000,0.000000,0.000000}%
\pgfsetfillcolor{currentfill}%
\pgfsetlinewidth{0.803000pt}%
\definecolor{currentstroke}{rgb}{0.000000,0.000000,0.000000}%
\pgfsetstrokecolor{currentstroke}%
\pgfsetdash{}{0pt}%
\pgfsys@defobject{currentmarker}{\pgfqpoint{0.000000in}{-0.048611in}}{\pgfqpoint{0.000000in}{0.000000in}}{%
\pgfpathmoveto{\pgfqpoint{0.000000in}{0.000000in}}%
\pgfpathlineto{\pgfqpoint{0.000000in}{-0.048611in}}%
\pgfusepath{stroke,fill}%
}%
\begin{pgfscope}%
\pgfsys@transformshift{6.160557in}{0.526234in}%
\pgfsys@useobject{currentmarker}{}%
\end{pgfscope}%
\end{pgfscope}%
\begin{pgfscope}%
\definecolor{textcolor}{rgb}{0.000000,0.000000,0.000000}%
\pgfsetstrokecolor{textcolor}%
\pgfsetfillcolor{textcolor}%
\pgftext[x=6.160557in,y=0.429012in,,top]{\color{textcolor}\rmfamily\fontsize{10.000000}{12.000000}\selectfont \(\displaystyle 10.0\)}%
\end{pgfscope}%
\begin{pgfscope}%
\definecolor{textcolor}{rgb}{0.000000,0.000000,0.000000}%
\pgfsetstrokecolor{textcolor}%
\pgfsetfillcolor{textcolor}%
\pgftext[x=5.087012in,y=0.250000in,,top]{\color{textcolor}\rmfamily\fontsize{10.000000}{12.000000}\selectfont time (s)}%
\end{pgfscope}%
\begin{pgfscope}%
\pgfsetbuttcap%
\pgfsetroundjoin%
\definecolor{currentfill}{rgb}{0.000000,0.000000,0.000000}%
\pgfsetfillcolor{currentfill}%
\pgfsetlinewidth{0.803000pt}%
\definecolor{currentstroke}{rgb}{0.000000,0.000000,0.000000}%
\pgfsetstrokecolor{currentstroke}%
\pgfsetdash{}{0pt}%
\pgfsys@defobject{currentmarker}{\pgfqpoint{-0.048611in}{0.000000in}}{\pgfqpoint{0.000000in}{0.000000in}}{%
\pgfpathmoveto{\pgfqpoint{0.000000in}{0.000000in}}%
\pgfpathlineto{\pgfqpoint{-0.048611in}{0.000000in}}%
\pgfusepath{stroke,fill}%
}%
\begin{pgfscope}%
\pgfsys@transformshift{3.906113in}{0.824262in}%
\pgfsys@useobject{currentmarker}{}%
\end{pgfscope}%
\end{pgfscope}%
\begin{pgfscope}%
\definecolor{textcolor}{rgb}{0.000000,0.000000,0.000000}%
\pgfsetstrokecolor{textcolor}%
\pgfsetfillcolor{textcolor}%
\pgftext[x=3.631421in,y=0.776037in,left,base]{\color{textcolor}\rmfamily\fontsize{10.000000}{12.000000}\selectfont \(\displaystyle 0.2\)}%
\end{pgfscope}%
\begin{pgfscope}%
\pgfsetbuttcap%
\pgfsetroundjoin%
\definecolor{currentfill}{rgb}{0.000000,0.000000,0.000000}%
\pgfsetfillcolor{currentfill}%
\pgfsetlinewidth{0.803000pt}%
\definecolor{currentstroke}{rgb}{0.000000,0.000000,0.000000}%
\pgfsetstrokecolor{currentstroke}%
\pgfsetdash{}{0pt}%
\pgfsys@defobject{currentmarker}{\pgfqpoint{-0.048611in}{0.000000in}}{\pgfqpoint{0.000000in}{0.000000in}}{%
\pgfpathmoveto{\pgfqpoint{0.000000in}{0.000000in}}%
\pgfpathlineto{\pgfqpoint{-0.048611in}{0.000000in}}%
\pgfusepath{stroke,fill}%
}%
\begin{pgfscope}%
\pgfsys@transformshift{3.906113in}{1.263323in}%
\pgfsys@useobject{currentmarker}{}%
\end{pgfscope}%
\end{pgfscope}%
\begin{pgfscope}%
\definecolor{textcolor}{rgb}{0.000000,0.000000,0.000000}%
\pgfsetstrokecolor{textcolor}%
\pgfsetfillcolor{textcolor}%
\pgftext[x=3.631421in,y=1.215098in,left,base]{\color{textcolor}\rmfamily\fontsize{10.000000}{12.000000}\selectfont \(\displaystyle 0.3\)}%
\end{pgfscope}%
\begin{pgfscope}%
\pgfsetbuttcap%
\pgfsetroundjoin%
\definecolor{currentfill}{rgb}{0.000000,0.000000,0.000000}%
\pgfsetfillcolor{currentfill}%
\pgfsetlinewidth{0.803000pt}%
\definecolor{currentstroke}{rgb}{0.000000,0.000000,0.000000}%
\pgfsetstrokecolor{currentstroke}%
\pgfsetdash{}{0pt}%
\pgfsys@defobject{currentmarker}{\pgfqpoint{-0.048611in}{0.000000in}}{\pgfqpoint{0.000000in}{0.000000in}}{%
\pgfpathmoveto{\pgfqpoint{0.000000in}{0.000000in}}%
\pgfpathlineto{\pgfqpoint{-0.048611in}{0.000000in}}%
\pgfusepath{stroke,fill}%
}%
\begin{pgfscope}%
\pgfsys@transformshift{3.906113in}{1.702384in}%
\pgfsys@useobject{currentmarker}{}%
\end{pgfscope}%
\end{pgfscope}%
\begin{pgfscope}%
\definecolor{textcolor}{rgb}{0.000000,0.000000,0.000000}%
\pgfsetstrokecolor{textcolor}%
\pgfsetfillcolor{textcolor}%
\pgftext[x=3.631421in,y=1.654158in,left,base]{\color{textcolor}\rmfamily\fontsize{10.000000}{12.000000}\selectfont \(\displaystyle 0.4\)}%
\end{pgfscope}%
\begin{pgfscope}%
\pgfsetbuttcap%
\pgfsetroundjoin%
\definecolor{currentfill}{rgb}{0.000000,0.000000,0.000000}%
\pgfsetfillcolor{currentfill}%
\pgfsetlinewidth{0.803000pt}%
\definecolor{currentstroke}{rgb}{0.000000,0.000000,0.000000}%
\pgfsetstrokecolor{currentstroke}%
\pgfsetdash{}{0pt}%
\pgfsys@defobject{currentmarker}{\pgfqpoint{-0.048611in}{0.000000in}}{\pgfqpoint{0.000000in}{0.000000in}}{%
\pgfpathmoveto{\pgfqpoint{0.000000in}{0.000000in}}%
\pgfpathlineto{\pgfqpoint{-0.048611in}{0.000000in}}%
\pgfusepath{stroke,fill}%
}%
\begin{pgfscope}%
\pgfsys@transformshift{3.906113in}{2.141445in}%
\pgfsys@useobject{currentmarker}{}%
\end{pgfscope}%
\end{pgfscope}%
\begin{pgfscope}%
\definecolor{textcolor}{rgb}{0.000000,0.000000,0.000000}%
\pgfsetstrokecolor{textcolor}%
\pgfsetfillcolor{textcolor}%
\pgftext[x=3.631421in,y=2.093219in,left,base]{\color{textcolor}\rmfamily\fontsize{10.000000}{12.000000}\selectfont \(\displaystyle 0.5\)}%
\end{pgfscope}%
\begin{pgfscope}%
\definecolor{textcolor}{rgb}{0.000000,0.000000,0.000000}%
\pgfsetstrokecolor{textcolor}%
\pgfsetfillcolor{textcolor}%
\pgftext[x=3.575865in,y=1.335803in,,bottom,rotate=90.000000]{\color{textcolor}\rmfamily\fontsize{10.000000}{12.000000}\selectfont energy (J)}%
\end{pgfscope}%
\begin{pgfscope}%
\pgfpathrectangle{\pgfqpoint{3.906113in}{0.526234in}}{\pgfqpoint{2.361798in}{1.619136in}}%
\pgfusepath{clip}%
\pgfsetrectcap%
\pgfsetroundjoin%
\pgfsetlinewidth{1.505625pt}%
\definecolor{currentstroke}{rgb}{0.000000,0.000000,1.000000}%
\pgfsetstrokecolor{currentstroke}%
\pgfsetdash{}{0pt}%
\pgfpathmoveto{\pgfqpoint{4.013467in}{0.599832in}}%
\pgfpathlineto{\pgfqpoint{4.014970in}{0.600834in}}%
\pgfpathlineto{\pgfqpoint{4.016903in}{0.605726in}}%
\pgfpathlineto{\pgfqpoint{4.019694in}{0.619894in}}%
\pgfpathlineto{\pgfqpoint{4.023559in}{0.652946in}}%
\pgfpathlineto{\pgfqpoint{4.028497in}{0.716298in}}%
\pgfpathlineto{\pgfqpoint{4.035153in}{0.833571in}}%
\pgfpathlineto{\pgfqpoint{4.045030in}{1.052357in}}%
\pgfpathlineto{\pgfqpoint{4.062636in}{1.443119in}}%
\pgfpathlineto{\pgfqpoint{4.069721in}{1.552605in}}%
\pgfpathlineto{\pgfqpoint{4.074874in}{1.603136in}}%
\pgfpathlineto{\pgfqpoint{4.078739in}{1.622807in}}%
\pgfpathlineto{\pgfqpoint{4.081315in}{1.626840in}}%
\pgfpathlineto{\pgfqpoint{4.082818in}{1.625798in}}%
\pgfpathlineto{\pgfqpoint{4.084751in}{1.620791in}}%
\pgfpathlineto{\pgfqpoint{4.087542in}{1.606364in}}%
\pgfpathlineto{\pgfqpoint{4.091407in}{1.572810in}}%
\pgfpathlineto{\pgfqpoint{4.096345in}{1.508679in}}%
\pgfpathlineto{\pgfqpoint{4.103001in}{1.390407in}}%
\pgfpathlineto{\pgfqpoint{4.113307in}{1.160936in}}%
\pgfpathlineto{\pgfqpoint{4.129840in}{0.795967in}}%
\pgfpathlineto{\pgfqpoint{4.136925in}{0.685942in}}%
\pgfpathlineto{\pgfqpoint{4.142293in}{0.633264in}}%
\pgfpathlineto{\pgfqpoint{4.146158in}{0.613950in}}%
\pgfpathlineto{\pgfqpoint{4.148519in}{0.610185in}}%
\pgfpathlineto{\pgfqpoint{4.150022in}{0.611005in}}%
\pgfpathlineto{\pgfqpoint{4.151955in}{0.615730in}}%
\pgfpathlineto{\pgfqpoint{4.154746in}{0.629774in}}%
\pgfpathlineto{\pgfqpoint{4.158611in}{0.662886in}}%
\pgfpathlineto{\pgfqpoint{4.163549in}{0.726704in}}%
\pgfpathlineto{\pgfqpoint{4.170205in}{0.845251in}}%
\pgfpathlineto{\pgfqpoint{4.180082in}{1.067045in}}%
\pgfpathlineto{\pgfqpoint{4.197902in}{1.468576in}}%
\pgfpathlineto{\pgfqpoint{4.204988in}{1.579196in}}%
\pgfpathlineto{\pgfqpoint{4.210141in}{1.630005in}}%
\pgfpathlineto{\pgfqpoint{4.214006in}{1.649568in}}%
\pgfpathlineto{\pgfqpoint{4.216367in}{1.653347in}}%
\pgfpathlineto{\pgfqpoint{4.217870in}{1.652485in}}%
\pgfpathlineto{\pgfqpoint{4.219803in}{1.647650in}}%
\pgfpathlineto{\pgfqpoint{4.222594in}{1.633352in}}%
\pgfpathlineto{\pgfqpoint{4.226459in}{1.599737in}}%
\pgfpathlineto{\pgfqpoint{4.231397in}{1.535135in}}%
\pgfpathlineto{\pgfqpoint{4.238053in}{1.415569in}}%
\pgfpathlineto{\pgfqpoint{4.248144in}{1.188056in}}%
\pgfpathlineto{\pgfqpoint{4.265106in}{0.807570in}}%
\pgfpathlineto{\pgfqpoint{4.272192in}{0.696395in}}%
\pgfpathlineto{\pgfqpoint{4.277559in}{0.643420in}}%
\pgfpathlineto{\pgfqpoint{4.281424in}{0.624213in}}%
\pgfpathlineto{\pgfqpoint{4.283786in}{0.620645in}}%
\pgfpathlineto{\pgfqpoint{4.285289in}{0.621640in}}%
\pgfpathlineto{\pgfqpoint{4.287221in}{0.626648in}}%
\pgfpathlineto{\pgfqpoint{4.290013in}{0.641209in}}%
\pgfpathlineto{\pgfqpoint{4.293877in}{0.675233in}}%
\pgfpathlineto{\pgfqpoint{4.298816in}{0.740504in}}%
\pgfpathlineto{\pgfqpoint{4.305472in}{0.861397in}}%
\pgfpathlineto{\pgfqpoint{4.315348in}{1.087049in}}%
\pgfpathlineto{\pgfqpoint{4.333169in}{1.494379in}}%
\pgfpathlineto{\pgfqpoint{4.340254in}{1.606163in}}%
\pgfpathlineto{\pgfqpoint{4.345407in}{1.657267in}}%
\pgfpathlineto{\pgfqpoint{4.349272in}{1.676732in}}%
\pgfpathlineto{\pgfqpoint{4.351634in}{1.680316in}}%
\pgfpathlineto{\pgfqpoint{4.353137in}{1.679278in}}%
\pgfpathlineto{\pgfqpoint{4.355069in}{1.674160in}}%
\pgfpathlineto{\pgfqpoint{4.357861in}{1.659342in}}%
\pgfpathlineto{\pgfqpoint{4.361725in}{1.624809in}}%
\pgfpathlineto{\pgfqpoint{4.366664in}{1.558742in}}%
\pgfpathlineto{\pgfqpoint{4.373320in}{1.436818in}}%
\pgfpathlineto{\pgfqpoint{4.383626in}{1.200137in}}%
\pgfpathlineto{\pgfqpoint{4.400158in}{0.823457in}}%
\pgfpathlineto{\pgfqpoint{4.407244in}{0.709797in}}%
\pgfpathlineto{\pgfqpoint{4.412611in}{0.655312in}}%
\pgfpathlineto{\pgfqpoint{4.416476in}{0.635276in}}%
\pgfpathlineto{\pgfqpoint{4.418838in}{0.631323in}}%
\pgfpathlineto{\pgfqpoint{4.420341in}{0.632123in}}%
\pgfpathlineto{\pgfqpoint{4.422273in}{0.636941in}}%
\pgfpathlineto{\pgfqpoint{4.425064in}{0.651347in}}%
\pgfpathlineto{\pgfqpoint{4.428929in}{0.685397in}}%
\pgfpathlineto{\pgfqpoint{4.433868in}{0.751107in}}%
\pgfpathlineto{\pgfqpoint{4.440523in}{0.873270in}}%
\pgfpathlineto{\pgfqpoint{4.450400in}{1.101988in}}%
\pgfpathlineto{\pgfqpoint{4.468221in}{1.516445in}}%
\pgfpathlineto{\pgfqpoint{4.475306in}{1.630770in}}%
\pgfpathlineto{\pgfqpoint{4.480459in}{1.683359in}}%
\pgfpathlineto{\pgfqpoint{4.484324in}{1.703677in}}%
\pgfpathlineto{\pgfqpoint{4.486686in}{1.707658in}}%
\pgfpathlineto{\pgfqpoint{4.488189in}{1.706822in}}%
\pgfpathlineto{\pgfqpoint{4.490121in}{1.701900in}}%
\pgfpathlineto{\pgfqpoint{4.492912in}{1.687242in}}%
\pgfpathlineto{\pgfqpoint{4.496777in}{1.652684in}}%
\pgfpathlineto{\pgfqpoint{4.501716in}{1.586172in}}%
\pgfpathlineto{\pgfqpoint{4.508371in}{1.462958in}}%
\pgfpathlineto{\pgfqpoint{4.518463in}{1.228332in}}%
\pgfpathlineto{\pgfqpoint{4.535425in}{0.835594in}}%
\pgfpathlineto{\pgfqpoint{4.542510in}{0.720694in}}%
\pgfpathlineto{\pgfqpoint{4.547878in}{0.665851in}}%
\pgfpathlineto{\pgfqpoint{4.551743in}{0.645887in}}%
\pgfpathlineto{\pgfqpoint{4.554104in}{0.642113in}}%
\pgfpathlineto{\pgfqpoint{4.555393in}{0.642781in}}%
\pgfpathlineto{\pgfqpoint{4.557110in}{0.646665in}}%
\pgfpathlineto{\pgfqpoint{4.559687in}{0.658855in}}%
\pgfpathlineto{\pgfqpoint{4.563337in}{0.688872in}}%
\pgfpathlineto{\pgfqpoint{4.568061in}{0.748590in}}%
\pgfpathlineto{\pgfqpoint{4.574287in}{0.858682in}}%
\pgfpathlineto{\pgfqpoint{4.583090in}{1.058289in}}%
\pgfpathlineto{\pgfqpoint{4.604991in}{1.571071in}}%
\pgfpathlineto{\pgfqpoint{4.611646in}{1.671841in}}%
\pgfpathlineto{\pgfqpoint{4.616585in}{1.717407in}}%
\pgfpathlineto{\pgfqpoint{4.620020in}{1.732859in}}%
\pgfpathlineto{\pgfqpoint{4.622167in}{1.735504in}}%
\pgfpathlineto{\pgfqpoint{4.623455in}{1.734482in}}%
\pgfpathlineto{\pgfqpoint{4.625388in}{1.729289in}}%
\pgfpathlineto{\pgfqpoint{4.628179in}{1.714123in}}%
\pgfpathlineto{\pgfqpoint{4.632044in}{1.678655in}}%
\pgfpathlineto{\pgfqpoint{4.636982in}{1.610675in}}%
\pgfpathlineto{\pgfqpoint{4.643638in}{1.485074in}}%
\pgfpathlineto{\pgfqpoint{4.653729in}{1.246420in}}%
\pgfpathlineto{\pgfqpoint{4.670691in}{0.847984in}}%
\pgfpathlineto{\pgfqpoint{4.677777in}{0.731808in}}%
\pgfpathlineto{\pgfqpoint{4.682930in}{0.678193in}}%
\pgfpathlineto{\pgfqpoint{4.686795in}{0.657335in}}%
\pgfpathlineto{\pgfqpoint{4.689156in}{0.653138in}}%
\pgfpathlineto{\pgfqpoint{4.690659in}{0.653888in}}%
\pgfpathlineto{\pgfqpoint{4.692592in}{0.658760in}}%
\pgfpathlineto{\pgfqpoint{4.695383in}{0.673482in}}%
\pgfpathlineto{\pgfqpoint{4.699033in}{0.706054in}}%
\pgfpathlineto{\pgfqpoint{4.703971in}{0.772563in}}%
\pgfpathlineto{\pgfqpoint{4.710413in}{0.892662in}}%
\pgfpathlineto{\pgfqpoint{4.719860in}{1.115739in}}%
\pgfpathlineto{\pgfqpoint{4.739184in}{1.578031in}}%
\pgfpathlineto{\pgfqpoint{4.746054in}{1.689370in}}%
\pgfpathlineto{\pgfqpoint{4.751207in}{1.741477in}}%
\pgfpathlineto{\pgfqpoint{4.754857in}{1.760104in}}%
\pgfpathlineto{\pgfqpoint{4.757219in}{1.763739in}}%
\pgfpathlineto{\pgfqpoint{4.758507in}{1.762906in}}%
\pgfpathlineto{\pgfqpoint{4.760440in}{1.757937in}}%
\pgfpathlineto{\pgfqpoint{4.763231in}{1.742967in}}%
\pgfpathlineto{\pgfqpoint{4.767096in}{1.707514in}}%
\pgfpathlineto{\pgfqpoint{4.772034in}{1.639121in}}%
\pgfpathlineto{\pgfqpoint{4.778690in}{1.512240in}}%
\pgfpathlineto{\pgfqpoint{4.788781in}{1.270347in}}%
\pgfpathlineto{\pgfqpoint{4.805743in}{0.864866in}}%
\pgfpathlineto{\pgfqpoint{4.812829in}{0.746009in}}%
\pgfpathlineto{\pgfqpoint{4.818196in}{0.689135in}}%
\pgfpathlineto{\pgfqpoint{4.822061in}{0.668308in}}%
\pgfpathlineto{\pgfqpoint{4.824423in}{0.664269in}}%
\pgfpathlineto{\pgfqpoint{4.825711in}{0.664878in}}%
\pgfpathlineto{\pgfqpoint{4.827429in}{0.668779in}}%
\pgfpathlineto{\pgfqpoint{4.830005in}{0.681197in}}%
\pgfpathlineto{\pgfqpoint{4.833656in}{0.711947in}}%
\pgfpathlineto{\pgfqpoint{4.838379in}{0.773297in}}%
\pgfpathlineto{\pgfqpoint{4.844606in}{0.886592in}}%
\pgfpathlineto{\pgfqpoint{4.853409in}{1.092281in}}%
\pgfpathlineto{\pgfqpoint{4.875524in}{1.625728in}}%
\pgfpathlineto{\pgfqpoint{4.882180in}{1.728676in}}%
\pgfpathlineto{\pgfqpoint{4.887118in}{1.774804in}}%
\pgfpathlineto{\pgfqpoint{4.890553in}{1.790094in}}%
\pgfpathlineto{\pgfqpoint{4.892700in}{1.792410in}}%
\pgfpathlineto{\pgfqpoint{4.894203in}{1.790693in}}%
\pgfpathlineto{\pgfqpoint{4.896351in}{1.783492in}}%
\pgfpathlineto{\pgfqpoint{4.899356in}{1.764168in}}%
\pgfpathlineto{\pgfqpoint{4.903436in}{1.721380in}}%
\pgfpathlineto{\pgfqpoint{4.908804in}{1.638578in}}%
\pgfpathlineto{\pgfqpoint{4.916104in}{1.486734in}}%
\pgfpathlineto{\pgfqpoint{4.928986in}{1.160970in}}%
\pgfpathlineto{\pgfqpoint{4.941225in}{0.873551in}}%
\pgfpathlineto{\pgfqpoint{4.948310in}{0.754708in}}%
\pgfpathlineto{\pgfqpoint{4.953463in}{0.700280in}}%
\pgfpathlineto{\pgfqpoint{4.957328in}{0.679470in}}%
\pgfpathlineto{\pgfqpoint{4.959690in}{0.675583in}}%
\pgfpathlineto{\pgfqpoint{4.960978in}{0.676319in}}%
\pgfpathlineto{\pgfqpoint{4.962696in}{0.680437in}}%
\pgfpathlineto{\pgfqpoint{4.965272in}{0.693280in}}%
\pgfpathlineto{\pgfqpoint{4.968922in}{0.724827in}}%
\pgfpathlineto{\pgfqpoint{4.973646in}{0.787511in}}%
\pgfpathlineto{\pgfqpoint{4.979872in}{0.902989in}}%
\pgfpathlineto{\pgfqpoint{4.988890in}{1.117793in}}%
\pgfpathlineto{\pgfqpoint{5.010361in}{1.645445in}}%
\pgfpathlineto{\pgfqpoint{5.017017in}{1.752352in}}%
\pgfpathlineto{\pgfqpoint{5.021955in}{1.801231in}}%
\pgfpathlineto{\pgfqpoint{5.025605in}{1.818838in}}%
\pgfpathlineto{\pgfqpoint{5.027752in}{1.821566in}}%
\pgfpathlineto{\pgfqpoint{5.029041in}{1.820468in}}%
\pgfpathlineto{\pgfqpoint{5.030973in}{1.814985in}}%
\pgfpathlineto{\pgfqpoint{5.033764in}{1.799034in}}%
\pgfpathlineto{\pgfqpoint{5.037629in}{1.761787in}}%
\pgfpathlineto{\pgfqpoint{5.042567in}{1.690454in}}%
\pgfpathlineto{\pgfqpoint{5.049223in}{1.558727in}}%
\pgfpathlineto{\pgfqpoint{5.059315in}{1.308547in}}%
\pgfpathlineto{\pgfqpoint{5.076062in}{0.895463in}}%
\pgfpathlineto{\pgfqpoint{5.083147in}{0.772397in}}%
\pgfpathlineto{\pgfqpoint{5.088515in}{0.713312in}}%
\pgfpathlineto{\pgfqpoint{5.092380in}{0.691506in}}%
\pgfpathlineto{\pgfqpoint{5.094742in}{0.687138in}}%
\pgfpathlineto{\pgfqpoint{5.096030in}{0.687656in}}%
\pgfpathlineto{\pgfqpoint{5.097747in}{0.691533in}}%
\pgfpathlineto{\pgfqpoint{5.100324in}{0.704125in}}%
\pgfpathlineto{\pgfqpoint{5.103759in}{0.733271in}}%
\pgfpathlineto{\pgfqpoint{5.108483in}{0.795110in}}%
\pgfpathlineto{\pgfqpoint{5.114495in}{0.905875in}}%
\pgfpathlineto{\pgfqpoint{5.123083in}{1.110081in}}%
\pgfpathlineto{\pgfqpoint{5.146486in}{1.689948in}}%
\pgfpathlineto{\pgfqpoint{5.152928in}{1.789747in}}%
\pgfpathlineto{\pgfqpoint{5.157651in}{1.833907in}}%
\pgfpathlineto{\pgfqpoint{5.161087in}{1.849085in}}%
\pgfpathlineto{\pgfqpoint{5.163234in}{1.851095in}}%
\pgfpathlineto{\pgfqpoint{5.164737in}{1.849058in}}%
\pgfpathlineto{\pgfqpoint{5.166884in}{1.841252in}}%
\pgfpathlineto{\pgfqpoint{5.169890in}{1.820800in}}%
\pgfpathlineto{\pgfqpoint{5.173969in}{1.775992in}}%
\pgfpathlineto{\pgfqpoint{5.179337in}{1.689799in}}%
\pgfpathlineto{\pgfqpoint{5.186637in}{1.532368in}}%
\pgfpathlineto{\pgfqpoint{5.199734in}{1.190185in}}%
\pgfpathlineto{\pgfqpoint{5.211758in}{0.900218in}}%
\pgfpathlineto{\pgfqpoint{5.218843in}{0.778540in}}%
\pgfpathlineto{\pgfqpoint{5.223996in}{0.723196in}}%
\pgfpathlineto{\pgfqpoint{5.227646in}{0.703049in}}%
\pgfpathlineto{\pgfqpoint{5.230008in}{0.698807in}}%
\pgfpathlineto{\pgfqpoint{5.231296in}{0.699440in}}%
\pgfpathlineto{\pgfqpoint{5.233014in}{0.703519in}}%
\pgfpathlineto{\pgfqpoint{5.235591in}{0.716517in}}%
\pgfpathlineto{\pgfqpoint{5.239241in}{0.748718in}}%
\pgfpathlineto{\pgfqpoint{5.243964in}{0.812976in}}%
\pgfpathlineto{\pgfqpoint{5.250191in}{0.931661in}}%
\pgfpathlineto{\pgfqpoint{5.258994in}{1.147173in}}%
\pgfpathlineto{\pgfqpoint{5.281109in}{1.706302in}}%
\pgfpathlineto{\pgfqpoint{5.287765in}{1.814269in}}%
\pgfpathlineto{\pgfqpoint{5.292703in}{1.862677in}}%
\pgfpathlineto{\pgfqpoint{5.296139in}{1.878747in}}%
\pgfpathlineto{\pgfqpoint{5.298286in}{1.881203in}}%
\pgfpathlineto{\pgfqpoint{5.299574in}{1.879854in}}%
\pgfpathlineto{\pgfqpoint{5.301506in}{1.873874in}}%
\pgfpathlineto{\pgfqpoint{5.304297in}{1.856959in}}%
\pgfpathlineto{\pgfqpoint{5.308162in}{1.817925in}}%
\pgfpathlineto{\pgfqpoint{5.313315in}{1.739832in}}%
\pgfpathlineto{\pgfqpoint{5.320186in}{1.596994in}}%
\pgfpathlineto{\pgfqpoint{5.330921in}{1.319093in}}%
\pgfpathlineto{\pgfqpoint{5.346380in}{0.927464in}}%
\pgfpathlineto{\pgfqpoint{5.353681in}{0.796838in}}%
\pgfpathlineto{\pgfqpoint{5.359048in}{0.736676in}}%
\pgfpathlineto{\pgfqpoint{5.362913in}{0.714791in}}%
\pgfpathlineto{\pgfqpoint{5.365275in}{0.710666in}}%
\pgfpathlineto{\pgfqpoint{5.366563in}{0.711409in}}%
\pgfpathlineto{\pgfqpoint{5.368281in}{0.715686in}}%
\pgfpathlineto{\pgfqpoint{5.370857in}{0.729084in}}%
\pgfpathlineto{\pgfqpoint{5.374507in}{0.762059in}}%
\pgfpathlineto{\pgfqpoint{5.379231in}{0.827642in}}%
\pgfpathlineto{\pgfqpoint{5.385457in}{0.948534in}}%
\pgfpathlineto{\pgfqpoint{5.394260in}{1.167727in}}%
\pgfpathlineto{\pgfqpoint{5.416161in}{1.730979in}}%
\pgfpathlineto{\pgfqpoint{5.422817in}{1.841721in}}%
\pgfpathlineto{\pgfqpoint{5.427755in}{1.891821in}}%
\pgfpathlineto{\pgfqpoint{5.431190in}{1.908828in}}%
\pgfpathlineto{\pgfqpoint{5.433338in}{1.911755in}}%
\pgfpathlineto{\pgfqpoint{5.434626in}{1.910645in}}%
\pgfpathlineto{\pgfqpoint{5.436558in}{1.904959in}}%
\pgfpathlineto{\pgfqpoint{5.439349in}{1.888327in}}%
\pgfpathlineto{\pgfqpoint{5.443214in}{1.849402in}}%
\pgfpathlineto{\pgfqpoint{5.448152in}{1.774770in}}%
\pgfpathlineto{\pgfqpoint{5.454808in}{1.636853in}}%
\pgfpathlineto{\pgfqpoint{5.464900in}{1.374765in}}%
\pgfpathlineto{\pgfqpoint{5.481862in}{0.937115in}}%
\pgfpathlineto{\pgfqpoint{5.488947in}{0.809435in}}%
\pgfpathlineto{\pgfqpoint{5.494100in}{0.750455in}}%
\pgfpathlineto{\pgfqpoint{5.497965in}{0.727456in}}%
\pgfpathlineto{\pgfqpoint{5.500327in}{0.722784in}}%
\pgfpathlineto{\pgfqpoint{5.501615in}{0.723275in}}%
\pgfpathlineto{\pgfqpoint{5.503333in}{0.727268in}}%
\pgfpathlineto{\pgfqpoint{5.505909in}{0.740356in}}%
\pgfpathlineto{\pgfqpoint{5.509344in}{0.770759in}}%
\pgfpathlineto{\pgfqpoint{5.514068in}{0.835378in}}%
\pgfpathlineto{\pgfqpoint{5.520080in}{0.951253in}}%
\pgfpathlineto{\pgfqpoint{5.528668in}{1.165062in}}%
\pgfpathlineto{\pgfqpoint{5.552286in}{1.777239in}}%
\pgfpathlineto{\pgfqpoint{5.558728in}{1.880615in}}%
\pgfpathlineto{\pgfqpoint{5.563451in}{1.925871in}}%
\pgfpathlineto{\pgfqpoint{5.566886in}{1.940993in}}%
\pgfpathlineto{\pgfqpoint{5.568819in}{1.942714in}}%
\pgfpathlineto{\pgfqpoint{5.570107in}{1.941131in}}%
\pgfpathlineto{\pgfqpoint{5.572039in}{1.934676in}}%
\pgfpathlineto{\pgfqpoint{5.574831in}{1.916819in}}%
\pgfpathlineto{\pgfqpoint{5.578695in}{1.876012in}}%
\pgfpathlineto{\pgfqpoint{5.583848in}{1.794805in}}%
\pgfpathlineto{\pgfqpoint{5.590719in}{1.646777in}}%
\pgfpathlineto{\pgfqpoint{5.601455in}{1.359610in}}%
\pgfpathlineto{\pgfqpoint{5.616699in}{0.960957in}}%
\pgfpathlineto{\pgfqpoint{5.623999in}{0.825446in}}%
\pgfpathlineto{\pgfqpoint{5.629367in}{0.762704in}}%
\pgfpathlineto{\pgfqpoint{5.633232in}{0.739601in}}%
\pgfpathlineto{\pgfqpoint{5.635593in}{0.735016in}}%
\pgfpathlineto{\pgfqpoint{5.636882in}{0.735601in}}%
\pgfpathlineto{\pgfqpoint{5.638599in}{0.739773in}}%
\pgfpathlineto{\pgfqpoint{5.641176in}{0.753238in}}%
\pgfpathlineto{\pgfqpoint{5.644611in}{0.784337in}}%
\pgfpathlineto{\pgfqpoint{5.649335in}{0.850249in}}%
\pgfpathlineto{\pgfqpoint{5.655561in}{0.973060in}}%
\pgfpathlineto{\pgfqpoint{5.664364in}{1.197546in}}%
\pgfpathlineto{\pgfqpoint{5.686909in}{1.794237in}}%
\pgfpathlineto{\pgfqpoint{5.693565in}{1.906236in}}%
\pgfpathlineto{\pgfqpoint{5.698503in}{1.955974in}}%
\pgfpathlineto{\pgfqpoint{5.701938in}{1.972084in}}%
\pgfpathlineto{\pgfqpoint{5.703871in}{1.974259in}}%
\pgfpathlineto{\pgfqpoint{5.705159in}{1.972936in}}%
\pgfpathlineto{\pgfqpoint{5.707091in}{1.966805in}}%
\pgfpathlineto{\pgfqpoint{5.709883in}{1.949268in}}%
\pgfpathlineto{\pgfqpoint{5.713747in}{1.908610in}}%
\pgfpathlineto{\pgfqpoint{5.718900in}{1.827066in}}%
\pgfpathlineto{\pgfqpoint{5.725771in}{1.677678in}}%
\pgfpathlineto{\pgfqpoint{5.736507in}{1.386653in}}%
\pgfpathlineto{\pgfqpoint{5.751966in}{0.975898in}}%
\pgfpathlineto{\pgfqpoint{5.759266in}{0.838604in}}%
\pgfpathlineto{\pgfqpoint{5.764633in}{0.775184in}}%
\pgfpathlineto{\pgfqpoint{5.768498in}{0.751955in}}%
\pgfpathlineto{\pgfqpoint{5.770860in}{0.747444in}}%
\pgfpathlineto{\pgfqpoint{5.772148in}{0.748120in}}%
\pgfpathlineto{\pgfqpoint{5.773866in}{0.752463in}}%
\pgfpathlineto{\pgfqpoint{5.776442in}{0.766296in}}%
\pgfpathlineto{\pgfqpoint{5.780092in}{0.800565in}}%
\pgfpathlineto{\pgfqpoint{5.784816in}{0.868945in}}%
\pgfpathlineto{\pgfqpoint{5.791043in}{0.995249in}}%
\pgfpathlineto{\pgfqpoint{5.799846in}{1.224614in}}%
\pgfpathlineto{\pgfqpoint{5.821961in}{1.819870in}}%
\pgfpathlineto{\pgfqpoint{5.828617in}{1.934871in}}%
\pgfpathlineto{\pgfqpoint{5.833555in}{1.986459in}}%
\pgfpathlineto{\pgfqpoint{5.836990in}{2.003607in}}%
\pgfpathlineto{\pgfqpoint{5.839137in}{2.006246in}}%
\pgfpathlineto{\pgfqpoint{5.840426in}{2.004825in}}%
\pgfpathlineto{\pgfqpoint{5.842358in}{1.998481in}}%
\pgfpathlineto{\pgfqpoint{5.845149in}{1.980505in}}%
\pgfpathlineto{\pgfqpoint{5.849014in}{1.938990in}}%
\pgfpathlineto{\pgfqpoint{5.854167in}{1.855904in}}%
\pgfpathlineto{\pgfqpoint{5.861038in}{1.703897in}}%
\pgfpathlineto{\pgfqpoint{5.871773in}{1.408107in}}%
\pgfpathlineto{\pgfqpoint{5.887232in}{0.991171in}}%
\pgfpathlineto{\pgfqpoint{5.894532in}{0.852035in}}%
\pgfpathlineto{\pgfqpoint{5.899900in}{0.787899in}}%
\pgfpathlineto{\pgfqpoint{5.903765in}{0.764520in}}%
\pgfpathlineto{\pgfqpoint{5.906127in}{0.760072in}}%
\pgfpathlineto{\pgfqpoint{5.907415in}{0.760831in}}%
\pgfpathlineto{\pgfqpoint{5.909132in}{0.765339in}}%
\pgfpathlineto{\pgfqpoint{5.911709in}{0.779532in}}%
\pgfpathlineto{\pgfqpoint{5.915359in}{0.814532in}}%
\pgfpathlineto{\pgfqpoint{5.920083in}{0.884214in}}%
\pgfpathlineto{\pgfqpoint{5.926309in}{1.012748in}}%
\pgfpathlineto{\pgfqpoint{5.935112in}{1.245928in}}%
\pgfpathlineto{\pgfqpoint{5.957227in}{1.850256in}}%
\pgfpathlineto{\pgfqpoint{5.963883in}{1.966781in}}%
\pgfpathlineto{\pgfqpoint{5.968822in}{2.018935in}}%
\pgfpathlineto{\pgfqpoint{5.972257in}{2.036170in}}%
\pgfpathlineto{\pgfqpoint{5.974404in}{2.038735in}}%
\pgfpathlineto{\pgfqpoint{5.975692in}{2.037221in}}%
\pgfpathlineto{\pgfqpoint{5.977625in}{2.030674in}}%
\pgfpathlineto{\pgfqpoint{5.980416in}{2.012266in}}%
\pgfpathlineto{\pgfqpoint{5.984281in}{1.969903in}}%
\pgfpathlineto{\pgfqpoint{5.989434in}{1.885276in}}%
\pgfpathlineto{\pgfqpoint{5.996304in}{1.730636in}}%
\pgfpathlineto{\pgfqpoint{6.007040in}{1.430026in}}%
\pgfpathlineto{\pgfqpoint{6.022499in}{1.006785in}}%
\pgfpathlineto{\pgfqpoint{6.029584in}{0.869076in}}%
\pgfpathlineto{\pgfqpoint{6.034952in}{0.802747in}}%
\pgfpathlineto{\pgfqpoint{6.038817in}{0.778079in}}%
\pgfpathlineto{\pgfqpoint{6.041393in}{0.772902in}}%
\pgfpathlineto{\pgfqpoint{6.042681in}{0.773738in}}%
\pgfpathlineto{\pgfqpoint{6.044399in}{0.778403in}}%
\pgfpathlineto{\pgfqpoint{6.046976in}{0.792946in}}%
\pgfpathlineto{\pgfqpoint{6.050626in}{0.828667in}}%
\pgfpathlineto{\pgfqpoint{6.055349in}{0.899646in}}%
\pgfpathlineto{\pgfqpoint{6.061576in}{1.030417in}}%
\pgfpathlineto{\pgfqpoint{6.070379in}{1.267447in}}%
\pgfpathlineto{\pgfqpoint{6.092279in}{1.876417in}}%
\pgfpathlineto{\pgfqpoint{6.098935in}{1.996125in}}%
\pgfpathlineto{\pgfqpoint{6.103873in}{2.050263in}}%
\pgfpathlineto{\pgfqpoint{6.107309in}{2.068626in}}%
\pgfpathlineto{\pgfqpoint{6.109456in}{2.071774in}}%
\pgfpathlineto{\pgfqpoint{6.110744in}{2.070562in}}%
\pgfpathlineto{\pgfqpoint{6.112677in}{2.064397in}}%
\pgfpathlineto{\pgfqpoint{6.115468in}{2.046388in}}%
\pgfpathlineto{\pgfqpoint{6.119333in}{2.004265in}}%
\pgfpathlineto{\pgfqpoint{6.124271in}{1.923525in}}%
\pgfpathlineto{\pgfqpoint{6.130927in}{1.774356in}}%
\pgfpathlineto{\pgfqpoint{6.141018in}{1.490943in}}%
\pgfpathlineto{\pgfqpoint{6.157980in}{1.017786in}}%
\pgfpathlineto{\pgfqpoint{6.160557in}{0.961550in}}%
\pgfpathlineto{\pgfqpoint{6.160557in}{0.961550in}}%
\pgfusepath{stroke}%
\end{pgfscope}%
\begin{pgfscope}%
\pgfsetrectcap%
\pgfsetmiterjoin%
\pgfsetlinewidth{0.803000pt}%
\definecolor{currentstroke}{rgb}{0.000000,0.000000,0.000000}%
\pgfsetstrokecolor{currentstroke}%
\pgfsetdash{}{0pt}%
\pgfpathmoveto{\pgfqpoint{3.906113in}{0.526234in}}%
\pgfpathlineto{\pgfqpoint{3.906113in}{2.145371in}}%
\pgfusepath{stroke}%
\end{pgfscope}%
\begin{pgfscope}%
\pgfsetrectcap%
\pgfsetmiterjoin%
\pgfsetlinewidth{0.803000pt}%
\definecolor{currentstroke}{rgb}{0.000000,0.000000,0.000000}%
\pgfsetstrokecolor{currentstroke}%
\pgfsetdash{}{0pt}%
\pgfpathmoveto{\pgfqpoint{6.267911in}{0.526234in}}%
\pgfpathlineto{\pgfqpoint{6.267911in}{2.145371in}}%
\pgfusepath{stroke}%
\end{pgfscope}%
\begin{pgfscope}%
\pgfsetrectcap%
\pgfsetmiterjoin%
\pgfsetlinewidth{0.803000pt}%
\definecolor{currentstroke}{rgb}{0.000000,0.000000,0.000000}%
\pgfsetstrokecolor{currentstroke}%
\pgfsetdash{}{0pt}%
\pgfpathmoveto{\pgfqpoint{3.906113in}{0.526234in}}%
\pgfpathlineto{\pgfqpoint{6.267911in}{0.526234in}}%
\pgfusepath{stroke}%
\end{pgfscope}%
\begin{pgfscope}%
\pgfsetrectcap%
\pgfsetmiterjoin%
\pgfsetlinewidth{0.803000pt}%
\definecolor{currentstroke}{rgb}{0.000000,0.000000,0.000000}%
\pgfsetstrokecolor{currentstroke}%
\pgfsetdash{}{0pt}%
\pgfpathmoveto{\pgfqpoint{3.906113in}{2.145371in}}%
\pgfpathlineto{\pgfqpoint{6.267911in}{2.145371in}}%
\pgfusepath{stroke}%
\end{pgfscope}%
\begin{pgfscope}%
\definecolor{textcolor}{rgb}{0.000000,0.000000,0.000000}%
\pgfsetstrokecolor{textcolor}%
\pgfsetfillcolor{textcolor}%
\pgftext[x=5.087012in,y=2.228704in,,base]{\color{textcolor}\rmfamily\fontsize{12.000000}{14.400000}\selectfont energy}%
\end{pgfscope}%
\end{pgfpicture}%
\makeatother%
\endgroup%
}
           \caption{Plotting values for Small $\theta$ with $\Delta t = 0.001$}
           \label{fig:CP310b}
        \end{center}
    \end{figure}
    
    \noindent
    Here we can see that, while energy isn't conserved entirely, it's doing a much better 
    job of it than it was in the previous example. We can also observe that the phase plot is 
    much more elliptical and therefore purely periodic, which we expect when analysing a pendulum 
    oscillating with small $\theta$. In order to analytically prove that the energy increases 
    with each step when looking at small angle approximations, we look at Equation \ref{eq:coupled pendulum}:

    \begin{equation*}
        \begin{split}
            &p_{i+1} = p_i - \Omega_0^2 \sin(q_i) \Delta t \\
            &q_{i+1} = q_i + p_i \Delta t \\
        \end{split}
    \end{equation*}

    \begin{center}
        At small angles we can approximate
    \end{center}
    \begin{equation*}
        \begin{split}
            &\sin\theta = \theta \\
            &(1-\cos\theta) = 1 \\
            \implies &p_{i+1} = p_i - \Omega_0^2 q_i \Delta t \\
        \end{split}
    \end{equation*}

    \begin{center}
        Now recall
    \end{center}
    \begin{equation*}
        \begin{split}
            &E = \frac{1}{2}mL^2 \omega^2 + mgL(1-\cos\theta) \\
            \implies &E = \frac{1}{2}mL^2 \omega^2 + mgL
        \end{split}
    \end{equation*}
    
    \noindent
    It's important to notice that this $E$ depends on $\omega$, which, when calculated 
    in the program, is done so using a Taylor expansion which is truncated after the 
    second term. While this error is not great, when squared and compounded 1000 times 
    as it is in the first two examples, it's obvious that this is the source of the error. 
    The only problem is that this error seems to be coming from a quadratic term, while the 
    increase in energy seems to be exponential, which we can't seem to make sense of. 
    \newline
    \newline
    One solution to the energy conservation problem is an adjustment to the method we use 
    to approximate the values of, in this case, $\omega$ and $\theta$. The equations are 
    valid for the entire interval of interest and so we can evaluate a point later in time 
    in order to better approximate the derivative of the function of interest. This is an 
    improvement to the Euler's method that we have been using and it doesn't quite solve the 
    issue of energy not being conserved, but it does give a bound to the energy, making it 
    periodic. This method is known as Symplectic Integration or the Symplectic Euler Method.
    Below, Figure \ref{fig:symplectic}, is the set of plots for this method.

    \begin{figure}[H]
        \begin{center}
           \scalebox{.7}{%% Creator: Matplotlib, PGF backend
%%
%% To include the figure in your LaTeX document, write
%%   \input{<filename>.pgf}
%%
%% Make sure the required packages are loaded in your preamble
%%   \usepackage{pgf}
%%
%% Figures using additional raster images can only be included by \input if
%% they are in the same directory as the main LaTeX file. For loading figures
%% from other directories you can use the `import` package
%%   \usepackage{import}
%% and then include the figures with
%%   \import{<path to file>}{<filename>.pgf}
%%
%% Matplotlib used the following preamble
%%
\begingroup%
\makeatletter%
\begin{pgfpicture}%
\pgfpathrectangle{\pgfpointorigin}{\pgfqpoint{6.400000in}{4.800000in}}%
\pgfusepath{use as bounding box, clip}%
\begin{pgfscope}%
\pgfsetbuttcap%
\pgfsetmiterjoin%
\definecolor{currentfill}{rgb}{1.000000,1.000000,1.000000}%
\pgfsetfillcolor{currentfill}%
\pgfsetlinewidth{0.000000pt}%
\definecolor{currentstroke}{rgb}{1.000000,1.000000,1.000000}%
\pgfsetstrokecolor{currentstroke}%
\pgfsetdash{}{0pt}%
\pgfpathmoveto{\pgfqpoint{0.000000in}{0.000000in}}%
\pgfpathlineto{\pgfqpoint{6.400000in}{0.000000in}}%
\pgfpathlineto{\pgfqpoint{6.400000in}{4.800000in}}%
\pgfpathlineto{\pgfqpoint{0.000000in}{4.800000in}}%
\pgfpathclose%
\pgfusepath{fill}%
\end{pgfscope}%
\begin{pgfscope}%
\pgfsetbuttcap%
\pgfsetmiterjoin%
\definecolor{currentfill}{rgb}{1.000000,1.000000,1.000000}%
\pgfsetfillcolor{currentfill}%
\pgfsetlinewidth{0.000000pt}%
\definecolor{currentstroke}{rgb}{0.000000,0.000000,0.000000}%
\pgfsetstrokecolor{currentstroke}%
\pgfsetstrokeopacity{0.000000}%
\pgfsetdash{}{0pt}%
\pgfpathmoveto{\pgfqpoint{0.835065in}{2.870679in}}%
\pgfpathlineto{\pgfqpoint{3.196863in}{2.870679in}}%
\pgfpathlineto{\pgfqpoint{3.196863in}{4.489815in}}%
\pgfpathlineto{\pgfqpoint{0.835065in}{4.489815in}}%
\pgfpathclose%
\pgfusepath{fill}%
\end{pgfscope}%
\begin{pgfscope}%
\pgfsetbuttcap%
\pgfsetroundjoin%
\definecolor{currentfill}{rgb}{0.000000,0.000000,0.000000}%
\pgfsetfillcolor{currentfill}%
\pgfsetlinewidth{0.803000pt}%
\definecolor{currentstroke}{rgb}{0.000000,0.000000,0.000000}%
\pgfsetstrokecolor{currentstroke}%
\pgfsetdash{}{0pt}%
\pgfsys@defobject{currentmarker}{\pgfqpoint{0.000000in}{-0.048611in}}{\pgfqpoint{0.000000in}{0.000000in}}{%
\pgfpathmoveto{\pgfqpoint{0.000000in}{0.000000in}}%
\pgfpathlineto{\pgfqpoint{0.000000in}{-0.048611in}}%
\pgfusepath{stroke,fill}%
}%
\begin{pgfscope}%
\pgfsys@transformshift{0.942419in}{2.870679in}%
\pgfsys@useobject{currentmarker}{}%
\end{pgfscope}%
\end{pgfscope}%
\begin{pgfscope}%
\definecolor{textcolor}{rgb}{0.000000,0.000000,0.000000}%
\pgfsetstrokecolor{textcolor}%
\pgfsetfillcolor{textcolor}%
\pgftext[x=0.942419in,y=2.773457in,,top]{\color{textcolor}\rmfamily\fontsize{10.000000}{12.000000}\selectfont \(\displaystyle 0.0\)}%
\end{pgfscope}%
\begin{pgfscope}%
\pgfsetbuttcap%
\pgfsetroundjoin%
\definecolor{currentfill}{rgb}{0.000000,0.000000,0.000000}%
\pgfsetfillcolor{currentfill}%
\pgfsetlinewidth{0.803000pt}%
\definecolor{currentstroke}{rgb}{0.000000,0.000000,0.000000}%
\pgfsetstrokecolor{currentstroke}%
\pgfsetdash{}{0pt}%
\pgfsys@defobject{currentmarker}{\pgfqpoint{0.000000in}{-0.048611in}}{\pgfqpoint{0.000000in}{0.000000in}}{%
\pgfpathmoveto{\pgfqpoint{0.000000in}{0.000000in}}%
\pgfpathlineto{\pgfqpoint{0.000000in}{-0.048611in}}%
\pgfusepath{stroke,fill}%
}%
\begin{pgfscope}%
\pgfsys@transformshift{1.479191in}{2.870679in}%
\pgfsys@useobject{currentmarker}{}%
\end{pgfscope}%
\end{pgfscope}%
\begin{pgfscope}%
\definecolor{textcolor}{rgb}{0.000000,0.000000,0.000000}%
\pgfsetstrokecolor{textcolor}%
\pgfsetfillcolor{textcolor}%
\pgftext[x=1.479191in,y=2.773457in,,top]{\color{textcolor}\rmfamily\fontsize{10.000000}{12.000000}\selectfont \(\displaystyle 2.5\)}%
\end{pgfscope}%
\begin{pgfscope}%
\pgfsetbuttcap%
\pgfsetroundjoin%
\definecolor{currentfill}{rgb}{0.000000,0.000000,0.000000}%
\pgfsetfillcolor{currentfill}%
\pgfsetlinewidth{0.803000pt}%
\definecolor{currentstroke}{rgb}{0.000000,0.000000,0.000000}%
\pgfsetstrokecolor{currentstroke}%
\pgfsetdash{}{0pt}%
\pgfsys@defobject{currentmarker}{\pgfqpoint{0.000000in}{-0.048611in}}{\pgfqpoint{0.000000in}{0.000000in}}{%
\pgfpathmoveto{\pgfqpoint{0.000000in}{0.000000in}}%
\pgfpathlineto{\pgfqpoint{0.000000in}{-0.048611in}}%
\pgfusepath{stroke,fill}%
}%
\begin{pgfscope}%
\pgfsys@transformshift{2.015964in}{2.870679in}%
\pgfsys@useobject{currentmarker}{}%
\end{pgfscope}%
\end{pgfscope}%
\begin{pgfscope}%
\definecolor{textcolor}{rgb}{0.000000,0.000000,0.000000}%
\pgfsetstrokecolor{textcolor}%
\pgfsetfillcolor{textcolor}%
\pgftext[x=2.015964in,y=2.773457in,,top]{\color{textcolor}\rmfamily\fontsize{10.000000}{12.000000}\selectfont \(\displaystyle 5.0\)}%
\end{pgfscope}%
\begin{pgfscope}%
\pgfsetbuttcap%
\pgfsetroundjoin%
\definecolor{currentfill}{rgb}{0.000000,0.000000,0.000000}%
\pgfsetfillcolor{currentfill}%
\pgfsetlinewidth{0.803000pt}%
\definecolor{currentstroke}{rgb}{0.000000,0.000000,0.000000}%
\pgfsetstrokecolor{currentstroke}%
\pgfsetdash{}{0pt}%
\pgfsys@defobject{currentmarker}{\pgfqpoint{0.000000in}{-0.048611in}}{\pgfqpoint{0.000000in}{0.000000in}}{%
\pgfpathmoveto{\pgfqpoint{0.000000in}{0.000000in}}%
\pgfpathlineto{\pgfqpoint{0.000000in}{-0.048611in}}%
\pgfusepath{stroke,fill}%
}%
\begin{pgfscope}%
\pgfsys@transformshift{2.552736in}{2.870679in}%
\pgfsys@useobject{currentmarker}{}%
\end{pgfscope}%
\end{pgfscope}%
\begin{pgfscope}%
\definecolor{textcolor}{rgb}{0.000000,0.000000,0.000000}%
\pgfsetstrokecolor{textcolor}%
\pgfsetfillcolor{textcolor}%
\pgftext[x=2.552736in,y=2.773457in,,top]{\color{textcolor}\rmfamily\fontsize{10.000000}{12.000000}\selectfont \(\displaystyle 7.5\)}%
\end{pgfscope}%
\begin{pgfscope}%
\pgfsetbuttcap%
\pgfsetroundjoin%
\definecolor{currentfill}{rgb}{0.000000,0.000000,0.000000}%
\pgfsetfillcolor{currentfill}%
\pgfsetlinewidth{0.803000pt}%
\definecolor{currentstroke}{rgb}{0.000000,0.000000,0.000000}%
\pgfsetstrokecolor{currentstroke}%
\pgfsetdash{}{0pt}%
\pgfsys@defobject{currentmarker}{\pgfqpoint{0.000000in}{-0.048611in}}{\pgfqpoint{0.000000in}{0.000000in}}{%
\pgfpathmoveto{\pgfqpoint{0.000000in}{0.000000in}}%
\pgfpathlineto{\pgfqpoint{0.000000in}{-0.048611in}}%
\pgfusepath{stroke,fill}%
}%
\begin{pgfscope}%
\pgfsys@transformshift{3.089508in}{2.870679in}%
\pgfsys@useobject{currentmarker}{}%
\end{pgfscope}%
\end{pgfscope}%
\begin{pgfscope}%
\definecolor{textcolor}{rgb}{0.000000,0.000000,0.000000}%
\pgfsetstrokecolor{textcolor}%
\pgfsetfillcolor{textcolor}%
\pgftext[x=3.089508in,y=2.773457in,,top]{\color{textcolor}\rmfamily\fontsize{10.000000}{12.000000}\selectfont \(\displaystyle 10.0\)}%
\end{pgfscope}%
\begin{pgfscope}%
\definecolor{textcolor}{rgb}{0.000000,0.000000,0.000000}%
\pgfsetstrokecolor{textcolor}%
\pgfsetfillcolor{textcolor}%
\pgftext[x=2.015964in,y=2.594444in,,top]{\color{textcolor}\rmfamily\fontsize{10.000000}{12.000000}\selectfont time (s)}%
\end{pgfscope}%
\begin{pgfscope}%
\pgfsetbuttcap%
\pgfsetroundjoin%
\definecolor{currentfill}{rgb}{0.000000,0.000000,0.000000}%
\pgfsetfillcolor{currentfill}%
\pgfsetlinewidth{0.803000pt}%
\definecolor{currentstroke}{rgb}{0.000000,0.000000,0.000000}%
\pgfsetstrokecolor{currentstroke}%
\pgfsetdash{}{0pt}%
\pgfsys@defobject{currentmarker}{\pgfqpoint{-0.048611in}{0.000000in}}{\pgfqpoint{0.000000in}{0.000000in}}{%
\pgfpathmoveto{\pgfqpoint{0.000000in}{0.000000in}}%
\pgfpathlineto{\pgfqpoint{-0.048611in}{0.000000in}}%
\pgfusepath{stroke,fill}%
}%
\begin{pgfscope}%
\pgfsys@transformshift{0.835065in}{3.258698in}%
\pgfsys@useobject{currentmarker}{}%
\end{pgfscope}%
\end{pgfscope}%
\begin{pgfscope}%
\definecolor{textcolor}{rgb}{0.000000,0.000000,0.000000}%
\pgfsetstrokecolor{textcolor}%
\pgfsetfillcolor{textcolor}%
\pgftext[x=0.452348in,y=3.210473in,left,base]{\color{textcolor}\rmfamily\fontsize{10.000000}{12.000000}\selectfont \(\displaystyle -0.1\)}%
\end{pgfscope}%
\begin{pgfscope}%
\pgfsetbuttcap%
\pgfsetroundjoin%
\definecolor{currentfill}{rgb}{0.000000,0.000000,0.000000}%
\pgfsetfillcolor{currentfill}%
\pgfsetlinewidth{0.803000pt}%
\definecolor{currentstroke}{rgb}{0.000000,0.000000,0.000000}%
\pgfsetstrokecolor{currentstroke}%
\pgfsetdash{}{0pt}%
\pgfsys@defobject{currentmarker}{\pgfqpoint{-0.048611in}{0.000000in}}{\pgfqpoint{0.000000in}{0.000000in}}{%
\pgfpathmoveto{\pgfqpoint{0.000000in}{0.000000in}}%
\pgfpathlineto{\pgfqpoint{-0.048611in}{0.000000in}}%
\pgfusepath{stroke,fill}%
}%
\begin{pgfscope}%
\pgfsys@transformshift{0.835065in}{3.680248in}%
\pgfsys@useobject{currentmarker}{}%
\end{pgfscope}%
\end{pgfscope}%
\begin{pgfscope}%
\definecolor{textcolor}{rgb}{0.000000,0.000000,0.000000}%
\pgfsetstrokecolor{textcolor}%
\pgfsetfillcolor{textcolor}%
\pgftext[x=0.560373in,y=3.632023in,left,base]{\color{textcolor}\rmfamily\fontsize{10.000000}{12.000000}\selectfont \(\displaystyle 0.0\)}%
\end{pgfscope}%
\begin{pgfscope}%
\pgfsetbuttcap%
\pgfsetroundjoin%
\definecolor{currentfill}{rgb}{0.000000,0.000000,0.000000}%
\pgfsetfillcolor{currentfill}%
\pgfsetlinewidth{0.803000pt}%
\definecolor{currentstroke}{rgb}{0.000000,0.000000,0.000000}%
\pgfsetstrokecolor{currentstroke}%
\pgfsetdash{}{0pt}%
\pgfsys@defobject{currentmarker}{\pgfqpoint{-0.048611in}{0.000000in}}{\pgfqpoint{0.000000in}{0.000000in}}{%
\pgfpathmoveto{\pgfqpoint{0.000000in}{0.000000in}}%
\pgfpathlineto{\pgfqpoint{-0.048611in}{0.000000in}}%
\pgfusepath{stroke,fill}%
}%
\begin{pgfscope}%
\pgfsys@transformshift{0.835065in}{4.101798in}%
\pgfsys@useobject{currentmarker}{}%
\end{pgfscope}%
\end{pgfscope}%
\begin{pgfscope}%
\definecolor{textcolor}{rgb}{0.000000,0.000000,0.000000}%
\pgfsetstrokecolor{textcolor}%
\pgfsetfillcolor{textcolor}%
\pgftext[x=0.560373in,y=4.053573in,left,base]{\color{textcolor}\rmfamily\fontsize{10.000000}{12.000000}\selectfont \(\displaystyle 0.1\)}%
\end{pgfscope}%
\begin{pgfscope}%
\definecolor{textcolor}{rgb}{0.000000,0.000000,0.000000}%
\pgfsetstrokecolor{textcolor}%
\pgfsetfillcolor{textcolor}%
\pgftext[x=0.396792in,y=3.680247in,,bottom,rotate=90.000000]{\color{textcolor}\rmfamily\fontsize{10.000000}{12.000000}\selectfont angle (rad)}%
\end{pgfscope}%
\begin{pgfscope}%
\pgfpathrectangle{\pgfqpoint{0.835065in}{2.870679in}}{\pgfqpoint{2.361798in}{1.619136in}}%
\pgfusepath{clip}%
\pgfsetrectcap%
\pgfsetroundjoin%
\pgfsetlinewidth{1.505625pt}%
\definecolor{currentstroke}{rgb}{0.000000,0.000000,1.000000}%
\pgfsetstrokecolor{currentstroke}%
\pgfsetdash{}{0pt}%
\pgfpathmoveto{\pgfqpoint{0.942419in}{4.415991in}}%
\pgfpathlineto{\pgfqpoint{0.944566in}{4.414161in}}%
\pgfpathlineto{\pgfqpoint{0.948860in}{4.405034in}}%
\pgfpathlineto{\pgfqpoint{0.953154in}{4.388698in}}%
\pgfpathlineto{\pgfqpoint{0.959596in}{4.351052in}}%
\pgfpathlineto{\pgfqpoint{0.966037in}{4.298400in}}%
\pgfpathlineto{\pgfqpoint{0.974625in}{4.206924in}}%
\pgfpathlineto{\pgfqpoint{0.985361in}{4.063629in}}%
\pgfpathlineto{\pgfqpoint{1.000390in}{3.825306in}}%
\pgfpathlineto{\pgfqpoint{1.034744in}{3.264333in}}%
\pgfpathlineto{\pgfqpoint{1.045479in}{3.127253in}}%
\pgfpathlineto{\pgfqpoint{1.054068in}{3.041940in}}%
\pgfpathlineto{\pgfqpoint{1.060509in}{2.994448in}}%
\pgfpathlineto{\pgfqpoint{1.066950in}{2.962293in}}%
\pgfpathlineto{\pgfqpoint{1.071244in}{2.949743in}}%
\pgfpathlineto{\pgfqpoint{1.075539in}{2.944457in}}%
\pgfpathlineto{\pgfqpoint{1.077686in}{2.944558in}}%
\pgfpathlineto{\pgfqpoint{1.079833in}{2.946488in}}%
\pgfpathlineto{\pgfqpoint{1.084127in}{2.955814in}}%
\pgfpathlineto{\pgfqpoint{1.088421in}{2.972345in}}%
\pgfpathlineto{\pgfqpoint{1.094862in}{3.010274in}}%
\pgfpathlineto{\pgfqpoint{1.101304in}{3.063190in}}%
\pgfpathlineto{\pgfqpoint{1.109892in}{3.154979in}}%
\pgfpathlineto{\pgfqpoint{1.120627in}{3.298586in}}%
\pgfpathlineto{\pgfqpoint{1.135657in}{3.537164in}}%
\pgfpathlineto{\pgfqpoint{1.167863in}{4.067059in}}%
\pgfpathlineto{\pgfqpoint{1.178599in}{4.209727in}}%
\pgfpathlineto{\pgfqpoint{1.187187in}{4.300574in}}%
\pgfpathlineto{\pgfqpoint{1.195776in}{4.366776in}}%
\pgfpathlineto{\pgfqpoint{1.202217in}{4.398643in}}%
\pgfpathlineto{\pgfqpoint{1.206511in}{4.410995in}}%
\pgfpathlineto{\pgfqpoint{1.210805in}{4.416081in}}%
\pgfpathlineto{\pgfqpoint{1.212952in}{4.415880in}}%
\pgfpathlineto{\pgfqpoint{1.215099in}{4.413850in}}%
\pgfpathlineto{\pgfqpoint{1.219394in}{4.404324in}}%
\pgfpathlineto{\pgfqpoint{1.223688in}{4.387598in}}%
\pgfpathlineto{\pgfqpoint{1.230129in}{4.349387in}}%
\pgfpathlineto{\pgfqpoint{1.236570in}{4.296208in}}%
\pgfpathlineto{\pgfqpoint{1.245159in}{4.204106in}}%
\pgfpathlineto{\pgfqpoint{1.255894in}{4.060187in}}%
\pgfpathlineto{\pgfqpoint{1.270924in}{3.821357in}}%
\pgfpathlineto{\pgfqpoint{1.303130in}{3.291726in}}%
\pgfpathlineto{\pgfqpoint{1.313865in}{3.149374in}}%
\pgfpathlineto{\pgfqpoint{1.322454in}{3.058842in}}%
\pgfpathlineto{\pgfqpoint{1.331042in}{2.992998in}}%
\pgfpathlineto{\pgfqpoint{1.337483in}{2.961419in}}%
\pgfpathlineto{\pgfqpoint{1.341778in}{2.949265in}}%
\pgfpathlineto{\pgfqpoint{1.346072in}{2.944379in}}%
\pgfpathlineto{\pgfqpoint{1.348219in}{2.944680in}}%
\pgfpathlineto{\pgfqpoint{1.350366in}{2.946810in}}%
\pgfpathlineto{\pgfqpoint{1.354660in}{2.956535in}}%
\pgfpathlineto{\pgfqpoint{1.358954in}{2.973456in}}%
\pgfpathlineto{\pgfqpoint{1.365396in}{3.011949in}}%
\pgfpathlineto{\pgfqpoint{1.371837in}{3.065391in}}%
\pgfpathlineto{\pgfqpoint{1.380425in}{3.157806in}}%
\pgfpathlineto{\pgfqpoint{1.391161in}{3.302034in}}%
\pgfpathlineto{\pgfqpoint{1.406190in}{3.541116in}}%
\pgfpathlineto{\pgfqpoint{1.438397in}{4.070478in}}%
\pgfpathlineto{\pgfqpoint{1.449132in}{4.212514in}}%
\pgfpathlineto{\pgfqpoint{1.457720in}{4.302730in}}%
\pgfpathlineto{\pgfqpoint{1.466309in}{4.368215in}}%
\pgfpathlineto{\pgfqpoint{1.472750in}{4.399506in}}%
\pgfpathlineto{\pgfqpoint{1.477044in}{4.411462in}}%
\pgfpathlineto{\pgfqpoint{1.481338in}{4.416148in}}%
\pgfpathlineto{\pgfqpoint{1.483486in}{4.415747in}}%
\pgfpathlineto{\pgfqpoint{1.485633in}{4.413516in}}%
\pgfpathlineto{\pgfqpoint{1.489927in}{4.403593in}}%
\pgfpathlineto{\pgfqpoint{1.494221in}{4.386477in}}%
\pgfpathlineto{\pgfqpoint{1.500662in}{4.347702in}}%
\pgfpathlineto{\pgfqpoint{1.507103in}{4.293997in}}%
\pgfpathlineto{\pgfqpoint{1.515692in}{4.201271in}}%
\pgfpathlineto{\pgfqpoint{1.526427in}{4.056734in}}%
\pgfpathlineto{\pgfqpoint{1.541457in}{3.817403in}}%
\pgfpathlineto{\pgfqpoint{1.573663in}{3.288314in}}%
\pgfpathlineto{\pgfqpoint{1.584399in}{3.146595in}}%
\pgfpathlineto{\pgfqpoint{1.592987in}{3.056695in}}%
\pgfpathlineto{\pgfqpoint{1.599428in}{3.005370in}}%
\pgfpathlineto{\pgfqpoint{1.605870in}{2.969140in}}%
\pgfpathlineto{\pgfqpoint{1.610164in}{2.953784in}}%
\pgfpathlineto{\pgfqpoint{1.614458in}{2.945652in}}%
\pgfpathlineto{\pgfqpoint{1.616605in}{2.944323in}}%
\pgfpathlineto{\pgfqpoint{1.618752in}{2.944824in}}%
\pgfpathlineto{\pgfqpoint{1.620899in}{2.947155in}}%
\pgfpathlineto{\pgfqpoint{1.625193in}{2.957276in}}%
\pgfpathlineto{\pgfqpoint{1.629488in}{2.974588in}}%
\pgfpathlineto{\pgfqpoint{1.635929in}{3.013644in}}%
\pgfpathlineto{\pgfqpoint{1.642370in}{3.067612in}}%
\pgfpathlineto{\pgfqpoint{1.650958in}{3.160648in}}%
\pgfpathlineto{\pgfqpoint{1.661694in}{3.305492in}}%
\pgfpathlineto{\pgfqpoint{1.676724in}{3.545072in}}%
\pgfpathlineto{\pgfqpoint{1.708930in}{4.073884in}}%
\pgfpathlineto{\pgfqpoint{1.719665in}{4.215285in}}%
\pgfpathlineto{\pgfqpoint{1.728254in}{4.304867in}}%
\pgfpathlineto{\pgfqpoint{1.734695in}{4.355925in}}%
\pgfpathlineto{\pgfqpoint{1.741136in}{4.391871in}}%
\pgfpathlineto{\pgfqpoint{1.745430in}{4.407032in}}%
\pgfpathlineto{\pgfqpoint{1.749725in}{4.414964in}}%
\pgfpathlineto{\pgfqpoint{1.751872in}{4.416193in}}%
\pgfpathlineto{\pgfqpoint{1.754019in}{4.415592in}}%
\pgfpathlineto{\pgfqpoint{1.756166in}{4.413161in}}%
\pgfpathlineto{\pgfqpoint{1.760460in}{4.402841in}}%
\pgfpathlineto{\pgfqpoint{1.764754in}{4.385335in}}%
\pgfpathlineto{\pgfqpoint{1.771195in}{4.345997in}}%
\pgfpathlineto{\pgfqpoint{1.777637in}{4.291768in}}%
\pgfpathlineto{\pgfqpoint{1.786225in}{4.198422in}}%
\pgfpathlineto{\pgfqpoint{1.796961in}{4.053270in}}%
\pgfpathlineto{\pgfqpoint{1.811990in}{3.813445in}}%
\pgfpathlineto{\pgfqpoint{1.844196in}{3.284913in}}%
\pgfpathlineto{\pgfqpoint{1.854932in}{3.143832in}}%
\pgfpathlineto{\pgfqpoint{1.863520in}{3.054568in}}%
\pgfpathlineto{\pgfqpoint{1.869962in}{3.003776in}}%
\pgfpathlineto{\pgfqpoint{1.876403in}{2.968114in}}%
\pgfpathlineto{\pgfqpoint{1.880697in}{2.953150in}}%
\pgfpathlineto{\pgfqpoint{1.884991in}{2.945417in}}%
\pgfpathlineto{\pgfqpoint{1.887138in}{2.944288in}}%
\pgfpathlineto{\pgfqpoint{1.889285in}{2.944990in}}%
\pgfpathlineto{\pgfqpoint{1.891432in}{2.947521in}}%
\pgfpathlineto{\pgfqpoint{1.895727in}{2.958040in}}%
\pgfpathlineto{\pgfqpoint{1.900021in}{2.975741in}}%
\pgfpathlineto{\pgfqpoint{1.906462in}{3.015359in}}%
\pgfpathlineto{\pgfqpoint{1.912903in}{3.069850in}}%
\pgfpathlineto{\pgfqpoint{1.921492in}{3.163505in}}%
\pgfpathlineto{\pgfqpoint{1.932227in}{3.308962in}}%
\pgfpathlineto{\pgfqpoint{1.947257in}{3.549032in}}%
\pgfpathlineto{\pgfqpoint{1.979463in}{4.077279in}}%
\pgfpathlineto{\pgfqpoint{1.990199in}{4.218040in}}%
\pgfpathlineto{\pgfqpoint{1.998787in}{4.306985in}}%
\pgfpathlineto{\pgfqpoint{2.005228in}{4.357510in}}%
\pgfpathlineto{\pgfqpoint{2.011669in}{4.392887in}}%
\pgfpathlineto{\pgfqpoint{2.015964in}{4.407654in}}%
\pgfpathlineto{\pgfqpoint{2.020258in}{4.415188in}}%
\pgfpathlineto{\pgfqpoint{2.022405in}{4.416217in}}%
\pgfpathlineto{\pgfqpoint{2.024552in}{4.415414in}}%
\pgfpathlineto{\pgfqpoint{2.026699in}{4.412784in}}%
\pgfpathlineto{\pgfqpoint{2.030993in}{4.402067in}}%
\pgfpathlineto{\pgfqpoint{2.035287in}{4.384171in}}%
\pgfpathlineto{\pgfqpoint{2.041729in}{4.344273in}}%
\pgfpathlineto{\pgfqpoint{2.048170in}{4.289520in}}%
\pgfpathlineto{\pgfqpoint{2.056758in}{4.195556in}}%
\pgfpathlineto{\pgfqpoint{2.067494in}{4.049795in}}%
\pgfpathlineto{\pgfqpoint{2.082523in}{3.809483in}}%
\pgfpathlineto{\pgfqpoint{2.114730in}{3.281524in}}%
\pgfpathlineto{\pgfqpoint{2.125465in}{3.141085in}}%
\pgfpathlineto{\pgfqpoint{2.134054in}{3.052459in}}%
\pgfpathlineto{\pgfqpoint{2.140495in}{3.002202in}}%
\pgfpathlineto{\pgfqpoint{2.146936in}{2.967110in}}%
\pgfpathlineto{\pgfqpoint{2.151230in}{2.952539in}}%
\pgfpathlineto{\pgfqpoint{2.155524in}{2.945204in}}%
\pgfpathlineto{\pgfqpoint{2.157672in}{2.944276in}}%
\pgfpathlineto{\pgfqpoint{2.159819in}{2.945178in}}%
\pgfpathlineto{\pgfqpoint{2.161966in}{2.947909in}}%
\pgfpathlineto{\pgfqpoint{2.166260in}{2.958825in}}%
\pgfpathlineto{\pgfqpoint{2.170554in}{2.976915in}}%
\pgfpathlineto{\pgfqpoint{2.176995in}{3.017093in}}%
\pgfpathlineto{\pgfqpoint{2.183437in}{3.072107in}}%
\pgfpathlineto{\pgfqpoint{2.192025in}{3.166378in}}%
\pgfpathlineto{\pgfqpoint{2.202760in}{3.312443in}}%
\pgfpathlineto{\pgfqpoint{2.217790in}{3.552995in}}%
\pgfpathlineto{\pgfqpoint{2.249996in}{4.080663in}}%
\pgfpathlineto{\pgfqpoint{2.260732in}{4.220778in}}%
\pgfpathlineto{\pgfqpoint{2.269320in}{4.309085in}}%
\pgfpathlineto{\pgfqpoint{2.275761in}{4.359074in}}%
\pgfpathlineto{\pgfqpoint{2.282203in}{4.393881in}}%
\pgfpathlineto{\pgfqpoint{2.286497in}{4.408255in}}%
\pgfpathlineto{\pgfqpoint{2.290791in}{4.415390in}}%
\pgfpathlineto{\pgfqpoint{2.292938in}{4.416218in}}%
\pgfpathlineto{\pgfqpoint{2.295085in}{4.415215in}}%
\pgfpathlineto{\pgfqpoint{2.299379in}{4.407733in}}%
\pgfpathlineto{\pgfqpoint{2.303674in}{4.393016in}}%
\pgfpathlineto{\pgfqpoint{2.310115in}{4.357712in}}%
\pgfpathlineto{\pgfqpoint{2.316556in}{4.307256in}}%
\pgfpathlineto{\pgfqpoint{2.325144in}{4.218392in}}%
\pgfpathlineto{\pgfqpoint{2.335880in}{4.077715in}}%
\pgfpathlineto{\pgfqpoint{2.348762in}{3.877207in}}%
\pgfpathlineto{\pgfqpoint{2.387410in}{3.247888in}}%
\pgfpathlineto{\pgfqpoint{2.398146in}{3.114186in}}%
\pgfpathlineto{\pgfqpoint{2.406734in}{3.032168in}}%
\pgfpathlineto{\pgfqpoint{2.413175in}{2.987407in}}%
\pgfpathlineto{\pgfqpoint{2.419616in}{2.958139in}}%
\pgfpathlineto{\pgfqpoint{2.423911in}{2.947570in}}%
\pgfpathlineto{\pgfqpoint{2.426058in}{2.945013in}}%
\pgfpathlineto{\pgfqpoint{2.428205in}{2.944286in}}%
\pgfpathlineto{\pgfqpoint{2.430352in}{2.945388in}}%
\pgfpathlineto{\pgfqpoint{2.434646in}{2.953071in}}%
\pgfpathlineto{\pgfqpoint{2.438940in}{2.967984in}}%
\pgfpathlineto{\pgfqpoint{2.445381in}{3.003573in}}%
\pgfpathlineto{\pgfqpoint{2.451823in}{3.054296in}}%
\pgfpathlineto{\pgfqpoint{2.460411in}{3.143478in}}%
\pgfpathlineto{\pgfqpoint{2.471147in}{3.284477in}}%
\pgfpathlineto{\pgfqpoint{2.486176in}{3.520897in}}%
\pgfpathlineto{\pgfqpoint{2.520530in}{4.084034in}}%
\pgfpathlineto{\pgfqpoint{2.531265in}{4.223501in}}%
\pgfpathlineto{\pgfqpoint{2.539853in}{4.311165in}}%
\pgfpathlineto{\pgfqpoint{2.546295in}{4.360618in}}%
\pgfpathlineto{\pgfqpoint{2.552736in}{4.394853in}}%
\pgfpathlineto{\pgfqpoint{2.557030in}{4.408834in}}%
\pgfpathlineto{\pgfqpoint{2.561324in}{4.415570in}}%
\pgfpathlineto{\pgfqpoint{2.563471in}{4.416197in}}%
\pgfpathlineto{\pgfqpoint{2.565618in}{4.414994in}}%
\pgfpathlineto{\pgfqpoint{2.569913in}{4.407113in}}%
\pgfpathlineto{\pgfqpoint{2.574207in}{4.392003in}}%
\pgfpathlineto{\pgfqpoint{2.580648in}{4.356130in}}%
\pgfpathlineto{\pgfqpoint{2.587089in}{4.305140in}}%
\pgfpathlineto{\pgfqpoint{2.595678in}{4.215640in}}%
\pgfpathlineto{\pgfqpoint{2.606413in}{4.074321in}}%
\pgfpathlineto{\pgfqpoint{2.621443in}{3.837633in}}%
\pgfpathlineto{\pgfqpoint{2.655796in}{3.274781in}}%
\pgfpathlineto{\pgfqpoint{2.666532in}{3.135640in}}%
\pgfpathlineto{\pgfqpoint{2.675120in}{3.048298in}}%
\pgfpathlineto{\pgfqpoint{2.681561in}{2.999114in}}%
\pgfpathlineto{\pgfqpoint{2.688003in}{2.965164in}}%
\pgfpathlineto{\pgfqpoint{2.692297in}{2.951380in}}%
\pgfpathlineto{\pgfqpoint{2.696591in}{2.944844in}}%
\pgfpathlineto{\pgfqpoint{2.698738in}{2.944317in}}%
\pgfpathlineto{\pgfqpoint{2.700885in}{2.945620in}}%
\pgfpathlineto{\pgfqpoint{2.705179in}{2.953701in}}%
\pgfpathlineto{\pgfqpoint{2.709473in}{2.969007in}}%
\pgfpathlineto{\pgfqpoint{2.715915in}{3.005164in}}%
\pgfpathlineto{\pgfqpoint{2.722356in}{3.056421in}}%
\pgfpathlineto{\pgfqpoint{2.730944in}{3.146239in}}%
\pgfpathlineto{\pgfqpoint{2.741680in}{3.287876in}}%
\pgfpathlineto{\pgfqpoint{2.756709in}{3.524830in}}%
\pgfpathlineto{\pgfqpoint{2.791063in}{4.087393in}}%
\pgfpathlineto{\pgfqpoint{2.801798in}{4.226208in}}%
\pgfpathlineto{\pgfqpoint{2.810387in}{4.313227in}}%
\pgfpathlineto{\pgfqpoint{2.816828in}{4.362141in}}%
\pgfpathlineto{\pgfqpoint{2.823269in}{4.395805in}}%
\pgfpathlineto{\pgfqpoint{2.827563in}{4.409392in}}%
\pgfpathlineto{\pgfqpoint{2.831858in}{4.415728in}}%
\pgfpathlineto{\pgfqpoint{2.834005in}{4.416155in}}%
\pgfpathlineto{\pgfqpoint{2.836152in}{4.414751in}}%
\pgfpathlineto{\pgfqpoint{2.840446in}{4.406471in}}%
\pgfpathlineto{\pgfqpoint{2.844740in}{4.390969in}}%
\pgfpathlineto{\pgfqpoint{2.851181in}{4.354528in}}%
\pgfpathlineto{\pgfqpoint{2.857623in}{4.303006in}}%
\pgfpathlineto{\pgfqpoint{2.866211in}{4.212871in}}%
\pgfpathlineto{\pgfqpoint{2.876946in}{4.070916in}}%
\pgfpathlineto{\pgfqpoint{2.891976in}{3.833698in}}%
\pgfpathlineto{\pgfqpoint{2.926329in}{3.271428in}}%
\pgfpathlineto{\pgfqpoint{2.937065in}{3.132941in}}%
\pgfpathlineto{\pgfqpoint{2.945653in}{3.046245in}}%
\pgfpathlineto{\pgfqpoint{2.952095in}{2.997601in}}%
\pgfpathlineto{\pgfqpoint{2.958536in}{2.964224in}}%
\pgfpathlineto{\pgfqpoint{2.962830in}{2.950834in}}%
\pgfpathlineto{\pgfqpoint{2.967124in}{2.944697in}}%
\pgfpathlineto{\pgfqpoint{2.969271in}{2.944371in}}%
\pgfpathlineto{\pgfqpoint{2.971418in}{2.945874in}}%
\pgfpathlineto{\pgfqpoint{2.975712in}{2.954354in}}%
\pgfpathlineto{\pgfqpoint{2.980007in}{2.970052in}}%
\pgfpathlineto{\pgfqpoint{2.986448in}{3.006776in}}%
\pgfpathlineto{\pgfqpoint{2.992889in}{3.058565in}}%
\pgfpathlineto{\pgfqpoint{3.001478in}{3.149016in}}%
\pgfpathlineto{\pgfqpoint{3.012213in}{3.291287in}}%
\pgfpathlineto{\pgfqpoint{3.027243in}{3.528768in}}%
\pgfpathlineto{\pgfqpoint{3.061596in}{4.090740in}}%
\pgfpathlineto{\pgfqpoint{3.072332in}{4.228898in}}%
\pgfpathlineto{\pgfqpoint{3.080920in}{4.315270in}}%
\pgfpathlineto{\pgfqpoint{3.087361in}{4.363644in}}%
\pgfpathlineto{\pgfqpoint{3.089508in}{4.376416in}}%
\pgfpathlineto{\pgfqpoint{3.089508in}{4.376416in}}%
\pgfusepath{stroke}%
\end{pgfscope}%
\begin{pgfscope}%
\pgfsetrectcap%
\pgfsetmiterjoin%
\pgfsetlinewidth{0.803000pt}%
\definecolor{currentstroke}{rgb}{0.000000,0.000000,0.000000}%
\pgfsetstrokecolor{currentstroke}%
\pgfsetdash{}{0pt}%
\pgfpathmoveto{\pgfqpoint{0.835065in}{2.870679in}}%
\pgfpathlineto{\pgfqpoint{0.835065in}{4.489815in}}%
\pgfusepath{stroke}%
\end{pgfscope}%
\begin{pgfscope}%
\pgfsetrectcap%
\pgfsetmiterjoin%
\pgfsetlinewidth{0.803000pt}%
\definecolor{currentstroke}{rgb}{0.000000,0.000000,0.000000}%
\pgfsetstrokecolor{currentstroke}%
\pgfsetdash{}{0pt}%
\pgfpathmoveto{\pgfqpoint{3.196863in}{2.870679in}}%
\pgfpathlineto{\pgfqpoint{3.196863in}{4.489815in}}%
\pgfusepath{stroke}%
\end{pgfscope}%
\begin{pgfscope}%
\pgfsetrectcap%
\pgfsetmiterjoin%
\pgfsetlinewidth{0.803000pt}%
\definecolor{currentstroke}{rgb}{0.000000,0.000000,0.000000}%
\pgfsetstrokecolor{currentstroke}%
\pgfsetdash{}{0pt}%
\pgfpathmoveto{\pgfqpoint{0.835065in}{2.870679in}}%
\pgfpathlineto{\pgfqpoint{3.196863in}{2.870679in}}%
\pgfusepath{stroke}%
\end{pgfscope}%
\begin{pgfscope}%
\pgfsetrectcap%
\pgfsetmiterjoin%
\pgfsetlinewidth{0.803000pt}%
\definecolor{currentstroke}{rgb}{0.000000,0.000000,0.000000}%
\pgfsetstrokecolor{currentstroke}%
\pgfsetdash{}{0pt}%
\pgfpathmoveto{\pgfqpoint{0.835065in}{4.489815in}}%
\pgfpathlineto{\pgfqpoint{3.196863in}{4.489815in}}%
\pgfusepath{stroke}%
\end{pgfscope}%
\begin{pgfscope}%
\definecolor{textcolor}{rgb}{0.000000,0.000000,0.000000}%
\pgfsetstrokecolor{textcolor}%
\pgfsetfillcolor{textcolor}%
\pgftext[x=2.015964in,y=4.573148in,,base]{\color{textcolor}\rmfamily\fontsize{12.000000}{14.400000}\selectfont \(\displaystyle \theta\)}%
\end{pgfscope}%
\begin{pgfscope}%
\pgfsetbuttcap%
\pgfsetmiterjoin%
\definecolor{currentfill}{rgb}{1.000000,1.000000,1.000000}%
\pgfsetfillcolor{currentfill}%
\pgfsetlinewidth{0.000000pt}%
\definecolor{currentstroke}{rgb}{0.000000,0.000000,0.000000}%
\pgfsetstrokecolor{currentstroke}%
\pgfsetstrokeopacity{0.000000}%
\pgfsetdash{}{0pt}%
\pgfpathmoveto{\pgfqpoint{3.906113in}{2.870679in}}%
\pgfpathlineto{\pgfqpoint{6.267911in}{2.870679in}}%
\pgfpathlineto{\pgfqpoint{6.267911in}{4.489815in}}%
\pgfpathlineto{\pgfqpoint{3.906113in}{4.489815in}}%
\pgfpathclose%
\pgfusepath{fill}%
\end{pgfscope}%
\begin{pgfscope}%
\pgfsetbuttcap%
\pgfsetroundjoin%
\definecolor{currentfill}{rgb}{0.000000,0.000000,0.000000}%
\pgfsetfillcolor{currentfill}%
\pgfsetlinewidth{0.803000pt}%
\definecolor{currentstroke}{rgb}{0.000000,0.000000,0.000000}%
\pgfsetstrokecolor{currentstroke}%
\pgfsetdash{}{0pt}%
\pgfsys@defobject{currentmarker}{\pgfqpoint{0.000000in}{-0.048611in}}{\pgfqpoint{0.000000in}{0.000000in}}{%
\pgfpathmoveto{\pgfqpoint{0.000000in}{0.000000in}}%
\pgfpathlineto{\pgfqpoint{0.000000in}{-0.048611in}}%
\pgfusepath{stroke,fill}%
}%
\begin{pgfscope}%
\pgfsys@transformshift{4.013467in}{2.870679in}%
\pgfsys@useobject{currentmarker}{}%
\end{pgfscope}%
\end{pgfscope}%
\begin{pgfscope}%
\definecolor{textcolor}{rgb}{0.000000,0.000000,0.000000}%
\pgfsetstrokecolor{textcolor}%
\pgfsetfillcolor{textcolor}%
\pgftext[x=4.013467in,y=2.773457in,,top]{\color{textcolor}\rmfamily\fontsize{10.000000}{12.000000}\selectfont \(\displaystyle 0.0\)}%
\end{pgfscope}%
\begin{pgfscope}%
\pgfsetbuttcap%
\pgfsetroundjoin%
\definecolor{currentfill}{rgb}{0.000000,0.000000,0.000000}%
\pgfsetfillcolor{currentfill}%
\pgfsetlinewidth{0.803000pt}%
\definecolor{currentstroke}{rgb}{0.000000,0.000000,0.000000}%
\pgfsetstrokecolor{currentstroke}%
\pgfsetdash{}{0pt}%
\pgfsys@defobject{currentmarker}{\pgfqpoint{0.000000in}{-0.048611in}}{\pgfqpoint{0.000000in}{0.000000in}}{%
\pgfpathmoveto{\pgfqpoint{0.000000in}{0.000000in}}%
\pgfpathlineto{\pgfqpoint{0.000000in}{-0.048611in}}%
\pgfusepath{stroke,fill}%
}%
\begin{pgfscope}%
\pgfsys@transformshift{4.550240in}{2.870679in}%
\pgfsys@useobject{currentmarker}{}%
\end{pgfscope}%
\end{pgfscope}%
\begin{pgfscope}%
\definecolor{textcolor}{rgb}{0.000000,0.000000,0.000000}%
\pgfsetstrokecolor{textcolor}%
\pgfsetfillcolor{textcolor}%
\pgftext[x=4.550240in,y=2.773457in,,top]{\color{textcolor}\rmfamily\fontsize{10.000000}{12.000000}\selectfont \(\displaystyle 2.5\)}%
\end{pgfscope}%
\begin{pgfscope}%
\pgfsetbuttcap%
\pgfsetroundjoin%
\definecolor{currentfill}{rgb}{0.000000,0.000000,0.000000}%
\pgfsetfillcolor{currentfill}%
\pgfsetlinewidth{0.803000pt}%
\definecolor{currentstroke}{rgb}{0.000000,0.000000,0.000000}%
\pgfsetstrokecolor{currentstroke}%
\pgfsetdash{}{0pt}%
\pgfsys@defobject{currentmarker}{\pgfqpoint{0.000000in}{-0.048611in}}{\pgfqpoint{0.000000in}{0.000000in}}{%
\pgfpathmoveto{\pgfqpoint{0.000000in}{0.000000in}}%
\pgfpathlineto{\pgfqpoint{0.000000in}{-0.048611in}}%
\pgfusepath{stroke,fill}%
}%
\begin{pgfscope}%
\pgfsys@transformshift{5.087012in}{2.870679in}%
\pgfsys@useobject{currentmarker}{}%
\end{pgfscope}%
\end{pgfscope}%
\begin{pgfscope}%
\definecolor{textcolor}{rgb}{0.000000,0.000000,0.000000}%
\pgfsetstrokecolor{textcolor}%
\pgfsetfillcolor{textcolor}%
\pgftext[x=5.087012in,y=2.773457in,,top]{\color{textcolor}\rmfamily\fontsize{10.000000}{12.000000}\selectfont \(\displaystyle 5.0\)}%
\end{pgfscope}%
\begin{pgfscope}%
\pgfsetbuttcap%
\pgfsetroundjoin%
\definecolor{currentfill}{rgb}{0.000000,0.000000,0.000000}%
\pgfsetfillcolor{currentfill}%
\pgfsetlinewidth{0.803000pt}%
\definecolor{currentstroke}{rgb}{0.000000,0.000000,0.000000}%
\pgfsetstrokecolor{currentstroke}%
\pgfsetdash{}{0pt}%
\pgfsys@defobject{currentmarker}{\pgfqpoint{0.000000in}{-0.048611in}}{\pgfqpoint{0.000000in}{0.000000in}}{%
\pgfpathmoveto{\pgfqpoint{0.000000in}{0.000000in}}%
\pgfpathlineto{\pgfqpoint{0.000000in}{-0.048611in}}%
\pgfusepath{stroke,fill}%
}%
\begin{pgfscope}%
\pgfsys@transformshift{5.623784in}{2.870679in}%
\pgfsys@useobject{currentmarker}{}%
\end{pgfscope}%
\end{pgfscope}%
\begin{pgfscope}%
\definecolor{textcolor}{rgb}{0.000000,0.000000,0.000000}%
\pgfsetstrokecolor{textcolor}%
\pgfsetfillcolor{textcolor}%
\pgftext[x=5.623784in,y=2.773457in,,top]{\color{textcolor}\rmfamily\fontsize{10.000000}{12.000000}\selectfont \(\displaystyle 7.5\)}%
\end{pgfscope}%
\begin{pgfscope}%
\pgfsetbuttcap%
\pgfsetroundjoin%
\definecolor{currentfill}{rgb}{0.000000,0.000000,0.000000}%
\pgfsetfillcolor{currentfill}%
\pgfsetlinewidth{0.803000pt}%
\definecolor{currentstroke}{rgb}{0.000000,0.000000,0.000000}%
\pgfsetstrokecolor{currentstroke}%
\pgfsetdash{}{0pt}%
\pgfsys@defobject{currentmarker}{\pgfqpoint{0.000000in}{-0.048611in}}{\pgfqpoint{0.000000in}{0.000000in}}{%
\pgfpathmoveto{\pgfqpoint{0.000000in}{0.000000in}}%
\pgfpathlineto{\pgfqpoint{0.000000in}{-0.048611in}}%
\pgfusepath{stroke,fill}%
}%
\begin{pgfscope}%
\pgfsys@transformshift{6.160557in}{2.870679in}%
\pgfsys@useobject{currentmarker}{}%
\end{pgfscope}%
\end{pgfscope}%
\begin{pgfscope}%
\definecolor{textcolor}{rgb}{0.000000,0.000000,0.000000}%
\pgfsetstrokecolor{textcolor}%
\pgfsetfillcolor{textcolor}%
\pgftext[x=6.160557in,y=2.773457in,,top]{\color{textcolor}\rmfamily\fontsize{10.000000}{12.000000}\selectfont \(\displaystyle 10.0\)}%
\end{pgfscope}%
\begin{pgfscope}%
\definecolor{textcolor}{rgb}{0.000000,0.000000,0.000000}%
\pgfsetstrokecolor{textcolor}%
\pgfsetfillcolor{textcolor}%
\pgftext[x=5.087012in,y=2.594444in,,top]{\color{textcolor}\rmfamily\fontsize{10.000000}{12.000000}\selectfont time (s)}%
\end{pgfscope}%
\begin{pgfscope}%
\pgfsetbuttcap%
\pgfsetroundjoin%
\definecolor{currentfill}{rgb}{0.000000,0.000000,0.000000}%
\pgfsetfillcolor{currentfill}%
\pgfsetlinewidth{0.803000pt}%
\definecolor{currentstroke}{rgb}{0.000000,0.000000,0.000000}%
\pgfsetstrokecolor{currentstroke}%
\pgfsetdash{}{0pt}%
\pgfsys@defobject{currentmarker}{\pgfqpoint{-0.048611in}{0.000000in}}{\pgfqpoint{0.000000in}{0.000000in}}{%
\pgfpathmoveto{\pgfqpoint{0.000000in}{0.000000in}}%
\pgfpathlineto{\pgfqpoint{-0.048611in}{0.000000in}}%
\pgfusepath{stroke,fill}%
}%
\begin{pgfscope}%
\pgfsys@transformshift{3.906113in}{3.258162in}%
\pgfsys@useobject{currentmarker}{}%
\end{pgfscope}%
\end{pgfscope}%
\begin{pgfscope}%
\definecolor{textcolor}{rgb}{0.000000,0.000000,0.000000}%
\pgfsetstrokecolor{textcolor}%
\pgfsetfillcolor{textcolor}%
\pgftext[x=3.523396in,y=3.209936in,left,base]{\color{textcolor}\rmfamily\fontsize{10.000000}{12.000000}\selectfont \(\displaystyle -0.5\)}%
\end{pgfscope}%
\begin{pgfscope}%
\pgfsetbuttcap%
\pgfsetroundjoin%
\definecolor{currentfill}{rgb}{0.000000,0.000000,0.000000}%
\pgfsetfillcolor{currentfill}%
\pgfsetlinewidth{0.803000pt}%
\definecolor{currentstroke}{rgb}{0.000000,0.000000,0.000000}%
\pgfsetstrokecolor{currentstroke}%
\pgfsetdash{}{0pt}%
\pgfsys@defobject{currentmarker}{\pgfqpoint{-0.048611in}{0.000000in}}{\pgfqpoint{0.000000in}{0.000000in}}{%
\pgfpathmoveto{\pgfqpoint{0.000000in}{0.000000in}}%
\pgfpathlineto{\pgfqpoint{-0.048611in}{0.000000in}}%
\pgfusepath{stroke,fill}%
}%
\begin{pgfscope}%
\pgfsys@transformshift{3.906113in}{3.680247in}%
\pgfsys@useobject{currentmarker}{}%
\end{pgfscope}%
\end{pgfscope}%
\begin{pgfscope}%
\definecolor{textcolor}{rgb}{0.000000,0.000000,0.000000}%
\pgfsetstrokecolor{textcolor}%
\pgfsetfillcolor{textcolor}%
\pgftext[x=3.631421in,y=3.632021in,left,base]{\color{textcolor}\rmfamily\fontsize{10.000000}{12.000000}\selectfont \(\displaystyle 0.0\)}%
\end{pgfscope}%
\begin{pgfscope}%
\pgfsetbuttcap%
\pgfsetroundjoin%
\definecolor{currentfill}{rgb}{0.000000,0.000000,0.000000}%
\pgfsetfillcolor{currentfill}%
\pgfsetlinewidth{0.803000pt}%
\definecolor{currentstroke}{rgb}{0.000000,0.000000,0.000000}%
\pgfsetstrokecolor{currentstroke}%
\pgfsetdash{}{0pt}%
\pgfsys@defobject{currentmarker}{\pgfqpoint{-0.048611in}{0.000000in}}{\pgfqpoint{0.000000in}{0.000000in}}{%
\pgfpathmoveto{\pgfqpoint{0.000000in}{0.000000in}}%
\pgfpathlineto{\pgfqpoint{-0.048611in}{0.000000in}}%
\pgfusepath{stroke,fill}%
}%
\begin{pgfscope}%
\pgfsys@transformshift{3.906113in}{4.102332in}%
\pgfsys@useobject{currentmarker}{}%
\end{pgfscope}%
\end{pgfscope}%
\begin{pgfscope}%
\definecolor{textcolor}{rgb}{0.000000,0.000000,0.000000}%
\pgfsetstrokecolor{textcolor}%
\pgfsetfillcolor{textcolor}%
\pgftext[x=3.631421in,y=4.054107in,left,base]{\color{textcolor}\rmfamily\fontsize{10.000000}{12.000000}\selectfont \(\displaystyle 0.5\)}%
\end{pgfscope}%
\begin{pgfscope}%
\definecolor{textcolor}{rgb}{0.000000,0.000000,0.000000}%
\pgfsetstrokecolor{textcolor}%
\pgfsetfillcolor{textcolor}%
\pgftext[x=3.467840in,y=3.680247in,,bottom,rotate=90.000000]{\color{textcolor}\rmfamily\fontsize{10.000000}{12.000000}\selectfont angle (rad)}%
\end{pgfscope}%
\begin{pgfscope}%
\pgfpathrectangle{\pgfqpoint{3.906113in}{2.870679in}}{\pgfqpoint{2.361798in}{1.619136in}}%
\pgfusepath{clip}%
\pgfsetrectcap%
\pgfsetroundjoin%
\pgfsetlinewidth{1.505625pt}%
\definecolor{currentstroke}{rgb}{0.000000,0.000000,1.000000}%
\pgfsetstrokecolor{currentstroke}%
\pgfsetdash{}{0pt}%
\pgfpathmoveto{\pgfqpoint{4.013467in}{3.680247in}}%
\pgfpathlineto{\pgfqpoint{4.034938in}{3.328503in}}%
\pgfpathlineto{\pgfqpoint{4.047821in}{3.153523in}}%
\pgfpathlineto{\pgfqpoint{4.056409in}{3.062012in}}%
\pgfpathlineto{\pgfqpoint{4.064998in}{2.995044in}}%
\pgfpathlineto{\pgfqpoint{4.071439in}{2.962603in}}%
\pgfpathlineto{\pgfqpoint{4.075733in}{2.949893in}}%
\pgfpathlineto{\pgfqpoint{4.080027in}{2.944480in}}%
\pgfpathlineto{\pgfqpoint{4.082174in}{2.944531in}}%
\pgfpathlineto{\pgfqpoint{4.084321in}{2.946420in}}%
\pgfpathlineto{\pgfqpoint{4.088616in}{2.955693in}}%
\pgfpathlineto{\pgfqpoint{4.092910in}{2.972204in}}%
\pgfpathlineto{\pgfqpoint{4.099351in}{3.010149in}}%
\pgfpathlineto{\pgfqpoint{4.105792in}{3.063107in}}%
\pgfpathlineto{\pgfqpoint{4.114381in}{3.154930in}}%
\pgfpathlineto{\pgfqpoint{4.125116in}{3.298436in}}%
\pgfpathlineto{\pgfqpoint{4.140146in}{3.536528in}}%
\pgfpathlineto{\pgfqpoint{4.174499in}{4.096270in}}%
\pgfpathlineto{\pgfqpoint{4.185235in}{4.233292in}}%
\pgfpathlineto{\pgfqpoint{4.193823in}{4.318653in}}%
\pgfpathlineto{\pgfqpoint{4.200264in}{4.366182in}}%
\pgfpathlineto{\pgfqpoint{4.206705in}{4.398334in}}%
\pgfpathlineto{\pgfqpoint{4.211000in}{4.410846in}}%
\pgfpathlineto{\pgfqpoint{4.215294in}{4.416058in}}%
\pgfpathlineto{\pgfqpoint{4.217441in}{4.415907in}}%
\pgfpathlineto{\pgfqpoint{4.219588in}{4.413916in}}%
\pgfpathlineto{\pgfqpoint{4.223882in}{4.404444in}}%
\pgfpathlineto{\pgfqpoint{4.228176in}{4.387737in}}%
\pgfpathlineto{\pgfqpoint{4.234618in}{4.349509in}}%
\pgfpathlineto{\pgfqpoint{4.241059in}{4.296287in}}%
\pgfpathlineto{\pgfqpoint{4.249647in}{4.204152in}}%
\pgfpathlineto{\pgfqpoint{4.260383in}{4.060337in}}%
\pgfpathlineto{\pgfqpoint{4.275412in}{3.821995in}}%
\pgfpathlineto{\pgfqpoint{4.307619in}{3.293293in}}%
\pgfpathlineto{\pgfqpoint{4.318354in}{3.150720in}}%
\pgfpathlineto{\pgfqpoint{4.326942in}{3.059836in}}%
\pgfpathlineto{\pgfqpoint{4.335531in}{2.993584in}}%
\pgfpathlineto{\pgfqpoint{4.341972in}{2.961720in}}%
\pgfpathlineto{\pgfqpoint{4.346266in}{2.949407in}}%
\pgfpathlineto{\pgfqpoint{4.350560in}{2.944396in}}%
\pgfpathlineto{\pgfqpoint{4.352708in}{2.944648in}}%
\pgfpathlineto{\pgfqpoint{4.354855in}{2.946739in}}%
\pgfpathlineto{\pgfqpoint{4.359149in}{2.956412in}}%
\pgfpathlineto{\pgfqpoint{4.363443in}{2.973314in}}%
\pgfpathlineto{\pgfqpoint{4.369884in}{3.011825in}}%
\pgfpathlineto{\pgfqpoint{4.376326in}{3.065310in}}%
\pgfpathlineto{\pgfqpoint{4.384914in}{3.157756in}}%
\pgfpathlineto{\pgfqpoint{4.395649in}{3.301879in}}%
\pgfpathlineto{\pgfqpoint{4.410679in}{3.540470in}}%
\pgfpathlineto{\pgfqpoint{4.442885in}{4.068908in}}%
\pgfpathlineto{\pgfqpoint{4.453621in}{4.211169in}}%
\pgfpathlineto{\pgfqpoint{4.462209in}{4.301738in}}%
\pgfpathlineto{\pgfqpoint{4.470797in}{4.367632in}}%
\pgfpathlineto{\pgfqpoint{4.477239in}{4.399207in}}%
\pgfpathlineto{\pgfqpoint{4.481533in}{4.411321in}}%
\pgfpathlineto{\pgfqpoint{4.485827in}{4.416131in}}%
\pgfpathlineto{\pgfqpoint{4.487974in}{4.415778in}}%
\pgfpathlineto{\pgfqpoint{4.490121in}{4.413587in}}%
\pgfpathlineto{\pgfqpoint{4.494415in}{4.403714in}}%
\pgfpathlineto{\pgfqpoint{4.498710in}{4.386616in}}%
\pgfpathlineto{\pgfqpoint{4.505151in}{4.347823in}}%
\pgfpathlineto{\pgfqpoint{4.511592in}{4.294075in}}%
\pgfpathlineto{\pgfqpoint{4.520180in}{4.201318in}}%
\pgfpathlineto{\pgfqpoint{4.530916in}{4.056889in}}%
\pgfpathlineto{\pgfqpoint{4.545946in}{3.818051in}}%
\pgfpathlineto{\pgfqpoint{4.578152in}{3.289879in}}%
\pgfpathlineto{\pgfqpoint{4.588887in}{3.147933in}}%
\pgfpathlineto{\pgfqpoint{4.597476in}{3.057679in}}%
\pgfpathlineto{\pgfqpoint{4.606064in}{2.992144in}}%
\pgfpathlineto{\pgfqpoint{4.612505in}{2.960858in}}%
\pgfpathlineto{\pgfqpoint{4.616800in}{2.948943in}}%
\pgfpathlineto{\pgfqpoint{4.621094in}{2.944334in}}%
\pgfpathlineto{\pgfqpoint{4.623241in}{2.944788in}}%
\pgfpathlineto{\pgfqpoint{4.625388in}{2.947080in}}%
\pgfpathlineto{\pgfqpoint{4.629682in}{2.957152in}}%
\pgfpathlineto{\pgfqpoint{4.633976in}{2.974446in}}%
\pgfpathlineto{\pgfqpoint{4.640417in}{3.013521in}}%
\pgfpathlineto{\pgfqpoint{4.646859in}{3.067532in}}%
\pgfpathlineto{\pgfqpoint{4.655447in}{3.160598in}}%
\pgfpathlineto{\pgfqpoint{4.666183in}{3.305333in}}%
\pgfpathlineto{\pgfqpoint{4.681212in}{3.544416in}}%
\pgfpathlineto{\pgfqpoint{4.713419in}{4.072317in}}%
\pgfpathlineto{\pgfqpoint{4.724154in}{4.213948in}}%
\pgfpathlineto{\pgfqpoint{4.732742in}{4.303886in}}%
\pgfpathlineto{\pgfqpoint{4.739184in}{4.355252in}}%
\pgfpathlineto{\pgfqpoint{4.745625in}{4.391493in}}%
\pgfpathlineto{\pgfqpoint{4.749919in}{4.406824in}}%
\pgfpathlineto{\pgfqpoint{4.754213in}{4.414897in}}%
\pgfpathlineto{\pgfqpoint{4.756360in}{4.416182in}}%
\pgfpathlineto{\pgfqpoint{4.758507in}{4.415628in}}%
\pgfpathlineto{\pgfqpoint{4.760654in}{4.413235in}}%
\pgfpathlineto{\pgfqpoint{4.764949in}{4.402964in}}%
\pgfpathlineto{\pgfqpoint{4.769243in}{4.385474in}}%
\pgfpathlineto{\pgfqpoint{4.775684in}{4.346117in}}%
\pgfpathlineto{\pgfqpoint{4.782125in}{4.291844in}}%
\pgfpathlineto{\pgfqpoint{4.790714in}{4.198469in}}%
\pgfpathlineto{\pgfqpoint{4.801449in}{4.053429in}}%
\pgfpathlineto{\pgfqpoint{4.816479in}{3.814103in}}%
\pgfpathlineto{\pgfqpoint{4.848685in}{3.286477in}}%
\pgfpathlineto{\pgfqpoint{4.859421in}{3.145162in}}%
\pgfpathlineto{\pgfqpoint{4.868009in}{3.055541in}}%
\pgfpathlineto{\pgfqpoint{4.874450in}{3.004441in}}%
\pgfpathlineto{\pgfqpoint{4.880891in}{2.968485in}}%
\pgfpathlineto{\pgfqpoint{4.885186in}{2.953351in}}%
\pgfpathlineto{\pgfqpoint{4.889480in}{2.945479in}}%
\pgfpathlineto{\pgfqpoint{4.891627in}{2.944294in}}%
\pgfpathlineto{\pgfqpoint{4.893774in}{2.944949in}}%
\pgfpathlineto{\pgfqpoint{4.895921in}{2.947442in}}%
\pgfpathlineto{\pgfqpoint{4.900215in}{2.957913in}}%
\pgfpathlineto{\pgfqpoint{4.904509in}{2.975599in}}%
\pgfpathlineto{\pgfqpoint{4.910951in}{3.015237in}}%
\pgfpathlineto{\pgfqpoint{4.917392in}{3.069772in}}%
\pgfpathlineto{\pgfqpoint{4.925980in}{3.163455in}}%
\pgfpathlineto{\pgfqpoint{4.936716in}{3.308798in}}%
\pgfpathlineto{\pgfqpoint{4.951745in}{3.548365in}}%
\pgfpathlineto{\pgfqpoint{4.983952in}{4.075714in}}%
\pgfpathlineto{\pgfqpoint{4.994687in}{4.216711in}}%
\pgfpathlineto{\pgfqpoint{5.003276in}{4.306015in}}%
\pgfpathlineto{\pgfqpoint{5.009717in}{4.356848in}}%
\pgfpathlineto{\pgfqpoint{5.016158in}{4.392519in}}%
\pgfpathlineto{\pgfqpoint{5.020452in}{4.407455in}}%
\pgfpathlineto{\pgfqpoint{5.024746in}{4.415127in}}%
\pgfpathlineto{\pgfqpoint{5.026894in}{4.416211in}}%
\pgfpathlineto{\pgfqpoint{5.029041in}{4.415455in}}%
\pgfpathlineto{\pgfqpoint{5.031188in}{4.412861in}}%
\pgfpathlineto{\pgfqpoint{5.035482in}{4.402191in}}%
\pgfpathlineto{\pgfqpoint{5.039776in}{4.384311in}}%
\pgfpathlineto{\pgfqpoint{5.046217in}{4.344391in}}%
\pgfpathlineto{\pgfqpoint{5.052659in}{4.289595in}}%
\pgfpathlineto{\pgfqpoint{5.061247in}{4.195604in}}%
\pgfpathlineto{\pgfqpoint{5.071982in}{4.049959in}}%
\pgfpathlineto{\pgfqpoint{5.087012in}{3.810152in}}%
\pgfpathlineto{\pgfqpoint{5.119218in}{3.283086in}}%
\pgfpathlineto{\pgfqpoint{5.129954in}{3.142407in}}%
\pgfpathlineto{\pgfqpoint{5.138542in}{3.053421in}}%
\pgfpathlineto{\pgfqpoint{5.144983in}{3.002855in}}%
\pgfpathlineto{\pgfqpoint{5.151425in}{2.967470in}}%
\pgfpathlineto{\pgfqpoint{5.155719in}{2.952731in}}%
\pgfpathlineto{\pgfqpoint{5.160013in}{2.945259in}}%
\pgfpathlineto{\pgfqpoint{5.162160in}{2.944276in}}%
\pgfpathlineto{\pgfqpoint{5.164307in}{2.945133in}}%
\pgfpathlineto{\pgfqpoint{5.166454in}{2.947827in}}%
\pgfpathlineto{\pgfqpoint{5.170749in}{2.958696in}}%
\pgfpathlineto{\pgfqpoint{5.175043in}{2.976772in}}%
\pgfpathlineto{\pgfqpoint{5.181484in}{3.016973in}}%
\pgfpathlineto{\pgfqpoint{5.187925in}{3.072030in}}%
\pgfpathlineto{\pgfqpoint{5.196514in}{3.166328in}}%
\pgfpathlineto{\pgfqpoint{5.207249in}{3.312274in}}%
\pgfpathlineto{\pgfqpoint{5.222279in}{3.552319in}}%
\pgfpathlineto{\pgfqpoint{5.254485in}{4.079099in}}%
\pgfpathlineto{\pgfqpoint{5.265220in}{4.219458in}}%
\pgfpathlineto{\pgfqpoint{5.273809in}{4.308125in}}%
\pgfpathlineto{\pgfqpoint{5.280250in}{4.358423in}}%
\pgfpathlineto{\pgfqpoint{5.286691in}{4.393522in}}%
\pgfpathlineto{\pgfqpoint{5.290986in}{4.408065in}}%
\pgfpathlineto{\pgfqpoint{5.295280in}{4.415336in}}%
\pgfpathlineto{\pgfqpoint{5.297427in}{4.416218in}}%
\pgfpathlineto{\pgfqpoint{5.299574in}{4.415261in}}%
\pgfpathlineto{\pgfqpoint{5.301721in}{4.412466in}}%
\pgfpathlineto{\pgfqpoint{5.306015in}{4.401397in}}%
\pgfpathlineto{\pgfqpoint{5.310309in}{4.383126in}}%
\pgfpathlineto{\pgfqpoint{5.316751in}{4.342645in}}%
\pgfpathlineto{\pgfqpoint{5.323192in}{4.287327in}}%
\pgfpathlineto{\pgfqpoint{5.331780in}{4.192724in}}%
\pgfpathlineto{\pgfqpoint{5.342516in}{4.046477in}}%
\pgfpathlineto{\pgfqpoint{5.357545in}{3.806196in}}%
\pgfpathlineto{\pgfqpoint{5.389752in}{3.279707in}}%
\pgfpathlineto{\pgfqpoint{5.400487in}{3.139668in}}%
\pgfpathlineto{\pgfqpoint{5.409075in}{3.051320in}}%
\pgfpathlineto{\pgfqpoint{5.415517in}{3.001290in}}%
\pgfpathlineto{\pgfqpoint{5.421958in}{2.966477in}}%
\pgfpathlineto{\pgfqpoint{5.426252in}{2.952132in}}%
\pgfpathlineto{\pgfqpoint{5.430546in}{2.945062in}}%
\pgfpathlineto{\pgfqpoint{5.432693in}{2.944280in}}%
\pgfpathlineto{\pgfqpoint{5.434840in}{2.945338in}}%
\pgfpathlineto{\pgfqpoint{5.439135in}{2.952959in}}%
\pgfpathlineto{\pgfqpoint{5.443429in}{2.967845in}}%
\pgfpathlineto{\pgfqpoint{5.449870in}{3.003442in}}%
\pgfpathlineto{\pgfqpoint{5.456311in}{3.054206in}}%
\pgfpathlineto{\pgfqpoint{5.464900in}{3.143429in}}%
\pgfpathlineto{\pgfqpoint{5.475635in}{3.284344in}}%
\pgfpathlineto{\pgfqpoint{5.490665in}{3.520302in}}%
\pgfpathlineto{\pgfqpoint{5.525018in}{4.082472in}}%
\pgfpathlineto{\pgfqpoint{5.535754in}{4.222189in}}%
\pgfpathlineto{\pgfqpoint{5.544342in}{4.310217in}}%
\pgfpathlineto{\pgfqpoint{5.550783in}{4.359978in}}%
\pgfpathlineto{\pgfqpoint{5.557225in}{4.394505in}}%
\pgfpathlineto{\pgfqpoint{5.561519in}{4.408652in}}%
\pgfpathlineto{\pgfqpoint{5.565813in}{4.415522in}}%
\pgfpathlineto{\pgfqpoint{5.567960in}{4.416203in}}%
\pgfpathlineto{\pgfqpoint{5.570107in}{4.415044in}}%
\pgfpathlineto{\pgfqpoint{5.574401in}{4.407223in}}%
\pgfpathlineto{\pgfqpoint{5.578695in}{4.392140in}}%
\pgfpathlineto{\pgfqpoint{5.585137in}{4.356257in}}%
\pgfpathlineto{\pgfqpoint{5.591578in}{4.305226in}}%
\pgfpathlineto{\pgfqpoint{5.600166in}{4.215686in}}%
\pgfpathlineto{\pgfqpoint{5.610902in}{4.074453in}}%
\pgfpathlineto{\pgfqpoint{5.625931in}{3.838230in}}%
\pgfpathlineto{\pgfqpoint{5.660285in}{3.276339in}}%
\pgfpathlineto{\pgfqpoint{5.671020in}{3.136945in}}%
\pgfpathlineto{\pgfqpoint{5.679609in}{3.049237in}}%
\pgfpathlineto{\pgfqpoint{5.686050in}{2.999745in}}%
\pgfpathlineto{\pgfqpoint{5.692491in}{2.965505in}}%
\pgfpathlineto{\pgfqpoint{5.696785in}{2.951556in}}%
\pgfpathlineto{\pgfqpoint{5.701080in}{2.944886in}}%
\pgfpathlineto{\pgfqpoint{5.703227in}{2.944306in}}%
\pgfpathlineto{\pgfqpoint{5.705374in}{2.945566in}}%
\pgfpathlineto{\pgfqpoint{5.709668in}{2.953587in}}%
\pgfpathlineto{\pgfqpoint{5.713962in}{2.968867in}}%
\pgfpathlineto{\pgfqpoint{5.720403in}{3.005035in}}%
\pgfpathlineto{\pgfqpoint{5.726845in}{3.056333in}}%
\pgfpathlineto{\pgfqpoint{5.735433in}{3.146190in}}%
\pgfpathlineto{\pgfqpoint{5.746168in}{3.287739in}}%
\pgfpathlineto{\pgfqpoint{5.761198in}{3.524225in}}%
\pgfpathlineto{\pgfqpoint{5.795551in}{4.085833in}}%
\pgfpathlineto{\pgfqpoint{5.806287in}{4.224904in}}%
\pgfpathlineto{\pgfqpoint{5.814875in}{4.312290in}}%
\pgfpathlineto{\pgfqpoint{5.821317in}{4.361513in}}%
\pgfpathlineto{\pgfqpoint{5.827758in}{4.395466in}}%
\pgfpathlineto{\pgfqpoint{5.832052in}{4.409218in}}%
\pgfpathlineto{\pgfqpoint{5.836346in}{4.415686in}}%
\pgfpathlineto{\pgfqpoint{5.838493in}{4.416166in}}%
\pgfpathlineto{\pgfqpoint{5.840640in}{4.414805in}}%
\pgfpathlineto{\pgfqpoint{5.844935in}{4.406584in}}%
\pgfpathlineto{\pgfqpoint{5.849229in}{4.391107in}}%
\pgfpathlineto{\pgfqpoint{5.855670in}{4.354654in}}%
\pgfpathlineto{\pgfqpoint{5.862111in}{4.303090in}}%
\pgfpathlineto{\pgfqpoint{5.870700in}{4.212917in}}%
\pgfpathlineto{\pgfqpoint{5.881435in}{4.071052in}}%
\pgfpathlineto{\pgfqpoint{5.896465in}{3.834305in}}%
\pgfpathlineto{\pgfqpoint{5.930818in}{3.272984in}}%
\pgfpathlineto{\pgfqpoint{5.941554in}{3.134238in}}%
\pgfpathlineto{\pgfqpoint{5.950142in}{3.047174in}}%
\pgfpathlineto{\pgfqpoint{5.956583in}{2.998221in}}%
\pgfpathlineto{\pgfqpoint{5.963024in}{2.964555in}}%
\pgfpathlineto{\pgfqpoint{5.967319in}{2.951001in}}%
\pgfpathlineto{\pgfqpoint{5.971613in}{2.944733in}}%
\pgfpathlineto{\pgfqpoint{5.973760in}{2.944355in}}%
\pgfpathlineto{\pgfqpoint{5.975907in}{2.945816in}}%
\pgfpathlineto{\pgfqpoint{5.980201in}{2.954237in}}%
\pgfpathlineto{\pgfqpoint{5.984495in}{2.969911in}}%
\pgfpathlineto{\pgfqpoint{5.990937in}{3.006648in}}%
\pgfpathlineto{\pgfqpoint{5.997378in}{3.058479in}}%
\pgfpathlineto{\pgfqpoint{6.005966in}{3.148967in}}%
\pgfpathlineto{\pgfqpoint{6.016702in}{3.291146in}}%
\pgfpathlineto{\pgfqpoint{6.031731in}{3.528153in}}%
\pgfpathlineto{\pgfqpoint{6.066085in}{4.089182in}}%
\pgfpathlineto{\pgfqpoint{6.076820in}{4.227603in}}%
\pgfpathlineto{\pgfqpoint{6.085408in}{4.314344in}}%
\pgfpathlineto{\pgfqpoint{6.091850in}{4.363027in}}%
\pgfpathlineto{\pgfqpoint{6.098291in}{4.396406in}}%
\pgfpathlineto{\pgfqpoint{6.102585in}{4.409761in}}%
\pgfpathlineto{\pgfqpoint{6.106879in}{4.415828in}}%
\pgfpathlineto{\pgfqpoint{6.109026in}{4.416106in}}%
\pgfpathlineto{\pgfqpoint{6.111174in}{4.414545in}}%
\pgfpathlineto{\pgfqpoint{6.115468in}{4.405923in}}%
\pgfpathlineto{\pgfqpoint{6.119762in}{4.390052in}}%
\pgfpathlineto{\pgfqpoint{6.126203in}{4.353031in}}%
\pgfpathlineto{\pgfqpoint{6.132644in}{4.300935in}}%
\pgfpathlineto{\pgfqpoint{6.141233in}{4.210132in}}%
\pgfpathlineto{\pgfqpoint{6.151968in}{4.067639in}}%
\pgfpathlineto{\pgfqpoint{6.160557in}{3.936006in}}%
\pgfpathlineto{\pgfqpoint{6.160557in}{3.936006in}}%
\pgfusepath{stroke}%
\end{pgfscope}%
\begin{pgfscope}%
\pgfsetrectcap%
\pgfsetmiterjoin%
\pgfsetlinewidth{0.803000pt}%
\definecolor{currentstroke}{rgb}{0.000000,0.000000,0.000000}%
\pgfsetstrokecolor{currentstroke}%
\pgfsetdash{}{0pt}%
\pgfpathmoveto{\pgfqpoint{3.906113in}{2.870679in}}%
\pgfpathlineto{\pgfqpoint{3.906113in}{4.489815in}}%
\pgfusepath{stroke}%
\end{pgfscope}%
\begin{pgfscope}%
\pgfsetrectcap%
\pgfsetmiterjoin%
\pgfsetlinewidth{0.803000pt}%
\definecolor{currentstroke}{rgb}{0.000000,0.000000,0.000000}%
\pgfsetstrokecolor{currentstroke}%
\pgfsetdash{}{0pt}%
\pgfpathmoveto{\pgfqpoint{6.267911in}{2.870679in}}%
\pgfpathlineto{\pgfqpoint{6.267911in}{4.489815in}}%
\pgfusepath{stroke}%
\end{pgfscope}%
\begin{pgfscope}%
\pgfsetrectcap%
\pgfsetmiterjoin%
\pgfsetlinewidth{0.803000pt}%
\definecolor{currentstroke}{rgb}{0.000000,0.000000,0.000000}%
\pgfsetstrokecolor{currentstroke}%
\pgfsetdash{}{0pt}%
\pgfpathmoveto{\pgfqpoint{3.906113in}{2.870679in}}%
\pgfpathlineto{\pgfqpoint{6.267911in}{2.870679in}}%
\pgfusepath{stroke}%
\end{pgfscope}%
\begin{pgfscope}%
\pgfsetrectcap%
\pgfsetmiterjoin%
\pgfsetlinewidth{0.803000pt}%
\definecolor{currentstroke}{rgb}{0.000000,0.000000,0.000000}%
\pgfsetstrokecolor{currentstroke}%
\pgfsetdash{}{0pt}%
\pgfpathmoveto{\pgfqpoint{3.906113in}{4.489815in}}%
\pgfpathlineto{\pgfqpoint{6.267911in}{4.489815in}}%
\pgfusepath{stroke}%
\end{pgfscope}%
\begin{pgfscope}%
\definecolor{textcolor}{rgb}{0.000000,0.000000,0.000000}%
\pgfsetstrokecolor{textcolor}%
\pgfsetfillcolor{textcolor}%
\pgftext[x=5.087012in,y=4.573148in,,base]{\color{textcolor}\rmfamily\fontsize{12.000000}{14.400000}\selectfont \(\displaystyle \omega\)}%
\end{pgfscope}%
\begin{pgfscope}%
\pgfsetbuttcap%
\pgfsetmiterjoin%
\definecolor{currentfill}{rgb}{1.000000,1.000000,1.000000}%
\pgfsetfillcolor{currentfill}%
\pgfsetlinewidth{0.000000pt}%
\definecolor{currentstroke}{rgb}{0.000000,0.000000,0.000000}%
\pgfsetstrokecolor{currentstroke}%
\pgfsetstrokeopacity{0.000000}%
\pgfsetdash{}{0pt}%
\pgfpathmoveto{\pgfqpoint{0.835065in}{0.526234in}}%
\pgfpathlineto{\pgfqpoint{3.196863in}{0.526234in}}%
\pgfpathlineto{\pgfqpoint{3.196863in}{2.145371in}}%
\pgfpathlineto{\pgfqpoint{0.835065in}{2.145371in}}%
\pgfpathclose%
\pgfusepath{fill}%
\end{pgfscope}%
\begin{pgfscope}%
\pgfsetbuttcap%
\pgfsetroundjoin%
\definecolor{currentfill}{rgb}{0.000000,0.000000,0.000000}%
\pgfsetfillcolor{currentfill}%
\pgfsetlinewidth{0.803000pt}%
\definecolor{currentstroke}{rgb}{0.000000,0.000000,0.000000}%
\pgfsetstrokecolor{currentstroke}%
\pgfsetdash{}{0pt}%
\pgfsys@defobject{currentmarker}{\pgfqpoint{0.000000in}{-0.048611in}}{\pgfqpoint{0.000000in}{0.000000in}}{%
\pgfpathmoveto{\pgfqpoint{0.000000in}{0.000000in}}%
\pgfpathlineto{\pgfqpoint{0.000000in}{-0.048611in}}%
\pgfusepath{stroke,fill}%
}%
\begin{pgfscope}%
\pgfsys@transformshift{1.401060in}{0.526234in}%
\pgfsys@useobject{currentmarker}{}%
\end{pgfscope}%
\end{pgfscope}%
\begin{pgfscope}%
\definecolor{textcolor}{rgb}{0.000000,0.000000,0.000000}%
\pgfsetstrokecolor{textcolor}%
\pgfsetfillcolor{textcolor}%
\pgftext[x=1.401060in,y=0.429012in,,top]{\color{textcolor}\rmfamily\fontsize{10.000000}{12.000000}\selectfont \(\displaystyle -0.1\)}%
\end{pgfscope}%
\begin{pgfscope}%
\pgfsetbuttcap%
\pgfsetroundjoin%
\definecolor{currentfill}{rgb}{0.000000,0.000000,0.000000}%
\pgfsetfillcolor{currentfill}%
\pgfsetlinewidth{0.803000pt}%
\definecolor{currentstroke}{rgb}{0.000000,0.000000,0.000000}%
\pgfsetstrokecolor{currentstroke}%
\pgfsetdash{}{0pt}%
\pgfsys@defobject{currentmarker}{\pgfqpoint{0.000000in}{-0.048611in}}{\pgfqpoint{0.000000in}{0.000000in}}{%
\pgfpathmoveto{\pgfqpoint{0.000000in}{0.000000in}}%
\pgfpathlineto{\pgfqpoint{0.000000in}{-0.048611in}}%
\pgfusepath{stroke,fill}%
}%
\begin{pgfscope}%
\pgfsys@transformshift{2.015965in}{0.526234in}%
\pgfsys@useobject{currentmarker}{}%
\end{pgfscope}%
\end{pgfscope}%
\begin{pgfscope}%
\definecolor{textcolor}{rgb}{0.000000,0.000000,0.000000}%
\pgfsetstrokecolor{textcolor}%
\pgfsetfillcolor{textcolor}%
\pgftext[x=2.015965in,y=0.429012in,,top]{\color{textcolor}\rmfamily\fontsize{10.000000}{12.000000}\selectfont \(\displaystyle 0.0\)}%
\end{pgfscope}%
\begin{pgfscope}%
\pgfsetbuttcap%
\pgfsetroundjoin%
\definecolor{currentfill}{rgb}{0.000000,0.000000,0.000000}%
\pgfsetfillcolor{currentfill}%
\pgfsetlinewidth{0.803000pt}%
\definecolor{currentstroke}{rgb}{0.000000,0.000000,0.000000}%
\pgfsetstrokecolor{currentstroke}%
\pgfsetdash{}{0pt}%
\pgfsys@defobject{currentmarker}{\pgfqpoint{0.000000in}{-0.048611in}}{\pgfqpoint{0.000000in}{0.000000in}}{%
\pgfpathmoveto{\pgfqpoint{0.000000in}{0.000000in}}%
\pgfpathlineto{\pgfqpoint{0.000000in}{-0.048611in}}%
\pgfusepath{stroke,fill}%
}%
\begin{pgfscope}%
\pgfsys@transformshift{2.630870in}{0.526234in}%
\pgfsys@useobject{currentmarker}{}%
\end{pgfscope}%
\end{pgfscope}%
\begin{pgfscope}%
\definecolor{textcolor}{rgb}{0.000000,0.000000,0.000000}%
\pgfsetstrokecolor{textcolor}%
\pgfsetfillcolor{textcolor}%
\pgftext[x=2.630870in,y=0.429012in,,top]{\color{textcolor}\rmfamily\fontsize{10.000000}{12.000000}\selectfont \(\displaystyle 0.1\)}%
\end{pgfscope}%
\begin{pgfscope}%
\definecolor{textcolor}{rgb}{0.000000,0.000000,0.000000}%
\pgfsetstrokecolor{textcolor}%
\pgfsetfillcolor{textcolor}%
\pgftext[x=2.015964in,y=0.250000in,,top]{\color{textcolor}\rmfamily\fontsize{10.000000}{12.000000}\selectfont angle (rad)}%
\end{pgfscope}%
\begin{pgfscope}%
\pgfsetbuttcap%
\pgfsetroundjoin%
\definecolor{currentfill}{rgb}{0.000000,0.000000,0.000000}%
\pgfsetfillcolor{currentfill}%
\pgfsetlinewidth{0.803000pt}%
\definecolor{currentstroke}{rgb}{0.000000,0.000000,0.000000}%
\pgfsetstrokecolor{currentstroke}%
\pgfsetdash{}{0pt}%
\pgfsys@defobject{currentmarker}{\pgfqpoint{-0.048611in}{0.000000in}}{\pgfqpoint{0.000000in}{0.000000in}}{%
\pgfpathmoveto{\pgfqpoint{0.000000in}{0.000000in}}%
\pgfpathlineto{\pgfqpoint{-0.048611in}{0.000000in}}%
\pgfusepath{stroke,fill}%
}%
\begin{pgfscope}%
\pgfsys@transformshift{0.835065in}{0.913717in}%
\pgfsys@useobject{currentmarker}{}%
\end{pgfscope}%
\end{pgfscope}%
\begin{pgfscope}%
\definecolor{textcolor}{rgb}{0.000000,0.000000,0.000000}%
\pgfsetstrokecolor{textcolor}%
\pgfsetfillcolor{textcolor}%
\pgftext[x=0.452348in,y=0.865492in,left,base]{\color{textcolor}\rmfamily\fontsize{10.000000}{12.000000}\selectfont \(\displaystyle -0.5\)}%
\end{pgfscope}%
\begin{pgfscope}%
\pgfsetbuttcap%
\pgfsetroundjoin%
\definecolor{currentfill}{rgb}{0.000000,0.000000,0.000000}%
\pgfsetfillcolor{currentfill}%
\pgfsetlinewidth{0.803000pt}%
\definecolor{currentstroke}{rgb}{0.000000,0.000000,0.000000}%
\pgfsetstrokecolor{currentstroke}%
\pgfsetdash{}{0pt}%
\pgfsys@defobject{currentmarker}{\pgfqpoint{-0.048611in}{0.000000in}}{\pgfqpoint{0.000000in}{0.000000in}}{%
\pgfpathmoveto{\pgfqpoint{0.000000in}{0.000000in}}%
\pgfpathlineto{\pgfqpoint{-0.048611in}{0.000000in}}%
\pgfusepath{stroke,fill}%
}%
\begin{pgfscope}%
\pgfsys@transformshift{0.835065in}{1.335802in}%
\pgfsys@useobject{currentmarker}{}%
\end{pgfscope}%
\end{pgfscope}%
\begin{pgfscope}%
\definecolor{textcolor}{rgb}{0.000000,0.000000,0.000000}%
\pgfsetstrokecolor{textcolor}%
\pgfsetfillcolor{textcolor}%
\pgftext[x=0.560373in,y=1.287577in,left,base]{\color{textcolor}\rmfamily\fontsize{10.000000}{12.000000}\selectfont \(\displaystyle 0.0\)}%
\end{pgfscope}%
\begin{pgfscope}%
\pgfsetbuttcap%
\pgfsetroundjoin%
\definecolor{currentfill}{rgb}{0.000000,0.000000,0.000000}%
\pgfsetfillcolor{currentfill}%
\pgfsetlinewidth{0.803000pt}%
\definecolor{currentstroke}{rgb}{0.000000,0.000000,0.000000}%
\pgfsetstrokecolor{currentstroke}%
\pgfsetdash{}{0pt}%
\pgfsys@defobject{currentmarker}{\pgfqpoint{-0.048611in}{0.000000in}}{\pgfqpoint{0.000000in}{0.000000in}}{%
\pgfpathmoveto{\pgfqpoint{0.000000in}{0.000000in}}%
\pgfpathlineto{\pgfqpoint{-0.048611in}{0.000000in}}%
\pgfusepath{stroke,fill}%
}%
\begin{pgfscope}%
\pgfsys@transformshift{0.835065in}{1.757887in}%
\pgfsys@useobject{currentmarker}{}%
\end{pgfscope}%
\end{pgfscope}%
\begin{pgfscope}%
\definecolor{textcolor}{rgb}{0.000000,0.000000,0.000000}%
\pgfsetstrokecolor{textcolor}%
\pgfsetfillcolor{textcolor}%
\pgftext[x=0.560373in,y=1.709662in,left,base]{\color{textcolor}\rmfamily\fontsize{10.000000}{12.000000}\selectfont \(\displaystyle 0.5\)}%
\end{pgfscope}%
\begin{pgfscope}%
\definecolor{textcolor}{rgb}{0.000000,0.000000,0.000000}%
\pgfsetstrokecolor{textcolor}%
\pgfsetfillcolor{textcolor}%
\pgftext[x=0.396792in,y=1.335803in,,bottom,rotate=90.000000]{\color{textcolor}\rmfamily\fontsize{10.000000}{12.000000}\selectfont velocity (\(\displaystyle \frac{rad}{s}\))}%
\end{pgfscope}%
\begin{pgfscope}%
\pgfpathrectangle{\pgfqpoint{0.835065in}{0.526234in}}{\pgfqpoint{2.361798in}{1.619136in}}%
\pgfusepath{clip}%
\pgfsetrectcap%
\pgfsetroundjoin%
\pgfsetlinewidth{1.505625pt}%
\definecolor{currentstroke}{rgb}{0.000000,0.000000,1.000000}%
\pgfsetstrokecolor{currentstroke}%
\pgfsetdash{}{0pt}%
\pgfpathmoveto{\pgfqpoint{3.089177in}{1.335802in}}%
\pgfpathlineto{\pgfqpoint{3.086508in}{1.299155in}}%
\pgfpathlineto{\pgfqpoint{3.081176in}{1.262598in}}%
\pgfpathlineto{\pgfqpoint{3.073194in}{1.226221in}}%
\pgfpathlineto{\pgfqpoint{3.062582in}{1.190115in}}%
\pgfpathlineto{\pgfqpoint{3.049366in}{1.154367in}}%
\pgfpathlineto{\pgfqpoint{3.033578in}{1.119066in}}%
\pgfpathlineto{\pgfqpoint{3.015258in}{1.084300in}}%
\pgfpathlineto{\pgfqpoint{2.994451in}{1.050153in}}%
\pgfpathlineto{\pgfqpoint{2.971208in}{1.016712in}}%
\pgfpathlineto{\pgfqpoint{2.945587in}{0.984059in}}%
\pgfpathlineto{\pgfqpoint{2.917650in}{0.952275in}}%
\pgfpathlineto{\pgfqpoint{2.887467in}{0.921438in}}%
\pgfpathlineto{\pgfqpoint{2.855113in}{0.891628in}}%
\pgfpathlineto{\pgfqpoint{2.820667in}{0.862916in}}%
\pgfpathlineto{\pgfqpoint{2.784216in}{0.835377in}}%
\pgfpathlineto{\pgfqpoint{2.745848in}{0.809078in}}%
\pgfpathlineto{\pgfqpoint{2.705661in}{0.784086in}}%
\pgfpathlineto{\pgfqpoint{2.663752in}{0.760465in}}%
\pgfpathlineto{\pgfqpoint{2.620228in}{0.738273in}}%
\pgfpathlineto{\pgfqpoint{2.575194in}{0.717568in}}%
\pgfpathlineto{\pgfqpoint{2.528765in}{0.698401in}}%
\pgfpathlineto{\pgfqpoint{2.481056in}{0.680821in}}%
\pgfpathlineto{\pgfqpoint{2.432184in}{0.664874in}}%
\pgfpathlineto{\pgfqpoint{2.382273in}{0.650600in}}%
\pgfpathlineto{\pgfqpoint{2.331447in}{0.638035in}}%
\pgfpathlineto{\pgfqpoint{2.279832in}{0.627212in}}%
\pgfpathlineto{\pgfqpoint{2.227558in}{0.618159in}}%
\pgfpathlineto{\pgfqpoint{2.174755in}{0.610898in}}%
\pgfpathlineto{\pgfqpoint{2.121555in}{0.605449in}}%
\pgfpathlineto{\pgfqpoint{2.068091in}{0.601825in}}%
\pgfpathlineto{\pgfqpoint{2.014497in}{0.600036in}}%
\pgfpathlineto{\pgfqpoint{1.960906in}{0.600086in}}%
\pgfpathlineto{\pgfqpoint{1.907453in}{0.601976in}}%
\pgfpathlineto{\pgfqpoint{1.854272in}{0.605700in}}%
\pgfpathlineto{\pgfqpoint{1.801494in}{0.611249in}}%
\pgfpathlineto{\pgfqpoint{1.749253in}{0.618608in}}%
\pgfpathlineto{\pgfqpoint{1.697678in}{0.627759in}}%
\pgfpathlineto{\pgfqpoint{1.646898in}{0.638678in}}%
\pgfpathlineto{\pgfqpoint{1.597041in}{0.651338in}}%
\pgfpathlineto{\pgfqpoint{1.548230in}{0.665704in}}%
\pgfpathlineto{\pgfqpoint{1.500588in}{0.681742in}}%
\pgfpathlineto{\pgfqpoint{1.454232in}{0.699410in}}%
\pgfpathlineto{\pgfqpoint{1.409279in}{0.718662in}}%
\pgfpathlineto{\pgfqpoint{1.365840in}{0.739451in}}%
\pgfpathlineto{\pgfqpoint{1.324023in}{0.761722in}}%
\pgfpathlineto{\pgfqpoint{1.283932in}{0.785420in}}%
\pgfpathlineto{\pgfqpoint{1.245667in}{0.810485in}}%
\pgfpathlineto{\pgfqpoint{1.209323in}{0.836854in}}%
\pgfpathlineto{\pgfqpoint{1.174990in}{0.864459in}}%
\pgfpathlineto{\pgfqpoint{1.142753in}{0.893233in}}%
\pgfpathlineto{\pgfqpoint{1.112691in}{0.923102in}}%
\pgfpathlineto{\pgfqpoint{1.084879in}{0.953992in}}%
\pgfpathlineto{\pgfqpoint{1.059387in}{0.985826in}}%
\pgfpathlineto{\pgfqpoint{1.036276in}{1.018524in}}%
\pgfpathlineto{\pgfqpoint{1.015604in}{1.052006in}}%
\pgfpathlineto{\pgfqpoint{0.997421in}{1.086189in}}%
\pgfpathlineto{\pgfqpoint{0.981774in}{1.120987in}}%
\pgfpathlineto{\pgfqpoint{0.968700in}{1.156314in}}%
\pgfpathlineto{\pgfqpoint{0.958231in}{1.192084in}}%
\pgfpathlineto{\pgfqpoint{0.950394in}{1.228208in}}%
\pgfpathlineto{\pgfqpoint{0.945207in}{1.264597in}}%
\pgfpathlineto{\pgfqpoint{0.942684in}{1.301161in}}%
\pgfpathlineto{\pgfqpoint{0.942830in}{1.337810in}}%
\pgfpathlineto{\pgfqpoint{0.945645in}{1.374455in}}%
\pgfpathlineto{\pgfqpoint{0.951123in}{1.411004in}}%
\pgfpathlineto{\pgfqpoint{0.959250in}{1.447368in}}%
\pgfpathlineto{\pgfqpoint{0.970005in}{1.483458in}}%
\pgfpathlineto{\pgfqpoint{0.983363in}{1.519184in}}%
\pgfpathlineto{\pgfqpoint{0.999290in}{1.554457in}}%
\pgfpathlineto{\pgfqpoint{1.017748in}{1.589192in}}%
\pgfpathlineto{\pgfqpoint{1.038689in}{1.623302in}}%
\pgfpathlineto{\pgfqpoint{1.062064in}{1.656702in}}%
\pgfpathlineto{\pgfqpoint{1.087814in}{1.689310in}}%
\pgfpathlineto{\pgfqpoint{1.115876in}{1.721044in}}%
\pgfpathlineto{\pgfqpoint{1.146179in}{1.751826in}}%
\pgfpathlineto{\pgfqpoint{1.178650in}{1.781579in}}%
\pgfpathlineto{\pgfqpoint{1.213208in}{1.810227in}}%
\pgfpathlineto{\pgfqpoint{1.249767in}{1.837701in}}%
\pgfpathlineto{\pgfqpoint{1.288237in}{1.863930in}}%
\pgfpathlineto{\pgfqpoint{1.328521in}{1.888848in}}%
\pgfpathlineto{\pgfqpoint{1.370521in}{1.912393in}}%
\pgfpathlineto{\pgfqpoint{1.414131in}{1.934504in}}%
\pgfpathlineto{\pgfqpoint{1.459244in}{1.955127in}}%
\pgfpathlineto{\pgfqpoint{1.505746in}{1.974208in}}%
\pgfpathlineto{\pgfqpoint{1.553523in}{1.991699in}}%
\pgfpathlineto{\pgfqpoint{1.602454in}{2.007556in}}%
\pgfpathlineto{\pgfqpoint{1.652419in}{2.021737in}}%
\pgfpathlineto{\pgfqpoint{1.703291in}{2.034208in}}%
\pgfpathlineto{\pgfqpoint{1.754946in}{2.044934in}}%
\pgfpathlineto{\pgfqpoint{1.807252in}{2.053890in}}%
\pgfpathlineto{\pgfqpoint{1.860080in}{2.061052in}}%
\pgfpathlineto{\pgfqpoint{1.913298in}{2.066402in}}%
\pgfpathlineto{\pgfqpoint{1.966773in}{2.069925in}}%
\pgfpathlineto{\pgfqpoint{2.020370in}{2.071613in}}%
\pgfpathlineto{\pgfqpoint{2.073957in}{2.071462in}}%
\pgfpathlineto{\pgfqpoint{2.127398in}{2.069472in}}%
\pgfpathlineto{\pgfqpoint{2.180561in}{2.065648in}}%
\pgfpathlineto{\pgfqpoint{2.233313in}{2.059999in}}%
\pgfpathlineto{\pgfqpoint{2.285521in}{2.052541in}}%
\pgfpathlineto{\pgfqpoint{2.337055in}{2.043293in}}%
\pgfpathlineto{\pgfqpoint{2.387788in}{2.032277in}}%
\pgfpathlineto{\pgfqpoint{2.437591in}{2.019524in}}%
\pgfpathlineto{\pgfqpoint{2.486341in}{2.005064in}}%
\pgfpathlineto{\pgfqpoint{2.533916in}{1.988936in}}%
\pgfpathlineto{\pgfqpoint{2.580198in}{1.971181in}}%
\pgfpathlineto{\pgfqpoint{2.625071in}{1.951843in}}%
\pgfpathlineto{\pgfqpoint{2.668424in}{1.930972in}}%
\pgfpathlineto{\pgfqpoint{2.710149in}{1.908621in}}%
\pgfpathlineto{\pgfqpoint{2.750142in}{1.884846in}}%
\pgfpathlineto{\pgfqpoint{2.788304in}{1.859708in}}%
\pgfpathlineto{\pgfqpoint{2.824540in}{1.833270in}}%
\pgfpathlineto{\pgfqpoint{2.858761in}{1.805599in}}%
\pgfpathlineto{\pgfqpoint{2.890881in}{1.776763in}}%
\pgfpathlineto{\pgfqpoint{2.920821in}{1.746837in}}%
\pgfpathlineto{\pgfqpoint{2.948508in}{1.715893in}}%
\pgfpathlineto{\pgfqpoint{2.973872in}{1.684009in}}%
\pgfpathlineto{\pgfqpoint{2.996851in}{1.651266in}}%
\pgfpathlineto{\pgfqpoint{3.017387in}{1.617743in}}%
\pgfpathlineto{\pgfqpoint{3.035432in}{1.583525in}}%
\pgfpathlineto{\pgfqpoint{3.050939in}{1.548696in}}%
\pgfpathlineto{\pgfqpoint{3.063872in}{1.513342in}}%
\pgfpathlineto{\pgfqpoint{3.074197in}{1.477550in}}%
\pgfpathlineto{\pgfqpoint{3.081889in}{1.441410in}}%
\pgfpathlineto{\pgfqpoint{3.086930in}{1.405009in}}%
\pgfpathlineto{\pgfqpoint{3.089308in}{1.368438in}}%
\pgfpathlineto{\pgfqpoint{3.089015in}{1.331786in}}%
\pgfpathlineto{\pgfqpoint{3.086054in}{1.295144in}}%
\pgfpathlineto{\pgfqpoint{3.080430in}{1.258603in}}%
\pgfpathlineto{\pgfqpoint{3.072159in}{1.222251in}}%
\pgfpathlineto{\pgfqpoint{3.061260in}{1.186179in}}%
\pgfpathlineto{\pgfqpoint{3.047761in}{1.150476in}}%
\pgfpathlineto{\pgfqpoint{3.031694in}{1.115230in}}%
\pgfpathlineto{\pgfqpoint{3.013100in}{1.080527in}}%
\pgfpathlineto{\pgfqpoint{2.992023in}{1.046454in}}%
\pgfpathlineto{\pgfqpoint{2.968517in}{1.013095in}}%
\pgfpathlineto{\pgfqpoint{2.942638in}{0.980533in}}%
\pgfpathlineto{\pgfqpoint{2.914452in}{0.948849in}}%
\pgfpathlineto{\pgfqpoint{2.884028in}{0.918121in}}%
\pgfpathlineto{\pgfqpoint{2.851440in}{0.888427in}}%
\pgfpathlineto{\pgfqpoint{2.816771in}{0.859841in}}%
\pgfpathlineto{\pgfqpoint{2.780105in}{0.832434in}}%
\pgfpathlineto{\pgfqpoint{2.741533in}{0.806275in}}%
\pgfpathlineto{\pgfqpoint{2.701152in}{0.781431in}}%
\pgfpathlineto{\pgfqpoint{2.659061in}{0.757963in}}%
\pgfpathlineto{\pgfqpoint{2.615366in}{0.735932in}}%
\pgfpathlineto{\pgfqpoint{2.570174in}{0.715392in}}%
\pgfpathlineto{\pgfqpoint{2.523599in}{0.696397in}}%
\pgfpathlineto{\pgfqpoint{2.475756in}{0.678994in}}%
\pgfpathlineto{\pgfqpoint{2.426765in}{0.663228in}}%
\pgfpathlineto{\pgfqpoint{2.376748in}{0.649139in}}%
\pgfpathlineto{\pgfqpoint{2.325829in}{0.636764in}}%
\pgfpathlineto{\pgfqpoint{2.274135in}{0.626134in}}%
\pgfpathlineto{\pgfqpoint{2.221797in}{0.617275in}}%
\pgfpathlineto{\pgfqpoint{2.168944in}{0.610212in}}%
\pgfpathlineto{\pgfqpoint{2.115709in}{0.604963in}}%
\pgfpathlineto{\pgfqpoint{2.062224in}{0.601539in}}%
\pgfpathlineto{\pgfqpoint{2.008623in}{0.599952in}}%
\pgfpathlineto{\pgfqpoint{1.955041in}{0.600204in}}%
\pgfpathlineto{\pgfqpoint{1.901612in}{0.602295in}}%
\pgfpathlineto{\pgfqpoint{1.848468in}{0.606219in}}%
\pgfpathlineto{\pgfqpoint{1.795743in}{0.611967in}}%
\pgfpathlineto{\pgfqpoint{1.743568in}{0.619524in}}%
\pgfpathlineto{\pgfqpoint{1.692074in}{0.628870in}}%
\pgfpathlineto{\pgfqpoint{1.641390in}{0.639981in}}%
\pgfpathlineto{\pgfqpoint{1.591641in}{0.652829in}}%
\pgfpathlineto{\pgfqpoint{1.542952in}{0.667381in}}%
\pgfpathlineto{\pgfqpoint{1.495445in}{0.683599in}}%
\pgfpathlineto{\pgfqpoint{1.449237in}{0.701442in}}%
\pgfpathlineto{\pgfqpoint{1.404444in}{0.720866in}}%
\pgfpathlineto{\pgfqpoint{1.361178in}{0.741819in}}%
\pgfpathlineto{\pgfqpoint{1.319545in}{0.764250in}}%
\pgfpathlineto{\pgfqpoint{1.279650in}{0.788101in}}%
\pgfpathlineto{\pgfqpoint{1.241591in}{0.813312in}}%
\pgfpathlineto{\pgfqpoint{1.205463in}{0.839819in}}%
\pgfpathlineto{\pgfqpoint{1.171355in}{0.867556in}}%
\pgfpathlineto{\pgfqpoint{1.139352in}{0.896453in}}%
\pgfpathlineto{\pgfqpoint{1.109533in}{0.926437in}}%
\pgfpathlineto{\pgfqpoint{1.081972in}{0.957434in}}%
\pgfpathlineto{\pgfqpoint{1.056738in}{0.989367in}}%
\pgfpathlineto{\pgfqpoint{1.033891in}{1.022155in}}%
\pgfpathlineto{\pgfqpoint{1.013489in}{1.055718in}}%
\pgfpathlineto{\pgfqpoint{0.995583in}{1.089972in}}%
\pgfpathlineto{\pgfqpoint{0.980215in}{1.124832in}}%
\pgfpathlineto{\pgfqpoint{0.967425in}{1.160213in}}%
\pgfpathlineto{\pgfqpoint{0.957244in}{1.196025in}}%
\pgfpathlineto{\pgfqpoint{0.949696in}{1.232183in}}%
\pgfpathlineto{\pgfqpoint{0.944801in}{1.268595in}}%
\pgfpathlineto{\pgfqpoint{0.942569in}{1.305173in}}%
\pgfpathlineto{\pgfqpoint{0.943008in}{1.341827in}}%
\pgfpathlineto{\pgfqpoint{0.946116in}{1.378465in}}%
\pgfpathlineto{\pgfqpoint{0.951885in}{1.414999in}}%
\pgfpathlineto{\pgfqpoint{0.960300in}{1.451337in}}%
\pgfpathlineto{\pgfqpoint{0.971342in}{1.487391in}}%
\pgfpathlineto{\pgfqpoint{0.984983in}{1.523071in}}%
\pgfpathlineto{\pgfqpoint{1.001190in}{1.558290in}}%
\pgfpathlineto{\pgfqpoint{1.019921in}{1.592961in}}%
\pgfpathlineto{\pgfqpoint{1.041132in}{1.626997in}}%
\pgfpathlineto{\pgfqpoint{1.064770in}{1.660314in}}%
\pgfpathlineto{\pgfqpoint{1.090777in}{1.692830in}}%
\pgfpathlineto{\pgfqpoint{1.119087in}{1.724464in}}%
\pgfpathlineto{\pgfqpoint{1.149632in}{1.755137in}}%
\pgfpathlineto{\pgfqpoint{1.182336in}{1.784772in}}%
\pgfpathlineto{\pgfqpoint{1.217117in}{1.813296in}}%
\pgfpathlineto{\pgfqpoint{1.253890in}{1.840636in}}%
\pgfpathlineto{\pgfqpoint{1.292563in}{1.866724in}}%
\pgfpathlineto{\pgfqpoint{1.333041in}{1.891495in}}%
\pgfpathlineto{\pgfqpoint{1.375222in}{1.914886in}}%
\pgfpathlineto{\pgfqpoint{1.419002in}{1.936837in}}%
\pgfpathlineto{\pgfqpoint{1.464272in}{1.957293in}}%
\pgfpathlineto{\pgfqpoint{1.510920in}{1.976203in}}%
\pgfpathlineto{\pgfqpoint{1.558829in}{1.993517in}}%
\pgfpathlineto{\pgfqpoint{1.607879in}{2.009192in}}%
\pgfpathlineto{\pgfqpoint{1.657950in}{2.023188in}}%
\pgfpathlineto{\pgfqpoint{1.708914in}{2.035468in}}%
\pgfpathlineto{\pgfqpoint{1.760646in}{2.046002in}}%
\pgfpathlineto{\pgfqpoint{1.813016in}{2.054762in}}%
\pgfpathlineto{\pgfqpoint{1.865894in}{2.061727in}}%
\pgfpathlineto{\pgfqpoint{1.919146in}{2.066877in}}%
\pgfpathlineto{\pgfqpoint{1.972641in}{2.070200in}}%
\pgfpathlineto{\pgfqpoint{2.026243in}{2.071687in}}%
\pgfpathlineto{\pgfqpoint{2.079820in}{2.071334in}}%
\pgfpathlineto{\pgfqpoint{2.133238in}{2.069142in}}%
\pgfpathlineto{\pgfqpoint{2.186362in}{2.065118in}}%
\pgfpathlineto{\pgfqpoint{2.239061in}{2.059270in}}%
\pgfpathlineto{\pgfqpoint{2.291202in}{2.051615in}}%
\pgfpathlineto{\pgfqpoint{2.342654in}{2.042172in}}%
\pgfpathlineto{\pgfqpoint{2.393291in}{2.030964in}}%
\pgfpathlineto{\pgfqpoint{2.442985in}{2.018022in}}%
\pgfpathlineto{\pgfqpoint{2.491612in}{2.003378in}}%
\pgfpathlineto{\pgfqpoint{2.539051in}{1.987070in}}%
\pgfpathlineto{\pgfqpoint{2.585184in}{1.969139in}}%
\pgfpathlineto{\pgfqpoint{2.629896in}{1.949630in}}%
\pgfpathlineto{\pgfqpoint{2.673076in}{1.928594in}}%
\pgfpathlineto{\pgfqpoint{2.714616in}{1.906084in}}%
\pgfpathlineto{\pgfqpoint{2.754414in}{1.882157in}}%
\pgfpathlineto{\pgfqpoint{2.792369in}{1.856874in}}%
\pgfpathlineto{\pgfqpoint{2.828389in}{1.830298in}}%
\pgfpathlineto{\pgfqpoint{2.862384in}{1.802495in}}%
\pgfpathlineto{\pgfqpoint{2.894269in}{1.773537in}}%
\pgfpathlineto{\pgfqpoint{2.923966in}{1.743495in}}%
\pgfpathlineto{\pgfqpoint{2.951401in}{1.712444in}}%
\pgfpathlineto{\pgfqpoint{2.976506in}{1.680463in}}%
\pgfpathlineto{\pgfqpoint{2.999220in}{1.647630in}}%
\pgfpathlineto{\pgfqpoint{3.019487in}{1.614027in}}%
\pgfpathlineto{\pgfqpoint{3.037255in}{1.579738in}}%
\pgfpathlineto{\pgfqpoint{3.052482in}{1.544847in}}%
\pgfpathlineto{\pgfqpoint{3.065130in}{1.509441in}}%
\pgfpathlineto{\pgfqpoint{3.075168in}{1.473607in}}%
\pgfpathlineto{\pgfqpoint{3.082571in}{1.437433in}}%
\pgfpathlineto{\pgfqpoint{3.087321in}{1.401009in}}%
\pgfpathlineto{\pgfqpoint{3.089406in}{1.364425in}}%
\pgfpathlineto{\pgfqpoint{3.088821in}{1.327770in}}%
\pgfpathlineto{\pgfqpoint{3.085567in}{1.291135in}}%
\pgfpathlineto{\pgfqpoint{3.079653in}{1.254610in}}%
\pgfpathlineto{\pgfqpoint{3.071093in}{1.218284in}}%
\pgfpathlineto{\pgfqpoint{3.059908in}{1.182249in}}%
\pgfpathlineto{\pgfqpoint{3.046125in}{1.146591in}}%
\pgfpathlineto{\pgfqpoint{3.029780in}{1.111400in}}%
\pgfpathlineto{\pgfqpoint{3.010911in}{1.076762in}}%
\pgfpathlineto{\pgfqpoint{2.989565in}{1.042763in}}%
\pgfpathlineto{\pgfqpoint{2.965796in}{1.009488in}}%
\pgfpathlineto{\pgfqpoint{2.939662in}{0.977018in}}%
\pgfpathlineto{\pgfqpoint{2.911227in}{0.945435in}}%
\pgfpathlineto{\pgfqpoint{2.880562in}{0.914817in}}%
\pgfpathlineto{\pgfqpoint{2.847742in}{0.885240in}}%
\pgfpathlineto{\pgfqpoint{2.812850in}{0.856780in}}%
\pgfpathlineto{\pgfqpoint{2.775970in}{0.829506in}}%
\pgfpathlineto{\pgfqpoint{2.737196in}{0.803489in}}%
\pgfpathlineto{\pgfqpoint{2.696622in}{0.778792in}}%
\pgfpathlineto{\pgfqpoint{2.654351in}{0.755479in}}%
\pgfpathlineto{\pgfqpoint{2.610486in}{0.733608in}}%
\pgfpathlineto{\pgfqpoint{2.565138in}{0.713235in}}%
\pgfpathlineto{\pgfqpoint{2.518418in}{0.694412in}}%
\pgfpathlineto{\pgfqpoint{2.470443in}{0.677186in}}%
\pgfpathlineto{\pgfqpoint{2.421334in}{0.661602in}}%
\pgfpathlineto{\pgfqpoint{2.371211in}{0.647699in}}%
\pgfpathlineto{\pgfqpoint{2.320201in}{0.635514in}}%
\pgfpathlineto{\pgfqpoint{2.268431in}{0.625076in}}%
\pgfpathlineto{\pgfqpoint{2.216030in}{0.616414in}}%
\pgfpathlineto{\pgfqpoint{2.163128in}{0.609549in}}%
\pgfpathlineto{\pgfqpoint{2.109859in}{0.604498in}}%
\pgfpathlineto{\pgfqpoint{2.056355in}{0.601276in}}%
\pgfpathlineto{\pgfqpoint{2.002750in}{0.599890in}}%
\pgfpathlineto{\pgfqpoint{1.949179in}{0.600343in}}%
\pgfpathlineto{\pgfqpoint{1.895774in}{0.602635in}}%
\pgfpathlineto{\pgfqpoint{1.842669in}{0.606760in}}%
\pgfpathlineto{\pgfqpoint{1.789998in}{0.612707in}}%
\pgfpathlineto{\pgfqpoint{1.737891in}{0.620461in}}%
\pgfpathlineto{\pgfqpoint{1.686480in}{0.630001in}}%
\pgfpathlineto{\pgfqpoint{1.635892in}{0.641304in}}%
\pgfpathlineto{\pgfqpoint{1.586253in}{0.654340in}}%
\pgfpathlineto{\pgfqpoint{1.537688in}{0.669077in}}%
\pgfpathlineto{\pgfqpoint{1.490317in}{0.685475in}}%
\pgfpathlineto{\pgfqpoint{1.444259in}{0.703494in}}%
\pgfpathlineto{\pgfqpoint{1.399628in}{0.723087in}}%
\pgfpathlineto{\pgfqpoint{1.356535in}{0.744205in}}%
\pgfpathlineto{\pgfqpoint{1.315088in}{0.766794in}}%
\pgfpathlineto{\pgfqpoint{1.275389in}{0.790797in}}%
\pgfpathlineto{\pgfqpoint{1.237537in}{0.816153in}}%
\pgfpathlineto{\pgfqpoint{1.201626in}{0.842799in}}%
\pgfpathlineto{\pgfqpoint{1.167745in}{0.870666in}}%
\pgfpathlineto{\pgfqpoint{1.135978in}{0.899686in}}%
\pgfpathlineto{\pgfqpoint{1.106403in}{0.929785in}}%
\pgfpathlineto{\pgfqpoint{1.079093in}{0.960888in}}%
\pgfpathlineto{\pgfqpoint{1.054117in}{0.992919in}}%
\pgfpathlineto{\pgfqpoint{1.031536in}{1.025796in}}%
\pgfpathlineto{\pgfqpoint{1.011405in}{1.059438in}}%
\pgfpathlineto{\pgfqpoint{0.993775in}{1.093763in}}%
\pgfpathlineto{\pgfqpoint{0.978688in}{1.128684in}}%
\pgfpathlineto{\pgfqpoint{0.966182in}{1.164116in}}%
\pgfpathlineto{\pgfqpoint{0.956288in}{1.199971in}}%
\pgfpathlineto{\pgfqpoint{0.949030in}{1.236161in}}%
\pgfpathlineto{\pgfqpoint{0.944426in}{1.272596in}}%
\pgfpathlineto{\pgfqpoint{0.942487in}{1.309186in}}%
\pgfpathlineto{\pgfqpoint{0.943219in}{1.345843in}}%
\pgfpathlineto{\pgfqpoint{0.946618in}{1.382474in}}%
\pgfpathlineto{\pgfqpoint{0.952678in}{1.418990in}}%
\pgfpathlineto{\pgfqpoint{0.961382in}{1.455302in}}%
\pgfpathlineto{\pgfqpoint{0.972710in}{1.491319in}}%
\pgfpathlineto{\pgfqpoint{0.986634in}{1.526954in}}%
\pgfpathlineto{\pgfqpoint{1.003119in}{1.562117in}}%
\pgfpathlineto{\pgfqpoint{1.022125in}{1.596722in}}%
\pgfpathlineto{\pgfqpoint{1.043604in}{1.630683in}}%
\pgfpathlineto{\pgfqpoint{1.067505in}{1.663917in}}%
\pgfpathlineto{\pgfqpoint{1.093767in}{1.696340in}}%
\pgfpathlineto{\pgfqpoint{1.122326in}{1.727872in}}%
\pgfpathlineto{\pgfqpoint{1.153111in}{1.758435in}}%
\pgfpathlineto{\pgfqpoint{1.186046in}{1.787952in}}%
\pgfpathlineto{\pgfqpoint{1.221050in}{1.816350in}}%
\pgfpathlineto{\pgfqpoint{1.258035in}{1.843556in}}%
\pgfpathlineto{\pgfqpoint{1.296911in}{1.869503in}}%
\pgfpathlineto{\pgfqpoint{1.337580in}{1.894126in}}%
\pgfpathlineto{\pgfqpoint{1.379942in}{1.917362in}}%
\pgfpathlineto{\pgfqpoint{1.423891in}{1.939152in}}%
\pgfpathlineto{\pgfqpoint{1.469317in}{1.959441in}}%
\pgfpathlineto{\pgfqpoint{1.516109in}{1.978178in}}%
\pgfpathlineto{\pgfqpoint{1.564149in}{1.995315in}}%
\pgfpathlineto{\pgfqpoint{1.613317in}{2.010808in}}%
\pgfpathlineto{\pgfqpoint{1.663491in}{2.024617in}}%
\pgfpathlineto{\pgfqpoint{1.714546in}{2.036708in}}%
\pgfpathlineto{\pgfqpoint{1.766354in}{2.047049in}}%
\pgfpathlineto{\pgfqpoint{1.818786in}{2.055613in}}%
\pgfpathlineto{\pgfqpoint{1.871711in}{2.062380in}}%
\pgfpathlineto{\pgfqpoint{1.924997in}{2.067330in}}%
\pgfpathlineto{\pgfqpoint{1.978510in}{2.070452in}}%
\pgfpathlineto{\pgfqpoint{2.032116in}{2.071738in}}%
\pgfpathlineto{\pgfqpoint{2.085682in}{2.071183in}}%
\pgfpathlineto{\pgfqpoint{2.139074in}{2.068791in}}%
\pgfpathlineto{\pgfqpoint{2.192158in}{2.064566in}}%
\pgfpathlineto{\pgfqpoint{2.244802in}{2.058519in}}%
\pgfpathlineto{\pgfqpoint{2.296874in}{2.050667in}}%
\pgfpathlineto{\pgfqpoint{2.348244in}{2.041029in}}%
\pgfpathlineto{\pgfqpoint{2.398783in}{2.029631in}}%
\pgfpathlineto{\pgfqpoint{2.448366in}{2.016500in}}%
\pgfpathlineto{\pgfqpoint{2.496869in}{2.001672in}}%
\pgfpathlineto{\pgfqpoint{2.544171in}{1.985184in}}%
\pgfpathlineto{\pgfqpoint{2.590154in}{1.967078in}}%
\pgfpathlineto{\pgfqpoint{2.634703in}{1.947399in}}%
\pgfpathlineto{\pgfqpoint{2.677709in}{1.926199in}}%
\pgfpathlineto{\pgfqpoint{2.719063in}{1.903531in}}%
\pgfpathlineto{\pgfqpoint{2.758663in}{1.879453in}}%
\pgfpathlineto{\pgfqpoint{2.796411in}{1.854024in}}%
\pgfpathlineto{\pgfqpoint{2.832213in}{1.827311in}}%
\pgfpathlineto{\pgfqpoint{2.865981in}{1.799378in}}%
\pgfpathlineto{\pgfqpoint{2.897630in}{1.770297in}}%
\pgfpathlineto{\pgfqpoint{2.927083in}{1.740141in}}%
\pgfpathlineto{\pgfqpoint{2.954266in}{1.708985in}}%
\pgfpathlineto{\pgfqpoint{2.979112in}{1.676906in}}%
\pgfpathlineto{\pgfqpoint{3.001561in}{1.643985in}}%
\pgfpathlineto{\pgfqpoint{3.021556in}{1.610303in}}%
\pgfpathlineto{\pgfqpoint{3.039048in}{1.575943in}}%
\pgfpathlineto{\pgfqpoint{3.053994in}{1.540992in}}%
\pgfpathlineto{\pgfqpoint{3.066358in}{1.505534in}}%
\pgfpathlineto{\pgfqpoint{3.076108in}{1.469659in}}%
\pgfpathlineto{\pgfqpoint{3.083221in}{1.433453in}}%
\pgfpathlineto{\pgfqpoint{3.087679in}{1.397008in}}%
\pgfpathlineto{\pgfqpoint{3.089472in}{1.360411in}}%
\pgfpathlineto{\pgfqpoint{3.088594in}{1.323754in}}%
\pgfpathlineto{\pgfqpoint{3.085049in}{1.287127in}}%
\pgfpathlineto{\pgfqpoint{3.078844in}{1.250619in}}%
\pgfpathlineto{\pgfqpoint{3.069995in}{1.214321in}}%
\pgfpathlineto{\pgfqpoint{3.058524in}{1.178322in}}%
\pgfpathlineto{\pgfqpoint{3.044459in}{1.142712in}}%
\pgfpathlineto{\pgfqpoint{3.027835in}{1.107577in}}%
\pgfpathlineto{\pgfqpoint{3.008692in}{1.073005in}}%
\pgfpathlineto{\pgfqpoint{2.987079in}{1.039081in}}%
\pgfpathlineto{\pgfqpoint{2.963048in}{1.005890in}}%
\pgfpathlineto{\pgfqpoint{2.936658in}{0.973513in}}%
\pgfpathlineto{\pgfqpoint{2.907975in}{0.942032in}}%
\pgfpathlineto{\pgfqpoint{2.877070in}{0.911525in}}%
\pgfpathlineto{\pgfqpoint{2.844019in}{0.882067in}}%
\pgfpathlineto{\pgfqpoint{2.808905in}{0.853733in}}%
\pgfpathlineto{\pgfqpoint{2.771813in}{0.826594in}}%
\pgfpathlineto{\pgfqpoint{2.732837in}{0.800717in}}%
\pgfpathlineto{\pgfqpoint{2.692073in}{0.776169in}}%
\pgfpathlineto{\pgfqpoint{2.649621in}{0.753011in}}%
\pgfpathlineto{\pgfqpoint{2.605589in}{0.731302in}}%
\pgfpathlineto{\pgfqpoint{2.560084in}{0.711096in}}%
\pgfpathlineto{\pgfqpoint{2.513221in}{0.692446in}}%
\pgfpathlineto{\pgfqpoint{2.465117in}{0.675398in}}%
\pgfpathlineto{\pgfqpoint{2.415890in}{0.659996in}}%
\pgfpathlineto{\pgfqpoint{2.365664in}{0.646280in}}%
\pgfpathlineto{\pgfqpoint{2.314565in}{0.634285in}}%
\pgfpathlineto{\pgfqpoint{2.262719in}{0.624040in}}%
\pgfpathlineto{\pgfqpoint{2.210257in}{0.615574in}}%
\pgfpathlineto{\pgfqpoint{2.157308in}{0.608907in}}%
\pgfpathlineto{\pgfqpoint{2.104007in}{0.604056in}}%
\pgfpathlineto{\pgfqpoint{2.050486in}{0.601034in}}%
\pgfpathlineto{\pgfqpoint{1.996878in}{0.599850in}}%
\pgfpathlineto{\pgfqpoint{1.943318in}{0.600505in}}%
\pgfpathlineto{\pgfqpoint{1.889939in}{0.602998in}}%
\pgfpathlineto{\pgfqpoint{1.836876in}{0.607323in}}%
\pgfpathlineto{\pgfqpoint{1.784260in}{0.613469in}}%
\pgfpathlineto{\pgfqpoint{1.732223in}{0.621419in}}%
\pgfpathlineto{\pgfqpoint{1.680896in}{0.631154in}}%
\pgfpathlineto{\pgfqpoint{1.630405in}{0.642648in}}%
\pgfpathlineto{\pgfqpoint{1.580878in}{0.655873in}}%
\pgfpathlineto{\pgfqpoint{1.532438in}{0.670793in}}%
\pgfpathlineto{\pgfqpoint{1.485206in}{0.687371in}}%
\pgfpathlineto{\pgfqpoint{1.439298in}{0.705565in}}%
\pgfpathlineto{\pgfqpoint{1.394830in}{0.725327in}}%
\pgfpathlineto{\pgfqpoint{1.351913in}{0.746609in}}%
\pgfpathlineto{\pgfqpoint{1.310652in}{0.769356in}}%
\pgfpathlineto{\pgfqpoint{1.271151in}{0.793510in}}%
\pgfpathlineto{\pgfqpoint{1.233507in}{0.819011in}}%
\pgfpathlineto{\pgfqpoint{1.197814in}{0.845793in}}%
\pgfpathlineto{\pgfqpoint{1.164160in}{0.873790in}}%
\pgfpathlineto{\pgfqpoint{1.132629in}{0.902932in}}%
\pgfpathlineto{\pgfqpoint{1.103299in}{0.933145in}}%
\pgfpathlineto{\pgfqpoint{1.076242in}{0.964353in}}%
\pgfpathlineto{\pgfqpoint{1.051526in}{0.996480in}}%
\pgfpathlineto{\pgfqpoint{1.029210in}{1.029445in}}%
\pgfpathlineto{\pgfqpoint{1.009351in}{1.063167in}}%
\pgfpathlineto{\pgfqpoint{0.991997in}{1.097561in}}%
\pgfpathlineto{\pgfqpoint{0.977192in}{1.132542in}}%
\pgfpathlineto{\pgfqpoint{0.964970in}{1.168025in}}%
\pgfpathlineto{\pgfqpoint{0.955364in}{1.203921in}}%
\pgfpathlineto{\pgfqpoint{0.948396in}{1.240142in}}%
\pgfpathlineto{\pgfqpoint{0.944083in}{1.276598in}}%
\pgfpathlineto{\pgfqpoint{0.942437in}{1.313200in}}%
\pgfpathlineto{\pgfqpoint{0.943461in}{1.349858in}}%
\pgfpathlineto{\pgfqpoint{0.947153in}{1.386481in}}%
\pgfpathlineto{\pgfqpoint{0.953503in}{1.422980in}}%
\pgfpathlineto{\pgfqpoint{0.962496in}{1.459264in}}%
\pgfpathlineto{\pgfqpoint{0.974110in}{1.495243in}}%
\pgfpathlineto{\pgfqpoint{0.988316in}{1.530830in}}%
\pgfpathlineto{\pgfqpoint{1.005079in}{1.565936in}}%
\pgfpathlineto{\pgfqpoint{1.024358in}{1.600475in}}%
\pgfpathlineto{\pgfqpoint{1.046106in}{1.634361in}}%
\pgfpathlineto{\pgfqpoint{1.070268in}{1.667510in}}%
\pgfpathlineto{\pgfqpoint{1.096785in}{1.699839in}}%
\pgfpathlineto{\pgfqpoint{1.125591in}{1.731269in}}%
\pgfpathlineto{\pgfqpoint{1.156616in}{1.761721in}}%
\pgfpathlineto{\pgfqpoint{1.189781in}{1.791119in}}%
\pgfpathlineto{\pgfqpoint{1.225007in}{1.819389in}}%
\pgfpathlineto{\pgfqpoint{1.262204in}{1.846461in}}%
\pgfpathlineto{\pgfqpoint{1.301281in}{1.872267in}}%
\pgfpathlineto{\pgfqpoint{1.342140in}{1.896740in}}%
\pgfpathlineto{\pgfqpoint{1.384681in}{1.919820in}}%
\pgfpathlineto{\pgfqpoint{1.428797in}{1.941449in}}%
\pgfpathlineto{\pgfqpoint{1.474379in}{1.961570in}}%
\pgfpathlineto{\pgfqpoint{1.521313in}{1.980134in}}%
\pgfpathlineto{\pgfqpoint{1.569482in}{1.997093in}}%
\pgfpathlineto{\pgfqpoint{1.618767in}{2.012403in}}%
\pgfpathlineto{\pgfqpoint{1.669043in}{2.026026in}}%
\pgfpathlineto{\pgfqpoint{1.720187in}{2.037927in}}%
\pgfpathlineto{\pgfqpoint{1.772070in}{2.048074in}}%
\pgfpathlineto{\pgfqpoint{1.824562in}{2.056443in}}%
\pgfpathlineto{\pgfqpoint{1.877533in}{2.063011in}}%
\pgfpathlineto{\pgfqpoint{1.930850in}{2.067761in}}%
\pgfpathlineto{\pgfqpoint{1.984380in}{2.070683in}}%
\pgfpathlineto{\pgfqpoint{2.037989in}{2.071767in}}%
\pgfpathlineto{\pgfqpoint{2.091542in}{2.071011in}}%
\pgfpathlineto{\pgfqpoint{2.144907in}{2.068417in}}%
\pgfpathlineto{\pgfqpoint{2.197949in}{2.063992in}}%
\pgfpathlineto{\pgfqpoint{2.250537in}{2.057747in}}%
\pgfpathlineto{\pgfqpoint{2.302538in}{2.049698in}}%
\pgfpathlineto{\pgfqpoint{2.353823in}{2.039866in}}%
\pgfpathlineto{\pgfqpoint{2.404264in}{2.028276in}}%
\pgfpathlineto{\pgfqpoint{2.453734in}{2.014958in}}%
\pgfpathlineto{\pgfqpoint{2.502112in}{1.999946in}}%
\pgfpathlineto{\pgfqpoint{2.549275in}{1.983278in}}%
\pgfpathlineto{\pgfqpoint{2.595106in}{1.964998in}}%
\pgfpathlineto{\pgfqpoint{2.639492in}{1.945150in}}%
\pgfpathlineto{\pgfqpoint{2.682321in}{1.923787in}}%
\pgfpathlineto{\pgfqpoint{2.723488in}{1.900961in}}%
\pgfpathlineto{\pgfqpoint{2.762890in}{1.876732in}}%
\pgfpathlineto{\pgfqpoint{2.800430in}{1.851160in}}%
\pgfpathlineto{\pgfqpoint{2.836013in}{1.824309in}}%
\pgfpathlineto{\pgfqpoint{2.869553in}{1.796247in}}%
\pgfpathlineto{\pgfqpoint{2.900965in}{1.767045in}}%
\pgfpathlineto{\pgfqpoint{2.930173in}{1.736776in}}%
\pgfpathlineto{\pgfqpoint{2.957103in}{1.705514in}}%
\pgfpathlineto{\pgfqpoint{2.981690in}{1.673339in}}%
\pgfpathlineto{\pgfqpoint{3.003872in}{1.640331in}}%
\pgfpathlineto{\pgfqpoint{3.023595in}{1.606570in}}%
\pgfpathlineto{\pgfqpoint{3.040810in}{1.572142in}}%
\pgfpathlineto{\pgfqpoint{3.055475in}{1.537131in}}%
\pgfpathlineto{\pgfqpoint{3.067554in}{1.501623in}}%
\pgfpathlineto{\pgfqpoint{3.077016in}{1.465707in}}%
\pgfpathlineto{\pgfqpoint{3.083839in}{1.429471in}}%
\pgfpathlineto{\pgfqpoint{3.088006in}{1.393004in}}%
\pgfpathlineto{\pgfqpoint{3.089506in}{1.356397in}}%
\pgfpathlineto{\pgfqpoint{3.088336in}{1.319738in}}%
\pgfpathlineto{\pgfqpoint{3.084499in}{1.283120in}}%
\pgfpathlineto{\pgfqpoint{3.078003in}{1.246631in}}%
\pgfpathlineto{\pgfqpoint{3.068866in}{1.210361in}}%
\pgfpathlineto{\pgfqpoint{3.057109in}{1.174401in}}%
\pgfpathlineto{\pgfqpoint{3.042762in}{1.138838in}}%
\pgfpathlineto{\pgfqpoint{3.025860in}{1.103761in}}%
\pgfpathlineto{\pgfqpoint{3.006444in}{1.069256in}}%
\pgfpathlineto{\pgfqpoint{2.984563in}{1.035408in}}%
\pgfpathlineto{\pgfqpoint{2.960271in}{1.002302in}}%
\pgfpathlineto{\pgfqpoint{2.933626in}{0.970020in}}%
\pgfpathlineto{\pgfqpoint{2.904697in}{0.938641in}}%
\pgfpathlineto{\pgfqpoint{2.873553in}{0.908245in}}%
\pgfpathlineto{\pgfqpoint{2.840272in}{0.878907in}}%
\pgfpathlineto{\pgfqpoint{2.804936in}{0.850700in}}%
\pgfpathlineto{\pgfqpoint{2.767634in}{0.823696in}}%
\pgfpathlineto{\pgfqpoint{2.728457in}{0.797962in}}%
\pgfpathlineto{\pgfqpoint{2.687503in}{0.773564in}}%
\pgfpathlineto{\pgfqpoint{2.644873in}{0.750561in}}%
\pgfpathlineto{\pgfqpoint{2.600674in}{0.729014in}}%
\pgfpathlineto{\pgfqpoint{2.555015in}{0.708976in}}%
\pgfpathlineto{\pgfqpoint{2.508010in}{0.690499in}}%
\pgfpathlineto{\pgfqpoint{2.459776in}{0.673630in}}%
\pgfpathlineto{\pgfqpoint{2.410434in}{0.658411in}}%
\pgfpathlineto{\pgfqpoint{2.360107in}{0.644882in}}%
\pgfpathlineto{\pgfqpoint{2.308919in}{0.633076in}}%
\pgfpathlineto{\pgfqpoint{2.257000in}{0.623026in}}%
\pgfpathlineto{\pgfqpoint{2.204478in}{0.614755in}}%
\pgfpathlineto{\pgfqpoint{2.151484in}{0.608286in}}%
\pgfpathlineto{\pgfqpoint{2.098152in}{0.603636in}}%
\pgfpathlineto{\pgfqpoint{2.044615in}{0.600815in}}%
\pgfpathlineto{\pgfqpoint{1.991006in}{0.599832in}}%
\pgfpathlineto{\pgfqpoint{1.937459in}{0.600688in}}%
\pgfpathlineto{\pgfqpoint{1.884108in}{0.603383in}}%
\pgfpathlineto{\pgfqpoint{1.831088in}{0.607908in}}%
\pgfpathlineto{\pgfqpoint{1.778529in}{0.614252in}}%
\pgfpathlineto{\pgfqpoint{1.726564in}{0.622399in}}%
\pgfpathlineto{\pgfqpoint{1.675321in}{0.632328in}}%
\pgfpathlineto{\pgfqpoint{1.624931in}{0.644013in}}%
\pgfpathlineto{\pgfqpoint{1.575517in}{0.657425in}}%
\pgfpathlineto{\pgfqpoint{1.527203in}{0.672529in}}%
\pgfpathlineto{\pgfqpoint{1.480110in}{0.689286in}}%
\pgfpathlineto{\pgfqpoint{1.434355in}{0.707654in}}%
\pgfpathlineto{\pgfqpoint{1.390051in}{0.727586in}}%
\pgfpathlineto{\pgfqpoint{1.347310in}{0.749031in}}%
\pgfpathlineto{\pgfqpoint{1.306237in}{0.771934in}}%
\pgfpathlineto{\pgfqpoint{1.266935in}{0.796239in}}%
\pgfpathlineto{\pgfqpoint{1.229500in}{0.821883in}}%
\pgfpathlineto{\pgfqpoint{1.194026in}{0.848802in}}%
\pgfpathlineto{\pgfqpoint{1.160601in}{0.876928in}}%
\pgfpathlineto{\pgfqpoint{1.129308in}{0.906190in}}%
\pgfpathlineto{\pgfqpoint{1.100223in}{0.936516in}}%
\pgfpathlineto{\pgfqpoint{1.073419in}{0.967829in}}%
\pgfpathlineto{\pgfqpoint{1.048963in}{1.000052in}}%
\pgfpathlineto{\pgfqpoint{1.026914in}{1.033104in}}%
\pgfpathlineto{\pgfqpoint{1.007327in}{1.066903in}}%
\pgfpathlineto{\pgfqpoint{0.990250in}{1.101366in}}%
\pgfpathlineto{\pgfqpoint{0.975726in}{1.136407in}}%
\pgfpathlineto{\pgfqpoint{0.963790in}{1.171939in}}%
\pgfpathlineto{\pgfqpoint{0.954472in}{1.207875in}}%
\pgfpathlineto{\pgfqpoint{0.947794in}{1.244126in}}%
\pgfpathlineto{\pgfqpoint{0.943773in}{1.280602in}}%
\pgfpathlineto{\pgfqpoint{0.942419in}{1.317215in}}%
\pgfpathlineto{\pgfqpoint{0.943735in}{1.353874in}}%
\pgfpathlineto{\pgfqpoint{0.947719in}{1.390487in}}%
\pgfpathlineto{\pgfqpoint{0.954359in}{1.426967in}}%
\pgfpathlineto{\pgfqpoint{0.963641in}{1.463222in}}%
\pgfpathlineto{\pgfqpoint{0.975540in}{1.499163in}}%
\pgfpathlineto{\pgfqpoint{0.990028in}{1.534701in}}%
\pgfpathlineto{\pgfqpoint{1.007069in}{1.569749in}}%
\pgfpathlineto{\pgfqpoint{1.026621in}{1.604220in}}%
\pgfpathlineto{\pgfqpoint{1.048636in}{1.638030in}}%
\pgfpathlineto{\pgfqpoint{1.073059in}{1.671093in}}%
\pgfpathlineto{\pgfqpoint{1.099830in}{1.703328in}}%
\pgfpathlineto{\pgfqpoint{1.128883in}{1.734654in}}%
\pgfpathlineto{\pgfqpoint{1.160146in}{1.764994in}}%
\pgfpathlineto{\pgfqpoint{1.193541in}{1.794272in}}%
\pgfpathlineto{\pgfqpoint{1.228987in}{1.822415in}}%
\pgfpathlineto{\pgfqpoint{1.266395in}{1.849351in}}%
\pgfpathlineto{\pgfqpoint{1.305671in}{1.875014in}}%
\pgfpathlineto{\pgfqpoint{1.346720in}{1.899338in}}%
\pgfpathlineto{\pgfqpoint{1.389439in}{1.922261in}}%
\pgfpathlineto{\pgfqpoint{1.433721in}{1.943727in}}%
\pgfpathlineto{\pgfqpoint{1.479456in}{1.963681in}}%
\pgfpathlineto{\pgfqpoint{1.526531in}{1.982071in}}%
\pgfpathlineto{\pgfqpoint{1.574829in}{1.998851in}}%
\pgfpathlineto{\pgfqpoint{1.624228in}{2.013979in}}%
\pgfpathlineto{\pgfqpoint{1.674606in}{2.027414in}}%
\pgfpathlineto{\pgfqpoint{1.725837in}{2.039124in}}%
\pgfpathlineto{\pgfqpoint{1.777793in}{2.049078in}}%
\pgfpathlineto{\pgfqpoint{1.830344in}{2.057250in}}%
\pgfpathlineto{\pgfqpoint{1.883360in}{2.063620in}}%
\pgfpathlineto{\pgfqpoint{1.936706in}{2.068171in}}%
\pgfpathlineto{\pgfqpoint{1.990251in}{2.070891in}}%
\pgfpathlineto{\pgfqpoint{2.043860in}{2.071774in}}%
\pgfpathlineto{\pgfqpoint{2.097400in}{2.070816in}}%
\pgfpathlineto{\pgfqpoint{2.150736in}{2.068021in}}%
\pgfpathlineto{\pgfqpoint{2.203735in}{2.063396in}}%
\pgfpathlineto{\pgfqpoint{2.256264in}{2.056953in}}%
\pgfpathlineto{\pgfqpoint{2.308193in}{2.048708in}}%
\pgfpathlineto{\pgfqpoint{2.359392in}{2.038682in}}%
\pgfpathlineto{\pgfqpoint{2.409733in}{2.026901in}}%
\pgfpathlineto{\pgfqpoint{2.459089in}{2.013396in}}%
\pgfpathlineto{\pgfqpoint{2.507339in}{1.998200in}}%
\pgfpathlineto{\pgfqpoint{2.554362in}{1.981354in}}%
\pgfpathlineto{\pgfqpoint{2.600041in}{1.962899in}}%
\pgfpathlineto{\pgfqpoint{2.644262in}{1.942883in}}%
\pgfpathlineto{\pgfqpoint{2.686914in}{1.921357in}}%
\pgfpathlineto{\pgfqpoint{2.727893in}{1.898374in}}%
\pgfpathlineto{\pgfqpoint{2.767095in}{1.873995in}}%
\pgfpathlineto{\pgfqpoint{2.804425in}{1.848279in}}%
\pgfpathlineto{\pgfqpoint{2.839789in}{1.821293in}}%
\pgfpathlineto{\pgfqpoint{2.873099in}{1.793102in}}%
\pgfpathlineto{\pgfqpoint{2.904274in}{1.763780in}}%
\pgfpathlineto{\pgfqpoint{2.933235in}{1.733398in}}%
\pgfpathlineto{\pgfqpoint{2.959912in}{1.702033in}}%
\pgfpathlineto{\pgfqpoint{2.984238in}{1.669763in}}%
\pgfpathlineto{\pgfqpoint{3.006153in}{1.636668in}}%
\pgfpathlineto{\pgfqpoint{3.025604in}{1.602830in}}%
\pgfpathlineto{\pgfqpoint{3.042542in}{1.568333in}}%
\pgfpathlineto{\pgfqpoint{3.056925in}{1.533264in}}%
\pgfpathlineto{\pgfqpoint{3.068719in}{1.497707in}}%
\pgfpathlineto{\pgfqpoint{3.077893in}{1.461752in}}%
\pgfpathlineto{\pgfqpoint{3.084426in}{1.425486in}}%
\pgfpathlineto{\pgfqpoint{3.088301in}{1.388999in}}%
\pgfpathlineto{\pgfqpoint{3.089508in}{1.352382in}}%
\pgfpathlineto{\pgfqpoint{3.088046in}{1.315723in}}%
\pgfpathlineto{\pgfqpoint{3.083916in}{1.279115in}}%
\pgfpathlineto{\pgfqpoint{3.077131in}{1.242645in}}%
\pgfpathlineto{\pgfqpoint{3.067705in}{1.206405in}}%
\pgfpathlineto{\pgfqpoint{3.055663in}{1.170484in}}%
\pgfpathlineto{\pgfqpoint{3.041034in}{1.134970in}}%
\pgfpathlineto{\pgfqpoint{3.023855in}{1.099951in}}%
\pgfpathlineto{\pgfqpoint{3.004167in}{1.065514in}}%
\pgfpathlineto{\pgfqpoint{2.982019in}{1.031744in}}%
\pgfpathlineto{\pgfqpoint{2.957465in}{0.998724in}}%
\pgfpathlineto{\pgfqpoint{2.930567in}{0.966537in}}%
\pgfpathlineto{\pgfqpoint{2.901392in}{0.935262in}}%
\pgfpathlineto{\pgfqpoint{2.870010in}{0.904978in}}%
\pgfpathlineto{\pgfqpoint{2.836500in}{0.875761in}}%
\pgfpathlineto{\pgfqpoint{2.800944in}{0.847682in}}%
\pgfpathlineto{\pgfqpoint{2.763432in}{0.820814in}}%
\pgfpathlineto{\pgfqpoint{2.724055in}{0.795223in}}%
\pgfpathlineto{\pgfqpoint{2.682913in}{0.770974in}}%
\pgfpathlineto{\pgfqpoint{2.640106in}{0.748129in}}%
\pgfpathlineto{\pgfqpoint{2.595741in}{0.726744in}}%
\pgfpathlineto{\pgfqpoint{2.549929in}{0.706875in}}%
\pgfpathlineto{\pgfqpoint{2.502784in}{0.688572in}}%
\pgfpathlineto{\pgfqpoint{2.454423in}{0.671881in}}%
\pgfpathlineto{\pgfqpoint{2.404967in}{0.656846in}}%
\pgfpathlineto{\pgfqpoint{2.354539in}{0.643504in}}%
\pgfpathlineto{\pgfqpoint{2.303265in}{0.631889in}}%
\pgfpathlineto{\pgfqpoint{2.251273in}{0.622032in}}%
\pgfpathlineto{\pgfqpoint{2.198693in}{0.613958in}}%
\pgfpathlineto{\pgfqpoint{2.145656in}{0.607688in}}%
\pgfpathlineto{\pgfqpoint{2.092295in}{0.603237in}}%
\pgfpathlineto{\pgfqpoint{2.038743in}{0.600617in}}%
\pgfpathlineto{\pgfqpoint{1.985134in}{0.599836in}}%
\pgfpathlineto{\pgfqpoint{1.931602in}{0.600894in}}%
\pgfpathlineto{\pgfqpoint{1.878282in}{0.603789in}}%
\pgfpathlineto{\pgfqpoint{1.825305in}{0.608514in}}%
\pgfpathlineto{\pgfqpoint{1.772805in}{0.615057in}}%
\pgfpathlineto{\pgfqpoint{1.720913in}{0.623400in}}%
\pgfpathlineto{\pgfqpoint{1.669758in}{0.633523in}}%
\pgfpathlineto{\pgfqpoint{1.619468in}{0.645399in}}%
\pgfpathlineto{\pgfqpoint{1.570168in}{0.658998in}}%
\pgfpathlineto{\pgfqpoint{1.521982in}{0.674284in}}%
\pgfpathlineto{\pgfqpoint{1.475030in}{0.691220in}}%
\pgfpathlineto{\pgfqpoint{1.429428in}{0.709762in}}%
\pgfpathlineto{\pgfqpoint{1.385291in}{0.729862in}}%
\pgfpathlineto{\pgfqpoint{1.342727in}{0.751470in}}%
\pgfpathlineto{\pgfqpoint{1.301843in}{0.774530in}}%
\pgfpathlineto{\pgfqpoint{1.262741in}{0.798984in}}%
\pgfpathlineto{\pgfqpoint{1.225517in}{0.824771in}}%
\pgfpathlineto{\pgfqpoint{1.190263in}{0.851825in}}%
\pgfpathlineto{\pgfqpoint{1.157068in}{0.880079in}}%
\pgfpathlineto{\pgfqpoint{1.126012in}{0.909462in}}%
\pgfpathlineto{\pgfqpoint{1.097174in}{0.939900in}}%
\pgfpathlineto{\pgfqpoint{1.070625in}{0.971316in}}%
\pgfpathlineto{\pgfqpoint{1.046429in}{1.003634in}}%
\pgfpathlineto{\pgfqpoint{1.024647in}{1.036772in}}%
\pgfpathlineto{\pgfqpoint{1.005333in}{1.070648in}}%
\pgfpathlineto{\pgfqpoint{0.988534in}{1.105178in}}%
\pgfpathlineto{\pgfqpoint{0.974292in}{1.140277in}}%
\pgfpathlineto{\pgfqpoint{0.962641in}{1.175857in}}%
\pgfpathlineto{\pgfqpoint{0.953611in}{1.211832in}}%
\pgfpathlineto{\pgfqpoint{0.947223in}{1.248112in}}%
\pgfpathlineto{\pgfqpoint{0.943494in}{1.284608in}}%
\pgfpathlineto{\pgfqpoint{0.942433in}{1.321230in}}%
\pgfpathlineto{\pgfqpoint{0.944042in}{1.357888in}}%
\pgfpathlineto{\pgfqpoint{0.948317in}{1.394492in}}%
\pgfpathlineto{\pgfqpoint{0.955248in}{1.430951in}}%
\pgfpathlineto{\pgfqpoint{0.964817in}{1.467176in}}%
\pgfpathlineto{\pgfqpoint{0.977002in}{1.503077in}}%
\pgfpathlineto{\pgfqpoint{0.991771in}{1.538566in}}%
\pgfpathlineto{\pgfqpoint{1.009089in}{1.573555in}}%
\pgfpathlineto{\pgfqpoint{1.028914in}{1.607958in}}%
\pgfpathlineto{\pgfqpoint{1.051195in}{1.641689in}}%
\pgfpathlineto{\pgfqpoint{1.075878in}{1.674666in}}%
\pgfpathlineto{\pgfqpoint{1.102902in}{1.706805in}}%
\pgfpathlineto{\pgfqpoint{1.132201in}{1.738027in}}%
\pgfpathlineto{\pgfqpoint{1.163702in}{1.768255in}}%
\pgfpathlineto{\pgfqpoint{1.197326in}{1.797412in}}%
\pgfpathlineto{\pgfqpoint{1.232991in}{1.825426in}}%
\pgfpathlineto{\pgfqpoint{1.270608in}{1.852226in}}%
\pgfpathlineto{\pgfqpoint{1.310084in}{1.877745in}}%
\pgfpathlineto{\pgfqpoint{1.351320in}{1.901918in}}%
\pgfpathlineto{\pgfqpoint{1.394215in}{1.924685in}}%
\pgfpathlineto{\pgfqpoint{1.438662in}{1.945988in}}%
\pgfpathlineto{\pgfqpoint{1.484550in}{1.965773in}}%
\pgfpathlineto{\pgfqpoint{1.531765in}{1.983989in}}%
\pgfpathlineto{\pgfqpoint{1.580189in}{2.000590in}}%
\pgfpathlineto{\pgfqpoint{1.629701in}{2.015534in}}%
\pgfpathlineto{\pgfqpoint{1.680179in}{2.028782in}}%
\pgfpathlineto{\pgfqpoint{1.731496in}{2.040301in}}%
\pgfpathlineto{\pgfqpoint{1.783523in}{2.050061in}}%
\pgfpathlineto{\pgfqpoint{1.836132in}{2.058036in}}%
\pgfpathlineto{\pgfqpoint{1.889190in}{2.064208in}}%
\pgfpathlineto{\pgfqpoint{1.942565in}{2.068558in}}%
\pgfpathlineto{\pgfqpoint{1.996123in}{2.071077in}}%
\pgfpathlineto{\pgfqpoint{2.049731in}{2.071758in}}%
\pgfpathlineto{\pgfqpoint{2.103255in}{2.070600in}}%
\pgfpathlineto{\pgfqpoint{2.156560in}{2.067604in}}%
\pgfpathlineto{\pgfqpoint{2.209514in}{2.062779in}}%
\pgfpathlineto{\pgfqpoint{2.261985in}{2.056137in}}%
\pgfpathlineto{\pgfqpoint{2.313840in}{2.047696in}}%
\pgfpathlineto{\pgfqpoint{2.364951in}{2.037476in}}%
\pgfpathlineto{\pgfqpoint{2.415190in}{2.025505in}}%
\pgfpathlineto{\pgfqpoint{2.464431in}{2.011813in}}%
\pgfpathlineto{\pgfqpoint{2.512553in}{1.996435in}}%
\pgfpathlineto{\pgfqpoint{2.559434in}{1.979410in}}%
\pgfpathlineto{\pgfqpoint{2.604958in}{1.960781in}}%
\pgfpathlineto{\pgfqpoint{2.649012in}{1.940597in}}%
\pgfpathlineto{\pgfqpoint{2.691487in}{1.918909in}}%
\pgfpathlineto{\pgfqpoint{2.732276in}{1.895771in}}%
\pgfpathlineto{\pgfqpoint{2.771278in}{1.871242in}}%
\pgfpathlineto{\pgfqpoint{2.808396in}{1.845384in}}%
\pgfpathlineto{\pgfqpoint{2.843539in}{1.818262in}}%
\pgfpathlineto{\pgfqpoint{2.876620in}{1.789944in}}%
\pgfpathlineto{\pgfqpoint{2.907555in}{1.760502in}}%
\pgfpathlineto{\pgfqpoint{2.936270in}{1.730009in}}%
\pgfpathlineto{\pgfqpoint{2.962692in}{1.698540in}}%
\pgfpathlineto{\pgfqpoint{2.986757in}{1.666176in}}%
\pgfpathlineto{\pgfqpoint{3.008405in}{1.632995in}}%
\pgfpathlineto{\pgfqpoint{3.027583in}{1.599081in}}%
\pgfpathlineto{\pgfqpoint{3.044243in}{1.564518in}}%
\pgfpathlineto{\pgfqpoint{3.058344in}{1.529391in}}%
\pgfpathlineto{\pgfqpoint{3.069852in}{1.493786in}}%
\pgfpathlineto{\pgfqpoint{3.078738in}{1.457792in}}%
\pgfpathlineto{\pgfqpoint{3.084980in}{1.421498in}}%
\pgfpathlineto{\pgfqpoint{3.088563in}{1.384993in}}%
\pgfpathlineto{\pgfqpoint{3.089478in}{1.348366in}}%
\pgfpathlineto{\pgfqpoint{3.087723in}{1.311709in}}%
\pgfpathlineto{\pgfqpoint{3.083302in}{1.275111in}}%
\pgfpathlineto{\pgfqpoint{3.076227in}{1.238662in}}%
\pgfpathlineto{\pgfqpoint{3.066513in}{1.202453in}}%
\pgfpathlineto{\pgfqpoint{3.054186in}{1.166572in}}%
\pgfpathlineto{\pgfqpoint{3.039276in}{1.131108in}}%
\pgfpathlineto{\pgfqpoint{3.021819in}{1.096149in}}%
\pgfpathlineto{\pgfqpoint{3.001859in}{1.061781in}}%
\pgfpathlineto{\pgfqpoint{2.979445in}{1.028088in}}%
\pgfpathlineto{\pgfqpoint{2.954632in}{0.995156in}}%
\pgfpathlineto{\pgfqpoint{2.927481in}{0.963065in}}%
\pgfpathlineto{\pgfqpoint{2.898060in}{0.931895in}}%
\pgfpathlineto{\pgfqpoint{2.866441in}{0.901724in}}%
\pgfpathlineto{\pgfqpoint{2.832703in}{0.872628in}}%
\pgfpathlineto{\pgfqpoint{2.796929in}{0.844679in}}%
\pgfpathlineto{\pgfqpoint{2.759207in}{0.817947in}}%
\pgfpathlineto{\pgfqpoint{2.719633in}{0.792500in}}%
\pgfpathlineto{\pgfqpoint{2.678302in}{0.768402in}}%
\pgfpathlineto{\pgfqpoint{2.635320in}{0.745714in}}%
\pgfpathlineto{\pgfqpoint{2.590791in}{0.724493in}}%
\pgfpathlineto{\pgfqpoint{2.544827in}{0.704793in}}%
\pgfpathlineto{\pgfqpoint{2.497543in}{0.686664in}}%
\pgfpathlineto{\pgfqpoint{2.449056in}{0.670153in}}%
\pgfpathlineto{\pgfqpoint{2.399488in}{0.655301in}}%
\pgfpathlineto{\pgfqpoint{2.348961in}{0.642147in}}%
\pgfpathlineto{\pgfqpoint{2.297602in}{0.630723in}}%
\pgfpathlineto{\pgfqpoint{2.245539in}{0.621061in}}%
\pgfpathlineto{\pgfqpoint{2.192903in}{0.613183in}}%
\pgfpathlineto{\pgfqpoint{2.139824in}{0.607111in}}%
\pgfpathlineto{\pgfqpoint{2.086435in}{0.602861in}}%
\pgfpathlineto{\pgfqpoint{2.032871in}{0.600442in}}%
\pgfpathlineto{\pgfqpoint{1.979264in}{0.599862in}}%
\pgfpathlineto{\pgfqpoint{1.925749in}{0.601121in}}%
\pgfpathlineto{\pgfqpoint{1.872459in}{0.604218in}}%
\pgfpathlineto{\pgfqpoint{1.819528in}{0.609143in}}%
\pgfpathlineto{\pgfqpoint{1.767088in}{0.615883in}}%
\pgfpathlineto{\pgfqpoint{1.715270in}{0.624423in}}%
\pgfpathlineto{\pgfqpoint{1.664204in}{0.634739in}}%
\pgfpathlineto{\pgfqpoint{1.614016in}{0.646805in}}%
\pgfpathlineto{\pgfqpoint{1.564833in}{0.660591in}}%
\pgfpathlineto{\pgfqpoint{1.516777in}{0.676060in}}%
\pgfpathlineto{\pgfqpoint{1.469967in}{0.693174in}}%
\pgfpathlineto{\pgfqpoint{1.424520in}{0.711889in}}%
\pgfpathlineto{\pgfqpoint{1.380549in}{0.732156in}}%
\pgfpathlineto{\pgfqpoint{1.338165in}{0.753926in}}%
\pgfpathlineto{\pgfqpoint{1.297471in}{0.777142in}}%
\pgfpathlineto{\pgfqpoint{1.258570in}{0.801745in}}%
\pgfpathlineto{\pgfqpoint{1.221557in}{0.827674in}}%
\pgfpathlineto{\pgfqpoint{1.186525in}{0.854863in}}%
\pgfpathlineto{\pgfqpoint{1.153560in}{0.883244in}}%
\pgfpathlineto{\pgfqpoint{1.122744in}{0.912746in}}%
\pgfpathlineto{\pgfqpoint{1.094153in}{0.943295in}}%
\pgfpathlineto{\pgfqpoint{1.067858in}{0.974814in}}%
\pgfpathlineto{\pgfqpoint{1.043924in}{1.007226in}}%
\pgfpathlineto{\pgfqpoint{1.022410in}{1.040448in}}%
\pgfpathlineto{\pgfqpoint{1.003369in}{1.074400in}}%
\pgfpathlineto{\pgfqpoint{0.986848in}{1.108997in}}%
\pgfpathlineto{\pgfqpoint{0.972888in}{1.144153in}}%
\pgfpathlineto{\pgfqpoint{0.961524in}{1.179781in}}%
\pgfpathlineto{\pgfqpoint{0.952782in}{1.215793in}}%
\pgfpathlineto{\pgfqpoint{0.946685in}{1.252101in}}%
\pgfpathlineto{\pgfqpoint{0.943248in}{1.288616in}}%
\pgfpathlineto{\pgfqpoint{0.942479in}{1.325246in}}%
\pgfpathlineto{\pgfqpoint{0.944380in}{1.361902in}}%
\pgfpathlineto{\pgfqpoint{0.948947in}{1.398495in}}%
\pgfpathlineto{\pgfqpoint{0.956168in}{1.434932in}}%
\pgfpathlineto{\pgfqpoint{0.966025in}{1.471126in}}%
\pgfpathlineto{\pgfqpoint{0.978494in}{1.506986in}}%
\pgfpathlineto{\pgfqpoint{0.993545in}{1.542425in}}%
\pgfpathlineto{\pgfqpoint{1.011140in}{1.577354in}}%
\pgfpathlineto{\pgfqpoint{1.031236in}{1.611687in}}%
\pgfpathlineto{\pgfqpoint{1.053783in}{1.645340in}}%
\pgfpathlineto{\pgfqpoint{1.078726in}{1.678229in}}%
\pgfpathlineto{\pgfqpoint{1.106002in}{1.710272in}}%
\pgfpathlineto{\pgfqpoint{1.135546in}{1.741389in}}%
\pgfpathlineto{\pgfqpoint{1.167283in}{1.771503in}}%
\pgfpathlineto{\pgfqpoint{1.201135in}{1.800538in}}%
\pgfpathlineto{\pgfqpoint{1.237018in}{1.828422in}}%
\pgfpathlineto{\pgfqpoint{1.274843in}{1.855085in}}%
\pgfpathlineto{\pgfqpoint{1.314517in}{1.880460in}}%
\pgfpathlineto{\pgfqpoint{1.355940in}{1.904482in}}%
\pgfpathlineto{\pgfqpoint{1.399011in}{1.927091in}}%
\pgfpathlineto{\pgfqpoint{1.443621in}{1.948230in}}%
\pgfpathlineto{\pgfqpoint{1.489660in}{1.967845in}}%
\pgfpathlineto{\pgfqpoint{1.537013in}{1.985887in}}%
\pgfpathlineto{\pgfqpoint{1.585562in}{2.002308in}}%
\pgfpathlineto{\pgfqpoint{1.635186in}{2.017068in}}%
\pgfpathlineto{\pgfqpoint{1.685762in}{2.030129in}}%
\pgfpathlineto{\pgfqpoint{1.737163in}{2.041456in}}%
\pgfpathlineto{\pgfqpoint{1.789260in}{2.051022in}}%
\pgfpathlineto{\pgfqpoint{1.841925in}{2.058801in}}%
\pgfpathlineto{\pgfqpoint{1.895024in}{2.064773in}}%
\pgfpathlineto{\pgfqpoint{1.948425in}{2.068924in}}%
\pgfpathlineto{\pgfqpoint{2.001996in}{2.071242in}}%
\pgfpathlineto{\pgfqpoint{2.055601in}{2.071721in}}%
\pgfpathlineto{\pgfqpoint{2.109108in}{2.070361in}}%
\pgfpathlineto{\pgfqpoint{2.162381in}{2.067164in}}%
\pgfpathlineto{\pgfqpoint{2.215288in}{2.062140in}}%
\pgfpathlineto{\pgfqpoint{2.267698in}{2.055300in}}%
\pgfpathlineto{\pgfqpoint{2.319478in}{2.046662in}}%
\pgfpathlineto{\pgfqpoint{2.370499in}{2.036250in}}%
\pgfpathlineto{\pgfqpoint{2.420635in}{2.024088in}}%
\pgfpathlineto{\pgfqpoint{2.469760in}{2.010210in}}%
\pgfpathlineto{\pgfqpoint{2.517751in}{1.994649in}}%
\pgfpathlineto{\pgfqpoint{2.564489in}{1.977446in}}%
\pgfpathlineto{\pgfqpoint{2.609858in}{1.958645in}}%
\pgfpathlineto{\pgfqpoint{2.653744in}{1.938294in}}%
\pgfpathlineto{\pgfqpoint{2.696039in}{1.916444in}}%
\pgfpathlineto{\pgfqpoint{2.736637in}{1.893150in}}%
\pgfpathlineto{\pgfqpoint{2.775438in}{1.868473in}}%
\pgfpathlineto{\pgfqpoint{2.812344in}{1.842473in}}%
\pgfpathlineto{\pgfqpoint{2.847265in}{1.815217in}}%
\pgfpathlineto{\pgfqpoint{2.880115in}{1.786772in}}%
\pgfpathlineto{\pgfqpoint{2.910811in}{1.757211in}}%
\pgfpathlineto{\pgfqpoint{2.939278in}{1.726607in}}%
\pgfpathlineto{\pgfqpoint{2.965445in}{1.695037in}}%
\pgfpathlineto{\pgfqpoint{2.989248in}{1.662579in}}%
\pgfpathlineto{\pgfqpoint{3.010627in}{1.629315in}}%
\pgfpathlineto{\pgfqpoint{3.029531in}{1.595325in}}%
\pgfpathlineto{\pgfqpoint{3.045913in}{1.560696in}}%
\pgfpathlineto{\pgfqpoint{3.059732in}{1.525512in}}%
\pgfpathlineto{\pgfqpoint{3.070954in}{1.489860in}}%
\pgfpathlineto{\pgfqpoint{3.079551in}{1.453829in}}%
\pgfpathlineto{\pgfqpoint{3.085502in}{1.417508in}}%
\pgfpathlineto{\pgfqpoint{3.088794in}{1.380985in}}%
\pgfpathlineto{\pgfqpoint{3.089416in}{1.344350in}}%
\pgfpathlineto{\pgfqpoint{3.087369in}{1.307695in}}%
\pgfpathlineto{\pgfqpoint{3.082657in}{1.271109in}}%
\pgfpathlineto{\pgfqpoint{3.075291in}{1.234682in}}%
\pgfpathlineto{\pgfqpoint{3.065290in}{1.198505in}}%
\pgfpathlineto{\pgfqpoint{3.052678in}{1.162665in}}%
\pgfpathlineto{\pgfqpoint{3.037487in}{1.127252in}}%
\pgfpathlineto{\pgfqpoint{3.019754in}{1.092354in}}%
\pgfpathlineto{\pgfqpoint{2.999523in}{1.058055in}}%
\pgfpathlineto{\pgfqpoint{2.976843in}{1.024442in}}%
\pgfpathlineto{\pgfqpoint{2.951770in}{0.991598in}}%
\pgfpathlineto{\pgfqpoint{2.924368in}{0.959604in}}%
\pgfpathlineto{\pgfqpoint{2.894702in}{0.928540in}}%
\pgfpathlineto{\pgfqpoint{2.862847in}{0.898483in}}%
\pgfpathlineto{\pgfqpoint{2.828882in}{0.869509in}}%
\pgfpathlineto{\pgfqpoint{2.792890in}{0.841690in}}%
\pgfpathlineto{\pgfqpoint{2.754961in}{0.815096in}}%
\pgfpathlineto{\pgfqpoint{2.715189in}{0.789794in}}%
\pgfpathlineto{\pgfqpoint{2.673672in}{0.765847in}}%
\pgfpathlineto{\pgfqpoint{2.630515in}{0.743317in}}%
\pgfpathlineto{\pgfqpoint{2.585824in}{0.722260in}}%
\pgfpathlineto{\pgfqpoint{2.539710in}{0.702730in}}%
\pgfpathlineto{\pgfqpoint{2.492288in}{0.684776in}}%
\pgfpathlineto{\pgfqpoint{2.443677in}{0.668444in}}%
\pgfpathlineto{\pgfqpoint{2.393997in}{0.653776in}}%
\pgfpathlineto{\pgfqpoint{2.343373in}{0.640810in}}%
\pgfpathlineto{\pgfqpoint{2.291931in}{0.629578in}}%
\pgfpathlineto{\pgfqpoint{2.239799in}{0.620110in}}%
\pgfpathlineto{\pgfqpoint{2.187107in}{0.612430in}}%
\pgfpathlineto{\pgfqpoint{2.133988in}{0.606557in}}%
\pgfpathlineto{\pgfqpoint{2.080574in}{0.602506in}}%
\pgfpathlineto{\pgfqpoint{2.026998in}{0.600289in}}%
\pgfpathlineto{\pgfqpoint{1.973394in}{0.599910in}}%
\pgfpathlineto{\pgfqpoint{1.919897in}{0.601371in}}%
\pgfpathlineto{\pgfqpoint{1.866641in}{0.604668in}}%
\pgfpathlineto{\pgfqpoint{1.813757in}{0.609793in}}%
\pgfpathlineto{\pgfqpoint{1.761379in}{0.616731in}}%
\pgfpathlineto{\pgfqpoint{1.709637in}{0.625467in}}%
\pgfpathlineto{\pgfqpoint{1.658661in}{0.635976in}}%
\pgfpathlineto{\pgfqpoint{1.608577in}{0.648232in}}%
\pgfpathlineto{\pgfqpoint{1.559511in}{0.662204in}}%
\pgfpathlineto{\pgfqpoint{1.511586in}{0.677855in}}%
\pgfpathlineto{\pgfqpoint{1.464919in}{0.695147in}}%
\pgfpathlineto{\pgfqpoint{1.419629in}{0.714034in}}%
\pgfpathlineto{\pgfqpoint{1.375827in}{0.734469in}}%
\pgfpathlineto{\pgfqpoint{1.333623in}{0.756400in}}%
\pgfpathlineto{\pgfqpoint{1.293120in}{0.779770in}}%
\pgfpathlineto{\pgfqpoint{1.254421in}{0.804522in}}%
\pgfpathlineto{\pgfqpoint{1.217621in}{0.830592in}}%
\pgfpathlineto{\pgfqpoint{1.182811in}{0.857916in}}%
\pgfpathlineto{\pgfqpoint{1.150078in}{0.886423in}}%
\pgfpathlineto{\pgfqpoint{1.119502in}{0.916043in}}%
\pgfpathlineto{\pgfqpoint{1.091159in}{0.946702in}}%
\pgfpathlineto{\pgfqpoint{1.065120in}{0.978323in}}%
\pgfpathlineto{\pgfqpoint{1.041448in}{1.010827in}}%
\pgfpathlineto{\pgfqpoint{1.020203in}{1.044134in}}%
\pgfpathlineto{\pgfqpoint{1.001436in}{1.078160in}}%
\pgfpathlineto{\pgfqpoint{0.985194in}{1.112822in}}%
\pgfpathlineto{\pgfqpoint{0.971516in}{1.148034in}}%
\pgfpathlineto{\pgfqpoint{0.960438in}{1.183709in}}%
\pgfpathlineto{\pgfqpoint{0.951985in}{1.219758in}}%
\pgfpathlineto{\pgfqpoint{0.946179in}{1.256093in}}%
\pgfpathlineto{\pgfqpoint{0.943033in}{1.292624in}}%
\pgfpathlineto{\pgfqpoint{0.942557in}{1.329262in}}%
\pgfpathlineto{\pgfqpoint{0.944751in}{1.365916in}}%
\pgfpathlineto{\pgfqpoint{0.949609in}{1.402495in}}%
\pgfpathlineto{\pgfqpoint{0.957119in}{1.438911in}}%
\pgfpathlineto{\pgfqpoint{0.967264in}{1.475072in}}%
\pgfpathlineto{\pgfqpoint{0.980018in}{1.510891in}}%
\pgfpathlineto{\pgfqpoint{0.995349in}{1.546278in}}%
\pgfpathlineto{\pgfqpoint{1.013220in}{1.581146in}}%
\pgfpathlineto{\pgfqpoint{1.033587in}{1.615409in}}%
\pgfpathlineto{\pgfqpoint{1.056399in}{1.648982in}}%
\pgfpathlineto{\pgfqpoint{1.081601in}{1.681782in}}%
\pgfpathlineto{\pgfqpoint{1.109130in}{1.713727in}}%
\pgfpathlineto{\pgfqpoint{1.138917in}{1.744738in}}%
\pgfpathlineto{\pgfqpoint{1.170890in}{1.774737in}}%
\pgfpathlineto{\pgfqpoint{1.204968in}{1.803650in}}%
\pgfpathlineto{\pgfqpoint{1.241069in}{1.831404in}}%
\pgfpathlineto{\pgfqpoint{1.279101in}{1.857929in}}%
\pgfpathlineto{\pgfqpoint{1.318971in}{1.883158in}}%
\pgfpathlineto{\pgfqpoint{1.360580in}{1.907029in}}%
\pgfpathlineto{\pgfqpoint{1.403824in}{1.929480in}}%
\pgfpathlineto{\pgfqpoint{1.448597in}{1.950454in}}%
\pgfpathlineto{\pgfqpoint{1.494785in}{1.969899in}}%
\pgfpathlineto{\pgfqpoint{1.542275in}{1.987766in}}%
\pgfpathlineto{\pgfqpoint{1.590948in}{2.004007in}}%
\pgfpathlineto{\pgfqpoint{1.640683in}{2.018583in}}%
\pgfpathlineto{\pgfqpoint{1.691355in}{2.031455in}}%
\pgfpathlineto{\pgfqpoint{1.742838in}{2.042591in}}%
\pgfpathlineto{\pgfqpoint{1.795004in}{2.051961in}}%
\pgfpathlineto{\pgfqpoint{1.847723in}{2.059543in}}%
\pgfpathlineto{\pgfqpoint{1.900862in}{2.065317in}}%
\pgfpathlineto{\pgfqpoint{1.954288in}{2.069267in}}%
\pgfpathlineto{\pgfqpoint{2.007869in}{2.071384in}}%
\pgfpathlineto{\pgfqpoint{2.061470in}{2.071662in}}%
\pgfpathlineto{\pgfqpoint{2.114957in}{2.070100in}}%
\pgfpathlineto{\pgfqpoint{2.168197in}{2.066703in}}%
\pgfpathlineto{\pgfqpoint{2.221056in}{2.061479in}}%
\pgfpathlineto{\pgfqpoint{2.273403in}{2.054441in}}%
\pgfpathlineto{\pgfqpoint{2.325106in}{2.045608in}}%
\pgfpathlineto{\pgfqpoint{2.376037in}{2.035002in}}%
\pgfpathlineto{\pgfqpoint{2.426068in}{2.022651in}}%
\pgfpathlineto{\pgfqpoint{2.475074in}{2.008586in}}%
\pgfpathlineto{\pgfqpoint{2.522934in}{1.992844in}}%
\pgfpathlineto{\pgfqpoint{2.569528in}{1.975464in}}%
\pgfpathlineto{\pgfqpoint{2.614740in}{1.956491in}}%
\pgfpathlineto{\pgfqpoint{2.658457in}{1.935972in}}%
\pgfpathlineto{\pgfqpoint{2.700571in}{1.913961in}}%
\pgfpathlineto{\pgfqpoint{2.740977in}{1.890514in}}%
\pgfpathlineto{\pgfqpoint{2.779575in}{1.865688in}}%
\pgfpathlineto{\pgfqpoint{2.816268in}{1.839547in}}%
\pgfpathlineto{\pgfqpoint{2.850967in}{1.812157in}}%
\pgfpathlineto{\pgfqpoint{2.883584in}{1.783587in}}%
\pgfpathlineto{\pgfqpoint{2.914039in}{1.753908in}}%
\pgfpathlineto{\pgfqpoint{2.942257in}{1.723195in}}%
\pgfpathlineto{\pgfqpoint{2.968169in}{1.691523in}}%
\pgfpathlineto{\pgfqpoint{2.991709in}{1.658973in}}%
\pgfpathlineto{\pgfqpoint{3.012820in}{1.625625in}}%
\pgfpathlineto{\pgfqpoint{3.031450in}{1.591561in}}%
\pgfpathlineto{\pgfqpoint{3.031450in}{1.591561in}}%
\pgfusepath{stroke}%
\end{pgfscope}%
\begin{pgfscope}%
\pgfsetrectcap%
\pgfsetmiterjoin%
\pgfsetlinewidth{0.803000pt}%
\definecolor{currentstroke}{rgb}{0.000000,0.000000,0.000000}%
\pgfsetstrokecolor{currentstroke}%
\pgfsetdash{}{0pt}%
\pgfpathmoveto{\pgfqpoint{0.835065in}{0.526234in}}%
\pgfpathlineto{\pgfqpoint{0.835065in}{2.145371in}}%
\pgfusepath{stroke}%
\end{pgfscope}%
\begin{pgfscope}%
\pgfsetrectcap%
\pgfsetmiterjoin%
\pgfsetlinewidth{0.803000pt}%
\definecolor{currentstroke}{rgb}{0.000000,0.000000,0.000000}%
\pgfsetstrokecolor{currentstroke}%
\pgfsetdash{}{0pt}%
\pgfpathmoveto{\pgfqpoint{3.196863in}{0.526234in}}%
\pgfpathlineto{\pgfqpoint{3.196863in}{2.145371in}}%
\pgfusepath{stroke}%
\end{pgfscope}%
\begin{pgfscope}%
\pgfsetrectcap%
\pgfsetmiterjoin%
\pgfsetlinewidth{0.803000pt}%
\definecolor{currentstroke}{rgb}{0.000000,0.000000,0.000000}%
\pgfsetstrokecolor{currentstroke}%
\pgfsetdash{}{0pt}%
\pgfpathmoveto{\pgfqpoint{0.835065in}{0.526234in}}%
\pgfpathlineto{\pgfqpoint{3.196863in}{0.526234in}}%
\pgfusepath{stroke}%
\end{pgfscope}%
\begin{pgfscope}%
\pgfsetrectcap%
\pgfsetmiterjoin%
\pgfsetlinewidth{0.803000pt}%
\definecolor{currentstroke}{rgb}{0.000000,0.000000,0.000000}%
\pgfsetstrokecolor{currentstroke}%
\pgfsetdash{}{0pt}%
\pgfpathmoveto{\pgfqpoint{0.835065in}{2.145371in}}%
\pgfpathlineto{\pgfqpoint{3.196863in}{2.145371in}}%
\pgfusepath{stroke}%
\end{pgfscope}%
\begin{pgfscope}%
\definecolor{textcolor}{rgb}{0.000000,0.000000,0.000000}%
\pgfsetstrokecolor{textcolor}%
\pgfsetfillcolor{textcolor}%
\pgftext[x=2.015964in,y=2.228704in,,base]{\color{textcolor}\rmfamily\fontsize{12.000000}{14.400000}\selectfont phase plot}%
\end{pgfscope}%
\begin{pgfscope}%
\pgfsetbuttcap%
\pgfsetmiterjoin%
\definecolor{currentfill}{rgb}{1.000000,1.000000,1.000000}%
\pgfsetfillcolor{currentfill}%
\pgfsetlinewidth{0.000000pt}%
\definecolor{currentstroke}{rgb}{0.000000,0.000000,0.000000}%
\pgfsetstrokecolor{currentstroke}%
\pgfsetstrokeopacity{0.000000}%
\pgfsetdash{}{0pt}%
\pgfpathmoveto{\pgfqpoint{3.906113in}{0.526234in}}%
\pgfpathlineto{\pgfqpoint{6.267911in}{0.526234in}}%
\pgfpathlineto{\pgfqpoint{6.267911in}{2.145371in}}%
\pgfpathlineto{\pgfqpoint{3.906113in}{2.145371in}}%
\pgfpathclose%
\pgfusepath{fill}%
\end{pgfscope}%
\begin{pgfscope}%
\pgfsetbuttcap%
\pgfsetroundjoin%
\definecolor{currentfill}{rgb}{0.000000,0.000000,0.000000}%
\pgfsetfillcolor{currentfill}%
\pgfsetlinewidth{0.803000pt}%
\definecolor{currentstroke}{rgb}{0.000000,0.000000,0.000000}%
\pgfsetstrokecolor{currentstroke}%
\pgfsetdash{}{0pt}%
\pgfsys@defobject{currentmarker}{\pgfqpoint{0.000000in}{-0.048611in}}{\pgfqpoint{0.000000in}{0.000000in}}{%
\pgfpathmoveto{\pgfqpoint{0.000000in}{0.000000in}}%
\pgfpathlineto{\pgfqpoint{0.000000in}{-0.048611in}}%
\pgfusepath{stroke,fill}%
}%
\begin{pgfscope}%
\pgfsys@transformshift{4.013467in}{0.526234in}%
\pgfsys@useobject{currentmarker}{}%
\end{pgfscope}%
\end{pgfscope}%
\begin{pgfscope}%
\definecolor{textcolor}{rgb}{0.000000,0.000000,0.000000}%
\pgfsetstrokecolor{textcolor}%
\pgfsetfillcolor{textcolor}%
\pgftext[x=4.013467in,y=0.429012in,,top]{\color{textcolor}\rmfamily\fontsize{10.000000}{12.000000}\selectfont \(\displaystyle 0.0\)}%
\end{pgfscope}%
\begin{pgfscope}%
\pgfsetbuttcap%
\pgfsetroundjoin%
\definecolor{currentfill}{rgb}{0.000000,0.000000,0.000000}%
\pgfsetfillcolor{currentfill}%
\pgfsetlinewidth{0.803000pt}%
\definecolor{currentstroke}{rgb}{0.000000,0.000000,0.000000}%
\pgfsetstrokecolor{currentstroke}%
\pgfsetdash{}{0pt}%
\pgfsys@defobject{currentmarker}{\pgfqpoint{0.000000in}{-0.048611in}}{\pgfqpoint{0.000000in}{0.000000in}}{%
\pgfpathmoveto{\pgfqpoint{0.000000in}{0.000000in}}%
\pgfpathlineto{\pgfqpoint{0.000000in}{-0.048611in}}%
\pgfusepath{stroke,fill}%
}%
\begin{pgfscope}%
\pgfsys@transformshift{4.550240in}{0.526234in}%
\pgfsys@useobject{currentmarker}{}%
\end{pgfscope}%
\end{pgfscope}%
\begin{pgfscope}%
\definecolor{textcolor}{rgb}{0.000000,0.000000,0.000000}%
\pgfsetstrokecolor{textcolor}%
\pgfsetfillcolor{textcolor}%
\pgftext[x=4.550240in,y=0.429012in,,top]{\color{textcolor}\rmfamily\fontsize{10.000000}{12.000000}\selectfont \(\displaystyle 2.5\)}%
\end{pgfscope}%
\begin{pgfscope}%
\pgfsetbuttcap%
\pgfsetroundjoin%
\definecolor{currentfill}{rgb}{0.000000,0.000000,0.000000}%
\pgfsetfillcolor{currentfill}%
\pgfsetlinewidth{0.803000pt}%
\definecolor{currentstroke}{rgb}{0.000000,0.000000,0.000000}%
\pgfsetstrokecolor{currentstroke}%
\pgfsetdash{}{0pt}%
\pgfsys@defobject{currentmarker}{\pgfqpoint{0.000000in}{-0.048611in}}{\pgfqpoint{0.000000in}{0.000000in}}{%
\pgfpathmoveto{\pgfqpoint{0.000000in}{0.000000in}}%
\pgfpathlineto{\pgfqpoint{0.000000in}{-0.048611in}}%
\pgfusepath{stroke,fill}%
}%
\begin{pgfscope}%
\pgfsys@transformshift{5.087012in}{0.526234in}%
\pgfsys@useobject{currentmarker}{}%
\end{pgfscope}%
\end{pgfscope}%
\begin{pgfscope}%
\definecolor{textcolor}{rgb}{0.000000,0.000000,0.000000}%
\pgfsetstrokecolor{textcolor}%
\pgfsetfillcolor{textcolor}%
\pgftext[x=5.087012in,y=0.429012in,,top]{\color{textcolor}\rmfamily\fontsize{10.000000}{12.000000}\selectfont \(\displaystyle 5.0\)}%
\end{pgfscope}%
\begin{pgfscope}%
\pgfsetbuttcap%
\pgfsetroundjoin%
\definecolor{currentfill}{rgb}{0.000000,0.000000,0.000000}%
\pgfsetfillcolor{currentfill}%
\pgfsetlinewidth{0.803000pt}%
\definecolor{currentstroke}{rgb}{0.000000,0.000000,0.000000}%
\pgfsetstrokecolor{currentstroke}%
\pgfsetdash{}{0pt}%
\pgfsys@defobject{currentmarker}{\pgfqpoint{0.000000in}{-0.048611in}}{\pgfqpoint{0.000000in}{0.000000in}}{%
\pgfpathmoveto{\pgfqpoint{0.000000in}{0.000000in}}%
\pgfpathlineto{\pgfqpoint{0.000000in}{-0.048611in}}%
\pgfusepath{stroke,fill}%
}%
\begin{pgfscope}%
\pgfsys@transformshift{5.623784in}{0.526234in}%
\pgfsys@useobject{currentmarker}{}%
\end{pgfscope}%
\end{pgfscope}%
\begin{pgfscope}%
\definecolor{textcolor}{rgb}{0.000000,0.000000,0.000000}%
\pgfsetstrokecolor{textcolor}%
\pgfsetfillcolor{textcolor}%
\pgftext[x=5.623784in,y=0.429012in,,top]{\color{textcolor}\rmfamily\fontsize{10.000000}{12.000000}\selectfont \(\displaystyle 7.5\)}%
\end{pgfscope}%
\begin{pgfscope}%
\pgfsetbuttcap%
\pgfsetroundjoin%
\definecolor{currentfill}{rgb}{0.000000,0.000000,0.000000}%
\pgfsetfillcolor{currentfill}%
\pgfsetlinewidth{0.803000pt}%
\definecolor{currentstroke}{rgb}{0.000000,0.000000,0.000000}%
\pgfsetstrokecolor{currentstroke}%
\pgfsetdash{}{0pt}%
\pgfsys@defobject{currentmarker}{\pgfqpoint{0.000000in}{-0.048611in}}{\pgfqpoint{0.000000in}{0.000000in}}{%
\pgfpathmoveto{\pgfqpoint{0.000000in}{0.000000in}}%
\pgfpathlineto{\pgfqpoint{0.000000in}{-0.048611in}}%
\pgfusepath{stroke,fill}%
}%
\begin{pgfscope}%
\pgfsys@transformshift{6.160557in}{0.526234in}%
\pgfsys@useobject{currentmarker}{}%
\end{pgfscope}%
\end{pgfscope}%
\begin{pgfscope}%
\definecolor{textcolor}{rgb}{0.000000,0.000000,0.000000}%
\pgfsetstrokecolor{textcolor}%
\pgfsetfillcolor{textcolor}%
\pgftext[x=6.160557in,y=0.429012in,,top]{\color{textcolor}\rmfamily\fontsize{10.000000}{12.000000}\selectfont \(\displaystyle 10.0\)}%
\end{pgfscope}%
\begin{pgfscope}%
\definecolor{textcolor}{rgb}{0.000000,0.000000,0.000000}%
\pgfsetstrokecolor{textcolor}%
\pgfsetfillcolor{textcolor}%
\pgftext[x=5.087012in,y=0.250000in,,top]{\color{textcolor}\rmfamily\fontsize{10.000000}{12.000000}\selectfont time (s)}%
\end{pgfscope}%
\begin{pgfscope}%
\pgfsetbuttcap%
\pgfsetroundjoin%
\definecolor{currentfill}{rgb}{0.000000,0.000000,0.000000}%
\pgfsetfillcolor{currentfill}%
\pgfsetlinewidth{0.803000pt}%
\definecolor{currentstroke}{rgb}{0.000000,0.000000,0.000000}%
\pgfsetstrokecolor{currentstroke}%
\pgfsetdash{}{0pt}%
\pgfsys@defobject{currentmarker}{\pgfqpoint{-0.048611in}{0.000000in}}{\pgfqpoint{0.000000in}{0.000000in}}{%
\pgfpathmoveto{\pgfqpoint{0.000000in}{0.000000in}}%
\pgfpathlineto{\pgfqpoint{-0.048611in}{0.000000in}}%
\pgfusepath{stroke,fill}%
}%
\begin{pgfscope}%
\pgfsys@transformshift{3.906113in}{0.925401in}%
\pgfsys@useobject{currentmarker}{}%
\end{pgfscope}%
\end{pgfscope}%
\begin{pgfscope}%
\definecolor{textcolor}{rgb}{0.000000,0.000000,0.000000}%
\pgfsetstrokecolor{textcolor}%
\pgfsetfillcolor{textcolor}%
\pgftext[x=3.631421in,y=0.877176in,left,base]{\color{textcolor}\rmfamily\fontsize{10.000000}{12.000000}\selectfont \(\displaystyle 0.2\)}%
\end{pgfscope}%
\begin{pgfscope}%
\pgfsetbuttcap%
\pgfsetroundjoin%
\definecolor{currentfill}{rgb}{0.000000,0.000000,0.000000}%
\pgfsetfillcolor{currentfill}%
\pgfsetlinewidth{0.803000pt}%
\definecolor{currentstroke}{rgb}{0.000000,0.000000,0.000000}%
\pgfsetstrokecolor{currentstroke}%
\pgfsetdash{}{0pt}%
\pgfsys@defobject{currentmarker}{\pgfqpoint{-0.048611in}{0.000000in}}{\pgfqpoint{0.000000in}{0.000000in}}{%
\pgfpathmoveto{\pgfqpoint{0.000000in}{0.000000in}}%
\pgfpathlineto{\pgfqpoint{-0.048611in}{0.000000in}}%
\pgfusepath{stroke,fill}%
}%
\begin{pgfscope}%
\pgfsys@transformshift{3.906113in}{1.561585in}%
\pgfsys@useobject{currentmarker}{}%
\end{pgfscope}%
\end{pgfscope}%
\begin{pgfscope}%
\definecolor{textcolor}{rgb}{0.000000,0.000000,0.000000}%
\pgfsetstrokecolor{textcolor}%
\pgfsetfillcolor{textcolor}%
\pgftext[x=3.631421in,y=1.513360in,left,base]{\color{textcolor}\rmfamily\fontsize{10.000000}{12.000000}\selectfont \(\displaystyle 0.3\)}%
\end{pgfscope}%
\begin{pgfscope}%
\definecolor{textcolor}{rgb}{0.000000,0.000000,0.000000}%
\pgfsetstrokecolor{textcolor}%
\pgfsetfillcolor{textcolor}%
\pgftext[x=3.575865in,y=1.335803in,,bottom,rotate=90.000000]{\color{textcolor}\rmfamily\fontsize{10.000000}{12.000000}\selectfont energy (J)}%
\end{pgfscope}%
\begin{pgfscope}%
\pgfpathrectangle{\pgfqpoint{3.906113in}{0.526234in}}{\pgfqpoint{2.361798in}{1.619136in}}%
\pgfusepath{clip}%
\pgfsetrectcap%
\pgfsetroundjoin%
\pgfsetlinewidth{1.505625pt}%
\definecolor{currentstroke}{rgb}{0.000000,0.000000,1.000000}%
\pgfsetstrokecolor{currentstroke}%
\pgfsetdash{}{0pt}%
\pgfpathmoveto{\pgfqpoint{4.013467in}{0.600210in}}%
\pgfpathlineto{\pgfqpoint{4.015615in}{0.600210in}}%
\pgfpathlineto{\pgfqpoint{4.017762in}{0.601510in}}%
\pgfpathlineto{\pgfqpoint{4.019909in}{0.610094in}}%
\pgfpathlineto{\pgfqpoint{4.024203in}{0.648699in}}%
\pgfpathlineto{\pgfqpoint{4.028497in}{0.714510in}}%
\pgfpathlineto{\pgfqpoint{4.034938in}{0.858305in}}%
\pgfpathlineto{\pgfqpoint{4.043527in}{1.113326in}}%
\pgfpathlineto{\pgfqpoint{4.067145in}{1.859332in}}%
\pgfpathlineto{\pgfqpoint{4.073586in}{1.988789in}}%
\pgfpathlineto{\pgfqpoint{4.077880in}{2.043221in}}%
\pgfpathlineto{\pgfqpoint{4.082174in}{2.069454in}}%
\pgfpathlineto{\pgfqpoint{4.084321in}{2.071621in}}%
\pgfpathlineto{\pgfqpoint{4.086468in}{2.066434in}}%
\pgfpathlineto{\pgfqpoint{4.088616in}{2.053945in}}%
\pgfpathlineto{\pgfqpoint{4.092910in}{2.007639in}}%
\pgfpathlineto{\pgfqpoint{4.097204in}{1.934561in}}%
\pgfpathlineto{\pgfqpoint{4.103645in}{1.781425in}}%
\pgfpathlineto{\pgfqpoint{4.112234in}{1.518432in}}%
\pgfpathlineto{\pgfqpoint{4.133704in}{0.835763in}}%
\pgfpathlineto{\pgfqpoint{4.140146in}{0.698875in}}%
\pgfpathlineto{\pgfqpoint{4.144440in}{0.638478in}}%
\pgfpathlineto{\pgfqpoint{4.148734in}{0.605688in}}%
\pgfpathlineto{\pgfqpoint{4.150881in}{0.600092in}}%
\pgfpathlineto{\pgfqpoint{4.153028in}{0.601792in}}%
\pgfpathlineto{\pgfqpoint{4.155175in}{0.610773in}}%
\pgfpathlineto{\pgfqpoint{4.159470in}{0.650149in}}%
\pgfpathlineto{\pgfqpoint{4.163764in}{0.716673in}}%
\pgfpathlineto{\pgfqpoint{4.170205in}{0.861370in}}%
\pgfpathlineto{\pgfqpoint{4.178793in}{1.117163in}}%
\pgfpathlineto{\pgfqpoint{4.200264in}{1.808201in}}%
\pgfpathlineto{\pgfqpoint{4.206705in}{1.953836in}}%
\pgfpathlineto{\pgfqpoint{4.211000in}{2.020902in}}%
\pgfpathlineto{\pgfqpoint{4.215294in}{2.060665in}}%
\pgfpathlineto{\pgfqpoint{4.217441in}{2.069763in}}%
\pgfpathlineto{\pgfqpoint{4.219588in}{2.071527in}}%
\pgfpathlineto{\pgfqpoint{4.221735in}{2.065938in}}%
\pgfpathlineto{\pgfqpoint{4.223882in}{2.053052in}}%
\pgfpathlineto{\pgfqpoint{4.228176in}{2.005983in}}%
\pgfpathlineto{\pgfqpoint{4.232471in}{1.932209in}}%
\pgfpathlineto{\pgfqpoint{4.238912in}{1.778212in}}%
\pgfpathlineto{\pgfqpoint{4.249647in}{1.442465in}}%
\pgfpathlineto{\pgfqpoint{4.266824in}{0.888795in}}%
\pgfpathlineto{\pgfqpoint{4.273265in}{0.736384in}}%
\pgfpathlineto{\pgfqpoint{4.279707in}{0.637204in}}%
\pgfpathlineto{\pgfqpoint{4.284001in}{0.605193in}}%
\pgfpathlineto{\pgfqpoint{4.286148in}{0.599996in}}%
\pgfpathlineto{\pgfqpoint{4.288295in}{0.602096in}}%
\pgfpathlineto{\pgfqpoint{4.290442in}{0.611474in}}%
\pgfpathlineto{\pgfqpoint{4.294736in}{0.651619in}}%
\pgfpathlineto{\pgfqpoint{4.299030in}{0.718854in}}%
\pgfpathlineto{\pgfqpoint{4.305472in}{0.864450in}}%
\pgfpathlineto{\pgfqpoint{4.314060in}{1.121006in}}%
\pgfpathlineto{\pgfqpoint{4.335531in}{1.811283in}}%
\pgfpathlineto{\pgfqpoint{4.341972in}{1.956015in}}%
\pgfpathlineto{\pgfqpoint{4.346266in}{2.022365in}}%
\pgfpathlineto{\pgfqpoint{4.350560in}{2.061352in}}%
\pgfpathlineto{\pgfqpoint{4.352708in}{2.070050in}}%
\pgfpathlineto{\pgfqpoint{4.354855in}{2.071411in}}%
\pgfpathlineto{\pgfqpoint{4.357002in}{2.065420in}}%
\pgfpathlineto{\pgfqpoint{4.359149in}{2.052138in}}%
\pgfpathlineto{\pgfqpoint{4.363443in}{2.004307in}}%
\pgfpathlineto{\pgfqpoint{4.367737in}{1.929838in}}%
\pgfpathlineto{\pgfqpoint{4.374178in}{1.774986in}}%
\pgfpathlineto{\pgfqpoint{4.384914in}{1.438481in}}%
\pgfpathlineto{\pgfqpoint{4.402091in}{0.885609in}}%
\pgfpathlineto{\pgfqpoint{4.408532in}{0.734062in}}%
\pgfpathlineto{\pgfqpoint{4.414973in}{0.635951in}}%
\pgfpathlineto{\pgfqpoint{4.419267in}{0.604720in}}%
\pgfpathlineto{\pgfqpoint{4.421414in}{0.599922in}}%
\pgfpathlineto{\pgfqpoint{4.423561in}{0.602422in}}%
\pgfpathlineto{\pgfqpoint{4.425709in}{0.612196in}}%
\pgfpathlineto{\pgfqpoint{4.430003in}{0.653109in}}%
\pgfpathlineto{\pgfqpoint{4.434297in}{0.721054in}}%
\pgfpathlineto{\pgfqpoint{4.440738in}{0.867543in}}%
\pgfpathlineto{\pgfqpoint{4.449327in}{1.124856in}}%
\pgfpathlineto{\pgfqpoint{4.470797in}{1.814351in}}%
\pgfpathlineto{\pgfqpoint{4.477239in}{1.958176in}}%
\pgfpathlineto{\pgfqpoint{4.481533in}{2.023806in}}%
\pgfpathlineto{\pgfqpoint{4.485827in}{2.062017in}}%
\pgfpathlineto{\pgfqpoint{4.487974in}{2.070315in}}%
\pgfpathlineto{\pgfqpoint{4.490121in}{2.071273in}}%
\pgfpathlineto{\pgfqpoint{4.492268in}{2.064880in}}%
\pgfpathlineto{\pgfqpoint{4.496563in}{2.030376in}}%
\pgfpathlineto{\pgfqpoint{4.500857in}{1.968187in}}%
\pgfpathlineto{\pgfqpoint{4.507298in}{1.828735in}}%
\pgfpathlineto{\pgfqpoint{4.515886in}{1.577233in}}%
\pgfpathlineto{\pgfqpoint{4.539504in}{0.826984in}}%
\pgfpathlineto{\pgfqpoint{4.545946in}{0.692922in}}%
\pgfpathlineto{\pgfqpoint{4.550240in}{0.634719in}}%
\pgfpathlineto{\pgfqpoint{4.554534in}{0.604269in}}%
\pgfpathlineto{\pgfqpoint{4.556681in}{0.599870in}}%
\pgfpathlineto{\pgfqpoint{4.558828in}{0.602770in}}%
\pgfpathlineto{\pgfqpoint{4.560975in}{0.612940in}}%
\pgfpathlineto{\pgfqpoint{4.565269in}{0.654620in}}%
\pgfpathlineto{\pgfqpoint{4.569564in}{0.723272in}}%
\pgfpathlineto{\pgfqpoint{4.576005in}{0.870651in}}%
\pgfpathlineto{\pgfqpoint{4.584593in}{1.128712in}}%
\pgfpathlineto{\pgfqpoint{4.606064in}{1.817405in}}%
\pgfpathlineto{\pgfqpoint{4.612505in}{1.960317in}}%
\pgfpathlineto{\pgfqpoint{4.616800in}{2.025228in}}%
\pgfpathlineto{\pgfqpoint{4.621094in}{2.062661in}}%
\pgfpathlineto{\pgfqpoint{4.623241in}{2.070558in}}%
\pgfpathlineto{\pgfqpoint{4.625388in}{2.071113in}}%
\pgfpathlineto{\pgfqpoint{4.627535in}{2.064319in}}%
\pgfpathlineto{\pgfqpoint{4.631829in}{2.029032in}}%
\pgfpathlineto{\pgfqpoint{4.636123in}{1.966116in}}%
\pgfpathlineto{\pgfqpoint{4.642565in}{1.825736in}}%
\pgfpathlineto{\pgfqpoint{4.651153in}{1.573426in}}%
\pgfpathlineto{\pgfqpoint{4.674771in}{0.824088in}}%
\pgfpathlineto{\pgfqpoint{4.681212in}{0.690976in}}%
\pgfpathlineto{\pgfqpoint{4.685506in}{0.633507in}}%
\pgfpathlineto{\pgfqpoint{4.689801in}{0.603839in}}%
\pgfpathlineto{\pgfqpoint{4.691948in}{0.599840in}}%
\pgfpathlineto{\pgfqpoint{4.694095in}{0.603139in}}%
\pgfpathlineto{\pgfqpoint{4.696242in}{0.613705in}}%
\pgfpathlineto{\pgfqpoint{4.700536in}{0.656151in}}%
\pgfpathlineto{\pgfqpoint{4.704830in}{0.725508in}}%
\pgfpathlineto{\pgfqpoint{4.711271in}{0.873772in}}%
\pgfpathlineto{\pgfqpoint{4.719860in}{1.132575in}}%
\pgfpathlineto{\pgfqpoint{4.741331in}{1.820444in}}%
\pgfpathlineto{\pgfqpoint{4.747772in}{1.962441in}}%
\pgfpathlineto{\pgfqpoint{4.752066in}{2.026628in}}%
\pgfpathlineto{\pgfqpoint{4.756360in}{2.063282in}}%
\pgfpathlineto{\pgfqpoint{4.758507in}{2.070779in}}%
\pgfpathlineto{\pgfqpoint{4.760654in}{2.070931in}}%
\pgfpathlineto{\pgfqpoint{4.762802in}{2.063735in}}%
\pgfpathlineto{\pgfqpoint{4.767096in}{2.027668in}}%
\pgfpathlineto{\pgfqpoint{4.771390in}{1.964026in}}%
\pgfpathlineto{\pgfqpoint{4.777831in}{1.822722in}}%
\pgfpathlineto{\pgfqpoint{4.786420in}{1.569612in}}%
\pgfpathlineto{\pgfqpoint{4.810038in}{0.821208in}}%
\pgfpathlineto{\pgfqpoint{4.816479in}{0.689050in}}%
\pgfpathlineto{\pgfqpoint{4.820773in}{0.632316in}}%
\pgfpathlineto{\pgfqpoint{4.825067in}{0.603432in}}%
\pgfpathlineto{\pgfqpoint{4.827214in}{0.599832in}}%
\pgfpathlineto{\pgfqpoint{4.829361in}{0.603531in}}%
\pgfpathlineto{\pgfqpoint{4.831508in}{0.614492in}}%
\pgfpathlineto{\pgfqpoint{4.835803in}{0.657702in}}%
\pgfpathlineto{\pgfqpoint{4.840097in}{0.727763in}}%
\pgfpathlineto{\pgfqpoint{4.846538in}{0.876908in}}%
\pgfpathlineto{\pgfqpoint{4.855126in}{1.136443in}}%
\pgfpathlineto{\pgfqpoint{4.876597in}{1.823469in}}%
\pgfpathlineto{\pgfqpoint{4.883039in}{1.964545in}}%
\pgfpathlineto{\pgfqpoint{4.887333in}{2.028008in}}%
\pgfpathlineto{\pgfqpoint{4.891627in}{2.063882in}}%
\pgfpathlineto{\pgfqpoint{4.893774in}{2.070978in}}%
\pgfpathlineto{\pgfqpoint{4.895921in}{2.070726in}}%
\pgfpathlineto{\pgfqpoint{4.898068in}{2.063130in}}%
\pgfpathlineto{\pgfqpoint{4.902362in}{2.026283in}}%
\pgfpathlineto{\pgfqpoint{4.906657in}{1.961916in}}%
\pgfpathlineto{\pgfqpoint{4.913098in}{1.819693in}}%
\pgfpathlineto{\pgfqpoint{4.921686in}{1.565790in}}%
\pgfpathlineto{\pgfqpoint{4.945304in}{0.818343in}}%
\pgfpathlineto{\pgfqpoint{4.951745in}{0.687142in}}%
\pgfpathlineto{\pgfqpoint{4.956040in}{0.631147in}}%
\pgfpathlineto{\pgfqpoint{4.960334in}{0.603046in}}%
\pgfpathlineto{\pgfqpoint{4.962481in}{0.599845in}}%
\pgfpathlineto{\pgfqpoint{4.964628in}{0.603944in}}%
\pgfpathlineto{\pgfqpoint{4.966775in}{0.615301in}}%
\pgfpathlineto{\pgfqpoint{4.971069in}{0.659273in}}%
\pgfpathlineto{\pgfqpoint{4.975363in}{0.730035in}}%
\pgfpathlineto{\pgfqpoint{4.981805in}{0.880057in}}%
\pgfpathlineto{\pgfqpoint{4.990393in}{1.140317in}}%
\pgfpathlineto{\pgfqpoint{5.011864in}{1.826480in}}%
\pgfpathlineto{\pgfqpoint{5.018305in}{1.966631in}}%
\pgfpathlineto{\pgfqpoint{5.022599in}{2.029367in}}%
\pgfpathlineto{\pgfqpoint{5.026894in}{2.064460in}}%
\pgfpathlineto{\pgfqpoint{5.029041in}{2.071154in}}%
\pgfpathlineto{\pgfqpoint{5.031188in}{2.070500in}}%
\pgfpathlineto{\pgfqpoint{5.033335in}{2.062503in}}%
\pgfpathlineto{\pgfqpoint{5.037629in}{2.024878in}}%
\pgfpathlineto{\pgfqpoint{5.041923in}{1.959789in}}%
\pgfpathlineto{\pgfqpoint{5.048364in}{1.816650in}}%
\pgfpathlineto{\pgfqpoint{5.056953in}{1.561962in}}%
\pgfpathlineto{\pgfqpoint{5.080571in}{0.815493in}}%
\pgfpathlineto{\pgfqpoint{5.087012in}{0.685254in}}%
\pgfpathlineto{\pgfqpoint{5.091306in}{0.629998in}}%
\pgfpathlineto{\pgfqpoint{5.095600in}{0.602682in}}%
\pgfpathlineto{\pgfqpoint{5.097747in}{0.599881in}}%
\pgfpathlineto{\pgfqpoint{5.099895in}{0.604378in}}%
\pgfpathlineto{\pgfqpoint{5.102042in}{0.616131in}}%
\pgfpathlineto{\pgfqpoint{5.106336in}{0.660865in}}%
\pgfpathlineto{\pgfqpoint{5.110630in}{0.732326in}}%
\pgfpathlineto{\pgfqpoint{5.117071in}{0.883219in}}%
\pgfpathlineto{\pgfqpoint{5.125660in}{1.144198in}}%
\pgfpathlineto{\pgfqpoint{5.147131in}{1.829475in}}%
\pgfpathlineto{\pgfqpoint{5.153572in}{1.968697in}}%
\pgfpathlineto{\pgfqpoint{5.157866in}{2.030705in}}%
\pgfpathlineto{\pgfqpoint{5.162160in}{2.065016in}}%
\pgfpathlineto{\pgfqpoint{5.164307in}{2.071309in}}%
\pgfpathlineto{\pgfqpoint{5.166454in}{2.070252in}}%
\pgfpathlineto{\pgfqpoint{5.168601in}{2.061854in}}%
\pgfpathlineto{\pgfqpoint{5.172896in}{2.023451in}}%
\pgfpathlineto{\pgfqpoint{5.177190in}{1.957642in}}%
\pgfpathlineto{\pgfqpoint{5.183631in}{1.813593in}}%
\pgfpathlineto{\pgfqpoint{5.192219in}{1.558127in}}%
\pgfpathlineto{\pgfqpoint{5.215837in}{0.812659in}}%
\pgfpathlineto{\pgfqpoint{5.222279in}{0.683385in}}%
\pgfpathlineto{\pgfqpoint{5.226573in}{0.628870in}}%
\pgfpathlineto{\pgfqpoint{5.230867in}{0.602340in}}%
\pgfpathlineto{\pgfqpoint{5.233014in}{0.599938in}}%
\pgfpathlineto{\pgfqpoint{5.235161in}{0.604835in}}%
\pgfpathlineto{\pgfqpoint{5.237308in}{0.616982in}}%
\pgfpathlineto{\pgfqpoint{5.241602in}{0.662477in}}%
\pgfpathlineto{\pgfqpoint{5.245897in}{0.734635in}}%
\pgfpathlineto{\pgfqpoint{5.252338in}{0.886395in}}%
\pgfpathlineto{\pgfqpoint{5.260926in}{1.148084in}}%
\pgfpathlineto{\pgfqpoint{5.282397in}{1.832456in}}%
\pgfpathlineto{\pgfqpoint{5.288838in}{1.970745in}}%
\pgfpathlineto{\pgfqpoint{5.293133in}{2.032023in}}%
\pgfpathlineto{\pgfqpoint{5.297427in}{2.065550in}}%
\pgfpathlineto{\pgfqpoint{5.299574in}{2.071442in}}%
\pgfpathlineto{\pgfqpoint{5.301721in}{2.069981in}}%
\pgfpathlineto{\pgfqpoint{5.303868in}{2.061184in}}%
\pgfpathlineto{\pgfqpoint{5.308162in}{2.022004in}}%
\pgfpathlineto{\pgfqpoint{5.312456in}{1.955477in}}%
\pgfpathlineto{\pgfqpoint{5.318898in}{1.810521in}}%
\pgfpathlineto{\pgfqpoint{5.327486in}{1.554286in}}%
\pgfpathlineto{\pgfqpoint{5.348957in}{0.863687in}}%
\pgfpathlineto{\pgfqpoint{5.355398in}{0.718313in}}%
\pgfpathlineto{\pgfqpoint{5.359692in}{0.651253in}}%
\pgfpathlineto{\pgfqpoint{5.363987in}{0.611298in}}%
\pgfpathlineto{\pgfqpoint{5.366134in}{0.602019in}}%
\pgfpathlineto{\pgfqpoint{5.368281in}{0.600018in}}%
\pgfpathlineto{\pgfqpoint{5.370428in}{0.605314in}}%
\pgfpathlineto{\pgfqpoint{5.372575in}{0.617855in}}%
\pgfpathlineto{\pgfqpoint{5.376869in}{0.664108in}}%
\pgfpathlineto{\pgfqpoint{5.381163in}{0.736961in}}%
\pgfpathlineto{\pgfqpoint{5.387605in}{0.889585in}}%
\pgfpathlineto{\pgfqpoint{5.396193in}{1.151976in}}%
\pgfpathlineto{\pgfqpoint{5.417664in}{1.835422in}}%
\pgfpathlineto{\pgfqpoint{5.424105in}{1.972774in}}%
\pgfpathlineto{\pgfqpoint{5.428399in}{2.033319in}}%
\pgfpathlineto{\pgfqpoint{5.432693in}{2.066063in}}%
\pgfpathlineto{\pgfqpoint{5.434840in}{2.071552in}}%
\pgfpathlineto{\pgfqpoint{5.436988in}{2.069689in}}%
\pgfpathlineto{\pgfqpoint{5.439135in}{2.060491in}}%
\pgfpathlineto{\pgfqpoint{5.443429in}{2.020537in}}%
\pgfpathlineto{\pgfqpoint{5.447723in}{1.953293in}}%
\pgfpathlineto{\pgfqpoint{5.454164in}{1.807435in}}%
\pgfpathlineto{\pgfqpoint{5.462753in}{1.550438in}}%
\pgfpathlineto{\pgfqpoint{5.484224in}{0.860611in}}%
\pgfpathlineto{\pgfqpoint{5.490665in}{0.716136in}}%
\pgfpathlineto{\pgfqpoint{5.494959in}{0.649789in}}%
\pgfpathlineto{\pgfqpoint{5.499253in}{0.610603in}}%
\pgfpathlineto{\pgfqpoint{5.501400in}{0.601721in}}%
\pgfpathlineto{\pgfqpoint{5.503547in}{0.600119in}}%
\pgfpathlineto{\pgfqpoint{5.505694in}{0.605814in}}%
\pgfpathlineto{\pgfqpoint{5.507842in}{0.618749in}}%
\pgfpathlineto{\pgfqpoint{5.512136in}{0.665760in}}%
\pgfpathlineto{\pgfqpoint{5.516430in}{0.739306in}}%
\pgfpathlineto{\pgfqpoint{5.522871in}{0.892788in}}%
\pgfpathlineto{\pgfqpoint{5.531459in}{1.155873in}}%
\pgfpathlineto{\pgfqpoint{5.550783in}{1.782219in}}%
\pgfpathlineto{\pgfqpoint{5.557225in}{1.935141in}}%
\pgfpathlineto{\pgfqpoint{5.563666in}{2.034595in}}%
\pgfpathlineto{\pgfqpoint{5.567960in}{2.066553in}}%
\pgfpathlineto{\pgfqpoint{5.570107in}{2.071641in}}%
\pgfpathlineto{\pgfqpoint{5.572254in}{2.069374in}}%
\pgfpathlineto{\pgfqpoint{5.574401in}{2.059777in}}%
\pgfpathlineto{\pgfqpoint{5.578695in}{2.019049in}}%
\pgfpathlineto{\pgfqpoint{5.582990in}{1.951091in}}%
\pgfpathlineto{\pgfqpoint{5.589431in}{1.804335in}}%
\pgfpathlineto{\pgfqpoint{5.598019in}{1.546584in}}%
\pgfpathlineto{\pgfqpoint{5.619490in}{0.857549in}}%
\pgfpathlineto{\pgfqpoint{5.625931in}{0.713978in}}%
\pgfpathlineto{\pgfqpoint{5.630226in}{0.648344in}}%
\pgfpathlineto{\pgfqpoint{5.634520in}{0.609929in}}%
\pgfpathlineto{\pgfqpoint{5.636667in}{0.601444in}}%
\pgfpathlineto{\pgfqpoint{5.638814in}{0.600242in}}%
\pgfpathlineto{\pgfqpoint{5.640961in}{0.606336in}}%
\pgfpathlineto{\pgfqpoint{5.643108in}{0.619664in}}%
\pgfpathlineto{\pgfqpoint{5.647402in}{0.667432in}}%
\pgfpathlineto{\pgfqpoint{5.651696in}{0.741668in}}%
\pgfpathlineto{\pgfqpoint{5.658138in}{0.896004in}}%
\pgfpathlineto{\pgfqpoint{5.668873in}{1.231840in}}%
\pgfpathlineto{\pgfqpoint{5.686050in}{1.785415in}}%
\pgfpathlineto{\pgfqpoint{5.692491in}{1.937471in}}%
\pgfpathlineto{\pgfqpoint{5.698932in}{2.035849in}}%
\pgfpathlineto{\pgfqpoint{5.703227in}{2.067022in}}%
\pgfpathlineto{\pgfqpoint{5.705374in}{2.071707in}}%
\pgfpathlineto{\pgfqpoint{5.707521in}{2.069038in}}%
\pgfpathlineto{\pgfqpoint{5.709668in}{2.059041in}}%
\pgfpathlineto{\pgfqpoint{5.713962in}{2.017540in}}%
\pgfpathlineto{\pgfqpoint{5.718256in}{1.948870in}}%
\pgfpathlineto{\pgfqpoint{5.724698in}{1.801221in}}%
\pgfpathlineto{\pgfqpoint{5.733286in}{1.542723in}}%
\pgfpathlineto{\pgfqpoint{5.754757in}{0.854501in}}%
\pgfpathlineto{\pgfqpoint{5.761198in}{0.711838in}}%
\pgfpathlineto{\pgfqpoint{5.765492in}{0.646920in}}%
\pgfpathlineto{\pgfqpoint{5.769786in}{0.609277in}}%
\pgfpathlineto{\pgfqpoint{5.771933in}{0.601189in}}%
\pgfpathlineto{\pgfqpoint{5.774081in}{0.600387in}}%
\pgfpathlineto{\pgfqpoint{5.776228in}{0.606879in}}%
\pgfpathlineto{\pgfqpoint{5.780522in}{0.641417in}}%
\pgfpathlineto{\pgfqpoint{5.784816in}{0.703446in}}%
\pgfpathlineto{\pgfqpoint{5.791257in}{0.842427in}}%
\pgfpathlineto{\pgfqpoint{5.799846in}{1.093269in}}%
\pgfpathlineto{\pgfqpoint{5.823464in}{1.844230in}}%
\pgfpathlineto{\pgfqpoint{5.829905in}{1.978745in}}%
\pgfpathlineto{\pgfqpoint{5.834199in}{2.037083in}}%
\pgfpathlineto{\pgfqpoint{5.838493in}{2.067468in}}%
\pgfpathlineto{\pgfqpoint{5.840640in}{2.071751in}}%
\pgfpathlineto{\pgfqpoint{5.842787in}{2.068679in}}%
\pgfpathlineto{\pgfqpoint{5.844935in}{2.058284in}}%
\pgfpathlineto{\pgfqpoint{5.849229in}{2.016011in}}%
\pgfpathlineto{\pgfqpoint{5.853523in}{1.946631in}}%
\pgfpathlineto{\pgfqpoint{5.859964in}{1.798094in}}%
\pgfpathlineto{\pgfqpoint{5.868552in}{1.538856in}}%
\pgfpathlineto{\pgfqpoint{5.890023in}{0.851468in}}%
\pgfpathlineto{\pgfqpoint{5.896465in}{0.709717in}}%
\pgfpathlineto{\pgfqpoint{5.900759in}{0.645517in}}%
\pgfpathlineto{\pgfqpoint{5.905053in}{0.608646in}}%
\pgfpathlineto{\pgfqpoint{5.907200in}{0.600956in}}%
\pgfpathlineto{\pgfqpoint{5.909347in}{0.600554in}}%
\pgfpathlineto{\pgfqpoint{5.911494in}{0.607445in}}%
\pgfpathlineto{\pgfqpoint{5.915788in}{0.642759in}}%
\pgfpathlineto{\pgfqpoint{5.920083in}{0.705511in}}%
\pgfpathlineto{\pgfqpoint{5.926524in}{0.845416in}}%
\pgfpathlineto{\pgfqpoint{5.935112in}{1.097069in}}%
\pgfpathlineto{\pgfqpoint{5.958730in}{1.847136in}}%
\pgfpathlineto{\pgfqpoint{5.965172in}{1.980697in}}%
\pgfpathlineto{\pgfqpoint{5.969466in}{2.038296in}}%
\pgfpathlineto{\pgfqpoint{5.973760in}{2.067893in}}%
\pgfpathlineto{\pgfqpoint{5.975907in}{2.071773in}}%
\pgfpathlineto{\pgfqpoint{5.978054in}{2.068299in}}%
\pgfpathlineto{\pgfqpoint{5.980201in}{2.057504in}}%
\pgfpathlineto{\pgfqpoint{5.984495in}{2.014462in}}%
\pgfpathlineto{\pgfqpoint{5.988789in}{1.944374in}}%
\pgfpathlineto{\pgfqpoint{5.995231in}{1.794952in}}%
\pgfpathlineto{\pgfqpoint{6.003819in}{1.534983in}}%
\pgfpathlineto{\pgfqpoint{6.025290in}{0.848449in}}%
\pgfpathlineto{\pgfqpoint{6.031731in}{0.707615in}}%
\pgfpathlineto{\pgfqpoint{6.036025in}{0.644134in}}%
\pgfpathlineto{\pgfqpoint{6.040320in}{0.608037in}}%
\pgfpathlineto{\pgfqpoint{6.042467in}{0.600745in}}%
\pgfpathlineto{\pgfqpoint{6.044614in}{0.600743in}}%
\pgfpathlineto{\pgfqpoint{6.046761in}{0.608032in}}%
\pgfpathlineto{\pgfqpoint{6.051055in}{0.644121in}}%
\pgfpathlineto{\pgfqpoint{6.055349in}{0.707595in}}%
\pgfpathlineto{\pgfqpoint{6.061791in}{0.848420in}}%
\pgfpathlineto{\pgfqpoint{6.070379in}{1.100877in}}%
\pgfpathlineto{\pgfqpoint{6.093997in}{1.850026in}}%
\pgfpathlineto{\pgfqpoint{6.100438in}{1.982629in}}%
\pgfpathlineto{\pgfqpoint{6.104732in}{2.039487in}}%
\pgfpathlineto{\pgfqpoint{6.109026in}{2.068295in}}%
\pgfpathlineto{\pgfqpoint{6.111174in}{2.071774in}}%
\pgfpathlineto{\pgfqpoint{6.113321in}{2.067897in}}%
\pgfpathlineto{\pgfqpoint{6.115468in}{2.056703in}}%
\pgfpathlineto{\pgfqpoint{6.119762in}{2.012892in}}%
\pgfpathlineto{\pgfqpoint{6.124056in}{1.942098in}}%
\pgfpathlineto{\pgfqpoint{6.130497in}{1.791796in}}%
\pgfpathlineto{\pgfqpoint{6.139086in}{1.531105in}}%
\pgfpathlineto{\pgfqpoint{6.160557in}{0.845445in}}%
\pgfpathlineto{\pgfqpoint{6.160557in}{0.845445in}}%
\pgfusepath{stroke}%
\end{pgfscope}%
\begin{pgfscope}%
\pgfsetrectcap%
\pgfsetmiterjoin%
\pgfsetlinewidth{0.803000pt}%
\definecolor{currentstroke}{rgb}{0.000000,0.000000,0.000000}%
\pgfsetstrokecolor{currentstroke}%
\pgfsetdash{}{0pt}%
\pgfpathmoveto{\pgfqpoint{3.906113in}{0.526234in}}%
\pgfpathlineto{\pgfqpoint{3.906113in}{2.145371in}}%
\pgfusepath{stroke}%
\end{pgfscope}%
\begin{pgfscope}%
\pgfsetrectcap%
\pgfsetmiterjoin%
\pgfsetlinewidth{0.803000pt}%
\definecolor{currentstroke}{rgb}{0.000000,0.000000,0.000000}%
\pgfsetstrokecolor{currentstroke}%
\pgfsetdash{}{0pt}%
\pgfpathmoveto{\pgfqpoint{6.267911in}{0.526234in}}%
\pgfpathlineto{\pgfqpoint{6.267911in}{2.145371in}}%
\pgfusepath{stroke}%
\end{pgfscope}%
\begin{pgfscope}%
\pgfsetrectcap%
\pgfsetmiterjoin%
\pgfsetlinewidth{0.803000pt}%
\definecolor{currentstroke}{rgb}{0.000000,0.000000,0.000000}%
\pgfsetstrokecolor{currentstroke}%
\pgfsetdash{}{0pt}%
\pgfpathmoveto{\pgfqpoint{3.906113in}{0.526234in}}%
\pgfpathlineto{\pgfqpoint{6.267911in}{0.526234in}}%
\pgfusepath{stroke}%
\end{pgfscope}%
\begin{pgfscope}%
\pgfsetrectcap%
\pgfsetmiterjoin%
\pgfsetlinewidth{0.803000pt}%
\definecolor{currentstroke}{rgb}{0.000000,0.000000,0.000000}%
\pgfsetstrokecolor{currentstroke}%
\pgfsetdash{}{0pt}%
\pgfpathmoveto{\pgfqpoint{3.906113in}{2.145371in}}%
\pgfpathlineto{\pgfqpoint{6.267911in}{2.145371in}}%
\pgfusepath{stroke}%
\end{pgfscope}%
\begin{pgfscope}%
\definecolor{textcolor}{rgb}{0.000000,0.000000,0.000000}%
\pgfsetstrokecolor{textcolor}%
\pgfsetfillcolor{textcolor}%
\pgftext[x=5.087012in,y=2.228704in,,base]{\color{textcolor}\rmfamily\fontsize{12.000000}{14.400000}\selectfont energy}%
\end{pgfscope}%
\end{pgfpicture}%
\makeatother%
\endgroup%
}
           \caption{Symplectic Integration Method with $\theta$ small}
           \label{fig:symplectic}
        \end{center}
    \end{figure}
    
    \noindent
    The code for this entire section is in Appendix 2 and it can be seen that it is practically 
    identical to the first section with the exception of the loop, which can be seen below.
    \newline
    \lstinputlisting[title=Symplectic Euler's Method Loop, style=inline, linerange=25-29, firstnumber=25]{CP3b.py}
    
    \noindent
    As can be seen in Figure \ref{fig:symplectic}, this method is much better at keeping the 
    energy bounded, with a total variation of $\approx 0.2$. Compared to the previous results, 
    this is a great improvement in terms of accurately approximating the motion. We can again 
    notice that the phase plot is purely elliptical, with no decay or growth, showing that our 
    energy is conserved to some extent. We have effectively removed the approximation from the 
    calculation, rather approximating the entire problem but accurately calculating the values. 
    \newline
    \newline
    Finally, we will investigate the dependence of period on either the amplitude or the length. 
    In our case, we will investigate length. In order to do this, we have written some code, 
    Appendix 3, that is in most senses the exact same as the first two sections with a few extra 
    arrays and calculations added. Most importantly, we added an extra loop so that we can simulate 
    the motion of the pendulum for a range of different lengths, namely 1m to 100m, evaluating every 
    metre. From this, we performed a Levenberg-Marquardt fit using the \texttt{curve\_fit} function 
    in \texttt{scipy.optimize} and could then extract the angular frequency and from that find the 
    period. Below are the most important lines from the code.
    \newline
    \lstinputlisting[title=Length Dependence of Period Code, style=inline, linerange=32-41, firstnumber=32]{CP3c.py}
    
    \noindent
    Line 40 is the calculation of the observed period and line 41 is a calculation of the expected 
    period given the angular frequency $\Omega_0$ supplied in line 37. Below, Figure \ref{fig:lengthperiod90}, 
    is the plot of the observed period and the predicted period. Importantly, the predicted period is 
    based off the small angle approximation, where $T = \frac{2\pi}{\Omega_0}$.

    \begin{figure}[H]
        \begin{center}
           \scalebox{.7}{%% Creator: Matplotlib, PGF backend
%%
%% To include the figure in your LaTeX document, write
%%   \input{<filename>.pgf}
%%
%% Make sure the required packages are loaded in your preamble
%%   \usepackage{pgf}
%%
%% Figures using additional raster images can only be included by \input if
%% they are in the same directory as the main LaTeX file. For loading figures
%% from other directories you can use the `import` package
%%   \usepackage{import}
%% and then include the figures with
%%   \import{<path to file>}{<filename>.pgf}
%%
%% Matplotlib used the following preamble
%%
\begingroup%
\makeatletter%
\begin{pgfpicture}%
\pgfpathrectangle{\pgfpointorigin}{\pgfqpoint{6.400000in}{4.800000in}}%
\pgfusepath{use as bounding box, clip}%
\begin{pgfscope}%
\pgfsetbuttcap%
\pgfsetmiterjoin%
\definecolor{currentfill}{rgb}{1.000000,1.000000,1.000000}%
\pgfsetfillcolor{currentfill}%
\pgfsetlinewidth{0.000000pt}%
\definecolor{currentstroke}{rgb}{1.000000,1.000000,1.000000}%
\pgfsetstrokecolor{currentstroke}%
\pgfsetdash{}{0pt}%
\pgfpathmoveto{\pgfqpoint{0.000000in}{0.000000in}}%
\pgfpathlineto{\pgfqpoint{6.400000in}{0.000000in}}%
\pgfpathlineto{\pgfqpoint{6.400000in}{4.800000in}}%
\pgfpathlineto{\pgfqpoint{0.000000in}{4.800000in}}%
\pgfpathclose%
\pgfusepath{fill}%
\end{pgfscope}%
\begin{pgfscope}%
\pgfsetbuttcap%
\pgfsetmiterjoin%
\definecolor{currentfill}{rgb}{1.000000,1.000000,1.000000}%
\pgfsetfillcolor{currentfill}%
\pgfsetlinewidth{0.000000pt}%
\definecolor{currentstroke}{rgb}{0.000000,0.000000,0.000000}%
\pgfsetstrokecolor{currentstroke}%
\pgfsetstrokeopacity{0.000000}%
\pgfsetdash{}{0pt}%
\pgfpathmoveto{\pgfqpoint{0.800000in}{0.528000in}}%
\pgfpathlineto{\pgfqpoint{5.760000in}{0.528000in}}%
\pgfpathlineto{\pgfqpoint{5.760000in}{4.224000in}}%
\pgfpathlineto{\pgfqpoint{0.800000in}{4.224000in}}%
\pgfpathclose%
\pgfusepath{fill}%
\end{pgfscope}%
\begin{pgfscope}%
\pgfsetbuttcap%
\pgfsetroundjoin%
\definecolor{currentfill}{rgb}{0.000000,0.000000,0.000000}%
\pgfsetfillcolor{currentfill}%
\pgfsetlinewidth{0.803000pt}%
\definecolor{currentstroke}{rgb}{0.000000,0.000000,0.000000}%
\pgfsetstrokecolor{currentstroke}%
\pgfsetdash{}{0pt}%
\pgfsys@defobject{currentmarker}{\pgfqpoint{0.000000in}{-0.048611in}}{\pgfqpoint{0.000000in}{0.000000in}}{%
\pgfpathmoveto{\pgfqpoint{0.000000in}{0.000000in}}%
\pgfpathlineto{\pgfqpoint{0.000000in}{-0.048611in}}%
\pgfusepath{stroke,fill}%
}%
\begin{pgfscope}%
\pgfsys@transformshift{0.979908in}{0.528000in}%
\pgfsys@useobject{currentmarker}{}%
\end{pgfscope}%
\end{pgfscope}%
\begin{pgfscope}%
\definecolor{textcolor}{rgb}{0.000000,0.000000,0.000000}%
\pgfsetstrokecolor{textcolor}%
\pgfsetfillcolor{textcolor}%
\pgftext[x=0.979908in,y=0.430778in,,top]{\color{textcolor}\rmfamily\fontsize{10.000000}{12.000000}\selectfont \(\displaystyle 0\)}%
\end{pgfscope}%
\begin{pgfscope}%
\pgfsetbuttcap%
\pgfsetroundjoin%
\definecolor{currentfill}{rgb}{0.000000,0.000000,0.000000}%
\pgfsetfillcolor{currentfill}%
\pgfsetlinewidth{0.803000pt}%
\definecolor{currentstroke}{rgb}{0.000000,0.000000,0.000000}%
\pgfsetstrokecolor{currentstroke}%
\pgfsetdash{}{0pt}%
\pgfsys@defobject{currentmarker}{\pgfqpoint{0.000000in}{-0.048611in}}{\pgfqpoint{0.000000in}{0.000000in}}{%
\pgfpathmoveto{\pgfqpoint{0.000000in}{0.000000in}}%
\pgfpathlineto{\pgfqpoint{0.000000in}{-0.048611in}}%
\pgfusepath{stroke,fill}%
}%
\begin{pgfscope}%
\pgfsys@transformshift{1.890836in}{0.528000in}%
\pgfsys@useobject{currentmarker}{}%
\end{pgfscope}%
\end{pgfscope}%
\begin{pgfscope}%
\definecolor{textcolor}{rgb}{0.000000,0.000000,0.000000}%
\pgfsetstrokecolor{textcolor}%
\pgfsetfillcolor{textcolor}%
\pgftext[x=1.890836in,y=0.430778in,,top]{\color{textcolor}\rmfamily\fontsize{10.000000}{12.000000}\selectfont \(\displaystyle 20\)}%
\end{pgfscope}%
\begin{pgfscope}%
\pgfsetbuttcap%
\pgfsetroundjoin%
\definecolor{currentfill}{rgb}{0.000000,0.000000,0.000000}%
\pgfsetfillcolor{currentfill}%
\pgfsetlinewidth{0.803000pt}%
\definecolor{currentstroke}{rgb}{0.000000,0.000000,0.000000}%
\pgfsetstrokecolor{currentstroke}%
\pgfsetdash{}{0pt}%
\pgfsys@defobject{currentmarker}{\pgfqpoint{0.000000in}{-0.048611in}}{\pgfqpoint{0.000000in}{0.000000in}}{%
\pgfpathmoveto{\pgfqpoint{0.000000in}{0.000000in}}%
\pgfpathlineto{\pgfqpoint{0.000000in}{-0.048611in}}%
\pgfusepath{stroke,fill}%
}%
\begin{pgfscope}%
\pgfsys@transformshift{2.801763in}{0.528000in}%
\pgfsys@useobject{currentmarker}{}%
\end{pgfscope}%
\end{pgfscope}%
\begin{pgfscope}%
\definecolor{textcolor}{rgb}{0.000000,0.000000,0.000000}%
\pgfsetstrokecolor{textcolor}%
\pgfsetfillcolor{textcolor}%
\pgftext[x=2.801763in,y=0.430778in,,top]{\color{textcolor}\rmfamily\fontsize{10.000000}{12.000000}\selectfont \(\displaystyle 40\)}%
\end{pgfscope}%
\begin{pgfscope}%
\pgfsetbuttcap%
\pgfsetroundjoin%
\definecolor{currentfill}{rgb}{0.000000,0.000000,0.000000}%
\pgfsetfillcolor{currentfill}%
\pgfsetlinewidth{0.803000pt}%
\definecolor{currentstroke}{rgb}{0.000000,0.000000,0.000000}%
\pgfsetstrokecolor{currentstroke}%
\pgfsetdash{}{0pt}%
\pgfsys@defobject{currentmarker}{\pgfqpoint{0.000000in}{-0.048611in}}{\pgfqpoint{0.000000in}{0.000000in}}{%
\pgfpathmoveto{\pgfqpoint{0.000000in}{0.000000in}}%
\pgfpathlineto{\pgfqpoint{0.000000in}{-0.048611in}}%
\pgfusepath{stroke,fill}%
}%
\begin{pgfscope}%
\pgfsys@transformshift{3.712691in}{0.528000in}%
\pgfsys@useobject{currentmarker}{}%
\end{pgfscope}%
\end{pgfscope}%
\begin{pgfscope}%
\definecolor{textcolor}{rgb}{0.000000,0.000000,0.000000}%
\pgfsetstrokecolor{textcolor}%
\pgfsetfillcolor{textcolor}%
\pgftext[x=3.712691in,y=0.430778in,,top]{\color{textcolor}\rmfamily\fontsize{10.000000}{12.000000}\selectfont \(\displaystyle 60\)}%
\end{pgfscope}%
\begin{pgfscope}%
\pgfsetbuttcap%
\pgfsetroundjoin%
\definecolor{currentfill}{rgb}{0.000000,0.000000,0.000000}%
\pgfsetfillcolor{currentfill}%
\pgfsetlinewidth{0.803000pt}%
\definecolor{currentstroke}{rgb}{0.000000,0.000000,0.000000}%
\pgfsetstrokecolor{currentstroke}%
\pgfsetdash{}{0pt}%
\pgfsys@defobject{currentmarker}{\pgfqpoint{0.000000in}{-0.048611in}}{\pgfqpoint{0.000000in}{0.000000in}}{%
\pgfpathmoveto{\pgfqpoint{0.000000in}{0.000000in}}%
\pgfpathlineto{\pgfqpoint{0.000000in}{-0.048611in}}%
\pgfusepath{stroke,fill}%
}%
\begin{pgfscope}%
\pgfsys@transformshift{4.623618in}{0.528000in}%
\pgfsys@useobject{currentmarker}{}%
\end{pgfscope}%
\end{pgfscope}%
\begin{pgfscope}%
\definecolor{textcolor}{rgb}{0.000000,0.000000,0.000000}%
\pgfsetstrokecolor{textcolor}%
\pgfsetfillcolor{textcolor}%
\pgftext[x=4.623618in,y=0.430778in,,top]{\color{textcolor}\rmfamily\fontsize{10.000000}{12.000000}\selectfont \(\displaystyle 80\)}%
\end{pgfscope}%
\begin{pgfscope}%
\pgfsetbuttcap%
\pgfsetroundjoin%
\definecolor{currentfill}{rgb}{0.000000,0.000000,0.000000}%
\pgfsetfillcolor{currentfill}%
\pgfsetlinewidth{0.803000pt}%
\definecolor{currentstroke}{rgb}{0.000000,0.000000,0.000000}%
\pgfsetstrokecolor{currentstroke}%
\pgfsetdash{}{0pt}%
\pgfsys@defobject{currentmarker}{\pgfqpoint{0.000000in}{-0.048611in}}{\pgfqpoint{0.000000in}{0.000000in}}{%
\pgfpathmoveto{\pgfqpoint{0.000000in}{0.000000in}}%
\pgfpathlineto{\pgfqpoint{0.000000in}{-0.048611in}}%
\pgfusepath{stroke,fill}%
}%
\begin{pgfscope}%
\pgfsys@transformshift{5.534545in}{0.528000in}%
\pgfsys@useobject{currentmarker}{}%
\end{pgfscope}%
\end{pgfscope}%
\begin{pgfscope}%
\definecolor{textcolor}{rgb}{0.000000,0.000000,0.000000}%
\pgfsetstrokecolor{textcolor}%
\pgfsetfillcolor{textcolor}%
\pgftext[x=5.534545in,y=0.430778in,,top]{\color{textcolor}\rmfamily\fontsize{10.000000}{12.000000}\selectfont \(\displaystyle 100\)}%
\end{pgfscope}%
\begin{pgfscope}%
\definecolor{textcolor}{rgb}{0.000000,0.000000,0.000000}%
\pgfsetstrokecolor{textcolor}%
\pgfsetfillcolor{textcolor}%
\pgftext[x=3.280000in,y=0.251766in,,top]{\color{textcolor}\rmfamily\fontsize{10.000000}{12.000000}\selectfont Length (m)}%
\end{pgfscope}%
\begin{pgfscope}%
\pgfsetbuttcap%
\pgfsetroundjoin%
\definecolor{currentfill}{rgb}{0.000000,0.000000,0.000000}%
\pgfsetfillcolor{currentfill}%
\pgfsetlinewidth{0.803000pt}%
\definecolor{currentstroke}{rgb}{0.000000,0.000000,0.000000}%
\pgfsetstrokecolor{currentstroke}%
\pgfsetdash{}{0pt}%
\pgfsys@defobject{currentmarker}{\pgfqpoint{-0.048611in}{0.000000in}}{\pgfqpoint{0.000000in}{0.000000in}}{%
\pgfpathmoveto{\pgfqpoint{0.000000in}{0.000000in}}%
\pgfpathlineto{\pgfqpoint{-0.048611in}{0.000000in}}%
\pgfusepath{stroke,fill}%
}%
\begin{pgfscope}%
\pgfsys@transformshift{0.800000in}{0.777112in}%
\pgfsys@useobject{currentmarker}{}%
\end{pgfscope}%
\end{pgfscope}%
\begin{pgfscope}%
\definecolor{textcolor}{rgb}{0.000000,0.000000,0.000000}%
\pgfsetstrokecolor{textcolor}%
\pgfsetfillcolor{textcolor}%
\pgftext[x=0.525308in,y=0.728887in,left,base]{\color{textcolor}\rmfamily\fontsize{10.000000}{12.000000}\selectfont \(\displaystyle 2.5\)}%
\end{pgfscope}%
\begin{pgfscope}%
\pgfsetbuttcap%
\pgfsetroundjoin%
\definecolor{currentfill}{rgb}{0.000000,0.000000,0.000000}%
\pgfsetfillcolor{currentfill}%
\pgfsetlinewidth{0.803000pt}%
\definecolor{currentstroke}{rgb}{0.000000,0.000000,0.000000}%
\pgfsetstrokecolor{currentstroke}%
\pgfsetdash{}{0pt}%
\pgfsys@defobject{currentmarker}{\pgfqpoint{-0.048611in}{0.000000in}}{\pgfqpoint{0.000000in}{0.000000in}}{%
\pgfpathmoveto{\pgfqpoint{0.000000in}{0.000000in}}%
\pgfpathlineto{\pgfqpoint{-0.048611in}{0.000000in}}%
\pgfusepath{stroke,fill}%
}%
\begin{pgfscope}%
\pgfsys@transformshift{0.800000in}{1.188507in}%
\pgfsys@useobject{currentmarker}{}%
\end{pgfscope}%
\end{pgfscope}%
\begin{pgfscope}%
\definecolor{textcolor}{rgb}{0.000000,0.000000,0.000000}%
\pgfsetstrokecolor{textcolor}%
\pgfsetfillcolor{textcolor}%
\pgftext[x=0.525308in,y=1.140282in,left,base]{\color{textcolor}\rmfamily\fontsize{10.000000}{12.000000}\selectfont \(\displaystyle 5.0\)}%
\end{pgfscope}%
\begin{pgfscope}%
\pgfsetbuttcap%
\pgfsetroundjoin%
\definecolor{currentfill}{rgb}{0.000000,0.000000,0.000000}%
\pgfsetfillcolor{currentfill}%
\pgfsetlinewidth{0.803000pt}%
\definecolor{currentstroke}{rgb}{0.000000,0.000000,0.000000}%
\pgfsetstrokecolor{currentstroke}%
\pgfsetdash{}{0pt}%
\pgfsys@defobject{currentmarker}{\pgfqpoint{-0.048611in}{0.000000in}}{\pgfqpoint{0.000000in}{0.000000in}}{%
\pgfpathmoveto{\pgfqpoint{0.000000in}{0.000000in}}%
\pgfpathlineto{\pgfqpoint{-0.048611in}{0.000000in}}%
\pgfusepath{stroke,fill}%
}%
\begin{pgfscope}%
\pgfsys@transformshift{0.800000in}{1.599902in}%
\pgfsys@useobject{currentmarker}{}%
\end{pgfscope}%
\end{pgfscope}%
\begin{pgfscope}%
\definecolor{textcolor}{rgb}{0.000000,0.000000,0.000000}%
\pgfsetstrokecolor{textcolor}%
\pgfsetfillcolor{textcolor}%
\pgftext[x=0.525308in,y=1.551676in,left,base]{\color{textcolor}\rmfamily\fontsize{10.000000}{12.000000}\selectfont \(\displaystyle 7.5\)}%
\end{pgfscope}%
\begin{pgfscope}%
\pgfsetbuttcap%
\pgfsetroundjoin%
\definecolor{currentfill}{rgb}{0.000000,0.000000,0.000000}%
\pgfsetfillcolor{currentfill}%
\pgfsetlinewidth{0.803000pt}%
\definecolor{currentstroke}{rgb}{0.000000,0.000000,0.000000}%
\pgfsetstrokecolor{currentstroke}%
\pgfsetdash{}{0pt}%
\pgfsys@defobject{currentmarker}{\pgfqpoint{-0.048611in}{0.000000in}}{\pgfqpoint{0.000000in}{0.000000in}}{%
\pgfpathmoveto{\pgfqpoint{0.000000in}{0.000000in}}%
\pgfpathlineto{\pgfqpoint{-0.048611in}{0.000000in}}%
\pgfusepath{stroke,fill}%
}%
\begin{pgfscope}%
\pgfsys@transformshift{0.800000in}{2.011297in}%
\pgfsys@useobject{currentmarker}{}%
\end{pgfscope}%
\end{pgfscope}%
\begin{pgfscope}%
\definecolor{textcolor}{rgb}{0.000000,0.000000,0.000000}%
\pgfsetstrokecolor{textcolor}%
\pgfsetfillcolor{textcolor}%
\pgftext[x=0.455863in,y=1.963071in,left,base]{\color{textcolor}\rmfamily\fontsize{10.000000}{12.000000}\selectfont \(\displaystyle 10.0\)}%
\end{pgfscope}%
\begin{pgfscope}%
\pgfsetbuttcap%
\pgfsetroundjoin%
\definecolor{currentfill}{rgb}{0.000000,0.000000,0.000000}%
\pgfsetfillcolor{currentfill}%
\pgfsetlinewidth{0.803000pt}%
\definecolor{currentstroke}{rgb}{0.000000,0.000000,0.000000}%
\pgfsetstrokecolor{currentstroke}%
\pgfsetdash{}{0pt}%
\pgfsys@defobject{currentmarker}{\pgfqpoint{-0.048611in}{0.000000in}}{\pgfqpoint{0.000000in}{0.000000in}}{%
\pgfpathmoveto{\pgfqpoint{0.000000in}{0.000000in}}%
\pgfpathlineto{\pgfqpoint{-0.048611in}{0.000000in}}%
\pgfusepath{stroke,fill}%
}%
\begin{pgfscope}%
\pgfsys@transformshift{0.800000in}{2.422691in}%
\pgfsys@useobject{currentmarker}{}%
\end{pgfscope}%
\end{pgfscope}%
\begin{pgfscope}%
\definecolor{textcolor}{rgb}{0.000000,0.000000,0.000000}%
\pgfsetstrokecolor{textcolor}%
\pgfsetfillcolor{textcolor}%
\pgftext[x=0.455863in,y=2.374466in,left,base]{\color{textcolor}\rmfamily\fontsize{10.000000}{12.000000}\selectfont \(\displaystyle 12.5\)}%
\end{pgfscope}%
\begin{pgfscope}%
\pgfsetbuttcap%
\pgfsetroundjoin%
\definecolor{currentfill}{rgb}{0.000000,0.000000,0.000000}%
\pgfsetfillcolor{currentfill}%
\pgfsetlinewidth{0.803000pt}%
\definecolor{currentstroke}{rgb}{0.000000,0.000000,0.000000}%
\pgfsetstrokecolor{currentstroke}%
\pgfsetdash{}{0pt}%
\pgfsys@defobject{currentmarker}{\pgfqpoint{-0.048611in}{0.000000in}}{\pgfqpoint{0.000000in}{0.000000in}}{%
\pgfpathmoveto{\pgfqpoint{0.000000in}{0.000000in}}%
\pgfpathlineto{\pgfqpoint{-0.048611in}{0.000000in}}%
\pgfusepath{stroke,fill}%
}%
\begin{pgfscope}%
\pgfsys@transformshift{0.800000in}{2.834086in}%
\pgfsys@useobject{currentmarker}{}%
\end{pgfscope}%
\end{pgfscope}%
\begin{pgfscope}%
\definecolor{textcolor}{rgb}{0.000000,0.000000,0.000000}%
\pgfsetstrokecolor{textcolor}%
\pgfsetfillcolor{textcolor}%
\pgftext[x=0.455863in,y=2.785861in,left,base]{\color{textcolor}\rmfamily\fontsize{10.000000}{12.000000}\selectfont \(\displaystyle 15.0\)}%
\end{pgfscope}%
\begin{pgfscope}%
\pgfsetbuttcap%
\pgfsetroundjoin%
\definecolor{currentfill}{rgb}{0.000000,0.000000,0.000000}%
\pgfsetfillcolor{currentfill}%
\pgfsetlinewidth{0.803000pt}%
\definecolor{currentstroke}{rgb}{0.000000,0.000000,0.000000}%
\pgfsetstrokecolor{currentstroke}%
\pgfsetdash{}{0pt}%
\pgfsys@defobject{currentmarker}{\pgfqpoint{-0.048611in}{0.000000in}}{\pgfqpoint{0.000000in}{0.000000in}}{%
\pgfpathmoveto{\pgfqpoint{0.000000in}{0.000000in}}%
\pgfpathlineto{\pgfqpoint{-0.048611in}{0.000000in}}%
\pgfusepath{stroke,fill}%
}%
\begin{pgfscope}%
\pgfsys@transformshift{0.800000in}{3.245481in}%
\pgfsys@useobject{currentmarker}{}%
\end{pgfscope}%
\end{pgfscope}%
\begin{pgfscope}%
\definecolor{textcolor}{rgb}{0.000000,0.000000,0.000000}%
\pgfsetstrokecolor{textcolor}%
\pgfsetfillcolor{textcolor}%
\pgftext[x=0.455863in,y=3.197256in,left,base]{\color{textcolor}\rmfamily\fontsize{10.000000}{12.000000}\selectfont \(\displaystyle 17.5\)}%
\end{pgfscope}%
\begin{pgfscope}%
\pgfsetbuttcap%
\pgfsetroundjoin%
\definecolor{currentfill}{rgb}{0.000000,0.000000,0.000000}%
\pgfsetfillcolor{currentfill}%
\pgfsetlinewidth{0.803000pt}%
\definecolor{currentstroke}{rgb}{0.000000,0.000000,0.000000}%
\pgfsetstrokecolor{currentstroke}%
\pgfsetdash{}{0pt}%
\pgfsys@defobject{currentmarker}{\pgfqpoint{-0.048611in}{0.000000in}}{\pgfqpoint{0.000000in}{0.000000in}}{%
\pgfpathmoveto{\pgfqpoint{0.000000in}{0.000000in}}%
\pgfpathlineto{\pgfqpoint{-0.048611in}{0.000000in}}%
\pgfusepath{stroke,fill}%
}%
\begin{pgfscope}%
\pgfsys@transformshift{0.800000in}{3.656876in}%
\pgfsys@useobject{currentmarker}{}%
\end{pgfscope}%
\end{pgfscope}%
\begin{pgfscope}%
\definecolor{textcolor}{rgb}{0.000000,0.000000,0.000000}%
\pgfsetstrokecolor{textcolor}%
\pgfsetfillcolor{textcolor}%
\pgftext[x=0.455863in,y=3.608650in,left,base]{\color{textcolor}\rmfamily\fontsize{10.000000}{12.000000}\selectfont \(\displaystyle 20.0\)}%
\end{pgfscope}%
\begin{pgfscope}%
\pgfsetbuttcap%
\pgfsetroundjoin%
\definecolor{currentfill}{rgb}{0.000000,0.000000,0.000000}%
\pgfsetfillcolor{currentfill}%
\pgfsetlinewidth{0.803000pt}%
\definecolor{currentstroke}{rgb}{0.000000,0.000000,0.000000}%
\pgfsetstrokecolor{currentstroke}%
\pgfsetdash{}{0pt}%
\pgfsys@defobject{currentmarker}{\pgfqpoint{-0.048611in}{0.000000in}}{\pgfqpoint{0.000000in}{0.000000in}}{%
\pgfpathmoveto{\pgfqpoint{0.000000in}{0.000000in}}%
\pgfpathlineto{\pgfqpoint{-0.048611in}{0.000000in}}%
\pgfusepath{stroke,fill}%
}%
\begin{pgfscope}%
\pgfsys@transformshift{0.800000in}{4.068270in}%
\pgfsys@useobject{currentmarker}{}%
\end{pgfscope}%
\end{pgfscope}%
\begin{pgfscope}%
\definecolor{textcolor}{rgb}{0.000000,0.000000,0.000000}%
\pgfsetstrokecolor{textcolor}%
\pgfsetfillcolor{textcolor}%
\pgftext[x=0.455863in,y=4.020045in,left,base]{\color{textcolor}\rmfamily\fontsize{10.000000}{12.000000}\selectfont \(\displaystyle 22.5\)}%
\end{pgfscope}%
\begin{pgfscope}%
\definecolor{textcolor}{rgb}{0.000000,0.000000,0.000000}%
\pgfsetstrokecolor{textcolor}%
\pgfsetfillcolor{textcolor}%
\pgftext[x=0.400308in,y=2.376000in,,bottom,rotate=90.000000]{\color{textcolor}\rmfamily\fontsize{10.000000}{12.000000}\selectfont Period T (s)}%
\end{pgfscope}%
\begin{pgfscope}%
\pgfpathrectangle{\pgfqpoint{0.800000in}{0.528000in}}{\pgfqpoint{4.960000in}{3.696000in}}%
\pgfusepath{clip}%
\pgfsetrectcap%
\pgfsetroundjoin%
\pgfsetlinewidth{1.505625pt}%
\definecolor{currentstroke}{rgb}{0.000000,0.000000,1.000000}%
\pgfsetstrokecolor{currentstroke}%
\pgfsetdash{}{0pt}%
\pgfpathmoveto{\pgfqpoint{1.025455in}{0.755586in}}%
\pgfpathlineto{\pgfqpoint{1.071001in}{0.916972in}}%
\pgfpathlineto{\pgfqpoint{1.116547in}{1.040804in}}%
\pgfpathlineto{\pgfqpoint{1.162094in}{1.145171in}}%
\pgfpathlineto{\pgfqpoint{1.207640in}{1.237110in}}%
\pgfpathlineto{\pgfqpoint{1.253186in}{1.321007in}}%
\pgfpathlineto{\pgfqpoint{1.298733in}{1.396502in}}%
\pgfpathlineto{\pgfqpoint{1.344279in}{1.467758in}}%
\pgfpathlineto{\pgfqpoint{1.389826in}{1.534439in}}%
\pgfpathlineto{\pgfqpoint{1.435372in}{1.598025in}}%
\pgfpathlineto{\pgfqpoint{1.480918in}{1.659351in}}%
\pgfpathlineto{\pgfqpoint{1.526465in}{1.717247in}}%
\pgfpathlineto{\pgfqpoint{1.572011in}{1.771417in}}%
\pgfpathlineto{\pgfqpoint{1.617557in}{1.822991in}}%
\pgfpathlineto{\pgfqpoint{1.663104in}{1.873143in}}%
\pgfpathlineto{\pgfqpoint{1.708650in}{1.922335in}}%
\pgfpathlineto{\pgfqpoint{1.754197in}{1.970467in}}%
\pgfpathlineto{\pgfqpoint{1.799743in}{2.017292in}}%
\pgfpathlineto{\pgfqpoint{1.845289in}{2.062673in}}%
\pgfpathlineto{\pgfqpoint{1.890836in}{2.106640in}}%
\pgfpathlineto{\pgfqpoint{1.936382in}{2.149352in}}%
\pgfpathlineto{\pgfqpoint{1.981928in}{2.191025in}}%
\pgfpathlineto{\pgfqpoint{2.027475in}{2.231877in}}%
\pgfpathlineto{\pgfqpoint{2.073021in}{2.272091in}}%
\pgfpathlineto{\pgfqpoint{2.118567in}{2.311794in}}%
\pgfpathlineto{\pgfqpoint{2.164114in}{2.351057in}}%
\pgfpathlineto{\pgfqpoint{2.209660in}{2.389897in}}%
\pgfpathlineto{\pgfqpoint{2.255207in}{2.428286in}}%
\pgfpathlineto{\pgfqpoint{2.300753in}{2.466171in}}%
\pgfpathlineto{\pgfqpoint{2.346299in}{2.503478in}}%
\pgfpathlineto{\pgfqpoint{2.391846in}{2.540128in}}%
\pgfpathlineto{\pgfqpoint{2.437392in}{2.576045in}}%
\pgfpathlineto{\pgfqpoint{2.482938in}{2.611163in}}%
\pgfpathlineto{\pgfqpoint{2.528485in}{2.645426in}}%
\pgfpathlineto{\pgfqpoint{2.574031in}{2.678794in}}%
\pgfpathlineto{\pgfqpoint{2.619578in}{2.711242in}}%
\pgfpathlineto{\pgfqpoint{2.665124in}{2.742761in}}%
\pgfpathlineto{\pgfqpoint{2.710670in}{2.773355in}}%
\pgfpathlineto{\pgfqpoint{2.756217in}{2.803036in}}%
\pgfpathlineto{\pgfqpoint{2.801763in}{2.831831in}}%
\pgfpathlineto{\pgfqpoint{2.847309in}{2.859771in}}%
\pgfpathlineto{\pgfqpoint{2.892856in}{2.886895in}}%
\pgfpathlineto{\pgfqpoint{2.938402in}{2.913245in}}%
\pgfpathlineto{\pgfqpoint{2.983949in}{2.938867in}}%
\pgfpathlineto{\pgfqpoint{3.029495in}{2.963807in}}%
\pgfpathlineto{\pgfqpoint{3.075041in}{2.988114in}}%
\pgfpathlineto{\pgfqpoint{3.120588in}{3.011835in}}%
\pgfpathlineto{\pgfqpoint{3.166134in}{3.035016in}}%
\pgfpathlineto{\pgfqpoint{3.211680in}{3.057702in}}%
\pgfpathlineto{\pgfqpoint{3.257227in}{3.079938in}}%
\pgfpathlineto{\pgfqpoint{3.302773in}{3.101764in}}%
\pgfpathlineto{\pgfqpoint{3.348320in}{3.123219in}}%
\pgfpathlineto{\pgfqpoint{3.393866in}{3.144340in}}%
\pgfpathlineto{\pgfqpoint{3.439412in}{3.165161in}}%
\pgfpathlineto{\pgfqpoint{3.484959in}{3.185715in}}%
\pgfpathlineto{\pgfqpoint{3.530505in}{3.206031in}}%
\pgfpathlineto{\pgfqpoint{3.576051in}{3.226136in}}%
\pgfpathlineto{\pgfqpoint{3.621598in}{3.246055in}}%
\pgfpathlineto{\pgfqpoint{3.667144in}{3.265812in}}%
\pgfpathlineto{\pgfqpoint{3.712691in}{3.285428in}}%
\pgfpathlineto{\pgfqpoint{3.758237in}{3.304923in}}%
\pgfpathlineto{\pgfqpoint{3.803783in}{3.324312in}}%
\pgfpathlineto{\pgfqpoint{3.849330in}{3.343614in}}%
\pgfpathlineto{\pgfqpoint{3.894876in}{3.362841in}}%
\pgfpathlineto{\pgfqpoint{3.940422in}{3.382008in}}%
\pgfpathlineto{\pgfqpoint{3.985969in}{3.401125in}}%
\pgfpathlineto{\pgfqpoint{4.031515in}{3.420204in}}%
\pgfpathlineto{\pgfqpoint{4.077062in}{3.439253in}}%
\pgfpathlineto{\pgfqpoint{4.122608in}{3.458281in}}%
\pgfpathlineto{\pgfqpoint{4.168154in}{3.477296in}}%
\pgfpathlineto{\pgfqpoint{4.213701in}{3.496303in}}%
\pgfpathlineto{\pgfqpoint{4.259247in}{3.515310in}}%
\pgfpathlineto{\pgfqpoint{4.304793in}{3.534320in}}%
\pgfpathlineto{\pgfqpoint{4.350340in}{3.553339in}}%
\pgfpathlineto{\pgfqpoint{4.395886in}{3.572371in}}%
\pgfpathlineto{\pgfqpoint{4.441433in}{3.591418in}}%
\pgfpathlineto{\pgfqpoint{4.486979in}{3.610483in}}%
\pgfpathlineto{\pgfqpoint{4.532525in}{3.629569in}}%
\pgfpathlineto{\pgfqpoint{4.578072in}{3.648677in}}%
\pgfpathlineto{\pgfqpoint{4.623618in}{3.667810in}}%
\pgfpathlineto{\pgfqpoint{4.669164in}{3.686969in}}%
\pgfpathlineto{\pgfqpoint{4.714711in}{3.706154in}}%
\pgfpathlineto{\pgfqpoint{4.760257in}{3.725367in}}%
\pgfpathlineto{\pgfqpoint{4.805803in}{3.744607in}}%
\pgfpathlineto{\pgfqpoint{4.851350in}{3.763876in}}%
\pgfpathlineto{\pgfqpoint{4.896896in}{3.783172in}}%
\pgfpathlineto{\pgfqpoint{4.942443in}{3.802497in}}%
\pgfpathlineto{\pgfqpoint{4.987989in}{3.821849in}}%
\pgfpathlineto{\pgfqpoint{5.033535in}{3.841229in}}%
\pgfpathlineto{\pgfqpoint{5.079082in}{3.860635in}}%
\pgfpathlineto{\pgfqpoint{5.124628in}{3.880068in}}%
\pgfpathlineto{\pgfqpoint{5.170174in}{3.899526in}}%
\pgfpathlineto{\pgfqpoint{5.215721in}{3.919009in}}%
\pgfpathlineto{\pgfqpoint{5.261267in}{3.938516in}}%
\pgfpathlineto{\pgfqpoint{5.306814in}{3.958045in}}%
\pgfpathlineto{\pgfqpoint{5.352360in}{3.977597in}}%
\pgfpathlineto{\pgfqpoint{5.397906in}{3.997169in}}%
\pgfpathlineto{\pgfqpoint{5.443453in}{4.016761in}}%
\pgfpathlineto{\pgfqpoint{5.488999in}{4.036372in}}%
\pgfpathlineto{\pgfqpoint{5.534545in}{4.056000in}}%
\pgfusepath{stroke}%
\end{pgfscope}%
\begin{pgfscope}%
\pgfpathrectangle{\pgfqpoint{0.800000in}{0.528000in}}{\pgfqpoint{4.960000in}{3.696000in}}%
\pgfusepath{clip}%
\pgfsetrectcap%
\pgfsetroundjoin%
\pgfsetlinewidth{1.505625pt}%
\definecolor{currentstroke}{rgb}{1.000000,0.000000,0.000000}%
\pgfsetstrokecolor{currentstroke}%
\pgfsetdash{}{0pt}%
\pgfpathmoveto{\pgfqpoint{1.025455in}{0.696000in}}%
\pgfpathlineto{\pgfqpoint{1.071001in}{0.832807in}}%
\pgfpathlineto{\pgfqpoint{1.116547in}{0.937784in}}%
\pgfpathlineto{\pgfqpoint{1.162094in}{1.026283in}}%
\pgfpathlineto{\pgfqpoint{1.207640in}{1.104252in}}%
\pgfpathlineto{\pgfqpoint{1.253186in}{1.174741in}}%
\pgfpathlineto{\pgfqpoint{1.298733in}{1.239563in}}%
\pgfpathlineto{\pgfqpoint{1.344279in}{1.299898in}}%
\pgfpathlineto{\pgfqpoint{1.389826in}{1.356565in}}%
\pgfpathlineto{\pgfqpoint{1.435372in}{1.410162in}}%
\pgfpathlineto{\pgfqpoint{1.480918in}{1.461141in}}%
\pgfpathlineto{\pgfqpoint{1.526465in}{1.509850in}}%
\pgfpathlineto{\pgfqpoint{1.572011in}{1.556568in}}%
\pgfpathlineto{\pgfqpoint{1.617557in}{1.601521in}}%
\pgfpathlineto{\pgfqpoint{1.663104in}{1.644896in}}%
\pgfpathlineto{\pgfqpoint{1.708650in}{1.686848in}}%
\pgfpathlineto{\pgfqpoint{1.754197in}{1.727507in}}%
\pgfpathlineto{\pgfqpoint{1.799743in}{1.766988in}}%
\pgfpathlineto{\pgfqpoint{1.845289in}{1.805386in}}%
\pgfpathlineto{\pgfqpoint{1.890836in}{1.842786in}}%
\pgfpathlineto{\pgfqpoint{1.936382in}{1.879262in}}%
\pgfpathlineto{\pgfqpoint{1.981928in}{1.914880in}}%
\pgfpathlineto{\pgfqpoint{2.027475in}{1.949697in}}%
\pgfpathlineto{\pgfqpoint{2.073021in}{1.983765in}}%
\pgfpathlineto{\pgfqpoint{2.118567in}{2.017130in}}%
\pgfpathlineto{\pgfqpoint{2.164114in}{2.049834in}}%
\pgfpathlineto{\pgfqpoint{2.209660in}{2.081916in}}%
\pgfpathlineto{\pgfqpoint{2.255207in}{2.113408in}}%
\pgfpathlineto{\pgfqpoint{2.300753in}{2.144343in}}%
\pgfpathlineto{\pgfqpoint{2.346299in}{2.174749in}}%
\pgfpathlineto{\pgfqpoint{2.391846in}{2.204653in}}%
\pgfpathlineto{\pgfqpoint{2.437392in}{2.234078in}}%
\pgfpathlineto{\pgfqpoint{2.482938in}{2.263046in}}%
\pgfpathlineto{\pgfqpoint{2.528485in}{2.291579in}}%
\pgfpathlineto{\pgfqpoint{2.574031in}{2.319695in}}%
\pgfpathlineto{\pgfqpoint{2.619578in}{2.347413in}}%
\pgfpathlineto{\pgfqpoint{2.665124in}{2.374748in}}%
\pgfpathlineto{\pgfqpoint{2.710670in}{2.401716in}}%
\pgfpathlineto{\pgfqpoint{2.756217in}{2.428331in}}%
\pgfpathlineto{\pgfqpoint{2.801763in}{2.454607in}}%
\pgfpathlineto{\pgfqpoint{2.847309in}{2.480557in}}%
\pgfpathlineto{\pgfqpoint{2.892856in}{2.506193in}}%
\pgfpathlineto{\pgfqpoint{2.938402in}{2.531525in}}%
\pgfpathlineto{\pgfqpoint{2.983949in}{2.556564in}}%
\pgfpathlineto{\pgfqpoint{3.029495in}{2.581320in}}%
\pgfpathlineto{\pgfqpoint{3.075041in}{2.605802in}}%
\pgfpathlineto{\pgfqpoint{3.120588in}{2.630020in}}%
\pgfpathlineto{\pgfqpoint{3.166134in}{2.653982in}}%
\pgfpathlineto{\pgfqpoint{3.211680in}{2.677695in}}%
\pgfpathlineto{\pgfqpoint{3.257227in}{2.701168in}}%
\pgfpathlineto{\pgfqpoint{3.302773in}{2.724406in}}%
\pgfpathlineto{\pgfqpoint{3.348320in}{2.747419in}}%
\pgfpathlineto{\pgfqpoint{3.393866in}{2.770210in}}%
\pgfpathlineto{\pgfqpoint{3.439412in}{2.792788in}}%
\pgfpathlineto{\pgfqpoint{3.484959in}{2.815158in}}%
\pgfpathlineto{\pgfqpoint{3.530505in}{2.837325in}}%
\pgfpathlineto{\pgfqpoint{3.576051in}{2.859296in}}%
\pgfpathlineto{\pgfqpoint{3.621598in}{2.881074in}}%
\pgfpathlineto{\pgfqpoint{3.667144in}{2.902666in}}%
\pgfpathlineto{\pgfqpoint{3.712691in}{2.924075in}}%
\pgfpathlineto{\pgfqpoint{3.758237in}{2.945306in}}%
\pgfpathlineto{\pgfqpoint{3.803783in}{2.966365in}}%
\pgfpathlineto{\pgfqpoint{3.849330in}{2.987254in}}%
\pgfpathlineto{\pgfqpoint{3.894876in}{3.007978in}}%
\pgfpathlineto{\pgfqpoint{3.940422in}{3.028540in}}%
\pgfpathlineto{\pgfqpoint{3.985969in}{3.048945in}}%
\pgfpathlineto{\pgfqpoint{4.031515in}{3.069196in}}%
\pgfpathlineto{\pgfqpoint{4.077062in}{3.089297in}}%
\pgfpathlineto{\pgfqpoint{4.122608in}{3.109250in}}%
\pgfpathlineto{\pgfqpoint{4.168154in}{3.129059in}}%
\pgfpathlineto{\pgfqpoint{4.213701in}{3.148727in}}%
\pgfpathlineto{\pgfqpoint{4.259247in}{3.168258in}}%
\pgfpathlineto{\pgfqpoint{4.304793in}{3.187652in}}%
\pgfpathlineto{\pgfqpoint{4.350340in}{3.206915in}}%
\pgfpathlineto{\pgfqpoint{4.395886in}{3.226048in}}%
\pgfpathlineto{\pgfqpoint{4.441433in}{3.245054in}}%
\pgfpathlineto{\pgfqpoint{4.486979in}{3.263935in}}%
\pgfpathlineto{\pgfqpoint{4.532525in}{3.282694in}}%
\pgfpathlineto{\pgfqpoint{4.578072in}{3.301333in}}%
\pgfpathlineto{\pgfqpoint{4.623618in}{3.319854in}}%
\pgfpathlineto{\pgfqpoint{4.669164in}{3.338260in}}%
\pgfpathlineto{\pgfqpoint{4.714711in}{3.356553in}}%
\pgfpathlineto{\pgfqpoint{4.760257in}{3.374734in}}%
\pgfpathlineto{\pgfqpoint{4.805803in}{3.392807in}}%
\pgfpathlineto{\pgfqpoint{4.851350in}{3.410772in}}%
\pgfpathlineto{\pgfqpoint{4.896896in}{3.428631in}}%
\pgfpathlineto{\pgfqpoint{4.942443in}{3.446388in}}%
\pgfpathlineto{\pgfqpoint{4.987989in}{3.464042in}}%
\pgfpathlineto{\pgfqpoint{5.033535in}{3.481596in}}%
\pgfpathlineto{\pgfqpoint{5.079082in}{3.499053in}}%
\pgfpathlineto{\pgfqpoint{5.124628in}{3.516412in}}%
\pgfpathlineto{\pgfqpoint{5.170174in}{3.533676in}}%
\pgfpathlineto{\pgfqpoint{5.215721in}{3.550847in}}%
\pgfpathlineto{\pgfqpoint{5.261267in}{3.567925in}}%
\pgfpathlineto{\pgfqpoint{5.306814in}{3.584913in}}%
\pgfpathlineto{\pgfqpoint{5.352360in}{3.601812in}}%
\pgfpathlineto{\pgfqpoint{5.397906in}{3.618623in}}%
\pgfpathlineto{\pgfqpoint{5.443453in}{3.635348in}}%
\pgfpathlineto{\pgfqpoint{5.488999in}{3.651987in}}%
\pgfpathlineto{\pgfqpoint{5.534545in}{3.668543in}}%
\pgfusepath{stroke}%
\end{pgfscope}%
\begin{pgfscope}%
\pgfsetrectcap%
\pgfsetmiterjoin%
\pgfsetlinewidth{0.803000pt}%
\definecolor{currentstroke}{rgb}{0.000000,0.000000,0.000000}%
\pgfsetstrokecolor{currentstroke}%
\pgfsetdash{}{0pt}%
\pgfpathmoveto{\pgfqpoint{0.800000in}{0.528000in}}%
\pgfpathlineto{\pgfqpoint{0.800000in}{4.224000in}}%
\pgfusepath{stroke}%
\end{pgfscope}%
\begin{pgfscope}%
\pgfsetrectcap%
\pgfsetmiterjoin%
\pgfsetlinewidth{0.803000pt}%
\definecolor{currentstroke}{rgb}{0.000000,0.000000,0.000000}%
\pgfsetstrokecolor{currentstroke}%
\pgfsetdash{}{0pt}%
\pgfpathmoveto{\pgfqpoint{5.760000in}{0.528000in}}%
\pgfpathlineto{\pgfqpoint{5.760000in}{4.224000in}}%
\pgfusepath{stroke}%
\end{pgfscope}%
\begin{pgfscope}%
\pgfsetrectcap%
\pgfsetmiterjoin%
\pgfsetlinewidth{0.803000pt}%
\definecolor{currentstroke}{rgb}{0.000000,0.000000,0.000000}%
\pgfsetstrokecolor{currentstroke}%
\pgfsetdash{}{0pt}%
\pgfpathmoveto{\pgfqpoint{0.800000in}{0.528000in}}%
\pgfpathlineto{\pgfqpoint{5.760000in}{0.528000in}}%
\pgfusepath{stroke}%
\end{pgfscope}%
\begin{pgfscope}%
\pgfsetrectcap%
\pgfsetmiterjoin%
\pgfsetlinewidth{0.803000pt}%
\definecolor{currentstroke}{rgb}{0.000000,0.000000,0.000000}%
\pgfsetstrokecolor{currentstroke}%
\pgfsetdash{}{0pt}%
\pgfpathmoveto{\pgfqpoint{0.800000in}{4.224000in}}%
\pgfpathlineto{\pgfqpoint{5.760000in}{4.224000in}}%
\pgfusepath{stroke}%
\end{pgfscope}%
\begin{pgfscope}%
\pgfsetbuttcap%
\pgfsetmiterjoin%
\definecolor{currentfill}{rgb}{1.000000,1.000000,1.000000}%
\pgfsetfillcolor{currentfill}%
\pgfsetfillopacity{0.800000}%
\pgfsetlinewidth{1.003750pt}%
\definecolor{currentstroke}{rgb}{0.800000,0.800000,0.800000}%
\pgfsetstrokecolor{currentstroke}%
\pgfsetstrokeopacity{0.800000}%
\pgfsetdash{}{0pt}%
\pgfpathmoveto{\pgfqpoint{0.897222in}{3.725543in}}%
\pgfpathlineto{\pgfqpoint{3.713200in}{3.725543in}}%
\pgfpathquadraticcurveto{\pgfqpoint{3.740978in}{3.725543in}}{\pgfqpoint{3.740978in}{3.753321in}}%
\pgfpathlineto{\pgfqpoint{3.740978in}{4.126778in}}%
\pgfpathquadraticcurveto{\pgfqpoint{3.740978in}{4.154556in}}{\pgfqpoint{3.713200in}{4.154556in}}%
\pgfpathlineto{\pgfqpoint{0.897222in}{4.154556in}}%
\pgfpathquadraticcurveto{\pgfqpoint{0.869444in}{4.154556in}}{\pgfqpoint{0.869444in}{4.126778in}}%
\pgfpathlineto{\pgfqpoint{0.869444in}{3.753321in}}%
\pgfpathquadraticcurveto{\pgfqpoint{0.869444in}{3.725543in}}{\pgfqpoint{0.897222in}{3.725543in}}%
\pgfpathclose%
\pgfusepath{stroke,fill}%
\end{pgfscope}%
\begin{pgfscope}%
\pgfsetrectcap%
\pgfsetroundjoin%
\pgfsetlinewidth{1.505625pt}%
\definecolor{currentstroke}{rgb}{0.000000,0.000000,1.000000}%
\pgfsetstrokecolor{currentstroke}%
\pgfsetdash{}{0pt}%
\pgfpathmoveto{\pgfqpoint{0.925000in}{4.050389in}}%
\pgfpathlineto{\pgfqpoint{1.202778in}{4.050389in}}%
\pgfusepath{stroke}%
\end{pgfscope}%
\begin{pgfscope}%
\definecolor{textcolor}{rgb}{0.000000,0.000000,0.000000}%
\pgfsetstrokecolor{textcolor}%
\pgfsetfillcolor{textcolor}%
\pgftext[x=1.313889in,y=4.001778in,left,base]{\color{textcolor}\rmfamily\fontsize{10.000000}{12.000000}\selectfont Observed Period}%
\end{pgfscope}%
\begin{pgfscope}%
\pgfsetrectcap%
\pgfsetroundjoin%
\pgfsetlinewidth{1.505625pt}%
\definecolor{currentstroke}{rgb}{1.000000,0.000000,0.000000}%
\pgfsetstrokecolor{currentstroke}%
\pgfsetdash{}{0pt}%
\pgfpathmoveto{\pgfqpoint{0.925000in}{3.856716in}}%
\pgfpathlineto{\pgfqpoint{1.202778in}{3.856716in}}%
\pgfusepath{stroke}%
\end{pgfscope}%
\begin{pgfscope}%
\definecolor{textcolor}{rgb}{0.000000,0.000000,0.000000}%
\pgfsetstrokecolor{textcolor}%
\pgfsetfillcolor{textcolor}%
\pgftext[x=1.313889in,y=3.808105in,left,base]{\color{textcolor}\rmfamily\fontsize{10.000000}{12.000000}\selectfont Small Angle Approximation Prediction}%
\end{pgfscope}%
\end{pgfpicture}%
\makeatother%
\endgroup%
}
           \caption{Length Dependence of Period with Starting $\theta = \ang{90}$}
           \label{fig:lengthperiod90}
        \end{center}
    \end{figure}

    \noindent
    As can be seen, the prediction underestimates the period by an increasingly significant amount. 
    However, when oscillating at small angles, as below in Figure \ref{fig:lengthperiod10}, the 
    small angle approximation predicts the length dependence of period quite accurately. It might not 
    be very visible but the two plots are practically on top of one another.

    \begin{figure}[H]
        \begin{center}
           \scalebox{.7}{%% Creator: Matplotlib, PGF backend
%%
%% To include the figure in your LaTeX document, write
%%   \input{<filename>.pgf}
%%
%% Make sure the required packages are loaded in your preamble
%%   \usepackage{pgf}
%%
%% Figures using additional raster images can only be included by \input if
%% they are in the same directory as the main LaTeX file. For loading figures
%% from other directories you can use the `import` package
%%   \usepackage{import}
%% and then include the figures with
%%   \import{<path to file>}{<filename>.pgf}
%%
%% Matplotlib used the following preamble
%%
\begingroup%
\makeatletter%
\begin{pgfpicture}%
\pgfpathrectangle{\pgfpointorigin}{\pgfqpoint{6.400000in}{4.800000in}}%
\pgfusepath{use as bounding box, clip}%
\begin{pgfscope}%
\pgfsetbuttcap%
\pgfsetmiterjoin%
\definecolor{currentfill}{rgb}{1.000000,1.000000,1.000000}%
\pgfsetfillcolor{currentfill}%
\pgfsetlinewidth{0.000000pt}%
\definecolor{currentstroke}{rgb}{1.000000,1.000000,1.000000}%
\pgfsetstrokecolor{currentstroke}%
\pgfsetdash{}{0pt}%
\pgfpathmoveto{\pgfqpoint{0.000000in}{0.000000in}}%
\pgfpathlineto{\pgfqpoint{6.400000in}{0.000000in}}%
\pgfpathlineto{\pgfqpoint{6.400000in}{4.800000in}}%
\pgfpathlineto{\pgfqpoint{0.000000in}{4.800000in}}%
\pgfpathclose%
\pgfusepath{fill}%
\end{pgfscope}%
\begin{pgfscope}%
\pgfsetbuttcap%
\pgfsetmiterjoin%
\definecolor{currentfill}{rgb}{1.000000,1.000000,1.000000}%
\pgfsetfillcolor{currentfill}%
\pgfsetlinewidth{0.000000pt}%
\definecolor{currentstroke}{rgb}{0.000000,0.000000,0.000000}%
\pgfsetstrokecolor{currentstroke}%
\pgfsetstrokeopacity{0.000000}%
\pgfsetdash{}{0pt}%
\pgfpathmoveto{\pgfqpoint{0.800000in}{0.528000in}}%
\pgfpathlineto{\pgfqpoint{5.760000in}{0.528000in}}%
\pgfpathlineto{\pgfqpoint{5.760000in}{4.224000in}}%
\pgfpathlineto{\pgfqpoint{0.800000in}{4.224000in}}%
\pgfpathclose%
\pgfusepath{fill}%
\end{pgfscope}%
\begin{pgfscope}%
\pgfsetbuttcap%
\pgfsetroundjoin%
\definecolor{currentfill}{rgb}{0.000000,0.000000,0.000000}%
\pgfsetfillcolor{currentfill}%
\pgfsetlinewidth{0.803000pt}%
\definecolor{currentstroke}{rgb}{0.000000,0.000000,0.000000}%
\pgfsetstrokecolor{currentstroke}%
\pgfsetdash{}{0pt}%
\pgfsys@defobject{currentmarker}{\pgfqpoint{0.000000in}{-0.048611in}}{\pgfqpoint{0.000000in}{0.000000in}}{%
\pgfpathmoveto{\pgfqpoint{0.000000in}{0.000000in}}%
\pgfpathlineto{\pgfqpoint{0.000000in}{-0.048611in}}%
\pgfusepath{stroke,fill}%
}%
\begin{pgfscope}%
\pgfsys@transformshift{0.979908in}{0.528000in}%
\pgfsys@useobject{currentmarker}{}%
\end{pgfscope}%
\end{pgfscope}%
\begin{pgfscope}%
\definecolor{textcolor}{rgb}{0.000000,0.000000,0.000000}%
\pgfsetstrokecolor{textcolor}%
\pgfsetfillcolor{textcolor}%
\pgftext[x=0.979908in,y=0.430778in,,top]{\color{textcolor}\rmfamily\fontsize{10.000000}{12.000000}\selectfont \(\displaystyle 0\)}%
\end{pgfscope}%
\begin{pgfscope}%
\pgfsetbuttcap%
\pgfsetroundjoin%
\definecolor{currentfill}{rgb}{0.000000,0.000000,0.000000}%
\pgfsetfillcolor{currentfill}%
\pgfsetlinewidth{0.803000pt}%
\definecolor{currentstroke}{rgb}{0.000000,0.000000,0.000000}%
\pgfsetstrokecolor{currentstroke}%
\pgfsetdash{}{0pt}%
\pgfsys@defobject{currentmarker}{\pgfqpoint{0.000000in}{-0.048611in}}{\pgfqpoint{0.000000in}{0.000000in}}{%
\pgfpathmoveto{\pgfqpoint{0.000000in}{0.000000in}}%
\pgfpathlineto{\pgfqpoint{0.000000in}{-0.048611in}}%
\pgfusepath{stroke,fill}%
}%
\begin{pgfscope}%
\pgfsys@transformshift{1.890836in}{0.528000in}%
\pgfsys@useobject{currentmarker}{}%
\end{pgfscope}%
\end{pgfscope}%
\begin{pgfscope}%
\definecolor{textcolor}{rgb}{0.000000,0.000000,0.000000}%
\pgfsetstrokecolor{textcolor}%
\pgfsetfillcolor{textcolor}%
\pgftext[x=1.890836in,y=0.430778in,,top]{\color{textcolor}\rmfamily\fontsize{10.000000}{12.000000}\selectfont \(\displaystyle 20\)}%
\end{pgfscope}%
\begin{pgfscope}%
\pgfsetbuttcap%
\pgfsetroundjoin%
\definecolor{currentfill}{rgb}{0.000000,0.000000,0.000000}%
\pgfsetfillcolor{currentfill}%
\pgfsetlinewidth{0.803000pt}%
\definecolor{currentstroke}{rgb}{0.000000,0.000000,0.000000}%
\pgfsetstrokecolor{currentstroke}%
\pgfsetdash{}{0pt}%
\pgfsys@defobject{currentmarker}{\pgfqpoint{0.000000in}{-0.048611in}}{\pgfqpoint{0.000000in}{0.000000in}}{%
\pgfpathmoveto{\pgfqpoint{0.000000in}{0.000000in}}%
\pgfpathlineto{\pgfqpoint{0.000000in}{-0.048611in}}%
\pgfusepath{stroke,fill}%
}%
\begin{pgfscope}%
\pgfsys@transformshift{2.801763in}{0.528000in}%
\pgfsys@useobject{currentmarker}{}%
\end{pgfscope}%
\end{pgfscope}%
\begin{pgfscope}%
\definecolor{textcolor}{rgb}{0.000000,0.000000,0.000000}%
\pgfsetstrokecolor{textcolor}%
\pgfsetfillcolor{textcolor}%
\pgftext[x=2.801763in,y=0.430778in,,top]{\color{textcolor}\rmfamily\fontsize{10.000000}{12.000000}\selectfont \(\displaystyle 40\)}%
\end{pgfscope}%
\begin{pgfscope}%
\pgfsetbuttcap%
\pgfsetroundjoin%
\definecolor{currentfill}{rgb}{0.000000,0.000000,0.000000}%
\pgfsetfillcolor{currentfill}%
\pgfsetlinewidth{0.803000pt}%
\definecolor{currentstroke}{rgb}{0.000000,0.000000,0.000000}%
\pgfsetstrokecolor{currentstroke}%
\pgfsetdash{}{0pt}%
\pgfsys@defobject{currentmarker}{\pgfqpoint{0.000000in}{-0.048611in}}{\pgfqpoint{0.000000in}{0.000000in}}{%
\pgfpathmoveto{\pgfqpoint{0.000000in}{0.000000in}}%
\pgfpathlineto{\pgfqpoint{0.000000in}{-0.048611in}}%
\pgfusepath{stroke,fill}%
}%
\begin{pgfscope}%
\pgfsys@transformshift{3.712691in}{0.528000in}%
\pgfsys@useobject{currentmarker}{}%
\end{pgfscope}%
\end{pgfscope}%
\begin{pgfscope}%
\definecolor{textcolor}{rgb}{0.000000,0.000000,0.000000}%
\pgfsetstrokecolor{textcolor}%
\pgfsetfillcolor{textcolor}%
\pgftext[x=3.712691in,y=0.430778in,,top]{\color{textcolor}\rmfamily\fontsize{10.000000}{12.000000}\selectfont \(\displaystyle 60\)}%
\end{pgfscope}%
\begin{pgfscope}%
\pgfsetbuttcap%
\pgfsetroundjoin%
\definecolor{currentfill}{rgb}{0.000000,0.000000,0.000000}%
\pgfsetfillcolor{currentfill}%
\pgfsetlinewidth{0.803000pt}%
\definecolor{currentstroke}{rgb}{0.000000,0.000000,0.000000}%
\pgfsetstrokecolor{currentstroke}%
\pgfsetdash{}{0pt}%
\pgfsys@defobject{currentmarker}{\pgfqpoint{0.000000in}{-0.048611in}}{\pgfqpoint{0.000000in}{0.000000in}}{%
\pgfpathmoveto{\pgfqpoint{0.000000in}{0.000000in}}%
\pgfpathlineto{\pgfqpoint{0.000000in}{-0.048611in}}%
\pgfusepath{stroke,fill}%
}%
\begin{pgfscope}%
\pgfsys@transformshift{4.623618in}{0.528000in}%
\pgfsys@useobject{currentmarker}{}%
\end{pgfscope}%
\end{pgfscope}%
\begin{pgfscope}%
\definecolor{textcolor}{rgb}{0.000000,0.000000,0.000000}%
\pgfsetstrokecolor{textcolor}%
\pgfsetfillcolor{textcolor}%
\pgftext[x=4.623618in,y=0.430778in,,top]{\color{textcolor}\rmfamily\fontsize{10.000000}{12.000000}\selectfont \(\displaystyle 80\)}%
\end{pgfscope}%
\begin{pgfscope}%
\pgfsetbuttcap%
\pgfsetroundjoin%
\definecolor{currentfill}{rgb}{0.000000,0.000000,0.000000}%
\pgfsetfillcolor{currentfill}%
\pgfsetlinewidth{0.803000pt}%
\definecolor{currentstroke}{rgb}{0.000000,0.000000,0.000000}%
\pgfsetstrokecolor{currentstroke}%
\pgfsetdash{}{0pt}%
\pgfsys@defobject{currentmarker}{\pgfqpoint{0.000000in}{-0.048611in}}{\pgfqpoint{0.000000in}{0.000000in}}{%
\pgfpathmoveto{\pgfqpoint{0.000000in}{0.000000in}}%
\pgfpathlineto{\pgfqpoint{0.000000in}{-0.048611in}}%
\pgfusepath{stroke,fill}%
}%
\begin{pgfscope}%
\pgfsys@transformshift{5.534545in}{0.528000in}%
\pgfsys@useobject{currentmarker}{}%
\end{pgfscope}%
\end{pgfscope}%
\begin{pgfscope}%
\definecolor{textcolor}{rgb}{0.000000,0.000000,0.000000}%
\pgfsetstrokecolor{textcolor}%
\pgfsetfillcolor{textcolor}%
\pgftext[x=5.534545in,y=0.430778in,,top]{\color{textcolor}\rmfamily\fontsize{10.000000}{12.000000}\selectfont \(\displaystyle 100\)}%
\end{pgfscope}%
\begin{pgfscope}%
\definecolor{textcolor}{rgb}{0.000000,0.000000,0.000000}%
\pgfsetstrokecolor{textcolor}%
\pgfsetfillcolor{textcolor}%
\pgftext[x=3.280000in,y=0.251766in,,top]{\color{textcolor}\rmfamily\fontsize{10.000000}{12.000000}\selectfont Length (m)}%
\end{pgfscope}%
\begin{pgfscope}%
\pgfsetbuttcap%
\pgfsetroundjoin%
\definecolor{currentfill}{rgb}{0.000000,0.000000,0.000000}%
\pgfsetfillcolor{currentfill}%
\pgfsetlinewidth{0.803000pt}%
\definecolor{currentstroke}{rgb}{0.000000,0.000000,0.000000}%
\pgfsetstrokecolor{currentstroke}%
\pgfsetdash{}{0pt}%
\pgfsys@defobject{currentmarker}{\pgfqpoint{-0.048611in}{0.000000in}}{\pgfqpoint{0.000000in}{0.000000in}}{%
\pgfpathmoveto{\pgfqpoint{0.000000in}{0.000000in}}%
\pgfpathlineto{\pgfqpoint{-0.048611in}{0.000000in}}%
\pgfusepath{stroke,fill}%
}%
\begin{pgfscope}%
\pgfsys@transformshift{0.800000in}{0.787547in}%
\pgfsys@useobject{currentmarker}{}%
\end{pgfscope}%
\end{pgfscope}%
\begin{pgfscope}%
\definecolor{textcolor}{rgb}{0.000000,0.000000,0.000000}%
\pgfsetstrokecolor{textcolor}%
\pgfsetfillcolor{textcolor}%
\pgftext[x=0.525308in,y=0.739322in,left,base]{\color{textcolor}\rmfamily\fontsize{10.000000}{12.000000}\selectfont \(\displaystyle 2.5\)}%
\end{pgfscope}%
\begin{pgfscope}%
\pgfsetbuttcap%
\pgfsetroundjoin%
\definecolor{currentfill}{rgb}{0.000000,0.000000,0.000000}%
\pgfsetfillcolor{currentfill}%
\pgfsetlinewidth{0.803000pt}%
\definecolor{currentstroke}{rgb}{0.000000,0.000000,0.000000}%
\pgfsetstrokecolor{currentstroke}%
\pgfsetdash{}{0pt}%
\pgfsys@defobject{currentmarker}{\pgfqpoint{-0.048611in}{0.000000in}}{\pgfqpoint{0.000000in}{0.000000in}}{%
\pgfpathmoveto{\pgfqpoint{0.000000in}{0.000000in}}%
\pgfpathlineto{\pgfqpoint{-0.048611in}{0.000000in}}%
\pgfusepath{stroke,fill}%
}%
\begin{pgfscope}%
\pgfsys@transformshift{0.800000in}{1.251869in}%
\pgfsys@useobject{currentmarker}{}%
\end{pgfscope}%
\end{pgfscope}%
\begin{pgfscope}%
\definecolor{textcolor}{rgb}{0.000000,0.000000,0.000000}%
\pgfsetstrokecolor{textcolor}%
\pgfsetfillcolor{textcolor}%
\pgftext[x=0.525308in,y=1.203643in,left,base]{\color{textcolor}\rmfamily\fontsize{10.000000}{12.000000}\selectfont \(\displaystyle 5.0\)}%
\end{pgfscope}%
\begin{pgfscope}%
\pgfsetbuttcap%
\pgfsetroundjoin%
\definecolor{currentfill}{rgb}{0.000000,0.000000,0.000000}%
\pgfsetfillcolor{currentfill}%
\pgfsetlinewidth{0.803000pt}%
\definecolor{currentstroke}{rgb}{0.000000,0.000000,0.000000}%
\pgfsetstrokecolor{currentstroke}%
\pgfsetdash{}{0pt}%
\pgfsys@defobject{currentmarker}{\pgfqpoint{-0.048611in}{0.000000in}}{\pgfqpoint{0.000000in}{0.000000in}}{%
\pgfpathmoveto{\pgfqpoint{0.000000in}{0.000000in}}%
\pgfpathlineto{\pgfqpoint{-0.048611in}{0.000000in}}%
\pgfusepath{stroke,fill}%
}%
\begin{pgfscope}%
\pgfsys@transformshift{0.800000in}{1.716190in}%
\pgfsys@useobject{currentmarker}{}%
\end{pgfscope}%
\end{pgfscope}%
\begin{pgfscope}%
\definecolor{textcolor}{rgb}{0.000000,0.000000,0.000000}%
\pgfsetstrokecolor{textcolor}%
\pgfsetfillcolor{textcolor}%
\pgftext[x=0.525308in,y=1.667965in,left,base]{\color{textcolor}\rmfamily\fontsize{10.000000}{12.000000}\selectfont \(\displaystyle 7.5\)}%
\end{pgfscope}%
\begin{pgfscope}%
\pgfsetbuttcap%
\pgfsetroundjoin%
\definecolor{currentfill}{rgb}{0.000000,0.000000,0.000000}%
\pgfsetfillcolor{currentfill}%
\pgfsetlinewidth{0.803000pt}%
\definecolor{currentstroke}{rgb}{0.000000,0.000000,0.000000}%
\pgfsetstrokecolor{currentstroke}%
\pgfsetdash{}{0pt}%
\pgfsys@defobject{currentmarker}{\pgfqpoint{-0.048611in}{0.000000in}}{\pgfqpoint{0.000000in}{0.000000in}}{%
\pgfpathmoveto{\pgfqpoint{0.000000in}{0.000000in}}%
\pgfpathlineto{\pgfqpoint{-0.048611in}{0.000000in}}%
\pgfusepath{stroke,fill}%
}%
\begin{pgfscope}%
\pgfsys@transformshift{0.800000in}{2.180511in}%
\pgfsys@useobject{currentmarker}{}%
\end{pgfscope}%
\end{pgfscope}%
\begin{pgfscope}%
\definecolor{textcolor}{rgb}{0.000000,0.000000,0.000000}%
\pgfsetstrokecolor{textcolor}%
\pgfsetfillcolor{textcolor}%
\pgftext[x=0.455863in,y=2.132286in,left,base]{\color{textcolor}\rmfamily\fontsize{10.000000}{12.000000}\selectfont \(\displaystyle 10.0\)}%
\end{pgfscope}%
\begin{pgfscope}%
\pgfsetbuttcap%
\pgfsetroundjoin%
\definecolor{currentfill}{rgb}{0.000000,0.000000,0.000000}%
\pgfsetfillcolor{currentfill}%
\pgfsetlinewidth{0.803000pt}%
\definecolor{currentstroke}{rgb}{0.000000,0.000000,0.000000}%
\pgfsetstrokecolor{currentstroke}%
\pgfsetdash{}{0pt}%
\pgfsys@defobject{currentmarker}{\pgfqpoint{-0.048611in}{0.000000in}}{\pgfqpoint{0.000000in}{0.000000in}}{%
\pgfpathmoveto{\pgfqpoint{0.000000in}{0.000000in}}%
\pgfpathlineto{\pgfqpoint{-0.048611in}{0.000000in}}%
\pgfusepath{stroke,fill}%
}%
\begin{pgfscope}%
\pgfsys@transformshift{0.800000in}{2.644832in}%
\pgfsys@useobject{currentmarker}{}%
\end{pgfscope}%
\end{pgfscope}%
\begin{pgfscope}%
\definecolor{textcolor}{rgb}{0.000000,0.000000,0.000000}%
\pgfsetstrokecolor{textcolor}%
\pgfsetfillcolor{textcolor}%
\pgftext[x=0.455863in,y=2.596607in,left,base]{\color{textcolor}\rmfamily\fontsize{10.000000}{12.000000}\selectfont \(\displaystyle 12.5\)}%
\end{pgfscope}%
\begin{pgfscope}%
\pgfsetbuttcap%
\pgfsetroundjoin%
\definecolor{currentfill}{rgb}{0.000000,0.000000,0.000000}%
\pgfsetfillcolor{currentfill}%
\pgfsetlinewidth{0.803000pt}%
\definecolor{currentstroke}{rgb}{0.000000,0.000000,0.000000}%
\pgfsetstrokecolor{currentstroke}%
\pgfsetdash{}{0pt}%
\pgfsys@defobject{currentmarker}{\pgfqpoint{-0.048611in}{0.000000in}}{\pgfqpoint{0.000000in}{0.000000in}}{%
\pgfpathmoveto{\pgfqpoint{0.000000in}{0.000000in}}%
\pgfpathlineto{\pgfqpoint{-0.048611in}{0.000000in}}%
\pgfusepath{stroke,fill}%
}%
\begin{pgfscope}%
\pgfsys@transformshift{0.800000in}{3.109153in}%
\pgfsys@useobject{currentmarker}{}%
\end{pgfscope}%
\end{pgfscope}%
\begin{pgfscope}%
\definecolor{textcolor}{rgb}{0.000000,0.000000,0.000000}%
\pgfsetstrokecolor{textcolor}%
\pgfsetfillcolor{textcolor}%
\pgftext[x=0.455863in,y=3.060928in,left,base]{\color{textcolor}\rmfamily\fontsize{10.000000}{12.000000}\selectfont \(\displaystyle 15.0\)}%
\end{pgfscope}%
\begin{pgfscope}%
\pgfsetbuttcap%
\pgfsetroundjoin%
\definecolor{currentfill}{rgb}{0.000000,0.000000,0.000000}%
\pgfsetfillcolor{currentfill}%
\pgfsetlinewidth{0.803000pt}%
\definecolor{currentstroke}{rgb}{0.000000,0.000000,0.000000}%
\pgfsetstrokecolor{currentstroke}%
\pgfsetdash{}{0pt}%
\pgfsys@defobject{currentmarker}{\pgfqpoint{-0.048611in}{0.000000in}}{\pgfqpoint{0.000000in}{0.000000in}}{%
\pgfpathmoveto{\pgfqpoint{0.000000in}{0.000000in}}%
\pgfpathlineto{\pgfqpoint{-0.048611in}{0.000000in}}%
\pgfusepath{stroke,fill}%
}%
\begin{pgfscope}%
\pgfsys@transformshift{0.800000in}{3.573474in}%
\pgfsys@useobject{currentmarker}{}%
\end{pgfscope}%
\end{pgfscope}%
\begin{pgfscope}%
\definecolor{textcolor}{rgb}{0.000000,0.000000,0.000000}%
\pgfsetstrokecolor{textcolor}%
\pgfsetfillcolor{textcolor}%
\pgftext[x=0.455863in,y=3.525249in,left,base]{\color{textcolor}\rmfamily\fontsize{10.000000}{12.000000}\selectfont \(\displaystyle 17.5\)}%
\end{pgfscope}%
\begin{pgfscope}%
\pgfsetbuttcap%
\pgfsetroundjoin%
\definecolor{currentfill}{rgb}{0.000000,0.000000,0.000000}%
\pgfsetfillcolor{currentfill}%
\pgfsetlinewidth{0.803000pt}%
\definecolor{currentstroke}{rgb}{0.000000,0.000000,0.000000}%
\pgfsetstrokecolor{currentstroke}%
\pgfsetdash{}{0pt}%
\pgfsys@defobject{currentmarker}{\pgfqpoint{-0.048611in}{0.000000in}}{\pgfqpoint{0.000000in}{0.000000in}}{%
\pgfpathmoveto{\pgfqpoint{0.000000in}{0.000000in}}%
\pgfpathlineto{\pgfqpoint{-0.048611in}{0.000000in}}%
\pgfusepath{stroke,fill}%
}%
\begin{pgfscope}%
\pgfsys@transformshift{0.800000in}{4.037796in}%
\pgfsys@useobject{currentmarker}{}%
\end{pgfscope}%
\end{pgfscope}%
\begin{pgfscope}%
\definecolor{textcolor}{rgb}{0.000000,0.000000,0.000000}%
\pgfsetstrokecolor{textcolor}%
\pgfsetfillcolor{textcolor}%
\pgftext[x=0.455863in,y=3.989570in,left,base]{\color{textcolor}\rmfamily\fontsize{10.000000}{12.000000}\selectfont \(\displaystyle 20.0\)}%
\end{pgfscope}%
\begin{pgfscope}%
\definecolor{textcolor}{rgb}{0.000000,0.000000,0.000000}%
\pgfsetstrokecolor{textcolor}%
\pgfsetfillcolor{textcolor}%
\pgftext[x=0.400308in,y=2.376000in,,bottom,rotate=90.000000]{\color{textcolor}\rmfamily\fontsize{10.000000}{12.000000}\selectfont Period T (s)}%
\end{pgfscope}%
\begin{pgfscope}%
\pgfpathrectangle{\pgfqpoint{0.800000in}{0.528000in}}{\pgfqpoint{4.960000in}{3.696000in}}%
\pgfusepath{clip}%
\pgfsetrectcap%
\pgfsetroundjoin%
\pgfsetlinewidth{1.505625pt}%
\definecolor{currentstroke}{rgb}{0.000000,0.000000,1.000000}%
\pgfsetstrokecolor{currentstroke}%
\pgfsetdash{}{0pt}%
\pgfpathmoveto{\pgfqpoint{1.025455in}{0.696696in}}%
\pgfpathlineto{\pgfqpoint{1.071001in}{0.851402in}}%
\pgfpathlineto{\pgfqpoint{1.116547in}{0.970111in}}%
\pgfpathlineto{\pgfqpoint{1.162094in}{1.070187in}}%
\pgfpathlineto{\pgfqpoint{1.207640in}{1.158359in}}%
\pgfpathlineto{\pgfqpoint{1.253186in}{1.238064in}}%
\pgfpathlineto{\pgfqpoint{1.298733in}{1.311364in}}%
\pgfpathlineto{\pgfqpoint{1.344279in}{1.379599in}}%
\pgfpathlineto{\pgfqpoint{1.389826in}{1.443674in}}%
\pgfpathlineto{\pgfqpoint{1.435372in}{1.504275in}}%
\pgfpathlineto{\pgfqpoint{1.480918in}{1.561923in}}%
\pgfpathlineto{\pgfqpoint{1.526465in}{1.617002in}}%
\pgfpathlineto{\pgfqpoint{1.572011in}{1.669830in}}%
\pgfpathlineto{\pgfqpoint{1.617557in}{1.720669in}}%
\pgfpathlineto{\pgfqpoint{1.663104in}{1.769728in}}%
\pgfpathlineto{\pgfqpoint{1.708650in}{1.817173in}}%
\pgfpathlineto{\pgfqpoint{1.754197in}{1.863150in}}%
\pgfpathlineto{\pgfqpoint{1.799743in}{1.907786in}}%
\pgfpathlineto{\pgfqpoint{1.845289in}{1.951194in}}%
\pgfpathlineto{\pgfqpoint{1.890836in}{1.993475in}}%
\pgfpathlineto{\pgfqpoint{1.936382in}{2.034715in}}%
\pgfpathlineto{\pgfqpoint{1.981928in}{2.074989in}}%
\pgfpathlineto{\pgfqpoint{2.027475in}{2.114360in}}%
\pgfpathlineto{\pgfqpoint{2.073021in}{2.152886in}}%
\pgfpathlineto{\pgfqpoint{2.118567in}{2.190618in}}%
\pgfpathlineto{\pgfqpoint{2.164114in}{2.227601in}}%
\pgfpathlineto{\pgfqpoint{2.209660in}{2.263878in}}%
\pgfpathlineto{\pgfqpoint{2.255207in}{2.299488in}}%
\pgfpathlineto{\pgfqpoint{2.300753in}{2.334466in}}%
\pgfpathlineto{\pgfqpoint{2.346299in}{2.368846in}}%
\pgfpathlineto{\pgfqpoint{2.391846in}{2.402658in}}%
\pgfpathlineto{\pgfqpoint{2.437392in}{2.435930in}}%
\pgfpathlineto{\pgfqpoint{2.482938in}{2.468688in}}%
\pgfpathlineto{\pgfqpoint{2.528485in}{2.500955in}}%
\pgfpathlineto{\pgfqpoint{2.574031in}{2.532753in}}%
\pgfpathlineto{\pgfqpoint{2.619578in}{2.564102in}}%
\pgfpathlineto{\pgfqpoint{2.665124in}{2.595020in}}%
\pgfpathlineto{\pgfqpoint{2.710670in}{2.625525in}}%
\pgfpathlineto{\pgfqpoint{2.756217in}{2.655633in}}%
\pgfpathlineto{\pgfqpoint{2.801763in}{2.685358in}}%
\pgfpathlineto{\pgfqpoint{2.847309in}{2.714714in}}%
\pgfpathlineto{\pgfqpoint{2.892856in}{2.743714in}}%
\pgfpathlineto{\pgfqpoint{2.938402in}{2.772371in}}%
\pgfpathlineto{\pgfqpoint{2.983949in}{2.800695in}}%
\pgfpathlineto{\pgfqpoint{3.029495in}{2.828698in}}%
\pgfpathlineto{\pgfqpoint{3.075041in}{2.856390in}}%
\pgfpathlineto{\pgfqpoint{3.120588in}{2.883781in}}%
\pgfpathlineto{\pgfqpoint{3.166134in}{2.910879in}}%
\pgfpathlineto{\pgfqpoint{3.211680in}{2.937694in}}%
\pgfpathlineto{\pgfqpoint{3.257227in}{2.964235in}}%
\pgfpathlineto{\pgfqpoint{3.302773in}{2.990508in}}%
\pgfpathlineto{\pgfqpoint{3.348320in}{3.016522in}}%
\pgfpathlineto{\pgfqpoint{3.393866in}{3.042283in}}%
\pgfpathlineto{\pgfqpoint{3.439412in}{3.067800in}}%
\pgfpathlineto{\pgfqpoint{3.484959in}{3.093078in}}%
\pgfpathlineto{\pgfqpoint{3.530505in}{3.118124in}}%
\pgfpathlineto{\pgfqpoint{3.576051in}{3.142945in}}%
\pgfpathlineto{\pgfqpoint{3.621598in}{3.167546in}}%
\pgfpathlineto{\pgfqpoint{3.667144in}{3.191932in}}%
\pgfpathlineto{\pgfqpoint{3.712691in}{3.216110in}}%
\pgfpathlineto{\pgfqpoint{3.758237in}{3.240084in}}%
\pgfpathlineto{\pgfqpoint{3.803783in}{3.263860in}}%
\pgfpathlineto{\pgfqpoint{3.849330in}{3.287442in}}%
\pgfpathlineto{\pgfqpoint{3.894876in}{3.310835in}}%
\pgfpathlineto{\pgfqpoint{3.940422in}{3.334044in}}%
\pgfpathlineto{\pgfqpoint{3.985969in}{3.357073in}}%
\pgfpathlineto{\pgfqpoint{4.031515in}{3.379926in}}%
\pgfpathlineto{\pgfqpoint{4.077062in}{3.402607in}}%
\pgfpathlineto{\pgfqpoint{4.122608in}{3.425121in}}%
\pgfpathlineto{\pgfqpoint{4.168154in}{3.447470in}}%
\pgfpathlineto{\pgfqpoint{4.213701in}{3.469659in}}%
\pgfpathlineto{\pgfqpoint{4.259247in}{3.491691in}}%
\pgfpathlineto{\pgfqpoint{4.304793in}{3.513569in}}%
\pgfpathlineto{\pgfqpoint{4.350340in}{3.535297in}}%
\pgfpathlineto{\pgfqpoint{4.395886in}{3.556878in}}%
\pgfpathlineto{\pgfqpoint{4.441433in}{3.578315in}}%
\pgfpathlineto{\pgfqpoint{4.486979in}{3.599610in}}%
\pgfpathlineto{\pgfqpoint{4.532525in}{3.620767in}}%
\pgfpathlineto{\pgfqpoint{4.578072in}{3.641789in}}%
\pgfpathlineto{\pgfqpoint{4.623618in}{3.662677in}}%
\pgfpathlineto{\pgfqpoint{4.669164in}{3.683436in}}%
\pgfpathlineto{\pgfqpoint{4.714711in}{3.704066in}}%
\pgfpathlineto{\pgfqpoint{4.760257in}{3.724571in}}%
\pgfpathlineto{\pgfqpoint{4.805803in}{3.744953in}}%
\pgfpathlineto{\pgfqpoint{4.851350in}{3.765215in}}%
\pgfpathlineto{\pgfqpoint{4.896896in}{3.785358in}}%
\pgfpathlineto{\pgfqpoint{4.942443in}{3.805384in}}%
\pgfpathlineto{\pgfqpoint{4.987989in}{3.825296in}}%
\pgfpathlineto{\pgfqpoint{5.033535in}{3.845096in}}%
\pgfpathlineto{\pgfqpoint{5.079082in}{3.864785in}}%
\pgfpathlineto{\pgfqpoint{5.124628in}{3.884366in}}%
\pgfpathlineto{\pgfqpoint{5.170174in}{3.903840in}}%
\pgfpathlineto{\pgfqpoint{5.215721in}{3.923210in}}%
\pgfpathlineto{\pgfqpoint{5.261267in}{3.942476in}}%
\pgfpathlineto{\pgfqpoint{5.306814in}{3.961641in}}%
\pgfpathlineto{\pgfqpoint{5.352360in}{3.980706in}}%
\pgfpathlineto{\pgfqpoint{5.397906in}{3.999673in}}%
\pgfpathlineto{\pgfqpoint{5.443453in}{4.018543in}}%
\pgfpathlineto{\pgfqpoint{5.488999in}{4.037318in}}%
\pgfpathlineto{\pgfqpoint{5.534545in}{4.056000in}}%
\pgfusepath{stroke}%
\end{pgfscope}%
\begin{pgfscope}%
\pgfpathrectangle{\pgfqpoint{0.800000in}{0.528000in}}{\pgfqpoint{4.960000in}{3.696000in}}%
\pgfusepath{clip}%
\pgfsetrectcap%
\pgfsetroundjoin%
\pgfsetlinewidth{1.505625pt}%
\definecolor{currentstroke}{rgb}{1.000000,0.000000,0.000000}%
\pgfsetstrokecolor{currentstroke}%
\pgfsetdash{}{0pt}%
\pgfpathmoveto{\pgfqpoint{1.025455in}{0.696000in}}%
\pgfpathlineto{\pgfqpoint{1.071001in}{0.850408in}}%
\pgfpathlineto{\pgfqpoint{1.116547in}{0.968889in}}%
\pgfpathlineto{\pgfqpoint{1.162094in}{1.068774in}}%
\pgfpathlineto{\pgfqpoint{1.207640in}{1.156774in}}%
\pgfpathlineto{\pgfqpoint{1.253186in}{1.236332in}}%
\pgfpathlineto{\pgfqpoint{1.298733in}{1.309493in}}%
\pgfpathlineto{\pgfqpoint{1.344279in}{1.377590in}}%
\pgfpathlineto{\pgfqpoint{1.389826in}{1.441547in}}%
\pgfpathlineto{\pgfqpoint{1.435372in}{1.502040in}}%
\pgfpathlineto{\pgfqpoint{1.480918in}{1.559577in}}%
\pgfpathlineto{\pgfqpoint{1.526465in}{1.614552in}}%
\pgfpathlineto{\pgfqpoint{1.572011in}{1.667281in}}%
\pgfpathlineto{\pgfqpoint{1.617557in}{1.718018in}}%
\pgfpathlineto{\pgfqpoint{1.663104in}{1.766973in}}%
\pgfpathlineto{\pgfqpoint{1.708650in}{1.814321in}}%
\pgfpathlineto{\pgfqpoint{1.754197in}{1.860212in}}%
\pgfpathlineto{\pgfqpoint{1.799743in}{1.904771in}}%
\pgfpathlineto{\pgfqpoint{1.845289in}{1.948109in}}%
\pgfpathlineto{\pgfqpoint{1.890836in}{1.990321in}}%
\pgfpathlineto{\pgfqpoint{1.936382in}{2.031490in}}%
\pgfpathlineto{\pgfqpoint{1.981928in}{2.071690in}}%
\pgfpathlineto{\pgfqpoint{2.027475in}{2.110986in}}%
\pgfpathlineto{\pgfqpoint{2.073021in}{2.149437in}}%
\pgfpathlineto{\pgfqpoint{2.118567in}{2.187095in}}%
\pgfpathlineto{\pgfqpoint{2.164114in}{2.224007in}}%
\pgfpathlineto{\pgfqpoint{2.209660in}{2.260215in}}%
\pgfpathlineto{\pgfqpoint{2.255207in}{2.295759in}}%
\pgfpathlineto{\pgfqpoint{2.300753in}{2.330674in}}%
\pgfpathlineto{\pgfqpoint{2.346299in}{2.364992in}}%
\pgfpathlineto{\pgfqpoint{2.391846in}{2.398743in}}%
\pgfpathlineto{\pgfqpoint{2.437392in}{2.431953in}}%
\pgfpathlineto{\pgfqpoint{2.482938in}{2.464648in}}%
\pgfpathlineto{\pgfqpoint{2.528485in}{2.496852in}}%
\pgfpathlineto{\pgfqpoint{2.574031in}{2.528585in}}%
\pgfpathlineto{\pgfqpoint{2.619578in}{2.559869in}}%
\pgfpathlineto{\pgfqpoint{2.665124in}{2.590720in}}%
\pgfpathlineto{\pgfqpoint{2.710670in}{2.621158in}}%
\pgfpathlineto{\pgfqpoint{2.756217in}{2.651198in}}%
\pgfpathlineto{\pgfqpoint{2.801763in}{2.680854in}}%
\pgfpathlineto{\pgfqpoint{2.847309in}{2.710143in}}%
\pgfpathlineto{\pgfqpoint{2.892856in}{2.739076in}}%
\pgfpathlineto{\pgfqpoint{2.938402in}{2.767667in}}%
\pgfpathlineto{\pgfqpoint{2.983949in}{2.795928in}}%
\pgfpathlineto{\pgfqpoint{3.029495in}{2.823869in}}%
\pgfpathlineto{\pgfqpoint{3.075041in}{2.851501in}}%
\pgfpathlineto{\pgfqpoint{3.120588in}{2.878834in}}%
\pgfpathlineto{\pgfqpoint{3.166134in}{2.905878in}}%
\pgfpathlineto{\pgfqpoint{3.211680in}{2.932642in}}%
\pgfpathlineto{\pgfqpoint{3.257227in}{2.959135in}}%
\pgfpathlineto{\pgfqpoint{3.302773in}{2.985363in}}%
\pgfpathlineto{\pgfqpoint{3.348320in}{3.011336in}}%
\pgfpathlineto{\pgfqpoint{3.393866in}{3.037060in}}%
\pgfpathlineto{\pgfqpoint{3.439412in}{3.062543in}}%
\pgfpathlineto{\pgfqpoint{3.484959in}{3.087790in}}%
\pgfpathlineto{\pgfqpoint{3.530505in}{3.112810in}}%
\pgfpathlineto{\pgfqpoint{3.576051in}{3.137606in}}%
\pgfpathlineto{\pgfqpoint{3.621598in}{3.162186in}}%
\pgfpathlineto{\pgfqpoint{3.667144in}{3.186556in}}%
\pgfpathlineto{\pgfqpoint{3.712691in}{3.210719in}}%
\pgfpathlineto{\pgfqpoint{3.758237in}{3.234682in}}%
\pgfpathlineto{\pgfqpoint{3.803783in}{3.258450in}}%
\pgfpathlineto{\pgfqpoint{3.849330in}{3.282026in}}%
\pgfpathlineto{\pgfqpoint{3.894876in}{3.305416in}}%
\pgfpathlineto{\pgfqpoint{3.940422in}{3.328624in}}%
\pgfpathlineto{\pgfqpoint{3.985969in}{3.351654in}}%
\pgfpathlineto{\pgfqpoint{4.031515in}{3.374511in}}%
\pgfpathlineto{\pgfqpoint{4.077062in}{3.397197in}}%
\pgfpathlineto{\pgfqpoint{4.122608in}{3.419718in}}%
\pgfpathlineto{\pgfqpoint{4.168154in}{3.442075in}}%
\pgfpathlineto{\pgfqpoint{4.213701in}{3.464274in}}%
\pgfpathlineto{\pgfqpoint{4.259247in}{3.486316in}}%
\pgfpathlineto{\pgfqpoint{4.304793in}{3.508207in}}%
\pgfpathlineto{\pgfqpoint{4.350340in}{3.529947in}}%
\pgfpathlineto{\pgfqpoint{4.395886in}{3.551542in}}%
\pgfpathlineto{\pgfqpoint{4.441433in}{3.572992in}}%
\pgfpathlineto{\pgfqpoint{4.486979in}{3.594303in}}%
\pgfpathlineto{\pgfqpoint{4.532525in}{3.615475in}}%
\pgfpathlineto{\pgfqpoint{4.578072in}{3.636512in}}%
\pgfpathlineto{\pgfqpoint{4.623618in}{3.657416in}}%
\pgfpathlineto{\pgfqpoint{4.669164in}{3.678190in}}%
\pgfpathlineto{\pgfqpoint{4.714711in}{3.698836in}}%
\pgfpathlineto{\pgfqpoint{4.760257in}{3.719357in}}%
\pgfpathlineto{\pgfqpoint{4.805803in}{3.739754in}}%
\pgfpathlineto{\pgfqpoint{4.851350in}{3.760030in}}%
\pgfpathlineto{\pgfqpoint{4.896896in}{3.780188in}}%
\pgfpathlineto{\pgfqpoint{4.942443in}{3.800228in}}%
\pgfpathlineto{\pgfqpoint{4.987989in}{3.820154in}}%
\pgfpathlineto{\pgfqpoint{5.033535in}{3.839967in}}%
\pgfpathlineto{\pgfqpoint{5.079082in}{3.859668in}}%
\pgfpathlineto{\pgfqpoint{5.124628in}{3.879261in}}%
\pgfpathlineto{\pgfqpoint{5.170174in}{3.898746in}}%
\pgfpathlineto{\pgfqpoint{5.215721in}{3.918126in}}%
\pgfpathlineto{\pgfqpoint{5.261267in}{3.937402in}}%
\pgfpathlineto{\pgfqpoint{5.306814in}{3.956575in}}%
\pgfpathlineto{\pgfqpoint{5.352360in}{3.975648in}}%
\pgfpathlineto{\pgfqpoint{5.397906in}{3.994622in}}%
\pgfpathlineto{\pgfqpoint{5.443453in}{4.013498in}}%
\pgfpathlineto{\pgfqpoint{5.488999in}{4.032278in}}%
\pgfpathlineto{\pgfqpoint{5.534545in}{4.050964in}}%
\pgfusepath{stroke}%
\end{pgfscope}%
\begin{pgfscope}%
\pgfsetrectcap%
\pgfsetmiterjoin%
\pgfsetlinewidth{0.803000pt}%
\definecolor{currentstroke}{rgb}{0.000000,0.000000,0.000000}%
\pgfsetstrokecolor{currentstroke}%
\pgfsetdash{}{0pt}%
\pgfpathmoveto{\pgfqpoint{0.800000in}{0.528000in}}%
\pgfpathlineto{\pgfqpoint{0.800000in}{4.224000in}}%
\pgfusepath{stroke}%
\end{pgfscope}%
\begin{pgfscope}%
\pgfsetrectcap%
\pgfsetmiterjoin%
\pgfsetlinewidth{0.803000pt}%
\definecolor{currentstroke}{rgb}{0.000000,0.000000,0.000000}%
\pgfsetstrokecolor{currentstroke}%
\pgfsetdash{}{0pt}%
\pgfpathmoveto{\pgfqpoint{5.760000in}{0.528000in}}%
\pgfpathlineto{\pgfqpoint{5.760000in}{4.224000in}}%
\pgfusepath{stroke}%
\end{pgfscope}%
\begin{pgfscope}%
\pgfsetrectcap%
\pgfsetmiterjoin%
\pgfsetlinewidth{0.803000pt}%
\definecolor{currentstroke}{rgb}{0.000000,0.000000,0.000000}%
\pgfsetstrokecolor{currentstroke}%
\pgfsetdash{}{0pt}%
\pgfpathmoveto{\pgfqpoint{0.800000in}{0.528000in}}%
\pgfpathlineto{\pgfqpoint{5.760000in}{0.528000in}}%
\pgfusepath{stroke}%
\end{pgfscope}%
\begin{pgfscope}%
\pgfsetrectcap%
\pgfsetmiterjoin%
\pgfsetlinewidth{0.803000pt}%
\definecolor{currentstroke}{rgb}{0.000000,0.000000,0.000000}%
\pgfsetstrokecolor{currentstroke}%
\pgfsetdash{}{0pt}%
\pgfpathmoveto{\pgfqpoint{0.800000in}{4.224000in}}%
\pgfpathlineto{\pgfqpoint{5.760000in}{4.224000in}}%
\pgfusepath{stroke}%
\end{pgfscope}%
\begin{pgfscope}%
\pgfsetbuttcap%
\pgfsetmiterjoin%
\definecolor{currentfill}{rgb}{1.000000,1.000000,1.000000}%
\pgfsetfillcolor{currentfill}%
\pgfsetfillopacity{0.800000}%
\pgfsetlinewidth{1.003750pt}%
\definecolor{currentstroke}{rgb}{0.800000,0.800000,0.800000}%
\pgfsetstrokecolor{currentstroke}%
\pgfsetstrokeopacity{0.800000}%
\pgfsetdash{}{0pt}%
\pgfpathmoveto{\pgfqpoint{0.897222in}{3.725543in}}%
\pgfpathlineto{\pgfqpoint{3.713200in}{3.725543in}}%
\pgfpathquadraticcurveto{\pgfqpoint{3.740978in}{3.725543in}}{\pgfqpoint{3.740978in}{3.753321in}}%
\pgfpathlineto{\pgfqpoint{3.740978in}{4.126778in}}%
\pgfpathquadraticcurveto{\pgfqpoint{3.740978in}{4.154556in}}{\pgfqpoint{3.713200in}{4.154556in}}%
\pgfpathlineto{\pgfqpoint{0.897222in}{4.154556in}}%
\pgfpathquadraticcurveto{\pgfqpoint{0.869444in}{4.154556in}}{\pgfqpoint{0.869444in}{4.126778in}}%
\pgfpathlineto{\pgfqpoint{0.869444in}{3.753321in}}%
\pgfpathquadraticcurveto{\pgfqpoint{0.869444in}{3.725543in}}{\pgfqpoint{0.897222in}{3.725543in}}%
\pgfpathclose%
\pgfusepath{stroke,fill}%
\end{pgfscope}%
\begin{pgfscope}%
\pgfsetrectcap%
\pgfsetroundjoin%
\pgfsetlinewidth{1.505625pt}%
\definecolor{currentstroke}{rgb}{0.000000,0.000000,1.000000}%
\pgfsetstrokecolor{currentstroke}%
\pgfsetdash{}{0pt}%
\pgfpathmoveto{\pgfqpoint{0.925000in}{4.050389in}}%
\pgfpathlineto{\pgfqpoint{1.202778in}{4.050389in}}%
\pgfusepath{stroke}%
\end{pgfscope}%
\begin{pgfscope}%
\definecolor{textcolor}{rgb}{0.000000,0.000000,0.000000}%
\pgfsetstrokecolor{textcolor}%
\pgfsetfillcolor{textcolor}%
\pgftext[x=1.313889in,y=4.001778in,left,base]{\color{textcolor}\rmfamily\fontsize{10.000000}{12.000000}\selectfont Observed Period}%
\end{pgfscope}%
\begin{pgfscope}%
\pgfsetrectcap%
\pgfsetroundjoin%
\pgfsetlinewidth{1.505625pt}%
\definecolor{currentstroke}{rgb}{1.000000,0.000000,0.000000}%
\pgfsetstrokecolor{currentstroke}%
\pgfsetdash{}{0pt}%
\pgfpathmoveto{\pgfqpoint{0.925000in}{3.856716in}}%
\pgfpathlineto{\pgfqpoint{1.202778in}{3.856716in}}%
\pgfusepath{stroke}%
\end{pgfscope}%
\begin{pgfscope}%
\definecolor{textcolor}{rgb}{0.000000,0.000000,0.000000}%
\pgfsetstrokecolor{textcolor}%
\pgfsetfillcolor{textcolor}%
\pgftext[x=1.313889in,y=3.808105in,left,base]{\color{textcolor}\rmfamily\fontsize{10.000000}{12.000000}\selectfont Small Angle Approximation Prediction}%
\end{pgfscope}%
\end{pgfpicture}%
\makeatother%
\endgroup%
}
           \caption{Length Dependence of Period with Starting $\theta = \ang{10}$}
           \label{fig:lengthperiod10}
        \end{center}
    \end{figure}
    
    \noindent
    From these two results we can conclude that at small angles Euler's Method, when using Symplectic 
    Integration, is quite good at modelling something like an simple oscillator. However, when 
    deviating slightly from the bounds of the small angle approximation, these numerical methods 
    do well but are not accurate enough to trust entirely. 
    \newline
    In terms of the actual result obtained from investigating the length dependence of period for 
    a simple pendulum, we can see that as length increases, so does the period, but at some sort of 
    logarithmic or inverse quadratic rate. From analysing the equations of motion for the pendulum, 
    we see that $T$ goes by $\frac{1}{\Omega_0}$, which itself goes by $\sqrt{\frac{1}{L}}$. Thus 
    $T \sim \sqrt{L}$, which tells us that this is an inverse quadratic relationship. 
    \newline
    \newline
    In conclusion, these numerical integration methods can be quite inaccurate but using a few tricks, 
    such as symplectic integration and approximations to simplify things, we can confidently use 
    these methods to model systems that have very complex, and often analytically impossible, solutions. 
    

    \newpage
    \section{Appendix}
    \setcounter{figure}{0} \renewcommand{\thefigure}{A.\arabic{figure}}

    \lstinputlisting[caption=Pendulum Update Code, style=appendix]{CP3a.py}
    \lstinputlisting[caption=Symplectic Euler Method, style=appendix]{CP3b.py}
    \lstinputlisting[caption=Length Dependence of Period, style=appendix]{CP3c.py}

\end{document}